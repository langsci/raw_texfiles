\documentclass[output=paper,colorlinks,citecolor=brown,
% hidelinks,
% showindex
]{langscibook}
\ChapterDOI{10.5281/zenodo.14833608}
\author{Zhen Li\affiliation{Peking University}}
\title{The expression of information structure in Teke-Kukuya} 

\IfFileExists{../localcommands.tex}{
   \addbibresource{../localbibliography.bib}
   % add all extra packages you need to load to this file

\usepackage{tabularx,multicol}
\usepackage{url}
\urlstyle{same}

\usepackage{listings}
\lstset{basicstyle=\ttfamily,tabsize=2,breaklines=true}

\usepackage{langsci-basic}
\usepackage{langsci-optional}
\usepackage{langsci-lgr}
\usepackage{langsci-osl}
% \usepackage{./langsci/styles/langsci-lgr}
% \usepackage{./langsci/styles/langsci-osl}
% \usepackage{langsci-gb4e}

\usepackage{tikz}
\usetikzlibrary{patterns,calc}
\pgfdeclarepatternformonly{south east lines}{\pgfqpoint{-0pt}{-0pt}}{\pgfqpoint{3pt}{3pt}}{\pgfqpoint{3pt}{3pt}}{
    \pgfsetlinewidth{0.6pt}
    \pgfpathmoveto{\pgfqpoint{0pt}{3pt}}
    \pgfpathlineto{\pgfqpoint{3pt}{0pt}}
    \pgfpathmoveto{\pgfqpoint{.2pt}{-.2pt}}
    \pgfpathlineto{\pgfqpoint{-.2pt}{.2pt}}
    \pgfpathmoveto{\pgfqpoint{3.2pt}{2.8pt}}
    \pgfpathlineto{\pgfqpoint{2.8pt}{3.2pt}}
    \pgfusepath{stroke}}
    
\usepackage{stmaryrd}
\usepackage{wasysym}
\usepackage{multirow}
\usepackage{caption}
\usepackage{subcaption}
\usepackage{mathrsfs}
\usepackage{qtree}

\usepackage{linguex}


   %pminos do not split footnotes
% \interfootnotelinepenalty=10000 %Footnote in Laporte chapters has to be split SN


%\DeclareIndexNameFormat{default}{%
%\nameparts{#1}%
%\usebibmacro{index:name}%
%{\index[names]}%
%{\namepartfamily}%
%{\namepartgiveni}%
% {}% L1
% {}% L2
%{\namepartprefix}% generates spurious space L3
%{\namepartsuffix}% generates spurious space L4
%}

%  {\DeclareIndexNameFormat{default}{%
%     \usebibmacro{index:name}{\index[names]}{#1}{#3}{#5}{#7}}}

%\DeclareIndexNameFormat{default}{%
%  \usebibmacro{index:name}{\sindex[nom]}{#1}{#3}{#5}{#7}}

%\DeclareIndexNameFormat{default}{%
%  \usebibmacro{index:name}{\sindex[person]}{#1}{#3}{#5}{#7}}
%\DeclareIndexNameFormat{default}{%
%\nameparts{#1} \usebibmacro{index:name}{\sindex[person]]}{\namepartfamily}{‌​\namepartgiven}{\nam‌​epartprefix}{\namepa‌​rtsuffix}}

%\newcommand{\smiley}{:)}

%\renewbibmacro*{index:name}[5]{%
%\usebibmacro{index:entry}{#1}%
%{\iffieldundef{usera}{}{\thefield{usera}\actualoperator}\mkbibindexname{#2}{#3}{#4}{#5}}}

% \newcommand{\noop}[1]{}

%remove for final
%\overfullrule=1mm

\newcommand{\tobi}[2]}}
\renewcommand{\S}[1]{\tobi{#1}{\textsc{*}}}

% this volume references
% puts: [this volume]
% already defined: \citetv
%\newcommand{\citepv}[1]{(\citeauthor{#1} \citeyear*{#1} [this volume])}
\newcommand{\citealtv}[1]{\citeauthor{#1} \citeyear*{#1} [this volume]}

%parentheses around example number
\newcommand{\pref}[1]{(\ref{#1})}

% in-text examples

\newcommand{\lnex}[1]{\textit{#1}} %target lang word
\newcommand{\lnlit}[1]{(lit.: `#1')} %literal reading
\newcommand{\lnlat}[1]{(#1)} % latinization
\newcommand{\lntrans}[1]{`#1'} %translation
\newcommand{\lnexl}[2]%
{\lnex{#1}{} \lnlat{#2}} % ex with latinization
\newcommand{\lnexlat}[3]{\lnex{#1}{} \lnlat{#2}{} \lntrans{#3}} % ex with latinization and tranl.

%ch01
\newcommand{\co}[1]{\mbox{\textbf{#1}}}

%ch09

\newcommand{\cyrbulg}[1]{\begin{otherlanguage*}{bulgarian}#1\end{otherlanguage*}}


%ch10
\newcommand{\nlp}{{\small NLP}}
\newcommand{\mwe}{{\small MWE}}
\newcommand{\rae}{{\small RAE}}
\newcommand{\lvc}{{\small LVC}}
\newcommand{\pos}{{\small P}o{\small S}}
%\newcommand{\todo}[1]{ \textcolor{red}{#1} }

%\renewcommand{\labelenumi}{\theenumi}
%\ainamefmt{{vv}{ll}{, ff}{, jj}} % fullname

\newcommand{\biberror}[1]{{\color{red}#1}}

\newcommand{\osenovaitem}{--~}
   %% hyphenation points for line breaks
%% Normally, automatic hyphenation in LaTeX is very good
%% If a word is mis-hyphenated, add it to this file
%%
%% add information to TeX file before \begin{document} with:
%% %% hyphenation points for line breaks
%% Normally, automatic hyphenation in LaTeX is very good
%% If a word is mis-hyphenated, add it to this file
%%
%% add information to TeX file before \begin{document} with:
%% %% hyphenation points for line breaks
%% Normally, automatic hyphenation in LaTeX is very good
%% If a word is mis-hyphenated, add it to this file
%%
%% add information to TeX file before \begin{document} with:
%% \include{localhyphenation}
\hyphenation{
    Beck-man
    Ngu-yen
    back-chan-nel
    back-chan-nels
    mo-not-o-nous
    ste-reo-typ-i-cal
}

\hyphenation{
    Beck-man
    Ngu-yen
    back-chan-nel
    back-chan-nels
    mo-not-o-nous
    ste-reo-typ-i-cal
}

\hyphenation{
    Beck-man
    Ngu-yen
    back-chan-nel
    back-chan-nels
    mo-not-o-nous
    ste-reo-typ-i-cal
}

   \boolfalse{bookcompile}
   \togglepaper[3]%%chapternumber
}{}


\title{The expression of information structure in Teke-Kukuya} 
\author{Zhen Li\affiliation{Peking University}}
\abstract{This chapter introduces the expression of information structure of Teke-Kukuya which is a Bantu language spoken in the Republic of Congo. I first discuss the canonical word order of the language and what discourse functions it can have. Then the chapter describes the functions and interpretation of a dedicated immediate-before-verb (IBV) focus position that can express various types of focus. Different kinds of topical elements in Kukuya tend to occur in the preverbal domain. Passiveness is expressed via two functional constructions. The language also makes use of cleft constructions to express focus, from which some connections with the IBV focus strategy are observed. In general, the Kukuya language behaves more discourse-configurational.}
\begin{document}
\maketitle
\label{ch:3}

\section{Introduction}\label{teke:sec:1}
This chapter provides a description on the expression of information structure in the Kukuya language ([kkw], B77a). Kukuya is a Bantu language spoken in the Lékana district of the Republic of Congo and is a variety in the Teke group (B70). The phonology and noun class inventories of the Teke-Kukuya variety have been documented in detail by \citet{Paulian1975} and some careful analysis on its prosodic domain was conducted by \citet{Hyman1987}. The data presented here were collected during my fieldwork in the summers of 2019 and 2021 in the Lékana district, where the Kukuya language is predominantly used in daily life, and also in Brazzaville with aged speakers who were born and brought up in Lékana. I have collected both elicited and spontaneous data during the fieldwork, but since the marked information structural expressions such as the use of the dedicated focus position are more obviously attested in elicited sentences, the examples presented in this chapter are largely based on the elicited data and I leave more elaborated investigation on the information structure of spontaneous speech for further research. The elicitation materials for investigating information structure expressions are largely based on the methodological guide ``The BaSIS basics of information structure" \citep{vanderWal2021a} developed as part of the Bantu Syntax and Information Structure (BaSIS) project.

Kukuya was said to be ``economic" \citep{Paulian1998, Paulian2001} in terms of verb derivational suffixes and object marking. However, compared to many other Teke varieties, there is less phonological reduction on the nominal and verb prefixes/stems and there is rich agreement morphology in Kukuya. As we see throughout the chapter, the preverbal subject in Kukuya is always present and is only very rarely omitted, which I take as evidence for its clause-internal status rather than being a dislocated topic. In Kukuya there is much crucial morphosyntactic variation that cannot be accounted for by the traditional point of view on the grammar but must be explained with reference to information structure. In this chapter I show that the word order in Kukuya is to a large extent determined by information structure more so than by traditional grammatical roles. All kinds of topical elements tend to occur in the preverbal domain, while focused elements are usually placed in a dedicated immediate-before-verb (IBV) focus position which is rarely attested in eastern and southern Bantu languages but seems to be an areal feature shared by some West-Coastal Bantu languages \citep{Grégoire1993, Hadermann1996, BostoenMundeke2011, BostoenMundeke2012, DeKind2014, KoniMuluwaBostoen2014, BostoenKoniMuluwa2021}. I show that in Teke-Kukuya, this IBV position is productively exploited, and the element that is placed in the IBV position must be itself focal or part of a larger focal constituent.


The chapter is organised as follows: \sectref{teke:sec:2} introduces the canonical word order SVO and various types of focus that this word order can express; \sectref{teke:sec:3} is dedicated to illustrating different functions of the IBV focus position and interpretations associated with it, as well as some morphological and tonal variation related to this position; \sectref{teke:sec:4} introduces the expressions of (multiple) topical elements in the preverbal domain, and functional passive constructions that combine the use of IBV focus position and topic fronting; \sectref{teke:sec:5} turns to different types of cleft constructions and tries to build up some connections between the (pseudo-)cleft and the IBV focus construction; \sectref{teke:sec:6} summarises.
\section{Canonical word order}\label{teke:sec:2}
In this section I will present the canonical SVO word order in Kukuya and show that the SVO word order can be used to express various types of focus such as argument and adjunct focus, VP focus and polarity focus. The SVO word order is also the most common way of expressing a thetic sentence.
\subsection{SVO as canonical word order}\label{teke:sec:2.1}
When talking about word order, it can be sometimes problematic to generalize what the ``canonical" word order is in a language, depending on different criteria and discourse types. Even synchronic variation within the language can provide different clues on its canonical word order. Here I follow the criterion that the canonical word order of a language is commonly reflected in a ``topic-comment articulation" where the subject of the sentence has a discourse function of topic representing presupposed or given information, and the rest of the sentence expresses new information \citep{Lambrecht1994, Andrews2007}. Thus the canonical word order is expected to show up in the answer to a question such as `What did s/he do' which places focus on the predicate, namely the VP. In \xref{1} we see that to answer such a question, the SVO word order is usually attested, which has been reported to be the canonical word order of most other Bantu languages (\citealt{Bearth2003, vanderWal2015, DowningMarten2019}, among others).
\begin{exe}
    \ex \label{1}
    \begin{xlist}
%%EAX
\ex
%%JUDGEMENT
%%LABEL
\label{1a}
%%CONTEXT
%%LINE2
\gll
Mu-kái kí-má káá-sî?\\
%%LINE3
1-woman 7-what 1\Sm{}.\Pst{}-do.\Pst{}\\
%%TRANS1
\trans ‘What did the woman do?’
%%TRANS2
%%EXEND

%%EAX
\ex
%%JUDGEMENT
%%LABEL
\label{1b}
%%CONTEXT
%%LINE2
\gll
Ndé áá-b\'{u}num-i baa-nts\'{u}\'{u}.\\
%%LINE3
1.\Pro{} 1\Sm{}.\Pst{}-feed-\Pst{} 2-chicken\\
%%TRANS1
\trans ‘She fed the chicken.’
%%TRANS2
%%EXEND

    \end{xlist}
\end{exe} 
However, as we will see shortly, the felicitous answer to a VP question is not restricted to SVO word order, but can also be SOV as shown in \xref{2}, although the occurrence of SVO for VP focus largely surpasses that of SOV in my corpus and is always the first intuition of the speakers. Based on these facts, I assume that the canonical word order of Kukuya, if there is one, should be SVO. We will also see later in this chapter that any deviation of the SVO word order, to a larger or smaller extent, involves some discourse-related manipulations. The SOV word order expressing VP focus is possibly used to mark contrast on the VP (see \sectref{teke:sec:3.2}), or it is in the process of being further grammaticalised from a more marked focus construction to a pragmatically neutral word order secondary to the canonical SVO. 
\begin{exe}
%%EAX
\ex
%%JUDGEMENT
%%LABEL
\label{2}
%%CONTEXT
(visual stimulus: what are the two women doing?)\\
%%LINE2
\gll
Bó ntálí bá-kâ-yílik-a.\\
%%LINE3
2.\Pro{} 9.bed 2\Sm{}-\Prs{}-clear.up-\Fv{}\\
%%TRANS1
\trans ‘They are making the bed.’
%%TRANS2
%%EXEND


\end{exe}
The canonical position of different kinds of adjuncts is usually postverbal and after the object(s) in a transitive construction, as illustrated in \xref{3}. Here I refer to adjuncts as adverbial phrases that add extra information (temporal, locative, manner) to the sentence, which is distinguished from adverbs which modify the verb. From \xref{3} we can also see that in a ditransitive construction in Kukuya, the recipient object always precedes the theme. Example \xref{4} shows that it is ungrammatical to place the locative phrase between the verb and the object, even if this adjunct is in focus. This also indicates that in Kukuya there is no IAV focus position which is well-known in many other Bantu languages. I thus conclude that the canonical constituent order in Kukuya is Subject-Verb-Object-Adjuncts.
\begin{exe}
%%EAX
\ex
%%JUDGEMENT
%%LABEL
\label{3}
%%CONTEXT
%%LINE2
\gll
Nkaaká áá-wî baa-ndzulí bvi-kídzá mu nkunkólo yi.\\
%%LINE3
1.grandmother 1\Sm{}.\Pst{}-give.\Pst{} 2-cat 8-food 18.\Loc{} 9.evening 9.\Dem{}\\
%%TRANS1
\trans ‘The grandmother gave the cats food this evening.’
%%TRANS2
%%EXEND

\end{exe}
\begin{exe} 
    \ex (answer to `where did you see Gilbert?') \label{4}
     \begin{xlist}
%%EAX
\ex
%%JUDGEMENT
[*]{
%%LABEL
\label{4a}
%%CONTEXT
%%LINE2
\gll
Me á-mún-i ku dzándu Gilbert.\\
%%LINE3
1\Sg{}.\Pro{} \Pst{}-1\Sg{}.\Sm{}.see-\Pst{} 17.\Loc{} 5.market Gilbert\\
%%TRANS1
\trans int: ‘I saw Gilbert at the market.’
%%TRANS2
}
%%EXEND

%%EAX
\ex
%%JUDGEMENT
[]{
%%LABEL
\label{4b}
%%CONTEXT
%%LINE2
\gll
Me á-mún-i Gilbert ku dzándu.\\
%%LINE3
1\Sg{}.\Pro{} \Pst{}-1\Sg{}.\Sm{}.see-\Pst{} Gilbert 17.\Loc{} 5.market\\
%%TRANS1
\trans ‘I saw Gilbert at the market.’
%%TRANS2
}
%%EXEND

%%EAX
\ex
%%JUDGEMENT
[]{
%%LABEL
\label{4c}
%%CONTEXT
%%LINE2
\gll
Me ku dzándu á-mún-i Gilbert.\\
%%LINE3
1\Sg{}.\Pro{} 17.\Loc{} 5.market \Pst{}-1\Sg{}.\Sm{}.see-\Pst{} Gilbert\\
%%TRANS1
\trans ‘I saw Gilbert at the market.’
%%TRANS2
}
%%EXEND

    \end{xlist}
\end{exe}
\subsection{Focus expressions in SVO}\label{teke:sec:2.2}
As said above, the canonical SVO word order is usually captured when the whole VP is in focus. In this subsection I will show that the SVO word order can also be used to express term focus, and in fact all types of arguments and adjuncts can be focused in their canonical linear position. In addition, the SVO word order can also express different types of predicate-centered focus (PCF) such as verb focus and polarity focus. I will discuss them in turn.

A question word, which is usually considered to be inherently focused, as well as its corresponding answer, are commonly seen as reliable diagnostics for focus expressions \citep{Rooth1992, Lambrecht1994, Krifka2007, vanderWal2016}. In Kukuya, an object can be focused in its canonical postverbal position. In \xref{5a} we see that the answer to an object question can be SVO with the object being focused in its canonical position. We find in \xref{5b} that this question can also be answered in an SOV order with the object focused in the IBV position, which will be discussed later in \sectref{teke:sec:3.1}. We may also notice in \xref{5} that the tone on the nominal prefix of the object DP and the shape of subject marker on the verb differ depending on the word order, which I will discuss in \sectref{teke:sec:5.2}.
\begin{exe}
    \ex (What did father sell yesterday?) \label{5}
    \begin{xlist}
%%EAX
\ex
%%JUDGEMENT
%%LABEL
\label{5a}
%%CONTEXT
%%LINE2
\gll
Ndé \textbf{áá}-ték-i \textbf{ba}a-ntaba\textsubscript{[\FOC{}]}.\\
%%LINE3
1.\Pro{} 1\Sm{}.\Pst{}-sell-\Pst{} 2-goat\\
%%TRANS1
\trans ‘He sold some \textit{goats}.’ \jambox*{[SVO object focus]}
%%TRANS2
%%EXEND

%%EAX
\ex
%%JUDGEMENT
%%LABEL
\label{5b}
%%CONTEXT
%%LINE2
\gll
Ndé \textbf{bá}a-ntabá\textsubscript{[\FOC{}]} \textbf{káá}-ték-i.\\
%%LINE3
1.\Pro{} 2-goat 1\Sm{}.\Pst{}-sell-\Pst{}\\
%%TRANS1
\trans ‘He sold some \textit{goats}.’ \jambox*{[SOV object focus]}
%%TRANS2
%%EXEND

    \end{xlist}
\end{exe} 
Adjuncts can also be focused in their original postverbal positions. In \xref{6} we see that when answering to a question on the location, the locative phrase providing new information can just occur in its canonical position. 
\begin{exe} 
%%EAX
\ex
%%JUDGEMENT
%%LABEL
\label{6}
%%CONTEXT
 (Where did father buy the wine?) \\
%%LINE2
\gll
Taará áá-fúum-i ma-lí [ku mfaí]\textsubscript{[\FOC{}]}.\\
%%LINE3
1.father 1\Sm{}.\Pst{}-buy-\Pst{} 6-wine {\db}17.\Loc{} 9.capital\\
%%TRANS1
\trans ‘Father bought the wine in Brazzaville.'
%%TRANS2
%%EXEND

\end{exe}
As we will see in \sectref{teke:sec:3.1.1}, question words are predominantly placed in the IBV position, but for some of the interrogative adjuncts, in particular the manner and reason question words \textit{bun\'{i}} `how' and \textit{mu kim\'{a}} `why', they are also widely attested to occur in their canonical postverbal position as in \xref{7}. 
\begin{exe} 
%%EAX
\ex
%%JUDGEMENT
%%LABEL
\label{7}
%%CONTEXT
%%LINE2
\gll
Muu-ndziá áá-dzí ma-ká bu-ní?\\
%%LINE3
1-foreigner 1\Sm{}.\Pst{}-eat.\Pst{} 6-cassava 14-which\\
%%TRANS1
\trans ‘How did the foreigner eat the cassavas?'
%%TRANS2
%%EXEND

\end{exe}
An interrogative adjunct can also be placed in its canonical position in the context of a rhetorical question as in \xref{8}, which usually expresses doubt about or opposition against the previous statement, but not necessarily requests an answer.
\begin{exe} 
%%EAX
\ex
%%JUDGEMENT
%%LABEL
\label{8}
%%CONTEXT
 (Context: the speaker thinks that it is impossible for the person to have seen Alain.) \\
%%LINE2
\gll
Ndé á-mún-i Alain \textbf{ku-ní}?\\
%%LINE3
1.\Pro{} 1\Sm{}.\Pst{}-see-\Pst{} Alain 17-which\\
%%TRANS1
\trans ‘He saw Alain, (but) where?’
%%TRANS2
%%EXEND

\end{exe}
An element modified by `only' is always associated with an exhaustive focus reading. In \xref{9a} and \xref{9b} we see that to place an object DP modified by `only' in its canonical postverbal position and in the IBV position are both grammatical. This shows that even exhaustive focus on the object can be expressed in the canonical SVO word order.
\begin{exe} 
    \ex  \label{9}
    \begin{xlist}
%%EAX
\ex
%%JUDGEMENT
%%LABEL
\label{9a}
%%CONTEXT
%%LINE2
\gll
Mu-loí áá-wî báana \textbf{wúna} maa-nkúru.\\
%%LINE3
1-teacher 1\Sm{}.\Pst{}-give.\Pst{} 2.children only 6-pen\\
%%TRANS1
\trans ‘The teacher gave the children only pens.’
%%TRANS2
%%EXEND

%%EAX
\ex
%%JUDGEMENT
%%LABEL
\label{9b}
%%CONTEXT
%%LINE2
\gll
Nkaaká \textbf{wúna} mvá káá-w\'{i} b\'{u}-ká.\\
%%LINE3
1.grandmother only 1.dog 1\Sm{}.\Pst{}-give.\Pst{} 14-cassava\\
%%TRANS1
\trans ‘The grandmother gave only the dog cassava.’
%%TRANS2
%%EXEND

\end{xlist}
\end{exe}
A subject can also be focused preverbally in SVO word order, as shown in the question-answer pair in \xref{10} and the preverbal subject modified by `only' in \xref{11}. The availability of preverbal subject focus in Kukuya is somehow exceptional considering the rigid constraint against the preverbal subject to be focal in many other Bantu languages \citep{Morimoto2000, Zerbian2006, vanderWal2009a, vanderWal2015, DowningMarten2019}. It should be noted here that I am not claiming that the preverbal subject is structurally focused \textit{in situ}, but in a position which is structurally different from the canonical subject position, though under both circumstances the linear word order is SVO. In \sectref{teke:sec:3.3}, I will distinguish the preverbal topical subject position from the focal subject in IBV position. Here I say that a topical and a focal subject, which structurally occupy different positions, overlap in their linear position in the preverbal domain. 
\begin{exe}
    \ex \label{10}
    \begin{xlist}
%%EAX
\ex
%%JUDGEMENT
%%LABEL
\label{10a}
%%CONTEXT
%%LINE2
\gll
Kí-má	kíí-súruk-i?\\
%%LINE3
7-what	7\Sm{}.\Pst{}-fall-\Pst{}\\
%%TRANS1
\trans ‘What fell down?’
%%TRANS2
%%EXEND

%%EAX
\ex
%%JUDGEMENT
%%LABEL
\label{10b}
%%CONTEXT
%%LINE2
\gll
Mpúku áá-súruk-i.\\
%%LINE3
1.rat 1\Sm{}.\Pst{}-fall-\Pst{}\\
%%TRANS1
\trans ‘A/The rat fell down.’
%%TRANS2
%%EXEND

    \end{xlist}
\end{exe}
\begin{exe}
%%EAX
\ex
%%JUDGEMENT
%%LABEL
\label{11}
%%CONTEXT
%%LINE2
\gll
Wúna	baa-ntsúú	báá-ból-i.\\
%%LINE3
only	2-chicken	2\Sm{}.\Pst{}-decompose-\Pst{}\\
%%TRANS1
\trans ‘Only the \textit{chicken} got bad.’
%%TRANS2
%%EXEND

\end{exe}
SVO is also compatible with various types of predicate-centered focus (PCF) \citep{Güldemann2003, Güldemann2009} including verb focus (state-of-affairs focus) and polarity focus or verum. In \xref{12b} SVO as the answer to the question in \xref{12a} expresses focus on the verb and SVO is used. In \xref{13b} SVO is used to correct the truth value of \xref{13a}, thus we see that SVO is also felicitous to express polarity focus.
 \begin{exe}
    \ex \label{12}
    \begin{xlist}
%%EAX
\ex
%%JUDGEMENT
%%LABEL
\label{12a}
%%CONTEXT
%%LINE2
\gll
Taará mi-féme kí-má ké káá-sí?\\
%%LINE3
1.father 4-pig 7-what 7.\Pro{} 1\Sm{}.\Pst{}-do.\Pst{}\\
%%TRANS1
\trans ‘What did father do to the pigs?’
%%TRANS2
%%EXEND

%%EAX
\ex
%%JUDGEMENT
%%LABEL
\label{12b}
%%CONTEXT
%%LINE2
\gll
Ndé áá-dzw\'{i} mi-f\'{e}me.\\
%%LINE3
1.\Pro{} 1\Sm{}.\Pst{}-kill.\Pst{} 4-pig\\
%%TRANS1
\trans ‘He killed the pigs.’
%%TRANS2
%%EXEND

    \end{xlist}
\end{exe}
\begin{exe}
     \ex \label{13}
    \begin{xlist}
%%EAX
\ex
%%JUDGEMENT
%%LABEL
\label{13a}
%%CONTEXT
%%LINE2
\gll
Gilbert ka-káá-bvúúr-í we mi-pará ni?\\
%%LINE3
Gilbert \Neg{}-1\Sm{}.\Pst{}-return-\Pst{} 2\Sg{}.\Pro{} 4-money \Neg{}\\
%%TRANS1
\trans ‘Gilbert did not return you the money?’
%%TRANS2
%%EXEND

%%EAX
\ex
%%JUDGEMENT
%%LABEL
\label{13b}
%%CONTEXT
%%LINE2
\gll
Ndé áá-bvúur-i me mi-pará.\\
%%LINE3
1.\Pro{} 1\Sm{}.\Pst{}-return-\Pst{} 1\Sg{}.\Pro{} 4-money\\
%%TRANS1
\trans ‘He (did) returned me the money’
%%TRANS2
%%EXEND

    \end{xlist}
\end{exe}
In the above I have shown that SVO can have different uses in terms of information structure. SVO is most commonly used as a ``topic-comment structure" where the subject functions as the topic and the whole VP is focused, and it can also express term focus and different types of PCF. There is no constraint against preverbal subject focus.
\subsection{Thetic sentences}\label{teke:sec:2.3}
In this subsection, I will show that the SVO word order can also be used as a thetic sentence. A thetic sentence is used to present all the information that the sentence carries in one piece, as opposed to a ``categorical" sentence in which the topic and comment can be further divided \citep{Kuroda1972, Sasse1987, Sasse1996}. The thetic sentence is also referred to as ``all-new" or ``all focus" utterance \citep{vanderWal2021a}. The answer to a question such as `what happens' can thus be used to investigate the formation of a thetic sentence, as this type of question often does not presuppose a topical referent and requires information on the whole event. In Kukuya, a thetic sentence usually surfaces in SVO. As shown in \xref{14}, to answer the question `what happened outside', only \xref{14a} with SVO is felicitous, while any deviation of this word order cannot be an appropriate answer. The answer in \xref{14b} is only felicitous when \textit{mw\'{a}ana} `child' has been already mentioned and is what the speakers are talking about. In \xref{15} the preverbal subject is indefinite and non-specific, which are not characteristics of topic. From the context we see that it expresses a thetic meaning as there is no old information presupposed. Here we see that the distinction on definiteness of the preverbal subject may help discern a thetic SVO sentence from a categorical one.
\begin{exe} 
    \ex (What happened outside?)\label{14}
    \begin{xlist}
%%EAX
\ex
%%JUDGEMENT
[]{
%%LABEL
\label{14a}
%%CONTEXT
%%LINE2
\gll
Mvá áá-bvî ku ntsá dzuná.\\
%%LINE3
1.dog 1\Sm{}.\Pst{}-fall.\Pst{} 17.\Loc{} inside 5.hole\\
%%TRANS1
\trans ‘A dog fell into the hole.’
%%TRANS2
}
%%EXEND

%%EAX
\ex
%%JUDGEMENT
[\#]{
%%LABEL
\label{14b}
%%CONTEXT
%%LINE2
\gll
Mwáana taará áá-béer-i.\\
%%LINE3
1.child 1.father 1\Sm{}.\Pst{}-beat-\Pst{}\\
%%TRANS1
\trans ‘The child was beaten by father.’
%%TRANS2
}
%%EXEND

    \end{xlist}
\end{exe}
\begin{exe}
%%EAX
\ex
%%JUDGEMENT
%%LABEL
\label{15}
%%CONTEXT
 (Context: you returned home and found some footprints on the floor, you said to your roommate:) \\
%%LINE2
\gll
\textbf{Mbuurú} áá-yení.\\
%%LINE3
1.person 1\Sm{}.\Pst{}-come.\Pst{}\\
%%TRANS1
\trans ‘Someone came.’ 
%%TRANS2
%%EXEND

\end{exe}
We have already seen above that the preverbal subject in SVO can be topical or focal. The availability of SVO to express the thetic meaning shows that the preverbal subject can also be non-topical (and non-focal), since there is no distinction on topic and comment in a thetic sentence. However, an answer to a question `what happened' may also contain a topic expression due to the tendency to ``accommodate information" in a sentence \citep{Lewis1979, Stalnaker2002, vonFintel2008, vanderWal2016}. Even if there is no identifiable topical referent in the common ground before the discourse starts, the interlocutors tend to accept the referent that occurs at the beginning of the dialogue, for example the `dog' in \xref{14a} above, to ensure a coherent communication. In this regard, SVO may never be really ``thetic", but the preverbal subject can function as an ``immediate topic" that can always rescue the discourse from not having a topic.

A question that asks about the reason may also have a thetic answer, as the reason may not contain any presupposed information known by the addressee. In the examples \xref{16} and \xref{17} below, a subject relative construction is used to answer this kind of \textit{why}-questions. At first glance, the subject relative clause looks like a dedicated strategy to express a thetic meaning, just as in French a cleft can be used in a thetic sentence in which the subject is detopicalised by relativisation. However, since the question word for `why' is formed as \textit{mu kima} which literally means `for what', the relativisation of the subject is more likely to nominalise the whole sentence to congruently answer the question, as the \textit{why}-question is actually a \textit{what}-question which targets at a nominal. For example in \xref{16} the question literally means `for what are the children afraid' which may target a certain object that causes the fear rather than a whole event, therefore the answer is interpreted as `\textit{for} the crocodile that is walking in the yard'. 
\begin{exe}
    \ex \label{16}
    \begin{xlist}
%%EAX
\ex
%%JUDGEMENT
%%LABEL
\label{16a}
%%CONTEXT
%%LINE2
\gll
Báana mu \textbf{ki-má} bá-li ya buokó?\\
%%LINE3
2.children 18.\Loc{} 7-what 2\Sm{}-\Cop{} with 14.fear\\
%%TRANS1
\trans ‘Why are the children afraid?’
%%TRANS2
%%EXEND

%%EAX
\ex
%%JUDGEMENT
%%LABEL
\label{16b}
%%CONTEXT
%%LINE2
\gll
Mu-ŋaaní \textbf{wu}-k\^{a}-dzíe ŋa kalá mbalí.\\
%%LINE3
1.crocodile 1\Rel{}-\Prs{}-walk 16.\Loc{} inside 9.yard\\
%%TRANS1
\trans ‘A crocodile that is walking in the yard.’
%%TRANS2
%%EXEND

    \end{xlist}
\end{exe}
\begin{exe} 
    \ex \label{17}
    \begin{xlist}
%%EAX
\ex
%%JUDGEMENT
%%LABEL
\label{17a}
%%CONTEXT
%%LINE2
\gll
Mu-kái mu \textbf{ki-má} ké káá-mal-í ŋa nzó?\\
%%LINE3
1-woman 18.\Loc{} 7-what 7.\Pro{} 1\Sm{}.\Pst{}-leave-\Pst{} 16.\Loc{} 9.house\\
%%TRANS1
\trans ‘Why did the woman leave home?’
%%TRANS2
%%EXEND

%%EAX
\ex
%%JUDGEMENT
%%LABEL
\label{17b}
%%CONTEXT
%%LINE2
\gll
Mwáana aa ndé \textbf{wǔ}-dzínim-i.\\
%%LINE3
1.child 1.\Conn{} 1.\Pro{} 1\Rel{}-disappear-\Pst{}\\
%%TRANS1
\trans ‘Her son who disappeared.’
%%TRANS2
%%EXEND

    \end{xlist}
\end{exe}
In \xref{18} we see that the \textit{mu} always requires a nominal or nominalised element following it. To answer the question in the context, only \xref{18a} is grammatical as the reason is nominalised thus can be selected by \textit{mu}, while \xref{18b} the sentence is ungrammatical due to the fact that the construction after \textit{mu} is still clausal rather than nominal. From \xref{18a} we also see that the relativisation of the subject can nominalise the clause after \textit{mu} as a whole, since the child becomes happy not because of the `father' himself, but of the \textit{fact} that `father bought him a small goat'. In this sense, the relativisation strategy above in \xref{16} and \xref{17} can also be considered as nominalising the whole clause for the sake of expressing the information as \textit{one piece} \citep{Sasse1987}. 
\begin{exe} 
    \ex  (Why is the child happy?) \label{18}
    \begin{xlist}
%%EAX
\ex
%%JUDGEMENT
[]{
%%LABEL
\label{18a}
%%CONTEXT
%%LINE2
\gll
Mwáana li y\v{a} kí-sáábí mu taará \textbf{wǔ}-fúum-i ndé ntaba.\\
%%LINE3
1.child \Cop{} with 7-happiness 18.\Loc{} 1.father 1\Rel{}-buy-\Pst{} 1.\Pro{} 1.goat\\
%%TRANS1
\trans ‘The child is happy that father bought him a goat.’
%%TRANS2
}
%%EXEND

%%EAX
\ex
%%JUDGEMENT
[*]{
%%LABEL
\label{18b}
%%CONTEXT
%%LINE2
\gll
Mwáana li y\v{a} kí-sáábí mu taará \textbf{áá}-fúum-i ndé ntaba.\\
%%LINE3
1.child \Cop{} with 7-happiness 18.\Loc{} 1.father 1\Sm{}.\Pst{}-buy-\Pst{} 1.\Pro{} 1.goat\\
%%TRANS1
\trans int. ‘The child is happy that father bought him a goat.’
%%TRANS2
}
%%EXEND

    \end{xlist}
\end{exe}
In summary, a thetic meaning is commonly expressed by the canonical SVO word order in Kukuya. A subject relativisation strategy may also be used to express thetic meaning, the motivation of which seems to be nominalising the whole information in the sentence as one chunk. However, the use of relativisation may also be due to the fact that some interrogative words tend to require a nominal answer.

In the previous sections, I have shown that the canonical word order SVO is compatible with different information structural constructions. It can be used to express VP focus, term focus, predicated-centered focus and theticity. In the next section I will introduce how a deviation of this canonical word order, namely the use of the dedicated IBV focus position, is associated with information structure.
\section{Dedicated IBV focus position}\label{teke:sec:3}
This section gives an overview on the availability and interpretation of the dedicated immediate-before-verb (IBV) focus position in Kukuya. I show that the IBV position is available for arguments including subject and object, for adjuncts and even for the infinitive predicate to get focused. Compared to the \textit{in situ} focus strategy introduced above, the element placed in IBV often has an identificational/contrastive focus reading. The interrogative words and contrastively focused elements more strictly occur in the IBV position than other focal elements. The focal interpretation can project from the IBV position to the whole VP.

\begin{table}
    % \centering
    \begin{tabularx}{0.5\textwidth}{XXXX}
    \lsptoprule
        \TOP{} & \TOP{} & \textbf{\FOC} & Verb\\
        \midrule
        S\textsubscript{\TOP} & & & V\\
        & & S\textsubscript{\FOC} & V\\
        O\textsubscript{\TOP} & (O\textsubscript{\TOP}) & S\textsubscript{\FOC} & V\\
        S\textsubscript{\TOP} & (O\textsubscript{\TOP}) & O\textsubscript{\FOC} & V\\
    \lspbottomrule
    \end{tabularx}
    \caption{Linear slots of the preverbal domain in Kukuya}
    \label{tab:kukuya-prev-slots}
\end{table}

An immediate question here is how to define the ``IBV" position in this language. Throughout the chapter, the notion ``IBV position" refers to a particular \textit{structural} position, whether occupied or empty, that is adjacent to the verb and no other constituent can intervene in between. Linearly, a topical or non-focal grammatical subject in the SVO word order can also occur immediately left-adjacent to the verb, but it is not structurally placed in the IBV position, because other constituents can still be inserted between it and the verb (see \sectref{teke:sec:3.1.2}). When the IBV focus position is not filled, a topical or non-focal subject just linearly overlaps the IBV position. If we consider all possible elements in the preverbal domain to occur in different slots which correspond to different structural positions (but not necessarily linear position), which is illustrated in \tabref{tab:kukuya-prev-slots}, we see that the IBV focus position can be clearly discerned. In this table, each line represents a particular construction that will be introduced in the rest of the chapter.
\subsection{Argument and adjunct focus in IBV}\label{teke:sec:3.1}
\subsubsection{Object and adjunct focus in IBV}\label{teke:sec:3.1.1}
In Kukuya, a question word is usually placed in the IBV position. As shown in \xref{19a} and \xref{19b}, the questioned objects \textit{kímá} `what' and \textit{ná} `who' are both placed in the IBV position. In \xref{19c} we see that the question word must be strictly adjacent to the verb and the intervention of another element turns the sentence ungrammatical. Example \xref{19d} shows that to place the questioned object \textit{kímá} `what' in its canonical postverbal position is also ungrammatical.
\begin{exe}
    \ex \label{19}
    \begin{xlist}
%%EAX
\ex
%%JUDGEMENT
[]{
%%LABEL
\label{19a}
%%CONTEXT
%%LINE2
\gll
Mvá \textbf{kí-má} káá-siib-i?\\
%%LINE3
1.dog 7-what 1\Sm{}.\Pst{}-catch-\Pst{}\\
%%TRANS1
\trans ‘What did the dog catch?’
%%TRANS2
}
%%EXEND

%%EAX
\ex
%%JUDGEMENT
[]{
%%LABEL
\label{19b}
%%CONTEXT
%%LINE2
\gll
Taará \textbf{n\'{a}} káá-m\'{u}n-í ku mu-s\'{u}ru?\\
%%LINE3
1.father 1.who 1\Sm{}.\Pst{}-see-\Pst{} 17.\Loc{} 3-forest\\
%%TRANS1
\trans ‘Who did father see in the forest?’
%%TRANS2
}
%%EXEND

%%EAX
\ex
%%JUDGEMENT
[*]{
%%LABEL
\label{19c}
%%CONTEXT
%%LINE2
\gll
Taará \textbf{n\'{a}} ku mu-s\'{u}ru káá-m\'{u}n-í?\\
%%LINE3
1.father 1.who 17.\Loc{} 3-forest 1\Sm{}.\Pst{}-see-\Pst{}\\
%%TRANS1
\trans int. ‘Who did father see in the forest?’
%%TRANS2
}
%%EXEND

%%EAX
\ex
%%JUDGEMENT
[*]{
%%LABEL
\label{19d}
%%CONTEXT
%%LINE2
\gll
Mvá áá-siib-i \textbf{ki-ma}?\\
%%LINE3
1.dog 1\Sm{}.\Pst{}-catch-\Pst{} 7-what\\
%%TRANS1
\trans int. ‘What did the dog catch?’
%%TRANS2
}
%%EXEND

    \end{xlist}
\end{exe}
In ditransitive constructions \xref{20a} and \xref{20b}, the recipient and the patient objects are questioned in the IBV position respectively, while the other non-focal objects also occur in the preverbal domain preceding the IBV position. We see from these examples that a questioned object is strictly placed in the IBV position. We can also see in \xref{20} that there is an agreeing pronoun following the question word. I leave the function of the pronoun to be discussed later and assume the question word and the pronoun to form one inseparable constituent in the IBV position.
\begin{exe}
    \ex \label{20}
    \begin{xlist}
%%EAX
\ex
%%JUDGEMENT
%%LABEL
\label{20a}
%%CONTEXT
%%LINE2
\gll
Nkaaká ma-désu \textbf{ná} ndé káá-wî?\\
%%LINE3
1.grandmother 6-bean 1.who 1.\Pro{} 1\Sm{}.\Pst{}-give.\Pst{}\\
%%TRANS1
\trans ‘To whom did grandmother give the beans?’
%%TRANS2
%%EXEND

%%EAX
\ex
%%JUDGEMENT
%%LABEL
\label{20b}
%%CONTEXT
%%LINE2
\gll
Nkaaká mvá \textbf{kí-má} k\'{e} káá-wî?\\
%%LINE3
1.grandmother 1.dog 7-what 7.\Pro{} 1\Sm{}.\Pst{}-give.\Pst{}\\
%%TRANS1
\trans ‘What did grandmother give to the dog?’
%%TRANS2
%%EXEND

    \end{xlist}
\end{exe}
The answer to an object question also tends to occur in the IBV position, though it is not restricted to this position. As introduced in \sectref{teke:sec:2} and also as in \xref{21}, we see that the answer to a questioned object question can be SVO and SOV, with the focal answer being either in IBV or its canonical postverbal position. Here we see that both preverbal and postverbal focus strategies are available in Kukuya. In the elicitation of question-answer pairs, I had a strong impression that when I put emphatic intonation on the focal answer in French, the speakers were more likely to use the preverbal focus strategy in the corresponding translation. I will discuss the interpretational differences of the IBV and \textit{in situ} object focus strategies in \sectref{teke:sec:3.4}. 
\begin{exe}
    \ex (What did mother buy yesterday?)\label{21}
    \begin{xlist}
%%EAX
\ex
%%JUDGEMENT
%%LABEL
\label{21a}
%%CONTEXT
%%LINE2
\gll
Ndé \textbf{áá}-fúum-i \textbf{mu}-ngwa.\\
%%LINE3
1.\Pro{} 1\Sm{}.\Pst{}-buy-\Pst{} 3-salt\\
%%TRANS1
\trans ‘She bought some \textit{salt}.’
%%TRANS2
%%EXEND

%%EAX
\ex
%%JUDGEMENT
%%LABEL
\label{21b}
%%CONTEXT
%%LINE2
\gll
Ndé \textbf{mú}-ngwa \textbf{káá}-fúum-i.\\
%%LINE3
1.\Pro{} 3-salt 1\Sm{}.\Pst{}-buy-\Pst{}\\
%%TRANS1
\trans ‘She bought some \textit{salt}.’
%%TRANS2
%%EXEND

    \end{xlist}
\end{exe} 
In an alternative question that asks for a choice or preference, as well as in its corresponding answer, the IBV focus strategy is always used as shown in \xref{22} and \xref{23}. SVO is viewed as infelicitous as in \xref{23c}. Here we see that when some (at least one) alternative is explicitly mentioned in the context, SOV must be used for exclusion and identification.
\begin{exe}
    \ex \label{22}
    \begin{xlist}
%%EAX
\ex
%%JUDGEMENT
%%LABEL
\label{22a}
%%CONTEXT
%%LINE2
\gll
Maamá lóoso káá-télek-i wó bú-ka?\\
%%LINE3
1.mother 5.rice 1\Sm{}.\Pst{}-prepare-\Pst{} or 14-cassava\\
%%TRANS1
\trans ‘Did mother cook the rice or the cassava?’
%%TRANS2
%%EXEND

%%EAX
\ex
%%JUDGEMENT
%%LABEL
\label{22b}
%%CONTEXT
%%LINE2
\gll
Ndé bú-ka káá-télek-i.\\
%%LINE3
1.\Pro{} 14-cassava 1\Sm{}.\Pst{}-prepare-\Pst{}\\
%%TRANS1
\trans ‘She prepared the cassava.’
%%TRANS2
%%EXEND

    \end{xlist}
\end{exe}
\begin{exe}
    \ex \label{23}
    \begin{xlist}
%%EAX
\ex
%%JUDGEMENT
[]{
%%LABEL
\label{23a}
%%CONTEXT
%%LINE2
\gll
We \textbf{báa-ntsúú} k\^{a}-dzií kí-dzá wó kí-wáli?\\
%%LINE3
2\Sg{}.\Pro{} 2-chicken 2\Sg{}.\Prs{}-like \Inf{}-eat or 7-duck\\
%%TRANS1
\trans ‘Do you like to eat chicken or duck?’
%%TRANS2
}
%%EXEND

%%EAX
\ex
%%JUDGEMENT
[]{
%%LABEL
\label{23b}
%%CONTEXT
%%LINE2
\gll
Me \textbf{kí-wál-í} k\^{a}-n-dzií kí-dzá.\\
%%LINE3
1\Sg{}.\Pro{} 7-duck \Prs{}-1\Sg{}.\Sm{}-like \Inf{}-eat \\
%%TRANS1
\trans ‘I like to eat duck.’
%%TRANS2
}
%%EXEND

%%EAX
\ex
%%JUDGEMENT
[\#]{
%%LABEL
\label{23c}
%%CONTEXT
%%LINE2
\gll
Me k\^{a}-n-dzií kí-dzá \textbf{kí-wáli}.\\
%%LINE3
1\Sg{}.\Pro{} \Prs{}-1\Sg{}.\Sm{}-like \Inf{}-eat 7-duck \\
%%TRANS1
\trans ‘I like to eat duck.’
%%TRANS2
}
%%EXEND

    \end{xlist}
\end{exe}
The questioned adjuncts such as \textit{munk\'{i}} `when', \textit{kun\'{i}} `where', \textit{bun\'{i}} `how' and \textit{mu kim\'{a}} `why', as inherently focal, are also most commonly placed in the IBV position, as shown in \xxref{24}{26}. As for the answer to an adjunct question, the focused adjunct in the answer can be either in IBV or its base position, as illustrated in \xref{24}.
\begin{exe}
    \ex \label{24}
    \begin{xlist}
%%EAX
\ex
%%JUDGEMENT
%%LABEL
\label{24a}
%%CONTEXT
%%LINE2
\gll
Mwáana \textbf{munkí} káá-dzí ntsúi?\\
%%LINE3
1.child when 1\Sm{}.\Pst{}-eat.\Pst{} 1.fish\\
%%TRANS1
\trans ‘When did the child eat the fish?’
%%TRANS2
%%EXEND

%%EAX
\ex
%%JUDGEMENT
%%LABEL
\label{24b}
%%CONTEXT
%%LINE2
\gll
Nd\'{e} ntsúi \textbf{mu} \textbf{ngwaalí} káá-dzí.\\
%%LINE3
1.\Pro{} 1.fish 18.\Loc{} 9.morning 1\Sm{}.\Pst{}-eat.\Pst{}\\
%%TRANS1
\trans ‘S/He ate the fish in the \textit{morning}.’
%%TRANS2
%%EXEND

%%EAX
\ex
%%JUDGEMENT
%%LABEL
\label{24c}
%%CONTEXT
%%LINE2
\gll
Nd\'{e} áá-dzí ntsúi \textbf{mu} \textbf{ngwaalí}.\\
%%LINE3
1.\Pro{} 1\Sm{}.\Pst{}-eat.\Pst{} 1.fish 18.\Loc{} 9.morning\\
%%TRANS1
\trans ‘S/He ate the fish in the \textit{morning}.’
%%TRANS2
%%EXEND

    \end{xlist}
\end{exe}
\begin{exe}
%%EAX
\ex
%%JUDGEMENT
%%LABEL
\label{25}
%%CONTEXT
%%LINE2
\gll
Li-dzwá nyama wúa, biáwe ndé \textbf{ku-ní} líi-kab-a? \\
%%LINE3
1\Pl{}.\Sm{}-kill 1.animal 1.\Dem{}.II 1\Pl{}.\Pro{} 1.\Pro{} 17-which 1\Pl{}.\Fut{}-share-\Fv{}\\
%%TRANS1
\trans ‘(As) we kill that animal, where will we share it?’
%%TRANS2
%%EXEND

\end{exe}
\begin{exe} 
    \ex \label{26}
    \begin{xlist}
%%EAX
\ex
%%JUDGEMENT
%%LABEL
\label{26a}
%%CONTEXT
%%LINE2
\gll
Mwáana ki-yinga \textbf{bu-ní} kíí-wir-i?\\
%%LINE3
1.child 7-festival 14-which 7\Sm{}.\Pst{}-pass-\Pst{}\\
%%TRANS1
\trans ‘How did the child pass the festival?’
%%TRANS2
%%EXEND

%%EAX
\ex
%%JUDGEMENT
%%LABEL
\label{26b}
%%CONTEXT
%%LINE2
\gll
Ndé ki-yinga \textbf{kí-bvé} kíí-wir-i.\\
%%LINE3
1.\Pro{} 7-festival 7-good 7\Sm{}.\Pst{}-pass-\Pst{}\\
%%TRANS1
\trans ‘S/He passed the festival \textit{well/happily}.’
%%TRANS2
%%EXEND

\end{xlist}
\end{exe}
Some interrogative adjuncts, in particular the manner and reason interrogatives \textit{bun\'{i}} `how' and \textit{mu kim\'{a}} `why', are also attested to occur in their canonical postverbal position as in \xref{27}, in free variation with their counterpart in the IBV position, without triggering interpretational differences. Some interrogative adjuncts occur in the canonical postverbal position in the context of a rhetorical question, see example \xref{8} above.
\begin{exe} 
%%EAX
\ex
%%JUDGEMENT
%%LABEL
\label{27}
%%CONTEXT
%%LINE2
\gll
Ngo káá-kwí \textbf{mu} \textbf{ki-ma}?\\
%%LINE3
1.leopard 1\Sm{}.\Pst{}-die.\Pst{} 18.\Loc{} 7-what\\
%%TRANS1
\trans ‘Why did the leopard die?'
%%TRANS2
%%EXEND

    \end{exe}
An element modified by `only' is always associated with an exhaustive focus reading. In \xref{9} above we have already seen that to place an object DP modified by `only' in the IBV position and the canonical postverbal position are both grammatical. This shows that exclusive focus is not necessarily expressed via the IBV position. Example \xref{28} shows that when excluding some alternatives in an explicitly mentioned set, the exclusively focused phrase can either occur in IBV or its canonical position. In spontaneous speech, I also found both postverbal and preverbal distribution of the \textit{only}-phrases, although the occurrence in the IBV position is more often attested.
\begin{exe} 
    \ex \label{28}
    \begin{xlist}
%%EAX
\ex
%%JUDGEMENT
%%LABEL
\label{28a}
%%CONTEXT
%%LINE2
\gll
Mu-kái áá-fúum-i ntaba y\v{a} má-sáani?\\
%%LINE3
1-woman 1\Sm{}.\Pst{}-buy-\Pst{} 1.goat with 6-plate\\
%%TRANS1
\trans ‘Did the woman buy a goat and some plates?’
%%TRANS2
%%EXEND

%%EAX
\ex
%%JUDGEMENT
%%LABEL
\label{28b}
%%CONTEXT
%%LINE2
\gll
Nd\'{e} w\'{u}na ma-s\'{a}aani k\'{a}-fúum-i.\\
%%LINE3
1.\Pro{} only 6-plate 1\Sm{}.\Pst{}-buy-\Pst{}\\
%%TRANS1
\trans ‘She only bought some plates.’
%%TRANS2
%%EXEND

%%EAX
\ex
%%JUDGEMENT
%%LABEL
\label{28c}
%%CONTEXT
%%LINE2
\gll
Nd\'{e} \'{a}-fúum-i w\'{u}na ma-s\'{a}aani.\\
%%LINE3
1.\Pro{} 1\Sm{}.\Pst{}-buy-\Pst{} only 6-plate\\
%%TRANS1
\trans ‘She only bought some plates.’
%%TRANS2
%%EXEND

\end{xlist}
\end{exe}
Interestingly, where I do find restrictions on the position of phrases with exclusive (exhaustive) focus is in yes-no questions. As \xref{29} shows, in a yes-no question the `\textit{only}'-phrase can only occur in the IBV position but is judged to be ungrammatical in the postverbal position. This restriction does not hold when there is no `only'-phrase in the sentence; both SOV and SVO word order are felicitous to form a yes-no question in that case.
\begin{exe}
     \ex \label{29}
    \begin{xlist}
%%EAX
\ex
%%JUDGEMENT
[]{
%%LABEL
\label{29a}
%%CONTEXT
%%LINE2
\gll
Taará \textbf{wúna} ma-sáání káá-swaak-í?\\
%%LINE3
1.father only 6-plate 1\Sm{}.\Pst{}-wash-\Pst{}\\
%%TRANS1
\trans ‘Did father only wash the plates?’
%%TRANS2
}
%%EXEND

%%EAX
\ex
%%JUDGEMENT
[*]{
%%LABEL
\label{29b}
%%CONTEXT
%%LINE2
\gll
Taará áá-swaak-í \textbf{wúna} ma-sáani?\\
%%LINE3
1.father 1\Sm{}.\Pst{}-wash-\Pst{} only 6-plate\\
%%TRANS1
\trans int. ‘Did father only wash the plates?’
%%TRANS2
}
%%EXEND

    \end{xlist}
\end{exe}
Contrastively focused objects and adjuncts also commonly occur in the IBV position, with rare exceptions. The postverbal locative phrase and the object in statement \xref{30a} are corrected respectively in \xref{30b} and \xref{30c} in the IBV position, while correcting them postverbally is degraded. In \xref{31} the instrumental phrase is also corrected in the IBV position. In \xref{30b} and \xref{31b} we also notice that the focal element in IBV can be preceded by multiple non-focal elements which can be subject, object and adjunct. I will return to discuss this in \sectref{teke:sec:4.1}. 
\begin{exe}
    \ex \label{30}
    \begin{xlist}
%%EAX
\ex
%%JUDGEMENT
%%LABEL
\label{30a}
%%CONTEXT
%%LINE2
\gll
Ngaŋwâ maamá káá-wéek-i mu-nkáání ku Djambala?\\
%%LINE3
9.truth 1.mother 1\Sm{}.\Pst{}-send-\Pst{} 3-letter 17.\Loc{} Djambala\\
%%TRANS1
\trans ‘Did mother send the letter to Djambala?’
%%TRANS2
%%EXEND

%%EAX
\ex
%%JUDGEMENT
%%LABEL
\label{30b}
%%CONTEXT
%%LINE2
\gll
Ambú, ndé mu-nkáání \textbf{mfaí} káá-wéek-i.\\
%%LINE3
no 1.\Pro{} 3-letter 3.capital 1\Sm{}.\Pst{}-send-\Pst{}\\
%%TRANS1
\trans ‘No, she sent the letter to \textit{Brazzaville}.’
%%TRANS2
%%EXEND

%%EAX
\ex
%%JUDGEMENT
%%LABEL
\label{30c}
%%CONTEXT
%%LINE2
\gll
Ambú, ndé \textbf{kí-dzídzilá} káá-wéek-i.\\
%%LINE3
no 1.\Pro{} 7-parcel 1\Sm{}.\Pst{}-send-\Pst{}\\
%%TRANS1
\trans ‘No, she sent a \textit{parcel}.’
%%TRANS2
%%EXEND

    \end{xlist}
\end{exe}  
\begin{exe}
    \ex \label{31}
    \begin{xlist}
%%EAX
\ex
%%JUDGEMENT
%%LABEL
\label{31a}
%%CONTEXT
%%LINE2
\gll
Ki-yélé kíí-nyánim-i kii-mbúli mu míaka.\\
%%LINE3
7-hare 7\Sm{}.\Pst{}-save-\Pst{} 7-lion 18.\Loc{} 4.hand\\
%%TRANS1
\trans ‘The hare saved the lion by hand.’
%%TRANS2
%%EXEND

%%EAX
\ex
%%JUDGEMENT
%%LABEL
\label{31b}
%%CONTEXT
%%LINE2
\gll
Ambú, ndé kii-mbúli \textbf{mu} \textbf{mu-siá} káá-nyánim-i.\\
%%LINE3
no 1.\Pro{} 7-lion 18.\Loc{} 3-rope 1\Sm{}.\Pst{}-save-\Pst{}\\
%%TRANS1
\trans ‘No, he saved the lion with a \textit{rope}.’
%%TRANS2
%%EXEND

    \end{xlist}
\end{exe} 
Some additional examples on contrastive focus with clear context are illustrated in \xref{32} and \xref{33}. Example \xref{32} is felicitous in the context when you did not feed the chicken and went out with your wife, when you returned home, you found that the chickens were full and there were beans on the ground. Your wife did not notice the beans and asked `did someone feed the chickens with rice?' and you corrected her with this sentence. The speaker also suggested some possible context for \xref{32} and \xref{33} as shown in the brackets. From these examples we see that when displacing an element in the IBV position from its canonical position, a set of alternatives is at least implicitly available from the context.
\begin{exe}
%%EAX
\ex
%%JUDGEMENT
%%LABEL
\label{32}
%%CONTEXT
 (Context: there were bags of beans and rice, you found that the chickens were full and only the beans were reduced.)\\
%%LINE2
\gll
Mbuurú baa-ntsúú \textbf{má-désu} káá-búnum-i.\\
%%LINE3
1.person 2-chicken 6-bean 1\Sm{}.\Pst{}-feed-\Pst{}\\
%%TRANS1
\trans ‘The person/Someone fed the chickens the \textit{beans}.’
%%TRANS2
%%EXEND

\end{exe}
\begin{exe}
%%EAX
\ex
%%JUDGEMENT
%%LABEL
\label{33}
%%CONTEXT
 (Context: you see that the child is sitting on the ground and crying, your friend asks from some distance away `did something hurt the child's legs?', and you correct her/him.)\\
%%LINE2
\gll
Ki-lóko mwáana \textbf{mú-tswé} kíí-búl-i.\\
%%LINE3
7-thing 1.child 3-head 7\Sm{}.\Pst{}-hurt-\Pst{}\\
%%TRANS1
\trans ‘The (particular) thing hurt the child's \textit{head}.’
%%TRANS2
%%EXEND

\end{exe}
An interim generalisation here is that a focal object or adjunct can be either placed in IBV position or its canonical postverbal position, while some types of foci such as question words and contrastively focused elements particularly favour the IBV position. This is in line with the idea that specific types of focus and different ``degrees" of contrast can be syntactically identified (\citealt{Cruschina2021} a.o.). The IBV position, as the more marked focus position than the canonical position  in terms of word order, is reserved for higher degree of contrast while the canonical postverbal position may encode less or no contrast. I will show in \sectref{teke:sec:3.4} that the IBV position usually expresses identificational focus, while assertively focused elements tend to stay in their canonical positions.
\subsubsection{Subject focus in IBV}\label{teke:sec:3.1.2}
Subject focus in Kukuya can be expressed in three ways, namely in the IBV position, by an OSV word order or by using a pseudo-cleft construction. We will see shortly that the former two means should be considered as different realisations of the same IBV subject focus strategy.

First, to question a subject, the pseudo-cleft construction seems to be the most widely used in my corpus, and an example is shown (\ref{34a}--\ref{34c}). In these constructions, the subordinated clause is a relative clause with a covert head, and the predicative focused subject DP occur sentence-finally. A copula linking the relative part and the predicative DP is only visible in negative context as in \xref{34c}, in which the subject marking on the copula is by default the class 7 subject marker \textit{kí}-.
\begin{exe}
    \ex \label{34}
    \begin{xlist}
%%EAX
\ex
%%JUDGEMENT
%%LABEL
\label{34a}
%%CONTEXT
%%LINE2
\gll
Ki-kí-túm-í mbaá ki-namá \textbf{kí-ma}?\\
%%LINE3
7\Rel{}-7\Sm{}-cause-\Pst{} 9.fire \Inf{}-burn 7-what\\
%%TRANS1
\trans ‘What caused the fire?’
%%TRANS2
%%EXEND

%%EAX
\ex
%%JUDGEMENT
%%LABEL
\label{34b}
%%CONTEXT
%%LINE2
\gll
Wǔ-fúum-i ma-li taará.\\
%%LINE3
1\Rel{}-buy-\Pst{} 6-wine 1.father\\
%%TRANS1
\trans ‘(The one) who bought the wine is father.’
%%TRANS2
%%EXEND

%%EAX
\ex
%%JUDGEMENT
%%LABEL
\label{34c}
%%CONTEXT
%%LINE2
\gll
Wǔ-dzí baa-ntsúú ka-kí-li mvá ni.\\
%%LINE3
1\Rel{}-eat.\Pst{} 2-chicken \Neg{}-7\Sm{}-\Cop{} 1.dog \Neg{}\\
%%TRANS1
\trans ‘(The one) who ate the chicken was not the dog.’
%%TRANS2
%%EXEND

    \end{xlist}
\end{exe}
An alternative strategy to focus the subject is to place the subject in the IBV position, as the question word in \xref{35}, the subject modified by `only' in \xref{36} and the answer to a subject question in \xref{37}. It is noteworthy that the subject is focused in the IBV position which is structurally different from its canonical preverbal position. When a focused subject appears preverbally, no other element can occur between this subject and the verb as shown in \xref{35b}, which is not characteristic of the topical subject, therefore the focused subject must be placed in a different structural position, which is the IBV.
\begin{exe}
    \ex \label{35}
    \begin{xlist}
%%EAX
\ex
%%JUDGEMENT
[]{
%%LABEL
\label{35a}
%%CONTEXT
%%LINE2
\gll
\textbf{Ná} áá-t\'{e}k-i mu-ngwa?\\
%%LINE3
1.who 1\Sm{}.\Pst{}-sell-\Pst{} 3-salt\\
%%TRANS1
\trans ‘Who sold the salt?’
%%TRANS2
}
%%EXEND

%%EAX
\ex
%%JUDGEMENT
[*]{
%%LABEL
\label{35b}
%%CONTEXT
%%LINE2
\gll
\textbf{Ná} mú-ngwa káá-t\'{e}k-i?\\
%%LINE3
1.who 3-salt 1\Sm{}.\Pst{}-sell-\Pst{}\\
%%TRANS1
\trans int. ‘Who sold the salt?’
%%TRANS2
}
%%EXEND

    \end{xlist}
\end{exe}
\begin{exe}
%%EAX
\ex
%%JUDGEMENT
%%LABEL
\label{36}
%%CONTEXT
%%LINE2
\gll
Wúna	\textbf{baa-ntsúú}	báá-ból-i.\\
%%LINE3
only	2-chicken	2\Sm{}.\Pst{}-rot-\Pst{}\\
%%TRANS1
\trans ‘Only the chicken rotted.’
%%TRANS2
%%EXEND

\end{exe}
\begin{exe}
%%EAX
\ex
%%JUDGEMENT
%%LABEL
\label{37}
%%CONTEXT
(`Who gave the child the oranges?')\\
%%LINE2
\gll
\textbf{Bí-búru} bíí-wî mwáana ma-láara.\\
%%LINE3
8-parent 8\Sm{}.\Pst{}-give.\Pst{} 1.child 6-orange\\
%%TRANS1
\trans ‘The \textit{parents} gave the child the oranges.’
%%TRANS2
%%EXEND


\end{exe}
Intriguingly, an exception to the legitimacy of a preverbal focal subject is the \textit{which}-phrase. According to many speakers, a \textit{which}-phrase cannot be placed in IBV position in the same way as other interrogative phrases, but can only occur in a reverse pseudo-cleft sentence. We see in \xref{38} and \xref{39} that the \textit{which}-phrase and \textit{whose}-phrase are not compatible with canonical subject marking, which indicates that they cannot function as the grammatical subject of the sentence, but can only occur in a pseudo-cleft construction. 
\begin{exe}
    \ex \label{38}
    \begin{xlist}
%%EAX
\ex
%%JUDGEMENT
[*]{
%%LABEL
\label{38a}
%%CONTEXT
%%LINE2
\gll
Mw\'{a}ana wu-ní \textbf{áá}-mún-i Zacharie?\\
%%LINE3
1.child 1-which 1\Sm{}.\Pst{}-see-\Pst{} Zacharie\\
%%TRANS1
\trans int. ‘Which child saw Zacharie?’
%%TRANS2
}
%%EXEND

%%EAX
\ex
%%JUDGEMENT
[]{
%%LABEL
\label{38b}
%%CONTEXT
%%LINE2
\gll
Mw\'{a}ana wu-ní \textbf{wǔ}-mún-i Zacharie?\\
%%LINE3
1.child 1-which 1\Rel{}-see-\Pst{} Zacharie\\
%%TRANS1
\trans ‘Which child is the one who saw Zacharie?’
%%TRANS2
}
%%EXEND

    \end{xlist}
\end{exe}
\begin{exe}
    \ex \label{39}
    \begin{xlist}
%%EAX
\ex
%%JUDGEMENT
%%LABEL
\label{39a}
%%CONTEXT
%%LINE2
\gll
Mu-káli wuu n\'{a} (\textbf{*á})-n\'{i}ak-i mw\'{a}ana?\\
%%LINE3
1-wife 1.\Conn{} 1.who {\db}1\Sm{}.\Pst{}-abandon-\Pst{} 1.child\\
%%TRANS1
\trans int. ‘Whose wife abandoned the child?’
%%TRANS2
%%EXEND

%%EAX
\ex
%%JUDGEMENT
%%LABEL
\label{39b}
%%CONTEXT
%%LINE2
\gll
Mu-káli wuu n\'{a} \textbf{wǔ}-n\'{i}ak-i mw\'{a}ana?\\
%%LINE3
1-wife 1.\Conn{} 1.who 1\Rel{}-abandon-\Pst{} 1.child\\
%%TRANS1
\trans ‘Whose wife abandoned the child?’
%%TRANS2
%%EXEND

    \end{xlist}
\end{exe}
A \textit{which}-phrase is usually considered to be discourse-linked and presupposes an antecedent in the given discourse, thus does not necessarily trigger discourse-new information \citep{Şener2010}. On the opposite, a non-discourse-linked interrogative phrase does not presuppose an antecedent and always functions as a focal phrase. As we will see throughout this chapter that the preverbal domain is available for elements of various information structural status, it thus seems unexpected that the D-linked \textit{which}- and \textit{whose}-phrases are not compatible with preverbal focus in a mono-clausal construction. One possible motivation for the D-linked interrogatives to prefer a cleft construction may be that the presupposed existence makes the question as selective, which patterns with the pseudo-cleft construction. Here again it shows that different types and ``degrees" of contrast and focus may be encoded through different grammatical strategies.

There are some interpretational differences between the pseudo-cleft construction and subject focus in IBV. In \xref{40a} the subject of the embedded clause is questioned in a pseudo-cleft, and in \xref{40b} it is questioned in the IBV position. According to the speakers, \xref{40a} is used in the context where there is a presupposed set of candidates who killed the king, which means the speaker has already a group of suspects; while in \xref{40b} there is no candidate invoked in the speaker's mind. In this sense the pseudo-cleft construction is more discourse-linked than the IBV subject focus strategy.
\begin{exe}
    \ex \label{40}
    \begin{xlist}
%%EAX
\ex
%%JUDGEMENT
%%LABEL
\label{40a}
%%CONTEXT
%%LINE2
\gll
Ndé kâ-tsuomó ndíri [\textbf{wǔ}-dzwí mu-kóko \textbf{na}].\\
%%LINE3
1.\Pro{} 1\Sm{}.\Prs{}-think 1.\Comp{} {\db}1\Rel{}-kill.\Pst{} 1-king 1.who\\
%%TRANS1
\trans ‘S/He is thinking about who killed the king.’
%%TRANS2
%%EXEND

%%EAX
\ex
%%JUDGEMENT
%%LABEL
\label{40b}
%%CONTEXT
%%LINE2
\gll
Ndé kâ-tsuomó ndíri [\textbf{ná} áá-dzwí mu-kóko].\\
%%LINE3
1.\Pro{} 1\Sm{}.\Prs{}-think 1.\Comp{} {\db}1.who 1\Sm{}.\Pst{}-kill.\Pst{} 1-king\\
%%TRANS1
\trans ‘S/He is thinking about who killed the king.’
%%TRANS2
%%EXEND

    \end{xlist}
\end{exe} 
Subject focus in the IBV position is very commonly accompanied by the fronting of topical object(s) to the preverbal domain. For example, the answer to a subject question in \xref{37} above can be alternatively expressed as in \xref{41} in which the subject is focused in the IBV position and the given objects are all preposed to the preverbal domain, surfacing an OOSV word order. Similarly in \xref{42}, both in the question and the answer, the focused subject is placed in the IBV position with some object being fronted to the preverbal domain. The OSV word order is in fact very commonly attested in Kukuya for expressing subject focus, and we will also see in \sectref{teke:sec:4.2} that the OSV can function as an equivalent of the passive construction. I assume that this OSV construction is not a third strategy for subject focus, but it is just a different realisation of the IBV focus strategy introduced above, and here in the OSV construction some element must be marked salient as topic of the sentence and fronted to the initial position. The preposing of other preverbal constituents can also help to identify the subject as being placed in the IBV focus position. 
\begin{exe}
%%EAX
\ex
%%JUDGEMENT
%%LABEL
\label{41}
%%CONTEXT
 (`Who gave the child the oranges?') \\
%%LINE2
\gll
Mwáana ma-láara \textbf{bí-búru} bíí-wî.\\
%%LINE3
1.child 6-orange 8-parent 8\Sm{}.\Pst{}-give.\Pst{}\\
%%TRANS1
\trans ‘The child was given the oranges by the \textit{parents}.’
%%TRANS2
%%EXEND

\end{exe}
\begin{exe}
    \ex \label{42}
    \begin{xlist}
%%EAX
\ex
%%JUDGEMENT
%%LABEL
\label{42a}
%%CONTEXT
%%LINE2
\gll
Taará téme \textbf{ná} áá-sonom-i?\\
%%LINE3
1.father 5.hoe 1.who 1\Sm{}.\Pst{}-lend-\Pst{}\\
%%TRANS1
\trans ‘Who lent father the hoe?’
%%TRANS2
%%EXEND

%%EAX
\ex
%%JUDGEMENT
%%LABEL
\label{42b}
%%CONTEXT
%%LINE2
\gll
Téme \textbf{nkaaká} áá-sonom-í taará.\\
%%LINE3
5.hoe 1.grandmother 1\Sm{}.\Pst{}-lend-\Pst{} 1.father\\
%%TRANS1
\trans ‘Grandmother lent father the hoe.’
%%TRANS2
%%EXEND

    \end{xlist}
\end{exe}
 I have shown in the previous subsection that focal objects and adjuncts can occur either in the IBV position or their canonical postverbal position. Here we see that the subject can also be focused in the IBV position, and we can generalise that all arguments and adjuncts in Kukuya can be focused in this verb-adjacent IBV position. Next I will investigate whether the IBV position can be used to express focus on an element which is structurally smaller than the argument/adjunct, namely a modifier, or on a larger constituent such as the VP.
\subsubsection{Sub-NP focus}\label{teke:sec:3.1.3}
In this subsection I show that sub-NP focus can also be expressed by placing the NP in the IBV position, as it can express focus on a modifier. In example \xref{43} we see that the interrogative quantifier \textit{kwê} `how many' occurs in the IBV position, following the NP that it modifies. From this example it is not clear whether it is the whole DP including the head noun and the quantifier that is placed in the IBV position, or solely the quantifier occupies the IBV position.
\begin{exe}
%%EAX
\ex
%%JUDGEMENT
%%LABEL
\label{43}
%%CONTEXT
%%LINE2
\gll
Ba-nziá ma-tsúku kwê bâ-sá ŋa ntsá bu-lá ba?\\
%%LINE3
2-foreigner 6-day how.many 2\Sm{}.\Fut{}-stay 16.\Loc{} inside 14-village 14.\Dem{}\\
%%TRANS1
\trans ‘How long will the foreigners stay in this village?’
%%TRANS2
%%EXEND

\end{exe}
The sentences in \xref{44} were elicited with a picture in which a woman is holding three knives in her hand. In \xref{44a} the numeral quantifier is correctively focused and takes the H tone prefix, while the nominal prefix of the head NP keeps the L tone; in \xref{44b} it is only the head NP that is focal and takes the H tone prefix, with the quantifier following it; \xref{44c} conveys focus on the whole quantified NP, and in this case the H tone prefix only appears on the head NP.\largerpage[2]


\begin{exe}
     \ex (visual stimuli: a woman is holding three knives in her hand.)\label{44}
    \begin{xlist}
%%EAX
\ex
%%JUDGEMENT
%%LABEL
\label{44a}
%%CONTEXT
 (Is the woman holding TWO knives?)\\
%%LINE2
\gll
Ndé maa-mbhielé \textbf{má}-tíri kâ-kwaal-a.\\
%%LINE3
1.\Pro{} 6-knive 6-three 1\Sm{}.\Prs{}-hold-\Fv{}\\
%%TRANS1
\trans ‘She is holding \textit{three} knives.’
%%TRANS2
%%EXEND

%%EAX
\ex
%%JUDGEMENT
%%LABEL
\label{44b}
%%CONTEXT
 (Is the woman holding three \textit{spoons}?)\\
%%LINE2
\gll
Ndé \textbf{má}a-mbhielé \textbf{ma}-tíri kâ-kwaal-a.\\
%%LINE3
1.\Pro{} 6-knive 6-three 1\Sm{}.\Prs{}-hold-\Fv{}\\
%%TRANS1
\trans ‘She is holding three \textit{knives}.’
%%TRANS2
%%EXEND

%%EAX
\ex
%%JUDGEMENT
%%LABEL
\label{44c}
%%CONTEXT
 (Is the woman holding \textit{two spoons}?/What is the woman holding?) \\
%%LINE2
\gll
Ndé \textbf{má}a-mbhielé ma-tíri kâ-kwaal-a.\\
%%LINE3
1.\Pro{} 6-knife 6-three 1\Sm{}.\Prs{}-hold-\Fv{}\\
%%TRANS1
\trans ‘She is holding \textit{three knives}.’
%%TRANS2
%%EXEND

    \end{xlist}
\end{exe}
From these examples we see that only one H tone prefix can occur on the preverbal elements, either on an NP or its modifier, and it is only when the modifier itself is focal that it can take the H tone prefix. One question that arises from \xref{44} is whether the H tone prefix always aligns with focus and with the IBV position. Here I suppose that the head NP and the modifier are separated in \xref{44a}, the head NP functions as a dislocated topic and only the modifier is focused in IBV, since the head NP in this case can also be elided; in \xref{44b} and \xref{44c} the head NP and the modifier are one constituent and only the prefix on the head NP can take the H tone which maps onto the focus reading and the IBV position.

So far in this section I have shown that the IBV focus position is available for argument focus including subject and object, adjunct focus as well as sub-NP focus on a modifier. In the next section I will look into how the IBV position is exploited beyond term focus, namely in predicate-centered focus.

\subsection{Predicate(-centered) focus and IBV}\label{teke:sec:3.2}
\subsubsection{VP focus and verb focus}\label{teke:sec:3.2.1}
In this section I refer to predicate focus as focus on the whole verb phrase. Predicate\hyp centered focus (PCF), first defined in \citet{Güldemann2003}, indicates the focus on \textit{part} of the predicate and can be further divided into state-of-affairs focus which is also referred to as verb focus, tense/aspect/mood (TAM) focus, and polarity focus/verum; the latter two are also referred to as operator focus.

As introduced in \sectref{teke:sec:2}, VP focus in Kukuya is most commonly expressed via the canonical SVO word order. Example \xref{45} is extracted from a written translation task done by the speakers in which they gave most answers to the VP questions in SVO word order. However, the IBV focus position can also be employed when answering a VP question, as shown in \xref{46c}. In elicitation the speakers usually cannot explain the interpretational differences between SVO and SOV when expressing VP focus, while there are in fact more pragmatic restrictions for the SOV word order to occur, as I will introduce shortly.
  \begin{exe}
    \ex \label{45}
    \begin{xlist}
%%EAX
\ex
%%JUDGEMENT
%%LABEL
\label{45a}
%%CONTEXT
%%LINE2
\gll
Huguette bu-k\'{i}a k\'{i}-m\'{a} k\^{a}-s\'{a}?\\
%%LINE3
Huguette 14-tomorrow 7-what 1\Sm{}.\Fut{}-do\\
%%TRANS1
\trans ‘What will Huguette do tomorrow?’
%%TRANS2
%%EXEND

%%EAX
\ex
%%JUDGEMENT
%%LABEL
\label{45b}
%%CONTEXT
%%LINE2
\gll
Bu-k\'{i}a, ndé \^{a}-y\'{e} k\'{i}e báa-nd\'{u}ku.\\
%%LINE3
14-tomorrow 1.\Pro{} 1\Sm{}.\Fut{}-go visit 2-friend \\
%%TRANS1
\trans ‘Tomorrow she will go to visit friends.’
%%TRANS2
%%EXEND

    \end{xlist}
\end{exe}
 \begin{exe}
    \ex \label{46}
    \begin{xlist}
%%EAX
\ex
%%JUDGEMENT
%%LABEL
\label{46a}
%%CONTEXT
%%LINE2
\gll
Taará kí-má ké káá-sí ŋa ngwaal\'{i}?\\
%%LINE3
1.father 7-what 7.\Pro{} 1\Sm{}.\Pst{}-do.\Pst{} 16.\Loc{} 9.morning\\
%%TRANS1
\trans ‘What did father do in the morning?’
%%TRANS2
%%EXEND

%%EAX
\ex
%%JUDGEMENT
%%LABEL
\label{46b}
%%CONTEXT
%%LINE2
\gll
Ndé áá-dzw\'{i} mi-f\'{e}me.\\
%%LINE3
1.\Pro{} 1\Sm{}.\Pst{}-kill.\Pst{} 4-pig\\
%%TRANS1
\trans ‘He killed some pigs.’
%%TRANS2
%%EXEND

%%EAX
\ex
%%JUDGEMENT
%%LABEL
\label{46c}
%%CONTEXT
%%LINE2
\gll
Ndé m\'{i}-f\'{e}me káá-dzw\'{i}.\\
%%LINE3
1.\Pro{} 4-pig 1\Sm{}.\Pst{}-kill.\Pst{}\\
%%TRANS1
\trans ‘He killed some pigs.’
%%TRANS2
%%EXEND

    \end{xlist}
\end{exe}
The availability of SOV word order to express VP focus could be viewed as a counterargument for the IBV focus position, as in this case the focus is not only on the element immediate before the verb, but is on the whole predicate that contains the IBV element. I propose that we do not need to reject the hypothesis of IBV as a focus position but can revise the hypothesis to say that the IBV element should at least be part of the focus, and the IBV element as the \textit{nucleus} of the focus set can project up to the whole verb phrase, which depends on the discourse context (\citealt[555]{Selkirk1995}, \citealt{Reinhart2006}, \citealt[241]{vanderWal2009a}).

An alternative question on the VP and its congruent answer can both be expressed by SOV as shown in \xref{47}. The yes-no question in \xref{48a} focuses the whole VP and also uses SOV, here we see that SVO in \xref{48b} and SOV in \xref{48c} can both be felicitous additive responses to this question, which again shows that both SVO and SOV word order can signal focus on the whole VP.
 \begin{exe}
    \ex \label{47}
    \begin{xlist}
%%EAX
\ex
%%JUDGEMENT
%%LABEL
\label{47a}
%%CONTEXT
%%LINE2
\gll
We má-s\'{a}\'{a}n\'{i} \'{a}-swaak\'{i} wó bi-báa-wî we áá-sî?\\
%%LINE3
2\Sg{}.\Pro{} 6-plate 2\Sg{}.\Pst{}-wash-\Pst{} or 8\Rel{}-2\Sm{}.\Pst{}-give.\Pst{} 2\Sg{}.\Pro{} 2\Sg{}.\Pst{}-do.\Pst{}\\
%%TRANS1
\trans ‘Did you wash the plates or do your homework?’
%%TRANS2
%%EXEND

%%EAX
\ex
%%JUDGEMENT
%%LABEL
\label{47b}
%%CONTEXT
%%LINE2
\gll
Me má-sáání á-n-swaak-í.\\
%%LINE3
1\Sg{}.\Pro{} 6-plates \Pst{}-1\Sg{}.\Sm{}-wash-\Pst{}\\
%%TRANS1
\trans ‘I washed the plates.’
%%TRANS2
%%EXEND

    \end{xlist}
\end{exe}
 \begin{exe}
    \ex \label{48}
    \begin{xlist}
%%EAX
\ex
%%JUDGEMENT
%%LABEL
\label{48a}
%%CONTEXT
%%LINE2
\gll
Ndé wúna bi-ko káá-swaak-í?\\
%%LINE3
1.\Pro{} only 8-clothes 1\Sm{}.\Pst{}-wash-\Pst{}\\
%%TRANS1
\trans ‘Did he only wash the clothes?’    
%%TRANS2
%%EXEND

%%EAX
\ex
%%JUDGEMENT
%%LABEL
\label{48b}
%%CONTEXT
%%LINE2
\gll
Ndé áá-búnum-i bii-ndomó hé.\\
%%LINE3
1.\Pro{} 1\Sm{}.\Pst{}-feed-\Pst{} 8-goat also\\
%%TRANS1
\trans ‘He also fed the goats.’
%%TRANS2
%%EXEND

%%EAX
\ex
%%JUDGEMENT
%%LABEL
\label{48c}
%%CONTEXT
%%LINE2
\gll
Ndé bíi-ndomó hé káá-búnum-i (hé).\\
%%LINE3
1.\Pro{} 8-goat also 1\Sm{}.\Pst{}-feed-\Pst{} {\db}also\\
%%TRANS1
\trans ‘He also fed the goats.’
%%TRANS2
%%EXEND

    \end{xlist}
\end{exe}
Verb focus, also known as state-of-affairs focus which locates focus on the lexical value of the verb, can be expressed in different ways in Kukuya. As mentioned in \sectref{teke:sec:2}, SVO can be used to signal verb focus. The answer to a question like `what did X do to Y?' can be used to diagnose verb focus expressions, in which the subject and the object are both topical since they are already given in the background and the focus is on the verb itself. Interestingly, we see in \xref{49} and \xref{50} that SVO and SOV can both signal verb focus, while OSV with the subject in IBV cannot be used as a felicitous answer.
\begin{exe}
    \ex \label{49}
    \begin{xlist}
%%EAX
\ex
%%JUDGEMENT
[]{
%%LABEL
\label{49a}
%%CONTEXT
%%LINE2
\gll
Ngolo Marie kí-má káá-sî?\\
%%LINE3
Ngolo Marie 7-what 1\Sm{}.\Pst{}-do.\Pst{}\\
%%TRANS1
\trans ‘What did Ngolo do to Marie?’
%%TRANS2
}
%%EXEND

%%EAX
\ex
%%JUDGEMENT
[]{
%%LABEL
\label{49b}
%%CONTEXT
%%LINE2
\gll
Ngolo áá-pfur-í Marie.\\
%%LINE3
Ngolo 1\Sm{}.\Pst{}-cheat-\Pst{} Marie\\
%%TRANS1
\trans ‘Ngolo betrayed Marie.’
%%TRANS2
}
%%EXEND

%%EAX
\ex
%%JUDGEMENT
[]{
%%LABEL
\label{49c}
%%CONTEXT
%%LINE2
\gll
Ngolo Marie káá-pfur-í.\\
%%LINE3
Ngolo Marie 1\Sm{}.\Pst{}-cheat-\Pst{}\\
%%TRANS1
\trans ‘Ngolo betrayed Marie.’
%%TRANS2
}
%%EXEND

%%EAX
\ex
%%JUDGEMENT
[\#]{
%%LABEL
\label{49d}
%%CONTEXT
%%LINE2
\gll
Marie Ngolo áá-pfur-í.\\
%%LINE3
Marie Ngolo 1\Sm{}.\Pst{}-cheat-\Pst{}\\
%%TRANS1
\trans int. ‘Ngolo betrayed Marie.’
%%TRANS2
}
%%EXEND

    \end{xlist}
\end{exe}
\begin{exe}
    \ex \label{50}
    \begin{xlist}
%%EAX
\ex
%%JUDGEMENT
[]{
%%LABEL
\label{50a}
%%CONTEXT
%%LINE2
\gll
Ng\'{u}ku baa-ntaba kí-má káá-sî?\\
%%LINE3
1.mother 2-goat 7-what 1\Sm{}.\Pst{}-do.\Pst{}\\
%%TRANS1
\trans ‘What did mother do to the goats?’
%%TRANS2
}
%%EXEND

%%EAX
\ex
%%JUDGEMENT
[]{
%%LABEL
\label{50b}
%%CONTEXT
%%LINE2
\gll
Ng\'{u}ku áá-dzwí baa-ntaba.\\
%%LINE3
1.mother 1\Sm{}.\Pst{}-kill.\Pst{} 2-goat\\
%%TRANS1
\trans ‘Mother killed the goats.’
%%TRANS2
}
%%EXEND

%%EAX
\ex
%%JUDGEMENT
[]{
%%LABEL
\label{50c}
%%CONTEXT
%%LINE2
\gll
Ng\'{u}ku \textbf{b\'{a}}a-ntaba káá-dzwí.\\
%%LINE3
1.mother 2-goat 1\Sm{}.\Pst{}-kill.\Pst{}\\
%%TRANS1
\trans ‘Mother killed the goats.’
%%TRANS2
}
%%EXEND

%%EAX
\ex
%%JUDGEMENT
[\#]{
%%LABEL
\label{50d}
%%CONTEXT
%%LINE2
\gll
Baa-ntaba ng\'{u}ku áá-dzwí.\\
%%LINE3
2-goat 1.mother 1\Sm{}.\Pst{}-kill.\Pst{}\\
%%TRANS1
\trans int. ‘Mother killed the goats.’
%%TRANS2
}
%%EXEND

    \end{xlist}
\end{exe}
Here it is somehow problematic to explain why SOV is applicable to express verb focus; if we hypothesise the IBV to be a dedicated focus position from which the focus can project up to the whole VP, we still cannot account for why the focus on the IBV can be ``transferred" to the verb. According to the focus projection hypothesis above, the object being placed in the IBV position is consistent with the whole VP being in focus, since the object is counted as within the scope of the VP focus. Here we may wonder whether the preposed objects in \xref{49c} and \xref{50c} indeed occupy the IBV position or they are just fronted as some topical elements. The most obvious evidence that they are placed in the IBV position rather than some higher positions lies in the H tone on the nominal prefix in \xref{50c}. I will present in \sectref{teke:sec:5.2} that this H tone marking only occurs when the preposed element is in IBV. So here we can confirm that SOV is indeed felicitous to express verb focus with the IBV position being occupied. In \xref{49d} and \xref{50d} we see that the answers become infelicitous when the subject is placed in IBV, this may be accounted for by an economy principle. Since verb focus here must involve something to be placed in the IBV position and both the subject and object in this question are topical, it may be easier to just place the object to the IBV position, rather than place the subject in the IBV while also topicalising the object.

Let us consider some more examples of verb focus. In the answers to the question in \xref{51a}, some other actions taken on the object `pig' are introduced in addition to just `washing' it, so the verb `to kill' in \xxref{51b}{51d} itself is focused. We see that \xref{51b} is felicitous with SVO, while \xref{51c} with SOV is infelicitous here with the additive particle \textit{hé} and it only implies that the grandmother must have killed other animals beforehand, thus the focus can only be on the object rather than the verb. SOV in \xref{51d} without the additive particle is felicitous to correctively focus the truth value of the verb, and in \xref{51e} SOV with the additive particle is felicitous when the whole VP is focused as an additive action that is not related to the pig.
\begin{exe}
     \ex \label{51}
    \begin{xlist}
%%EAX
\ex
%%JUDGEMENT
[]{
%%LABEL
\label{51a}
%%CONTEXT
%%LINE2
\gll
Nkaaká áá-swaak-í mu-féme?\\
%%LINE3
1.grandmother 1\Sm{}.\Pst{}-wash-\Pst{} 4-pig\\
%%TRANS1
\trans ‘Did grandmother wash the pig?’
%%TRANS2
}
%%EXEND

%%EAX
\ex
%%JUDGEMENT
[]{
%%LABEL
\label{51b}
%%CONTEXT
%%LINE2
\gll
Ndé áá-dzwí hé mú-féme.\\
%%LINE3
1.\Pro{} 1\Sm{}.\Pst{}-kill.\Pst{} also 4-pig\\
%%TRANS1
\trans ‘She also \textit{killed} the pig.’
%%TRANS2
}
%%EXEND

%%EAX
\ex
%%JUDGEMENT
[\#]{
%%LABEL
\label{51c}
%%CONTEXT
%%LINE2
\gll
Ndé mú-fémé (hé) káá-dzwí (hé).\\
%%LINE3
1.\Pro{} 4-pig also 1\Sm{}.\Pst{}-kill.\Pst{} {\db}also\\
%%TRANS1
\trans int. ‘She also \textit{killed} the pig.’\\‘She also killed the \textit{pig}.’
%%TRANS2
}
%%EXEND

%%EAX
\ex
%%JUDGEMENT
[]{
%%LABEL
\label{51d}
%%CONTEXT
%%LINE2
\gll
Ndé mú-fémé káá-dzwí.\\
%%LINE3
1.\Pro{} 4-pig 1\Sm{}.\Pst{}-kill.\Pst{}\\
%%TRANS1
\trans ‘She \textit{killed} the pig.’
%%TRANS2
}
%%EXEND

%%EAX
\ex
%%JUDGEMENT
[]{
%%LABEL
\label{51e}
%%CONTEXT
%%LINE2
\gll
Ndé báa-ntsúú hé káá-ká-i.\\
%%LINE3
1.\Pro{} 2-chicken also 1\Sm{}.\Pst{}-grill-\Pst{}\\
%%TRANS1
\trans  ‘She also \textit{grilled the chicken}.’
%%TRANS2
}
%%EXEND

    \end{xlist}
\end{exe}
The infelicity in \xref{51c} above is unexpected given that we have already seen examples above in which the SOV word order can be used to express verb focus. I suggest that this infelicity is due to the presence of the additive particle \textit{hé} which is always associated with the focal element in a sentence. This particle may disambiguate the nucleus of the focus from the domain to which it may project up, thus in \xref{51c} it is more intuitive for the speakers to interpret the focus on the object only. In the absence of this additive particle, the SOV word order becomes a possible way of expressing verb focus as in \xref{51d}. We also see that in \xref{51e} the presence of \textit{hé} does not prevent the SOV word order from expressing VP focus, this may be explained by the fact that the whole VP in \xref{51e} is new, so the VP focus reading can be rescued from the intervention of the additive particle, thus it can be an appropriate answer to \xref{51a}.

From above I have shown that when the IBV position is occupied by an object DP, it can be used to express VP focus and verb focus. Next I present another strategy for expressing predicate-centered focus, which also involves the use of the IBV focus position, namely the predicate doubling construction.
\subsubsection{Predicate doubling}\label{teke:sec:3.2.2}
Predicate doubling was first documented by \citet[121]{Meeussen1967} as the ``advance verb construction" that can express polarity focus, intensity and concession. In many other Bantu languages, predicate doubling is a common strategy to express state-of-affairs focus and polarity focus on the verb, and is reported to be situated in different stages in the grammaticalisation path to the progressive and future tense \citep{GüldemannEtAl2010, GüldemannEtAl2014, Morimoto2016}. In some neighbouring languages of Teke, such as in the Kikongo group of Zone H and other Zone B languages, some of which also favour the IBV focus position, the predicate doubling construction is also well attested expressing verb focus and polarity focus, as well as progressive and future tense \citep{Hadermann1996, DeKind2014, DeKindEtAl2015, GüldemannFiedler2022}. Some examples from these languages illustrate the phenomenon in \xxref{52}{55} below. 
\begin{exe}
%%EAX
\ex
%%JUDGEMENT
%%LABEL
\label{52}
%%CONTEXT
%%LINE2
\gll
Ku-tá:nga ndyeká-tá:nga.\\
%%LINE3
\Inf{}-read 1\Sg{}:\Fut{}-read\\
%%TRANS1
\trans ‘I will \textit{read}.’     \jambox*{[Suundi H31b, \citealt[161]{Hadermann1996}] (verb focus)}
%%TRANS2
%%EXEND

\end{exe}
\begin{exe}
%%EAX
\ex
%%JUDGEMENT
%%LABEL
\label{53}
%%CONTEXT
%%LINE2
\gll
Mona mbwene N-kenda za zula ki-ame kina.\\
%%LINE3
\Inf{}.see 1\Sg{}.see.\Perf{} 10-affliction 10.\Gen{} 7.people 7-1\Sg{}.\Poss{} 7.\Dem{}\\
%%TRANS1
\trans ‘I have surely seen the affliction of that people of mine there.’     \jambox*{[Ndibu H16, \citealt[12]{DeKindEtAl2015}] (polarity focus)}
%%TRANS2
%%EXEND

\end{exe}
\begin{exe}
%%EAX
\ex
%%JUDGEMENT
%%LABEL
\label{54}
%%CONTEXT
%%LINE2
\gll
Ba-ka:s\'{\textschwa} bá-ná:, vád\'{\textschwa} b\^{a}:vád{\textschwa} pénda.\\
%%LINE3
2-woman 2-\Dem{} \Inf{}.cultivate 2-cultivate groundnut\\
%%TRANS1
\trans ‘These women, they are cultivating groundnuts.’     \jambox*{[Nzebi B52, \citealt[162]{Hadermann1996}] (progressive)}
%%TRANS2
%%EXEND

\end{exe}
\begin{exe}
%%EAX
\ex
%%JUDGEMENT
%%LABEL
\label{55}
%%CONTEXT
%%LINE2
\gll
Vuumbuka yi-vuumbuka.\\
%%LINE3
\Inf{}-dress 1\Sg{}.\Sm{}-dress \\
%%TRANS1
\trans ‘I'll dress myself.’     \jambox*{[Yaka H33, \citealt[36]{DeKindEtAl2015}] (future)}
%%TRANS2
%%EXEND

\end{exe}
In this subsection I introduce the predicate doubling construction in Kukuya. In Kukuya, the predicate doubling construction is mainly attested as IBV doubling, while topic doubling is judged to be infelicitous and cleft doubling to be quite marginal (see the definition of these predicate doubling types in the introductory chapter of this volume). In \xref{56}, to exclusively focus the lexical value of a verb while excluding some alternatives, we see that SOV can be appropriately used in \xref{56b} with the exclusive focus particle \textit{wúna} preceding the preposed object, while to place the particle immediately in front of the verb in either SVO or SOV is judged to be ungrammatical as shown in \xref{56c} and \xref{56d}. It seems that the exclusive particle can only modify nominal elements or a VP but not a bare verb. In \xref{56e} we see that verb focus can also be expressed by placing an infinitive form of the verb immediately before the inflected verb.
\begin{exe}
    \ex \label{56}
    \begin{xlist}
     \judgewidth{??}
%%EAX
\ex
%%JUDGEMENT
[]{
%%LABEL
\label{56a}
%%CONTEXT
%%LINE2
\gll
Ngúku áá-télek-i bu-ká áá-dzí?\\
%%LINE3
1.mother 1\Sm{}.\Pst{}-prepare-\Pst{} 14-cassava 1\Sm{}.\Pst{}-eat.\Pst{}\\
%%TRANS1
\trans ‘Did mother prepare and eat the cassava?’
%%TRANS2
}
%%EXEND

%%EAX
\ex
%%JUDGEMENT
[]{
%%LABEL
\label{56b}
%%CONTEXT
%%LINE2
\gll
Ndé wúna bu-ká káá-télek-i.\\
%%LINE3
1.\Pro{} only 14-cassava 1\Sm{}.\Pst{}-prepare-\Pst{}\\
%%TRANS1
\trans ‘She only \textit{prepared} the cassava.’
%%TRANS2
}
%%EXEND

%%EAX
\ex
%%JUDGEMENT
[??]{
%%LABEL
\label{56c}
%%CONTEXT
%%LINE2
\gll
Ndé wúna áá-télek-i bu-ká.\\
%%LINE3
1.\Pro{} only 1\Sm{}.\Pst{}-prepare-\Pst{} 14-cassava\\
%%TRANS1
\trans int. ‘She only \textit{prepared} the cassava.’
%%TRANS2
}
%%EXEND

%%EAX
\ex
%%JUDGEMENT
[*]{
%%LABEL
\label{56d}
%%CONTEXT
%%LINE2
\gll
Ndé bu-ká wúna káá-télek-i.\\
%%LINE3
1.\Pro{} 14-cassava only 1\Sm{}.\Pst{}-prepare-\Pst{}\\
%%TRANS1
\trans int. ‘She only \textit{prepared} the cassava.’
%%TRANS2
}
%%EXEND

%%EAX
\ex
%%JUDGEMENT
[]{
%%LABEL
\label{56e}
%%CONTEXT
%%LINE2
\gll
Ndé bu-ká wúna \textbf{ki-téléké} káá-télek-i.\\
%%LINE3
1.\Pro{} 14-cassava only \Inf{}-prepare 1\Sm{}.\Pst{}-prepare-\Pst{}\\
%%TRANS1
\trans ‘She only \textit{prepared} the cassava.’
%%TRANS2
}
%%EXEND

    \end{xlist}
\end{exe}
One additional example of predicate doubling expressing verb focus in Kukuya is given in \xref{57}. There is an important interpretational difference between the use of SOV and predicate doubling in expressing verb focus: while \xref{56b}, \xref{56e} and \xref{57b} all express exclusive focus on the verb, \xref{56b} with SOV indicates that the event is completed and the mother only prepared the cassava but does not need to go on making it, while \xref{56e} and \xref{57b} imply that the event is still continuing and there must be other things that need to be done with the cassava and the goats.
\begin{exe}
    \ex \label{57}
    \begin{xlist}
%%EAX
\ex
%%JUDGEMENT
%%LABEL
\label{57a}
%%CONTEXT
%%LINE2
\gll
Maamá áá-dzwî baa-ntabá áá-ték-i.\\
%%LINE3
1.mother 1\Sm{}.\Pst{}-kill.\Pst{} 2-goat 1\Sm{}.\Pst{}-sell-\Pst{}\\
%%TRANS1
\trans ‘Mother killed the goats (and) sold (them).’
%%TRANS2
%%EXEND

%%EAX
\ex
%%JUDGEMENT
%%LABEL
\label{57b}
%%CONTEXT
%%LINE2
\gll
Ambú, ndé bó wúna \textbf{ki-téké} \textbf{káa-ték-i}.\\
%%LINE3
no 1.\Pro{} 2.\Pro{} only \Inf{}-sell 1\Sm{}.\Pst{}-sell-\Pst{}\\
%%TRANS1
\trans ‘No, she only \textit{sold} them.’
%%TRANS2
%%EXEND

    \end{xlist}
\end{exe} 
Polarity focus on the verb can be expressed neither by predicate doubling nor by SOV in Kukuya. In \xref{58} and \xref{59} we see that to correct a negative truth value on the verb, there is no other construction than the canonical SVO, and the speakers tend to put some intonational emphasis on the verb to express the polarity focus. SOV in \xref{58c} and \xref{59c} is infelicitous here, while it can actually express focus on the object or the VP or the lexical value of the verb. The predicate doubling in \xref{58d} and \xref{59d} is also infelicitous and implies that there are other actions that need to be done with the oranges, expressing verb focus. We see that both the SOV order and predicate doubling can trigger alternatives either on the object or on the verb and imply a contrast with other actions or tasks that remain to be done.
\begin{exe}
     \ex \label{58}
    \begin{xlist}
%%EAX
\ex
%%JUDGEMENT
[]{
%%LABEL
\label{58a}
%%CONTEXT
%%LINE2
\gll
Taará ka-káá-kí ma-láala ni?\\
%%LINE3
1.father \Neg{}-1\Sm{}.\Pst{}-pick.\Pst{} 6-orange \Neg{}\\
%%TRANS1
\trans ‘Did father not pick the oranges?’
%%TRANS2
}
%%EXEND

%%EAX
\ex
%%JUDGEMENT
[]{
%%LABEL
\label{58b}
%%CONTEXT
%%LINE2
\gll
Ndé áá-kí ma-láala.\\
%%LINE3
1.\Pro{} 1\Sm{}.\Pst{}-pick.\Pst{} 6-orange\\
%%TRANS1
\trans ‘He \textit{did} pick the oranges.’
%%TRANS2
}
%%EXEND

%%EAX
\ex
%%JUDGEMENT
[\#]{
%%LABEL
\label{58c}
%%CONTEXT
%%LINE2
\gll
Ndé má-láálá káá-kí.\\
%%LINE3
1.\Pro{} 6-orange 1\Sm{}.\Pst{}-pick.\Pst{}\\
%%TRANS1
\trans int. ‘He \textit{did} pick the oranges.’
%%TRANS2
}
%%EXEND

%%EAX
\ex
%%JUDGEMENT
[\#]{
%%LABEL
\label{58d}
%%CONTEXT
%%LINE2
\gll
Ndé ma-láala \textbf{kí-ká} káá-kí.\\
%%LINE3
1.\Pro{} 6-orange \Inf{}-pick 1\Sm{}.\Pst{}-pick.\Pst{}\\
%%TRANS1
\trans int. ‘He \textit{did} pick the oranges.’
%%TRANS2
}
%%EXEND

    \end{xlist}
\end{exe}
\begin{exe}
     \ex \label{59}
    \begin{xlist}
%%EAX
\ex
%%JUDGEMENT
[]{
%%LABEL
\label{59a}
%%CONTEXT
%%LINE2
\gll
Ndé ka-káá-bvúúr-í we mi-pará ni?\\
%%LINE3
1.\Pro{} \Neg{}-1\Sm{}.\Pst{}-return-\Pst{} 2\Sg{}.\Pro{} 4-money \Neg{}\\
%%TRANS1
\trans ‘Did s/he not return you the money?’
%%TRANS2
}
%%EXEND

%%EAX
\ex
%%JUDGEMENT
[]{
%%LABEL
\label{59b}
%%CONTEXT
%%LINE2
\gll
Ndé áá-bvúur-i me mi-pará.\\
%%LINE3
1.\Pro{} 1\Sm{}.\Pst{}-return-\Pst{} 1\Sg{}.\Pro{} 4-money\\
%%TRANS1
\trans ‘S/He \textit{did} return me the money.’
%%TRANS2
}
%%EXEND

%%EAX
\ex
%%JUDGEMENT
[\#]{
%%LABEL
\label{59c}
%%CONTEXT
%%LINE2
\gll
Ndé me mí-para káá-bvúur-i.\\
%%LINE3
1.\Pro{} 1\Sg{}.\Pro{} 4-money 1\Sm{}.\Pst{}-return-\Pst{}\\
%%TRANS1
\trans int. ‘He \textit{did} return me the money.’
%%TRANS2
}
%%EXEND

%%EAX
\ex
%%JUDGEMENT
[\#]{
%%LABEL
\label{59d}
%%CONTEXT
%%LINE2
\gll
Ndé me mi-pará \textbf{kí-bvúúrá} káá-bvúur-i.\\
%%LINE3
1.\Pro{} 1\Sg{}.\Pro{} 4-money \Inf{}-return 1\Sm{}.\Pst{}-return-\Pst{}\\
%%TRANS1
\trans int. ‘He \textit{did} return me the money.’
%%TRANS2
}
%%EXEND

    \end{xlist}
\end{exe}
The predicate doubling construction that expresses verb focus in Kukuya looks quite like the IBV focus construction that encodes narrow focus on the preposed DP, and predicate doubling in this case is just a particular realisation of the IBV focus, in which the predicate is doubled as an infinitive form and is focused in the IBV position. In this sense, the predicate doubling and the SOV/OSV word order are actually the same structure that places focus in the IBV position, which is also consistent with the fact that infinitives are also DPs in Kukuya and most other Bantu languages. If this is true, we may expect the fronted infinitive and a preverbal question word to be in complementary distribution as they should compete for the unique IBV position, and this is borne out as the ungrammaticality in \xref{60} and \xref{61}, which also shows that there is only one preverbal focus site in this language. In these examples the predicate doubling is intended to be used for expressing a \textit{progressive} meaning, which I will introduce shortly.
\begin{exe}
%%EAX
\ex
%%JUDGEMENT
[*]{
%%LABEL
\label{60}
%%CONTEXT
%%LINE2
\gll
Ndé \textbf{kí-má} \textbf{kí-dzá} kâ-dzá?\\
%%LINE3
1.\Pro{} 7-what \Inf{}-eat 1.\Sm{}.\Prs{}-eat\\
%%TRANS1
\trans int. ‘What is he/she eating?’
%%TRANS2
}
%%EXEND

\end{exe}
\begin{exe}
%%EAX
\ex
%%JUDGEMENT
[*]{
%%LABEL
\label{61}
%%CONTEXT
%%LINE2
\gll
Ná \textbf{kí-tsúka} kâ-tsúka?\\
%%LINE3
1.who \Inf{}-speak 1.\Sm{}.\Prs{}-speak\\
%%TRANS1
\trans int. ‘Who is talking?’
%%TRANS2
}
%%EXEND

\end{exe}
We have seen above that the IBV position is associated with argument and adjunct focus, as well as VP focus and verb focus. It is not clear here whether the infinitive in the predicate doubling construction should be viewed as an argument of the verb, if so, the predicate doubling in analogous to term focus on an argument DP. In fact, predicate doubling and term focus in the IBV position have some important interpretational similarities: predicate doubling usually implies the potential occurrence of other actions, while term focus in IBV also hints that some alternatives should be available for the proposition. I will discuss more on these interpretational properties in \sectref{teke:sec:3.4}.

Similar to many other Bantu languages, predicate doubling in Kukuya can express progressive aspect. In examples \xref{62} and \xref{63} the fronted infinitive expresses a neutral progressive meaning without focusing on the verb itself. Verb focus and progressive reading are often said to have a close semantic and pragmatic relation and the progressive is considered to be an inherently focused verb category in which the ``ongoing nature of the event described by the verb" constitutes the focus domain of the sentence \citep{HymanWatters1984, Güldemann2003, DeKind2014, DeKindEtAl2015}. The predicate doubling with progressive reading is sometimes ambiguous and can only be distinguished from PCF through the pragmatic context. Example \xref{62} can be a felicitous corrective response to focus on the progressive aspect expressing TAM focus, while predicate doubling in \xref{63} is used outside the PCF context. In Kukuya there is a dedicated aspect marker \mbox{-\textit{k\^{a}}-} that can mark habitual as well as progressive aspect without the fronting of an infinitive verb, so the predicate doubling is not the only way of expressing progressive in Kukuya.
\begin{exe}
%%EAX
\ex
%%JUDGEMENT
%%LABEL
\label{62}
%%CONTEXT
 (Have they already eaten?)\\
%%LINE2
\gll
Bó kí-dzá bá-\textbf{kâ}-dzá.\\
%%LINE3
2.\Pro{} \Inf{}-eat 2\Sm{}-\Prog{}-eat\\
%%TRANS1
\trans ‘They are eating.’
%%TRANS2
%%EXEND

\end{exe}
\begin{exe}
%%EAX
\ex
%%JUDGEMENT
%%LABEL
\label{63}
%%CONTEXT
%%LINE2
\gll
Mwáana wu-kái wu-kí-kwî ngúku á-yiká \textbf{kí-líla} kâ-líl-a.\\
%%LINE3
1.child 1-female 1\Rel{}-7\Sm{}-die.\Pst{} 1.mother 1\Sm{}-\Impf{}  \Inf{}-cry 1\Sm{}.\Prog{}-cry-\Fv{}\\
%%TRANS1
\trans ‘The girl whose mother died is crying.’
%%TRANS2
%%EXEND

\end{exe}
In Kukuya grammar, the expression of immediate future tense also involves the SOV order, as in \xref{64}. The predicate doubling construction in Kukuya can also have the immediate future reading, as shown in \xref{65}. 
\begin{exe}
%%EAX
\ex
%%JUDGEMENT
%%LABEL
\label{64}
%%CONTEXT
%%LINE2
\gll
Bó má-ko báa-fúum-a.\\
%%LINE3
2.\Pro{}  6-banana 2\Sm{}.\Fut{}-buy-\Fv{}\\
%%TRANS1
\trans ‘They'll buy some bananas.’
%%TRANS2
%%EXEND

\end{exe}
\begin{exe}
     \ex \label{65}
    \begin{xlist}
%%EAX
\ex
%%JUDGEMENT
%%LABEL
\label{65a}
%%CONTEXT
%%LINE2
\gll
We ka-á-bvúúr-í ndé mi-pará ni?\\
%%LINE3
2\Sg{}.\Pro{} \Neg{}-2\Sg{}.\Pst{}-return-\Pst{} 1.\Pro{} 4-money \Neg{}\\
%%TRANS1
\trans ‘You did not return her the money?’
%%TRANS2
%%EXEND

%%EAX
\ex
%%JUDGEMENT
%%LABEL
\label{65b}
%%CONTEXT
%%LINE2
\gll
Me mi-pará \textbf{kí-bvúúrá} kâ-n-bvúur-a.\\
%%LINE3
1\Sg{}.\Pro{} 4-money \Inf{}-return \Prs{}-1\Sg{}.\Sm{}-return-\Fv{}\\
%%TRANS1
\trans ‘I am (surely) going to return the money.’
%%TRANS2
%%EXEND

    \end{xlist}
\end{exe}
The response in \xref{65b} has a verum reading, meaning that the speaker will definitely return the money and does not imply that there are other things to be done with the object `money', which differs from the interpretation in example \xref{59d} above. The contrast between \xref{59d} and \xref{65b} is that, in \xref{59d} the alternative could be `borrow again' the money in addition to just returning it, while in \xref{65b} the alternative is `not to return' the money as opposed to returning it. We see that when predicate doubling has an immediate future reading, it can express polarity focus/verum, which may be due to the SOV word order grammaticalising to express certain tense, thus becoming pragmatically equal to the canonical word order; for the predicate doubling in other tenses, it cannot be used to express polarity focus/verum.

From the above presentation on the expressions of different types of predicate \mbox{(-centered)} focus, we see that in Kukuya VP focus (predicate focus) can be expressed by SVO as well as SOV. The use of the IBV focus position to express VP focus can be explained by the focus projection account. Verb focus (state-of-affairs focus) can also be realised via SVO and SOV, while OSV cannot express verb focus. It remains to be investigated why the IBV focus position is also involved in expressing verb focus. The predicate doubling construction is used mostly to express verb focus and usually triggers alternatives to the verb, while polarity focus is commonly expressed by the canonical SVO word order.
\subsection{IBV as a dedicated focus position}\label{teke:sec:3.3}
So far we have encountered and discussed many examples which suggest that the IBV position is always associated to some type of focus. In this subsection I will investigate some intrinsic properties of the IBV position. First I will provide more tests on whether the IBV position is really a dedicated focus position. Then I will discuss the interpretational differences between the IBV focus strategy and \textit{in situ} focus that was introduced in the previous section, showing that the IBV position is reserved for expressing identificational focus, while the \textit{in situ} focus strategy seems to be more frequently used to express assertive focus.  

Many Bantu languages have been reported to have a dedicated focus position, most of which are the so-called immediate-after-verb (IAV) position that are commonly attested in languages such as Aghem \citep{Watters1979, HymanPolinsky2010}, Bemba \citep{CostaKula2008}, Matengo \citep{Yoneda2011}, Makhuwa \citep{vanderWal2009a}, and Zulu \citep{Buell2009}. The immediate-before-verb (IBV) focus position is much more rarely attested only in some West-Coastal Bantu languages (WCB), which has already been described in detail for Mbuun (B87, \citealt{BostoenMundeke2011, BostoenMundeke2012}), Nsong (B85, \citealt{KoniMuluwaBostoen2019}) and in the Kikongo cluster \citep{Hadermann1996, DeKind2014, DeKindEtAl2015}. Here I provide some more evidence to show that the IBV in Kukuya is indeed a dedicated focus position.

At this moment, we consider first non-subject elements (the IBV and the canonical subject position will be disentangled later). We see that if some element is placed in the IBV position, it must be focal or at least within the scope of focus, while elements in other positions cannot be focal at the same time. This is illustrated in \xref{66} and \xref{67}. In \xref{66b} we see that a question word cannot co-occur with another element being placed in the IBV position, the ungrammaticality can only be explained by the focal status of the adverb in IBV and the generalisation that multiple foci are not allowed; \xref{66c} is not a felicitous answer to \xref{66a}, as what is placed in the IBV position is a manner adverb but not the object which is the target of the question; \xref{66c} can only be an appropriate answer to the question `\textit{How} did the person eat the cassava', which indicates that the adverb in the IBV must be focal. Similarly in \xref{67}, only \xref{67b} can be a felicitous answer to \xref{67a} while \xref{67c} can only be the answer to the question that asks for the location. From these examples we see that if there are multiple preverbal elements (in an affirmative sentence), the IBV slot must be occupied by a focal element, while other elements in the sentence are prohibited to be focused.
\begin{exe} 
    \ex \label{66}
    \begin{xlist}
%%EAX
\ex
%%JUDGEMENT
[]{
%%LABEL
\label{66a}
%%CONTEXT
%%LINE2
\gll
Mbuurú \textbf{kí-má} káá-dzí tswáatswáa?\\
%%LINE3
1.person 7-what 1\Sm{}.\Pst{}-eat.\Pst{} fast\\
%%TRANS1
\trans ‘What did the person eat quickly?’
%%TRANS2
}
%%EXEND

%%EAX
\ex
%%JUDGEMENT
[*]{
%%LABEL
\label{66b}
%%CONTEXT
%%LINE2
\gll
Mbuurú kí-má \textbf{tswáatswáa} káá-dzí?\\
%%LINE3
1.person 7-what fast 1\Sm{}.\Pst{}-eat.\Pst{}\\
%%TRANS1
\trans int. ‘What did the person eat quickly?’
%%TRANS2
}
%%EXEND

%%EAX
\ex
%%JUDGEMENT
[\#]{
%%LABEL
\label{66c}
%%CONTEXT
%%LINE2
\gll
Mbuurú bu-ka \textbf{tswáatswáa} káá-dzí.\\
%%LINE3
1.person 14-cassava fast 1\Sm{}.\Pst{}-eat.\Pst{}\\
%%TRANS1
\trans ‘The person ate the cassava \textit{quickly}.’
%%TRANS2
}
%%EXEND

    \end{xlist}
\end{exe}
\begin{exe} 
    \ex \label{67}
    \begin{xlist}
%%EAX
\ex
%%JUDGEMENT
[]{
%%LABEL
\label{67a}
%%CONTEXT
%%LINE2
\gll
We \textbf{ná} á-mún-i ku dzándu?\\
%%LINE3
2\Sg{}.\Pro{} who 2\Sg{}.\Pst{}-see-\Pst{} 17.\Loc{} 5.market\\
%%TRANS1
\trans ‘Who did you see at the market?’
%%TRANS2
}
%%EXEND

%%EAX
\ex
%%JUDGEMENT
[]{
%%LABEL
\label{67b}
%%CONTEXT
%%LINE2
\gll
Me \textbf{Gilbert} á-mún-i ku dzándu.\\
%%LINE3
1\Sg{}.\Pro{} Gilbert 1\Sg{}.\Pst{}-see-\Pst{} 17.\Loc{} 5.market\\
%%TRANS1
\trans ‘I saw Gilbert at the market.’
%%TRANS2
}
%%EXEND

%%EAX
\ex
%%JUDGEMENT
[\#]{
%%LABEL
\label{67c}
%%CONTEXT
%%LINE2
\gll
Me Gilbert \textbf{ku} \textbf{dzándu} á-mún-i.\\
%%LINE3
1\Sg{}.\Pro{} Gilbert 17.\Loc{} 5.market 1\Sg{}.\Pst{}-see-\Pst{}\\
%%TRANS1
\trans ‘I saw Gilbert \textit{at the marker}.’
%%TRANS2
}
%%EXEND

    \end{xlist}
\end{exe}
Neither SOV nor OSV can be used to answer the question such as `What happened' as in \xref{68a} which requires a thetic answer in which the whole utterance provides the new information, thus no topic or focus is subdivided in the sentence (\citealt{Kuroda1972, Sasse1987, Sasse1996}; also see \sectref{teke:sec:2.3}). Here we see that only SVO in \xref{68b} can be felicitous. The answers in \xref{68c} and \xref{68d} are both inappropriate here, as there must be some focal reading triggered by the IBV position being occupied, namely the object `child' in \xref{68c} and the subject `father' in \xref{68d}, thus they are both incompatible with the thetic requirement.
\begin{exe} 
    \ex \label{68}
    \begin{xlist}
%%EAX
\ex
%%JUDGEMENT
[]{
%%LABEL
\label{68a}
%%CONTEXT
%%LINE2
\gll
Me a-n-yúk-i nkelé ku mbali, kí-má kí-s\^{i}?\\
%%LINE3
1\Sg{}.\Pro{} \Pst{}-1\Sg{}.\Sm{}-hear-\Fv{} 9.noise 17.\Loc{} 9.outside 7-what 7\Sm{}.\Pst{}-do.\Pst{}\\
%%TRANS1
\trans ‘I heard some noise outside, what happened?’
%%TRANS2
}
%%EXEND

%%EAX
\ex
%%JUDGEMENT
[]{
%%LABEL
\label{68b}
%%CONTEXT
%%LINE2
\gll
Taará áá-béer-i mwáana.\\
%%LINE3
1.father 1\Sm{}.\Pst{}-beat-\Pst{} 1.child\\
%%TRANS1
\trans ‘Father beat the child.’
%%TRANS2
}
%%EXEND

%%EAX
\ex
%%JUDGEMENT
[\#]{
%%LABEL
\label{68c}
%%CONTEXT
%%LINE2
\gll
Taará mwááná káá-béer-i.\\
%%LINE3
1.father 1.child 1\Sm{}.\Pst{}-beat-\Pst{}\\
%%TRANS1
\trans ‘Father beat the \textit{child}.’
%%TRANS2
}
%%EXEND

%%EAX
\ex
%%JUDGEMENT
[\#]{
%%LABEL
\label{68d}
%%CONTEXT
%%LINE2
\gll
Mwáána taará áá-béer-i.\\
%%LINE3
1.child 1.father 1\Sm{}.\Pst{}-beat-\Pst{}\\
%%TRANS1
\trans ‘\textit{Father} beat the child.’
%%TRANS2
}
%%EXEND

    \end{xlist}
\end{exe}
Idiom tests can also help to justify that the IBV position is indeed associated with focus function \citep{vanderWal2016, vanderWal2021a}. In idiom sentences, the idiomatic reading arises as a whole chunk and is thus considered to be non-compositional. We would predict that any part of an idiom sentence cannot be focused, since no expressions in an idiom refers to something that is accessible in the reality, thus no alternatives can be triggered for focus. Examples \xxref{69}{71} illustrate several idioms in Kukuya, and all these idiom sentences surface as SVO, which is further evidence for SVO as the canonical order. Crucially, we find that when the word order is shifted to SOV, the sentence is still grammatical but the idiomatic reading is not retained, and the sentence can only have the literal meaning. These idiom tests show that the formation of SOV must have involved some discourse-related operations, namely the IBV element must be focal, as the translations indicate.
\begin{exe}
     \ex \label{69}
    \begin{xlist}
%%EAX
\ex
%%JUDGEMENT
%%LABEL
\label{69a}
%%CONTEXT
%%LINE2
\gll
Ndé áá-tín-i ko li-búi.\\
%%LINE3
1.\Pro{} 1\Sm{}.\Pst{}-pick-\Pst{} 5.banana 5-immature\\
%%TRANS1
\trans ‘S/He had a sexual relation with a child.’\\
%%TRANS2
    lit.: `S/He picked the unripe banana.'
%%EXEND

%%EAX
\ex
%%JUDGEMENT
%%LABEL
\label{69b}
%%CONTEXT
%%LINE2
\gll
Ndé ko li-búi káá-tín-i.\\
%%LINE3
1.\Pro{} 5.banana 5-immature 1\Sm{}.\Pst{}-pick-\Pst{}\\
%%TRANS1
\trans *‘S/He had a sexual relation with a child.’\\
%%TRANS2
    `S/He picked the unripe banana.'
%%EXEND

    \end{xlist}
\end{exe}
\begin{exe}
     \ex \label{70}
    \begin{xlist}
%%EAX
\ex
%%JUDGEMENT
%%LABEL
\label{70a}
%%CONTEXT
%%LINE2
\gll
Me a-n-dzw\^{i} ntaalí mu kíí.\\
%%LINE3
1\Sg{}.\Pro{} \Pst{}-1\Sg{}.\Sm{}-pick.\Pst{} 1.snake 18.\Loc{} 7.pipe\\
%%TRANS1
\trans ‘I have lost all.’\\
%%TRANS2
    lit.: `I killed a snake with a pipe.'
%%EXEND

%%EAX
\ex
%%JUDGEMENT
%%LABEL
\label{70b}
%%CONTEXT
%%LINE2
\gll
Me ntaalí mu kíí n-dzw\^{i}.\\
%%LINE3
1\Sg{}.\Pro{} 1.snake 18.\Loc{} 7.pipe \Pst{}-1\Sg{}.\Sm{}-pick.\Pst{}\\
%%TRANS1
\trans *‘I have lost all.’\\
%%TRANS2
    `I killed a snake with a \textit{pipe}.'
%%EXEND

    \end{xlist}
\end{exe}
\begin{exe}
     \ex \label{71}
    \begin{xlist}
%%EAX
\ex
%%JUDGEMENT
%%LABEL
\label{71a}
%%CONTEXT
%%LINE2
\gll
Maa-nkala máá-dzí mbúlu.\\
%%LINE3
6-charcoal 6\Sm{}.\Pst{}-eat.\Pst{} 9.blanket\\
%%TRANS1
\trans ‘The problem becomes burning (rather than coldness when you use too much charcoal).’\\
%%TRANS2
%%EXEND
    lit.: `The coal is eating the blanket.'
%%EAX
\ex
%%JUDGEMENT
%%LABEL
\label{71b}
%%CONTEXT
%%LINE2
\gll
Maa-nkala mbúlú máá-dzí.\\
%%LINE3
6-charcoal 9.blanket 6\Sm{}.\Pst{}-eat.\Pst{}\\
%%TRANS1
\trans *‘The problem becomes burning (rather than coldness when you use too much charcoal).’\\
%%TRANS2
%%EXEND

    `The coal is eating the \textit{blanket}.'
    \end{xlist}
\end{exe}
So far we have seen that any non-subject constituent that is in IBV position must be interpreted as focused; now I provide examples to illustrate that it is also a dedicated focus position for a subject. Example \xref{72} is partially repeated from \xref{35} above, in which we see that the interrogative subject in \xref{72a} seemingly occupies the same near position as the grammatical subject in canonical word order; however, from \xref{72b} and \xref{72c} we see that the focal and topical subjects are subject to different constraints on their linear position: the focal subject can only occur in the IBV position but cannot be followed by other elements in the preverbal domain as in \xref{72b}, while the topical subject can be followed by other DPs, such as by focused object in the IBV position in \xref{72c}. In other words, the focal subject has an IBV requirement while the topical subject does not, therefore they must stay in different structural positions. Similarly in \xref{73}, we see that the answer to a subject question must be adjacent to the verb as in \xref{73a} and \xref{73c}, and another DP cannot intervene as in \xref{73b}.
\begin{exe}
    \ex \label{72}
    \begin{xlist}
%%EAX
\ex
%%JUDGEMENT
[]{
%%LABEL
\label{72a}
%%CONTEXT
%%LINE2
\gll
\textbf{Ná} áá-t\'{e}k-i mu-ngwa?\\
%%LINE3
1.who 1\Sm{}.\Pst{}-sell-\Pst{} 3-salt\\
%%TRANS1
\trans ‘Who sold the salt?’
%%TRANS2
}
%%EXEND

%%EAX
\ex
%%JUDGEMENT
[*]{
%%LABEL
\label{72b}
%%CONTEXT
%%LINE2
\gll
\textbf{Ná} mú-ngwa káá-t\'{e}k-i?\\
%%LINE3
1.who 3-salt 1\Sm{}.\Pst{}-sell-\Pst{}\\
%%TRANS1
\trans int. ‘Who sold the salt?’
%%TRANS2
}
%%EXEND

%%EAX
\ex
%%JUDGEMENT
[]{
%%LABEL
\label{72c}
%%CONTEXT
 (What did the grandmother sell?) \\
%%LINE2
\gll
\textbf{Nkaaká} mú-ngwa káá-t\'{e}k-i?\\
%%LINE3
1.grandmother 3-salt 1\Sm{}.\Pst{}-sell-\Pst{}\\
%%TRANS1
\trans ‘The grandmother sold some \textit{salt}.’
%%TRANS2
}
%%EXEND

    \end{xlist}
\end{exe}
\begin{exe}
     \ex  (Who brought the dog?)\label{73}
    \begin{xlist}
%%EAX
\ex
%%JUDGEMENT
[]{
%%LABEL
\label{73a}
%%CONTEXT
%%LINE2
\gll
\textbf{Taará} áá-yi-í mvá.\\
%%LINE3
1.father 1\Sm{}.\Pst{}-bring-\Pst{} 1.dog\\
%%TRANS1
\trans ‘\textit{Father} brought the dog.’
%%TRANS2
}
%%EXEND

%%EAX
\ex
%%JUDGEMENT
[\#]{
%%LABEL
\label{73b}
%%CONTEXT
%%LINE2
\gll
Taará mvá káá-yi-í.\\
%%LINE3
1.father 1.dog 1\Sm{}.\Pst{}-bring-\Pst{}\\
%%TRANS1
\trans int.‘\textit{Father} brought the dog.’
%%TRANS2
}
%%EXEND

%%EAX
\ex
%%JUDGEMENT
[]{
%%LABEL
\label{73c}
%%CONTEXT
%%LINE2
\gll
Mvá taará áá-yi-í.\\
%%LINE3
1.dog 1.father 1\Sm{}.\Pst{}-bring-\Pst{}\\
%%TRANS1
\trans ‘\textit{Father} brought the dog.’
%%TRANS2
}
%%EXEND

    \end{xlist}
\end{exe}
In \xref{74a} the subject precedes the negative marker on the verb, and the focus is on the polarity of the sentence rather than on the subject; while \xref{74b} expresses constituent negation and the subject is somehow ``inserted" between the negative marker and the verb, providing evidence that it must be situated in a different position than the subject in \xref{74a}. The interpretation in \xref{74b} is that it is not `father' but someone else that killed the leopard, so the focus is apparently on the subject. From the minimal pair in \xref{74} the canonical subject position and the IBV can be distinguished: in \xref{74a} the subject appears in the canonical subject position, while in \xref{74b} the subject is placed in the IBV position. The position of the negative marker here may also support that the IBV position is indeed ``immediate" before the verb, since when the IBV slot is empty as in \xref{74a}, the negative morpheme is always prefixed to the verb and prosodically phrased together with it. 
\begin{exe}
     \ex \label{74}
    \begin{xlist}
%%EAX
\ex
%%JUDGEMENT
%%LABEL
\label{74a}
%%CONTEXT
%%LINE2
\gll
Ngo \textbf{taará} ka-káá-dzwí ni.\\
%%LINE3
1.leopard 1.father \Neg{}-1\Sm{}.\Pst{}-kill.\Pst{} \Neg{}\\
%%TRANS1
\trans ‘The leopard, father did \textit{not} kill (it).’
%%TRANS2
%%EXEND

%%EAX
\ex
%%JUDGEMENT
%%LABEL
\label{74b}
%%CONTEXT
%%LINE2
\gll
Ngo ka \textbf{taará} áá-dzwí ni.\\
%%LINE3
1.leopard \Neg{} 1.father 1\Sm{}.\Pst{}-kill.\Pst{} \Neg{}\\
%%TRANS1
\trans ‘The leopard was not killed by father (but by someone else).’
%%TRANS2
%%EXEND

    \end{xlist}
\end{exe}
The analysis above has provided strong evidence on the presence of a dedicated focus position in Kukuya, i.e. everything that is in this position is focused, and this position is located immediately left-adjacent to the verb. I have also shown that the subject is focused in the IBV position which is distinct from its canonical preverbal position. When this IBV position is filled, the sentence must have undergone some discourse-related operations for information packaging, in most cases some argument or adjunct gets focused. Recall that in \sectref{teke:sec:2} we have seen that an element such as a postverbal object can also be focused in its canonical postverbal position, next I will discuss the distinction between IBV focus and non-IBV \textit{in situ} focus with regard to their interpretation.
\subsection{Interpretational properties of IBV focus}\label{teke:sec:3.4}
In the introduction on the expressions of term focus, we have already noticed that in some examples the IBV focus strategy is preferred over the \textit{in situ} focus strategy: in the answer to an alternative question; in a contrastive focus expression; in the predicate doubling construction; and in most of the SOV sentences with a clear context in which some overt alternatives are available for the focused element. In this subsection I show more details on the interpretational distinction between IBV focus and the non-IBV focus strategies, arguing that the IBV position is usually, if not in all cases, used to express identificational focus. Here I refer to identificational focus as a focus type that identifies a referent in an existential presupposition; for example, in the English sentence `What I like is sunshine', where the presupposition is that there is something that I like and this something is identified as sunshine. The concept of identificational focus is also used as a hypernym of contrastive and exhaustive foci.


I begin with comparing a minimal sentence pair that only differs in the order of the constituents. Both the sentences in \xref{75a} and \xref{75b} can be felicitous answers to the question `what did father eat?', while they differ in interpretation as shown in the contexts. Example \xref{75a} with SOV word order is used to identify exactly what father ate; while \xref{75b} with SVO word order is used just to provide some new information. For example \xref{75a} the speakers clearly told me that there must be some alternatives invoked in mind and you want to identify what exactly the correct answer is.
\begin{exe} 
     \ex \label{75}
    \begin{xlist}
%%EAX
\ex
%%JUDGEMENT
%%LABEL
\label{75a}
%%CONTEXT
 (Context: there were many dishes and in fact father ate only some fish, and you may suspect him to have eaten something else.) \\
%%LINE2
\gll
Taará \textbf{báa-ntsúí} káá-dzí.\\
%%LINE3
1.father 2-fish 1\Sm{}.\Pst{}-eat.\Pst{}\\
%%TRANS1
\trans ‘The father ate some \textit{fish}.’
%%TRANS2
%%EXEND

%%EAX
\ex
%%JUDGEMENT
%%LABEL
\label{75b}
%%CONTEXT
 (Context: there were some fish and the father ate them all, and you just wanted to know what father ate.) \\
%%LINE2
\gll
Taará áá-dzí \textbf{baa-ntsúi}.\\
%%LINE3
1.father 1\Sm{}.\Pst{}-eat.\Pst{} 2-fish\\
%%TRANS1
\trans ‘The father ate some \textit{fish}.’
%%TRANS2
%%EXEND

    \end{xlist}
\end{exe}
The same distinction is attested in \xref{76}, which is a sentence extracted from a written task done by two speakers. One speaker was asked to write a letter in Kukuya to another speaker, and at the beginning of the letter after some greetings, the speaker asked the other if he saw me, using the SVO sentence in \xref{76a}. When I asked them if this sentence can be replaced by SOV as \xref{76b}, both of them judged it as infelicitous, saying that \xref{76b} is used only when the speaker thought the other had seen someone and wanted to know who exactly he saw. From this minimal pair we see again that SOV is used for identification and SVO simply provides new information.
\begin{exe}
    \ex (At the beginning of a letter: `How are you? Did you see Zhen yesterday?'...)\label{76}
    \begin{xlist}
%%EAX
\ex
%%JUDGEMENT
[]{
%%LABEL
\label{76a}
%%CONTEXT
%%LINE2
\gll
We á-mún-i Zhen?\\
%%LINE3
2\Sg{}.\Pro{} 2\Sg{}.\Pst{}-see-\Pst{} Zhen\\
%%TRANS1
\trans ‘Did you see Zhen?’
%%TRANS2
}
%%EXEND

%%EAX
\ex
%%JUDGEMENT
[\#]{
%%LABEL
\label{76b}
%%CONTEXT
%%LINE2
\gll
We Zhen á-mún-i?\\
%%LINE3
2\Sg{}.\Pro{} Zhen 2\Sg{}.\Pst{}-see-\Pst{}\\
%%TRANS1
\trans ‘Did you see \textit{Zhen}?’ 
%%TRANS2
}
%%EXEND

    \end{xlist}
\end{exe}
Example \xref{77a} is used when someone is asking about your profession. Here an identificational reading can also be deduced, since a person's career is usually the regular and unique activity that s/he is involved, and \xref{77a} implies that the speaker lives by only selling goats but not other animals; \xref{77b} is used in case where the speaker has a farm and s/he is just telling the others what s/he sells, in which the goats are not necessarily the only animal that the speaker sells. 
\begin{exe}
    \ex \label{77}
    \begin{xlist}
%%EAX
\ex
%%JUDGEMENT
%%LABEL
\label{77a}
%%CONTEXT
%%LINE2
\gll
Me \textbf{báa-ntabá} kâ-n-téke.\\
%%LINE3
1\Sg{}.\Pro{} 2-goat \Prs{}-1\Sg{}.\Sm{}-sell\\
%%TRANS1
\trans ‘I sell goats.’
%%TRANS2
%%EXEND

%%EAX
\ex
%%JUDGEMENT
%%LABEL
\label{77b}
%%CONTEXT
%%LINE2
\gll
Me kâ-n-téké \textbf{báa-ntabá}.\\
%%LINE3
1\Sg{}.\Pro{} \Prs{}-1\Sg{}.\Sm{}-sell 2-goat\\
%%TRANS1
\trans ‘I sell goats.’
%%TRANS2
%%EXEND

    \end{xlist}
\end{exe}
The sentences in \xref{78} intend to express focus on the subject and are both felicitous as answers to a subject question. While \xref{78a} is used when `you see that child crying and you want to know whether the father or the mother beat the child', \xref{78b} according to the speakers can also mean `it is father but not someone else that beat the child', here it seems that both subjects here may have been placed in the same IBV position.
\begin{exe}
     \ex \label{78}
    \begin{xlist}
%%EAX
\ex
%%JUDGEMENT
%%LABEL
\label{78a}
%%CONTEXT
 (Context: you see that child crying and you want to know whether the father or the mother beat the child.) \\
%%LINE2
\gll
Mwáana \textbf{taará} áá-béer-i.\\
%%LINE3
1.child 1.father 1\Sm{}.\Pst{}-beat-\Pst{}\\
%%TRANS1
\trans ‘The child is beaten by \textit{Father}.’
%%TRANS2
%%EXEND

%%EAX
\ex
%%JUDGEMENT
%%LABEL
\label{78b}
%%CONTEXT
 (Context: it is father but not someone else that beat the child.) \\
%%LINE2
\gll
\textbf{Taará} áá-béer-i mwáana.\\
%%LINE3
1.father 1\Sm{}.\Pst{}-beat-\Pst{} 1.child\\
%%TRANS1
\trans ‘Father beat the child.’
%%TRANS2
%%EXEND

    \end{xlist}
\end{exe}
The sentences in \xref{79} are examples of the construction that functions as an equivalent of the passive in Kukuya, which I will introduce in the next section. \xref{79a} is used in the context where you discovered the theft and were worrying about your things to have been all stolen, after checking you found that only the necklace was missing, while \xref{79b} is used when simply telling a truth that the thieves had come and a necklace was stolen. 
\begin{exe}
     \ex \label{79}
    \begin{xlist}
%%EAX
\ex
%%JUDGEMENT
%%LABEL
\label{79a}
%%CONTEXT
%%LINE2
\gll
\textbf{Mú-dzirá} báá-túr-i.\\
%%LINE3
3-necklace 2\Sm{}.\Pst{}-steal-\Pst{}\\
%%TRANS1
\trans ‘The necklace was stolen.’
%%TRANS2
%%EXEND

%%EAX
\ex
%%JUDGEMENT
%%LABEL
\label{79b}
%%CONTEXT
%%LINE2
\gll
Báá-túr-i \textbf{mu-dzirá}.\\
%%LINE3
2\Sm{}.\Pst{}-steal-\Pst{} 3-necklace\\
%%TRANS1
\trans ‘They stole the necklace. (The necklace was stolen)’
%%TRANS2
%%EXEND

    \end{xlist}
\end{exe}
Some more evidence comes from the interpretation on the word \textit{mbuurú} `person' in different positions, which is inspired by the same test used for diagnosing exclusive focus in \citet{vanderWal2016}. In Kukuya, the expression \textit{mbuurú} can have the reading `person' or `someone/anyone', which depends on the context. In \xref{80a} when \textit{mbuurú} is placed in the IBV position, it can only have a generic reading as `human-being' that contrasts with an animal; while in \xref{80b} \textit{mbuurú} can have either the reading `someone' or `person'. The generic reading in \xref{80a} is consistent with the hypothesis we make here on the IBV position being an identificational focus position \citep{vanderWal2016, vanderWal2021a}. The reading of `someone' is indefinite so is never identifiable, while the reading `person' can only be identified when contrasted with `non-person', namely the animals.
\begin{exe}
     \ex \label{80}
    \begin{xlist}
%%EAX
\ex
%%JUDGEMENT
%%LABEL
\label{80a}
%%CONTEXT
%%LINE2
\gll
Ngo \textbf{mbuurú} káá-dzí.\\
%%LINE3
1.leopard 1.person 1\Sm{}.\Pst{}-eat.\Pst{}\\
%%TRANS1
\trans ‘The leopard ate a \textit{person} (not an animal).’
%%TRANS2
%%EXEND

%%EAX
\ex
%%JUDGEMENT
%%LABEL
\label{80b}
%%CONTEXT
%%LINE2
\gll
Ngo áá-dzí \textbf{mbuurú}.\\
%%LINE3
1.leopard 1\Sm{}.\Pst{}-eat.\Pst{} 1.person\\
%%TRANS1
\trans ‘The leopard ate someone/the person/a person.’
%%TRANS2
%%EXEND

    \end{xlist}
\end{exe}
The generic reading is also attested in OSV as in \xref{81} that expresses focus on the subject. Example \xref{81} is used when you saw a dead leopard and you were wondering how the leopard died until you found an arrow on its body which indicated that it was killed by a human. Notice here that the `person' reading, though it can be definitely referring to a given person, can only show the contrast between  `this person' and `that person' when demonstrative modifiers are present, thus in \xref{80a} above and \xref{81} the \textit{mbuurú} in the IBV position cannot express contrast between different `persons' but can only have the generic reading.
\begin{exe}
%%EAX
\ex
%%JUDGEMENT
%%LABEL
\label{81}
%%CONTEXT
 (Context: you saw a dead leopard and you were wondering how the leopard died until you found an arrow on its body.)\\
%%LINE2
\gll
Ngo \textbf{mbuurú} áá-dzwî.\\
%%LINE3
1.leopard 1.person 1\Sm{}.\Pst{}-kill.\Pst{}\\
%%TRANS1
\trans ‘The leopard was killed by a \textit{person}.’
%%TRANS2
%%EXEND

\end{exe}
Another crucial piece of evidence supporting the IBV position as an identificational focus site lies in the negation strategy on the focal elements. In examples \xref{82} and \xref{83} we see that to negate the element in the IBV position, the often omitted copula can somehow ``show up" with the negative marker and precede the IBV item. Example \xref{82} means that the gecko was not eaten by the dog but by some other animals, and the negation targets only the subject (dog) and does not negate the action/sentence. Example \xref{83} means that father bought some other things instead of the bed. Given that the copula has an identifying function, its being placed immediately before the IBV focused element suggests that the IBV element is identificationally focused. The possible presence of the copula can also provide evidence on the origin and the nature of the IBV position, namely its connection with the cleft construction that is dedicated for identification and specification, which will be discussed in \sectref{teke:sec:5.2}.
\begin{exe}
%%EAX
\ex
%%JUDGEMENT
%%LABEL
\label{82}
%%CONTEXT
%%LINE2
\gll
Ngwangúlu ka-kí-li \textbf{mvá} áá-dzí ni.\\
%%LINE3
1.gecko \Neg{}-7\Sm{}-\Cop{} 1.dog 1\Sm{}.\Pst{}-eat.\Pst{} \Neg{}\\
%%TRANS1
\trans ‘The gecko was not eaten by the \textit{dog}.’
%%TRANS2
%%EXEND

\end{exe}
\begin{exe}
%%EAX
\ex
%%JUDGEMENT
%%LABEL
\label{83}
%%CONTEXT
%%LINE2
\gll
Taará ka(-kí-li) \textbf{ntáli} káá-sí me ni.\\
%%LINE3
1.father \Neg{}-7\Sm{}-\Cop{} 9.bed 1\Sm{}.\Pst{}-make.\Pst{} 1\Sg{}.\Pro{} \Neg{}\\
%%TRANS1
\trans ‘Father did not make a \textit{bed} for me.’
%%TRANS2
%%EXEND

\end{exe}
Although there is much evidence on the identificational nature of the IBV position, this position is not necessarily a dedicated exclusive focus position, from which we may see the difference between identificational and exclusive focus in this language. If the IBV position is used to express exclusive focus, we expect that an element modified by the strong quantifiers `every/each' and `all' should be incompatible with the IBV position, since a DP modified by these quantifiers is not exclusive \citep{É.Kiss1998, vanderWal2009a, vanderWal2011, vanderWal2016}. However, an \textit{every}-phrase can occur in the IBV position as shown in \xref{84a}, and to specify a set of alternatives such as `every chicken' to contrast with `every fish' in this example is possible but is judged to be unnecessary according to the speakers. The context of \xref{84a} can be either `there were several species of fish and you tasted each' or `there were many dishes and you only tasted each fish but not other meat'. In \xref{84b} we see that a DP modified by the universal quantifier `all' is also compatible with the IBV position, and here again to explicitly specify the alternatives such as `all the cakes' to show a contrast is possible but not necessary.
\begin{exe}
    \ex \label{84}
    \begin{xlist}
%%EAX
\ex
%%JUDGEMENT
%%LABEL
\label{84a}
%%CONTEXT
%%LINE2
\gll
Me \textbf{ná} \textbf{ntsúi} á-n-dziin-i.\\
%%LINE3
1\Sg{}.\Pro{} every 1.fish \Pst{}-1\Sg{}.\Sm{}-taste-\Pst{}\\
%%TRANS1
\trans ‘I tasted each fish.’
%%TRANS2
%%EXEND

%%EAX
\ex
%%JUDGEMENT
%%LABEL
\label{84b}
%%CONTEXT
%%LINE2
\gll
Me \textbf{báa-ntsúi} \textbf{bwě} á-n-dziin-i.\\
%%LINE3
1\Sg{}.\Pro{} 2-fish 2.all \Pst{}-1\Sg{}.\Sm{}-taste-\Pst{}\\
%%TRANS1
\trans ‘I tasted all the fish.’
%%TRANS2
%%EXEND

    \end{xlist}
\end{exe}
In \xref{85} we see that a DP modified by a scalar additive particle `even' that does not exclude the alternatives can occur at IBV. In \xref{86} the reply to an incomplete question with the additive particle `also' can surface in the SOV word order, which again indicates that the IBV position is not necessarily an exclusive focus position.
\begin{exe}
%%EAX
\ex
%%JUDGEMENT
%%LABEL
\label{85}
%%CONTEXT
 (Context: there is a lazy boy who never did any housework but today he has washed many things, the clothes, the curtains, the plates, and...)\\
%%LINE2
\gll
Ndé ntswê \textbf{ki-tséké} kíí me káá-swaak-í.\\
%%LINE3
1.\Pro{} even 7-hat 7.\Conn{} 1\Sg{}.\Pro{} 1\Sm{}.\Pst{}-wash-\Pst{}\\
%%TRANS1
\trans ‘He even washed my hat.’
%%TRANS2
%%EXEND

\end{exe}
\begin{exe}
%%EAX
\ex
%%JUDGEMENT
%%LABEL
\label{86}
%%CONTEXT
 (Did Gilbert wash the clothes?)\\
%%LINE2
\gll
Ndé \textbf{bí-ko} káá-swaak-í, ndé hé \textbf{má-saaní} káá-swaak-í.\\
%%LINE3
1.\Pro{} 8-clothes 1\Sm{}.\Pst{}-wash-\Pst{} 1.\Pro{} also 6-plate 1\Sm{}.\Pst{}-wash-\Pst{}\\
%%TRANS1
\trans ‘He washed the clothes, and he also washed the plates.’
%%TRANS2
%%EXEND

\end{exe}
There are also some counterarguments against the IBV position as being an identificational focus site. The first puzzle that remains to be explained is what we have already seen above: since the question words show the strongest tendency to be placed in the IBV position and if this preference is related to the identificational nature for most content questions, it is unexpected that SOV and SVO are both acceptable as the answer, if only the IBV position is employed for identificational focus.

Moreover, if the IBV position is identificational in nature which must have a presupposition of existence, a question with the IBV interrogative phrase cannot have an empty set answer, since the existence of a possible candidate is contained in the presupposition. In \xref{87} we see that the content question can be answered by `nobody', which indicates that there is no presupposition in the question, thus it is not necessarily identificational focus.
\begin{exe}
    \ex \label{87}
    \begin{xlist}
%%EAX
\ex
%%JUDGEMENT
%%LABEL
\label{87a}
%%CONTEXT
%%LINE2
\gll
We \textbf{ná} á-mún-i ku mu-súru?\\
%%LINE3
2\Sg{}.\Pro{} 1.who 2\Sg{}.\Pst{}-see-\Pst{} 17.\Loc{} 3-forest\\
%%TRANS1
\trans ‘Who did you see the the forest?’
%%TRANS2
%%EXEND

%%EAX
\ex
%%JUDGEMENT
%%LABEL
\label{87b}
%%CONTEXT
%%LINE2
\gll
Mbuurú ni.\\
%%LINE3
1.person \Neg{}\\
%%TRANS1
\trans ‘Nobody.’
%%TRANS2
%%EXEND

    \end{xlist}
\end{exe}
For these counterarguments against the IBV to be an identificational focus position, I will leave them open for now. The assumption is that, at an earlier stage the IBV position was indeed innovated for the sake of expressing identificational focus, which can be deduced from its possible origin from a cleft construction that I will discuss in \sectref{teke:sec:5.2}, but synchronically not all the uses of IBV position in all contexts are necessarily identificational, and in fact the IBV position has been observed to be in a further grammaticalisation process to become pragmatically neutral.


In this section I have introduced some syntactic and interpretational properties of the IBV focus position in Kukuya. I argued that the IBV position is a dedicated focus position which is structurally different from the canonical subject in the SVO order. I have shown that the IBV focus position is available for argument focus including subject and object, adjunct focus, sub-NP focus on a modifier as well as various types of predicate-centered focus such as VP focus and verb focus. The element that is placed in the IBV position can be an argument NP, an adjunct PP, or an infinitive verb in the predicate doubling construction. While focus can also be expressed postverbally for non-subject constituents, IBV focus tend to have an identificational reading but in some contexts it becomes pragmatically neutral. After investigating focus expressions in this language, in the next section I introduce topic expressions.

\section{Topical elements in the preverbal domain}\label{teke:sec:4}
In Bantu languages and in general cross-linguistically, topical elements show the general tendency to occur in the left periphery or the preverbal domain of the sentence \citep{Gundel1988, Henderson2006, vanderWal2009a, vanderWal2015, KerrEtAl2023}. Likewise, the topical elements in Kukuya also tend to occur in the preverbal domain. In this section I will first introduce that in Kukuya there are multiple types of topical elements and they all tend to co-occur in the preverbal domain. As illustrated above, there is a dedicated IBV focus position in this language, and in fact this IBV position can also interact with topical expressions. We will see in this section that in many sentences in which the IBV focus slot is occupied, all other non-focal elements tend to occur in the left periphery preceding the IBV slot, leaving the verb to the right boundary of the clause. Then I will present two specific constructions that can function as the equivalent of a passive, namely the OSV and the impersonal \textit{ba}- constructions, which can functionally compensate the absence of morphological passive marking in this language.
\subsection{Multiple topics in the preverbal domain}\label{teke:sec:4.1}
I will start by classifying different types of topical elements in this language. According to different syntactic and interpretational properties, at least four types of topical elements can be distinguished, which are the topical subject, the topical object, the scene-setting topic, and the secondary topic, which all precede the IBV position. Though in this section I will not investigate in detail the structural positions that these preverbal elements may occupy, their syntactic properties that are relevant to the discussion will also be mentioned. Next I present these elements one by one.
\subsubsection{Topical subject}\label{teke:sec:4.1.1}
First I will investigate how the topical subject in Kukuya behaves in terms of both syntactic and information structural status. I will show that a topical subject in Kukuya can occur in various preverbal positions, while its interpretational characteristics can differ. It should be noted that the available positions for a topical subject that I mention are not necessarily different structural positions but different linear positions \textit{relative to} other preverbal elements.

The first possible position that a topical subject can occur in is the initial position of a sentence, for example the topical subject in the SVO or SOV word order. Some examples are given in \xref{88}: in \xref{88a} the subject is topical while the focus is on the VP, and the pronominal subject in the congruent answer \xref{88b} is also topical since it is given and is what the predicate is about.
\begin{exe}
    \ex \label{88}
    \begin{xlist}
%%EAX
\ex
%%JUDGEMENT
%%LABEL
\label{88a}
%%CONTEXT
%%LINE2
\gll
Nkaaká kí-má káá-s\^{i}?\\
%%LINE3
1.grandmother 7-what 1\Sm{}.\Pst{}-do.\Pst{}\\
%%TRANS1
\trans ‘What did grandmother do?’
%%TRANS2
%%EXEND

%%EAX
\ex
%%JUDGEMENT
%%LABEL
\label{88b}
%%CONTEXT
%%LINE2
\gll
Ndé áá-tól-i ma-buokó ma-kí-ték-e.\\
%%LINE3
1.\Pro{} 1\Sm{}.\Pst{}-collect-\Pst{} 6-mushroom 6\Rel{}-7\Sm{}-sell-\Fv{}\\
%%TRANS1
\trans ‘She collected mushrooms to sell.’
%%TRANS2
%%EXEND

    \end{xlist}
\end{exe}
A topical subject that occurs in the sentence-initial position can often be followed by multiple other topical elements, in which case the IBV focus position cannot be empty but is always filled by a focal element. The other in-between topical elements in the preverbal domain are usually the objects of the verb. I will introduce the latter type in detail as ``secondary topic" in \sectref{teke:sec:4.1.3}. In example \xref{89} we see that the subject \textit{nkaaká} `grandmother', which controls the class 1 subject marking on the verb, is followed by the object of the verb \textit{buká} `cassava' and the interrogative word in the IBV position. In \xref{90} and \xref{91} two answers are illustrated in which the object and the adjunct are focused in IBV, and again the subject occurs in the sentence-initial position with another topical object sandwiched between the subject and the IBV element.
\begin{exe}
%%EAX
\ex
%%JUDGEMENT
%%LABEL
\label{89}
%%CONTEXT
%%LINE2
\gll
\textbf{Nkaaká} bu-ka ná ndé káá-bí-í kí-wâ?\\
%%LINE3
1.grandmother 14-cassava who 1.\Pro{} 1\Sm{}.\Pst{}-refuse-\Pst{} \Inf{}-give\\
%%TRANS1
\trans ‘To whom didn't the grandma give the cassava?’
%%TRANS2
%%EXEND

\end{exe}
\begin{exe}
%%EAX
\ex
%%JUDGEMENT
%%LABEL
\label{90}
%%CONTEXT
 (Did the grandma give the beans to the \textit{cats}?)\\
%%LINE2
\gll
Ambú, \textbf{ndé} ma-désu báa-mvá káá-wí.\\
%%LINE3
no 1.\Pro{} 6-bean 2-dog 1\Sm{}.\Pst{}-give.\Pst{}\\
%%TRANS1
\trans ‘No, she gave the beans to the \textit{dogs}.’
%%TRANS2
%%EXEND

\end{exe}
\begin{exe}
%%EAX
\ex
%%JUDGEMENT
%%LABEL
\label{91}
%%CONTEXT
 (How did father go to Djambala?) \\
%%LINE2
\gll
\textbf{Ndé} Dzambála mu miilí káá-yení.\\
%%LINE3
1.\Pro{} Djambala 18.\Loc{} 4.leg 1\Sm{}.\Pst{}-go.\Pst{}\\
%%TRANS1
\trans ‘He went to Djambala on foot.’
%%TRANS2
%%EXEND

\end{exe}
A difference between these two kinds of sentence-initial subjects with regard to whether they are followed by other topical elements is that, when the grammatical subject is the sole argument in the preverbal domain, it can be indefinite; when the subject is followed by other preverbal elements, namely in a SXXV construction, it cannot be indefinite. In \xref{92} the indefinite and non-specific reading can be deduced from the given context; in \xref{93} the subject is modified by a strong quantifier `every'. Since an indefinite non-specific element or a subject NP modified by strong quantifiers such as `all' and `every/each' cannot be dislocated nor a discourse topic \citep{Rizzi1986, Zerbian2006, Zeller2008, vanderWal2009a}, these examples suggest that there is at least one non-dislocated subject position in the preverbal domain. In example \xref{94} we see that in the SOV word order where the object gets focused in IBV, the subject can also be indefinite according to the context.
\begin{exe}
%%EAX
\ex
%%JUDGEMENT
%%LABEL
\label{92}
%%CONTEXT
 (Context: you returned home and found some footprints on the floor, you say to your roommate:) \\
%%LINE2
\gll
\textbf{Mbuurú} (nguumó) áá-yení.\\
%%LINE3
1.person {\db}1.one 1\Sm{}.\Pst{}-come.\Pst{}\\
%%TRANS1
\trans ‘Someone came.’ 
%%TRANS2
%%EXEND

\end{exe}
\begin{exe} 
%%EAX
\ex
%%JUDGEMENT
%%LABEL
\label{93}
%%CONTEXT
 (Context: the headmaster came to the class and distributed the candies to each of the children.) \\
%%LINE2
\gll
\textbf{Ná} \textbf{mwáana} áá-bák-i ba-bonbon.\\
%%LINE3
every 1.child 1\Sm{}.\Pst{}-get-\Pst{} 2-candy\\
%%TRANS1
\trans ‘Every child got candies.’ 
%%TRANS2
%%EXEND

\end{exe}
\begin{exe}
%%EAX
\ex
%%JUDGEMENT
%%LABEL
\label{94}
%%CONTEXT
 (Context: you returned home and found things are in disorder, after checking what was missing you realised that the money was stolen:) \\
%%LINE2
\gll
\textbf{Mbuurú} mí-pará káá-túr-i.\\
%%LINE3
1.person 4-money 1\Sm{}.\Pst{}-steal-\Pst{}\\
%%TRANS1
\trans ‘Someone stole the \textit{money}.’ 
%%TRANS2
%%EXEND

\end{exe}
When the initial subject is followed by other topical objects, the subject cannot be indefinite and non-specific. In examples \xref{95} and \xref{96}, we find that while the NPs \textit{mbuurú} and \textit{kilóko} can have both the indefinite and definite reading which depends on the context, they can only have the definite reading when followed by other topical elements. In other words, the initial subject must be topical if followed by other topical elements in the preverbal domain. According to most speakers it is infelicitous to place the modifier \textit{nguumó} `one' with the initial subject in the presence of other preverbal topical elements, as shown in \xref{96a}; the sentence can only become appropriate if the subject is the only preverbal element as in \xref{96b}. However, there is some intra-speaker variation on the judgement of \xref{96a}; it can be felicitous according to some speakers in the context of contrast on the direct object. 
\begin{exe}
%%EAX
\ex
%%JUDGEMENT
%%LABEL
\label{95}
%%CONTEXT
%%LINE2
\gll
Ki-lóko mwáana \textbf{mú-tswê} kíí-bólik-i.\\
%%LINE3
7-thing 1.child 3-head 7\Sm{}.\Pst{}-hurt-\Pst{}\\
%%TRANS1
\trans ‘The thing in question hurt the child's \textit{head}.’
%%TRANS2
%%EXEND

\end{exe}
\begin{exe}
    \ex \label{96}
    \begin{xlist}
    \judgewidth{\%}
%%EAX
\ex
%%JUDGEMENT
[\%]{
%%LABEL
\label{96a}
%%CONTEXT
%%LINE2
\gll
\textbf{Mbuurú} \textbf{nguumó} baa-ntsúú má-désu káá-búnum-i.\\
%%LINE3
1.person 1.one 2-chicken 6-bean 1\Sm{}.\Pst{}-feed-\Pst{}\\
%%TRANS1
\trans ‘The person/Someone fed the chicken the \textit{beans}.’
%%TRANS2
}
%%EXEND

%%EAX
\ex
%%JUDGEMENT
[]{
%%LABEL
\label{96b}
%%CONTEXT
%%LINE2
\gll
\textbf{Mbuurú} \textbf{nguumó} áá-búnum-i baa-ntsúú ma-désu.\\
%%LINE3
1.person 1.one 1\Sm{}.\Pst{}-feed-\Pst{} 2-chicken 6-bean\\
%%TRANS1
\trans ‘One person fed the chicken the \textit{beans}.’
%%TRANS2
}
%%EXEND

    \end{xlist}
\end{exe}
A topical subject can also occur in a non-initial position preceded by another topical element which also seems to be the subject of the sentence. In example \xref{97} and \xref{98} the sentence can be ambiguous whether it is actually about the `father' and the `child' or the `hoe' and the `lamp', respectively. The initial elements in both sentences are obviously the possessor or at least the user of the syntactic subjects that control subject marking on the verb, which looks like the ``possessor-raising" construction as in \xref{99}. In a similar construction in \xref{100}, the initial element `child' is not necessarily the possessor of the syntactic subject `festival' but should be an `experiencer', and the sentence is indeed about the `child' rather than the `festival' since the Q-A pair targets the information on the `feeling' of the `child'.
\begin{exe}
%%EAX
\ex
%%JUDGEMENT
%%LABEL
\label{97}
%%CONTEXT
%%LINE2
\gll
Taará \textbf{téme} ku-ní líí-dzinim-i?\\
%%LINE3
1.father 5.hoe 17-which 5\Sm{}.\Pst{}-disappear-\Pst{}\\
%%TRANS1
\trans ‘Where did father lose the hoe?' (lit.: `As for father, where did (his) hoe disappear?')
%%TRANS2
%%EXEND

\end{exe}
\begin{exe} 
%%EAX
\ex
%%JUDGEMENT
%%LABEL
\label{98}
%%CONTEXT
%%LINE2
\gll
Mwáana \textbf{múnda} wu-kí-fúúm-í maamá ku dzándu áá-dzínim-i.\\
%%LINE3
1.child 3.lamp 3\Rel{}-7\Sm{}-buy-\Pst{} 1.mother 17.\Loc{} 5.market 3\Sm{}.\Pst{}-disappear-\Pst{}\\
%%TRANS1
\trans ‘The child lost the lamp that mother bought at the market.’ (lit.: `As for the child, the lamp that mother bought at the market disappeared.')
%%TRANS2
%%EXEND

\end{exe}
\begin{exe} 
%%EAX
\ex
%%JUDGEMENT
%%LABEL
\label{99}
%%CONTEXT
%%LINE2
\gll
Mu-kokó áá-tsilik-í \textbf{{\textltailm}íibi} \textbf{mu-líeme}.\\
%%LINE3
1-king 1\Sm{}.\Pst{}-cut-\Pst{} 1.thief 3-finger\\
%%TRANS1
\trans ‘The king cut the thief the/his finger.’
%%TRANS2
%%EXEND

\end{exe}
\begin{exe} 
    \ex \label{100}
    \begin{xlist}
%%EAX
\ex
%%JUDGEMENT
%%LABEL
\label{100a}
%%CONTEXT
%%LINE2
\gll
Mwáana \textbf{ki-yinga} bu-ní kí-wir-i?\\
%%LINE3
1.child 7-festival 14-which 7\Sm{}.\Pst{}-pass-\Pst{}\\
%%TRANS1
\trans ‘The child, how did the festival pass (for him/her)?’
%%TRANS2
%%EXEND

%%EAX
\ex
%%JUDGEMENT
%%LABEL
\label{100b}
%%CONTEXT
%%LINE2
\gll
Ndé \textbf{ki-yinga} kí-bvé kí-wir-i.\\
%%LINE3
1.\Pro{} 7-festival 7-good 7\Sm{}.\Pst{}-pass-\Pst{}\\
%%TRANS1
\trans ‘(For) him/her, the festival passed well.’
%%TRANS2
%%EXEND

\end{xlist}
\end{exe}
These examples are reminiscent of the ``double subject construction" as is attested in Chinese and Japanese, but from these examples we can observe that the initial element is clearly not an argument of the verb, so it cannot be the grammatical subject, but should be analysed as a ``scene-setting" topic, which is the second type of preverbal topical element that I would like to introduce next.
\subsubsection{Scene-setting topic}\label{teke:sec:4.1.2}
A scene-setting topic usually sets the ``spatial, temporal or individual framework" of the rest of the sentence \citep{Chafe1976, LiThompson1976}. Some examples of scene-setting topics in Kukuya are given in \xxref{101}{103}. In \xref{101} the sentence-initial topic is a DP which is co-referential to the pronominal subject in the pseudo-cleft construction that follows; in \xref{102} and \xref{103} the scene-setting topics are adverbial phrases. What distinguishes the scene-setting topics from topical subjects or objects besides their semantic function is that a scene-setting element never functions as an argument of the verb, thus is not originated from the rest of the sentence \citep{Lambrecht1994}, it occurs only for the sake of limiting the frame of the proposition or semantically relate the event described by the core sentence to an ``external topic". In addition, there is a further division on the relation of the scene-setting elements and the rest of the sentence in examples \xxref{100}{103}. In \xref{100} and \xref{101}, the initial element is what the sentence is ``about" as the whole sentence is telling something on the `child' and the `woman'; while in \xref{102} and \xref{103} the locative and temporal phrases only set the background or the scene of the sentence and the aboutness topic expression is the 1st person pronoun. 
\begin{exe}
%%EAX
\ex
%%JUDGEMENT
%%LABEL
\label{101}
%%CONTEXT
%%LINE2
\gll
\textbf{Mu-kái} wu-ká-búr-í ndé mú-kái wó balaka?\\
%%LINE3
1-woman 1\Rel{}-1\Sm{}.\Pst{}-give.birth-\Pst{} 1.\Pro{} 1-female or 1.male\\
%%TRANS1
\trans ‘The woman, whom she gave birth to was a girl or a boy?’
%%TRANS2
%%EXEND

\end{exe}
\begin{exe}
%%EAX
\ex
%%JUDGEMENT
%%LABEL
\label{102}
%%CONTEXT
%%LINE2
\gll
\textbf{Mu} \textbf{mu-súru}, me á-mún-i ba-kái bá-k\^{a}-tólo nkwî.\\
%%LINE3
18.\Loc{} 3-forest 1\Sg{}.\Pro{} \Pst{}-1\Sg{}.see-\Pst{} 2-women 2\Sm{}-\Prog{}-cut 9.firewood\\
%%TRANS1
\trans ‘In the forest, I saw women cutting the firewood.’
%%TRANS2
%%EXEND

\end{exe}
\begin{exe}
%%EAX
\ex
%%JUDGEMENT
%%LABEL
\label{103}
%%CONTEXT
%%LINE2
\gll
\textbf{Mvúla} \textbf{wǔ-yá} me â-m-fúúm-á báa-ntaba nkáma.\\
%%LINE3
3.year 3\Rel{}-come 1\Sg{}.\Pro{} \Fut{}-1\Sg{}.\Sm{}-buy-\Fv{} 2-goat hundred\\
%%TRANS1
\trans ‘Next year I will buy a hundred goats.’
%%TRANS2
%%EXEND

\end{exe}
\subsubsection{Secondary topic}\label{teke:sec:4.1.3}
The third type of topical element is what I label as ``secondary topic". Crosslinguistically, an utterance can contain more than one topic under discussion simultaneously, which is often attested in the predicate-focus structure as shown in \xref{104}. In \xref{104a} the question is on some relation between mother and the goats, and the answer in \xref{104b} adds information to both the mother and the goats, thus here the two arguments should be both counted as topics. The question now is how to determine primary and secondary topichood.
\begin{exe}
    \ex \label{104}
    \begin{xlist}
%%EAX
\ex
%%JUDGEMENT
%%LABEL
\label{104a}
%%CONTEXT
%%LINE2
\gll
Ng\'{u}ku baa-ntaba kí-má káá-sî?\\
%%LINE3
1.mother 2-goat 7-what 1\Sm{}.\Pst{}-do.\Pst{}\\
%%TRANS1
\trans ‘What did mother do to the goats?’
%%TRANS2
%%EXEND

%%EAX
\ex
%%JUDGEMENT
%%LABEL
\label{104b}
%%CONTEXT
%%LINE2
\gll
Ng\'{u}ku áá-dzwí baa-ntaba.\\
%%LINE3
1.mother 1\Sm{}.\Pst{}-kill.\Pst{} 2-goat\\
%%TRANS1
\trans ‘Mother killed the goats.’
%%TRANS2
%%EXEND

    \end{xlist}
\end{exe}
In \citet{Nikolaeva2001}, a secondary topic is defined as ``an entity such that the utterance is construed to be about the \textit{relationship} between it and the primary topic". The primary topic is considered to be more pragmatically salient and is closely associated with the subject function \citep{DalrympleNikolaeva2011}; as the secondary topic would often be realised as the object of the verb, which corresponds to some assumption that in historical terms, objects are grammaticalised secondary topics \citep{Givón1984, Givón1990, Givón2001}. In \citegen{Vallduví1992} approach, the old information in the utterance can be further split into informationally more and less prominent material, namely the ``link" and ``tail" which correspond to the primary and secondary topic we discuss here. While the ``link" is what the new information is anchored to, the ``tail" entails the presence of the ``link" and implies that some update is to be carried out to complete the information on the relation between it and the ``link". In other words, the primary and secondary topics stand in a certain presupposed relation: the secondary topic presupposes the existence of the primary topic, and the proposition is to add new knowledge to some relation between the primary and the secondary topics \citep{DalrympleNikolaeva2011}. In the above example \xref{104}, the primary topic is the `mother' and the secondary topic is the `goats', since the question is on what actions are done on the goats, thus the topic `goats' as the patient entails the presence of the `mother' as the agent.


In particular in the Kukuya language, I propose that the distinction on primary and secondary topics is grammatically encoded via word order: if there are more than one topical elements in the preverbal domain, only the primary topic can be placed sentence-initially (excluding the scene-setting topics), while the secondary topic should be non-initial. There are three informational types that involve secondary topic in Kukuya: the first type is as in example \xref{104} in which the focus extends over the transitive predicate only, namely the verb focus expression where both arguments of the verb are given; the second type is the possessive secondary topic as in example \xref{97} above, in which the two preverbal topics are in a possessive relation and the possessor functions as the primary topic, while the possessum is the secondary topic and the syntactic subject \citep{Nikolaeva2001}; a similar example is given in \xref{105}.
\begin{exe}
%%EAX
\ex
%%JUDGEMENT
%%LABEL
\label{105}
%%CONTEXT
%%LINE2
\gll
Ngúku \textbf{ndzulí} ku-ní áá-dzinim-i?\\
%%LINE3
1.mother 1.cat 17-which 1\Sm{}.\Pst{}-disappear-\Pst{}\\
%%TRANS1
\trans ‘Where did mother lose the cat?' (lit: `As for mother, where did (her) cat disappear?')
%%TRANS2
%%EXEND

\end{exe}
The third type of secondary topic is attested in adjunct or argument focus constructions. In example \xref{106} and \xref{107}, when the locative interrogative word is focused in the IBV position, the object which is given in the context occurs between the initial subject and the IBV focused element, resulting in two topical elements in the preverbal domain. In \xref{106b} the assertion updates the addressee's knowledge on the relation between uncle and the rice by adding information that it was yesterday that uncle ate the rice; here the `uncle' functions as the primary topic and `rice' as secondary topic. Similarly in \xref{107}, there are two preverbal secondary topics `falling' and `plates' which are the two objects of the ditransitive verb `to launch'. The word order pattern in \xref{106} and \xref{107} is very commonly attested in the formulation of question-answer pairs in Kukuya.

\begin{exe}
    \ex \label{106}
    \begin{xlist}
%%EAX
\ex
%%JUDGEMENT
%%LABEL
\label{106a}
%%CONTEXT
%%LINE2
\gll
Mu-pfúru \textbf{lóoso} munkí káá-dzí?\\
%%LINE3
1-uncle 5.rice when 1\Sm{}.\Pst{}-eat.\Pst{}\\
%%TRANS1
\trans ‘When did uncle eat the rice?’
%%TRANS2
%%EXEND

%%EAX
\ex
%%JUDGEMENT
%%LABEL
\label{106b}
%%CONTEXT
%%LINE2
\gll
Ndé \textbf{lóoso} má-tsíká káá-dzí.\\
%%LINE3
1.\Pro{} 5.rice 6-yesterday 1\Sm{}.\Pst{}-eat.\Pst{}\\
%%TRANS1
\trans ‘He ate the rice \textit{yesterday}.’
%%TRANS2
%%EXEND

    \end{xlist}

%%EAX
\ex
%%JUDGEMENT
%%LABEL
\label{107}
%%CONTEXT
%%LINE2
\gll
Taará \textbf{ma-sáani} \textbf{bví} ku-ní káá-tí?\\
%%LINE3
1.father 6-plate 9.falling 17-which 1\Sm{}.\Pst{}-launch-\Pst{}\\
%%TRANS1
\trans ‘Where did father throw the plates?’
%%TRANS2
%%EXEND

\end{exe}
In \xref{108} and \xref{109} the division between primary and secondary topic is seen in the context of argument focus. In \xref{108} the recipient object of the ditransitive verb is focused in IBV, and the topical theme object is placed in the preverbal domain as the secondary topic; in \xref{109} it is the subject that gets focused in IBV, and both objects of the verb `to give' are placed preverbally, in this case the theme `oranges' is the secondary topic.

\begin{exe}
%%EAX
\ex
%%JUDGEMENT
%%LABEL
\label{108}
%%CONTEXT
 (Did the grandfather give the food to the \textit{dogs}?) \\
%%LINE2
\gll
Ambú, ndé \textbf{bvi-kídzá} báa-ndzulí káá-wî.\\
%%LINE3
no 1.\Pro{} 8-food 2-cat 1\Sm{}.\Pst{}-give.\Pst{}\\
%%TRANS1
\trans ‘No, he gave the food to the \textit{cats}.’
%%TRANS2
%%EXEND


%%EAX
\ex
%%JUDGEMENT
%%LABEL
\label{109}
%%CONTEXT
 (`Who gave the child the oranges?') \\
%%LINE2
\gll
Mwáana ma-láara \textbf{bí-búru} bíí-wî.\\
%%LINE3
1.child 6-orange 8-parent 8\Sm{}.\Pst{}-give.\Pst{}\\
%%TRANS1
\trans ‘The child was given the oranges by the \textit{parents}.’
%%TRANS2
%%EXEND

\end{exe}
In the above examples, it is interesting to see that the exploitation of IBV focus is usually accompanied by the ``fronting" of other topical elements to the preverbal domain, while it is grammatical that the topical objects and adjuncts remain in their base positions, i.e. postverbally. The exact trigger of this topic fronting, whether syntactic or pragmatic, is left for further research. Here I propose that the Kukuya language can grammatically distinguish the primary and secondary topic by word order: the sentence-initial (excluding the scene-setting) topic is always primary while the non-initial one is secondary. Since the primary topic usually sets the most important framework and aboutness of the main predication, while a secondary topic is less important and continuous in terms of referential accessibility and thematic importance \citep{Givón1990, Nikolaeva2001, Croft1991, Tsao1987, Shi2000}, it is necessary that the primary topic scopes over the secondary topic, so the former is placed in the initial position.


The secondary topic must have a definite reading. In \xref{110} we see that it is infelicitous to have an indefinite object `someone/one person' occur in the preverbal domain and function as a secondary topic; in \xref{111} the preposed object \textit{kilóko} `thing' can only be interpreted as some particular thing that has been mentioned before but cannot be indefinite non-specific, as can be deduced from the context.

\begin{exe}
\judgewidth{??}
%%EAX
\ex
%%JUDGEMENT
[??]{
%%LABEL
\label{110}
%%CONTEXT
 (Context: you are traveling in a very quiet small town and you did not see anyone on the street. Your friend said she saw a person's figure on the way and you ask her:)\\
%%LINE2
\gll
We \textbf{mbuurú} \textbf{nguumó} ku-ní áá-mún-i?\\
%%LINE3
2\Sg{}.\Pro{} 1.person 1.one 17-which 2\Sg{}.\Pst{}-see-\Pst{}\\
%%TRANS1
\trans int. ‘Where did you see someone/one person?’
%%TRANS2
}
%%EXEND

%%EAX
\ex
%%JUDGEMENT
[]{
%%LABEL
\label{111}
%%CONTEXT
 (Context: you have a precious gift in your home. One day you found a theft but fortunately the precious thing was not stolen.)\\
%%LINE2
\gll
{\textltailm}íibi \textbf{ki-lóko} ka-káá-túr-i ni.\\
%%LINE3
1.thief 7-thing \Neg{}-1\Sm{}.\Pst{}-steal-\Pst{} \Neg{}\\
%%TRANS1
\trans ‘The thief did not steal the thing/*anything.’\\
%%TRANS2
    *‘The thief stole nothing.’    
}
%%EXEND

\end{exe}
In example \xref{112} some sentences with minimal difference on the position of the object \textit{mbuurú} `person' are illustrated. For \xref{112a} and \xref{112b}, as implied from the context, the preposed object can only be interpreted as definite: \xref{112a} and \xref{112b} have the same interpretation and can both be appropriately used in the context of (a) but neither can be used in the context of (b), which shows neither sentence can have the reading of `I saw nobody'; in \xref{112c} the object is placed in the IBV position and can only have a generic reading as `human-being'; while in \xref{112d} the object in its canonical postverbal position can have the indefinite reading and functions as a negative polarity item (NPI), or a definite reading can also arise according to the context.\largerpage[2]

\begin{exe}
     \ex \label{112}
    \begin{xlist}
%%EAX
\ex
%%JUDGEMENT
%%LABEL
\label{112a}
%%CONTEXT
 (Context: your uncle asked you to call a certain person sitting under a tree nearby to come. You went but did not find the person. You returned and said to the uncle:) \\
%%LINE2
\gll
\textbf{Mbuurú} me ka-á-mún-i ni.\\
%%LINE3
1.person 1\Sg{}.\Pro{} \Neg{}-\Pst{}-1\Sg{}.see-\Pst{} \Neg{}\\
%%TRANS1
\trans ‘I did not see the person/*anyone.’
%%TRANS2
%%EXEND

%%EAX
\ex
%%JUDGEMENT
%%LABEL
\label{112b}
%%CONTEXT
 (\textsuperscript{\#}Context: your mum and you are entering a dark hall. You are walking in front and your mum asked you from behind if you saw anyone in the hall.) \\
%%LINE2
\gll
Me \textbf{mbuurú} ka-á-mún-i ni.\\
%%LINE3
1\Sg{}.\Pro{} 1.person \Neg{}-\Pst{}-1\Sg{}.see-\Pst{} \Neg{}\\
%%TRANS1
\trans ‘I did not see the person/*anyone.’
%%TRANS2
%%EXEND

%%EAX
\ex
%%JUDGEMENT
%%LABEL
\label{112c}
%%CONTEXT
 (Context: you saw a `monster' in the forest; you did not know what animal it was, and after coming back someone asked you if you see anybody in the forest.) \\
%%LINE2
\gll
Me ka \textbf{mbuurú} á-mún-i ni.\\
%%LINE3
1\Sg{}.\Pro{} \Neg{} 1.person \Pst{}-1\Sg{}.see-\Pst{} \Neg{}\\
%%TRANS1
\trans ‘I did not see a \textit{person}/*anyone.’
%%TRANS2
%%EXEND

%%EAX
\ex
%%JUDGEMENT
%%LABEL
\label{112d}
%%CONTEXT
 (felicitous in the context of both (a) and (b)) \\
%%LINE2
\gll
Me ka-á-mún-i \textbf{mbuurú} ni.\\
%%LINE3
1\Sg{}.\Pro{} \Neg{}-\Pst{}-1\Sg{}.see-\Pst{} 1.person \Neg{}\\
%%TRANS1
\trans ‘I did not see anyone/the person.’
%%TRANS2
%%EXEND

    \end{xlist}
\end{exe}
In this subsection I have shown that there can be multiple preverbal topical elements, which can be further divided into primary topics which include sentence-initial topical subject (also object, see next section) and scene-setting topics, and secondary topics which are usually objects of the verb. The interpretation on these topics with regard to definiteness and specificity may depend to a large extent on their relative position in the preverbal domain. An initial subject, if it is not the sole preverbal argument, and a secondary topic must be definite. There are still some further questions that need to be investigated, such as potential restrictions on the order of the preverbal elements, and the connection between IBV focus and topic fronting.

As mentioned at the beginning of the section, there is a fourth type of topical element in Kukuya, which is the topicalised object that occurs in the initial position of the sentence. The sentence-initial object is usually attested in an OSV word order as in \xref{113} and an impersonal \textit{ba}-construction as in \xref{114}, which can serve as functional \textit{passives} in this language. In the next section I will introduce in detail these two functional passive constructions.
\largerpage[-1]

\begin{exe}
%%EAX
\ex
%%JUDGEMENT
%%LABEL
\label{113}
%%CONTEXT
%%LINE2
\gll
\textbf{Bii-ndomó}	kíi-mbúlí	kí-dzí.\\
%%LINE3
8-sheep	7-lion	7\Sm{}.\Pst{}-eat.\Pst{}\\
%%TRANS1
\trans ‘The sheep were eaten by the lion.’ (lit.: `The sheep, the lion ate them.')
%%TRANS2
%%EXEND

%%EAX
\ex
%%JUDGEMENT
%%LABEL
\label{114}
%%CONTEXT
%%LINE2
\gll
\textbf{Mu-tí}	\textbf{mu}	\textbf{máa-ŋgúlu}	\^{a}li	báa-tsílik-i	mbvúlá  wǔ-fíŋ-a.\\
%%LINE3
3-tree 18.\Conn{} 6-mango \Aux{}.\Rpst{} 2\Sm{}.\Pst{}-cut.down-\Pst{} 3.year 3\Rel{}-pass-\Fv{}\\
%%TRANS1
\trans ‘The mango tree was cut down last year.’
%%TRANS2
%%EXEND

\end{exe}
\subsection{Functional passives}\label{teke:sec:4.2}
In this subsection I introduce how Kukuya makes use of the IBV focus position and the topic fronting tendency to express the passive meaning. Two particular structures are presented, namely the OSV and the impersonal \textit{ba}- constructions. I first discuss how the passive reading is generated through these constructions, and then display some basic syntactic and interpretational properties of both structures as well as their restrictions in use. Some of the presentation here is part of my previous work in \citet{Li2020}, and is primarily inspired by the pioneering work of \citet{BostoenMundeke2011} on similar functional constructions in another West-Coastal Bantu language Mbuun (B87).

In most Bantu languages, the passive is typically encoded by a verbal derivational suffix and a shift of grammatical roles of the arguments. In the Swahili example \xref{115}, we see that the passive marker \mbox{-\textit{iw}-} is used, the patient is promoted to the subject position and controls subject marking on the verb, while the agent can be optionally expressed by a prepositional phrase. The Kukuya language systematically lacks verbal derivational suffixes, with only some unproductive residues, thus we may wonder how passiveness is expressed in Kukuya, compensating the absence of morphological passive marking. 

\begin{exe}
%%EAX
\ex
%%JUDGEMENT
%%LABEL
\label{115}
%%CONTEXT
%%LINE2
\gll
Vy-akula vi-li-l-\textbf{iw}-a (na wa-toto).\\
%%LINE3
8-food 8\Sm{}-\Pst{}-eat-\Pass{}-\Fv{} {\db}by 2-child\\
%%TRANS1
\trans ‘The food was eaten by children.’ \jambox*{[Swahili G42]}
%%TRANS2
%%EXEND

\end{exe} 
\subsubsection{The OSV construction}\label{teke:sec:4.2.1}
The first functional passive construction in Kukuya is the OSV structure in which the object is fronted to the sentence-initial position while the subject is placed in the IBV position, as shown in \xref{116} and \xref{117}. From these examples we can also see that both animate and inanimate subjects can have the agent reading in this functional passive construction.
\begin{exe}
%%EAX
\ex
%%JUDGEMENT
%%LABEL
\label{116}
%%CONTEXT
%%LINE2
\gll
Mbaá	mvúlá	áá-dzíib-i.\\
%%LINE3
9.fire	3.rain	3\Sm{}.\Pst{}-extinguish-\Pst{}\\
%%TRANS1
\trans ‘The fire was extinguished by the rain.’  (lit.: `The fire, the rain put it out.') \jambox*{\citep[4]{Li2020}}
%%TRANS2
%%EXEND

\end{exe}
\begin{exe}
%%EAX
\ex
%%JUDGEMENT
%%LABEL
\label{117}
%%CONTEXT
%%LINE2
\gll
Bu-ká búú mwáana nzulí áá-wool-i.\\
%%LINE3
14-cassava 14.\Conn{} 1.child 1.cat 1\Sm{}-snatch-\Pst{}\\
%%TRANS1
\trans ‘The child's cassava was snatched by the cat.’ (lit.: `The cassava of the child, the cat snatched it.')
%%TRANS2
%%EXEND

\end{exe} 
 In the ditransitive constructions \xref{118} and \xref{119}, we see that both the theme and the patient object can be fronted. The passive reading can be verified in the elicitation: when I asked the speakers to translate the French passive into Teke and there was an explicit agent in the sentence, the OSV structure was always used.
\begin{exe}
    \ex \label{118}
    \begin{xlist}
%%EAX
\ex
%%JUDGEMENT
%%LABEL
\label{118a}
%%CONTEXT
%%LINE2
\gll
Báana ngúku áá-télek-i bvi-kídza.\\
%%LINE3
2.children 1.mother 1\Sm{}.\Pst{}-prepare-\Pst{} 8-food\\
%%TRANS1
\trans ‘The children were prepared the food by mother.’
%%TRANS2
%%EXEND

%%EAX
\ex
%%JUDGEMENT
%%LABEL
\label{118b}
%%CONTEXT
%%LINE2
\gll
Bvi-kídza ngúku áá-télek-i báana.\\
%%LINE3
8-food 1.mother 1\Sm{}.\Pst{}-prepare-\Pst{} 2.children\\
%%TRANS1
\trans ‘The food was prepared for the children by mother.’
%%TRANS2
%%EXEND

    \end{xlist}
\end{exe}
\begin{exe}
    \ex \label{119}
    \begin{xlist}
%%EAX
\ex
%%JUDGEMENT
%%LABEL
\label{119a}
%%CONTEXT
%%LINE2
\gll
Mu-safuká mú-káí áá-kwá-i mu mbhiele.\\
%%LINE3
3-safou.tree 1-woman 1\Sm{}.\Pst{}-chop-\Pst{} 18.\Loc{} 9.knife\\
%%TRANS1
\trans ‘The safou tree was chopped with a knife by the woman.’
%%TRANS2
%%EXEND

%%EAX
\ex
%%JUDGEMENT
%%LABEL
\label{119b}
%%CONTEXT
%%LINE2
\gll
Mbhiele mú-kái áá-kwá-í mu mu-safuká.\\
%%LINE3
9.knife 1-woman 1\Sm{}.\Pst{}-chop-\Pst{} 18.\Loc{} 3-safou.tree\\
%%TRANS1
\trans ‘A knife was used to chop the safou tree by the woman.’
%%TRANS2
%%EXEND

    \end{xlist}
\end{exe}
For the examples above I only gave the passive translation as a stimulus. However, the OSV construction itself does not show apparent grammatical means that are dedicated to passive expression, and here I want to decompose the OSV structure to see how the passive reading has emerged. Pragmatically, passiveness is often considered as a ``foregrounding and backgrounding operation” \citep{KeenanDryer2007} in which the patient is foregrounded to the sentence-initial position while the agent is backgrounded or unspecified. In this sense, a passive construction is similar to the topicalisation operation in which the patient is fronted to the sentence-initial position to become the topic of the sentence, while the agent can remain in the original position or be demoted to a less/non-topical position. In other words, a passivised element is usually made topical. The availability of OSV structure to express passive is thus consistent with the generalisation in the above subsection that in Kukuya topical elements tend to occur in the preverbal domain, so the topical object in OSV is expressed in the sentence-initial position. Nevertheless, in the OSV construction the agent subject is always explicitly expressed, which is not expected in a canonical passive construction. In addition, recall that the OSV structure is what I have introduced for subject focus (see \sectref{teke:sec:3.1.2}) and is always felicitous as an answer to a subject question, as shown in \xref{120} and \xref{121}. The focal status of the agent is pragmatically incompatible with a prototypical passive construction in which the agent is usually demoted or even deleted.
\begin{exe}
    \ex \label{120}
    \begin{xlist}
%%EAX
\ex
%%JUDGEMENT
%%LABEL
\label{120a}
%%CONTEXT
%%LINE2
\gll
Mwáana láana ná áá-wî?\\
%%LINE3
1.child 5.orange 1.who 1\Sm{}.\Pst{}-give.\Pst{}\\
%%TRANS1
\trans ‘Who gave the child the orange?’
%%TRANS2
%%EXEND

%%EAX
\ex
%%JUDGEMENT
%%LABEL
\label{120b}
%%CONTEXT
%%LINE2
\gll
Mwáana láala taará áá-wî.\\
%%LINE3
1.child 5.orange 1.father 1\Sm{}.\Pst{}-give.\Pst{}\\
%%TRANS1
\trans ‘\textit{Father} gave the orange to the child.’
%%TRANS2
%%EXEND

    \end{xlist}
\end{exe}
\begin{exe}
    \ex \label{121}
    \begin{xlist}
%%EAX
\ex
%%JUDGEMENT
%%LABEL
\label{121a}
%%CONTEXT
%%LINE2
\gll
Wǔ-fúum-i ma-li ná ndé?\\
%%LINE3
1\Rel{}-buy-\Pst{} 6-wine 1.who 1.\Pro{}\\
%%TRANS1
\trans ‘Who bought the wine?’
%%TRANS2
%%EXEND

%%EAX
\ex
%%JUDGEMENT
%%LABEL
\label{121b}
%%CONTEXT
%%LINE2
\gll
Ma-li taará áá-fúum-i.\\
%%LINE3
6-wine 1.father 1\Sm{}.\Pst{}-buy-\Pst{}\\
%%TRANS1
\trans ‘The wine was bought by \textit{Father}.’ \jambox*{\citep[15]{Li2020}}
%%TRANS2
%%EXEND

    \end{xlist} 
\end{exe}   
In \xref{122a} we can see that the OSV functional passive construction cannot have a questioned adjunct, since the IBV focus position is occupied by the agent and there is usually only one focused element in a Kukuya sentence, the interrogative phrase becomes infelicitous even in its base position; the only possible rephrasing is \xref{122b} in which the passive reading is lost. 
\begin{exe}
    \ex \label{122}
    \begin{xlist}
%%EAX
\ex
%%JUDGEMENT
[*]{
%%LABEL
\label{122a}
%%CONTEXT
%%LINE2
\gll
Mwáana taará áá-béer-i munkí?\\
%%LINE3
1.child 1.father 1\Sm{}.\Pst{}-beat-\Pst{} when\\
%%TRANS1
\trans int. ‘When was the child beaten by father?’
%%TRANS2
}
%%EXEND

%%EAX
\ex
%%JUDGEMENT
[]{
%%LABEL
\label{122b}
%%CONTEXT
%%LINE2
\gll
Taará mwáana munkí káá-béer-i?\\
%%LINE3
1.father 1.child when 1\Sm{}.\Pst{}-beat-\Pst{}\\
%%TRANS1
\trans ‘When did father beat the child?’
%%TRANS2
}
%%EXEND

    \end{xlist}
\end{exe}
When I intend to elicit a sentence like `What was stolen by X?' in which the ``passivised" object is an interrogative phrase, the speakers still use the OSV word order as in \xref{123}. At first glance, we see that a question word can occur in the initial position of the OSV construction to express passive. However, as will be discussed in the next section, the sentence in \xref{123a} is actually a cleft construction in which the class 1 subject marker shifts from \textit{á}- to \textit{ká}-; in \xref{123b} we find that the initial question word is incompatible with the canonical subject marker. The OSV word order in \xref{123} is not the OSV functional passive construction that we are discussing here.
\begin{exe}
    \ex \label{123}
    \begin{xlist}
%%EAX
\ex
%%JUDGEMENT
[]{
%%LABEL
\label{123a}
%%CONTEXT
%%LINE2
\gll
\textbf{Kí-má} {\textltailm}íibi \textbf{káá}-túr-i?\\
%%LINE3
7-what 1.thief 1\Sm{}.\Pst{}-steal-\Pst{}\\
%%TRANS1
\trans ‘What was stolen by the thief?’
%%TRANS2
}
%%EXEND

%%EAX
\ex
%%JUDGEMENT
[*]{
%%LABEL
\label{123b}
%%CONTEXT
%%LINE2
\gll
\textbf{Kí-má} {\textltailm}íibi \textbf{áá}-túr-i?\\
%%LINE3
7-what 1.thief 1\Sm{}.\Pst{}-steal-\Pst{}\\
%%TRANS1
\trans int. ‘What was stolen by the thief?’
%%TRANS2
}
%%EXEND

    \end{xlist}
\end{exe}
Therefore it shows that the OSV structure, though it can function as a translational equivalent of a canonical passive construction, is by no means dedicated to express passive and is at least pragmatically different from a true passive (see \citealt{BostoenMundeke2011} for a similar proposal for Mbuun). The primary function of the OSV construction is to clearly delimit the different discourse roles of the subject and object, in which the object is topicalised and fronted to the initial position; the subject is focused in the IBV position and the focus reading is somehow ``strengthened" by fronting the topical object. Pragmatically, the OSV construction can function as the equivalent of passive but is used only when the agent serves as the new or contrasted information, thus needs to be explicitly expressed.
\subsubsection{Impersonal \textit{ba}- construction}\label{teke:sec:4.2.2}
The second equivalent of passive in Kukuya is the so-called impersonal \textit{ba}- construction. In this construction, the verb always takes the class 2 subject marker \textit{ba}- which is not anaphoric to any lexical or pronominal subject in the sentence or the discourse. The patient object can occur either postverbally or preverbally, while the agent is in most cases deleted or unspecified, and this is why the construction is labeled as ``impersonal". Some examples are illustrated below. In \xref{124} and \xref{125}, the patient object occurs preverbally, and the agent is unknown and suppressed; while in \xref{126} and \xref{127} there is no preverbal element and the patient object occurs after the \textit{ba}- verb.
\begin{exe}
%%EAX
\ex
%%JUDGEMENT
%%LABEL
\label{124}
%%CONTEXT
 (visual stimuli: What about the food?) \\
%%LINE2
\gll
\textbf{Bviilá} \textbf{báá}-tél-i bví ku mfúúlá.\\
%%LINE3
8.food 2\Sm{}.\Pst{}-throw-\Pst{} 9.falling 17.\Loc{} 9.road\\
%%TRANS1
\trans ‘The food was thrown onto the road.’
%%TRANS2
%%EXEND

\end{exe}
\begin{exe}
%%EAX
\ex
%%JUDGEMENT
%%LABEL
\label{125}
%%CONTEXT
%%LINE2
\gll
\textbf{Mu-ŋwâ} wu-kí-som-í báa-mpúku \textbf{báá}-kí-i.\\
%%LINE3
3.hole 3\Rel{}-7\Sm{}-go.out-\Pst{} 2-rat 2\Sm{}.\Pst{}-fill-\Pst{}\\
%%TRANS1
\trans ‘The hole where the rats went out was filled.’
%%TRANS2
%%EXEND

\end{exe}
\begin{exe}
%%EAX
\ex
%%JUDGEMENT
%%LABEL
\label{126}
%%CONTEXT
 (Context: in a story, a candle was extinguished due to some unclear reason...)\\
%%LINE2
\gll
Níŋáa \textbf{báá}-dzíib-i \textbf{bu-dzí}.\\
%%LINE3
suddenly 2\Sm{}.\Pst{}-extinguish-\Pst{} 14-candle\\
%%TRANS1
\trans ‘Suddenly the candle was extinguished.’ \jambox*{\citep[31]{Li2020}}
%%TRANS2
%%EXEND

\end{exe}
\begin{exe}
%%EAX
\ex
%%JUDGEMENT
%%LABEL
\label{127}
%%CONTEXT
%%LINE2
\gll
\textbf{Báá}-tí \textbf{ndé} bví ku mbali.\\
%%LINE3
2\Sm{}.\Pst{}-throw.\Pst{} 1.\Pro{} 9.falling 17.\Loc{} 9.outside\\
%%TRANS1
\trans ‘It was thrown outside.’ \jambox*{(Saint Matthieu V:13)}
%%TRANS2
%%EXEND

\end{exe}
In example \xref{128} we see from the context that the agent should be `I', and the subject marking on the verb is still \textit{ba}-, which shows the impersonal nature of the class 2 subject marker in this construction.
\begin{exe}
%%EAX
\ex
%%JUDGEMENT
%%LABEL
\label{128}
%%CONTEXT
 (Context: you cut some firewood in the morning and you gave it to your brother who could not work.) \\
%%LINE2
\gll
\textbf{Nkwíi} yi-m-baal-í me \textbf{báá}-wî ngândukú aa me.\\
%%LINE3
9.firewood 9\Rel{}-1\Sg{}.\Sm{}-cut-\Pst{} 1\Sg{}.\Pro{} 2\Sm{}.\Pst{}-give.\Pst{} 1.brother 1.\Conn{} 1\Sg{}.\Pro{}\\
%%TRANS1
\trans ‘The firewood that I cut was given to my brother.’
%%TRANS2
%%EXEND

\end{exe}
Example \xref{129} shows that both objects of the ditransitive verb can be preposed in this functional passive construction. Interestingly, from the context we see that the preverbal theme object is topical in \xref{129a}, while in \xref{129b} the preverbal recipient object is focal. The different discourse status of the preverbal object here is reminiscent of the information structure of the preverbal subject discussed in \sectref{teke:sec:3.1.2}. I propose that the preverbal objects in \xref{129} occupy different structural positions; the preverbal object in \xref{129b} is in the IBV focus position. In this regard, the preverbal DP of the \textit{ba}- construction behaves more like a preverbal subject which can be either topical or focal.
\begin{exe}
    \ex (To whom did mother give the keys?)\label{129}
    \begin{xlist}
%%EAX
\ex
%%JUDGEMENT
%%LABEL
\label{129a}
%%CONTEXT
%%LINE2
\gll
\textbf{Ma-fungúla} báá-wî taará.\\
%%LINE3
6-key 2\Sm{}.\Pst{}-give.\Pst{} 1.father\\
%%TRANS1
\trans ‘The keys were given to father.’
%%TRANS2
%%EXEND

%%EAX
\ex
%%JUDGEMENT
%%LABEL
\label{129b}
%%CONTEXT
%%LINE2
\gll
\textbf{Taará} báá-wî ma-fungúla.\\
%%LINE3
1.father 2\Sm{}.\Pst{}-give.\Pst{} 6-key\\
%%TRANS1
\trans ‘Father was given the keys.’
%%TRANS2
%%EXEND

    \end{xlist}
\end{exe}
In example \xref{130} we see that the \textit{ba}- construction is used when the preverbal DP is contrastively focused and placed in the IBV position, and \xref{131} shows that a preverbal interrogative word can occur in the \textit{ba}- construction. In this sense, the \textit{ba}- construction also shows deviance from the canonical passive construction in that the functionally ``passivised" element is not always topical but can also be focal.  
\begin{exe}
    \ex \label{130}
    \begin{xlist}
%%EAX
\ex
%%JUDGEMENT
%%LABEL
\label{130a}
%%CONTEXT
%%LINE2
\gll
\textbf{Bi-ko} bvi-kí-dzílík-í mú-kái ku ngulu aa nzó báá-túr-i.\\
%%LINE3
8-clothes 8\Rel{}-7\Sm{}-reserve-\Pst{} 1-woman 17.\Loc{} 9.inside 9.\Conn{} 9.house 2\Sm{}.\Pst{}-steal-\Pst{}\\
%%TRANS1
\trans ‘The clothes that the lady kept in the house were stolen.’
%%TRANS2
%%EXEND

%%EAX
\ex
%%JUDGEMENT
%%LABEL
\label{130b}
%%CONTEXT
%%LINE2
\gll
Ambú, ndé \textbf{mí-pará} \textbf{báá}-túr-i.\\
%%LINE3
no 1.\Pro{} 4-money 2\Sm{}-steal-\Pst{}\\
%%TRANS1
\trans ‘No, her \textit{money} was stolen.’
%%TRANS2
%%EXEND

    \end{xlist}
\end{exe}
\begin{exe} 
%%EAX
\ex
%%JUDGEMENT
%%LABEL
\label{131}
%%CONTEXT
 (Context: you found that the bananas on the table disappeared, and you asked father.)\\
%%LINE2
\gll
Ma-ko \textbf{ná} báá-wî?\\
%%LINE3
6-banana 1.who 2\Sm{}.\Pst{}-give.\Pst{}\\
%%TRANS1
\trans ‘The bananas were given to whom?’/‘Who was given the bananas?’
%%TRANS2
%%EXEND

\end{exe}
This functional passive construction with class 2/3rd person plural subject marking is actually commonly attested in Bantu languages and beyond \citep{Frajzyngier1982, KeenanDryer2007, CobbinahLüpke2009}. A number of Bantu languages such as Bàsàá \citep{HamlaouiMakasso2013}, Mbuun \citep{BostoenMundeke2011}, Bemba \citep{KulaMarten2010}, Lunda \citep{Kawasha2007} and Matengo \citep{vanderWal2015} have reported this construction as a functional passive. In all these languages the patient can either precede or follow the verb in this construction. As for the agent, in Bàsàá, Mbuun and Matengo, it is always unspecified and can not be present even via an oblique phrase, while in Bemba and Lunda an oblique agent is allowed and even preferred. In Kukuya, the agent is usually deleted but sometimes it can be introduced by an oblique phrase headed by a class 18 locative pronoun \textit{mu}. However, two situations need to be distinguished.

There are some cases in which the DP introduced by the oblique phrase seems to be the agent of the verb, as shown in \xref{132} and \xref{133}. Though these expressions are considered to be quite marginal and rare in use, the speakers often give the active constructions as equivalent translations to them. However, \xref{132} and \xref{133} can be used in various contexts in which the DP in the oblique phrase does not necessarily function as the agent but rather a ``causer" of the event. In \xref{132} the context can be that someone else gave the child the orange due to father's commission or network, while in \xref{133} it was not necessarily your wife who caught you but perhaps your wife reported you to the police or you committed a crime due to your wife. Given that the class 18 pronoun often introduces a reason, here the oblique phrases in these two examples should be interpreted as reason phrases rather than the demoted agents.
\begin{exe}
%%EAX
\ex
%%JUDGEMENT
%%LABEL
\label{132}
%%CONTEXT
%%LINE2
\gll
Mwáana báá-wî láala \textbf{mu} \textbf{taará}.\\
%%LINE3
1.child 2\Sm{}.\Pst{}-give.\Pst{} 5.orange 18.\Loc{} 1.father\\
%%TRANS1
\trans ‘The child was given an orange because of father.’
%%TRANS2
%%EXEND

\end{exe}
\begin{exe}
%%EAX
\ex
%%JUDGEMENT
%%LABEL
\label{133}
%%CONTEXT
%%LINE2
\gll
Me báá-siib-i \textbf{mu} \textbf{mu-káli}.\\
%%LINE3
1\Sg{}.\Pro{} 2\Sm{}.\Pst{}-catch-\Pst{} 18.\Loc{} 1.wife\\
%%TRANS1
\trans ‘I was caught because of the wife.’
%%TRANS2
%%EXEND

\end{exe}
The \textit{ba}- construction with an oblique phrase cannot be a felicitous answer to a subject question. To answer the subject question in \xref{134a}, the OSV structure in \xref{134b} is the answer par excellence, while \xref{134c} is infelicitous here because the oblique phrase can only be interpreted as a purpose or a reason. The question-answer congruence may also have some effect here, since the question in \xref{134a} does not involve the \textit{ba}- construction, \xref{134c} is not expected to be a felicitous answer.
\begin{exe}
    \ex \label{134}
    \begin{xlist}
%%EAX
\ex
%%JUDGEMENT
[]{
%%LABEL
\label{134a}
%%CONTEXT
%%LINE2
\gll
Nzó yǐ yá mú-táliki \textbf{ná} ndé áá-tsú-i?\\
%%LINE3
9.house 9\Rel{} with 3-height 1.who 1.\Pro{} 1\Sm{}.\Pst{}-build-\Pst{}\\
%%TRANS1
\trans ‘The tall building was built by whom?’
%%TRANS2
}
%%EXEND

%%EAX
\ex
%%JUDGEMENT
[]{
%%LABEL
\label{134b}
%%CONTEXT
%%LINE2
\gll
Yó \textbf{míi-ndéle} míí-tsú-i.\\
%%LINE3
9.\Pro{} 4-foreigner 4\Sm{}.\Pst{}-build-\Pst{}\\
%%TRANS1
\trans ‘It was built by the foreigners.’
%%TRANS2
}
%%EXEND

%%EAX
\ex
%%JUDGEMENT
[\#]{
%%LABEL
\label{134c}
%%CONTEXT
%%LINE2
\gll
Yó báa-tsú-i \textbf{mu} \textbf{mii-ndéle}.\\
%%LINE3
9.\Pro{} 2\Sm{}-build-\Pst{} 18.\Loc{} 4-foreigner\\
%%TRANS1
\trans int. ‘It was built by the foreigners.’\\
%%TRANS2
‘It was built for/because of the foreigners.’    \jambox*{\citep[30]{Li2020}}
}
%%EXEND


\end{xlist}
\end{exe} 
Based on all these examples on the oblique phrase in the \textit{ba}- construction, I would rather conclude that the DP introduced by \textit{mu} is never a true agent but can only function as a reason, a purpose or a method, though sometimes it can be ambiguously interpreted as the agent. In this sense, it is more plausible to still label the \textit{ba}- construction as impersonal. Compared to the OSV structure, the \textit{ba}- construction is used when the agent is unspecified or there is no need to express it.

To summarise, I have presented two functional passive constructions in Kukuya, namely the OSV and the impersonal \textit{ba}- construction. Both of the constructions can serve as the translational equivalent of a prototypical passive structure. However, their syntactic and pragmatic properties differ from each other and also from the prototypical passive. The OSV construction is used when the utterance is about the patient and the agent needs to be explicitly expressed as new or contrasted information. The impersonal \textit{ba}- construction looks more similar to the canonical passive as the agent is usually deleted, but the preverbal object can either be topical or focal, which differs from the canonical passivised element. Here these two constructions only partially overlap with some properties of the canonical passive construction and can only be treated as functional equivalents.  


In this section I have shown that different types of topic expressions tend to occur in the preverbal domain in Kukuya. The topic expressions can be divided into primary and secondary topics: a primary topic often includes the topical subject or the scene-setting topics, which occur sentence-initially; a secondary topic is non-initial and is often attested in the preverbal domain accompanied by the IBV focus position being occupied. Two functional passive constructions are presented, namely the OSV construction and the impersonal \textit{ba}-construction, which are used in different pragmatic contexts and both make use of the topic fronting tendency and the IBV focus strategy to express passive. A scheme on the mapping of word order and information structure of Kukuya is illustrated in \figref{fig:teke:fromex:135}. In the next section I introduce cleft constructions and their connection with the IBV focus strategy.

\begin{figure}
\caption{A scheme of information structure and word order in Kukuya}
\label{fig:teke:fromex:135}
\fbox
{\shortstack[l]{
    \textbf{scene-setting TOP}\\
    non-argument}}
\fbox
{\shortstack[l]{
    \textbf{subject TOP}\\
    argument}}
    \fbox
{\shortstack[l]{
    \textbf{secondary TOP}\\
    argument}}
\\
\fbox
{\shortstack[l]{
    \textbf{FOC}\\
    argument/adjunct}}
\textbf{VERB}
\fbox
{\shortstack[l]{
    non-TOP/non-FOC\\
    argument/adjunct}}
\end{figure}

\section{Cleft constructions}\label{teke:sec:5}
Clefts are one of the well-known focus marking devices in Bantu languages \citep{Demuth1987, SabelZeller2006, ChengDowning2013, HamlaouiMakasso2015, LafkiouiEtAl2016}. In this section I present different types of cleft constructions in Kukuya and their functions in information packaging. I first give a description on the formation and interpretation of the basic cleft and (reverse) pseudo-cleft constructions, then I introduce a special construction that I label as a ``reduced" cleft. I also propose and show evidence that the IBV focus construction in this language is very likely to have its origin in the cleft construction, and different intermediate grammaticalisation stages can be identified.
\subsection{Basic cleft and pseudo-cleft constructions}\label{teke:sec:5.1}
As for a basic cleft, here I refer to a construction parallel to the English sentence `it was a pancake that we ate', and it can also be labeled as the \textit{it}-cleft. Syntactically, a basic cleft usually consists of two clauses: one contains a nominal predicate and one contains a free relative clause. The focus reading arises from the combination of the relative clause and the nominal predicate. The relative part of the cleft is presented as the maximal group of referents to
which the predicate applies and is equated to the referent in the nominal predicate, and in this way an identificational and exclusive focus reading is rendered \citep{vanderWalManiacky2015}.

In Kukuya, a basic cleft can be used to express focus on arguments and adjuncts. An example of a basic cleft in Kukuya that fulfills all the syntatic properties mentioned above is illustrated in \xref{136}. We see that the sentence contains an initial copula that takes a default class 7 subject marker, a nominal predicate that takes a H tone prefix and a relative clause with segmental relative marking. In fact, this kind of ``complete" cleft construction is never uttered in natural speech. The copula is usually omitted in affirmative sentences, so a cleft construction in Kukuya is mostly formed just by a nominal predicate followed by a free relative clause. In \xref{136} the focus is on the clefted object that occurs in the initial position, and the following relative clause is used to exclusively identify it. When using a cleft as in \xref{136}, the speaker intends to express that the person only bought a shelf but nothing else.
\begin{exe}
%%EAX
\ex
%%JUDGEMENT
%%LABEL
\label{136}
%%CONTEXT
%%LINE2
\gll
(Kí-li) \textbf{kí-taabí} ki-káá-fúúm-í ndé ku dzándu.\\
%%LINE3
7\Sm{}.\Pst{}-\Cop{} 7-shelf 7\Rel{}-1\Sm{}.\Pst{}-buy-\Pst{} 1.\Pro{} 17.\Loc{} 5.market\\
%%TRANS1
\trans ‘It was a \textit{shelf} that s/he bought at the market.’
%%TRANS2
%%EXEND

\end{exe}
In example \xref{137a} we see that the object cleft sentence can only be a proper answer to an object question but not to a VP question, so apparently the focus reading cannot be extended to a larger constituent in a cleft. We also see that \xref{137a} cannot be continued with an additive sentence such as `and also some sheep', showing that the cleft sentence expresses exclusive focus. In \xref{137b} we see that in the negative counterpart of the cleft sentence, the copula shows up and hosts the negative prefix. Here the scope of negation is not the whole sentence but only the focus. A subject cleft sentence is given in \xref{138}. In all these examples, the clefted arguments receive an exclusive focus reading.
\begin{exe}
    \ex \label{137}
    \begin{xlist}
%%EAX
\ex
%%JUDGEMENT
%%LABEL
\label{137a}
%%CONTEXT
%%LINE2
\gll
Báa-ntaba ba-kí-fúúm-í mú-kái.\\
%%LINE3
2-goat 2\Rel{}-7\Sm{}.\Pst{}-buy-\Pst{} 1-woman\\
%%TRANS1
\trans ‘It was some \textit{goats} that the woman bought.’ \\ \jambox*{`\textit{What did the woman buy?}' \cmark}
\jambox*{`\textit{What did the woman do?}' \xmark} \jambox*{`\textit{...and also some sheep}' \xmark}
%%TRANS2
%%EXEND

%%EAX
\ex
%%JUDGEMENT
%%LABEL
\label{137b}
%%CONTEXT
%%LINE2
\gll
Ka-kí-li báa-ntaba ba-kí-fúúm-í mú-kái ni.\\
%%LINE3
\Neg{}-7\Sm{}.\Pst{}-\Cop{} 2-goat 2\Rel{}-7\Sm{}.\Pst{}-buy-\Pst{} 1-woman \Neg{}\\
%%TRANS1
\trans ‘It was not some \textit{goats} that the woman bought.’
%%TRANS2
%%EXEND

    \end{xlist}
\end{exe}
\begin{exe}
%%EAX
\ex
%%JUDGEMENT
%%LABEL
\label{138}
%%CONTEXT
%%LINE2
\gll
Wúna mvá wu-á-wî baa-ntaba buókó.\\
%%LINE3
only 1.dog 1\Rel{}-1\Sm{}-give.\Pst{} 2-goat 14.fear\\
%%TRANS1
\trans ‘It was only the dog who scared the goats.’
%%TRANS2
%%EXEND

\end{exe}
A pseudo-cleft refers to a construction that equates the referent of a headless relative clause with a nominal predicate, for example the English sentence `what we want is pizza', and is also known as \textit{wh}-cleft. The pseudo-cleft construction seems to be more frequently attested in Kukuya than the basic cleft and is usually used to express subject focus (see \sectref{teke:sec:3.1.2}), as shown in \xref{139}. In \xref{140} the alternative question begins with a dislocated topic \textit{mu-kái} `woman' and is followed by a pseudo-cleft construction sentence with the predicative focal object at the end. 
\begin{exe}
    \ex \label{139}
    \begin{xlist}
%%EAX
\ex
%%JUDGEMENT
%%LABEL
\label{139a}
%%CONTEXT
%%LINE2
\gll
Ki-kí-túm-í mbaá ki-namá kí-ma?\\
%%LINE3
7\Rel{}-7\Sm{}-cause-\Pst{} 9.fire \Inf{}-burn 7-what\\
%%TRANS1
\trans ‘What caused the fire?’
%%TRANS2
%%EXEND

%%EAX
\ex
%%JUDGEMENT
%%LABEL
\label{139b}
%%CONTEXT
%%LINE2
\gll
Baá-fúum-i ma-li ba-na?\\
%%LINE3
2\Rel{}-buy-\Pst{} 6-wine 2-who\\
%%TRANS1
\trans ‘Who (\textit{pl.}) bought the wine?’
%%TRANS2
%%EXEND

%%EAX
\ex
%%JUDGEMENT
%%LABEL
\label{139c}
%%CONTEXT
%%LINE2
\gll
Wǔ-dzí baa-ntsúú ka-kí-li mvá ni.\\
%%LINE3
1\Rel{}-eat.\Pst{} 2-chicken \Neg{}-7\Sm{}-\Cop{} 1.dog \Neg{}\\
%%TRANS1
\trans ‘(The one) who ate the chicken was not the dog.’
%%TRANS2
%%EXEND

    \end{xlist}
\end{exe}
\begin{exe}
    \ex \label{140}
    \begin{xlist}
%%EAX
\ex
%%JUDGEMENT
%%LABEL
\label{140a}
%%CONTEXT
%%LINE2
\gll
Mu-kái wu-ká-búr-í ndé \textbf{mú-kái} wó \textbf{balaka}?\\
%%LINE3
1-woman 1\Rel{}-1\Sm{}.\Pst{}-give.birth-\Pst{} 1.\Pro{} 1-female or 1.male\\
%%TRANS1
\trans ‘The woman gave birth to a girl or a boy?’\\lit.: `The woman, to whom she gave birth was a girl or a boy?'
%%TRANS2
%%EXEND

%%EAX
\ex
%%JUDGEMENT
%%LABEL
\label{140b}
%%CONTEXT
%%LINE2
\gll
Wu-ká-búr-í ndé \textbf{balaka}.\\
%%LINE3
1\Rel{}-1\Sm{}-give.birth-\Pst{} 1.\Pro{} 1.male\\
%%TRANS1
\trans ‘The one she gave birth to was a boy.’
%%TRANS2
%%EXEND

    \end{xlist}
\end{exe}
In \xref{141} a reverse pseudo-cleft sentence is illustrated. Here again we see that the reverse pseudo-cleft cannot be continued by a sentence like `and also some sheep', which shows that it expresses exclusive focus. 
\begin{exe}
%%EAX
\ex
%%JUDGEMENT
%%LABEL
\label{141}
%%CONTEXT
%%LINE2
\gll
Báa-ntaba (bá-li) ba-kí-fúúm-í mú-kái.\\
%%LINE3
2-goat {\db}2\Sm{}.\Pst{}-\Cop{} 2\Rel{}-7\Sm{}.\Pst{}-buy-\Pst{} 1-woman\\
%%TRANS1
\trans ‘The \textit{goats} were what the woman bought.’ \\\jambox*{`\textit{...and also some sheep}' \xmark }
%%TRANS2
%%EXEND

\end{exe}
There is also a commonly seen construction which surfaces in the OSV word order and in which the focus is placed on the initial element, as illustrated in \xref{142}. I would analyse this construction as a somehow ``reduced" version of a basic cleft rather than a monoclausal construction with initial focus, for the reasons that will become clear shortly. This cleft construction is reduced in the sense that there is no segmental relative marker on the verb, but there are clues of relative marking. In \xref{142} we see that the class 1 subject marking on the verb takes the form \textit{ka}- rather than the canonical form \textit{a}-, which is an indicator of relative marking on the verb. This construction is a natural way of expressing exclusive focus on the initial element but never on the whole VP, which corresponds more to the cleft construction than the IBV focus construction. Prosodically, the initial focused element is always independently phrased from the rest of the sentence, which can also show evidence for the cleft nature of this construction \citep{ChengDowning2013}. Therefore, I label this construction as a reduced cleft and will hypothesise that it can reflect an intermediate stage of the grammaticalisation process from the cleft to the IBV focus strategy. It is worth noting that this construction should be distinguished from the OSV construction presented in \sectref{teke:sec:4.2.1} in which the focus is in IBV, though they have the same linear word order.
\begin{exe}
%%EAX
\ex
%%JUDGEMENT
%%LABEL
\label{142}
%%CONTEXT
%%LINE2
\gll
Má-biríki taará káá-fúum-i ku mfaí.\\
%%LINE3
6-brick 1.father 1\Sm{}.\Pst{}-buy-\Pst{} 17.\Loc{} 9.capital \\
%%TRANS1
\trans ‘It was some bricks that father bought from Brazzaville.’\jambox*{`\textit{...and also a motorbike}' \xmark }
%%TRANS2
%%EXEND

\end{exe}
Some more examples of this reduced cleft construction are given in \xref{143} and \xref{144}. The construction is most commonly attested as a content question as in \xref{143}, in which the speakers place the interrogative word at the start of the sentence. In \xref{144} the focus is on the quantifier of the initial NP, while the whole NP occurs in the initial position. The reduced cleft is only discernible when the initial focused element is a non-subject, since a reduced subject cleft cannot be distinguished from the canonical word order when there is no relative marker, no subject marking allomorphy or word order change.
\begin{exe}
%%EAX
\ex
%%JUDGEMENT
%%LABEL
\label{143}
%%CONTEXT
%%LINE2
\gll
\textbf{Munkí} mwáana káa-dzí ntsúi?\\
%%LINE3
when 1.child 1\Sm{}.\Pst{}-eat.\Pst{} 1.fish\\
%%TRANS1
\trans ‘When did the child eat fish?’
%%TRANS2
%%EXEND

\end{exe}
\begin{exe}
%%EAX
\ex
%%JUDGEMENT
%%LABEL
\label{144}
%%CONTEXT
 (Context: the thief would have stolen more goats, but it was only a /textit{few}.)\\
%%LINE2
\gll
Baa-ntaba \textbf{bá}-bíibi {\textltailm}íibi káá-túr-i.\\
%%LINE3
2-goat 2-few 1.thief 1\Sm{}.\Pst{}-steal-\Pst{}\\
%%TRANS1
\trans ‘The thief stole \textit{few} goats.’
%%TRANS2
%%EXEND

\end{exe}
In this subsection I have presented three main types of cleft constructions in Kukuya, namely the basic cleft, (reverse) pseudo-cleft and the reduced cleft. I showed that all these constructions express exclusive focus on the clefted element. Some further research need to be carried out on the pragmatic distinctions on the cleft construction and the IBV focus strategy when they both express exclusive focus. 
\subsection{Connection between IBV focus and cleft constructions}\label{teke:sec:5.2}
In this subsection I provide some evidence on the connection between the cleft and the IBV focus constructions, proposing that the IBV focus strategy has its origin in the cleft. Apart from identification focus interpretation that they are both used to express, I take evidence from some shared grammatical properties on the two constructions, namely the H tone prefix on the focused NP, subject marking alternation, the verb-final H tone and the negation strategy. Next I illustrate each point with examples from both constructions.

The hypothesis that the IBV focus strategy originates from a bi-clausal cleft construction was first made in \citet{DeKind2014} in his analysis on the preverbal focus strategy in Kisikongo. One argument he  provided was that in the relative clause of the cleft construction and the SOV word order, the class 1 subject takes the same allomorphic subject prefix \textit{ka}-, and this is also attested in Kukuya. In example \xref{145} we see that in a subject relative the preverbal class 1 pronoun takes the subject marker \textit{a}- on the verb; in a non-subject relative the postverbal class 1 pronominal subject takes the subject marker \textit{ka}-. In \xref{146} as well as in many examples above, we see that when a non-subject constituent is focused in IBV position, the class 1 subject marker alternates from \textit{a}- to \textit{ka}-.
\begin{exe}
    \ex \label{145}
    \begin{xlist}
%%EAX
\ex
%%JUDGEMENT
%%LABEL
\label{145a}
%%CONTEXT
%%LINE2
\gll
Ndé wu-\textbf{á}-banám-i áá-tok-í ndziimi.\\
%%LINE3
1.\Pro{} 1\Rel{}-1\Sm{}.\Pst{}-wake.suddenly-\Pst{} 1\Sm{}.\Pst{}-sweat-\Pst{} much\\
%%TRANS1
\trans ‘S/He who woke up suddenly sweated a lot.’ \\\jambox*{[subject relative]}
%%TRANS2
%%EXEND

%%EAX
\ex
%%JUDGEMENT
%%LABEL
\label{145b}
%%CONTEXT
%%LINE2
\gll
Ki-sáli ki-\textbf{ká}-lil-í \textbf{ndé} ka-kí-li tsítse ni.\\
%%LINE3
7-reason 7\Rel{}-1\Sm{}.\Pst{}-cry-\Pst{} 1.\Pro{} \Neg{}-7\Sm{}.\Pst{}-\Cop{} clear \Neg{}\\
%%TRANS1
\trans ‘The reason why s/he cried was not clear.’ \jambox*{[nonsubject relative]}
%%TRANS2
%%EXEND

    \end{xlist}
\end{exe}
\begin{exe}
%%EAX
\ex
%%JUDGEMENT
%%LABEL
\label{146}
%%CONTEXT
%%LINE2
\gll
Ndé má-\textbf{láálá} \textbf{káá}-fúum-i.\\
%%LINE3
1.\Pro{} 6-orange 1\Sm{}.\Pst{}-buy-\Pst{}\\
%%TRANS1
\trans ‘S/He bought the \textit{oranges}.’
%%TRANS2
%%EXEND

\end{exe}
In \citet{NdongaMfuwa1995}'s description on Kisikongo relatives and also noted in \citet{DeKind2014}, the subject maker on relative verbs and in the SOV word order bears a H tone, which can corroborate the connection between the preverbal focus and the cleft construction. In Kukuya, the 1\Pl{} subject marker in the remote past tense in the SVO word order is \textit{lii}- with a L tone, and in the relative construction \xref{147} and the SOV order \xref{148} below we see that the 1\Pl{} subject marker is realised as \textit{líi}- in which a H tone is inserted. I assume that this is a grammatical H tone that marks relative, and in these two examples we see again that the verb form in the IBV focus construction has retained some relative residue.
\begin{exe}
%%EAX
\ex
%%JUDGEMENT
%%LABEL
\label{147}
%%CONTEXT
%%LINE2
\gll
li-meé li-\textbf{líi}-li \textbf{líi}-tí \textbf{bhií} bví\\
%%LINE3
5-stone 5\Rel{}-1\Pl{}.\Rpst{}-\Cop{} 1\Pl{}.\Rpst{}-launch.\Pst{} 1\Pl{}.\Pro{} 9.falling\\
%%TRANS1
\trans ‘the stone that we had thrown away’
%%TRANS2
%%EXEND

\end{exe}
\begin{exe}
%%EAX
\ex
%%JUDGEMENT
%%LABEL
\label{148}
%%CONTEXT
%%LINE2
\gll
Bhií \^{a}li kí-má \textbf{líi}-fúum-i?\\
%%LINE3
1\Pl{}.\Pro{} \Pst{} 7-what 2\Pl{}.\Pst{}-buy-\Pst{}\\
%%TRANS1
\trans ‘What had we bought?’
%%TRANS2
%%EXEND

\end{exe}
Another piece of tonal evidence that can show the relative origin of the verb in the IBV focus construction is the verb-final H tone attested in non-subject extraction. In the subject relative in \xref{149a}, the tone on the verb stem is realised as HL as in its citation form, while in the non-subject relative in \xref{149b} the tone on the verb stem is realised as H and is carried over onto the the prefix of the postverbal subject NP. Here it shows that there is an emergent grammatical H tone occurring verb-finally when a non-subject argument is relativised.
\begin{exe}
    \ex \label{149}
    \begin{xlist}
%%EAX
\ex
%%JUDGEMENT
%%LABEL
\label{149a}
%%CONTEXT
%%LINE2
\gll
mu-kái wǔ-fúum-\textbf{i} \textbf{mi}-féme\\
%%LINE3
1-woman 1\Rel{}-buy-\Pst{} 4-pig\\
%%TRANS1
\trans ‘the woman who bought the pigs’ \jambox*{[subject relative]}
%%TRANS2
%%EXEND

%%EAX
\ex
%%JUDGEMENT
%%LABEL
\label{149b}
%%CONTEXT
%%LINE2
\gll
mi-féme mi-kí-fúúm-\textbf{í} \textbf{mú}-kái\\
%%LINE3
4-pig 4\Rel{}-7\Sm{}.\Pst{}-buy-\Pst{} 1-woman\\
%%TRANS1
\trans ‘the pigs that the woman bought’ \jambox*{[non-subject relative]}
%%TRANS2
%%EXEND

    \end{xlist}
\end{exe}
When a non-subject element is focused in the IBV position, the verb-final H tone is also attested. Since in Kukuya there is an utterance-final tone lowering rule and the verb tends to occur in the right boundary of the clause in the IBV focus construction, the verb-final H tone is only attested when there is a non-focal postverbal element in the IBV focus construction. In \xref{150a} the subject can be either topical/non-focal or focal and the tone pattern of the verb is HL with no final H tone; in \xref{150b} the theme object is focused in IBV, and the tone pattern on the verb is realised as H and is spread onto the prefix of the following recipient object. Similarly in \xref{151}, when the locative phrase is focused in IBV, the tone pattern on the verb shifts from HL to H and the H tone is again carried over onto the following prefix. Here again the tone pattern on the verb in the IBV focus construction clearly shows relative properties.
\begin{exe}
    \ex \label{150}
    \begin{xlist}
%%EAX
\ex
%%JUDGEMENT
%%LABEL
\label{150a}
%%CONTEXT
%%LINE2
\gll
Taará áá-w\textbf{î} \textbf{ba}a-ndzulí ma-désu.\\
%%LINE3
1.father 1\Sm{}.\Pst{}-give.\Pst{} 2-cat 6-bean\\
%%TRANS1
\trans ‘Father gave beans to the cats.’
%%TRANS2
%%EXEND

%%EAX
\ex
%%JUDGEMENT
%%LABEL
\label{150b}
%%CONTEXT
%%LINE2
\gll
Taará lóoso káá-w\textbf{í} \textbf{bá}a-nzulí.\\
%%LINE3
1.father 5.rice 1\Sm{}.\Pst{}-give.\Pst{} 2-cat\\
%%TRANS1
\trans ‘Father gave the \textit{rice} to the cats.’
%%TRANS2
%%EXEND

    \end{xlist}
\end{exe}
\begin{exe}
%%EAX
\ex
%%JUDGEMENT
%%LABEL
\label{151}
%%CONTEXT
 (Where did father buy the wine?) \\
%%LINE2
\gll
Ndé ku dzándú káá-fúúm-\textbf{í} \textbf{má}-lí.\\
%%LINE3
1.\Pro{} 17.\Loc{} 5.market 1\Sm{}.\Pst{}-buy-\Pst{} 6-wine\\
%%TRANS1
\trans ‘He bought the wine \textit{at the market}.’
%%TRANS2
%%EXEND

\end{exe}
So far I have shown that the verb form in the IBV focus construction resembles that in the relative clause. Now we consider how the predicative part of the cleft construction is associated with the IBV focus strategy. In many examples above we have seen that the prefix of a focused NP in the IBV position always bears a H tone. In example \xref{152a} the postverbally focused NP does not have a H tone prefix, while in \xref{152b} the IBV focused NP does. Similarly in \xref{153}, we see that while there are three preverbal NPs in this sentence, only the focal NP in the IBV position has a H tone prefix.
\begin{exe}
    \ex (What did father buy?)\label{152}
    \begin{xlist}
%%EAX
\ex
%%JUDGEMENT
%%LABEL
\label{152a}
%%CONTEXT
%%LINE2
\gll
Ndé áá-fúum-i \textbf{ma}-láala.\\
%%LINE3
1.\Pro{} 1\Sm{}.\Pst{}-buy-\Pst{} 6-orange\\
%%TRANS1
\trans ‘He bought some oranges.’
%%TRANS2
%%EXEND

%%EAX
\ex
%%JUDGEMENT
%%LABEL
\label{152b}
%%CONTEXT
%%LINE2
\gll
Ndé \textbf{má-láálá} káá-fúum-i.\\
%%LINE3
1.\Pro{} 6-orange 1\Sm{}.\Pst{}-buy-\Pst{}\\
%%TRANS1
\trans ‘He bought some \textit{oranges}.’
%%TRANS2
%%EXEND

    \end{xlist}
\end{exe}
\begin{exe}
%%EAX
\ex
%%JUDGEMENT
%%LABEL
\label{153}
%%CONTEXT
 (Did the woman give the fish to the \textit{dogs}?)\\
%%LINE2
\gll
Mu-kái baa-ntsúi \textbf{bá}a-ndzuli káá-wî.\\
%%LINE3
1-woman 2-fish 2-cat 1\Sm{}.\Pst{}-give.\Pst{}\\
%%TRANS1
\trans ‘The woman gave the fish to the \textit{cats}.’
%%TRANS2
%%EXEND

\end{exe}
In Kukuya, a predicative NP also has a H tone prefix. Since the copula is omitted in affirmative sentences, a predicative construction is usually expressed by juxtaposition of two NPs and the predicative one is marked by a H tone prefix, as illustrated in examples \xref{154} and \xref{155}.
\begin{exe}
%%EAX
\ex
%%JUDGEMENT
%%LABEL
\label{154}
%%CONTEXT
%%LINE2
\gll
Ndé \textbf{mú}-tsúli.\\
%%LINE3
1.\Pro{} 1-goldsmith\\
%%TRANS1
\trans ‘S/He is a goldsmith.’ \jambox*{[cf. \textit{mu-tsúli} `goldsmith']}
%%TRANS2
%%EXEND

\end{exe}
\begin{exe}
%%EAX
\ex
%%JUDGEMENT
%%LABEL
\label{155}
%%CONTEXT
%%LINE2
\gll
Ki-báka \textbf{kí}-báka, bu-bila.nkele \textbf{mú}u-nkwáárá.\\
%%LINE3
7-obtain 7-obtain 14-question 3-keeping\\
%%TRANS1
\trans ‘To obtain is to obtain, the question is (how) to keep.’     \jambox*{\citep[194]{Paulian1975}}   
%%TRANS2
%%EXEND

\end{exe}

In the (pseudo-)cleft constructions presented in \sectref{teke:sec:5.1}, we have seen that the nominal predicate always has a H tone prefix, which is also shown in the pseudo-cleft in example \xref{156}. Since the nominal predicate is usually the focused part in a (pseudo-)cleft, I propose that the H tone prefix of the IBV focused element corresponds to that in the cleft construction. Now we have seen the association of the IBV focus construction with both the relative clause part and the nominal predication part of the cleft.
\begin{exe}
%%EAX
\ex
%%JUDGEMENT
%%LABEL
\label{156}
%%CONTEXT
%%LINE2
\gll
Kǐ-n-dzií me ki-nywâ \textbf{má}-dzá maa-mfé.\\
%%LINE3
7\Rel{}-1\Sg{}.\Sm{}-please 1\Sg{}.\Pro{} \Inf{}-drink 6-water 6-cold\\
%%TRANS1
\trans ‘What I like to drink is cold water.’
%%TRANS2
%%EXEND

\end{exe}
To negate the focused element in the IBV focus construction, a copula with the negative prefix often appears immediately before the focused element, as shown in examples \xref{157} and \xref{158}. The occurrence of the copula can corroborate the predicative origin of the IBV focused element and the cleft origin of the IBV focus construction.
\begin{exe}
%%EAX
\ex
%%JUDGEMENT
%%LABEL
\label{157}
%%CONTEXT
%%LINE2
\gll
Ngwangúlu ka-kí-li \textbf{mvá} áá-dzí ni.\\
%%LINE3
1.gecko \Neg{}-7\Sm{}-\Cop{} 1.dog 1\Sm{}.\Pst{}-eat.\Pst{} \Neg{}\\
%%TRANS1
\trans ‘The gecko was not eaten by the \textit{dog}.’/`The \textit{dog} did not eat the gecko.'
%%TRANS2
%%EXEND

\end{exe}
\begin{exe}
%%EAX
\ex
%%JUDGEMENT
%%LABEL
\label{158}
%%CONTEXT
%%LINE2
\gll
Taará ka(-kí-li) \textbf{ntáli} káá-sí me ni.\\
%%LINE3
1.father \Neg{}-7\Sm{}-\Cop{} 9.bed 1\Sm{}.\Pst{}-make.\Pst{} 1\Sg{}.\Pro{} \Neg{}\\
%%TRANS1
\trans ‘Father did not make me a \textit{bed}.’
%%TRANS2
%%EXEND

\end{exe}
Based on all the shared grammatical properties of the IBV focus strategy and the cleft construction illustrated above, I propose that the IBV focus construction has its origin in the cleft construction. In this chapter I do not discuss in detail the grammaticalisation process of the IBV focus construction, but only propose some important intermediate stages in the grammaticalisation. In \xref{159} there are three parallel constructions that co-exist in this language, and I suppose that diachronically the \xref{159b} is an intermediate construction derived from \xref{159a} whereby the relative marker in the cleft is deleted and the postverbal subject becomes preverbal; the IBV focus construction in \xref{159c} is derived from \xref{159b}, whereby the preverbal subject is fronted to the sentence-initial position, leaving the focused element in the IBV position.
\begin{exe}
    \ex \label{159}
    \begin{xlist}
%%EAX
\ex
%%JUDGEMENT
%%LABEL
\label{159a}
%%CONTEXT
%%LINE2
\gll
(Kí-li) \textbf{bá}a-ntaba ba-\textbf{kí}-fúúm-í mú-kái.\\
%%LINE3
7\Sm{}-\Cop{} 2-goat 2\Rel{}-7\Sm{}.\Pst{}-buy-\Pst{} 1-woman\\
%%TRANS1
\trans ‘It was the \textit{goats} that the woman bought.’
%%TRANS2
%%EXEND

%%EAX
\ex
%%JUDGEMENT
%%LABEL
\label{159b}
%%CONTEXT
%%LINE2
\gll
\textbf{Bá}a-ntaba mu-kái \textbf{káá}-fúum-i.\\
%%LINE3
2-goat 1-woman 1\Sm{}.\Pst{}-buy-\Pst{} 1-woman\\
%%TRANS1
\trans ‘The \textit{goats} were (what) the woman bought.’
%%TRANS2
%%EXEND

%%EAX
\ex
%%JUDGEMENT
%%LABEL
\label{159c}
%%CONTEXT
%%LINE2
\gll
 Mu-kái \textbf{bá}a-ntaba \textbf{káá}-fúum-i.\\
%%LINE3
1-woman 2-goat  1\Sm{}.\Pst{}-buy-\Pst{}\\
%%TRANS1
\trans ‘The woman bought the \textit{goats}.’
%%TRANS2
%%EXEND

    \end{xlist}
\end{exe}
In this section I have presented different types of cleft constructions, namely the basic cleft, the (reverse) pseudo-cleft and the reduced cleft constructions, and I have shown that these constructions all express exclusive focus on the nominal predicate. I have also shown some shared grammatical features between the cleft and the IBV focus construction, claiming that the IBV focus strategy has been grammaticalised from the cleft construction.

\section{Summary}\label{teke:sec:6}
In the first part of this chapter, I have demonstrated that the Kukuya language has a canonical SVO word order, while any deviation from this word order is produced for the purpose of information packaging. I have shown that a focused constituent, be it an argument or an adjunct of the verb, can be placed in its canonical position or in the IBV position, while the IBV position is preferred. VP focus and and verb focus can also be expressed through the canonical SVO word order or by placing the object/infinitive verb in the IBV position. Based on these facts and some additional tests, I concluded that the IBV position is really a dedicated focus position in the language, even though the focused elements are not obligatorily placed there. There is some interpretational difference between the IBV and \textit{in situ} focus strategies, in which the IBV focus site is more often associated with identificational focus, and the other often expresses assertive focus. As for topical elements, they all tend to occur in the preverbal domain as in most other Bantu languages, and several types of topical elements can be distinguished, namely the scene-setting topics, primary and secondary topics. Interestingly, the occurrence of some topical elements in the preverbal domain depends on whether the IBV focus position is occupied. I also gave a detailed introduction on two particular constructions that can function as translational equivalents of the passive construction. Different types of cleft constructions were also discussed. At the end of the chapter, I showed some shared grammatical properties that can connect the IBV focus strategy and the cleft construction. Based on these connections, I propose that the IBV focus strategy, which characterises the expression of information structure in this language, has its origin in a cleft construction.

\section*{Acknowledgements}
This research was mainly supported by the China Scholarship Council and also partly funded by the BaSIS “Bantu Syntax and Information Structure” project at Leiden University. I thank my consultants in Congo-Brazzaville Zacharie Ngouloubi, Gilbert Mbou, Gabriel Ntsiebele and Alain Mbiambourou for sharing their insights on their language with us. I also acknowledge Jenneke van der Wal, Elisabeth Kerr, Maarten Mous and three anonymous reviewers for their helpful comments. Any remaining errors are mine alone.

\section*{Abbreviations}
\begin{tabular}{@{}ll@{}}
%%% All Leipzig abbreviations are commented out, following the LangSci guidelines of only listing non-Leipzig abbreviations.
% \Aux{} & auxiliary\\
% \Comp{} & complementiser\\
\Conn{} & connective\\
% \Cop{} & copula\\
% \Dem{} & demonstrative\\
% \Fut{} & future tense\\
\Fv{} & final vowel\\
% \Gen{} & genitive\\
\Impf{} & imperfect\\
% \Inf{} & infinitive\\
% \Loc{} & locative\\
% \Neg{} & negative
% \Pass{} & passive\\
% \Pl{} & plural\\
% \Poss{} & possessive\\
\Pro{} & pronoun\\
% \Prog{} & progressive\\
% \Prs{} & present tense\\
% \Pst{} & past tense\\
% \Rel{} & relative marker\\
\Rpst{} & remote past\\
% \Sg{} & singular\\
\Sm{} & subject marker
\end{tabular}

% \section*{References}
% Andrews, Avery D. 2007. The major functions of the noun phrase. In: Shopen T. (ed.), \textit{Language Typology and Syntactic Description}, 132-223. Cambridge: Cambridge University Press.

% Bearth, Thomas. 2003. Syntax. In Derek Nurse \& Gérard Philippson (eds.), \textit{The Bantu languages}, 121-142. London, New York: Routledge.

% Bostoen, Koen \& L\'{e}on Mundeke. 2011. Passiveness and inversion in Mbuun (Bantu, B87, DRC). \textit{Studies in Language} 35\xref{1}: 72–111.

% Bostoen, Koen \& L\'{e}on Mundeke. 2012. Subject marking, object-verb order and focus in Mbuun (Bantu, B87), \textit{Southern African Linguistics and Applied Language Studies}, 30:2, 139-154

% Bostoen, Koen, Bernard Clist, Charles Doumenge, Rebecca Grollemund, Jean-Marie Hombert, Joseph K. Muluwa \& Jean Maley. 2015. Middle to late Holocene Paleoclimatic change and the early Bantu expansion in the rain forests of Western Central Africa. \textit{Current Anthropology}, 56\xref{3}, 354-384

% Bostoen, Koen \& Joseph Koni Muluwa. 2021. ‘The immediate before verb focus position in West-Coastal Bantu: some comparative data’. Workshop presentation presented at the \textit{Typological perspectives on focus marking in African languages}, Online workshop, May 27.

% Buell, Leston. 2009. Evaluating the immediate postverbal position as a focus position in Zulu. In \textit{Selected proceedings of the 38th annual conference on African linguistics} (pp. 166-172). Cascadilla Proceedings Project Somerville, MA.

% Chafe, William. 1976. Givenness, contrastiveness, definiteness, subjects, topics and point of view. In \textit{Subject and Topic}, edited by C. N. Li, 27-55. New York: Academic Press.

% Cheng, Lisa L.-S. \& Laura Downing. 2013. ‘Clefts in Durban Zulu’. In K. Hartmann \& T. Veenstra (eds), \textit{Cleft structures}. Amsterdam: John Benjamins, pp. 141‑164.

% Cobbinah, Alexander \& Friederike Lüpke. 2009. No cut to fit – zero coded passives in African languages. In Mathias Brenzinger \& Anna-Maria Fehn (eds.), \textit{Proceedings of the 6th World Congress of African Linguistics}, 153-165. Cologne: Köppe.

% Costa, João \& Nancy C. Kula. 2008. Focus at the interface: Evidence from Romance and Bantu. In \textit{The Bantu-Romance connection}, edited by C. De Cat \& K. Demuth, 293-322. Amsterdam: John Benjamins.

% Croft, William. 1991. \textit{Syntactic categories and grammatical relations: The cognitive organization of information}. Chicago: University of Chicago Press.

% Cruschina, Silvio. 2021. The greater the contrast, the greater the potential: On the effects of focus in syntax. \textit{Glossa: A Journal of General Linguistics} 6\xref{1}. 1–30.

% Dalrymple, Mary \& Irina Nikolaeva. 2011. \textit{Objects and information structure}. Cambridge: Cambridge University Press.

% De Kind, Jasper. 2014. Pre-verbal focus in Kisikongo (H16a, Bantu). \textit{ZAS Papers in Linguistics}, 57, 95-122.

% De Kind, Jasper, Sebastian Dom, Gilles-Maurice de Schryver \& Koen Bostoen. 2015. Event-Centrality and the Pragmatics-Semantics Interface in Kikongo: From Predication Focus to Progressive Aspect and Vice Versa. \textit{Folia Linguistica Historica} 36: 113-163.

% Demuth, Katherine. 1987. ‘Discourse functions of word order in Sesotho acquisition’. In R.S. Tomlin (ed.), \textit{Coherence and Grounding in Discourse}. Amsterdam: John Benjamins, pp. 91‑108.

% Downing, Laura J. \& Lutz Marten. 2019. Clausal morphosyntax and information structure. In M. Van de Velde et al. (eds.), \textit{The Bantu Languages}, Second Edition, 270–307. London: Routledge.

% É.Kiss, Katalin. 1998. Identificational focus versus information focus. \textit{Language} 74: 245–73.

% Hadermann, Pascale. 1996. Grammaticalisation de la structure infinitif + verbe conjugué dans quelques langues bantoues. \textit{Studies in African Linguistics} 25,2: 155-169.

% Frajzyingier, Zygmunt. 1982. Indefinite agent, passive and impersonal passive: A functional study. \textit{Lingua} 58. 267-290.

% Givón, Talmy. 1984. Direct object and dative shifting: semantic and pragmatic case. In Frans Plank (ed.), \textit{Objects: Towards a Theory of Grammatical Relations}, 151–182. London and Orlando: Academic Press.

% Givón, Talmy. 1990. Syntax. \textit{A Functional Typological Introduction}, vol. 2. Amsterdam and Philadelphia: Benjamins.

% Grégoire, Claire. 1993. L’ordre SOV? Du Nen: Exception Ou Généralisation D’une Tendance Répandue En Bantoue Du Nord-Ouest. Paper presented at \textit{23rd Colloquium on African Languages and Linguistics}, Leiden.

% Güldemann, Tom. 2003. Present progressive vis-à-vis predication focus in Bantu: A verbal category between semantics and pragmatics. Studies in Language. \textit{International Journal sponsored by the Foundation “Foundations of Language”}, 27\xref{2}, 323-360.

% Güldemann, Tom, Ines Fiedler, Yukiko Morimoto \& Kirill Prokhorov. 2010. Preposed verb doubling and predicate-centered focus. In \textit{International Conference of the SFB} (Vol. 632, pp. 8-10).

% Güldemann, Tom, Ines Fiedler \& Yukiko Morimoto. 2014. The verb in the preverbal domain across Bantu: infinitive “fronting” and predicate-centered focus. \textit{Unpublished manuscript. Humboldt University Berlin}.

% Güldemann, Tom, Sabine Zerbian \& Malte Zimmermann. 2015. Variation in information structure with special reference to Africa. \textit{Annu. Rev. Linguist}., 1\xref{1}, 155-178.

% Güldemann, Tom. \& Ines Fiedler. 2022. Predicate partition for predicate-centred focus and Meeussen’s Proto-Bantu “advance verb construction”. In Koen Bostoen, Gilles-Maurice de Schryver, Rozenn Guérois \& Sara Pacchiarotti (eds.), \textit{On Reconstructing Proto-Bantu Grammar}. Berlin: Language Science Press.

% Hamlaoui, Fatima \& Emmanuel-Moselly Makasso. 2013. Bàsàa object left dislocation, topicalization and the syntax-phonology mapping of intonation phrases. Paper presented at the \textit{5th International Conference on Bantu Languages}, Paris, 12-15 June. ZAS MA thesis.

% Hamlaoui, Fatima \& Emmanuel-Moselly Makasso. 2015. ‘Focus marking and the unavailability of inversion structures in the Bantu language Bàsàá (A43)’. \textit{Lingua} 154: 35‑64.

% Henderson, Brent. 2006. \textit{The Syntax and Typology of Bantu Relative Clauses}. Ph. D. dissertation, University of Illinois, Urbana-Champaign.

% Hyman, Larry. M. 1987. Prosodic domains in Kukuya. \textit{Natural Language \& Linguistic Theory}, 311-333.

% Hyman, Larry. M. \& John R. Watters. 1984. Auxiliary focus. \textit{Studies in African linguistics}, 15\xref{3}, 233-274.

% Hyman, Larry. M. \& Maria Polinsky. 2010. Focus in Aghem. \textit{Information structure: Theoretical, typological, and experimental perspectives}, 206-233.

% Krifka, Manfred. 2007. The semantics of questions and the focusation of answers. In Chungmin Lee, Matthew Gordon \& Daniel Büring (eds.), \textit{Topic and focus, crosslinguistic perspectives on meaning and intonation}, 139–150. Dordrecht: Springer

% Gundel, Jeanette. 1988. Universals of topic-comment structure. In \textit{Studies in syntactic typology}, edited by M. Hammond, E. A. Moravcsik and J. R. Wirth, 209-242. Amsterdam: John Benjamins.

% Kawasha, Boniface. 2007. Passivization in Lunda. \textit{Journal of African Languages
% and Linguistics} 28. 37-56.

% Keenan, Edward L. \& Matthew S. Dryer. 2007. “Passive in the World’s Languages”. \textit{Clause Structure, Language Typology and Syntactic Description}, 1, 325-361.

% Koni Muluwa, Joseph \& Koen Bostoen. 2014. The immediate before the verb focus position in Nsong (Bantu B85d, DR Congo): a corpus-based exploration. \textit{ZAS Papers in Linguistics}, 57, 123-135.

% Koni Muluwa, Joseph \& Koen Bostoen. 2019. Nsong (B85d). In Mark van de Velde, Koen Bostoen, Derek. Nurse \& Gérard Philippson (eds.), The Bantu Languages (Second Edition), 414-448. Oxford: Routledge.

% Kula, Nancy C. \& Lutz Marten. 2010. Argument structure and agency in Bemba passives. In Karsten Legère \& Christina Thornell (eds.), \textit{Bantu languages: Analyses, description and theory}, 115-130. Cologne: Rüdiger Köppe.

% Kuroda, Sige-Yuki. 1972. The categorical and the thetic judgment: Evidence from Japanese syntax. \textit{Foundations of language}, 153-185.

% Lafkioui, Mena B., Ernest Nshemezimana \& Koen Bostoen. 2016. Cleft constructions and focus in Kirundi. \textit{Africana Linguistica} 22: 71-106.

% Lambrecht, Knud. 1994. \textit{Information structure and sentence form}. Cambridge: Cambridge University Press [Cambridge Studies in Linguistics].

% Lewis, David. 1979. Scorekeeping in a language game. \textit{Journal of Philosophical Logic} 8. 339–359

% Li, Charles N. \& Sandra A. Thompson. 1976. Subject and topic: a new typology of language. In \textit{Subject and Topic}, edited by C. Li, 457-490. New York: Academic Press.

% Li, Zhen. 2020. A note on the functional passives in Teke-Kukuya (Bantu B77, Congo). In Van der Wal, Jenneke, Heleen Smits, Sara Petrollino, Victoria Nyst, and Maarten Kossmann (eds.), \textit{Essays in African languages and linguistics: in honour of Maarten Mous}, 267-286. Leiden: ASCL Occasional Publication 41.

% Li, Zhen. forthcoming. Word order, agreement and information structure in Teke. Ph.D thesis in preparation, Leiden University.

% Meeussen, Achille Emile. 1967. Bantu grammatical reconstructions. \textit{Annales du Musée Royal de l’Afrique Centrale} 61.

% Morimoto, Yukiko. 2000. Discourse conigurationality in Bantu Morphosyntax. PhD dissertation, Stanford: Stanford University.

% Morimoto, Yukiko. 2016. Verb doubling vs. the conjoint/disjoint alternation.
% (Paper presented at Bantu 6, Helsinki).

% Ndonga Mfuwa, Manuel. 1995. \textit{Systématique grammaticale du kisikongo (Angola)}. Paris, Université René Descartes, Paris V.

% Nikolaeva, Irina. 2001. Secondary Topic as a Relation in Information Structure. \textit{Linguistics} 39.1: 1–49.

% Pacchiarotti, Sara, Natalia Chousou-Polydouri \& Koen Bostoen. 2019. Untangling the West-Coastal Bantu mess : identification, geography and phylogeny of the Bantu B50-80 languages. \textit{Africana Linguistica}, 25, 155–229

% Paulian, Christiane. 1975. \textit{Le kukuya: langue teke du Congo}. Paris: SELAF.

% Reinhart, Tanya. 2006. \textit{Interface strategies}. Cambridge, Massachusetts: MIT Press.

% Rizzi, Luigi. 1986. On the status of subject clitics in Romance. In \textit{Studies in Romance linguistics}, edited by O. Jaeggli and C. Silva-Corvalan, 391-420. Dordrecht: Foris.

% Rooth, Mats. 1992. A theory of focus interpretation. \textit{Natural Language Semantics} 1 \xref{1}. 75–116

% Sabel, Joachim \& Jochen Zeller. 2006. ‘Wh‑question formation in Nguni’. In J. Mugane, J. Hutchison \& D. Worman (eds),\textit{African Languages and Linguistics in Broad Perspective (Selected Proceedings of the 35th Annual Conference of African Linguistics, Harvard, Cambridge)}. Somerville: Cascadilla Proceedings Press, pp. 271‑283.

% Sasse, Hans-Jürgen. 1987. The thetic/categorical distinction revisited. In: \textit{Linguistics}
% 25 : 511–80.

% Sasse, Hans-Jürgen. 1996. \textit{Theticity}. Arbeitspapiere Cologne: Institut für Sprachwissenschat der Universität zu Köln.

% Selkirk, Elizabeth. 1995. The Prosodic structure of function words. In \textit{Papers in Optimality Theory}, edited by J. Beckman, L. Walsh Dickey and S. Urbanczyk, 439-469. Amherst, Massachusetts: GLSA.

% Şener, Sekan. 2010. (Non-)peripheral matters in Turkish syntax. Doctoral dissertation, University of Connecticut, Storrs

% Shi, Dingxu. 2000. Topic and topic-comment constructions in Mandarin Chinese. \textit{Language}, 76\xref{2}, 383-408.

% Stalnaker, Robert. 2002. Common ground. \textit{Linguistics and Philosophy} 25. 701–721

% Tsao, Fengfu. 1987. A topic-comment approach to the \textit{ba} construction. \textit{Journal of Chinese Linguistics} 15.1-54.

% Vallduví, Enric. 1992. The Informational Component. New York: Garland

% Van der Wal, Jenneke. 2009. \textit{Word order and Information Structure in Makhuwa-Enahara}. Utrecht: Landelijke Onderzoekschool Taalwetenschap (LOT).

% Van der Wal, Jenneke. 2011. Focus excluding alternatives: Conjoint/disjoint marking in Makhuwa. \textit{Lingua}, 121\xref{11}, 1734-1750.

% Van der Wal, J. 2015. \textit{Bantu Syntax}. Oxford Handbooks Online

% Van der Wal, Jenneke \& Jacky Maniacky. 2015. How ‘person’ got into focus: Grammaticalization of clefts in Lingala and Kikongo areas. \textit{Linguistics}, 53\xref{1}, 1-52.

% Van der Wal, Jenneke. 2016. Diagnosing focus, \textit{Studies in Language} 40:2, 259–301

% Van der Wal, Jenneke. 2021. \textit{The BaSIS basics of information structure}. (https://bantusyntaxinformationstructure.files.wordpress.com/2021/08/the-basis-basics-of-information-structure-complete-6.pdf)

% von Fintel, Kai. 2008. What is presupposition accommodation, again? \textit{Philosophical perspectives}
% 22 \xref{1}. 137–170.

% Watters, John R. 1979. \textit{Focus in Aghem: A study of its formal correlates and typology} (Doctoral dissertation, Univ. Los Angeles).

% Yoneda, Nobuko. 2011. Word order in Matengo (N13): Topicality and informational roles. \textit{Lingua}, 121\xref{5}, 754-771.

% Zeller, Jochen. 2008. On agreement, focus and subject positions in Bantu. Paper presented at the conference \textit{Movement and word order} in Bantu, Leiden.

% Zerbian, Sabine. 2006. \textit{Expression of information structure in Northern Sotho}. PhD dissertation, Humboldt University.

\sloppy\printbibliography[heading=subbibliography,notkeyword=this]
\end{document}
