\documentclass[output=paper]{langscibook}
\ChapterDOI{10.5281/zenodo.14833612}
\author{Ernest Nshemezimana\orcid{}\affiliation{Université de Burundi} and Jenneke van der Wal\orcid{}\affiliation{Leiden University}}
\title{The expression of information structure in Kirundi}
\abstract{This chapter reports on different linguistic means used in Kirundi to construct informational units in a sentence. We demonstrate that information structure has a fundamental influence on the morphosyntax of Kirundi. Information structure determines the word order to a large degree, with the requirements for topics in the preverbal domain and a final focus position leading to a range of subject inversion constructions. It can also influence the verbal inflection, particularly in verb forms distinguishing the conjoint/disjoint alternation. Other linguistic phenomena linked to information structure such as predicate doubling, the agreeing particle ~-\textit{ó} as a topic marker and copular constructions (clefts, pseudoclefts and reverse pseudoclefts) pinpointing the focus interpretation are also described in this chapter.}
\IfFileExists{../localcommands.tex}{
  \addbibresource{../localbibliography.bib}
  \usepackage{langsci-optional}
\usepackage{langsci-gb4e}
\usepackage{langsci-lgr}

\usepackage{listings}
\lstset{basicstyle=\ttfamily,tabsize=2,breaklines=true}

%added by author
% \usepackage{tipa}
\usepackage{multirow}
\graphicspath{{figures/}}
\usepackage{langsci-branding}

  
\newcommand{\sent}{\enumsentence}
\newcommand{\sents}{\eenumsentence}
\let\citeasnoun\citet

\renewcommand{\lsCoverTitleFont}[1]{\sffamily\addfontfeatures{Scale=MatchUppercase}\fontsize{44pt}{16mm}\selectfont #1}
   
  %% hyphenation points for line breaks
%% Normally, automatic hyphenation in LaTeX is very good
%% If a word is mis-hyphenated, add it to this file
%%
%% add information to TeX file before \begin{document} with:
%% %% hyphenation points for line breaks
%% Normally, automatic hyphenation in LaTeX is very good
%% If a word is mis-hyphenated, add it to this file
%%
%% add information to TeX file before \begin{document} with:
%% %% hyphenation points for line breaks
%% Normally, automatic hyphenation in LaTeX is very good
%% If a word is mis-hyphenated, add it to this file
%%
%% add information to TeX file before \begin{document} with:
%% \include{localhyphenation}
\hyphenation{
affri-ca-te
affri-ca-tes
an-no-tated
com-ple-ments
com-po-si-tio-na-li-ty
non-com-po-si-tio-na-li-ty
Gon-zá-lez
out-side
Ri-chárd
se-man-tics
STREU-SLE
Tie-de-mann
}
\hyphenation{
affri-ca-te
affri-ca-tes
an-no-tated
com-ple-ments
com-po-si-tio-na-li-ty
non-com-po-si-tio-na-li-ty
Gon-zá-lez
out-side
Ri-chárd
se-man-tics
STREU-SLE
Tie-de-mann
}
\hyphenation{
affri-ca-te
affri-ca-tes
an-no-tated
com-ple-ments
com-po-si-tio-na-li-ty
non-com-po-si-tio-na-li-ty
Gon-zá-lez
out-side
Ri-chárd
se-man-tics
STREU-SLE
Tie-de-mann
} 
  \togglepaper[5]%%chapternumber
}{}

\begin{document}
\maketitle 
\label{ch:5}

\section{Introduction}
\begin{sloppypar}
Kirundi is a Bantu language spoken in Burundi in the Great Lakes region of central and eastern Africa. In \citegen{Guthrie1948} referential classification, updated by \citet{Maho2009}, Kirundi is JD62, and it has ISO code [run]. Its speakers are estimated to be over 11 million in Burundi, where it has the status of national and official language according to Article 5 of the National Constitution dated 7 June 2018. The country’s second official language is French. Kirundi is mutually intelligible with a number of neighbouring languages, most importantly Kinyarwanda (JD61), but also some languages spoken in the regions of Tanzania bordering Burundi, such as Kiha (JD66) spoken in Buha (south of Kirundi), Kihangaza (JD65) and Kishubi (JD64) in Bushubi (north-east). Kirundi is also close to Kivinza (JD67), spoken in Uvinza in Tanzania, south of Buha.
\end{sloppypar}

The current chapter mainly draws on \citet{Nshemezimana2016}, referring to other work on Kirundi where appropriate, and also drawing on the BantUgent monolingual corpus of Kirundi. Further data come from the first author as a native speaker, whose data for the BaSIS project have been stored in an Online Language Database via Dative, which is to be archived in The Language Archive. Examples created based on corpus examples are indicated as “derived”. 

While word order \citep{Morimoto1999,Morimoto2006,Bukuru2003}, the conjoint/disjoint alternation \citep{Ndayiragije1999,Sabimana1986,NshemezimanaBostoen2017}, and clefts \citep{LafkiouiEtAl2016} have been described and analysed before, the phenomena of predicate doubling and the marker -\textit{ó} are newly described here. We follow the Kirundi orthography \citep{Ntahokaja1994}. The tone marking in this chapter follows the Kirundi tradition started by \citet{Meeussen1959} and does not (necessarily) represent surface tones.

The aim of the chapter is to provide an overview of the different linguistic means that Kirundi uses to structure information in a sentence. Specifically, the conjoint/disjoint alternation is recapitulated in \sectref{bkm:Ref81485420}; we discuss word order, including the relatively large array of subject inversion constructions, in \sectref{bkm:Ref81485307}; predicate doubling is described in \sectref{bkm:Ref75783502}; \sectref{bkm:Ref72249299} describes the function of the marker -\textit{ó} as a contrastive topic marker; and \sectref{bkm:Ref72334782} discusses four types of copular constructions, including clefts and pseudoclefts. For further explanation on general terms in information structure we refer to \textcite{chapters/intro}.

\section{Conjoint and disjoint verb forms}
\label{bkm:Ref81485420}
Like many other eastern and southern Bantu languages, Kirundi has certain conjugations with two forms, the conjoint and the disjoint. These terms were coined for Kirundi by \citet[215--216]{Meeussen1959} to express the relation between the verb and the following element as together (French \textit{conjoint}) or separated (French \textit{disjoint}). The alternation has been linked to differences in information structure, with conjoint (\CJ{}) expressing focus on a constituent following the verb, and disjoint (\DJ{}) associated with predicate-centred focus (see \citealt{vanderWalHyman2017} for an overview). \citet{Nshemezimana2016} and \citet{NshemezimanaBostoen2017} provide a detailed corpus-based analysis of the alternation and its connection with information structure, which we summarise here, and add to.

Key to the conjoint/disjoint alternation is the fact that the conjoint verb form cannot appear clause-finally (i.e. some element has to follow), and there is a direct or indirect relation with information structure \citep[15]{vanderWal2017}. In Kirundi, the alternation is directly dependent on information structure, as \citet{Ndayiragije1999}, \citet{Bukuru2003}, and \citet{NshemezimanaBostoen2017} show, and as we also discuss below.

The alternation can be found in three conjugations in Kirundi: the present, near past, and remote past. We illustrate each below. In all tenses, the \CJ{} form is never constituent-final, and in the present and near past tense, it loses a lexical H tone if it has one. The \DJ{} form retains a possible lexical H, and is marked by a prefix -\textit{ra}- in the present and remote past, and by -\textit{a}- in the near past.

\ea
Present \citep[79,~80,~and~derived]{Nshemezimana2016}
%%EAX
\ea
\begin{xlist}
\exi{\CJ}{
%%JUDGEMENT
%%LABEL
%%CONTEXT
%%LINE1
Nkomeza-magúfa ifasha *(umutíma waawe).\\
%%LINE2
\gll
n-komer-i-a  ma-gúfa  i-$\varnothing$-fásh-a  u-mu-tíma  u-awe\\
%%LINE3
9-be.strong-\CAUS{}-\FV{}  6-bone  9\SM{}-\PRS.\CJ{}-help-\FV{}  \AUG{}-3-heart  3-\POSS.2\SG{}\\
%%TRANS1
\glt
‘Carbohydrates help your heart.’ (corpus UGhent)\\
%%TRANS2
}
\end{xlist}
%%EXEND


\ex
\begin{xlist}
%%EAX
\exi{\DJ{}}
%%JUDGEMENT
%%LABEL
%%CONTEXT
Nkomeza-magúfa irafasha.\\
%%LINE1
%%LINE2
\gll
n-komer-i-a  ma-gúfa  i-ra-fásh-a  \\
%%LINE3
9-be.strong-\CAUS{}-\FV{}  6-bone  9\SM-\PRS.\DJ{}-help-\FV{}\\
%%TRANS1
\glt
‘Carbohydrates help.’ (derived)\\
%%TRANS2
%%EXEND

\end{xlist}
\z

\ex
Near past \citep[79,~80,~and~derived]{Nshemezimana2016}

\ea
%\label{ex:kirundi:186}
\begin{xlist}
%%EAX
\exi{\CJ{}}
%%JUDGEMENT
%%LABEL
%%CONTEXT
%%LINE1
Ma, nabonye *(Bikíra Mariyá) …\\
%%LINE2
\gll
ma  n-a-$\varnothing$-bón-ye  Bikíra  Mariyá\\
%%LINE3
mother  1\SG.\SM-\N.\PST-\CJ{}-see-\PFV{}  virgin  Mary\\
%%TRANS1
\glt
‘Mother, I’ve seen the Virgin Mary!’ (derived)\\
%%TRANS2
%%EXEND

\end{xlist}

\ex
\begin{xlist}
%%EAX
\exi{\DJ{}}
%%JUDGEMENT
%%LABEL
%%CONTEXT
%%LINE1
Ma, naabóonye Bikíra Mariyá …\\
%%LINE2
\gll
ma  n-a-a-bón-ye  Bikíra  Mariyá\\
%%LINE3
mother  1\SG.\SM{}-\N.\PST-\DJ{}-see-\PFV{}  virgin  Mary\\
%%TRANS1
\glt
‘Mother, I’ve seen the Virgin Mary!’ (corpus of BantUgent)\\
%%TRANS2
%%EXEND

\end{xlist}

\ex
\begin{xlist}
%%EAX
\exi{\CJ{}}
%%JUDGEMENT
%%LABEL
%%CONTEXT
%%LINE1
*Namuboonye.\\
%%LINE2
\gll
\N{}-a-$\varnothing$-mu-bón-ye\\
%%LINE3
1\SG.\SM-\N.\PST-\CJ-1\OM{}-see-\PFV{}\\
%%TRANS1
\glt
‘I’ve seen her.’ (derived)\\
%%TRANS2
%%EXEND

\end{xlist}

\ex
\begin{xlist}
%%EAX
\exi{\DJ{}}
%%JUDGEMENT
%%LABEL
%%CONTEXT
%%LINE1
Naamubóonye.\\
%%LINE2
\gll
\N{}-a-a-bón-ye\\
%%LINE3
1\SG.\SM-\N.\PST-\DJ-1\OM{}-see-\PFV{}\\
%%TRANS1
\glt
‘I’ve seen her.’ (derived)\\
%%TRANS2
%%EXEND

\end{xlist}

\z

\ex
Remote past \citep[79,~80,~and~derived]{Nshemezimana2016}\\
\ea
%\label{ex:kirundi:186}
\begin{xlist}
%%EAX
\exi{\CJ{}}
%%JUDGEMENT
%%LABEL
%%CONTEXT
%%LINE1
… yatéeye *(akáamo abanyeepolitiíke) …\\
%%LINE2
\gll
a-á-$\varnothing$-teer-ye  a-ka-áamo  a-ba-nyeepolitiíke\\
%%LINE3
1\SM-\RMT.\PST-\CJ{}-throw-\PFV{}  \AUG{}-12-call  \AUG{}-2-politician\\
%%TRANS1
\glt
‘…he has appealed to the politicians….’ (corpus UGhent)\\
%%TRANS2
%%EXEND

\end{xlist}

\ex
\begin{xlist}
%%EAX
\exi{\DJ{}}
%%JUDGEMENT
%%LABEL
%%CONTEXT
%%LINE1
… yaráteeye akáamo abéenegíhugu bó mu Kiruundo.\\
%%LINE2
\gll
a-á-ra-téer-ye  a-ka-áamo  a-ba-éenegíhugu  ba-ó mu  Kirundo \\
%%LINE3
1\SM-\RMT.\PST-\DJ{}-throw-\PFV{}  \AUG{}-12-call  \AUG{}-2-inhabitant  2-\CONN{} 18  Kirundo \\
%%TRANS1
\glt
‘…he has appealed to the inhabitants of Kirundo.’ (corpus UGhent)\\
%%TRANS2
%%EXEND

\end{xlist}

\ex
\begin{xlist}
%%EAX
\exi{\DJ{}}
%%JUDGEMENT
%%LABEL
%%CONTEXT
%%LINE1
Yarágateeye.\\
%%LINE2
\gll
a-á-ra-ka-téer-ye\\
%%LINE3
1\SM-\RMT.\PST-\DJ{}-12\OM{}-throw-\PFV{}\\
%%TRANS1
\glt
‘He has thrown it.’ / ‘He has appealed.’ (derived)\\
%%TRANS2
%%EXEND

\end{xlist}

\z
\z


The fact that the \DJ{} form can be followed by a non-dislocated object or adjunct already shows that the alternation is not determined by constituency, as \citet{NshemezimanaBostoen2017} and \citet{Nshemezimana2016} also argue. 

\citet[397]{NshemezimanaBostoen2017} add that the potential and subsecutive conjugation also show the tonal effects with H-toned verbs, and show that the alternation is not present in negative and relative conjugations (in line with the tendencies in other languages, see \citealt{vanderWal2017}). They summarise the marking as in \tabref{tab:run-cjdjforms}.

\begin{table}
\begin{tabularx}{\textwidth}{XlXX}
\lsptoprule
conjugation & TAM-marker & disjoint & conjoint\\
\midrule
present & -$\varnothing$- & -ra- /+lexical H & -$\varnothing$- /--lexical H\\
near past & -a- & -a- /+lexical H & -$\varnothing$- /--lexical H\\
remote past & -á- & -ra- /+lexical H & -$\varnothing$- /--lexical H\\
potential & -oo- & -$\varnothing$- /+lexical H & -$\varnothing$- /--lexical H\\
subsecutive & -ka- & -$\varnothing$- /+lexical H & -$\varnothing$- /--lexical H\\
\lspbottomrule
\end{tabularx}
\caption{Conjoint and disjoint verb forms \citep[397]{NshemezimanaBostoen2017}}
\label{tab:run-cjdjforms}
\end{table}


\citeauthor{NshemezimanaBostoen2017} propose that the \DJ{} marker \textit{a}-/\textit{ra-} should be analysed as a marker of ``event focus" (predicate-centred focus), and moreover that the \CJ{} form is the default, without any focus function (cf. \citeauthor{Ndayiragije1999}'s \citeyear{Ndayiragije1999} analysis of \textit{a}-/\textit{ra}- as an antifocus marker). However, further details from introspective tests by Ernest Nshemezimana reveal that this cannot be the whole story. If the \CJ{} form is unmarked in form and function, the incompatibility with idioms and postverbal NPs modified by ‘even’ is unexpected. These instead require a \DJ{} form – with the \CJ{} form, the idiomatic reading is lost \xref{bkm:Ref75772351} or the sentence becomes ungrammatical \xref{bkm:Ref81548679}. In these sentences, the focus is clearly not on the event.

\ea
\label{bkm:Ref75772351}
%\label{ex:kirundi:56}
Parts of idioms cannot follow the \CJ{} form\\
\ea
\begin{xlist}
%%EAX
\exi{\DJ}
%%JUDGEMENT
%%LABEL
%%CONTEXT
%%LINE1
Turafáshe imbwá amabóko.\\
%%LINE2
\gll
tu-ra-fát-iye  i-n-bwá  a-ma-bóko\\
%%LINE3
1\PL.\SM-\DJ{}-hold-\PFV{}  \AUG{}-9-dog  \AUG{}-6-arm\\
%%TRANS1
\glt
‘We really landed ourselves in a nasty situation.’\\
%%TRANS2
lit. ‘We have held the dog by the arms.’

%%EXEND

\end{xlist}

\ex
\begin{xlist}
%%EAX
\exi{\CJ{}}
%%JUDGEMENT
%%LABEL
%%CONTEXT
%%LINE1
Dufashe imbwá amabóko.\\
%%LINE2
\gll
tu--$\varnothing$-fát-iye  i-n-bwá  a-ma-bóko\\
%%LINE3
1\PL.\SM{}-\CJ{}-hold-\PFV{}  \AUG{}-9-dog  \AUG{}-6-arm\\
%%TRANS1
\glt
‘We have held the dog by the arms.’\\
%%TRANS2
*`We really landed ourselves in a nasty situation.’

%%EXEND

\end{xlist}

\z
\z

%%EAX
\ea
%%JUDGEMENT
%%LABEL
\label{bkm:Ref81548679}
%%CONTEXT
Objects modified by ‘even’ cannot follow the \CJ{} form\\
%%LINE1
Yohana ararya/*arya ndetse na ifi.\\
%%LINE2
\gll
Yohana  a-ra-ri-a/a-ri-a  ndetse  na  i-fi\\
%%LINE3
1.John  1\SM-\PRS.\DJ{}-eat-fv/1\SM{}-eat-\FV{}  even  and  \AUG{}-9.fish\\
%%TRANS1
\glt
‘John eats even fish.’\\
%%TRANS2
%%EXEND

\z

Furthermore, a cognate object is only accepted when modified to become specific \xref{bkm:Ref75772356}, and the same goes for the word \textit{umuntu} ‘person’ after a \CJ{} form: it either needs to be specified in order to be grammatical, or else it is interpreted as generic, as opposed to the corresponding \DJ{} form which freely accepts an indefinite reading. This too would be unexpected if the \CJ{} form were an information-structurally neutral form.

\ea
\label{bkm:Ref75772356}
%\label{ex:kirundi:56}
Cognate objects cannot as such follow the \CJ{} form\\
\ea
\begin{xlist}
%%EAX
\exi{\DJ{}}
%%JUDGEMENT
%%LABEL
%%CONTEXT
%%LINE1
Naaróose indóoto.\\
%%LINE2
\gll
\N{}-a-a-róot-ye  i-n-róoto\\
%%LINE3
1\SG.\SM-\N.\PST-\DJ{}-dream-\PFV{}  \AUG{}-9-dream\\
%%TRANS1
\glt
‘I dreamt a dream.’\\
%%TRANS2
%%EXEND

\end{xlist}

\ex
\begin{xlist}
%%EAX
\exi{\CJ{}}
%%JUDGEMENT
%%LABEL
%%CONTEXT
%%LINE1
Naroose indóoto *(zitari nziiza).\\
%%LINE2
\gll
\N{}-a-$\varnothing$-róot-ye  i-n-róoto  zi-ta-ri  n-ziiza\\
%%LINE3
1\SG.\SM-\N.\PST-\CJ{}-dream-\PFV{}  \AUG{}-10-dream  10\SM-\NEG{}-be.\REL{}  10-good\\
%%TRANS1
\glt
‘I dreamt dreams *(that were not pleasant).’\\
%%TRANS2
%%EXEND

\end{xlist}

\z
\z

\ea
Indefinite non-specific cannot as such follow the \CJ{} form\\
%%EAX
\ea
\begin{xlist}
\exi{\CJ}
%%JUDGEMENT
%%LABEL
%\label{ex:kirundi:56}
%%CONTEXT
%%LINE1
Twaboonye umuntu *(yambaye nk’ umusazi).\\
%%LINE2
\gll
tu-a-$\varnothing$-bón-ye  u-mu-ntu  a-ambar-ye\textsuperscript{H} nka  u-mu-sazi \\
%%LINE3
1\PL.\SM{}-\N.\PST-\CJ{}-see-\PFV{}  \AUG{}-1-person  1\SM{}-get.dressed-\PFV{}.\REL{} like  \AUG{}-1-fool ‘We saw someone *(dressed like a fool).’\\
%%TRANS1
\glt
‘We saw a human being.’\\
%%TRANS2
\end{xlist}
%%EXEND

\ex
\begin{xlist}
%%EAX
\exi{\DJ{}}
%%JUDGEMENT
%%LABEL
%%CONTEXT
%%LINE1
Twaabóonye umuntu.\\
%%LINE2
\gll
u-a-a-bón-ye  u-mu-ntu\\
%%LINE3
1\PL.\SM-\N.\PST-\DJ{}-see-\PFV{}  \AUG{}-1-person\\
%%TRANS1
\glt
‘We saw someone.’\\
%%TRANS2
%%EXEND

\end{xlist}

\z
\z

This suggests that there is more at hand than the \CJ{} form as a neutral form, specifically that some alternatives must be available (and possibly excluded) for the focused referent after a \CJ{} form. The exclusion of alternatives would explain the incompatibility of NPs with ‘even’, as these do not allow exclusion: with ‘even’,  the predicate is true for all alternatives, even the least likely one. A constituent modified by ‘even’ cannot follow the \CJ{} form if this requires exclusion of alternatives for the following constituent. The generic and the modified interpretation of the indefinite \textit{umuntu} ‘person’ also follows, as these are the two interpretations that allow for exclusion of alternatives: unlike the indefinite non-specific interpretation ‘someone’ (which does not exclude anyone), a person dressed like a fool excludes everyone not dressed as a fool, and ‘human being’ excludes the alternatives of cats, chickens, etc. For the same reason, exclusive focus on the element following the \CJ{} verb form also predicts incompatibility with the universal quantifier ‘all’. As shown in \xref{bkm:Ref76372322}, this is not directly ungrammatical in Kirundi, but the context shows that a contrast is interpreted with another set, thus allowing exclusion: all the sweet potatoes but not (all) the beans.\pagebreak

\ea
Compatibility with universal quantifier
\ea
(Context: Two children are supposed to share the potatoes. The first child comes in and eats all the food; when the second child comes in, s/he complains. The parent asks ‘What happened?’)\\
\begin{xlist}
%%EAX
\exi{\DJ}
%%JUDGEMENT
%%LABEL
%\label{ex:kirundi:56}
%%CONTEXT
%%LINE1
Yaariiye vya ibijumbu vyóóse.\\
%%LINE2
\gll
a-a-a-ri-ye  bi-a  i-bi-jumbu  bi-óóse\\
%%LINE3
1\SM-\N.\PST-\DJ{}-eat-\PFV{}  8-\DEM{}\textsubscript{5}  \AUG{}-8-sweet.potato  8-all\\
%%TRANS1
\glt
‘S/he has eaten all the sweet potatoes.’\\
%%TRANS2
%%EXEND
    
\end{xlist}

\ex
(Two children are supposed to share the food. The first child comes in and eats more than s/he should; when the second child comes in, s/he complains. The parent asks ‘What is left?’)\\
\begin{xlist}
%%EAX
\exi{\CJ}
%%JUDGEMENT
%%LABEL
\label{bkm:Ref76372322}
%%CONTEXT
%%LINE1
Yariiye ibjumbu vyóóse, asigaza ibiharage (gusa).\\
%%LINE2
\gll
a-a-$\varnothing$-ri-ye  i-bi-jumbu  bi-óóse  a-sigaz-a   i-bi-harage  gusa \\
%%LINE3
1\SM-\N.\PST{}-\CJ{}-eat-\PFV{}  \AUG{}-8-sweet.potato  8-all  1\SM{}-leave-\FV{}   \AUG{}-8-beans  only \\
%%TRANS1
\glt
  ‘S/he has eaten all the sweet potatoes, s/he left the beans (only).’\\
%%TRANS2
%%EXEND

\end{xlist}
\z
\z


While these diagnostics thus suggest an interpretation of exclusive focus after the \CJ{} form, the following test shows that this cannot be \textit{exhaustive} focus. The theoretical difference is that for exclusivity at least \textit{some} alternatives must be excluded, while for exhaustivity \textit{all} alternatives are excluded. Correcting an incomplete assertion with a \CJ{} form cannot start with ‘no’, as shown in \xref{bkm:Ref76372401}. This shows that the negation cannot apply to the hypothesised exhaustivity in the question: the negation can only apply to the truth of the statement, and since it is true that you drank milk (even though that is not the complete truth), this cannot be denied. From this answer, we deduce that there is no such exhaustivity present in the question, that is, the focus expressed by the \CJ{} form cannot be exhaustive, i.e. the question in \xref{bkm:Ref76372401} with the conjoint form is not interpreted as ‘Did you drink only milk?’. This differs from when the question contains an explicit exhaustive marker ‘only’ as in \xref{bkm:Ref76372415}, in which case ‘no’ in the answer denies the exhaustivity: it is not exhaustively milk that I drank.

\ea
\label{bkm:Ref76372401}
\begin{xlist}[Q.\CJ:]
%%EAX
\exi{Q.\CJ:}
%%JUDGEMENT
%%LABEL
%%CONTEXT
%%LINE1
Mbega wanyooye amata?\\
%%LINE2
\gll
mbega  u-a-$\varnothing$-nyo-ye  a-ma-ta\\
%%LINE3
Q  2\SG.\SM-\N.\PST{}-\CJ{}-drink-\PFV{}  \AUG{}-6-milk\\
%%TRANS1
\glt
‘Did you drink milk?’\\
%%TRANS2
%%EXEND

%%EAX
\exi{A:}
%%JUDGEMENT
%%LABEL
%%CONTEXT
%%LINE1
Ego/\textsuperscript{\#}Oya, n’ifanta nayinyóoye.\\
%%LINE2
\gll
ego/oya  ni  i-fanta  \N{}-a-nyó-ye\textsuperscript{H}\\
%%LINE3
yes/no  \COP{}  \AUG{}-9.fanta  1\SG{}.\SM{}-\N{}.\PST{}-drink-\PFV{}.\REL{}\\
%%TRANS1
\glt
‘Yes/\textsuperscript{\#}No, I also drank Fanta.’\\
%%TRANS2
%%EXEND

\end{xlist}
\z

\ea
\label{bkm:Ref76372415}
\begin{xlist}[Q.\CJ:]
%%EAX
\exi{Q.\CJ:}
%%JUDGEMENT
%%LABEL
%%CONTEXT
%%LINE1
Mbega wanyooye amata gusa?\\
%%LINE2
\gll
mbega  u-a-$\varnothing$-nyó-ye  a-ma-ta  gusa\\
%%LINE3
Q  2\SG{}.\SM{}-\N.\PST{}-\CJ{}-drink-\PFV{}  \AUG{}-6-milk  only\\
%%TRANS1
\glt
‘Did you drink only milk?’\\
%%TRANS2
%%EXEND

%%EAX
\exi{A:}
%%JUDGEMENT
%%LABEL
%%CONTEXT
%%LINE1
Oya/\textsuperscript{\#}Ego, n’ifanta nayinyóoye.\\
%%LINE2
\gll
oya/ego  na  i-fanta  \N{}-a-a-yi-nyó-ye\\
%%LINE3
no/yes  also  \AUG{}-9.fanta  1\SG{}.\SM-\N.\PST-\DJ-9\OM{}-drink-\PFV{}\\
%%TRANS1
\glt
‘No/\textsuperscript{\#}Yes, I also drank Fanta.’\\
%%TRANS2
%%EXEND

\end{xlist}
\z

An exclusive (even if not exhaustive) interpretation also predicts that the \CJ{} form will be incompatible with a context in which there are multiple correct answers. In such a ``mention some" context, either form is acceptable in Kirundi, but crucially, adding ‘for example’ throws off the \CJ{} form, as in \xref{bkm:Ref75856545}.

\ea
(What do tourists in Burundi typically do?)\\
\ea
\begin{xlist}
%%EAX
\exi{\CJ{}}
%%JUDGEMENT
%%LABEL
\label{bkm:Ref75856545}
%%CONTEXT
%%LINE1
Bagenda kuri Tanganyika (\#nk’ akakorero).\\
%%LINE2
\gll
ba-$\varnothing$-gend-a  kuri  Tanganyika  nka  a-ka-korero\\
%%LINE3
2\SM{}-\PRS.\CJ{}-walk-\FV{}  17  Tanganyika  like  \AUG{}-12-example\\
%%TRANS1
\glt
‘They visit Tanganyika (\#for example).’\\
%%TRANS2
%%EXEND

\end{xlist}

\ex
\begin{xlist}
%%EAX
\exi{\DJ{}}
%%JUDGEMENT
%%LABEL
%%CONTEXT
%%LINE1
Baragenda kuri Tanganyika (nk’ akakorero).\\
%%LINE2
\gll
ba-ra-gend-a  kuri  Tanganyika  nka  a-ka-korero\\
%%LINE3
2\SM-\PRS.\DJ{}-walk-\FV{}  17  Tanganyika  like  \AUG{}-12-example\\
%%TRANS1
\glt
‘They visit Tanganyika, for example.’\\
%%TRANS2
%%EXEND

\end{xlist}

\z
\z

These data suggest an analysis in which the \CJ{} form triggers alternatives for the postverbal (clause-final, see \sectref{bkm:Ref81485890}) constituent. This analysis also explains the well-known restriction that postverbal question words, which are inherently focused, require the \CJ{} form \xref{bkm:Ref75857421}, as do contrasted NPs, and NPs modified by ‘only’ \xref{bkm:Ref75857415}.

%%EAX
\ea
%%JUDGEMENT
%%LABEL
\label{bkm:Ref75857421}
%%CONTEXT
%%LINE1
A(*ra)kora iki?\\
%%LINE2
\gll
a-ra-kór-a  ikí\\
%%LINE3
1\SM-\PRS.\DJ{}-do-\FV{}  what\\
%%TRANS1
\glt
‘What does s/he do?’\\
%%TRANS2
%%EXEND

\z

%%EAX
\ea
%%JUDGEMENT
%%LABEL
\label{bkm:Ref75857415}
%%CONTEXT
%%LINE1
Aríko ya(*ra)vyáara abakoóbwa gusa.\\
%%LINE2
\gll
ariko  a-á-(ra-)vyáar-a  a-ba-koóbwa  gusa\\
%%LINE3
but  1\SM-\RMT.\PST{}-(\DJ{}-)give.birth-\FV{}  \AUG{}-2-girl  only\\
%%TRANS1
\glt
‘But she gave birth to girls only.’\\
%%TRANS2
%%EXEND

 \citep[403]{NshemezimanaBostoen2017}
\z

We thus propose a characterisation of the \CJ{} form not as a default but as a form triggering alternatives for a postverbal constituent. This in turn suggests a characterisation of the \DJ{} form as triggering an alternative set NOT for a postverbal constituent. This negative characterisation is in line with its underspecified interpretation: the \DJ{} form is used for different types of predicate-centred focus \citep{Güldemann2010}, specifically TAM focus, verum, and state-of-affairs focus (see extensive argumentation in \citealt{NshemezimanaBostoen2017}). It is also used in thetic inversion constructions such as \xref{bkm:Ref76374640}, where the main motivation is the detopicalisation of the logical subject (see further \sectref{bkm:Ref75783474}). If any alternatives are generated, then they concern the whole assertion – \citegen{Lambrecht1994} ``sentence focus". Crucially, the \CJ{} form in subject inversion, if possible, expresses focus on the postverbal logical subject, as in \xref{bkm:Ref78543650}.

\ea
\label{bkm:Ref76374640}
\ea
(Context: Reporting what you saw at the neighbour’s to someone who was not there.)\\
\begin{xlist}
%%EAX
\exi{\DJ{}}
%%JUDGEMENT
%%LABEL
%%CONTEXT
%%LINE1
Harapfuuye impené.\\
%%LINE2
\gll
ha-ra-pfú-ye  i-n-hené\\
%%LINE3
16\SM-\DJ{}-die-\PFV{}  \AUG{}-9-goat\\
%%TRANS1
\glt
‘A/the goat has died.’\\
%%TRANS2
%%EXEND

\end{xlist}

\ex
%\label{ex:kirundi:56}
\label{bkm:Ref78543650}
%\label{ex:kirundi:1}\\
(Context: Reporting what happened to someone who, hearing that an animal in his neighbour’s herd has just died, wants to know which animal it is.)\\
\begin{xlist}
%%EAX
\exi{\CJ{}}
%%JUDGEMENT
%%LABEL
%%CONTEXT
%%LINE1
Hapfuuye impené.\\
%%LINE2
\gll
ha-$\varnothing$-pfú-ye  i-n-hené\\
%%LINE3
16\SM{}-\CJ{}-die-\PFV{}  \AUG{}-9-goat\\
%%TRANS1
\glt
‘A/the goat has died.’\\
%%TRANS2
%%EXEND

\end{xlist}
\z
\z

\citet{NshemezimanaBostoen2017} mention that there is one context in which the choice of verb form is ``up to the speaker" and no systematic difference in interpretation can be found. This is in environments for VP focus (Lambrecht’s predicate focus). Nevertheless, in specific examples we can detect a difference between VP focus with a \CJ{} or with a \DJ{} form. In \xref{bkm:Ref76375155}, the \CJ{} form suggests that washing dishes is what s/he did, possibly to the exclusion of doing other things, whereas the \DJ{} form has a more casual interpretation, suggesting other activities were also performed. 

\ea
\label{bkm:Ref76375155}(\textit{Yakora iki?} ‘What did s/he do?’)

%\label{ex:kirundi:56}
\ea
\begin{xlist}
%%EAX
\exi{\CJ{}}
%%JUDGEMENT
%%LABEL
%%CONTEXT
%%LINE1
Yooza ivyombo.\\
%%LINE2
\gll
a-á-$\varnothing$-óoz-a  i-bi-ombo\\
%%LINE3
1\SM-\RMT.\PST{}-\CJ{}-wash-\FV{}  \AUG{}-8-dishes\\
%%TRANS1
\glt
‘S/he washed dishes.’ (habitually)\\
%%TRANS2
%%EXEND

\end{xlist}

\ex
\begin{xlist}
%%EAX
\exi{\DJ{}}
%%JUDGEMENT
%%LABEL
%%CONTEXT
%%LINE1
Yarooza ivyombo.\\
%%LINE2
\gll
a-á-ra-óoz-a  i-bi-ombo\\
%%LINE3
1\SM{}-\RMT.\PST{}-\DJ{}-wash-\FV{}  \AUG{}-8-dishes\\
%%TRANS1
\glt
‘S/he washed dishes.’ (sometimes, and sometimes s/he went to work on the land)\\
%%TRANS2
%%EXEND

\end{xlist}

\z
\z


These new data thus help us to understand that the \CJ{} form is not the unmarked ``elsewhere" form that \citet{NshemezimanaBostoen2017} propose it to be, but triggers alternatives on the element following the \CJ{} verb form. This generalisation holds in those environments where the \DJ{} form is grammatically also possible, were it not for the context. This excludes relative and negative conjugations, for example, as these do not show an alternation but only allow one form (formally equal to the \CJ{} form). In line with \citet{NshemezimanaBostoen2017}, we conclude that the \CJ{}/\DJ{} alternation in Kirundi is directly affected by information structure, unlike in neighbouring Kinyarwanda, which \citet{NgobokaZeller2017} analyse as constituent-based.

\section{Word order}
\label{bkm:Ref81485307}
Word order in Kirundi is largely determined by information structure. The preverbal domain is restricted to non-focal elements and highly preferred for topical elements, whereas the postverbal domain is characterised as non-topical, with the clause-final position reserved for focus. Example \xref{bkm:Ref74312415} illustrates that it is the clause-final constituent that is in focus.\largerpage[2]

\ea
\label{bkm:Ref74312415}
%%EAX
\ea
%%JUDGEMENT
%%LABEL
%%CONTEXT
%%LINE1
Nya muvyéeyi yaheereye amatá abáana mu nzu.\\
%%LINE2
\gll
nya  mu-vyéeyi  a-a-$\varnothing$-há-ir-ye  a-ma-tá  a-bá-ana   mu  n-zu \\
%%LINE3
1.\DEM{}\textsubscript{7}  1-mother  1\SM-\N.\PST{}-\CJ{}-give-\APPL-\PFV{}  \AUG-{}6-milk  \AUG-{}2-child   18  9-house \\
%%TRANS1
\glt
    ‘The mother (that we were talking about) gave milk to the children in [the \textit{house}]\textsubscript{FOC} (not outside).’\\
%%TRANS2
%%EXEND

%%EAX
\ex
%%JUDGEMENT
%%LABEL
%\label{ex:kirundi:101}
%%CONTEXT
%%LINE1
Nya muvyéeyi yaheereye abáana mu nzu amatá.\\
%%LINE2
\gll
nya  mu-vyéeyi  a-a-$\varnothing$-há-ir-ye  a-ba-áana  mu  n-zu  a-ma-tá\\
%%LINE3
1.\DEM{}\textsubscript{7}  1-mother  1\SM-\N.\PST-\CJ{}-give-\APPL-\PFV{}  \AUG-2{}-child  18  9-house  \AUG-6{}-milk\\
%%TRANS1
\glt
‘The mother (that we were talking about) gave [the \textit{milk}]\textsubscript{FOC} (not the bread) to the children in the house.’\\
%%TRANS2
%%EXEND

%%EAX
\ex
%%JUDGEMENT
%%LABEL
%%CONTEXT
%%LINE1
Nya muvyéeyi yaheereye amatá mu nzu abáana.\\
%%LINE2
\gll
nya  mu-vyéeyi  a-a-$\varnothing$-há-ir-ye  a-ma-tá  mu  n-zu  a-ba-áana\\
%%LINE3
1.\DEM{}\textsubscript{7}  1-mother  1\SM-\N.\PST-\CJ{}-give-\APPL-\PFV{}  \AUG-6{}-milk  18  9-house  \AUG-2{}-child\\
%%TRANS1
\glt
‘The mother (that we were talking about) gave the milk to [the \textit{children}]\textsubscript{FOC} (not the husband) in the house.’\\
%%TRANS2
%%EXEND

\z
\z

We discuss the interpretational restrictions and preferences for the preverbal and postverbal domain subsequently, paying special attention to sentence-final focus (\sectref{bkm:Ref81485890}) and subject inversion constructions (\sectref{bkm:Ref75783474}).

\subsection{Preverbal domain}

As in other Bantu languages (see \citealt{Zerbian2006a,Yoneda2011,vanderWal2009} among others), in Kirundi there is a ban on focus in a preverbal position, as shown for content question words in \xref{bkm:Ref72332402:a} and \xref{bkm:Ref105321835:a} and elements modified by ‘only’ in \xref{bkm:Ref105321684:a} and \xref{bkm:Ref142557423:a}.\footnote{Content questions and their answers are inherently focused, and the particle ‘only’ is an exhaustive focus-sensitive particle, which therefore modifies a focused element. As described in \textcite{chapters/intro}, these are used as diagnostics for focus.} Instead, the focused element occurs postverbally, or in a cleft construction. The postverbal focus is illustrated in \xref{bkm:Ref105321835:b} and \xref{bkm:Ref142557423:b} for objects, and in the subject inversion construction in \xref{bkm:Ref72332402:c} and \xref{bkm:Ref105321835} (see further in \sectref{bkm:Ref72335068} on subject inversion). Example \xref{bkm:Ref72332402:b} illustrates a clefted subject (see \sectref{bkm:Ref72334782} on clefts). We do not illustrate all the possible grammatical versions but refer to the relevant sections – for now, the take-away point is that focus may not occur preverbally.

\ea
\label{bkm:Ref72332402}
%%EAX
\ea
%%JUDGEMENT
[*]{
%%LABEL
\label{bkm:Ref72332402:a}
%%CONTEXT
%%LINE1
Ndé yiinjíye/aríinjiye?\\
%%LINE2
\gll
ndé  a-$\varnothing$-íinjir-ye / a-$\varnothing$-ra-íinjir-ye\\
%%LINE3
1.who  1\SM-\PRS{}-come.in-\PFV{} / 1\SM-\PRS-\DJ{}-come.in-\PFV{}\\
%%TRANS1
\glt
int. ‘Who comes in?’\\
%%TRANS2
}
%%EXEND

%\label{ex:kirundi:101}
 %\label{ex:kirundi:1}
%%EAX
\ex
%%JUDGEMENT
[]{
%%LABEL
\label{bkm:Ref72332402:b}
%%CONTEXT
%%LINE1
Ni ndé asohotse?  \jambox*{[cleft]}
%%LINE2
\gll
ni  ndé  a-$\varnothing$-sohok-ye\textsuperscript{H}\\
%%LINE3
\COP{}  1.who  1\SM-\PRS-{}go.out-\PFV.\REL{}\\
%%TRANS1
\glt
‘Who (is the one who) goes out?’\\
%%TRANS2
}
%%EXEND

%%EAX
\ex
%%JUDGEMENT
[]{
%%LABEL
\label{bkm:Ref72332402:c}
%%CONTEXT
%%LINE1
Haje ndé?    \jambox*{[subject inversion]}
%%LINE2
\gll
ha-$\varnothing$-əz-ye  ndé\\
%%LINE3
\EXP{}-\PRS.\CJ{}-come-\PFV{}  who\\
%%TRANS1
\glt
‘Who comes?’\\
%%TRANS2
}
%%EXEND

\z
\z

\ea\label{bkm:Ref105321835}
%%EAX
\ea
%%JUDGEMENT
[*]{
%%LABEL
\label{bkm:Ref105321835:a}
%%CONTEXT
%%LINE1
%%LINE2
\gll 
Iki  u-a-bón-ye? / u-a-a-bón-ye\\
%%LINE3
what  2\SG.\SM-\N.\PST-{}see-\PFV{} / 2\SG.\SM-\N.\PST-\DJ-{}see-\PFV{}\\
%%TRANS1
\glt
int. ‘What did you see?’\\
%%TRANS2
}
%%EXEND


%%EAX
\ex
%%JUDGEMENT
[]{
%%LABEL
\label{bkm:Ref105321835:b}
%%CONTEXT
%%LINE1
Wabonye iki?\\
%%LINE2
\gll
u-a-$\varnothing$-bón-ye  iki?\\
%%LINE3
2\SG.\SM-\N.\PST-\CJ{}-{}see-\PFV{}  what?\\
%%TRANS1
\glt
‘What did you see?’\\
%%TRANS2
}
%%EXEND

\z
\z

\ea
\label{bkm:Ref105321684}
%%EAX
\ea
%%JUDGEMENT
[*]{
%%LABEL
\label{bkm:Ref105321684:a}
%%CONTEXT
%%LINE1
Abagabo babiri gusa barasohotse.\\
%%LINE2
\gll
abagabo  babiri  gusa  ba-ra-sohok-ye\\
%%LINE3
2.man  2.two  only  2\SM-\PRS.\DJ{}-go.out-\PFV{}\\
%%TRANS1
\glt
Int: ‘Only two men went out.’\\
%%TRANS2
}
%%EXEND

%%EAX
\ex
%%JUDGEMENT
[]{
%%LABEL
\label{bkm:Ref105321684:b}
%%CONTEXT
%%LINE1
Hasohotse abagabo babiri gusa.      \jambox*{[subject inversion]}
%%LINE2
\gll
ha-$\varnothing$-sohok-ye  abagabo  babiri  gusa\\
%%LINE3
\EXP-\PRS.\CJ{}-go.out-\PFV{}  2.man  2.two  only\\
%%TRANS1
\glt
‘Only two men go out.’\\
%%TRANS2
}
%%EXEND

\z\z

\ea
\label{bkm:Ref142557423}
%%EAX
\ea
%%JUDGEMENT
[*]{
%%LABEL
\label{bkm:Ref142557423:a}
%%CONTEXT
%%LINE1
Imbwá gusa naabónye.\\
%%LINE2
\gll
imbwa  gusa  \N{}-a-a-bón-ye\\
%%LINE3
9.dog  only  1\SG.\SM-\N.\PST-\DJ{}-see-\PFV{}\\
%%TRANS1
\glt
int. ‘I saw only a dog.’\\
%%TRANS2
}
%%EXEND

%\label{ex:kirundi:101}
%%EAX
\ex
%%JUDGEMENT
[]{
%%LABEL
\label{bkm:Ref142557423:b}
%%CONTEXT
%%LINE1
Naabónye imbwá gusa.\\
%%LINE2
\gll
\N{}-a-$\varnothing$-bón-ye  imbwa  gusa\\
%%LINE3
1\SG.\SM-\N.\PST-\CJ{}-see-\PFV{}  9.dog  only\\
%%TRANS1
\glt
int. ‘I saw only a dog.’\\
%%TRANS2
}
%%EXEND

\z\z

The preference for topics to occur preverbally can be seen in left-dislocated arguments, such as \textit{abaana} ‘the children’ in \xref{bkm:Ref73810020} and frame-setting elements like \textit{harya} ‘there’ in \xref{bkm:Ref73810045}, both given in the context. (See \citealt{chapters/intro} for further background on the different types of topics.) The expression of shift topics remains to be examined.

\ea
(Where were you taking the kids to this morning?)\\
%%EAX
\ea
%%JUDGEMENT
%%LABEL
\label{bkm:Ref73810020}
%%CONTEXT
%%LINE1
Abáana, twaari túbajanye kw’ishuúre.\\
%%LINE2
\gll
a-ba-áana  tu-a-ri  tu-ba-gi-an-ye  kw’  i-shuúre\\
%%LINE3
\AUG-2{}-child  1\PL.\SM-\N.\PST{}-be  1\PL.\SM-2\OM-{}go-\ASS-\PFV{}  17  \AUG-5{}.school\\
%%TRANS1
\glt
‘The kids, we were taking them to school.’\\
%%TRANS2
%%EXEND

%%EAX
\ex
%%JUDGEMENT
%%LABEL
\label{bkm:Ref73810045}
%%CONTEXT
%%LINE1
Harya twaari túbajanye kw’ishuúre.\\
%%LINE2
\gll
ha-rya  tu-a-ri  tu-ba-gi-an-ye  kw’  i-shuúre\\
%%LINE3
16-\DEM{}\textsubscript{3}  1\PL.\SM-\N.\PST{}-be  1\PL.\SM-2\OM{}-go-\ASS-\PFV{}  17  \AUG-5{}.school\\
%%TRANS1
\glt
‘There (where you saw me) we were taking them to school.’\\
%%TRANS2
%%EXEND

\z
\z

It is also possible for multiple topic expressions to occur preverbally, either specifying ever smaller subsets, for example of location in \xref{bkm:Ref74316186}, or narrowing down the referent by intersection of time and place, as in \xref{bkm:Ref74318587}.

\ea
(Context: The journalist gives information in a radio broadcast.)\\
%%EAX
\ea
%%JUDGEMENT
%%LABEL
%\label{ex:kirundi:131}
\label{bkm:Ref74316186}
%%CONTEXT
%%LINE1
I Ngozi, muri zone Mwumba, ku musozi Karungura, hariho umuvyéeyi yiibáarutse ubushuuri.\\
%%LINE2
\gll
  i  Ngozi  muri  zone  Mwumba  ku  mu-sozi  Karungura  ha-$\varnothing$-ri=hó  u-mu-vyéeyi  a-a-íibáaruk-ye\textsuperscript{H}  u-bu-shuuri\\
%%LINE3
  19  Ngozi  18  5.area  Mwumba  17  3-hill  Karungura  \EXP-\PRS{}-be=16  \AUG-1-{}mother  1\SM-\N.\PST{}-give.birth-\PFV.\REL{}  \AUG{}-14-triplet\\
%%TRANS1
\glt
  ‘In Ngozi, in the Mwumba area, on Karungura hill, there is a mother who gave birth to triplets.’\\
%%TRANS2
%%EXEND

%%EAX
\ex
%%JUDGEMENT
%%LABEL
\label{bkm:Ref74318587}
%%CONTEXT
%%LINE1
Ku mugórooba w’eéjo, i Ngozi, harihó umuvyéeyi yiíbáarutse ubushuuri.\\
%%LINE2
\gll
ku  mu-górooba  u-a  ejó  i  Ngozi  ha-$\varnothing$-ri=hó  u-mu-vyéeyi   a-a-íibáaruk-ye\textsuperscript{H}  u-bu-shuuri \\
%%LINE3
17  3-evening  3-\CONN{}  yesterday  19  Ngozi  \EXP-\PRS{}-be=16  \AUG-1-{}mother   1\SM-\N.\PST{}-give.birth-\PFV.\REL{}  \AUG{}-14-triplet \\
%%TRANS1
\glt
  ‘Yesterday evening, in Ngozi, there is a mother who gave birth to triplets.’\\
%%TRANS2
%%EXEND

\z
\z

Multiple active participants may also be expressed as preverbal full NPs or pronouns, illustrated in \xref{bkm:Ref142552207}. Note that when both topical referents are expressed pronominally, the first referent can be analysed as an aboutness topic (indicating to the addressee which referent the following information is about), and the second is interpreted as a contrastive topic (i.e. an identifiable typically given referent that is contrasted with another given referent), as shown in \xxref{bkm:Ref142552207:b}{bkm:Ref142552207:c}.

\ea
\label{bkm:Ref142552207}(Context: Reaction to the question asking whether the dogs normally like (eating) sweet potatoes.)
%\label{ex:kirundi:3}
%%EAX
\ea
%%JUDGEMENT
%%LABEL
\label{bkm:Ref142552207:a}
%%CONTEXT
%%LINE1
Imbwá, ibijumbu, zirabíkunda.\\
%%LINE2
\gll
i-n-bwa  i-bi-jumbu  zi-ra-bi-kúund-a\\
%%LINE3
\AUG-10{}-dog  \AUG-8-{}sweet.potato  10\SM-\PRS.\DJ{}-8\OM{}-like-\FV{}\\
%%TRANS1
\glt
‘Dogs, the sweet potatoes, they like them.’\\
%%TRANS2
%%EXEND

%%EAX
\ex
%%JUDGEMENT
%%LABEL
\label{bkm:Ref142552207:b}
%%CONTEXT
%%LINE1
Zó, vyó, zirabikunda.\\
%%LINE2
\gll
zi-ó  bi-ó  zi-ra-bi-kúund-a\\
%%LINE3
10-\PRO{}  8-\PRO{}  10\SM-\PRS.\DJ-8\OM-{}like-\FV{}\\
%%TRANS1
\glt
‘Them (dogs), as for as them (sweet potatoes), they like them.’ (but not the beans)\\
%%TRANS2
%%EXEND

%%EAX
\ex
%%JUDGEMENT
%%LABEL
\label{bkm:Ref142552207:c}
%%CONTEXT
%%LINE1
Vyó, zó, zirabikunda.\\
%%LINE2
\gll
bi-ó  zi-ó  zi-ra-bi-kúund-a\\
%%LINE3
8-\PRO{}  10-\PRO{}  10\SM-\PRS.\DJ-8\OM-{}like-\FV{}\\
%%TRANS1
\glt
‘Them (sweet potatoes), as for as them (dogs), they like them.’ (but the cat doesn't)\\
%%TRANS2
%%EXEND

\z
\z

However, it is also possible for non-topical subjects to occur preverbally, as illustrated in \xref{bkm:Ref75348847} and \xref{bkm:Ref75339354}. These examples show that indefinite non-specific subjects, which cannot function as topics, can occupy a preverbal position.

%%EAX
\ea
%%JUDGEMENT
%%LABEL
\label{bkm:Ref75348847}
%%CONTEXT
(Why are the people shouting?)\\
%%LINE1
Umuntu arapfúye.\\
%%LINE2
\gll
u-mu-ntu  a-ra-pfú-ye\\
%%LINE3
\AUG{}-1-person  1\SM-\PRS.\DJ{}-die-\PFV{}\\
%%TRANS1
\glt
‘Someone has (just) died.’\\
%%TRANS2
%%EXEND

\z

%%EAX
\ea
%%JUDGEMENT
%%LABEL
\label{bkm:Ref75339354}
%%CONTEXT
(Context: Reaction to seeing the milk spilled on the floor.)\\
%%LINE1
Umuntu yaasheeshe amatá.\\
%%LINE2
\gll
u-mu-ntu  a-a-sees-ye  a-ma-tá\\
%%LINE3
\AUG-1{}-person  1\SM-\N.\PST.\DJ{}-pour-\PFV{}  \AUG{}-6-milk\\
%%TRANS1
\glt
‘Someone poured the milk.’\\
%%TRANS2
%%EXEND

\z

In summary, the preverbal domain may not be occupied by focused constituents, and is preferred for topics, which may be contrastive. Nevertheless, the preverbal domain cannot be characterised as only topical, since indefinite non-specific subjects may also appear preverbally. 

\subsection{Postverbal domain}
\label{bkm:Ref72335068}
The verb phrase typically functions as the comment to the preverbal topic. The whole VP can form the focus, as in the storyline in \xref{bkm:Ref142555660} or with the cognate object in \xref{bkm:Ref142555662}, or the focus can be only one constituent within the comment, which appears in clause-final position (on which more in \sectref{bkm:Ref81485890}). 

%%EAX
\ea
%%JUDGEMENT
%%LABEL
\label{bkm:Ref142555660}
%%CONTEXT
%%LINE1
Baratérama, [barasoma akúuki]\textsubscript{FOC}…  \\
%%LINE2
\gll
ba-ra-teram-a  ba-ra-som-a  a-ka-úuki… \\
%%LINE3
2\SM{}-\DJ{}-rest-\FV{}  2\SM{}-\DJ{}-drink-\FV{}  \AUG{}-12-honey.drink\\
%%TRANS1
\glt
‘They rested and drank the honey drink, …'\\
%%TRANS2
%%EXEND

\z

\ea
\label{bkm:Ref142555662}
(Context: Having arrived in class, a teacher finds that the students have not yet done the homework s/he left them the day before and asks them the following:)

\begin{xlist}
%%EAX
\exi{Q:}{
%%JUDGEMENT
%%LABEL
%%CONTEXT
%%LINE1
None ejó mwaákoze ikí?\\
%%LINE2
\gll
none  ejó  mu-á-$\varnothing$-kór-ye  ikí\\
%%LINE3
then  yesterday  2\PL{}.\SM{}-\RMT.\PST{}-\CJ{}-do-\PFV{}  what\\
%%TRANS1
\glt
 ‘What did you do yesterday then?’ \\
%%TRANS2
}
%%EXEND

%%EAX
\exi{A:}{
%%JUDGEMENT
%%LABEL
%%CONTEXT
%%LINE1
Twaárarírimvye indírimbo.\\
%%LINE2
\gll
tu-á-ra-rírimb-ye  i-n-rírimbo\\
%%LINE3
1\PL{}.\SM{}-\RMT.\PST{}-\DJ{}-sing-\PFV{}  \AUG{}-10-song\\
%%TRANS1
\glt
  ‘We sang songs.’\\
%%TRANS2
}
%%EXEND

\end{xlist}
\z

As in other languages, the right periphery of the sentence can also host a right-dislocated element, interpreted as an afterthought, like \textit{iyo} \textit{mbwa} ‘that dog’ in \xref{bkm:Ref73818128}.

%%EAX
\ea
%%JUDGEMENT
%%LABEL
\label{bkm:Ref73818128}
%%CONTEXT
%%LINE1
Duhejeje kuyigaburira, iyo mbwá.\\
%%LINE2
\gll
tu-$\varnothing$-hez-ye  ku-yi-gaburira  i-i-o  n-bwá.\\
%%LINE3
1\PL.\SM-\PRS.\CJ-{}finish-\PFV{}  15-9\OM{}-feed  \AUG-9-\DEM{}\textsubscript{2}  9-dog\\
%%TRANS1
\glt
‘We just fed it, that dog.’\\
%%TRANS2
%%EXEND

\z

Material that is neither topical nor focal can also appear in the postverbal domain. This is typically the case for subject inversion constructions, discussed and further illustrated in \sectref{bkm:Ref75783474}. The postverbal domain thus hosts non-topical information, and this may be focal, but only in clause-final position, as shown in the next section.

\subsubsection{Sentence-final focus}
\label{bkm:Ref81485890}
Focused elements appear in sentence-final position or in a cleft construction. The clefts are presented in \sectref{bkm:Ref72334782}, and here we discuss the clause-final focus position. This position, as identified by \citet{Sabimana1986} and \citet{Ndayiragije1999}, is shown for Theme and Recipient arguments of ditransitive verbs in the following examples: content questions in \xref{bkm:Ref72332331} and \xref{bkm:Ref79838134}, answers to those questions in \xref{bkm:Ref72332345} and \xref{bkm:Ref72332347}, and arguments modified by the exhaustive focus-sensitive particle ‘only’ in \xref{bkm:Ref72332358} and \xref{bkm:Ref72332359}. In each example we see that the focused element (the question word, answer, or constituent modified by ‘only’) must appear as the second of the two objects, in the clause-final focus position.

\ea
\ea
%%EAX
\begin{xlist}
\exi{\CJ}
%%JUDGEMENT
[]{
%%LABEL
\label{bkm:Ref72332331}
%%CONTEXT
%%LINE1
Uhaaye umwáana iki?\\
%%LINE2
\gll
u-$\varnothing$-há-ye  u-mw-áana  ikí?\\
%%LINE3
2\SG.\SM-\PRS{}-give-\PFV{}  \AUG{}-1-child  what\\
%%TRANS1
\glt
‘What have you given to the child?’\\
%%TRANS2
}
\end{xlist}
%%EXEND

\ex
%%EAX
\begin{xlist}
\exi{\CJ}
%%JUDGEMENT
[*]{
%%LABEL
%%CONTEXT
%%LINE1
%%LINE2
\gll
U-$\varnothing$-há-ye  ikí  u-mw-áana?\\
%%LINE3
2\SG.\SM-\PRS{}-give-\PFV{}  what  \AUG{}-1-child\\
%%TRANS1
\glt
%%TRANS2
}
\end{xlist}
%%EXEND


\z
\z


\ea
\ea
%%EAX
\begin{xlist}
\exi{\CJ}
%%JUDGEMENT
[]{
%%LABEL
\label{bkm:Ref79838134}
%%CONTEXT
%%LINE1
Uhaaye ikijumbu ndé?\\
%%LINE2
\gll
u-$\varnothing$-há-ye  i-ki-jumbu  ndé?\\
%%LINE3
2\SG.\SM-\PRS{}-give-\PFV{}  \AUG{}-7-sweet.potato  who\\
%%TRANS1
\glt
‘Who do you give a sweet potato?’\\
%%TRANS2
}
\end{xlist}
%%EXEND

\ex
%%EAX
\begin{xlist}
\exi{\CJ}
%%JUDGEMENT
[*]{
%%LABEL
%\label{ex:kirundi:101}
%%CONTEXT
%%LINE1
%%LINE2
\gll
U-$\varnothing$-há-ye  ndé  i-ki-jumbu?\\
%%LINE3
2\SG.\SM-\PRS{}-give-\PFV{}  who \AUG{}-7-sweet.potato\\
%%TRANS1
\glt
%%TRANS2
}
\end{xlist}
%%EXEND

\z
\z

\ea
\label{bkm:Ref72332345}
(What do you give to the child ?)
\ea
%%EAX
\begin{xlist}
\exi{\CJ{}}
%%JUDGEMENT
[]{
%%LABEL
%%CONTEXT
%%LINE1
Mpaaye umwáana ikijumbu.\\
%%LINE2
\gll
\N{}-$\varnothing$-há-ye  u-mw-áana  i-ki-jumbu.\\
%%LINE3
1\SG.\SM-\PRS.\CJ{}-give-\PFV{}  \AUG{}-1-child  \AUG{}-7-sweet.potato\\
%%TRANS1
\glt
‘I give the child a sweet potato.’\\
%%TRANS2
}
\end{xlist}
%%EXEND
\pagebreak
\ex
%%EAX
\begin{xlist}
\exi{\CJ{}}
%%JUDGEMENT
[]{
%%LABEL
%%CONTEXT
%%LINE1
Ndamuhaaye ikijumbu.\\
%%LINE2
\gll
\N{}.ra-$\varnothing$-mu-ha-ye  i-ki-jumbu.\\
%%LINE3
1\SG.\SM-\PRS.\CJ-1\OM-{}give-\PFV{}  \AUG{}-7-sweet.potato\\
%%TRANS1
\glt
‘I give him/her a sweet potato.’\\
%%TRANS2
}
\end{xlist}
%%EXEND

\ex 
%%EAX
\begin{xlist}
\exi{\CJ}
%%JUDGEMENT
[\textsuperscript{\#}]{
%%LABEL
%%CONTEXT
%%LINE1
%%LINE2
\gll
\N{}.ra-$\varnothing$-mu-ha-ye  i-ki-jumbu  u-mw-áana\\
%%LINE3
1\SG.\SM-\PRS.\CJ-1\OM-{}give-\PFV{}  \AUG{}-7-sweet.potato  \AUG{}-1-child\\
%%TRANS1
\glt
%%TRANS2
}
\end{xlist}
%%EXEND


\z
\z

\ea
\label{bkm:Ref72332347}
(Who do you give sweet potato?)\\
\ea
%%EAX
\begin{xlist}
\exi{\CJ}
%%JUDGEMENT
%%LABEL
%%CONTEXT
%%LINE2
\gll
\N-{}$\varnothing$-há-ye  i-ki-jumbu  u-mw-áana.\\
%%LINE3
1\SG.\SM-\PRS.\CJ{}-give-\PFV{}  \AUG{}-7-sweet.potato  \AUG{}-1-child\\
%%TRANS1
\glt
‘I give the sweet potato to a child.’\\
%%TRANS2
\end{xlist}
%%EXEND

\ex  
%%EAX
\begin{xlist}
\exi{\CJ}
%%JUDGEMENT
%%LABEL
%%CONTEXT
%%LINE1
%%LINE2
\gll
\N-{}$\varnothing$-ki-há-ye  u-mw-áana.\\
%%LINE3
1\SG.\SM-\PRS.\CJ{}-7\OM{}-give-\PFV{}  \AUG{}-1-child\\
%%TRANS1
\glt
      ‘I give it to a/the child.’\\
%%TRANS2
\end{xlist}
%%EXEND

\z
\z



\ea
\label{bkm:Ref72332358}

%%EAX
\ea
%%JUDGEMENT
[]{
%%LABEL
%%CONTEXT
%%LINE1
%%LINE2
\gll
Tu-a-\{mu\}-ha-ye  \{Kabura\}  i-bi-jumbu  gusa\\
%%LINE3
1\PL.\SM-\N.\PST.\CJ{}-\{1\OM{}\}-give-\PFV{}  {\db}1.Kabura  \AUG{}-8-sweet.potato  only\\
%%TRANS1
\glt
‘We gave Kabura/him only sweet potatoes.’\\
%%TRANS2
}
%%EXEND

%%EAX
\ex
%%JUDGEMENT
[*]{
%%LABEL
%%CONTEXT
%%LINE1
%%LINE2
\gll
Tu-a-(mu-)ha-ye  i-bi-jumbu  gusa  Kabura.\\
%%LINE3
1\PL.\SM-\N.\PST.\CJ{}-(1\OM{}-)give-\PFV{}  \AUG{}-8-sweet.potato  only  1.Kabura\\
%%TRANS1
\glt
%%TRANS2
}
%%EXEND


\z
\z

\ea
%%EAX
\ea
%%JUDGEMENT
[]{
%%LABEL
%\label{ex:kirundi:101}
\label{bkm:Ref72332359}
%%CONTEXT
%%LINE1
%%LINE2
\gll
Tu-a-\{bi-\}ha-ye  {i-bi-jumbu}  Kabura  gusa.\\
%%LINE3
1\PL.\SM-\N.\PST.\CJ-\{8\OM-\}give-\PFV{}  {\AUG{}-8-sweet.potato}  1.Kabura  only\\
%%TRANS1
\glt
    ‘We gave sweet potatoes/them to Kabura only.’\\
%%TRANS2
}
%%EXEND

%%EAX
\ex
%%JUDGEMENT
[*]{
%%LABEL
%%CONTEXT
%%LINE1
%%LINE2
\gll
Tu-a-(bi-)ha-ye  Kabura  gusa  i-bi-jumbu.\\
%%LINE3
1\PL.\SM-\N.\PST.\CJ-(8\OM-)give-\PFV{}  1.Kabura  only  \AUG{}-8-sweet.potato\\
%%TRANS1
\glt
%%TRANS2
}
%%EXEND

\z
\z

Kirundi thus shows evidence for a dedicated clause-final focus position.

\subsubsection{Subject inversion constructions}
\label{bkm:Ref75783474}
In keeping with the tendency for preverbal elements to be topical and focal elements to be postverbal, the logical subject \textit{must} stay in a postverbal position when it is focal and \textit{prefers} a postverbal position when it is non-topical (alternatives being the various clefts described in \sectref{bkm:Ref72334782}). Such word orders are known as subject inversion constructions \citep{Zerbian2006a}. Kirundi is exceptional in showing all seven types of subject inversion constructions listed by \citet{MartenvanderWal2014}, including Patient Inversion, Semantic and Formal Locative Inversion, Instrument Inversion, Complement Inversion, Default Agreement Inversion and Agreeing Inversion. These are discussed extensively by \citet{Nshemezimana2016,Ndayiragije1999,Morimoto2000,Morimoto2006,Morimoto2009} and \citet{Sabimana1986}, and are presented in turn below.

In Patient Inversion, the patient occupies a preverbal position, determining subject agreement on the verb, and the logical subject is clause-final. In \xref{bkm:Ref141643390}, for example, the theme \textit{amaazi} ‘water’ is preverbal, determining subject marking in class 6, and the logical subject ‘Yohana’ appears postverbally. The postverbal logical subject is interpreted as focused in Patient Inversion, as shown by the compatibility with \textit{gusa} ‘only’ in \xref{bkm:Ref73976448}, and the simple focus in answer to a content question in \xref{bkm:Ref75346778}. Patient Inversion takes the conjoint verb form, and cannot be used in thetic contexts. For pragmatic restrictions on animacy in Patient Inversion, see discussion in \citet{Morimoto2006} and \citet{Nshemezimana2016}.

%%EAX
\ea
%%JUDGEMENT
%%LABEL
\label{bkm:Ref73976448}
%%CONTEXT
(When will you be healed?)\\
%%LINE1
Sinzí. Ivyo biizi Imáana gusa.\\
%%LINE2
\gll
si-n-zí  i-bi-o  bi-zi  i-máana  gusa\\
%%LINE3
\NEG-1\SG.\SM{}-know  \AUG-8-\DEM{}\textsubscript{2}  8\SM-{}know  \AUG-9{}.God  only\\
%%TRANS1
\glt
‘I don’t know, only God knows.’\\
%%TRANS2
%%EXEND

\z

%%EAX
\ea
%%JUDGEMENT
%%LABEL
%%CONTEXT
(Did Christian take pictures yesterday?)\\
%%LINE1
Amafoto yafashe Ernest, nayó Christian sinzí.\\
%%LINE2
\gll
a-ma-foto  a-a-$\varnothing$-fát-ye  Ernest  nayó  Christian  si-n-zi\\
%%LINE3
\AUG-6-{}photo  6\SM-\N.\PST-\CJ{}-take-\PFV{}  1.Ernest  but  1.Christian  \NEG-1\SG.\SM{}-know\\
%%TRANS1
\glt
‘\textit{Ernest} took pictures, but Christian I don’t know.’\\
%%TRANS2
%%EXEND

\z

\ea
\label{bkm:Ref75346778}
\label{bkm:Ref141643390}(Who brought water?)

%%EAX
\ea
%%JUDGEMENT
[]{
%%LABEL
%%CONTEXT
%%LINE1
Amáazi  aazanye  Yohana.\\
%%LINE2
\gll
amáazi  a-a-$\varnothing$-zan-ye  Yohana\\
%%LINE3
6.water  6\SM-\N.\PST-\CJ{}-bring-\PFV{}  1.John\\
%%TRANS1
\glt
  ‘\textit{John} brought water.’\\
%%TRANS2
}
%%EXEND

%%EAX
\ex
%%JUDGEMENT
[\textsuperscript{\#}]{
%%LABEL
%%CONTEXT
%%LINE1
Yohana  aazanye  amaazi.\\
%%LINE2
\gll
Yohana  a-a-$\varnothing$-zan-ye  amaazi\\
%%LINE3
1.John  1\SM-\N.\PST-\CJ{}-bring-\PFV{}  6.water\\
%%TRANS1
\glt
%%TRANS2
}
%%EXEND


\z
\z

In Formal Locative Inversion, a topical locative expression appears preverbally, subject marking is always class 16 \textit{ha}-, and the logical subject appears in final position forming the focus, as illustrated in \xref{bkm:Ref73976458}. The subject marker \textit{ha}- is formally the same as the expletive (see Default Agreement Inversion), but a difference is spotted in the fact that it can still refer to a location, as in \xref{bkm:Ref81464317}. Locative Inversion, too, is not found in thetic contexts but only with subject focus.

%%EAX
\ea
%%JUDGEMENT
%%LABEL
\label{bkm:Ref73976458}
%%CONTEXT
(Context: At a party, we want to know how many visitors have already been welcomed. Some people have been received in the garden and others in the house. The speaker gives the report for those who have been received in the house.)\\
%%LINE1
Mu nzu haakiiririwe abantu batanu.\\
%%LINE2
\gll
mu  n-zu  ha-a-$\varnothing$-akiir-ir-w-ye  a-ba-ntu  ba-tanu\\
%%LINE3
18  9-house  16\SM-\N.\PST-\CJ{}-reiceive-\APPL-\PASS-\PFV{}  \AUG-{}2-person  2-five\\
%%TRANS1
\glt
‘In the house there have been received five people.’\\
%%TRANS2
%%EXEND

\z

%%EAX
\ea
%%JUDGEMENT
%%LABEL
\label{bkm:Ref81464317}
%%CONTEXT
(Context: Seeing tracks in the sand and pointing, ‘What has passed here?’)\\
%%LINE1
  Haciiyeho imodoka.\\
%%LINE2
\gll
  ha-cí-ye=hó  i-modoka\\
%%LINE3
  16\SM{}-pass-\PFV{}=16  \AUG-9.car\\
%%TRANS1
\glt
  ‘Here passed a car.’\\
%%TRANS2
%%EXEND


\z

Apart from the locative class 16 subject marking, the verb in Locative Inversion can also have an enclitic in -\textit{o}, coreferent with one of the four locative classes: =\textit{hó} (cl.16), =\textit{mwó} (cl.18), =\textit{kó} (cl.17) and =\textit{yó} (cl.19) \citep[64]{Meeussen1959}. See also \citet{DevosEtAl2017} on Kirundi locative enclitics. The class 18 =\textit{mwó} is illustrated in \xref{bkm:Ref79856645}.

%%EAX
\ea
%%JUDGEMENT
%%LABEL
\label{bkm:Ref79856645}
%%CONTEXT
\citep[137]{Nshemezimana2016}\\
%%LINE1
Muri aya mavuriro, harabonekamwo umwahwa.\\
%%LINE2
\gll
murí  a-a-a  ma-vuuriro  ha-ra-bónek-a=mwó   u-mu-aáhwa \\
%%LINE3
18  \AUG-6-\DEM{}\textsubscript{1}  6-health.center  16\SM-\PRS.\DJ-{}be.seen-\FV=18{}   \AUG-3{}-traditional.medicine \\
%%TRANS1
\glt
  ‘In these health centres, there can be found traditional medicine.’\\
%%TRANS2
%%EXEND

\z

The enclitic is used when the locative noun phrase is dislocated, suggesting that the enclitic is a (resumptive) pronoun. This renders the locative phrase free in its position, being able to appear in initial \xref{bkm:Ref105422954:a}, central \xref{bkm:Ref105422954:b}, or final position \xref{bkm:Ref105422954:c}.\pagebreak

\ea
\label{bkm:Ref105422954}\citep[134]{Nshemezimana2016}
%\label{ex:kirundi:23}
%%EAX
\ea
%%JUDGEMENT
%%LABEL
\label{bkm:Ref105422954:a}
%%CONTEXT
%%LINE1
Murí aka gakino, harimwó ibihiímba bitatu.\\
%%LINE2
\gll
Murí  a-ka-a  ka-kino  ha-$\varnothing$-ri=mwó   i-bi-hiímba  bi-tatu \\
%%LINE3
18  \AUG-12-\DEM{}\textsubscript{1}  12-skit  16\SM-\PRS-{}be=18   \AUG-8-{}part  8-three \\
%%TRANS1
\glt
  ‘In this sketch, there are three parts.’\\
%%TRANS2
%%EXEND

  
%%EAX
\ex
%%JUDGEMENT
%%LABEL
\label{bkm:Ref105422954:b}
%%CONTEXT
%%LINE1
Harimwó, murí aka gakino, ibihiímba bitatu.\\
%%LINE2
\gll
ha-$\varnothing$-ri=mwó  murí  a-ka-a  ka-kino   i-bi-hiímba  bi-tatu \\
%%LINE3
16\SM-\PRS-{}be=18  18  \AUG-12-\DEM\textsubscript{1}  12-skit   \AUG-8-{}part  8-three \\
%%TRANS1
\glt
  ‘There are, in this sketch, three parts.’\\
%%TRANS2
%%EXEND

%%EAX
\ex
%%JUDGEMENT
%%LABEL
\label{bkm:Ref105422954:c}
%%CONTEXT
%%LINE1
Harimwó ibihiímba bitatu, murí aka gakino.\\
%%LINE2
\gll
ha-$\varnothing$-ri=mwó  i-bi-hiímba  bi-tatu  murí  a-ka-a  ka-kino\\
%%LINE3
16\SM-\PRS-{}be=18  \AUG-8-part  8-three  18  \AUG-12-\DEM\textsubscript{1}  12-skit\\
%%TRANS1
\glt
‘There are three parts, in this sketch.’\\
%%TRANS2
%%EXEND

\z\z


While \textit{ha}- is underspecified for locative semantics, the enclitic agrees in class with the dislocated locative noun phrase and specifies the exact location: among (with =\textit{mwó}) \xref{bkm:Ref80456712}, on (with =\textit{kó}) \xref{bkm:Ref80456733}, inside (with =\textit{yó}) \xref{bkm:Ref80456754},\footnote{Note that class 16 =\textit{ho} and 19 =\textit{yo} (but not 17 and 18) can be used to refer to locative noun phrases of class 18.} or with existential reference (=\textit{hó}) \xref{bkm:Ref80456783}.\footnote{It is interesting to note that the corpus study by \citet{NshemezimanaMberamihigo2021} reveals that the majority of locative enclitics is the class 18 =\textit{mwó}, with 95.74\% in the written part of the corpus, and 66.15\% in the oral part. This can perhaps be linked to its larger variation in use, alternating with =\textit{yo} in certain contexts, as in \xref{ex:kirundi:whatinhere}.

\ea 
Aha harimwo/yo iki?
\label{ex:kirundi:whatinhere}
%\label{ex:kirundi:28}
%%EAX
  \ea
%%JUDGEMENT
%%LABEL
%%CONTEXT
%%LINE1
%%LINE2
\gll
  {a-ha-a  ha-$\varnothing$-ri-mwó  ikí?}\\
%%LINE3
    \AUG-16-\DEM{}\textsubscript{1}  16\SM-\PRS{}-be=18  what\\
%%TRANS1
\glt
%%TRANS2
%%EXEND

%%EAX
  \ex
%%JUDGEMENT
%%LABEL
%%CONTEXT
%%LINE1
%%LINE2
\gll
  {a-ha-a  ha-$\varnothing$-ri-yó  ikí?}\\
%%LINE3
    \AUG-16-\DEM{}\textsubscript{1}  16\SM-\PRS{}-be=19  what\\
%%TRANS1
\glt
‘What’s in it here?’\\
%%TRANS2
%%EXEND

  \z
  \zlast
} 
The information-structural interpretation is still the same.


%%EAX
\ea
%%JUDGEMENT
%%LABEL
\label{bkm:Ref80456712}
%%CONTEXT
%%LINE1
Muri abo bantu harimwo uwugwaye korona.\\
%%LINE2
\gll
murí  a-ba-o  ba-ntu  ha-$\varnothing$-ri=mwó  u-wu-gwáar-ye  korona\\
%%LINE3
18  \AUG-2-\DEM{}\textsubscript{2}  2-person  16\SM-\PRS{}-be=18  \AUG-{}1-be.sick-\PFV{}  1.corona\\
%%TRANS1
\glt
‘Among these people, there is one who suffers from the corona virus.’\\
%%TRANS2
%%EXEND

\z

%%EAX
\ea
%%JUDGEMENT
%%LABEL
\label{bkm:Ref80456733}
%%CONTEXT
%%LINE1
Ku méezá haryamyeko akayáabu.\\
%%LINE2
\gll
ku  méezá  ha-ryáam-ye=kó  a-ka-yáabu\\
%%LINE3
17  table  16\SM{}-sleep-\PFV={}17  \AUG-{}12-cat\\
%%TRANS1
\glt
‘On the table a cat is sleeping.’\\
%%TRANS2
%%EXEND

\z

%%EAX
\ea
%%JUDGEMENT
%%LABEL
\label{bkm:Ref80456754}
%%CONTEXT
%%LINE1
Mu nzu hinjiyeyo inzoka.\\
%%LINE2
\gll
mu  n-zu  ha-a-$\varnothing$-injir-ye=yó  i-n-zoka\\
%%LINE3
18  9-house  16\SM{}-\N.\PST-\CJ{}-enter-\PFV{}=19  \AUG{}-9-snake\\
%%TRANS1
\glt
‘In(to) the house has entered a snake.’\\
%%TRANS2
%%EXEND

%%EAX
\ex
%%JUDGEMENT
%%LABEL
\label{bkm:Ref80456783}
%%CONTEXT
%%LINE1
Aha hantu hariho abantu badutwaye amakaramu.\\
%%LINE2
\gll
a-ha-a  ha-ntu  ha-ri=hó a-ba-ntu  ba-tu-twáar-ye  a-ma-karamu\\
%%LINE3
\AUG-16-\DEM{}\textsubscript{1}  16-place  16\SM{}-be=16  \AUG{}-2-person  2\SM-1\PL.\OM{}-take-\PFV{}  \AUG{}-6-pen\\
%%TRANS1
\glt
‘In this place are people who took our pens.’\\
%%TRANS2
%%EXEND

\z

The enclitic can also be used for disambiguation: without the locative enclitic, \xref{bkm:Ref80278403} is ambiguous between a goal reading of the locative phrase (‘going to Bujumbura’) and a source reading (‘leaving from Bujumbura’). With the enclitic, as in \xref{bkm:Ref80278372}, the locative can only be interpreted as the goal.

%%EAX
\ea
%%JUDGEMENT
%%LABEL
\label{bkm:Ref80278403}
%%CONTEXT
(Among the given people, who went to/from Bujumbura?)\\
%%LINE1
I Bujumbura hagiye Minani.\\
%%LINE2
\gll
i  Bujumbura  ha-a-$\varnothing$-gi-ye  Minani\\
%%LINE3
19  Bujumbura  16\SM/\EXP-\N.\PST-\CJ{}-go-\PFV{}  1.Minani\\
%%TRANS1
\glt
‘Minani went to Bujumbura/Minani went from Bujumbura (to Europe).’\\
%%TRANS2
%%EXEND

%%EAX
\ex
%%JUDGEMENT
%%LABEL
\label{bkm:Ref80278372}
%%CONTEXT
%%LINE1
I Bujumbura hagiyeyo Minani.\\
%%LINE2
\gll
i  Bujumbura  ha-a-$\varnothing$-gi-ye=yó  Minani\\
%%LINE3
19  Bujumbura  16\SM-\N.\PST-\CJ{}-go-\PFV{}=19  1.Minani\\
%%TRANS1
\glt
‘Minani went to Bujumbura.’\\
%%TRANS2
%%EXEND

\z

Kirundi also features Semantic Locative Inversion (SLI). SLI received its name from the fact that the locative interpretation of the preverbal constituent is only present semantically in denoting a spatial or temporal localisation \citep{Buell2007}. This differs from the formally locative phrase that is crucially marked by the locative prefix (see above). In SLI, the verb takes its subject marker in the noun class of the preverbal (semantically but not formally) locative phrase in the subject function, as seen in \xref{bkm:Ref149303804} for the class 8 location `these eyebrows'; in Formal Locative Inversion the subject marker is in locative class 16.\largerpage[2]

%%EAX
\ea
%%JUDGEMENT
%%LABEL
\label{bkm:Ref149303804}
%%CONTEXT
%%LINE1
Ivyo bigóhe birikó ubwoóya bwiínshi.\\
%%LINE2
\gll
i-bi-o  bi-góhe  bi-ri=kó  u-bu-oóya  bu-iínshi.\\
%%LINE3
\AUG-8-\DEM{}\textsubscript{2}  8-eyebrow  \SM8-{}be=17  \AUG-14-hair  14-much\\
%%TRANS1
\glt
‘In these eyebrows there are a lot of hairs.’ (\textit{Anon.1990}, Education, 1990s – BantUGhent corpus)\\
%%TRANS2
%%EXEND

\z

In Instrument Inversion, intransitive as well as transitive predicates may occur, but the instrument must crucially be an argument of the verb. This is done by deriving the verb as a causative, marked by the suffix -\textit{ish}/-\textit{esh}, as illustrated in \xref{ex:implements} and \xref{ex:drinkstraw:c}, the alternative being the instrument in the function of an adjunct, as in \xref{ex:drinkstraw:a} and \xref{ex:drinkstraw:d} with the preposition \textit{na} ‘with’. In \xref{ex:implements}, the instrument \textit{ivyo bikoreesho} ‘these implements’ precedes the verb and determines subject marking in class 8, and the logical subject ‘Christians and non-Christians’ follows the verb. As \citet{Zeller2013} also notes for Zulu, the predicate must prototypically take an instrument for inversion to be possible (e.g. write with a pen, but not climb with a ladder), for example in \xref{ex:drinkstraw} drinking with a straw. The straw (\textit{umukeenke}) is an adjunct in \xref{ex:drinkstraw:a}, it is a postverbal argument in \xref{ex:drinkstraw:b} where the verb is not -\textit{nyw}- but the causative -\textit{nyweesh}-, and it appears preverbally in the subject inversion construction in \xref{ex:drinkstraw:c}, determining subject marking on the verb in class 3. In all instances of Instrument Inversion, the postverbal logical subject is in focus, and again the construction cannot be used in a thetic context.

\ea
\label{ex:implements}\citep[120--122]{Nshemezimana2016}

%%EAX
\ea
\begin{xlist}[SVOOOOO]
\exi{O\textsubscript{[INST]}VS:}
%%JUDGEMENT
[]{
%%LABEL
\label{ex:implements:a}
%%CONTEXT
%%LINE1
Ivyo bikóreesho bikoreesha uwa Krístu n’úwutari wé.\\
%%LINE2
\gll
i-bi-o  bi-kóreesho  bi-$\varnothing$-kór-ish-a u-u-a  Kristu  n’  u-u-ta-ri  wé\\
%%LINE3
\AUG-8-\DEM{}\textsubscript{2}  8-implement  8\SM-\PRS.\CJ-{}do-\CAUS-\FV{}   \AUG-1-\CONN{}  Christ  and  \AUG-1-\NEG-{}be  1.\PRO{}\\
%%TRANS1
\glt
‘These implements are used by Christians as well as non-Christians.’ (CU101004Ukwege, Paix, 2010s)
}
\end{xlist}
%%TRANS2
%%EXEND

%%EAX
\ex
\begin{xlist}[SVOOOOO]
\exi{SVO\textsubscript{[INST]}:}
%%JUDGEMENT
[]{
%%LABEL
\label{ex:implements:b}
%%CONTEXT
%%LINE1
%%LINE2
\gll
U-u-a  Kristu  n’  u-u-ta-ri  wé   ba-$\varnothing$-kór-ish-a  i-bi-o  bi-kóreesho. \\
%%LINE3
\AUG-1-\CONN{}  Christ  and  \AUG-1-\NEG-{}be  1.\PRO{}   2\SM-\PRS.\CJ-{}do-\CAUS{}-\FV{}  \AUG{}-8-\DEM{}\textsubscript{2}  8-implement \\
%%TRANS1
\glt
  ‘Both Christians and non-Christians use these implements.’
%%TRANS2
}
\end{xlist}
%%EXEND
\z

\ex
\label{ex:drinkstraw}
%%EAX
\ea
\begin{xlist}[SVOOOOO]
\exi{SVO\textsubscript{[INST]}:}
%%JUDGEMENT
[]{
%%LABEL
\label{ex:drinkstraw:a}
%%CONTEXT
%%LINE1
Nyaa mugabo akuunda kunywá inzogá n’úmukeenke.\\
%%LINE2
\gll
nyaa  mu-gabo  a-$\varnothing$-kúund-a  ku-nyóa  i-\N-{}zogá   n’  u-mu-keenke \\
%%LINE3
1-\DEM{}\textsubscript{7}  1-man  1\SM-\PRS.\CJ-{}like-\FV{}  {}15-{}drink  \AUG-9-{}drink    with  \AUG-3-{}straw \\
%%TRANS1
\glt
  ‘The man in question often drinks the drink with the straw.’
%%TRANS2
}
\end{xlist}
%%EXEND

%%EAX
\ex
\begin{xlist}[SVOOOOO]
\exi{SVO\textsubscript{[INST]}:}
%%JUDGEMENT
[]{
%%LABEL
\label{ex:drinkstraw:b}
%%CONTEXT
%%LINE1
Nyaa mugabo akuunda kunywéesha inzogá umukenke.\\
%%LINE2
\gll
nyaa  mu-gabo  a-$\varnothing$-kúund-a  ku-nyó-ish-a  i-\N-{}zogá   u-mu-keenke \\
%%LINE3
1.\DEM{}\textsubscript{7}  1-man  1\SM-\PRS.\CJ-{}like-\FV{}  {}15-{}drink-\CAUS{}-\FV{}  \AUG-9-{}drink   \AUG-3-{}straw \\
%%TRANS1
\glt
‘The man in question often drinks the drink with the straw.’\\
%%TRANS2
}
\end{xlist}
%%EXEND

%%EAX
\ex
\begin{xlist}[SVOOOOO]
\exi{O\textsubscript{[INST]}VS:}
%%JUDGEMENT
[]{
%%LABEL
\label{ex:drinkstraw:c}
%%CONTEXT
%%LINE1
Umukeenke ukuunda kunywéesha (inzogá) nyáa mugabo.\\
%%LINE2
\gll
u-mu-keenke  u-$\varnothing$-kúund-a  ku-nyó-[ish]-a (i-\N-{}zogá)   nyaa  mu-gabo \\
%%LINE3
\AUG-3-{}straw  3\SM-\PRS.\CJ-{}like-\FV{}  15-{}drink-\CAUS-\FV{}   {\db}\AUG{}\textsubscript{9}-9-drink  1.\DEM{}\textsubscript{7}  1-man\\
%%TRANS1
\glt
  ‘The man in question (not John) often drinks the drink with the straw.’\\
%%TRANS2
}
\end{xlist}
%%EXEND


%%EAX
\ex
\begin{xlist}[SVOOOOO]
\exi{*O\textsubscript{[INST]}VS:}
%%JUDGEMENT
[*]{
%%LABEL
\label{ex:drinkstraw:d}
%%CONTEXT
%%LINE1
%%LINE2
\gll
Na  u-mu-keenke  u-$\varnothing$-kúund-a  ku-nyóa   (i-\N-{}zogá)  nyaa  mu-gabo\\
%%LINE3
  with  \AUG-3-{}straw  3\SM-\PRS.\CJ-{}like-\FV{}  {}15-{}drink  {\db}\AUG-9-{}drink  1.\DEM{}\textsubscript{7}  1-man\\
%%TRANS1
\glt
%%TRANS2
}
\end{xlist}
%%EXEND

\z
\z

Complement Inversion involves a CP triggering subject marking in (default) class 8 on the verb, and the postverbal subject is in focus. Unlike in Kinyarwanda \citep[193]{Kimenyi1980}, the CP does not appear in preverbal position in Kirundi, but is typically right-dislocated (presumably because it is phonologically heavy), as shown in \xref{bkm:Ref75348528:a}.

\ea
\label{bkm:Ref75348528}\citep[124]{Nshemezimana2016}
%%EAX
\ea
\begin{xlist}[SVOOOOO]
\exi{VSO\textsubscript{[PROP]}:}
%%JUDGEMENT
%%LABEL
\label{bkm:Ref75348528:a}
%%CONTEXT
%%LINE1
Vyaávuze abakurambere bíwaácu, kó yooba ari amatá avyaayé amasoro.\\
%%LINE2
\gll
bi-á-$\varnothing$-vúg-ye  a-ba-kurambere  ba-íwaácu  kó  a-oo-bá-a   a-ri  a-ma-tá  a-vyáar-ye\textsuperscript{H}  a-ma-soro \\
%%LINE3
8\SM-\RMT.\PST-\CJ{}-say-\PFV{}  \AUG{}-2-{}older.people  2-\POSS.1\PL{}  \COMP{}  6\SM-\POT-{}be-\FV{}   \EXP-{}be  \AUG-6-{}milk  6\SM-{}generate-\PFV{}.\REL{}  \AUG{}-6-{}butter \\
%%TRANS1
\glt
  ‘Our elders (not someone else) said it, that it would be the milk that generates the butter.’ (Kabizi141114revu, Magazines, 2010s)\\
%%TRANS2
\end{xlist}
%%EXEND

%%EAX
\ex
\begin{xlist}[SVOOOOO]
\exi{SVO\textsubscript{[PROP]}:}
%%JUDGEMENT
%%LABEL
\label{bkm:Ref75348528:b}
%%CONTEXT
%%LINE1
Abakurambere bíwaácu báavuze kó yoobá ari amatá avyaayé amasoro.\\
%%LINE2
\gll
a-ba-kurambere  ba-íwaácu  ba-á-$\varnothing$-vúg-ye  kó  a-oo-bá-a   a-ri  a-ma-tá  a-vyáar-ye\textsuperscript{H}  a-ma-soro \\
%%LINE3
\AUG-{}2-elder  2-\POSS.1\PL{}  2\SM-\RMT.\PST-\CJ{}-say-\PFV{}  \COMP{}  6\SM-\POT{}-be-\FV{}   \EXP-{}be  \AUG-6{}-milk  6\SM-{}generate-\PFV.\REL{}  \AUG-6{}-butter \\
%%TRANS1
\glt
  ‘Our elders said that it would be the milk that generates the butter.’ (Adapted from \xref{bkm:Ref75348528:a})\\
%%TRANS2
\end{xlist}
%%EXEND

\z
\z

In Default Agreement Inversion (DAI), the subject marker is the expletive \textit{ha}-, which here does not refer to a particular place (compare to Locative Inversion above). When the verb takes a conjoint form (see \sectref{bkm:Ref81485420}), the clause-final logical subject is interpreted as focus, seen in the question word in \xref{bkm:Ref73976476}, and the subject modified by ‘only’ in \xref{bkm:Ref75346884}, as well as the incompatibility with ‘even’ \xref{bkm:Ref105424268}; the latter test suggests an exclusive focus interpretation.

%%EAX
\ea
%%JUDGEMENT
%%LABEL
\label{bkm:Ref73976476}
%%CONTEXT
(Context: The teacher wants to know how many students the headmaster needs to go unload school materials in the car. He asks her:)\\
%%LINE1
Haaze abanyéeshuúre bangáahé?\\
%%LINE2
\gll
ha-əz-e  a-ba-nyéeshuúre  ba-ngáahé?\\
%%LINE3
\EXP{}-come-\SBJV{}  \AUG{}-2-student  2-how.many\\
%%TRANS1
\glt
‘How many students should come?’\\
%%TRANS2
%%EXEND

\z

%%EAX
\ea
%%LABEL
%\label{ex:kirundi:3}
\label{bkm:Ref75346884}
%%CONTEXT
(Have all people died in the accident?)\\
%%LINE1
%%LINE2
\gll
Ha-a-$\varnothing$-pfú-ye  Kabura  gusa.\\
%%LINE3
\EXP-\N.\PST-\CJ-{}die-\PFV{}  kabura  only\\
%%TRANS1
\glt
‘Only Kabura died.’\\
%%TRANS2

%%EXEND

\z

%%EAX
\ea
%%JUDGEMENT
%%LABEL
\label{bkm:Ref105424268}
%%CONTEXT
(Are the women the ones who trap moles?)\\
%%LINE1
Ahanini, hatega (*ndetse na) abagabo.\\
%%LINE2
\gll
ahanini  ha-teg-a  ndetse  na  a-ba-gabo\\
%%LINE3
usually  \EXP{}-trap-\FV{}  even  and  \AUG{}-2-man\\
%%TRANS1
\glt
‘Usually men trap (them).’\\
%%TRANS2
%%EXEND

\z

When the verb takes its disjoint form, the entire sentence is within the scope of the assertion – unlike the conjoint form, it does not imply any presupposition. These are known as ``thetic sentences", and these can be seen as sentences without a topic expression, i.e. in which the subject is detopicalised  \citep{LambrechtPolinsky1997,Lambrecht2000,Sasse1987}. As expected, it is ungrammatical to modify the postverbal constituent with \textit{gusa} ‘only’, as in \xref{bkm:Ref79131990}. This DAI construction can be compared to the presentational construction with \textit{hari/hariho} (see \sectref{bkm:Ref75348623}), which differs in both form and function.

\ea
(What happened?)\\
%%EAX
\ea
%%JUDGEMENT
%%LABEL
%\label{ex:kirundi:131}
%%CONTEXT
%%LINE1
Harakómeretse umuntu.\\
%%LINE2
\gll
Ha-ra-kómerek-ye  u-mu-ntu\\
%%LINE3
\EXP{}-\DJ{}-be.hurt-\PFV{}  \AUG{}-1-person\\
%%TRANS1
\glt
‘Someone has just been hurt.’\\
%%TRANS2
%%EXEND

%%EAX
\ex
%%JUDGEMENT
%%LABEL
\label{bkm:Ref79131990}
%%CONTEXT
%%LINE1
Harakómeretse abaana (*gusa).\\
%%LINE2
\gll
Ha-ra-kómerek-ye  a-ba-áana  gusa\\
%%LINE3
\EXP-\DJ-{}be.hurt-\PFV{}  \AUG-2{}-child  only\\
%%TRANS1
\glt
‘(\textsuperscript{$*$}Only) the children have been hurt.’\\
%%TRANS2
%%EXEND

\z
\z

%%EAX
\ea
%%JUDGEMENT
%%LABEL
%%CONTEXT
\citep[152]{Nshemezimana2016}\\
%%LINE1
Haraaza kaándi abatóozakoóri ngo babatízwe.\\
%%LINE2
\gll
ha-$\varnothing$-ra-ǝz-a  kaándi  a-ba-tóozakoóri   ngo  ba-batíz-u-e \\
%%LINE3
\EXP-\PRS-\DJ-{}come-\FV{} also  \AUG-2-{}tax.collector   for  2\SM-{}be.baptized-\PASS-\SBJV{}   \\
%%TRANS1
\glt
  ‘There also came tax collectors to be baptized.’\\
%%TRANS2
(\textit{UbwuzureBushasha}, Religion, 1960s)
%%EXEND

\z


The same use of the expletive \textit{ha}- is seen in weather expressions (also thetic), and as mentioned, contrasts with the locative interpretation in Locative Inversion.

%%EAX
\ea
%%JUDGEMENT
%%LABEL
%%CONTEXT
%%LINE1
Harakanye.\\
%%LINE2
\gll
ha-ra-kany-ye\\
%%LINE3
\EXP{}-\DJ{}-become.cold-\PFV{}\\
%%TRANS1
\glt
‘It’s cold.’\\
%%TRANS2
%%EXEND

\z

Transitive predicates may also participate in DAI, whereby both the agent and the patient argument appear postverbally. The only possible interpretation is subject focus (in line with \citet{CarstensMletshe2015}'s observation for Xhosa), and the subject must, as expected, occur in clause-final position, resulting in VOS order \xref{bkm:Ref75339373}. Here, only the conjoint form is accepted, and a thetic interpretation is impossible.

\ea
\label{bkm:Ref75339373}
(Who has poured the milk?)
%%EAX
\ea
%%JUDGEMENT
[]{
%%LABEL
%\label{ex:kirundi:3}
%%CONTEXT
%%LINE1
Hasheeshe amatá Yohana.\jambox[1in]{(VOS)}
%%LINE2
\gll
Ha-a-sees-ye  a-ma-tá  Yohana\\
%%LINE3
\EXP-\N.\PST{}-pour-\PFV{}  \AUG{}-6-milk  John\\
%%TRANS1
\glt
  ‘\textit{John} (not Peter) had poured the milk.’\\
%%TRANS2
}
%%EXEND

%%EAX
\ex
%%JUDGEMENT
[\textsuperscript{\#}]{
%%LABEL
%%CONTEXT
%%LINE1
Hasheeshe  Yohana  amatá.\\
%%LINE2
{\gll
ha-a-$\varnothing$-sees-ye  Yohana  a-ma-tá.\\
%%LINE3
\EXP-\N.\PST-\CJ{}-pour-\PFV{} John  \AUG{}-6-milk\\}\jambox[1in]{(\#VSO)}
%%TRANS1
\glt
  ‘\textit{John} (not Peter) had poured the milk.’\\
%%TRANS2
}
%%EXEND

\z
\z

In Agreeing Inversion, the subject marker agrees with the postverbal logical subject. Here, the logical subject has to be indefinite and only the disjoint verb form may be used. In this construction, the postverbal logical subject is part of the new information (though not focal), and the construction as such is used in a thetic environment, as in \xref{bkm:Ref73950839} and \xref{bkm:Ref79132665}. The DAI and presentational alternatives (see further in \sectref{bkm:Ref75348623}) are also felicitous here as indicated in \xref{bkm:Ref116983035} and \xref{bkm:Ref116983036}, respectively; DAI is seen as very natural, and AI is associated with colloquial use.

\ea
\label{bkm:Ref73950839}(Why are the people shouting?)
%\label{ex:kirundi:3}
%%EAX
\ea
%%JUDGEMENT
%%LABEL
%%CONTEXT
%%LINE1
%%LINE2
\gll
A-$\varnothing$-ra-pfú-ye  u-mu-ntu.\\
%%LINE3
  1\SM-\PRS-\DJ-{}die-\PFV{}  \AUG{}-1-person\\
%%TRANS1
\glt
%%TRANS2
%%EXEND

%%EAX
\ex
%%JUDGEMENT
%%LABEL
\label{bkm:Ref116983035}
%%CONTEXT
%%LINE1
%%LINE2
\gll
Ha-$\varnothing$-ra-pfú-ye  u-mu-ntu.\\
%%LINE3
  \EXP-\PRS-\DJ{}-die-\PFV{}  \AUG{}-1-person\\
%%TRANS1
\glt
%%TRANS2
%%EXEND



%%EAX
\ex
%%JUDGEMENT
%%LABEL
\label{bkm:Ref116983036}
%%CONTEXT
%%LINE2
\gll
Ha-$\varnothing$-ri  u-mu-ntu  a-pfú-ye\textsuperscript{H}.\\
%%LINE3
\EXP-\PRS{}-be  aug-1-person  1\SM-{}die-\PFV.\REL{}\\
%%TRANS1
\glt
  ‘Someone has just died/is dying.’\\
%%TRANS2
%%EXEND

\z
\z


%%EAX
\ea
%%JUDGEMENT
%%LABEL
\label{bkm:Ref79132665}
%%CONTEXT
(Context: You have visited the neighbours and when you come back your husband asks ‘Anything new?’)\\
%%LINE1
Yaapfuuye impené kwa Kabura.\\
%%LINE2
\gll
i-a-a-pfú-ye  i-n-hene  kwa  Kabura\\
%%LINE3
9\SM{}-\N.\PST{}-\DJ{}-die-\PFV{}  \AUG{}-9-goat  17  Kabura\\
%%TRANS1
\glt
‘A goat has died at Kabura’s house.’\\
%%TRANS2
%%EXEND

\z

VS word order is not only encountered in subject inversion, but also when the subject is right-dislocated. However, we can see a clear difference between AI on the one hand, and a right-dislocated subject on the other hand. In AI, the subject is part of the new information; it is in a low syntactic position, scoping under negation, as in \xref{bkm:Ref75349904}; and there is no prosodic break between V and S.

%%EAX
\ea
%%JUDGEMENT
%%LABEL
\label{bkm:Ref75349904}
%%CONTEXT
(Are the schools open?)\jambox*{[Agreeing Inversion]}
%%LINE1
Ntarúugurura (amashuúre) yóóse.\\
%%LINE2
\gll
nti-a-raa-ugurur-a  a-ma-shuure  a-óóse\\
%%LINE3
\NEG{}-6\SM{}-\INCP{}-open-\FV{}  \AUG{}-6-school  6-all\\
%%TRANS1
\glt
‘Not all schools are open yet.’ (but some are)\\
%%TRANS2
%%EXEND

\z

When the subject is right-dislocated, however, the subject is given information (added as an afterthought); the quantified subject scopes over negation, as in \xref{bkm:Ref75350086}; and there is typically a prosodic break, indicated by the comma in \xref{bkm:Ref75350104}.

%%EAX
\ea
%%JUDGEMENT
%%LABEL
\label{bkm:Ref75350086}
%%CONTEXT
(Are all schools open?’)\jambox*{[right-dislocation]}
%%LINE1
Ntaruugurura, amashuure yoose.\\
%%LINE2
\gll
nti-a-raa-uguru-a  a-ma-shuure  a-ose\\
%%LINE3
\NEG{}-6\SM{}-\INCP{}-open-\FV{}  \AUG{}-6-school  6-all\\
%%TRANS1
\glt
‘They are not open yet, all schools.’ (none are open)\\
%%TRANS2
%%EXEND

\z

%%EAX
\ea
%%JUDGEMENT
%%LABEL
\label{bkm:Ref75350104}
%%CONTEXT
\citep[57]{Nshemezimana2016}\\
%%LINE1
Yaráhejeje arageenda, nya mwáana.\\
%%LINE2
\gll
a-á-ra-hér-i-ye  a-ra-geend-a  nyaa  mu-áana\\
%%LINE3
1\SM-\RMT.\PST-\DJ-{}finish-\CAUS{}-\PFV{}  {}1\SM{}-\DJ{}-go-\FV{}  {}1.\DEM{}\textsubscript{7}  1-child\\
%%TRANS1
\glt
‘Eventually he left, the aforementioned child.’\\
%%TRANS2
  (Mushingantahe, Paix, 2000s)

%%EXEND

\z

In summary, the logical subject appears postverbally in subject inversion constructions, where it is interpreted as focused when the verb is in its conjoint form, and as non-topical in thetics with a disjoint form. When compared to other Bantu languages, Kirundi is exceptional in allowing a wide range of inversion constructions, as also discussed in \citet{KerrEtAl2023}.

\section{Predicate doubling}
\label{bkm:Ref75783502}
It is possible in Kirundi to use the infinitive together with an inflected form of the same predicate, a construction known as ``predicate doubling" (see overview in \citealt{GüldemannFiedler2022}). Of the three types of predicate doubling, Kirundi allows topic doubling, and shows a special nominalised in-situ doubling. We discuss these in turn.

In Topic doubling, the infinitive occurs in the left periphery as a topic. For intransitive verbs, there are four possible interpretations. The interpretation can be a) verum as in \xref{bkm:Ref72248348}, b) implied contrast with another action/verb as in \xref{bkm:Ref72248357}, c) depreciative \citep[cf.~][]{Meeussen1967}, d) intensive/excessive; both c and d are illustrated in \xref{bkm:Ref72248384}.

%%EAX
\ea
%%JUDGEMENT
%%LABEL
\label{bkm:Ref72248348}
%%CONTEXT
%%LINE1
Kwiruka ndiiruka.\\
%%LINE2
\gll
ku-iruka  \N{}-ra-iruk-a\\
%%LINE3
15-run  1\SG.\SM-\PRS.\DJ-{}run-\FV{}\\
%%TRANS1
\glt
verum: ‘I do run!’ (you might think I don’t)\\
%%TRANS2
%%EXEND

\z

%%EAX
\ea
%%JUDGEMENT
%%LABEL
\label{bkm:Ref72248357}
%%CONTEXT
%%LINE1
Inka, kuzigaburira naazigaburiye.\\
%%LINE2
\gll
i-n-ka  ku-zi-gabur-ir-a  \N{}-a-a-zi-gabur-ir-ye\\
%%LINE3
\AUG-10{}-cow  15-10\OM-{}feed-\APPL-\FV{}  1\SG.\SM-\N.\PST-\DJ-10\OM{}-feed-\APPL-\PFV{}\\
%%TRANS1
\glt
contrastive: ‘I did feed the cows.’ (implying that the other task of cleaning the house is not done)\\
%%TRANS2
%%EXEND

\z

%%EAX
\ea
%%JUDGEMENT
%%LABEL
\label{bkm:Ref72248384}
%%CONTEXT
%%LINE1
Kwandika uraanditse.\\
%%LINE2
\gll
ku-andika  u-ra-andik-ye\\
%%LINE3
15-write  2\SG.\SM-\DJ{}-write-\PFV{}\\
%%TRANS1
\glt
Intensive: ‘You have really written a lot!’ (you expected one page and s/he has written five)\\
%%TRANS2
  Depreciative: ‘At least you have written (something).’ (the answers on a student’s exam are not good enough)

%%EXEND

\z

For topic doubling with a transitive verb, the object can either follow the infinitive, or the inflected verb. If the object follows the infinitive, as in \xref{bkm:Ref80458229}, the focus is on the VP and the interpretation is verum or depreciative; if the object follows the inflected verb, as in \xref{bkm:Ref78492154}, the focus is on the object, as indicated in the context of an object question.

%%EAX
\ea
%%JUDGEMENT
%%LABEL
\label{bkm:Ref80458229}
%%CONTEXT
%%LINE1
Kurya imboga (kó), ndazirya, ariko…\\
%%LINE2
\gll
ku-ria  i-n-boga  ku-ó  ni-ra-zi-rí-a  ariko…\\
%%LINE3
15-eat  \AUG{}-10-vegetable  15-\CM{}  1\SG.\SM-\PRS.\DJ-10\OM{}-eat-\FV{}  but\\
%%TRANS1
\glt
‘Even though I eat vegetables, ...’\\
%%TRANS2
‘I do eat vegetables, but... (it doesn’t help)’

%%EXEND

\z

%%EAX
\ea
%%JUDGEMENT
%%LABEL
\label{bkm:Ref78492154}
%%CONTEXT
(Context: In a restaurant, when asked what you want to order.)\\
%%LINE1
(U)kuryá, ndya inyama, kunywá nywa ifanta.\\
%%LINE2
\gll
u-ku-ryá  \N{}-rí-a  i-nyama  ku-nywá  \N{}-nyó-a  i-fanta\\
%%LINE3
\AUG-15{}-eat  1\SG.\SM{}-eat-\FV{}  \AUG-10.meat  15-drink  1\SG.\SM{}-drink-\FV{}  \AUG{}-9.fanta\\
%%TRANS1
\glt
‘For eating, I eat meat, for drinking, I drink fanta.’\\
%%TRANS2
%%EXEND

\z

Since topicalised constituents can be marked by the contrastive topic marker -\textit{ó} (see \sectref{bkm:Ref72249299}), the initial infinitival phrase can be marked as such, too, by \textit{kó}. The particle seems to foreground the depreciative meaning, as in \xref{bkm:Ref81494620}, or adds an exclamative flavour (see \xref{bkm:Ref81509745} below).

\ea
\label{bkm:Ref81494620}
\begin{xlist}
\exi{A:} I need to lose weight. \\
\exi{B:} Maybe if you do sport?\\
%%EAX
\exi{A:}
%%JUDGEMENT
%%LABEL
%%CONTEXT
%%LINE1
Kugira ikarashishi kó ndayigira.\\
%%LINE2
\gll
ku-gira  i-karashishi  ku-ó  \N{}-ra-i-gir-a\\
%%LINE3
15-do  \AUG{}-9.sport  15-\CM{}  1\SG.\SM-\PRS.\DJ-9\OM{}-do-\FV{}\\
%%TRANS1
\glt
‘Even if I do sports… (I don’t lose weight).’\\
%%TRANS2
%%EXEND

\end{xlist}
\z

A second type of predicate doubling, cleft doubling, is not acceptable in Kirundi with the same predicate appearing twice. If the infinitive is clefted, the predicate cannot be doubled (as is possible in other languages, such as Fwe \citep{Gunnink2018} and Kîîtharaka \citep{chapters/kiitharaka}), but instead the light verb -\textit{kora} ‘do, make’ must be used, as in \xref{bkm:Ref116983834:b}.

\ea
\label{bkm:Ref116983834}
%%EAX
\ea
%%JUDGEMENT
[*]{
%%LABEL
\label{bkm:Ref116983834:a}
%%CONTEXT
%%LINE1
Ni kurírimba ndiríimba.\\
%%LINE2
\gll
ni  ku-rírimba  \N{}-rírimb-a\textsuperscript{H}\\
%%LINE3
\COP{}  15-sing  1\SG.\SM{}-sing-\FV.\REL{}\\
%%TRANS1
\glt
‘It’s singing that I sing.’\\
%%TRANS2
}
%%EXEND

%%EAX
\ex
%%JUDGEMENT
[]{
%%LABEL
\label{bkm:Ref116983834:b}
%%CONTEXT
%%LINE1
Ni kuririmba nkora.\\
%%LINE2
\gll
ni  ku-rírimba  \N{}-kór-a\textsuperscript{H}\\
%%LINE3
\COP{}  15-sing  1\SG{}.\SM{}-do-\FV.\REL{}\\
%%TRANS1
\glt
‘It’s singing that I do.’\\
%%TRANS2
}
%%EXEND

\z
\z

Equally ungrammatical is the third type of predicate doubling: in-situ doubling with a bare infinitive is not accepted (which is possible in other languages, e.g. Kîîtharaka and Kinyakyusa – see \textcite{chapters/kiitharaka} and \textcite{chapters/kinyakyusa}):

%%EAX
\ea
%%JUDGEMENT
[*]{
%%LABEL
%%CONTEXT
%%LINE1
Niirutse kwiiruka.\\
%%LINE2
\gll
ni-$\varnothing$-íiruk-ye  ku-íiruka\\
%%LINE3
1\SG.\SM-\PRS.\CJ-{}run-\PFV{}  15-run\\
%%TRANS1
\glt
int. ‘I really ran.’\\
%%TRANS2
}
%%EXEND

\z

However, two other nomino-verbal forms can be used in situ with the same inflected predicate: either a derivation in class 14 \textit{bu-} with a final vowel -\textit{e} (equal to the subjunctive), as in \xref{bkm:Ref78492908} and \xref{bkm:Ref78492919}, or with \textit{nya} preceding the infinitive, as in \xref{bkm:Ref78492942}.

%%EAX
\ea
%%JUDGEMENT
%%LABEL
\label{bkm:Ref78492908}
%%CONTEXT
%%LINE1
Heba ivyaha buhebé.\\
%%LINE2
\gll
$\varnothing$-héb-a  i-bi-áaha  bu-héb-e\\
%%LINE3
2\SG.\SM{}-leave-\IMP{}  \AUG{}-8-sin  14-leave-\SBJV{}\\
%%TRANS1
\glt
‘Stop sinning for real.’\\
%%TRANS2
%%EXEND

\z

%%EAX
\ea
%%JUDGEMENT
%%LABEL
\label{bkm:Ref78492919}
%%CONTEXT
%%LINE1
Ushaaka ngende bugende?\\
%%LINE2
\gll
u-shaak-a  \N{}-gend-e  bu-gend-e\\
%%LINE3
2\SG.\SM{}-want-\FV{}  1\SG.\SM-{}go-\SBJV{}  14-go-\SBJV{}\\
%%TRANS1
\glt
‘You really want me to go?’\\
%%TRANS2
%%EXEND

\z

%%EAX
\ea
%%JUDGEMENT
%%LABEL
\label{bkm:Ref78492942}
%%CONTEXT
%%LINE1
Uzoozé tubaané nya kubáana.\\
%%LINE2
\gll
u-zoo-əz-e  tu-báan-e  nya  ku-báana\\
%%LINE3
2\SG.\SM{}-\FUT{}-come-\SBJV{}  1\PL.\SM-{}live.together-\SBJV{}  very  15-live.together\\
%%TRANS1
\glt
‘Come, we should live together properly.’\\
%%TRANS2
  ‘We should be living-together living-together.’

%%EXEND

\z

Both constructions with \textit{bu}- and \textit{nya} bring about a reading of ‘really’, ‘properly’, referring to a prototype (as in the English reduplication ‘I want a DRINK-drink’, see \citealt{GhomeshiEtAl2004}), but also touching on verum and unexpectedness. The precise interpretations and use of this and the other predicate doubling constructions remain for future investigations.

\section{Particle -ó for contrastive given information}
\label{bkm:Ref72249299}
Kirundi has a particle -\textit{ó} that originates as a pronoun (the ``-o of reference" in \citealt{Ashton1945}) but also functions as a marker of contrastive given information. This is also found in Kîîtharaka and Rukiga with similar functions (see \citealt{AsiimwevanderWal2021}, \textcite{chapters/rukiga} and \textcite{chapters/kiitharaka}). In its function as a pronoun, we encounter -\textit{ó} as an independent pronoun – in \xref{bkm:Ref75439017} the pronominal object, after the preposition \textit{na} ‘with’ in \xref{bkm:Ref74930335}, and after the comparative \textit{nka} ‘like’ in \xref{bkm:Ref74930344} – and also as an enclitic when referring to locations \xref{bkm:Ref75359891}.

%%EAX
\ea
%%JUDGEMENT
%%LABEL
\label{bkm:Ref74930335}
%%CONTEXT
%%LINE1
Natambanye \textbf{nabó}.\\
%%LINE2
\gll
\N{}-a-táamb-an-ye  na=ba-ó\\
%%LINE3
1\SG.\SM-\N.\PST-{}dance-\ASS-\PFV{}  with=2-\PRO{}\\
%%TRANS1
\glt
‘I danced with them.’\\
%%TRANS2
%%EXEND

\z
\pagebreak

%%EAX
\ea
%%JUDGEMENT
%%LABEL
\label{bkm:Ref74930344}
%%CONTEXT
%%LINE1
Atamba nkaá\textbf{twe}.\\
%%LINE2
\gll
a-$\varnothing$-taamb-a  nka=twe\\
%%LINE3
1\SM-\PRS.\CJ{}-dance-\FV{}  like=1\PL.\PRO{}\\
%%TRANS1
\glt
‘He dances like us.’\\
%%TRANS2
%%EXEND

\z

%%EAX
\ea
%%JUDGEMENT
%%LABEL
\label{bkm:Ref75439017}
%%CONTEXT
(Context: You are talking about someone and say that they are always late. I haven’t heard well and ask ‘Are you talking about me?!’ You say:)\\
%%LINE1
Mvuze wé.\\
%%LINE2
\gll
\N{}-$\varnothing$-vúg-ye  wé\\
%%LINE3
1\SG.\SM-\PRS-\CJ-{}say-\PFV{}  1.\PRO{}\\
%%TRANS1
\glt
‘I’m talking about him/her.’\\
%%TRANS2
%%EXEND

\z

%%EAX
\ea
%%JUDGEMENT
%%LABEL
\label{bkm:Ref75359891}
%%CONTEXT
\citep[134]{Nshemezimana2016}\\
%%LINE1
Kurí kimwé, haanditse\textbf{kó} aya majaambo.\\
%%LINE2
\gll
kurí  ki-mwé  ha-$\varnothing$-aandik-ye=kó a-a-a   ma-jaambo\\
%%LINE3
17  7-one  16\SM-\CJ-{}write-\PFV{}=17{} \AUG{}-6-\DEM{}\textsubscript{1}  6-word\\
%%TRANS1
\glt
‘On one, there are written these words.’\\
%%TRANS2
  (Ifaranga, Education, 1980s)

%%EXEND

\z

However, -\textit{ó} can also be used in a different way, to mark given information as contrastive. This is shown in the contrast between \xref{bkm:Ref74930350:a} and \xref{bkm:Ref74930350:b}, and discussed in more detail below.

\ea
\label{bkm:Ref74930350}
%%EAX
\ea
%%JUDGEMENT
%%LABEL
\label{bkm:Ref74930350:a}
%%CONTEXT
%%LINE1
Igifaraánsa ndakívuga.\\
%%LINE2
\gll
i-ki-faraánsa  \N{}-ra-ki-vúga\\
%%LINE3
\AUG-7-{}french  1\SG.\SM-\PRS.\DJ-7\OM{}-speak-\FV{}\\
%%TRANS1
\glt
‘French, I speak it.’\\
%%TRANS2
%%EXEND

%\label{ex:kirundi:23}
%%EAX
\ex
%%JUDGEMENT
%%LABEL
\label{bkm:Ref74930350:b}
%%CONTEXT
%%LINE1
Igifaraánsa có ndakívuga.\\
%%LINE2
\gll
i-ki-faraánsa  ki-ó  \N{}-ra-ki-vúg-a\\
%%LINE3
\AUG-7-{}french  7-\CM{}  1\SG.\SM-\PRS.\DJ-7\OM{}-speak-\FV{}\\
%%TRANS1
\glt
‘As for as French, I speak it (unlike English or Lingala, for example).’\\
%%TRANS2
%%EXEND

\z
\z

We first present the formal properties of the pronoun/marker, and then discuss the interpretational effects it has as a contrastive marker.

\subsection{Formal properties of -\textit{o}}
\largerpage

The pronoun/particle reflects noun class and for each class there is a short and long series, as illustrated in \tabref{tab:run-nclass-o}.

\begin{table}
\begin{tabularx}{\textwidth}{XXlXX}
\lsptoprule
\multirow{2}{\linewidth}{noun class and prefix} & \multirow{2}{\linewidth}{example noun} & \multirow{2}{*}{gloss} & \multicolumn{2}{c}{particle/pronoun}\\
&  &  & short series & long series\\
\midrule
1. mu- & umuntu & ‘person’ & wé & wéewé\\
2. ba- & abantu & ‘persons’ & bó & bóobó\\
3. mu- & umusózi & ‘mountain’ & wó & wóowó\\
4. mi- & imisózi & ‘mountains’ & yó & yóoyó\\
5. ri-/$\varnothing$- & irigí & ‘egg’ & ryó & ryóoryó\\
6. ma- & amagí & ‘eggs’ & yó & yóoyó\\
7. ki- & igiteérwa & ‘plant’ & có & cóocó\\
8. bi- & ibiteérwa & ‘plants’ & vyó & vyóovyó\\
9. n- & inká & ‘cow’ & yó & yóoyó\\
10. n- & inká & ‘cows’ & zó & zóozó\\
11. ru- & urutoke & ‘finger’ & rwó & rwóorwó\\
12. ka- & akayáabu & ‘cat’ & kó & kókó\\
13. tu- & utuyáabu & ‘cats’ & twó & twótwó\\
14. bu- & ububaasha & ‘power, capacity’ & bwó & bwóbwó\\
15. ku- & kugenda & ‘to go’ & kwó & kwókwó\\
16. ha-  & ahantu & ‘place’ & hó & hóhó\\
17. ku & ku nzu & ‘on the house’ & hó & hóhó\\
18. mu & mu nzu & ‘in the house’ & hó & hóhó\\
19. i & i Bujumbura & ‘in Bujumbura’ & hó & hóhó\\
\lspbottomrule
\end{tabularx}
\caption{Noun classes and the particle -ó in Kirundi}
\label{tab:run-nclass-o}
\end{table}

When combining with a personal pronoun referring to a speech act participant, the (formally locative) form -\textit{ho} is used. The forms are given in \tabref{tab:run-pro-o}, and their use is illustrated in \xref{bkm:Ref70288900}.


\begin{table}
\begin{tabularx}{0.66\textwidth}{lXX}
\lsptoprule
1\SG & je/jeewé & jehó\\
2\SG & we/wewé & wehó\\
1\PL & twe/tweebwé & twehó\\
2\PL & mwe/mweebwé & mwehó\\
\lspbottomrule
\end{tabularx}
\caption{Personal pronouns for speech act participants and the particle -ó in Kirundi}
\label{tab:run-pro-o}
\end{table}

%%EAX
\ea
%%JUDGEMENT
%%LABEL
\label{bkm:Ref70288900}
%%CONTEXT
%%LINE1
Mweebwé genda; \textbf{tweho} tuzoza ejo.\\
%%LINE2
\gll
mweebwé  $\varnothing$-geend-a  twe-hó  tu-zo-əz-a  ejó\\
%%LINE3
2\PL.\PRO{}  2\PL.\SM-{}go-\IMP{}  1\PL.\PRO{}-\CM{}  1\PL.\SM-\FUT-{}come-\FV{}  tomorrow\\
%%TRANS1
\glt
‘You go; as for us, we shall go tomorrow.’\\
%%TRANS2
%%EXEND

\z

As a contrastive topic marker, the particle typically follows the topical, left-peripheral noun, as in \xref{bkm:Ref74757572}. It cannot precede the noun it refers to, as shown in \xref{bkm:Ref74757617}, but the particle can appear initially as in \xref{bkm:Ref74757916}, or finally as in \xref{bkm:Ref74757932}, when referring to a pronominal referent that is indexed on the verb – in this case the subject marker in class 2.

\ea
%%EAX
\ea
%%JUDGEMENT
%%LABEL
\label{bkm:Ref74757572}
%%CONTEXT
%%LINE1
Abáana bó baamaze kugenda kw’ishuúre.\\
%%LINE2
\gll
a-ba-áana  ba-ó  ba-a-a-mar-ye  ku-geenda  kw’  i-shuúre\\
%%LINE3
\AUG-2-{}child  2-\CM{}  2.\SM-\N.\PST-\DJ-{}finish-\PFV{}  15-go  17  \AUG-5.school\\
%%TRANS1
\glt
‘As for the children, they have already gone to school.’ (but the women haven’t)\\
%%TRANS2
%%EXEND

%%EAX
\ex
%%JUDGEMENT
%%LABEL
\label{bkm:Ref74757617}
%%CONTEXT
%%LINE1
(*Bó) abáana baamaze kugenda kw’ishuúre.\\
%%LINE2
\gll
ba-ó  a-ba-áana  ba-a-a-mar-ye  ku-geenda  kw’  i-shuúre\\
%%LINE3
2-\CM{}  \AUG-2-{}child  2.\SM-\N.\PST-\DJ-{}finish-\PFV{}  15-go  17  \AUG-5.school\\
%%TRANS1
\glt
int. ‘As for the children, they have already gone to school.’\\
%%TRANS2
%%EXEND

%\label{ex:kirundi:166}
%%EAX
\ex
%%JUDGEMENT
%%LABEL
\label{bkm:Ref74757916}
%%CONTEXT
%%LINE1
Bó, baamaze kugenda kw’ishuúre.\\
%%LINE2
\gll
ba-ó  ba-a-a-mar-ye  ku-geenda  kw’  i-shuúre\\
%%LINE3
2-\CM{}  2.\SM-\N.\PST-\DJ-{}finish-\PFV{}  15-go  17  \AUG-5.school\\
%%TRANS1
\glt
‘As for them, they have already gone to school.’\\
%%TRANS2
%%EXEND

%%EAX
\ex
%%JUDGEMENT
%%LABEL
\label{bkm:Ref74757932}
%%CONTEXT
Abáana baamaze kugenda kw’ishuúre, bó.\\
%%LINE1
%%LINE2
\gll
a-ba-áana  ba-a-a-mar-ye  ku-geenda  kw’  i-shuúre  ba-ó\\
%%LINE3
\AUG-2-{}child  2.\SM-\N.\PST-\DJ-{}finish-\PFV{}  15-go  17  \AUG-5.school  2-\CM{} \\
%%TRANS1
\glt
‘Children have already gone at school, them.’ (as opposed to women)\\
%%TRANS2
%%EXEND

\z
\z

The same holds for objects that are pronominally represented by an object marker: the coreferring left- or right-dislocated phrase can be marked by -\textit{o}, as shown in \xref{bkm:Ref74758345}. Note that the object marker must be present here.

\ea
\label{bkm:Ref74758345}
(Context: Someone was supposed to buy charcoal, maize, and oil.)\\
%%EAX
\ea
%%JUDGEMENT
%%LABEL
%\label{ex:kirundi:56}
%%CONTEXT
%%LINE1
Amakára yó yaayaguze.\\
%%LINE2
\gll
a-ma-kára  a-ó  a-a-a-*(ya)-gur-ye\\
%%LINE3
\AUG-6-{}coal  6-\CM{}  1\SM-\N.\PST-\DJ-6\OM{}-buy-\PFV{}\\
%%TRANS1
\glt
‘As for as the charcoal, s/he bought it.’ (but the rest not)\\
%%TRANS2
%%EXEND

%%EAX
\ex
%%JUDGEMENT
%%LABEL
%%CONTEXT
%%LINE2
\gll
A-a-a-*(ya)-gur-ye,  a-ma-kára  a-ó.\\
%%LINE3
1\SM-\N.\PST-\DJ-6\OM{}-buy-\PFV{}  \AUG-6-{}coal  6-\CM{}\\
%%TRANS1
\glt
  ‘S/he bought it, (the) charcoal.’ (but the rest not)\\
%%TRANS2
%%EXEND

\z
\z

The marker/pronoun also appears independently in the right context, both right- and left-dislocated, as illustrated in \xref{bkm:Ref78532151:B} B and B$'$. This seems to fulfil both functions, as a pronoun, and a contrastive topic marker.

\ea
\label{bkm:Ref78532151}
\begin{xlist}[B$'$:]
  \exi{A:}{Have you eaten the beans?}
  \exi{B:}{No.}
  \exi{A:}{What about the banana?}
%%EAX
\exi{B:}{
%%JUDGEMENT
%%LABEL
\label{bkm:Ref78532151:B}
%%CONTEXT
%%LINE1
Naawuríiye, wó. \\
%%LINE2
\gll
\N{}-a-a-wu-rí-ye  u-ó \\
%%LINE3
1\SG.\SM-\N.\PST-\DJ-3\OM{}-eat-\PFV{}  3-\CM{} \\
%%TRANS1
\glt
‘Thát I have eaten.’\\
%%TRANS2
}
%%EXEND

%%EAX
\exi{B$'$:}{
%%JUDGEMENT
%%LABEL
\label{bkm:Ref78532151:B'}
%%CONTEXT
%%LINE1
Wó, naawuriiye.\\
%%LINE2
\gll
u-ó  \N{}-a-a-wu-ri-ye \\
%%LINE3
3-\CM{}  1\SG.\SM-\N.\PST-\DJ-3\OM{}-eat-\PFV{} \\
%%TRANS1
\glt
‘Thát I have eaten.’\\
%%TRANS2
}
%%EXEND

\end{xlist}
\z

Having seen the formal properties of the pronoun/particle, we can discuss its use and interpretation in more detail.

\subsection{Functional properties of -\textit{o}}

The presence of the particle adds a contrastive meaning, that is, the referent is contrasted with another referent. However, it can only be used with given referents, not with newly presented, focal referents. This is shown in the incompatibility of the particle with a noun in a basic cleft or pseudocleft (both constructions expressing focus, see \sectref{bkm:Ref72334782}), shown in \xref{bkm:Ref75175945} and \xref{bkm:Ref75175946}, respectively.\pagebreak

%%EAX
\ea
%%JUDGEMENT
%%LABEL
\label{bkm:Ref75175945}
%%CONTEXT
%%LINE1
Ni abaana (*bo) baajé.  \jambox*{[basic cleft]}
%%LINE2
\gll
ni  a-ba-áana  ba-ó  ba-əz-ye\textsuperscript{H}\\
%%LINE3
\COP{}  \AUG-2{}-child  2-\CM{}  2\SM{}-come-\PFV.\REL{}\\
%%TRANS1
\glt
‘It’s the children who have come.’\\
%%TRANS2
%%EXEND

\z

%%EAX
\ea
%%JUDGEMENT
%%LABEL
\label{bkm:Ref75175946}
%%CONTEXT
%%LINE1
Abaáje ni abaana (*bó).  \jambox*{[pseudocleft]}
%%LINE2
\gll
a-ba-əz-ye\textsuperscript{H}  ni  a-ba-áana  ba-ó\\
%%LINE3
\AUG-2\SM-{}come-\PFV.\REL{}  \COP{}  \AUG-2{}-child  2-\CM{}\\
%%TRANS1
\glt
‘The ones who have come are the children.’\\
%%TRANS2
%%EXEND

\z

It is telling that instead of the ungrammatical \xref{bkm:Ref75175946} with the particle, the sentence in \xref{bkm:Ref75176045} comes more naturally. Here, the particle modifies the initial free relative \textit{abaáje} ‘the ones who have come’, which is given information, as indicated in the context, and functions as the topic.

%%EAX
\ea
%%JUDGEMENT
%%LABEL
\label{bkm:Ref81816342}
%%CONTEXT
\label{bkm:Ref75176045}(Context: We were expecting three groups of people: adults, elderly, and children. Not all have arrived.)\\
%%LINE1
Abaáje bó ni abáana.\\
%%LINE2
\gll
a-ba-əz-ye\textsuperscript{H}  ba-ó  ni  a-ba-áana\\
%%LINE3
\AUG-2\SM-{}come-\PFV.\REL{}  2-\CM{}  \COP{}  \AUG{}-2-child\\
%%TRANS1
\glt
‘As for the ones who have come, it’s the children.’\\
%%TRANS2
%%EXEND

\z

In the same way, the postverbal logical subject cannot be modified by -\textit{ó}, as shown for Locative Inversion in \xref{bkm:Ref75179471}. This is expected, as the postverbal logical subject is the focus (see \sectref{bkm:Ref75783474}). Equally expected is the fact that the preverbal locative \textit{can} be modified by -\textit{ó}, as in \xref{bkm:Ref75179462}, since this has been shown to function as the topic \citep[62]{Nshemezimana2020}.

%%EAX
\ea
%%JUDGEMENT
%%LABEL
\label{bkm:Ref75179471}
%%CONTEXT
%%LINE1
Mu Kirundo hagiiyeyó abáana babiri (*bó).\\
%%LINE2
\gll
mu  Kirundo  ha-a-gi-ye=yó  a-ba-áana  ba-biri  ba-ó\\
%%LINE3
18  Kirundo  16\SM{}-\N.\PST{}-go-\PFV{}=19  \AUG{}-2-child  2-two  2-\CM{}\\
%%TRANS1
\glt
‘Kirundo, two children went there.’\\
%%TRANS2
%%EXEND

\z

%%EAX
\ea
%%JUDGEMENT
%%LABEL
\label{bkm:Ref75179462}
%%CONTEXT
(How many children have gone to Bujumbura and how many to Kirundo?)\\
%%LINE1
Mu  Kirundo hó hagiiyeyó abáana babiri.\\
%%LINE2
\gll
mu  Kirundo  ha-ó  ha-a-gi-ye=yó  a-ba-áana  ba-biri\\
%%LINE3
18  Kirundo  16-\CM{}  16\SM{}-\N{}.\PST{}-go-\PFV{}=19  \AUG{}-2-child  2-two\\
%%TRANS1
\glt
‘As for Kirundo, two children went there.’\\
%%TRANS2
%%EXEND

\z

Finally, -\textit{ó} cannot modify the object when the object is in focus following a conjoint form, as in \xref{bkm:Ref75180460}, or even when it is included in VP focus following a disjoint form, as in \xref{bkm:Ref75180621}. The particle can only be added when the object is dislocated, as seen by the presence of the object marker on the verb \xref{bkm:Ref75180614}.

%%EAX
\ea
%%JUDGEMENT
%%LABEL
\label{bkm:Ref75180460}
%%CONTEXT
(What have you eaten?)\\
%%LINE1
Nariiye umuhwi (*wó).\\
%%LINE2
\gll
\N{}-a-$\varnothing$-rí-ye  u-mu-hwí  u-ó\\
%%LINE3
1\SG.\SM-\N.\PST-\CJ{}-eat-\PFV{}  \AUG{}-3{}-banana  3-\CM{}\\
%%TRANS1
\glt
‘I have eaten a banana.’\\
%%TRANS2
%%EXEND

\z

%%EAX
\ea
%%JUDGEMENT
%%LABEL
\label{bkm:Ref75180621}
%%CONTEXT
(What have you done?)\\
%%LINE1
Naaríiye umuhwí (*wó).\\
%%LINE2
\gll
\N{}-a-a-rí-ye  u-mu-hwí  u-ó\\
%%LINE3
1\SG.\SM-\N.\PST-\DJ{}-eat-\PFV{}  \AUG-{}3-banana  3-\CM{}\\
%%TRANS1
\glt
‘I have eaten a banana.’\\
%%TRANS2
%%EXEND

\z

%%EAX
\ea
%%JUDGEMENT
%%LABEL
\label{bkm:Ref75180614}
%%CONTEXT
%%LINE1
Naawuríiye, umuhwí (wó).\\
%%LINE2
\gll
\N{}-a-a-wu-rí-ye  u-mu-hwí  u-ó\\
%%LINE3
1\SG.\SM-\N.\PST-\DJ-3\OM{}eat-\PFV{}  \AUG-{}3-banana  3-\CM{}\\
%%TRANS1
\glt
‘I have eaten it, the banana.’\\
%%TRANS2
%%EXEND

\z

While the particle is syntactically optional in \xref{bkm:Ref75180614}, the contexts for felicitous use with and without the particle are crucially different: without -\textit{ó} , \xref{bkm:Ref75180614} is an answer to ‘What have you done with the banana?’ (focus just on the verb, the object being given information), while the presence of -\textit{ó} requires that some alternative edible referent has been mentioned before, as in ‘So if you haven’t eaten the avocado, then what about the banana?’. This again shows the contrastive interpretation associated with the particle -\textit{ó}.

As -\textit{ó} can only combine with dislocated NPs, and these necessarily represent given information, the only interpretation that \textit{umuntu} ‘person, someone, human being’ can receive in \xref{bkm:Ref75182159} is generic. Furthermore, this allows for alternatives and hence contrast with other living beings.

%%EAX
\ea
%%JUDGEMENT
%%LABEL
\label{bkm:Ref75182159}
%%CONTEXT
%%LINE1
Umuntu wé, azoobazwa ivyó yakóze.\\
%%LINE2
\gll
u-mu-ntu  wé  a-zoo-báz-w-a  i-bi-ó  a-a-kór-ye\textsuperscript{H}\\
%%LINE3
\AUG{}-1-person  1.\CM{}  1\SM{}-\FUT{}-ask-\PASS{}-\FV{}  \AUG{}-8-\PRO{}  1\SM{}-\N.\PST{}-do-\PFV{}.\REL{}\\
%%TRANS1
\glt
‘Man will be held responsible for what he has done.’ (as opposed to animals or trees)\\
%%TRANS2
%%EXEND

\z

When a contrast is made between two active referents, it is infelicitous to mark the first referent by -\textit{ó}, but it is preferred for the second referent to have the particle, either by itself or with \textit{na}-, both illustrated in \xref{bkm:Ref81509490}.

%%EAX
\ea
%%JUDGEMENT
%%LABEL
\label{bkm:Ref81509490}
%%CONTEXT
(What is the man holding and what is the woman holding? + QUIS picture)\\
%%LINE1
Umugabo (*wé) afise igikombe; umugoré nawé/wé afise isáhaáni.\\
%%LINE2
\gll
u-mu-gabo  wé  a-fit-ye  i-ki-kombe  u-mu-goré  na-we/we   a-fit-ye  i-sahani \\
%%LINE3
\AUG-1{}-man  1.\CM{}  1\SM{}-have-\PFV{}  \AUG-7-{}cup  \AUG-1{}-woman  and-1\PRO/1.\CM{}   1\SM-have-\PFV{}  \AUG{}-9.plate \\
%%TRANS1
\glt
  ‘The man has a cup; the woman (as for her), she has a plate.’\\
%%TRANS2
%%EXEND

\z

When a non-subject is contrasted in the same way, it is naturally fronted and marked by the particle, as in \xref{bkm:Ref75184712} and \xref{bkm:Ref75184713}. In both examples, a superset is introduced in the preceding question, and two contrasting subsets are mentioned in the answer.

%%EAX
\ea
%%JUDGEMENT
%%LABEL
\label{bkm:Ref75184712}
%%CONTEXT
(Does Yona buy coffee for his colleagues?)\\
%%LINE1
Yona agurira agahawá shéebuja aríko bagenziwé (bó) abagurira icáayi.\\
%%LINE2
\gll
Yona  a-$\varnothing$-gur-ir-a  a-ka-hawá  shéebuja  aríko  ba-genzi-wé   ba-ó  a-$\varnothing$-ba-gur-ir-a  i-ki-áayi \\
%%LINE3
Jonas  1\SM-\PRS.\CJ{}-buy-\APPL-\FV{}  \AUG{}-12-coffee  his.boss  but  2-colleague-\POSS.1{}   2-\CM{}  1\SM-\PRS.\CJ-2\OM{}-buy-\APPL-\FV{}  \AUG{}-7-tea \\
%%TRANS1
\glt
  ‘Jonas buys coffee for his boss but (as for) his colleagues, he buys them tea.’\\
%%TRANS2
%%EXEND

\z

%%EAX
\ea
%%JUDGEMENT
%%LABEL
\label{bkm:Ref75184713}
%%CONTEXT
(Did s/he iron the clothes?)\\
%%LINE1
Amashaáti yaayagooroye aríko amasume (yó) yayaretse uko.\\
%%LINE2
\gll
a-ma-shaáti  a-a-a-ya-gooror-ye  aríko  a-ma-sume  a-ó   a-a-a-ya-rek-ye  uko \\
%%LINE3
\AUG-6-{}shirt  1\SM-\N.\PST-\DJ-6\OM{}-iron-\PFV{}  but  \AUG-{}6-towel  6-\CM{}   1\SM-\N.\PST-\DJ-6\OM-{}leave-\PFV{}  like.that \\
%%TRANS1
\glt
  ‘The shirts, s/he has ironed them but (as for) the towels, s/he has left them like that.’\\
%%TRANS2
%%EXEND

\z

As the particle associates with topics, it can be added to the predicate doubling construction discussed in \sectref{bkm:Ref75783502}. When present, it can highlight the concessive aspect of meaning, or give a mirative or exclamative flavour, as the underspecification of interpretations in \xref{bkm:Ref75263857} and \xref{bkm:Ref75264568} shows.

%%EAX
\ea
%%JUDGEMENT
%%LABEL
\label{bkm:Ref81509745}
%%CONTEXT
%%LINE1
\label{bkm:Ref75263857}Ukwooga kó aróoze!\\
%%LINE2
\gll
u-ku-óoga  ku-ó  a-ra-óog-ye\\
%%LINE3
\AUG{}-15-swim  15-\CM{}  1\SM{}-\PRS.\DJ{}-swim-\PFV{}\\
%%TRANS1
\glt
‘She has swum a lot.’ (more than usual or than expected)\\
%%TRANS2
%%EXEND

\z

%%EAX
\ea
%%JUDGEMENT
%%LABEL
\label{bkm:Ref75264568}
%%CONTEXT
%%LINE1
Kwandika kó uraanditse.\\
%%LINE2
\gll
ku-andika  ku-ó  u-ra-andik-ye\\
%%LINE3
15-write  15-\CM{}  2\SG.\SM-\PRS.\DJ{}-write-\PFV{}\\
%%TRANS1
\glt
‘You have really written a lot!’ (more than expected)\\
%%TRANS2
  ‘Well, at least you have written.’ (but it isn’t very good)

%%EXEND

\z

The same mirative/exclamative flavour can be present when -\textit{ó} is used with a (non-infinitival) NP, as illustrated in \xref{bkm:Ref75263788} and \xref{bkm:Ref75263790}. The contrastive or exclamative interpretation is dependent on the context. Here, prosody/intonation plays an important role in removing ambiguity. For example, in the exclamative context, the construction has an exclamatory intonational reading, as in \xref{bkm:Ref75263788}, while in a contrastive context, we have two intonational contours, one ascending and the other descending which respectively affect the first part and the second part of the sentence, as in \xref{bkm:Ref75263790}.

%%EAX
\ea
%%JUDGEMENT
%%LABEL
\label{bkm:Ref75263788}
%%CONTEXT
(Context: You come outside after it has rained and see lots of puddles and even broken branches.)\\
%%LINE1
Imvúra yó iraguuye!\\
%%LINE2
\gll
i-n-vúra  i-ó  i-ra-gu-ye\\
%%LINE3
\AUG{}-9-rain  9-\CM{}  9\SM-\PRS.\DJ{}-fall-\PFV{}\\
%%TRANS1
\glt
‘It has really rained (a lot)!’\\
%%TRANS2
%%EXEND

\z

%%EAX
\ea
%%JUDGEMENT
%%LABEL
\label{bkm:Ref75263790}
%%CONTEXT
Abaana bó, arabafise.\\
%%LINE1
%%LINE2
\gll
a-ba-áana  ba-ó  a-ra-ba-fit-ye\\
%%LINE3
\AUG-2{}-child  2-\CM{}  1\SM-\DJ-2\OM{}-have\\
%%TRANS1
\glt
%%TRANS2
%%EXEND

\ea
‘As for children, s/he has them.’ (but the rest not) \\
(Contrastive situation: Does s/he have a house, cows, and children now?)
\ex ‘Children s/he has enough.’\\
(Exclamative situation: I am surprised that s/he has so many children!)
\z
\z

The surprise at the excessive extent of the event as a whole can also be marked on just the object as in \xref{bkm:Ref75265013}, reinforced by the ideophone \textit{pé}. There is no contrastive interpretation of \textit{amáazi} ‘water’ here.

%%EAX
\ea
%%JUDGEMENT
%%LABEL
\label{bkm:Ref75265013}
%%CONTEXT
(Context: You are swimming in the lake, I go away for some shopping, when I come back you’re still swimming there.)\\
%%LINE1
Amáazi yó, urayooze pé!\\
%%LINE2
\gll
a-ma-zi  a-ó  u-ra-ya-óog-ye  pe\\
%%LINE3
\AUG-6-{}water  6-\CM{}  2\SG.\SM{}-\PRS.\DJ-6\OM{}-swim-\PFV{}  \IDEO{}\\
%%TRANS1
\glt
‘You have swum a long time!’\\
%%TRANS2
%%EXEND

\z

To summarise, -\textit{ó} originates as a pronoun and still functions as such, but has further developed as a marker to indicate contrast on given (topical) referents. A next step in the research on this particle should include the interaction of the particle with specific contexts to specify in detail the possibilities for mirative and exclamative interpretations.

\section{Copular constructions}
\label{bkm:Ref72334782}
There are four constructions in Kirundi that involve a copula and some form or relative clause: 1) the presentational, introduced by \textit{hari(ho)}; 2) the basic cleft; 3) the pseudocleft; and 4) the reverse pseudocleft\slash left\hyp peripheral NP + cleft. These constructions can be distinguished not only by their syntactic configuration but also their information structure. We discuss these in turn, after we present general information on the copula and verb ‘to be’, which are relevant to all four constructions. Throughout this section, we build on \citet{LafkiouiEtAl2016}, who provide an in-depth discussion of these constructions on the basis of corpus data. We add to their analysis by providing further tests for the exact interpretation of each of the constructions.

\subsection{Copular verbs}

There are four markers involved in nominal predication: invariant \textit{ni}/\textit{si}, inflected verbs -\textit{ri} and \textit{ba}-, and negative presentational marker \textit{ntaa}. We also refer to \citet{LafkiouiEtAl2016} for discussion of copulas in Kirundi.

The invariant copula used in each of the constructions mentioned is used to create a nominal predicate. As such, it does not accept valency-changing or other derivational morphology, and cannot inflect for tense, aspect, mood, and person. The only variation is the affirmative \textit{ni} vs. negative \textit{si}. We show its use in a predicational copular clause in \xref{bkm:Ref73705366}, and in a cleft in \xref{bkm:Ref75358266} and \xref{bkm:Ref74741046}.
\largerpage[2]

%%EAX
\ea
%%JUDGEMENT
%%LABEL
\label{bkm:Ref73705366}
%%CONTEXT
%%LINE1
Iryo koóti ryaawe ni/si rishaásha.\\
%%LINE2
\gll
i-ri-o  koóti  ri-aawe  ni/si  ri-shaásha\\
%%LINE3
\AUG-5-\DEM{}\textsubscript{3}  jacket  5-\POSS.2\SG{}  \COP/\COP.\NEG{}  5-new\\
%%TRANS1
\glt
‘This jacket of yours is/is not new.’\\
%%TRANS2
%%EXEND

\z

%%EAX
\ea
%%JUDGEMENT
%%LABEL
\label{bkm:Ref75358266}
%%CONTEXT
(What did you bring us from market?)\\
%%LINE1
Ni ibitúumbura nabaázaniye.\\
%%LINE2
\gll
ni  i-bi-túumbura  \N{}-a-ba-zan-ir-ye\textsuperscript{H}\\
%%LINE3
\COP{}  \AUG-8{}-doughnut  1\SG.\SM-\N.\PST-2\OM{}-bring-\APPL-\PFV.\REL{}\\
%%TRANS1
\glt
‘It is doughnuts that I brought you.’\\
%%TRANS2
%%EXEND

\z


%%EAX
\ea
%%JUDGEMENT
%%LABEL
\label{bkm:Ref74741046}
%%CONTEXT
(This water was poured by Kabura.)\\
%%LINE1
Oya, si we yayasheshe.\\
%%LINE2
\gll
oya,  si  wé  a-a-a-sees-ye\textsuperscript{H}\\
%%LINE3
no,  \COP.\NEG{}  1.\PRO{}  1\SM-\N.\PST-6\OM{}-pour-\PFV.\REL{}\\
%%TRANS1
\glt
‘No, it wasn’t him who poured it out.’\\
%%TRANS2
%%EXEND

\z


The invariant copula \textit{ni/si} is only used to express the present tense – when tense or aspect other than the general present needs to be indicated, the copular verb -\textit{ri} is used, a reflex of Proto-Bantu *\textit{d\`{ɪ}}. This verb is not restricted in inflection, taking prefixes for tense and a subject marker, as shown for a simple predicative clause in \xref{bkm:Ref75184052}, for a cleft in \xref{bkm:Ref73960454}, and for a presentational construction in \xref{bkm:Ref75184028}.

%%EAX
\ea
%%JUDGEMENT
%%LABEL
\label{bkm:Ref75184052}
%%CONTEXT
(How many were the children you told me about?)\\
%%LINE1
Abo báana baari baké caane.\\
%%LINE2
\gll
a-ba-o  ba-áana  ba-a-ri  ba-ké  caane\\
%%LINE3
\AUG-2-\DEM{}\textsubscript{2}  \AUG-2-{}child  2\SM-\N.\PST{}-be  2-few  \INT{}\\
%%TRANS1
\glt
‘These children were very few.’\\
%%TRANS2
%%EXEND

\z

%%EAX
\ea
%%JUDGEMENT
%%LABEL
\label{bkm:Ref73960454}
%%CONTEXT
(Who you were talking with?)\\
%%LINE1
Yari Jeanine aje kundamutsa.\\
%%LINE2
\gll
a-a-ri  Jeanine  a-əz-ye\textsuperscript{H}  ku-n-ramutsa\\
%%LINE3
1\SM-\N.\PST{}-be  Jeanine  1\SM-{}come-\PFV.\REL{}  15-1\SG.\OM{}-greet\\
%%TRANS1
\glt
‘It was Jeanine who came to greet me.’\\
%%TRANS2
%%EXEND

\z

\ea
\label{bkm:Ref75184028}
\begin{xlist}
\exi{A:} Why are there so many people in your home? (after seeing a crowd of people at B’s place.)\\
%%EAX
\exi{B:}
%%JUDGEMENT
%%LABEL
%%CONTEXT
%%LINE1
Hari abantu baatuúzaniye akayoga.\\
%%LINE2
\gll
ha-$\varnothing$-ri  a-ba-ntu  ba-a-tu-zan-ir-ye\textsuperscript{H}  a-ka-yogá\\
%%LINE3
\EXP-\PRS{}-be  \AUG-2-{}person  2\SM-\N.\PST-1\PL.\OM{}bring-\APPL-\PFV.\REL{}  \AUG{}-12-beer\\
%%TRANS1
\glt
‘There are people who brought us beer.’\\
%%TRANS2
%%EXEND

\end{xlist}
\z

The verb -\textit{ri} can and must also be used in a subordinate clause, in the past as in \xref{bkm:Ref75355595}, but strikingly also in the present tense, with an impersonal subject marker \textit{a}- as in \xref{bkm:Ref73716709}, called an expletive marker by \citet{LafkiouiEtAl2016}. The invariant copula \textit{ni/si} is unacceptable, shown in \xref{bkm:Ref73716671}, and so is normal subject inflection, as in \xref{bkm:Ref75247211}.

\ea
\label{bkm:Ref75355595}
(Context: About the noise heard the night before.)
%%EAX
\ea
%%JUDGEMENT
[]{
%%LABEL
%%CONTEXT
%%LINE1
Biyumviira kó baarí abasumá baarí baaje kwiíba imódoka yíiwé.\\
%%LINE2
\gll
Ba-iyumviir-a  kó  ba-a-rí  a-ba-sumá  ba-a-ri\textsuperscript{H}   ba-əz-ye\textsuperscript{H}  ku-íiba  i-módoka  i-íiwé \\
%%LINE3
2\SM{}-think-\FV{}  \COMP{}  2\SM-\N.\PST{}-be  \AUG-2{}-thief  2\SM-\N.\PST{}-be.\REL{}   2\SM-{}come-\PFV.\REL{}  15-steal  \AUG-{}9-car  9-\POSS.1{} \\
%%TRANS1
\glt
  ‘They think it was the thieves who came to steal his car.’\\
%%TRANS2
}
%%EXEND

%%EAX
\ex
%%JUDGEMENT
[]{
%%LABEL
\label{bkm:Ref73716709}
%%CONTEXT
%%LINE1
%%LINE2
\gll
(…)  kó  a-$\varnothing$-rí  a-ba-sumá  ba-a-ri\textsuperscript{H}  ba-əz-ye\textsuperscript{H}   ku-íiba  i-módoka  i-íiwé \\
%%LINE3
(…)  \COMP{}  \EXP-\PRS{}-be  \AUG-2{}-thief  2\SM-\N.\PST{}-be.\REL{}  2\SM-{}come-\PFV.\REL{}   15-steal  \AUG-{}9-car  9-\POSS.1{} \\
%%TRANS1
\glt
  ‘(…) that it is the thieves who came to steal his car.’\\
%%TRANS2
}
%%EXEND

%%EAX
\ex
%%JUDGEMENT
[*]{
%%LABEL
\label{bkm:Ref75247211}
%%CONTEXT
%%LINE1
%%LINE2
\gll
(…)  kó  ba-$\varnothing$-rí  a-ba-sumá  ba-a-ri\textsuperscript{H}  ba-əz-ye\textsuperscript{H}   ku-íiba  i-módoka  i-íiwé \\
%%LINE3
(…)  \COMP{}  2\SM-\PRS{}-be  \AUG-2{}-thief  2\SM-\N.\PST{}-be.\REL{}  2\SM-{}come-\PFV.\REL{}   15-steal  \AUG-{}9-car  9-\POSS.1{} \\
%%TRANS1
\glt
  int. ‘(…) that it is the thieves who came to steal his car.’\\
%%TRANS2
}
%%EXEND

%%EAX
\ex
%%JUDGEMENT
[*]{
%%LABEL
\label{bkm:Ref73716671}
%%CONTEXT
%%LINE1
%%LINE2
\gll
(…)  kó  ni/si  a-ba-sumá  ba-a-ri\textsuperscript{H}  ba-əz-ye\textsuperscript{H}   ku-íiba  i-módoka  i-íiwé \\
%%LINE3
(…)  \COMP{}  \COP{}  \AUG-2{}-thief  2\SM-\N.\PST{}-be.\REL{}  2\SM-{}come-\PFV.\REL{}   15-steal  \AUG-{}9-car  9-\POSS.1{} \\
%%TRANS1
\glt
  int. ‘(…) that it was the thieves who came to steal his car.’\\
%%TRANS2
}
%%EXEND

\z
\z

There is a third verb of existence, which is used in presentationals. This is -\textit{bá}, which can be translated as ‘be, exist’ in predicational constructions \xref{bkm:Ref75356327}, or as ‘live’ \xref{bkm:Ref75356329}. Both -\textit{ri} and -\textit{bá} are analysed by \citet[145]{Meeussen1959} as ``defective verbs". While -\textit{bá} is not used in clefts and pseudoclefts, it does occur in presentational constructions, as shown in \xref{bkm:Ref75253864}.\largerpage[2]

\ea
\label{bkm:Ref75356327}
\begin{xlist}
\exi{A:}  How do you appreciate my children now? (A asks B, showing him his children whom he had seen some time ago.)\\
%%EAX
\exi{B:}
%%JUDGEMENT
%%LABEL
%%CONTEXT
%%LINE1
Mbona baárabáaye beezá caane.\\
%%LINE2
\gll
\N{}-bón-a  ba-á-ra-bá-ye  ba-iizá  caane\\
%%LINE3
1\SG.\SM-{}see-\FV{}  2\SM-\RMT.\PST-\DJ-be-\PFV{}  2-beautiful  \INT{}\\
%%TRANS1
\glt
‘I see that they have become very beautiful.’\\
%%TRANS2
%%EXEND
\end{xlist}

\ex
\label{bkm:Ref75356329}

%%EAX
%%JUDGEMENT
%%LABEL
%%CONTEXT
(Context: A says to B, showing her people who are sitting next to them.)\\
%%LINE1
Aba bantu baba iwacu.\\
%%LINE2
\gll
a-ba-a  ba-ntu  ba-$\varnothing$-bá-a  iwacu\\
%%LINE3
\AUG-2-\DEM{}\textsubscript{1}  \AUG-{}2-person  2\SM-\PRS{}-live-\FV{}  our.home\\
%%TRANS1
\glt
‘These people live with us.’\\
%%TRANS2
%%EXEND
 
\ex
\label{bkm:Ref75253864}
%%EAX
%%JUDGEMENT
%%LABEL
%%CONTEXT
(Context: Telling a story of the panther.)\\
%%LINE1
Haábaaye ingwe yavyáaye ibibuguru ndwi, irahéza yingiinga icuúya ngo kiyisigáranire abáana.\\
%%LINE2
\gll
ha-á-$\varnothing$-bá-ye  i-\N-{}gwe  i-á-vyáar-ye\textsuperscript{H}   i-bi-buguru  ndwi,  i-ra-héz-a  i-íingiing-a  i-ki-uúya   ngo  ki-yi-sigár-an-ir-e  a-ba-áana\\
%%LINE3
\EXP-\RMT.\PST-\CJ{}-be-\PFV{}  \AUG{}-9-{}panther  9\SM-\RMT.\PST-{}give.birth-\PFV.\REL{}   \AUG-8-{}cub  seven  9\SM-\DJ-{}finish-\FV{}  {}9\SM{}-{}beg-\FV{}  \AUG{}-7-{}serval   in.order  7\SM-9\OM-{}keep-\ASS{}-\APPL{}-\SBJ{}  \AUG{}-2-{}child\\
%%TRANS1
\glt
  ‘Once upon a time there was a panther who gave birth to seven cubs; she begged the serval to keep her children for her (during her absence).’ (\textit{Imigani}, Contes, 1940s – BantUGhent corpus)\\
%%TRANS2
%%EXEND

\z

The two forms -\textit{bá} and -\textit{ri} are in complementary distribution, even in the presentational construction. For example, -\textit{ri} cannot be used in future tense \xref{bkm:Ref75258689} or perfective aspect \xref{bkm:Ref75258679}, where instead -\textit{bá} needs to be used.

\ea
\label{bkm:Ref75258689}
%%EAX
\ea
%%JUDGEMENT
[]{
%%LABEL
%%CONTEXT
%%LINE1
Ejó hazooba ináama y’abashíingamáteeká.\\
%%LINE2
\gll
ejó  ha-zoo-bá-a  i-náama  i-a  a-ba-shíingamáteeká\\
%%LINE3
tomorrow  \EXP{}-\FUT{}-be-\FV{}  \AUG{}-5.meeting  5-\CONN{}  \AUG{}-2-parliamentarian\\
%%TRANS1
\glt
‘Tomorrow there will be a meeting of parliamentarians.’\\
%%TRANS2
}
%%EXEND

%%EAX
\ex
%%JUDGEMENT
[*]{
%%LABEL
%%CONTEXT
%%LINE1
%%LINE2
\gll
Ejó  ha-zoo-ri  i-náama  i-a  a-ba-shíingamáteeká\\
%%LINE3
tomorrow  \EXP{}-\FUT{}-be-\FV{}  \AUG{}-5.meeting  5-\CONN{}  \AUG{}-2-parliamentarian\\
%%TRANS1
\glt
%%TRANS2
}
%%EXEND

\z
\ex
\label{bkm:Ref75258679}

%%EAX
\ea
%%JUDGEMENT
[]{
%%LABEL
%%CONTEXT
%%LINE1
Aho heepfó harabáaye isaánganya mu mwaánya uhezé.\\
%%LINE2
\gll
a-ha-o  heepfó  ha-$\varnothing$-ra-bá-ye  i-saánganya  mu  mu-aánya   u-her-ye\textsuperscript{H}\\
%%LINE3
\AUG-16-\DEM{}\textsubscript{2}  down  \EXP-\PRS-\DJ-{}be-\PFV{}  \AUG-5{}.accident  18  3-time   3\SM{}-finish-\PFV.\REL{}\\
%%TRANS1
\glt
  ‘Down there, an accident happens at the last moment.’\\
%%TRANS2
}
%%EXEND

%%EAX
\ex
%%JUDGEMENT
[*]{
%%LABEL
%%CONTEXT
%%LINE1
%%LINE2
\gll
A-ha-o  heepfó  ha-$\varnothing$-ra-ri  i-saánganya  mu  mu-aánya  u-her-ye\textsuperscript{H}\\
%%LINE3
\AUG-16-\DEM{}\textsubscript{2}  down  \EXP-\PRS-\DJ-{}be-\PFV{}  \AUG-5{}.accident  18  3-time  3\SM{}-finish-\PFV.\REL{}\\
%%TRANS1
\glt
%%TRANS2
}
%%EXEND

\z
\z

A question that is relevant for basic clefts is whether the invariant copula \textit{ni} has further lost its predicative functions in a cleft and is now simply a discourse operator \citep{Lambrecht1994,Muller2002,Blanche-Benvensite2002}. This has been argued for Kîîtharaka \citep{AbelsMuriungi2008} and Kikuyu \citep{Schwarz2007}, for example, and is suggested for Kirundi by \citet{LafkiouiEtAl2016}. However, considering the variation between \textit{ni} and -\textit{ri} in Kirundi, fulfilling the same function, as well as the fact that the lexical verb in the cleft is marked as relative (by a high tone) suggest that the biclausal cleft structure is still present and the copula functions to create a nominal predicate.

  Finally, there is a negative presentational marker \textit{ntaa}, used in simple assertions as in \xref{bkm:Ref75419400} and \xref{bkm:Ref75419402}, and in presentational constructions as in \xref{bkm:Ref75354090}. 

%%EAX
\ea
%%JUDGEMENT
%%LABEL
\label{bkm:Ref75419400}
%%CONTEXT
(How many members did you have in your meeting?)\\
%%LINE1
Ntaa na bátatu.\\
%%LINE2
\gll
ntaa  na  ba-tatu\\
%%LINE3
\NEG.\COP{}  even  2-three\\
%%TRANS1
\glt
‘There were not even three.’\\
%%TRANS2
%%EXEND

\z

\ea
\label{bkm:Ref75419402}
\begin{xlist}
\exi{A:} I will come to you tomorrow to buy maize.\\
%%EAX
\exi{B:}
%%JUDGEMENT
%%LABEL
%%CONTEXT
%%LINE1
Ntaa bigóori birího.\\
%%LINE2
\gll
Ntaa  bi-góori  bi-ri=hó.\\
%%LINE3
\NEG{}.\COP{}  8-maize  8-be.\REL{}=16.\PRO{}\\
%%TRANS1
\glt
‘There is no maize (that is here).’\\
%%TRANS2
%%EXEND

\end{xlist}
\z

\ea
\label{bkm:Ref75354090}
\begin{xlist}
\exi{A:}  Who were you talking to?\\
%%EAX
\exi{B:}
%%JUDGEMENT
%%LABEL
%%CONTEXT
%%LINE1
Ntaa muntu twavugana. Nari ndirimvye gusa. \\  %%LINE2
\gll
ntaa  mu-ntu  tu-a-vúg-an-a\textsuperscript{H}  \N{}-a-ri   \N{}-rírimb-ye  gusa \\
%%LINE3
\NEG.\COP{}  1-person  1\PL.\SM-\N.\PST-{}talk-\ASS-\FV.\REL{}  1\SG.\SM-\N.\PST{}-be   1\SG.\SM{}-sing-\PFV{}  only\\
%%TRANS1
\glt
  ‘We weren't talking to anyone. I had just sung.’ (lit. there wasn’t anyone we talked to)\\
%%TRANS2
%%EXEND

\end{xlist}
\z

\textit{Ntaa} can be analysed as consisting of negation \textit{nti}- and a stem -\textit{a} \citep{LafkiouiEtAl2016}. Like -\textit{ri}, it changes to the expletive marker \textit{a}- in a dependent clause, followed by negation -\textit{ta}- and stem -\textit{a}. These are illustrated in \xref{bkm:Ref73956939} and \xref{bkm:Ref73719164}, respectively. This suggests that we are not dealing with a predicate of existence but one of possession ‘have’; these are known to participate in presentational constructions (cf. French \textit{il y a}, Swahili \textit{ku-na}, see \citealt{Marten2013}).

\ea
(How many students found the exam?)\\
%%EAX
\ea
%%JUDGEMENT
%%LABEL
\label{bkm:Ref73956939}
%%CONTEXT
%%LINE1
Ntaa n’umwe yabitoóye.\\
%%LINE2
\gll
ntaa  n’  u-mwé  a-á-bi-tóo-ye\textsuperscript{H}\\
%%LINE3
\NEG.\COP{}  even  1-one  1\SM-\RMT.\PST-8\OM-{}find-\PFV.\REL{}\\
%%TRANS1
\glt
‘There is no one who found it.’\\
%%TRANS2
%%EXEND

%%EAX
\ex
%%JUDGEMENT
%%LABEL
\label{bkm:Ref73719164}
%%CONTEXT
%%LINE1
Mbona (kó) ataa n’umwe yabitoóye.\\
%%LINE2
\gll
\N{}-bón-a  (kó)  a-taa  n’  u-mwe  a-á-bi-tóor-ye\textsuperscript{H}\\
%%LINE3
1\SG.\SM{}-see-\FV{}  (\COMP{})  \EXP-\NEG.\COP{}  even  1-one  1\SM-\RMT.\PST-8\OM-{}find-\PFV.\REL{}\\
%%TRANS1
\glt
‘I see that there is no one who found it.’\\
%%TRANS2
%%EXEND

\z
\z

\subsection{Presentational}
\label{bkm:Ref75348623}
This section summarises the data and analysis by \citet[chapter~6]{Nshemezimana2016} and \citet{NshemezimanaMberamihigo2021}. We refer to these sources for further data and analysis of the presentational construction. The presentational construction with \textit{hari(ho)} consists of the copular verb -\textit{ri} inflected with the expletive subject marker \textit{ha}- and an optional enclitic \textit{=ho}, a noun phrase or pronoun, and a relative clause, as in \xref{bkm:Ref74054848}.

%%EAX
\ea
%%JUDGEMENT
%%LABEL
\label{bkm:Ref74054848}
%%CONTEXT
(Why are you here on the road?)\\
%%LINE1
Hari abantu baajé kuturamutsa; turabáherekeje.\\
%%LINE2
\gll
ha-$\varnothing$-ri  a-ba-ntu  ba-a-əz-ye\textsuperscript{H}  ku-tu-ramutsa   tu-ra-ba-herekez-ye \\
%%LINE3
\EXP{}-\PRS{}-be  \AUG-2-{}people  2\SM-\N.\PST-{}come-\PFV.\REL{}  15-1\PL.\OM{}-greet   1\PL.\SM-\DJ-2\OM-{}accompany-\PFV{} \\
%%TRANS1
\glt
  ‘There are people who came to visit us. We accompany them.’\\
%%TRANS2
%%EXEND

\z

The NP can also be introduced by the inflected\footnote{Note that \textit{fit}- can only be inflected in a limited number of TAM conjugations.} verb \textit{fit-} ‘have’ \xref{bkm:Ref93471637} or a marker \textit{nga}, as in \xref{bkm:Ref106269434}. For further discussion on these variants we refer to \citet{Nshemezimana2016} as well as the discussion in \citet{Marten2013} for Swahili, and here concentrate on the construction with \textit{hari(ho)}.

%%EAX
\ea
%%JUDGEMENT
%%LABEL
\label{bkm:Ref93471637}
%%CONTEXT
(Ex.385, \citealt[260]{Nshemezimana2016})\\
%%LINE1
Urafíse inaanga yiivúza.\\
%%LINE2
\gll
u-ra-fit-ye  i-nanga  i-i-vúz-a\textsuperscript{H}\\
%%LINE3
2\SG.\SM-\PRS.\DJ-{}have-\PFV{}  \AUG{}-9-{}zither  9\SM-\REFL-{}play-\FV.\REL{}\\
%%TRANS1
\glt
‘You have a self-playing zither.’\\
%%TRANS2
  (\textit{IragiNdanga}, Traditional culture, 2000s)

%%EXEND

%%EAX
\ex
%%JUDGEMENT
%%LABEL
\label{bkm:Ref106269434}
%%CONTEXT
(Ex.396, \citealt[260]{Nshemezimana2016})\\
%%LINE1
Ngaabó abaáhaamvye umugabo waawe bari ku ruugi.\\
%%LINE2
\gll
nga-a-ba-o  a-ba-á-haamb-ye\textsuperscript{H}  u-mu-gabo  u-aawe   ba-ri  ku  ru-uugi \\
%%LINE3
\PRSNT-\AUG-2-\DEM{}\textsubscript{2}  \AUG-2-\RMT.\PST-{}bury-\PFV{}.\REL{}  \AUG{}-1-{}husband  1-\POSS.2\SG{}   2\SM-{}be  17  11-door \\
%%TRANS1
\glt
‘Here they are, those who buried your husband are at the door.’  (\textit{UbwuzuBushasha}, Religion, 1960s)\\
%%TRANS2
%%EXEND

\z

The subject marker \textit{ha}- is expletive in this construction, not referring to an actual location. When the clitic =\textit{ho} is added, the resulting interpretation is slightly different, as discussed below.

%%EAX
\ea
%%JUDGEMENT
%%LABEL
\label{bkm:Ref76658329}
%%CONTEXT
\label{bkm:Ref73366995}(Ex. 43, \citealt[41]{Nshemezimana2016})\\
%%LINE1
Harihó umusóre yiítooye aja kurésha umukoóbwa\\
%%LINE2
\gll
ha-$\varnothing$-ri=hó  u-mu-sóre  a-á-i-tóor-ye\textsuperscript{H}   a-gi-a  ku-résha  u-mu-koóbwa \\
%%LINE3
\EXP-\PRS-{}be=16  \AUG-1-young.man  1\SM-\RMT.\PST-\REFL-{}prepare-\PFV.\REL{}   1\SM-{}go-\FV{}  {}15-{}entice  \AUG{}-1-{}girl \\
%%TRANS1
\glt
  ‘There is a young man who went looking for a fiancée.’\\
%%TRANS2
%%EXEND

\z

While in \xref{bkm:Ref76658329} both \textit{ha}- and =\textit{ho} are expletive, referring – if to anything – to the current point in the discourse, they can also refer to an actual location. In this case, it is no longer a presentational construction but rather a case of Locative Inversion. For example, in \xref{bkm:Ref73366998}, \textit{ha}- and =\textit{ho} refer to \textit{háno mu rugó} ‘here in the courtyard’, indicating the exact place of the referent that forms the argument of the verb. The relative clause in this example only modifies the NP and could be omitted, in contrast to the presentational constructions, where the relative clause forms part of the construction.\largerpage[2]

%%EAX
\ea
%%JUDGEMENT
%%LABEL
\label{bkm:Ref73366998}
%%CONTEXT
%%LINE1
Háno mu rugo harihó imbwá, irikó iríinyeza.\\
%%LINE2
\gll
hano  mu  ru-gó  ha-$\varnothing$-ri=hó  i-\N-{}bwa  i-ri=kó\textsuperscript{H}   i-i-nyegez-a \\
%%LINE3
here  18  11-courtyard  16\SM-\PRS-{}be=16  \AUG-9-{}dog  9\SM-{}be=17.\REL{}   9\SM-\REFL-{}hide-\FV{} \\
%%TRANS1
\glt
  ‘Here in the courtyard is a dog, which is hiding.’\\
%%TRANS2
%%EXEND

\z

An argument could be made to analyse \textit{hari(ho)} as a unit, functioning as a ``presentational marker". If this were the case, the optionality of =\textit{ho} is unexplained, and the ambiguity with Locative Inversion just shown would be unexpected. Furthermore, it would be unclear why the verb is marked as relative. This is instead understood naturally if these are biclausal structures: they consist of a main clause presenting the referent by means of the predicator -\textit{ri}, and a relative clause. Another test would be to see if tense marking is added if the presentational is in the past tense.

What is special about the relative clause in this construction, though, is that it does not contain the typical backgrounded or presupposed information known from a cleft (see \sectref{bkm:Ref81550959}). Instead, the predicate in the relative clause forms part of the presented situation. The newly presented content is not merely the existence or presence of an entity (as in DAI), but a whole situation. The relative clause thus seems to add the same predication as in its non-presentative basic structure in \xref{bkm:Ref75262552} and \xref{bkm:Ref75262560}.

%%EAX
\ea
%%JUDGEMENT
%%LABEL
\label{bkm:Ref75262552}
%%CONTEXT
(basic structure corresponding to \xref{bkm:Ref74054848})\\
%%LINE1
Abantu baaje kuturamutsa.\\
%%LINE2
\gll
a-ba-ntu  ba-a-a-əz-ye  ku-tu-ramutsa\\
%%LINE3
\AUG-2-{}people  2\SM-\N.\PST-\DJ-{}come-\PFV{}  15-1\PL.\OM{}-greet\\
%%TRANS1
\glt
‘People came to visit us.’\\
%%TRANS2
%%EXEND

\z

%%EAX
\ea
%%JUDGEMENT
%%LABEL
\label{bkm:Ref75262560}
%%CONTEXT
(basic structure corresponding to \xref{bkm:Ref73366995})\\
%%LINE1
Umusóre yariítooye aja kurésha umukoóbwa.\\
%%LINE2
\gll
u-mu-sóre  a-á-ra-i-tóor-ye  a-gi-a  ku-résha   u-mu-koóbwa \\
%%LINE3
\AUG-1-{}young.man  1\SM-\RMT.\PST-\DJ-\REFL{}-prepare-\PFV{}  {}1\SM{}-{}go-\FV{}  {}15-{}entice   \AUG-1-{}girl \\
%%TRANS1
\glt
  ‘A young man went looking for a fiancée.’\\
%%TRANS2
%%EXEND

\z

This is also where the construction in \xref{bkm:Ref73366998} differs, as the relative clause here merely provides background information and the main point of information is the existence/presence of the dog in the courtyard.

  The context in which these presentationals are used also indicates that all the information is provided as new. Therefore, these are thetic sentences, used to introduce a referent and at the same time add information about that referent. Crucially, the referent to which the NP in this construction refers is presented as non-topical, but cannot be in focus. This is evident in the unacceptibility of interrogatives in the \textit{hari} construction \xref{bkm:Ref74740077} and its incompatibility with \textit{gusa} ‘only’ \xref{bkm:Ref75426224}, for example.

%%EAX
\ea
%%JUDGEMENT
[*]{
%%LABEL
\label{bkm:Ref74740077}
%%CONTEXT
%%LINE1
Hari iki muriko murarondera?\\
%%LINE2
\gll
ha-ri  iki  mu-ronder-a\textsuperscript{H}\\
%%LINE3
\EXP{}-be  what  2\PL.\SM{}-look.for-\FV.\REL{}\\
%%TRANS1
\glt
‘What are you looking for?’\\
%%TRANS2
}
%%EXEND

\z

%%EAX
\ea
%%JUDGEMENT
%%LABEL
\label{bkm:Ref75426224}
%%CONTEXT
%%LINE1
Harihó umusóre (*gusa) yiítooye aja kurésha umukoóbwa.\\
%%LINE2
\gll
ha-$\varnothing$-ri=hó  u-mu-sóre  (*gusa)  a-á-i-tóor-ye\textsuperscript{H}   a-gi-a  ku-résha  u-mu-koóbwa \\
%%LINE3
\EXP-\PRS-{}be=16  \AUG-1-young.man  (*only)  1\SM-\RMT.\PST-\REFL-{}prepare-\PFV.\REL{}   1\SM-{}go-\FV{}  {}15-{}entice \AUG{}-1-{}girl \\
%%TRANS1
\glt
  ‘There is (*only) a young man who went looking for a fiancée.’\\
%%TRANS2
%%EXEND

\z

The contexts for \xref{bkm:Ref75427022} also indicate that the whole sentence is presented as one piece of new information (\citeauthor{Lambrecht1994}'s \citeyear{Lambrecht1994} ``sentence focus"), as the construction with \textit{hari} cannot be used in a corrective context.

%%EAX
\ea
%%JUDGEMENT
%%LABEL
\label{bkm:Ref75427022}
%%CONTEXT
(Context 1: What happened, why are you here?\\
\textsuperscript{\#}Context 2: Are you looking for the keys?)\\
%%LINE1
Hari amaherá yataakáye turíko turarondera.\\
%%LINE2
\gll
ha-ri  a-ma-herá  a-a-táakar-ye\textsuperscript{H}  tu-ri=kó\textsuperscript{H}   tu-ra-ronder-a \\
%%LINE3
\EXP{}-be  \AUG-6{}-money  6\SM-\N.\PST{}-loose-\PFV.\REL{}  1\PL.\SM{}-be=17.\REL{}   1\PL.\SM-\PRS.\DJ{}-look.for-\FV{} \\
%%TRANS1
\glt
  ‘There is some lost money that we are looking for.’\\
%%TRANS2
%%EXEND

\z

Because of its presentational function, there is a definiteness effect: the presented referent should be newly introduced, and hence proper names of familiar people are not accepted.

%%EAX
\ea
%%JUDGEMENT
[*]{
%%LABEL
%%CONTEXT
(adapted from \xref{bkm:Ref74054848})\\
%%LINE1
%%LINE2
\gll
Ha-ri  Petero  a-a-əz-ye\textsuperscript{H}  ku-tu-ramutsa\\
%%LINE3
\EXP{}-be  1.Peter  1\SM-\N.\PST-{}come-\PFV.\REL{}  15-1\PL.\OM{}-greet\\
%%TRANS1
\glt
‘There is Peter who came to visit us.’\\
%%TRANS2
}
%%EXEND

\z

%%EAX
\ea
%%JUDGEMENT
[*]{
%%LABEL
%%CONTEXT
(adapted from \xref{bkm:Ref73366995})\\
%%LINE1
%%LINE2
\gll
Ha-ri=hó  Petero  a-á-i-tóor-ye\textsuperscript{H}   a-gi-a  ku-résha  u-mu-koóbwa \\
%%LINE3
\EXP-{}be=16  1.Peter  1\SM-\RMT.\PST-\REFL-{}prepare-\PFV.\REL{}   1\SM-{}go-\FV{}  {}15-{}entice  \AUG{}-1-{}girl \\
%%TRANS1
\glt
  ‘There is Peter who went looking for a fiancée.’\\
%%TRANS2
}
%%EXEND

\z

We conclude that the presentational construction in Kirundi is used for thetic sentences, and consists of an indefinite predicate marked by \textit{hari}(=\textit{ho}) and a relative clause. It therefore formally fits with other types of clefts, as discussed in the next subsections.

\subsection{Basic clefts}
\label{bkm:Ref81550959}
Clefts are another copular construction in Kirundi, consisting of a non-verbal predicate (copula + NP) and a relative clause, as is familiar from languages all around the world. Kirundi being a pro-drop language, it does not feature an expletive in the pre-copular position; hence we follow \citet{Nshemezimana2016} and \citet{LafkiouiEtAl2016} and call it a ``basic cleft" instead of an ``it-cleft". The basic cleft is illustrated in \xref{bkm:Ref76915025}. The relative clause is marked by a high tone, which is variable in its surface position, but indicated in the gloss as part of the final vowel.

\ea
\label{bkm:Ref76915025}
(Context: B and friends have come to A’s place)
\begin{xlist}
\exi{A:}  Who are you looking for?\\
%%EAX
\exi{B:}
%%JUDGEMENT
%%LABEL
%%CONTEXT
%%LINE1
Ni wewé turondéra.\\
%%LINE2
\gll
ni  wewé  tu-ronder-a\textsuperscript{H}\\
%%LINE3
\COP{}  2\SG.\PRO{}  1\PL.\SM{}-look.for-\FV.\REL{}\\
%%TRANS1
\glt
‘It’s you we’re looking for.’\\
%%TRANS2
%%EXEND

\end{xlist}
\z

For further formal description of basic clefts in Kirundi we refer to \citet{LafkiouiEtAl2016}, and here we concentrate on the interpretation. Basic clefts typically express focus on the clefted constituent (the predicative noun following the copula), with the relative clause providing the presupposed, given information, and the clefted constituent being asserted as the referent to which this information applies. This can be seen in the question-answer pair in \xref{bkm:Ref76915336}. Note that the basic cleft can be used for subjects and objects alike.

\ea
\label{bkm:Ref76915336}
\begin{xlist}
%%EAX
\exi{A:}
%%JUDGEMENT
%%LABEL
%%CONTEXT
%%LINE1
Ni nde yavunye iyo ntébe?\\
%%LINE2
\gll
ni  nde  a-a-vun-ye\textsuperscript{H}  i-i-o  n-tébe\\
%%LINE3
\COP{}  who  1\SM-\N.\PST{}-break-\PFV.\REL{}  \AUG-9-\DEM{}\textsubscript{2}  9-chair\\
%%TRANS1
\glt
‘Who broke that chair?’\\
%%TRANS2
%%EXEND

%%EAX
\exi{B:}
%%JUDGEMENT
%%LABEL
%%CONTEXT
%%LINE1
Ni Kabura yayivunyé.\\
%%LINE2
\gll
ni  Kabura  a-a-yi-vún-ye\textsuperscript{H}\\
%%LINE3
\COP{}  1.Kabura  1\SM-\N.\PST-9\OM{}-break-\PFV.\REL{}\\
%%TRANS1
\glt
‘It was Kabura who broke it.’\\
%%TRANS2
%%EXEND

\end{xlist}
\z

Unlike the typical exhaustive interpretation that is found in it-clefts in English and other languages \citep[e.g.][]{Horn1981,É.Kiss1998,Declerck1988,Hedberg2000,BeaverClark2008}, the type of focus expressed by the basic cleft in Kirundi seems to be underspecified, as it occurs in various contexts. We present some of these. Apart from simple focus as in Q-A pairs, as in \xref{bkm:Ref142572186}, clefts can be used in ``mention some" contexts \xref{bkm:Ref79134504}, which are necessarily non-exhaustive, but they function equally well in a corrective context \xref{bkm:Ref79134487}, and are compatible with exhaustive \textit{gusa} ‘only’ \xref{bkm:Ref79134496}. 

%%EAX
\ea
%%JUDGEMENT
%%LABEL
\label{bkm:Ref142572186}
%%CONTEXT
(Who is sitting under the tree?)\\
%%LINE1
Ni Kabura yiicáye muusi y’ígití.\\
%%LINE2
\gll
ni  Kabura  a-iicar-ye\textsuperscript{H}  musi  y-a  i-ki-ti\\
%%LINE3
\COP{}  1.Kabura  1\SM{}-sit-\PFV.\REL{}  18.under  19-\CONN{}  \AUG{}-7-tree\\
%%TRANS1
\glt
‘It is Kabura who is sitting under the tree.’\\
%%TRANS2
%%EXEND

\z

%%EAX
\ea
%%JUDGEMENT
%%LABEL
\label{bkm:Ref79134504}
%%CONTEXT
(What sort of milk can I drink?)\\
%%LINE1
Ni ay’íinká woonywa, nk’ akarorero.\\
%%LINE2
\gll
ni  a-a  i-n-ká  u-oo-nyo-a\textsuperscript{H}  nk’  a-ka-rorero\\
%%LINE3
\COP{}  6-\CONN{}  \AUG{}-9-cow  2\SG.\SM-\POT-{}drink-\FV.\REL{}  for  \AUG{}-12-example\\
%%TRANS1
\glt
‘You can drink cow’s milk, for example.’\\
%%TRANS2
%%EXEND

\z

%%EAX
\ea
%%JUDGEMENT
%%LABEL
\label{bkm:Ref79134487}
%%CONTEXT
(Did you drink beer?)\\
%%LINE1
Oya ni ifanta nanyóonye (si ikiyeri).\\
%%LINE2
\gll
oya  ni  ifanta  \N{}-a-nyó-ye\textsuperscript{H}  (si  i-ki-yeri)\\
%%LINE3
no  \COP{}  9.fanta  1\SG.\SM-\N.\PST-{}drink-\PFV.\REL{}  (not  \AUG{}-7-beer)\\
%%TRANS1
\glt
‘No, it’s a fanta that I drank (not a beer).’\\
%%TRANS2
%%EXEND

\z

%%EAX
\ea
%%JUDGEMENT
%%LABEL
\label{bkm:Ref79134496}
%%CONTEXT
%%LINE1
Ni umuceri (gusa) nariyé (s’umuceri n’inyama).\\
%%LINE2
\gll
ni  u-mu-ceri  (gusa)  \N{}-a-ri-ye\textsuperscript{H}  (si  u-mu-ceri  n’  i-nyama)\\
%%LINE3
\COP{}  \AUG{}-3-rice  only  1\SG.\SM-\N.\PST{}-eat-\PFV.\REL{}  (not  \AUG-{}3-rice  and  10-meat)\\
%%TRANS1
\glt
‘It’s only the rice that I ate (not rice and meat).’\\
%%TRANS2
%%EXEND

\z

It seems that basic clefts in Kirundi can even be used in thetic contexts, where no information is presupposed. The relative clause here functions similarly to its use in the presentationals (see further \sectref{bkm:Ref75348623}). \citet[98]{LafkiouiEtAl2016} also note this use, and question whether these constructions are actual basic clefts. Silvio Cruschina (p.c.) suggests that: ``They could actually be specificational copular sentences or pseudoclefts with an omitted but implicit initial constituent related to the context and a relative clause attached to the postcopular nominal phrase: e.g. ‘The driver believed that [what caused the noise] was a tree branch that hit the car’, or ‘[The unusual noise that we heard] is a child who had fallen down’.''


%%EAX
\ea
%%JUDGEMENT
%%LABEL
%%CONTEXT
(Context: A dog jumps into the passing truck without the driver’s knowledge. The driver hears some kind of unusual noise but does not stop. A person who saw this recounts what happened with the driver, indicating what the driver believed the noise heard to be.)\\
%%LINE1
Umushoferi yagira ngo ni ishami ry’igiti rikubise ku muduga.\\
%%LINE2
\gll
u-mu-shoferi  a-a-gir-a  ngo  ni  i-shami  ri-a   i-ki-ti  ri-kubit-ye\textsuperscript{H}  ku  mu-duga \\
%%LINE3
\AUG-1-{}driver  1\SM-\N.\PST{}-believe-\FV{}  \QUOT{}  \COP{}  \AUG{}-branch  5-\CONN{}   \AUG{}-7-tree  5\SM-{}hit-\PFV.\REL{}  17  3-car \\
%%TRANS1
\glt
  ‘The driver believed that it was a tree branch hitting the car.’\\
%%TRANS2
%%EXEND

\z

\ea
\begin{xlist}
\exi{A:}  What happened? (after hearing an unusual noise)\\
%%EAX
\exi{B:}
%%JUDGEMENT
%%LABEL
%%CONTEXT
%%LINE1
Ni umwáana yiitúuye haasí.\\
%%LINE2
\gll
ni   u-mu-áana  a-i-túur-ye\textsuperscript{H}  haasí\\
%%LINE3
\COP{}  \AUG-1{}-child  1\SM-\REFL{}-fall-\PFV.\REL{}  down\\
%%TRANS1
\glt
‘It is a child who falls down.’\\
%%TRANS2
%%EXEND

\end{xlist}
\z

An in-between case is presented in \xref{bkm:Ref75343434}, where the clefted constituent forms the answer to the immediate question, but the addition of the relative clause (which does not contain presupposed information) seems to answer a hidden question ‘What was going on?’ or ‘Why were you talking to this person?’, thus resulting in a larger constituent being in focus.

\ea
\label{bkm:Ref75343434}
\begin{xlist}
\exi{A:}  Who were you talking to outside?
%%EAX
\exi{B:}
%%JUDGEMENT
%%LABEL
%%CONTEXT
Yari Jeanne yaríkó araánsiguurira ibiháruuro.\\
%%LINE1
%%LINE2
\gll
a-a-ri  Jeanne  a-a-ri=ko\textsuperscript{H}  a-ra-n-siguur-ir-a i-bi-háruuro \\
%%LINE3
1\SM-\N.\PST{}-be  1.Jeanne  1\SM-\N.\PST{}-be=17.\REL{}  1\SM-\DJ-1\SG.\OM{}-explain-\FV{}   \AUG{}-8-mathematic \\
%%TRANS1
\glt
  ‘It was Jeanne who explained mathematics to me.’\\
%%TRANS2
%%EXEND

\end{xlist}
\z

Interestingly, the clefted constituent can be modified by \textit{na} ‘and, also, even’ (in line with the wider non-exclusive use of the basic cleft), but the addition of the scalar additive particle \textit{ndetse} ‘even’ renders the construction ungrammatical. At the moment we do not have a satisfying explanation.

%%EAX
\ea
%%JUDGEMENT
%%LABEL
%%CONTEXT
%%LINE1
Ni (*ndetse) n’ ifanta nanyóoye.\\
%%LINE2
\gll
ni  ndetse  n’  i-fanta  \N{}-a-nyó-ye\textsuperscript{H}\\
%%LINE3
\COP{}  even  and  \AUG{}-9-fanta  1\SG.\SM-\N.\PST{}-drink-\PFV.\REL{}\\
%%TRANS1
\glt
‘It is even a fanta that I drank.’\\
%%TRANS2
%%EXEND

\z

The overall underspecified focus interpretation of the basic cleft means that its precise interpretation is highly context-sensitive, as also concluded by \citet{LafkiouiEtAl2016}.

\subsection{Pseudoclefts}

Pseudoclefts are complex structures consisting of two parts \citep{DenDikken2017,Apothéloz2012,Roubaud2000}. The first part is a free relative clause, and the second a predicate noun, in the following example marked by the copula \textit{ni}.


%%EAX
\ea
%%JUDGEMENT
%%LABEL
%%CONTEXT
%%LINE1
Abaádutaahanye ni abahuúngu baácu.\\
%%LINE2
\gll
a-ba-á-tu-taahan-ye\textsuperscript{H}  ni  a-ba-huúngu  ba-áacu\\
%%LINE3
\AUG{}-2-\RMT.\PST-1\PL.\OM{}-drive-\PFV.\REL{}  \COP{}  \AUG{}-2-son  2-\POSS.1\PL{}\\
%%TRANS1
\glt
‘Those who drove us home are our sons.’\\
%%TRANS2
%%EXEND

\z


The relative clause here functions as an NP. When the antecedent of the relative clause is a subject, we see a headless relative (``relative autonome" in \citealt[133]{Meeussen1959}) – a nominalisation marked on the verb by an augment and noun class marker corresponding to the noun class of the antecedent, as in \xref{bkm:Ref78794744}. Non\hyp subject relatives, on the other hand, are marked by a pronoun (``pronom précessif" in \citealt{Meeussen1959}), as illustrated in \xref{bkm:Ref75856897} where the pronoun \textit{icó} precedes the relative verb (which is marked by a high tone).


%%EAX
\ea
%%JUDGEMENT
%%LABEL
\label{bkm:Ref78794744}
%%CONTEXT
%%LINE1
Icaádukijije ni ubuntu bwa Yeésu.\\
%%LINE2
\gll
i-ki-á-tu-kiz-ye\textsuperscript{H}  ni  u-bu-ntu  bu-a  Yeésu\\
%%LINE3
\AUG{}-7-\RMT.\PST{}-1\PL{}.\OM{}-save-\PFV.\REL{}  \COP{}  \AUG{}-14-grace  14-\CONN{}  Jesus\\
%%TRANS1
\glt
‘What saved us is the grace of Jesus.’\\
%%TRANS2
%%EXEND

\z

%%EAX
\ea
%%JUDGEMENT
%%LABEL
\label{bkm:Ref75856897}
%%CONTEXT
%%LINE1
Icó twuúmviise ni ico tukubwiíye.\\
%%LINE2
\gll
  i-ki-ó  tu-á-úumv-ye\textsuperscript{H}  ni  i-ki-o tu-$\varnothing$-ku-bwiír-ye\textsuperscript{H} \\
%%LINE3
  \AUG{}-7-\PRO{}  1\PL.\SM-\RMT.\PST{}-hear-\PFV.\REL{}  \COP{}  \AUG{}-7-\DEM{}\textsubscript{2} 1\PL.\SM-\PRS-2\SG.\OM{}-tell-\PFV.\REL{}\\
%%TRANS1
\glt
  ‘What we heard is what we tell you.’\\
%%TRANS2
%%EXEND

\z


The two parts of the pseudocleft have a distinctive intonation, where the first part has a rising intonational contour indicating a continuation, and the second ending with a concluding intonation. This differs from the intonation of basic clefts, which form one intonational phrase. The segmentation is indicated by the brackets in \xref{bkm:Ref75764861}. See again \citet{LafkiouiEtAl2016} for further discussion.


%%EAX
\ea
%%JUDGEMENT
%%LABEL
\label{bkm:Ref75764861}
%%CONTEXT
%%LINE1
(Icó twiipfúuza) (ni ukuguma ku Mana yáacu).\\
%%LINE2
\gll
i-ki-ó  tu-íipfuuz-a\textsuperscript{H}  ni  u-ku-guma  ku  Mana  i-áacu\\
%%LINE3
\AUG{}-7-\PRO{}  1\PL.\SM{}-wish-\FV.\REL{}  \COP{}  \AUG{}-15-stay  17  9.God  9-\POSS.1\PL{}\\
%%TRANS1
\glt
‘What we wish is to stay with our God.’\\
%%TRANS2
%%EXEND

\z

Formally, pseudoclefts are copular clauses of which the first part is a free relative. The term ``pseudocleft", however, is only used for specificational copular clauses \xref{bkm:Ref75772945} and not predicational ones, even though it is perfectly possible to create a predicational copular clause of which the first term is a free relative, as in \xref{bkm:Ref75772952}. The difference can be seen in the fact that the identificational copular clause or pseudocleft in \xref{bkm:Ref75772945}, but not the predicational copular clause in \xref{bkm:Ref75772952}, can be used to answer ‘What did you bring?’ (see \citealt{DenDikken2013,DenDikken2017} on predicational and specificational cleft sentences).

%%EAX
\ea
%%JUDGEMENT
%%LABEL
\label{bkm:Ref75772945}
%%CONTEXT
%%LINE1
Icó twaazanyé ni amaherá y’íshuúre.  \jambox*{[specificational]}
%%LINE2
\gll
i-ki-ó  tu-a-zan-ye\textsuperscript{H}  ni  a-ma-herá  a-a  i-shuúre\\
%%LINE3
\AUG{}-7-\PRO{}  1\PL.\SM-\N.\PST{}-bring-\PFV.\REL{}  \COP{}  \AUG{}-6-maney  6-\CONN{}  \AUG{}-5-school\\
%%TRANS1
\glt
‘What we brought is the school fees.’\\
%%TRANS2
%%EXEND

%%EAX
\ex
%%JUDGEMENT
%%LABEL
\label{bkm:Ref75772952}
%%CONTEXT
%%LINE1
Icó twaazanyé ni ciza.  \jambox*{[predicational]}
%%LINE2
\gll
i-ki-ó  tu-a-zan-ye\textsuperscript{H}  ni  ki-iizá\\
%%LINE3
\AUG{}-7-\PRO{}  1\PL.\SM-\N.\PST{}-bring-\PFV.\REL{}  \COP{}  7-good\\
%%TRANS1
\glt
‘What we brought is good.’\\
%%TRANS2
%%EXEND

\z

From an interpretational point of view, the free relative forms the presupposition (hence needs to be referential) and the predicative noun is the assertion. Together, they result in identificational focus. This is illustrated in \xref{bkm:Ref75860417}, where the free relative establishes that there exists something that threw him into disarray, and the postcopular noun phrase identifies this as the death of his child.
\pagebreak

%%EAX
\ea
%%JUDGEMENT
%%LABEL
\label{bkm:Ref75860417}
%%CONTEXT
%%LINE1
Icaámuteeye agahiinda ni urupfu rw’umwáana wíiwé.\\
%%LINE2
\gll
[i-ki-á-mu-téer-ye  a-ka-hiinda]\textsuperscript{\TOP}  ni  [u-ru-pfu  ru-a   u-ma-áana  u-íiwé]\textsuperscript{\FOC} \\
%%LINE3
{\db}\AUG{}-7-\RMT.\PST-1\OM{}-cause-\PFV.\REL{}  \AUG-{}12-disarray  \COP{}  {\db}\AUG{}-11-death  11-\CONN{}   \AUG-{}1-child  1-\POSS.1{} \\
%%TRANS1
\glt
  ‘What threw him into disarray was the death of his child.’\\
%%TRANS2
%%EXEND

\z

As the first part of the pseudocleft (the free relative) relates to the previous discourse, the pseudocleft can function as a ``bridge" between the old and new information \citep[244]{Nshemezimana2016}. This is illustrated in \xref{bkm:Ref106310149}. Since the free relative thus functions as a topic, it can also be marked by the contrastive topic marker -\textit{o}, as shown in \xref{bkm:Ref81816342} above.

%%EAX
\ea
%%JUDGEMENT
%%LABEL
\label{bkm:Ref106310149}
%%CONTEXT
(\citealt[244]{Nshemezimana2016}, \citealt[90]{LafkiouiEtAl2016})\\
(Context: Two people are discussing about the hunting profession that was practised in the past but which is no longer allowed for reasons of environmental protection. Speaking of those who still support this practice, one tells it in these terms:) \\
%%LINE1
Abashígikiye uwo mucó ni abahiígi bakiriho.\\
%%LINE2
\gll
a-ba-shígikir-ye\textsuperscript{H} u-u-o  mu-có  ni  a-ba-hiígi   ba-ki-ri=ho\textsuperscript{H} \\
%%LINE3
\AUG-2{}-support-\PFV.\REL{}  \AUG-3-\DEM{}\textsubscript{2}  3-practice  \COP{}  \AUG-2{}-hunter   2\SM-\PERS{}-be=16.\REL{} \\
%%TRANS1
\glt
‘Those who support this practice are the hunters who are still alive.’\\ 
(\textit{IragiNdanga,} Culture traditionnelle, 2000s)\\
%%TRANS2
%%EXEND

\z

In answers to content questions, pseudoclefts first repeat the given information from the question and then add the focused answer, as illustrated in \xref{bkm:Ref119734746}.

\ea
\label{bkm:Ref119734746}
(\citealt[245]{Nshemezimana2016}, \citealt[93--94]{LafkiouiEtAl2016})\\
%%EAX
\ea
%%JUDGEMENT
%%LABEL
%%CONTEXT
%%LINE1
Abaróongoora Ekleziyá ni baandé? \\
%%LINE2
\gll
a-ba-roongoor-a\textsuperscript{H}\textsuperscript{} ekleziyá  ni  ba-ndé?\\
%%LINE3
\AUG-2{}-lead-\FV.\REL{}  church  \COP{}  2-who\\
%%TRANS1
\glt
‘Who leads the (Catholic) church?’\\
%%TRANS2
%%EXEND

%%EAX
\ex
%%JUDGEMENT
%%LABEL
%%CONTEXT
%%LINE1
Abaróongoora Ekleziyá, ni Paapa n’ábeépiskoópi.\\
%%LINE2
\gll
a-ba-róongoor-a\textsuperscript{H} ekleziyá  ni  paapa  na  a-ba-épiskoópi.\\
%%LINE3
\AUG-2{}-lead-\FV.\REL{}  church  \COP{}  1.pope  and  \AUG-2{}-bishop\\
%%TRANS1
\glt
‘Those who lead the (Catholic) Church are the pope and the bishops.’\\
%%TRANS2
  (\textit{Yaga}, Religion, 1960s)

%%EXEND

\z
\z

Pseudoclefts can also be used in a contrastive context, as illustrated in \xref{bkm:Ref119734770}.

%%EAX
\ea
%%JUDGEMENT
%%LABEL
\label{bkm:Ref119734770}
%%CONTEXT
(\citealt[246]{Nshemezimana2016}, \citealt[91]{LafkiouiEtAl2016})\\
%%LINE1
Ivyó dutuunzé si ivyaácu ni ivyáawe.\\
%%LINE2
\gll
i-bi-ó  tu-túung-ye\textsuperscript{H} si  i-bi-aácu ni  i-bi-áawe \\
%%LINE3
\AUG-8-\PRCS{}  1\PL.\SM{}-have-\PFV.\REL{}  \COP{}  \AUG-8-\POSS.1\PL{} \COP{}  \AUG-8-\POSS.2\SG{} \\
%%TRANS1
\glt
‘What we have is not ours, it is yours.’\\
(\textit{Karaba}, Theatre, 1960)\\
%%TRANS2
%%EXEND

\z

The fact that the pseudocleft can be used in ``mention some" contexts, as in \xref{bkm:Ref79138286}, suggests that the focus is not exclusive or exhaustive. The same conclusion is reached for the correction of an incomplete statement, as in \xref{bkm:Ref79138377}: if the statement were interpreted as exhaustive, the corrective reply should have started with ‘no’ (negating the exhaustivity), so the fact that a natural reply starts with ‘yes’ indicates a non-exhaustive focus.

%%EAX
\ea
%%JUDGEMENT
%%LABEL
\label{bkm:Ref79138286}
%%CONTEXT
(Where can I buy a book?)\\
%%LINE1
Aho wokigurira ni harya, nk’akarorero.\\
%%LINE2
\gll
a-ha-ó  u-oo-ki-gur-ir-a\textsuperscript{H}  ni  ha-rya  nka  a-ka-rorero\\
%%LINE3
\AUG-16-\PRO{}  2\SG.\SM-\POT-7\OM{}-buy-\APPL-\FV.\REL{}  \COP{}  16-\DEM{}\textsubscript{3}  for  \AUG-{}12-example\\
%%TRANS1
\glt
‘Where you can buy it is there, for example.’\\
%%TRANS2
%%EXEND

\z

\ea
%%EAX
\ea
%%JUDGEMENT
%%LABEL
\label{bkm:Ref79138377}
%%CONTEXT
%%LINE1
Icó bakenéye ni ugushika urugó.\\
%%LINE2
\gll
i-ki-ó  ba-kener-ye\textsuperscript{H}  ni  u-ku-shika  u-ru-gó\\
%%LINE3
\AUG-7-\PRO{}  2\SM{}-need-\PFV.\REL{}  \COP{}  \AUG-15{}-get.used.to  \AUG{}-11-home\\
%%TRANS1
\glt
‘What they need is to get used to the home.’\\
%%TRANS2
%%EXEND

%%EAX
\ex
%%JUDGEMENT
%%LABEL
%\label{ex:kirundi:56}
%%CONTEXT
%%LINE1
Ego ariko barakenéye n’ugufashwa.\\
%%LINE2
\gll
ego  ariko  ba-ra-kener-ye  na  u-ku-fásh-w-a\\
%%LINE3
yes  but  2\SM-\DJ-{}need-\PFV{}  also  \AUG{}-15-help-\PASS-\FV{}\\
%%TRANS1
\glt
‘Yes, but they also need to be helped.’\\
%%TRANS2
%%EXEND

\z
\z

Example \xref{bkm:Ref142585865} shows how the cleft and pseudocleft can be used in comparable contexts, here contrasting two subjects; ‘the man’ appears in a pseudocleft and ‘the woman’ in a basic cleft.

%%EAX
\ea
%%JUDGEMENT
%%LABEL
\label{bkm:Ref142585865}
%%CONTEXT
(Who throws the stone and who throws the bottle? + photo of a man throwing a bottle and a girl throwing a stone.)\\
%%LINE1
Icupa uwuriteeye ni umugabo, ariko ibuye ni umukoobwa ariteeye.\\
%%LINE2
\gll
i-cupa  u-wu-ri-teer-ye\textsuperscript{H}  ni  u-mu-gabo  ariko  i-buye  ni  u-mu-koobwa  a-ri-teer-ye\textsuperscript{H}   \\
%%LINE3
\AUG{}-5.bottle  \AUG{}-1-5\OM{}-throw-\PFV.\REL{}  \COP{}  \AUG{}-1-man  but \AUG{}-5.stone  \COP{}  \AUG{}-1-girl  1\SM-5\OM{}-throw-\PFV.\REL{} \\
%%TRANS1
\glt
‘The bottle, the one who throws it is the man, but the stone, it is the girl who throws it.’\\
%%TRANS2
%%EXEND

\z

In summary, pseudoclefts in Kirundi (as in other languages) are used to identify a referent, but they can be used in a range of contexts.

\subsection{Reverse pseudocleft / left-peripheral NP + cleft}

In the pseudoclefts discussed above, the precopular part is a free relative and the postcopular noun identifies the referent described by the free relative. These two parts can also be reversed: If the precopular constituent is a referential noun and the copula is followed by a relative, the construction is called a ``reverse pseudocleft", e.g. ‘[Unlimited internet access] is [what I want]’. Kirundi can form two such constructions, which were not yet distinguished by \citet{Nshemezimana2016} or \citet{LafkiouiEtAl2016}. They indicate that the initial constituent plays a double pragmatic role as both topic and focus. Here, we show that there are two underlying structures, and we show that the apparent double role as topic and focus may actually be split between the initial noun phrase and the clefted pronoun.

The first form of reverse pseudocleft is illustrated in \xref{bkm:Ref118966697}. The postcopular part starts with a demonstrative, which either functions as a relative clause marker when it has a high tone as in \xref{bkm:Ref118966697:b}, or as a deictic demonstrative without the high tone, as in \xref{bkm:Ref118966697:a}. Note that in this reverse order, the precopular part is still the topic, and the free relative is the comment/focus, as indicated in the contexts.

\ea
\label{bkm:Ref118966697}
Context 1: We talked about your travels and you told me you’re travelling with someone. A bit later in the conversation, you mention Jean. I ask ‘Which Jean are you talking about?’ (focus on ‘the one we will travel with’)\\
\textsuperscript{\#}Context 2: You tell me you will travel with someone. I ask ‘Who will you travel with?’ (focus on ‘Jean’)

%\label{ex:kirundi:192}
%%EAX
\ea
%%JUDGEMENT
%%LABEL
\label{bkm:Ref118966697:a}
%%CONTEXT
%%LINE1
Jean ni uwo tuzóojáana.\\
%%LINE2
\gll
Jean  ni  u-u-o  tu-zoo-gi-an-a\textsuperscript{H}\\
%%LINE3
1.Jean  \COP{}  \AUG-1-\DEM{}\textsubscript{2}  1\PL.\SM-\FUT{}-go-\ASS{}-\FV.\REL{}\\
%%TRANS1
\glt
‘Jean is \textbf{that} one (that we just talked about) that we will travel with.’\\
%%TRANS2
%%EXEND

%%EAX
\ex
%%JUDGEMENT
%%LABEL
\label{bkm:Ref118966697:b}
%%CONTEXT
%%LINE1
Jean ni uwó tuzóojáana.\\
%%LINE2
\gll
Jean  ni  u-u-o  tu-zoo-gi-an-a\textsuperscript{H}\\
%%LINE3
1.Jean  \COP{}  \AUG-1-\DEM{}\textsubscript{2}  1\PL.\SM-\FUT{}-go-\ASS{}-\FV.\REL{}\\
%%TRANS1
\glt
‘Jean is \textbf{the} one we will travel with.’\\
%%TRANS2
%%EXEND

\z
\z

The other form of reverse pseudocleft also has an initial noun phrase, a copula, and a relative clause, but here we find no demonstrative but a personal pronoun as the clefted constituent. The verb is still marked relative, as in the basic cleft. In \xref{bkm:Ref78803758}, the NP \textit{abáana} ‘children’ and the pronoun \textit{bó} both refer to the same referent, namely the children.

%%EAX
\ea
%%JUDGEMENT
%%LABEL
\label{bkm:Ref78803758}
%%CONTEXT
%%LINE1
Abáana bakirí bató ni bó turungíka kw’ishuúre.\\
%%LINE2
\gll
a-ba-áana  ba-ki-ri\textsuperscript{H}  ba-tó  ni  ba-ó  tu-rungík-a\textsuperscript{H} ku  i-shuúre\\
%%LINE3
\AUG-2-{}child  2\SM-\PERS{}-be.\REL{}  2-young  \COP{}  2-\PRO{}  1\PL.\SM{}-send-\FV.\REL{}  17  \AUG{}-5.school\\
%%TRANS1
\glt
‘Young children (they) are the ones we send to school.’\\
%%TRANS2
‘Young children, it’s them we send to school.’
%%EXEND

\z

The initial NP seems to be in focus here, as seen in the contexts for \xref{bkm:Ref118967248}, and given the fact that it can form the answer to a content question, as in \xref{bkm:Ref78803230:b}, as alternative to the basic cleft.

%%EAX
\ea
%%JUDGEMENT
%%LABEL
\label{bkm:Ref118967248}
%%CONTEXT
\textsuperscript{\#}Context 1: We talked about your travels and you told me you’re travelling with someone. A bit later in the conversation, you mention Jean. I ask ‘Who is Jean?’ (focus on ‘the one we will travel with’)\\
Context 2: You tell me you will travel with someone. I ask ‘Who will you travel with?’ (focus on ‘Jean’)\\
%%LINE1
Jean ni wé tuzóojáana.\\
%%LINE2
\gll
Jean  ni  wé  tu-zoo-gi-an-a\textsuperscript{H}\\
%%LINE3
1.Jean  \COP{}  1.\PRO{}  1\PL.\SM-\FUT{}-go-\ASS{}-\FV.\REL{}\\
%%TRANS1
\glt
‘Jean is who we will travel with.’\\
%%TRANS2
%%EXEND

\z
\pagebreak

\ea
\label{bkm:Ref78803230}(Who ate my food ?)
%\label{ex:kirundi:56}
%%EAX
\ea
%%JUDGEMENT
%%LABEL
\label{bkm:Ref78803230:a}
%%CONTEXT
%%LINE1
Ni Kabura yabiriye.  \jambox*{[basic cleft]}
%%LINE2
\gll
ni  Kabura  a-a-bi-ri-ye\textsuperscript{H}\\
%%LINE3
\COP{}  Kabura  1\SM{}-\N.\PST-8\OM{}-eat-\PFV.\REL{}\\
%%TRANS1
\glt
‘It is Kabura who ate it.’\\
%%TRANS2
%%EXEND

%%EAX
\ex
%%JUDGEMENT
%%LABEL
\label{bkm:Ref78803230:b}
%%CONTEXT
%%LINE1
Kabura, ni wé yabiriye.  \jambox*{[rev. pseudocleft]}
%%LINE2
\gll
Kabura  ni  wé  a-a-bi-ri-ye\textsuperscript{H}\\
%%LINE3
1.Kabura  \COP{}  1.\PRO{}  1\SM-\N.\PST-8\OM{}-eat-\PFV.\REL{}\\
%%TRANS1
\glt
‘Kabura, it is him who ate it.’\\
%%TRANS2
‘Kabura is who ate it.’

%%EXEND

\z
\z

An alternative analysis of this ``reverse pseudocleft" construction views the initial NP as a separate phrase, and the post-NP part as a basic cleft in which the pronoun is the clefted constituent. Comparing \xref{bkm:Ref78803230:b} with the basic cleft in \xref{bkm:Ref78803230:a}, this analysis takes the pronoun \textit{wé} to be in focus here. The initial NP actually shows some characteristics of a topic, for example the fact that a pause can follow, as indicated by the comma in \xref{bkm:Ref78803230:b} above, and the fact that the initial NP cannot be questioned, shown in \xref{bkm:Ref79142314}. The construction with a question word as in this example can only be interpreted as an echo question, when you haven’t heard well.

%%EAX
\ea
%%JUDGEMENT
[*]{
%%LABEL
\label{bkm:Ref79142314}
%%CONTEXT
%%LINE1
Nde ni we tuzoojaana?\\
%%LINE2
\gll
ndé  ni  wé  tu-zoo-gi-an-a\textsuperscript{H}\\
%%LINE3
who  \COP{}  1.\PRO{}  1\PL.\SM-\FUT{}-go-\ASS-\FV.\REL{}\\
%%TRANS1
\glt
int. ‘Who is the one you will travel with?’\\
%%TRANS2
}
%%EXEND

\z

What forms the focus is thus not the NP but the pronoun in the cleft: \textit{wé} in \xref{bkm:Ref78803230:b}, \textit{bó} in \xref{bkm:Ref78803758}. \citet{Nshemezimana2016} thus proposes that the referent that both the NP and the pronoun refer to thus fulfils a double pragmatic function: it is referential, taking a topic function, and at the same time it is asserted or identified, forming the focus. In the syntax, these functions are neatly distributed over the NP and the pronoun. In example \xref{bkm:Ref81411744}, it it clear that the initial NP ‘the neighbours’ is given information in the context – what is in focus here is the fact that they are identified as the ones who compromise us with our parents.
\pagebreak

%%EAX
\ea
%%JUDGEMENT
%%LABEL
\label{bkm:Ref81411744}
%%CONTEXT
(\citealt[248]{Nshemezimana2016}, \citealt[94]{LafkiouiEtAl2016})\\
%%LINE1
Igitúma tudashobóra gutéembeerana ukó dushaaká ni ukó \textbf{ababáanyi} bé n’ábaándi batuboná. Kenshi ababáanyi ni bó baduteéranya n’ábavyéeyi.\\
%%LINE2
\gll
i-ki-tum-a  tu-ta-shóbor-a  ku-téembeer-an-a  ukó tu-shaak-a\textsuperscript{H}\textsuperscript{} ni  ukó  a-ba-báanyi  be  na  a-ba-ndi ba-tu-bón-a\textsuperscript{H}  kenshi  a-ba-báanyi  ni  ba-ó  ba-tu-téerany-a\textsuperscript{H}  na  a-ba-vyéeyi\\
%%LINE3
\AUG-7-{}make-\FV.\REL{}  1\PL.\SM-\NEG{}-can-\FV{}  15-move-\ASS-\FV{}  as  1\PL.\SM{}-want-\FV.\REL{}  \COP{}  that  \AUG-2-{}neighbour  with  and  \AUG-2{}-other  2\SM-1\PL.\OM{}-see-\FV.\REL{}  often  \AUG-2{}-neighbour  \COP{}  2-\PRO{}  2\SM-1\PL.\OM{}-compromise-\FV.\REL{}  with  \AUG-2{}-parent\\
%%TRANS1
\glt
‘What makes it impossible for us to go out together like we want is that neighbours and others could see us. Often, the neighbours are the ones who compromise us with the parents.’\\
%%TRANS2
(\textit{Abahungu}, Education, 1980)
%%EXEND

\z

However, this analysis as a topic + basic cleft is suboptimal for contexts where the initial NP does function as a focus, for example as the answer to a question, as in \xref{bkm:Ref80957516}, in a correction, as in \xref{bkm:Ref80957526}, or when modified by the exhaustive particle ‘only’, as in \xref{bkm:Ref118968124}.

\ea
\begin{xlist}
%%EAX
\exi{A:}{
%%JUDGEMENT
%%LABEL
\label{bkm:Ref80957516}
%%CONTEXT
%%LINE1
Ni ndé yandiíriye umukaaté?\\
%%LINE2
\gll
ni  ndé  a-a-\N{}-rí-ir-ye\textsuperscript{H}  u-mu-kaaté\\
%%LINE3
\COP{}  who  1\SM-\N.\PST-1\SG.\OM{}-eat-\APPL-\PFV.\REL{}  \AUG-{}3-bread\\
%%TRANS1
\glt
  ‘Who ate my bread?'\\
%%TRANS2
}
%%EXEND

%%EAX
\exi{B:}{
%%JUDGEMENT
%%LABEL
%%CONTEXT
%%LINE1
Pita ni wé yawuriye.\\
%%LINE2
\gll
  Pita  ni  wé  a-a-wu-rí-ye\textsuperscript{H}\\
%%LINE3
  Peter  \COP{}  1.\PRO{}  1\SM-\N.\PST-3\OM{}-eat-\PFV.\REL{}\\
%%TRANS1
\glt
  ‘Peter, it’s him who ate it.’ / ‘Peter is the one who ate it.’\\
%%TRANS2
 }
%%EXEND

\end{xlist}
\z

\ea
\label{bkm:Ref80957526}
\begin{xlist}
%%EAX
\exi{A:}{
%%JUDGEMENT
%%LABEL
%%CONTEXT
%%LINE1
Bukuru asa na sé caane.\\
%%LINE2
\gll
Bukuru  a-sa-a  na  sé  caane\\
%%LINE3
1.Bukuru  1\SM{}-look.like-\FV{}  with  his.father  \INT{}\\
%%TRANS1
\glt
‘Bukuru looks a lot like his father.’\\
%%TRANS2
}
%%EXEND

%%EAX
\exi{B:}{
%%JUDGEMENT
%%LABEL
%%CONTEXT
%%LINE1
Oya, Butoyi ni wé basa cane.\\
%%LINE2
\gll
  oya  Butoyi  ni  wé  ba-sa-a\textsuperscript{H}  caane\\
%%LINE3
  no  1.Butoyi  \COP{}  1.\PRO{}  2\SM{}-look.like-\FV.\REL{}  \INT{}\\
%%TRANS1
\glt
  ‘No, Butoyi, he’s the one who looks a lot like him.’\\
%%TRANS2
}
%%EXEND

\end{xlist}
\z
\pagebreak
%%EAX
\ea
%%JUDGEMENT
%%LABEL
\label{bkm:Ref118968124}
%%CONTEXT
(Tell me about your siblings.)\\
%%LINE1
Abahuúngu gusa ni bó tuvukána.\\
%%LINE2
\gll
a-ba-huúngu  gusa  ni  ba-ó  tu-vúukan-a\textsuperscript{H}\\
%%LINE3
\AUG-{}2-boy  only  \COP{}  2-\PRO{}  1\PL.\SM-{}be.sibling-\FV.\REL{}\\
%%TRANS1
\glt
‘We only have brothers.’\\
%%TRANS2
lit. ‘Boys only; it’s them that are sibling with us.’

%%EXEND

\z

These could be analysed either as a reverse pseudocleft (i.e. a copular construction with an identificational initial NP that is in focus), or alternatively as a fragment answer followed by a basic cleft with a coreferential pronoun. It is very possible that the construction has multiple underlying structures for what looks to be the same on the surface, but further research is required to confirm this. Further research can also confirm the focus interpretation in this construction: if what follows the initial NP is indeed a basic cleft (with the clefted personal pronoun), we would expect the focus interpretation to be the same as that of the basic cleft. 

\section{Conclusion}

Information structure has a fundamental influence on the morphosyntax of Kirundi. It determines the word order to a large degree, with the requirements for topics in the preverbal domain and a final focus position leading to a range of subject inversion constructions. Verbal inflection is also partly determined by information structure, as shown for the behaviour of the conjoint and disjoint verb forms in a larger range of tests than considered in the literature for Kirundi so far. Newly described in this chapter are the predicate doubling constructions: topic doubling is used for verum (with pragmatic extensions of contrast, intensity, and depreciation), and doubling with an in-situ nominalisation of the same predicate results in a prototypicality reading. Equally new is the description of the agreeing particle -\textit{ó} as a topic marker of a contrastive topic, or inclusive addition when used with \textit{na}. Finally, the chapter has extended earlier descriptions of various copular constructions, pinpointing the focus interpretation of basic clefts, pseudoclefts, and reverse pseudoclefts.

\section*{Acknowledgements}\largerpage

This research was supported by NWO Vidi grant 276-78-001 as part of the BaSIS “Bantu Syntax and Information Structure” project at Leiden University. We thank the two reviewers for their helpful comments, and the BaSIS colleagues for their support. Any remaining errors are ours alone.

\section*{Abbreviations and symbols}

Numbers refer to noun classes, unless followed by \SG{}/\PL{}, in which case the number (1 or 2) refers to first or second person. Tone marking follows Kirundi tradition; high tones are marked by an acute accent; low tones remain unmarked. Capital N indicates a place-assimilating nasal.

\begin{multicols}{2}
\begin{tabbing}
MMMM \= ungrammatical\kill
%%% All Leipzig abbreviations are commented out, following the LangSci guidelines of only listing non-Leipzig abbreviations.
* \> ungrammatical\\
\textsuperscript{?} \> degraded grammaticality\\
\textsuperscript{\#} \> infelicitous in the given \\ \>  context\\
*(X) \> the presence of X is obligatory \\ \> and cannot grammatically \\ \>  be omitted\\
(*X) \> the presence of X would make    \\ \>  the sentence ungrammatical\\
(X) \> the presence of X is optional\\
{}[ ]\textsuperscript{\FOC} \> focus\\
{}[ ]\textsuperscript{H} \> melodic/floating high tone\\
{}[ ]\textsuperscript{\TOP} \> topic\\
% \APPL{} \> applicative\\
\ASS{} \> associative\\
\AUG{} \> augment\\
% \CAUS{} \> causative\\
\CJ{} \> conjoint\\
\CM{} \> contrastive marker\\
\CONN{} \> connective\\
% \COP{} \> copular\\
DAI \> default agreement inversion\\
\DEM{}\textsubscript{X} \> demonstrative of series X\\
\DJ{} \> disjoint\\
\EXP{} \> expletive\\
% \FUT{} \> future\\
\FV{} \> final vowel\\
\IDEO{} \> ideophone\\
% \IMP{} \> imperative\\
\INCP{} \> inceptive\\
\INT{} \> intensifier\\
\N.\PST{} \> near past\\
% \NEG{} \> negative\\
\OM{} \> object marker\\
% Q \> question marker\\
% \PASS{} \> passive \\
\PERS{} \> persistive\\
% \PFV{} \> perfective\\
% \PL{} \> plural\\
% \POSS{} \> possessive \\
\POT{} \> potential\\
\PRCS{} \> precessive pronoun\\
\PRO{} \> pronoun\\
% \PROP{} \> proposition\\
% \PRS{} \> present\\
\PRSNT{} \> presentative\\
\QUOT{} \> quotative\\
% \REL{} \> relative\\
% \REFL{} \> reflexive\\
\RMT.\PST{} \> remote past\\
% \SBJV{} \> subjunctive\\
% \SG{} \> singular\\
\SM{} \> subject marker\\
VP \> verb phrase 
\end{tabbing}
\end{multicols}

\sloppy\printbibliography[heading=subbibliography,notkeyword=this]

\end{document}
