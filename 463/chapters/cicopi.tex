\documentclass[output=paper]{langscibook}
\ChapterDOI{10.5281/zenodo.14833620}
\author{Nelsa Nhantumbo\orcid{}\affiliation{Universidade Eduardo Mondlane} and Jenneke van der Wal\orcid{}\affiliation{Leiden University} }
\title{The expression of information structure in Cicopi}
\abstract{This chapter describes the information structure in Cicopi, discussing and illustrating the information-structural function of verbal inflection, word order, three types of cleft constructions, two types of predicate doubling, and the choice of referent expression. Cicopi has three different forms to mark the present tense: conjoint, disjoint/habitual, and progressive; and the perfective is marked by \nobreakdash-\textit{ile/\nobreakdash-ite} or \textit{to-.} The conditions under which these are used seems to depend not just on constituent-finality and aspect, but also on focus and maybe evidentiality. There is no dedicated focus position in Cicopi and the three clefts, as well as subject inversion, are used to express various types of focus.
}
\IfFileExists{../localcommands.tex}{
  \addbibresource{../localbibliography.bib}
  \usepackage{langsci-optional}
\usepackage{langsci-gb4e}
\usepackage{langsci-lgr}

\usepackage{listings}
\lstset{basicstyle=\ttfamily,tabsize=2,breaklines=true}

%added by author
% \usepackage{tipa}
\usepackage{multirow}
\graphicspath{{figures/}}
\usepackage{langsci-branding}

  
\newcommand{\sent}{\enumsentence}
\newcommand{\sents}{\eenumsentence}
\let\citeasnoun\citet

\renewcommand{\lsCoverTitleFont}[1]{\sffamily\addfontfeatures{Scale=MatchUppercase}\fontsize{44pt}{16mm}\selectfont #1}
   
  %% hyphenation points for line breaks
%% Normally, automatic hyphenation in LaTeX is very good
%% If a word is mis-hyphenated, add it to this file
%%
%% add information to TeX file before \begin{document} with:
%% %% hyphenation points for line breaks
%% Normally, automatic hyphenation in LaTeX is very good
%% If a word is mis-hyphenated, add it to this file
%%
%% add information to TeX file before \begin{document} with:
%% %% hyphenation points for line breaks
%% Normally, automatic hyphenation in LaTeX is very good
%% If a word is mis-hyphenated, add it to this file
%%
%% add information to TeX file before \begin{document} with:
%% \include{localhyphenation}
\hyphenation{
affri-ca-te
affri-ca-tes
an-no-tated
com-ple-ments
com-po-si-tio-na-li-ty
non-com-po-si-tio-na-li-ty
Gon-zá-lez
out-side
Ri-chárd
se-man-tics
STREU-SLE
Tie-de-mann
}
\hyphenation{
affri-ca-te
affri-ca-tes
an-no-tated
com-ple-ments
com-po-si-tio-na-li-ty
non-com-po-si-tio-na-li-ty
Gon-zá-lez
out-side
Ri-chárd
se-man-tics
STREU-SLE
Tie-de-mann
}
\hyphenation{
affri-ca-te
affri-ca-tes
an-no-tated
com-ple-ments
com-po-si-tio-na-li-ty
non-com-po-si-tio-na-li-ty
Gon-zá-lez
out-side
Ri-chárd
se-man-tics
STREU-SLE
Tie-de-mann
} 
  \togglepaper[9]%%chapternumber
}{}

\begin{document}
\maketitle 
\label{ch:9}
%\shorttitlerunninghead{}%%use this for an abridged title in the page headers


\section{Introduction}

This chapter gives a general overview of the expression of information structure in Cicopi. Cicopi is a Bantu language codified as S61 in the \citet{Guthrie1948} classification, updated by \citet{Maho2009}, with ISO code [cce]. It is spoken predominantly in South Mozambique, in the provinces of Gaza and Inhambane, by approximately 336,020 people \citep{INE2017}. According to \citet{NgungaFaquir2011}, this language has 6 variants: Cindonje, Cilenge, Citonga, Cicopi, Cilambwe and Cikhambani. Cicopi was the variant used in this research because it is the reference variant \citep[see][]{NgungaFaquir2011}. Data were collected during the month of July 2019 in Chidenguele, Mozambique, with two male speakers and one female speaker aged 24--42, and were supplemented with data that were checked with these and other native speakers. The data were entered into an Online Language Database accessible via Dative, and transcribed according to the orthography \citep{NgungaFaquir2011}, with added tone marking and indication of vowel lengthening. We have done our best to mark surface tone, including downstep, but we are not certain of all tone marking (especially on the examples from the Frog Story), and further study is necessary to verify the tones and intonation and analyse the tone system.

Copi is a little-studied language. Among the studies carried out, the following stand out: \citet{DosSantos1941} with \citetitle{DosSantos1941}; \citet{Bailey1976} on the phonology and morphotonology; \citet{Nhantumbo2009} describing morpho-phonological processes resulting from marking the past in Copi; \citet{Nhantumbo2014} describing the verb in present and future; \citet{Nhantumbo2019} on the phonology and morphology of the verb; and \citet{Nhantumbo2019a} describing negation marking in Cicopi.

In this chapter, the aim is to provide an overview of how information structure in Cicopi is expressed. We do this by describing the context-dependent interpretation of a number of linguistic strategies in Cicopi, i.e. from form to function. We start with verbal inflection in Cicopi, to show the use and interpretation of different verb forms (\sectref{bkm:Ref141344108}), followed by the word order (\sectref{bkm:Ref141344123}), discussing the basic word order in Cicopi, the absence of a dedicated focus position in the sentence, and subject inversion constructions. We then continue to present the form and function of three types of cleft constructions (\sectref{bkm:Ref121993097}), and predicate doubling (\sectref{bkm:Ref141344221}). We close with remarks on referent expression (\sectref{bkm:Ref127267714}).

\section{Verbal inflection}
\label{bkm:Ref141344108}\label{bkm:Ref141358518}
Earlier work \citep{Bailey1976,Nhantumbo2014,Nhantumbo2019} primarily described the formal aspects (the morphology, tonology and morphophonology) of verbal inflection. In this section, we survey not just the form but investigate the use and meaning of the conjugations in more detail.

In some tense/aspect conjugations, Cicopi has more than one verb form. There are three forms in the present tense (\sectref{bkm:Ref141367382}), and two forms in the perfective (\sectref{bkm:Ref141367401}), whose distribution seems to be influenced or even determined by information structure (but see the discussion below for the influence of aspect and evidentiality, too, in the present tense). Their forms are given in \tabref{tab:key:1}. Two of the forms in the present tense resemble the so-called conjoint/disjoint alternation as found in other southern and eastern Bantu languages (see \citealt{vanderWalHyman2017} for an overview). This alternation is characterised by the two verb forms typically sharing the same tense/aspect semantics, but showing a difference in their distribution (conjoint cannot appear clause-finally whereas disjoint can) and a difference in information structure (the conjoint form being followed by focused information, and the disjoint not). The progressive form (which \citealt{Bailey1976} calls ``continuous") constitutes its own category, but interacts with the other two forms in applying to the present situation \citep{Nhantumbo2005}. The two perfective forms do not seem to vary in aspect. We discuss some information-structural differences below, but further research is needed to establish the exact form-meaning mappings.

\begin{table}
\begin{tabularx}{\textwidth}{lllX}
\lsptoprule
 & affirmative & translation & negative\\
\midrule
present \CJ{} & SM-VB-a & \multirow{2}{*}{I cook/am cooking} & \multirow{2}{*}{SM-ka-VB-i}\\
present \DJ/habitual & SM-a-VB-a &  & \\
\addlinespace
present progressive & SM-o-VB-a & I am cooking & SM-ka-VB-i (+tone)\\
\addlinespace
perfective & SM\nobreakdash-VB\nobreakdash-ile/\nobreakdash-ite/\nobreakdash-e & I (have) cooked & \multirow{2}{*}{SM-yá-VB-a}\\
TO & SM-to-VB-a & I (have) cooked & \\
\lspbottomrule
\end{tabularx}
\caption{Overview of tenses with possible influence of information structure in Cicopi}
\label{tab:key:1}
\end{table}

We discuss the form, distribution, and interpretation of the present and perfective conjugations in turn.

\subsection{Present tense}
\label{bkm:Ref141367382}
The present tense conjoint and disjoint verb form in Cicopi belong to the same conjugation, as they share one negative form (see \tabref{tab:key:1}), and they can occur in question-answer pairs, with the question in \xref{bkm:Ref141369089} containing the unmarked conjoint present form, and the answer the disjoint form marked by \textit{a\nobreakdash-}.

\ea
\label{bkm:Ref141369089}
%%EAX
\begin{xlist}
\exi{Q:}{
%%JUDGEMENT
%%LABEL
%%CONTEXT
%%LINE1
{Ina matutu:ma cikolwa:ni?}\footnote{No tone marking.}\jambox*{[conjoint]}
%%LINE2
\gll
ina mu-a-tutum-a   ci-kolwa-ni\\
%%LINE3
Q 2\PL.\SM-\DJ{}-run-\FV{}  7-school-loc\\
%%TRANS1
\glt ‘Do you run at school?’ \\
%%TRANS2
}
%%EXEND

%%EAX
\exi{A:}{
%%JUDGEMENT
%%LABEL
%%CONTEXT
%%LINE1
{Hatútû:ma.}\jambox*{[disjoint]}
%%LINE2
\gll
hi-a-tutum-a\\
%%LINE3
1\PL.\SM-\DJ-run-\FV{}\\
%%TRANS1
\glt ‘We (do) run.’
%%TRANS2
}
%%EXEND

%%EAX
\exi{A':}{
%%JUDGEMENT
%%LABEL
%%CONTEXT
%%LINE1
{Hígó:nda mabhu:ku.}\jambox*{[conjoint]}
%%LINE2
\gll
hi-gond-a ma-bhuku\\
%%LINE3
1\PL.\SM-read-\FV{} 6-book\\
%%TRANS1
\glt ‘We read books.’
%%TRANS2
}
%%EXEND

\end{xlist}
\z

The present tense forms differ in their distribution: As defining for the conjoint/disjoint alternation \citep{vanderWal2017}, the conjoint form cannot appear in sentence-final position of a main clause, as shown in \xref{bkm:Ref120705766:a} and \xref{bkm:Ref120705751:b}. Instead, the disjoint form must be used, as in \xref{bkm:Ref120705766:b}, or the conjoint form must appear in non-final position as in \xref{bkm:Ref120705766:c}. The progressive can also appear in sentence-final position, as shown in \xref{bkm:Ref120705751:a}.

\ea
\label{bkm:Ref120705766}
%%EAX
\ea
%%JUDGEMENT
[*]{
%%LABEL
\label{bkm:Ref120705766:a}
%%CONTEXT
%%LINE1
{Dikanéká hidiwo:na.}\jambox*{[conjoint]}
%%LINE2
\gll
di-kaneka  hi-di-won-a\\
%%LINE3
5-mug  1\PL.\SM-5\OM-see-\FV\\
%%TRANS1
\glt
int. ‘The mug, we see it.’\\
%%TRANS2
}
%%EXEND


%%EAX
\ex
%%JUDGEMENT
[]{
%%LABEL
\label{bkm:Ref120705766:b}
%%CONTEXT
%%LINE1
{Dikanéká hadiwo:na.}\jambox*{[disjoint]}
%%LINE2
\gll
di-kaneka  hi-a-di-won-a\\
%%LINE3
5-mug  1\PL.\SM-\DJ-5\OM-see-\FV\\
%%TRANS1
\glt
‘The mug, we see it.’\\
%%TRANS2
}
%%EXEND


%%EAX
\ex
%%JUDGEMENT
[]{
%%LABEL
\label{bkm:Ref120705766:c}
%%CONTEXT
%%LINE1
{Hiwóná dikanê:ka.}\jambox*{[conjoint]}
%%LINE2
\gll
hi-won-a  di-kaneka \\
%%LINE3
1\PL.\SM-see-\FV{}  5-mug \\
%%TRANS1
\glt
‘We see a/the mug.’\\
%%TRANS2
}
%%EXEND

\z
\z

\ea
\label{bkm:Ref120705751}
(What are you doing?)
%%EAX
\ea
%%JUDGEMENT
[]{
%%LABEL
\label{bkm:Ref120705751:a}
%%CONTEXT
%%LINE1
{Hótútû:ma.}\jambox*{[progressive]}
%%LINE2
\gll
hi-o-tutum-a\\
%%LINE3
1\PL.\SM-\PROG-run-\FV\\
%%TRANS1
\glt
‘We are running.’\\
%%TRANS2
}
%%EXEND


  %%EAX
\ex
%%JUDGEMENT
[*]{
%%LABEL
\label{bkm:Ref120705751:b}
%%CONTEXT
%%LINE1
{Hítútû:ma.}\jambox*{[conjoint]}
%%LINE2
\gll
hi-tutum-a\\
%%LINE3
1\PL.\SM-run-\FV\\
%%TRANS1
\glt
int. ‘We are running. / We run.’\\
%%TRANS2
}
%%EXEND

\z
\z

Interestingly, the conjoint form is accepted in final position as a yes/no question, as in \xref{bkm:Ref120705945}. This is a use that has not been described in other languages, and at the moment it is not yet well understood.

%%EAX
\ea
%%JUDGEMENT
%%LABEL
\label{bkm:Ref120705945}
%%CONTEXT
(A dog with rabies is coming towards us)\\
%%LINE1
{Hítútû:ma?}\jambox*{[conjoint]}
%%LINE2
\gll
hi-tutum-a\\
%%LINE3
1\PL.\SM-run-\FV\\
%%TRANS1
\glt ‘Do we run? / Should we run?’
%%TRANS2
%%EXEND

\z

There are indications in the penultimate lengthening that the conjoint and the progressive forms are phrased together with the following constituent(s). Based on analyses for other eastern and southern Bantu languages (e.g. \citealt{Zerbian2006} for Northern Sotho; \citealt{ZellerEtAl2017,Halpert2017} for Zulu; \citealt{Devos2008} for Makwe among others), we hypothesise that penultimate lengthening indicates the right boundary of a phonological phrase. The conjoint and progressive forms do not show lengthening on the verb (only on the object), whereas the disjoint form shows lengthening on both verb and object, as illustrated in \xref{bkm:Ref120706029}. However, there seems to be variation, and dedicated investigation is needed to show the interaction between syntax, prosodic phrasing, and information structure.

\ea
\label{bkm:Ref120706029}
%%EAX
\ea
%%JUDGEMENT
%%LABEL
(What are you drawing?/\textsuperscript{\#}Are you doing what I told you?)\\
%%CONTEXT
%%LINE1
{Hireká mâ:ti.}\jambox*{[conjoint]}
%%LINE2
\gll
hi-rek-a  mati\\
%%LINE3
1\PL.\SM-draw-\FV{}  6.water\\
%%TRANS1
\glt ‘We are drawing water.’
%%TRANS2
%%EXEND

%%EAX
\ex
%%JUDGEMENT
%%LABEL
%%CONTEXT
(Are you doing what I said to do?/You don’t draw water.)\\
%%LINE1
{A:thú hárê:ka (mâ:ti).}\jambox*{[disjoint]}
%%LINE2
\gll
athu  hi-a-rek-a  mati\\
%%LINE3
1\PL.\PRO{}  1\PL.\SM-\DJ-draw-\FV{}  6.water\\
%%TRANS1
\glt ‘We (do) draw (water).’\\
%%TRANS2
‘We \textit{are} drawing (water).’
%%EXEND

%%EAX
\ex
%%JUDGEMENT
%%LABEL
(What are you doing now?)\\
%%CONTEXT
%%LINE1
{A:thú ho:rê:ká mâ:ti.}\jambox*{[progressive]}
%%LINE2
\gll
athu  hi-o-rek-a  mati\\
%%LINE3
1\PL.\PRO{}  1\PL.\SM{}-\PROG{}-draw-\FV{}  6.water\\
%%TRANS1
\glt ‘We are drawing water.’
%%TRANS2
%%EXEND

\z
\z

When a postverbal element is in focus, the conjoint or progressive form has to be used; for example an inherently focused content question word \xref{bkm:Ref120706074} or an object modified by the exhaustive focus-sensitive particle ‘only’ \xref{bkm:Ref120706082}. The disjoint form is not accepted here (with or without penultimate length), as indicated for both examples.

\ea
\label{bkm:Ref120706074}
%%EAX
\ea
%%JUDGEMENT
[]{
%%LABEL
%%CONTEXT
%%LINE1
{Muthúmá ca:ni?}\jambox*{[conjoint]}
%%LINE2
\gll
mu-thum-a  cani\\
%%LINE3
2\PL.\SM{}-do-\FV{}  what\\
%%TRANS1
\glt
‘What are you (pl) doing?’\\
%%TRANS2
}
%%EXEND


%%EAX
\ex
%%JUDGEMENT
[*]{
%%LABEL
%%CONTEXT
%%LINE1
{Mathúmá ca:ni?}\jambox*{[disjoint]}
%%LINE2
\gll
mu-a-thum-a  cani\\
%%LINE3
2\PL.\SM-\DJ{}-do-\FV{}  what\\
%%TRANS1
\glt
int. ‘What are you (pl) doing?’\\
%%TRANS2
}
%%EXEND


%%EAX
\ex
%%JUDGEMENT
[]{
%%LABEL
%%CONTEXT
%%LINE1
{Mothúmá ca:ni?}\jambox*{[progressive]}
%%LINE2
\gll
mu-o-thum-a  cani\\
%%LINE3
2\PL.\SM{}-\PROG{}-do-\FV{}  what\\
%%TRANS1
\glt
‘What are you (pl) doing?’\\
%%TRANS2
}
%%EXEND

\z
\z

%%EAX
\ea
%%JUDGEMENT
%%LABEL
\label{bkm:Ref120706082}
%%CONTEXT
%%LINE1
%%LINE2
\gll
Hi-xav-a   / * h-a-xav-a   / h-o-xav-a  ma-fá:ka  dwé.\\
%%LINE3
1\PL.\SM{}-buy-\FV{}  / {} 1\PL.\SM{}-\DJ{}-buy-\FV{}   / 1\PL.\SM{}-\PROG{}-buy-\FV{}  6-maize  only\\
%%TRANS1
\glt
‘We buy/are buying only maize.’\\
%%TRANS2
%%EXEND

\z

Nevertheless, while focus occurs with a conjoint form, the inverse is not true: post-conjoint constituents can be unfocussable items such as parts of idioms, as in \xref{bkm:Ref120706130}, and cognate objects, as in \xref{bkm:Ref120706137}. Cognate objects and parts of idioms cannot be focused, because they cannot generate any alternatives in the idiomatic interpretation \citep[see][]{vanderWal2016,vanderWal2021a} – what else would one be dreaming if not a dream?

%%EAX
\ea
%%JUDGEMENT
%%LABEL
\label{bkm:Ref120706130}
%%CONTEXT
%%LINE1
Nila:vwá / nala:vwá / nola:vwá ngu pápî:lu.\\
%%LINE2
\gll
ni-lav-w-a  / ni-a-lav-w-a  / ni-o-lav-w-a  ngu  papilu\\
%%LINE3
1\SG.\SM{}-want-\PASS{}-\FV{}  / 1\SG.\SM{}-\DJ{}-want-\PASS{}-\FV{}  / 1\SG.\SM{}-\PROG{}-want-\PASS{}-\FV{}  \PREP{}  9.letter\\
%%TRANS1
\glt
literal: ‘I am wanted by the letter.’\\
%%TRANS2
idiomatic: ‘I need to go to the toilet.’

%%EXEND
\z
\pagebreak
\ea
\label{bkm:Ref120706137}
(Context: A child is asleep and making movements.)
%%EAX
\ea
%%JUDGEMENT
%%LABEL
%%CONTEXT
%%LINE1
{Álórá ḿlo:ro.}\jambox*{[conjoint]}
%%LINE2
\gll
a-lor-a  m-loro\\
%%LINE3
1\SM{}-dream-\FV{}  3-dream\\
%%TRANS1
\glt
‘S/he is dreaming a dream.’\\
%%TRANS2
%%EXEND


%%EAX
\ex
%%JUDGEMENT
%%LABEL
%%CONTEXT
%%LINE1
{Ól\textsuperscript{!}ó:rá ḿlo:ro.}\jambox*{[progressive]}
%%LINE2
\gll
a-o-lor-a  m-loro\\
%%LINE3
1\SM{}-\PROG{}-dream-\FV{}  3-dream\\
%%TRANS1
\glt
‘S/he is dreaming a dream.’\\
%%TRANS2
%%EXEND

\z
\z

Thus, the conjoint form in Cicopi does not express (exclusive) focus on a postverbal element. This can also be seen in the possibility to modify an object by \textit{hambi} ‘even’ \xref{bkm:Ref120706251}, which is incompatible with an inherently exclusive meaning, as ‘even’ means that all other more likely items have also been washed, not excluding any alternatives.

%%EAX
\ea
%%JUDGEMENT
%%LABEL
\label{bkm:Ref120706251}
%%CONTEXT
%%LINE1
{Jhoáná ázilé ákuwúlá hambí ni máláso:ro      }\jambox*{[conjoint]}
%%LINE2
\gll
Joana  a-z-ile  a-kuwul-a  hambi  ni  ma-lasoro\\
%%LINE3
1.Joana  1\SM{}-\LIM{}-\PFV{}  1\SM{}-wash-\FV{}  even  and  6-sheet\\
%%TRANS1
\glt
‘Joana washed even the sheets.’\\
%%TRANS2
%%EXEND

\z

The disjoint form has been called the habitual form \citep{Nhantumbo2005}, as it expresses actions that are regularly performed. The progressive form, on the other hand, refers to an ongoing action. The difference between these two verb forms in their aspectual interpretation is illustrated in \xref{bkm:Ref120706241}.

\ea
\label{bkm:Ref120706241}
%%EAX
\ea
%%JUDGEMENT
%%LABEL
%%CONTEXT
%%LINE1
{Hárê:k-á mâ:ti.     }\jambox*{[disjoint]}
%%LINE2
\gll
hi-a-rek-a  mati\\
%%LINE3
1\PL.\SM{}-\DJ{}-draw-\FV{}  6.water\\
%%TRANS1
\glt
‘We draw water (habitually, not at this moment).’\\
%%TRANS2
%%EXEND

  %%EAX
\ex
%%JUDGEMENT
%%LABEL
%%CONTEXT
%%LINE1
{Horê:k-á mâ:ti.    }\jambox*{[progressive]}
%%LINE2
\gll
hi-o-rek-a  mati\\
%%LINE3
1\PL.\SM-\PROG{}-draw-\FV{}  6.water\\
%%TRANS1
\glt
‘We are drawing water (right now).’\\
%%TRANS2
%%EXEND

\z
\z

This does not mean that the disjoint form is obligatorily used for habituals, as can be seen in \xref{bkm:Ref127266023}, which uses the conjoint form in a habitual meaning/context, because the postverbal element is in focus.

%%EAX
\ea
%%JUDGEMENT
%%LABEL
\label{bkm:Ref127266023}
%%CONTEXT
(Habitually, does she eat rice or shima?)\\
%%LINE1
{E:né ngu cíhê:ne ádyá mpû:nga.    }\jambox*{[conjoint]}
%%LINE2
\gll
ene  ngu  ci-hene  a-dy-a  mpunga\\
%%LINE3
1.\PRO{}  \PREP{}  7-habit  1\SM{}-eat-\FV{}  3.rice\\
%%TRANS1
\glt ‘S/he habitually eats rice.’
%%TRANS2
%%EXEND

\z

As mentioned, the progressive is used for actions that are ongoing, as illustrated again in \xref{bkm:Ref127266052}, but the other two forms are acceptable as well for ongoing actions. The only difference indicated by our speakers was that \textit{osinya} (progressive) is used when you’re seeing it now, and \textit{yasinya} (disjoint) when you’re informing someone else.

%%EAX
\ea
%%JUDGEMENT
%%LABEL
\label{bkm:Ref127266052}
%%CONTEXT
%%LINE1
María yásî:nyá / ósî:nyá / ásînya ko:nku.\\
%%LINE2
\gll
Maria  a-a-siny-a  / a-o-siny-a   / a-siny-a  konku\\
%%LINE3
1.Maria  1\SM{}-\DJ{}-dance-\FV{}  / 1\SM{}-\PROG{}-dance-\FV{} /   1\SM{}-dance-\FV{}  now\\
%%TRANS1
\glt
‘Maria is dancing now.’\\
%%TRANS2
%%EXEND

\z

Nevertheless, focus again trumps aspect, because if the postverbal object is focused, the conjoint form is preferred even if the action is carried out at the time of speaking, as in \xref{bkm:Ref127267122}.

%%EAX
\ea
%%JUDGEMENT
%%LABEL
\label{bkm:Ref127267122}
%%CONTEXT
(While someone is up in the tree: Are you picking these oranges for Helena or for Ana?)\\
%%LINE1
Madímwa yá nihaphé:lá/\textsuperscript{\#}nóhaphé:lá A:na.\\
%%LINE2
\gll
ma-dimwa  ya  ni-haph-el-a  / ni-o-haph-el-a  Ana\\
%%LINE3
6-orange  6.\DEM{}.\PROX{}  1\SG.\SM{}-pick-\APPL{}-\FV{}  / 1\SG.\SM{}-\PROG{}-pick-\APPL{}-\FV{}  1.Ana\\
%%TRANS1
\glt
‘These oranges, I’m picking (them) for Ana.’\\
%%TRANS2
%%EXEND

\z

While the exact difference in use and interpretation between the three present tense forms requires more in-depth investigation, the contexts sketched by the speakers for the disjoint and the progressive form indicate not just an aspectual difference but suggest a difference in evidentiality  too, as can be seen in the contexts for the two forms in \xxref{bkm:Ref127267131}{bkm:Ref127267133}. The disjoint form seems to indicate that the speaker had direct visual evidence; the evidential interpretation of the conjoint and progressive forms are not clear yet.\largerpage[2.25]

\ea
\label{bkm:Ref127267131}
%%EAX
\ea
%%JUDGEMENT
%%LABEL
%%CONTEXT
%%LINE1
Káphínda mǒ:vha. \jambox*{[disjoint]}
%%LINE2
\gll
ku-a-phind-a  movha\\
%%LINE3
17\SM{}-\DJ{}-pass-\FV{}   3.car\\
%%TRANS1
\glt
‘A car is passing by.’ (You see it.)\\
%%TRANS2
%%EXEND



  %%EAX
\ex
%%JUDGEMENT
%%LABEL
%%CONTEXT
%%LINE1
Kóph\textsuperscript{!}índa mǒ:vha. \jambox*{[progressive]}
%%LINE2
\gll
ku-o-phind-a  movha\\
%%LINE3
17\SM{}-\PROG{}-pass-\FV{}  3.car\\
%%TRANS1
\glt
‘A car is passing by.’ (Someone else tells/ask you.)\\
%%TRANS2
%%EXEND

\z
\z

\ea
%%EAX
\ea
%%JUDGEMENT
%%LABEL
%%CONTEXT
(Context: I see someone hitting the bulls and want to inform others that are inside.)\\
%%LINE1
{Tápé(:)kwá tího:mú.}\jambox*{[disjoint]}
%%LINE2
\gll
ti-a-pek-w-a  ti-homu\\
%%LINE3
10\SM{}-\DJ{}-hit-\PASS{}-\FV{}  10-bull\\
%%TRANS1
\glt
‘The bulls are (being) beaten.’\\
%%TRANS2
%%EXEND


  %%EAX
\ex
%%JUDGEMENT
%%LABEL
%%CONTEXT
(Why do the bulls have these marks? -- The sentence is not accepted out of the blue.)\\
%%LINE1
{Tópé(:)kwá tího:mú.}\jambox*{[progressive]}
%%LINE2
\gll
ti-o-pek-w-a  ti-homu\\
%%LINE3
10\SM{}-\PROG{}-hit-\PASS{}-\FV{}  10-bull\\
%%TRANS1
\glt
‘The bulls are (being) beaten.’\\
%%TRANS2
%%EXEND

\z
\z

\ea
\label{bkm:Ref127267133}
%%EAX
\ea
%%JUDGEMENT
%%LABEL
%%CONTEXT
(Context: A friend passed by your field and tells you s/he has seen birds eating the maize.)\\
%%LINE1
Sinya:na sâ:dyá mafa:ka. \jambox*{[disjoint]}
%%LINE2
\gll
si-nyana  si-a-dy-a  ma-faka\\
%%LINE3
8-bird  8\SM-\DJ{}-eat-\FV{}  6-maize\\
%%TRANS1
\glt
‘(The) birds are eating (the) maize.’\\
%%TRANS2
%%EXEND

 


  %%EAX
\ex
%%JUDGEMENT
%%LABEL
%%CONTEXT
(generic)\\
%%LINE1
{Tikhu:mba tídyá khô:ndze.}\jambox*{[conjoint]}
%%LINE2
\gll
ti-khumba  ti-dy-a  khondze\\
%%LINE3
10-pig  10\SM{}-eat-\FV{}  grass\\
%%TRANS1
\glt
‘Pigs eat grass.’\\
%%TRANS2
%%EXEND

\z
\z

Further and more targeted research is required to establish the precise factors determining the use of each verb form in the present tense, but aspect (habitual vs. progressive), focus (postverbal or predicate-centred), and evidentiality all seem to play a role.

\subsection{Perfective}
\label{bkm:Ref141367401}
As mentioned, Cicopi has two forms in the perfective conjugation. The first is created with the suffix -\textit{ile/-ite} and the second with the prefix \textit{to}-. As shown in \tabref{tab:key:1} above, the two conjugations share the same negation, suggesting that their main difference is not in tense/aspect semantics.\footnote{We note, though, that both forms are accepted with inchoative verbs, but there seems to be a restriction on stative verbs, e.g. for \nobreakdash-\textit{ziva} ‘know’ the \textit{to}\nobreakdash- form is not accepted while the \nobreakdash-\textit{ile} form is. We thank a reviewer for asking about these predicates, but have to leave this for future research.} Both encode that the event has been finished \citep{Nhantumbo2009}. The suffixes \nobreakdash-\textit{ile} and \nobreakdash-\textit{ite} are not different morphemes, but allomorphs of the same morpheme. The allomorph \nobreakdash-\textit{ite} is used with stems ending in \nobreakdash-\textit{l} or \textit{\nobreakdash-t} \citep{Nhantumbo2014}, as in \xref{bkm:Ref136287899} and \xref{bkm:Ref136287901}, and with monoconsonantal verb stems (e.g. \nobreakdash-\textit{w}\nobreakdash- ‘fall’ and \nobreakdash-\textit{dy}\nobreakdash- ‘eat’). We refer to \citet{Nhantumbo2019} for further argumentation on the allomorphy.

%%EAX
\ea
%%JUDGEMENT
%%LABEL
\label{bkm:Ref136287899}
%%CONTEXT
%%LINE1
Maria abháté dipápí:lo.\\
%%LINE2
\gll
Maria   a-bhal-ile  dipapilo\\
%%LINE3
1.Maria  1\SM{}-write-\PFV{}  5-letter\\
%%TRANS1
\glt
‘Maria wrote a letter.’\\
%%TRANS2
%%EXEND

%%EAX
\ex
%%JUDGEMENT
%%LABEL
\label{bkm:Ref136287901}
%%CONTEXT
%%LINE1
Mwanâ:na adukê:te dibhúlú:ku.\\
%%LINE2
\gll
mw-anana   a-duket-ile  di-bhuluku\\
%%LINE3
1-child  1\SM{}-try-\PFV{}   5-trousers\\
%%TRANS1
\glt
‘The child tried on the trousers.’\\
%%TRANS2
%%EXEND

\z

We discuss the forms \textit{{}-ile/-ite} and the form \textit{to-} in this chapter, as we see some information-structural differences between them. The \textit{to}{}- form is felicitous in expressing state-of-affairs focus, i.e. supplying or contrasting the lexical value of the verb, as indicated in \xref{bkm:Ref127267278:a}, \xref{bkm:Ref127267346}, and \xref{bkm:Ref127267428}, but it is not felicitous in verum contexts, i.e. emphatic focus on the truth value, as shown in \xref{bkm:Ref127267278:b} and \xref{bkm:Ref141371619:a}. The perfective shows the opposite behaviour and is felicitously used to express verum, as seen in \xref{bkm:Ref127267286} and \xref{bkm:Ref141371619:b}.\largerpage[2]

%%EAX
\ea
%%JUDGEMENT
%%LABEL
\label{bkm:Ref127267278}
%%CONTEXT
%%LINE1
Átô:dya.\\
%%LINE2
\gll
a-to-dy-a\\
%%LINE3
1\SM{}-\TO{}-eat-\FV{}\\
%%TRANS1
\glt
‘S/he ate (it).’\\
%%TRANS2
%%EXEND

\ea[]{
\label{bkm:Ref127267278:a} answer to ‘What happened with the food on the table?'  \jambox*{[SoA focus]}
}
\ex[\textsuperscript{\#}]{
\label{bkm:Ref127267278:b} correction of ‘She didn’t eat.’  \jambox*{[verum]}
}
\z

%%EAX
\ex
%%JUDGEMENT
%%LABEL
\label{bkm:Ref127267286}
%%CONTEXT
(She didn’t eat.)\\
%%LINE1
(i:m) Ádyî:te.  \jambox*{[verum]}
%%LINE2
\gll
ii  a-dy-ile\\
%%LINE3
yes   1\SM{}-eat-\PFV{}\\
%%TRANS1
\glt
‘(Yes) S/he ate.’ / `S/he \textit{did} eat.\\
%%TRANS2
%%EXEND

\z

%%EAX
\ea
%%JUDGEMENT
%%LABEL
\label{bkm:Ref127267346}
%%CONTEXT
%%LINE1
{Vátótútû:má kûmbe vátôsámbê:la?  }\jambox*{[SoA focus]}
%%LINE2
\gll
va-to-tutum-a  kumbe  va-to-sambel-a\\
%%LINE3
2\SM{}-\TO{}-run-\FV{}  or  2\SM{}-\TO{}-swim-\FV{}\\
%%TRANS1
\glt
‘Did they run or did they swim?’\\
%%TRANS2
%%EXEND

\z

%%EAX
\ea
%%JUDGEMENT
%%LABEL
\label{bkm:Ref127267428}
%%CONTEXT
(Did Paulo wash the beans?) \jambox*{[SoA focus]}
%%LINE1
Ihî:mhim, Paulu aákuwúlá feijáu, átô:bhǐ:ká dwé.\\
%%LINE2
\gll
ihimhim  Paulo  a-a-kuwul-a  feijao  a-to-bhik-a  dwe\\
%%LINE3
no  1.Paolo  1\SM{}-\NEG{}-wash-\FV{}  9.bean  1\SM{}-\TO{}-cook-\FV{}  only\\
%%TRANS1
\glt
‘No, Paulo didn’t wash the beans, he only cooked (them).’\\
%%TRANS2
%%EXEND

\z

\ea\label{bkm:Ref141371619}(Paulo didn’t cook the beans.) You’re lying, …  \jambox*{[verum]}

%%EAX
\ea
%%JUDGEMENT
[\textsuperscript{\#}]{
%%LABEL
\label{bkm:Ref141371619:a}
%%CONTEXT
%%LINE1
Páúlú átôbǐ:ká feijáu\\
%%LINE2
\gll
Paulo  a-to-bhik-a  feijao\\
%%LINE3
1.Paulo  1\SM{}-\TO{}-cook-\FV{}  9.bean\\
%%TRANS1
\glt
‘Paulo cooked (the) beans.’\\
%%TRANS2
\textsuperscript{\#}`Paulo did cook (the) beans.’
}
%%EXEND


%%EAX
\ex
%%JUDGEMENT
[]{
%%LABEL
\label{bkm:Ref141371619:b}
%%CONTEXT
%%LINE1
Páúlú abhíkî:le feijáu.\\
%%LINE2
\gll
Paulo  a-bhik-ile  feijao\\
%%LINE3
1.Paulo  1\SM{}-cook-\PFV{}  9.bean\\
%%TRANS1
\glt
‘Paulo cooked (the) beans.’\\
%%TRANS2
‘Paulo did cook (the) beans.’
}
%%EXEND

\z
\z

Either form is acceptable with a postverbal focus on the object, as illustrated for content question words in \xref{bkm:Ref127267469} and for answers to object questions in \xref{bkm:Ref127267478}. With VP focus either form is acceptable too, as shown in \xref{bkm:Ref127267501}.

\ea\label{bkm:Ref127267469}
%%EAX
\ea
%%JUDGEMENT
%%LABEL
%%CONTEXT
%%LINE1
Mutómahá câ:ni?\\
%%LINE2
\gll
mu-to-mah-a  cani\\
%%LINE3
2\PL.\SM{}-\TO{}-do-\FV{}  what\\
%%TRANS1
\glt
‘What did you do?’\\
%%TRANS2
%%EXEND


  %%EAX
\ex
%%JUDGEMENT
%%LABEL
%%CONTEXT
%%LINE1
A:nu mumahilé câ:ni?\\
%%LINE2
\gll
anu  mu-mah-ile  cani\\
%%LINE3
2\PL{}.\PRO{}  2\PL.\SM{}-do-\PFV{}  what\\
%%TRANS1
\glt
‘What did you do?’\\
%%TRANS2
%%EXEND

\z
\z
\pagebreak
\ea\label{bkm:Ref127267478}
%%EAX
\ea
%%JUDGEMENT
%%LABEL
%%CONTEXT
(What did you drink?)\\
%%LINE1
Nitóselá sérvhe:jha.\\
%%LINE2
\gll
ni-to-sel-a  servhejha\\
%%LINE3
1\SG.\SM{}-\TO{}-drink-\FV{}  beer\\
%%TRANS1
\glt
‘I drank beer.’\\
%%TRANS2
%%EXEND


%%EAX
\ex
%%JUDGEMENT
%%LABEL
%%CONTEXT
(Who did grandma give the mangoes?)\\
%%LINE1
Áníngílé titíyá Lu:rdi.\\
%%LINE2
\gll
a-ning-ile  titiya  Lurdi\\
%%LINE3
1\SM{}-give-\PFV{}  1.aunt  1.Lurdes\\
%%TRANS1
\glt
‘She gave (them to) aunt Lurdes.’\\
%%TRANS2
%%EXEND

\z
\z

\ea
\label{bkm:Ref127267501}
(What did you do?)\\
%%EAX
\ea
%%JUDGEMENT
%%LABEL
%%CONTEXT
%%LINE1
Hikaláhatshi:lé hibhulá ní na:wé.\\
%%LINE2
\gll
hi-khalahatsi-ile  hi-bhul-a  ni  ni-awe\\
%%LINE3
1\PL.\SM{}-sit.down-\PFV{}  1\PL.\SM{}-talk-\FV{}  with  with-2\SG{}.\PRO{}\\
%%TRANS1
\glt
‘We sat and talked with you.’\\
%%TRANS2
%%EXEND


%%EAX
\ex
%%JUDGEMENT
%%LABEL
%%CONTEXT
%%LINE1
Hitókhaláhátshi hibhulá ní na:we.\\
%%LINE2
\gll
hi-to-khalahatshi  hi-bhul-a  ni  ni-awe\\
%%LINE3
1\PL.\SM{}-\TO{}-sit.down  1\PL.\SM{}-talk-\FV{}  with  with-2\SG{}.\PRO{}\\
%%TRANS1
\glt
‘We sat and talked with you.’\\
%%TRANS2
%%EXEND

\z
\z

In summary, information structure influences the choice of verbal conjugations in Cicopi, at least in the perfective and the present, although the exact determining factors remain to be investigated.

\section{Word order}
\label{bkm:Ref141344123}
As a ``basic word order",\footnote{See \citet{KerrEtAl2023} for discussion on the extent to which grammatical roles like subject and object are useful in describing word order in Bantu languages.} Cicopi uses (S)VO order in the context of VP focus. When the subject is highly active, it is preferably expressed by just the subject marker;  a full NP subject is present when (re)activating the subject referent. This word order is illustrated in example \xref{bkm:Ref121988028} from the Frog story, where the boy is introduced in the first sentence, and the next line comments on this boy, adding the information that he has a frog and a dog.
\largerpage[-1]\pagebreak

%%EAX
\ea
%%JUDGEMENT
%%LABEL
\label{bkm:Ref121988028}
%%CONTEXT
(Context: There was a boy who was fond of animals.)\\
%%LINE1
Se m’fáná wúwa, átífúyá díkhélé ni yi:mbwá.\\
%%LINE2
\gll
se  m’-fana  wuwa  a-ti-fuy-a  di-khele  ni  yi-mbwa\\
%%LINE3
so  1-boy  1.\DEM{}.\PROX{}  1\SM{}-\IPFV{}-raise-\FV{}  5-frog  and  9-dog\\
%%TRANS1
\glt
‘So this boy raised a frog and a dog.’\\
%%TRANS2
%%EXEND

\z

In this section, we further discuss the information-structural functions found in the preverbal domain (Sections~\ref{bkm:Ref121988240} and~\ref{bkm:Ref124153637}), the lack of a dedicated focus position in the sentence (\sectref{bkm:Ref121988262}), subject inversion constructions and their interpretation (\sectref{bkm:Ref121988307}), and the right periphery (\sectref{bkm:Ref121988320}).


\subsection{No preverbal focus (?)}
\label{bkm:Ref120693404}\label{bkm:Ref121988240}
In terms of word order and the influence that information structure has on it, there seems to be a split topic-V-nontopic in Cicopi, but there is no dedicated focus position. Focused constituents occur in the postverbal domain (see Sections~\ref{bkm:Ref121988262} and~\ref{bkm:Ref121988307}) and, as is familiar from other Bantu languages (see for example \textcite{chapters/makhuwa}), focus is not allowed in the preverbal domain. This shown for question words in \xref{bkm:Ref121988341} and answers in \xref{bkm:Ref121988348}, both focused subjects. 

%%EAX
\ea
%%JUDGEMENT
[*]{
%%LABEL
\label{bkm:Ref121988341}
%%CONTEXT
%%LINE1
Mání afí:lé?\\
%%LINE2
\gll
mani  a-f-ile\\
%%LINE3
who    1\SM{}-die-\PFV{}\\
%%TRANS1
\glt
‘Who died?’\\
%%TRANS2
}
%%EXEND

%%EAX
\ex
%%JUDGEMENT
[]{
%%LABEL
\label{bkm:Ref121988348}
%%CONTEXT
(Who is cooking rice?)\\
%%LINE1
\textsuperscript{\#} Luiza abhika mpu:nga.\footnote{This example lacks tone marking.}\\
%%LINE2
\gll
{\phantom{\textsuperscript{\#}}} Luisa  a-bhik-a  mpunga\\
%%LINE3
{} 1.Luisa  1\SM{}-cook-\FV{}  3.rice\\
%%TRANS1
\glt
\phantom{\textsuperscript{\#}} ‘Luisa is cooking rice.’\\
%%TRANS2
}
%%EXEND

\z

Based on this generalisation, we would expect that a constituent modified by the exhaustive focus particle ‘only’ is also rejected. While it is true that preverbal non-subjects can never be modified by \textit{dwe} ‘only’ \xref{bkm:Ref141347018}, it does seem acceptable for preverbal subjects – compare \xxref{bkm:Ref141347018}{bkm:Ref141347020}, with the relevant constituent marked in boldface. Further research is needed here, as we also noted a preference in \xref{bkm:Ref124149613} for the exhaustive particle to apply to the predicate rather than to the subject.

%%EAX
\ea
%%JUDGEMENT
[]{
%%LABEL
\label{bkm:Ref141347018}
%%CONTEXT
%%LINE1
\textbf{Mpu:nga} (*\textbf{dwé}) Luiza óbhǐ:ka.\\
%%LINE2
\gll
mpunga dwe  Luisa  a-o-bhik-a\\
%%LINE3
3.rice  only  1.Luisa  1\SM{}-\PROG{}-cook-\FV{}\\
%%TRANS1
\glt
‘(*only) The rice, Luisa is cooking (it).’\\
%%TRANS2
}
%%EXEND

\z

%%EAX
\ea
%%JUDGEMENT
[]{
%%LABEL
%%CONTEXT
%%LINE1
Kupíndílé tixaka tá tî:ngí tá síha:ri, kámbé \textbf{pho:ngó} \textbf{dwé} yíwî:te.\\
%%LINE2
\gll
ku-pind-ile  ti-xaka  t-a   t-ingi   t-a   si-hari   kambe  \textbf{phongo}  \textbf{dwe}  yi-w-ile\\
%%LINE3
17\SM{}-pass-\PFV{}  10-kind  10-\CONN{}  10-many  10-\CONN{}  8-animal but  9.goat  only  9\SM{}-fall-\PFV{}\\
%%TRANS1
\glt
‘There passed many types of animals, but only a/the goat fell.’
%%TRANS2
}
%%EXEND
 
\z

%%EAX
\ea
%%JUDGEMENT
[]{
%%LABEL
%%CONTEXT
%%LINE1
\textbf{Páulu} \textbf{dwé} átsu:tí sóntô:ni.\\
%%LINE2
\gll
Paulo  dwe  a-tsul-ile  sonto-ni\\
%%LINE3
1.Paulo  only  1\SM{}-leave-\PFV{}  church-\LOC{}\\
%%TRANS1
\glt
‘Only Paulo went to church.’\\
%%TRANS2
}
%%EXEND

\z

%%EAX
\ea
%%JUDGEMENT
[\textsuperscript{\#}]{
%%LABEL
\label{bkm:Ref124149613}\label{bkm:Ref141347020}
%%CONTEXT
%%LINE1
\textbf{Mimvhúnja} \textbf{dwé} yíbê:te.\\
%%LINE2
\gll
mi-mvhunja  dwe  yi-bel-ile\\
%%LINE3
4-rabbit  only  4\SM{}-enter-\PFV{}\\
%%TRANS1
\glt
\textsuperscript{?}`Only the rabbits entered.’\\
%%TRANS2
‘The rabbits just entered (and didn’t do anything else).’
}
%%EXEND

\z

\subsection{Preverbal topics}
\label{bkm:Ref124153637}
We do typically find topics in the preverbal domain. Although active familiar topics are naturally expressed pronominally, for example by the object marker B's answer in \xref{bkm:Ref120091592}, topics may also be full NPs in the left periphery, as in \xref{bkm:Ref121987443}, where the action ‘to break a coconut’ is presented in the question, and occurs in the left periphery in the answer.

\ea
(Context: QUIS map task, with speaker A explaining the route to speaker B.)\\
\begin{xlist}
%%EAX
\exi{A:}
%%JUDGEMENT
%%LABEL

%%CONTEXT
%%LINE1
%%LINE2
\gll
Ni-mán-á  dí-kávhâ:lu.\\
%%LINE3
1\SG.\SM{}-find-\FV{}  5-horse\\
%%TRANS1
\glt
‘I find a horse.’\\
%%TRANS2
%%EXEND

%%EAX
\exi{B:}
%%JUDGEMENT
%%LABEL
\label{bkm:Ref120091592}
%%CONTEXT
%%LINE1
%%LINE2
\gll
Ní-\textbf{dí}{}-wô:n-i.\\
%%LINE3
1\SG.\SM{}-5\OM{}-see-\PFV{}\\
%%TRANS1
\glt
‘I have seen it.’
%%TRANS2
%%EXEND

\end{xlist}
\z

%%EAX
\ea
%%JUDGEMENT
%%LABEL
\label{bkm:Ref121987443}
%%CONTEXT
(How can I break a coconut?)\\
%%LINE1
\textbf{Kufá:ya} \textbf{ka} \textbf{dikho:kho}, kuni mamahelo mambí:dí.\\
%%LINE2
\gll
ku-faya  k-a  di-khokho  ku=ni  ma-mahelo  ma-mbidi\\
%%LINE3
15-break  15-\CONN{}  5-coconuts  17\SM{}=with  6-way  6-two\\
%%TRANS1
\glt
‘There are two ways to break coconuts.’\\
%%TRANS2
%%EXEND

\z

Contrastive topics are expressed by an NP in the left periphery, requiring an object marker if the preposed constituent is a human object. The contrast may be explicit, mentioning both of the contrasting referents, as in \xref{bkm:Ref141347619} and \xref{bkm:Ref141347629}, or implicit, as in \xref{bkm:Ref141347637}.

%%EAX
\ea
%%JUDGEMENT
%%LABEL
\label{bkm:Ref141347619}
%%CONTEXT
(What did grandma give the girls?)\\
%%LINE1
Áníngílé Lídíà mafá:ka, \textbf{Rozí:tá} \textbf{ní} \textbf{Laurí:nya} á\textbf{vá}níngá máphê:ra.\\
%%LINE2
\gll
a-ning-ile  Lidia  ma-faka  Rozita  ni Laurinya  a-va{}-ning-a  ma-phera\\
%%LINE3
1\SM{}-give-\PFV{}  1.Lidia  6-maize  1.Rozita  and  1.Laurinha  1\SM{}-2\OM{}-give-\FV{}  6-pear\\
%%TRANS1
\glt
‘She gave Lidia maize; to Rosita and Laurinha she gave pears.’\\
%%TRANS2
%%EXEND

\z

%%EAX
\ea
%%JUDGEMENT
%%LABEL
\label{bkm:Ref141347629}
%%CONTEXT
(Four of my siblings are girls.)\\
%%LINE1
\textbf{Vavámbî:dí} vóbhilívî:la\\
%%LINE2
\gll
va-va-mbidi  va-o-bhilivil-a\\
%%LINE3
2-?\CONN{}-two  2\SM{}-\PROG{}-be.light-\FV{}\\
%%TRANS1
\glt
‘Two are light-skinned’\\
%%TRANS2
%%EXEND

%%EAX
\sn
%%JUDGEMENT
%%LABEL
%%CONTEXT
%%LINE1
ní \textbf{vavambî:dí} vántí:má vo:kô:ma.\\
%%LINE2
\gll
ni  va-va-mbidi  va-ntima  va-a-ku-koma\\
%%LINE3
and  2-?\CONN{}-two  2-black  2-\CONN{}-15-be.short\\
%%TRANS1
\glt
‘and two (are) black and short.’
%%TRANS2
%%EXEND

\z

%%EAX
\ea
%%JUDGEMENT
%%LABEL
\label{bkm:Ref141347637}
%%CONTEXT
(Was it the house and the yard that s/he swept?)\\
%%LINE1
\textbf{Nyumbá:ni} ayáhiyê:la.\\
%%LINE2
\gll
nyumba-ini  a-ya-hiyel-a\\
%%LINE3
9.house-\LOC{}  1\SM{}-\NEG{}-sweep-\FV{}\\
%%TRANS1
\glt
‘S/he didn’t sweep the house.’\\
%%TRANS2
%%EXEND

\z

A topic may also be indicated by the locative preposition \textit{ka}, which is used when a subset of the initial topic referent is then selected in the comment, as illustrated in \xref{bkm:Ref120093070} and \xref{bkm:Ref120093071}.

%%EAX
\ea
%%JUDGEMENT
%%LABEL
\label{bkm:Ref120093070}
%%CONTEXT
(Are these people wearing hats? +QUIS picture of two women without hats and two men with hats.)\\
%%LINE1
\textbf{Ka} vá:thu váva, avá vákúni sígo:ko majǎ:ha.\\
%%LINE2
\gll
ka  va-thu  vava  ava  va=ku=ni  si-goko  ma-jaha\\
%%LINE3
\LOC{}  2-people  2.\DEM{}.\MED{}  2.\PRO{}  2\SM=\REL{}=with  8-hat  \COP{}.6-man\\
%%TRANS1
\glt
‘Of/between these people, the ones that have hats are the men.’\\
%%TRANS2
%%EXEND

\z

%%EAX
\ea
%%JUDGEMENT
%%LABEL
\label{bkm:Ref120093071}
%%CONTEXT
%%LINE1
\textbf{Ká} kó:ká ní fâ:nta, utósélá kô:ka dwé?\\
%%LINE2
\gll
ka  koka  ni  fanta  u-to-sel-a  koka  dwe\\
%%LINE3
\LOC{}  coke  and  fanta  2\SG.\SM{}-\TO{}-drink-\FV{}  coke  only\\
%%TRANS1
\glt
‘Between coke and fanta, did you drink only coke?’\\
%%TRANS2
%%EXEND

\z

When the topic shifts to a different referent, the new topic is also expressed as an NP in the preverbal domain, as in \xref{bkm:Ref121991624} from the Frog story, where, after a number of lines about the frogs, the boy is reactivated as the topic and subject and is marked by the distal demonstrative \textit{wule} (penultimate lengthening is not transcribed in this example).

%%EAX
\ea
%%JUDGEMENT
%%LABEL
\label{bkm:Ref121991624}
%%CONTEXT
(Moving on, they saw many more frogs, in addition to the two they had seen. Those big ones who saw them turned out to be the parents of the other frogs.  So, the big ones that they were seeing were jumping.)\\
%%LINE1
Sê, \textbf{m’pfáná} \textbf{wúlé} átsakí:lé já ángé adí díkhéle dá:kwé di angatídilá:va.\\
%%LINE2
\gll
se  m’-fana  wu-le  a-tsak-ile  ja  a-nge  a-di     di-khele  di-akwe  di  a-nga-ti-di-lav-a\\
%%LINE3
so  1-boy  1-\DEM.\DIST{}  1\SM{}-be.happy-\PFV{}  so  1\SM{}-say  \AUG{}-5.\DEM{} 5.\COP-frog  5-\POSS.1  5.\DEM{}  1\SM{}-\REL{}-\IPFV{}-5\OM{}-want-\FV{}\\
%%TRANS1
\glt
‘So, that boy was happy, so he said: this is his frog that he was looking for.’
%%TRANS2
%%EXEND

\z

Shift topics may be marked by a pronoun \mbox{-\textit{ona}}, as in \xref{bkm:Ref121989506}. In combination with \textit{ni} ‘and’, it expresses that there is another referent apart from the topic referent for which the predicate also holds: \textit{ni yona} ‘and him/her too’, as in \xref{bkm:Ref121989699}. Both examples are from a recounting of the Frog story.

%%EAX
\ea
%%JUDGEMENT
%%LABEL
\label{bkm:Ref121989506}
%%CONTEXT
(The boy stood there in pain.)\\
%%LINE1
Sê bhonyáni \textbf{yó:ná} yítsú:té.\\
%%LINE2
\gll
se  bhonyani  y-ona  yi-tsul-ile\\
%%LINE3
and  9.mouse  9-\PRO{}  9\SM{}-leave-\PFV{}\\
%%TRANS1
\glt
‘As for the mouse, it is gone.’\\
%%TRANS2
%%EXEND

\z

%%EAX
\ea
%%JUDGEMENT
%%LABEL
\label{bkm:Ref121989699}
%%CONTEXT
(He turned the table over to see if by chance it would be there under the table and saw that it wasn’t.)\\
%%LINE1
Sê, yí:mbwá \textbf{ni} \textbf{yó:ná}, yapfététéla makó:tá to khé:né, mákotákotá yáwá ingáwá cídíhúmé:té kúyá há:yi. \\
%%LINE2
\gll
se  yi-mbwa  \textbf{ni}  yi-ona  yi-a-pf-etetel-a  ma-kota  to.khene ma-kota$\sim$kota  y-awa  ingawa  ci-di-hum-ete  ku-ya  hayi \\
%%LINE3
then  9-dog  and  9-\PRO{}  9\SM{}-\PRS{}-hear-\STAT.\APPL{}-\FV{}  6-cry  \COMP{} 6-cry$\sim$\RED{}  6.\DEM{}  could  7\SM{}-\DEP{}-leave-\PFV{}  15-go  where \\
%%TRANS1
\glt
‘Then the dog too is hearing the cry to see where the cry comes from.’\\
%%TRANS2
%%EXEND

\z

In shifting from one to the next event, Cicopi also uses tail-head linking, whereby an event mentioned in (the final part of) one sentence is repeated at the start of the next sentence. This is illustrated in \xref{bkm:Ref121989204} from the Frog story, where the action of scratching the nose is first repeated in the next sentence before moving the story forward.

%%EAX
\ea
%%JUDGEMENT
%%LABEL
\label{bkm:Ref121989204}
%%CONTEXT
%%LINE1
I bhónyá:ní yíku múlé:yá \textbf{yíngámúnyára} \textbf{thó:mbvú}.\\
%%LINE2
\gll
i  bhonyani  yi=ku  mu-leya  yi-nga-mu-nyar-a  thombvu\\
%%LINE3
\COP{}  9.mouse  9\SM{}=\REL{}  18-\DEM.\DIST{}  9\SM{}-\REL{}-1\OM{}-scratch-\FV{}  9.nose\\
%%TRANS1
\glt
‘It was a bush mouse that is there that scratched him on the nose.’\\
%%TRANS2
%%EXEND

%%EAX
\sn
%%JUDGEMENT
%%LABEL
%%CONTEXT
%%LINE1
Já \textbf{yídímúrányílé} \textbf{thó:mbvú}, e:ne atósá:lá acítín’o:té thó:mbvú yílé inkú:pfá kúpá:ndá. \\
%%LINE2
\gll
ja  yi-di-mu-rany-il-e  thombvu  ene  a-to-sal-a  a-ci-ti-n’ol-ile  thombvu  yi-le  in-ku-pfa  kupanda  \\
%%LINE3
so  9\SM{}-\DEP{}-1\OM{}-scratch-\PFV{}  9.nose  1.\PRO{}  1\SM{}-\TO{}-stay-\FV{}  1\SM{}-\CON{}-\IPFV{}-touch-\PFV{}  9.nose  9-\DEM.\MED{}  \LINK{}-15-feel   15.pain\\
%%TRANS1
\glt
‘After scratching him on the nose, he stood there touching his nose in so much pain he felt.’
%%TRANS2
%%EXEND

\z

There can also be multiple topics in the left periphery, be they arguments or scene-setting adverbs, as illustrated in \xref{bkm:Ref122335950} and \xref{bkm:Ref146270227}. The object \textit{m’pawu} ‘cassava’ in \xref{bkm:Ref122335950} can be seen as a secondary topic here (as defined by \citealt{DalrympleNikolaeva2011}), meaning that “the utterance is construed to be about the relation that holds between it and the primary topic” \citep[2]{Nikolaeva2001}.

%%EAX
\ea
%%JUDGEMENT
%%LABEL
\label{bkm:Ref122335950}
%%CONTEXT
(What did mother do with the cassava?)\\
%%LINE1
Má:mí, m’pá:wú, axávísí:lé ayáwǔ:dya.\\
%%LINE2
\gll
Mami  m’-pawu  a-xavis-ile  a-ya-wu-dy-a\\
%%LINE3
1.mother  3-cassava  1\SM{}-sell-\PFV{}  1\SM{}-\NEG{}-3\OM{}-eat-\FV{}\\
%%TRANS1
\glt
‘Mother, the cassava, she sold (it), she didn’t eat (it).’\\
%%TRANS2
%%EXEND

\z

%%EAX
\ea
%%JUDGEMENT
%%LABEL
\label{bkm:Ref146270227}
%%CONTEXT
(Context: When in Gaza, someone comments: you do not have mangos here. A person who knows the area better replies.)\\
%%LINE1
Mámá:ngá, ~Chidéngué:lé, ka Gá:za mâ:ngi.\\
%%LINE2
\gll
ma-manga  Chidenguele  ka  Gaza  ma-ngi\\
%%LINE3
6-mango  Chidenguele  \LOC{}  Gaza  \COP{}.6{}-many\\
%%TRANS1
\glt
‘Mangos, in Chidenguele in Gaza, there are many.’\\
%%TRANS2
%%EXEND

\z

As subjects are typical topics, and canonically appear preverbally, it can be difficult to see in which (structural) position they are. The pronominal subject \textit{e:ne} ‘s/he’ in \xref{bkm:Ref141350066}, repeated from \xref{bkm:Ref127266023} above, is in a left-peripheral position, separated from the verb by the PP ‘by habit’.

%%EAX
\ea
%%JUDGEMENT
%%LABEL
\label{bkm:Ref141350066}
%%CONTEXT
%%LINE1
E:né ngu cíhê:ne ádyá mpû:nga.\\
%%LINE2
\gll
ene  ngu  ci-hene  a-dy-a  mpunga\\
%%LINE3
1.\PRO{}  \PREP{}  7-habit  1\SM{}-eat-\FV{}  3.rice\\
%%TRANS1
\glt
‘S/he habitually eats rice.’\\
%%TRANS2
%%EXEND

\z

However, the preverbal subject can also be an indefinite, as in \xref{bkm:Ref141350076}, which cannot be left-dislocated and hence provides evidence for a non-topical subject position as well. A systematic study of the prosodic properties of subjects (and other constituents) in the preverbal domain may well shine light on the syntactic status and marking of the different positions.

%%EAX
\ea
%%JUDGEMENT
%%LABEL
\label{bkm:Ref141350076}
%%CONTEXT
%%LINE1
I:nthu áréthémúkí:lé â:wa.\\
%%LINE2
\gll
n-thu  a-rethemuk-ile  a-w-a\\
%%LINE3
1-person  1\SM{}-slip-\PFV{}  1\SM{}-fall-\FV{}\\
%%TRANS1
\glt
‘Someone slipped and fell.’\\
%%TRANS2
%%EXEND

\z

The left periphery also hosts scene-setting topics, such as the adverbial phrases \textit{ahá ká bhasíkêni} ‘at the bicycle’ in \xref{bkm:Ref121993048}, and in \xref{bkm:Ref121993050} the phrases \textit{ití ní wúsíkú dímwání dítshíkú} ‘one day at night’ and \textit{inkama angadipfuxela} ‘the time that he visited him’. Each topic in these examples is indicated by square brackets.

%%EAX
\ea
%%JUDGEMENT
%%LABEL
\label{bkm:Ref121993048}
%%CONTEXT
%%LINE1
[Ahá ká bhasíkê:ni], nákwélé:lá ngu cíne:ne.\\
%%LINE2
\gll
[aha  ka  bhasikeni]  n-a-kwel-el-a  ngu  cinene\\
%%LINE3
16.\DEM.\PROX{}  \LOC{}  5.bicycle  1\SM{}-\PRS{}-go.up-\APPL{}-\FV{}  \PREP{}  7.right\\
%%TRANS1
\glt
‘From the bicycle here I go up to the right.’\\
%%TRANS2
%%EXEND

 
\z
\pagebreak

%%EAX
\ea
%%JUDGEMENT
%%LABEL
\label{bkm:Ref121993050}
%%CONTEXT
%%LINE1
{}[Ití ní wúsí:kú dímwányání dítshí:kú], [ínkámá angádípfúxê:lá], díkhê:lé dítíngádí kô:ná múlé:yá ndání ka díbhójhé:lá. \\
%%LINE2
\gll
{}[i-ti  ni  wusiku  di-mwanyani  di-tshiku]  [in-kama  a-nga-di-pfuxel-a]  di-khele     di-ti-nga-di  kona  mu-leya  ndani  ka  di-bhojhela \\
%%LINE3
{\db}\COP{}-\PST{}  and  night  5-other  5-day  {\db}9-time  1\SM-\REL{}-5\OM{}-visit-\FV{}  5-frog 5\SM{}-\IPFV{}-\NEG{}-be  17.\PRO{}  18-\DEM.\DIST{}  inside  \LOC{}  5-bottle \\
%%TRANS1
\glt
‘One day at night, visiting him, the frog was not inside the bottle.’\\
%%TRANS2
%%EXEND

\z

As further illustration of topics, the following example \xref{bkm:Ref146270541} shows a scene-setting topic (\textit{cibhabha ca mina} ‘to my left’), and in the reaction a contrastive topic (\textit{ani} ‘me’, as opposed to you), with the two contrastive topics indicated in boldface. The exchange also shows a contrastive focus with \textit{tihomu} ‘bulls’ being contrasted to \textit{timbwa} ‘dogs’, in italics in the example.

\ea
(Context: QUIS map task, where speaker A explains the route to speaker B.)\\
\begin{xlist}
%%EAX
\exi{A:}
%%JUDGEMENT
%%LABEL
\label{bkm:Ref146270541}
%%CONTEXT
%%LINE1
\textbf{Cibhabha} \textbf{ca} \textbf{mî:ná} nimaná \textit{tî:mbwa tírâ:ru}.\\
%%LINE2
\gll
ci-bhabha  ci-a  mina  ni-man-a  \textit{ti-mbwa}  \textit{ti-raru}\\
%%LINE3
7-left  7-\CONN{}  1\SG.\PRO{}  1\SG.\SM{}-find-\FV{}  10-dog  10-three\\
%%TRANS1
\glt
‘To my left, I find three dogs.’\\
%%TRANS2
%%EXEND

%%EAX
\exi{B:}
%%JUDGEMENT
%%LABEL
%%CONTEXT
%%LINE1
\textbf{A:ni} niwóná \textit{tíhô:mu tí-râ:ru}.\\
%%LINE2
\gll
ani  ni-won-a  \textit{ti-homu  ti-raru}\\
%%LINE3
1\SG.\PRO{}  1\SG.\SM{}-see-\FV{}  10-bull  10-three\\
%%TRANS1
\glt
‘As for me, I see three bulls.’
%%TRANS2
%%EXEND

\end{xlist}
\z

In summary, Cicopi does not seem to allow preverbal foci (but ‘only’ remains unclear) and prefers preverbal topics. There are multiple positions in the left periphery to which subjects and non-subjects can move, one of which is a non-topical subject position. In the next section, we turn our attention to the postverbal domain. 

\subsection{No dedicated focus position}
\label{bkm:Ref121988262}
As focused constituents cannot appear in the preverbal domain, they must appear postverbally, or in a cleft (see \sectref{bkm:Ref121993097}). This is again shown for the correction in \xref{bkm:Ref146270704}, where the focused object simply appears in a postverbal position.

\ea
(Context: QUIS picture of a girl pulling a chair.)\\
\begin{xlist}
%%EAX
\exi{A:}
%%JUDGEMENT
%%LABEL
\label{bkm:Ref146270704}
%%CONTEXT
%%LINE1
Mwanáná ándíndá mé:za?\\
%%LINE2
\gll
mw-anana  a-ndind-a  meza\\
%%LINE3
1-child  1\SM{}-pull-\FV{}  3.table\\
%%TRANS1
\glt
‘Is the child pulling a/the table?’\\
%%TRANS2
%%EXEND

%%EAX
\exi{B:}
%%JUDGEMENT
%%LABEL
%%CONTEXT
%%LINE1
Ihî:mhím, (mwaná:ná) ándíndá \textbf{cítu:lu}.\\
%%LINE2
\gll
ihimhim  mw-anana  a-ndind-a  ci-tulu\\
%%LINE3
no  1-child  1\SM{}-pull-\FV{}  7.chair\\
%%TRANS1
\glt
‘No, the child is pulling a/the chair.’
%%TRANS2
%%EXEND

\end{xlist}
\z

While focus is restricted to the postverbal domain in a non-clefted sentence, there is no dedicated position for focus in Cicopi. Within the postverbal domain, there is no requirement for the focused element to appear adjacent to the verb (as in various other zone S languages, e.g. \citealt{Buell2009} for Zulu) nor phrase-finally (like in Kirundi, see \cite{chapters/kirundi}). We show this for the Theme and Recipient arguments in a ditransitive: either position can host a question word, as shown in \xref{bkm:Ref121149061} and \xref{bkm:Ref121149063}, either can form the answer to a content question, as in \xref{bkm:Ref121149077} and \xref{bkm:Ref121149078}.

\ea\label{bkm:Ref121149061}Recipient question

  %%EAX
\ea
%%JUDGEMENT
%%LABEL
%%CONTEXT
%%LINE1
Vánánîngá másenórá ma:ní?\\
%%LINE2
\gll
va-na-ning-a  ma-senora  mani\\
%%LINE3
2\SM{}-\FUT{}-give-\FV{}  6-carrot  who\\
%%TRANS1
\glt
‘Who will they give carrots?’\\
%%TRANS2
%%EXEND


  %%EAX
\ex
%%JUDGEMENT
%%LABEL
%%CONTEXT
%%LINE1
Hinán\textsuperscript{!}íngá ma:ní díkáne:ka?\\
%%LINE2
\gll
hi-na-ning-a  mani  di-kaneka\\
%%LINE3
1\PL.\SM{}-\FUT{}-give-\FV{}  who  5-mug\\
%%TRANS1
\glt
‘Who will we give the mug to?’\\
%%TRANS2
%%EXEND

\z
\z

\ea\label{bkm:Ref121149063}Theme question

  %%EAX
\ea
%%JUDGEMENT
%%LABEL
%%CONTEXT
%%LINE1
Vánáníngá ndíyâ:wé câ:ni?\\
%%LINE2
\gll
va-na-ning-a  ndiya-awe  cani\\
%%LINE3
2\SM{}-\FUT{}-give-\FV{}  1.sister-\POSS{}.1  what\\
%%TRANS1
\glt
‘What will they give his sister?’\\
%%TRANS2
%%EXEND


  %%EAX
\ex
%%JUDGEMENT
%%LABEL
%%CONTEXT
%%LINE1
Váná(mú)n\textsuperscript{!}íngá cá:ní ndíyâ:wé?\\
%%LINE2
\gll
va-na-mu-ning-a  cani  ndiya-awe\\
%%LINE3
2\SM{}-\FUT{}-1\OM{}-give-\FV{}  what  1.sister-\POSS{}.1\\
%%TRANS1
\glt
‘What will they give his sister?’\\
%%TRANS2
%%EXEND

\z
\z

%%EAX
\ea
%%JUDGEMENT
%%LABEL
\label{bkm:Ref121149077}
%%CONTEXT
(Who will they give carrots?)\\
%%LINE1
Váná(*mu)nî:ngá (másenó:rá) Gô:mex.\\
%%LINE2
\gll
va-na-mu-ning-a  ma-senora  Gomes\\
%%LINE3
2\SM{}-\FUT{}-1\OM{}-give-\FV{}  6-carrots  1.Gomes\\
%%TRANS1
\glt
‘They will give Gomes carrots.’\\
%%TRANS2
%%EXEND

\z

%%EAX
\ea
%%JUDGEMENT
%%LABEL
\label{bkm:Ref121149078}\label{bkm:Ref141359654}
%%CONTEXT
(Will you cook rice for the visitors?)\\
%%LINE1
Ninábhikélá vapfumba mba:ba.\\
%%LINE2
\gll
ni-ná-bhik-el-a  va-pfumba  mbaba\\
%%LINE3
1\SG.\SM{}-\FUT{}-cook-\APPL{}-\FV{}  2-visitors  3.shima\\
%%TRANS1
\glt
‘I will cook shima for the visitors.’\\
%%TRANS2
%%EXEND

\z

Interrogative adverbs can also appear in either position, as illustrated for \textit{njani} ‘how’ and \textit{hayi} ‘where’ in \xref{bkm:Ref141356959} and \xref{bkm:Ref141356961}, respectively.

\ea\label{bkm:Ref141356959}
%%EAX
\ea
%%JUDGEMENT
%%LABEL
%%CONTEXT
%%LINE1
Álúngisile \textbf{nja:ní} mo:vha?\\
%%LINE2
\gll
a-lungis-ile   njani  movha\\
%%LINE3
1\SM{}-repair-\PFV{}  how  3.car\\
%%TRANS1
\glt
‘How did s/he repair the car?’\\
%%TRANS2
%%EXEND


  \ex  Álúngisile mo:vha \textbf{nja:ní}?
\z
\z

\ea\label{bkm:Ref141356961}
%%EAX
\ea
%%JUDGEMENT
%%LABEL
%%CONTEXT
%%LINE1
Urumété \textbf{ha:yi} mípâ:wu?\\
%%LINE2
\gll
u-rum-el-ile  hayi  mi-pawu\\
%%LINE3
2\SG.\SM{}-send-\APPL{}-\PFV{}  where  4-cassava\\
%%TRANS1
\glt
‘Where did you send the cassava?’\\
%%TRANS2
%%EXEND


  \ex  Urumété mípâ:wu \textbf{ha:yi}?
\z
\z

In the same way, phrases modified by the exhaustive focus particle \textit{dwe} ‘only’ can also appear in either position, as in \xref{bkm:Ref121149095} and \xref{bkm:Ref121149097}.\footnote{The logical possibilities are judged as follows, with bold indicating the focus (as diagnosed by contrast in a following clause):\\
\begin{tabular}{ll}
\textbf{Theme} dwe Recipient &  Theme dwe \textbf{Recipient}\\
\textbf{\textsuperscript{\#}Theme} Recipient dwe & Theme \textbf{Recipient} dwe\\
Recipient \textbf{Theme} dwe & \textsuperscript{\#}Recipient Theme dwe\\
Recipient dwe \textbf{Theme} &  \textbf{Recipient} dwe Theme\\
\end{tabular}
% \gllll
% {\textbf{Theme} dwe Recipient} {Theme dwe \textbf{Recipient}}\\
% {\textbf{\textsuperscript{\#}Theme} Recipient dwe } {Theme \textbf{Recipient} dwe}\\
% {Recipient \textbf{Theme} dwe } {\textsuperscript{\#}Recipient Theme dwe}\\
% {Recipient dwe \textbf{Theme}} {\textbf{Recipient} dwe Theme}\\
}
The prosodic break in \xref{bkm:Ref121149095:a} and \xref{bkm:Ref121149097:b} could be indicative of a requirement that the focus be final in some constituent, but further research into prosodic phrasing (and its relation with syntactic phrasing) is necessary to confirm this.

\ea\label{bkm:Ref121149095}
%%EAX
\ea
%%JUDGEMENT
%%LABEL
\label{bkm:Ref121149095:a}
%%CONTEXT
\textbf{Theme} only, Recipient \\
%%LINE1
Vánáningá mábho:mú dwé, tshándza:na -- vánambímún\textsuperscript{!}íngá cíkê:ta.\\
%%LINE2
\gll
va-na-ning-a  ma-bhomu  dwe  tshandzana  va-na-mbi-mu-ning-a  ci-keta\\
%%LINE3
2\SM{}-\FUT{}-give-\FV{}  6-lemons  only  1.niece 2\SM{}-\FUT{}-\NEG{}-1\OM{}-give-\FV{}  7-pineapple\\
%%TRANS1
\glt
‘They will give only lemons to the niece, they will not give her pineapple.’
%%TRANS2
%%EXEND

%%EAX
\ex
%%JUDGEMENT
%%LABEL
\label{bkm:Ref121149095:b}
%%CONTEXT
Recipient \textbf{Theme} only\\
%%LINE1
Vánáningá tshándzá:na mábho:mu dwé -- vánambímún\textsuperscript{!}íngá cíkê:ta.\\
%%LINE2
\gll
va-na-ning-a  tshandzana  ma-bhomu  dwe  va-na-mbi-mu-ning-a  ci-keta\\
%%LINE3
2\SM{}-\FUT{}-give-\FV{}  1.niece  6-lemons  only  2\SM{}-\FUT{}-\NEG{}-1\OM{}-give-\FV{}  7-pineapple\\
%%TRANS1
\glt
‘They will give only lemons to the niece, they will not give her pineapple.’
%%TRANS2
%%EXEND

\z
\z

\ea\label{bkm:Ref121149097}
%%EAX
\ea
%%JUDGEMENT
%%LABEL
\label{bkm:Ref121149097:a}
%%CONTEXT
Theme \textbf{Recipient} only\\
%%LINE1
Vánáningá mábho:mú tshándza:na dwé -- vánamb\textsuperscript{!}íníngá ndíyâ:we.\\
%%LINE2
\gll
va-na-ning-a  ma-bhomu  tshandzana  dwe     va-na-mbi-ning-a  ndiya-awe\\
%%LINE3
2\SM{}-\FUT{}-give-\FV{}  6-lemon  1.niece  only  2\SM{}-\FUT{}-\NEG{}-give-\FV{}  1.sister-\POSS{}.1\\
%%TRANS1
\glt
‘They will give lemons to the niece only, they will not give to her sister.’
%%TRANS2
%%EXEND


%%EAX
\ex
%%JUDGEMENT
%%LABEL
\label{bkm:Ref121149097:b}
%%CONTEXT
\textbf{Recipient} only, Theme\\
%%LINE1
Vánáningá tshándzá:na dwé, mábho:mu -- vánamb\textsuperscript{!}íníngá ndíyâ:we.\\
%%LINE2
\gll
va-na-ning-a  tshandzana  dwe  ma-bhomu    va-na-mbi-ning-a  ndiya-awe\\
%%LINE3
2\SM{}-\FUT{}-give-\FV{}  1.niece  only  6-lemon 2\SM{}-\FUT{}-\NEG{}-give-\FV{}  1.sister-\POSS{}.1\\
%%TRANS1
\glt
‘They will give lemons to the niece only, they will not give to her sister.’
%%TRANS2
%%EXEND

\z
\z

In fact, Cicopi allows multiple content question words in the postverbal domain, as illustrated in \xref{bkm:Ref122338299}.
\pagebreak

%%EAX
\ea
%%JUDGEMENT
%%LABEL
\label{bkm:Ref122338299}
%%CONTEXT
%%LINE1
Vhalério áxávheté cá:ní mâ:ni?\\
%%LINE2
\gll
Vhalerio  a-xavh-el-ile  cani  mani\\
%%LINE3
1.Valerio  1\SM{}-buy-\APPL{}-\PFV{}  what  who\\
%%TRANS1
\glt
‘Who did Valerio buy what?’\\
%%TRANS2
%%EXEND

\z

For non-arguments, such multiple questions are also possible, and the word order is flexible, as shown in \xref{bkm:Ref141356914} and \xref{bkm:Ref141356920}.

\ea\label{bkm:Ref141356914}
%%EAX
\ea
%%JUDGEMENT
%%LABEL
%%CONTEXT
%%LINE1
Álúngisilé câ:ni nja:ní?\\
%%LINE2
\gll
a-lungis-ile  cani  njani\\
%%LINE3
1\SM{}-repair-\PFV{}  what  how\\
%%TRANS1
\glt
‘How did s/he repair what?’\\

%%EXEND

  %%EAX
\ex
%%JUDGEMENT
%%LABEL
%%CONTEXT
%%LINE1
Álúngisilá nja:ní câ:ni?\\
%%LINE2
\gll
a-lungis-ile  njani  cani\\
%%LINE3
1\SM{}-repair-\PFV{}  how  what\\
%%TRANS1
\glt
‘How did s/he repair what?’\\

%%EXEND

\z
\z

\ea\label{bkm:Ref141356920}
%%EAX
\ea
%%JUDGEMENT
%%LABEL
%%CONTEXT
%%LINE1
Pédrú ámáné c\textsuperscript{!}á:ní aha:yi?\\
%%LINE2
\gll
Pedro  a-man-e  cani  ahayi\\
%%LINE3
1.Pedro  1\SM{}-find-\PFV{}  what  where\\
%%TRANS1
\glt
‘What did Pedro find where?’\\
%%TRANS2
%%EXEND


  %%EAX
\ex
%%JUDGEMENT
%%LABEL
%%CONTEXT
%%LINE1
Pédrú ámáné ha:yí ca:ni?\\
%%LINE2
\gll
Pedro  a-man-e  hayi  cani\\
%%LINE3
1.Pedro  1\SM{}-find-\PFV{}  where  what\\
%%TRANS1
\glt
‘What did Pedro find where?’\\
%%TRANS2
%%EXEND

\z
\z

Note, however, that multiple focus seems to be possible only for question words and not for other focus constituents, as only one postverbal constituent can be modified by \textit{dwe} ‘only’, as seen in the ungrammaticality of example \xref{bkm:Ref124149352}.

%%EAX
\ea
%%JUDGEMENT
[*]{
%%LABEL
\label{bkm:Ref124149352}
%%CONTEXT
%%LINE1
Nigǒndísílé vanáná dwé cíco:pi dwé.\\
%%LINE2
\gll
ni-gond-is-ile  va-nana  dwe  ci-copi  dwe\\
%%LINE3
1\SG.\SM{}-read-\CAUS{}-\PFV{}  2-child  only  7-copi  only\\
%%TRANS1
\glt
int. ‘I taught only (the) children only Cicopi.’\\
%%TRANS2
}
%%EXEND

\z

Therefore, while focus appears in the postverbal domain in Cicopi, there is no dedicated focus position for objects or adverbs. We now turn to postverbal subjects, which work slightly differently.

\subsection{Subject inversion}
\label{bkm:Ref121988307}
The postverbal domain also hosts the logical subject in subject inversion constructions. These are constructions in which the logical subject appears postverbally and a possible topic appears in a preverbal position (see \citealt{MartenvanderWal2014} for an overview of subject inversion constructions in Bantu). Cicopi has two such constructions: Agreeing Inversion and Default Agreement Inversion (DAI), discussed in more detail below. Other possible inversion constructions are not accepted in Cicopi, such as Locative Inversion and Patient Inversion, in which respectively a locative or theme constituent appears preverbally and determines subject marking on the verb. Their unacceptability is shown in \xref{bkm:Ref141358804} for Locative Inversion,\footnote{Locative Inversion is unacceptable with or without applicative morphology} and in \xref{bkm:Ref141358815} for Patient Inversion.

%%EAX
\ea
%%JUDGEMENT
[*]{
%%LABEL
\label{bkm:Ref141358804}
%%CONTEXT
%%LINE1
Ntini mulé múbéte Joau.\\
%%LINE2
\gll
in-t-ini  mu-le  mu-bet-e  Joao\\
%%LINE3
3-home-\LOC{}  18-\DEM{}.\MED{}  18\SM{}-enter-\PFV{}  1.Joao\\
%%TRANS1
\glt
‘In that home/compound entered João.’\\
%%TRANS2
}
%%EXEND

%%EAX
\ex
%%JUDGEMENT
[]{
%%LABEL
\label{bkm:Ref141358815}
%%CONTEXT
(Context: QUIS picture of a girl pulling a chair.)\\
%%LINE1
Cítúlú cí cindí:nda mwáná:ná.\\
%%LINE2
\gll
ci-tulu  ci  ci-ndind-a  mw-anana\\
%%LINE3
7-chair  7.\DEM{}.\PROX{}  7\SM{}-pull-\FV{}  1-child\\
%%TRANS1
\glt
\textsuperscript{?} ‘The chair is pulling the child.’\\
%%TRANS2
* `The child is pulling the chair.’
}
%%EXEND

\z

In Agreeing Inversion, the subject marker on the verb agrees in noun class with the postverbal logical subject, as illustrated in \xref{bkm:Ref120691553} where the postverbal \textit{mafaka} ‘maize’ determines the subject marker in noun class 6.

%%EAX
\ea
%%JUDGEMENT
%%LABEL
\label{bkm:Ref120691553}
%%CONTEXT
%%LINE1
Mapháyî:lwe mafáká thémbwe:ni?\\
%%LINE2
\gll
ma-phay-il-w-e  ma-faka  themw-ini\\
%%LINE3
6\SM{}-sow-\APPL{}-\PASS{}-\PFV{}  6-maize  field-\LOC{}\\
%%TRANS1
\glt
‘Was maize sown in the field?’\\
%%TRANS2
%%EXEND

\z

Agreeing Inversion can be used in a corrective context, as in \xref{bkm:Ref146271618}, and without necessary exclusion of alternatives (given that the additive \textit{ni} `and/also' can be used).\largerpage[-1]\pagebreak

%%EAX
\ea
%%JUDGEMENT
%%LABEL
\label{bkm:Ref146271618}
%%CONTEXT
(\textit{Mbvhuta yidya senora?} `Does the sheep eat carrot?')\\
%%LINE1
Ihî:mhím hingá mbvhú:tá dwé yídyâ:ku, \textbf{yí:dyá} \textbf{ní} \textbf{fu:tu}.\\
%%LINE2
\gll
ihimhim  hinga  mbvhuta  dwe  yi-dy-a=ku  yi-dy-a  ni  futu\\
%%LINE3
no  \COP.\NEG{}  9.sheep  only  9\SM{}-eat-\FV=\REL{}  9\SM{}-eat-\FV{}  and  9.tortoise\\
%%TRANS1
\glt
‘No, it’s not only the sheep; the tortoise also eats.’\\
%%TRANS2
%%EXEND

\z

It is unclear at this point which form of the verb is acceptable in Agreeing Inversion (see \sectref{bkm:Ref141358518} on conjoint and disjoint verb forms), and which types of predicates can occur. Further research, for example on the scope of the postverbal subject relative to negation, is needed to establish the underlying structure of Agreeing Inversion in Cicopi, as well as its precise interpretation and use.

The second subject inversion construction, Default Agreement Inversion (DAI), is possible in Cicopi with unaccusative \xref{bkm:Ref127259247}, unergative \xref{bkm:Ref127259301}, passive, and transitive \xref{bkm:Ref127259279} predicates. The subject marker in this construction is insensitive to the person/number/gender features of the subject, remaining a default class 17, and the postverbal subject can be a question word or an answer, as illustrated in the following examples. The interpretation is discussed below.

\ea\label{bkm:Ref127259247}
%%EAX
\ea
%%JUDGEMENT
%%LABEL
%%CONTEXT
%%LINE1
Kúwî:té ma:ní?\\
%%LINE2
\gll
ku-w-ile   mani\\
%%LINE3
17\SM{}-fall-\PFV{}  who\\
%%TRANS1
\glt
‘Who fell?’\\
%%TRANS2
%%EXEND


  %%EAX
\ex
%%JUDGEMENT
%%LABEL
%%CONTEXT
%%LINE1
Kúwí:te Marî:ya.\\
%%LINE2
\gll
ku-w-ile  Mariya\\
%%LINE3
17\SM{}-fall-\PFV{}  1.Maria\\
%%TRANS1
\glt
‘It was Maria who fell.’\\
%%TRANS2
%%EXEND

\z
\z

\ea\label{bkm:Ref127259279}(Context: QUIS picture of a girl pulling a chair.)

  %%EAX
\ea
%%JUDGEMENT
%%LABEL
\label{bkm:Ref127259279:a}
%%CONTEXT
%%LINE1
Kúndíndá ma:ni cítu:lu? \jambox*{[VSO]}
%%LINE2
\gll
ku-ndind-a  mwani  ci-tulu\\
%%LINE3
17\SM{}-pull-\FV{}  who  7-chair\\
%%TRANS1
\glt
‘Who is pulling the chair?’\\
%%TRANS2
%%EXEND


  %%EAX
\ex
%%JUDGEMENT
%%LABEL
\label{bkm:Ref127259279:b}
%%CONTEXT
%%LINE1
Kúndíndá mwán\textsuperscript{!}á:ná cítu:lu. \jambox*{[VSO]}
%%LINE2
\gll
ku-ndind-a  mw-anana  ci-tulu\\
%%LINE3
17\SM{}-pull-\FV{}  1-child  7-chair\\
%%TRANS1
\glt
‘A child is pulling the chair.’\\
%%TRANS2
(also OK as answer to ‘Is a dog pulling the chair?’)

%%EXEND

\z
\z

DAI can have a thetic interpretation (“out of the blue”, see \citealt{chapters/intro} and \citealt{Sasse1996,Sasse2006}), or narrow subject focus. The thetic use is illustrated in \xref{bkm:Ref121990638} and \xref{bkm:Ref121990820} with an existential/presentational or announcing function.

%%EAX
\ea
%%JUDGEMENT
%%LABEL
\label{bkm:Ref121990638}
%%CONTEXT
(Frog story)\\
%%LINE1
Ká:sí kúvéní híngá n’dó:ngá wówó:má wúkú hálé. \\
%%LINE2
\gll
kasi  kuveni  hinga  n’-donga  w-a-woma  wu-ku  hale\\
%%LINE3
but  [adv]  \COP.\NEG{}  3-tree  3-\CONN{}-dry  3\SM=\REL{}  16.\DEM.\MED{} \\
%%TRANS1
\glt
‘But, after all, it isn’t a dry tree that is there.’\\
%%TRANS2
%%EXEND

%%EAX
\sn
%%JUDGEMENT
%%LABEL
%%CONTEXT
%%LINE1
\textbf{Kútíní} \textbf{cíhâ:rí} \textbf{címwé:cô} vácídhánâká ku ínyá:rá, cíhá:rí cí cíngáyáé:má. \\
%%LINE2
\gll
ku-ti=ni  ci-hari  ci-mweco  va-ci-dhan-ak-a=ku  i-nyara  ci-hari ci  ci-nga-ya-em-a\\
%%LINE3
17\SM{}-\IPFV{}=with  7-animal  7\SM{}-one  2\SM-7\OM{}-call-\DUR{}-\FV{}=\REL{}  9-buffalo  7\SM{}-animal  7.\DEM.\PROX{}  7\SM{}-\REL{}-go-stand-\FV{} \\
%%TRANS1
\glt
‘There was an animal called a buffalo that was standing.’
%%TRANS2
%%EXEND

%%EAX
\ex
%%JUDGEMENT
%%LABEL
\label{bkm:Ref121990820}\label{bkm:Ref127259301}
%%CONTEXT
%%LINE1
Kúná ndzú:mà.\\
%%LINE2
\gll
ku-n-a  ndzuma\\
%%LINE3
17\SM{}-rain-\FV{}  9.rain\\
%%TRANS1
\glt
‘It's raining.’
%%TRANS2
%%EXEND

\z

The interpretation with narrow focus on the postverbal subject is illustrated in \xref{bkm:Ref121990826} and \xref{bkm:Ref121990832} with the exhaustive particle \textit{dwe} ‘only’.\largerpage

%%EAX
\ea
%%JUDGEMENT
%%LABEL
\label{bkm:Ref121990826}
%%CONTEXT
%%LINE1
Kúhókile mwaná:ná dwe.\\
%%LINE2
\gll
ku-hok-ile  mw-anana  dwe\\
%%LINE3
17\SM{}-arrive-\PFV{}  1-child  only\\
%%TRANS1
\glt
‘Only the child arrived.’\\
%%TRANS2
%%EXEND

%%EAX
\ex
%%JUDGEMENT
%%LABEL
\label{bkm:Ref121990832}
%%CONTEXT
%%LINE1
Kútútumilé Gó:mes dwé.\\
%%LINE2
\gll
ku-tutum-ile  Gomes  dwe\\
%%LINE3
17\SM{}-run-\PFV{}  1.Gomes  only\\
%%TRANS1
\glt
‘Only Gomes ran.’\\
%%TRANS2
%%EXEND

\z

With a transitive verb, narrow subject focus is the only interpretation that is allowed for DAI, regardless of the word order: both VOS and VSO are acceptable but only with subject focus, as indicated in the contexts for \xref{bkm:Ref136287722} and \xref{bkm:Ref121990860}; see also in \xref{bkm:Ref127259279:b} above.\footnote{Note, though, that there is a pause between O and S in the VOS order. As mentioned, more research is needed to understand the role of prosody.} This is in line with what \citet{CarstensMletshe2015} found for Xhosa VSO transitive expletive constructions.

%%EAX
\ea
%%JUDGEMENT
%%LABEL
\label{bkm:Ref136287722}
%%CONTEXT
(Context 1, subject focus: I can see that someone is filling the water containers and ask: who is drawing water? \\
\textsuperscript{\#}Context 2, object focus: What is Nelsa drawing?)\\
%%LINE1
Kúr\textsuperscript{!}éká mâ:tí Nê:lsa.    \jambox*{[VOS]}
%%LINE2
\gll
ku-rek-a   mati  Nelsa\\
%%LINE3
17\SM{}-draw-\FV{}  6.water  1.Nelsa\\
%%TRANS1
\glt
‘Nelsa is drawing water.’
%%TRANS2
%%EXEND

\z

%%EAX
\ea
%%JUDGEMENT
%%LABEL
\label{bkm:Ref121990860}
%%CONTEXT
(Who is cooking rice? / \textsuperscript{\#}What is Luisa cooking?)\\
%%LINE1
Kubhiká Lúízá mpû:nga.  \jambox*{[VSO]}
%%LINE2
\gll
ku-bhik-a  Luiza  mpunga\\
%%LINE3
17\SM{}-cook-\FV{}  1.Luisa  3.rice\\
%%TRANS1
\glt
‘Luisa cooked rice.’\\
%%TRANS2
%%EXEND

\z

Note that this postverbal focus in VSO is not necessarily exhaustive, as modification of the subject by the inclusive particle \textit{hambi} ‘even’ is possible in example \xref{bkm:Ref146272393}, suggesting that there are others who also ate shima (see also \citealt{chapters/intro} for diagnostics):

%%EAX
\ea
%%JUDGEMENT
%%LABEL
\label{bkm:Ref146272393}
%%CONTEXT
%%LINE1
Kúdyî:té \textbf{hámbí} \textbf{Rózá:ríó} mba:ba.   \jambox*{[VSO]}
%%LINE2
\gll
ku-dy-ile  hambi  Rosario  mbaba\\
%%LINE3
17\SM{}-eat-\PFV{}  even  1.Rosario  3.shima\\
%%TRANS1
\glt
‘Even Rosario ate shima.’\\
%%TRANS2
%%EXEND

\z

Questions in transitive DAI require the question word to be placed adjacent to the verb, as in \xref{bkm:Ref120691162}, in contrast to object question words in SVOO order, as shown in \sectref{bkm:Ref121988262} above, examples \xxref{bkm:Ref121149061}{bkm:Ref141359654}. This is of course with the exception of multiple questions, where both question words occur postverbally but not necessarily verb-adjacent, as in \xref{bkm:Ref120691168}.

\ea\label{bkm:Ref120691162}
%%EAX
\ea
%%JUDGEMENT
[]{
%%LABEL
%%CONTEXT
%%LINE1
Kúndíndá \textbf{ma:ni} cítu:lu? \jambox*{[VSO]}
%%LINE2
\gll
ku-ndind-a  mani  ci-tulu\\
%%LINE3
17\SM{}-pull-\FV{}  who  7-chair\\
%%TRANS1
\glt
‘Who is pulling the chair?’\\
%%TRANS2
}
%%EXEND


  %%EAX
\ex
%%JUDGEMENT
[*]{
%%LABEL
%%CONTEXT
%%LINE1
Kubhiká mpû:nga \textbf{mâ:ni}? \jambox*{[*VOS]}
%%LINE2
\gll
ku-bhik-a  mpunga  mani\\
%%LINE3
17\SM{}-cook-\FV{}  3.rice  who\\
%%TRANS1
\glt
‘Who is cooking rice?’\\
%%TRANS2
}
%%EXEND

\z
\z
%%EAX
\ea
%%JUDGEMENT
[]{
%%LABEL
\label{bkm:Ref120691168}
%%CONTEXT
%%LINE1
Kúbhálá mâ:ní câ:ni?\footnote{It is uncertain whether the inverse order of question words (\textit{kúbhǎ:lá c\textsuperscript{!}á:ní ma:ni?}) is also accepted, and if so, in which contexts.}\\
%%LINE2
\gll ku-bhal-a  mani  cani\\ 
%%LINE3
17\SM{}-write-\FV{}  who  what\\\jambox*{[VSO]}
%%TRANS1
\glt
‘Who is writing what?’\\
%%TRANS2
}
%%EXEND

\z

All three present tense verb forms (see \sectref{bkm:Ref141358518}) are acceptable in DAI, with slight differences in meaning. The progressive form is used as a question or as a progressive action \xref{bkm:Ref136237672:a}, the disjoint form is used to express habitual action, as we can see in \xref{bkm:Ref136237672:b}, and the conjoint form for punctual action as in \xref{bkm:Ref136237672:c}. The precise conditions for the use of each form, and their interpretations in combination with subject inversion, remain to be investigated.

\ea\label{bkm:Ref136237672}
%%EAX
\ea
%%JUDGEMENT
%%LABEL
\label{bkm:Ref136237672:a}
%%CONTEXT
(Context: You are wondering what noise is that you hear, and a friend answers this.)\\
%%LINE1
Kóph\textsuperscript{!}índa mǒ:vha. \jambox*{[progressive]}
%%LINE2
\gll
ku-o-phind-a  movha\\
%%LINE3
17\SM{}-\PROG{}-pass-\FV{}  3.car\\
%%TRANS1
\glt
‘A car is passing by.’\\
%%TRANS2
%%EXEND


  %%EAX
\ex
%%JUDGEMENT
%%LABEL
\label{bkm:Ref136237672:b}
%%CONTEXT
(Context: There’s is a road that seems to be narrow for a car or looks unused and someone ask if cars pass there. )\\
%%LINE1
Káphínda mǒ:vha. \jambox*{[disjoint]}
%%LINE2
\gll
ku-a-phind-a  movha\\
%%LINE3
17\SM{}-dj-pass-\FV{}  3.car\\
%%TRANS1
\glt
‘A car passes by.’ (in a particular road, usually)\\
%%TRANS2
%%EXEND


  %%EAX
\ex
%%JUDGEMENT
%%LABEL
\label{bkm:Ref136237672:c}
%%CONTEXT
(Context: Someone hears a noise and asks ‘What is that noise?’ / ‘What is passing by?’)\\
%%LINE1
Kúphínda mǒ:vha. \jambox*{[conjoint]}
%%LINE2
\gll
ku-phind-a  movha\\
%%LINE3
17\SM{}-pass-\FV{}  3.car\\
%%TRANS1
\glt
‘A car is passing by.’\\
%%TRANS2
%%EXEND

\z
\z

In summary, Cicopi accepts only two subject inversion constructions, Agreeing Inversion and Default Agreement Inversion. Not much can be said at this stage about the former, but the latter is possible with predicates of different valencies: with intransitives it is used in thetic environments as well as with focus on the postverbal logical subject, and with transitives the logical subject must be focused. The possibilities for conjoint and disjoint verb forms and their interpretations in subject inversion remain for further research. 

\subsection{Right periphery}
\label{bkm:Ref121988320}
It is not always clear whether the postverbal logical subject is part of an inversion construction or a right-dislocated constituent. Example \xref{bkm:Ref120095706} may look like an Agreeing Inversion construction, but the context indicates that the postverbal subject is not presented as new information (because it is given in the question) but should be analysed as a right-dislocated constituent. The same is true for \xref{bkm:Ref124151476}, from the Frog story, where the frog is part of the core of the story, but has not been mentioned for a while. The speaker, having referred to the frog first as ‘the animal he was looking for’ and then specifying in the right periphery as ‘his frog’.

%%EAX
\ea
%%JUDGEMENT
%%LABEL
\label{bkm:Ref120095706}
%%CONTEXT
(Why do the bulls have these marks?)\\
%%LINE1
Tópé:kwá \textbf{tího:mú}.\\
%%LINE2
\gll
ti-o-pek-w-a  ti-homu\\
%%LINE3
10\SM{}-\PROG{}-hit-pass-\FV{}  10-bulls\\
%%TRANS1
\glt
‘They are (being) beaten, the bulls.’\\
%%TRANS2
%%EXEND

\z

%%EAX
\ea
%%JUDGEMENT
%%LABEL
\label{bkm:Ref124151476}
%%CONTEXT
(When the boy passed, he went to climb that tree because he saw that where he had been looking, where the mouse was, it was not the place he was looking for.)\\
%%LINE1
Ayamáná cíhárí ci angátícílá:vá ca \textbf{díkhelé} \textbf{dá:kwé}. \\
%%LINE2
\gll
a-ya-man-a  ci-hari  ci  a-nga-ti-ci-lav-a  ci-a   di-khele  di-akwe \\
%%LINE3
1\SM{}-\NEG{}-find-\FV{}  7-animal  7\textsc{.dem}.\PROX{}  1\SM-\REL{}-\IPFV{}-7\OM{}-want-\FV{}  7-\CONN{}  5-frog  5-\textsc{poss.1}\\
%%TRANS1
\glt
‘He did not find the animal he was looking for, his frog.’
%%TRANS2
%%EXEND

\z

A clear afterthought interpretation of a right-peripheral constituent is also seen in \xref{bkm:Ref121989942}: the narrator of the Frog story assumes that the hearer knows who\slash what was hitting the boy, but adds the information that this was the owl, in case it was not clear. Then, in the second part, \textit{mámé wúlé} ‘the boy’ is right-dislocated too, and also modified by a distal demonstrative.
\pagebreak

%%EAX
\ea
%%JUDGEMENT
%%LABEL
\label{bkm:Ref121989942}
%%CONTEXT
(That owl got tired of hearing his noise, went out and hit him.)\\
%%LINE1
Cídimúpékílé ngu típá:pá, \textbf{cíkhóvhá} \textbf{cílé,} ató:wá háhá:tsí \textbf{mámé} \textbf{wú:lé}. \\
%%LINE2
\gll
ci-di-mu-pek-ile  ngu  ti-papa  ci-khovha  ci-le  a-to-w-a  ha-hatsi  mame  wu-le  \\
%%LINE3
7\SM{}-\DEP{}-1\OM{}-hit-\PFV{}  \PREP{}  10-wing  7-owl  7\SM{}-\DEM{}.\MED{}  1\SM{}-\PST{}-fall-\FV{}  16-down 1.boy  1\SM{}-\DEM.\MED{}\\
%%TRANS1
\glt
‘When he hit him with its wings, the/that owl, he fell to the ground, the/that boy.’
%%TRANS2
%%EXEND

\z

A prosodic break may also indicate right dislocation, as in \xref{bkm:Ref146700838} and \xref{bkm:Ref141360524} for subject and objects.

%%EAX
\ea
%%JUDGEMENT
%%LABEL
\label{bkm:Ref146700838}
%%CONTEXT
%%LINE1
Átô:bhǐka, Páulu, feijáu.\\
%%LINE2
\gll
a-to-bhik-a  Paulu  feijau\\
%%LINE3
1\SM{}-\TO{}-cook-\FV{}  1.Paulo  10.bean\\
%%TRANS1
\glt
‘He cooks (them) well, Paulo, the beans.’\\
%%TRANS2
%%EXEND

\z

%%EAX
\ea
%%JUDGEMENT
%%LABEL
\label{bkm:Ref141360524}
%%CONTEXT
(Description of definiteness pictures Bloom Ström. A man goes to the market. He sees a pineapple.)\\
%%LINE1
Digwǐta díwóná míko:mbva, diyâyídhu:ndha (, \textbf{miko:mbva}).\\
%%LINE2
\gll
di-gwit-a  di-won-a  mi-kombva  di-ya-yi-dhundh-a  mi-kombva\\
%%LINE3
5\SM{}-finish-\FV{}  5\SM{}-see-\FV{}  4-banana  5\SM{}-\NEG{}-4\OM{}-like-\FV{}  4.banana\\
%%TRANS1
\glt
‘He looks at some bananas. He didn’t like them (the bananas).’\\
%%TRANS2
%%EXEND

\z

Object marking seems to be optional for right-peripheral constituents, as shown in the comparison between \xref{bkm:Ref124152723:a} with object marker and \xref{bkm:Ref124152723:b} without. Note also the lengthening on \textit{hayi} ‘where’ in \xref{bkm:Ref124152723:a} but not \xref{bkm:Ref124152723:b}, likely indicative of the right edge of a prosodic phrase. These properties can be interpreted as the object being dislocated in \xref{bkm:Ref124152723:a} but not \xref{bkm:Ref124152723:b}.

\ea\label{bkm:Ref124152723}
%%EAX
\ea
%%JUDGEMENT
%%LABEL
\label{bkm:Ref124152723:a}
%%CONTEXT
%%LINE1
M\textbf{ii}xavílé ha:yí \textbf{fóni} \textbf{ilé}?\\
%%LINE2
\gll
mu-yi-xav-ile  hayi  foni  ile\\
%%LINE3
2\PL.\SM{}-9\OM{}-buy-\PFV{}  where  9.phone  9.\DEM{}.\MED{}\\
%%TRANS1
\glt
‘Where did you buy that phone?’\\
%%TRANS2
%%EXEND


  %%EAX
\ex
%%JUDGEMENT
%%LABEL
\label{bkm:Ref124152723:b}
%%CONTEXT
%%LINE1
Muxavílé hayí \textbf{fóni} \textbf{ilé}?\\
%%LINE2
\gll
mu-xav-ile  hayi  foni  ile\\
%%LINE3
2\PL.\SM{}-buy-\PFV{}  where  9.phone  9.\DEM{}.\MED{}\\
%%TRANS1
\glt
‘Where did you buy that phone?’\\
%%TRANS2
%%EXEND

\z
\z

The optional object marking contrasts with left-dislocated constituents, which seems to require object marking when human, as shown in \xref{bkm:Ref127260966:a}; compared with right-peripheral location of the same constituent in \xref{bkm:Ref127260966:b}, with optional object marking.

\ea\label{bkm:Ref127260966}
%%EAX
\ea
%%JUDGEMENT
%%LABEL
\label{bkm:Ref127260966:a}
%%CONTEXT
%%LINE1
\textbf{Vaná:ná}, ngóndî:sí á\textbf{*(vá)}ningíle mápâ:xta.\\
%%LINE2
\gll
va-nana  n-gondisi  a-va-ning-ile  ma-paxta\\
%%LINE3
2-child  1-teacher  1\SM{}-2\OM{}-give-\PFV{}  6-bag\\
%%TRANS1
\glt
‘The children, the teacher gave them bags.’\\
%%TRANS2
%%EXEND


  %%EAX
\ex
%%JUDGEMENT
%%LABEL
\label{bkm:Ref127260966:b}
%%CONTEXT
%%LINE1
Ngóndî:si á\textbf{(vá)}níngílé hambí ní mápa:xta \textbf{vanâ:na}.\\
%%LINE2
\gll
n-gondisi  a-va-ning-ile  hambi  ni  ma-paxta  va-nana\\
%%LINE3
1-teacher  1\SM{}-2\OM{}-give-\PFV{}  even  and  6-bag  2-child\\
%%TRANS1
\glt
‘The teacher gave even bags to the children.’\\
%%TRANS2
%%EXEND

\z
\z

Object marking would be a potential further diagnostic to determine the status of constituents inside or outside the core clause, but object marking in Cicopi is still poorly understood (see \sectref{bkm:Ref127267714}).

To summarise this section on word order, the preverbal domain in Cicopi is restricted to non-focal constituents, and the postverbal domain contains non-topical constituents. Topics are preferably expressed in initial position, and focus constituents can appear in any postverbal position, although there may be a preference for the immediate-after-verb position. The right periphery contains backgrounded constituents such as afterthoughts. This is summarised in \figref{fig:cicopiwo}. 

\begin{figure}
%% FIGTAB
%% Not treating this as a table -- doesn't make sense to use formatting of a regular table here (e.g. no vertical lines, double horizontal lines)
%% If needed, this can be replaced with a figure without cell boundaries
\begin{tabular}{|c|c|c|c|c|}
\hline
topic & subject & V & non-topic & non-topic/non-focus\\
\hline
\end{tabular}
\caption{Template for Cicopi word order}
\label{fig:cicopiwo}
\end{figure}

\section{Cleft}
\label{bkm:Ref121993097}
Cleft constructions consist of a relative clause expressing the given/presupposed information, and a clefted focus constituent. Before we introduce the cleft constructions in Cicopi, we first provide information on the marking of cleft constructions, specifically the copulas and relative marking. The clefted constituent is marked by a preceding copula, and the copula can appear in three forms: segmentally as \textit{ngu} or \textit{i}, or suprasegmentally as a high tone. The copula \textit{ngu} is illustrated in \xref{bkm:Ref141428691}; the high tone copula can be seen in comparing \xxref{bkm:Ref141428713:a}{bkm:Ref141428713:b}. Example \xref{bkm:Ref141428713:a} is a cleft, with a high tone on \textit{mpú:nga} and compatible with modification by \textit{dwe} ‘only’ (see further below), whereas in \xref{bkm:Ref141428713:b} \textit{mpu:nga} `rice' is a low-toned left-dislocated topic, which hence cannot be modified by ‘only’. We gloss the high tone as \COP{}, together with the gloss of the first segmental morpheme.

%%EAX
\ea
%%JUDGEMENT
%%LABEL
\label{bkm:Ref141428691}
%%CONTEXT
%%LINE1
Ngú ha:yí údímâ:ku?\\
%%LINE2
\gll
ngu  hayi  u-dim-a=ku\\
%%LINE3
\COP{}  where  2\SG.\SM{}-farm-\FV{}=\REL{}\\
%%TRANS1
\glt
‘Where do you dig/farm?’ / ‘Where are you digging?’\\
%%TRANS2
%%EXEND

\z

\ea\label{bkm:Ref141428713}
%%EAX
\ea
%%JUDGEMENT
%%LABEL
\label{bkm:Ref141428713:a}
%%CONTEXT
%%LINE1
Mpú:nga (dwé) Luiza abhikâ:ku.\\
%%LINE2
\gll
mpunga  dwe  Luisa  a-bhik-a=ku\\
%%LINE3
\COP{}.3.rice  only  1.Luisa  1\SM{}-cook-\FV{}=\REL{}\\
%%TRANS1
\glt
‘It’s (only) the rice Luisa is cooking.’\\
%%TRANS2
%%EXEND


  %%EAX
\ex
%%JUDGEMENT
%%LABEL
\label{bkm:Ref141428713:b}
%%CONTEXT
%%LINE1
Mpu:nga (*dwé) Luíza óbhǐ:ka.\\
%%LINE2
\gll
mpunga  dwe  Luisa  a-o-bhik-a\\
%%LINE3
3.rice  only  1.Luisa  1\SM{}-\PROG{}-cook-\FV{}\\
%%TRANS1
\glt
‘The rice, Luisa is cooking (it).’\\
%%TRANS2
%%EXEND

\z
\z

Relativisation is marked in \xref{bkm:Ref120692880} for the present tense by the enclitic =\textit{ku}, and in other tenses as a prefix \textit{nga}\nobreakdash-, illustrated for an object relative clause in \xref{bkm:Ref127261683}.

%%EAX
\ea
%%JUDGEMENT
%%LABEL
\label{bkm:Ref120692880}
%%CONTEXT
(\textit{I ncani yiku ka dipanela?} ‘What is in the pot?’)\\
%%LINE1
Í féjaú yíkú ka dípáne:la\\
%%LINE2
\gll
i  fejau  yi=ku  ka  di-panela\\
%%LINE3
\COP{}  9.bean  9\SM{}=\REL{}  \LOC{}  5-pot\\
%%TRANS1
\glt
‘It’s beans that are in the pot.’\\
%%TRANS2
%%EXEND

\z

%%EAX
\ea
%%JUDGEMENT
%%LABEL
\label{bkm:Ref127261683}
%%CONTEXT
%%LINE1
Vanáná (vá) ningavawo:na váwí:te.\\
%%LINE2
\gll
va-nana  va  ni-nga-va-von-a  va-w-ile\\
%%LINE3
2-child  2.\DEM.\PROX{}  1\textsc{sg.sm-rel-2om}{}-see-\FV{}  2\SM{}-fall-\PFV{}\\
%%TRANS1
\glt
‘The(se) children that I saw fell.’\\
%%TRANS2
%%EXEND

\z

As in many languages, cleft constructions in Cicopi are used to focus constituents, and Cicopi has three related constructions to do that: the basic cleft, the pseudocleft, and the reverse pseudocleft (which may be analysed better as a NP\,+ basic cleft). We discuss these in turn.

\subsection{Basic cleft}

The basic cleft in Cicopi is marked by a copula \textit{i} or \textit{ngu} preceding the clefted constituent, and a relative clause following it, as we can see in \xref{bkm:Ref120692880} above and \xref{bkm:Ref120693136}, where the relative is marked by the enclitic =\textit{ku}.

%%EAX
\ea
%%JUDGEMENT
%%LABEL
\label{bkm:Ref120693136}
%%CONTEXT
(\textit{Kudya mani senora}? Who is eating a/the carrot? + drawing of a sheep and a tortoise eating a carrot.)\\
%%LINE1
Í mbvhû:tá ní fu:tu sídyâ:ku séno:ra.\\
%%LINE2
\gll
i  mbvhuta  ni  futu  si-dy-a=ku  senora\\
%%LINE3
\COP{}  9.sheep and  9.tortoise  8\SM{}-eat-\FV{}=\REL{}  5.carrot\\
%%TRANS1
\glt
‘It is the sheep and the tortoise that are eating a/the carrot.’\\
%%TRANS2
%%EXEND

\z

Basic clefts can naturally be used for interrogatives, clefting the inherently focused question word, as in \xref{bkm:Ref126844879} and \xref{bkm:Ref126844882}.

%%EAX
\ea
%%JUDGEMENT
%%LABEL
\label{bkm:Ref126844879}
%%CONTEXT
%%LINE1
Í nc\textsuperscript{!}ání Teré:za angatíyo:ka?\\
%%LINE2
\gll
i  cani  Tereza  a-nga-ti-yok-a\\
%%LINE3
\COP{}  what  1.Teresa  1\SM{}-\REL{}-\IPFV{}-bake-\FV{}\\
%%TRANS1
\glt
‘What was Teresa baking?’\\
%%TRANS2
%%EXEND

\z

%%EAX
\ea
%%JUDGEMENT
%%LABEL
\label{bkm:Ref126844882}
%%CONTEXT
%%LINE1
Í m\textsuperscript{!}ání ándíndâ:ku cítu:lu?\\
%%LINE2
\gll
i  mani  a-ndind-a=ku  ci-tulu\\
%%LINE3
\COP{}  who  1\SM{}-pull-\FV{}=\REL{}  7-chair\\
%%TRANS1
\glt
‘Who is pulling the chair?’\\
%%TRANS2
%%EXEND

\z

As already mentioned in \sectref{bkm:Ref120693404}, the question word by itself is ungrammatical in preverbal position, i.e. without the copula. Answers to content questions may also appear in a cleft, as shown in (\ref{bkm:Ref120693442}A2), although they do not need to, as shown in (\ref{bkm:Ref120693442}A1).

\ea
(What was Teresa baking?)\\
\begin{xlist}[A2:]
%%EAX
\exi{A1:}
%%JUDGEMENT
%%LABEL
\label{bkm:Ref120693442}
%%CONTEXT
%%LINE1
Téréza átíy\textsuperscript{!}óká díbhoólu.\\
%%LINE2
\gll
Tereza  a-ti-yok-a  di-bholu\\
%%LINE3
1.Teresa  1\SM{}-\IPFV{}-bake-\FV{}  5-cake\\
%%TRANS1
\glt
‘Teresa was baking a/the cake.’\\
%%TRANS2
%%EXEND

%%EAX
\exi{A2:}
%%JUDGEMENT
%%LABEL
%%CONTEXT
%%LINE1
I díbholú angatíy\textsuperscript{!}óká Téréza.\\
%%LINE2
\gll
i  dibholu  a-nga-ti-yok-a  Tereza\\
%%LINE3
\COP{}  5-cake  1\SM{}-\REL{}-\IPFV{}-bake-\FV{}  1.Teresa\\
%%TRANS1
\glt
‘It was a/the cake that Teresa was baking.’
%%TRANS2
%%EXEND

\end{xlist}
\z

Similarly, in answering the question in \xref{bkm:Ref120693497}, we also have two possibilities: the basic cleft as in A1 or the DAI as in A2.
\pagebreak

\ea
\begin{xlist}[A2:]
\exi{Q:} ‘Who is pulling the chair?’
%%EAX
\exi{A1:}
%%JUDGEMENT
%%LABEL
\label{bkm:Ref120693497}
%%CONTEXT
%%LINE1
Í mwán\textsuperscript{!}á:ná ándíndâ:ku.  \jambox*{[cleft]}
%%LINE2
\gll
i  mw-anana  a-ndind-a=ku\\
%%LINE3
\COP{}  1-child  1\SM{}-pull-\FV{}=\REL{}\\
%%TRANS1
\glt
'It’s the child who is pulling.’\\
%%TRANS2
%%EXEND

%%EAX
\exi{A2:}
%%JUDGEMENT
%%LABEL
%%CONTEXT
%%LINE1
Kúndíndá mwán\textsuperscript{!}á:ná cítu:lu.  \jambox*{[VSO]}
%%LINE2
\gll
ku-ndind-a  mw-anana  ci-tulu\\
%%LINE3
17\SM{}-pull-\FV{}  1-child  7-chair\\
%%TRANS1
\glt
‘A child is pulling the chair.’
%%TRANS2
%%EXEND

\end{xlist}
\z

Clefts are not only used for eliciting and providing new information, but also to express a contrast, as in the correction from the QUIS map task illustrated in \xref{bkm:Ref120693591}.

\ea\label{bkm:Ref120693591}
\begin{xlist}
\exi{A:}{(I have walked and I find a bicycle.)} \\

%%EAX
\exi{B:}{
%%JUDGEMENT
%%LABEL
%%CONTEXT
%%LINE1
Hingá bhasíké:ní, í m\textsuperscript{!}ó:vha níwónâ:ku.\\
%%LINE2
\gll
hinga  bhasikeni,  i  movha  ni-won-a=ku\\
%%LINE3
\COP.\NEG{}  9.bicycle  \COP{}  5.car  1\SG.\SM{}-see-\FV{}=\REL{}\\
%%TRANS1
\glt
‘It’s not a bicycle, it’s a car that I see.' 
%%TRANS2
}
%%EXEND

\end{xlist}
\z


The focus on the clefted constituent comes out as exhaustive when the expression \textit{dwe} (only) is used. Note that \textit{dwe} can be separated from the clefted constituent but still modify it, as in the answer in \xref{bkm:Ref120693710}, where \textit{dwe} appears in final position, but is interpreted with the clefted subject \textit{Gomes}.

\ea
\label{bkm:Ref120693710}
%%EAX
\begin{xlist}
\exi{Q:}
%%JUDGEMENT
%%LABEL
%%CONTEXT
%%LINE1
Í m\textsuperscript{!}ání angatutu:ma?\\
%%LINE2
\gll
i  mani  a-nga-tutum-a\\
%%LINE3
\COP{}  who  1\SM{}-\REL{}-run-\FV{}\\
%%TRANS1
\glt
‘Who ran?’\\
%%TRANS2
%%EXEND

%%EAX
\exi{A:}{
%%JUDGEMENT
%%LABEL
%%CONTEXT
%%LINE1
Í Go:mes angatutu:ma dwé.\\
%%LINE2
\gll
i  Gomes  a-nga-tutum-a  dwe\\
%%LINE3
\COP{}  1.Gomes  1\SM{}-\REL{}-run-\FV{}  only\\
%%TRANS1
\glt
‘It was only Gomes that ran.’
%%TRANS2
}
%%EXEND

\end{xlist}
\z

The exhaustive interpretation seems to be an implied part of the interpretation of the basic cleft, as visible in the answer to an incomplete question. The fact that the answer in \xref{bkm:Ref120694427} contains \textit{ihimhim} ‘no’, rather than \textit{iim} ‘yes’, shows that the cleft in the question can be interpreted as exhaustive: the cleft question in \xref{bkm:Ref120694427} asks, ‘Is [the sheep]\textsubscript{EXH} eating a carrot?’ and the negation applies to the exhaustivity: it is true that the sheep is eating the carrot, but it is not \textit{only} the sheep (also the tortoise). However, an answer with ‘yes’ was also given to a similarly ``incomplete" question, as in \xref{bkm:Ref120694485}, showing that exhaustivity is not inherent to the basic cleft.

\ea
(Context: a drawing with a tortoise and sheep eating a carrot.)\\
%%EAX
\begin{xlist}
\exi{Q:}{
%%JUDGEMENT
%%LABEL
\label{bkm:Ref120694427}
%%CONTEXT
%%LINE1
I mbvhu:tá yídyáku sénó:rá?\\
%%LINE2
\gll
i  mbvhuta  yi-dy-a=ku  senora\\
%%LINE3
\COP{}  9.sheep  9\SM{}-eat-\FV{}=\REL{}  5.carrot\\
%%TRANS1
\glt
‘Is it the sheep that is eating a/the carrot?’\\
%%TRANS2
}
%%EXEND

%%EAX
\exi{A:}{
%%JUDGEMENT
%%LABEL
%%CONTEXT
%%LINE1
Ihî:mhím, í mbvhû:ta ní fu:tu, hingá mbvhú:tá dwé yídyâ:ku séno:ra.\\
%%LINE2
\gll
ihimhim  i  mbvhuta  ni  futu   hinga  mbvhuta  dwe  yi-dy-a=ku  senora\\
%%LINE3
no  \COP{}  9.sheep  and  9.tortoise \COP.\NEG{}  9.sheep  only  9\SM{}-eat-\FV{}=\REL{}  5.carrot\\
%%TRANS1
\glt
‘No, it is the sheep and tortoise, it's not only the sheep that eats a/the carrot.’
%%TRANS2
}
%%EXEND

\end{xlist}
\z

\ea
(Context: a photo of a woman selling tomatoes and onions.)\\
%%EAX
\begin{xlist}
\exi{Q:}{
%%JUDGEMENT
%%LABEL
\label{bkm:Ref120694485}
%%CONTEXT
%%LINE1
Símá:tí axávísá:kú wansika:ti?\\
%%LINE2
\gll
si-mati  a-xav-is-a=ku  wansikati\\
%%LINE3
\COP{}.8-tomato  1\SM{}-buy-\CAUS{}-\FV{}=\REL{}  1.woman\\
%%TRANS1
\glt
‘Is it tomatoes the woman is selling?’\\
%%TRANS2
}
%%EXEND

%%EAX
\exi{A:}{
%%JUDGEMENT
%%LABEL
%%CONTEXT
%%LINE1
Iim/Ihî:mhím sím\textsuperscript{!}á:tí ní tíny\textsuperscript{!}á:lá áxávísâ:kú wánsíka:ti.\\
%%LINE2
\gll
Iim/ ihimhim  si-mati  ni  ti-nyala  a-xav-is-a=ku  wansikati\\
%%LINE3
yes/ no  \COP{}.8-tomato  and  10-onion  1\SM{}-buy-\CAUS{}-\FV{}=\REL{}  1.woman\\
%%TRANS1
\glt
‘Yes/no, it’s tomatoes and onions that the woman is selling.’
%%TRANS2
}
%%EXEND

\end{xlist}
\z

The answer in \xref{bkm:Ref127262426} suggests that the cleft conveys an exhaustive focus: just by using the cleft, the “overcomplete” statement in the question can be corrected to exclude the cat mentioned in the question in \xref{bkm:Ref127262426}.\largerpage[-1]\pagebreak 

\ea\label{bkm:Ref127262426}(Context: QUIS picture of Maria with a rabbit.)
%%EAX
\begin{xlist}
\exi{Q:}{
%%JUDGEMENT
%%LABEL
%%CONTEXT
%%LINE1
Í mvh\textsuperscript{!}ú:njá ni cíwô:ngá ákúnású Marî:ya?\footnote{The enclitic object marker is used in relative clauses and with SM=na=\CL{} ‘be with’; in other environments we find a prefixal object marker.}\\
%%LINE2
\gll
i  mvhunja  ni  ci-wonga  a-ku=na=su  Mariya\\
%%LINE3
\COP{}  5.rabbit  and  7-cat  1\SM{}-\REL{}=with=8\OM{}  1.Mary\\
%%TRANS1
\glt
‘It is a rabbit and a cat that Mary has?’
%%TRANS2
}
%%EXEND


%%EAX
\exi{A:}{
%%JUDGEMENT
%%LABEL
%%CONTEXT
%%LINE1
Ihî:mhím, i mvhu:njá akuna:cú, akáná cíwó:ngá.\\
%%LINE2
\gll
ihimhim  i  mvhunja  a-ku=na=cu  a-ka=na  ci-wonga\\
%%LINE3
no  \COP{}  9.rabbit  1\SM{}-\REL{}=with=7\OM{}  1\SM{}-\NEG{}=with  7.cat\\
%%TRANS1
\glt
‘No, it is a rabbit that she has. She doesn’t have a cat.’
%%TRANS2
}
%%EXEND

\end{xlist}
\z

The fact that inclusive \textit{hambi} ‘even’ is not acceptable in a cleft \xref{bkm:Ref120694799} illustrates the same point: if she has fed the goats even rice, as the least likely in a range of other foodstuffs that she also gave, then rice is not the exhaustive referent for which the predicate is true.

%%EAX
\ea
%%JUDGEMENT
[*]{
%%LABEL
\label{bkm:Ref120694799}
%%CONTEXT
%%LINE1
Í hambí mpû:nga Laurínyá átiningile típhô:ngo.\\
%%LINE2
\gll
i  hambi  mpunga  Laurinya  a-ti-ning-ile  ti-phongo\\
%%LINE3
\COP{}  even  3.rice  1.Laurinha  1\SM{}-10\OM{}-give-\PFV{}  10-goat\\
%%TRANS1
\glt
int. ‘It’s even rice that Laurinha gave to the goats.’\\
%%TRANS2
}
%%EXEND

\z

We have conflicting evidence for the universal quantifier \textit{sotshe} ‘all’. In principle, ‘all’ cannot exclude any alternatives and is therefore incompatible with exclusive focus. The unacceptability of \xref{bkm:Ref120694808} therefore suggests that the cleft has an exclusive interpretation; but the acceptability of \xref{bkm:Ref120694822} suggests differently. The explanation for this acceptability might be found in the context for examples like \xref{bkm:Ref120694845} and \xref{bkm:Ref120694861}: the universal quantifier is opposed to the alternative of ‘some’ (i.e. ‘not all’), thereby excluding at least that alternative. This opposition makes it compatible with an exclusive reading.

%%EAX
\ea
%%JUDGEMENT
[*]{
%%LABEL
\label{bkm:Ref120694808}
%%CONTEXT
%%LINE1
I so:tshe siketa ningaxa:va.\footnote{This example lacks tone marking.}\\
%%LINE2
\gll
i  si-otshe  si-keta  ni-nga-xav-a\\
%%LINE3
\COP{}  8-all  8-pineapple  1\SM{}-\REL{}-buy-\FV{}\\
%%TRANS1
\glt
int. ‘It’s all the pineapples that I bought.’\\
%%TRANS2
}
%%EXEND

\z
\largerpage[-1]\pagebreak

%%EAX
\ea
%%JUDGEMENT
[]{
%%LABEL
\label{bkm:Ref120694822}
%%CONTEXT
\label{bkm:Ref120694845}(Context: There’s a pest and someone wants to know which animals will die if they don’t fumigate.)\\
%%LINE1
Síhǎrí sô:tshé sínô:fa.\\
%%LINE2
\gll
si-hari  si-otshe  si-na-o-f-a\\
%%LINE3
\COP{}.8-animal  8-all  8\SM{}-\FUT{}-\REL{}-die-\FV{}\\
%%TRANS1
\glt
‘It’s all the animals that will die.’\\
%%TRANS2
}
%%EXEND

\z

%%EAX
\ea
%%JUDGEMENT
[]{
%%LABEL
\label{bkm:Ref120694861}
%%CONTEXT
(Did the cat really break every single thing?)\\
%%LINE1
Í s\textsuperscript{!}ó:tshé ciwo:ngá cíngáfa:ya.\\
%%LINE2
\gll
I  si-otshe  ci-wonga  ci-nga-fay-a\\
%%LINE3
\COP{}  8-all  7-cat  7\SM{}-\REL{}-break-\FV{}\\
%%TRANS1
\glt
‘It’s everything that the cat broke.’\\
%%TRANS2
}
%%EXEND

\z

A ``mention some" question can also be used to test exhaustivity (see the BaSIS methodology, \citealt{vanderWal2021a}). Such a question has multiple good answers; for the question in \xref{bkm:Ref120694878} there are typically various places in which spring onions can be bought, and this means that the answer to such a question cannot contain an exhaustive focus strategy (as it would not be true that the mentioned place is the only place where onions can be bought). The question here can be felicitously answered by the SVO sentence in \xref{bkm:Ref120694878:b}, but for the cleft in \xref{bkm:Ref120694878:a} it was indicated that this means it would be the only place you can get them (which is not true and therefore infelicitous). 

\ea\label{bkm:Ref120694878}(Where can I buy spring onions?)
%%EAX
\ea
%%JUDGEMENT
[\textsuperscript{\#}]{
%%LABEL
\label{bkm:Ref120694878:a}
%%CONTEXT
%%LINE1
Í bazá:rá ungóxává:kú cíbhi:la.\\
%%LINE2
\gll
i  bazara  u-nga-ku-xav-a=ku  ci-bhila\\
%%LINE3
\COP{}  market  2\SG.\SM{}-\POT{}-15-buy-\FV{}=\REL{}  7-spring.onion\\
%%TRANS1
\glt
‘It’s on the market that you can buy spring onion.’\\
%%TRANS2
}
%%EXEND


%%EAX
\ex
%%JUDGEMENT
[]{
%%LABEL
\label{bkm:Ref120694878:b}
%%CONTEXT
%%LINE1
Ungáxává bazâ:ra.\\
%%LINE2
\gll
u-nga-xav-a  bazara\\
%%LINE3
2\SG.\SM{}-\POT{}-buy-\FV{}  market\\
%%TRANS1
\glt
‘You can buy (them) on the market.’\\
%%TRANS2
}  
%%EXEND

\z
\z

Given these data suggesting the exclusive or exhaustive interpretation of the basic cleft, it is surprising that the basic cleft is accepted with cognate objects \xref{bkm:Ref141364676} and parts of idioms \xref{bkm:Ref141364685}, retaining the idiomatic interpretation; as mentioned, cognate objects and parts of idioms are ``unfocusable" as they cannot generate any alternatives in the idiomatic interpretation \citep[see][]{vanderWal2016,vanderWal2022}. Note also that these are thetic contexts, which may be of influence on their unexpected acceptance.

%%EAX
\ea
%%JUDGEMENT
%%LABEL
\label{bkm:Ref141364676}
%%CONTEXT
(Where did you get this idea?)\\
%%LINE1
\'{M}l\textsuperscript{!}ó:ró ningalo:ra.\\
%%LINE2
\gll
m-loro  ni-nga-lor-a\\
%%LINE3
\COP.3-dream  1\SG.\SM{}-\REL{}-dream-\FV{}\\
%%TRANS1
\glt
‘It’s a dream that I dreamt.’\\
%%TRANS2
%%EXEND

\z

%%EAX
\ea
%%JUDGEMENT
%%LABEL
\label{bkm:Ref141364685}
%%CONTEXT
(What happened with him/her?)\\
%%LINE1
Díbh\textsuperscript{!}á:vhú ángákha:va.\\
%%LINE2
\gll
di-bhavhu  a-nga-khav-a\\
%%LINE3
\COP{}.5-bucket  1\SM{}-\REL{}/\PFV{}-kick-\FV{}\\
%%TRANS1
\glt
‘It’s a bucket that s/he kicked.’\\
%%TRANS2
‘S/He passed away.’

%%EXEND

\z

Finally, as Cicopi does not have a restriction against multiple foci, the focus on a clefted constituent can be combined with focus elsewhere in the clause, as illustrated in \xref{bkm:Ref120693735}. In \xref{bkm:Ref120693735:a} the exhaustive marker \textit{dwé} gives exhaustive focus to the object \textit{tikaneta} ‘pens’, which means that only pens were given to the children by the teacher, whereas in \xref{bkm:Ref120693735:b} the focus is on \textit{vanana} ‘children’, which means that only children were given pens (which presupposes that in addition to children other people such as school staff and others could have received the pens). \textit{Dwe} in these cases modifies the constituent immediately to its left.

\ea\label{bkm:Ref120693735}
%%EAX
\ea
%%JUDGEMENT
%%LABEL
\label{bkm:Ref120693735:a}
%%CONTEXT
(Context: Someone sees pens and pencils with the children and asks: Who gave those things to the children?)\\
%%LINE1
I n’góndísi angánínga tíkáné:ta dwé, váná:na.\\
%%LINE2
\gll
i  n’gondisi  a-nga-ning-a  ti-kaneta  dwé,  va-nana\\
%%LINE3
\COP{}  1.teacher  1\SM{}-\REL{}-give-\FV{}  10-pen  only  2-child\\
%%TRANS1
\glt
‘It was the teacher that gave only pens to children.’\\
%%TRANS2
%%EXEND


   %%EAX
\ex
%%JUDGEMENT
%%LABEL
\label{bkm:Ref120693735:b}
%%CONTEXT
(Context: Someone sees the children and the adults with pens and ask: Who gave pens to them?)\\
%%LINE1
I n’góndísi angáníngá tíkáné:ta, váná:na dwé.\\
%%LINE2
\gll
i  n’gondisi  a-nga-ning-a  ti-kaneta  va-nana  dwé\\
%%LINE3
\COP{}  1.teacher  1\SM{}-\REL{}-give-\FV{}  10-pen  2-child  only\\
%%TRANS1
\glt
‘It was the teacher that gave pens to the children only.’\\
%%TRANS2
%%EXEND

\z
\z

Multiple questions with a cleft are also accepted; the subject or object can be clefted while the other interrogative remains postverbal.\pagebreak

\ea
%%EAX
\ea
%%JUDGEMENT
%%LABEL
%%CONTEXT
%%LINE1
Í nc\textsuperscript{!}á:ní angawo:mbá ma:ni?\\
%%LINE2
\gll
i  cani  a-nga-womb-a  mani\\
%%LINE3
\COP{}  what  1\SM{}-\REL{}-say-\FV{}  who\\
%%TRANS1
\glt
‘Who said what?’, lit. ‘It’s what that who said?’\\
%%TRANS2
%%EXEND


  %%EAX
\ex
%%JUDGEMENT
%%LABEL
%%CONTEXT
%%LINE1
Í m!ání angawo:mbá ca:ni?\\
%%LINE2
\gll
i  mani  a-nga-womb-a  cani\\
%%LINE3
\COP{}  who  1\SM{}-\REL{}-say-\FV{}  what\\
%%TRANS1
\glt
‘Who said what?’, lit. ‘It’s who that said what?’\\
%%TRANS2
%%EXEND

\z
\z

In summary, the clefted constituent shows some properties of an exclusive or even exhaustive focus interpretation, but its use seems to be broader than that, given the felicity in thetic contexts with idioms and cognate objects.

\subsection{Pseudocleft}

In a pseudocleft, the copula joins a free relative and a noun phrase. So, we find first the relative clause (on its left), then the copula and a noun phrase or pronoun (on its right), as we can see in \xref{bkm:Ref120695311}.

%%EAX
\ea
%%JUDGEMENT
%%LABEL
\label{bkm:Ref120695311}
%%CONTEXT
(What did they see?)\\
%%LINE1
Ací vángácíwó:na i cíwó:nga.\\
%%LINE2
\gll
a-ci  va-nga-ci-won-a  i  ci-wonga\\
%%LINE3
\AUG{}-7.\DEM.\PROX{}  2\SM{}-\REL{}-7\OM{}-see-\FV{}  \COP{}  7-cat\\
%%TRANS1
\glt
‘What they saw is a cat.’\\
%%TRANS2
%%EXEND

\z

The free relative clause in the pseudocleft is headed by the demonstrative pronoun. This proximate demonstrative pronoun is joined to the augment \textit{a\nobreakdash-} at the beginning to form \textit{awu} in \xref{bkm:Ref141365224} and \textit{aci} in \xref{bkm:Ref120696902}. The free relative clause describes an entity, and that entity is then identified by the focus (which is the referent of the NP or pronoun). To illustrate: the relative clause \textit{awu ningatimulosa} ‘the one I was greeting’ in \xref{bkm:Ref120695361} presupposes that there is someone greeting someone else, and the second part identifies exactly who was being greeted: \textit{Marta}. The context questions also indicate that the postcopular NP forms the focus, not the precopular part.\largerpage

%%EAX
\ea
%%JUDGEMENT
%%LABEL
\label{bkm:Ref120695361}
%%CONTEXT
\label{bkm:Ref141365224}(Who were you greeting? / \textsuperscript{\#}Who is Marta?)\\
%%LINE1
Awú ningatímúló:sá í Ma:rtá.\\
%%LINE2
\gll
a-wu  ni-nga-ti-mu-los-a  i  Marta\\
%%LINE3
\AUG{}-1\textsc{.dem.prox}  1\SG.\SM{}-\REL{}-\IPFV{}-greet-\FV{}  \COP{}  1.Marta\\
%%TRANS1
\glt
‘The one I was greeting is Marta.’\\
%%TRANS2
%%EXEND

\z

Nevertheless, the presupposition of existence is not necessarily present, considering that the answer to a pseudocleft question can be the empty set, as seen in the felicitous answer to the question in \xref{bkm:Ref136287478}.

\ea
\label{bkm:Ref136287478}
%%EAX
\begin{xlist}[A2:]
\exi{Q:}{
%%JUDGEMENT
%%LABEL
%%CONTEXT
%%LINE1
Awú atsímbítisáku mo:vhá í má:ní?\\
%%LINE2
\gll
a-wu  a-tsimbitis-a=ku  mo:vha  i  ma:ni\\
%%LINE3
\AUG{}-1.\DEM{}  1\SM{}-drive-\FV{}=\REL{}  3.car  \COP{}  who\\
%%TRANS1
\glt
‘Who is driving/can drive a/the car?’, lit. The one driving the car is who?\\
%%TRANS2
}
%%EXEND

%%EAX
\exi{A1:}{
%%JUDGEMENT
%%LABEL
%%CONTEXT
%%LINE1
Hingá:nthu.\\
%%LINE2
\gll
hinga nthu.\\
%%LINE3
\COP.\NEG{} 1.person\\
%%TRANS1
\glt
‘Nobody.’
%%TRANS2
}
%%EXEND

%%EAX
\exi{A2:}{
%%JUDGEMENT
%%LABEL
%%CONTEXT
%%LINE1
Awú atsímbítisáku mo:vhá hingá:nthu.\\
%%LINE2
\gll
a-wu  a-tsimbitis-a=ku  mo:vha  hinga  nthu\\
%%LINE3
\AUG{}-1.\DEM{}  1\SM{}-drive-\FV{}=\REL{}  3.car  \COP.\NEG{} 1.person\\
%%TRANS1
\glt
‘Who can drive a/the car is nobody.’\footnote{The meaning can be ‘who is driving the car’, but a reviewer points out that if a car is driven, it is (in the pre-self-driving era) driven by someone. The interpretation of this sentence can thus only be in a situation of finding out who has a driving licence, translated as ‘who can drive a car’.}
%%TRANS2
}
%%EXEND

\end{xlist}
\z


Pseudoclefts can be used to ask and answer questions. In the answer to a content question, the free relative repeats the given information and then the predicative NP brings the focus answer, as we can see in \xref{bkm:Ref120696779}.

\ea
\label{bkm:Ref120696779}

%%EAX
\begin{xlist}
\exi{Q:}{
%%JUDGEMENT
%%LABEL
%%CONTEXT
%%LINE1
Àwú atsímbítísákú mo:vhá i má:ní?\\
%%LINE2
\gll
a-wu  a-tsimbitis-a=ku  movha  i  mani\\
%%LINE3
\AUG{}-1.\DEM{}.\PROX{}  1\SM{}-drive-\FV{}=\REL{}  3.car  \COP{}  who\\
%%TRANS1
\glt
‘Who is driving the car?’\\
%%TRANS2
}
%%EXEND

%%EAX
\exi{A:}{
%%JUDGEMENT
%%LABEL
%%CONTEXT
%%LINE1
Àwú atsímbítísakú mo:vhá i Sá:ra.\\
%%LINE2
\gll
a-wu  a-tsimbitis-a=ku  movha  i  Sara\\
%%LINE3
\AUG{}-1.\DEM.\PROX{}  1\SM{}-drive-\FV{}=\REL{}  3.car  \COP{}  1.Sara \\
%%TRANS1
\glt
‘The one who’s driving the car is Sara.’
%%TRANS2
}
%%EXEND

\end{xlist}
\z

The focus interpretation is also visible in \xref{bkm:Ref120696892}, where the pseudocleft structure results in ungrammaticality when applying the ‘even’ test which is marked by \textit{hambi} in Cicopi – this, in fact, suggests an exclusive reading. In \xref{bkm:Ref120696902}, the example also shows the unacceptability of parts of an idiomatic expression, in this case ‘the bucket’. Although the sentence is grammatical, it does not retain its idiomatic meaning, because alternatives are generated on the level of the object \{a stone, a ball, a bucket\} and not the whole idiom (pass away).

%%EAX
\ea
%%JUDGEMENT
[*]{
%%LABEL
\label{bkm:Ref120696892}
%%CONTEXT
%%LINE1
Awu angádya ciké:ta i hambí Arli:ndu.\\
%%LINE2
\gll
a-wu  a-nga-dy-a  ci-keta  i  hambi  Arlindu\\
%%LINE3
\AUG{}-1.\DEM.\PROX{}  1\SM{}-\REL{}-eat-\FV{}  7-pineapple  \COP{}  even  1.Arlindu\\
%%TRANS1
\glt
int. ‘The one who ate pineapple was even Arlindo.’\\
%%TRANS2
}
%%EXEND

 
\z

%%EAX
\ea
%%JUDGEMENT
%%LABEL
\label{bkm:Ref120696902}
%%CONTEXT
%%LINE1
Ací angákhá:vá i díbhá:vhu.\\
%%LINE2
\gll
a-ci  a-nga-khav-a  i  di-bhavhu\\
%%LINE3
\AUG{}-\DEM.\PROX{}  1\SM-\REL{}-kick-\FV{}  \COP{}  5-bucket\\
%%TRANS1
\glt
‘What s/he kicked is a bucket.’\\
%%TRANS2
* `S/He passed away’

%%EXEND

\z

Pseudoclefts are naturally used in answers to alternative questions, identifying the referent, as illustrated in \xref{bkm:Ref120696808} and \xref{bkm:Ref120696809}, as well as in selective answers such as \xref{bkm:Ref120696812}, again identifying a subset.

%%EAX
\ea
%%JUDGEMENT
%%LABEL
\label{bkm:Ref120696808}
%%CONTEXT
(Who cut the banana, the big boy or the small boy?)\\
%%LINE1
Awú angawílá nkô:mbvá ḿ’fá:ná wándo:to.\\
%%LINE2
\gll
a-wu  a-nga-wil-a  nkombva  m’-fana  w-a  ndoto\\
%%LINE3
\AUG{}-1.\DEM{}.\PROX{}  1\SM{}-\REL{}-cut-\FV{}  9.banana  \COP{}.1-boy  1-\CONN{}  small\\
%%TRANS1
\glt
‘The one who cut the banana is the small boy.’\\
%%TRANS2
%%EXEND

\z

%%EAX
\ea
%%JUDGEMENT
%%LABEL
\label{bkm:Ref120696809}
%%CONTEXT
(Who has a parrot, Thomas or Samuel? + QUIS picture of Samuel with a parrot.)\\
%%LINE1
Aw’ ákúni cinyáná:na í S\textsuperscript{!}ámué:le.\\
%%LINE2
\gll
a-wu  a=ku=ni  ci-nyanana  i  Samuel\\
%%LINE3
\AUG{}-1.\DEM{}.\PROX{}  1\SM{}-\REL{}=with  7-parrot  \COP{}  1.Samuel\\
%%TRANS1
\glt
‘The one who has a parrot is Samuel.’\\
%%TRANS2
%%EXEND

\z

%%EAX
\ea
%%JUDGEMENT
%%LABEL
\label{bkm:Ref120696812}
%%CONTEXT
(Do these people wear hats? +QUIS picture of two women without hats and two men with hats. )\\
%%LINE1
Ka váthu vá:va, avá vákúni sígo:ko májǎ:há.\\
%%LINE2
\gll
ka  va-thu  vava  a-va  va=ku=ni  si-goko  ma-jaha\\
%%LINE3
\LOC{}  2-people  2.\DEM{}.\PROX{}  \AUG{}-2.\PRO{}  2\SM=\REL{}=with  8-hats  \COP{}.6-man\\
%%TRANS1
\glt
‘Of/between these people, the ones that have hats are the men.’\\
%%TRANS2
%%EXEND

\z

As in other languages, the pseudocleft in Cicopi is thus typically used for identifying a referent, although it remains to be seen if this is an exhaustive identification.

\subsection{Reverse pseudocleft\slash left-peripheral NP + cleft}

In principle, the two parts of the copular construction in a pseudocleft can be reversed, referred to as a reverse pseudocleft, e.g. ‘the men are the ones that have hats’. However, in this construction in Cicopi, the demonstrative in the relative clause is no longer marked with the augment, as shown in \xref{bkm:Ref127264015}, where we see \textit{wu} rather than \textit{awu}. In fact, it can be an independent pronoun, as illustrated by \textit{yona} in \xref{bkm:Ref127264630}. We give two translations, foreshadowing the alternative analysis discussed later.

%%EAX
\ea
%%JUDGEMENT
%%LABEL
\label{bkm:Ref127264015}
%%CONTEXT
(Context: We were talking about school and I mentioned Marta, but you don’t know Marta. So, you ask: ‘Who is Marta?’)\\
%%LINE1
Ma:rta, i \textbf{wú} angárépwé:la cikólwá:ni.\\
%%LINE2
\gll
Marta  i  wu  a-nga-repwel-a  ci-kolwa-ni\\
%%LINE3
1.Marta  \COP{}  1.\DEM{}.\PROX{}  1\SM{}-\REL{}.\PST{}-fail-\FV{}  7-school-\LOC{}\\
%%TRANS1
\glt
‘Marta is the one who failed at school.’ / ‘Marta, it is her who failed at school.’\\
%%TRANS2
%%EXEND

%%EAX
\ex
%%JUDGEMENT
%%LABEL
\label{bkm:Ref127264630}
%%CONTEXT
(Having climbed onto that trunk, his dog also climbed, showing him by scent that the animal we are looking for may be there. Then he sniffed.)\\
%%LINE1
Yí:mbwá, ngu \textbf{yóna} yifémbá:kú ngu makô:ta yá yó:na.\\
%%LINE2
\gll
yi-mbwa  ngu  yi-ona  yi-femb-a=ku  ngu  ma-kota  y-a  y-ona\\
%%LINE3
9-dog  \COP{}  9-\PRO{}  9\SM{}-sniff-\FV{}=\REL{}  \PREP{}  6-nose  6-\CONN{}  9-\PRO{}\\
%%TRANS1
\glt
‘The dog is what sniffs with its snout.’\\
%%TRANS2
%%EXEND

\z

As in other Bantu languages (see \textcite{chapters/kirundi}, \textcite{chapters/rukiga}, \textcite{chapters/kinyakyusa}, \textcite{chapters/kiitharaka}), this apparent ``reverse pseudocleft" may be better analysed as a left-dislocated NP followed by a basic cleft. In this analysis, the clefted constituent is the demonstrative or personal pronoun, referring to the same referent as the initial NP. We suggest that the precopular NP functions as the topic – this would be compatible with the following prosodic break in \xref{bkm:Ref127264015} and \xref{bkm:Ref127264630} above. The clefted pronoun (\textit{wu}/\textit{yona}) then forms the focus.

This suggests that the referent to which the initial NP and the demonstrative refer must be topical information and be in focus at the same time. This is shown in \xref{bkm:Ref120697758}, where the girls are mentioned in the question and can hence be taken up as a (contrastive) topic, but since they are selected to the exclusion of the boys, they are in focus too. The topic function is expressed by the NP and the focus by the demonstrative in the cleft.

%%EAX
\ea
%%JUDGEMENT
%%LABEL
\label{bkm:Ref120697758}
%%CONTEXT
(Did the boys and girls wash their hands?)\\
%%LINE1
\textbf{Mahórá:na} ngu \textbf{wó:na} mangasa:mba mandza:na, majá:há mayásá:mbá.\\
%%LINE2
\gll
ma-horana  ngu  w-ona  ma-nga-samb-a  ma-ndza-ini  ma-jaha  ma-ya-samb-a\\
%%LINE3
6-girl  \COP{}  6-\PRO{}  6\SM{}-\REL{}-wash-\FV{}  6-hand-\LOC{}  6-boy  6\SM{}-\NEG{}-wash-\FV{}\\
%%TRANS1
\glt
‘It was the girls who washed the hands, the boys didn’t.’\\
%%TRANS2
‘(as for) The girls, it’s them who washed hands, the boys didn’t.’

%%EXEND

\z

Idiomatic expressions do not retain their idiomatic reading in this construction, as seen in \xref{bkm:Ref127264997} – this is expected because neither topic nor focus can involve idiomatic expressions.\largerpage

%%EAX
\ea
%%JUDGEMENT
%%LABEL
\label{bkm:Ref127264997}
%%CONTEXT
%%LINE1
Dibhá:vhú, i ci éné angákhá:va.\\
%%LINE2
\gll
di-bhávhú  i  ci  ene  a-nga-khav-a\\
%%LINE3
5-bucket  \COP{}  7.\DEM.\PROX{}  1.\PRO{}  1\SM-\REL{}-kick-\FV{}\\
%%TRANS1
\glt
‘The bucket, it’s what s/he kicked.’\\
%%TRANS2
* `Passing away is what s/he did.’
%%EXEND

\z

The fact that the initial NP cannot be a question word, as shown in \xref{bkm:Ref120698646}, also suggests that this is not just a reverse pseudocleft. Nevertheless, the initial NP can be modified by \textit{dwe} ‘only’, as in \xref{bkm:Ref120698660}. Considering the pause after \textit{Tereza dwe} ‘only Teresa’, we propose that this can be analysed as a fragment answer, followed by a basic cleft, as indicated in the translations.

%%EAX
\ea
%%JUDGEMENT
[*]{
%%LABEL
\label{bkm:Ref120698646}
%%CONTEXT
%%LINE1
Ma:ni i wú híngámuwó:na?\\
%%LINE2
\gll
mani  i  wu  hi-nga-mu-won-a\\
%%LINE3
who  \COP{}  1.\DEM.\PROX{}  1\SM.\PL{}-\REL{}-1\OM{}-see-\FV{}\\
%%TRANS1
\glt
int. ‘Who is it / Who is the one that we saw?’\\
%%TRANS2
}
%%EXEND

%%EAX
\ex
%%JUDGEMENT
[]{
%%LABEL
\label{bkm:Ref120698660}
%%CONTEXT
(We expect to receive Pedro, Paulo, Teresa and others. Did they come?)\\
%%LINE1
Tere:za dwé, i wu angá:ta ntí:ni.\\
%%LINE2
\gll
Tereza  dwe  i  wu  a-nga-t-a  n-t-ini\\
%%LINE3
1.Teresa  only  \COP{}  1.\DEM.\PROX{}  1\SM-\REL{}-come-\FV{}  3\textsc{-}home\textsc{{}-loc}\\
%%TRANS1
\glt
‘It was only Teresa who came home.’\\
%%TRANS2
‘Only Teresa. It’s her who came home.’
}
%%EXEND

\z

In summary, there is some evidence to show that what may at first sight look like a reverse pseudocleft, actually combines an initial NP, which functions as a topic or a fragment answer, with a basic cleft in which a demonstrative pronoun (coreferent with the NP) is in focus. This may be a strategy to express both the topical and the focal function of a referent, but divided the two over the NP and clefted pronoun.

\section{Predicate doubling}
\label{bkm:Ref141344221}
In a predicate doubling construction, the same predicate occurs twice: once in a finite and once in a non-finite form. Out of the three types of predicate doubling that \citet{GüldemannFiedler2022} describe, Cicopi has two predicate doubling constructions: topic doubling and in-situ doubling (discussed below). It does not allow cleft doubling, as shown in \xref{bkm:Ref120698959}, where an infinitive forms the clefted constituent.

%%EAX
\ea
%%JUDGEMENT
[*]{
%%LABEL
\label{bkm:Ref120698959}
%%CONTEXT
%%LINE1
I kúsé:ka hisékâ:ku.\\
%%LINE2
\gll
i  ku-sek-a  hi-sek-a=ku\\
%%LINE3
\COP{}  15-laugh  1\PL.\SM{}-laugh-\FV{}=\REL{}\\
%%TRANS1
\glt
lit. ‘It’s laughing that we laugh.’\\
%%TRANS2
}
%%EXEND

\z

In topic doubling, the infinitive precedes the inflected form of the same verb, as in \xref{bkm:Ref121154546} for the verb \nobreakdash-\textit{bhika} ‘cook’.

%%EAX
\ea
%%JUDGEMENT
%%LABEL
\label{bkm:Ref121154546}
%%CONTEXT
(Context: A mother went out to work and when she returns, she can see that the children are weak. She asks the help ‘Are you cooking for these children?’)\\
%%LINE1
Kubhí:ka hábhî:ka.\\
%%LINE2
\gll
ku-bhika  hi-a-bhik-a\\
%%LINE3
15-cook  1\PL.\SM-\DJ{}-cook-\FV{}\\
%%TRANS1
\glt
‘We do cook (but they don’t eat).’\\
%%TRANS2
%%EXEND

\z

Topic doubling typically implies a contrast with an alternative predicate (e.g. cooking versus eating), and this contrast can be made explicit, as illustrated in \xref{bkm:Ref120700101}. Example \xref{bkm:Ref120700108} shows the same contrast with a transitive predicate, where both verb and object are preposed.
\pagebreak

%%EAX
\ea
%%JUDGEMENT
%%LABEL
\label{bkm:Ref120700101}
%%CONTEXT
(Context: You meet someone on the street and you don’t even greet -- s/he asks whether you’re annoyed.)\\
%%LINE1
Niyákwâ:ta max kuja:ha nijáhi:le.\\
%%LINE2
\gll
ni-ya-kwat-a  max  ku-jaha  ni-jah-ile\\
%%LINE3
1\SG.\SM-\NEG{}-be.angry-\FV{}  but  15-hurry  1\SG.\SM-hurry-\PFV{}\\
%%TRANS1
‘I’m not angry, but I am in a hurry.’
%%TRANS2
%%EXEND

%%EAX
\ex
%%JUDGEMENT
%%LABEL
\label{bkm:Ref120700108}
%%CONTEXT
\label{bkm:Ref141367009}(Context: You were left at home with tasks of washing (clothes) and cooking beans. When mum comes home and sees you sitting, she is annoyed: ‘You didn’t do anything, you’re just sitting here watching television!’)\\
%%LINE1
Niyákuwû:la ká:mbe [kubhika tifeijáu] nibhíkî:le.\\
%%LINE2
\gll
ni-ya-kuwul-a  kambe  ku-bhika  ti-feijau  ni-bhik-ile\\
%%LINE3
1\SG.\SM-\NEG{}-wash-\FV{}  but  15-cook  10-bean  1\SG.\SM{}-cook-\PFV{}\\
%%TRANS1
\glt
‘I didn’t wash, but I \textit{did} cook the beans.’\\
%%TRANS2
%%EXEND

\z

The object can also follow the inflected verb, but the interpretation will not be the same, as shown in \xref{bkm:Ref120700122}. This example can be an answer to ‘what did you cook?’ (object focus) or to confirm that s/he really cooked as recommended. In contrast, when the whole verb phrase (infinitive plus object) precedes the inflected form, as in \xref{bkm:Ref141367009}, object focus is not a possible interpretation.

%%EAX
\ea
%%JUDGEMENT
%%LABEL
\label{bkm:Ref120700122}
%%CONTEXT
%%LINE1
Kubhi:ka nibhíkíle tifeijau.\\
%%LINE2
\gll
ku-bhika  ni-bhik-ile  ti-feijau\\
%%LINE3
15-cook  1\SG.\SM{}-cook-\PFV{}  10-bean\\
%%TRANS1
\glt
‘I cooked (the) beans.’\\
%%TRANS2
%%EXEND

\z

Apart from the implied or explicit contrast, the interpretation may be one of verum (emphatic focus on the truth), as illustrated in \xref{bkm:Ref121149920} and \xref{bkm:Ref103676347}.

%%EAX
\ea
%%JUDGEMENT
%%LABEL
\label{bkm:Ref121149920}
%%CONTEXT
(Context: Someone gave a task and wants to confirm that it is done, saying ‘You are not doing what I said’.)\\
%%LINE1
Kuthu:ma háthû:ma!\\
%%LINE2
\gll
ku-thuma  hi-a-thum-a\\
%%LINE3
15-work  1\PL.\SM-\DJ{}-work-\FV{}\\
%%TRANS1
\glt
‘We are actually doing it.’\\
%%TRANS2
%%EXEND

%%EAX
\ex
%%JUDGEMENT
%%LABEL
\label{bkm:Ref103676347}
%%CONTEXT
(You are not eating the cake that I bought. It’ll go bad.)\\
%%LINE1
Ku:dya hâ:dyá.\\
%%LINE2
\gll
ku-dya  hi-a-dy-a\\
%%LINE3
15-eat  1\PL.\SM{}-\DJ{}-eat-\FV{}\\
%%TRANS1
\glt
‘We \textit{are} eating (it).’\\
%%TRANS2
%%EXEND

\z

Topic doubling can in the right context also have a depreciative meaning, as illustrated in \xref{bkm:Ref120700175} and \xref{bkm:Ref141429345}, or an intensive reading, as in \xref{bkm:Ref121149950}.

%%EAX
\ea
%%JUDGEMENT
%%LABEL
\label{bkm:Ref120700175}
%%CONTEXT
(Context: You’re talking with your friends and someone else is further away; he comes towards you and you change the topic of conversation and he asks why you’re laughing at him, but you say no, we’re just happy.)\\
%%LINE1
Kuse:ka hasê:ka.\\
%%LINE2
ku-seka  hi-a-sek-a\\
%%LINE3
15-laugh  1\PL.\SM-\DJ{}-laugh-\FV{}\\
%%TRANS1
‘We’re just laughing.’
%%TRANS2
%%EXEND

%%EAX
\ex
%%JUDGEMENT
%%LABEL
\label{bkm:Ref141429345}
%%CONTEXT
(Context: There are people with a bad body odour and someone appears asking what is going on with these ones.)\\
%%LINE1
Kusá:mbá vásá:mbá.\\
%%LINE2
\gll
ku-samba  va-a-samb-a\\
%%LINE3
15-bathe  2\SM-\DJ{}-bathe-\FV{}\\
%%TRANS1
\glt
‘They do take a bath (but they don’t get clean).’\\
%%TRANS2
%%EXEND

%%EAX
\ex
%%JUDGEMENT
%%LABEL
\label{bkm:Ref121149950}
%%CONTEXT
(Context: Mary is getting fat and her aunt is surprised with her. What is happening with Mary?)\\
%%LINE1
Ku:dya wâ:dya.\\
%%LINE2
\gll
ku-dya  w-a-dy-a\\
%%LINE3
15-eat  1\SM{}-\DJ{}-eat-\FV{}\\
%%TRANS1
\glt
‘She is eating too much.’\\
%%TRANS2
%%EXEND

\z

The second type of predicate doubling is called in-situ doubling. It features the infinitive in a postverbal position. The verb form of the inflected verb in the present tense can be each of the conjoint, disjoint, or progressive form (see \sectref{bkm:Ref141358518}), as shown in \xref{bkm:Ref120700227} – we do not know if this corresponds to a difference in meaning or use.

%%EAX
\ea
%%JUDGEMENT
%%LABEL
\label{bkm:Ref120700227}
%%CONTEXT
(Why are you laughing?)\\
%%LINE1
Hi-/ha-/ho-seka kúsê:ka.\\
%%LINE2
\gll
hi-/hi-a-/  hi-o-sek-a  ku-seka\\
%%LINE3
1\textsc{pl.sm-/1pl.sm-dj-/}  1\PL.\SM-\PROG{}-laugh-\FV{}  15-laugh\\
%%TRANS1
\glt ‘We are just laughing.’
%%TRANS2
%%EXEND

\z

For in-situ doubling, there are two main interpretations. The first is an intensive reading, as illustrated in \xref{bkm:Ref120700365} and \xref{bkm:Ref120700370}. Note that \xref{bkm:Ref120700365} also illustrates the possibility of a post-infinitival object in in-situ doubling.

%%EAX
\ea
%%JUDGEMENT
%%LABEL
\label{bkm:Ref120700365}
%%CONTEXT
(Context: They hit the child more than a normal spanking.)\\
%%LINE1
Vapékíle kúpê:ka (mwanâ:na).\\
%%LINE2
\gll
va-pek-ile  ku-peka  mw-anana\\
%%LINE3
2\SM{}-hit-\PFV{}  15-hit  1-child\\
%%TRANS1
\glt
‘They really hit the child.’\\
%%TRANS2
%%EXEND

\z

%%EAX
\ea
%%JUDGEMENT
%%LABEL
\label{bkm:Ref120700370}
%%CONTEXT
(Context: They walked a longer distance than usual, perhaps as far as Xai-Xai from Chidenguele.)\\
%%LINE1
Vátsímbíté kútsímbî:la.\\
%%LINE2
\gll
va-tsimbil-ile  ku-tsimbila\\
%%LINE3
2\SM{}-walk-\PFV{}  15-walk\\
%%TRANS1
\glt
‘They really walked!’\\
%%TRANS2
%%EXEND

\z

The second interpretation is again the depreciative, as we can see in \xref{bkm:Ref121149946}.

%%EAX
\ea
%%JUDGEMENT
%%LABEL
\label{bkm:Ref121149946}
%%CONTEXT
%%LINE1
Híthé:te kuthê:la (mâ:ti).\\
%%LINE2
\gll
Hi-thel-ile    ku-thela  mati\\
%%LINE3
1\PL.\SM{}-water-\PFV{}  15-water  6.water\\
%%TRANS1
\glt
‘We watered (the crops) (even if they don’t bear fruit).’\\
%%TRANS2
%%EXEND

\z

We summarise what we know about predicate doubling in Cicopi in \tabref{tab:cicopi-preddoub}.

\begin{table}
\begin{tabularx}{\textwidth}{XXl}
\lsptoprule
 & form & Interpretation\\
\midrule
topic doubling & INF (O) V-fin & verum, depreciative, intensive\\
& INF V-fin O & object focus, confirmation\\
\addlinespace
in-situ doubling & V-fin INF (O) & intensive, depreciative\\
\lspbottomrule
\end{tabularx}
\caption{Predicate doubling options in Cicopi}
\label{tab:cicopi-preddoub}
\end{table}

The precise uses of topic doubling and in-situ doubling in Cicopi, as well as the possibilities with regard to the position of arguments (subject, object), and what these tell us about the underlying syntactic and semantic structure, remain for further research.

\section{Referent expression}
\label{bkm:Ref127267714}
Whether referents are more active or less active in the hearer’s (and speaker’s) mind has an influence on the way they are referred to \citep{Chafe1987}. Highly active referents need less material for successful reference \citep{Ariel1990,GundelEtAl1993}, and it is therefore unsurprising that in Cicopi such active referents can be referred to by a mere subject marker when they function as subjects, as illustrated earlier in this chapter and again in \xref{bkm:Ref124154447}, where we indicate a “null subject” by the empty set symbol.

%%EAX
\ea
%%JUDGEMENT
%%LABEL
\label{bkm:Ref124154447}
%%CONTEXT
(The boy and the dog were still looking for the frog and they found bees on their way and started to run.)\\
%%LINE1
Se, vácípíndí:le a:hu, yímbwá yíngâdí yicótútú:ma, ${\varnothing}$ \textbf{yi}tsú:la. \\
%%LINE2
\gll
se  va-ci-pind-ile  ahu  yi-mbwa  yi-nga-di  yi-ci-o-tutum-a  yi-tsul-a\\
%%LINE3
so  2\SM{}-pass-\PFV{}  16.\DEM{}.\MED{}   9-dog  9\SM{}-still-be  9\SM{}-\CON{}-\PROG{}-run-\FV{}  9\SM{}-leave-\FV{} \\
%%TRANS1
\glt
‘After that, the dog kept running, and left.’\\
%%TRANS2
%%EXEND

%%EAX
\sn
%%JUDGEMENT
%%LABEL
%%CONTEXT
%%LINE1
${\varnothing}$ \textbf{Yi}tsú:té \textbf{yí}yátúmbé:lá, akáyíwô:ne.\\
%%LINE2
\gll
yi-tsul-ile  yi-ya-tumbel-a  a-ka-yi-won-i\\
%%LINE3
9\SM{}-leave-\PFV{}  9\SM{}-go-hide-\FV{}  1\SM{}-\NEG{}-9\OM{}-see-\NEG{}\\
%%TRANS1
\glt
‘He went to hide, he doesn’t see him.’\\
%%TRANS2
%%EXEND

\z

The active object can equally be expressed pronominally, by object marking on the verb, as in \xref{bkm:Ref124152314} from the QUIS map task: speaker A activates the concept ‘butterfly’, and speaker B then refers to it by an object marker \textit{di\nobreakdash-}.

\ea
\begin{xlist}
%%EAX
\exi{A:}
%%JUDGEMENT
%%LABEL
\label{bkm:Ref124152314}
%%CONTEXT
%%LINE1
Nákwélélá ngú cíné:né; nímána dipháphálátá:ni.\\
%%LINE2
\gll
n-a-kwel-el-a  ngu  cinene;  ni-man-a  di-phaphalatani\\
%%LINE3
1\SM{}-\PRS{}-go.up-\APPL{}-\FV{}  \PREP{}  7.right   1\SG.\SM-{}find-\FV{}  5-butterfly\\
%%TRANS1
\glt
‘I go to the right, I find a butterfly.’\\
%%TRANS2
%%EXEND

%%EAX
\exi{B:}
%%JUDGEMENT
%%LABEL
%%CONTEXT
%%LINE1
Ní\textbf{dí}má:ne, ni\textbf{dí}má:ne.\\
%%LINE2
\gll
ni-di-man-e  ni-di-man-e\\
%%LINE3
1\SG.\SM{}-5\OM{}-find-\PFV{}  1\SG.\SM{}-5\OM{}-find-\PFV{}\\
%%TRANS1
\glt
‘I have found it.’
%%TRANS2
%%EXEND

\end{xlist}
\z

Out of context, it seems that the object marker can only function as a pronoun. Example \xref{bkm:Ref127267640:a} shows that the object marker on the verb and the coreferent NP cannot be in the same domain – the object marker is only allowed when the coreferent NP is extraposed, as in \xref{bkm:Ref127267640:b}. This was tested with question words, since we know for sure that those cannot be dislocated, and therefore the object-marked object preceding the question word (e.g. \textit{ndiyawe} ‘his sister’) must also be in situ in the same domain. The same is illustrated in \xref{bkm:Ref141373727}, where the object marker \textit{ma}\nobreakdash- cannot be present if the coreferent object \textit{mamanga} ‘mangos’ is in the same domain. 

\ea\label{bkm:Ref127267640}
%%EAX
\ea
%%JUDGEMENT
[*]{
%%LABEL
\label{bkm:Ref127267640:a}
%%CONTEXT
%%LINE1
Váná\textbf{mú}n\textsuperscript{!}íngá \textbf{ndíyâ:wé} câ:ni?\\
%%LINE2
\gll
va-na-mu-ning-a  ndiya-awe  cani\\
%%LINE3
2\SM{}-\FUT{}-1\OM{}-give-\FV{}  1.sister-\POSS{}.1  what\\
%%TRANS1
\glt
‘What will they give his sister?’\\
%%TRANS2
}
%%EXEND


  %%EAX
\ex
%%JUDGEMENT
[]{
%%LABEL
\label{bkm:Ref127267640:b}
%%CONTEXT
%%LINE1
Váná\textbf{mú}n\textsuperscript{!}íngá cá:ní \textbf{ndíyâ:wé}?\\
%%LINE2
\gll
va-na-mu-ning-a  cani  ndiya-awe\\
%%LINE3
2\SM{}-\FUT{}-1\OM{}-give-\FV{}  what  1.sister-\POSS{}.1\\
%%TRANS1
\glt
‘What will they give his sister?’\\
%%TRANS2
}
%%EXEND

\z

%%EAX
\ex
%%JUDGEMENT
%%LABEL
\label{bkm:Ref141373727}
%%CONTEXT
%%LINE1
U(*\textbf{ma})xaveté \textbf{máma:nga} mâ:ni?\\
%%LINE2
\gll
u-ma-xav-el-ile  ma-manga  mani\\
%%LINE3
2\SG.\SM{}-6\OM{}-buy-\APPL{}-\PFV{}  6-mangos  who\\
%%TRANS1
\glt
‘Who did you buy mangoes for?’\\
%%TRANS2
%%EXEND

\z

However, we do find examples of the cooccurrence of an object and an object marker, in contexts where the object referent is active. In examples \xref{bkm:Ref127267749} and \xref{bkm:Ref127267750}, the verb shows penultimate lengthening, so the object seems to be phrased separately from the verb, but there is no pause to indicate dislocation. We leave further analysis of Cicopi object marking for future research, but refer to \citegen{SikukuDiercks2021} findings for Lubukusu, where doubling is accepted in verum and mirative contexts when the object is given.\largerpage

\ea
\begin{xlist}
%%EAX
\exi{A:}
%%JUDGEMENT
%%LABEL
\label{bkm:Ref127267749}
%%CONTEXT
%%LINE1
Nikwélé:lá nimáná nyú:mbá yóbhílívi:la.\\
%%LINE2
\gll
ni-kwelel-a  ni-man-a  nyumba  yi-a-ku-bhilivila\\
%%LINE3
1\SG.\SM{}-climb-\FV{}  1\SG.\SM{}-find-\FV{}  9.house  9-\CONN{}-15-be.red\\
%%TRANS1
\glt
‘I go up and find a red house.’\\
%%TRANS2
%%EXEND

%%EAX
\exi{B:}
%%JUDGEMENT
%%LABEL
%%CONTEXT
%%LINE1
Na\textbf{yí}wó:na nyú:mba, má:ji áni já níyíwóná háhá:tshí.\\
%%LINE2
\gll
ni-a-yi-won-a  nyumba,  maji  ani  ja  ni-yi-won-a  ha-hatsi\\
%%LINE3
1\SG.\SM{}-\DJ{}-9\OM{}-see-\FV{}  9.house  but  1\SG{}.\PRO{}  now  1\SG.\SM{}-9\OM{}-see-\FV{}  16-down\\
%%TRANS1
\glt
‘I see (it) the house, but below.
%%TRANS2
%%EXEND

\end{xlist}

%%EAX
\ex
%%JUDGEMENT
%%LABEL
\label{bkm:Ref127267750}
%%CONTEXT
(Could it be that Pedro found the phone?)\\
%%LINE1
Pédrú áyím\textsuperscript{!}ání fó:ní sála:ni.\\
%%LINE2
\gll
Pedro  a-yi-man-e  foni  sala-ini\\
%%LINE3
1.Pedro  1\SM{}-9\OM{}-find-\PFV{}  9.phone  5.room-\LOC{}\\
%%TRANS1
\glt
‘Pedro found the phone in the room.’\\
%%TRANS2
%%EXEND

\z

Note that it is also perfectly acceptable to completely drop the object without any object marking, as illustrated in \xref{bkm:Ref124153589} and \xref{bkm:Ref124153591}.

%%EAX
\ea
%%JUDGEMENT
%%LABEL
\label{bkm:Ref124153589}
%%CONTEXT
%%LINE1
Niwoné mu:ti. […] Ntó(yí)da:ya.\\
%%LINE2
\gll
ni-won-e  moti  ni-to-yi-day-a\\
%%LINE3
1\SG.\SM{}-see-\PFV{}  9.gazella  1\SG.\SM{}-\TO{}-9\OM{}-kill-\FV{}\\
%%TRANS1
\glt
‘I saw a gazella. I killed (it).’\\
%%TRANS2
%%EXEND

\z

%%EAX
\ea
%%JUDGEMENT
%%LABEL
\label{bkm:Ref124153591}
%%CONTEXT
(Did you eat (the) bread?)\\
%%LINE1
I:ná, nídyí:té.\\
%%LINE2
\gll
ina  ni-dy-ile\\
%%LINE3
yes  1\SG.\SM{}-eat-\PFV{}\\
%%TRANS1
\glt
‘Yes, I ate (it).’\\
%%TRANS2
%%EXEND

\z

We saw in Sections~\ref{bkm:Ref124153637} and~\ref{bkm:Ref121988320} that mentally active referents may be expressed by a noun phrase in the left or right periphery, and in this section we have seen that they may also be expressed by a subject or object marker on the verb, or (at least in the case of objects) be omitted completely. Further research is needed to establish what determines whether an object is expressed by an object marker or dropped altogether.

\section{Conclusion}

As we said in the introduction, the aim of this chapter is to give a general overview of the expression of information structure in Cicopi, and we have provided a first description and illustration of the information-structural functions of verbal inflection, word order, three types of clefts, predicate doubling, and referent expression. We can summarise the chapter as follows.

First, Cicopi has three forms in the present tense (conjoint, which is marked by zero morpheme; disjoint/habitual, marked by \textit{a-}; and progressive, marked by \textit{o-}). The precise conditions under which these are used require further investigation, but it seems to depend on a complex interaction between constituent-finality, focus, aspect, and perhaps evidentiality. The conjoint/disjoint alternation in any case seems to be determined more by constituency than by focus directly. In the perfective, there are two forms, \textit{{}-ile/-ite} and \textit{to}{}-. The \textit{to-} form is felicitous in expressing state-of-affairs focus (contrasting the lexical value of the verb), but is not felicitous in a verum context. The perfective \textit{{}-ile/-ite} shows the opposite behaviour and is felicitously used to express verum. The alternating verb forms in Cicopi are thus restricted by a less straightforward set of conditions than is known from other languages (see e.g. the parametric variation described by \citealt{vanderWal2017}); especially the potential link with evidentiality is remarkable.

Second, as in many other Bantu languages, the preverbal domain is restricted to non-focal constituents and prefers preverbal topics, and the postverbal domain contains non-topical and focal constituents. There may be an immediate-after-verb preference for focus, but Cicopi shows no restriction to a specific focus position, and focused constituents can appear in any postverbal position. Moreover, Cicopi allows multiple question words postverbally. The relatively complex set of factors determining the use of the conjoint and disjoint verb forms may have a correlation with the lack of a focus position in Cicopi: we can imagine that the (direct or indirect) relation between verb form and focus is not as clear as in a language with a fixed focus position (such as Zulu or Makhuwa, for example), and that therefore other factors (aspect, evidentiality) are more prominent than for other languages.

Third, the basic cleft shows some properties of an exclusive focus interpretation, but its uses seem to be broader, also being accepted in thetic environments, with cognate objects and parts of idioms. Pseudoclefts are also used to express focus, and what at first sight looks like a reverse pseudocleft was shown to be analysable as an initial NP functioning as a topic or a fragment answer, followed by a clefted (demonstrative) pronoun, similar to the construction in Kirundi, Rukiga, Kîîtharaka, and Kinyakyusa (see the other chapters in \citealt{langsci-current-book}). Further investigation is needed to establish the precise interpretations and the underlying structure of these cleft constructions.

Fourth, Cicopi has two predicate doubling constructions: topic doubling and in-situ doubling. The topic doubling shows an interpretation of verum and can also have a depreciative and intensive meaning (as also found in Kirundi, Rukiga, and Kîîtharaka), but further research is necessary to determine the precise uses of each predicate doubling construction, as well as the possibilities with regard to the position of arguments.

Finally, we have seen that active referents in Cicopi, in line with universal tendencies \citep{Gundel1988, GundelEtAl1993}, may be expressed by a noun phrase in the left or right periphery, particularly when indicating a shift topic, and may also be expressed just by a subject or object marker on the verb, or be omitted completely.

We hope that this chapter forms the beginning of many further discoveries about Cicopi and how it structures its information between speaker and addressee.

\section*{Acknowledgements}

This research was supported by NWO Vidi grant 276-78-001 as part of the BaSIS “Bantu Syntax and Information Structure” project at Leiden University. We thank Constância Zaida Mussavele, Arlindo João Nhantumbo, Gomes David Chemane, Hortência Ernesto, Gervásio Chambo, Engrácia Ernesto for sharing their insights on their language with us; we thank two reviewers for their helpful comments, and the BaSIS colleagues for their support. Any remaining errors are ours alone.

\section*{Abbreviations and symbols}

Numbers refer to noun classes unless followed by \SG{}/\PL{}, in which case the number (1 or 2) refers to first or second person. The orthography for Cicopi has been followed, including the following conventions: b [ɓ], bh [b], d [\textsf{ɗ}], dh [d], v [ʋ], vh [v], c [c], ch [c\textsuperscript{h}], j [ʒ], x [ʃ]. Tone marking indicates surface tone including intonation (to the best of our ability, and likely incorrect in places); high tones are marked by an acute accent; low tones remain unmarked, falling tones are marked by a circumflex accent. An apostrophe (as in \textit{m’fana}) indicates a syllabic nasal. Vowel length is indicated by /:/ and vowel nasalisation is indicated by m in the coda, as in \textit{ihimhim} [ihḭ:hḭ].

%%% All Leipzig abbreviations are commented out, following the LangSci guidelines of only listing non-Leipzig abbreviations.
\begin{multicols}{2}
\begin{tabbing}
MMMM \= ungrammatical\kill
* \> ungrammatical\\
\textsuperscript{\#} \> infelicitous in the given \\ \> context\\
\textsuperscript{!} \> downstep\\
*(X) \> the presence of X is obligatory \\ \> and cannot grammatically \\ \> be omitted\\
(*X) \> the presence of X would make \\ \> the sentence ungrammatical\\
(X) \> the presence of X is optional\\
% \APPL{} \> applicative\\
\AUG{} \> augment \\
% \CAUS{} \> causative\\
\CJ{} \> conjoint\\
\CL{} \> class marker\\
% \COMP{} \> complementiser\\
\CON{} \> consecutive\\
\CONN{} \> connective\\
% \COP{} \> copula\\
DAI \> Default Agreement Inversion\\
\DEP{} \> dependent conjugation\\
% \DEM{} \> demonstrative\\
% \DIST{} \> distal\\
\DJ{} \> disjoint\\
% \FUT{} \> future\\
\FV{} \> final vowel\\
% \GEN{} \> genitive\\
% \IPFV{} \> imperfective\\
\LIM{} \> limit (‘end up V-ing’)\\
\LINK{} \> linker (unsure)\\
% \LOC{} \> locative\\
\MED{} \> medial\\
% \NEG{} \> negation\\
\OM{} \> object marker\\
\PREP{} \> preposition\\
Q \> question\\
QUIS \> Questionnaire on Information \\ \> Structure \citep{SkopeteasEtAl2006}\\
% \PFV{} \> perfective\\
% \PL{} \> plural\\
% \POSS{} \> possessive\\
\POT{} \> potential\\
\PRO{} \> pronoun\\
% \PROG{} \> progressive\\
% \PROX{} \> proximal\\
% \PRS{} \> present\\
% \PST{} \> past\\
\RED{} \> reduplication\\
% \REL{} \> relative\\
% \SG{} \> singular\\
\SM{} \> subject marker\\
\STAT{} \> stative\\
\TO{} \> \textit{to}{}- morpheme in perfective \\ \> conjugation\\
\end{tabbing}
\end{multicols}

\printbibliography[heading=subbibliography,notkeyword=this]
\end{document}
