\documentclass[output=paper]{langscibook}
\ChapterDOI{10.5281/zenodo.14833616}
\author{Amani Lusekelo\orcid{}\affiliation{University of Dar es Salaam} and Jenneke van der Wal\orcid{}\affiliation{Leiden University} and Simon Msovela\orcid{}\affiliation{University of Gothenburg}}
\title{The expression of information structure in Kinyakyusa}
\abstract{This chapter presents strategies for the expression of topic and focus in sentences of Kinyakyusa (spoken in Tanzania and Malawi). In Kinyakyusa, there is no dedicated focus position immediately after the verb or elsewhere, and remarkably, focus may occur preverbally. Topic doubling, which is common in Kinyakyusa, can provide a verum, intensive or depreciative reading. Furthermore, the interpretation of the focused element in the cleft is mainly for identificational purposes. Lastly, Kinyakyusa has the V augment and CV prefix in the nominal structure. The CV prefix is an exhaustive marker.   }
\IfFileExists{../localcommands.tex}{
  \addbibresource{../localbibliography.bib}
  % add all extra packages you need to load to this file

\usepackage{tabularx,multicol}
\usepackage{url}
\urlstyle{same}

\usepackage{listings}
\lstset{basicstyle=\ttfamily,tabsize=2,breaklines=true}

\usepackage{langsci-basic}
\usepackage{langsci-optional}
\usepackage{langsci-lgr}
\usepackage{langsci-osl}
% \usepackage{./langsci/styles/langsci-lgr}
% \usepackage{./langsci/styles/langsci-osl}
% \usepackage{langsci-gb4e}

\usepackage{tikz}
\usetikzlibrary{patterns,calc}
\pgfdeclarepatternformonly{south east lines}{\pgfqpoint{-0pt}{-0pt}}{\pgfqpoint{3pt}{3pt}}{\pgfqpoint{3pt}{3pt}}{
    \pgfsetlinewidth{0.6pt}
    \pgfpathmoveto{\pgfqpoint{0pt}{3pt}}
    \pgfpathlineto{\pgfqpoint{3pt}{0pt}}
    \pgfpathmoveto{\pgfqpoint{.2pt}{-.2pt}}
    \pgfpathlineto{\pgfqpoint{-.2pt}{.2pt}}
    \pgfpathmoveto{\pgfqpoint{3.2pt}{2.8pt}}
    \pgfpathlineto{\pgfqpoint{2.8pt}{3.2pt}}
    \pgfusepath{stroke}}
    
\usepackage{stmaryrd}
\usepackage{wasysym}
\usepackage{multirow}
\usepackage{caption}
\usepackage{subcaption}
\usepackage{mathrsfs}
\usepackage{qtree}

\usepackage{linguex}


  %pminos do not split footnotes
% \interfootnotelinepenalty=10000 %Footnote in Laporte chapters has to be split SN


%\DeclareIndexNameFormat{default}{%
%\nameparts{#1}%
%\usebibmacro{index:name}%
%{\index[names]}%
%{\namepartfamily}%
%{\namepartgiveni}%
% {}% L1
% {}% L2
%{\namepartprefix}% generates spurious space L3
%{\namepartsuffix}% generates spurious space L4
%}

%  {\DeclareIndexNameFormat{default}{%
%     \usebibmacro{index:name}{\index[names]}{#1}{#3}{#5}{#7}}}

%\DeclareIndexNameFormat{default}{%
%  \usebibmacro{index:name}{\sindex[nom]}{#1}{#3}{#5}{#7}}

%\DeclareIndexNameFormat{default}{%
%  \usebibmacro{index:name}{\sindex[person]}{#1}{#3}{#5}{#7}}
%\DeclareIndexNameFormat{default}{%
%\nameparts{#1} \usebibmacro{index:name}{\sindex[person]]}{\namepartfamily}{‌​\namepartgiven}{\nam‌​epartprefix}{\namepa‌​rtsuffix}}

%\newcommand{\smiley}{:)}

%\renewbibmacro*{index:name}[5]{%
%\usebibmacro{index:entry}{#1}%
%{\iffieldundef{usera}{}{\thefield{usera}\actualoperator}\mkbibindexname{#2}{#3}{#4}{#5}}}

% \newcommand{\noop}[1]{}

%remove for final
%\overfullrule=1mm

\newcommand{\tobi}[2]}}
\renewcommand{\S}[1]{\tobi{#1}{\textsc{*}}}

% this volume references
% puts: [this volume]
% already defined: \citetv
%\newcommand{\citepv}[1]{(\citeauthor{#1} \citeyear*{#1} [this volume])}
\newcommand{\citealtv}[1]{\citeauthor{#1} \citeyear*{#1} [this volume]}

%parentheses around example number
\newcommand{\pref}[1]{(\ref{#1})}

% in-text examples

\newcommand{\lnex}[1]{\textit{#1}} %target lang word
\newcommand{\lnlit}[1]{(lit.: `#1')} %literal reading
\newcommand{\lnlat}[1]{(#1)} % latinization
\newcommand{\lntrans}[1]{`#1'} %translation
\newcommand{\lnexl}[2]%
{\lnex{#1}{} \lnlat{#2}} % ex with latinization
\newcommand{\lnexlat}[3]{\lnex{#1}{} \lnlat{#2}{} \lntrans{#3}} % ex with latinization and tranl.

%ch01
\newcommand{\co}[1]{\mbox{\textbf{#1}}}

%ch09

\newcommand{\cyrbulg}[1]{\begin{otherlanguage*}{bulgarian}#1\end{otherlanguage*}}


%ch10
\newcommand{\nlp}{{\small NLP}}
\newcommand{\mwe}{{\small MWE}}
\newcommand{\rae}{{\small RAE}}
\newcommand{\lvc}{{\small LVC}}
\newcommand{\pos}{{\small P}o{\small S}}
%\newcommand{\todo}[1]{ \textcolor{red}{#1} }

%\renewcommand{\labelenumi}{\theenumi}
%\ainamefmt{{vv}{ll}{, ff}{, jj}} % fullname

\newcommand{\biberror}[1]{{\color{red}#1}}

\newcommand{\osenovaitem}{--~} 
  %% hyphenation points for line breaks
%% Normally, automatic hyphenation in LaTeX is very good
%% If a word is mis-hyphenated, add it to this file
%%
%% add information to TeX file before \begin{document} with:
%% %% hyphenation points for line breaks
%% Normally, automatic hyphenation in LaTeX is very good
%% If a word is mis-hyphenated, add it to this file
%%
%% add information to TeX file before \begin{document} with:
%% %% hyphenation points for line breaks
%% Normally, automatic hyphenation in LaTeX is very good
%% If a word is mis-hyphenated, add it to this file
%%
%% add information to TeX file before \begin{document} with:
%% \include{localhyphenation}
\hyphenation{
    Beck-man
    Ngu-yen
    back-chan-nel
    back-chan-nels
    mo-not-o-nous
    ste-reo-typ-i-cal
}

\hyphenation{
    Beck-man
    Ngu-yen
    back-chan-nel
    back-chan-nels
    mo-not-o-nous
    ste-reo-typ-i-cal
}

\hyphenation{
    Beck-man
    Ngu-yen
    back-chan-nel
    back-chan-nels
    mo-not-o-nous
    ste-reo-typ-i-cal
}
 
  \togglepaper[1]%%chapternumber
}{}

\begin{document}
\maketitle 
\label{ch:7}
%\shorttitlerunninghead{}%%use this for an abridged title in the page headers


\section{Introduction}

Kinyakyusa is spoken in south-west Tanzania and north-east Malawi by more than one million people \citep{Felberg1996,LoT2009}. It is classified as M31 in \citegen{Maho2009} update of \citegen{Guthrie1948} classification and has ISO code [nny]. The three dialects of Kinyakyusa include the southern dialect called Ngonde (spoken along the shores of Lake Nyasa/Malawi in Malawi), the eastern dialect called Selya (spoken in the mountainous parts north-east of Lake Nyasa/Malawi in Tanzania) and the western dialect called Mwamba (spoken in the mountanous parts north-west of Lake Nyasa/Malawi Tanzania). The existing literature covers the emphatic marker(s) \citep{MwangokaVoorhoeve1960}, locative clitics \citep{Persohn2017}, tense, aspect, and modality \citep{Persohn2020}, object marking \citep{LusekeloFut}, the structure of conditional sentences \citep{Lusekelo2016}, noun phrase structure \citep{Lusekelo2009}, the V and CV ``augment" \citep{vanderWalLusekelo2022}, and subject inversion \citep{MsovelaEtAl2023}. 

The description is based on data gathered from three native speakers of Kinyakyusa aged between 48 and 66 during fieldwork in November 2020 in Kiwira and come from spontaneous conversation, narratives, and elicitation. Additional data were provided by the first author, who is a Kinyakyusa native speaker, and checked with speakers remotely. We used the BaSIS project methodology, available through the Leiden Repository.\footnote{\url{https://scholarlypublications.universiteitleiden.nl/handle/1887/3608096}} The data were transcribed and stored in an Online Language Database accessible through the Dative user interface that allows data sharing in collaborative research. More information about Dative can be accessed via htps://www.dative.ca (and see the introduction to this book). This database will be accessible through The Language Archive. We also refer to the introduction to this book for further background on the terms and diagnostics used for information structure.

Kinyakyusa is not a tonal language as it lost its inherent (lexical) tone \citep[38]{Persohn2020}. While other Kinyakyusa variants have 7 contrastive vowels, the variant spoken in Kiwira (probably much of Mwamba dialect) does not make a phonological distinction between ɪ/i and ʊ/u. Variation between ʊ/u is mostly absent in speech, and although two of our speakers did produce both ɪ and i, neither they nor the first author differentiated them meaningfully as far as we could test. We therefore represent five vowels only. Similarly, we write /l/ for the tap sound that has [l] and [ɾ] as allophonic variants. We do not represent the vowel length caused by automatic compensatory lengthening in syllables with a labialised or palatalised onset (Cy-/Cw-) and/or a prenasalised coda (-nC). Finally, note that there are two series of subject markers, where series 2 (class 1 u-/i-, class 2 bi-, class 6 gi-) is used preceding the tense marker -\textit{ku}-, and series 1 (class 1 a-, class 2 ba-, class 6 ga-) is used elsewhere \citep[see][3]{Persohn2020}.

The aim of this chapter is to present the morphosyntactic ways in which Kinyakyusa speakers organise the information in a sentence. We start in \sectref{bkm:Ref136514368} by showing to what extent information structure influences word order, including subject inversion constructions. Then in \sectref{bkm:Ref114648496}, we show the interpretations of two predicate doubling constructions: topic doubling and in-situ doubling. A topic marker \textit{po} is discussed in \sectref{bkm:Ref123889293}; three types of clefts are analysed regarding their form and function in \sectref{bkm:Ref136514487} (the basic cleft, pseudocleft, and reverse pseudocleft), and the CV exhaustive marker is shown in \sectref{bkm:Ref122533116}. 

\section{Word order}
\label{bkm:Ref136514368}
Kinyakyusa shows interesting word order properties compared to other Bantu languages. Three properties stand out. First, there is no dedicated focus position (\sectref{bkm:Ref125381586}){.} Second, the left periphery of the sentence is preferred for various kinds of topic, though nouns in focus can also occur there (\sectref{bkm:Ref114653225}). Third, while some verbs allow locative inversion, in general only agreeing inversion is attested in the language (\sectref{bkm:Ref122073759}). See also \citet{KerrEtAl2023} and \citet{MsovelaEtAl2023}.

\subsection{No dedicated focus position postverbally}
\label{bkm:Ref125381586}
Kinyakyusa canonical word order can be characterised as SVO, although given referents are typically expressed pronominally (in the form of subject and object markers). There is no dedicated focus position immediately after the verb or elsewhere. This section discusses the possibilities of focusing a noun postverbally, drawing on \citet{KerrEtAl2023}. 

In the postverbal position, both the Recipient and Theme can be in focus, in either position. One of the tests for a dedicated position of focus concerns question words, as they are inherently focused. Both the Recipient and the Theme can be questioned postverbally. The Theme is questioned in \xref{bkm:Ref136343831}, while the Recipient is questioned in \xref{ex:postv-qrecipient}.

\ea
\label{bkm:Ref136343831}
%%EAX
\ea
%%JUDGEMENT
%%LABEL
\label{bkm:Ref136343831:a}
%%CONTEXT
%%LINE1
Untupe ukump’ \textbf{ifiki} unsekele?\\
%%LINE2
\gll
u-n-tupe  a-ku-m-p-a  i-fi-ki  u-n-sekele\\
%%LINE3
\AUG{}-1-fat  1\SM{}-\PRS{}-1\OM{}-give-\FV{}  \AUG{}-8-what  \AUG{}-1-thin\\
%%TRANS1
\glt ‘What is the fat one giving the thin one?'
%%TRANS2
%%EXEND

%%EAX
\ex
%%JUDGEMENT
%%LABEL
\label{bkm:Ref136343831:b}
%%CONTEXT
%%LINE1
{Untupe} {ukumpa} {unsekel’} \textbf{ifiki?}\\
%%LINE2
\gll
{u-n-tupe}  {a-ku-m-p-a}  {u-n-sekele}  i-fi-ki\\
%%LINE3
{\AUG{}-1-fat}  {1\SM-\PRS-1\OM{}-give-\FV{}}  {\AUG{}-1-thin}  {\AUG{}-8-what}\\
%%TRANS1
\glt
‘What is the fat one giving the thin one?’\\
%%TRANS2
%%EXEND

\z

\ex\label{ex:postv-qrecipient}
%%EAX
\ea
%%JUDGEMENT
%%LABEL
\label{ex:postv-qrecipient:a}
%%CONTEXT
%%LINE1
Untupe ukumpa \textbf{juani} ikipale?\\
%%LINE2
\gll
u-n-tupe  a-ku-m-p-a  ju-ani  i-ki-pale\\
%%LINE3
\AUG{}-1-fat  1\SM{}-\PRS{}-1\OM{}-give-\FV{}  1-who  \AUG{}-7-calabash\\
%%TRANS1
\glt
‘Who is the fat one giving a calabash?’\\
%%TRANS2
%%EXEND

%%EAX
\ex
%%JUDGEMENT
%%LABEL
\label{ex:postv-qrecipient:b}
%%CONTEXT
%%LINE1
{Untupe} {ukumpa} {ikipale} \textbf{juani?}\\
%%LINE2
\gll
{u-n-tupe}  {a-ku-m-p-a}  {i-ki-pale}  ju-ani\\
%%LINE3
{\AUG{}-1-fat}  {1\SM-\PRS-1\OM{}-give-\FV{}}  {\AUG{}-7-calabash}  {1-who}\\
%%TRANS1
\glt
‘Who is the fat one giving the calabash?’\\
%%TRANS2
%%EXEND

\z
\z


Kinyakyusa allows any noun, the Recipient or the Theme, to be an answer to a content question (and hence be in focus) postverbally, as illustrated in \xref{bkm:Ref90047925}. Both examples in \xref{bkm:Ref90047925} can be an answer to either \xref{bkm:Ref136343831:a} or \xref{bkm:Ref136343831:b} and also \xref{ex:postv-qrecipient:a} and \xref{ex:postv-qrecipient:b}. Therefore, both Theme and Recipient can be focused in any postverbal position. Note also that the relative animacy of the two objects has no influence on the word order.

\ea
\label{bkm:Ref90047925}
%%EAX
\ea
%%JUDGEMENT
%%LABEL
%%CONTEXT
%%LINE1
Untupe ukumpa ikipale unsekele.\\
%%LINE2
\gll
u-n-tupe  a-ku-m-p-a  i-ki-pale  u-n-sekele\\
%%LINE3
\AUG{}-1-fat  1\SM{}-\PRS{}-1\OM{}-give-\FV{}  \AUG{}-7-calabash  \AUG{}-1-thin\\
%%TRANS1
\glt
‘The fat one gives the thin one a calabash.’\\
%%TRANS2
%%EXEND

%%EAX
\ex
%%JUDGEMENT
%%LABEL
%%CONTEXT
%%LINE1
{Untupe} {ukumpa} {unsekele} {ikipale}.\\
%%LINE2
\gll
{u-n-tupe}  {a-ku-m-p-a}  {u-n-sekele}  {i-ki-pale}\\
%%LINE3
{\AUG{}-1-fat}  {1\SM-\PRS-1\OM{}-give-\FV{}}  {\AUG{}-1-thin}  {\AUG{}-7-calabash}\\
%%TRANS1
\glt
‘The fat one gives the thin one a calabash.’\\
%%TRANS2
%%EXEND

\z\z


Modification by the exhaustive particle  -\textit{ene} ‘only’ is also used to test the noun in focus in any postverbal position, because the particle associates with focus. In \xref{bkm:Ref136343822}, the Recipient can be modified by ‘only’ in postverbal position. Likewise, the Theme modified by ‘only’ can occur after the Recipient, as illustrated in \xref{bkm:Ref90048089}.

\ea
%%EAX
\label{soda}
\ea
%%JUDGEMENT
%%LABEL
\label{bkm:Ref136343822}
%%CONTEXT
%%LINE1
Ampele mwene Sekela isoda.\\
%%LINE2
\gll
a-m-p-ile  mu-ene  Sekela  i-soda\\
%%LINE3
1\SM{}-1\OM{}-give-\PFV{}  1-only  1.Sekela  \AUG{}-9.soda\\
%%TRANS1
\glt
‘S/he has given only Sekela soda.’\\
%%TRANS2
%%EXEND

%%EAX
\ex
%%JUDGEMENT
%%LABEL
\label{bkm:Ref90048089}
%%CONTEXT
%%LINE1
{Ampele} {Sekela} {jeene} {isoda}.\\
%%LINE2
\gll
{a-m-p-ile}  {Sekela}  {ji-ene}  {i-soda}\\
%%LINE3
{1\SM-1\OM{}-give-\PFV{}}  {1.Sekela}  {9-only}  {\AUG{}-9.soda}\\
%%TRANS1
\glt
‘S/he has given Sekela only soda.’\\
%%TRANS2
%%EXEND

\z
\z

The focus constituent therefore does not need to appear in a dedicated position, but can appear anywhere postverbally, as shown with respect to the arguments in \xxref{bkm:Ref136343831}{soda} above. The same is true also for adverbs, as illustrated in \xref{bkm:Ref90636091}, where we also show the possibility for multiple questions.

\ea
\label{bkm:Ref90636091}
%%EAX
\ea
%%JUDGEMENT
%%LABEL
\label{bkm:Ref90636091:a}
%%CONTEXT
%%LINE1
%%LINE2
\gll
Atu  a-biik-ile  i-ki-pale   kugu?\\
%%LINE3
1.Atu  1\SM{}-put-\PFV{}  \AUG{}-7-calabash   17.where~\\
%%TRANS1
\glt
‘Where did Atu put the calabash?’\\
%%TRANS2
%%EXEND


%%EAX
\ex
%%JUDGEMENT
%%LABEL
\label{bkm:Ref90636091:b}
%%CONTEXT
%%LINE1
%%LINE2
\gll
Atu  a-biik-ile  kugu   i-ki-pale?\\
%%LINE3
1.Atu  1\SM{}-put-\PFV{}  17.where   \AUG{}-7-calabash~\\
%%TRANS1
\glt
‘Where did Atu put the calabash?’\\
%%TRANS2
%%EXEND



%%EAX
\ex
%%JUDGEMENT
%%LABEL
\label{bkm:Ref90636091:c}
%%CONTEXT
%%LINE1
Abiikile ifiki kugu?\\
%%LINE2
\gll
a-biik-ile  i-fi-ki  kugu\\
%%LINE3
1\SM{}-put-\PFV{}  \AUG{}-8-what  17.where\\
%%TRANS1
\glt
‘What did s/he put where?’\\
%%TRANS2
%%EXEND

\z
\z


In constructions where multiple elements are questioned, however, the order is restricted. This is shown for adverbs whereby the reverse of \xref{bkm:Ref90636091:c} is not allowed; hence example \xref{bkm:Ref136343872:a} is not accepted. Furthermore, there seems to be a restriction for sentences with multiple question words, as not all combinations of arguments and/or adverbials are accepted. This is illustrated in \xref{bkm:Ref136343872:b}. Further research is necessary to determine what the precise restrictions are.

\ea
\label{bkm:Ref136343872}
%%EAX
\ea
%%JUDGEMENT
[*]{
%%LABEL
\label{bkm:Ref136343872:a}
%%CONTEXT
%%LINE1
Abiikile kugu ifiki?\\
%%LINE2
\gll
a-biik-ile  kugu  i-fi-ki\\
%%LINE3
1\SM{}-put-\PFV{}  17.where  \AUG{}-8-what\\
%%TRANS1
\glt
‘What did s/he put where?’\\
%%TRANS2
}
%%EXEND

%%EAX
\ex
%%JUDGEMENT
[*]{
%%LABEL
\label{bkm:Ref136343872:b}
%%CONTEXT
%%LINE1
Untupe ukumpa juani ifiki?\\
%%LINE2
\gll
u-n-tupe  a-ku-m-p-a  ju-ani  i-fi-ki\\
%%LINE3
\AUG{}-1-fat  1\SM{}-\PRS{}-1\OM{}-give-\FV{}  1-who  \AUG{}-8-what\\
%%TRANS1
\glt
‘Who is the fat one giving what?’\\
%%TRANS2
}
%%EXEND

\z
\z


To summarise, constituents can be focused in the postverbal domain, without positional restrictions, as was shown for adverbs and object arguments (see \sectref{bkm:Ref122073759} for subjects). There are restrictions on multiple content question words, but these require further investigation.

\subsection{Topic and focus in the left periphery}
\label{bkm:Ref114653225}
Kinyakyusa prefers the preverbal domain for topics. While familiar topics are typically expressed by just a subject marker, a full NP is used for shift topics, contrastive topics, and scene-setting topics. Shift topics are illustrated in part of a narrative given in \xref{bkm:Ref122078941}. Note that the distal demonstrative (in boldface) is also used here to mark the shift in topic.\largerpage[-1]\pagebreak

\ea
\label{bkm:Ref122078941}(That old woman asked: ``Why do you sing and lament?" She replied: ``My friend said to throw away the child; she is the one who cheated me; she is called Kisugujila." Then...)\\

%%EAX
\ea
%%JUDGEMENT
%%LABEL
%%LINE1
 Ubibi \textbf{jula} atile ``mma isaga umyande amatiti aga kuti ngupe umwana".\\
%%LINE2
\gll
u-bibi  ju-la  a-ti-ile  mma  is-ag-a  u-myand-e  a-ma-titi  aga  kuti  n-ku-p-e  u-mu-ana.\\
%%LINE3
\AUG{}-1.grandmother  1-\DEM.\DIST{}  1\SM{}-say-\PFV{}  no  come-\HAB{}-\IMP{}  2\SG.\SM{}-lick-\SBJV{} \AUG{}-6-eye.discharge  6.\DEM.\PROX{}  \COMP{}  1\SG.\SM-2\SG.\OM{}-give-\SBJV{}  \AUG{}-1-child\\
%%TRANS1
\glt ‘That old woman said ``No, you should come and lick these sleepies and I will give you a baby".’
%%TRANS2
%%EXEND

%%EAX
\ex
%%JUDGEMENT
%%LABEL
%%LINE1
Looli unkiikulu \textbf{jula} amyandile amatiti mwa bibi jula.\\
%%LINE2
\gll
looli  u-n-kiikulu  ju-la  a-myand-ile  a-ma-titi  mu-a  bibi  ju-la\\
%%LINE3
but  \AUG{}-1-woman  1-\DEM.\DIST{}  1\SM{}-lick-\PFV{}  \AUG{}-6-eye.discharge  18-\CONN{} 1.grandmother  1-\DEM.\DIST{}\\
%%TRANS1
\glt ‘Then the woman licked the sleepies (on the eyes) of that old woman.’
%%TRANS2
%%EXEND

\z

% This context interrupts the nested xlist, so I've manually set the xnumii counter for the following example.
(When she finished that old woman told her ``Go to that house, when you enter you will find a baby." When she arrived in there, she indeed found a child.)\\

%%EAX
\begin{xlist}
\setcounter{xnumii}{2}
\ex
%%JUDGEMENT
%%LABEL
%%LINE1
Looli umwana \textbf{jula} ali n’ ikilundi kimo kyene; akalinakyo ikilundi ikingi.\\
%%LINE2
\gll
looli  u-mu-ana  ju-la  a-li  na  i-ki-lundi  ki-mo  ki-ene a-ka-li-na=kio  i-ki-lundi  i-ki-ngi.\\
%%LINE3
but  \AUG{}-1-child  1-\DEM.\DIST{}  1\SM{}-be  with  \AUG{}-7-leg  7-one  7-only  1\SM-\NEG-{}be-with=7.\PRO{}  \AUG{}-7-leg  \AUG{}-7-other\\
%%TRANS1
\glt ‘But that child had one leg, he did not have the other leg.'
%%TRANS2
%%EXEND

\end{xlist}
\z

Having presented shift topics, we now turn to contrastive topics. Examples \xref{bkm:Ref136344955} and \xref{bkm:Ref136344956} illustrate an explicit comparison between people, referring to them by proper names. Note that subjects and objects can be placed in the preverbal domain when functioning as contrastive topics.\pagebreak

%%EAX
\ea
%%JUDGEMENT
%%LABEL
\label{bkm:Ref136344955}
%%CONTEXT
%%LINE1
Amani numbwene; Saimoni ngambona.\\
%%LINE2
\gll
Amani  n-m-bon-ile  Saimoni  n-ka-m-bon-a\\
%%LINE3
Amani  1\SG.\SM-1\OM{}-see-\PFV{}  1.Simon  1\SG.\SM-\NEG-1\OM{}-see-\FV{}\\
%%TRANS1
\glt
‘Amani I have seen; Simon I have not seen.’\\
%%TRANS2
%%EXEND

\z


%%EAX
\ea
%%JUDGEMENT
%%LABEL
\label{bkm:Ref136344956}
%%CONTEXT
(How many votes did Leo and his friends get?)\\
%%LINE1
ULeo n’ abinaake bakakabile nyingi. ULeo ihano, ujungi uju sita.\\
%%LINE2
\gll
u-Leo  na  a-ba-in-aake  ba-ka-kab-ile  nyingi  u-Leo  i-hano  u-ju-ngi  uju  sita\\
%%LINE3
\AUG{}-1.Leo  and  \AUG{}-2-friend-\POSS{}.1  2\SM-\NEG{}-get-\PFV{}  many  \AUG{}-Leo  \AUG{}-9.five  \AUG{}-1-other  1.\DEM.\PROX{}  9.six\\
%%TRANS1
\glt
‘Leo and his friends did not get many (votes). Leo got five, another got six.’
%%TRANS2
%%EXEND

\z

Scene-setting topics are illustrated for a temporal adverb in \xref{bkm:Ref105322476} and a location in \xref{bkm:Ref105322491}.

%%EAX
\ea
%%JUDGEMENT
%%LABEL
\label{bkm:Ref105322476}
%%CONTEXT
(Context: Beginning of a recipe.)\\
%%LINE1
Na mmajolo nnyonyilwe ukuti mbiije umpunga.\\
%%LINE2
\gll
na   mu-ma-jolo   n-nyonyu-il-e   ukuti   n-piij-e   u-m-punga\\
%%LINE3
and   18-6-yesterday  1\SG.\SM{}-desire-\PFV{}   \COMP{}   1\SG.\SM{}-cook-\SBJV{}   \AUG{}-3-rice\\
%%TRANS1
\glt
‘Yesterday, I felt that I should cook rice.’\\
%%TRANS2
%%EXEND

\z


%%EAX
\ea
%%JUDGEMENT
%%LABEL
\label{bkm:Ref105322491}
%%CONTEXT
(QUIS map task)\\
%%LINE1
Looli pang’ombe apa une nsyagile ina.\\
%%LINE2
\gll
looli  pa-ng’ombe  apa  une  n-si-ag-ile  ina\\
%%LINE3
but  16-10.cow  16.\DEM.\PROX{}  1\SG.\PRO{}  1\SG.\SM-10\OM{}-find-\PFV{}  four\\
%%TRANS1
\glt
‘But here at the place of cows, I found four (cows).’\\
%%TRANS2
%%EXEND

\z


Multiple topics may also appear preverbally, as in \xref{bkm:Ref105322512}, taken from a recounting of the Frog Story. The comment ‘they looked at the frog’ is anchored to an ever more specified referent: temporal ‘one day’ is specified to ‘at night’ (scene-setting), then we learn that this statement is about Jackson and his dog (shifting back from the frog to Jackson), and then the time is even further specified as ‘when they wanted to go to sleep’.\pagebreak

%%EAX
\ea
%%JUDGEMENT
%%LABEL
\label{bkm:Ref105322512}
%%CONTEXT
(Jackson has a dog which he loves very much. Also, he has a frog which he put in a bottle. It stayed and slept there.)\\
%%LINE1
{}[Akabalilo kamo] [pakilo] [Jakisoni n’mbwa jake] [bo bikulonda ukubuuka nkulambalala] bakikeetile ikyula.\\
%%LINE2
\gll
a-ka-balilo  ka-mo  pa-kilo  Jakisoni  na  mbwa  ji-ake  bo  ba-ku-lond-a  u-ku-buuka  mu-ku-lambalala  ba-ki-keet-ile  i-ki-ula\\
%%LINE3
\AUG{}-12-time  12-one  16-night  1.Jackson  and  9.dog  9-\POSS.1{}  when 2\SM-\PRS{}-want-\FV{}  \AUG{}-15-go  18-15-lie.down  2\SM-7\OM{}-look-\PFV{}  \AUG{}-7-frog\\
%%TRANS1
\glt ‘One day, at night, when Jackson and his dog wanted to go to sleep, they looked at the frog.’
%%TRANS2
%%EXEND

\z

Nevertheless, the preverbal domain is not reserved for topics in Kinyakyusa. First, an indefinite noun can appear preverbally, for example in \xref{bkm:Ref105351277} and \xref{bkm:Ref105351278}. Note that in example \xref{bkm:Ref105351277} the word \textit{mundu} ‘person’ does not have an augment and could thus be analysed as focused/clefted (see the discussion further on in this section and in \sectref{bkm:Ref122533658}).

%%EAX
\ea
%%JUDGEMENT
%%LABEL
\label{bkm:Ref105351277}
%%CONTEXT
%%LINE1
Ngimba mundu ali pakusenga iliisu.\\
%%LINE2
\gll
ngimba  mu-ndu  a-li  pa-ku-senga  i-li-isu\\
%%LINE3
\EXCLAM{}  1-person  1\SM{}-be  16-15-slash  \AUG{}-5-grass\\
%%TRANS1
\glt
‘Oh! Someone is slashing grass / it is someone slashing grass.’\\
%%TRANS2
%%EXEND

\z


%%EAX
\ea
%%JUDGEMENT
%%LABEL
\label{bkm:Ref105351278}
%%CONTEXT
%%LINE1
Linga siku umundu linga ikukubuula gwinogonengepo.\\
%%LINE2
\gll
linga  siku  u-mu-ndu  linga  a-ku-ku-buul-a  gu-inogon-ang-e=po\\
%%LINE3
\COND{}  9.day  \AUG{}-1-person  \COND{}  1\SM-\PRS-2\SG.\OM{}-tell-\FV{}  2\SG.\SM{}-think-?-\SBJV{}=16\\
%%TRANS1
\glt
‘If during another day someone tells you something you must think.’\\
%%TRANS2
%%EXEND

\z


Second, the subject may be preverbal in a thetic sentence, as in the out-of-the-blue sentences in \xref{bkm:Ref90303738} and \xref{bkm:Ref90303744}.

%%EAX
\ea
%%JUDGEMENT
%%LABEL
\label{bkm:Ref90303738}
%%CONTEXT
%%LINE1
Ifula jikutima kula.\\
%%LINE2
\gll
i-fula  ji-ku-tim-a  ku-la\\
%%LINE3
\AUG{}-9.rain  9\SM-\PRS{}-rain-\FV{}  17-\DEM.\DIST{}\\
%%TRANS1
\glt
‘It’s raining there.’\\
%%TRANS2
%%EXEND

\z

%%EAX
\ea
%%JUDGEMENT
%%LABEL
\label{bkm:Ref90303744}
%%CONTEXT
(Context: You are sitting in a house as a small group. Someone stares out through the window.
Another person asks: Why do you stare through the window?)\\
%%LINE1
Ikyula kikolile ulubwele.\\
%%LINE2
\gll
i-ki-ula  ki-kol-ile  u-lu-bwele\\
%%LINE3
\AUG{}-7-frog  7\SM{}-catch-\PFV{} \AUG{}-11-fly\\
%%TRANS1
\glt ‘A/The frog caught a fly!' \citep[13]{KerrEtAl2023}\\
%%TRANS2
%%EXEND
\z

Third, focus may appear in the preverbal position. This is discussed more extensively in \sectref{bkm:Ref122533658} on basic clefts, and is evidenced by the fact that the preverbal subject can be questioned \xref{bkm:Ref90303243}, it can form the answer to a question \xref{bkm:Ref90303382}, and it can be modified by exhaustive ‘only’ \xref{bkm:Ref105352219}.

%%EAX
\ea
%%JUDGEMENT
%%LABEL
\label{bkm:Ref90303243}
%%CONTEXT
%%LINE1
Juani akuuliile ifilato?\\
%%LINE2
\gll
ju-ani  a-ku-ul-il-ile  i-fi-lato\\
%%LINE3
1-who  1\SM{}-2\SG{}.\OM{}-buy-\APPL{}-\PFV{}  \AUG{}-8-shoe\\
%%TRANS1
\glt
‘Who bought you shoes?’\\
%%TRANS2
%%EXEND

\z


\ea
\label{bkm:Ref90303382}
%%EAX
\ea
%%JUDGEMENT
%%LABEL
%%CONTEXT
%%LINE1
(*i)Fiki fisatwike?\\
%%LINE2
\gll
i-fi-ki  fi-satuk-ile\\
%%LINE3
\AUG{}-8-what  8\SM{}-fall-\PFV{}\\
%%TRANS1
\glt
‘What has fallen?’\\
%%TRANS2
%%EXEND

%%EAX
\ex
%%JUDGEMENT
%%LABEL
%%CONTEXT
%%LINE1
{Imbwa} {jasatwike} {paasi.}\\
%%LINE2
\gll
{i-mbwa}  {ji-a-satuk-ile}  {pa-asi.}\\
%%LINE3
{\AUG{}-9.dog}  {9\SM-\PST{}-fall-\FV{}}  {16-down}\\
%%TRANS1
\glt
‘The dog fell down.’ \citep[11]{KerrEtAl2023}\\
%%TRANS2
%%EXEND

\z
\z

%%EAX
\ea
%%JUDGEMENT
%%LABEL
\label{bkm:Ref105352219}
%%CONTEXT
%%LINE1
Beene abapuuti batikubomba imbombo (abangi boosa bikubomba).\\
%%LINE2
\gll
{ba-ene}  {a-ba-puuti}  {ba-ti-ku-bomb-a}  i-mbombo  a-ba-ngi ba-osa  ba-ku-bomb-a \\
%%LINE3
2-only  {\AUG{}-2-priest}  {2\SM-\NEG-\PRS{}-work-\FV{}}  {\AUG{}-9.job}  {\AUG{}-2-other} 2-all  2\SM-\PRS{}-work-\FV{} \\
%%TRANS1
\glt
‘Only the priests do not work (all the other people work).’ \citep[11]{KerrEtAl2023}\\
%%TRANS2
%%EXEND

\z

The CV prefix, which introduces exhaustive focus (see \sectref{bkm:Ref122533116} and \citealt{vanderWalLusekelo2022}), may also occur on preverbal nouns as in \xxref{bkm:Ref122598796}{bkm:Ref123743341}, although this possibility needs specific contexts for it to express exclusive reading.

%%EAX
\ea
%%JUDGEMENT
%%LABEL
\label{bkm:Ref122598796}
%%CONTEXT
(Context: Friends visit a park. One asks about the behavior of animals in the park. A special hunter replies.)\\
%%LINE1
Jingalamu jikulya ifinyamaana ifinine.\\
%%LINE2
\gll
ji-n-galamu  ji-ku-li-a  i-fi-nyamaana  i-fi-nine\\
%%LINE3
\EXH{}-9-lion  9\SM{}-\PRS{}-eat-\FV{}  \AUG{}-8-animal  \AUG{}-8-friend\\
%%TRANS1
\glt
‘(Only) the lion eats other animals.’\\
%%TRANS2
%%EXEND

\z


%%EAX
\ea
%%JUDGEMENT
%%LABEL
%%CONTEXT
(Context: The priests have the ability to work but they have been exempted from doing so.)\\
%%LINE1
Babapuuti batikubomba imbombo (abangi boosa bikubomba).\\
%%LINE2
\gll
ba-ba-puuti  ba-ti-ku-bomb-a  i-mbombo a-ba-ngi  ba-osa  ba-ku-bomb-a\\
%%LINE3
\EXH{}-2-priest  2\SM-\NEG-\PRS{}-work-\FV{}  \AUG{}-9.job  \AUG{}-2-other  2-all  2\SM-\PRS{}-work-\FV{}\\
%%TRANS1
\glt ‘Only priests do not work (all the other people work).’
%%TRANS2
%%EXEND

\z

%%EAX
\ea
%%JUDGEMENT
%%LABEL
\label{bkm:Ref123743341}
%%CONTEXT
(Context: There are banana plants, avocado trees, and firewood trees; out of these, the firewood trees have dried.)\\
%%LINE1
Gimipiki (gyene) gyumile -- ifijinja fikuuma.\\
%%LINE2
\gll
gi-mi-piki  gi-ene  gi-um-ile  i-fi-jinja  fi-ka-um-a\\
%%LINE3
\EXH{}-4-tree  4-only  4\SM{}-dry-\PFV{}  \AUG{}-8-banana.tree  8\SM-\NEG{}-dry-\FV{}\\
%%TRANS1
\glt ‘Only the trees have dried -- the banana plants are not dry.’
%%TRANS2
%%EXEND

\z


Object nouns with the CV exhaustive prefix cannot appear preverbally \xref{ex:luggagenomoney} – they can only occur initially with the marking for a reverse pseudocleft (\textit{fyo ifi} in this case), further discussed in \sectref{bkm:Ref122079499}.

%%EAX
\ea
%%JUDGEMENT
%%LABEL
\label{ex:luggagenomoney}
%%CONTEXT
%%LINE1
Fifitwalo *(fyo ifi) batwele. Indalama bakatwala.\\
%%LINE2
\gll
fi-fi-twalo  fyo  ifi  ba-twal-ile  i-ndalama  ba-ka-twal-a\\
%%LINE3
\EXH{}-8-luggage  8.\IDCOP{}  8.\DEM.\PROX{}  2\SM{}-bring-\PFV{}  \AUG{}-9.money  2\SM-\NEG{}-bring-\FV{}\\
%%TRANS1
\glt
‘Only the luggage that is what they brought. As for money, they did not bring (it).’ \citep[336]{vanderWalLusekelo2022}\\
%%TRANS2
%%EXEND

\z


In summary, unlike in many other Bantu languages (see for example \chapref{ch:8} on Makhuwa\hyp Enahara and \chapref{ch:9} on Cicopi in this volume, as well as for example \citealt{Zerbian2006} and \citealt{Yoneda2011} among others), whose preverbal domains are strictly dedicated to topics, the preverbal domain in Kinyakyusa may in addition host focal subjects, although preposing topical constituents is common.

\subsection{Subject inversion}
\label{bkm:Ref122073759}

The logical subject in Kinyakyusa can appear preverbally, as in \xref{bkm:Ref122598826:a}, or postverbally, as in \xref{bkm:Ref122598826:b} for the subject \textit{amiisi} ‘water’.\footnote{This section is based on \citet{MsovelaEtAl2023}. For a detailed analysis, reference to the article is recommended.} 
 Unlike many other eastern Bantu languages, Kinyakyusa only features Agreeing Inversion (henceforth AI), whereby the subject marker on the verb agrees with the postverbal subject. In the case of example \REF{bkm:Ref122598826:b}, the postverbal subject is a class 6 noun, determining the subject marker on the verb in class 6 \textit{ga-}.

\ea
\label{bkm:Ref122598826}
%%EAX
\ea
%%JUDGEMENT
%%LABEL
\label{bkm:Ref122598826:a}
%%CONTEXT
%%LINE1
Amiisi gingiile nnyumba.\\
%%LINE2
\gll
a-ma-isi  ga-ingil-ile  mu-nyumba\\
%%LINE3
\AUG{}-6-water  6\SM{}-enter-\PFV{}  18-9.house\\
%%TRANS1
\glt
‘Water entered the house.’\\
%%TRANS2
%%EXEND



%%EAX
\ex
%%JUDGEMENT
%%LABEL
\label{bkm:Ref122598826:b}
%%CONTEXT
%%LINE1
Kusofu gingile amiisi.\\
%%LINE2
\gll
ku-sofu  ga-ingil-ile  a-ma-isi\\
%%LINE3
17-9.bedroom  6\SM{}-enter-\PFV{}  \AUG{}-6-water\\
%%TRANS1
\glt
‘In the bedroom entered water.’ \citep[158]{MsovelaEtAl2023}\\
%%TRANS2
%%EXEND

\z
\z

Locative inversion (LI) is a syntactic manifestation when the locative NP becomes a grammatical subject of the inverted construction, and the logical subject follows the verb. The locative subject in this case determines agreement on the verb
(\citealt[2]{BresnanKanerva1989}, \citealt[336]{Thwala2006}). While LI is quite pervasive throughout the Bantu area, it is not used as a productive construction in the Kiwira variant of Kinyakyusa. When we asked the speakers for grammaticality judgements, LI constructions were judged as unacceptable, as shown in \xxref{bkm:Ref122598852:a}{bkm:Ref122598852:b}, and the suggestion for improving the sentences involved a change to AI, as in \xref{bkm:Ref122598852:c}.

\ea
\label{bkm:Ref122598852}
%%EAX
\ea
%%JUDGEMENT
[*]{
%%LABEL
\label{bkm:Ref122598852:a}
%%CONTEXT
%%LINE1
%%LINE2
\gll
Pa-chunya  pa-bon-ik-e  a-ma-bwe.\\
%%LINE3
16-chunya  16\SM{}-see-\STAT{}-\PFV{}  \AUG{}-6-stone\\
%%TRANS1
\glt
int. ‘Minerals are discovered at Chunya.’\\
%%TRANS2
}
%%EXEND

%%EAX
\ex
%%JUDGEMENT
[*]{
%%LABEL
\label{bkm:Ref122598852:b}
%%CONTEXT
%%LINE1
%%LINE2
\gll
Mu-chunya  mu-bon-ik-e  a-ma-bwe.\\
%%LINE3
18-chunya  18\SM{}-see-\STAT{}-\PFV{}  \AUG{}-6-stone\\
%%TRANS1
\glt
int. ‘Minerals are discovered in Chunya.’\\
%%TRANS2
}
%%EXEND

%%EAX
\ex
%%JUDGEMENT
[]{
%%LABEL
\label{bkm:Ref122598852:c}
%%CONTEXT
%%LINE1
%%LINE2
\gll
Mu-chunya  ga-bon-ik-e  a-ma-bwe.\\
%%LINE3
18-chunya  6\SM{}-see-\STAT{}-\PFV{}  \AUG{}-6-stone\\
%%TRANS1
\glt
‘Minerals are discovered in Chunya.’ \citep[161]{MsovelaEtAl2023}\\
%%TRANS2
}
%%EXEND

\z
\z

An interesting exception where locative inversion seems to be acceptable is found with alternating verbs, specifically -\textit{soka} ‘exit’ in the ‘bloody nose’ construction \xref{bkm:Ref122598890}, and -\textit{fwana} ‘be enough’ for ‘be fitting’ \xref{bkm:Ref122598897}, where either the locative or the figure can be the subject. The precise semantic and/or pragmatic differences in interpretation and use between the two variants of these sentences remain to be determined.

\ea
\label{bkm:Ref122598890}
%%EAX
\ea
%%JUDGEMENT
%%LABEL
%%CONTEXT
%%LINE1
%%LINE2
\gll
Mu-m-bulo  mu-ku-sok-a  i-i-noge.\\
%%LINE3
18-9-nose  18\SM{}-\PRS{}-exit-\FV{}  \AUG{}-5-nose.blood\\
%%TRANS1
\glt
‘In the nose comes out blood.’\\
%%TRANS2
%%EXEND

%%EAX
\ex
%%JUDGEMENT
%%LABEL
%%CONTEXT
%%LINE1
%%LINE2
\gll
I-i-noge  li-ku-sok-a  mu-m-bulo.\\
%%LINE3
\AUG{}-5-nose.{blood}  5\SM{}-\PRS{}-exit-\FV{}  18-9-nose\\
%%TRANS1
\glt
‘Blood comes out the nose.’ \citep[162]{MsovelaEtAl2023}\\
%%TRANS2
%%EXEND

\z
\z

\ea
\label{bkm:Ref122598897}
%%EAX
\ea
%%JUDGEMENT
%%LABEL
%%CONTEXT
%%LINE1
%%LINE2
\gll
Mw-igali  mu-no  mu-ku-fwan-a  a-ba-ndu  ba-na.\\
%%LINE3
18-9.car  18-\DEM.\PROX{}  18\SM{}-\PRS{}-be.enough-\FV{}  \AUG{}-2-person  2-four\\
%%TRANS1
\glt
‘(Inside) this car fits four people.’ \citep[162]{MsovelaEtAl2023}\\
%%TRANS2
%%EXEND

%%EAX
\ex
%%JUDGEMENT
%%LABEL
%%CONTEXT
%%LINE1
%%LINE2
\gll
A-ba-ndu  ba-na  bi-ku-fwana  mw-igali  mu-no.\\
%%LINE3
\AUG{}-2-person  2-four  2\SM-\PRS{}-be.enough-\FV{}  18-9.car  18-\DEM.\PROX{}\\
%%TRANS1
\glt
‘Four people fit in this car.’\\
%%TRANS2
%%EXEND

\z
\z

We can analyse these examples either as locative inversion that is lexically restricted, or as exhibiting a causative-inchoative alternation – we will leave this open. That the alternation is present in a restricted number of predicates (as is typical in other languages, see e.g., \citealt[3--7]{Haspelmath1993}, \citealt{Creissels2022}) can be seen in the fact that a similar predicate, \textit{ingila} ‘enter’, does not allow the alternation, as in \xref{bkm:Ref147926994}. Systematic research into which predicates belong to the group of alternating verbs is welcome to clarify the interpretational and/or structural restrictions at play here.

\ea
\label{bkm:Ref147926994}
%%EAX
\ea
%%JUDGEMENT
%%LABEL
%%CONTEXT
%%LINE1
Amasiugusi gingiile musukali. \\
%%LINE2
\gll
a-ma-siugusi   ga-ingil-ile   mu-sukali \\
%%LINE3
\AUG{}-6-ant   6\SM{}-enter-\PFV{}   18-9.sugar\\
%%TRANS1
\glt
‘Ants entered in(to) the sugar.’\\
%%TRANS2
%%EXEND

%%EAX
\ex
%%JUDGEMENT
%%LABEL
%%LINE1
Musukali gi/*mw-ingiile amasiungusi. \\
%%LINE2
\gll
mu-sukali   mu-/ga-ingil-ile   a-ma-siungusi \\
%%LINE3
18-9.sugar   18\SM{}-/6\SM{}-enter-\PFV{}   \AUG{}-6-ant \\
%%TRANS1
\glt ‘Into the sugar entered ants. \citep[162]{MsovelaEtAl2023}’
%%TRANS2
%%EXEND

\z
\z

It is also interesting to note that, although the Mwamba dialect is said to cover both villages of Kiwira and Lwanga, Kinyakyusa spoken in the former (source of this study) differs in locative agreement with the one spoken in the latter described by \citet{Persohn2020}. The examples provided by \citet[95]{Persohn2020} were not accepted by our speakers (see \citealt{MsovelaEtAl2023} for further comparison). 

Regarding the valency types that AI can occur with in Kinyakyusa, it is accepted with the copula ‘be’, the two types of intransitives (unergatives and unaccusatives), and passives, but is not acceptable in VSO and VOS sentences. AI with the verb ‘be’ is illustrated in \xref{bkm:Ref122598914} and \xref{bkm:Ref122598922}. Although LI is not accepted, note that the verb in these inversions takes the locative enclitic (=\textit{po}, =\textit{ko}, =\textit{mo}).

%%EAX
\ea
%%JUDGEMENT
%%LABEL
\label{bkm:Ref122598914}
%%CONTEXT
%%LINE1
Leelo baaliko abandu bahano.\\
%%LINE2
\gll
leelo  ba-a-li=ko  a-ba-ndu  ba-hano\\
%%LINE3
but  2\SM{}-\PST{}-be=17  \AUG{}-2-person  2-five\\
%%TRANS1
\glt
‘But there were five people (contestants).’ \citep[166]{MsovelaEtAl2023}\\
%%TRANS2
%%EXEND

\z


%%EAX
\ea
%%JUDGEMENT
%%LABEL
\label{bkm:Ref122598922}
%%CONTEXT
%%LINE1
Keeta silimo imbatata mundeko.\\
%%LINE2
\gll
keeta  si-li=mo  i-mbatata  mu-ndeko\\
%%LINE3
look  10\SM{}-be=18  \AUG{}-10.potato  18-9.pot\\
%%TRANS1
\glt
‘Look, there are potatoes in the pot.’ \citep[166]{MsovelaEtAl2023}\\
%%TRANS2
%%EXEND

\z


For unaccusative and passive predicates, the single argument takes the patient role. AI with unaccusatives in Kinyakyusa is exemplified in \xref{bkm:Ref122598932} and \xref{bkm:Ref122598938}, and AI with a passive verb is illustrated in \xref{bkm:Ref122598944}.

%%EAX
\ea
%%JUDGEMENT
%%LABEL
\label{bkm:Ref122598932}
%%CONTEXT
%%LINE1
Kusofu gingiile amiisi.\\
%%LINE2
\gll
ku-sofu  gi-ingil-ile  a-ma-isi\\
%%LINE3
17-bedroom  6\SM{}-enter-\PFV{}  \AUG{}-6-water\\
%%TRANS1
\glt
‘Water entered in the bedroom.’ \citep[158]{MsovelaEtAl2023}\\
%%TRANS2
%%EXEND

\z


%%EAX
\ea
%%JUDGEMENT
%%LABEL
\label{bkm:Ref122598938}
%%CONTEXT
%%LINE1
Gyumile gimipiki.\\
%%LINE2
\gll
gi-um-ile  gi-mi-piki\\
%%LINE3
4\SM{}-dry-\PFV{}  \EXH{}-4-tree\\
%%TRANS1
\glt
‘Only the trees dried.’\\
%%TRANS2
%%EXEND

\z


%%EAX
\ea
%%JUDGEMENT
%%LABEL
\label{bkm:Ref122598944}
%%CONTEXT
%%LINE1
Sikulondwa syene milioni ibili ukumalikisya inyumba.\\
%%LINE2
\gll
si-ku-lond-w-a  si-ene  milioni  i-bili  u-ku-mal-ik-isy-a     i-nyumba \\
%%LINE3
10\SM-\PRS{}-want-\PASS-\FV{}  10-only  10.million  \AUG{}-10.two  \AUG{}-15-finish-\STAT{}-\CAUS-\FV{} \AUG{}-9.house \\
%%TRANS1
\glt
‘Only two million (not more) are needed to finish the house.’ \citep[166]{MsovelaEtAl2023}\\
%%TRANS2
%%EXEND

\z

Unergatives also allow subject inversion. These are intransitive verbs that, unlike unaccusatives, have an Agent argument. They include verbs such as \textit{run}, \textit{talk}, \textit{laugh} etc. We illustrate in \xref{bkm:Ref135207788} AI with the verb \textit{-bopa} ‘run’.

%%EAX
\ea
%%JUDGEMENT
%%LABEL
\label{bkm:Ref135207788}
%%CONTEXT
%%LINE1
Lyabopile ijenje!\\
%%LINE2
\gll
li-a-bop-ile  i-jenje\\
%%LINE3
5\SM{}-\PST{}-run-\PFV{}  \AUG{}-5.cockroach\\
%%TRANS1
\glt
‘There ran a cockroach!’ \citep[167]{MsovelaEtAl2023}\\
%%TRANS2
%%EXEND

\z


AI in Kinyakyusa cannot occur with both the subject and the object expressed as NPs postverbally; neither VOS nor VSO order is accepted in \xref{bkm:Ref122427174} and \xref{bkm:Ref122427176}, respectively, whether in a thetic context or with simple focus on the subject or object (i.e. as answer to a subject or object content question). We refer to \citet{MsovelaEtAl2023} for evidence that the restriction is not on transitive predicates as such in Kinyakyusa, but rather the postverbal appearance of both the subject and object NP. This is because subject inversion with a transitive predicate is accepted if the object NP is either dislocated or pronominalised as an object marker.

%%EAX
\ea
%%JUDGEMENT
[*]{
%%LABEL
\label{bkm:Ref122427174}
%%CONTEXT
%%LINE1
Aapiijiile ifindu uSekela.\\
%%LINE2
\gll
a-a-piij-ile  i-fi-ndu  u-Sekela\\
%%LINE3
1\SM-\PST{}-cook-\PFV{}  \AUG{}-8-food  \AUG{}-1.Sekela\\
%%TRANS1
\glt
‘Sekela has cooked (some) food.’ \citep[167]{MsovelaEtAl2023}\\
%%TRANS2
}
%%EXEND

\z


%%EAX
\ea
%%JUDGEMENT
[*]{
%%LABEL
\label{bkm:Ref122427176}
%%CONTEXT
%%LINE1
Akuunyile underefwa umpiki.\\
%%LINE2
\gll
a-kuuny-ile  u-n-delefwa  u-m-piki\\
%%LINE3
1\SM{}-push-\PFV{}  \AUG{}-1-driver  \AUG{}-3-tree\\
%%TRANS1
\glt
‘The driver hit a/the tree.’ \citep[167]{MsovelaEtAl2023}\\
%%TRANS2
}
%%EXEND

\z


Besides the formal properties of inversion that we just presented, the interpretational aspects presented hereunder cover the contexts in which VS order is used. In short, the postverbal subject in Kinyakyusa is non-topical – we find VS order being used in thetic sentences and in focusing the postverbal logical subject. In comparison with SV order, VS order is found to express a contrast. We discuss these contexts and interpretations in turn.

An SV sentence has a topic-comment articulation. The statement identifies a referent (i.e., the topic) and then comments on that referent, adding new information. In a thetic sentence, the entire proposition presents a state of affairs as new information. This means that thetics do not feature a topic expression. A thetic sentence typically marks the subject as non-topical \citep{Sasse1996,Sasse2006,Lambrecht1994}. In many Bantu languages, this detopicalisation is expressed by placing the subject in a postverbal position, as also happens in Kinyakyusa.

Thetic sentences are typically used presentationally, for example at the beginning of a story as in \xref{bkm:Ref136349676}; they are used to state the existence of a referent, as in \xref{bkm:Ref136349684}; they are also found in ``hot news"/``out of the blue" announcements as in \xref{bkm:Ref136349695}.

%%EAX
\ea
%%JUDGEMENT
%%LABEL
\label{bkm:Ref136349676}
%%CONTEXT
%%LINE1
Ulwa ijolo fiijo [aliko unnyambala jumo]. Aali n’ abakiikulu babili.\\
%%LINE2
\gll
u-lu-a  ijolo  fiijo  a-a-li=ko  u-n-nyambala  ju-mo  a-a-li  na   a-ba-kiikulu  ba-bili \\
%%LINE3
\AUG{}-11-\CONN{}  long  \INT{}  1\SM-\PST{}-be=17  \AUG{}-1-man  1-one  1\SM{}-\PST{}-be  with \AUG{}-2-woman  2-two \\
%%TRANS1
\glt
‘A long time ago, there was a man. He had two wives.’ \citep[170]{MsovelaEtAl2023}\\
%%TRANS2
%%EXEND

\z

%%EAX
\ea
%%JUDGEMENT
%%LABEL
\label{bkm:Ref136349684}
%%CONTEXT
(QUIS map task)
%%LINE1
Linga ufikile apo, kupingama papo apo silipo injila ibili.\\
%%LINE2
\gll
linga  u-fik-ile  apo  ku-pingam-a  papo  apo [si-li=po  i-njila  i-bili]\\
%%LINE3
\COND{}  2\SG.\SM{}-arrive-\PFV{}  16.\DEM.\MED{}  2\SG.\SM{}-turn-\FV{}  as  16.\DEM.\MED{} {\db}10\SM-be=16  \AUG{}-10.path  \AUG{}-10.two\\
%%TRANS1
\glt
‘If you have arrived there, turn as there are only two paths there.’ \citep[170]{MsovelaEtAl2023}\\
%%TRANS2
%%EXEND

\z

%%EAX
\ea
%%JUDGEMENT
%%LABEL
\label{bkm:Ref136349695}
%%CONTEXT
%%LINE1
Ukulinga nkiina; ikwaga jikusoka imbeba.\\
%%LINE2
\gll
u-ku-ling-a  mu-ki-ina  a-ku-ag-a  [ji-ku-sok-a  i-mbeba]\\
%%LINE3
1\SM{}-\PRS{}-peep-\FV{}  18-7-hole  1\SM-\PRS{}-find-\FV{}  {\db}9\SM-\PRS{}-get.out-\FV{}  \AUG{}-9.rat\\
%%TRANS1
\glt
‘He peeped in the hole; he saw a rat getting out.’ \citep[170]{MsovelaEtAl2023}\\
%%TRANS2
%%EXEND

\z


The postverbal subject may also be in focus (not just detopicalised). Simple (new information) focus can be seen in question-answer pairs, illustrated in \xref{bkm:Ref122117043} and \xref{bkm:Ref122117045}: The interrogatives are inherently focused; the constituents that replace them in the answers are also in focus.

\ea
\label{bkm:Ref122117043}
%%EAX
\ea
%%JUDGEMENT
%%LABEL
%%CONTEXT
%%LINE1
Bafwile (a)baani?\\
%%LINE2
\gll
ba-fw-ile  a-ba-ani\\
%%LINE3
2\SM{}-die-\PFV{}  \AUG{}-2-who\\
%%TRANS1
\glt
‘Who(pl) died?’\\
%%TRANS2
%%EXEND

%%EAX
\ex
%%JUDGEMENT
%%LABEL
%%CONTEXT
%%LINE1
Bafwile Abdala, Hamisi na Juma.\\
%%LINE2
\gll
ba-fw-ile  Abdala  Hamisi  na  Juma\\
%%LINE3
2\SM{}-die-\PFV{}  1.Abdala  1.Hamisi  and  1.Juma\\
%%TRANS1
\glt
‘Abdala, Hamisi, and Juma died.’ \citep[171]{MsovelaEtAl2023}\\
%%TRANS2
%%EXEND

\z
\z

\ea
\label{bkm:Ref122117045}
%%EAX
\ea
%%JUDGEMENT
%%LABEL
%%CONTEXT
%%LINE1
Jo jiliku (iji)  jifwile?\\
%%LINE2
\gll
jo  ji-liku   iji   ji-fw-ile\\
%%LINE3
9.\IDCOP{}   9-which  9.\DEM{}.\PROX{}   9\SM{}-die-\PFV{}\\
%%TRANS1
\glt
‘Which (animal) died?’\\
%%TRANS2
%%EXEND


%%EAX
\ex
%%JUDGEMENT
%%LABEL
%%CONTEXT
%%LINE1
Jifwile indemba indiitu.\\
%%LINE2
\gll
ji-fw-ile   i-n-temba   i-n-titu\\
%%LINE3
9\SM{}-die-\PFV{}   \AUG{}-9-hen   \AUG{}-9-black\\
%%TRANS1
\glt
‘A/the black hen died.’ \citep[171]{MsovelaEtAl2023}\\
%%TRANS2
%%EXEND

\z
\z

Among the possible interpretations of subject inversion, we note that exhaustively focused subjects are also accepted postverbally, as illustrated using the particle \textit{ene} ‘only’ and the CV prefix in \xref{bkm:Ref148767651}.

%%EAX
\ea
%%JUDGEMENT
%%LABEL
\label{bkm:Ref148767651}
%%CONTEXT
(What exact thing has fallen?)\\
%%LINE1
Jisatwike jeene jisimbilo.\\
%%LINE2
\gll
ji-satuk-ile  ji-ene  ji-simbilo\\
%%LINE3
9\SM{}-fall-\PFV{}  9-only  \EXH{}-9.pen\\
%%TRANS1
\glt
‘Only the pen has fallen.’\\
%%TRANS2
%%EXEND

\z


Given the fact that Kinyakyusa also allows preverbal focus, a follow-up research question is: What is the difference in interpretation between the preverbal and postverbal focused subject? In \citet{MsovelaEtAl2023}, we suggest that if a contrast is present with alternatives or expectations, the subject stays in a postverbal position, whereas if the referent is topical or no contrast set is present, the subject moves to a preverbal position. The contexts for the preverbal vs. postverbal position of \textit{juani} ‘who’ in \xref{bkm:Ref122427381} illustrate this: The SV order is interpreted as an open question in \xref{bkm:Ref122427381:a}, whereas the VS order suggests a selection from a set of alternatives in \xref{bkm:Ref122427381:b}.

\ea
\label{bkm:Ref122427381}
%%EAX
\ea
%%JUDGEMENT
%%LABEL
\label{bkm:Ref122427381:a}
%%CONTEXT
(Context: The chief has passed away and we are not sure whether the person who will now lead us will be as good.)\\
%%LINE1
Juani atulongolele?\\
%%LINE2
\gll
ju-ani  a-tu-longol-el-e\\
%%LINE3
1-who  1\SM{}-1\PL.\OM{}-lead-\APPL-\SBJV{}\\
%%TRANS1
\glt
‘Who should lead us?’\\
%%TRANS2
%%EXEND


%%EAX
\ex
%%JUDGEMENT
%%LABEL
\label{bkm:Ref122427381:b}
%%CONTEXT
(Context: In a choir, each of the singers can be the leader.)\\
%%LINE1
Atulongolele juani?\\
%%LINE2
\gll
a-tu-longol-el-e  ju-ani\\
%%LINE3
1\SM{}-1\PL.\OM{}-lead-\APPL{}-\SBJV{}  1-who\\
%%TRANS1
\glt
‘Who should lead us?’ \citep[173]{MsovelaEtAl2023}\\
%%TRANS2
%%EXEND

\z
\z

For further data and discussion of subject inversion in Kinyakyusa, we again refer to \citet{MsovelaEtAl2023}.

\section{Predicate doubling}
\label{bkm:Ref114648496}
In predicate doubling, a finite and non-finite form of the same predicate co-occur in the sentence. Kinyakyusa exhibits two types of predicate doubling, i.e. topic doubling and in-situ doubling, but it does not allow the third type, cleft doubling, which is attested in other languages \citep{GüldemannFiedler2022}. The ungrammaticality of cleft doubling is shown in \xref{bkm:Ref122520263}, and we discuss the form and interpretation of the other two types of predicate doubling in turn.

%%EAX
\ea
%%JUDGEMENT
[*]{
%%LABEL
\label{bkm:Ref122520263}
%%CONTEXT
%%LINE1
Ko kulima uku balimile. \\
%%LINE2
\gll
ko  ku-lima  uku  ba-lim-ile  \\
%%LINE3
15.\IDCOP{}  15-cultivate  15.\DEM.\PROX{}  2\SM{}-cultivate-\PFV{}\\
%%TRANS1
\glt
  int. ‘It is cultivating that they did.’\\
%%TRANS2
}
%%EXEND

\z

\subsection{Topic doubling}

Topic doubling is common in Kinyakyusa. In topic doubling, the infinitive precedes an inflected form of the same predicate. The infinitive either appears as it is, in class 15, and takes the augment, as illustrated in \xref{bkm:Ref123889229}, or alternatively it may be derived into a locative class 16 with \textit{pa}-, as illustrated in \xref{bkm:Ref122521250}.

%%EAX
\ea
%%JUDGEMENT
%%LABEL
\label{bkm:Ref123889229}
%%CONTEXT
(Context: Friends are talking about sugarcanes during rain season when sugarcane is not harvested. One friend asks: Do you guys want to eat sugarcane? The other friends reply.)\\

%%LINE1
%%LINE2
\gll
U-ku-londa{\footnotemark}   tu-ku-lond-a.\\
%%LINE3
\AUG{}-15-want  1\PL.\SM{}-\PRS{}-want-\FV{}\\
\footnotetext{The final vowel is not separated or glossed in the infinitive, as we view these as nouns.}
%%TRANS1
\glt
‘We \textit{do} want it.’\\
%%TRANS2
lit. ‘As for wanting, we want (but we cannot get sugarcane this season).’
%%EXEND

\z


%%EAX
\ea
%%JUDGEMENT
%%LABEL
\label{bkm:Ref122521250}
%%CONTEXT
(Context: You and your wife have quarreled, and a friend wants to know whether this has gone beyond repair. You answer this, to say that there is a conflict, but there is still conversation.)\\
%%LINE1
%%LINE2
\gll
Pa-ku-joba  tu-ku-job-a.\\
%%LINE3
16-15-say  1\PL.\SM{}-\PRS{}-say-\FV{}\\
%%TRANS1
\glt
‘We (do still) talk.’\\
%%TRANS2
lit. ‘As for talking, we talk.’

%%EXEND

\z


The interpretation of topic doubling depends largely on the context. A first possible interpretation is that of polarity focus or verum, as illustrated in \xref{bkm:Ref110850439}. The marker \textit{ko} (a pronoun used as a topic marker) optionally follows the topical infinitive and is further discussed in \sectref{bkm:Ref123889293}.

%%EAX
\ea
%%JUDGEMENT
%%LABEL
\label{bkm:Ref110850439}
%%CONTEXT
(Context: Father told us to sweep when he left. Now he comes back, and we are sitting watching TV. He says: ‘Why are you lazy watching TV and haven’t swept?’)\\
%%LINE1
Ukupyagila ko tupyagiile!\\
%%LINE2
\gll
u-ku-pyagila  ko  tu-pyagil-ile\\
%%LINE3
\AUG{}-15-sweep  15.\PRO{}  1\PL.\SM{}-sweep-\PFV{}\\
%%TRANS1
\glt
‘We DID sweep!’
%%TRANS2
%%EXEND

\z

Second, we find an intensive reading, where the action described in the predicate is carried out above expectation, as in \xref{bkm:Ref136353381}. The unexpectedness can be reinforced by the exclamation \textit{mwé,} suggesting a mirative interpretation, as in \xref{bkm:Ref136353384} and \xref{walking}.

%%EAX
\ea
%%JUDGEMENT
%%LABEL
\label{bkm:Ref136353381}
%%CONTEXT
(Context: Speaker compliments on someone’s outfit.)\\
%%LINE1
Ukufwala afwele.\\
%%LINE2
\gll
u-ku-fwala  a-fwal-ile\\
%%LINE3
\AUG{}-15-wear  1\SM{}-wear-\PFV{}\\
%%TRANS1
\glt
‘He has really dressed up!’\\
%%TRANS2
%%EXEND

\z


%%EAX
\ea
%%JUDGEMENT
%%LABEL
\label{bkm:Ref136353384}
%%CONTEXT
(Context: Speaker is astonished by good state of the house.)\\
%%LINE1
Mwé ukujenga ajengile.\\
%%LINE2
\gll
mwe  u-ku-jenga  a-jeng-ile\\
%%LINE3
\EXCLAM{}  \AUG{}-15-build  1\SM{}-build-\PFV{}\\
%%TRANS1
\glt
‘He really built something.’ (a special house)\\
%%TRANS2
%%EXEND

\z
\pagebreak
%%EAX
\ea
%%JUDGEMENT
%%LABEL
\label{walking}
%%CONTEXT
(Context: From Mbeya to Kiwira takes 10 hours to walk but someone arrives in 6 hours, which is extraordinary.)\\
%%LINE1
Mwé, ukwenda umwana endile!\footnote{Note also that the lexical subject here is placed after the infinitive, showing that the initial infinitive is indeed a topic in the left periphery.} \\
%%LINE2
\gll
mwe   u-ku-enda  u-mu-ana  a-end-ile\\
%%LINE3
\EXCLAM{}  \AUG{}-15-walk  \AUG{}-1-child  1\SM{}-walk-\PFV{}\\
%%TRANS1
\glt
‘The child (really) walked!’\\
%%TRANS2
%%EXEND

\z


A depreciative reading is also possible, indicating the bare minimum that would count as carrying out the action in the predicate but nothing that reaches expectations. In \xref{bkm:Ref122522449} and \xref{bkm:Ref122522450}, the pejorative suffix \nobreakdash-\textit{any}-,\footnote{In the contexts discussed in this chapter, insertion of the suffix \nobreakdash-\textit{any}- adds pejorative meaning to the verb. A more detailed analysis is offered in \citet[108]{Persohn2020} who analyses the form \nobreakdash-\textit{ania}- as a “combination of the reciprocal and causative extensions often gives a pluractional reading. The range of possible meanings includes reiteration, intensification or the involvement of multiple subjects or objects.”} which occurs on the verb in the clause that follows the clause with predicate doubling, highlights the depreciative meaning.

%%EAX
\ea
%%JUDGEMENT
%%LABEL
\label{bkm:Ref122522449}
%%CONTEXT
%%LINE1
Pakulima (po) alimile (lelo asengeenye).\\
%%LINE2
\gll
pa-ku-lima  po  a-lim-ile  leelo  a-seng-any-ile\\
%%LINE3
16-15-cultivate  16.\PRO{}  1\SM{}-cultivate-\PFV{}  but  1\SM{}-slash.grass-\PEJ{}-\PFV{}\\
%%TRANS1
\glt
‘As for cultivation, s/he did cultivate (but s/he did so sloppily).’\\
%%TRANS2
%%EXEND

\z


%%EAX
\ea
%%JUDGEMENT
%%LABEL
\label{bkm:Ref122522450}
%%CONTEXT
%%LINE1
Pakusona umwenda (ko/po) asonile leelo asonenie.\\
%%LINE2
\gll
pa-ku-sona  u-mu-enda  ko/po  a-son-ile  leelo  a-son-any-ile\\
%%LINE3
16-15-sew  \AUG{}-3-clothes  15.\PRO{}/16.\PRO{}  1\SM{}-sew-\PFV{}  but  1\SM{}-sew-\PEJ{}-\PFV{}\\
%%TRANS1
\glt
‘As for the sewing the clothes, s/he sewed, but did so badly.’\\
%%TRANS2
%%EXEND

\z


As already shown in example \xref{bkm:Ref110850439}, and here in examples \xref{bkm:Ref122522449} and \xref{bkm:Ref122522450}, the pronoun/topic marker \textit{po}/\textit{ko} may be added to introduce the infinitival topic (see \sectref{bkm:Ref123889293}). Example \xref{bkm:Ref122522407} illustrates the use of the marker once more.\largerpage[-1]\pagebreak

%%EAX
\ea
%%JUDGEMENT
%%LABEL
\label{bkm:Ref122522407}
%%CONTEXT
(Context: Speaker A was talking about the contribution to water supply in the village that Speaker B did contribute. Speaker A wanted to insist on the offer Speaker B provided. Speaker A says:)\\
%%LINE1
Ukubomba ko ubombile amiisi.\\
%%LINE2
\gll
u-ku-bomba  \textbf{ko}  u-bomb-ile  a-ma-isi\\
%%LINE3
\AUG{}-15-work  15.\PRO{}  2\SG.\SM{}-work-\PFV{}  \AUG{}-6-water\\
%%TRANS1
\glt
‘You did well for the water.’\\
%%TRANS2
lit. ‘As for doing, you did well, as regards the water.’
%%EXEND

\z

\subsection{In-situ doubling}
\label{bkm:Ref123889421}
The second type of predicate doubling is in-situ doubling, shown in \xref{bkm:Ref122522414}. The infinitive here follows an inflected form of the same predicate, and it must occur with the CV exhaustive prefix and cannot occur with the V augment (see \sectref{bkm:Ref122533116} on the CV prefix), as shown in \xref{bkm:Ref122533092}.

%%EAX
\ea
%%JUDGEMENT
%%LABEL
\label{bkm:Ref122522414}
%%CONTEXT
(Is Lydia washing dishes and cooking?)\\
%%LINE1
Ikusuka kukusuka.\\
%%LINE2
\gll
a-ku-suk-a  ku-ku-suka\\
%%LINE3
1\SM{}-\PRS{}-wash-\FV{}  \EXH{}-15-wash\\
%%TRANS1
\glt
‘She is just washing.’\\
%%TRANS2
%%EXEND

\z


%%EAX
\ea
%%JUDGEMENT
%%LABEL
\label{bkm:Ref122533092}
%%CONTEXT
%%LINE1
Alimile kukulima / *ukulima.\\
%%LINE2
\gll
a-lim-ile  ku-ku-lima   / u-ku-lima\\
%%LINE3
1\SM{}-cultivate-\PFV{}  \EXH{}-15-cultivate  / \AUG{}-15-cultivate\\
%%TRANS1
\glt
‘He (only) cultivated.’\\
%%TRANS2
%%EXEND

\z


When an object is present in this construction, it follows the infinitive and not the inflected verb, as shown in \xref{bkm:Ref122523250}.

\ea
\label{bkm:Ref122523250}(Context: Guests visit their friend. They are offered tea while food is being prepared. They agree on taking only tea.)
%%EAX
\ea
%%JUDGEMENT
%%LABEL
%%CONTEXT
%%LINE1
Tunwe kukunwa ikyai. (Tungagulila ifindu.)\\
%%LINE2
\gll
tu-nu-e  ku-ku-nua  i-kyai  tu-nga-agul-il-a i-fi-ndu\\
%%LINE3
1\PL.\SM{}-drink-\SBJV{}  \EXH{}-15-drink  \AUG{}-9.tea  1\PL.\SM{}-\PROH{}-wait-\APPL-\FV{} \AUG{}-8-food\\
%%TRANS1
\glt
‘Let us only drink tea. (We should not wait for food.)’\\
%%TRANS2
%%EXEND


%%EAX
\ex
%%JUDGEMENT
[*]{
%%LABEL
%%CONTEXT
%%LINE1
Tunwe ikyai kukunwa.\\
%%LINE2
\gll
tu-nu-e  i-kyai  ku-ku-nua\\
%%LINE3
1\PL.\SM{}-drink-\SBJV{}  \AUG{}-9.tea  \EXH{}-15-drink   \\
%%TRANS1
\glt
int. ‘Let us only drink tea.’\\
%%TRANS2
}
%%EXEND

\z
\z

In-situ doubling brings about an interpretation as state-of-affairs focus, and (because of the CV prefix) an exhaustive reading on the predicate: Only this action and no other was performed, as seen in the context and following clause in \xref{bkm:Ref122523250} above, and illustrated again in the contrastive context in \xref{bkm:Ref122523373}. Although superfluous, speakers may also add the exhaustive particle -\textit{ene} ‘only’ to emphasize on the action, as in \xref{bkm:Ref123830829}.

%%EAX
\ea
%%JUDGEMENT
%%LABEL
\label{bkm:Ref122523373}
%%CONTEXT
(Context: Malundi was supposed to do two activities, graze cows and cultivate a farm. He did one activity. Speaker A asks: Did Malundi graze cows and cultivate the farm? Speaker B knows exactly what Malundi did and replies.)\\
%%LINE1
Mma atimile kukutima ing’ombe.\\
%%LINE2
\gll
mma  a-tim-ile  ku-ku-tima  i-ng’ombe\\
%%LINE3
no  1\SM{}-graze-\PFV{}  \EXH{}-15-graze  \AUG{}-10.cow\\
%%TRANS1
\glt
‘No. He only grazed cows.’\\
%%TRANS2
%%EXEND

\z


%%EAX
\ea
%%JUDGEMENT
%%LABEL
\label{bkm:Ref123830829}
%%CONTEXT
(Context: The parents travelled and expected the children to wash clothes, sweep the yard and water the flowers. Upon their return, they ask: Did you complete your tasks?)\\
%%LINE1
Twasukile (kwene) kukusuka.\\
%%LINE2
\gll
tu-a-suk-ile  ku-ene  ku-ku-suka\\
%%LINE3
1\PL.\SM{}-\PST{}-wash-\PFV{}  15-only  \EXH{}-15-wash\\
%%TRANS1
\glt
‘We only washed.’\\
%%TRANS2
%%EXEND

\z


In summary, topic doubling in Kinyakyusa is used to express verum and a contrast between predicates, with additional pragmatically-licensed intensive, mirative, and depreciative interpretations. In-situ doubling uses the postverbal infinitive with the CV prefix and expresses exhaustive state-of-affairs focus.

\section{Topic markers \textit{ko} and \textit{po}}
\label{bkm:Ref123889293}
The pronominal expressions \textit{ko} and \textit{po}, which we call topic markers, can be used in topic doubling in Kinyakyusa (see \sectref{bkm:Ref114648496}), and in marking a conditional clause as topic. Depending on whether the initial predicate is an infinitive in class 15 or a locative in class 16, the topic marker is \textit{ko} or \textit{po}, respectively, as shown in \xref{bkm:Ref122429137} and \xref{bkm:Ref122429135}. Considering that both \textit{ko}/\textit{po} and the topic doubling construction mark the predicate as the (contrastive) topic, their combination is said to give extra emphasis.

%%EAX
\ea
%%JUDGEMENT
%%LABEL
\label{bkm:Ref122429137}
%%CONTEXT
%%LINE1
Mwé, ukwenda \textbf{ko} endile!\\
%%LINE2
\gll
mwe  u-ku-enda  ko  a-end-ile\\
%%LINE3
\EXCLAM{}  \AUG{}-15-walk  15.\PRO{}  1\SM{}-walk-\PFV{}\\
%%TRANS1
\glt
‘He (really) walked!’\\
%%TRANS2
%%EXEND

\z


%%EAX
\ea
%%JUDGEMENT
%%LABEL
\label{bkm:Ref122429135}
%%CONTEXT
%%LINE1
Ikinyangwa pakusya (\textbf{po}) bikusya.\\
%%LINE2
\gll
i-ki-nyangwa  pa-ku-sya  po  ba-ku-sy-a\\
%%LINE3
\AUG{}-7-banana.flour  16-15-grind  16.\PRO{}  2\SM{}-\PRS{}-grind-\FV{}\\
%%TRANS1
\glt
‘As for banana flour, they do the grinding.’\\
%%TRANS2
%%EXEND

\z


In conditional sentences with \textit{linga} ‘if/when’, \textit{po} may be added at the end of the protasis, as illustrated in \xref{bkm:Ref122427981}. The marker \textit{po} is analysed as occuring in the protasis here, because it appears to be cliticised onto the verb and may be followed by a pause.

%%EAX
\ea
%%JUDGEMENT
%%LABEL
\label{bkm:Ref122427981}
%%CONTEXT
(QUIS map task)\\
%%LINE1
Linga gwendile\textbf{po} kusyaga ing’ombe itatu.\\
%%LINE2
\gll
linga  gu-end-ile=po  ku-si-ag-a  i-ng’ombe  i-tatu\\
%%LINE3
\COND{}  2\SG.\SM{}-walk-\PFV{}=16  2\SG.\SM{}-10.\OM-find-\FV{}  \AUG{}-10.cow  10-three\\
%%TRANS1
\glt
‘When you have walked some distance, you will find three cows.’\\
%%TRANS2
%%EXEND

\z


Note that \textit{po} may also be added in the apodosis, as illustrated in \xref{bkm:Ref122427969}. Here the marker is said to be in the apodosis because a possible prosodic pause would be placed before \textit{po}.

%%EAX
\ea
%%JUDGEMENT
%%LABEL
\label{bkm:Ref122427969}
%%CONTEXT
(QUIS map task)\\
%%LINE1
Linga ufikile piikolokotwa pala \textbf{po}\footnote{One of the reviewers thinks that \textit{po} in the apodosis reads like the Swahili expression \textit{ndipo} ‘it is there/there it is…’. This reading is not straightly available in Kinyakyusa. The available reading of \textit{po} in the apodosis in Kinyakyusa is ‘then…’.} kupingama.\\
%%LINE2
\gll
linga  u-fik-ile  pa-i-kolokotwa  pala  po ku-pingam-a\\
%%LINE3
\COND{}  2\SG.\SM{}-arrive-\PFV{}  16-5-butterfly  16.\DEM{}.\DIST{}  16.\PRO{} 2\SG.\SM{}-turn-\FV{}\\
%%TRANS1
\glt ‘When you arrive at the place of the butterfly, (then) you turn.’
%%TRANS2
%%EXEND

\z

It is, however, not obligatory, as illustrated in \xref{bkm:Ref122595953}, though it sounds more natural with the \textit{po} than without.\footnote{This example also illustrates a narrative use of the infinitive \textit{kujaaga}, i.e., the absence of subject inflection on the verb, and the \textit{de facto} non-expression of the subject.}

%%EAX
\ea
%%JUDGEMENT
%%LABEL
\label{bkm:Ref122595953}
%%CONTEXT
(QUIS map task)\\
%%LINE1
Linga kwisa kumyangu, kujaaga injila iji jikufyuka bwalulu. Kufyuka najo.\\
%%LINE2
\gll
linga  ku-is-a  ku-mi-angu  ku-ji-aga  i-njila iji   ji-ku-fyuk-a  bwalulu  ku-fyuka  na=jo \\
%%LINE3
\COND{}  2\SG.\SM{}-come-\FV{}  17-\POSS.1\SG{}  15-9\OM{}-find  \AUG{}-9.path 9.\DEM{}.\PROX{}   9\SM{}-\PRS{}-climb-\FV{}  northward  15-climb  with=9 \\
%%TRANS1
\glt
‘If you want to reach my place, you will find a path that moves northward. You use that path.’\\
%%TRANS2
%%EXEND

\z

The information-structuring use in conditionals seems to have developed from the use of \textit{po} as a temporal pronoun translated as ‘then’, illustrated in \xref{bkm:Ref123745644} from the Frog Story.

%%EAX
\ea
%%JUDGEMENT
%%LABEL
\label{bkm:Ref123745644}
%%CONTEXT
%%LINE1
Po akasya kalyandile ukubopa.\\
%%LINE2
\gll
po  a-ka-sya  ka-li-and-ile  u-ku-bopa\\
%%LINE3
then  \AUG{}-12-antelope  12\SM-\PST{}-start-\PFV{}  \AUG{}-15-run\\
%%TRANS1
\glt
‘Then the antelope began running.’\\
%%TRANS2
%%EXEND

\z


It seems that the marker is further developing into a pragmatic particle, introducing new topics, as in \xref{bkm:Ref123745954} – this, however, requires further investigation and is outside of the scope of this chapter.

%%EAX
\ea
%%JUDGEMENT
%%LABEL
\label{bkm:Ref123745954}
%%CONTEXT
(Context: Discussion about elections and a local candidate.)\\
%%LINE1
Po ikampeni akomelaga kugu?\\
%%LINE2
\gll
po  i-kampeni  a-kom-el-ag-a  kugu\\
%%LINE3
then  \AUG{}-10.campaign  1\SM{}-hit-\APPL-\HAB-\FV{}  17.where\\
%%TRANS1
\glt
‘Now where did he do campaigns?’\\
%%TRANS2
%%EXEND

\z


Unlike in some other Bantu languages (see Rukiga, Kîîtharaka, and Kirundi in this volume), the -\textit{o} pronoun is not used as a topic marker for other noun classes, as the failed attempt at a contrastive topic marking in \xref{bkm:Ref123743976} shows. This is potentially the case because in Kinyakyusa the pronoun has developed into an identificational copula, as shown in the contrast between predicational and identificational non-verbal predication in \sectref{bkm:Ref122533698}.

%%EAX
\ea
%%JUDGEMENT
%%LABEL
\label{bkm:Ref123743976}
%%CONTEXT
%%LINE1
Abalimi (*bo) mbapeele ifijinja; abafwimi (*bo) ngabapa.\\
%%LINE2
\gll
a-ba-limi  bo  n-ba-p-ile  i-fi-jinja  a-ba-fwimi  bo  n-ka-ba-p-a \\
%%LINE3
\AUG{}-2-farmer  2.\PRO{}  1\SG.\SM-2\OM{}-give-\PFV{}  \AUG{}-8-banana.tree \AUG{}-2-hunter  2.\PRO{}  1\SG.\SM-\NEG-2\OM{}-give-\FV{} \\
%%TRANS1
\glt
‘The farmers I gave banana trees; the hunters I didn’t give.’\\
%%TRANS2
%%EXEND

\z


To summarise, the pronoun in -\textit{o} (\posscitealt{Ashton1945} \mbox{-\textit{o}} of reference) has a meaning ‘then’ in class 16 \textit{po}, and can be used to indicate the protasis or apodosis in a conditional clause. Both \textit{po} and \textit{ko} (class 15) are used in topic doubling constructions to mark the non-finite verb as the topic, thus functioning as a topic marker, but this use is not encountered for topics in other noun classes.

\section{Clefts}
\label{bkm:Ref136514487}
Like other Bantu languages, Kinyakyusa has various types of cleft(-like) constructions. We present in turn the basic cleft (\sectref{bkm:Ref122533658}), pseudocleft (\sectref{bkm:Ref122533667}), and reverse pseudocleft (\sectref{bkm:Ref122079499}). As all types of clefts feature nominal predication, we first briefly introduce the three ways in which Kinyakyusa marks non-verbal predication in \sectref{bkm:Ref122533698}.

\subsection{Non-verbal predication}
\label{bkm:Ref122533698}
The main strategy for non-verbal predication is the omission of the augment on nouns and adjectives, as illustrated in \xref{bkm:Ref122533754:b}.

\ea
\label{bkm:Ref122533754}
%%EAX
\ea
%%JUDGEMENT
%%LABEL
\label{bkm:Ref122533754:a}
%%CONTEXT
%%LINE1
%%LINE2
\gll
u-m-manyisi\\
%%LINE3
\AUG{}-1-teacher\\
%%TRANS1
\glt
‘a/the teacher’\\
%%TRANS2
%%EXEND

%%EAX
\ex
%%JUDGEMENT
%%LABEL
\label{bkm:Ref122533754:b}
%%CONTEXT
(What does Hobokera do for a living?)\\
%%LINE1
%%LINE2
\gll
Hobokela  m-manyisi.\\
%%LINE3
1.Hobokela  1-teacher\\
%%TRANS1
\glt
‘Hobokera is a teacher.’\\
%%TRANS2
%%EXEND

\z
\z

Another strategy is to use an inflected form of the verb ‘to be’ and omission of the augment on the nominal complement. In the past tense, the inflected verb -\textit{li} ‘to be’ is used, as in \xref{bkm:Ref122533774:a}. In the future tense, it manifests as -\textit{ja} ‘to be’, as in \xref{bkm:Ref122533774:b}.

\ea
\label{bkm:Ref122533774}
%%EAX
\ea
%%JUDGEMENT
%%LABEL
\label{bkm:Ref122533774:a}
%%CONTEXT
%%LINE1
Imbunda jaa\textbf{li} mbiine.\\
%%LINE2
\gll
i-n-bunda  ji-a-li  n-biine\\
%%LINE3
\AUG{}-9-donkey  9\SM{}-\PST{}-be  9-ill\\
%%TRANS1
\glt
‘The donkey was ill.’\\
%%TRANS2
%%EXEND

%%EAX
\ex
%%JUDGEMENT
%%LABEL
\label{bkm:Ref122533774:b}
%%CONTEXT
%%LINE1
Jibagile uku\textbf{ja} mbiiki, jibagile uku\textbf{ja} ngambaku.\\
%%LINE2
\gll
ji-bag-ile  u-ku-ja  mbiiki  ji-bag-ile  u-ku-ja  ngambaku\\
%%LINE3
9\SM{}-be.able-\PFV{}  \AUG{}-15-be  9.female  9\SM{}-be.able-\PFV{}  \AUG{}-15-be  9.male\\
%%TRANS1
\glt
‘It may be a female one, it may be a male one.’\\
%%TRANS2
%%EXEND

\z
\z

The pronominal expression -\textit{o} is the last strategy that marks non-verbal predication. The use of \textit{-o} (in addition to the omission of the augment) results in an identificational reading, as illustrated in \xref{bkm:Ref122533754}. While its origin is probably pronominal, it is analysable in contemporary Kinyakyusa as an identificational copula and we hence gloss it as such.

%%EAX
\ea
%%JUDGEMENT
%%LABEL
\label{bkm:Ref148770976}
%%CONTEXT
(Who is the teacher in this classroom?)\\
%%LINE1
Ummanyisi *(jo) nkiikulu uju.\\
%%LINE2
\gll
u-m-manyisi  jo  n-kiikulu  uju\\
%%LINE3
\AUG{}-1-teacher  1.\IDCOP{}  1-woman  1.\DEM.\PROX{}\\
%%TRANS1
\glt
  ‘The teacher is this woman.’
%%TRANS2
%%EXEND

\z

\subsection{Basic cleft}
\label{bkm:Ref122533658}
In Kinyakyusa, cleft sentences consist of a predicative noun followed by a relative clause. Non-subject clefts are clearly analysable as such, as they are marked by a predicative noun, and by the relativiser. We illustrate both in the following examples. Predication in the basic cleft is marked by omitting the augment as in \xref{bkm:Ref122533754} and sometimes by adding the identificational copula -\textit{o} as in \xref{bkm:Ref148770976}. The proximal demonstrative functions as the relativiser, as seen in the simple relative clause in \xref {bkm:Ref148771161}, where we indicate the relative clause in square brackets. The same relativiser also appears in the basic clefts in \xref{bkm:Ref122534076} and \xref{bkm:Ref98081032}. For non-subject relatives, the relativiser is obligatorily present, as indicated by the parentheses for *(\textit{ifi}).

%%EAX
\ea
%%JUDGEMENT
%%LABEL
\label{bkm:Ref122534076}
%%CONTEXT
%%LINE1
\label{bkm:Ref148771161}Findu fiki *(ifi) apiijile?\\
%%LINE2
\gll
fi-ndu  fi-ki  ifi  a-piij-ile\\
%%LINE3
8-food  8-what  8.\DEM.\PROX{}  1\SM{}-cook-\PFV{}\\
%%TRANS1
\glt
‘What food is it that s/he cooked?’\\
%%TRANS2
%%EXEND

\z


%%EAX
\ea
%%JUDGEMENT
%%LABEL
\label{bkm:Ref98081032}
%%CONTEXT
%%LINE1
Kyo kikota kiliku iki Sekela akonywile?\\
%%LINE2
\gll
kio  ki-kota  ki-liku  iki  Sekela  a-konyol-ile\\
%%LINE3
7.\IDCOP{}  7-chair  7-which  7.\DEM.\PROX{}  1.Sekela  1\SM{}-break-\PFV{}\\
%%TRANS1
\glt
‘Which chair did Sekela break?’\\
%%TRANS2
%%EXEND

\z

%%EAX
\ea
%%JUDGEMENT
%%LABEL
%%CONTEXT
%%LINE1
Ukatagege mmiisi umwana [uju tupapile].\\
%%LINE2
\gll
u-ka-taag-ag-e  mu-ma-isi  u-mu-ana uju  tu-paap-ile  \\
%%LINE3
2\SG.\SM-\NEG{}-throw-\HAB-\SBJV{}  18-6-water  \AUG{}-1-child 1.\DEM.\PROX{}  1\PL.\SM{}-give.birth-\PFV{}\\
%%TRANS1
\glt ‘How dare you throw into the river the child who we parented.’
%%TRANS2
%%EXEND

\z

For subject ``clefts" it is less straightforward to determine whether they are clefts and what their structure is. On the one hand, the augment is absent on the focused noun, as in \xref{bkm:Ref136505713}, and often the identificational copula is present, indicating that it is a predicative noun as expected in a cleft. But on the other hand, the relativiser is optional, as seen in \xref{bkm:Ref98081490}. This can lead to ambiguity as in \xref{bkm:Ref98081471}, where \textit{aba} may be interpreted as the proximal demonstrative, or as the relativiser, as indicated in the translations.

%%EAX
\ea
%%JUDGEMENT
%%LABEL
\label{bkm:Ref136505713}
%%CONTEXT
(Who has stolen the soda (between the adults and the children)?)\\
%%LINE1
(\textsuperscript{\#}A)Baana biibile.\\
%%LINE2
\gll
ba-ana  bi-ib-ile\\
%%LINE3
2-children  2\SM{}-steal-\PFV{}\\
%%TRANS1
\glt
‘It’s the children that have stolen.’\\
%%TRANS2
%%EXEND

\z


%%EAX
\ea
%%JUDGEMENT
%%LABEL
\label{bkm:Ref98081490}
%%CONTEXT
%%LINE1
Jo jiliku (iji) jifwile?\\
%%LINE2
\gll
jo  ji-liku  iji  ji-fw-ile\\
%%LINE3
9.\IDCOP{}  9-which  9.\DEM.\PROX{}  9\SM{}-die-\PFV{}\\
%%TRANS1
\glt
‘It is which (animal) that died?’\\
%%TRANS2
%%EXEND

\z


%%EAX
\ea
%%JUDGEMENT
%%LABEL
\label{bkm:Ref98081471}
%%CONTEXT
%%LINE1
Bo baana aba biibile.\\
%%LINE2
\gll
bo  ba-ana  aba  bi-ib-ile\\
%%LINE3
2.\IDCOP{}  2-children  2.\DEM.\PROX{}  2\SM{}-steal-\PFV{}\\
%%TRANS1
\glt
‘It’s the children who have stolen.’ (demonstrative = relativiser)\\
%%TRANS2
%%EXEND

‘It’s these children who have stolen.’ (demonstrative = deictic)\\
\z

We postulate that the construction with either or both of the relativiser and the identificational copula is structurally a cleft, whereas the construction with neither relativiser nor copula merely focuses the preverbal element without a cleft structure (see \sectref{bkm:Ref114653225}). This is also supported by the interpretational difference given for \xref{bkm:Ref98082217}: Without \textit{uju}, the implication is that the speaker does not know for certain if someone went to Mbeya, but with \textit{uju} the speaker knows that someone went, but not the identity of the person. That is, the presence of \textit{uju} adds a presupposition of existence.

%%EAX
\ea
%%JUDGEMENT
%%LABEL
\label{bkm:Ref98082217}
%%CONTEXT
%%LINE1
Juani (uju) ikubuuka kumbeje?\\
%%LINE2
\gll
ju-ani  uju  a-ku-buuk-a  ku-mbeje\\
%%LINE3
1-who  1.\DEM.\PROX{}  1\SM-\PRS{}-go-\FV{}  17-Mbeya\\
%%TRANS1
\glt
(no \textit{uju}) ‘Who will go to Mbeya?’\\
%%TRANS2
(with \textit{uju}) ‘Who (of these people) is it that will go to Mbeya?’

%%EXEND

\z


The presupposition of existence is confirmed in the impossibility to answer \xref{bkm:Ref98083788} with \textit{najumo} ‘nobody’. This can be compared to the question without \textit{uju} in the preverbal focus position (not cleft) in \xref{bkm:Ref114653070}.\footnote{We are not certain whether the presence of \textit{=po} in the one but not the other example makes a difference in meaning.}

%%EAX
\ea
\begin{xlist}
\exi{Q:}
%%JUDGEMENT
[]{
%%LABEL
\label{bkm:Ref98083788}
%%CONTEXT
%%LINE1
Juani uju afwilepo (apa)?\\
%%LINE2
\gll
ju-ani  uju  a-fw-ile=po  apa\\
%%LINE3
1-who  1.\DEM{}.\PROX{}  1\SM{}-die-\PFV{}=16  16.\DEM{}.\PROX{}\\
%%TRANS1
\glt
‘Who is it that has died here?’\\
%%TRANS2
}
%%EXEND

\exi{A:}
[\textsuperscript{\#}]{
Najumo.\\
\glt ‘Nobody.’
}
\end{xlist}
\z

%%EAX
\ea
\begin{xlist}
\exi{Q:}
%%JUDGEMENT
[]{
%%LABEL
\label{bkm:Ref114653070}
%%CONTEXT
%%LINE1
Juani afwile (apa)?\\
%%LINE2
\gll
ju-ani  a-fw-ile  apa\\
%%LINE3
1-who  1\SM{}-die-\PFV{}  16.\DEM.\PROX{}\\
%%TRANS1
\glt
‘Who has died here?’\\
%%TRANS2
}
%%EXEND

\exi{A:}
[]{
Najumo.\\
\glt ‘Nobody.’
}

\end{xlist}
\z

The interpretation of the focused element in the cleft can thus be said to at least be identificational. Interrogatives may be clefted, as illustrated above, and answers to such questions can similarly occur in a cleft, as in \xref{bkm:Ref98083872}. In this example, we can see that this is a cleft by the identificational copula.\largerpage[-1]\pagebreak

%%EAX
\ea
%%JUDGEMENT
%%LABEL
\label{bkm:Ref98083872}
%%CONTEXT
(Who died here?)\\
%%LINE1
Jo umpalamani afwile.\\
%%LINE2
\gll
jo  u-m-palamani  a-fw-ile\\
%%LINE3
1.\IDCOP{}  \AUG{}-1-neighbour  1\SM{}-die-\PFV{}\\
%%TRANS1
\glt
‘It’s the neighbour who died.’\\
%%TRANS2
%%EXEND

\z


Furthermore, the identificational (and not correcting) function can be seen in the acceptability of \xref{bkm:Ref98082928} in context 1 but not context 2.

%%EAX
\ea
%%JUDGEMENT
%%LABEL
\label{bkm:Ref98082928}
%%CONTEXT
(Context 1: ‘Who is it that you saw?’\\
\textsuperscript{\#}Context 2: ‘Did you see \textit{this} child?’)\\
%%LINE1
Jo jula (uju) nalimbwene.\\
%%LINE2
\gll
jo  ju-la  uju  n-ali-m-bon-ile\\
%%LINE3
1.\IDCOP{}  1-\DEM.\DIST{}  1.\DEM.\PROX{}  1\SG.\SM-\PST{}-1\OM{}-see-\PFV{}\\
%%TRANS1
\glt ‘It’s that one (that) I saw.’
%%TRANS2
%%EXEND

\z


Similarly, the ungrammaticality of clefting ‘nothing’ as in \xref{bkm:Ref98083554} can also be explained by incompatibility with identificational focus.

%%EAX
\ea
%%JUDGEMENT
[*]{
%%LABEL
\label{bkm:Ref98083554}
%%CONTEXT
%%LINE1
Fyo nafimo ifi fyonangike.\\
%%LINE2
\gll
fio  nafimo  ifi  fi-onang-ik-ile\\
%%LINE3
8.\IDCOP{}  8.nothing  8.\DEM.\PROX{}  8\SM{}-destroy-\STAT-\PFV{}\\
%%TRANS1
\glt
int. ‘It is nothing that was destroyed.’\\
%%TRANS2
}
%%EXEND

\z


The basic cleft should be compared to the preverbal focus described in \sectref{bkm:Ref114653225} and the reverse pseudocleft described in \sectref{bkm:Ref98081626}. Relevant for the analysis of the reverse pseudocleft are the facts that personal and demonstrative pronouns may also be clefted, as shown in \xxref{bkm:Ref98085850}{bkm:Ref98085919} – this will become clearer in \sectref{bkm:Ref98081626}.

%%EAX
\ea
%%JUDGEMENT
%%LABEL
\label{bkm:Ref98085850}
%%CONTEXT
(Did they put the flask outside or did you?)\\
%%LINE1
(Une) jo une mbiikile.\\
%%LINE2
\gll
une  jo  une  n-biik-ile\\
%%LINE3
1\SG.\PRO{}  1.\IDCOP{}  1\SG.\PRO{}  1\SG.\SM{}-put-\PFV{}\\
%%TRANS1
\glt
‘(Me,) It’s me who put (it) (there).’\\
%%TRANS2
%%EXEND

\z


%%EAX
\ea
%%JUDGEMENT
%%LABEL
%%CONTEXT
(Context: You see that someone has climbed into the tree, you see traces, and there are various people you suspect. You ask one of them.)\\
%%LINE1
Jo gwe gwakwelile mumpiki ugu?\\
%%LINE2
\gll
jo  gwe  gu-a-kwel-ile  mu-m-piki  ugu\\
%%LINE3
1.\IDCOP{}  2\SG.\PRO{}  2\SG.\SM-\PST{}-climb-\PFV{}  18-3-tree  3.\DEM.\PROX{}\\
%%TRANS1
\glt
‘Is it you who climbed this tree?’\\
%%TRANS2
%%EXEND

\z


%%EAX
\ea
%%JUDGEMENT
%%LABEL
\label{bkm:Ref98085919}
%%CONTEXT
(Recipe)
%%LINE1
Po lo ulu ngwega n’ umwoto ngubiika pamwanya.\\
%%LINE2
\gll
po  lo  ulu  n-ku-eg-a  na  u-moto   n-ku-biik-a  pa-mwanya \\
%%LINE3
then  11.\IDCOP{}  11.\DEM.\PROX{}  1\SG.\SM{}-\PRS{}-take-\FV{}  and  \AUG{}-3.fire  1\SG.\SM-\PRS{}-put-\FV{}  16-top \\
%%TRANS1
\glt
‘Then it is at this time that I put fire on top (of the rice pot).'\\
%%TRANS2
%%EXEND\largerpage[2]

%%EAX
\ex
%%JUDGEMENT
%%LABEL
%%CONTEXT
(Did you vote at Ibililo?)\\
%%LINE1
Eee ko kuno twasalilaga.\\
%%LINE2
\gll
eee  ko  ku-no  tu-a-sal-il-ag-a\\
%%LINE3
yes  17.\IDCOP{}  17-\DEM.\PROX{}  1\PL.\SM-\PST{}-choose-\APPL-\HAB-\FV{}\\
%%TRANS1
\glt
‘Yes that is (the place) where we voted / it’s there that we voted.’\\
%%TRANS2
%%EXEND

\z


In summary, Kinyakyusa basic clefts are marked by either or both of the identificational copula (plus absence of the augment) and the proximal demonstrative functioning as the relative marker. The clefted NP is in identificational focus, and this is where it differs from the simple preverbal focused subject, which does not come with the presupposition+identification interpretation.

\subsection{Pseudocleft}
\label{bkm:Ref122533667}
Pseudoclefts consist of a free relative clause followed by a predicative identificational noun phrase. The free relative is marked as such by the relativiser (proximal demonstrative), as in \xref{bkm:Ref98083691} with \textit{uju}. This sets up the presupposition that an entity described by the relative exists – in this case that there is someone who wrote this. This presupposition of existence is confirmed by the infelicity of answering the pseudocleft question in \xref{bkm:Ref98083949} by \textit{najumo} ‘nobody’.\largerpage[2]

%%EAX
\ea
%%JUDGEMENT
%%LABEL
\label{bkm:Ref98083691}
%%CONTEXT
(Who wrote this?)\\
%%LINE1
Uju aasimbile jila, mwana gwa nywili imbimba.\\
%%LINE2
\gll
uju  a-a-simb-ile  ji-la  mu-ana  gu-a  nywili  i-mbimba\\
%%LINE3
1.\DEM.\PROX{}  1\SM-\PST{}-write-\PFV{}  9-\DEM.\DIST{}  1-child  1-\CONN{}  10.hair  \AUG{}-10.short\\
%%TRANS1
\glt
‘The one who wrote this is a/the child with short hair.’\\
%%TRANS2
%%EXEND

\ex
%%EAX
%%JUDGEMENT
%%LABEL
\label{bkm:Ref98083949}
%%CONTEXT
(Context: You saw a group running past the window but didn’t identify anyone.)\\
%%LINE1
\begin{xlist}
\exi{Q:}{
Aba ubabwene bo baani?\\
%%LINE2
\gll
aba  u-ba-bon-ile  bo  ba-ani\\
%%LINE3
2.\DEM.\PROX{}  2\SG.\SM-2\OM{}-see-\PFV{}  2.\IDCOP{}  2-who\\
%%TRANS1
\glt
‘Who(pl) did you see?’, lit. ‘The ones that you saw are who?’\\}
%%TRANS2
%%EXEND

%%EAX
\exi{A:}{
%%JUDGEMENT
%%LABEL
%%CONTEXT
%%LINE1
\textsuperscript{\#}Najumo./\textsuperscript{\#}Nabamo.\\
%%LINE2
\gll
na-ju-mo/na-ba-mo\\
%%LINE3
and-1-one/and-2-one\\
%%TRANS1
\glt
‘Nobody.’\\}
%%TRANS2
%%EXEND
\end{xlist}
\z

The entity is then identified by the nominal predicate, and because of this specificational function, predication prefers the identificational copula -\textit{o} \xref{bkm:Ref97720875}. Note that a similar structure without the copula can be interpreted as a predicational copular clause, attributing a property to the referent of the free relative: In \xref{bkm:Ref97720755}, the referent ‘what hurts me’ is said to be dangerous. There is no identification or specification here, i.e., it is not a pseudocleft.

%%EAX
\ea
%%JUDGEMENT
%%LABEL
\label{bkm:Ref97720875}
%%CONTEXT
(Context: A soda has been stolen. There is a group of adults and a group of children – who stole the soda?)\\
%%LINE1
Aba bahiijile isooda bo baana (aba).        \jambox*{[specificational]}
%%LINE2
\gll
aba  ba-hij-ile  i-sooda  bo  ba-ana  aba\\
%%LINE3
2.\DEM.\PROX{}  2\SM{}-steal-\PFV{}  \AUG{}-9.soda  2.\IDCOP{}  2-children  2.\DEM.\PROX{}\\
%%TRANS1
\glt
‘The ones who stole the soda are (these) children.’\\
%%TRANS2
%%EXEND

\z


%%EAX
\ea
%%JUDGEMENT
%%LABEL
\label{bkm:Ref97720755}
%%CONTEXT
%%LINE1
Ifi fikumbaba fipala fiijo.      \jambox*{[predicational}
%%LINE2
\gll
ifi  fi-ku-m-bab-a  fi-pala  fiijo\\
%%LINE3
8.\DEM.\PROX{}  8\SM-\PRS-1\SG.\OM{}-hurt-\FV{}  8-dangerous  \INT{}\\
%%TRANS1
\glt
‘What hurts me is very dangerous.’\\
%%TRANS2
%%EXEND

\z


The referent that is predicated in the pseudocleft is in identificational focus. The referent that is identified must therefore be specific enough. For the focused \textit{mundu} ‘person’ in \xref{bkm:Ref98084318}, the speakers commented that “It is grammatical, but it does not make sense” – this is because the free relative already gives away that it concerns a person, as it refers to class 1.

%%EAX
\ea
%%JUDGEMENT
[\textsuperscript{?}]{
%%LABEL
\label{bkm:Ref98084318}
%%CONTEXT
%%LINE1
Uju aponile jo mundu.\\
%%LINE2
\gll
uju  a-pon-ile  jo  mu-ndu\\
%%LINE3
1.\DEM.\PROX{}  1\SM{}-recover-\PFV{}  1.\IDCOP{}  1-person\\
%%TRANS1
\glt
‘The one who recovered is a person/someone.’\\
%%TRANS2
}
%%EXEND

\z


Modification by ‘even’ \xref{bkm:Ref98084827} and ‘all’ \xref{bkm:Ref98084841} is ungrammatical, as these do not sufficiently identify a specific referent. They also test for exclusivity, but as shown below, the identification seems to play a more central role here than exclusivity.

%%EAX
\ea
%%JUDGEMENT
%%LABEL
\label{bkm:Ref98084827}
%%CONTEXT
%%LINE1
Uju nalyaganiile nagwe jo (*joope) n’uFrida.\\
%%LINE2
\gll
uju  n-ali-aganil-ile  na-gwe  jo  ju-ope  na  u-Frida\\
%%LINE3
1.\DEM.\PROX{}  1\SG.\SM-\PST{}-meet-\PFV{}  with-1.\PRO{}  1.\IDCOP{}  1-even  and  \AUG{}-1.Frida\\
%%TRANS1
\glt
‘The one I met is (even) Frida.’\\
%%TRANS2
%%EXEND

\z


%%EAX
\ea
%%JUDGEMENT
%%LABEL
\label{bkm:Ref98084841}
%%CONTEXT
%%LINE1
Ifi aagogile Kato fitana (*fyosa).\\
%%LINE2
\gll
ifi  a-a-gog-ile  Kato  fi-tana  fi-osa\\
%%LINE3
8.\DEM.\PROX{}  1\SM-\PST{}-kill-\PFV{}  1.Kato  8-cup  8-all\\
%%TRANS1
\glt
‘What Kato broke is (*all) cups.’\\
%%TRANS2
%%EXEND

\z


Identificational focus may here be distinguished from exclusive focus, as the interpretation of a numeral in the pseudocleft is the lower boundary and not the exact amount as would be expected if the interpretation were exclusive (compare to the CV prefix in \xref{bkm:Ref144796821}).

%%EAX
\ea
%%JUDGEMENT
%%LABEL
%%CONTEXT
%%LINE1
Isi tukulonda ndalama imilioni ibili ukumalikisya inyumba.\\
%%LINE2
\gll
isi  tu-ku-lond-a  ndalama  i-milioni  i-bili u-ku-mal-ik-isy-a  i-nyumba\\
%%LINE3
10.\DEM.\PROX{}  1\PL.\SM-\PRS{}-want-\FV{}  10.money  \AUG{}-10.million  \AUG{}-10.two \AUG{}-15-finish-\STAT-\CAUS-\FV{}  \AUG{}-9.house\\
%%TRANS1
\glt ‘What we need is two million to finish the house.’ (can be more)
%%TRANS2
%%EXEND

\z

In summary, the pseudocleft describes an entity in the free relative and identifies that entity as the referent of the noun phrase that follows the identificational copula, expressing identificational focus.

\subsection{Reverse pseudocleft/left-dislocation + cleft}
\label{bkm:Ref98081626}\label{bkm:Ref122079499}
Swapping the free relative and the identifying NP results in a reverse pseudocleft (NP \COP{} FR), as in \xref{bkm:Ref98085466}. The focus here, indicated by underlining, can be on the postcopular FR as in \xref{bkm:Ref98085466:a} or the precopular NP as in \xref{bkm:Ref98085466:b}.

\ea
\label{bkm:Ref98085466}
%%EAX
\ea
%%JUDGEMENT
%%LABEL
\label{bkm:Ref98085466:a}
%%CONTEXT
(Context 1: I’m talking about Bahati, but the name doesn’t ring a bell for you. At that point she enters the store that we are in.)\\
%%LINE1
Bahati jo uju ikuingila.\\
%%LINE2
\gll
Bahati  jo  uju  a-ku-ingil-a\\
%%LINE3
1.Bahati  1.\IDCOP{}  1.\DEM.\PROX{}  1\SM-\PRS{}-enter-\FV{}\\
%%TRANS1
\glt
‘Bahati is the one who is entering.’\\
%%TRANS2
%%EXEND

%%EAX
\ex
%%JUDGEMENT
%%LABEL
\label{bkm:Ref98085466:b}
%%CONTEXT
(Context 2: Someone enters the store that we are in, and you ask, ‘Who is the one that is entering?’)\\
%%LINE1
Bahati jo uju ikuingila.\\
%%LINE2
\gll
Bahati  jo  uju  a-ku-ingil-a\\
%%LINE3
1.Bahati  1.\IDCOP{}  1.\DEM.\PROX{}  1\SM-\PRS{}-enter-\FV{}\\
%%TRANS1
\glt
‘Bahati is the one who is entering.’\\
%%TRANS2
%%EXEND

\z
\z

We analyse the sentence with focus on the postcopular part (the FR, as in \xref{bkm:Ref98085466:a}) as a straightforward copular construction. For the one with precopular focus (as in \xref{bkm:Ref98085466:b}), we suspect that two underlying structures are possible: the first as a copular construction (i.e., a reverse pseudocleft: ‘Bahati is who is entering’) and the second with a left-peripheral NP followed by a basic cleft (‘Bahati, it’s her who is entering’), as also suggested for Kîîtharaka, Kirundi, Rukiga, and Cicopi (see chapters in this volume). We discuss properties of the sentences with focus on the initial referent to show evidence for both underlying structures, starting with the reverse pseudocleft and then indicating properties for left-dislocation.

In spontaneous discussion and narratives, the construction is typically used to identify a referent, as illustrated in \xref{bkm:Ref123833612} and \xref{bkm:Ref123834425}. In \xref{bkm:Ref123833612} we indicate the initial constituent in square brackets.

%%EAX
\ea
%%JUDGEMENT
%%LABEL
\label{bkm:Ref123833612}
%%CONTEXT
%%LINE1
[Ing’ombe iji jikulile buno] jo iji tukuti indama \textit{yaani} jikutama.\\
%%LINE2
\gll
i-ng’ombe  iji  ji-kul-ile  buno  jo  iji  tu-ku-ti i-ndama  yaani  ji-ku-tam-a\\
%%LINE3
\AUG-9{}.cow  9.\DEM.\PROX{}  9\SM{}-grow-\PFV{}  like.this  9.\PRO{}  9.\DEM.\PROX{}  1\PL.\SM-\PRS{}-say \AUG{}-9.calf  that.is(<Sw.)  9\SM-\PRS{}-moo-\FV{}\\
%%TRANS1
\glt ‘The cow which has grown like this, it is the one we call a calf, as it moos.’
%%TRANS2
%%EXEND

%%EAX
\ex
%%JUDGEMENT
%%LABEL
\label{bkm:Ref123834425}
%%CONTEXT
(How many votes did Leo and his friends get?)\\
%%LINE1
Edom jo uju akabile ikula nyingi.\\
%%LINE2
\gll
Edom  jo  uju  a-kab-ile  i-kula  nyingi\\
%%LINE3
1.Edom  1.\IDCOP{}  1.\DEM.\PROX{}  1\SM{}-get-\PFV{}  \AUG{}-10.vote  10.many\\
%%TRANS1
\glt
‘Edom is (the one) who got the majority votes.’\\
%%TRANS2
%%EXEND

\z


Interestingly, the construction can be used for subject questions but not object or adverb questions, as shown in \xxref{bkm:Ref122708498}{bkm:Ref122708500}. Further research is needed to pinpoint why this is.\largerpage[2]

%%EAX
\ea
%%JUDGEMENT
[]{
%%LABEL
\label{bkm:Ref122708498}
%%CONTEXT
%%LINE1
Juani jo uju ati atulongosye?\\
%%LINE2
\gll
ju-ani  jo  uju  a-ti  a-tu-longosy-e\\
%%LINE3
1-who  1.\IDCOP{}  1.\DEM.\PROX{}  1\SM{}-say  1\SM{}-1\PL.\OM{}-lead-\SBJV{}\\
%%TRANS1
\glt
‘Who is it that will lead us?’ (lit. ‘Who is the one that says s/he should lead us?’)\\
%%TRANS2
}
%%EXEND

\z

%%EAX
\ea
%%JUDGEMENT
[*]{
%%LABEL
%%CONTEXT
%%LINE1
Fiki fyo ifi uliile?\\
%%LINE2
\gll
fi-ki  fio  ifi  u-li-ile\\
%%LINE3
8-what  8.\IDCOP{}  8.\DEM.\PROX{}  2\SG.\SM{}-eat-\PFV{}\\
%%TRANS1
\glt
int. ‘What have you eaten?’ (lit. ‘What is what you’ve eaten?’)\\
%%TRANS2
}
%%EXEND

\z

%%EAX
\ea
%%JUDGEMENT
[*]{
%%LABEL
\label{bkm:Ref122708500}
%%CONTEXT
%%LINE1
Ndili lo ulu aakubuja?\\
%%LINE2
\gll
ndiri  lo  ulu  a=a-ku-buj-a\\
%%LINE3
when  11.\IDCOP{}  11.\DEM.\PROX{}  \FUT{}=1\SM{}-\PRS{}-return-\FV{}\\
%%TRANS1
\glt
int. ‘When is it that s/he will return?’ (lit. ‘When is when s/he will return?’)\\
%%TRANS2
}
%%EXEND

\z

The answer to an alternative question may also be phrased in a reverse pseudocleft, as in \xref{bkm:Ref123834919}, translated into English with an it-cleft to reflect the interpretation rather than the structure.

%%EAX
\ea
%%JUDGEMENT
%%LABEL
\label{bkm:Ref123834919}
%%CONTEXT
(Context: speakers see a drawing of two women entering through a door, with a question ‘Did two women or two children enter the house?’)\\
%%LINE1
Abakiikulu babili bo aba baalingile nnyumba.\\
%%LINE2
\gll
a-ba-kiikulu  ba-bili  bo  aba  ba-ali-ingil-ile  mu-n-yumba\\
%%LINE3
\AUG{}-2-woman  2-two  2.\IDCOP{}  2.\DEM.\PROX{}  2\SM{}-\PST{}-enter-\PFV{}  18-9-house\\
%%TRANS1
\glt
‘It is two women who entered the house.’\\
%%TRANS2
%%EXEND

\z


Furthermore, the construction may be used to correct the interlocutor on the identity of the subject, as illustrated in \xref{bkm:Ref136510603} and \xref{bkm:Ref136510605}.

%%EAX
\ea
%%JUDGEMENT
%%LABEL
\label{bkm:Ref136510603}
%%CONTEXT
(Are the cows jumping around in the field? No, the cows are not jumping in the ground, …)\\
%%LINE1
looli imbene syo isi sikunyela nkibanje.\\
%%LINE2
\gll
looli  i-mbene  syo  isi  si-ku-nyel-a  mu-ki-banje\\
%%LINE3
but  \AUG{}-10.goat  10.\IDCOP{}  10.\DEM.\PROX{}  10\SM-\PRS{}-jump-\FV{}  18-7-ground\\
%%TRANS1
\glt
‘But it is goats which play in the ground.’\\
%%TRANS2
%%EXEND

\z


%%EAX
\ea
%%JUDGEMENT
%%LABEL
\label{bkm:Ref136510605}
%%CONTEXT
(I did not find him/her in the river.)\\
%%LINE1
Ubibi jumo jo uju ambele.\\
%%LINE2
\gll
u-bibi  ju-mo  jo  uju  a-m-pa-ile\\
%%LINE3
\AUG{}-1.grandmother  1-one  1.\IDCOP{}  1.\DEM.\PROX{}  1\SM-1\SG.\OM{}-give-\PFV{}\\
%%TRANS1
\glt
‘An old woman gave (him/her) to me.’\\
%%TRANS2
%%EXEND

\z


The initial NP in the reverse pseudocleft may be modified by the exhaustive particle -\textit{ene} ‘only’ \xref{bkm:Ref148771823} and \xref{bkm:Ref148771845}, but not by ‘even’ \xref{bkm:Ref98087453} or ‘all’ \xref{bkm:Ref98087461}, suggesting an exclusive focus interpretation. This is because for ‘even’, it must be true that the proposition is true for other referents lower on the scale, and therefore no alternatives can be excluded. For example  \xref{bkm:Ref98087453}, this means that in addition to Salima, who is not likely to laugh, others also laughed. The same reasoning holds for ‘all’, as this includes all the members of the set.

%%EAX
\ea
%%JUDGEMENT
[]{
%%LABEL
\label{bkm:Ref148771823}
%%CONTEXT
(Context: Speakers are sure that all attendees went to the dancing party on foot except Peter who came by car.) \\
%%LINE1
Mwene Pita jo uju alyendile mwigali.\\
%%LINE2
\gll
mu-ene  Pita  jo  uju  a-ali-end-ile  mu-i-gali\\
%%LINE3
1-only  1.Peter  1.\IDCOP{}  1.\DEM.\PROX{}  1\SM-\PST{}-walk-\PFV{}  18-5-car\\
%%TRANS1
\glt
\textsuperscript{?}‘Only Peter it is who came by car.’\\
%%TRANS2
}
%%EXEND

\z


%%EAX
\ea
%%JUDGEMENT
[]{
%%LABEL
\label{bkm:Ref148771845}
%%CONTEXT
(Context: Speakers expected many people to arrive but only one person came.)\\
%%LINE1
Mwene Salima jo uju ikufika.\\
%%LINE2
\gll
mu-ene  Salima  jo  uju  a-ku-fik-a\\
%%LINE3
1-only  1.Salima  1.\PRO{}  1.\DEM.\PROX{}  1\SM-\PRS{}-arrive-\FV{}\\
%%TRANS1
\glt
‘Only Salima is the one who arrives.’\\
%%TRANS2
}
%%EXEND

\z


%%EAX
\ea
%%JUDGEMENT
[]{
%%LABEL
\label{bkm:Ref98087453}
%%CONTEXT
%%LINE1
Joope Salima (*jo uju) asekilepo.\\
%%LINE2
\gll
ju-ope  Salima  jo  uju  a-sek-ile=po\\
%%LINE3
1-even  1.Salima  1.\PRO{}  1.\DEM.\PROX{}  1\SM-{}laugh-\PFV{}=16\\
%%TRANS1
\glt
‘Even Salima (*is the one who) laughed.’\\
%%TRANS2
}
%%EXEND

\z


%%EAX
\ea
%%JUDGEMENT
[*]{
%%LABEL
\label{bkm:Ref98087461}
%%CONTEXT
%%LINE1
Abandu boosa bo aba bikufwala ifitili.\\
%%LINE2
\gll
a-ba-ndu  ba-osa  bo  aba  ba-ku-fwal-a  i-fi-tili\\
%%LINE3
\AUG{}-2-person  2-all  2.\PRO{}  2.\DEM.\PROX{}  2\SM-\PRS{}-wear-\FV{}  \AUG{}-8-hat\\
%%TRANS1
\glt
int. ‘All people are the ones wearing hats.’\\
%%TRANS2
}
%%EXEND

\z


The interpretation of the construction may be exclusive but is perhaps not inherently exhaustive, as the answer to an incomplete question can be answered by ‘yes’, as in \xref{bkm:Ref122709175}. It is true that Moses washed shirts, therefore ‘yes’ is a good answer, but it is not true that he washed \textit{only} shirts. Therefore, if the exhaustive interpretation were inherent to this construction, we would expect the answer to be ‘no’. This can be compared to the examples in \xref{bkm:Ref122709209} and \xref{bkm:Ref122709211} in the section on the CV exhaustive marker, where the answer is indeed ‘no’.

%%EAX
\ea
%%JUDGEMENT
%%LABEL
\label{bkm:Ref122709175}\label{bkm:Ref136517649}
%%CONTEXT
(Context: Speakers are shown a picture of a clothesline with various washed sheets and clothes, including shirts.)\\
%%LINE1
Bule isyati syo isi Mose asukile?\\
%%LINE2
\gll
bule  i-syati  syo  isi  Mose  a-suk-ile\\
%%LINE3
Q  \AUG{}-10.shirt  10.\IDCOP{}  10.\DEM{}.\PROX{}  1.Mose  1\SM{}-wash-\PFV{}\\
%%TRANS1
\glt
‘Is it shirts that Moses washed?’\\
%%TRANS2
%%EXEND

%%EAX
\sn
%%JUDGEMENT
%%LABEL
%%CONTEXT
%%LINE1
Eena Mose asukile isyati n’ imyenda igingi.\\
%%LINE2
\gll
eena  Mose  a-suk-ile  i-syati  na  i-mi-enda  i-gi-ngi\\
%%LINE3
yes  1.Moses  1\SM{}-wash-\PFV{}  \AUG{}-10.shirt  and  \AUG{}-4-clothes  \AUG{}-4-other\\
%%TRANS1
\glt
‘Yes. Moses washed shirts and other clothes.’\\
%%TRANS2
%%EXEND

\z

While all these examples and contexts indicate the initial NP as a focus constituent in a reverse pseudocleft, there is some evidence that a second underlying structure is also possible. In this second analysis, the initial NP is in the left periphery, followed by a cleft in which the coreferring demonstrative is clefted (compare the chapters on Kîîtharaka, Kirundi, Rukiga, and Cicopi for similar discussion), with a literal translation of \xref{bkm:Ref122709175} as ‘Shirts,  is it those that Moses washed?’. The same surface structure can thus be parsed in two different ways, represented in \tabref{tab:nny-twostructures}. Note that in the first structure, the demonstrative functions as the relativiser/head of the free relative clause, and in the second structure, the demonstrative is the clefted element and the relativisation is unmarked (see the optionality of the relative marker discussed in \sectref{bkm:Ref122533658}).

\begin{table}
\begin{tabularx}{\textwidth}{lllllX}
\lsptoprule
 & NP & \IDCOP{} & \DEM{}.\PROX{} & (S) V & \\
\midrule
1 & NP & is & \multicolumn{2}{c}{free relative} & \textit{‘Sara is (the one) who we like’}\\
& & & & & reverse pseudocleft\\
\addlinespace
2 & (NP) & is & pro\textsubscript{FOC} & relative & \textit{‘Sara, it is HER that we like’}\\
& & & & & left dislocation + cleft\\
\lspbottomrule
\end{tabularx}
\caption{Two underlying structures}
\label{tab:nny-twostructures}
\end{table}

There are at least two indications for the second analysis existing next to the reverse pseudocleft analysis. First, it is possible for a prosodic break to occur between the left-peripheral NP and the cleft, as in \xref{bkm:Ref125385688} and \xref{bkm:Ref125385689}.\largerpage[2.25]

%%EAX
\ea
%%JUDGEMENT
%%LABEL
\label{bkm:Ref125385688}
%%CONTEXT
%%LINE1
Amiisi aga, go aga tukupiijila.\\
%%LINE2
\gll
a-ma-isi  aga  go  aga  tu-ku-piij-il-a\\
%%LINE3
\AUG{}-6-water  6.\DEM.\PROX{}  6.\IDCOP{}  6.\DEM.\PROX{}  1\PL.\SM-\PRS{}-cook-\APPL-\FV{}\\
%%TRANS1
\glt
‘This water, it’s this that we cook with.’\\
%%TRANS2
%%EXEND

\z


%%EAX
\ea
%%JUDGEMENT
%%LABEL
\label{bkm:Ref125385689}
%%CONTEXT
(Context: We find a calabash among other calabashes and want to indicate that this particular one belongs to the thin person.)\\
%%LINE1
Ikipale iki, kyo iki apeeligwe unsekele.\\
%%LINE2
\gll
i-ki-pale  iki  kio   iki  a-p-el-igw-ile   u-n-sekele \\
%%LINE3
\AUG{}-7-calabash  7.\DEM.\PROX{}  7.\IDCOP{}  7.\DEM.\PROX{}  1\SM{}-give-\APPL-\PASS-\PFV{} \AUG{}-1-thin \\
%%TRANS1
\glt
‘This calabash, it is this one that the thin one was given.’\\
%%TRANS2
%%EXEND

\z

Second, the existence of the left-peripheral NP needs to somehow be presupposed, which makes sense if it is a topic. This can be seen in the felicitous and infelicitous contexts for \xref{bkm:Ref98086853}, where some set of boys must be present, and also in the clearly topical interpretation in \xref{bkm:Ref98087411}: ‘As for tea, I prefer this type’.

%%EAX
\ea
%%JUDGEMENT
%%LABEL
\label{bkm:Ref98086853}
%%CONTEXT
(Context 1: Which boy cut the banana?\\
Context 2: Did the big boy or the small boy cut the banana?\\
Context 3: \textsuperscript{\#}Did the big boy cut the banana?)\\
%%LINE1
(Mma) Undumyana unandi jo uju aasengile itoki.\\
%%LINE2
\gll
mma  u-m-lumyana  u-nandi  jo   uju  a-a-seng-ile  i-toki\\
%%LINE3
no  \AUG{}-1-boy  \AUG{}-little  1.\IDCOP{}  1.\DEM.\PROX{}  1\SM-\PST{}-cut-\PFV{}  \AUG{}-5.banana\\
%%TRANS1
\glt ‘(No) The little boy (it is him who) cut the banana.’
%%TRANS2
%%EXEND

\z

%%EAX
\ea
%%JUDGEMENT
%%LABEL
\label{bkm:Ref98087411}
%%CONTEXT
(Context 1: There are different types of tea to choose from (with milk, ginger, black)\\
Context 2: \textsuperscript{\#}Do you want tea or coffee?\\
Context 3: \textsuperscript{\#}You want coffee, right?)\\
%%LINE1
Ikyai jo iji ngulonda.\\
%%LINE2
\gll
i-kyai  jo  iji  n-ku-lond-a\\
%%LINE3
\AUG{}-9.tea  9.\IDCOP{}  9.\DEM.\PROX{}  1\SG.\SM-\PRS{}-want-\FV{}\\
%%TRANS1
\glt 
‘(as for) Tea, it’s \textit{this} that I want.’\\
%%TRANS2
*‘Tea is what I want.’
%%EXEND

\z

It seems, therefore, that the initial NP forms a topic expression, and the demonstrative is in focus. Since both the NP and the demonstrative refer to the same referent, for example \textit{undumyana} and \textit{uju} in \xref{bkm:Ref98086853} refer to the same boy, this construction manages to simultaneously express a topical/given status as well as a focus function of that referent.

Summarising the discussion on the three types of clefts, Kinyakyusa shows an identificational basic cleft and pseudocleft; the difference between the two requires further study in spontaneous texts and discourse. A third focus construction is a reverse pseudocleft (copular construction NP = FR), which can alternatively be analysed as an initial NP followed by a basic cleft. In this latter construction, the clefted demonstrative refers to the same referent as the initial NP, thereby in some way allowing that referent to be both topical and focal at the same time.

\section{The CV exhaustive marker}
\label{bkm:Ref122533116}
Apart from the V augment, nouns in Kinyakyusa can also feature a CV prefix to the noun. This has been called the ``CV augment", but \citet{vanderWalLusekelo2022} show that the CV prefix behaves quite differently from the V augment and is in fact better analysed as an exhaustive marker. The following section is taken from \citet{vanderWalLusekelo2022} and shows the exhaustive interpretation of the CV prefix. We will here not go into the background and reconstruction of the augment but refer to \citet{VandeVelde2019} and \citet{HalpertFut} for general overviews.

\citet{MwangokaVoorhoeve1960} translate nouns with a CV marker in Kinyakyusa with ‘only’, and we confirm and consolidate their analysis. If the CV marker is present, the resulting interpretation is exhaustive focus on the noun, which may project to the larger phrase. That the focus encoded by the CV marker is not just simple/new information focus, but exhaustive focus can be proven by the following tests. 

  First, the context and co-text provided by the speakers indicate that alternatives must be present for the noun bearing the CV exhaustivity prefix. The spontaneous follow-up in \xref{bkm:Ref56943215} shows a contrast, and the failed attempt at an additive continuation in \xref{bkm:Ref56945748} shows that the alternatives must be excluded.

%%EAX
\ea
%%JUDGEMENT
%%LABEL
\label{bkm:Ref56943215}
%%CONTEXT
%%LINE1
Abakangale batwele \textbf{fi}fitwalo (indalama bakatwala).\\
%%LINE2
\gll
a-ba-kangale  ba-twal-ile  fi-fi-twalo  i-ndalama  ba-ka-twal-a\\
%%LINE3
\AUG{}-2-elder  2\SM{}-bring-\PFV{}  \EXH{}-8-luggage  \AUG{}-10.money  2\SM-\NEG{}-bring-\FV{}\\
%%TRANS1
\glt
‘The elders brought only the luggage. (They did not bring money.)’ \citep[336]{vanderWalLusekelo2022}\\
%%TRANS2
%%EXEND

\z
\pagebreak

%%EAX
\ea
%%JUDGEMENT
%%LABEL
\label{bkm:Ref56945748}
%%CONTEXT
%%LINE1
Anwile \textbf{ji}nywamu \textsuperscript{\#}n’iinandi.\\
%%LINE2
\gll
a-nu-ile  ji-nywamu  na  i-nandi\\
%%LINE3
1\SM{}-drink-\PFV{}  \EXH{}-9.big  and  \AUG{}-9.small\\
%%TRANS1
\glt
‘He drank (only) the big one \textsuperscript{\#}and also the small one.’ \citep[336]{vanderWalLusekelo2022}\\
%%TRANS2
%%EXEND

\z


The context for the in-situ doubling construction in \xref{bkm:Ref56943235}, where the infinitive takes a CV marker (see also \sectref{bkm:Ref123889421}), indicates exclusion of one of the supposed tasks as well. Additionally, the translation provided by the speakers frequently included Swahili \textit{tu} or English ‘only’.

%%EAX
\ea
%%JUDGEMENT
%%LABEL
\label{bkm:Ref56943235}
%%CONTEXT
(Context: He was supposed to cook and sweep.)\\
%%LINE1
Apiijile \textbf{ku}kupiija.\\
%%LINE2
\gll
a-piij-ile  ku-ku-piija\\
%%LINE3
1\SM{}-cook-\PFV{}  \EXH{}-15-cook\\
%%TRANS1
\glt
‘He only cooked.’ \citep[336]{vanderWalLusekelo2022}\\
%%TRANS2
%%EXEND

\z


A second argument showing the exhaustivity of the CV marker is the compatibility with the focus-sensitive particle ‘only’ (preceding or following the noun, though not both), as shown in \xref{bkm:Ref122591578} and \xref{bkm:Ref122591580}.

%%EAX
\ea
%%JUDGEMENT
%%LABEL
\label{bkm:Ref122591578}
%%CONTEXT
%%LINE1
Uulile (kyene) \textbf{ki}kitala (kyene).\\
%%LINE2
\gll
a-ul-ile  ki-ene  ki-ki-tala\\
%%LINE3
1\SM{}-buy-\PFV{}  7-only  \EXH{}-7-bed\\
%%TRANS1
\glt
‘S/he bought only the bed.’ \citep[336]{vanderWalLusekelo2022}\\
%%TRANS2
%%EXEND

\z


%%EAX
\ea
%%JUDGEMENT
%%LABEL
\label{bkm:Ref122591580}
%%CONTEXT
%%LINE1
Ampele (mwene) \textbf{ju}nnandi (mwene).\\
%%LINE2
\gll
a-m-p-ile  mu-ene  ju-n-nandi\\
%%LINE3
1\SM{}-1\OM{}-give-\PFV{}  1-only  \EXH{}-1-young\\
%%TRANS1
\glt
‘S/he has given (it) only to the young one.’ \citep[337]{vanderWalLusekelo2022}\\
%%TRANS2
%%EXEND

\z


In contrast, the CV exhaustive marker is incompatible with the scalar particle -\textit{ope} ‘even’ and the additive particle \textit{na} ‘also/even’ which are inclusive in nature, as shown in \xref{bkm:Ref57358115:a}, and \xref{bkm:Ref57358117}. Nevertheless, \xref{bkm:Ref57358115:b} was accepted, which we do not understand at present – a reviewer suggests that there could be a contrast between ‘together’ and ‘separately’ here.\pagebreak

\ea
\label{bkm:Ref57358115}
(Context: Robert does not like cabbage. He will eat any other thing. But this time he has even eaten cabbage.)\\
%%EAX
\ea
%%JUDGEMENT
[*]{
%%LABEL
\label{bkm:Ref57358115:a}
%%CONTEXT
%%LINE1
Lobati aliile na jikabiki (joope).\\
%%LINE2
\gll
Lobati  a-li-ile  na  ji-kabiki  ji-ope\\
%%LINE3
1.Robert  1\SM{}-eat-\PFV{}  and  \EXH{}-9.cabbage  9-even\\
%%TRANS1
\glt ‘Robert has even eaten cabbage.’
%%TRANS2
}
%%EXEND


%%EAX
\ex
%%JUDGEMENT
[]{
%%LABEL
\label{bkm:Ref57358115:b}
%%CONTEXT
%%LINE1
Aliile (ifindu) fyosa na jikabiki kolumo.\\
%%LINE2
\gll
a-li-ile  i-fi-ndu  fi-osa  na  ji-kabiki  kolumo\\
%%LINE3
1\SM{}-eat-\PFV{}  \AUG{}-8-food  8-all  and  \EXH{}-9.cabbage  together\\
%%TRANS1
\glt
‘He has eaten all (types of) food, even cabbage.’ \citep[337]{vanderWalLusekelo2022}\\
%%TRANS2
}
%%EXEND

\z
\z

%%EAX
\ea
%%JUDGEMENT
%%LABEL
\label{bkm:Ref57358117}
%%CONTEXT
%%LINE1
(*Boope) Babaana baliile.\\
%%LINE2
\gll
ba-ope  ba-ba-ana  ba-li-ile\\
%%LINE3
2-even  \EXH{}-2-child  2\SM{}-eat-\PFV{}\\
%%TRANS1
\glt
‘(*Even) Only the children have eaten.’ \citep[337]{vanderWalLusekelo2022}\\
%%TRANS2
%%EXEND

\z


Third, the CV marker is not accepted with universal quantifiers like ‘every’ and ‘all’, again because no alternatives can be excluded – see \xref{bkm:Ref56944146:a}. However, exclusion of alternatives becomes possible when subsets can be created using a restrictive relative clause, as in \xref{bkm:Ref56944146:b}, or if the whole set is contrasted to another set, as in \xref{bkm:Ref56944170}. These examples therefore allow the presence of a CV marker.

\ea
\label{bkm:Ref56944146}
%%EAX
\ea
%%JUDGEMENT
[*]{
%%LABEL
\label{bkm:Ref56944146:a}
%%CONTEXT
%%LINE1
Ipyana aagonjile \textbf{fi}fisyesye fyosa.\\
%%LINE2
\gll
Ipyana  a-a-gonj-ile  fi-fi-syesye  fi-osa\\
%%LINE3
1.Ipyana  1\SM{}-\PST{}-taste-\PFV{}  \EXH{}-8-baked.good  8-all\\
%%TRANS1
\glt
int. ‘Ipyana tasted only all cakes.’\\
%%TRANS2
}
%%EXEND

%%EAX
\ex
%%JUDGEMENT
[]{
%%LABEL
\label{bkm:Ref56944146:b} 
%%CONTEXT
%%LINE1
Ipyana aagonjile fifisyesye (fyosa) ifi atendekiisye unna.\\
%%LINE2
\gll
Ipyana  a-a-gonj-ile  fi-fi-syesye  fi-osa ifi  a-tendekesy-ile  u-n-na \\
%%LINE3
1.Ipyana  1\SM{}-\PST{}-taste-\PFV{}  \EXH{}-8-baked.good  8-all 8.\DEM{}.\PROX{}  1\SM{}-bake-\PFV{}   \AUG{}-1-mother\\
%%TRANS1
\glt ‘Ipyana tasted only all the cakes that her mother baked (but did not taste any other cakes).’ \citep[337--338]{vanderWalLusekelo2022}
%%TRANS2
}
%%EXEND

\z
\z
\pagebreak

%%EAX
\ea
%%JUDGEMENT
%%LABEL
\label{bkm:Ref76978930}
%%CONTEXT
%%LINE1
\label{bkm:Ref56944170}Babandu boosa bikutuuja.\\
%%LINE2
\gll
ba-ba-ndu  ba-osa  ba-ku-tuuj-a\\
%%LINE3
\EXH{}-2-person  2-all  2\SM{}-\PRS{}-breathe-\FV{}\\
%%TRANS1
\glt
\textsuperscript{\#}`All people breathe.’\\
%%TRANS2
%%EXEND

‘Only all humans breathe.’ (follow-up reaction: ‘But cows breathe too!’) \citep[338]{vanderWalLusekelo2022}\\
\z

Fourth, the CV marker is not accepted with non-specific indefinites, as here too there are no alternatives that can be excluded. In \xref{bkm:Ref76978930}, instead the word \textit{umundu} ‘person’, which could otherwise be interpreted as ‘someone’ must here be interpreted as a generic ‘human being’. Under our hypothesis, the CV marker necessarily triggers and excludes alternatives, which is only possible if \textit{umundu} is interpreted as generic (excluding other species) and not if it is interpreted as indefinite non-specific (including anyone and everyone).

%%EAX
\ea
%%JUDGEMENT
%%LABEL
%%CONTEXT
(Context: You visit a national park, expecting to see trees and different animals, but instead…)\\
%%LINE1
Numbwene \textbf{ju}mundu.\\
%%LINE2
\gll
n-m-bon-ile  ju-mu-ndu\\
%%LINE3
1\SG.\SM{}-1\OM{}-see-\PFV{}  \EXH{}-1-person\\
%%TRANS1
\glt
‘I saw only a human/person.’\\
%%TRANS2
*`I saw someone.’ \citep[338]{vanderWalLusekelo2022}

%%EXEND

\z


Fifth, idioms and cognate objects are “unfocussable” as they have no referential meaning and therefore cannot trigger alternatives. We thus predict them to be incompatible with the CV marker. At first sight, the acceptance of \xxref{bkm:Ref56944486}{bkm:Ref56944487} seems to contradict this prediction, because the idiomatic object can take a CV marker. However, if we look at the context, we see that a contrast is indicated with other \textit{actions} and not with other \textit{objects}. This means that the given sentences are interpreted with the exclusion on the level of the verb phrase, and the set of alternatives is being formed for the whole idiom in the case of \xref{bkm:Ref114666832}, and the whole action in the case of the cognate objects in \xref{bkm:Ref114666844} and \xref{bkm:Ref114666853}, and not just the object.\largerpage

%%EAX
\ea
%%JUDGEMENT
%%LABEL
\label{bkm:Ref114666832}
%%CONTEXT
(Context: As soon as he gets up in the morning, he drinks, and straight from work he goes to the bar.\label{bkm:Ref56944486})\\
%%LINE1
Ikukoma \textbf{ga}miisi.\\
%%LINE2
\gll
a-ku-kom-a  ga-ma-isi\\
%%LINE3
1\SM{}-\PRS{}-hit-\FV{}  \EXH{}-6-water\\
%%TRANS1
\glt
‘He is only hitting water.’\\
%%TRANS2
‘He is only getting drunk.’ \citep[338]{vanderWalLusekelo2022}

%%EXEND

\z


%%EAX
\ea
%%JUDGEMENT
%%LABEL
\label{bkm:Ref114666844}
%%CONTEXT
(Context: The calves stay at home and need to be fed, and the larger cattle are taken out to graze. Gwamaka is not interested in feeding the cows at home, he only goes out to do the herding.)\\
%%LINE1
Gwamaka ikutiima \textbf{gu}ntiimo.\\
%%LINE2
\gll
Gwamaka  a-ku-tiim-a  gu-n-tiimo\\
%%LINE3
1.Gwamaka  1\SM{}-\PRS{}-graze-\FV{}  \EXH{}-3-grazing\\
%%TRANS1
\glt
‘Gwamaka only grazes (a/the graze).’ \citep[338]{vanderWalLusekelo2022}\\
%%TRANS2
%%EXEND


%%EAX
\ex
%%JUDGEMENT
%%LABEL
\label{bkm:Ref114666853}
%%CONTEXT
(Context: Why are you being so quiet?\label{bkm:Ref56944487})\\
%%LINE1
Ngwinogona \textbf{si}nyinogono.\\
%%LINE2
\gll
n-ku-inogon-a  si-nyinogono\\
%%LINE3
1\SG.\SM{}-\PRS{}-think-\FV{}  \EXH{}-10.thought\\
%%TRANS1
\glt
‘I’m only thinking thoughts.’ \citep[339]{vanderWalLusekelo2022}\\
%%TRANS2
%%EXEND

\z


A sixth test involves the focussing of a numeral. As explained in the introduction to this volume, numerals lose their upward-entailing quality in exhaustive focus and refer only to the exact quantity, because other amounts are excluded. In Kinyakyusa, a numeral in a DP with a CV marker is interpreted as the exact amount, as illustrated by the infelicity of the follow-up ‘maybe more’ in \xref{bkm:Ref76979279} and \xref{bkm:Ref56945623:a}. This constitutes evidence for the exhaustive interpretation, especially when compared to the use with the V augment in \xref{bkm:Ref56945623:b}, where a continuation ‘maybe more’ is felicitous (but also notice that \xref{bkm:Ref56945623:a} uses the reverse pseudocleft construction, whereas \xref{bkm:Ref56945623:b} does not).\largerpage[2.25]

%%EAX
\ea
%%JUDGEMENT
%%LABEL
\label{bkm:Ref144796821}
\label{bkm:Ref76979279}
%%CONTEXT
%%LINE1
Bahati ikukaba \textbf{ji}-milioni jimo kukyinja.\\
%%LINE2
\gll
Bahati  a-ku-kab-a  ji-milioni  ji-mo  ku-ki-inja\\
%%LINE3
1.Bahati  1\SM{}-\PRS{}-get-\FV{}  \EXH{}-9.million  9-one  17-7-year\\
%%TRANS1
\glt
‘Bahati earns (exactly) one million a year.’ \citep[339]{vanderWalLusekelo2022}\\
%%TRANS2
%%EXEND


\ex
\label{bkm:Ref56945623}
%%EAX
\ea
%%JUDGEMENT
%%LABEL
\label{bkm:Ref56945623:a}
%%CONTEXT
%%LINE1
\textbf{Si}nguku ntandatu syo isi syalyulisigwe (\textsuperscript{\#}pamo n’ iisiingi).\\
%%LINE2
\gll
si-n-guku  ntandatu  si-o  isi  si-ali-ul-is-igw-e pamo  na  i-si-ngi\\
%%LINE3
\EXH{}-10-chicken  10.six  10-\IDCOP{}  10.\DEM{}.\PROX{}  10\SM{}-\PST{}-buy-\CAUS-\PASS{}-\PFV{} maybe  and  \AUG{}-10-other\\
%%TRANS1
\glt ‘It’s six chickens exactly that were sold (\textsuperscript{\#}maybe more).’
%%TRANS2
%%EXEND

%%EAX
\ex
%%JUDGEMENT
%%LABEL
\label{bkm:Ref56945623:b}
%%CONTEXT
%%LINE1
Inguku ntandatu syalyulisiigwe (pamo n’ iisingi).\\
%%LINE2
\gll
i-n-guku  ntandatu  si-ali-ul-is-igw-e  pamo  na  i-si-ngi\\
%%LINE3
\AUG{}-10-chicken  10.six  10\SM{}-\PST{}-buy-\CAUS-\PASS{}-\FV{}  maybe  and  \AUG{}-10-other\\
%%TRANS1
\glt
‘Six chickens were sold (maybe more).’ \citep[339]{vanderWalLusekelo2022}\\
%%TRANS2
%%EXEND

\z
\z

Seventh, negation targets the exhaustivity (rather than the truth) of the sentence when the CV marker is present on the object. That is, \xref{bkm:Ref76979472} does not deny that they drank soda, but rather negates that it was \textit{only} soda that they drank. The fact that a grammatical operation like negation can target the exhaustivity also shows that exhaustivity is an inherent aspect of the meaning of the CV marker, and not a mere pragmatic implication.

%%EAX
\ea
%%JUDGEMENT
%%LABEL
\label{bkm:Ref76979472}
%%CONTEXT
%%LINE1
Bakanwile \textbf{si}sooda (baaliile/baanwile n’ ifingi).\\
%%LINE2
\gll
ba-ka-nu-ile   si-sooda  ba-a-li-ile  /ba-a-nu-ile  na  i-fi-ngi\\
%%LINE3
2\SM{}-\NEG{}-drink-\PFV{}  \EXH{}-10.soda  2\SM{}-\PST{}-eat-\PFV{}  /2\SM{}-\PST{}-drink-\PFV{}  and  \AUG{}-8-other\\
%%TRANS1
\glt
‘They didn’t drink only soda (they also drank other things).’ \citep[340]{vanderWalLusekelo2022}\\
%%TRANS2
%%EXEND

\z


Finally, the corrective answer to an incomplete yes/no question with the CV marker needs to be ‘no’ and cannot be ‘yes’ – compare to the same test in \xref{bkm:Ref136517649} above. This negation in the answer targets the exhaustivity encoded by the CV marker in the question, and can be compared to the felicitous answer ‘yes’ to an equally incomplete question with the V augment in \xref{bkm:Ref56946442}. The question in both cases asks about a subset of the true answers (only shirts, where other things have been washed too), making the predicate true for this subset (he did wash the shirts, after all), but making the exhaustivity false (he did not wash only the shirts).

\ea
\label{bkm:Ref122709209}
(Context: Speakers are shown a picture of a clothesline with various washed sheets and clothes, including shirts.)\\
%%EAX
\ea
%%JUDGEMENT
%%LABEL
%%CONTEXT
%%LINE1
Bule Mose asukile \textbf{si}syati?\\
%%LINE2
\gll
bule  Mose  a-suk-ile  si-syati\\
%%LINE3
Q  1.Moses  1\SM{}-wash-\PFV{}  \EXH{}-10.shirt\\
%%TRANS1
\glt
‘Did Moses wash only shirts?’\\
%%TRANS2
%%EXEND

%%EAX
\ex
%%JUDGEMENT
%%LABEL
%%CONTEXT
%%LINE1
Mma/\textsuperscript{\#}eena, Mose asukile isyati n’ imyenda igingi.\\
%%LINE2
\gll
mma/eena  Mose  a-suk-ile  i-syati   na  i-mi-enda  i-gi-ngi.\\
%%LINE3
no/yes  1.Moses  1\SM{}-wash-\PFV{}  \AUG{}-10.shirt  and  \AUG{}-4-clothes  \AUG{}-4-other\\
%%TRANS1
\glt
‘No/\textsuperscript{\#}Yes. Moses washed shirts and other clothes.’ \citep[340]{vanderWalLusekelo2022}\\
%%TRANS2
%%EXEND

\z
\z

\ea
\label{bkm:Ref56946442}\label{bkm:Ref122709211}
(Context: Speakers are shown a picture of a clothesline with various washed sheets and clothes, including shirts.)\\
%%EAX
\ea
%%JUDGEMENT
%%LABEL
%%CONTEXT
%%LINE1
Bule Mose asukile isyati?\\
%%LINE2
\gll
bule  Mose  a-suk-ile  i-syati  \\
%%LINE3
Q  1.Moses  1\SM{}-wash-\PFV{}  \AUG{}-10.shirt\\
%%TRANS1
\glt ‘Did Moses wash shirts?’
%%TRANS2
%%EXEND

%%EAX
\ex
%%JUDGEMENT
%%LABEL
%%CONTEXT
%%LINE1
Eena Mose asukile isyati pa-li-kimo n’ amagolole.\\
%%LINE2
\gll
eena  Mose  a-suk-ile  i-syati  pa-li-kimo  na  a-ma-golole\\
%%LINE3
yes  1.Moses  1\SM{}-wash-\PFV{}  \AUG{}-10.shirt  16-be-one  with  \AUG{}-6-sheet\\
%%TRANS1
\glt
‘Yes. Moses washed shirts together with sheets.’ \citep[340]{vanderWalLusekelo2022}\\
%%TRANS2
%%EXEND

\z
\z

In summary, there is overwhelming evidence that exhaustivity is inherent to the CV marker. \Citet{vanderWalLusekelo2022} therefore propose that it should be analysed as an exhaustive marker; we refer to that paper for further details on the formal and interpretational properties of the marker.

\section{Conclusion}\largerpage

Four concluding remarks can be summarized for this chapter. First, Kinyakyusa has no dedicated position for focus postverbally, nor does it have a dedicated position for topic preverbally. Unlike many eastern Bantu languages, Kinyakyusa allows preverbal focus \citep[see][]{KerrEtAl2023}, and it features only Agreeing Inversion as a productive subject inversion construction \citep[see][]{MsovelaEtAl2023}.  Second, the pronominal expression \textit{po} functions as a topic marker in the language. Third, as in other languages, cleft constructions are employed to express focus in the language. The basic cleft and pseudocleft involve identificational focus, and a construction with an initial NP and relative clause is shown to have two possible underlying structures: either a reverse pseudocleft or a topical initial NP followed by a basic cleft in which the demonstrative coreferring to the same referent is focused. Lastly, the CV prefix is shown to function as an exhaustive marker \citep[see][]{vanderWalLusekelo2022} – it selects a noun out of the available alternatives and excludes those alternatives as false. This chapter forms the first overview of the morphosyntactic strategies used in Kinyakyusa to express information structure. While it is obviously quite incomplete, and much remains to be discovered, we hope that it may inspire further research on Kinyakyusa, as well as other Bantu languages.

\section*{Acknowledgements}

This research was supported by NWO Vidi grant 276-78-001 as part of the BaSIS “Bantu Syntax and Information Structure” project at Leiden University. We thank Bahati Laikon Mwakasege, Peter Mwasyika Mwaipyana, and Yona Mwaipaja for sharing their insights on their language with us. We also thank the two reviewers for their comments, and we thank our BaSIS colleagues for their support and discussions. Any remaining errors are ours alone.

\section*{Abbreviations and symbols}

Numbers refer to noun classes unless followed by \SG{}/\PL{}, in which case the number refers to first or second person. 

\begin{multicols}{2}
\begin{tabbing}
MMMM \= ungrammatical\kill
%%% All Leipzig abbreviations are commented out, following the LangSci guidelines of only listing non-Leipzig abbreviations.
* \> ungrammatical\\
\textsuperscript{\#} \> infelicitous in the given \\ \> context\\
*(X) \> the presence of X is obligatory \\ \> and cannot grammatically be \\ \> omitted\\
(*X) \> the presence of X would make \\ \> the sentence ungrammatical\\
(X) \> the presence of X is optional\\
AI \> agreeing inversion\\
% \APPL{} \> applicative\\
\AUG{} \> augment\\
% \CAUS{} \> causative\\
% \COMP{} \> complementiser\\
% \COND{} \> conditional\\
\CONN{} \> connective\\
% \COP{} \> copula \\
% \DEM{} \>  demonstrative\\
% \DIST{} \> distal (demonstrative)\\
\EXH{} \> exhaustive marker\\
\EXCLAM{} \> exclamative \\
% \FUT{} \> future \\
\FV{} \>  final vowel\\
FR \>  free relative\\
\IDCOP{} \> identificational copula\\
% \IMP{} \>  imperative\\
int. \> intended\\
\INT{} \>  intensifier\\
\HAB{} \> habitual\\
\MED{} \> medial (demonstrative) \\
% \NEG{} \> negative\\
N \> place-assimilating nasal\\
NP \> noun phrase\\
\OM{} \> object marker \\
% \PASS{} \> passive\\
\PEJ{} \> pejorative\\
% \PFV{} \> perfective \\
% \PL{} \> plural\\
% \POSS{} \> possessive \\
\PRO{} \> pronoun \\
% \PROH{} \> prohibitive\\
% \PROX{} \> proximal (demonstrative) \\
% \PRS{} \> present\\
% \PST{} \> past \\
% Q \> question marker\\
QUIS \> questionnaire on information \\ \> structure \citep{SkopeteasEtAl2006}\\
% \SBJV{} \> subjunctive \\
% \SG{} \> singular \\
\SM{} \> subject marker\\
\STAT{} \> stative\\
Sw. \> Swahili
\end{tabbing}
\end{multicols}

\printbibliography[heading=subbibliography,notkeyword=this]
\end{document}
