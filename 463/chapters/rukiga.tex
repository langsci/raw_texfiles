\documentclass[output=paper]{langscibook}
\author{Allen Asiimwe\orcid{}\affiliation{Makerere University} and Jenneke van der Wal\orcid{}\affiliation{Leiden University}}
\ChapterDOI{10.5281/zenodo.14833614}
\title{The expression of information structure in Rukiga}
\abstract{This chapter offers a systematic descriptive analysis of the various strategies the language uses to express information structure in Rukiga. Notably, word order is determined primarily by discourse roles, the augment on modifiers encodes a restrictive reading, and predicate doubling is associated with a contrastive interpretation among other multiple readings. The particle~-\textit{o}~functions as a contrastive topic marker, which is often used in topic doubling constructions to reinforce a given interpretation. Although there is a tonal remainder of the conjoint/disjoint alternation, in Rukiga, it does not directly influence information structure. Rukiga also uses three types of clefts, a common strategy used to express focus as in many other languages of the world.}
\IfFileExists{../localcommands.tex}{
  \addbibresource{../localbibliography.bib}
  \usepackage{langsci-optional}
\usepackage{langsci-gb4e}
\usepackage{langsci-lgr}

\usepackage{listings}
\lstset{basicstyle=\ttfamily,tabsize=2,breaklines=true}

%added by author
% \usepackage{tipa}
\usepackage{multirow}
\graphicspath{{figures/}}
\usepackage{langsci-branding}

  
\newcommand{\sent}{\enumsentence}
\newcommand{\sents}{\eenumsentence}
\let\citeasnoun\citet

\renewcommand{\lsCoverTitleFont}[1]{\sffamily\addfontfeatures{Scale=MatchUppercase}\fontsize{44pt}{16mm}\selectfont #1}
   
  %% hyphenation points for line breaks
%% Normally, automatic hyphenation in LaTeX is very good
%% If a word is mis-hyphenated, add it to this file
%%
%% add information to TeX file before \begin{document} with:
%% %% hyphenation points for line breaks
%% Normally, automatic hyphenation in LaTeX is very good
%% If a word is mis-hyphenated, add it to this file
%%
%% add information to TeX file before \begin{document} with:
%% %% hyphenation points for line breaks
%% Normally, automatic hyphenation in LaTeX is very good
%% If a word is mis-hyphenated, add it to this file
%%
%% add information to TeX file before \begin{document} with:
%% \include{localhyphenation}
\hyphenation{
affri-ca-te
affri-ca-tes
an-no-tated
com-ple-ments
com-po-si-tio-na-li-ty
non-com-po-si-tio-na-li-ty
Gon-zá-lez
out-side
Ri-chárd
se-man-tics
STREU-SLE
Tie-de-mann
}
\hyphenation{
affri-ca-te
affri-ca-tes
an-no-tated
com-ple-ments
com-po-si-tio-na-li-ty
non-com-po-si-tio-na-li-ty
Gon-zá-lez
out-side
Ri-chárd
se-man-tics
STREU-SLE
Tie-de-mann
}
\hyphenation{
affri-ca-te
affri-ca-tes
an-no-tated
com-ple-ments
com-po-si-tio-na-li-ty
non-com-po-si-tio-na-li-ty
Gon-zá-lez
out-side
Ri-chárd
se-man-tics
STREU-SLE
Tie-de-mann
} 
  \togglepaper[6]%%chapternumber
}{}

\begin{document}
\maketitle 
\label{ch:6}
%\shorttitlerunninghead{}%%use this for an abridged title in the page headers





\section{Introduction}\label{sec:intro}

This chapter gives a general overview of the expression of information structure in Rukiga. Rukiga is a Bantu language (Guthrie classification JE14, ISO code [cgg]) of the Nyoro-Ganda group, spoken predominantly in South-Western Uganda by approximately 2.3m people \citep{U.B.O.S.2016}. Rukiga is closely related to Runyankore with a lexical similarity of up to 94\% \citep{EberhardEtAl2020}. Because of this high mutual intelligibility and a high level of lexical and grammatical affinity, the two languages are often clustered and studied together as one language (e.g. \citealt{Taylor1985,Turamyomwe2011,Asiimwe2014,Ndoleriire2020}, among others). Together the two languages form the language cluster: Runyankore-Rukiga. Some studies on (Runyankore-)Rukiga exist. Two descriptive grammars are available, namely \citet{MorrisKirwan1972} and \citet{Taylor1985}. Various other studies on Runyankore-Rukiga have been carried out, including ones on tense and aspect \citep{Turamyomwe2011,Asiimwe2024b}, definiteness and specificity \citep{Asiimwe2014}, and the syntax of relative clause constructions \citep{Asiimwe2019}. Two articles on aspects of information structure in Rukiga have been published within the Bantu Syntax and Information Structure (BaSIS) project: \citet{vanderWalAsiimwe2020} on the conjoint/disjoint alternation and \citet{AsiimwevanderWal2021} on the contrastive marker \textit{-o}. This chapter presents the first detailed (descriptive) study of information structure in Rukiga. It examines different strategies the language employs to express the various categories of information structure.

Data for this chapter are based on the Runyaifo variety largely spoken in Ndorwa county in Kabale district. Other dominant varieties of Rukiga include Rusigi, Ruhimba and Runyangyenzi. The rest of the dialects share a common grammar, and are quite distinct from Runyaifo. As part of the BaSIS research, data were collected during the month of January 2019 using the BaSIS project methodology, available through the Leiden Repository. Data were mainly collected through elicitation with three native speakers of Rukiga. In addition, data from natural speech in the form of narratives and recipes were also collected through interactions with the three native speakers. Additional data come from the first author who is a native speaker of Rukiga, and were checked by the three native speakers that participated in the elicitation sessions. The data were transcribed and stored in an Online Language Database accessible through the Dative user interface that allows data sharing in a collaborative research. This database will be accessible through The Language Archive. More information about Dative can be accessed via \url{https://www.dative.ca} (and see the introduction to this book). We also refer to the introduction to this book for further background on the terms and diagnostics used for information structure.

The chapter is structured as follows: \sectref{sec:cjdj-residue} disusses the tonal conjoint\slash disjoint alternation in Rukiga. \sectref{sec:wordorder} gives an extensive discussion on word order and notes that word order in Rukiga is influenced mainly by discourse. \sectref{sec:particle-o} then looks at particle \mbox{-\textit{o}} as a marker of contrastive topics, which also occurs in  predicate doubling constructions, as discussed in \sectref{sec:preddoubling}. This section observes that predicate doubling contructions are associated with multiple readings and these readings are context-dependent. In \sectref{sec:augment}, we discuss the optional augment on nominal modifiers in Rukiga, which is shown to express restrictiveness. \sectref{sec:cleft-constructions} is concerned with the role of cleft constructions in the expression of focus and \sectref{sec:objmarking} examines object marking in Rukiga with a focus on the role of pragmatic object doubling. A summary of the chapter is given in \sectref{sec:summary}.

\section{Conjoint/disjoint residue}\label{sec:cjdj-residue}

Some eastern and southern Bantu languages show a segmental morphological alternation in verbal conjugations known as the conjoint\slash disjoint alternation (see \chapref{ch:5}, \cite{chapters/kirundi} on Kirundi and \chapref{ch:8}, \cite{chapters/makhuwa} on Makhuwa). This alternation can function as a marker of focus, as illustrated for Kimatuumbi in \xref{bkm:Ref135644345}, where the disjoint verb form is marked by \textit{eenda} whereas the conjoint form is unmarked. Using the disjoint form results in focus on the predicate, whereas the conjoint form indicates focus on the constituent following the verb (see \citealt{vanderWal2017, vanderWalFut}, for an overview of the conjoint/disjoint alternation across Bantu). Crucially, the conjoint verb form cannot appear clause-finally, whereas the disjoint form can – this is consistent across the Bantu languages.

\ea
\label{bkm:Ref135644345}Kimatuumbi (P13, \citealt[60--61]{Odden1996}, glosses added)
\ea
%%LINE1
\begin{xlist}
%%EAX
\exi{\CJ{}}
%%JUDGEMENT
%%LABEL
%%CONTEXT
%%LINE2
\gll
Ni-kat-a  *(kaámba).\\
%%LINE3
1\SG{}.\SM{}-cut-\FV{}  rope\\
%%TRANS1
\glt
‘I am cutting \textit{rope} (not something else).’\\
%%TRANS2
%%EXEND
\end{xlist}


\ex
%%LINE1
\begin{xlist}
%%EAX
\exi{\DJ{}}
%%JUDGEMENT
%%LABEL
%%CONTEXT
%%LINE2
\gll
Eendá-kaat-á.\\
%%LINE3
1\SG{}.\SM{}.\PROG{}.\DJ{}-cut-\FV{}\\
%%TRANS1
\glt
‘He is cutting.’\\
%%TRANS2
%%EXEND
\end{xlist}


\ex
%%LINE1
\begin{xlist}
%%EAX
\exi{\DJ{}}
%%JUDGEMENT
%%LABEL
%%CONTEXT
%%LINE2
\gll
Eendá-kaat-á  kaámba.\\
%%LINE3
1\SG{}.\SM{}.\PROG{}.\DJ{}-cut-\FV{}  rope\\
%%TRANS1
\glt
‘He is \textit{cutting} rope (not doing something else to it).’\\
%%TRANS2
%%EXEND
\end{xlist}


\z
\z

Rukiga shows only a tonal residue of the alternation, as we argue in \citet{vanderWalAsiimwe2020}. Rukiga is the first Bantu language for which a purely tonal alternation has been described – all other languages that are known to have the alternation mark it by segmental morphology in at least one conjugation. 

In Rukiga, the tone of the verb is in some tenses affected by a process of tonal reduction (TR), as earlier described for Haya \citep{Hyman1999}. \Citet{vanderWalAsiimwe2020} show how tonal reduction applies to the verb when it is not clause-final. Compare the tonal pattern of the clause-final verb in \xref{bkm:Ref99000590:a} with high tones on the TAM marker and verb stem, with that in \xref{bkm:Ref99000590:b} where the verb is not final and surfaces with only a suffixal high tone on the final vowel. The tone of the following constituent remains unaffected, as far as we could see.

\ea
\label{bkm:Ref99000590}
%%EAX
\ea
\begin{xlist}[TR]
\exi{~}
%%JUDGEMENT
%%LABEL
\label{bkm:Ref99000590:a}
%%CONTEXT
%%LINE1
%%LINE2
\gll
María  y-áá-híing-a.\\
%%LINE3
1.Maria  1\SM{}-\N{}.\PST{}-dig-\FV{}\\
%%TRANS1
\glt
‘Maria has dug.’\\
%%TRANS2
\end{xlist}
%%EXEND

\ex
\label{bkm:Ref99000590:b}
%%EAX
\begin{xlist}[TR]
\exi{TR}
%%JUDGEMENT
%%LABEL
%%CONTEXT
%%LINE1
%%LINE2
\gll
María  y-aa-hing-á  o-mu-siri.{\footnotemark} \\
%%LINE3
1.Maria  1\SM{}-\N{}.\PST{}-dig-\FV{}  \AUG{}-3-field\\
\footnotetext{In natural speech, this is pronounced with liaison as \textit{yaahing’ ómusiri}, and the final H appears on the augment of the object.}
%%TRANS1
\glt
‘Maria has dug the field.’\\
%%TRANS2
\citep[44]{vanderWalAsiimwe2020}
\end{xlist}
%%EXEND

\z
\z


Tonal reduction can still be seen as marking the conjoint/disjoint alternation here, because it is not an automatic process, but is restricted to a subset of tenses~-- just as is the case for the conjoint/disjoint alternation in other languages. If TR were a tonal process applying as a general rule, we would expect it to apply across the board to all sequences of verb and following element. Instead, only the present/habitual, yesterday past, remote past, and near past conjugations in Rukiga show tonal reduction when the verb is not clause-final. For table overviews and details on the tonal behaviour of verb and object we refer to \citet{vanderWalAsiimwe2020}.

Given the sentence-final restriction of TR being the same as that of the conjoint/disjoint alternation, we investigated whether TR has an effect on information structure as well, as is the case for other Bantu languages with the alternation \citep{vanderWal2017}, but in Rukiga the only determining factor for the form of the verb is its appearance in final position: Tonally reduced verb forms cannot appear in final position in a main clause, as illustrated for the present habitual and the yesterday past in \xref{bkm:Ref99002375}. 

\ea
\label{bkm:Ref99002375}
%%EAX
\ea
%%JUDGEMENT
%%LABEL
%%CONTEXT
%%LINE1
%%LINE2
\gll
A-b-áana  ba-záan-a / *ba-zaan-a.    \\
%%LINE3
\AUG{}-2{}-children  2\SM{}-play-\FV{}\\ \jambox*{[present habitual]}
%%TRANS1
\glt
‘Children play.’\\
%%TRANS2
%%EXEND

%%EAX
\ex
%%JUDGEMENT
%%LABEL
%%CONTEXT
%%LINE1
Ekikópo, Hélen akitwííre / *akitwiiré.  \jambox*{[yesterday past]}
%%LINE2
\gll
e-ki-kopo  Helen  a-ki-twar-ire\\
%%LINE3
\AUG{}-7-cup  1.Hellen  1\SM{}-7\OM{}-take-\PFV{}\\
%%TRANS1
\glt
‘The cup, Hellen took it.’\\
%%TRANS2
 \citep[48]{vanderWalAsiimwe2020}
%%EXEND

\z
\z


Even when the verb is in focus (a typical environment for the disjoint/non-reduced form), as in \xref{bkm:Ref99002467} and \xref{bkm:Ref99002400}, clause-finality determines the form of the verb in Rukiga: final = no TR, as shown in \xref{bkm:Ref99002467}; non-final = no TR, as shown in \xref{bkm:Ref99002400:a} – the non-reduced form is not acceptable, as shown in \xref{bkm:Ref99002400:b}.

%%EAX
\ea
%%JUDGEMENT
%%LABEL
\label{bkm:Ref99002467}
%%CONTEXT
%%LINE1
%%LINE2
\gll
Tí-ba-a-karang’  é-bi-nyôbwa,  bá-á-bi-shékur-a.\\
%%LINE3
\NEG{}-2\SM{}-\N{}.\PST{}-roast  \AUG{}-8-groundnuts  2\SM{}-\N{}.\PST{}-8\OM{}-pound-\FV{}\\
%%TRANS1
\glt
‘They didn’t roast the groundnuts, they pounded them.’\\
%%TRANS2
\citep[51]{vanderWalAsiimwe2020}
%%EXEND

\z

\ea
\label{bkm:Ref99002400}
\ea
\label{bkm:Ref99002400:a}
%%LINE1
\begin{xlist}[TR]
%%EAX
\exi{TR}
%%JUDGEMENT
[]{
%%LABEL
%%CONTEXT
%%LINE2
\gll
E-nyonyi  tí-z-aa-tambur-a  júba  kwonká  z-\textbf{aa}-guruk-a  júba.\\
%%LINE3
\AUG{}-10.birds  \NEG{}-10\SM{}-\N{}.\PST{}-walk-\FV{}  quickly  but  10\SM{}-\N{}.\PST{}-fly-\FV{}  quickly\\
%%TRANS1
\glt
‘The birds have not walked quickly, they have flown quickly.’\\
%%TRANS2
}
%%EXEND
\end{xlist}


\ex
\begin{xlist}[TR]
\exi{~}
[*]{
\label{bkm:Ref99002400:b}
Enyonyi tízatambura júba konká z\textbf{áá}guruka júba.  \\
}
\end{xlist}
\z
\z

There is no correlation between the absence of TR and verb focus, and neither do we find a correlation between the presence of TR and focus following the verb. This can be shown by placing an idiomatic object in postverbal position: since idiomatic objects can only be interpreted together with the verb, generating alternatives for a focused object results in a loss of the idiomatic meaning \citep{vanderWal2021}. Therefore,  if TR on the preceding verb would induce focus, only the literal meaning should remain, not the idiomatic one. Example \xref{bkm:Ref99002946} shows that the idiomatic reading is present, and that the TR form is required (because the verb is not final).

\ea
\label{bkm:Ref99002946}
\ea
\begin{xlist}[TR]
%%EAX
\exi{TR}
%%JUDGEMENT
[]{
%%LABEL
%%CONTEXT
%%LINE1
N\textbf{aaye}yaguz’ órugusyo.\\
%%LINE2
\gll
n-aa-e-yaguz-a  o-ru-gusyo\\
%%LINE3
1\SG{}.\SM{}-\N{}.\PST{}-\REFL{}-scratch.\CAUS{}-\FV{}  \AUG{}-11-shard\\
%%TRANS1
\glt
‘I was in a bad situation.’\\
%%TRANS2
      lit. ‘I scratched myself with a shard.’
}
%%EXEND

\end{xlist}

\ex
\begin{xlist}[TR]
\exi{~}
[*]{
N\textbf{ááyé}yaguz’ orugúsyó.
}
\end{xlist}
\z
\z


TR equally applies in default agreement inversion (see \sectref{bkm:Ref116291412}), illustrated in \xref{bkm:Ref99003032}, where the interpretation is thetic.

\ea
\label{bkm:Ref99003032}
\begin{xlist}[TR]
%%EAX
\exi{TR}
%%JUDGEMENT
%%LABEL
%%CONTEXT
%%LINE1
Hiij’ ómuntu.\\
%%LINE2
\gll
ha-aa-ij-a  o-mu-ntu\\
%%LINE3
16\SM{}-\N{}.\PST{}-come-\FV{}  \AUG{}-1-person\\
%%TRANS1
\glt
‘Someone has come.’\\
%%TRANS2
%%EXEND

\end{xlist}


\z


\Citet[56]{vanderWalAsiimwe2020} conclude: 
\begin{quote}
    [...] that there is never a true minimal choice between applying TR or not, that is, there is no alternation depending on information structure, but rather a tonal rule that is sensitive to (some) constituency boundaries. There is no direct tonal marking of focus \citep[see][]{Hyman1999}. The options available to the speaker are to phrase a postverbal element within or outside of the same constituent as the verb, and the form of the verb follows automatically. \citep[56]{vanderWalAsiimwe2020}
\end{quote}

We refer to \citet{vanderWalAsiimwe2020} for more examples and a detailed exposition and argumentation of this tonal residue of the conjoint/disjoint alternation. 

\section{Word order}\label{sec:wordorder}

Word order in Rukiga is partly determined by information structure and therefore shows more flexibility than a characterisation as SVO can do justice to, as was observed already for many other Bantu languages (e.g. \citealt{Morimoto2000,Zerbian2006,vanderWal2009a,Yoneda2011,BostoenMundeke2012,KerrEtAl2023}, and others). Hence, word order can be viewed as enabling both syntactic and discourse functions. In this section, we show that word order in Rukiga is determined by discourse roles more than grammatical roles \citep{KerrEtAl2023}. If a canonical order has to be specified, we indicate that the best answer to a VP question is the order given in \xref{bkm:Ref53392113}, with the preverbal \textit{Pamela} functioning as the topic, and the verb and Theme being the comment, i.e. providing the new information anchored to the topic.

%%EAX
\ea
%%JUDGEMENT
%%LABEL
\label{bkm:Ref53392113}
%%CONTEXT
(What will Pamela do?)\\
%%LINE1
%%LINE2
\gll
Paméla  a-ryá-téek-a  muhógo\\
%%LINE3
1.Pamela  1\SM{}-\FUT{}-cook-\FV{}  9.cassava\\
%%TRANS1
\glt
‘Pamela will cook cassava.’\\
%%TRANS2
%%EXEND

\z

We also find that it is common to find all active arguments expressed by subject and object markers on the verb as illustrated in \xref{bkm:Ref113368013}.

%%EAX
\ea
%%JUDGEMENT
%%LABEL
\label{bkm:Ref113368013}
%%CONTEXT
(Has grandmother given the children the mangos?)\\
%%LINE1
%%LINE2
\gll
Y-áa-gi-bá-h-a.\\
%%LINE3
1\SM{}-\N{}.\PST{}-4\OM{}-2\OM{}-give-\FV{}\\
%%TRANS1
\glt
‘She (grandmother) has given them (the mangoes) to them (the children).’\\
%%TRANS2
%%EXEND

\z

The canonical word order is also used in the context of focus on the predicate, that is, State-of-Affairs focus \xref{bkm:Ref116997318:a} and polarity focus \xref{bkm:Ref116997318:b}. Note that it is more natural to pronominalise given arguments, as in the TAM focus in \xref{bkm:Ref116997318:c}.

\ea
\label{bkm:Ref116997318}
%%EAX
\ea
%%JUDGEMENT
%%LABEL
\label{bkm:Ref116997318:a}
%%CONTEXT
(Did you write the book?)\\
%%LINE1
Nshomir’ ékitabó kyônka,\footnote{Note that \textit{kyonka} ‘only’ does not agree with \textit{ekitabo} ‘book’ – as an adverb it takes this invariant form (ignoring dialectal variation with \textit{kwonka}), as in \textit{Náárya kyonka} ‘I only ate (I didn't drink)’.} tindákíhandiikire.\\
%%LINE2
\gll
n-shom-ire  e-ki-tabo  ki-onka  ti-n-ra-ki-handiik-ire\\
%%LINE3
1\SG{}.\SM{}-read-\PFV{}  \AUG{}-7-book  7-only  \NEG-1\SG.\SM-\F.\PST-7\OM{}-write-\PFV{}\\
%%TRANS1
\glt
‘I only \textit{read} the book, I didn’t write it.’\\
%%TRANS2
%%EXEND



%%EAX
\ex
%%JUDGEMENT
%%LABEL
\label{bkm:Ref116997318:b}
%%CONTEXT
(Are you sure mother bought bananas; I can’t see them?)\\
%%LINE1
%%LINE2
\gll
Máama  y-aa-gur-a  é-mi-nekye.  Ronda  gye.\\
%%LINE3
1.Mother  1\SM{}-\N{}.\PST{}-buy-\FV{}  \AUG{}-4-banana.{} look.for.\IMP{} well\\
%%TRANS1
\glt
‘Mother \textit{did} buy bananas. Check properly.’\\
%%TRANS2
%%EXEND

%%EAX
\ex
%%JUDGEMENT
%%LABEL
\label{bkm:Ref116997318:c}
%%CONTEXT
(Have you bathed the children?)\\
%%LINE1
Íngaaha,~kwonká\textbf{~}ninzá kubánaabisa.~\\
%%LINE2
\gll
ngaaha  kwonka  ni-n-z-a  ku-ba-naab-is-a\\
%%LINE3
no  but  1\SG{}.\SM{}-\PRS{}-go-\FV{}  15-2\OM{}-bathe-\CAUS{}-\FV{}\\
%%TRANS1
\glt
‘No but I \textit{will} bathe them.’\\
%%TRANS2
%%EXEND


\z
\z


In the rest of this section, we discuss the preverbal and postverbal positions and show the extent to which information structure influences word order in Rukiga. Arguments can be left- or right-dislocated and there are discourse interpretational variations depending on the order of constituents in a sentence.

\subsection{Preverbal position}
\subsubsection{No preverbal focus}

It is generally the case that the preverbal domain is associated with topics, and focused elements are not permitted in the preverbal domain. It is therefore ungrammatical to use an interrogative word preverbally \xref{bkm:Ref105427809:a}, or put an answer to an interrogative element in the preverbal domain \xref{bkm:Ref105427809:b}. 

\ea
\label{bkm:Ref105427809}
%%EAX
\ea
%%JUDGEMENT
[*]{
%%LABEL
\label{bkm:Ref105427809:a}
%%CONTEXT
%%LINE1
%%LINE2
\gll
Kí  Jóvani  y-aa-twar-a?\\
%%LINE3
what  1.Jovan  1\SM{}-\N{}.\PST{}-{}take-\FV{}\\
%%TRANS1
\glt
int. ‘What has Jovan taken?’\\
%%TRANS2
}
%%EXEND


\ex[]{
\label{bkm:Ref105427809:b}
(What has Jovan taken?)\\}
%%EAX
\sn
%%JUDGEMENT
[\textsuperscript{\#}]{
%%LABEL
%%CONTEXT
%%LINE1
%%LINE2
\gll
E-n-tébe  Jóvani  y-áá-twár-a.\\
%%LINE3
\AUG{}-9-chair  1.Jovan  1\SM{}-\N{}.\PST{}-{}take-\FV{}\\
%%TRANS1
\glt
int. ‘Jovan has taken a chair.’\\
%%TRANS2
}
%%EXEND

\z
\z


Grammatical subjects as focal elements cannot be questioned in the preverbal domain either. Instead, a cleft construction \xref{bkm:Ref118629450:a} can be used, or a pseudocleft \xref{bkm:Ref118629450:b}, or default agreement inversion (DAI) \xref{bkm:Ref118629450:c} – see \sectref{bkm:Ref116291412} for more information on DAI, and \sectref{sec:cleft-constructions} on clefts.

\ea
\label{bkm:Ref118629450}
%%EAX
\ea
%%JUDGEMENT
%%LABEL
\label{bkm:Ref118629450:a}
%%CONTEXT
%%LINE1
Nooh’ ówíija?\\
%%LINE2
\gll
ni  o-ha  o-u-aa-ij-a\\
%%LINE3
\COP{}  1-who  \AUG{}-1\RM{}-\N{}.\PST{}-{}come-\FV{}\\
%%TRANS1
\glt
lit. ‘It is who who came?\\
%%TRANS2
‘Who has come?’\\
%%EXEND


%%EAX
\ex
%%JUDGEMENT
%%LABEL
\label{bkm:Ref118629450:b}
%%CONTEXT
%%LINE1
Owíija n’ ooha?\\
%%LINE2
\gll
o-u-aa-ij-a  ni  o-ha\\
%%LINE3
\AUG{}-1\RM{}-\N{}.\PST{}-come-\FV{}  \COP{}  1-who\\
%%TRANS1
\glt
‘Who has come?’\\
%%TRANS2
%%EXEND

%%EAX
\ex
%%JUDGEMENT
%%LABEL
\label{bkm:Ref118629450:c}
%%CONTEXT
%%LINE1
Haija oha?\\
%%LINE2
\gll
Ha-ij-a  o-ha\\
%%LINE3
16\SM{}-come-\FV{}  1-who\\
%%TRANS1
\glt
‘Who has come?'\\
%%TRANS2
%%EXEND

\z
\z

Equally, a preverbal argument cannot be modified by the focus particle ‘only’ as the ungrammaticality of \REF{bkm:Ref105432915} and \REF{bkm:Ref105432947} show.

%%EAX
\ea
%%JUDGEMENT
[*]{
%%LABEL
\label{bkm:Ref105432915}
%%CONTEXT
%%LINE1
Táátá wenká yíij-a\\
%%LINE2
\gll
  Taata  w-enka  a-aa-ij-a\\
%%LINE3
  1.father  1-only  1\SM{}-\N{}.\PST{}-come-\FV{}\\
%%TRANS1
\glt
  ‘Only dad came.’\\
%%TRANS2
}
%%EXEND


\z

%%EAX
\ea
%%JUDGEMENT
[*]{
%%LABEL
\label{bkm:Ref105432947}
%%CONTEXT
%%LINE1
Emigaatí  yonká  omukáma  aguririre  ábéegi.\\
%%LINE2
\gll
e-mi-gaati  y-onka  o-mu-kama  a-gur-ir-ire  a-ba-egi\\
%%LINE3
\AUG{}-4-bread  4-only  \AUG{}-1-king  1\SM{}-buy-\APPL{}-\PFV{}  \AUG{}-2-student\\
%%TRANS1
\glt
`Only bread the king bought for the students.’\\
%%TRANS2
}
%%EXEND

\z


\subsubsection{Preverbal topics}

Topics typically appear in the preverbal domain. By topic, here we mean ``what the sentence is about" \citep{Reinhart1981}, or the “spatial, temporal, or individual framework within which the main predication holds” \citep[50]{Chafe1976} (see also \cite{chapters/intro}). This is complemented by the comment, in which some information is added to the topic. Theme and Recipient arguments appear postverbally when they are part of the comment, but when  topical, they preferably occur preverbally, as illustrated in \xref{bkm:Ref111449218}. When an object appears preverbally, the presence of an object marker on the verb is required (\textit{ba}- in \xref{bkm:Ref111449218}). Note that both \textit{abaana} ‘children’ and \textit{kaaka} ‘grandmother’ in \xref{bkm:Ref111449218} can be analysed as topics.\footnote{Left-dislocated subjects (as \textit{kaaka}) occur as contrastive topics, see \sectref{bkm:Ref111448990}.}

%%EAX
\ea
%%JUDGEMENT
%%LABEL
\label{bkm:Ref111449218}
%%CONTEXT
(Context: Children are seen leaving their grandmother’s house, one carrying a basket on her head.)\\
%%LINE1
  Abáána kááka yaa*(ba)há ki?\\
%%LINE2
\gll
  a-ba-ana  kaaka  a-aa-ba-h-a  ki\\
%%LINE3
  \AUG{}-2-child  1.grandmother  1\SM{}-\N{}.\PST{}-2\OM{}-give-\FV{}  what\\
%%TRANS1
\glt `The children, what did grandmother give them?'\\
%%TRANS2
%%EXEND

\z

Deriving a passive verb may also be used to promote objects in the active counterpart not just to subjects but also to topics, as exemplified in \xref{bkm:Ref116291168} and further discussed in \sectref{bkm:Ref111461978}. In this example, the suffix \mbox{-\textit{w}} creates a passive verb which promotes the Theme to a subject function (as seen in the subject marking), while leaving the Actor argument (the wind) in a postverbal position.\footnote{Note that in Rukiga the demoted Actor does not require further marking, i.e. no preposition such as ‘by’ is needed.}

%%EAX
\ea
%%JUDGEMENT
%%LABEL
\label{bkm:Ref116291168}
%%CONTEXT
(Who opened the window?)\\
%%LINE1
Edirísá ekaigur\textbf{w}’ ómuyaga.\\
%%LINE2
\gll
e-dirisa  e-ka-igur-\textbf{w}-a  o-mu-yaga\\
%%LINE3
\AUG{}-9.window  9\SM{}-\F{}.\PST{}-open-\PASS{}-\FV{}  \AUG{}-3-wind\\
%%TRANS1
\glt
‘The window was opened by the wind.’\\
%%TRANS2
%%EXEND


\z


Locative inversion constructions equally present topical locative phrases in the left periphery, such as \textit{aha rutindo} ‘on the bridge’ in \xref{bkm:Ref111449185} – see the discussion on inversion constructions in \sectref{bkm:Ref100083142}.

%%EAX
\ea
\nolistbreak
%%JUDGEMENT
%%LABEL
\label{bkm:Ref111449185}
%%CONTEXT
%%LINE1
Aha rutindo haarabah’émótoka\\
%%LINE2
\gll
a-ha  ru-tindo  ha-aa-rab-a=ho  e-motoka\\
%%LINE3
\AUG{}-16  11-bridge  16\SM{}-\N{}.\PST{}-{}pass-\FV{}=16  \AUG{}-10.car\\
%%TRANS1
\glt
‘Cars have passed on the bridge.’\\
%%TRANS2
%%EXEND


\z


Locative and temporal expressions also appear sentence-initially if they help to set the scene, as illustrated in \xref{bkm:Ref116291272} and \xref{bkm:Ref116291274}. Note that these may be separated from the rest of the sentence by a pause, as indicated by the comma in \xref{bkm:Ref116291274}.

%%EAX
\ea
%%JUDGEMENT
%%LABEL
\label{bkm:Ref116291272}
%%CONTEXT
%%LINE1
Omu mbága obugiineyó baahe?\\
%%LINE2
\gll
o-mu  n-baga  o-bugan-ire=yo  ba-he?\\
%%LINE3
\AUG{}-18  9-party  2\SG{}.\SM{}-find-\PFV{}=23{}  2-who\\
%%TRANS1
\glt
‘At the party, whom did you meet?’\\
%%TRANS2
%%EXEND


\z

%%EAX
\ea
%%JUDGEMENT
%%LABEL
\label{bkm:Ref116291274}
%%CONTEXT
%%LINE1
Erizóobá (,) Píta yaateek’ ákahûnga.\\
%%LINE2
\gll
e-ri-zooba  Pita  a-aa-teek-a  a-ka-hunga\\
%%LINE3
\AUG{}-5-day  1.Peter  1\SM{}-\N{}.\PST{}-cook-\FV{}  \AUG{}-12-posho\\
%%TRANS1
\glt
‘Today, Peter has cooked posho.’\\
%%TRANS2
%%EXEND


\z

The preverbal position is not exclusively reserved for topics, however, even if \citet[79]{Taylor1985} notes that the initial position is the only reliable marker of topicality. Subjects in thetic sentences may appear preverbally, as in \xref{bkm:Ref116291368}, and this suggests that Rukiga allows non-topical elements in the preverbal position \citep{KerrEtAl2023}, because the subject in a thetic sentence is detopicalised \citep{Sasse1996,Lambrecht1994}.

%%EAX
\ea
%%JUDGEMENT
%%LABEL
\label{bkm:Ref116291368}
%%CONTEXT
(What is the matter?)\\
%%LINE1
%%LINE2
\gll
O-mu-gôngo  ni-gu-n-sháash-a.\\
%%LINE3
\AUG{}-3-back  \IPFV{}-3\SM{}-1\SG{}.\OM{}-hurt-\FV{} \\
%%TRANS1
\glt
`My back is hurting.’\\
%%TRANS2
%%EXEND


\z

It is also possible for an indefinite subject such as \textit{omuntu} ‘person, someone’ to occur in a preverbal position \xxref{bkm:Ref113432268}{bkm:Ref113432286}. As indefinite non-specific referents cannot form topics, this too suggests that the preverbal position in Rukiga is not a dedicated topic position.

%%EAX
\ea
%%JUDGEMENT
%%LABEL
\label{bkm:Ref113432268}
%%CONTEXT
(Context: We are three and have different jobs to do, but don’t worry about the grazing, there   is somebody from outside that will do that.)\\
%%LINE1
Ente zó, omuntu naazá kuziríisa.\\
%%LINE2
\gll
e-n-te  z-o  o-mu-ntu  n-aa-za  ku-zi-ri-is-a\\
%%LINE3
\AUG{}-10-cows  10-\CM{}  \AUG{}-{}1-person  \IPFV{}-1\SM{}-go  15-10\OM{}-eat-\CAUS{}-\FV{}\\
%%TRANS1
\glt
`As for the cows, someone will graze them.’\\
%%TRANS2
%%EXEND


\z

%%EAX
\ea
%%JUDGEMENT
%%LABEL
\label{bkm:Ref113432286}
%%CONTEXT
(Have you heard a loud bang?)\\
%%LINE1
Ekintu kyó kyáhirima.\footnote{The contrastive topic marker \textit{kyo} here is optional; it adds intensity in this case.}\\
%%LINE2
\gll
e-ki-ntu  ki-o  ki-aa-hirim-a\\
%%LINE3
\AUG{}-7-thing  7-\PRO{}  7\SM{}-\N{}.\PST{}-fall-\FV{}\\
%%TRANS1
\glt
`Something (indeed) has fallen.’ (and made a very loud noise)\\
%%TRANS2
%%EXEND


\z

Note, though, that it is more natural to use a presentational subject inversion construction here (see \sectref{bkm:Ref116291412}), as in \xref{bkm:Ref116291393}, to compare with \xref{bkm:Ref113432268}.

%%EAX
\ea
%%JUDGEMENT
%%LABEL
\label{bkm:Ref116291393}
%%CONTEXT
%%LINE1
%%LINE2
\gll
E-n-te  z-ó  \textbf{ha-ine}  \textbf{ó-mu-ntu}  ó-ríku-z-á ku-zi-ríis-a \\
%%LINE3
\AUG{}-10-cows  10-\CM{}  16\SM{}-{}have  \AUG{}-1-person  1\RM{}-\IPFV{}-{}go-\FV{} 15-{}10\OM{}-{}eat-\CAUS{}-\FV{} \\
%%TRANS1
\glt
‘As for the cows, there is someone who will graze them.’\\
%%TRANS2
%%EXEND

\z


Furthermore, an indefinite (non-topical) interpretation does not seem to be acceptable for preverbal objects \xref{bkm:Ref111449315}, suggesting that there is a dedicated non-dislocated preverbal subject position in addition to the topic positions in the left periphery.

\ea
\label{bkm:Ref111449315}
(Has s/he bought something?)\\
%%EAX
\sn[*]{
%%JUDGEMENT
%%LABEL
%%CONTEXT
%%LINE1
Ekintu kyó yáákígura.\\
%%LINE2
\gll
e-ki-ntu  ki-o  a-aa-ki-gur-a\\
%%LINE3
\AUG{}-7-thing  7-\CM{}  1\SM{}-\N{}.\PST{}-7\OM{}-buy-\FV{}\\
%%TRANS1
\glt
int. ‘The thing, s/he bought it.’\\
%%TRANS2
}
%%EXEND

\z

In summary, there is a preference for topical constituents to appear preverbally, either fronted or assuming a subject function through passivisation or subject inversion, but non-topical subjects may also appear in the preverbal domain.

\subsubsection{Multiple topics}
\label{bkm:Ref127956542}
Multiple topics are allowed in the preverbal domain, both arguments \xxref{bkm:Ref113433194}{bkm:Ref117001570} and adverbs \xxref{bkm:Ref117001503}{bkm:Ref117001902}. The topical constituents are indicated by square brackets in these examples. As for preverbal objects, they must be resumed by an object marker. The adverbs are scene-setting topics, and the arguments may be familiarity topics (active from previous discourse) or contrastive topics (see next section). Some could also be analysed as ``secondary topics", meaning “an entity such that the utterance is construed to be about the relationship between it and the primary topic” \citep{Nikolaeva2001} – a typical example of a secondary topic is \textit{ebihimba} ‘the beans’ in \xref{bkm:Ref113433194}.

%%EAX
\ea
%%JUDGEMENT
%%LABEL
\label{bkm:Ref113433194}
%%CONTEXT
(Did father cook the beans?)\\
%%LINE1
Táát’ ebihîmba abiteekíre.\\
%%LINE2
\gll
[taata]  [e-bi-himba]  a-bi-teek-ire\\
%%LINE3
{\db}1.father  {\db}\AUG{}-8-bean  1\SM{}-8\OM{}-cook-\PFV{}\\
%%TRANS1
\glt
`Father, the beans, he cooked them.’\\
%%TRANS2
%%EXEND


\z

%%EAX
\ea
%%JUDGEMENT
%%LABEL
\label{bkm:Ref117001570}
%%CONTEXT
%%LINE1
Omwán’ ámaté yáágánywa?\\
%%LINE2
\gll
{}[o-mw-ana]  [a-ma-te]  a-aa-ga-nyw-a?\\
%%LINE3
{\db}\AUG{}-1-child  {\db}\AUG{}-6-milk  1\SM{}-\N{}.\PST{}-6\OM{}-drink-\FV{}\\
%%TRANS1
\glt
‘Has the child drunk the milk?’\\
%%TRANS2
%%EXEND


\z

%%EAX
\ea
%%JUDGEMENT
%%LABEL
\label{bkm:Ref117001503}
%%CONTEXT
(Context: Herdsmen passing on information to the cattle owner.)\\
%%LINE1
Nyómwébazy’ ómu kashéeshe tutwire énte kunywa ámáizi.\\
%%LINE2
\gll
{}[nyomwebazyo]  [o-mu  kasheeshe]  tu-twar-ire  e-n-te  ku-nywa     a-ma-izi \\
%%LINE3
{\db}yesterday  {\db}\AUG{}-18  12.morning  1\PL{}.\SM{}-take-\PFV{}  \AUG{}-10-cow  15-drink   \AUG{}-6-water \\
%%TRANS1
\glt
  \glt ‘Yesterday morning, we took the cows to drink water.’\\
%%TRANS2
%%EXEND

\z

%%EAX
\ea
%%JUDGEMENT
%%LABEL
\label{bkm:Ref117001902}
%%CONTEXT
%%LINE1
Omu bwire bwa Yés’ ábant’ ábaabaire baba bain’ éndwára nk’ébibémbe, hamwé n’ézíndi ndwára bakabá babashoróora.\\
%%LINE2
\gll
{}[o-mu  bu-ire  bu-a  Yesu]  [a-ba-ntu  a-ba-aba-ire  ba-ba ba-ine  e-n-dwara  nka  e-bi-bembe  hamwe  na  e-zi-ndi n-dwara]  ba-ka-b-a  ba-ba-shoroor-a\\
%%LINE3
{\db}\AUG{}-18  14-time  14-\CONN{}  1.Jesus  {\db}\AUG{}-2-person  \AUG{}-2\RM{}-be-\PFV{}  2\SM{}-be 2\SM{}-have  \AUG{}-10-disease  like  \AUG{}-8-leprosy  and  and  \AUG{}-10-other 10-disease  2\SM{}-\F.\PST{}-be-\FV{}  2\SM{}-2\OM{}-discriminate-\FV{}\\
%%TRANS1
\glt
‘During Jesus’ time, people who were suffering from diseases like leprosy and other diseases were discriminated against.’, lit. `... they discriminated them.'\\
%%TRANS2
%%EXEND


\z

\subsubsection{Contrastive topics}
\label{bkm:Ref111448990}
A preverbal element can also form a contrastive topic, as seen for the independent pronouns \textit{íwe} and \textit{nyówe} in \xref{bkm:Ref111452864}, and the adverbial \textit{omu mushana} ‘during the day’ in \xref{bkm:Ref111452876}, which is contrasted with \textit{nyekiro} ‘at night’.

%%EAX
\ea
%%JUDGEMENT
%%LABEL
\label{bkm:Ref111452864}
%%CONTEXT
(QUIS map task)\\
%%LINE1
\textbf{Íwe} oine piki ya burúrú kusha \textbf{nyowe} tíhó nd’ ááho.\\
%%LINE2
\gll
iwe  o-ine  piki  y-a  bururu  kusha  nyowe  ti=ho  n-ri  a-ho\\
%%LINE3
2\SG{}.\PRO{}  2\SG{}.\SM{}-have  9.motorcycle  9-\CONN{}  blue  but  1\SG{}.\PRO{}  \NEG{}.\COP{}=16  1\SG{}.\SM{}-be  \DEM{}-16.\PROX{}\\
%%TRANS1
\glt
‘You have a blue motorcycle. But for me, that is not where I am.’\\
%%TRANS2
%%EXEND


\z

%%EAX
\ea
%%JUDGEMENT
%%LABEL
\label{bkm:Ref111452876}
%%CONTEXT
%%LINE1
Kikáá nikirond’ émére nyékiro; \textbf{omumushaná} kinyam’ áhitagi ryómutí murungi; kishwek’ámíísho kitagahúmbya góona.\\
%%LINE2
\gll
  Ki-ka-b-a  ni-ki-rond-a  e-mere  nyekiro  o-mu  mu-shana    ki-nyam-a  a-ha  i-taagi   ri-a  o-mu-ti  mu-rungi  ki-shwek-a    a-ma-isho  ki-ta-ga-humby-a  ga-ona\\
%%LINE3
7\SM{}-\F{}.\PST{}-be-\FV{}  \IPFV{}-7\SM{}-look.for-\FV{}  \AUG{}-10.food  night  \AUG{}-18  3-day  7\SM{}-sleep-\FV{}  \AUG{}-16  5-branch  5-\CONN{}  \AUG{}-3-tree  3-good  7\SM{}-cover-\FV{}    \AUG{}-6-eye  7\SM{}-\NEG{}-6\OM{}-close-\FV{}  6-all\\
%%TRANS1
\glt
  (about the owl) ‘It would look for food at night and during the day sleep on a nice tree branch; when it sleeps, it does not close the eyes completely.’\\
%%TRANS2
%%EXEND


\z


A contrastive topic can also be indicated by the particle -\textit{o} – see \sectref{sec:particle-o} for a discussion on this particle. This is illustrated for the adverb \textit{nyomwebazo} ‘yesterday’ in \xref{bkm:Ref135651945}, where the contrast marker comes out as \textit{bwe}.

%%EAX
\ea
%%JUDGEMENT
%%LABEL
\label{bkm:Ref135651945}
%%CONTEXT
(Did you go to school yesterday and today?)\\
%%LINE1
Nyómwébázyó bwé tinshomíre.\\
%%LINE2
\gll
nyomwebazyo  bu-o  ti-n-shom-ire\\
%%LINE3
yesterday  14-\CM{}  \NEG{}-1\SG{}.\SM{}-read-\PFV{}\\
%%TRANS1
\glt
‘Yesterday I did not go to study.’ (but I did study today)\\
%%TRANS2
%%EXEND

\z

Independent pronouns are also used to mark contrastive topics. As such, they typically occur in the preverbal position (see \xref{bkm:Ref113439970:b} and \xref{bkm:Ref113439996:a}), although they can also come in the final position as in \xref{bkm:Ref113439996:b}. These pronouns are optional in Rukiga. When absent, a contrastive reading on the topic is not obvious, as can be seen in the comparison of \xref{bkm:Ref113439970:a} and \xref{bkm:Ref113439970:b}.

\ea
\label{bkm:Ref113439970}
%%EAX
\ea
%%JUDGEMENT
%%LABEL
\label{bkm:Ref113439970:a}
%%CONTEXT
%%LINE1
Naaruk’ékíibo.\\
%%LINE2
\gll
n-aa-ruk-a  e-ki-ibo\\
%%LINE3
1\SG{}.\SM{}-\N{}.\PST{}-weave-\FV{}  \AUG{}-7-basket\\
%%TRANS1
\glt
‘I weaved a basket.’\\
%%TRANS2
%%EXEND

%%EAX
\ex
%%JUDGEMENT
%%LABEL
\label{bkm:Ref113439970:b}
%%CONTEXT
%%LINE1
Nyowé naaruk’ékíibo.\\
%%LINE2
\gll
\textbf{nyowe}  n-aa-ruk-a  e-ki-ibo\\
%%LINE3
1\SG{}.\PRO{}  1\SG{}.\SM{}-\N{}.\PST{}-weave-\FV{}  \AUG{}-7-basket\\
%%TRANS1
\glt
‘Me, I weaved a basket.’ (maybe others did not weave baskets but other kinds of crafts or did other activities)\\
%%TRANS2
%%EXEND

\z
\z

\ea
\label{bkm:Ref113439996}
%%EAX
\ea
%%JUDGEMENT
%%LABEL
\label{bkm:Ref113439996:a}
%%CONTEXT
%%LINE1
\textbf{Imwe} mushitam’ ómumíisho.\\
%%LINE2
\gll
imwe  mu-shitam-e  o-mu  ma-isho\\
%%LINE3
2\PL{}.\PRO{}  2\PL{}.\SM{}-sit-\SBJV{}  \AUG{}-18  6-front\\
%%TRANS1
\glt
‘(For you) You sit in front.’\\
%%TRANS2
%%EXEND

  %%EAX
\ex
%%JUDGEMENT
%%LABEL
\label{bkm:Ref113439996:b}
%%CONTEXT
%%LINE1
Mushitam’ ómumíísh’ \textbf{íimwe}.\\
%%LINE2
\gll
mu-shitam-e  o-mu  ma-isho  imwe\\
%%LINE3
2\PL{}.\SM{}-sit-\SBJV{}  \AUG{}-18  6-front  2\PL{}.\PRO{} \\
%%TRANS1
\glt
‘(For you) You sit in front.’\\
%%TRANS2
%%EXEND

\z
\z

In summary, we have seen in this section that in Rukiga, the preverbal position is not dedicated to topics since it accommodates thetics and indefinite subjects which are non-topical. We have further demonstrated that it is possible to have multiple topics in the preverbal position, both arguments and adverbs. Objects as topics must, however, be resumed on the verb. And lastly, we have noted that contrastive topics are also marked, by being expressed as the independent personal pronoun, or by an additional contrastive particle in \textit{-o}. Before discussing this particle further in \sectref{sec:particle-o}, we first continue our presentation of word order and how it reflects information structure. The next subsection \xref{bkm:Ref127958714} focuses on the postverbal position. As we have shown that there is no preverbal focus in Rukiga,  focused elements must come after the verb.

\subsection{Postverbal focus}
\label{bkm:Ref127958714}
Focused elements typically appear postverbally, as seen in questions and answers for Themes and Locatives in \xref{bkm:Ref113441667} and \xref{bkm:Ref113441670}.

\ea
\label{bkm:Ref113441667}
%%EAX
\ea
%%JUDGEMENT
%%LABEL
%%CONTEXT
%%LINE1
Hélen atwire \textbf{ki}?\\
%%LINE2
\gll
Helen  a-twar-ire  ki\\
%%LINE3
1.Hellen  1\SM{}-take-\PFV{}  what\\
%%TRANS1
\glt
‘What did Hellen take?’\\
%%TRANS2
%%EXEND

%%EAX
\ex
%%JUDGEMENT
%%LABEL
%%CONTEXT
%%LINE1
Hélen atwir’ \textbf{ékikópo}.\\
%%LINE2
\gll
Helen  a-twar-ire  e-ki-kopo\\
%%LINE3
1.Hellen  1\SM{}-take-\PFV{}  \AUG{}-7-cup \\
%%TRANS1
\glt
‘Hellen took a cup.’\\
%%TRANS2
%%EXEND


\z
\z


\ea
\label{bkm:Ref113441670}
%%EAX
\ea
%%JUDGEMENT
%%LABEL
%%CONTEXT
%%LINE1
Amahúrire nibagagurá \textbf{nkáhe}?\\
%%LINE2
\gll
a-ma-hurire  ni-ba-ga-gur-a  nkahi\\
%%LINE3
\AUG{}-6-newspaper  \IPFV{}-2\SM{}-6\OM{}-buy-\FV{}  where\\
%%TRANS1
\glt
‘Where do I buy a newspaper?’\\
%%TRANS2
%%EXEND


 %%EAX
\ex
%%JUDGEMENT
%%LABEL
%%CONTEXT
%%LINE1
Nibagagurá aha \textbf{mídia sénta}.\\
%%LINE2
\gll
ni-ba-ga-gur-a  a-ha  midia.senta\\
%%LINE3
\IPFV{}-2\SM{}-6\OM{}-buy-\FV{}  \AUG{}-16  9.media.centre\\
%%TRANS1
\glt
‘They buy them at the Media Centre.’\\
%%TRANS2
‘They are bought at the Media Centre.’

%%EXEND

\z
\z

Focused elements are preferably adjacent to the verb in Rukiga, in the immediate-after-verb (IAV) position (see \citealt{Watters1979} for coining the term, and \citealt{Yoneda2011,vanderWal2009a,Buell2009} for claims of an IAV focus position in Matengo, Makhuwa-Enahara, and Zulu, respectively). To show that the IAV position is preferred for focus, consider that an interrogative word needs to be in the IAV position, as shown in \xref{bkm:Ref111453770} and \xref{bkm:Ref118704899}: different word orders are possible as long as the interrogative word is in the IAV position. In \xxref{bkm:Ref118704899:a}{bkm:Ref118704899:c}, the interrogative word \textit{oha} ‘who’ must occur in the IAV position. As illustrated in \xref{bkm:Ref118704899:d}, the construction becomes ungrammatical once there is an intervening element between the verb and the interrogative word.

\ea
\label{bkm:Ref111453770}
%%EAX
\ea
%%JUDGEMENT
[]{
%%LABEL
%%CONTEXT
%%LINE1
Kááka yaaha ky’ ábáana?\\
%%LINE2
\gll
Kaaka  ya-aa-h-a  ki  a-ba-ana\\
%%LINE3
1.grandmother  1\SM{}-\N{}.\PST{}-give  what  \AUG{}-2-child \\
%%TRANS1
\glt
‘What has grandmother given the children?’\\
%%TRANS2
}
%%EXEND

%%EAX
\ex
%%JUDGEMENT
[*]{
%%LABEL
%%CONTEXT
%%LINE1
Kááka yaah’ ábáána ki?\\
%%LINE2
\gll
kaaka  a-aa-h-a  a-ba-ana  ki\\
%%LINE3
1.grandmother  1\SM{}-\N{}.\PST{}-give-\FV{}  \AUG{}-2-child  what\\
%%TRANS1
\glt
int. ‘What has grandmother given the children?’\\
%%TRANS2
}
%%EXEND


\z
\z

\ea
\label{bkm:Ref118704899}
%%EAX
\ea
%%JUDGEMENT
[]{
%%LABEL
\label{bkm:Ref118704899:a}
%%CONTEXT
%%LINE1
Káák’ émiyembe agihiir’ óha?\\
%%LINE2
\gll
kaaka  e-mi-yembe  a-gi-h-ire  o-ha\\
%%LINE3
1.grandmother  \AUG{}-4-mango  1\SM{}-4\OM{}-give-\PFV{}  1-who\\
%%TRANS1
\glt
‘Who did grandmother give the mangoes?’\\
%%TRANS2
}
%%EXEND


%%EAX
\ex
%%JUDGEMENT
[]{
%%LABEL
\label{bkm:Ref118704899:b}
%%CONTEXT
%%LINE1
Kááka ahiir’ óhá emiyembe?\\
%%LINE2
\gll
kaaka  a-h-ire  o-ha  e-mi-yembe\\
%%LINE3
1.grandmother  1\SM{}-give-\PFV{}  1-who  \AUG{}-4-mango\\
%%TRANS1
\glt
‘Who did grandmother give mangoes?’\\
%%TRANS2
}
%%EXEND

%%EAX
\ex
%%JUDGEMENT
[]{
%%LABEL
\label{bkm:Ref118704899:c}
%%CONTEXT
%%LINE1
Emiyembe káák’ agihiir’ óha?\\
%%LINE2
\gll
e-mi-yembe  kaaka  a-gi-h-ire  o-ha\\
%%LINE3
\AUG{}-4-mango  1.grandmother  1\SM{}-4\OM{}-give-\PFV{}  1-who\\
%%TRANS1
\glt
‘Who did grandmother give the mangoes?’\\
%%TRANS2
}
%%EXEND


%%EAX
\ex
%%JUDGEMENT
[*]{
%%LABEL
\label{bkm:Ref118704899:d}
%%CONTEXT
%%LINE1
Kááka ahiir’ émiyemb’ óha?\\
%%LINE2
\gll
kaaka  a-h-ire  e-mi-yemba  o-ha\\
%%LINE3
1.grandmother  1\SM{}-give-\PFV{}  \AUG{}-4-mango  1-who\\
%%TRANS1
\glt
int. ‘Who did grandmother give mangos?’\\
%%TRANS2
}
%%EXEND

\z
\z

Although the interrogative word must appear in the IAV position, the answer does not need to. As illustrated in \xref{bkm:Ref113442937}, the Theme ‘hat’ can be an answer to an interrogative word in its canonical (non-IAV) position (independently of the animacy of the objects). This suggests that interrogative words are more restricted in word order than their answers.

\ea
\label{bkm:Ref113442937}
%%EAX
\ea
%%JUDGEMENT
%%LABEL
%%CONTEXT
%%LINE1
Waaha kí Jéini?\\
%%LINE2
\gll
u-aa-ha  ki  Jeini?\\
%%LINE3
2\SG{}.\SM{}-\N{}.\PST{}-give-\FV{}  what  1.Jane\\
%%TRANS1
\glt
‘What have you given Jane?’\\
%%TRANS2
%%EXEND

%%EAX
\ex
%%JUDGEMENT
%%LABEL
%%CONTEXT
%%LINE1
Naaha Jéín’ \textbf{énkofiira}.\\
%%LINE2
\gll
n-aa-h-a  Jeini  e-n-kofiira\\
%%LINE3
1\SG{}.\SM{}-\N{}.\PST{}-give-\FV{}  1.Jane  \AUG{}-9-hat\\
%%TRANS1
\glt
‘I have given a hat to Jane.’\\
%%TRANS2
%%EXEND


\z
\z

Unlike arguments, as we just saw, not all questioned adverbs are restricted to the IAV position. The interrogative adverb of time ‘when’ in Rukiga can appear in the IAV or in a non-IAV position, as shown in \xref{bkm:Ref113443019}. Compare with the adverb ‘where’ in \xref{bkm:Ref113443274}, which prefers to be in the IAV position. 

\ea
\label{bkm:Ref113443019}
%%EAX
\ea
%%JUDGEMENT
%%LABEL
%%CONTEXT
%%LINE1
Okaza Kampálá \textbf{ryári}?\\
%%LINE2
\gll
o-ka-z-a  Kampala  ryari \\
%%LINE3
2\SG{}.\SM{}-\F{}.\PST{}-go-\FV{}  23.Kampala  when\\
%%TRANS1
\glt
‘When did you go to Kampala?’\\
%%TRANS2
%%EXEND


%%EAX
\ex
%%JUDGEMENT
%%LABEL
%%CONTEXT
%%LINE1
Okaza \textbf{ryarí} Kampala?\\
%%LINE2
\gll
o-ka-z-a  ryari  Kampala\\
%%LINE3
2\SG{}.\SM{}-\F{}.\PST{}-go-\FV{}  when  23.Kampala\\
%%TRANS1
\glt
‘When did you go to Kampala?’\\
%%TRANS2
%%EXEND


\z
\z

\ea
\label{bkm:Ref113443274}
%%EAX
\ea
%%JUDGEMENT
[]{
%%LABEL
%%CONTEXT
%%LINE1
Tuguré \textbf{nkah}’ ébitookye?\\
%%LINE2
\gll
tu-gur-e  nkahe  e-bi-tookye\\
%%LINE3
1\PL{}.\SM{}-buy-\SBJV{}  where  \AUG{}-8-plantain\\
%%TRANS1
\glt
‘Where should we buy plantains?’\\
%%TRANS2
}
%%EXEND


%%EAX
\ex
%%JUDGEMENT
[\textsuperscript{?}]{
%%LABEL
%%CONTEXT
%%LINE1
Tugur’ ébitookye \textbf{nkahe}?\\
%%LINE2
\gll
tu-gur-e  e-bi-tookye  nkahe\\
%%LINE3
1\PL{}.\SM{}-buy-\SBJV{}  \AUG{}-8-plantains  where\\
%%TRANS1
\glt
int. ‘Where should we buy plantains?’\\
%%TRANS2
}
%%EXEND


\z
\z


Objects modified by ‘only’ equally require an IAV position and do not allow right-dislocation, as shown in \xref{bkm:Ref111462661}. Both the tonally reduced and the non-reduced form of the verb are indicated, and the order is unacceptable for either.

\ea
\label{bkm:Ref111462661}
%%EAX
\ea[]{
%%JUDGEMENT
%%LABEL
%%CONTEXT
%%LINE1
Píta yaateek’ \textbf{ákahúngá kónk’} érizóoba.\\
%%LINE2
\gll
Pita  a-aa-teek-a  a-ka-hunga  ka-onka  e-ri-zooba\\
%%LINE3
1.Peter  1\SM{}-\N{}.\PST{}-cook-\FV{}  \AUG{}\textbf{-12-posho}  12-only  \AUG{}-5-day\\
%%TRANS1
\glt
‘Peter cooked only posho today.’}
%%TRANS2
%%EXEND


%%EAX
\ex
%%JUDGEMENT
[*]{
%%LABEL
%%CONTEXT
%%LINE1
Pita yaa(ka)teeká/yáátéeka erizooba \textbf{akahúnga kónka}.\\
%%LINE2
\gll
Pita  a-aa-teek-a  e-ri-zooba  a-ka-hunga  ka-onka\\
%%LINE3
1.Peter  1\SM{}-\N{}.\PST{}-cook-\FV{}  \AUG{}-5-day  \AUG{}-12-posho  12-only\\
%%TRANS1
\glt
‘Peter cooked only posho today.’\\
%%TRANS2
}
%%EXEND


\z
\z

For completeness, we mention that multiple argument questions are ungrammatical in Rukiga, neither in situ as in \xref{bkm:Ref116292843}, nor with a cleft as in \xref{bkm:Ref116292868} (unlike in for example Cicopi).

\ea
\label{bkm:Ref116292843}
(Context: At a charity, someone gave various people various clothes.)\\
%%EAX
\sn
%%JUDGEMENT
[*]{
%%LABEL
%%CONTEXT
%%LINE1
Yaah’oha énki? / *Yaah’énki oha?\\
%%LINE2
\gll
a-a-h-a  o-ha  enki  / enki  o-ha\\
%%LINE3
1\SM{}-\PST{}-give-\FV{}  1-who  what  / what  1-who\\
%%TRANS1
\glt
int. ‘Who did s/he give what?’ / ‘What did s/he give who?’\\
%%TRANS2
}
%%EXEND


\z

%%EAX
\ea
%%JUDGEMENT
[*]{
%%LABEL
\label{bkm:Ref116292868}
%%CONTEXT
%%LINE1
N’ ooh’ órikukurur’ énki?\footnote{The sentence is equally unacceptable with the clitic form for ‘what’, =\textit{ki}.}\\
%%LINE2
\gll
ni  o-ha  o-riku-kurur-a  enki\\
%%LINE3
\COP{}  1-who  1\SM{}.\REL{}-\IPFV{}-pull-\FV{}  what\\
%%TRANS1
\glt
int. ‘Who is pulling what?’\\
%%TRANS2
}
%%EXEND


\z

The present analysis confirms that Rukiga has an IAV position, although it is not a strict one because answers to interrogative questions and some adverbs are not required to be in this position. Further discussion of the word order and interpretation can be found in \citet{KerrEtAl2023}. Non-topical subjects also appear postverbally, but a subject inversion construction is used in that case, as shown in the next section.

\subsection{Subject inversion}
\label{bkm:Ref100083142}
In subject inversion constructions, the logical subject comes after the verb and is non-topical. \citet{MartenvanderWal2014} identify seven subject inversion constructions in Bantu languages. These are: formal locative inversion, semantic locative inversion, instrument inversion, patient inversion, complement inversion, default agreement inversion and agreeing inversion. Passive constructions are added to the list as they present related features to the inversion constructions. Bantu languages differ in terms of the inversion constructions each language allows (for a detailed analysis of these constructions; we refer to \citealt{MartenvanderWal2014}). Below we show subject inversion constructions that are possible in Rukiga.

\subsubsection{No patient inversion or instrument inversion}

Rukiga does not allow patient inversion \xref{bkm:Ref135654256} or instrument inversion \xref{bkm:Ref135654278}, in which the preverbal element is a Theme or Instrument agreeing with the verb and the logical subject is in a postverbal position. 

\ea
\label{bkm:Ref135654256}
%%EAX
\ea
%%JUDGEMENT
[]{
%%LABEL
%%CONTEXT
%%LINE1
Abacáína nibombek’ énkuuto.\\
%%LINE2
\gll
a-ba-caina  ni-ba-ombek-a  e-n-kuuto\\
%%LINE3
\AUG{}-2-chinese  \IPFV{}-2\SM{}-build-\FV{}  \AUG{}-10-road\\
%%TRANS1
\glt
‘The Chinese are building roads.’\\
%%TRANS2
}
%%EXEND


 %%EAX
\ex
%%JUDGEMENT
[*]{
%%LABEL
%%CONTEXT
%%LINE1
Enkuuto nizibombek’ abacaina.\\
%%LINE2
\gll
e-n-kuuto  ni-zi-bombek-a  a-ba-caina\\
%%LINE3
\AUG{}-10-road  \IPFV{}-10\SM{}-build-\FV{}  \AUG{}-2-chinese\\
%%TRANS1
\glt
int. ‘The \textit{Chinese} are building roads.’\\
%%TRANS2
}
%%EXEND


\z
\z

\ea
\label{bkm:Ref135654278}
%%EAX
\ea
%%JUDGEMENT
[]{
%%LABEL
%%CONTEXT
%%LINE1
Táát’ akahandiikis’ ákacúmu.\\
%%LINE2
\gll
taata  a-ka-handiik-is-a  a-ka-cumu\\
%%LINE3
1.father  1\SM{}-\F{}.\PST{}-write-\CAUS{}-\FV{}  \AUG{}-12-pen\\
%%TRANS1
\glt
‘Father wrote with a pen.’\\
%%TRANS2
}
%%EXEND


 %%EAX
\ex
%%JUDGEMENT
[*]{
%%LABEL
%%CONTEXT
%%LINE1
Akacúmu kakahandiikisa táata.\\
%%LINE2
\gll
a-ka-cumu  ka-ka-handiik-is-a  taata\\
%%LINE3
\AUG{}-12-pen  12\SM{}-\F{}.\PST{}-write-\CAUS{}-\FV{}  1.father\\
%%TRANS1
\glt
int. ‘\textit{Father} wrote with a pen.’\\
%%TRANS2
}
%%EXEND


\z
\z

\subsubsection{Locative inversion (LI)}

Locative inversion is possible, but is restricted with respect to the predicate, the locative marking, as well as the locative noun classes. We first discuss the restricted locative noun classes of Rukiga. Class 17 is disappearing from Rukiga; it is not available as a locative prefix\footnote{While the locative augment and prefix are written separately from the noun, following the orthography, they do not function as prepositions (yet) but form part of the NP.} (see examples in \xref{bkm:Ref113443792}) as in other related Bantu languages such as Luganda (JE15) \citep{Gregoire1975}.\footnote{Note that the augment on the noun cannot be present when preceded by a locative prefix.}

\ea
\label{bkm:Ref113443792}

\ea[]{
\gll
a-ha  n-tebe\\
\AUG{}-16  9-chair\\
\glt
‘on the chair’   \\
}

\ex[]{
\gll
o-mu  mu-ti \\
\AUG{}-18  3-tree \\
\glt
‘in the tree’ 
}

\ex[*]{
\gll
o-ku  n-tebe\\
\AUG{}-17  9-chair  \\
\glt
int. ‘on/to the chair’
}
\z
\z


However, \textit{ku} (class 17) can be found as a noun class prefix in one lexical entry \textit{okuzimu} (underground) and is also used in locative demonstratives \citep[see][]{Asiimwe2024} as in \textit{kunu} ‘here/this place’, \textit{okwo} ‘there/that place near speaker’ and \textit{kuri}(\textit{ya}) ‘there/that place (place far from both the speaker and hearer)’. 

Class 18 \textit{omu} is used productively to derive locative nouns with a meaning of containment. However, there is neither a subject nor an object marker for class 18, and class 18 nouns use the class 16 subject and object prefix marker. Only the class 16 subject and object marker \textit{ha}- is used for all the three noun classes as shown in \xxref{bkm:Ref113444051}{bkm:Ref113444066} \citep[but see][]{Asiimwe2014, BeermannAsiimweFut}. Note that only classes 16 and 18 are used as enclitics to the verb, and in addition also =\textit{yo} of class 23 \xref{bkm:Ref127966346}, which can co-occur with noun phrases in any of the locative classes.

%%EAX
\ea
%%JUDGEMENT
%%LABEL
\label{bkm:Ref113444051}
%%CONTEXT
%%LINE1
Omu rufûnjo hamezirey’ ébihîmba.\\\nobreak
%%LINE2
\gll
o-mu  ru-funjo  \textbf{ha}-mer-ire=\textbf{yo}  e-bi-himba\\
%%LINE3
\AUG{}-18  11-swamp  16\SM{}-sprout-\PFV{}=23  \AUG{}-8-bean\\
%%TRANS1
\glt
‘Beans sprout in the swamp.’\\
%%TRANS2
%%EXEND


\z

%%EAX
\ea
%%JUDGEMENT
%%LABEL
%%CONTEXT
%%LINE1
(Omu kibira) nim\textbf{pa}kûnda.\\
%%LINE2
\gll
o-mu  ki-bira  ni-n-\textbf{ha}-kund-a\\
%%LINE3
\AUG{}-18  7-forest  \IPFV{}-1\SG{}.\SM{}-16\OM{}-like-\FV{}\\
%%TRANS1
\glt
‘(In the forest) I like (it) there.’\\
%%TRANS2
%%EXEND


\z

%%EAX
\ea
%%JUDGEMENT
%%LABEL
\label{bkm:Ref113444066}
\label{bkm:Ref127966346}
%%CONTEXT
%%LINE1
%%LINE2
\gll
O-ku-zímu  ti-\textbf{há}-ri=\textbf{yó}  ky-erérezi.\\
%%LINE3
\AUG{}-17-underground  \NEG{}-16\SM{}-be=23{}  7-light\\
%%TRANS1
\glt
`Underground there is no light.’\\
%%TRANS2
%%EXEND

 \citep[144]{Asiimwe2014}
\z

Typical locative inversion in Rukiga needs formal locative marking on the locative noun, as shown in \xref{bkm:Ref135655452} for class 18 \textit{omu-muti} ‘in the tree’ (and many examples below). 

%%EAX
\ea
%%JUDGEMENT
%%LABEL
\label{bkm:Ref135655452}
%%CONTEXT
%%LINE1
Omu mut’ ómwo niharááramú enyonyi.\\
%%LINE2
\gll
o-mu  mu-ti  omwo  ni-ha-raar-a=mu  e-nyonyi\\
%%LINE3
\AUG{}-18  3-tree  18.\DEM{}.\MED{}  \IPFV{}-16\SM{}-sleep-\FV{}=18  \AUG{}-10.birds\\
%%TRANS1
\glt
‘Birds sleep in that tree.’\\
%%TRANS2
%%EXEND


\z


Nevertheless, we also find examples with a formally unmarked locative noun in initial position, as in \xref{bkm:Ref116293261}. Note that the subject marker here is still in the locative class 16 (to be distinguished from semantic locative inversion) and there is a locative enclitic (to be distinguished from Default Agreement Inversion). 


%%EAX
\ea
%%JUDGEMENT
%%LABEL
\label{bkm:Ref116293261}
%%CONTEXT
%%LINE1
Ishomer’ éeri nihegyéramw’ ábántu bakúru.\\
%%LINE2
\gll
e-i-shomero  e-ri  ni-\textbf{ha}-egyer-a=\textbf{mu}  a-ba-ntu  ba-kuru\\
%%LINE3
\AUG{}-5-school  \DEM{}-5.\PROX{}  \IPFV{}-{}16\SM{}-learn-\FV{}=18  \AUG{}-2-people  2-big\\
%%TRANS1
\glt
‘Older people study at this school.’\\
%%TRANS2
lit. ‘This school studies older people.’

%%EXEND

\z


So-called Semantic Locative Inversion, where the preverbal NP is semantically locative but is not formally marked as such and subject marking agrees with the preverbal locative \citep{MartenvanderWal2014,Buell2007}, is attested only with a restricted number of predicates and in specific circumstances (that are yet to be determined precisely); some examples are given in \xxref{bkm:Ref116293263}{bkm:Ref151453758}. A locative enclitic is obligatorily present on the verb in both types of locative inversion, in these examples =\textit{mu} and \textit{=ho}.

%%EAX
\ea
%%JUDGEMENT
%%LABEL
\label{bkm:Ref116293263}
%%CONTEXT
%%LINE1
Ebicére bikabá nibituur’ ómu kidiba kiríkushangw’ ómwihamba ekidiba kirimw’ ámíizi.\\
%%LINE2
\gll
e-bi-cere  bi-ka-b-a  ni-bi-tuur-a  o-mu  ki-diba  ki-riku-shang-w-a   o-mu  i-hamba,  e-ki-diba  ki-ri=\textbf{mu}  a-ma-izi \\
%%LINE3
\AUG{}-8-frog  8\SM{}-\F{}.\PST{}-be-\FV{}  \IPFV{}-8\SM{}-live-\FV{}  \AUG{}-18  7-pond  7\RM{}-\IPFV{}-find-\PASS{}-\FV{}   \AUG{}-18  5-forest  \AUG{}-7-pond  7\SM{}-{}be=18  \AUG{}-6-water \\
%%TRANS1
\glt
`The frogs used to stay in a pond in a forest. There used to be water in the pond.’\\
%%TRANS2
%%EXEND

%%EAX
\ex
%%JUDGEMENT
%%LABEL
%%CONTEXT
%%LINE1
%%LINE2
\gll
O-mw-enda  gw-a-z-a=\textbf{mu}  o-bu-rofa.\\
%%LINE3
\AUG{}-3-cloth  3-\N{}.\PST{}-go-\FV{}=18  \AUG{}-14-dirt\\
%%TRANS1
\glt
‘Dirt has gone into the cloth.’ / ‘The cloth has become dirty.’\\
%%TRANS2
%%EXEND

%%EAX
\ex
%%JUDGEMENT
%%LABEL
%%CONTEXT
%%LINE1
Orutookye rumeziremw’~ámóozi.\\
%%LINE2
\gll
o-ru-tookye  ru-mer-ire=\textbf{mu}~  a-ma-ozi\\
%%LINE3
\AUG{}-11-banana.plantation  11\SM{}-germinate-\PFV{}=18  \AUG{}-6-pumkin\\
%%TRANS1
\glt
‘Pumpkins germinated in the banana plantation.’ (nobody planted them there)\\
%%TRANS2
%%EXEND

%%EAX
\ex
%%JUDGEMENT
%%LABEL
\label{bkm:Ref151453758}
%%CONTEXT
%%LINE1
%%LINE2
\gll
E-meezá  y-aa-yaatik-a=\textbf{ho}  ámá-izi.\\
%%LINE3
\AUG{}-9.table  9\SM{}-\N{}.\PST{}-pour-\FV{}=16  \AUG{}-6-water\\
%%TRANS1
\glt
‘Water is poured on the table.’\\
%%TRANS2
%%EXEND


\z


Formal locative inversion is only found with intransitive predicates (both unaccusative \xref{bkm:Ref113445288} and unergative \xref{bkm:Ref113445298}) and passivised predicates \xref{bkm:Ref113445357}; transitive predicates are not accepted in inversion constructions \xref{bkm:Ref113445323}; these are systematically passivised to ameliorate the attempted construction, as in \xref{bkm:Ref113445357}. We translate the sentences into idiomatic English, but note that this reflects only the basic content and not the information structure.\largerpage

%%EAX
\ea
%%JUDGEMENT
[]{
%%LABEL
\label{bkm:Ref113445288}
%%CONTEXT
unaccusative:\\
%%LINE1
Omu nj’óomu hagwiremw’ ómugurúsi.\\
%%LINE2
\gll
o-mu  n-ju  o-mu  ha-gw-ire=mu  o-mu-gurusi\\
%%LINE3
\AUG{}-18  9.house  \AUG{}-18{}  16\SM{}-fall-\PFV{}=18  \AUG{}-1-old.man\\
%%TRANS1
\glt
‘In this house an old man fell.’\\
%%TRANS2
}
%%EXEND

%%EAX
\ex
%%JUDGEMENT
[]{
%%LABEL
\label{bkm:Ref113445298}
%%CONTEXT
unergative:\\
%%LINE1
Omu rufûnjo hamezirey’ ébihîmba.\\
%%LINE2
\gll
o-mu  ru-funjo  ha-mer-ire=yo  e-bi-himba\\
%%LINE3
\AUG{}-18  11-swamp  16\SM{}-sprout-\PFV{}=23  \AUG{}-8-bean\\
%%TRANS1
\glt
‘In the swamp beans germinated.’\\
%%TRANS2
}
%%EXEND


\z

\ea[]{
\label{bkm:Ref113445323}
transitive:\\}
%%EAX
\sn
%%JUDGEMENT
[*]{
%%LABEL
%%CONTEXT
%%LINE1
Omu musiri habingiremu abahingi enyonyi / enyonyi abahingi.\\
%%LINE2
\gll
o-mu  mu-siri  ha-bing-ire=mu  a-ba-hingi   e-nyonyi \\
%%LINE3
\AUG{}-18  3-field  16\SM{}-chase-\PFV{}=18  \AUG{}-2-farmer  \AUG{}-10.bird\\
%%TRANS1
\glt
‘On the field the farmers chased the birds.’\\
%%TRANS2
}
%%EXEND


\z

%%EAX
\ea
%%JUDGEMENT
[]{
%%LABEL
\label{bkm:Ref113445357}
%%CONTEXT
passive of transitive:\\
%%LINE1
Omu musiri habingirwemw’ ényonyi (*abahíngi).\\
%%LINE2
\gll
o-mu  mu-siri  ha-bing-w-ire=mu  e-nyonyi  a-ba-hingi\\
%%LINE3
\AUG{}-18  3-field  16\SM{}-chase-\PASS{}-\PFV{}=18  \AUG{}-10.bird  \AUG{}-2-farmer\\
%%TRANS1
\glt
`From the garden the birds were chased (*by farmers).’\\
%%TRANS2
}
%%EXEND


\z

The preverbal locatives in inversion constructions function as true subjects, as they can be relativised using the subject relative strategy \xref{bkm:Ref117062915}\footnote{See \citet{Asiimwe2019} for a detailed analysis of the syntax of relative clauses in Runyankore-Rukiga.} (even if this is not used naturally, it is judged fully grammatical), and because it triggers subject marking in complex tenses, i.e. on both the auxiliary and the lexical verb \xref{bkm:Ref117063055}.

%%EAX
\ea
%%JUDGEMENT
%%LABEL
\label{bkm:Ref117062915}
%%CONTEXT
%%LINE1
Omu nj’ óomw’ ahaagwamw’ ómugurúsi, harimw’ émbeba.\\
%%LINE2
\gll
  o-mu  n-ju  o-mu  a-ha-aa-gw-a=mu    o-mu-gurusi  ha-ri=mu  e-m-beba\\
%%LINE3
\AUG{}-18  9-house  \DEM{}-18.\PROX{}  \AUG{}-16.\REL{}-\N{}.\PST{}-fall-\FV{}=18  \AUG{}-1-old.man  16\SM{}-be=18  \AUG{}-9-rat\\
%%TRANS1
\glt 
`In this house where an old man fell, there is a rat.'\\
%%TRANS2
%%EXEND

\z

%%EAX
\ea
%%JUDGEMENT
%%LABEL
\label{bkm:Ref117063055}
%%CONTEXT
%%LINE1
Omu katáre \textbf{ha}kabá \textbf{ha}gwiremw’ ómukázi.\\
%%LINE2
\gll
o-mu  ka-tare  ha-ka-b-a  ha-gw-ire=mu  o-mu-kazi\\
%%LINE3
\AUG{}-18  12-market  16\SM{}-\F{}.\PST{}-be-\FV{}  16\SM{}-fall-\PFV{}=18  \AUG{}-1-woman\\
%%TRANS1
\glt
‘In the market a woman had fallen.’\\
%%TRANS2
%%EXEND


\z

Locative inversion can be used for narrow focus on the postverbal logical subject, as shown for the question-answer pairs in \xref{bkm:Ref111455291} and (\ref{bkm:Ref111455300:a}--\ref{bkm:Ref111455300:c}), or in thetic\slash presentational contexts, as in \xref{bkm:Ref111455300:c} in answer to \xref{bkm:Ref111455300:b}.

\ea
\label{bkm:Ref111455291}
%%EAX
\ea
%%JUDGEMENT
%%LABEL
%%CONTEXT
%%LINE1
Omu mut’ óomwo niharááramú ki?\\
%%LINE2
\gll
o-mu  mu-ti  omwo  ni-ha-raar-a=mu  ki\\
%%LINE3
\AUG{}-18  3-tree  18-\DEM{}.\MED{}  \IPFV{}-16\SM{}-sleep-\FV{}=18  what\\
%%TRANS1
\glt
‘What sleeps in that tree?\\
%%TRANS2
%%EXEND


 %%EAX
\ex
%%JUDGEMENT
%%LABEL
%%CONTEXT
%%LINE1
Omu mut’ óomwo niharáaramw’ ényonyi.\\
%%LINE2
\gll
o-mu  mu-ti  omwo  ni-ha-raar-a=mu  e-nyonyi\\
%%LINE3
\AUG{}-18  3-tree  18.\DEM{}.\MED{}  \IPFV{}-16\SM{}-sleep-\FV{}=18  \AUG{}-10.bird\\
%%TRANS1
\glt
‘Birds sleep in that tree.’\\
%%TRANS2
%%EXEND


\z
\z

\ea
\label{bkm:Ref111455300}

%%EAX
\ea
%%JUDGEMENT
%%LABEL
\label{bkm:Ref111455300:a}
%%CONTEXT
%%LINE1
Aha rutindo haarabahó ki?\\
%%LINE2
\gll
a-ha  ru-tindo  ha-aa-rab-a=ho  ki\\
%%LINE3
\AUG{}-16  11-bridge  16\SM{}-\N{}.\PST{}-pass-\FV{}=16  what\\
%%TRANS1
\glt
‘What has passed on the bridge?\\
%%TRANS2
%%EXEND

 %%EAX
\ex
%%JUDGEMENT
%%LABEL
\label{bkm:Ref111455300:b}
%%CONTEXT
%%LINE1
Orutindo rwaba ki?\\
%%LINE2
\gll
o-ru-tindo  ru-aa-b-a  ki\\
%%LINE3
\AUG{}-11-bridge  11\SM{}-\N{}.\PST{}-be-\FV{}  what\\
%%TRANS1
\glt
‘What has happened to the bridge?’\\
%%TRANS2
%%EXEND


%%EAX
\ex
%%JUDGEMENT
%%LABEL
\label{bkm:Ref111455300:c}
%%CONTEXT
%%LINE1
Aha rutindo haarabah’ émotoka nyîngi.\\
%%LINE2
\gll
a-ha  ru-tindo  ha-aa-rab-a=ho  e-motoka  ny-ingi\\
%%LINE3
\AUG{}-16  11-bridge  16\SM{}-\N{}.\PST{}-pass-\FV{}=16  \AUG{}-10.car  10-many\\
%%TRANS1
\glt
‘On the bridge many cars have passed.’\\
%%TRANS2
%%EXEND


\z
\z

\subsubsection{Default Agreement Inversion (DAI)}
\label{bkm:Ref116291412}
More frequently used than locative inversion is default agreement inversion (DAI), where the subject marker is in class 16, and there is no locative enclitic on the verb. Nothing needs to precede the verb in DAI. As with LI, DAI also requires the tonally reduced form of the verb, as illustrated in \xref{bkm:Ref116293644}.

\ea
\label{bkm:Ref116293644}
%%EAX
\ea
%%JUDGEMENT
[]{
%%LABEL
%%CONTEXT
%%LINE1
%%LINE2
{\gll
Ha-a-shohor-a  Píta.  \\
%%LINE3
16\SM{}-\N{}.\PST{}-move.out-\FV{}  1.Peter\\
}\jambox*{[with TR]}
%%TRANS1
\glt
‘Peter has left.’ / ‘It is Peter who has moved out.’\\
%%TRANS2
}
%%EXEND


 %%EAX
\ex
%%JUDGEMENT
[*]{
%%LABEL
%%CONTEXT
%%LINE1
%%LINE2
{\gll
Há-á-shohor-a  Píta.  \\
%%LINE3
16\SM{}-\N{}.\PST{}-move.out-\FV{}  1.Peter\\
}\jambox*{[no TR]}
%%TRANS1
\glt
‘Peter has left.’\\
%%TRANS2
}
%%EXEND


\z
\z

DAI can be used when introducing a new referent, as in \xref{bkm:Ref111455458} and \xref{bkm:Ref111455460}, in content questions regarding the postverbal logical subject and answers to those questions, as in \xref{bkm:Ref111455483}, as well as when contrasting or correcting an alternative referent, shown in \xref{bkm:Ref113452874}.\pagebreak

%%EAX
\ea
%%JUDGEMENT
%%LABEL
\label{bkm:Ref111455458}
%%CONTEXT
(Context: Out-of-the-blue statement.)\\
%%LINE1
Harihó ekintú kyangy’ékíbuzire.\\
%%LINE2
\gll
ha-ri=ho  e-ki-ntu  ki-angye  e-ki-bur-ire\\
%%LINE3
16\SM{}-be=16  \AUG{}-7-thing  7-\POSS{}.1\SG{}  \AUG{}-7\SM{}.\REL{}-get.lost-\PFV{}\\
%%TRANS1
\glt
‘There is something that I lost.’\\
%%TRANS2
%%EXEND

%%EAX
\ex
%%JUDGEMENT
%%LABEL
\label{bkm:Ref111455460}
%%CONTEXT
%%LINE1
Hiij’ ómuntu.\\
%%LINE2
\gll
ha-aa-ij-a  o-mu-ntu\\
%%LINE3
16\SM{}-\N{}.\PST{}-come-\FV{}  \AUG{}-1-person\\
%%TRANS1
\glt
‘Someone has come.’\\
%%TRANS2
%%EXEND
\ex
\label{bkm:Ref111455483}
 (Context: You see people running and gathering, and you wonder what is going on.)\footnote{Although in this context, a basic cleft is preferred: \textit{N’oha owaija}?}

%%EAX
\ea
%%JUDGEMENT
%%LABEL
%%CONTEXT
%%LINE1
Haij’ óoha?\\
%%LINE2
\gll
ha-aa-ij-a  o-ha\\
%%LINE3
16\SM{}-\N{}.\PST{}-come-\FV{}  1-who\\
%%TRANS1
\glt
‘Who has come?’\\
%%TRANS2
%%EXEND


%%EAX
\ex
%%JUDGEMENT
%%LABEL
%%CONTEXT
%%LINE1
Haija purésidenti.\\
%%LINE2
\gll
ha-aa-ij-a  puresidenti\\
%%LINE3
16\SM{}-\N{}.\PST{}-come-\FV{}  1.president\\
%%TRANS1
\glt
‘The President has come.’\\
%%TRANS2
%%EXEND


\z
\z

%%EAX
\ea
%%JUDGEMENT
%%LABEL
\label{bkm:Ref113452874}
%%CONTEXT
(Is it Ron who left?)\\
%%LINE1
Ingaaha,  haagyenda  Jack.\\
%%LINE2
\gll
ngaaha,  ha-aa-gyend-a  Jack.\\
%%LINE3
no  16\SM{}-\N.\PST{}-go-\FV{}  1.Jack\\
%%TRANS1
\glt
‘No. It is Jack who has gone.’\\
%%TRANS2
%%EXEND


\z

Furthermore, the postverbal logical subject can be modified by ‘only’, as shown in \xref{bkm:Ref111455511}, and also by ‘also/even’ \xref{bkm:Ref111455562}. This suggests that the postverbal logical subject may be in focus, but is not inherently interpreted as exclusive: the interpretation as ‘also/even’ means that the proposition is true for other referents besides Daniel, which means that the construction in which it occurs (the DAI) does not come with an inherently exclusive focus interpretation.

%%EAX
\ea
%%JUDGEMENT
%%LABEL
\label{bkm:Ref111455511}
%%CONTEXT
%%LINE1
Haagambíre Dániel wénka.\\
%%LINE2
\gll
ha-aa-gamb-ire  Daniel  w-enka\\
%%LINE3
16\SM{}-\N{}.\PST{}-talk-\PFV{}  1.Daniel  1-only\\
%%TRANS1
\glt
‘Only Daniel talked.’\\
%%TRANS2
%%EXEND


\z

%%EAX
\ea
%%JUDGEMENT
%%LABEL
\label{bkm:Ref111455562}
%%CONTEXT
%%LINE1
Haabyama ná Dániel.\\
%%LINE2
\gll
ha-aa-byam-a  na  Daniel\\
%%LINE3
16\SM{}-\N{}.\PST{}-{}sleep-\FV{}  and  1.Daniel\\
%%TRANS1
\glt
‘Even/also Daniel slept.’\\
%%TRANS2
%%EXEND


\z

In neither LI nor DAI can an object marker be used, whether for the Theme or the Agent, as shown in \xref{bkm:Ref111455934} and \xref{bkm:Ref111455935}, respectively.

%%EAX
\ea
%%JUDGEMENT
[*]{
%%LABEL
\label{bkm:Ref111455934}
%%CONTEXT
%%LINE1
Omu musiri ha\textbf{zi}bingiremu abahingi.\\
%%LINE2
\gll
o-mu  mu-siri  ha-zi-bing-ire=mu  a-ba-hingi\\
%%LINE3
\AUG{}-18  3-field  16\SM{}-10\OM{}-chase-\PFV{}=18  \AUG{}-2-farmers\\
%%TRANS1
\glt
‘On the field chased the farmers them.’\\
%%TRANS2
}
%%EXEND


\z

%%EAX
\ea
%%JUDGEMENT
[*]{
%%LABEL
\label{bkm:Ref111455935}
%%CONTEXT
%%LINE1
Aha rutindo haa\textbf{zi}rabahó.\\
%%LINE2
\gll
a-ha  ru-tindo  ha-aa-zi-rab-a=ho\\
%%LINE3
\AUG{}-16  11-bridge  16\SM{}-\N{}.\PST{}-10\OM{}-pass-\FV{}=16\\
%%TRANS1
\glt
‘On the bridge have passed they.’\\
%%TRANS2
}
%%EXEND


\z

\subsubsection{Agreeing inversion}

Rukiga also seems to allow agreeing inversion, whereby the subject marker on the verb agrees with the postverbal subject. It can be difficult to distinguish this from a verb with a right-dislocated subject as in an afterthought (see further in \sectref{bkm:Ref100082426}). However, for afterthoughts we would expect a pause between the verb and the subject, and an indefinite interpretation of the postverbal subject would be unacceptable for an afterthought. The fact that the postverbal subject in \xref{bkm:Ref111454720} is not preceded by a pause, and there is liaison between verb and subject, suggests that this is an instance of agreeing inversion, and the indefinite interpretation is unacceptable for a dislocated phrase. Example \xref{bkm:Ref113453625} is felicitous in a thetic context (but not with contrastive focus on the subject), and could (with the right prosody) also be used as an afterthought. Note that the tones on the verb remain as in SV order, that is, the verb does not undergo TR (e.g. see \xref{bkm:Ref113453625:a}), unlike in default agreement inversion \citep[see][]{vanderWalAsiimwe2020}.

\ea
\label{bkm:Ref111454720}
%%EAX
\ea
%%JUDGEMENT
[]{
%%LABEL
%%CONTEXT
%%LINE1
Yííj’ ómuntu.    \jambox*{[no TR]}
%%LINE2
\gll
a-aa-ij-a  o-mu-ntu\\
%%LINE3
1\SM{}-\N{}.\PST{}-come-\FV{}  \AUG{}-1-person\\
%%TRANS1
\glt
‘Someone/a person has come.’\\
%%TRANS2
}
%%EXEND

%%EAX
\ex
%%JUDGEMENT
[*]{
%%LABEL
%%CONTEXT
%%LINE1
Yiij’ ómuntu.    \jambox*{[with TR]}
%%LINE2
\gll
a-aa-ij-a  o-mu-ntu\\
%%LINE3
1\SM{}-\N{}.\PST{}-come-\FV{}  \AUG{}-1-person\\
%%TRANS1
\glt
‘Someone/ a person has  come.’\\
%%TRANS2
}
%%EXEND


\z
\z

\ea
\label{bkm:Ref113453625}
%%EAX
\ea
%%JUDGEMENT
[]{
%%LABEL
\label{bkm:Ref113453625:a}
%%CONTEXT
%%LINE1
Yááyéésyamur’ émbúzi (\textsuperscript{\#}tí nte).  \jambox*{[no TR]}
%%LINE2
\gll
e-aa-esyamur-a  e-n-buzi  ti  n-te\\
%%LINE3
9\SM{}-\N{}.\PST{}-sneeze-\FV{}  \AUG{}-9-goat  \NEG{}  9-cow\\
%%TRANS1
\glt
‘The goat sneezed (\textsuperscript{\#}not the cow).’ (thetic)\\
%%TRANS2
‘It did sneeze, the goat.’ (right-dislocated)
}
%%EXEND

%%EAX
\ex
%%JUDGEMENT
[*]{
%%LABEL
\label{bkm:Ref113453625:b}
%%CONTEXT
%%LINE1
Yaayeesyamur’ émbúzi.  \jambox*{[with TR]}
%%LINE2
\gll
e-aa-esyamur-a  e-m-buzi\\
%%LINE3
9\SM{}-\N{}.\PST{}-sneeze-\FV{}  \AUG{}-9-goat\\
%%TRANS1
\glt
‘The goat sneezed.’\\
%%TRANS2
}
%%EXEND


\z
\z

Further research may elucidate the exact properties and use of Rukiga agreeing inversion.

\subsubsection{Passive}
\label{bkm:Ref111461978}
The passive resembles patient inversion in Rukiga, because the Agent can be present without further marking (no ``by-phrase"), as shown in \xref{bkm:Ref113453727:b}. The difference is that in the passive the Agent may be omitted, whereas in the (other) inversion constructions it is obligatorily present \xref{bkm:Ref113453727:c}. Furthermore, the verb is morphologically marked as passive by the extension -\textit{(g)w}- (with its allomorphs such as -\textit{ibw}-, -\textit{ebw}-, -\textit{ew}-).

\ea
\label{bkm:Ref113453727}
%%EAX
\ea
%%JUDGEMENT
%%LABEL
\label{bkm:Ref113453727:a}
%%CONTEXT
%%LINE1
Abakázi baahing’ ómusiri.\\
%%LINE2
\gll
a-ba-kazi  ba-aa-hing-a  o-mu-siri\\
%%LINE3
\AUG{}-2-woman  2\SM{}-\N{}.\PST{}-dig-\FV{}  \AUG{}-3-field\\
%%TRANS1
\glt
‘(The) women dug a/the field.’\\
%%TRANS2
%%EXEND

%%EAX
\ex
%%JUDGEMENT
%%LABEL
\label{bkm:Ref113453727:b}
%%CONTEXT
%%LINE1
Omusíri gwahingwá (abakázi).    \jambox*{[passive]}
%%LINE2
\gll
o-mu-siri  gu-aa-hing-w-a  a-ba-kazi\\
%%LINE3
\AUG{}-3-field  3\SM{}-\N{}.\PST{}-dig-\PASS{}-\FV{}  \AUG{}-2-women\\
%%TRANS1
\glt
‘The field was dug (by women).’\\
%%TRANS2
%%EXEND

%%EAX
\ex
%%JUDGEMENT
%%LABEL
\label{bkm:Ref113453727:c}
%%CONTEXT
%%LINE1
Omu musiri hakarabamw’ *(ábakázi).  \jambox*{[LI]}
%%LINE2
\gll
o-mu  mu-siri  ha-ka-rab-a=mu  *(a-ba-kazi).\\
%%LINE3
\AUG{}-18{}  3-field  16\SM{}-\F{}.\PST{}-dig-\FV{}  \AUG{}-2-women\\
%%TRANS1
\glt
‘On the field the women have dug.’\\
%%TRANS2
%%EXEND


\z
\z


The preverbal referent in the passive is interpreted as topical, and the postverbal Agent may be the focus, as illustrated in the question-answer pair in \xref{bkm:Ref111462190}.

\ea
\label{bkm:Ref111462190}
%%EAX
\ea
%%JUDGEMENT
%%LABEL
%%CONTEXT
%%LINE1
Enju ya shwénkuru enkombekw’ óha?\\
%%LINE2
\gll
e-n-ju  y-a  shwenkuru  e-ka-ombek-w-a  o-ha\\
%%LINE3
\AUG{}-9-house  9-\CONN{}  1.grandfather  9\SM{}-\F{}.\PST{}-build-\PASS{}-\FV{}  1-who\\
%%TRANS1
\glt
‘Who built grandfather’s house?’\\
%%TRANS2
%%EXEND


 %%EAX
\ex
%%JUDGEMENT
%%LABEL
%%CONTEXT
%%LINE1
Ekombekwa Róbati.\\
%%LINE2
\gll
e-ka-ombek-w-a  Robati\\
%%LINE3
9\SM{}-\F{}.\PST{}-build-\PASS{}-\FV{}  1.Robert\\
%%TRANS1
\glt
‘It was built by Robert.’\\
%%TRANS2
%%EXEND


\z
\z

Apart from the passive morpheme -\textit{(g)w}, Rukiga uses the class 2 prefix \textit{ba}- in impersonal constructions, as in other Bantu languages (e.g. \citealt{KulaMarten2010} for Bemba, \citealt{vanderWal2016} for Matengo; see also other chapters in \citealt{langsci-current-book}). The impersonal \textit{ba}- is used when the Agent is unknown or unimportant, or when it needs to be kept anonymous. The construction contains no logical subject NP and the attention is on the object. The \textit{ba}- construction is structurally not a passive because the preverbal object, although promoted to the IS function of topic, has not assumed the grammatical role of subject. This can be seen in the fact that it is marked on the verb with coreferential object marker – in \xref{bkm:Ref113456530}, the preverbal object \textit{esimu yangye} ‘my phone’ is marked on the verb by the object marker \textit{gi-}, and the same for \textit{enkuuto} ‘the road’ in example \xref{bkm:Ref113456853}. Example \xref{bkm:Ref127351280} shows the impersonal interpretation of the \textit{ba}-construction.

%%EAX
\ea
%%JUDGEMENT
%%LABEL
\label{bkm:Ref113456530}
%%CONTEXT
(Context: A girl is checking her bag, removing and throwing everything down and when her friend asks what she’s doing, she replies:)\\
%%LINE1
%%LINE2
\gll
E-símu  y-angye  b-áá-gí-ib-a.\\
%%LINE3
\AUG{}-{}9.phone  9-\POSS{}.1\SG{}  2\SM{}-\N{}.\PST{}-9\OM{}-{}steal-\FV{}\\
%%TRANS1
\glt
‘They have stolen my phone.’ / ‘My phone has been stolen.’\\
%%TRANS2
%%EXEND


\z

%%EAX
\ea
%%JUDGEMENT
%%LABEL
\label{bkm:Ref127351280}
%%CONTEXT
(Context: Father comes back home and his daughter tells him that someone she does not know was looking for him.)\\
%%LINE1
%%LINE2
\gll
Taata  ba-a-b-a  ni-ba-ku-rond-a.\\
%%LINE3
1.Father  2\SM{}-\N{}.\PST{}-{}be-\FV{}  \IPFV{}-2\SM{}-2\SG{}.\OM{}-{}look.for-\FV{}\\
%%TRANS1
\glt
‘Father, they were looking for you.’ / ‘Someone was looking for you, father.’\\
%%TRANS2
%%EXEND


\z
\pagebreak
%%EAX
\ea
%%JUDGEMENT
%%LABEL
\label{bkm:Ref113456853}
%%CONTEXT
(Context: Mother asks why we have come back home late.)\\
%%LINE1
Twakyererwa kuhik’ ómuka ahabwókuba omu kugaruka twashang’ \textbf{énkuuto bagísibire}. Náhabwékyo twabanza kwétooroora.\\
%%LINE2
\gll
tu-aa-kyererw-a  ku-hika  o-mu  ka  ahabwokuba  o-mu  ku-garuka  tu-aa-shang-a  e-n-kuuto  ba-gi-sib-ire.  nahabwekyo  tu-aa-banz-a  ku-etooroora.\\
%%LINE3
1\PL{}.\SM{}-\N{}.\PST{}-{}delay-\FV{}  15-reach  \AUG{}-{}18  9.home  because  \AUG{}-{}18  15-return  1\PL{}.\SM{}-\N{}.\PST{}-{}find-\FV{}  \AUG{}-{}9-road  2\SM{}-{}9\OM{}-close-\PFV{}.{}  therefore  1\PL{}.\SM{}-\N{}.\PST{}-{}be.first-\FV{}  15-go.round\\
%%TRANS1
\glt
‘We delayed to arrive home because on our way back, we found that the road had been closed. So, we had to take a longer route.’\\
%%TRANS2
%%EXEND


\z

In this subsection, we have shown that locative inversion, default agreement inversion and to some extent agreeing inversion are available in Rukiga. In addition we briefly discussed passive construcions where the object is promoted to topic in the preverbal position while the logical subject may or may not be present in the postverbal position. We generally note that the element that appears in the preverbal position is topicalised, whereas the postverbal logical subject forms part of the new or contrasted information (whether in a thetic interpretation or as narrow focus on the subject).

\subsection{Right periphery}
\label{bkm:Ref100082426}
What is not topical, but not focal either, can appear in the right periphery. This is for example the case for any constituents that follow the IAV focus, as in \xxref{bkm:Ref111462821}{bkm:Ref113457070}. Note that object marking in this case is optional.

%%EAX
\ea
%%JUDGEMENT
%%LABEL
\label{bkm:Ref111462821}
%%CONTEXT
%%LINE1
Kááka yaa(ba)ha ky’ \textbf{ábáána}?\\
%%LINE2
\gll
kaaka  a-aa-ba-h-a  ki  a-ba-ana\\
%%LINE3
1.grandmother  1\SM{}-\N{}.\PST{}-2\OM{}-give-\FV{}  what  \AUG{}-2-child\\
%%TRANS1
\glt
‘What has grandmother given the children?’\\
%%TRANS2
%%EXEND

%%EAX
\ex
%%JUDGEMENT
%%LABEL
\label{bkm:Ref113457070}
%%CONTEXT
%%LINE1
Nitubaasá ku(bí)gura nkah’ \textbf{ébitookye}?\\
%%LINE2
\gll
ni-tu-baas-a  ku-bi-gura  nkahe  e-bi-tookye\\
%%LINE3
\IPFV{}-1\PL{}.\SM{}-be.able-\FV{}  15-8\OM{}-buy  where  \AUG{}-8-plantains\\
%%TRANS1
\glt
‘Where can we buy plantains?’\\
%%TRANS2
%%EXEND

\z


Other examples involve an afterthought, that is, a full NP “used to clarify the referent of an earlier pronoun” \citep[414]{Lopez2016}, as illustrated in \xref{bkm:Ref113460217}. These are in Rukiga preceded by a pause.

\ea
\label{bkm:Ref113460217}
%%EAX
\ea
%%JUDGEMENT
%%LABEL
%%CONTEXT
(Has Peter cooked posho?)\\
%%LINE1
Píta yáákatéeka, \textbf{ákahúngá}.\\
%%LINE2
\gll
Pita  a-aa-ka-teek-a  a-ka-hunga\\
%%LINE3
1.Peter  1\SM{}-\N{}.\PST{}-12\OM{}-cook-\FV{}  \AUG{}-12-{}posho\\
%%TRANS1
\glt
‘Peter cooked it, posho.’\\
%%TRANS2
%%EXEND

%%EAX
\ex
%%JUDGEMENT
%%LABEL
%%CONTEXT
(Is the posho well cooked?)\\
%%LINE1
%%LINE2
\gll
Ka-sy-á  gye  \textbf{a-ka-hunga}\\
%%LINE3
12\SM{}-be.well.cooked-\FV{}  well \AUG{}-12-posho\\
%%TRANS1
\glt
‘It is well cooked, the posho.’\\
%%TRANS2
%%EXEND

\z
\z

A secondary topic may also appear in the right periphery of the sentence \citep{AsiimwevanderWal2021}, illustrated in \xref{bkm:Ref116294567}. Note that this secondary topic is marked by the contrastive topic marker \textit{go}. The prosody indicates that it is not right-dislocated, as \textit{amaizi go} cannot be preceded by a pause/prosodic break. Had the contrastive marker \textit{go} been absent, a pause would have been acceptable in that position. Another difference is that in the presence of -\textit{o}, both primary and secondary topics require co-indexing on the verb (compare to the optional object marking in \xref{bkm:Ref111462821} for example).


%%EAX
\ea
%%JUDGEMENT
%%LABEL
\label{bkm:Ref116294567}
%%CONTEXT
\citep[9]{AsiimwevanderWal2021}\\
(Did the cows drink the water?)\\
%%LINE1
Ente záá*(gá)nyw’ ámíizi go.\\
%%LINE2
\gll
e-n-te  zi-aa-ga-nyw-a  a-ma-izi  ga-o\\
%%LINE3
\AUG{}-10-cow  10\SM{}-\N{}.\PST{}-6\OM{}-drink-\FV{}  \AUG{}-6-water  6-\CM{}\\
%%TRANS1
\glt
`The cows, as for the water, they have drunk it.’\\
%%TRANS2
%%EXEND


\z


To summarise the word order properties of Rukiga, the preverbal domain preferably contains topics and may host non-topical subjects. Furthermore, Rukiga allows multiple topics and contrastive topics in the preverbal domain. However, focused elements are not allowed to appear preverbally. Interrogative constituents need to appear in IAV position (or in a cleft), and other focused constituents, for example some adverbs, may appear non-adjacent to the verb. Non-focal/non-topical constituents appear postverbally as well, exemplified by thetic subject inversions, afterthoughts, and secondary topics in the right periphery. Relevant to the marking of contrastive topics, in the next section, we discuss particle -\textit{o} present in Rukiga as a contrastive topic marker which also performs other pragmatic roles.

\section{Particle \textit{-o}}\label{sec:particle-o}
Although Rukiga does not have dedicated focus or topic particles, it has a morphological particle that marks contrastive topics, as we discuss in \citet{AsiimwevanderWal2021}. Its presence in a sentence triggers an interpretational difference as indicated in \xref{bkm:Ref111495084}. This particle is also found in Kîîtharaka and Kirundi with similar functions, see \textcite{chapters/kiitharaka} and \textcite{chapters/kirundi}.

\ea
\label{bkm:Ref111495084}
%%EAX
\ea
%%JUDGEMENT
%%LABEL
%%CONTEXT
%%LINE1
Enjojo  záija.\\
%%LINE2
\gll
e-n-jojo  zi-aa-ij-a.\\
%%LINE3
\AUG{}-10-elephant  10\SM{}-\N{}.\PST{}-come-\FV{}\\
%%TRANS1
\glt
'(The) elephants have come.'\\
%%TRANS2
%%EXEND

%%EAX
\ex
%%JUDGEMENT
%%LABEL
%%CONTEXT
%%LINE1
Enjojo  \textbf{zó}  záija.\\
%%LINE2
\gll
e-n-jojo  z-o  zi-aa-ij-a.\\
%%LINE3
\AUG{}-10-elephant  10-\CM{}  10\SM{}-\N{}.\PST{}-come-\FV{}\\
%%TRANS1
\glt
‘As for the elephants, they have come (maybe the antelopes, the zebras, the lions etc. have not shown up).’\\
%%TRANS2
%%EXEND


\z
\z

According to \citet{Taylor1985}, the particle encodes contrastiveness or mere emphasis such as in \xref{bkm:Ref113545408:b}, as the particle \textit{ko} emphasises ‘pen’ in a contrastive manner.\largerpage[2.25]

\ea
\label{bkm:Ref113545408} \citep[74, glosses adapted]{Taylor1985}
%%EAX
\ea
%%JUDGEMENT
%%LABEL
\label{bkm:Ref113545408:a}
%%CONTEXT
%%LINE1
%%LINE2
\gll
Y-aa-reet-a  é-ki-tabo,  a-ka-cumu  ka-buz-ire.\\
%%LINE3
1\SM{}-\N{}.\PST{}-bring-\FV{}  \AUG{}-7-book  \AUG{}-12-pen  12\SM{}-lose-\PFV{}\\
%%TRANS1
\glt
‘He brought the book and lost his pen.’ (sic)\\
%%TRANS2
%%EXEND

%%EAX
\ex
%%JUDGEMENT
%%LABEL
\label{bkm:Ref113545408:b}
%%CONTEXT
%%LINE1
%%LINE2
\gll
Y-aa-reet-a  é-ki-tabo,  a-ka-cumu  \textbf{k-ó}  ka-buz-ire.\\
%%LINE3
1\SM{}-\N{}.\PST{}-bring-\FV{}  \AUG{}-7-book  \AUG{}-12-pen  12-\CM{}  12\SM{}-lose-\PFV{}\\
%%TRANS1
\glt
‘He brought the book, but the pen is lost.’\\
%%TRANS2
%%EXEND


\z
\z


In addition to Taylor’s observation, \citet{Asiimwe2014} argues that the particle triggers a contrastive reading in a sentence between referents that are familiar \xref{bkm:Ref111495139}.

%%EAX
\ea
%%JUDGEMENT
%%LABEL
\label{bkm:Ref111495139}
%%CONTEXT
(Adapted from \citealt[236]{Asiimwe2014})\\
(Context: To counter the claim that nobody was invited, not even the teachers.)\\
%%LINE1
Abashomésa \textbf{bó} twábéeta.\\
%%LINE2
\gll
a-ba-shomesa  ba-o  tu-aa-ba-et-a\\
%%LINE3
\AUG{}-2-teacher  2-\CM{}  1\PL{}.\SM{}-\N{}.\PST{}-2\OM{}-call-\FV{}\\
%%TRANS1
\glt
`As for the teachers, we have invited them.’\\
%%TRANS2
%%EXEND


\z

Without wanting to repeat the whole description and analysis in \citet{AsiimwevanderWal2021}, we will summarise the main morphosyntactic properties in \sectref{bkm:Ref135664271}, then proceed to the interpretation in \sectref{bkm:Ref135664287} and finally present its combination with \textit{na} ‘and’ in \sectref{bkm:Ref135664306}.

\subsection{Morphosyntactic properties of the particle}
\label{bkm:Ref135664271}

The particle stands as an independent morpheme and like all the other nominal elements, it is marked for noun class as \tabref{tab:nyn-morph-o} shows.

\begin{table}[h]
\begin{tabular}{l@{~}llll}
\lsptoprule
\multicolumn{2}{l}{Noun class\,+\,prefix} & Example noun & Gloss & Particle\\
\midrule
1  & mu- & omuhara & girl & we\\
2  & ba- & abahara & girls & bo\\
3  & mu- & omuyembe & mango & gwo/gwe\\
4  & mi- & emiyembe & mangoes & yo\\
5  & ri-/i- & eihuri & egg & ryo\\
6  & ma- & amahuri & eggs & go\\
7  & ki- & ekihumi & granary & kyo\\
8  & bi- & ebihumi & granaries & byo\\
9  & n- & ente & cow & yo\\
10 & n- & ente & cows & zo\\
11 & ru- & orushare & calabash & rwo/rwe\\
12 & ka- & akatare & market & ko\\
13 & tu- & oturo & sleep & two/twe\\
14 & bu- & obumanzi & bravery & bwo/bwe\\
15 & ku- & okuguru & leg & kwo/kwe\\
16 & ha- & aheeru & outside & ho\\
17 & ku- & okuzimu & hell & yo\\
18 & mu- & omwiguru & in heaven & yo/ho/mwo/mwe\\
\lspbottomrule
\end{tabular}
\caption{Morphological structure of the particle -\textit{o} \citep[4]{AsiimwevanderWal2021}}
\label{tab:nyn-morph-o}
\end{table}

The particle typically follows the noun it refers to, as in \xref{bkm:Ref111495200:a}. However, it is free to move to the prenominal position, seen in \xref{bkm:Ref111495200:b}. It can also appear after the verb as exemplified in \xref{bkm:Ref111495200:c}.

\pagebreak
\ea
\label{bkm:Ref111495200} 
\citep[5]{AsiimwevanderWal2021}
\settowidth\jamwidth{[Post-N]}
%%EAX
\ea
%%JUDGEMENT
%%LABEL
\label{bkm:Ref111495200:a}
%%CONTEXT
%%LINE1
%%LINE2
{\gll
E-n-te  \textbf{z-ó}  \'{A}mos  n-aa-zá  ku-zi-ríis-a.    \\
%%LINE3
\AUG{}-{}10.cows  10-\CM{}  1.Amos  \IPFV{}-1\SM{}-{}go  15-10\OM{}-{}feed-\FV{}\\
}\jambox*{[Post-N]}
%%TRANS1
\glt
‘For the sake of the cows, Amos will graze them.’\\
%%TRANS2
%%EXEND


%%EAX
\ex
%%JUDGEMENT
%%LABEL
\label{bkm:Ref111495200:b}
%%CONTEXT
%%LINE1
%%LINE2
{\gll
…kwónka  \textbf{z-ó}  e-n-taama  z-áá-nyw-a.    \\
%%LINE3
but  10-\CM{}  \AUG{}-{}10-sheep  10\SM{}-\N{}.\PST{}-{}drink-\FV{}\\
}\jambox*{[Pre-N]}
%%TRANS1
\glt
‘…but as for the sheep, they drank’\\
%%TRANS2
%%EXEND


%%EAX
\ex
%%JUDGEMENT
%%LABEL
\label{bkm:Ref111495200:c}
%%CONTEXT
%%LINE1
%%LINE2
{\gll
E-n-te  ni-n-zá  ku-zi-ríis-a  \textbf{z-o}.  \\
%%LINE3
\AUG{}-{}10-cow  \IPFV{}-1\SG{}.\SM{}-{}go  15-10\OM{}-{}feed-\FV{}  10-\CM{}\\
}\jambox*{[Post-V]}
%%TRANS1
\glt
‘As for the cows, I will graze them.’\\
%%TRANS2
%%EXEND


\z
\z

The particle can be used pronominally, for highly accessible referents, with just a subject \xref{bkm:Ref113699913:a} or an object marker \xref{bkm:Ref113699913:b}. Note that although the particle is free to appear in the postverbal position, in the examples given \xref{bkm:Ref113699913} below  it is in the preverbal position because topics are typically marked in the preverbal position.


\ea
\label{bkm:Ref113699913}
\citep[5]{AsiimwevanderWal2021}
%%EAX
\ea
%%JUDGEMENT
%%LABEL
\label{bkm:Ref113699913:a}
%%CONTEXT
%%LINE1
\textbf{Bó} baateek’ ómucéeri.    \jambox*{[\CM{} + \SM{}]}
%%LINE2
\gll
ba-o  ba-aa-teek-a  o-mu-ceeri\\
%%LINE3
2-\CM{}  {}2\SM{}-\N{}.\PST{}-cook-\FV{}  \AUG{}-{}3-rice\\
%%TRANS1
\glt
‘As for them (the women), they have cooked rice.’\\
%%TRANS2
%%EXEND


%%EAX
\ex
%%JUDGEMENT
%%LABEL
\label{bkm:Ref113699913:b}
%%CONTEXT
%%LINE1
\textbf{Gw}’ ábakázi báágutéeka.  \jambox*{[\CM{} + \OM{}]}
%%LINE2
\gll
\textbf{gu-o}  a-ba-kazi  ba-aa-gu-teek-a\\
%%LINE3
3-\CM{}  \AUG{}-{}2-woman  2\SM{}-\N{}.\PST{}-3\OM{}-{}cook-\FV{}\\
%%TRANS1
\glt
‘As for it (the rice) the women have cooked it.’\\
%%TRANS2
%%EXEND


\z
\z

The particle occurs with both arguments and adverbials, as illustrated in \xref{bkm:Ref111472803:a} for a subject NP and \xref{bkm:Ref111472803:b} for an adverbial.

\ea
\label{bkm:Ref111472803}
%%EAX
\ea
%%JUDGEMENT
%%LABEL
\label{bkm:Ref111472803:a} 
%%CONTEXT
%%LINE1
Ebihímba \textbf{byó} tíbikamezire.\\
%%LINE2
\gll
e-bi-himba  bi-ó  tí-bi-ka-mer-ire\\
%%LINE3
\AUG{}-8-bean  8-\CM{}  \NEG{}-8\SM{}-\NEG{}-germinate-\PFV{}\\
%%TRANS1
\glt
`As for the beans, they have not yet germinated.’\\
%%TRANS2
%%EXEND

%%EAX
\ex
%%JUDGEMENT
%%LABEL
\label{bkm:Ref111472803:b}
%%CONTEXT
%%LINE1
Nyómwébázyo  \textbf{bwé}  tegwîre.\\
%%LINE2
\gll
nyomwebazyo  bu-o  ti-e-gw-ire.\\
%%LINE3
yesterday  14-\CM{}  \NEG{}-9\SM{}-fall-\PFV{}\\
%%TRANS1
\glt
‘Yesterday it did not rain (it rained on other days).’\\
%%TRANS2
%%EXEND


\z
\z

In case of conjoined clauses, the particle can appear in either the first or second clause (but preferrably in the second clause and not both – that would be overdoing it), as in \xref{bkm:Ref37771777}.

\ea
\label{bkm:Ref37771777}
%%EAX
\ea
%%JUDGEMENT
%%LABEL
%%CONTEXT
(What is the woman eating and what is the man eating? + QUIS picture)\\
%%LINE1
  Omukázi arikuryá ápo, kándi \textbf{wé} omushíija arikurya’ ómunekye.\\
%%LINE2
\gll
  o-mu-kazi  a-riku-ri-a  apo  kandi  w-o  o-mu-shaija    a-riku-ri-a  o-mu-nekye\\
%%LINE3
\AUG{}-{}1-woman  1\SM{}-\IPFV{}-eat-\FV{}  9.apple  and  1-\CM{}  \AUG{}-{}1-man 1\SM{}-\IPFV{}-eat-\FV{}  \AUG{}-{}3-banana\\
%%TRANS1
\glt
  \glt ‘The woman is eating an apple while the man is eating a banana.’\\
%%TRANS2
%%EXEND

%%EAX
\ex
%%JUDGEMENT
%%LABEL
%%CONTEXT
%%LINE1
Omukázi \textbf{wé} arikuryá ápo, omushíija arikury’ ómunekye.\\
%%LINE2
\gll
  o-mu-kazi  w-o  a-riku-ri-a  apo  o-mu-shaija  a-riku-ri-a    o-mu-nekye\\
%%LINE3
\AUG{}-{}1-woman  1-\CM{}  1\SM{}-\IPFV{}-{}eat-\FV{}  9.apple  \AUG{}-{}1-man  1\SM{}-\IPFV{}-{}eat-\FV{}  \AUG{}-{}3-banana\\
%%TRANS1
\glt
  \glt ‘The woman is eating an apple, the man is eating a banana.’\\
%%TRANS2
%%EXEND

\z
\z

\subsection{Functions of -\textit{o}}
\label{bkm:Ref135664287}
The particle combines with topic referents and is infelicitous in focus environments \citep{AsiimwevanderWal2021}. For example, the particle is incompatible with content questions which are inherently focused as shown in \xref{bkm:Ref113716034}, and equally infelicitous in an answer to a content question \xref{bkm:Ref113716050}.

%%EAX
\ea
%%JUDGEMENT
[]{
%%LABEL
\label{bkm:Ref113716034}
%%CONTEXT
\citep[7]{AsiimwevanderWal2021}\\
%%LINE1
%%LINE2
\gll
Saúda  y-aa-teek-á  ki  (*\textbf{ky-o})?\\
%%LINE3
1.Sauda  1\SM{}.\SG{}-\N{}.\PST{}-{}cook-\FV{}  what  {\db}7-\CM{}\\
%%TRANS1
\glt
‘What has Sauda cooked?’\\
%%TRANS2
}
%%EXEND


\ex[]{
\label{bkm:Ref113716050}
\citep[7]{AsiimwevanderWal2021}\\
(Who broke the cup?)
}
%%EAX
\sn
%%JUDGEMENT
[\textsuperscript{\#}]{
%%LABEL
%%CONTEXT
%%LINE1
Omwáná \textbf{wé} akyasíre.\\
%%LINE2
\gll
o-mu-ana  w-o  a-ki-at-ire\\
%%LINE3
\AUG{}-{}1-child  1-\CM{}  1\SM{}.\SG{}-7\OM{}-{}break-\PFV{}\\
%%TRANS1
\glt
`The child broke it.’\\
%%TRANS2
}
%%EXEND

\z

Instead, we argued that the particle -\textit{o} is a contrastive topic marker. We can see that it is contrastive in the fact that it is infelicitous in an environment where no alternative referents are expected, as illustrated in \xref{bkm:Ref37772311}.

%%EAX
\ea
%%JUDGEMENT
%%LABEL
\label{bkm:Ref37772311}
%%CONTEXT
\citep[8]{AsiimwevanderWal2021}\\
(Context: You only have sheep and perhaps you have come back from shepherding.)\\
%%LINE1
Entaama (\textsuperscript{\#}\textbf{zó}) zaanyw’ ámíizi.\\
%%LINE2
\gll
e-n-taama  z-o  zi-aa-nyw-a  a-ma-izi\\
%%LINE3
\AUG{}-10-sheep  10-\CM{}  10\SM{}-\N{}.\PST{}-{}drink-\FV{}  \AUG{}-{}6-water\\
%%TRANS1
\glt
`The sheep have drunk water.’\\
%%TRANS2
%%EXEND


\z


The particle thus evokes a salient alternative topic that is either explicit or implicit. The implicature in \xref{bkm:Ref113716525} is that the maize garden they have not weeded, for example.

%%EAX
\ea
%%JUDGEMENT
%%LABEL
\label{bkm:Ref113716525}
%%CONTEXT
\citep[11]{AsiimwevanderWal2021}\\
%%LINE1
Orutookye \textbf{rwó} báárubágara.\\
%%LINE2
\gll
o-ru-tookye  ru-o  ba-aa-ru-bagar-a\\
%%LINE3
\AUG{}-11-banana.plantation  11-\CM{}  1\SM{}-\N{}.\PST{}-11\OM{}-weed-\FV{}\\
%%TRANS1
\glt
`As for the banana plantation, they have weeded it.’\\
%%TRANS2
%%EXEND


\z


The particle is equally felicitous in situations where only a subset is mentioned. The response \xref{bkm:Ref113716593:b} to the question in \xref{bkm:Ref113716593:a} with the contrastive particle indicates that the set of referents contains different kinds of food including, for example, \textit{posho}, rice and bananas and out of the different kinds of food that were being cooked, only bananas were ready for serving. Note that with the presence of the particle, the contrasted referents need not be mentioned, and in fact the proposition is not necessarily false for the alternatives that the referent is contrasted with. As shown in \xref{bkm:Ref113716593:c}, the speaker may choose to say ‘others I don’t know’.

\ea
\label{bkm:Ref113716593}
%%EAX
\ea
%%JUDGEMENT
%%LABEL
\label{bkm:Ref113716593:a}
%%CONTEXT
%%LINE1
Ebyókuryá byáhíire?\\
%%LINE2
\gll
e-byokurya  bi-aa-sy-ire?\\
%%LINE3
\AUG{}-8.food  8\SM{}-\N{}.\PST{}-be.ready-\PFV{}\\
%%TRANS1
\glt
‘Is the food ready?’\\
%%TRANS2
%%EXEND

%%EAX
\ex
%%JUDGEMENT
%%LABEL
\label{bkm:Ref113716593:b}
%%CONTEXT
%%LINE1
Ebitookye byó byáhíire.\\
%%LINE2
\gll
e-bi-tookye  bi-o  bi-aa-sy-ire\\
%%LINE3
\AUG{}-8-banana  8-\CM{}  8\SM{}-\N{}.\PST{}-be.ready-\PFV{}\\
%%TRANS1
\glt
‘As for the bananas, they are ready.'\\
%%TRANS2
%%EXEND


%%EAX
\ex
%%JUDGEMENT
%%LABEL
\label{bkm:Ref113716593:c}
%%CONTEXT
%%LINE1
Ebitookye byó byáhíire, ebíndi tíbikahíire/tindíkumanya.\\
%%LINE2
\gll
    e-bi-tookye  bi-o  bi-aa-sy-ire  e-bi-ndi     ti-bi-ka-sy-ire  /ti-n-riku-many-a \\
%%LINE3
\AUG{}-8-banana  8-\CM{}  8\SM{}-\N{}.\PST{}-be.ready-\PFV{}  \AUG{}-8-{}other \NEG-8\SM-\NEG-be.ready-\PFV{}  /\NEG-1\SG.\SM-\IPFV{}-know-\FV{}\\
%%TRANS1
\glt
 	\glt ‘As for the bananas, they are ready; the rest are not ready / I don’t know.’\\
%%TRANS2
%%EXEND

\z
\z

The particle is equally found to mark shift topics. This is commonly observed in news anchoring where the particle is used when switching to a new news item. For this particular use, the particle occurs in the initial position preceding the topic as in the illustration in \xref{bkm:Ref113716936}.


%%EAX
\ea
%%JUDGEMENT
%%LABEL
\label{bkm:Ref113716936}
%%CONTEXT
\citep[13]{AsiimwevanderWal2021}\\
(Recorded on TV West 08-05-2020 from the 8pm news.)\\
%%LINE1
\textbf{Bó} abanyamakúru omurí Mbarara baatung’ óbuhwezi bw’óbuhúnga...\\
%%LINE2
\gll
ba-o  a-ba-nya-makuru  o-mu-ri  Mbarara  ba-aa-tung-a   o-bu-hwezi  bu-a  o-bu-hunga... \\
%%LINE3
2-\CM{}  \AUG{}-2-\NMLZ{}-news  \AUG{}-18-be  23.Mbarara  2\SM{}-\N{}.\PST{}-get-\FV{} \AUG{}-14-help  14-\CONN{}  \AUG{}-14-posho \\
%%TRANS1
\glt
`Journalists in Mbarara (district) have received aid in form of \textit{posho}…’\\
%%TRANS2
%%EXEND

\z

The particle may in given contexts also express the speaker’s surprise at an event or situation that is beyond expectation. This gives rise to a mirative reading, as illustrated in \xref{bkm:Ref113717125} \citep[see also][]{Asiimwe2023}. Example \xref{bkm:Ref38213858} expresses a polarity focus but at the same time also gives a counterexpectation reading – it is surprising to the speaker that gorillas can sing.\largerpage


%%EAX
\ea
%%JUDGEMENT
%%LABEL
\label{bkm:Ref113717125}
%%CONTEXT
\citep[19]{AsiimwevanderWal2021}\\
(Context: Someone has a function or has organised an event and sends out invitations. For one reason or another, s/he does not expect many guests to turn up. Many guests turn up to the surprise of the host.)\\
%%LINE1
Abantu \textbf{bó} bíija.\\
%%LINE2
\gll
a-ba-ntu  ba-o  ba-ij-a\\
%%LINE3
\AUG{}-2-person  2-\CM{}  2\SM{}.\N{}.\PST{}-{}come-\FV{}\\
%%TRANS1
\glt
`People really came.’ (many people turned up, more than those expected)\\
%%TRANS2
%%EXEND

%%EAX
\ex
%%JUDGEMENT
%%LABEL
\label{bkm:Ref38213858}
%%CONTEXT
(Is it true that the gorillas sang for you?)\\
%%LINE1
Engagi zó záátwéshongorerá!\\
%%LINE2
\gll
e-n-gagi  zi-o  zi-aa-tu-eshongor-er-a\\
%%LINE3
\AUG{}-{}10-gorilla  10-\CM{}  10\SM{}-\N{}.\PST{}-1\PL{}.\OM{}-sing-\APPL{}-\FV{}\\
%%TRANS1
\glt
`(It is true) They have indeed sung for us!’\\
%%TRANS2
%%EXEND


\z

We further note in \citet{AsiimwevanderWal2021} that two particles can occur referring to one entity as indicated in \xref{bkm:Ref113717468}.

%%EAX
\ea
%%JUDGEMENT
%%LABEL
\label{bkm:Ref113717468}
%%CONTEXT
(Context: Mother is amazed by the love and special care her two-year-old twins show each other.)\\
%%LINE1
%%LINE2
\gll
Mbwénu  \textbf{b-ó}  a-bá  \textbf{b-o}…!\\
%%LINE3
\DM{}  2-\CM{}  \DEM{}-2.\PROX{}  2-\CM{} \\
%%TRANS1
\glt
`As for those ones…!’ \citep[26]{AsiimwevanderWal2021}\\
%%TRANS2
%%EXEND


\z

The particle, given the right context, is also associated with other pragmatic interpretations, namely intensity and verum, and may also give rise to a depreciative interpretation as shown in \xref{bkm:Ref136078566}.

%%EAX
\ea
%%JUDGEMENT
%%LABEL
\label{bkm:Ref136078566}
%%CONTEXT
%%LINE1
%%LINE2
\gll
E-n-júra  y-ó  y-áa-gw-a.\\
%%LINE3
\AUG{}-{}9-rain  9-\CM{}  9-\N{}.\PST{}-{}fall-\FV{}.{}\\
%%TRANS1
\glt
‘It has rained.’\\
%%TRANS2
%%EXEND

\settowidth\jamwidth{[depreciative]}
% \settowidth\jamwidth{[polarity]}
\sn
Context 1: Someone is in doubt whether it rained in my area. \jambox*{[verum]{\footnotemark}}
\footnotetext{The verum interpretation comes out more naturally with a discourse marker \textit{nangwa} ‘truly’ or \textit{buzima} ‘indeed/truly’ used in the construction.}
\sn
Context 2: It rained really heavily and/or for a long time. \jambox*{[intensity]}
\sn
Context 3: It has rained but it is no use since the crops have already withered. \jambox*{[depreciative]}

\z

Furthermore, the particle may be used in predicate doubling constructions \xref{bkm:Ref136338129}, as we elaborate further in \sectref{sec:preddoubling} with the various interpretations.

%%EAX
\ea
%%JUDGEMENT
%%LABEL
\label{bkm:Ref136338129}
%%CONTEXT
%%LINE1
Okugyenda (\textbf{kó}/\textbf{kwé}), tákagyenziré... konká naazá kugyenda.\\
%%LINE2
\gll
o-ku-gyenda  ku-o  ti-a-ka-gyend-ire...  konka  ni-a-z-a  ku-gyenda\\
%%LINE3
\AUG{}-15-go  15-\CM{}  \NEG{}-1\SM{}-not.yet-go-\PFV{}...  but  \IPFV{}-1\SM{}-go-\FV{}  15-go\\
%%TRANS1
\glt
‘S/he has not yet gone, but... s/he will/must go.’\\
%%TRANS2
%%EXEND


\z


The particle may also be used ironically as in the context in \xref{bkm:Ref113717991}.

%%EAX
\ea
%%JUDGEMENT
%%LABEL
\label{bkm:Ref113717991}
%%CONTEXT
(Context: In the context of Covid-19 pandemic, the speaker heard that the government announced that it will provide free face masks to all its citizens)\\
%%LINE1
%%LINE2
\gll
Z-ó  ni-zi-ij-á  ryarí  báitu?\\
%%LINE3
10-\CM{}  \IPFV{}-10-come-\FV{}  when  by.the.way\\
%%TRANS1
\glt
`By the way when are they (the masks) coming?’\\
%%TRANS2
%%EXEND


\z

In summary, the particle not only realises a contrastive topic reading but also a wide range of pragmatic meanings including mirativity. A detailed discussion on the roles of the particle and its origin are presented in \citet{AsiimwevanderWal2021}.

\subsection{Na + \PRO{}}
\label{bkm:Ref135664306}
The pronoun in \textit{-o} realises an additive meaning when attached to \textit{na} which may mean ‘with’, ‘and’, ‘also’ or ‘even’, as illustrated in \xxref{bkm:Ref111487876}{bkm:Ref111487879}. The combination \textit{na}+\PRO{} follows a noun, and it marks the non-initial topic, that is, some topic has to have been mentioned before in order to felicitously use \textit{na}+\PRO{} after another topic. In \xref{bkm:Ref111487876}, the crested cranes are mentioned first, and it is added that the same proposition also holds for doves. In \xref{bkm:Ref135665621}, no prior referent is mentioned explicitly in the sentence itself, but this can only be said in reaction to an earlier statement mentioning other referents (participating in the same action).

%%EAX
\ea
%%JUDGEMENT
%%LABEL
\label{bkm:Ref111487876}
%%CONTEXT
(Did you see the crested cranes?)\\
%%LINE1
%%LINE2
\gll
N-aa-reeb-a  é-n-tuuha,  e-n-dahi  \textbf{na-zó}    n-áá-zí-reeb-a. \\
%%LINE3
1\SG{}.\SM{}-see-\FV{}  \AUG{}-{}10-crested.cranes  \AUG{}-10-dove  and-10.\PRO{}   1\SG{}.\SM{}-\N{}.\PST{}-10\OM{}-see-\FV{} \\
%%TRANS1
\glt
`I saw crested cranes, and also doves.’\\
%%TRANS2
%%EXEND

\z

%%EAX
\ea
%%JUDGEMENT
%%LABEL
\label{bkm:Ref135665621}
%%CONTEXT
(Tomorrow we will iron bedsheets, trousers and skirts.)\\
%%LINE1
%%LINE2
\gll
E-sááti  \textbf{na-zó},  nyénsákare  ni-tu-z-á  ku-zí-gorora\\
%%LINE3
\AUG{}-10.shirts  and-10.\PRO{}  tomorrow  \IPFV{}-1\PL{}.\SM{}-go-\FV{}  15-10\OM{}-iron\\
%%TRANS1
\glt
`The shirts too, we will iron them tomorrow.’\\
%%TRANS2
%%EXEND


\z

%%EAX
\ea
%%JUDGEMENT
%%LABEL
\label{bkm:Ref111487879}
%%CONTEXT
(Context: Other animals had already sought advice from the (clever) owl.) \\
%%LINE1
Wakamé yaayitw’émbého \textbf{nayó} yaayebuuz’ékyokukóra yaaza kubúúz’ ékihuunyira\\
%%LINE2
\gll
wakame  y-aa-it-w-a  e-n-beho  na-yo  y-aa-e-buuza  e-ki-a  o-ku-kora  y-aa-za  ku-buuz-a  e-ki-huunyira\\
%%LINE3
9.Hare  9\SM{}-\N{}.\PST{}-kill-\PASS{}-\FV{}  \AUG{}-9-coldness  and-9.\PRO{}  9\SM{}-\N{}.\PST{}-\REFL{}-ask-\FV{}  \AUG{}-7-\CONN{}  \AUG{}-15-do  9\SM{}-\N{}.\PST{}-go  15-ask-\FV{}  \AUG{}-7-owl\\
%%TRANS1
\glt
‘The hare also felt very cold, and wondered about what to do and went to ask the owl.’\\
%%TRANS2
%%EXEND


\z

The additive meaning is also clear in the following example from a recipe.

%%EAX
\ea
%%JUDGEMENT
%%LABEL
%%CONTEXT
(Make sure you wash the bucket where you put the porridge. It is thoroughly washed, there is no dirt at all.)\\
%%LINE1
Reero wáamara orondé esafuriya yaawe \textbf{nay}’ ógibóneze.\\
%%LINE2
\gll
reero  u-aa-mar-a  o-rond-e  e-safuriya    i-a-we  na-yo  o-gi-bonez-e\\
%%LINE3
then  2\SG{}.\SM{}-\N{}.\PST{}-finish-\FV{}  2\SG{}.\SM{}-look.for-\SBJV{}  \AUG{}-9.saucepan  9-\POSS{}.2\SG{}  and-9.\PRO{}  2\SG{}.\SM{}-9\OM{}-clean-\SBJV{}\\
%%TRANS1
\glt ‘Then you get a saucepan and clean it as well.’\\
%%TRANS2
%%EXEND


\z

The combination \textit{na}+\PRO{} facilitates topic shift. The topic in \xref{bkm:Ref111488233} shifts from good chairs to bad chairs and the marker can only occur with the second topic as the ungrammaticality of \xref{bkm:Ref111488233:b} shows.

\ea
\label{bkm:Ref111488233}
%%EAX
\ea
%%JUDGEMENT
%%LABEL
\label{bkm:Ref111488233:a}
%%CONTEXT
%%LINE1
Yaareeb’ éntéb’ énungi n’embí \textbf{nazó} yáázíreeba.\\
%%LINE2
\gll
a-aa-reeb-a  e-n-tebe  e-n-rungi  na e-n-bi  na-zo  a-aa-zi-reeb-a \\
%%LINE3
1\SM{}-\N{}.\PST{}-see-\FV{}  \AUG{}-10-chairs  \AUG{}-10-good  and \AUG{}-10-bad  and-10.\PRO{}  1\SM{}-\N{}.\PST{}-10\OM{}-see-\FV{} \\
%%TRANS1
\glt
‘S/he saw good chairs, and bad ones s/he also saw.’\\
%%TRANS2
%%EXEND

%%EAX
\ex
%%JUDGEMENT
%%LABEL
\label{bkm:Ref111488233:b}
%%CONTEXT
%%LINE1
Yaareeb’ éntéb’ (é)nnungi (*\textbf{nazó}) n’embí yáázíreeba.\\
%%LINE2
\gll
a-aa-reeb-a  e-n-tebe  e-n-rungi   na-zo  na     e-n-bi  a-aa-zi-reeba \\
%%LINE3
1\SM{}-\N{}.\PST{}-see-\FV{}  \AUG{}-10-chairs  \AUG{}-10-good   and-10.\PRO{}  and \AUG{}-10-bad  1\SM{}-\N{}.\PST{}-10\OM{}-see \\
%%TRANS1
\glt
‘S/he saw good chairs, and bad ones s/he also saw.’\\
%%TRANS2
%%EXEND

\z
\z

The difference with the contrastive particle -\textit{o} used by itself is that with \textit{na}+\PRO{}, the same predicate applies to both topics, as opposed to contrasting truth values or alternative predicates for the referents marked by -\textit{o} alone.

We conclude that the particle -\textit{o} and the combination \textit{na}+-\textit{o} are used in Rukiga to mark topics, either contrastively or additively. In the next section we turn to predicate doubling, in which one type also involves contrastive topicalisation, and as mentioned, this topic doubling can be combined with the -\textit{o} particle.\pagebreak

\section{Predicate doubling}\label{sec:preddoubling}

In predicate doubling, the same predicate occurs twice in one clause. Rukiga shows two types of predicate doubling: topic doubling and in-situ doubling. These will be presented in turn below. 

\subsection{Topic doubling}

In topic doubling, an infinitive form of the verb precedes an inflected form of the same verb, as in \xref{bkm:Ref98832788}. The marker \textit{kwo}/\textit{kwe} (see \sectref{sec:particle-o}) can also be added here to reinforce the various interpretations, showing that the initial infinitive typically functions as a contrastive topic. Relevant in comparison with \xref{bkm:Ref100084512} below, it is ungrammatical to add \textit{na} ‘and, also, even’ when the construction contains the topic marker -\textit{o} as shown in \xref{bkm:Ref98832788:b}.

\ea
\label{bkm:Ref98832788}
%%EAX
\ea
%%JUDGEMENT
[]{
%%LABEL
\label{bkm:Ref98832788:a}
%%CONTEXT
%%LINE1
Okuhínga (\textbf{kwé}) nimpînga.\\
%%LINE2
\gll
o-ku-hing-a  ku-o  ni-n-hing-a\\
%%LINE3
\AUG{}-{}15-dig-\FV{}  15-\CM{}  \IPFV{}-1\SG{}.\SM{}-{}dig-\FV{}\\
%%TRANS1
\glt
‘For the case of digging, I can dig.’ / ‘Digging I can do…’\\
%%TRANS2
}
%%EXEND


 %%EAX
\ex
%%JUDGEMENT
[*]{
%%LABEL
\label{bkm:Ref98832788:b}
%%CONTEXT
%%LINE1
N’okuhínga (\textbf{kwé}) nimpínga.\\
%%LINE2
\gll
na  o-ku-hing-a  ku-o  ni-n-hing-a\\
%%LINE3
and  \AUG{}-{}15-dig-\FV{}  15-\CM{}  \IPFV{}-1\SG{}.\SM{}-{}dig-\FV{}\\
%%TRANS1
\glt
int. ‘For the case of digging, I can also dig.’ / ‘Even digging I can do…’\\
%%TRANS2
}
%%EXEND


\z
\z

The construction can be used in a range of contexts with varying interpretations. A prototypical interpretation contrasts the topical infinitive with another action. The contrastive interpretation is made explicit in \xref{bkm:Ref98833538} by the following negative clause.

%%EAX
\ea
%%JUDGEMENT
%%LABEL
\label{bkm:Ref98833538}
%%CONTEXT
(Context: The Hare has been very lazy while the other animals worked on the field. The Hare could say:)\\
%%LINE1
Okukóra (kwé), tínaakora, konká ninzá kurya.\\
%%LINE2
\gll
o-ku-kora  ku-o  ti-n-a-kor-a  kwonka  ni-n-z-a  ku-ria\\
%%LINE3
\AUG{}-15-work  15-\CM{}  \NEG{}-1\SG{}.\SM{}\N.\PST{}-work-\FV{}  but  \IPFV{}-1\SG{}.\SM{}-go-\FV{}  15-eat\\
%%TRANS1
\glt
‘I’ve not worked, but I will eat.’\\
%%TRANS2
%%EXEND


\z

The contrastive interpretation is inherent to the strategy, as becomes evident from example \xref{bkm:Ref98833573}: even if nothing else is said, it is clear that eating is contrasted to something else.

%%EAX
\ea
%%JUDGEMENT
%%LABEL
\label{bkm:Ref98833573}
%%CONTEXT
(Context: You are visiting someone and have only been given food. When you’re asked how it is, you can say this and the host will know that you expected something else too, for example a drink; or you did not get satisfied.)\\
%%LINE1
Okuryá náarya…\\
%%LINE2
\gll
o-ku-ria  n-aa-ri-a\\
%%LINE3
\AUG{}-15-eat  1\SG{}.\SM{}-\N{}.\PST{}-eat-\FV{}\\
%%TRANS1
\glt
‘Eating I did...’\\
%%TRANS2
%%EXEND


\z

Topic doubling can also be used to express polarity focus and verum,\footnote{Polarity focus may be used in yes/no contexts; verum is used in a corrective context, to prevent the hearer from assuming the negative statement.} as in \xref{bkm:Ref98834331}. The contrast here is with the negative value of not having done the action.

%%EAX
\ea
%%JUDGEMENT
%%LABEL
\label{bkm:Ref98834331}
%%CONTEXT
(Have you spread the sorghum? Context: You want to emphasise that you have sowed enough seeds because the other person cannot see the seeds.)\\
%%LINE1
Okugutéera náágutéera.\\
%%LINE2
\gll
o-ku-gu-teera  n-aa-gu-teer-a\\
%%LINE3
\AUG{}-15-3\OM{}-beat  1\SG{}.\SM{}-\N{}.\PST{}-3\OM{}-beat-\FV{}\\
%%TRANS1
\glt
‘I DID scatter them.’\\
%%TRANS2
%%EXEND


\z


Apart from the contrastive and polarity/verum readings, the interpretation can also be what \citet{Meeussen1967} called ``concessive" and \citet{vanderWalJerro2022a} have named ``depreciative", as in \xref{bkm:Ref98834437}. In this interpretation, the action is evaluated as being not worth a lot, and/or as the low quality potentially preventing further actions or achievements. Furthermore, it can be intensive/to a high degree, as illustrated in \xref{bkm:Ref98834517}.

%%EAX
\ea
%%JUDGEMENT
%%LABEL
\label{bkm:Ref98834437}
%%CONTEXT
%%LINE1
%%LINE2
\gll
O-ku-támbura  kw-é  tw-á-támbur-a…\\
%%LINE3
\AUG{}-15{}-walk  15-\CM{}  1\PL{}.\SM{}-\N{}.\PST{}-walk-\FV{}\\
%%TRANS1
\glt
‘Although we walked… (I don’t know whether we’ll ever arrive).’\\
%%TRANS2
%%EXEND


\z
\largerpage[-1]\pagebreak
%%EAX
\ea
%%JUDGEMENT
%%LABEL
\label{bkm:Ref98834517}
%%CONTEXT
(Context: You’re telling somebody that you really played football, you’ve done it with a passion. You have maybe not done anything other than playing football.)\\
%%LINE1
Okutéér’ ómupííra gwé náágutééra.\footnote{Note that the contrastive marker here modifies and agrees with the object in class 3, not the infinitive.}\\
%%LINE2
\gll
o-ku-teera  o-mu-piira  gu-o  n-aa-gu-teer-a.\\
%%LINE3
\AUG{}-15-beat  \AUG{}-3-ball  3-\CM{}  1\SG{}.\SM{}-\N{}.\PST{}-3\OM{}-beat-fv\\
%%TRANS1
\glt
‘I really played football.’\\
%%TRANS2
%%EXEND


\z


A mirative reading is also possible with a predicate doubling construction, as in \xref{bkm:Ref117247263}, further illustrated in context 5 in \xref{bkm:Ref98834552}, with the speaker expressing surprise that the event described indeed took place.

%%EAX
\ea
%%JUDGEMENT
%%LABEL
\label{bkm:Ref117247263}
%%CONTEXT
(Context: There is a function at school and teachers join students on the dance floor, something that is totally unexpected by the students.)\\
%%LINE1
%%LINE2
\gll
O-ku-zína  b-aa-zín-a.\\
%%LINE3
\AUG{}-15-dance  2\SM{}-\N{}.\PST{}-dance-\FV{}\\
%%TRANS1
\glt
‘Dancing, they did (to the surprise of the students)!’\\
%%TRANS2
%%EXEND


\z

That these interpretations are fully context-dependent can be seen in \xref{bkm:Ref98834552}: the sentence is the same, also in terms of prosody, but the possible interpretations are many.

%%EAX
\ea
%%JUDGEMENT
%%LABEL
\label{bkm:Ref98834552}
%%CONTEXT
%%LINE1
%%LINE2
\gll
O-ku-hínga  tu-hing-íre.\\
%%LINE3
\AUG{}-15{}-dig  1\PL{}.\SM{}-dig-\PFV{}\\
%%TRANS1
\glt
%%TRANS2
%%EXEND

\settowidth\jamwidth{[depreciative]}
\sn Context 1: Did you really plough? \\
‘We actually ploughed.’  \jambox*{[polarity]}
\sn Context 2: We were expected to dig and feed the animals\\
‘Digging we did (but we didn’t feed the animals).’  \jambox*{[contrast]}
\sn Context 3: It’s planting season but there is no rain.  \\ 
‘We (went ahead and) ploughed anyway…’  \jambox*{[depreciative]}
\sn Context 4: The size of the ploughed land is big. \\
‘We really ploughed a lot!’  \jambox*{[intensive]}
\sn Context 5: We were expected to plough only a small part of the field but to our surprise, we ploughed all of it in a short time.\\
‘We ploughed a surprising amount!’  \jambox*{[mirative]}
\z

The intensive and mirative interpretations can easily overlap as one may be surprised at something that is done to a high degree. Nevertheless, example \xref{bkm:Ref116286044} also shows that the two can be distinguished by their contexts.

%%EAX
\ea
%%JUDGEMENT
%%LABEL
\label{bkm:Ref116286044}
%%CONTEXT
%%LINE1
\label{bkm:Ref127888871}Okuryá fene náágírya.\\
%%LINE2
\gll
o-ku-ria  fene  n-aa-gi-ri-a\\
%%LINE3
\AUG{}-15-eat  9.jackfruit  1\SG{}.\SM{}-\N{}.\PST{}-9\OM{}-eat-\FV{}\\
%%TRANS1
\glt
‘Eating jackfruit, I have eaten it.’\\
%%TRANS2
%%EXEND

\settowidth\jamwidth{[intensive]}
\sn Context 1:  I have eaten a whole jackfruit. These fruits are usually big. \jambox*{[intensive]}
\sn
Context 2:  I don’t usually eat jackfruit. I am surprised that I have eaten it and in big quantity. \jambox*{[mirative]}

\z

We noted in \sectref{sec:particle-o} that the pragmatic interpretations that are associated with particle -\textit{o} are possible with predicate doubling as demonstrated in \xref{bkm:Ref98834552}. The particle \textit{-o} can, however, be used together with the predicate doubling construction. In fact, it is typical for the particle to appear in predicate doubling. With both linguistic strategies marking contrastive topics, their combination can be described as reinforcing the various interpretations as we illustrate in \xref{bkm:Ref118708342}.

%%EAX
\ea
%%JUDGEMENT
%%LABEL
\label{bkm:Ref118708342}
%%CONTEXT
%%LINE1
%%LINE2
\gll
O-kw-óga  (\textbf{kw-é})  n-áá-yog-a.\\
%%LINE3
\AUG{}-15-swim  {\db}15-\CM{}  1\SG{}.\SM{}-\N{}.\PST{}-swim-\FV{}\\
%%TRANS1
\glt `I have really/indeed swum (but…).’\\
%%TRANS2
%%EXEND

\settowidth\jamwidth{[depreciative]}
\sn
Context 1: Pool attendant sees me walking away from the pool area showing no sign that I entered the water. \jambox*{[verum]}
\sn
Context 2: I was expected to swim and play baseball. \jambox*{[contrast]}
\sn
Context 3: The water was too cold but I went ahead and swam anyway. \jambox*{[depreciative]}
\sn
Context 4: I went into the pool and swam for a long time with lots of energy. \jambox*{[intensity]}
\sn
Context 5: I have always feared to get into the water but hey I can swim! \jambox*{[mirative]}

\z

Note that topic doubling is not used to express VP focus or state-of-affairs (SoA) focus – when these interpretations were assessed in examples \xref{bkm:Ref98834957} and \xref{bkm:Ref98835964}, respectively, the interpretation given by the speakers is one of polarity focus. Example \xref{bkm:Ref98834957} is not felicitous as answer to the VP question ‘what are you doing?’ (VP focus) and example \xref{bkm:Ref98835964} cannot be used to correct ‘washing’ by ‘ironing’ (SoA focus). 

%%EAX
\ea
%%JUDGEMENT
%%LABEL
\label{bkm:Ref98834957}
%%CONTEXT
(Context 1: Can you dig? E.g. when you want to give somebody a job, or test them\\
\textsuperscript{\#}Context 2: What are you doing?)\\
%%LINE1
Okuhînga nimpînga.\\
%%LINE2
\gll
o-ku-hinga  ni-n-hing-a\\
%%LINE3
\AUG{}-15-dig  \IPFV{}-1\SG{}.\SM{}-dig-\FV{}\\
%%TRANS1
\glt ‘I can dig (but...).’\\
%%TRANS2
%%EXEND

\z

%%EAX
\ea
%%JUDGEMENT
%%LABEL
\label{bkm:Ref98835964}
%%CONTEXT
(Context: Two pictures of Lydia washing the sheets and ironing the sheets; ‘Did she wash the sheets?’)\\
%%LINE1
(Yeego, konká) n’ ókuzígorora azigorwîre.\\
%%LINE2
\gll
yeego  kwonka  na  o-ku-zi-gorora  a-zi-goror-ire\\
%%LINE3
yes  but  and  \AUG{}-15-10\OM{}-iron  1\SM{}-10\OM{}-iron-\PFV{}\\
%%TRANS1
\glt
‘(Yes, but) she ironed them too.’\\
%%TRANS2
%%EXEND

\textsuperscript{\#}`(No), she ironed them.’\\
\z

For transitive predicates, when the object is included in the infinitive, the interpretation is still one of the above-mentioned (contrastive, polarity, depreciative, intensive, mirative). The intensive and mirative interpretations are illustrated in \xref{bkm:Ref127888871} above, and see \xref{bkm:Ref98835064} for a contrast on different actions.

%%EAX
\ea
%%JUDGEMENT
%%LABEL
\label{bkm:Ref98835064}
%%CONTEXT
(Context: There is one task left to do, which is mingling \textit{karo} ‘millet bread’; the others will get the water or do the weeding. Now you volunteer to do the mingling.)\footnote{‘Mingle’ is Ugandan English for stirring and preparing thick porridge-like substances.}\\
%%LINE1
%%LINE2
\gll
  [O-ku-góyá  a-ka-ró]  tu-ryá-ka-góy-a.\\
%%LINE3
  {\db}\AUG{}-15-stir  \AUG{}-12-millet.bread  1\PL{}.\SM{}-\N.\FUT{}-12\OM{}-stir-\FV{}\\
%%TRANS1
\glt
  \glt ‘Mingling millet bread, we will do it.’\\
%%TRANS2
%%EXEND

\z

When the object instead follows the inflected verb, however, the most natural interpretation is that of object focus, as indicated in the preceding question in \xref{bkm:Ref98835146}.

%%EAX
\ea
%%JUDGEMENT
%%LABEL
\label{bkm:Ref98835146}
%%CONTEXT
(What will you mingle?)\\
%%LINE1
%%LINE2
\gll
O-ku-góyá  tu-ryá-góy-á  á-ká-ro.\\
%%LINE3
\AUG{}-15-stir  1\PL{}.\SM{}-\N.\FUT{}-stir-\FV{}  \AUG{}-12-millet.bread\\
%%TRANS1
\glt
‘(As for mingling,) We will mingle millet bread.’\\
%%TRANS2
%%EXEND


\z

If an object marker is present, it should be present on both verbs – neither can be omitted, as shown in \xref{bkm:Ref98836854}.

\ea
\label{bkm:Ref98836854}
%%EAX
\ea
%%JUDGEMENT
%%LABEL
%%CONTEXT
%%LINE1
Okuziríisa kó, \'{A}mós naazá ku*(zi)ríisa.\\
%%LINE2
\gll
o-ku-zi-ri-is-a  ku-ó  Amos  n-aa-z-á  ku-zi-ri-is-a\\
%%LINE3
\AUG{}-15-10\OM{}-eat-\CAUS{}-\FV{}  15-\CM{}  1.Amos  \IPFV{}-1\SM{}-go-\FV{}  15-10\OM{}-eat-\CAUS{}-\FV{}\\
%%TRANS1
\glt
‘For the case of grazing them, Amos can do it.’\\
%%TRANS2
%%EXEND

 %%EAX
\ex
%%JUDGEMENT
%%LABEL
%%CONTEXT
%%LINE1
Oku*(zi)ríisa kó, \'{A}mos naazá kuziríisa.\\
%%LINE2
\gll
o-ku-zi-ri-is-a  ku-o  Amos  n-aa-z-á  ku-zi-ri-is-a\\
%%LINE3
\AUG{}-15-10\OM{}-eat-\CAUS{}-\FV{}  15-\CM{}  1.Amos  \IPFV{}-1\SM{}-go-\FV{}  15-10\OM{}-eat-\CAUS{}-\FV{}\\
%%TRANS1
\glt
‘For the case of grazing them, Amos can do it.’\\
%%TRANS2
%%EXEND

\z
\z

Note also that if the object is made explicit in the first phrase, the object marker cannot be omitted on the inflected verb, as shown in \xref{bkm:Ref116295118}.

%%EAX
\ea
%%JUDGEMENT
%%LABEL
\label{bkm:Ref116295118}
%%CONTEXT
%%LINE1
Okukárya ákahúnga kó Jein yáá*(ká)rya.\\
%%LINE2
\gll
o-ku-ka-ria  a-ka-hunga  ka-o  Jein  a-aa-ka-ri-a\\
%%LINE3
\AUG{}-15-12-eat  \AUG{}-12-posho  12-\CM{}  1.Jane  1\SM{}-\N{}.\PST{}-12\OM{}-eat-\FV{}\\
%%TRANS1
\glt
‘Jane has truly eaten (the posho).’\\
%%TRANS2
%%EXEND


\z


The subject, also functioning as a topic, can either precede the topical infinitive or the inflected verb, as shown in \xref{bkm:Ref111465657}. 

\ea
\label{bkm:Ref111465657}
%%EAX
\ea
%%JUDGEMENT
%%LABEL
%%CONTEXT
%%LINE1
Jéín ókuryá yáarya.\\
%%LINE2
\gll
Jein  o-ku-ria  a-aa-ri-a\\
%%LINE3
1.Jane  \AUG{}-15-eat  1\SM{}-\N{}.\PST{}-eat-\FV{}\\
%%TRANS1
\glt
‘Jane has eaten (it is true).’\\
%%TRANS2
%%EXEND

%%EAX
\ex
%%JUDGEMENT
%%LABEL
%%CONTEXT
%%LINE1
Okurya (kwó) Jein yáárya.\\
%%LINE2
\gll
o-ku-ria  ku-o  Jein  a-aa-ri-a\\
%%LINE3
\AUG{}-15-eat  15-\CM{}  1.Jane  1\SM{}-\N{}.\PST{}-eat-\FV{}\\
%%TRANS1
\glt
‘Jane has eaten (it is true/has eaten a lot).’\\
%%TRANS2
%%EXEND

\z
\z

To summarise, in topic doubling, an infinitive form of the verb functions as the contrastive topic, and it is followed by an inflected form of the same verb. The interpretation can be that of polarity focus, a contrast with other actions, depreciative, intensive, or mirative, depending on the context of use. We now turn to the second predicate doubling construction.

\subsection{In-situ doubling}

In two cases can the non-finite form follow the same inflected verb:\footnote{This is called in-situ focus doubling by \citet{GüldemannFiedler2022} as the non-finite form seems to function as an object in the unmoved postverbal position.} when nominalised in class 14 with \textit{bu}-, and when the infinitive is preceded by \textit{na} ‘and/with’. The \textit{bu}- doubling, illustrated in \xref{bkm:Ref98835711}, can be seen as the verbal parallel to a nominal reduplication procedure illustrated in \xref{bkm:Ref98835727}, resulting in a dismissive reading, indicated in the translation by ‘just, merely’.

%%EAX
\ea
%%JUDGEMENT
%%LABEL
\label{bkm:Ref98835711}
%%CONTEXT
(Context: I came home late, didn’t have supper.)\\
%%LINE1
%%LINE2
\gll
N-aa-byam-a  bu-byáma.\\
%%LINE3
1\SG{}.\SM{}-\N{}.\PST{}-sleep-\FV{}  14-sleep\\
%%TRANS1
\glt
‘I just went to sleep.’\\
%%TRANS2
%%EXEND


\z

%%EAX
\ea
%%JUDGEMENT
%%LABEL
\label{bkm:Ref98835727}
%%CONTEXT
%%LINE1
Ente  bute  neetumá  wáíruk-a  munónga?\\
%%LINE2
\gll
e-n-te  bu-te  ni-e-tum-a  u-aa-iruk-a  munonga\\
%%LINE3
\AUG{}-9-cow  14-cow  \IPFV{}-9\SM{}-cause-\FV{}  2.\SG.\SM{}-\N.\PST{}-run-\FV{}  fast/for.long\\
%%TRANS1
\glt
‘Can a mere cow cause you to run so much?’\\
%%TRANS2
%%EXEND


\z


The second strategy is illustrated in \xref{bkm:Ref98835774}, where either order of infinitive and inflected verb is allowed, as long as the infinitive is preceded by \textit{na} ‘and, also, even’. On the scale of expectation, the additive \textit{na} adds an above expectation reading; in this example perhaps it is to some degree expected that one might have a 5-minute power nap, but dreaming is one step further than napping.

\ea
\label{bkm:Ref98835774}
(Context: You fell asleep in the office and wake up surprised.) 
%%EAX
\ea
%%JUDGEMENT
%%LABEL
%%CONTEXT
%%LINE1
%%LINE2
\gll
N-áá-róot-a   *(n’)  ó-ku-róóta!\\
%%LINE3
1\SG{}.\SM{}-\PST{}-dream-\FV{}  even  \AUG{}-15-dream\\
%%TRANS1
\glt
‘I even dreamed!’\\
%%TRANS2
%%EXEND


\ex   N’ ókuróóta nááróóta!\\
    \glt ‘I even dreamed!’
\z
\z


The object can be added after either the inflected or the infinitive verb. The examples in \xref{bkm:Ref100084512} show different word orders (with an optional object noun because of the presence of the object marker), but the interpretation remains the same: you expected that they would only wash the bicycles, but they in addition repaired them, too.


\ea
\label{bkm:Ref100084512}(Context: You thought they only wash bicycles.)
%%EAX
\ea
%%JUDGEMENT
%%LABEL
%%CONTEXT
%%LINE1
N’ókubukánika (óbugaari) nibabukanika.\\
%%LINE2
\gll
Na  o-ku-bu-kanika  o-bu-gaari  ni-ba-bu-kanik-a\\
%%LINE3
Even/also  \AUG{}-15-14\OM{}-repair  \AUG{}-14-bicycle  \IPFV{}-2\SM{}-14\OM{}-repair-\FV{}\\
%%TRANS1
\glt
‘They even/also repair them (talking of bicycles).’\\
%%TRANS2
%%EXEND


%%EAX
\ex
%%JUDGEMENT
%%LABEL
%%CONTEXT
%%LINE1
Nibabukaniká n’okubukánika (óbugáari).\\
%%LINE2
\gll
ni-ba-bu-kanik-a  na  o-ku-bu-kanik-a o-bu-gaari\\
%%LINE3
\IPFV{}-2\SM{}-14\OM{}-repair-\FV{}  even/also  \AUG{}-15-14\OM{}-repair \AUG{}-14-bicycle\\
%%TRANS1
\glt
‘They even/also repair them (talking of bicycles).’\\
%%TRANS2
%%EXEND


%%EAX
\ex
%%JUDGEMENT
%%LABEL
%%CONTEXT
%%LINE1
N’ókubukánika nibabukanika (obugáari).\\
%%LINE2
\gll
na  o-ku-bu-kanika  ni-ba-bu-kanik-a  o-bu-gaari\\
%%LINE3
even/also  \AUG{}-15-14\OM{}-repair  \IPFV{}-2\SM{}-14\OM{}-repair-\FV{} \AUG{}-14-bicycle\\
%%TRANS1
\glt
‘They even/also repair them (talking of bicycles).’\\
%%TRANS2
%%EXEND

%%EAX
\ex   
%%JUDGEMENT
%%LABEL
%%CONTEXT
%%LINE1
Nibabukanik-a (óbugáari) n’okubukánika.\\
%%LINE2
\gll
  ni-ba-bu-kanik-a  o-bu-gaari na  o-ku-bu-kanik-a\\
%%LINE3
  \IPFV{}-2\SM{}-14-repair-\FV{}  \AUG{}-14-bicycle even/also  \AUG{}-15-14-repair-\FV{}\\
%%TRANS1
\glt `They even/also repair them (talking of bicycles).’\\
%%TRANS2
%%EXEND


\z

\z


A third type of predicate doubling known as cleft doubling (see the Kîîtharaka chapter, \citetv{chapters/kiitharaka}), is not possible in Rukiga as the ungrammatical example shows in \xref{bkm:Ref127994561}. In cleft doubling, an infinitive forms the focused constituent in a cleft (see further \sectref{bkm:Ref111469884}), while the same predicate is also the main predicate.

%%EAX
\ea
%%JUDGEMENT
[*]{
%%LABEL
\label{bkm:Ref127994561}
%%CONTEXT
%%LINE1
%%LINE2
\gll
Ni  o-ku-hing-a  a-hing-ire.\\
%%LINE3
\COP{}  \AUG{}-{}15-dig-\FV{}  1\SM{}-{}dig-\PFV{}\\
%%TRANS1
\glt
int. ‘It is digging that s/he did.’, ‘S/he \textit{dug}.’\\
%%TRANS2
}
%%EXEND


\z



In this section, we have shown that Rukiga, just like Kîîtharaka, Kinyakyusa, Makhuwa, Kirundi and Cicopi (see chapters in \citealt{langsci-current-book}), uses predicate doubling as a strategy to express information structure. Two kinds of predicate doubling, namely topic and in-situ doubling, are identified in Rukiga. In-situ doubling with \textit{bu}- is less prevalent and allows a dismissive interpretation, while in-situ doubling with na+infinitive is associated with a degree higher than expectation as well as mirativity. Topic doubling is more prevalent and is associated with various context-induced interpretations. It expresses contrastive topics, verum, intensity, depreciative and mirative interpretations. We noted that the contrastive topic reading is inherent to the strategy. We further showed that the particle~-\textit{o}~as a contrastive topic marker (discussed in \sectref{sec:particle-o}) is often used in topic doubling constructions to reinforce a given interpretation. However, further research should be carried out to determine the precise circumstances under which the two strategies co-occur.

\section{Augment}\label{sec:augment}
Rukiga presents an augment morpheme in its grammar, in the form of a vowel preceding the noun class prefix on nouns. Besides occurring on nouns, it optionally appears in the morphology of various nominal modifiers, specifically adjectives, possessives, relatives, numerals and some quantifiers. In some previous studies, the presence of an optional augment on nominal modifiers has been associated with definiteness \citep{MorrisKirwan1972,Taylor1972,Taylor1985}. By ``definite" they mean that the speaker has a particular referent in mind, and expects the hearer to uniquely distinguish it from other referents. Although the augment has been associated with various semantic and pragmatic roles (see \citealt{MorrisKirwan1972,Taylor1972,Taylor1985,Asiimwe2014} on Runyankore-Rukiga), in this section we argue that the presence of the augment on nominal modifiers marks a restrictive reading, that is, it selects a subset out of a set of alternatives \citep{AsiimweEtAl2023} and therefore has the effect of exclusive focus. We only summarise the main points here and refer to \citet{AsiimweEtAl2023} for a detailed analysis.\footnote{The analysis of the augment presented in \citet{AsiimweEtAl2023} compares the augment to the phenomenon of determiner spreading in Greek.} 

We illustrate the interpretation of the augment as a restrictive marker when it attaches to relative clauses, adjectives, possessives and some quantifiers, beginning with relative clauses. Relative clauses in Rukiga take an optional augment. When the augment is present, it triggers a restrictive reading that is unattainable when the augment is absent. Compare \xref{bkm:Ref113525219:a} and \xref{bkm:Ref113525219:b}:
\largerpage[-1]
\pagebreak

\ea
\label{bkm:Ref113525219}\citep[1288]{AsiimweEtAl2023}
%%EAX
\ea
%%JUDGEMENT
%%LABEL
\label{bkm:Ref113525219:a}
%%CONTEXT
\textit{non-restrictive}\\
%%LINE1
%%LINE2
\gll
e-n-yungu  \textbf{yí}  w-aa-goy-a=mu  á-ká-ro\\
%%LINE3
\AUG{}-9-pot  9\REL{}.\PRO{}  2\SG{}.\SM{}-\N{}.\PST{}-mingle-\FV{}=18  \AUG{}-12-millet.bread\\
%%TRANS1
\glt
    \glt ‘the/a pot, which you cooked millet bread in’ \\
%%TRANS2
(we already know which pot, there is one pot)
%%EXEND


%%EAX
\ex
%%JUDGEMENT
%%LABEL
\label{bkm:Ref113525219:b}
%%CONTEXT
%%LINE1
\textit{restrictive}\\
%%LINE2
\gll
e-n-yungw’  \textbf{é-yí}  w-aa-goy-a=mu  á-ká-ro\\
%%LINE3
\AUG{}-9-pot  \AUG{}-9\REL{}.\PRO{}  2\SG{}.\SM{}-\N{}.\PST{}-mingle-\FV{}=18  \AUG{}-12-millet.bread\\
%%TRANS1
\glt
‘the pot that you cooked millet bread in’ (not the other pot)\\
%%TRANS2
%%EXEND


    \z
\z


The above analysis is in contrast to \citegen{Taylor1985} claim that the augment is itself a relative clause marker, as a relative clause reading is attainable even when the augment is absent (see \xref{bkm:Ref113525219:a} above). Instead, the relative meaning is marked as a variation in tone patterns \citep[see][]{Asiimwe2019}.

We predicted that, if the augment on relatives marks a restrictive referent, it should be incompatible with unique referents, since there are no alternatives. This was borne out. It is infelicitous to use an augment on a relative clause that modifies a unique referent, such as the sun or the Pope. The augment on the subject relative in \xref{bkm:Ref111493905} triggers a set of alternative suns – yet in daily life outside of astronomy there are no alternative suns to consider.

%%EAX
\ea
%%JUDGEMENT
%%LABEL
\label{bkm:Ref111493905}
%%CONTEXT
\citep[1289]{AsiimweEtAl2023}\\
%%LINE1
Ndeebir’ éízóób’ (\textsuperscript{\#}é)lirí hale.\\
%%LINE2
\gll
n-reeb-ire  e-i-zooba  e-ri-ri  hare\\
%%LINE3
1\SG{}.\SM{}-{}see-\PFV{}  \AUG{}-{}5-sun  \AUG{}-{}5\RM{}-be  far\\
%%TRANS1
\glt
‘I saw the sun, which is far.’\\
%%TRANS2
%%EXEND


\z



Furthermore, we used ‘which’ questions to test whether the augment is indeed associated with restrictive interpretation. A ‘which’ question typically selects one member from a set. It is expected that in the answer to the question, a subset is selected, and the presence of the augment as in \xref{bkm:Ref111493926:b} is indeed preferred. 
\largerpage[-1]
\pagebreak

\ea
\label{bkm:Ref111493926}\citep[1290]{AsiimweEtAl2023}
%%EAX
\ea
%%JUDGEMENT
%%LABEL
\label{bkm:Ref111493926:a}
%%CONTEXT
Context: At the market when looking at pieces of cloth in different colors. \\
%%LINE1
Orugóye nooyendá kugura ruuha?\\
%%LINE2
\gll
o-ru-goye  ni-o-end-a  ku-gura  ru-ha\\
%%LINE3
\AUG{}-11-cloth  \IPFV{}-2\SG{}.\SM{}-want-\FV{}  15-buy  11-which\\
%%TRANS1
\glt
‘Which cloth do you want to buy?’\\
%%TRANS2
%%EXEND


%%EAX
\ex
%%JUDGEMENT
%%LABEL
\label{bkm:Ref111493926:b}
%%CONTEXT
%%LINE1
Niinyendá kugur’ órugóy’ \textsuperscript{\#}(ó)ruríkutukura. \\
%%LINE2
\gll
ni-n-end-a  ku-gura  o-ru-goye  o-ru-riku-tukur-a\\
%%LINE3
\IPFV{}-{}1\SG{}-want-\FV{}  15-buy  \AUG{}-11-cloth  \AUG{}-11\RM{}-\IPFV{}-be.red-\FV{}\\
%%TRANS1
\glt
`I want to buy a/the red cloth.’\\
%%TRANS2
lit. ‘I want to buy a/the cloth that is red.’

%%EXEND

\z
\z

Based on such tests (for more see \citealt{AsiimweEtAl2023}), we concluded that the presence of the augment on relative clauses triggers a restrictive reading while its absence means that there are no alternatives to select from.

The same holds for adjectives. \citet{Taylor1972, Taylor1985} equates the augment on adjectives to the definite marker in the Indo-European languages when he suggests that the presence of the augment on the adjective in \xref{bkm:Ref111493957} renders the noun \textit{omushaija} definite while its absence in \xref{bkm:Ref111493957:b} signifies an indefinite referent. However, we show that \xref{bkm:Ref111493957:b} can also be used with an indefinite interpretation and therefore the question is what function the augment has on the adjective.

\ea
\label{bkm:Ref111493957}(\citealt[74]{Taylor1972}; glosses added)
%%EAX
\ea
%%JUDGEMENT
%%LABEL
\label{bkm:Ref111493957:a}
%%CONTEXT
%%LINE1
%%LINE2
\gll
o-mu-sháija  mu-rungi\\
%%LINE3
\AUG{}-1-man  1-good\\
%%TRANS1
\glt
‘a good man’\\
%%TRANS2
%%EXEND


%%EAX
\ex
%%JUDGEMENT
%%LABEL
\label{bkm:Ref111493957:b}
%%CONTEXT
%%LINE1
%%LINE2
\gll
o-mu-sháíj’  \textbf{ó}-mu-rúngi\\
%%LINE3
\AUG{}-1-man  \AUG{}-1-good\\
%%TRANS1
\glt
‘the good man’\\
%%TRANS2
%%EXEND


\z
\z

Building on the work of \citet{Asiimwe2014} and discussed in detail in \citet{AsiimweEtAl2023}, we propose that the augment realises a restrictive reading on adjectives: while the absence of the augment on the adjective \textit{mbisi} ‘unripe’ in \xref{bkm:Ref111494100:b} gives no special interpretation, its presence on \textit{embisi} in \xref{bkm:Ref111494100:a} means that there are alternative pineapples that are ripe and that the buying is restricted to the subset that is unripe.\pagebreak

\ea
\label{bkm:Ref111494100}\citep[1294]{AsiimweEtAl2023}
\settowidth\jamwidth{[+A]}
%%EAX
\ea
%%JUDGEMENT
%%LABEL
\label{bkm:Ref111494100:a}
%%CONTEXT
%%LINE1
Naagur’ énanáás’ \textbf{émbísi}.  \jambox*{[+A]}
%%LINE2
\gll
n-aa-gur-a  e-nanaasi  e-n-bisi\\
%%LINE3
1\SG{}.\SM{}-\N{}.\PST{}-buy-\FV{}  \AUG{}-9.pineapple  \AUG{}-9-unripe\\
%%TRANS1
\glt
‘I have bought the unripe pineapple.’ (as opposed to a ripe pineapple)\\
%%TRANS2
%%EXEND

%%EAX
\ex
%%JUDGEMENT
%%LABEL
\label{bkm:Ref111494100:b}   
%%CONTEXT
%%LINE1
Naagur’ énanaasi \textbf{mbísi}.  \jambox*{[-A]}
%%LINE2
\gll
    n-aa-gur-a  e-nanaasi  n-bisi\\
%%LINE3
1\SG{}.\SM{}-\N{}.\PST{}-{}buy-\FV{}  \AUG{}-9.pineapple  9-unripe\\
%%TRANS1
\glt ‘I have bought an unripe pineapple.’\\
%%TRANS2
%%EXEND


\z
\z

The example in \xref{bkm:Ref111494149} involves a ‘which’ question again. The hearer is not expected to respond to \xref{bkm:Ref111494149:a} with an augmentless adjective because the question targets one referent from a set of given alternatives. Therefore, it is natural for the hearer to answer with an augment on the adjective, selecting big cups (the alternative being small cups). It is also infelicitous for the augment to be used with the adjective when the entities to select from include forks, plates, knives etc.

\ea
\label{bkm:Ref111494149}\citep[1295]{AsiimweEtAl2023}
%%EAX
\ea
%%JUDGEMENT
%%LABEL
\label{bkm:Ref111494149:a}
%%CONTEXT
%%LINE1
%%LINE2
\gll
E-bi-kópo  w-aa-gur-a  bi-iha?\\
%%LINE3
\AUG{}-8-cup  2\SG{}.\SM{}-\N{}.\PST{}-buy-\FV{}  8-which\\
%%TRANS1
\glt
‘Which cups have you bought?’\\
%%TRANS2
%%EXEND


%%EAX
\ex
%%JUDGEMENT
%%LABEL
\label{bkm:Ref111494149:b}
%%CONTEXT
%%LINE1
%%LINE2
\gll
N-aa-gur’  e-bi-kóp’  é-bi-hángo.\\
%%LINE3
1\SM{}-\N{}.\PST{}-buy  \AUG{}-8-cup  \AUG{}-8-big\\
%%TRANS1
\glt
‘I have bought the big cups.’\\
%%TRANS2
%%EXEND

%%EAX
\ex
%%JUDGEMENT
[\textsuperscript{\#}]{
%%LABEL
\label{bkm:Ref111494149:c}   
%%CONTEXT
%%LINE1
%%LINE2
\gll
N-aa-gur’  e-bi-kopo  bi-hángo.\\
%%LINE3
1\SM{}-\N{}.\PST{}-buy  \AUG{}-{}8-cup  8-big\\
%%TRANS1
\glt
%%TRANS2
}
%%EXEND

\z
\z

To further illustrate, the sentence in \xref{bkm:Ref111466869} was said as part of the instructions in the QUIS \citep{SkopeteasEtAl2006} map task, in which one speaker has to lead another speaker through a map with various entities on crossroads. When asked whether the adjective ‘big’ could have an augment here, it was indicated that this would mean the animal has various tails from which it could choose.

%%EAX
\ea
%%JUDGEMENT
%%LABEL
\label{bkm:Ref111466869}
%%CONTEXT
(‘In the middle of the road, there is a fox… no, a mongoose.’)\\
%%LINE1
%%LINE2
\gll
O-mu-terere  gw-in’  ó-mu-kira  \_\_-mu-hângo  gw-a  kitaka.\\
%%LINE3
\AUG{}-3-mongoose  3\SM{}-have  \AUG{}-3-tail  3-big  3-of  brown\\
%%TRANS1
\glt
`The mongoose has a big brown tail.’\\
%%TRANS2
%%EXEND


\z

By using an adjective with an augment, the speaker intends to provide the hearer with extra information, so that the hearer learns that there is a choice between referents.

Possessives too allow an optional augment. We again use a context containing a ‘which’ question in \xref{bkm:Ref111494278} to show that the augment attached to possessives selects one referent from a set (see again \citealt{AsiimweEtAl2023} for further evidence).

\ea
\label{bkm:Ref111494278}
(Context: Which garden has Mr Elephant dug?)\\
%%EAX
\ea
%%JUDGEMENT
[\textsuperscript{\#}]{
%%LABEL
%%CONTEXT
%%LINE1
%%LINE2
\gll
Warujojo  y-aa-hing-á  o-mu  mu-siri  \textbf{gw-é(ye)}.\\
%%LINE3
1.Elephant  1\SM{}-\N{}.\PST{}-dig-\FV{}  \AUG{}-18  3-garden  3.\POSS{}.1\\
%%TRANS1
\glt
`Mr Elephant cultivated in his garden.’\\
%%TRANS2
}
%%EXEND


%%EAX
\ex
%%JUDGEMENT
[]{
%%LABEL
%%CONTEXT
%%LINE1
%%LINE2
\gll
Warujojo  y-aa-hing-á  ó-mu  mu-siri  \textbf{o-gw-é(ye)}.\\
%%LINE3
1.Elephant  1\SM{}-\N{}.\PST{}-dig-\FV{}  \AUG{}-18  3-garden  \AUG{}-3.\POSS{}.1\\
%%TRANS1
\glt
`Mr Elephant cultivated in his own garden (e.g. not in Mr Hare’s garden).’\\
%%TRANS2
}
%%EXEND


%%EAX
\ex
%%JUDGEMENT
[\textsuperscript{\#}]{
%%LABEL
%%CONTEXT
%%LINE1
%%LINE2
\gll
Y-aa-hing-a  ó-mu  mu-siri  \textbf{gw-a}  Wakame.\\
%%LINE3
1\SM{}-\N{}.\PST{}-dig-\FV{}  \AUG{}-18  3-garden  3-\CONN{}  Hare\\
%%TRANS1
\glt
`He cultivated in Mr Hare’s garden.’\\
%%TRANS2
}
%%EXEND


%%EAX
\ex
%%JUDGEMENT
[]{
%%LABEL
%%CONTEXT
%%LINE1
%%LINE2
\gll
Y-aa-hing-a  ó-mu  mu-siri  \textbf{ó-gw-a}  Wakame.\\
%%LINE3
1\SM{}-\N{}.\PST{}-dig-\FV{}  \AUG{}-18  3-garden  \AUG{}-3-\CONN{}  Hare\\
%%TRANS1
\glt
`He cultivated in Mr Hare’s garden (and not in his).’\\
%%TRANS2
}
%%EXEND


\z
\z

We further show that the use of the augment with quantifiers restricts a subset of referents. Indefinite quantifiers such as -\textit{ingi} ‘many, -\textit{kye} ‘few’, -\textit{mwe} ‘some’ as observed with relative clauses, adjectives and possessives allow an optional augment that restricts a subset of the noun. The presence of an augment selects a subset of gardens that are many in \xref{bkm:Ref135724099:b} leaving the subset of gardens that are few or the rest of the gardens.

\ea
\label{bkm:Ref135724099}
%%EAX
\ea
%%JUDGEMENT
%%LABEL
\label{bkm:Ref135724099:a}
%%CONTEXT
%%LINE1
%%LINE2
\gll
E-mi-siri  y-a  Wakamé  \textbf{mí-ngi}  e-hing-ire.\\
%%LINE3
\AUG{}-4-garden  4-\CONN{}  1.Hare  4-many  4\SM{}-dig-\PFV{}\\
%%TRANS1
\glt
‘Many gardens belonging to Mr Hare are ploughed.’\\
%%TRANS2
%%EXEND


%%EAX
\ex
%%JUDGEMENT
%%LABEL
\label{bkm:Ref135724099:b}
%%CONTEXT
%%LINE1
%%LINE2
\gll
E-mi-siri  y-a  Wakame  \textbf{e-mi-ngi}  e-hingire.\\
%%LINE3
\AUG{}-4-garden  4-\CONN{}  Hare  4-many  4\SM{}-dig-\PFV{}\\
%%TRANS1
\glt
‘Most of the Mr Hare’s gardens are ploughed.’\\
%%TRANS2
%%EXEND


\z
\z

The quantifier -\textit{mwe} ‘some’ also expresses meaning about referents excluded from a given set.\footnote{The quantifier without the augment can also mean ‘certain’.} Like -\textit{ingi} ‘many’/-\textit{kye} ‘few’, the augment is optional on -\textit{mwe}. The quantifier -\textit{mwe} with the augment appears to restrict some members and exclude others in a given set. Although the referents are non-specific, in \xref{bkm:Ref111494382:b} the augment selects an unspecified number of shirts that are not ironed.

\ea
\label{bkm:Ref111494382}
%%EAX
\ea
%%JUDGEMENT
%%LABEL
\label{bkm:Ref111494382:a} 
%%CONTEXT
%%LINE1
E-saati \textbf{zi-mwé} ti-zi-gorwíre.\\
%%LINE2
\gll
    e-saati  zi-mwe  ti-zi-goror-íre\\
%%LINE3
\AUG{}-10.shirt  10-some  \NEG{}-10\SM{}-iron-\IPFV{}\\
%%TRANS1
\glt
    \glt ‘Some shirts are not ironed.’\\
%%TRANS2
%%EXEND


%%EAX
\ex
%%JUDGEMENT
%%LABEL
\label{bkm:Ref111494382:b}
%%CONTEXT
%%LINE1
Esaati \textbf{ézimwé} tizigorwîre.\\
%%LINE2
\gll
e-saati  e-zi-mwe  ti-zi-goror-íre\\
%%LINE3
\AUG{}-10.shirt  \AUG{}-10-some  \NEG{}-10\SM{}-iron-\IPFV{}\\
%%TRANS1
\glt
‘Some of the shirts are not ironed.’\\
%%TRANS2
%%EXEND


\z
\z

Further evidence that the augment is restrictive comes from the fact that the quantifier -\textit{ona} ‘all’, which cannot trigger alternatives within its set, does not permit the augment \xref{bkm:Ref111494451}. On the other hand, the quantifier -\textit{ndi} which entails the presence of alternatives always takes an augment \xref{bkm:Ref111494462}.

%%EAX
\ea
%%JUDGEMENT
%%LABEL
\label{bkm:Ref111494451}
%%CONTEXT
%%LINE1
Enyamaishwá  (*e)zoona  kú  ziizire  kukóra…
%%LINE2
\gll
e-nyamaishwa  e-zi-ona  ku  zi-ij-ire  ku-kora\\
%%LINE3
\AUG{}-10.animal  \AUG{}-10-all  when  10\SM{}-come-\PFV{}  15-work\\
%%TRANS1
\glt
‘When (the) other animals came to work…’\\
%%TRANS2
%%EXEND


\z

%%EAX
\ea
%%JUDGEMENT
%%LABEL
\label{bkm:Ref111494462}
%%CONTEXT
%%LINE1
Enyamaishwa *(é)zíndi kú ziizire kukóra…
%%LINE2
\gll
e-nyamaishwa e-zi-ndi    ku  zi-ij-ire  ku-kora…\\
%%LINE3
 \AUG{}-10.animal  \AUG{}-10-other when  10\SM{}-come-\PFV{}  15-work\\
%%TRANS1
\glt
‘When (the) other animals came to work…’\\
%%TRANS2
%%EXEND


\z


We postulate that the augment is one of the strategies Rukiga uses to contrast or exclude referents for which the predicate does not hold. We further assert that the presence of an optional augment on modifiers brings about a restrictive meaning, such that a set of alternatives must be present from which the asserted referent is selected. The alternatives in this case are triggered at the level of the modifier, that is, at the sub-NP level, e.g. focus on an adjective like ‘tall’ triggers the alternative ‘short’ (and possibly other intermediate measures). 

  Although the selection from alternatives at first sight relates to exclusive focus, \citet{AsiimweEtAl2023} show that alternatives on the level of the noun phrase are not necessarily excluded in the examples in \xref{bkm:Ref118715311} and \xref{bkm:Ref118715313}. If the noun phrase containing the augmented modifier had been interpreted as exclusive,\footnote{If other sizes between small and large, or old and new, are also taken into account, then those could potentially be excluded, and in that case, the test only shows that this cannot be an exhaustive interpretation in which \textit{all} alternatives are excluded.} we would not expect the acceptability of the second clause in these examples, as these state that the predicate is not exclusively true for the subset mentioned in the first clause. This ``mismatch" between the restrictive/exclusive interpretation on the sub-NP level and the non-exclusive interpretation on the NP level remains a topic for further research.


%%EAX
\ea
%%JUDGEMENT
%%LABEL
\label{bkm:Ref118715311}
%%CONTEXT
\citep[1332]{AsiimweEtAl2023}\\
%%LINE1
%%LINE2
\gll
Yakóbo  y-aa-gabur-ir’  é-nyamaíshwa.   \textbf{Pusi}  \textbf{e-n-tó}  z-aa-b-a  zi-ine  é-n-jara na  \textbf{púsi}  \textbf{é-n-kuru}  z-aa-b-a  zi-ine  é-n-jara\\
%%LINE3
1.Jacob  1\SM{}-\N{}.\PST{}-feed-\APPL{}-\FV{}  \AUG{}-10.animal 10.cat  \AUG{}-10-young  10\SM{}-\N{}.\PST{}-be-\FV{}  10\SM{}-have  \AUG{}-9-hunger~ and  10.cat  \AUG{}-10-old  10\SM{}-\N{}.\PST{}-be-\FV{}  10\SM{}-have  \AUG{}-9-hunger\\
%%TRANS1
\glt
‘Jacob fed the animals. The young cats were hungry, and also the old cats were hungry.’
%%TRANS2
%%EXEND
\z

%%EAX
\ea
%%JUDGEMENT
%%LABEL
\label{bkm:Ref118715313}\citep[1332]{AsiimweEtAl2023}
%%CONTEXT
%%LINE1
Yaareeb \textbf{entéb’ (é)nungí} n'\textbf{eémbí} nazó yáázíreeba.\\
%%LINE2
\gll
a-aa-reeb-a  e-n-tebe  e-n-rungi   na  e-n-bi  na-zo  a-aa-zi-reeb-a  \\
%%LINE3
1\SM{}-\N{}.\PST{}-see-\FV{}  \AUG{}-10-chair  \AUG{}-10-good and  \AUG{}-10-bad  and-10.\PRO{}  1\SM{}-\N{}.\PST{}-10\OM{}-see-\FV{} \\
%%TRANS1
\glt
‘S/he saw good chairs, and bad ones s/he also saw.’\\
%%TRANS2
%%EXEND

\z

Another question is how the contrast on a modifier (sub-NP level) interacts with the information structure in the clause. In preliminary data, there does not seem to be any restriction: noun phrases functioning as the topic, and noun phrases functioning as the focus can occur with or without an augment on the modifier, as shown in the left-peripheral topic in \xref{bkm:Ref118715421} and the cleft in \xref{bkm:Ref127995600}.


%%EAX
\ea
%%JUDGEMENT
%%LABEL
\label{bkm:Ref118715421}
%%CONTEXT
\citep[1332]{AsiimweEtAl2023}\\
%%LINE1
%%LINE2
\gll
E-bi-kóp’  (é-)bi-hángo  n-aa-bi-teer-a=mu  á-ba-gyenyi.\\
%%LINE3
\AUG{}-8-cup  {\db}\AUG{}-8-big  1\SG{}.\SM{}-\N{}.\PST{}-8\OM{}-put-\FV{}=18.\LOC{}  \AUG{}-2-visitor\\
%%TRANS1
\glt
‘As for the big cups, I have served the visitors tea in them.’\\
%%TRANS2
%%EXEND


\z

%%EAX
\ea
%%JUDGEMENT
%%LABEL
\label{bkm:Ref127995600}
%%CONTEXT
%%LINE1
N’ébíkóp’ (é)biháng’ ébí naateeramw’ ábagyenyi. \\
%%LINE2
\gll
ni  e-bi-kopo  (e)-bi-hango  e-bi  n-aa-te-er-a=mu  a-ba-gyenyi\\
%%LINE3
\COP{}  \AUG{}-8-cup  {\db}\AUG{}-8-big  \AUG{}-\REL{}.\PRO{} 1\SG{}.\SM{}-\N{}.\PST{}-put-\APPL{}-\FV{}=18  \AUG-2-visitor\\
%%TRANS1
\glt ‘It is the big cups that I have served the visitors tea in.’\\
%%TRANS2
%%EXEND


\z


We conclude in the paper that “there seems to be no correlation between “focus within the DP” (the restrictive reading of the augment) and focus in the clause: they are independent and all combinations occur” \citep[1333]{AsiimweEtAl2023}. This too remains an interesting field for further investigation.

\section{Cleft constructions}\label{sec:cleft-constructions}

Rukiga features three constructions that can be described as ``cleft": the basic cleft, the pseudocleft, and what looks like a reverse pseudocleft but turns out to be an NP constituent followed by a clefted pronoun. Each consists of three elements: 1) the copula \textit{ni}, 2) the clefted constituent, and 3) the relative clause. We present relative marking here, as it will be relevant for all the three constructions.\footnote{We acknowledge Melle Groen and Nina van der Vlugt for their help in investigating Rukiga clefts.}

Non-subject relatives are marked by the proximal demonstrative functioning as the relative pronoun, as in \xref{bkm:Ref98922886:b}. We gloss it here as \REL{}.\PRO{}.

\ea
\label{bkm:Ref98922886}
\citep[1287--1288]{AsiimweEtAl2023}
%%EAX
\ea
%%JUDGEMENT
%%LABEL
\label{bkm:Ref98922886:a}
%%CONTEXT
%%LINE1
%%LINE2
\gll
W-aa-teek’  á-ka-ró  o-mu  n-yúngu.\\
%%LINE3
2\PL{}.\SM{}-\N{}.\PST{}-cook  \AUG{}-12-millet.bread  \AUG{}-18  9-pot\\
%%TRANS1
\glt
‘You have prepared millet bread in a pot.’\\
%%TRANS2
%%EXEND


 %%EAX
\ex
%%JUDGEMENT
%%LABEL
\label{bkm:Ref98922886:b}
%%CONTEXT
%%LINE1
%%LINE2
\gll
e-n-yungw’  \textbf{(é)-yí}  w-aa-goy-a=mu  á-ká-ro\\
%%LINE3
\AUG{}-9-pot  \AUG{}-9\REL{}.\PRO{}  2\SG{}.\SM{}-\N{}.\PST{}-{}mingle-\FV{}=18.\LOC{}  \AUG{}-12-millet.bread\\
%%TRANS1
\glt
‘the pot that you prepared millet bread in’\\
%%TRANS2
%%EXEND


\z
\z


Subject relatives are marked by a different tone pattern, as shown in \xref{bkm:Ref98922875}.\largerpage[2]

\ea
\citep[1288]{AsiimweEtAl2023}
\label{bkm:Ref98922875}
%%EAX
\ea
%%JUDGEMENT
%%LABEL
%%CONTEXT
%%LINE1
%%LINE2
\gll
Wakame  y-áá-záár-a.\\
%%LINE3
9.rabbit  9\SM{}-\N{}.\PST{}-give.birth-\FV{}\\
%%TRANS1
\glt
‘A/the rabbit has given birth.’\\
%%TRANS2
%%EXEND


 %%EAX
\ex
%%JUDGEMENT
%%LABEL
%%CONTEXT
%%LINE1
%%LINE2
\gll
wakamé  y-aa-záar-a\\
%%LINE3
9.rabbit  9\RM{}-\N{}.\PST{}-give.birth-\FV{}\\
%%TRANS1
\glt
‘a/the rabbit which has given birth’\\
%%TRANS2
%%EXEND

\z
\z

As discussed in \sectref{sec:augment}, relative clauses may be preceded by an augment, which \citet{AsiimweEtAl2023} (see also \citealt{Asiimwe2019}) argue marks a restrictive relative clause.

\subsection{Basic cleft}
\label{bkm:Ref111469884}
The basic cleft consists of the copula \textit{ni} (or negative copula \textit{ti}) preceding the clefted constituent, and a relative clause following it, marked with the usual markers explained above, illustrated in \xref{bkm:Ref111468375} and \xref{bkm:Ref111468376}.

%%EAX
\ea
%%JUDGEMENT
%%LABEL
\label{bkm:Ref111468375}
%%CONTEXT
(What has Maria swept?)\\
%%LINE1
N’ ékibúge éki María yaakondóora.\\
%%LINE2
\gll
ni   e-ki-buga  e-ki   Maria  a-aa-kondoor-a\\
%%LINE3
\COP{}   \AUG{}-7{}-compound  \AUG{}-7\REL{}.\PRO{}   1.Maria   1\SM{}-\N{}.\PST{}-{}sweep-\FV{}\\
%%TRANS1
\glt
‘It’s the compound that Maria has swept.’\\
%%TRANS2
%%EXEND


\z

%%EAX
\ea
%%JUDGEMENT
%%LABEL
\label{bkm:Ref111468376}
%%CONTEXT
(What will Pamela cook?)\\
%%LINE1
Ni muhógo eyí Paméla aryátéeka.\\
%%LINE2
\gll
ni  muhogo  e-yí   Pamela  a-rya-teek-a\\
%%LINE3
\COP{}  9.cassava  \AUG{}-9.\REL{}.\PRO{}  1.Pamela  1\SM{}-\FUT{}-cook-\FV{}\\
%%TRANS1
\glt
‘It’s cassava that Pamela will cook.’\\
%%TRANS2
%%EXEND


\z


Basic clefts are not very commonly used, and a construction with a left\hyp peripheral NP + cleft (see \sectref{bkm:Ref100080636}) is preferred for noun phrases. We do find clefts naturally with interrogatives, as in \xref{bkm:Ref98944488} and \xref{bkm:Ref115082992}, and with personal pronouns, as in \xxref{bkm:Ref111469643}{bkm:Ref135725703}. Note that with clefted non-subject pronouns, the relative marker is not present, also shown in \xref{bkm:Ref111469645} and \xref{bkm:Ref135725703}. Note also that example \xref{bkm:Ref135725730} shows the use of the negative copula \textit{ti}.

%%EAX
\ea
%%JUDGEMENT
%%LABEL
\label{bkm:Ref98944488}
%%CONTEXT
%%LINE1
N’ oh’ ógyénzire?\\
%%LINE2
\gll
ni  o-ha  o-gyend-ire\\
%%LINE3
\COP{}  1-who  1\SM{}.\REL{}-go-\PFV{}\\
%%TRANS1
\glt
‘Who left?’\\
%%TRANS2
%%EXEND


\z

%%EAX
\ea
%%JUDGEMENT
%%LABEL
\label{bkm:Ref115082992}
%%CONTEXT
%%LINE1
Ni nkahé áh’ oseeriir’ ómugúsha?\\
%%LINE2
\gll
ni  nkahe  a-hu  o-s-er-ire  o-mu-gusha\\
%%LINE3
\COP{}  where  \AUG{}-{}16.\REL{}.\PRO{}  2\SG{}.\SM{}-grind-\APPL{}-\PFV{}  \AUG{}{-3-}sorghum\\
%%TRANS1
\glt
‘Where did you grind the sorghum from?’\\
%%TRANS2
%%EXEND


\z

%%EAX
\ea
%%JUDGEMENT
%%LABEL
\label{bkm:Ref135725730}
\label{bkm:Ref111469643}
%%CONTEXT
%%LINE1
Ekitábo nkishomíre konka \textbf{tíinye naakihandííkire}.\\
%%LINE2
\gll
e-ki-tabo  n-ki-shom-ire  konka  ti  nye n-a-ki-handiik-ire\\
%%LINE3
\AUG{}-7-book  1\SG{}.\SM{}-7\OM{}-read-\PFV{}  but  \NEG{}.\COP{}  1\SG{}.\PRO{}  1\SG.\SM-\PST-7\OM{}-write-\PFV{}\\
%%TRANS1
\glt
‘The book, I have read it but I’m not the one who wrote it.’\\
%%TRANS2
%%EXEND

%%EAX
\ex
%%JUDGEMENT
%%LABEL
\label{bkm:Ref111469645}
%%CONTEXT
%%LINE1
Nizó (*ezí) naabuganáho.\\
%%LINE2
\gll
ni  z-o  e-zi  n-aa-bugan-a=ho\\
%%LINE3
\COP{}  10-\PRO{}  \AUG{}-\REL{}.\PRO{}  1\SG{}.\SM{}-\N{}.\PST{}-meet-\FV{}=16.\LOC{}\\
%%TRANS1
\glt
‘They are the ones that I have met there.’\\
%%TRANS2
%%EXEND
%%EAX
\ex
%%JUDGEMENT
%%LABEL
\label{bkm:Ref135725703}
%%CONTEXT
%%LINE1
Níinye waaréeba.\\
%%LINE2
\gll
ni  inye  u-aa-reeb-a\\
%%LINE3
\COP{}  1\SG{}.\PRO{}  2\SG{}.\SM{}-\N{}.\PST{}-see-\FV{}\\
%%TRANS1
\glt
‘It is me that you saw.’\\
%%TRANS2
%%EXEND
\z

When tested, the focus on the clefted constituent in the basic cleft comes out as exhaustive. The basic cleft cannot be followed up by a clause asserting the truth for another referent, as in \xref{bkm:Ref111494752}; and it cannot be modified by ‘primarily’, as in \xref{bkm:Ref98944693}, or the universal quantifier ‘all’ as in \xref{bkm:Ref135726145}. These facts follow straightforwardly if the basic cleft has an inherent exhaustive meaning: ‘primarily’ indicates that the predicate is also true for other referents (other people spoke besides Sara), and ‘all’ does not exclude any referents in the set (there are no cups that did not fall); just as John also cooking posho means that Sara is not the only one. Thus, the alternatives that are necessarily present for the referents in \xxref{bkm:Ref135726191}{bkm:Ref135726145}, and for which the proposition is also true, are incompatible with the exhaustive interpretation of the cleft construction, requiring that the proposition be false for all alternatives.\largerpage

%%EAX
\ea
%%JUDGEMENT
[]{
%%LABEL
\label{bkm:Ref111494752}\label{bkm:Ref135726191}
%%CONTEXT
(Who cooked posho?)\\
%%LINE1
Ni Sáár’ ówaateekir’ ákahúnga (*, na Jóoni nawe).\\
%%LINE2
\gll
ni  Saara  o-u-aa-teek-ire  a-ka-hunga  na  Jooni  na-we\\
%%LINE3
\COP{}  1.Sara  \AUG{}-1\SM{}-\N{}.\PST{}-cook-\PFV{}  \AUG{}-12-posho  and  1.John  and-1.\PRO{} \\
%%TRANS1
\glt
int. ‘It is Sara who cooked posho (*, and John also).’\\
%%TRANS2
}
%%EXEND

%%EAX
\ex
%%JUDGEMENT
[*]{
%%LABEL
\label{bkm:Ref98944693}
%%CONTEXT
%%LINE1
Owáágamba ni Sáár’ okukira.\\
%%LINE2
\gll
o-u-aa-gamb-a  ni  Saara  okukira\\
%%LINE3
1\SM{}.\REL{}-1\SM{}-\N{}.\PST{}-speak-\FV{}  \COP{}  1.Saara  primarily\\
%%TRANS1
\glt
int. ‘The one who spoke is primarily Sara.’\\
%%TRANS2
}
%%EXEND

\z
%%EAX
\ea
%%JUDGEMENT
[*]{
%%LABEL
\label{bkm:Ref135726145}
%%CONTEXT
%%LINE1
Ni byón’ ebikóp’ ébyâgwa.\\
%%LINE2
\gll
ni  bi-ona  e-bi-kopo  e-bi-aa-gw-a\\
%%LINE3
\COP{}  8-all   \AUG{}-8-cup  \AUG{}-8\SM{}.\REL{}-\N{}.\PST{}-fall-\FV{}\\
%%TRANS1
\glt
‘It is all the cups that fell.’\\
%%TRANS2
}
%%EXEND


\z


\subsection{Pseudocleft}

In a pseudocleft, the copula joins a free relative (FR) on its left with a noun on its right: [FR] \COP{} [NP]. What looks like a verb in \xref{bkm:Ref98945724} is a relative clause that functions as a noun phrase, which is known as a free relative. The free relative typically creates a presupposition of existence, and the described entity is then identified by the focused noun, as in \xref{bkm:Ref98945724} we describe the existence of some who welcomed us, and this person is then identified as Peace.

%%EAX
\ea
%%JUDGEMENT
%%LABEL
\label{bkm:Ref98945724}
%%CONTEXT
(Who welcomed you?)\\
%%LINE1
Owaatwákíira ni Píisi.\\
%%LINE2
\gll
o-u-aa-tu-akiir-a  ni  Piisi\\
%%LINE3
\AUG{}-1\SM{}.\REL{}-\PST{}-1\PL{}.\OM{}-receive-\FV{}  \COP{}  1.Peace\\
%%TRANS1
\glt
‘The one who welcomed us is Peace.’\\
%%TRANS2
%%EXEND


\z

The existence presupposition can be seen in the oddness to answer the pseudocleft question in \xref{bkm:Ref98945731} with ‘nobody’, i.e. there must be someone who took the salt. Relevant to the question in this example, note that there is an asymmetry between subjects and non-subjects here: whereas subjects can be questioned in a pseudocleft, this is unacceptable for non-subjects, presumably because they have the possibility to be questioned postverbally.

%%EAX
\ea
%%JUDGEMENT
[]{
%%LABEL
\label{bkm:Ref98945731}
%%CONTEXT
%%LINE1
Owaatwar’ ómwónyo n’ooha? \\
%%LINE2
\gll
o-u-aa-twar-a  o-mu-onyo  ni  o-ha\\
%%LINE3
\AUG{}-1\SM{}.\REL{}-\N{}.\PST{}-take-\FV{}  \AUG{}-3-salt  \COP{}  1-who\\
%%TRANS1
\glt
‘Who has taken the salt?’\\
%%TRANS2
}
%%EXEND

%%EAX
\sn
%%JUDGEMENT
[\textsuperscript{\#}]{
%%LABEL
%%CONTEXT
%%LINE1
Tihárího.\\
%%LINE2
\gll
ti-ha-ri=ho\\
%%LINE3
\NEG{}-16\SM{}-be=16\\
%%TRANS1
\glt
‘Nobody.’\\
%%TRANS2
lit. ‘There isn’t (who has taken the salt).’\\
}
%%EXEND


\z


While the NP expresses identificational focus, the focus seems to differ from that in a basic cleft in allowing modification by ‘primarily’ \xref{bkm:Ref98946942}, thus arguing against inherent exhaustivity, but still disallowing ‘even’ \xref{bkm:Ref100085040}, ‘all’ \xref{bkm:Ref111494883}, and ‘for example’ \xref{bkm:Ref100085050}, which are also tests for exclusivity and exhaustivity. We suggest that this is due to the function of identification, which should select one primary referent: inclusive ‘even’ and ‘all’ do not select, and ‘for example’ is not specific enough for proper identification.

%%EAX
\ea
%%JUDGEMENT
[]{
%%LABEL
\label{bkm:Ref98946942}
%%CONTEXT
(Who spoke?)\\
%%LINE1
Okukír’ ówáagamba ni Sáara.\\
%%LINE2
\gll
okukira  o-u-aa-gamb-a  ni  Saara\\
%%LINE3
primarily  1\SM{}.\REL{}-1\SM{}-\N{}.\PST{}-speak-\FV{}  \COP{}  1.Saara\\
%%TRANS1
\glt
‘It is primarily the case that it was Sara who spoke.’\\
%%TRANS2
}
%%EXEND

\ex[]{
\label{bkm:Ref100085040}
(What else has Jane cooked?)\\
}
%%EAX
\sn
%%JUDGEMENT
[*]{
%%LABEL
%%CONTEXT
%%LINE1
Eki Jéin yaateeka ni n’ ákáro.\\
%%LINE2
\gll
e-ki  Jein  a-aa-teek-a  ni  na  a-ka-ro\\
%%LINE3
\AUG{}-7.\REL{}.\PRO{}   1.Jane   1\SM{}-\N{}.\PST{}-{}cook-\FV{}  \COP{}  and  \AUG{}-{}12-millet.bread\\
%%TRANS1
\glt
int. ‘What Jane has prepared is even/also millet bread.’\\
%%TRANS2
}
%%EXEND


%%EAX
\ex
%%JUDGEMENT
[]{
%%LABEL
\label{bkm:Ref111494883}
%%CONTEXT
(What drank water?)\\
%%LINE1
Ekyanyw’ ámíizi n’éntaama (*zóona). \\
%%LINE2
\gll
e-ki-a-nyw-a  a-ma-izi  ni  e-n-taama  zi-ona\\
%%LINE3
\AUG{}-7\SM-drink-\FV{}  \AUG{}-6-water  \COP{}  \AUG{}-10-sheep  10-all\\
%%TRANS1
\glt
`It is (*all) the sheep that drank water.’\\
%%TRANS2
}
%%EXEND

%%EAX
\ex
%%JUDGEMENT
[]{
%%LABEL
\label{bkm:Ref100085050}
%%CONTEXT
(Who has a pen?)\\
%%LINE1
Oyine péeni ni (*nka) Rónald.\\
%%LINE2
\gll
o-ine  peeni  ni  nka  Ronald\\
%%LINE3
1\SM{}.\REL{}-have  9.pen  \COP{}  like  1.Ronald\\
%%TRANS1
\glt
‘Who has a pen is (*for example) Ronald.’\\
%%TRANS2
}
%%EXEND


\z


\subsection{Left-peripheral NP + cleft}
\label{bkm:Ref100080636}
\largerpage
A direct reverse of the pseudocleft is not grammatical in Rukiga: compare the pseudocleft in \xref{bkm:Ref98947401:a} with the attempt in \xref{bkm:Ref98947401:b}. Instead, an independent pronoun in -\textit{o} must be used in this construction (see also \sectref{sec:particle-o}), as in \xref{bkm:Ref98947401:c}.

\ea
\label{bkm:Ref98947401}
%%EAX
\ea
%%JUDGEMENT
[]{
%%LABEL
\label{bkm:Ref98947401:a}
%%CONTEXT
%%LINE1
%%LINE2
\gll
[E-kí  Bíiru  y-aa-yozy-á]  ni  sókisi.\\
%%LINE3
\AUG{}-7\REL{}.\PRO{}  1.Bill  1\SM{}-\N{}.\PST{}-wash.\CAUS{}-\FV{}  \COP{}  10.sock\\
%%TRANS1
\glt
‘What Bill washed is socks.’\\
%%TRANS2
}
%%EXEND

%%EAX
\ex
%%JUDGEMENT
[*]{
%%LABEL
\label{bkm:Ref98947401:b}  
%%CONTEXT
%%LINE1
%%LINE2
\gll
Sókisi  n’  [\textbf{e-kí}/\textbf{e-zí}   Bíiru  y-aa-yózy-a].\\
%%LINE3
10.sock  \COP{}  {\db}\AUG-7/10.\REL.\PRO{}  1.Bill  1\SM{}-\PST{}-wash.\CAUS{}-\FV{}\\
%%TRANS1
\glt
%%TRANS2
}
%%EXEND

%%EAX
\ex
%%JUDGEMENT
[]{
%%LABEL
\label{bkm:Ref98947401:c}
%%CONTEXT
%%LINE1
%%LINE2
\gll
Sókisi  ni-\textbf{zó}  Bíiru  y-aa-yózy-a.\\
%%LINE3
10.sock  \COP{}-10.\PRO{}  1.Bill  1\SM{}-\PST{}-wash.\CAUS{}-\FV{}\\
%%TRANS1
\glt
‘\textit{Socks} Bill washed.’ / ‘Socks is what Bill washed.’\\
%%TRANS2
}
%%EXEND


\z
\z

There are further indications, however, that the construction in \xref{bkm:Ref98947401:c} is not in fact a reverse pseudocleft, but a noun phrase in the left periphery followed by a basic cleft in which the pronoun is the clefted element. First, the initial NP can be separated from the rest of the construction, as in \xref{bkm:Ref98947671}, where both \textit{amaizi} ‘water’ and \textit{ente} ‘cows’ are topics in the left periphery.

%%EAX
\ea
%%JUDGEMENT
%%LABEL
\label{bkm:Ref98947671}
%%CONTEXT
(Did the cows eat the food I left for them?)\\
%%LINE1
%%LINE2
\gll
A-má-ízi  e-n-te  ni-gwó  z-áá-nyw-a.\\
%%LINE3
\AUG{}-6-water  \AUG{}-10-cow  \COP{}-6.\PRO{}  10\SM{}-\N{}.\PST{}-drink-\FV{}\\
%%TRANS1
\glt
‘It is water that the cows have drunk.’, \\
%%TRANS2
lit. ‘Water, the cows, it is that that they have drunk.’\\
%%EXEND

\z


Second, an optional prosodic break is possible between the initial NP and the rest of the clause, as in \xref{bkm:Ref111495005}.

%%EAX
\ea
%%JUDGEMENT
%%LABEL
\label{bkm:Ref111495005}
%%CONTEXT
%%LINE1
%%LINE2
\gll
Sókisi,  ni-zó  Bíiru  y-aa-yózy-a.\\
%%LINE3
10.socks  \COP{}-10.\PRO{}  1.Bill  1\SM{}-\PST{}-wash.\CAUS{}-\FV{}\\
%%TRANS1
\glt
‘Socks is what Bill washed.’\\
%%TRANS2
lit. ‘Socks, it is that/them that Bill washed.’

%%EXEND

\z


The next task is then to determine the information-structural function of the initial NP. In some contexts it functions as a topic, and can co-occur with the contrastive topic marker, as in \xref{bkm:Ref116287297} and \xref{bkm:Ref116287506}. Note that the dispute in the context shows that the referent is known, and also that the combination of the contrastive marker and cleft in this context result in an emphatic verum interpretation: this is true, end of discussion.

\ea
\label{bkm:Ref116287297}
\begin{xlist}
\exi{A:} Carol baked mandaazi, Liz prepared chapati; and I think Jonah baked pancakes.\\
\exi{B:} No, Kate baked pancakes.\\
\exi{A:} Sure? \\
%%EAX
\exi{B:}
%%JUDGEMENT
%%LABEL
%%CONTEXT
%%LINE1
Kéeti wé niwé yaateek’  óbubânda.\\
%%LINE2
\gll
Kate  w-o  ni-we  a-a-teek-a  o-bu-banda\\
%%LINE3
1.Kate  1-\CM{}  \COP{}-1.\REL{}.\PRO{}  1\SM{}-\PST{}-cook-\FV{}  \AUG{}-14-pancakes\\
%%TRANS1
\glt
‘It is Kate who has made pancakes.’, ‘As for Kate, it is her who made pancakes.’
%%TRANS2
%%EXEND

\end{xlist}
\z

%%EAX
\ea
%%JUDGEMENT
%%LABEL
\label{bkm:Ref116287506}
%%CONTEXT
(Context: There is an argument as to whether it is sheep or cows that Juma grazed.)\\
%%LINE1
Entaama zó nizó yaaríisa.\\
%%LINE2
\gll
e-n-taama  z-o  ni-z-o  a-aa-ri-is-a\\
%%LINE3
\AUG{}-10-sheep  10-\CM{}  \COP{}-10-\REL{}.\PRO{}  1\SM{}-\N{}.\PST{}-{}eat-\CAUS{}-\FV{}\\
%%TRANS1
\glt
‘He has (only) grazed the sheep.’, ‘The sheep, it’s them that he grazed.’\\
%%TRANS2
%%EXEND


\z


The impression of focus on the initial NP in these examples is derived from the fact that the initial NP (e.g. socks) and the clefted pronoun (e.g. \textit{zo}) refer to the same referent. Expressing it in this construction allows for the expression of both properties: it is accessible, topical (as expressed by the NP), but also in exhaustive focus (as indicated by the clefted pronoun).

The initial NP cannot be analysed as a regular focus, since an initial interrogative in this construction is ungrammatical, as seen in \xref{bkm:Ref98947909}. As interrogative pronouns are taken to be in focus, their appearance initially would have been expected had the initial position be one of focus in this construction.

\ea
\label{bkm:Ref98947909}
%%EAX
\ea
%%JUDGEMENT
[*]{
%%LABEL
%%CONTEXT
%%LINE1
%%LINE2
\gll
Enki/ki  ni-kyo  Paméra  a-ryá-téek-a? \\
%%LINE3
what  \COP{}-7\PRO{}  1.Pamela  1\SM{}-\FUT{}-cook-\FV{}\\
%%TRANS1
\glt
‘What is it that Pamela will cook?’\\
%%TRANS2
}
%%EXEND

 %%EAX
\ex
%%JUDGEMENT
[*]{
%%LABEL
%%CONTEXT
%%LINE1
%%LINE2
\gll
Oha  ni-we  o-waa-shohor-a?\\
%%LINE3
1.who  \COP{}-1.\PRO{}  1\SM{}.\REL{}-\N{}.\PST{}-move.out-\FV{}\\
%%TRANS1
\glt
‘Who has moved out?’\\
%%TRANS2
}
%%EXEND

\z
\z

If the initial NP is a topic, as we proposed, it is unexpected that the exhaustive ‘only’ is accepted as a modifier of the initial NP. Nevertheless, this is what we find in \xref{bkm:Ref98948346}, as this associates with focus and not topic.

%%EAX
\ea
%%JUDGEMENT
%%LABEL
\label{bkm:Ref98948346}
%%CONTEXT
(Which animals drank water?)\\
%%LINE1
Entaamá  zonká  ni-zó  zanyw’  ámíizi.
%%LINE2
\gll
e-n-taama  zi-onka  ni-zo  z-aa-nyw-a  a-ma-izi.\\
%%LINE3
\AUG{}-10-sheep  10-only  \COP{}-10.\PRO{}  10\SM{}-\N{}.\PST{}-drink-\FV{}  \AUG{}-6-water\\
%%TRANS1
\glt
‘Only the sheep drank water.’\\
%%TRANS2
%%EXEND


\z


There are two possible options to analyse this: either the construction is grammaticalising to become integrated as a focus construction with a left-peripheral focus position, or the initial NP forms a phrase by itself, comparable to a fragment answer. If the construction is moving to a monoclausal focus construction with an initial focus position (which for some reason excludes interrogatives), then the initial focused constituent must have moved from inside the clause. This movement is expected to show reconstruction effects, i.e. the referent must be interpretable in the position where it moved from. We can test this by using a universal quantifier and a possessive pronoun, as in \xref{bkm:Ref98948563}. If the initial NP with the possessive pronoun has moved and reconstructs, then the pronoun should be able to be bound by the universal quantifier \textit{buri} ‘every’ in the subject, resulting in a reading that each parent loves their own child (the distributive reading). But in fact we only get the non-distributive reading, in which there is one particular child (belonging to a third person) that every parent loves. This shows us that the initial NP is not moved and that the first analysis is not likely to be correct (at this stage of the language – it may of course grammaticalise further).

%%EAX
\ea
%%JUDGEMENT
%%LABEL
\label{bkm:Ref98948563}
%%CONTEXT
%%LINE1
%%LINE2
\gll
[O-mw-ana  wé]  ni-wé   buri   mu-zíir’   a-ríku-kûnd-a.\\
%%LINE3
\AUG{}-1-child  1-\POSS{}.1  \COP{}-1.\PRO{}  every  1-parent  1\SM{}-\IPFV{}-love-\FV{}\\
%%TRANS1
\glt
‘[His/her]\textsubscript{k/*i} child is the one that [every parent]\textsubscript{i} loves.’\\
%%TRANS2
(only non-distributive)

%%EXEND

\z


The second option, the initial NP being a fragment answer, seems to account also for sentences like \xref{bkm:Ref98948713}, where the initial NP is modified by ‘only’ but separated from the rest by the adverb ‘yesterday’.

%%EAX
\ea
%%JUDGEMENT
%%LABEL
\label{bkm:Ref98948713}
%%CONTEXT
%%LINE1
%%LINE2
\gll
E-saatí  z-ônká,  nyómwébazó,  ni-z-ó     zi-gw-ir’  á-hâ-nsi. \\
%%LINE3
\AUG{}-10.shirts   10-only  yesterday   \COP{}-10-\PRO{}     10\SM{}-fall-\PFV{}  \AUG{}-16-down \\
%%TRANS1
    \glt ‘It was only the shirts, yesterday, that fell down.’\\
%%TRANS2
lit. `The shirts, yesterday, it is them that fell down.'\\
%%EXEND

\z

If the initial NP can form a fragment answer, which is then followed up by an explicating cleft, we would expect this construction to be possible in answering a content question, which is indeed the case, as shown in \xref{bkm:Ref98949372} – note that the cleft is not obligatory.

%%EAX
\ea
%%JUDGEMENT
%%LABEL
\label{bkm:Ref98949372}
%%CONTEXT
(Which mats did Jovia weave?)\\
%%LINE1
Emigúfu (niyó Jóviya arukíre).\\
%%LINE2
\gll
e-mi-gufu  ni-yo  Jovia  a-ruk-ire\\
%%LINE3
\AUG{}-4-short  \COP{}-4.\PRO{}  1.Jovia  1\SM{}-weave-\PFV{}\\
%%TRANS1
\glt
‘The short ones (it’s them that she weaved).’\\
%%TRANS2
%%EXEND


\z


The construction is also used in a corrective context, as in \xref{bkm:Ref116288150}.

%%EAX
\ea
%%JUDGEMENT
%%LABEL
\label{bkm:Ref116288150}
%%CONTEXT
(The cook has finally come.)\\
%%LINE1
Íngaaha, mááma niwé yíija, tí mutéeki.\\
%%LINE2
\gll
ngaaha  maama  ni-we  a-aa-ij-a  ti  mu-teeki\\
%%LINE3
no  1.mother  \COP{}-1.\PRO{}  1\SM{}-\N{}.\PST{}-{}come-\FV{}  \NEG{}  1-cook\\
%%TRANS1
\glt
‘No, mother has come, not the cook.’\\
%%TRANS2
%%EXEND


\z


As the pronoun in the basic cleft is interpreted as exhaustive (see \sectref{bkm:Ref111469884}), it is expected that the initial coreferential NP cannot be non-exhaustive, as illustrated in \xref{bkm:Ref111469925} with ‘for example’ and in \xref{bkm:Ref115098927} for ‘even’.

\ea
[]{
\label{bkm:Ref111469925}
(I’m looking for someone who can lend me a pen.)\\
}
%%EAX
\sn
%%JUDGEMENT
[*]{
%%LABEL
%%CONTEXT
%%LINE1
Nka Rónald niw’ áine péeni.\\
%%LINE2
\gll
nka   Ronald  ni-we  a-ine  peeni\\
%%LINE3
like   1.Ronald  \COP{}-1.\PRO{}  1\SM{}.\REL{}-have  9.pen\\
%%TRANS1
\glt
int. ‘For example Ronald is the one who has a pen.’\\
%%TRANS2
}
%%EXEND


\z

%%EAX
\ea
%%JUDGEMENT
[*]{
%%LABEL
\label{bkm:Ref115098927}
%%CONTEXT
%%LINE1
Nab’ ábáaná nibó omushomésa abahiir’ ékarámu.\\
%%LINE2
\gll
na-bo  a-ba-ana  ni-ba-o  o-mu-shomesa  a-ba-h-ire   e-karamu \\
%%LINE3
and-2  \AUG{}-2-children  \COP{}-2-\PRO{}  \AUG{}-1-teacher  1\SM{}-2\OM{}-give-\PFV{} \AUG{}-10.pencil \\
%%TRANS1
\glt
‘It is even children the teacher gave pencils.’\\
%%TRANS2
}
%%EXEND

\z

Which syntactic analysis turns out to be preferable for this construction remains a topic for further research. For now, we conclude that this is not a direct reverse of the pseudocleft (i.e. it is not a copular construction), but that the initial NP may be a topic or a fragment answer, followed by a basic cleft. As for the interpretation, it seems to combine exhaustive focus brought about by the basic cleft with the givenness expressed by the topic in initial position or the simple focus expressed in a fragment answer.

\pagebreak
\section{Object marking}\label{sec:objmarking}

The object marker in Rukiga can in principle not co-occur with the coreferential noun phrase in the same domain and object marking in Rukiga is hence characterised as ``non-doubling" (see \citealt{vanderWal2022} and references therein for discussion on doubling object marking). This is shown in \xref{bkm:Ref111488765}, where the object marker may only be present if the object is dislocated to the right of the adverb ‘today’. 

\ea
\label{bkm:Ref111488765}
%%EAX
\ea
%%JUDGEMENT
[*]{
%%LABEL
\label{bkm:Ref111488765:a}
%%CONTEXT
%%LINE1
%%LINE2
\gll
Píta  y-aa-ka-teek-a  a-ka-húunga  e-ri-zóoba.\\
%%LINE3
1.Peter  1\SM{}-\N{}.\PST{}-12\OM{}-cook-\FV{}  \AUG{}-12-posho  \AUG{}-5-day \\
%%TRANS1
\glt
int. ‘Peter cooked posho today.’\\
%%TRANS2
}
%%EXEND


%%EAX
\ex
%%JUDGEMENT
[]{
%%LABEL
\label{bkm:Ref111488765:b}
%%CONTEXT
%%LINE1
%%LINE2
\gll
Píta  y-aa-ka-teek’  é-ri-zóob’  a-ka-húúnga.\\
%%LINE3
1.Peter  1\SM{}-\N{}.\PST{}-12\OM{}-cook  \AUG{}-5-day  \AUG{}-12-posho\\
%%TRANS1
\glt
‘Peter cooked it today, posho.’\\
%%TRANS2
}
%%EXEND


\z
\z


Note that the verb takes the tonally reduced form in \xref{bkm:Ref111488765}, indicating that the object marker and the object NP are in the same domain. The non-reduced verb form on the other hand is grammatical with the object marker, see \xref{bkm:Ref111488925:a}, which indicates that the verb is final and the following object is not in the same domain.

\ea
\label{bkm:Ref111488925}
%%EAX
\ea
%%JUDGEMENT
[]{
%%LABEL
\label{bkm:Ref111488925:a}
%%CONTEXT
%%LINE1
Bamukomiré Káto.    \jambox*{[no TR]}
%%LINE2
\gll
ba-mu-kom-ire  Káto\\
%%LINE3
2\SM{}-1\OM{}-tie-\PFV{}  1.Kato\\
%%TRANS1
\glt
‘They tied Kato.’\\
%%TRANS2
}
%%EXEND


 %%EAX
\ex
%%JUDGEMENT
[*]{
%%LABEL
\label{bkm:Ref111488925:b}
%%CONTEXT
%%LINE1
Bamukomire Káto.  \jambox*{[TR]}
%%LINE2
\gll
ba-mu-kom-ire  Káto\\
%%LINE3
2\SM{}-1\OM{}-tie-\PFV{}  1.Kato\\
%%TRANS1
\glt
int. ‘They tied Kato.’\\
%%TRANS2
}
%%EXEND


\z
\z

Given these properties, there are two interesting aspects in Rukiga with respect to object marking. The first is that we do not see a lot of object drop, that is, the complete absence of any form referring to the object (compare e.g. with Kîîtharaka in \chapref{ch:5} \parencite{chapters/kirundi} and Makhuwa in \chapref{ch:8} \parencite{chapters/makhuwa}). In a recipe for sorghum porridge, for example, reference to given objects is always marked through an object marker (or subject marker), as can be seen in the fragment in \xref{bkm:Ref135728579}. Whether object drop is possible at all (and if so, when it occurs), remains for further research.

%%EAX
\ea
%%JUDGEMENT
%%LABEL
\label{bkm:Ref135728579}
%%CONTEXT
(When the water (cl 6) cools, you get back to that other bucket (cl 9) with a cooking stick (cl 11) and start stirring to mix the flour (cl 9) with that other porridge (cl 14) so that they both get well cooked.)\\
%%LINE1
Byámara kus’ okwaté gá míízí agáárugirem’ ékyóya o\textbf{ga}shuké mu baketi reero o\textbf{bu}reke buráareho \\
%%LINE2
\gll
bi-aa-mar-a  ku-sya  o-kwat-e  g-a  ma-izi  a-ga-aa-rug-ir-e=mu  e-ki-oya  o-\textbf{ga}-shuk-e  mu  baketi  reero  o-\textbf{bu}-rek-e  bu-raar-e=ho\\
%%LINE3
8\SM{}-\N{}.\PST{}-finish  15-get.ready  2\SG{}.\SM{}-get-\SBJV{}  6-\DEM{}  6-water  \AUG{}-6\RM{}-\N{}.\PST{}-come.from-\APPL{}-\SBJV{}=18  \AUG{}-7-heat  2\SG{}.\SM{}-6\OM{}-pour-\SBJV{} 18 9.bucket then  2\SG{}.\SM{}-14\OM{}-leave-\SBJV{}  14\SM{}-stay.for.night-\SBJV{}=16 \\
%%TRANS1
\glt
‘When they are both well-cooked you get that other water which has been cooled and you pour it in the bucket and leave the porridge to stay for a night.’ \\
%%TRANS2
%%EXEND

%%EAX
\sn
%%JUDGEMENT
%%LABEL
%%CONTEXT
%%LINE1
Nínga \textbf{bw}ába \textbf{bu}rí bwíngí o\textbf{bu}cwánuurire nk’omu bíndi bisafuríya kugira ngu \textbf{bu}hore reeró nyenkyakare mu\textbf{bú}nywe búgiziré kí… burafuka.\\
%%LINE2
\gll
nainga  bw-a-ba  bu-ri  bw-ingi  o-bu-cwanuur-ir-e  nka  o-mu  bí-ndi  bi-safuriya  ku-gira  ngu  bu-hor-e  reero  nyenkyakare  mu-bu-nyw-e  bu-giz-ire  ki…  bu-ra-fuk-a\\
%%LINE3
or    14\SM{}-\N{}.\PST{}-be  14\SM{}-be  14-much  2\SG{}.\SM{}-14\OM{}-reduce-\APPL{}-\SBJV{}  like  \AUG{}-18  8-other  8-saucepan  15-say  \COMP{}  14\SM{}-cool-\SBJV{}  then  tomorrow  2\PL{}.\SM{}-14\OM{}-drink-\SBJV{}  14\SM{}-do-\PFV{}  what...  14\SM{}-\IPFV{}-cool-\FV{} \\
%%TRANS1
\glt
‘Or when the porridge is much, you divide it and put in other big saucepans so that it cools and you drink it tomorrow [the next day] when it has cooled.’
%%TRANS2
%%EXEND

\z

A second interesting point is that despite the fact that the object marker may not double the coreferring noun phrase in the same domain, both \textit{can} co-occur under specific pragmatic contexts, as \citet{SikukuDiercks2021} first noted for Lubukusu \citep[see also][]{LippardEtAlFut}. The interpretation is one of verum, that is, an emphatic focus on the truth of the proposition (\citealt{Höhle1992}, cf. \citealt{RomeroHan2004}), as illustrated in the acceptable and unacceptable contexts in \xref{bkm:Ref116461196}. The first context elicits VP focus, and the second object focus, neither of which is acceptable. Only the corrective verum in the third context is acceptable. Note, however, that this is only possible with the form of the verb that is not tonally reduced, indicating that the object NP is not in the same domain as the verb (compare \xref{bkm:Ref111488765} and \xref{bkm:Ref116461196}).

%%EAX
\ea
%%JUDGEMENT
%%LABEL
\label{bkm:Ref116461196}
%%CONTEXT
(\textsuperscript{\#}Context 1: ‘What did Peter do today?’\\
\textsuperscript{\#}Context 2: ‘What did Peter cook today?’\\
Context 3: ‘I don’t believe that Peter cooked posho today!’)\\
%%LINE1
%%LINE2
\gll
Píta  y-áá-ka-téek-a  a-ka-húúnga  e-ri-zóoba.\\
%%LINE3
1.Peter  1\SM{}-\N{}.\PST{}-12\OM{}-cook-\FV{}  \AUG{}-12-posho  \AUG{}-5-day \\
%%TRANS1
\glt
‘Peter cooked posho today.’ (adapted from \citealt[52]{vanderWalAsiimwe2020})
%%TRANS2
%%EXEND

\z


In addition, object doubling is possible under a mirative or counterexpecation reading, as in \xref{bkm:Ref116461205}, as well as an intensity interpretation \xref{bkm:Ref114323509}.

%%EAX
\ea
%%JUDGEMENT
%%LABEL
\label{bkm:Ref116461205}
%%CONTEXT
(Context: It was expected that the sorghum would be ground well, which was not the case.)\\
%%LINE1
%%LINE2
\gll
W-áá-gu-s-a  o-mu-gúsha  (báasi)!\\
%%LINE3
2\SG{}.\SM{}-\N{}.\PST{}-3\OM{}-{}grind-\FV{}  \AUG{}-{}3-sorghum  (really)\\
%%TRANS1
\glt
‘You have ground the sorghum (what happened? It is not fine-ground)!’\\
%%TRANS2
%%EXEND


\z

%%EAX
\ea
%%JUDGEMENT
%%LABEL
\label{bkm:Ref114323509}
%%CONTEXT
(Context: The sorgum is so fine-ground.)\\
%%LINE1
%%LINE2
\gll
W-áá-gu-s-a  o-mu-gúsha  (buzima)!\\
%%LINE3
2\SG{}.\SM{}-\N{}.\PST{}-3\OM{}-{}grind-\FV{}  \AUG{}-{}3-sorghum  (really)\\
%%TRANS1
\glt
‘You have ground the sorghum (it is fine)!’\\
%%TRANS2
%%EXEND


\z

Further research into the exact contexts in which the object marker can be present is needed, specifically paying attention to what \citet{LippardEtAlFut} call ``emphatic" interpretations, and to the interaction between word order and tonal marking on the verb.

\section{Chapter summary}\label{sec:summary}

The chapter has provided a descriptive analysis of information structure in Rukiga. We note that there are various strategies the language employs, using prosody, morphology, and syntax. One of the strategies is word order. Word order in Rukiga is flexible, and is determined more by discourse roles than grammatical roles. We note that topics are preferred in the preverbal position. For example, in a locative inversion construction, the locative phrase appears in the preverbal position and acts as the topic. In the same vein, in passive constructions, erstwhile objects are promoted to subject and moved to the preverbal position to function as topics. In addition, Rukiga possesses a particle -\textit{o} that marks contrastive topics, and there are other pragmatic interpretations associated with it, namely polarity focus/verum, depreciative, intensive and mirative. Focused elements do not occur in the preverbal domain, but are typically marked in the immediate after verb position. Non-focal and non-topical elements also appear postverbally, for example the subject in a thetic sentence (primarily in locative inversion), and other non-topical non-focal elements appear in the right periphery. Predicate doubling is another strategy that is prevalently used in Rukiga for multiple readings depending on context: in topic doubling, an infinitive functions as a contrastive focus and the resulting interpretations include polarity focus/verum, depreciation, intension, and mirativity (the same interpretations that are encoded by particle \textit{-o}), whereas in-situ doubling with a class 14 nominalisation creates a dismissive reading, and in-situ doubling with \textit{na} and an infinitive is used to indicate an event happening above expectation. Clefts are a common strategy used to express focus as observed in the other languages described in this book. A detailed syntactic account of cleft constructions remains for further research, but we concluded that the basic cleft has an exhaustive interpretation and that the pseudocleft is associated with identification. A third construction that superficially is reminiscent of a reverse pseudocleft was shown to involve an independent (topic or fragment answer) NP followed by a basic cleft in which a coreferential pronoun is clefted. Further research should also consider whether object drop in Rukiga is possible (and if so, under which circumstances), the role of prosody in expressing information structure, and it would also be interesting to conduct a diachronic investigation of particle \textit{-o} and study this particle comparatively between Rukiga and other Bantu languages that have it.

\section*{Acknowledgements}\label{sec:acknowledgements}

This research was supported by an NWO Vidi grant 276-78-001 as part of the Bantu Syntax and Information Structure (BaSIS) project at Leiden University. We thank our speakers Pamellah Geiga Birungi, the late Joel Tumusiime, Ronald Twesigomwe for their enthusiasm and patience in working with us; we thank Nina van der Vlugt and Melle Groen for their work on Rukiga cleft constructions; we thank the BaSIS colleagues for their support; and we thank the reviewers of this chapter for their helpful comments. Any remaining errors are ours alone.

\section*{Abbreviations and symbols}\label{sec:abbsymbs}

Numbers refer to noun classes unless followed by \SG{}/\PL{}, in which case the number (1 or 2) refers to first or second person. High tones are marked by an acute accent; low tones remain unmarked. We stick to orthography as much as possible. Orthographic 〈k〉 and 〈g〉 before [i], as well as 〈ky〉 and 〈gy〉 before other vowels, are pronounced [tʃ] and [dʒ], respectively. Although sometimes speakers pronounce [l], there is no 〈l〉 in orthography (instead 〈r〉 is used). Liaison between words is indicated by an apostrophe. Vowels before a prenasalised consonants and vowels after palatalised and labialised consonants are automatically lengthened, but written with only one symbol.

\begin{multicols}{2}
\begin{tabbing}
MMMM \= ungrammatical\kill
%%% All Leipzig abbreviations are commented out, following the LangSci guidelines of only listing non-Leipzig abbreviations.
* \> ungrammatical\\
\textsuperscript{\#} \> infelicitous in the given \\ \> context\\
*(X) \> the presence of X is obligatory \\ \> and cannot grammatically \\ \> be omitted\\
(*X) \> the presence of X would make \\ \> the sentence ungrammatical\\
(X) \> the presence of X is optional\\
% \APPL{} \> applicative\\
\AUG{} \> augment\\
% \CAUS{} \> causative\\
\CM{} \> contrastive marker\\
\CONN{} \> connective\\
% \COP{} \> copula\\
DAI \> default agreement inversion\\
% \DEM{} \> demonstrative \\
\F{}.\PST{} \> far past\\
% \FUT{} \> future \\
\FV{} \> final vowel\\
% \IMP{} \> imperative (mood)\\
% \IPFV{} \> imperfective aspect\\
LI \> locative inversion\\
% \LOC{} \> locative\\
\MED{} \> medial (demonstrative) \\
\N{}.\FUT{} \> near future tense \\
\N{}.\PST{} \> near past tense\\
% \NEG{} \> negative\\
% \NMLZ{} \> nominaliser\\
\OM{} \> object marker \\
% \PASS{} \> passive\\
% \PFV{} \> perfective \\
% \PL{} \> plural\\
% \POSS{} \> possessive \\
\PRO{} \> pronoun \\
% \PROG{} \> progressive\\
% \PROX{} \> proximal (demonstrative) \\
% \PST{} \> past \\
% \REFL{} \> reflexive\\
% \REL{} \> relative \\
\RM{} \> relative marker \\
% \REL{}.\PRO{} \> relative pronoun \\
% \SBJV{} \> subjunctive \\
% \SG{} \> singular \\
\SM{} \> subject marker\\
TR \> tonal reduction\\
\end{tabbing}
\end{multicols}

\printbibliography[heading=subbibliography,notkeyword=this]\label{sec:refs}
\end{document}
