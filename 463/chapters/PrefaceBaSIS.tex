% \documentclass[output=paper]{langscibook}
% \author{add author}
% \title{Preface}
% \IfFileExists{../localcommands.tex}{
%   \addbibresource{../localbibliography.bib}
%   \usepackage{langsci-optional}
\usepackage{langsci-gb4e}
\usepackage{langsci-lgr}

\usepackage{listings}
\lstset{basicstyle=\ttfamily,tabsize=2,breaklines=true}

%added by author
% \usepackage{tipa}
\usepackage{multirow}
\graphicspath{{figures/}}
\usepackage{langsci-branding}

%   
\newcommand{\sent}{\enumsentence}
\newcommand{\sents}{\eenumsentence}
\let\citeasnoun\citet

\renewcommand{\lsCoverTitleFont}[1]{\sffamily\addfontfeatures{Scale=MatchUppercase}\fontsize{44pt}{16mm}\selectfont #1}
  
%   %% hyphenation points for line breaks
%% Normally, automatic hyphenation in LaTeX is very good
%% If a word is mis-hyphenated, add it to this file
%%
%% add information to TeX file before \begin{document} with:
%% %% hyphenation points for line breaks
%% Normally, automatic hyphenation in LaTeX is very good
%% If a word is mis-hyphenated, add it to this file
%%
%% add information to TeX file before \begin{document} with:
%% %% hyphenation points for line breaks
%% Normally, automatic hyphenation in LaTeX is very good
%% If a word is mis-hyphenated, add it to this file
%%
%% add information to TeX file before \begin{document} with:
%% \include{localhyphenation}
\hyphenation{
affri-ca-te
affri-ca-tes
an-no-tated
com-ple-ments
com-po-si-tio-na-li-ty
non-com-po-si-tio-na-li-ty
Gon-zá-lez
out-side
Ri-chárd
se-man-tics
STREU-SLE
Tie-de-mann
}
\hyphenation{
affri-ca-te
affri-ca-tes
an-no-tated
com-ple-ments
com-po-si-tio-na-li-ty
non-com-po-si-tio-na-li-ty
Gon-zá-lez
out-side
Ri-chárd
se-man-tics
STREU-SLE
Tie-de-mann
}
\hyphenation{
affri-ca-te
affri-ca-tes
an-no-tated
com-ple-ments
com-po-si-tio-na-li-ty
non-com-po-si-tio-na-li-ty
Gon-zá-lez
out-side
Ri-chárd
se-man-tics
STREU-SLE
Tie-de-mann
}
%   \togglepaper[1]%%chapternumber
% }{}
%
% \begin{document}
% \maketitle
% %\shorttitlerunninghead{}%%use this for an abridged title in the page headers
%
% In December 2017, the NWO Vidi project ‘Bantu Syntax and Information Structure’ (BaSIS) started at Leiden University, aiming to better understand the influence that information structure has on the (morpho)syntax of Bantu languages. Having now come to the end of the project, we present our findings in this book: 8 chapters on the languages we have investigated, plus an introduction explaining the basics of information structure and our methodology (which is available for other researchers to use).
%
% We wish to acknowledge the help we received during the project and in the writing of this book. Our heartfelt gratitude in the first place goes to all the speakers who patiently, willingly, and with lots of laughter helped us analyse their languages: Bahati Laikon Mwakasege, Peter Mwasyika Mwaipyana, Yona Mwaipaja, Pamellah Geiga Birungi, the late Joel Tumusiime, Ronald Twesigomwe, Dennis Muriuki Katheru, Philip Murithi Nyamu, Onesmus Mugambi Kamwara, Jonah Tajiri, Tabitha Giti, Jane Gacheri, Constancia Zaida Mussavele, Arlindo João Nhanthumbo, Gomes David Chemane, Hortência Ernesto, Gervásio Chambo, Engrácia Ernesto, Ali Menrage Buananli, Joaquim Nazário, N'gamo Saida Aly, Patient Batal, Edmond Biloungloung, Acteur Enganayat, Daniel Mbel, Angel Molel, Pierre Molel, Étienne Ondjem, and Jeanne Ong'omolaleba.
%
% We also thank András Bárány, Maarten Mous, Lutz Marten, Stavros Skopeteas, Patricia Schneider-Zioga, Radek Šimík, and Malte Zimmermann for their input during the project, and the Leiden University Centre for Linguistics (LUCL) for their support.
%
% Finally, we wish to thank the editors of the series (Michael Marlo and Laura Downing), and the reviewers of the individual chapters: Eva-Marie Bloom-Ström, Koen Bostoen, Silvio Cruschina (x2), Ines Fiedler, Hazel Gray, Mira Grubic, Rozenn Guérois, Hilde Gunnink, Nancy Hedberg, Roland Kiessling, Claudius Kihara, Aurore Montebran, Davety Mpiuka, Steve Nicolle, Jean Paul Ngoboka, Ruth Raharimanantsoa, Teresa Poeta, Saskia van Putten, Daisuke Shinagawa, and Gerrit de Wit. As customary, the chapters and any errors within them are the responsibility of the authors alone.
%
% It is our hope that these chapters may provide insights into the individual languages, form the basis for further comparative work, and inspire others to take information structure into account in their descriptions of other languages. We look forward to seeing further developments in the field of Bantu syntax and information structure.
%
% \noindent
% The BaSIS team\\
% Allen Asiimwe, Patrick N. Kanampiu, Elisabeth J. Kerr, Zhen Li, Amani Lusekelo, Simon Msovela, Nelsa Nhantumbo, Ernest Nshemezimana, and Jenneke van der Wal
%
% \sloppy\printbibliography[heading=subbibliography,notkeyword=this]
% \end{document}
