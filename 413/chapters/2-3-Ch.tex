\chapter{Chorote} \label{ch}

This chapter deals with the historical phonology of Chorote [chor1274] (\sectref{intro-ch}). \sectref{pm-to-ch} discusses the development of PM~consonants, vowels, and prosody from the PM stage to Proto-Chorote. \sectref{ch-dialects} is concerned with the diversification of the Chorote varieties.

For the Iyojwa’aja’ variety, spoken in Argentina, there is a detailed vocabulary by \citet{AG79}, a dictionary by \citet{ND09}, grammatical descriptions by \citet{AG78} and \citet{JC14a}, and a detailed description of its phonology by \citet{JC14b}. For the Iyo’awujwa’ variety, also spoken in Argentina, there is a grammatical description and a vocabulary by \citet{AG83}. \citet{GS10} documents multiple terms for plant species in Iyojwa’aja’ and Iyo’awujwa’. For Manjui, spoken in Paraguay, there is a dictionary by \citet{JC18}, which includes a morphological sketch, and \cits{JC-amer} paper on phonological and phonetic issues (recall that our use of the term ``Manjui'' excludes variety of San Eugenio/San Agustín, see \sectref{intro-ch}). In addition to these sources, we rely on Carol's field notes on all three varieties of Chorote, particularly on Iyojwa’aja’ and Manjui.

The consonantal inventory we assume for Proto-Chorote is given in \tabref{pch-inv-cons}. Note that \intxt{*hw} and \intxt{*hl} are analyzed as complex segments due to their distributional properties,\footnote{Treating them as complex segments rather than clusters allows to explain the existence of forms such as \word{Ijw}{ʔinhlɛ́s}{one’s children} or \wordnl{ʔinhwɛ́s}{one’s wing} without postulating complex onsets or codas. The two-phase realization is especially noticeable after a stressed vowel, where an intrusive ``echo vowel'' often appears, as already noticed by \citet[24–26]{AG83}; see also \citet[80]{JC14b} for acoustic data. For example, \word{Ijw}{táhle}{comes from (a distant place)} usually surfaces as \phonetic{ˈtahăleʔ}.} whereas other similar combinations (\intxt{*ht}, \intxt{*hj}, \intxt{*hm}, etc.) are treated as clusters. In coda position, however, */hw/ and */hl/ are realized as \intxt{*ʍ} and \intxt{*ɬ}, respectively, with significant gestural overlap. The vocalic inventory we assume for Proto-Chorote includes six vowel phonemes, */i~e~a~å~o~u/; the seventh vowel, reconstructed as \intxt{*ᵊ}, was an intrusive (nonphonemic) vowel.

\begin{table}
\caption{Proto-Chorote consonants}
\label{pch-inv-cons}
\fittable{
 \begin{tabular}{rcccccc}
  \lsptoprule
            & labial & dental & alveolar & velar & uvular & glottal\\\midrule
  plain stops & *p & *t & & *k & *q & *ʔ\\
  ejective stops & *p’ & *t’ & *ts’ & *k’ & *q’ & \\
  fricatives & & & *s & & & *h \\
  plain approximants & *w & *l & \multicolumn{2}{c}{*j} & &\\
  glottalized approximants & *ˀw & *ˀl & \multicolumn{2}{c}{*ˀj} & &\\
  preaspirated approximants & *hw & *hl & & & &\\
  plain nasals & *m & *n & & & &\\
  glottalized nasals & *ˀm & *ˀn & & & &\\
  \lspbottomrule
 \end{tabular}
 }
\end{table}

Individual Chorote lects, however, show drastically different  inventories. Their consonant systems lack a velar–uvular distinction; palatalized velars are opposed to plain velars instead. Many other palatalized consonants have arisen by means of palatalization processes, but their synchronic phonological status is disputed. The contemporary Chorote lects no longer retain \intxt{*å} as a speech sound (IPA~*[ɑ]), though \citet{JC14b} does posit an underlying distinction between /a/ and /å/ for Iyojwa’aja’ based on their behavior. In all contemporary lects, /i~u~e~o/ have lowered allophones in certain contexts, and in some cases the lowered allophones of /i~u/ are phonetically very close to the default (non-lowered) allophones of /e~o/.

\section{From Proto-Mataguayan to Proto-Chorote} \label{pm-to-ch}

This section deals with the development of PM~consonants (\sectref{ch-cons}), vowels (\sectref{ch-vow}), and prosody (\sectref{ch-prosody}) from the Proto-Mataguayan stage to Proto-Chorote.

\subsection{Consonants}\label{ch-cons}

The historical development of the PM consonants in Chorote includes the following sound changes: the sound change \sound{PM}{*ts} > \sound{PCh}{*s} (\sectref{ch-ts}), the merger of \sound{PM}{*k} and \sound{PM}{*q} in the coda position (\sectref{ch-k-q}), the unpacking of \sound{PM}{*ɸ} and \intxt{*ɬ} to \sound{PCh}{*hw} and \intxt{*hl}, respectively (\sectref{ch-f}), the merger of the fricatives \sound{PM}{*x}, \intxt{*χ}, and \intxt{*h} > \sound{PCh}{*h} or \intxt{*hw} in certain environments (\sectref{ch-j-jj}), the change of word-initial \sound{PM}{*ji\mbox{-}} to \sound{PCh}{*ʔi\mbox{-}} (\sectref{ch-ji}), the insertion of *[ʔ] word-finally after vowels and \intxt{*j} (\sectref{ch-saltillo}), the sporadic glottalization of sonorants in some words (\sectref{ch-spontaneous-glot}), the glottal dissimilation (\sectref{ch-glot-dissim}), the deglottalization of glottalized non-sonorant codas (\sectref{ch-preglottalized}), the fortition of glottalized fricatives (\sectref{ch-glott-fric}), and the evolution of syllabic consonants (\sectref{ch-syll-c}). The evolution of Proto-Mataguayan consonant clusters is described in \sectref{ch-consonant-dorsal} (for clusters whose second element is a guttural fricative) and \sectref{ch-clusters} (for all other clusters).

\subsubsection{\sound{PM}{*ts}}\label{ch-ts}

\sloppy
\sound{Proto-Mataguayan}{*ts} yielded \sound{PCh}{*s} in both onsets and codas, thus merging with \sound{PM}{*s} (though see \sectref{pch-s} for possible remnants of \intxt{*ts} in the Iyo’awujwa’ variety of Chorote). In the contemporary varieties of Chorote, the pronunciation of its default reflex varies between \phonetic{s}, \phonetic{xs}, and \phonetic{hs} whenever preceded by a vowel, as detailed in \sectref{pch-s}.

\begin{exe}
    \ex \centipede
    \ex \sisinlaw
    \ex \rootn
    \ex \suncho
    \ex \palm
    \ex \dew
    \ex \grandchildmpl
    \ex \chaguark
    \ex \oldn
    \ex \chaniart
    \ex \offspring
    \ex \bow
    \ex \majan
    \ex \quick
    \ex \jabiru
    \ex \starn
    \ex \gutscw
    \ex \limpkin
    \ex \eyelash
    \ex \basetrunk
    \ex \trunk
    \ex \smoke
    \ex \plits
    \ex \demts
    \ex \tsaqaq
    \ex \chaja
    \ex \spillcw
    \ex \duraznillo
    \ex \silkfloss
    \ex \fullriver
    \ex \tsofa
    \ex \redbrocket
    \ex \healthy
    \ex \knee
    \ex \peccary
    \ex \caracara
    \ex \chaguara
    \ex \wildpepper
\end{exe}

The Iyojwa'aja' reflexes suggest that the deaffrication may have failed to apply between a \intxt{*j} and a vowel. We propose that Proto-Chorote */s/ was articulated as *[ts] in that environment, and reconstruct \word{PCh}{*ˀ[n]ǻjtsiʔ}{to feel disgust} and \wordnl{*\mbox{-}kéjtsås}{grandchildren} (underlying representations: */n\mbox{-}ʔǻjsi/, */\mbox{-}kéjsås/).

\subsubsection{\sound{PM}{*k}, \intxt{*q}, and their glottalized counterparts}\label{ch-k-q}

This subsection deals with the development of \sound{Proto-Mataguayan}{*k(’)} and \intxt{*q(’)} in Proto-Chorote.

The Proto-Chorote reflexes of these sounds in the onset position are represented in this book as \intxt{*k(’)} and \intxt{*q(’)}, respectively. It is in fact likely that \sound{PCh}{*k(’)} was articulated as a prevelar stop (IPA [k̟(’)]) in onsets, since contemporary Chorote lects show palatalized reflexes in a development shared with Wichí: [kʲ] (\sectref{ch-k}) for the plain stop and [kʲ’] or [ʔʲ] (\sectref{ch-k'}) for the ejective stop. In addition, [k̟(’)] is still a usual realization of the reflex of \sound{PCh}{*k(’)} in Manjui before [e], and before in all Chorote lects before [i]. We do not reconstruct \sound{PCh}{*k} and \intxt{*k’} as \intxt{*kʲ} and \intxt{*kʲ’}, respectively, because these phonemes were subject to the so-called first palatalization, which applied independently across the differentiated Chorote lects (\sectref{ch-pal1}). 
Similarly, we propose that \sound{PCh}{*q(’)} was articulated as uvular, even though its reflexes in the daughter lects are sometimes articulated as velar in the contemporary varieties of Chorote (therefore, the velar/uvular contrast is no longer existent in contemporary Chorote). Reconstructing a uvular value for \sound{PCh}{*q(’)} helps to account for its failure to undergo the first palatalization in the contemporary Chorote varieties (\sectref{ch-pal1}) and for the lowering effect it causes in the preceding vowels (\sectref{ch-lowering}). Also note that in early loanwords from Spanish /k/ is rendered as modern \intxt{kʲ} (from \sound{PCh}{*k}) rather than \intxt{k} (from \sound{PCh}{*q}), as in \wordng{Ijw}{wákʲa} from \word{Spanish}{vaca \phonetic{ˈβ̞aka}}{cow} \citep[101, fn. 37]{JC14b}.

In a number of cases, however, \sound{Proto-Mataguayan}{*k} is reflected as \sound{PCh}{*q}.\footnote{We do not include the pair \wordng{PCh}{*taqám} \recind \word{PW}{*tákʲam}{pacu fish}, where in addition to the anomalous correspondence \sound{PCh}{*q} \recind \sound{PW}{*kʲ} one finds a mismatch between the placement of the accent. These words are likely related via borrowing and are not true cognates.} This is likely regular when \sound{PM}{*k} occurs as part of the cluster \intxt{*kh} word-medially, as in \REF{k-k-kha}, possibly due to the fact that \intxt{*k} may still have been syllabified as a coda when the merger of \intxt{*k} and \intxt{*q} took place (see below in this subsection; later on, clusters of the shape \intxt{*Ch} typically underwent metathesis; see \sectref{ch-consonant-dorsal}).\footnote{We do not rule out the possibility that the cluster \intxt{*kh} should also be reconstructed in the Proto-Mataguayan terms for\gloss{wild cat} (\wordng{PM}{*slǻqhaj} in our current proposal) and\gloss{fog} (\wordng{PM}{*xnáqhaj} in our current proposal), which could allow including the Maká homonyms \wordnl{xunkhaj}{wild cat} and \wordnl{xunkhaj}{fog} into the respective etymologies (in our current proposal, both are tentatively considered loans from Nivaĉle). In both cases, one finds \sound{PCh}{*hq} and \sound{PW}{*qh}, which could hypothetically be considered regular reflexes of \sound{PM}{*kh} and not only \sound{PM}{*qh}.} In other examples, \sound{Proto-Mataguayan}{*k} is backed to \intxt{*q} before the vowel \sound{PCh}{*u}.

\begin{exe}
    \ex \kha \label{k-k-kha}
    \ex \vomitv
    \ex \vomitn
    \ex \feed \label{k-k-feed}
\end{exe}

This exceptional development is not shared with Wichí, and the backing of \sound{PM}{*k} before \intxt{*u} cannot be viewed as regular, because numerous counterexamples are known. Compare especially \REF{k-kj-eatkun} with its causative \REF{k-k-feed}.

\begin{exe}
    \ex \eatkun \label{k-kj-eatkun}
\end{exe}

There is also a very rare correspondence between \sound{Ijw}{kʲ} and \sound{I'w/Mj}{k}, which is attributed to \sound{PCh}{*kw} in this book. This cluster goes back to \sound{PM}{*kɸ} and will be discussed in \sectref{ch-clusters}.

In the coda position, \sound{PM}{*k} and \intxt{*q} merged in Proto-Chorote. It is unclear whether the resulting sound was articulated as velar or uvular; we symbolize it as \sound{PCh}{*k}.

\begin{exe}
    \ex \food
    \ex \honeycomb
    \ex \goaway
    \ex \mortar
    \ex \hidev
    \ex \palm
    \ex \thread
    \ex \yicalhuk
    \ex \powder
    \ex \feces
    \ex \zorzal
    \ex \wildmanioc
    \ex \rope
    \ex \fence
    \ex \belt
    \ex \cat
    \ex \river
    \ex \plate
    \ex \blind
    \ex \unclesg
    \ex \chaja
    \ex \duraznillo
    \ex \leniosasg
    \ex \badmood
    \ex \allrcpr
    \ex \headn
    \ex \straw
    \ex \palosanto
    \ex \firewoodhuk
    \ex \cord
\end{exe}

It is often possible to determine whether \sound{PCh}{*k} in the stem-final position goes back to \sound{PM}{*k} or \intxt{*q} by adding a vowel-initial suffix: instances of \sound{PCh}{*k} that go back to \sound{PM}{*k} expectedly resyllabify as \sound{PCh}{*k} (which yields \intxt{kʲ} in the contemporary Chorote varieties), whereas those instances of \sound{PCh}{*k} that go back to \sound{PM}{*q} resyllabify as \sound{PCh}{*q} (which yields \intxt{k} in the contemporary Chorote varieties). In \REF{ex:k-kj-alt:ch}, this is shown for the Iyojwa'aja' reflexes of \word{PM}{*h\mbox{-}åk}{I went away} and \wordnl{*ɬ\mbox{-}åq}{her/his food}.

\booltrue{listing}
\ea\label{ex:k-kj-alt:ch}
Iyojwa'aja'
    \begin{xlist}
        \ex \wordnl{ʔá\mbox{-}k}{I went away} / \wordnl{ʔá\mbox{-}kʲ\mbox{-}eʔ}{then I went away}\\
        \ex \wordnl{hl-ák}{her/his food} / \wordnl{hl-ák-e}{with her/his food}
    \end{xlist}
\z
\boolfalse{listing}

\subsubsection{\sound{PM}{*ɸ} and \intxt{*ɬ}}\label{ch-f}

\sound{Proto-Mataguayan}{*ɸ} and \intxt{*ɬ} unpack to \sound{PCh}{*hw} and \intxt{*hl}, respectively, in onsets, and yield \intxt{*ʍ} and \intxt{*ɬ}, respectively, in codas. We consider \intxt{*ʍ} and \intxt{*ɬ} to be positionally conditioned allophones of */hw/, */hl/. The two-phase realization of \sound{PCh}{*hw} and \intxt{*hl} is especially noticeable in the daughter languages after a stressed vowel, where an intrusive ``echo vowel'' often appears, as already noticed by \citet[24–26]{AG83}; see also \citet[80]{JC14b} for acoustic data. That way, \word{Ijw}{táhleʔ}{comes from (a distant place)} usually surfaces as \phonetic{ˈtahăleʔ}, and \word{Ijw}{tɔ́hwe}{is far away from} as \phonetic{ˈtɔhɔ̆we}.

Some examples illustrating the evolution of \sound{PM}{*ɸ} follow. The major allophones represented in our notation include \intxt{*hw} (in onsets) and \intxt{*ʍ} (in codas). \citet[20–21]{AG83} describes its pronunciation in onsets as varying between [fʷ], [xʷ], and [hʷ] in Iyojwa’aja’, whereas [xʷ] is reported as the predominating allophone in Iyo’awujwa’ and Manjui (in the latter variety, [f] and [xʷ] are reported as rare variants). In our data, [hw] (or -- less frequently -- [xw]) is the default realization of /hw/ in onsets in all Chorote varieties, whereas before a pause /hw/ may surface as [wʍ] or [wh] at least in Iyojwa’aja’ \citep[87]{JC14b}.

\begin{exe}
    \ex \wing
    \ex \companion
    \ex \coal
    \ex \disease
    \ex \shoulder
    \ex \shoulderblade
    \ex \firef
    \ex \centipede
    \ex \cutdown
    \ex \algarrobof
    \ex \algarrobot
    \ex \rightn
    \ex \tell
    \ex \sisinlaw
    \ex \soninlaw
    \ex \fieldn
    \ex \flyv
    \ex \mortar
    \ex \rootn
    \ex \notafraid
    \ex \coldweather
    \ex \hidev
    \ex \pocote
    \ex \dreamv
    \ex \dreamn
    \ex \crab
    \ex \ankle
    \ex \throwpush
    \ex \fart
    \ex \neighbor
    \ex \spouse
    \ex \durmili
    \ex \chachalaca
    \ex \acquainted
    \ex \spend
    \ex \suckb
    \ex \tsofa
    \ex \woman
    \ex \ashamedcw
    \ex \pigeon
\end{exe}

The evolution of \sound{PM}{*ɬ} is exemplified below. Note that the Chorote sequence /hl/ is represented as \intxt{ɬ} even in onsets by \citet{LC-VG-07}. \citet[628]{LC-VG-10} further state that [hl], [xl], and [l] are innovative realizations found in the speech of younger Iyo’awujwa’ speakers. We surmise that \cits{LC-VG-10} attestation of [ɬ] in older speakers' speech reflects the pronunciation of individuals bilingual in Chorote and Wichí or Nivaĉle, since in our data [ɬ] occurs only as an allophone of /hl/ in coda position, and [hl] -- or even [hV̆l] after stressed vowels, as stated above -- is the default realization of /hl/, attested by Carol in old speakers' speech (more than 60 years old) in all varieties of Chorote. \citet[26]{AG78,AG79,AG83} also documents it as \intxt{ˣl} or \intxt{xVl} (and not as [ɬ]). Before a pause, /hl/ may surface as [ll̥] or [l] in Iyojwa’aja’ \citep[87]{JC14b}.

\begin{exe}
    \ex \burn
    \ex \mortar
    \ex \dreamv
    \ex \dreamn
    \ex \breath
    \ex \redquebracho
    \ex \answer
    \ex \flu
    \ex \demlh
    \ex \louse
    \ex \defecate
    \ex \lightfire
    \ex \firewoodlhet
    \ex \thread
    \ex \yicalhuk
    \ex \daylhuma
    \ex \girl
    \ex \dayworld
    \ex \rain
    \ex \sprout
    \ex \carrysh
    \ex \spinsew
    \ex \silkfloss
    \ex \tired
    \ex \rhea
    \ex \rib
    \ex \climb
    \ex \ask
    \ex \fatalha
    \ex \iguana
    \ex \ashamedcw
    \ex \urinate
    \ex \urine
    \ex \puma
\end{exe}

An exception arises when \sound{PM}{*ɬ} was syllabified as a nucleus in the protolanguage: in this case, one finds \intxt{*hᵊ} (see \sectref{ch-syll-c}).

\subsubsection{\sound{PM}{*h}, \intxt{*x} and \intxt{*χ}}\label{ch-j-jj}

In most cases, \sound{Proto-Mataguayan}{*x} and \intxt{*χ} yielded \sound{PCh}{*h} both in onsets and codas, thus merging with \sound{PM}{*h}. In the contemporary Chorote varieties, the reflexes of \sound{PCh}{*h} are typically articulated as [h] or [x],\footnote{\citet{AG83} represents the phoneme in question as /x/ in all three contemporary varieties. \citet[79]{JC14b} notes that it patterns with /ʔ/ in being transparent to a specific kind of vowel assimilation, but at the same time it also patterns with supraglottal consonants in being subject to palatalization. In this book, we follow \citep{JC14b} in conventionally representing the segment in question as /h/ in all Chorote varieties as well as in Proto-Chorote.} except in certain environments where \sound{PCh}{*h} is altogether lost (\sectref{ch-caida-de-h-final}--\ref{ch-caida-de-h-hw-hl}).

Some examples showing the default reflex of \sound{PM}{*x} in Chorote follow.

\begin{exe}
    \ex \bite
    \ex \cutdown
    \ex \rightn
    \ex \fieldn
    \ex \truev
    \ex \hole
    \ex \arrowkaxe
    \ex \youngerbro
    \ex \wash
    \ex \bow
    \ex \languageword
    \ex \smelln
    \ex \pathn
    \ex \abdcavity
    \ex \carrysh
    \ex \dig
    \ex \burrow
    \ex \stagnant
    \ex \night
    \ex \headn
    \ex \jelayuk
    \ex \vrbpl
    \ex \grass
    \ex \maguari
    \ex \skin
\end{exe}

The following examples show the default reflex of \sound{PM}{*χ} in Chorote.

\begin{exe}
    \ex \fatv
    \ex \crab
    \ex \north
    \ex \suncho
    \ex \sandisaj
    \ex \dividev
    \ex \barnowl
    \ex \oldn
    \ex \quick
    \ex \jabiru
    \ex \anteater
    \ex \pseudo
    \ex \shoot
    \ex \smoke
    \ex \fullriver
    \ex \tired
    \ex \rhea
    \ex \night
    \ex \tuscaf
    \ex \caracara
    \ex \jararaca
    \ex \peccary
    \ex \mistolf
    \ex \hurt
    \ex \argentineboa
    \ex \chaguara
    \ex \wildbean
    \ex \widower
    \ex \firei
    \ex \bro
    \ex \puma
\end{exe}

After rounded vowels, special reflexes are found. In that position, \sound{PM}{*χ} changes to \sound{PCh}{*hw} if a vowel follows, but to \intxt{*h} in the coda position.

\begin{exe}
    \ex \najendup
    \ex \centipede
    \ex \many
    \ex \deep
    \ex \far
    \ex \largefat
    \ex \snakeatuj
\end{exe}

In fact, \intxt{*h} and \intxt{*hw} alternate synchronically in such cases in Chorote, as the following examples show.

\booltrue{listing}
\ea\label{ex:h-hw:i'w}
Iyojwa'aja' \citep[152, 157]{ND09}
    \begin{xlist}
        \ex \wordnl{wúh}{it is big} / \wordnl{wúhw\mbox{-}aˀm}{it is thick}\\
        \ex \wordnl{tɔ́h\mbox{-}}{it is high, tall} / \wordnl{tɔ́hw-e}{it is far}, \wordnl{tɔ́hw-iʔ}{it is deep}
    \end{xlist}
\z

\ea\label{ex:h-hw:mj}
Manjui \citep{JC18}
    \begin{xlist}
        \ex \wordnl{ʔa\mbox{-}tɔ́h\mbox{-}ʔiˀm}{I am far from} / \wordnl{ʔa\mbox{-}tɔ́hw-ej}{it is far from}, \wordnl{ʔa\mbox{-}tɔ́hw-aˀm}{s/he is far from}\\
        \ex \wordnl{wúh}{it is big} / \wordnl{wúhw\mbox{-}aˀm}{it is thick}\\
    \end{xlist}
\z
\boolfalse{listing}

By contrast, \sound{PM}{*x} changes to PCh~*/hw/ only after \intxt{*u}, but not after \intxt{*o}, and in this case it is irrelevant whether the segment in question is syllabified as an onset or a coda (in the latter case, the allophone \intxt{*ʍ} occurs). That way, \sound{PM}{*x} merges with \sound{PM}{*ɸ} when preceded by an \intxt{*u}.

\begin{exe}
    \ex \eatvt
    \ex \aunt
\end{exe}

There is some evidence that suggests that word-initial guttural fricatives are deleted if the syllable is unstressed, as in the first-person active suffix \sound{PM}{*ha\mbox{-}}, whose Chorote reflex is \sound{PM}{*ʔa\mbox{-}}. If what follows is a rounded vowel followed by \intxt{*n}, some modifications may take place: the unstressed sequence \sound{PM}{*X₁₃on\mbox{-}} is reflected as \sound{PCh}{*ʔån\mbox{-} \recind *ʔan\mbox{-}}, as in \REF{hon-an-nightncw}--\REF{hon-an-earthcw}, and the sequence \sound{PM}{*X₁₃un\mbox{-}} as \sound{PCh}{*ʔin\mbox{-}}, as in \REF{hon-an-tuscaf}--\REF{hon-an-tuscag}. Guttural fricatives are also deleted in word-initial consonant clusters, as discussed in \sectref{ch-consonant-dorsal}.

\begin{exe}
    \ex \shaman
    \ex \nightncw \label{hon-an-nightncw}
    \ex \earthcw \label{hon-an-earthcw}
    \ex \tuscaf \label{hon-an-tuscaf}
    \ex \tuscat \label{hon-an-tuscat}
    \ex \tuscag \label{hon-an-tuscag}
\end{exe}

Guttural fricatives are also sometimes deleted in intervocalic position in unstressed syllables after vowels such as \intxt{*a} and \intxt{*o}. The following vowel is assimilated to the preceding low vowel, and the resulting vowel sequence is exceptionally not resolved by an automatic glottal stop (Chorote does not otherwise allow onsetless syllables).

\begin{exe}
    \ex \mouth
    \ex \watch
    \ex \earcw
\end{exe}

As a consequence of the intervocalic loss of guttural fricatives, Chorote shows synchronically active alternations between \sound{PCh}{*h} and zero at morpheme boundaries.

\ea
Iyojwa'aja' \citep{JC14a}
    \begin{xlist}
        \ex \gll ʔi-mʲá-jiˀn-eʔ~/i\mbox{-}må\mbox{-}hajin\mbox{-}ʔe/\\
        3.\textsc{i.rls}-sleep-\CAUS-\APPL:punctual\\
        \glt `s/he makes sleep'
        \ex \gll ʔi-ˀjá-jihn-iʔ~/i\mbox{-}ˀjå\mbox{-}hajin\mbox{-}hi(j)/\\
        3.\textsc{i.rls}-drink-\CAUS-\APPL:inside\\
        \glt `s/he gives to drink'
    \end{xlist}
\z

\ea
Iyo'awujwa' \citep[105]{AG83}
    \begin{xlist}
        \ex \gll -má-juʔ~/\mbox{-}ma\mbox{-}haju/\\
        -sleep-\textsc{desid}\\
        \glt `to feel sleepy'
    \end{xlist}
\z

However, not all suffixes are subject to the \intxt{*h}\mbox{-}loss: the \intxt{*h} at the left margin of applicatives and some other suffixes is never deleted in Iyojwa'aja'. This is the case in the verbal plural or `downwards' applicative suffix /\mbox{-}hen/, the `inside' applicative suffix /\mbox{-}hi(j)/, and the locative/dative applicative suffix /\mbox{-}håm/.\footnote{In Manjui, unlike Iyojwa’aja’, \intxt{*h} is lost in such cases, as in \wordnl{ʔi\mbox{-}ˀjé\mbox{-}ejʔ}{s/he drinks}, but this must be a post-Proto-Chorote development. Like in Iyojwa'aja', the vowel that follows \intxt{*h} regularly assimilates to the one that precedes it.} Note that the PM etyma of these suffixes contain a velar fricative, which could be a coincidence or not (by contrast, the suffixes where \intxt{*h} is lost after a low vowel go back to \intxt{*h}\mbox{-}initial morphemes of Proto-Mataguayan, such as \wordnl{*$=$hajuʔ}{desiderative}).

\ea
Iyojwa'aja' \citep{JC14a}
    \begin{xlist}
        \ex \gll ʔi-ˀjá-haʔ~\recind~ʔi-ˀjá-heʔ~/i\mbox{-}ˀjå\mbox{-}hi(j)/\\
        3.\textsc{i.rls}-drink-\APPL:inside\\
        \glt `s/he drinks'
    \end{xlist}
\z

Another instance where \intxt{*h} is preserved intervocalically after a low vowel is at the left margin of roots (perhaps due to the fact that the syllable in question is typically stressed: \word{PCh}{*ʔa\mbox{-}hǻåkeʔ}{your ditch}, \wordnl{*ʔa\mbox{-}hétek}{your head}, \wordnl{*ʔa\mbox{-}hóʔ}{I go}) and at the right margin of suffixes when these are followed by a vowel, such as the Iyojwa’aja’ first-person plural active suffix \intxt{\mbox{-}ah\mbox{-}}, incompletive \intxt{\mbox{-}tah\mbox{-}}, and in the applicatives of the shape \intxt{\mbox{-}ah\mbox{-}} (either underlying or derived by translaryngeal assimilation).

\subsubsection{\sound{PM}{*ji\mbox{-}}}\label{ch-ji}

The sequence \sound{PM}{*ji} is reflected as \sound{PCh}{*ʔi} in the word-initial position.

\begin{exe}
    \ex \dew
    \ex \wax
    \ex \water
    \ex \mancw
    \ex \truev
\end{exe}

When followed by a glottalized consonant and a low vowel (\sound{PM}{*a} or \intxt{*å}, but not \intxt{*ä}), \sound{PM}{*ji} > \intxt{*ʔi} changed to \sound{PCh}{*ʔa} word-initially (\sectref{pm-ch-ji-a}).

\begin{exe}
    \ex \jaguar
    \ex \treen
    \ex \vulture
\end{exe}

Word-medially, no change occurs.

\begin{exe}
    \ex \costumepl
    \ex \bloodpl
\end{exe}

\subsubsection{*\phonetic{ʔ}-insertion}\label{ch-saltillo}

A glottal stop is inserted after word-final vowels and after the approximant \intxt{*j} in Chorote, in stressed and unstressed syllables alike. The opposition \intxt{*ʔ} vs zero is thus neutralized in Proto-Chorote word-finally. \citet[85--89]{JC14b} argues that even synchronically the word-final instances of [ʔ] in Iyojwa’aja’ are best analyzed as inserted, whereas words that phonetically end in a vowel or a non-glottalized sonorant actually end in an underlying /h/, which is deleted before a pause in unstressed syllables.

\begin{exe}
    \ex \lick
    \ex \thorne
    \ex \namen
    \ex \algarrobof
    \ex \flyv
    \ex \elbow
    \ex \mancw
    \ex \tooln
    \ex \hand
    \ex \armadillo
    \ex \bottomn
    \ex \hornclub
    \ex \coldn
    \ex \squash
    \ex \daylhuma
    \ex \sleep
    \ex \savannahhawk
    \ex \newadj
    \ex \penis
    \ex \heel
    \ex \rain
    \ex \medicine
    \ex \costume
    \ex \wildcat
    \ex \movev
    \ex \worm
    \ex \belly
    \ex \neck
    \ex \price
    \ex \woman
    \ex \juice
    \ex \ripe
\end{exe}

The glottalized approximant \sound{PM}{*ˀj} is likewise reflected as \sound{PCh}{*jʔ} word-finally, thus merging with \sound{PM}{*j}.

\begin{exe}
    \ex \yicaay
    \ex \notafraid
    \ex \sunn
\end{exe}

\subsubsection{Sporadic glottalization}\label{ch-spontaneous-glot}
In a very restricted number of roots, Chorote has a glottalized sonorant where other Mataguayan languages have a plain one. We attribute this sound correspondence to a sporadic sound change whereby some sonorants irregularly became glottalized in Chorote.

\begin{exe}
    \ex \drinkv
    \ex \water
    \ex \many
    \ex \leg
\end{exe}

An anonymous reviewer brought our attention to the fact that sporadic glottalization seems to affect forms that otherwise contain \sound{PCh}{*ʔ}, but we have been unable to formulate a precise predictor of the process in question in terms of a regular, contextually conditioned sound change.

\subsubsection{Glottal dissimilation}\label{ch-glot-dissim}

When two consecutive syllables have glottalized consonants as their onsets in PM, Chorote deglottalizes the onset of the first syllable in a development shared with Wichí (\sectref{wi-glot-dissim}). \REF{ch-degl-hornero} shows some irregularities regarding the place of articulation of the dissimilating consonants.

\begin{exe}
    \ex \monkparakeet
    \ex \hornero \label{ch-degl-hornero}
    \ex \saliva
    \ex \hiccup
\end{exe}
    
\subsubsection{Deglottalization of preglottalized codas}\label{ch-preglottalized}

Most preglottalized codas of Proto-Mataguayan merge with their plain counterparts in Chorote by means of deglottalization. This includes the codas \intxt{*ˀp}, \intxt{*ˀt}, \intxt{*ˀts}, \intxt{*ˀk}, \intxt{*ˀɸ}, \intxt{*ˀɬ}, \intxt{*ˀs}, \intxt{*ˀx}, \intxt{*ˀχ}, and \intxt{*ˀj} (the latter coda actually yields \sound{PCh}{*jʔ}, but so does plain \sound{PM}{*j} in the word-final position thanks to the \intxt{*ʔ}\mbox{-}insertion process).

\begin{exe}
    \ex \honeycomb
    \ex \drinkn
    \ex \son
    \ex \yicaay
    \ex \cutdown
    \ex \fieldn
    \ex \notafraid
    \ex \hidev
    \ex \fart
    \ex \dew
    \ex \wax
    \ex \water
    \ex \tail
    \ex \hole
    \ex \sunn
    \ex \answer
    \ex \oldn
    \ex \wash
    \ex \winter
    \ex \languageword
    \ex \thread
    \ex \yicalhuk
    \ex \powder
    \ex \nose
    \ex \smelln
    \ex \lippaset
    \ex \fence
    \ex \lid
    \ex \starn
    \ex \vein
    \ex \sprout
    \ex \abdcavity
    \ex \suckb
    \ex \shoot
    \ex \spinsew
    \ex \carrysh
    \ex \dig
    \ex \blind
    \ex \uncle
    \ex \duraznillo
    \ex \nest
    \ex \badmood
    \ex \burrow
    \ex \walk
    \ex \climb
    \ex \night
    \ex \headn
    \ex \earth
    \ex \palosanto
    \ex \sandyplace
    \ex \firewoodhuk
    \ex \knee
    \ex \jaguar
    \ex \snakeatuj
    \ex \chest
\end{exe}

By contrast, the examples below show that \sound{PM}{*ˀm}, \intxt{*ˀn}, \intxt{*ˀl} are preserved in Chorote. In Manjui and most likely in Iyo'awujwa', they still contrast with their non-glottalized equivalents. Iyojwa'aja' has innovated in that all word-final sonorants are now glottalized in that language, and the glottalization has ceased to be contrastive in that position.

\begin{exe}
    \ex \brightness
    \ex \pronominal
    \ex \locustcw
    \ex \thorncutjan
    \ex \defecate
    \ex \kingvulture
    \ex \meat
\end{exe}

\subsubsection{\sound{PM}{*ɸ’}, \intxt{*s’}, \intxt{*ɬ’} > \sound{PCh}{*p’}, \intxt{*t’}}\label{ch-glott-fric}

Another sound change in Chorote, shared with Wichí and Nivaĉle but not with Maká, consists of the fortition of the Proto-Mataguayan glottalized fricatives (phonologically possibly analyzable as tautosyllabic sequences of a fricative and a glottal stop) to glottalized stops: \sound{PM}{*ɸ’}, \intxt{*s’}, \intxt{*ɬ’}~>~\sound{PCh}{*p’}, \intxt{*ts’}, \intxt{*t’}. (The sequence \intxt{*kɸ’}, however, changed to \sound{PCh}{*k’w} or possibly \intxt{*k’}.)

\begin{exe}
    \ex \poor
    \ex \samsam
    \ex \femalebreastits
    \ex \skinits
    \ex \meatits
    \ex \juiceits
    \ex \urinateyou
    \ex \urineits
\end{exe}

As a result of the sound change \sound{PM}{*ɬ’}~>~\intxt{*t’}, Chorote now displays a morphophonological rule which converts the underlying sequence */hl+ʔ/ into \intxt{*t’} (rather than \intxt{ɬ’}, as in Maká). The rule is no longer entirely productive in Chorote, since the sequence of /hl/ and /ʔ/ actually yields \intxt{hʔl} at the stem--suffix/enclitic boundary, as in Iyojwa'aja'~/táhl+ʔe/ → \wordnl{táhʔleʔ}{exits from}.

\subsubsection{Syllabic consonants}\label{ch-syll-c}

The syllabic consonants of Proto-Mataguayan are reflected in Chorote as sequences of the shape \sound{PCh}{*Cᵊ} (see \sectref{pm-ch-ep-v} on the status of \sound{PCh}{*ᵊ}), except for the syllabic nasal \intxt{*n̩}. This is seen in the allomorphy pattern of several prefixes, which show up as \sound{PCh}{*Cᵊ\mbox{-}} before supraglottal consonants, but as \sound{PCh}{*C\mbox{-}} before vowels (a position where the prefixes in question were not syllabic in PM) and before glottal consonants. In addition, the prefixes in question fuse with a stem-initial \intxt{*ʔ\mbox{-}}, resulting in a glottalized consonant (see also \sectref{ch-glott-fric}). Prefixes that show such allomorphy include the third-person T-class verbal prefix (\wordng{PCh}{*tᵊ\mbox{-} / *t\mbox{-} / *t\mbox{-}’…}), the third-person possessive prefix (\wordng{PCh}{*hᵊ\mbox{-} / *hl\mbox{-} / *t\mbox{-}’…}), the second-person active prefix (\wordng{PCh}{*hᵊ\mbox{-} / *hl\mbox{-} / *hᵊt\mbox{-}’…}), the feminine prefix in demonstratives (\wordng{PCh}{*ha\mbox{-} \recind *hå\mbox{-} / *hl\mbox{-}}).\footnote{We presently have no explanation for the occurrence of a low vowel -- as opposed to \intxt{*ᵊ} -- in the preconsonantal allomorph of the feminine prefix in demonstratives.} A similar pattern is seen in the first-person inactive realis prefix \wordng{PCh}{*sᵊ\mbox{-} / *s\mbox{-} / *ts\mbox{-}’…}, though its Proto-Mataguayan etymon is not known to have contained a syllabic consonant. The syllabic allomorphs in each Chorote lect are illustrated below; note that \sound{PCh}{*ᵊ} is typically reflected as \intxt{i} in the contemporary varieties.

\ea\label{ex:syllc:ijw}
Iyojwa'aja' \citep{JC14a}
    \begin{xlist}
        \ex \gll ti-més\\
            3.\textsc{t.rls}-be\_two\\
            \glt `they are two'
        \ex \gll ti-lʲákiˀn\\
            3.\textsc{t.rls}-play/dance\\
            \glt `s/he plays/dances'
        \ex \gll ta-kásit\\
            3.\textsc{t.rls}-stand\\
            \glt `s/he stands'
        \ex\gll hi-kʲóʔ \\
            3.{\textsc{poss}}-hand\\
            \glt `his/her hand'
        \ex\gll hi-tʲét-e \\
            2.{\textsc{act}}-throw-\APPL\\
            \glt `you throw it for her/him'
        \ex\gll ha-na\\
            {\textsc{f}}-{\textsc{dem}}:outside\_hands'\_reach\\
            \glt `this.\textsc{f} (outside one’s hands’ reach)'
    \end{xlist}
\z

\ea\label{ex:syllc:i'w}
Iyo'awujwa' \citep[66, 74, 75]{AG83}
    \begin{xlist}
        \ex \gll ti-lákʲen\\
             3.\textsc{t.rls}-play\\
             \glt `s/he plays'
        \ex \gll te-kénisʲen\\
             3.\textsc{t.rls}-sing\\
             \glt `s/he sings'
        \ex\gll hi-póʔo\\
            3.{\textsc{poss}}-heel\\
            \glt `his/her heel'
        \ex\gll hi-pén\\
            2.{\textsc{act}}-cook\\
            \glt `you cook'
    \end{xlist}
\z

\ea\label{ex:syllc:mj}
Manjui \citep{JC18}
    \begin{xlist}
        \ex \gll ti-khán\\
             3.\textsc{t.rls}-dig\\
             \glt `s/he digs'
        \ex\gll hi-lʲáhwa-aj\\
            3.{\textsc{poss}}-pet-\PL\\
            \glt `his/her pets'
        \ex\gll hi-ˀwɛ́n\\
            2.{\textsc{act}}-see\\
            \glt `you see her/him/it'
        \ex\gll ha-na\\
            {\textsc{f}}-{\textsc{dem}}:outside\_hands'\_reach\\
            \glt `this.\textsc{f} (outside one’s hands’ reach)'
    \end{xlist}
\z

The non-moraic allomorphs (identical to those found before vowels) also occur before underlying \sound{PCh}{*h} < \sound{PM}{*x}, and \sound{PCh}{*h} is then elided, as in \word{PCh}{*hl\mbox{-}étek}{her/his head} (from \wordnl{*\mbox{-}hétek}{head}); \wordnl{*hl\mbox{-}óʔ}{you go} (from \wordnl{*\mbox{-}hóʔ}{to go}). This is quite likely an innovation.

There is a potential correspondence between the reportative enclitic \wordng{PCh}{*$=$hᵊn} (>~\wordng{Ijw}{$=$heˀn}, \wordng{I'w/Mj}{$=$hen}) and \word{Ni}{$=$ɬån}{id.}. The comparison is doubtful since the vowel correspondences are not regular, but it is conceivable that the Proto-Chorote form derives from an earlier (pre-Proto-Chorote) \intxt{*$=$ɬ̩n} > \intxt{*$=$hᵊn}. Interestingly, the initial consonant of \wordng{Ijw/I'w/Mj}{$=$he(ˀ)n} never labializes to \intxt{hw} after a rounded vowel, feeding translaryngeal vowel assimilation instead (as in \word{Mj}{ʔi\mbox{-}jó\mbox{-}hon}{s/he/it became (hearsay)}), in stark contrast with the homonymous suffix \word{Ijw/I'w}{\mbox{-}he(ˀ)n}{downwards; verbal plural}.

\subsubsection{Consonant + guttural fricative}\label{ch-consonant-dorsal}

Proto-Mataguayan clusters of the shape \intxt{*MX} (where \intxt{M} stands for a sonorant and \intxt{X} for any of \intxt{x}, \intxt{*χ}, or \intxt{*h}) yield \sound{PCh}{*hM}. When the first consonant in the cluster is a plosive (\intxt{*PX}), the outcome is \sound{PCh}{*P} except after a stressed vowel, in which case the reflex is \sound{PCh}{*hP}, and word-initially, where the reflex \sound{PCh}{*PVh} is found. Note that in all known cases the clusters of the shape \sound{PCh}{*hP} (where \intxt{P} stands for a stop) go back to \sound{PM}{*Ph} (as opposed to \sound{PM}{*Px} or \intxt{*Pχ}), which could be a coincidence or not. The clusters of the shape \sound{PM}{*tsX} yield \sound{PCh}{*s} (synchronically, [s] and [xs]/[hs] do not contrast in any Chorote variety, but rather occur as possible realizations of /s/ of any origin after vowels). The clusters of the shape \sound{PM}{*Fx} and \sound{PM}{*Fχ}, where \intxt{*F} is a fricative, lose the dorsal fricative in Proto-Chorote and evolve just like \sound{PM}{*F}. Recall that clusters of the shape \sound{PM}{*Fh} are banned in Proto-Mataguayan (\sectref{fricative-h}).

The examples below show the development \sound{PM}{*MX} > \sound{PCh}{*hM}. \REF{ch-jx-teach} is an exception, where the metathesis is prevented by irregular vowel insertion.

\begin{exe}
    \ex \ankle
    \ex \armadillo
    \ex \youngersis
    \ex \ropepl
    \ex \noseobl
    \ex \sleepiness
    \ex \smellv
    \ex \snore
    \ex \mucus
    \ex \pathpl
    \ex \fishwithhook
    \ex \bilecwpl
    \ex \spousewh
    \ex \marry
    \ex \tuscaf
    \ex \tuscat
    \ex \tuscag
    \ex \wildbean
    \ex \teach \label{ch-jx-teach}
\end{exe}

The following examples show that \sound{PM}{*Ph} normally yielded \sound{PCh}{*hP} after a stressed vowel. We are not aware of any clear examples of \sound{PM}{*Px} or \intxt{*Pχ} in that environment, so we technically do not know what the Chorote reflexes of \sound{PM}{*Px}, \intxt{*Pχ} would be after a stressed vowel.

\begin{exe}
    \ex \wildcat
    \ex \platepl
    \ex \paloflojot
    \ex \headpl
    \ex \heartcw
\end{exe}

Word-initially, \sound{PCh}{*hC} and \intxt{*Ch} are not permitted, and a vowel is then inserted to break up the illicit cluster.

\begin{exe}
    \ex \cactus
    \ex \up
\end{exe}

The examples below show the development of \sound{PM}{*PX} after an unstressed vowel.\footnote{Whenever a stop is followed by an applicative/adpositional suffix starting with \sound{PM}{*x}, or by \word{PCh}{*\mbox{-}heˀn(eʔ)}{downwards; verbal plural} < \wordng{PM}{*-xäˀn(eʔ)}, Iyo'awujwa' and Manjui show the reflex \intxt{hP} rather than the expected reflex \intxt{*P}, as in \word{Mj}{tɛ́wahkʲ\mbox{-}ap}{by the river}, from \wordnl{tɛ́wak}{river} and \wordnl{\mbox{-}hap}{by, surrounding} < \wordng{PM}{*xop}. It is possible to account for this by positing an analogical leveling based on the default development of \sound{PM}{*x} > \sound{PCh}{*h}. Examples \REF{ch-px-p-beard} and \REF{ch-tx-t-uncle} instantiate the regular development. Furthermore, applicatives/adpositions and \wordng{PCh}{\mbox{-}heˀn(eʔ)} might correspond to a phonological domain beyond the scope of the rule \sound{PM}{*Px} > \sound{PCh}{*P}.}

\begin{exe}
    \ex \thorncutjan
    \ex \beard \label{ch-px-p-beard}
    \ex \uncle \label{ch-tx-t-uncle}
    \ex \knee
\end{exe}

Clusters of the shape \sound{PM}{*Fx} and \intxt{*Fχ} always lose the guttural fricative (no clusters of the shape ``fricative + \intxt{*h}'' existed in Proto-Mataguayan; see \sectref{fricative-h}). Likewise, the cluster \sound{PM}{*tsh} yields \sound{PCh}{*s}; note that /s/ is often pronounced as \phonetic{xs} or \phonetic{hs} in the contemporary varieties of Chorote (see \sectref{pch-s}), but there is no contrast between [s] and [xs], thus the latter is not a true consonant cluster.

\begin{exe}
    \ex \centipedepl
    \ex \killbird
    \ex \finger
    \ex \redquebracho
    \ex \oldpl
    \ex \healthy
    \ex \eggits
    \ex \headits
    \ex \caracarapl
    \ex \teach
    \ex \meat
\end{exe}

In a few cases, diagnostic cognates are lacking, and we have been unable to determine which guttural fricative is to be reconstructed for PM.

\begin{exe}
    \ex \heavyv
    \ex \roast
    \ex \nightmonkey
    \ex \girl
    \ex \orphancwpl
    \ex \throwcw
    \ex \rhea
    \ex \doradocw
\end{exe}

The word-final clusters \sound{PM}{*jʰ} and \intxt{*lʰ} (underlying */jh/ and */lh/) are preserved in Chorote.

\begin{exe}
    \ex \plaj
    \ex \distal
    \ex \soul
\end{exe}

\subsubsection{Other consonant clusters}\label{ch-clusters}

Word-initially, multiple consonant clusters -- such as \sound{PM}{*ɸk}, \intxt{*ɸts}, \intxt{*tk}, \intxt{*wk}, \intxt{*kt}, \intxt{*kɸ}, \intxt{*sl}, \intxt{*tl} -- undergo vowel insertion in Chorote. Most of them are broken by a \intxt{*ᵊ} (compare this to the evolution of PM syllabic consonants, described in \sectref{ch-syll-c}), but after \intxt{*k} (possibly articulated as [k̟]; \sectref{ch-k-q}) an \intxt{*i} is inserted instead. Unexpectedly, an inserted \intxt{*i} -- rather than \intxt{**ᵊ} -- is also seen in \REF{ch-fts-palm}. \REF{ch-fts-suncho} is also an exception; in this example, the word-initial consonant is altogether lost. The status of \sound{PCh}{*ᵊ} is discussed in \sectref{pm-ch-ep-v}.

\begin{exe}
    \ex \north
    \ex \suncho \label{ch-fts-suncho}
    \ex \palm \label{ch-fts-palm}
    \ex \skunk
    \ex \tortoise
    \ex \whitealgarrobof
    \ex \wildcat
    \ex \precipice
    \ex \blind
    \ex \metal
\end{exe}

In the same position, the Proto-Mataguayan onset \intxt{*st} receives a prothetic \intxt{*ᵊ} in Proto-Chorote.

\begin{exe}
    \ex \whitequebracho
    \ex \kingvulture
    \ex \cardon
    \ex \chachalaca
\end{exe}

\sound{PM}{*l} is lost before another consonant in Chorote if the cluster occurs word-initially.

\begin{exe}
    \ex \flu
    \ex \squash
\end{exe}

The cluster \sound{PM}{*kɸ} changed to \sound{PCh}{*kw}, which yields \sound{Ijw}{kʲ} and \sound{I'w/Mj}{k} (see \sectref{ch-kw}). Similarly, the cluster \intxt{*kɸ’} changed to \sound{PCh}{*k’w} or possibly \intxt{*k’}.

\begin{exe}
    \ex \bite
    \ex \tornkf
    \ex \frighten
\end{exe}

The Proto-Mataguayan sequences \intxt{*nj} and \intxt{*ˀnj} lose the palatal approximant in Chorote.

\begin{exe}
    \ex \smelln
    \ex \cavy
\end{exe}

Word-medially, vowel insertion is possibly found in the cluster \sound{PM}{*tsn} > \sound{PCh}{*sVn}.

\begin{exe}
    \ex \toad
\end{exe}

\sound{PM}{*ɸ}, \intxt{*n}, \intxt{*q}, \intxt{*w}, and \intxt{*ˀw} are lost before another consonant in the word-medial position. In the cluster \intxt{*qk}, the loss of \intxt{*q} induces a compensatory doubling of the preceding vowel.

\begin{exe}
    \ex \elbow
    \ex \welln
    \ex \cavy
    \ex \fingerjaqsi
    \ex \rib
    \ex \belly
\end{exe}

Clusters with a PM~guttural fricative followed by another consonant lose the guttural stem-initially – as in \REF{ch-xn-spring}, \REF{ch-xp-shadow}, \REF{ch-xw-moon}, \REF{ch-xm-fox} – except in \REF{ch-xp-straw}, where \sound{PM}{*Xp} yields \sound{PCh}{*ʔip}. Word-medially (at least before a stop), the guttural consonant yields \sound{PCh}{*h}, and a vowel (a copy of the preceding vowel) is inserted to break the cluster apart, as in \REF{ch-nxt-cavy}, \REF{ch-xt-femalebreast}.

\begin{exe}
    \ex \cavy \label{ch-nxt-cavy}
    \ex \spring \label{ch-xn-spring}
    \ex \straw \label{ch-xp-straw}
    \ex \fox \label{ch-xm-fox}
    \ex \shadow \label{ch-xp-shadow}
    \ex \moon \label{ch-xw-moon}
    \ex \femalebreast \label{ch-xt-femalebreast}
\end{exe}

The clusters \sound{PM}{*sˀw} and \intxt{*stw}, which are only found before \sound{PM}{*u}, yield \sound{PCh}{*sᵊʔ} and \intxt{*ʔᵊstV}, respectively.

\begin{exe}
    \ex \kingvulture
    \ex \anteater
    \ex \likelove
\end{exe}

The clusters \sound{PM}{*lʔ} – as in \REF{ch-l-doveula} – and \intxt{*mʔ} – as in \REF{ch-m-rat} – are apparently retained in the environment \intxt{*\_…h\#}. Otherwise the glottal stop is lost, as in \REF{ch-l-soninlaw}–\REF{ch-m-daylhuma}.

\begin{exe}
    \ex \rat \label{ch-m-rat}
    \ex \doveula \label{ch-l-doveula}
    \ex \soninlaw \label{ch-l-soninlaw}
    \ex \daylhuma \label{ch-m-daylhuma}
\end{exe}

Finally, a few clusters are retained in the medial position without any special change. These include \intxt{*lɸ}, \intxt{*lts}, \intxt{*sk’}.

\begin{exe}
    \ex \spouse
    \ex \majan
    \ex \widower
\end{exe}

\subsection{Vowels}\label{ch-vow}

Chorote shows more or less the same reflexes of PM vowels as Wichí: most vowels are preserved intact except for \sound{PM}{*ä}, which merges with \intxt{*e} or, if an accented syllable follows (\sectref{pm-ch-ae}), with \intxt{*i}. Three minor innovations shared with Wichí are the lowering of \intxt{*e} to \intxt{*a} before a \intxt{*χ} in the coda position (\sectref{pm-ch-ejj-ah}; also shared with Maká), the lowering of \intxt{*i} to \intxt{*e} in the environment \intxt{*At/x…ts} (\sectref{pm-ch-atits-ates}) and to \intxt{*a} in the environment \intxt{*\#ʔ…C'Á} (\sectref{pm-ch-ji-a}), and the rounding of \intxt{*e} before the clusters \intxt{*kw} (\sectref{pm-ch-ekw-okw}).

\subsubsection{Reflexes of \sound{PM}{*ä}} \label{pm-ch-ae}
The default reflex of \sound{PM}{*ä} in Chorote is \sound{PCh}{*e}. An irregular reflex is seen in \REF{ch-ae-rootn}. The reflex \sound{PCh}{*i} in \REF{ch-ae-soninlaw}, as opposed to \intxt{*e} in \REF{ch-ae-sisinlaw}, is due to harmonic rising triggered by the following \intxt{*u}, a process that might be regular in the environment \mbox{\intxt{*W\_Lu}}, where \intxt{W} stands for a labial and \intxt{L} for a coronal. Compare \word{PCh}{*\mbox{-}pél}{shadow}, but \word{Mj}{\mbox{-}péilik}{shadow} < \intxt{*\mbox{-}píl\mbox{-}uk}; \word{PM}{*ɸ’elxVtsé\mbox{-}ts}{poor}, but \word{PCh}{*p’ihlusé\mbox{-}s}{poor}.

\begin{exe}
    \ex \burn
    \ex \wing
    \ex \yicaay
    \ex \goawayyou
    \ex \putv
    \ex \flyv
    \ex \tell
    \ex \sisinlaw \label{ch-ae-sisinlaw}
    \ex \soninlaw \label{ch-ae-soninlaw}
    \ex \fieldn
    \ex \rootn \label{ch-ae-rootn}
    \ex \coldweather
    \ex \dreamn
    \ex \crab
    \ex \killbird
    \ex \hole
    \ex \spouse
    \ex \stretchout
    \ex \dividev
    \ex \chaniarf
    \ex \chaniart
    \ex \flu
    \ex \nightmonkey
    \ex \hither
    \ex \smellv
    \ex \tapeti
    \ex \acquainted
    \ex \abdcavity
    \ex \basetrunk
    \ex \trunk
    \ex \bilecw
    \ex \allrcpr
    \ex \burrow
    \ex \walk
    \ex \seev
    \ex \placen
    \ex \egg
    \ex \vrbpl
    \ex \headn
\end{exe}

The regular reflex in Chorote seems to be \intxt{*i} rather than \intxt{*e} if an accented syllable follows. \REF{ch-ae-i-deep} further suggests that it is the position of the accent in PM (as opposed to PCh) that matters.

\begin{exe}
    \ex \deep \label{ch-ae-i-deep}
    \ex \cat
    \ex \duraznillo
    \ex \meat
\end{exe}

\subsubsection{Lowering of \intxt{*e} before \intxt{*χ}}\label{pm-ch-ejj-ah}

Before the uvular fricative \intxt{*χ}, \sound{PM}{*e} has a special lowered reflex, \sound{PCh}{*a}. This is shared with Maká (\sectref{mk-uvul-retr}) and Wichí (\sectref{pm-wi-ejj-ajj}).

\begin{exe}
    \ex \fatv
    \ex \jabiru
    \ex \quick
    \ex \smoke
    \ex \fullriver
    \ex \peccary
    \ex \hurt
    \ex \chaguara
    \ex \wildbean
    \ex \mistolf
    \ex \puma
\end{exe}

The lowering induced by the uvular fricative left behind a synchronically active alternation in Chorote. In forms that go back to PM~etyma with a \intxt{*χ}, the lowering applies, and one finds \sound{PCh}{*a}. By contrast, the reflexes of PM~forms derived from the vocalic stems of the same etyma (see \sectref{jj-suff}) show no lowering, because \sound{PM}{*χ} was absent in the respective protoforms. Consequently, one finds \sound{PCh}{*e}, raised to \intxt{i} in the unstressed position in the contemporary varieties.

\booltrue{listing}
\ea
    Iyojwa’aja’ \citep[96, 143, 144]{ND09}
    \begin{xlist}
        \ex \intxt{pánsa} /pánsah/\gloss{fast, quick.\SG} → \intxt{pánsi-s} /pánsi-s/\gloss{fast, quick.\PL}
        \ex \intxt{p’élisʲe} /p’ílusah/\gloss{poor.\SG} → \intxt{p’ihlʲúxsi-s} /p’ilúsi-s/\gloss{poor.\PL}
        \ex \intxt{ʔáʔtʲeh-eʔ} /ʔǻʔtah-hi(j)/\gloss{it hurts} → \intxt{ʔáʔti-s-iʔ} /ʔǻʔti-s-hi(j)/\gloss{they hurt}
    \end{xlist}
\z
\ea
        Iyo’awujwa’ \citep[120, 166]{AG83}
    \begin{xlist}
        \ex \intxt{álisa} /ʔálᵊsah/\gloss{cháguar.\SG} → \intxt{álisi-s} /ʔálᵊsi-s/\gloss{cháguar.\PL}
        \ex \intxt{tóxsa} /túsah/\gloss{smoke.\SG} → \intxt{tóxsi-s} /túsi-s/\gloss{smoke.\PL}
    \end{xlist}
\z
\ea
        Manjui \citep{JC18}
    \begin{xlist}
        \ex \intxt{p’ilisáh} /p’ilVsáh/\gloss{poor.\SG} → \intxt{p’ilisɛ́-s} /p’ilVsé-s/\gloss{poor.\PL}
    \end{xlist}
\z
\boolfalse{listing}

\subsubsection{Lowering of \intxt{*i} in the environment \intxt{*At/x…ts}}\label{pm-ch-atits-ates}

In Chorote, \sound{PM}{*i} lowers to \intxt{*e} before \intxt{*ts}, provided that there is a low vowel (\intxt{*a} or \intxt{*å}) in the preceding syllable. This most regularly happens when the syllable has \intxt{*t} as the onset, but one example with \sound{PM}{*x} > \sound{PCh}{*h} has also been identified. As a consequence, the nominal plural suffix \intxt{*\mbox{-}is} shows the allomorph \intxt{*\mbox{-}es} in Proto-Chorote, an alternation best described as an instance of progressive height harmony. This innovation is shared with Wichí (\sectref{pm-wi-atits-atets}); in addition, a similar process operates dialectally in Nivaĉle (\sectref{ni-chishamnee-lowering}).

\begin{exe}
    \ex \drinknpl
    \ex \waterpl
    \ex \starn
    \ex \earthpl
    \ex \skinpl
\end{exe}

\subsubsection{Lowering of \intxt{*i} before glottalized consonants followed by a low vowel}\label{pm-ch-ji-a}

We have already seen that the sequence \sound{PM}{*ji} changed to \intxt{*ʔi} word-initially in Proto-Chorote (\sectref{ch-ji}). However, when followed by a glottalized consonant and a low vowel (\sound{PM}{*a} or \intxt{*å}, but not \intxt{*ä}), the vowel \intxt{*i} was lowered, yielding \intxt{*ʔa}. The development \sound{PM}{*ji} > \intxt{*ʔi} > \intxt{*ʔa} in this environment is shared with Wichí (\sectref{pm-wi-ji-ha}).

\begin{exe}
    \ex \jaguar
    \ex \treen
    \ex \vulture
\end{exe}

\subsubsection{Rounding of vowels next to \intxt{*k(’)w}}\label{pm-ch-ekw-okw}

In two examples, accented \sound{PM}{*é} and \intxt{*ä́} appear to have acquired rounding in Chorote next to \intxt{*kw} (from \sound{PM}{*kɸ}) or \intxt{*k’(w)} (from \sound{PM}{*kɸ’}).

\begin{exe}
    \ex \bite
    \ex \tornkf
\end{exe}

Unaccented instances of \intxt{*e} remained unaffected in Proto-Chorote. However, in the only known example vowel rounding is seen in the Iyojwa’aja’ variety, as shown in \REF{ch-e-o-yokwes} below.

\begin{exe}
    \ex \frighten
\end{exe}

\subsubsection{The emergence of \sound{Proto-Chorote}{*ᵊ}}\label{pm-ch-ep-v}

The insertion of an intrusive \intxt{*ᵊ} in certain consonant clusters (\sectref{ch-clusters}) and the decomposition of syllabic consonants into sequences of the type \intxt{*Cᵊ} (\sectref{ch-syll-c}) is shared by all Chorote varieties and must have been complete by the Proto-Chorote stage. All Chorote varieties have since merged \intxt{*ᵊ} with other vowels, especially \intxt{*i} (\sectref{ch-schwa-dial}), but this latter merger took place independently in the varieties of Chorote: \sound{PCh}{*ᵊ} differs from \sound{PCh}{*i} in not constituting the environment for the first palatalization (\sectref{ch-pal1}). However, the reflexes of both sounds did feed the second palatalization, which occurred in Iyojwa’aja’ and, with some restrictions, in Manjui (\sectref{ch-pal2}).

In \cits{NH06} typology of inserted vowels, \sound{PCh}{*ᵊ} is probably better characterized as an intrusive vowel rather than as a full-fledged epenthetic segment: its quality does not match any other vowel phoneme already present in the inventory, and it typically occurs in heterorganic clusters. Its only property untypical of intrusive vowels is that its main function is that of repairing illicit structures. It is therefore quite possible that \sound{PCh}{*ᵊ} was absent from the phonological representations of Proto-Chorote forms, as in */wkínah/ (likely pronunciation: *\phonetic{\mbox{wəˈk̟inah}}) `metal'. However, in the contemporary Chorote lects its reflexes are clearly segmental, which is in any case a common fate of erstwhile intrusive vowels in many languages \citep[422–424]{NH06}. 

It is difficult to reconstruct the exact phonetic realization of the intrusive vowel symbolized as \intxt{*ᵊ} here; possible values include [ɨ], [ə], and [ɪ]. It was certainly distinct from \sound{PCh}{*e} (which also sometimes yields [i] in the modern varieties), since the sound change \sound{PCh}{*e} > modern Chorote [i] fed the first palatalization, as in \word{PCh}{*ʔa\mbox{-}selǻn\mbox{-}eh}{I prepare, I make} > \word{I'w}{a\mbox{-}silʲén\mbox{-}e}{id.}.

\subsubsection{Other vowel changes} \label{pm-ch-vowel-resid}

There are some cases of \sound{PM}{*a} > \sound{PCh}{*o} in the environment \intxt{*\_k(’)ó}. 

\begin{exe}
    \ex \heel
    \ex \face
    \ex \eyebrow
\end{exe}

Before the plural non-human suffix \intxt{*\mbox{-}wáʔ}, found in demonstratives, the vowels \intxt{*a}, \intxt{*å}, and \intxt{*e} change to \intxt{*o}, as in the forms \wordnl{*ko\mbox{-}wáʔ}{those (outside the speaker’s sight)}, \wordnl{*no\mbox{-}wáʔ}{these (outside one’s hands’ reach)}, \wordnl{*ˀno\mbox{-}wáʔ}{these (within one’s hands’ reach)}, \wordnl{*po\mbox{-}wáʔ}{those (outside the speaker’s sight and never seen before)}, \wordnl{*so\mbox{-}wáʔ}{those (within the speaker’s sight)} (compare the masculine singular forms \intxt{*kǻʔ}, \intxt{*náʔ}, \intxt{*ˀnáʔ}, \intxt{*páʔ \recind *pǻʔ}, \intxt{*séʔ}). In the form \wordnl{*ha\mbox{-}wáʔ \recind *hå\mbox{-}wáʔ}{those (outside the speaker’s sight but seen before)}, the rounding of the vowel is perhaps prevented by the preceding glottal fricative (in the Manjui variety this form has subsequently changed to \intxt{ho\mbox{-}wa}, thus eliminating the irregularity).

\subsection{Word-level prosody}\label{ch-prosody}

Chorote has contrastive stress. In our proposal, Iyo’awujwa’ and Manjui are conservative with regard to the position of the stress, whereas Iyojwa’aja’ underwent stress retraction in some cases, as will be shown in \sectref{ch-ijw-prosody}. Synchronically, the stress of any given Chorote word form can be determined based on the accentual properties of individual morphemes as follows. The leftmost underlying accent is the one that appears in the surface realization, whereas all subsequent accents are deleted. If no morpheme in a given Chorote word contains an underlying accent, a default accent is inserted in the peninitial syllable (or in the only syllable in the case of monosyllabic words). In the Manjui examples in \REF{mj-stress-survive}, the underlying accents are indicated by an acute, and the surface accent is shown by means of the IPA~symbol [ˈ] in the phonetic transcriptions.\footnote{A note is due on the realization of the prefixes in the examples below. The prefixes /i\mbox{-}/, /hl\mbox{-}/, /s\mbox{-}/, /Vn\mbox{-}/ take moraic allomorphs (\intxt{ʔi\mbox{-}}, \intxt{hi\mbox{-}}, \intxt{ʃi\mbox{-}}, \intxt{ʔin\mbox{-}}) before supraglottal consonants; non-moraic allomorphs (\intxt{ˀj\mbox{-}}, \intxt{t\mbox{-}’…}, \intxt{ts\mbox{-}’…}, \intxt{ˀn\mbox{-}}) before /ʔ/; and maintain the underlying moraicity distinction before /h/ (as \intxt{ʔi\mbox{-}}, \intxt{hl\mbox{-}}, \intxt{s\mbox{-}}, \intxt{ʔin\mbox{-}}). Before vowels, /i\mbox{-}/, /hl\mbox{-}/, and /s\mbox{-}/ take non-moraic allomorphs (\intxt{j\mbox{-}}, \intxt{hl\mbox{-}}, \intxt{s\mbox{-}}), and /Vn\mbox{-}/ remains moraic (\intxt{ʔin\mbox{-}}).} The lowering of the pretonic vowel in \REF{mj-aseis} is not a productive process (see also \sectref{ch-pret-v-l-mj}).

\ea\label{mj-stress-survive}
Manjui \citep{JC18,GH94}\\
    \begin{xlist}
        \ex\gll /hl-úp-ís/~\phonetic{ˈhlʊpis}\\
                3.\textsc{poss}-nest-\PL\\
                \glt `its nests'
        \ex\gll /tós-ís/~\phonetic{ˈtɔxʃis}\\
                snake-\PL\\
                \glt `snakes'
        \ex\gll /túsah/~\phonetic{ˈtʊxsa}\\
                smoke\\
                \glt `smoke'
        \ex\gll /ʔis-ís/~\phonetic{ʔaxˈseis}\\
                good-\PL\\
                \glt `they are good'\label{mj-aseis}
        \ex\gll /hup-ájh/~\phonetic{huˈpajh}\\
                maize-\PL\\
                \glt `grass'
        \ex\gll /i-kʲoj/~\phonetic{ʔixˈʃoj}\\
                1\SG.\textsc{poss}-hand\\
                \glt `my hand'
        \ex\gll /kihwijh/~\phonetic{kiˈhwijh}\\
                below\\
                \glt `inside, below, beneath'
        \ex\gll /kʲoweh/~\phonetic{kʲoˈwɛh}\\
                hole\\
                \glt `burrow'
        \ex\gll /hl-túsah/~\phonetic{hiˈtʲuxsa}\\
                3.\textsc{poss}-smoke\\
                \glt `its smoke'
        \ex\gll /Vn-láhwah-ájh/~\phonetic{ʔinˈlahwaaj}\\
                {\textsc{gnr}}-pet-\PL\\
                \glt `one's pets'
        \ex\gll /Vn-kʲoj-ájh/~\phonetic{ʔinkiˈjejh}\\
                {\textsc{gnr}}-hand-\PL\\
                \glt `one's hands'
        \ex\gll /Vn-ˀlih-ájh/~\phonetic{ʔinʔlaˈhajh}\\
                {\textsc{gnr}}-language-\PL\\
                \glt `one's words'
        \ex\gll /s-kihwijh/~\phonetic{ʃiˈkeihwi}\\
                1\PL-below\\
                \glt `inside us, below us, beneath us'
        \ex\gll /hl-kʲoweh/~\phonetic{hiˈkʲowe}\\
                3.\textsc{poss}-hole\\
                \glt `her/his abdomen'
    \end{xlist}
\z

We propose that the Chorote stress straightforwardly continues the accent of Proto-Mataguayan with minor changes, and that the underlying accentual properties of specific morphemes were also inherited from PM. The accented vowels of Proto-Mataguayan are normally reflected as stressed in Chorote, and the unaccented ones as unstressed. As discussed in \chapref{prosody}, already in Proto-Mataguayan only the leftmost underlying accent in any given word made it to the surface, whereas all subsequent underlying accents were eliminated; this rule is still active in (Proto-)Chorote. In addition, as shown in \sectref{corta-larga-corta}, Proto-Mataguayan had a rule whereby a default peninitial accent is inserted in words without an underlying accent within the trisyllabic window at the left edge: ˘˘˘(…)~→~˘¯˘(…). This rule has  extended its operation to shorter words in  (Proto-)Chorote: unlike Proto-Mataguayan, where some monosyllabic or disyllabic words (including content words) may lack an accent altogether, Chorote requires that at least one syllable in a word be stressed, with the possible exception of some grammatical elements.

The Chorote reflexes of unaccented monosyllabic words of Proto-Mataguayan receive stress on their only syllable, as shown below.

\begin{exe}
    \ex \goaway
    \ex \cry
    \ex \fooditssg
    \ex \thorneitssg
    \ex \sprout
    \ex \spillcwimp
    \ex \grass
    \ex \cordits
    \ex \diecw
    \ex \stepv
    \ex \skinits
    \ex \good
\end{exe}

The Chorote reflexes of unaccented disyllabic words of Proto-Mataguayan receive stress on their final syllable, as shown below.

\begin{exe}
    \ex \jaguar
    \ex \treensg
    \ex \mancwsg
    \ex \vulturesg
    \ex \holeabs
    \ex \two
    \ex \starnsg
    \ex \bromelia
    \ex \nightncw
    \ex \earthcwsg
    \ex \fatalhaitssg
    \ex \aloja
    \ex \doradocwsg
    \ex \meatitssg
\end{exe}

The same combination obtains when an unaccented moraic prefix is added to an unaccented monosyllabic root. The following roots typically show up with a moraic prefix:

\begin{exe}
    \ex \tailsg
    \ex \torn
    \ex \handsg
    \ex \hornclubsg
    \ex \languagewordsg
    \ex \toolnsg
    \ex \lousesg
    \ex \yicalhuksg
    \ex \sleep
    \ex \smelln
    \ex \fatpesg
    \ex \fence
    \ex \lid
    \ex \basetrunk
    \ex \spillcw
    \ex \placen
    \ex \necksg
    \ex \clothes
\end{exe}

Most verbs that took a zero third-person realis prefix in Proto-Mataguayan underwent a morphological change in Chorote: they now take the third-person realis prefix \intxt{*ʔi\mbox{-}}. The verbs that were affected by this change are underlyingly unaccented in Proto-Mataguayan; in Chorote, they receive a default stress on the peninitial syllable.

\begin{exe}
    \ex \suckb
    \ex \swallow
    \ex \invite
    \ex \dig
    \ex \eatvt
    \ex \shoot
    \ex \carrysh
    \ex \walk
\end{exe}

Chorote retains the mobile paradigms of Proto-Mataguayan to some extent. For example, underlying unaccented monosyllables retain their behavior in Chorote: when they are followed by an underlyingly accented plural suffix, the stress moves to the suffix.

\booltrue{listing}
\ea
    Iyojwa’aja’ \citep[92]{JC14b}
    \begin{xlist}
        \ex \intxt{ʔés}\gloss{it is good} → \intxt{ʔiʃ-ís}\gloss{they are good}
        \ex \intxt{t-’ák}\gloss{its rope, cord} → \intxt{t-’ak-áʔ \recind t-’ak-áʔl}\gloss{its ropes, cords}
        \ex \intxt{t-’áx}\gloss{its skin} → \intxt{t-’ɛh-ɛ́s}\gloss{its skins}
    \end{xlist}
\z
\newpage
\ea
    Iyo’awujwa’ \citep[176]{AG83}
    \begin{xlist}
        \ex \intxt{hóp}\gloss{maize} (etymologically\gloss{grass.\SG}) → \intxt{hup-áj}\gloss{grass} (etymologically\gloss{grass.\PL})
    \end{xlist}
\z
\ea
    Manjui \citep{JC18}
    \begin{xlist}
        \ex \intxt{hʊ́p}\gloss{maize.\SG} → \intxt{hup-ájh}\gloss{maize.\PL, grass}
        \ex \intxt{ʔéis}\gloss{it is good} → \intxt{ʔas-éis}\gloss{they are good}
    \end{xlist}
\z

This differs from the behavior of underlyingly accented monosyllables, which keep their stress even when followed by an underlyingly accented plural suffix.

\ea
    Iyojwa’aja’ \citep[131, 132]{ND09}
    \begin{xlist}
        \ex \intxt{hl-ɛ́ʔ}\gloss{her/his/its name} → \intxt{hl-ɛ́j-is}\gloss{her/his/its names}
        \ex \intxt{hl-óp}\gloss{its nest} → \intxt{hl-óp-is}\gloss{its nests}
    \end{xlist}
\z
\ea
    Iyo’awujwa’ \citep[125, 176, 176, 183]{AG83}
    \begin{xlist}
        \ex \intxt{-éj}\gloss{yica bag} → \intxt{-éj-is}\gloss{yica bags}
        \ex \intxt{hl-úp}\gloss{its nest} → \intxt{hl-úp-is}\gloss{its nests}
        \ex \intxt{hók}\gloss{palo santo tree} → \intxt{hók-iʔ}\gloss{palo santo trees}
        \ex \intxt{tóxs}\gloss{snake} → \intxt{tóxs-is}\gloss{snakes}
    \end{xlist}
\z
\ea
    Manjui \citep{JC18}
    \begin{xlist}
        \ex \intxt{-át}\gloss{drink.\SG} → \intxt{-át-es}\gloss{drink.\PL}
        \ex \intxt{-ɛ́jʔ}\gloss{name} → \intxt{-ɛ́j-is}\gloss{names}
        \ex \intxt{-ɛ́jʔ}\gloss{yica bag} → \intxt{-ɛ́j-is}\gloss{yica bags}
        \ex \intxt{ˀmɔ́k}\gloss{zorzal bird} → \intxt{ˀmɔ́k-is}\gloss{zorzal birds}
        \ex \intxt{hɔ́k}\gloss{palo santo tree} → \intxt{hɔ́k-ej}\gloss{palo santo trees}
        \ex \intxt{hɔ́t}\gloss{sand.\SG (small quantity of sand)} → \intxt{hɔ́t-ej}\gloss{sand.\PL (large patch of sand)}
        \ex \intxt{hl-ʊ́p}\gloss{its nest} → \intxt{hl-ʊ́p-is}\gloss{its nests}
        \ex \intxt{tɔ́s}\gloss{snake} → \intxt{tɔ́xʃ-is}\gloss{snakes}
    \end{xlist}
\z

Chorote also retains the behavior of underlyingly unaccented disyllabic nouns and adpositions. When they occur without a prefix, they receive a default peninitial stress on their \emph{second} syllable, as explained above. However, when a moraic prefix is added, the default peninitial stress falls on the \emph{first} syllable of the stem.

\newpage
\ea
    Iyojwa’aja’ \citep[92]{JC14b}
    \begin{xlist}
        \ex \intxt{k’ijé}\gloss{for} → \intxt{si-kʲ’óje}\gloss{for us}
        \ex \intxt{ʔapɛ́ʔɛ}\gloss{above} → \intxt{si-típeʔe}\gloss{above us}
        \ex \intxt{kʲahwéh}\gloss{below} → \intxt{si-kʲáhwe}\gloss{below us}
    \end{xlist}
\z

\ea
   Manjui \citep{JC18,GH94}
    \begin{xlist}
        \ex \intxt{ʔijéʔ}\gloss{for} → \intxt{hi-ʔʲójeʔ}\gloss{for her/him}
        \ex \intxt{ʔapɛ́ʔɛʔ}\gloss{above} → \intxt{hi-tɛ́peʔeʔ}\gloss{on top of it}
        \ex \intxt{kihwíjh}\gloss{below} → \intxt{ʃi-kéihwi}\gloss{below us}
    \end{xlist}
\z
\boolfalse{listing}

In verbs, however, the pattern in question no longer occurs. Instead, fixed stem-initial stress was apparently generalized in verbs in these cases, as in \word{PCh}{*qásit}{stand up!} (compare \word{'Wk}{qasít}{id.}).

\section{From Proto-Chorote to the contemporary varieties} \label{ch-dialects}

In terms of the nature of the linguistic differences, Chorote shows more dialectal diversity than any other Mataguayan language. The variety spoken by the Iyojwa’aja’ people of Argentina, also known as Riverine Chorote or variety \#1 ($=$~V1), is particularly divergent, whereas all other varieties are closer to each other and are collectively referred to as Forest Chorote or variety \#2 ($=$~V2). This latter group of dialects, in turn, is subdivided into what we call Iyo’awujwa’ (spoken in Argentina as well in the community of San Eugenio, located in the surroundings of Pedro P. Peña, Paraguay) and Manjui (spoken especially in Misión Santa Rosa $=$ Wonta and Abizai). Note that the Iyo’awujwa’ speakers from San Eugenio are locally known as Manjui.

Iyojwa’aja’, Iyo’awujwa’, and Manjui are all further subdivided into a number of subvarieties. Subdialectal variation within these varieties remains understudied, however. \citet{AG78} states that the Iyojwa’aja’ are divided into \wordnl{Isiam jlele’}{Downriver People} and \wordnl{Pijiam jlele’}{Upriver People}, a claim whose linguistic validity we have been unable to confirm (perhaps due to drastic demographic changes that affected the Iyojwa’aja’ people during the 20\textsuperscript{th} century), though there certainly are lexical differences between subvarieties of Iyojwa’aja’. The Iyo’awujwa’ were historically (before the Chaco War) subdivided into two groups, \wordnl{Jla’wáj jlele’}{Lake People}\footnote{\citet[8]{JC14a} mistakenly analyzes \cits{AS82} attestation of this ethnonym as \wordnl{Jlawá'a jlele’}{Outsiders}.} and \wordnl{Jwej jlele’}{Field People}; it is unclear whether this division is related to the linguistic variation attested within contemporary Iyo’awujwa’. The Manjui are subdivided into \wordnl{Jlimnájnas}{Forest People} and \wordnl{Jlawá'a Wos}{Outsiders}, which historically spoke slightly different subdialects, according to \citet[5--8]{JC14a} and \citet[5]{GH94}. Although nowadays descendants of both groups have settled in Santa Rosa (Wonta), and the subdialects in question have mixed to some extent in the speech of the speakers born in the 1970s or later \citep[8]{JC18}, some minor lexical and phonetic differences persist \citep{GH94,JC-amer}.

This section describes the phonological evolution of Iyojwa’aja’, Iyo’awujwa’, and Manjui.

\subsection{Palatalization}\label{ch-palat}

Palatalization is a pervasive phenomenon in Chorote. It affects consonants, but only in the onset position. Most consonants palatalize by acquiring a secondary articulation, i.e., \intxt{*C}~>~[Cʲ]: \intxt{*t}~>~[tʲ], \intxt{*m}~>~[mʲ], \intxt{*l}~>~[lʲ], etc., a phenomenon known as \conc{secondary palatalization} \citep{NB07}. For others, it involves a change in the place of articulation (\cits{NB07} \conc{full palatalization}). This is the case with \intxt{*kʲ(’)}; \intxt{*s} and \intxt{*ts'} (except in Iyojwa'aja', where palatalization is most commonly realized as [Cʲ]); and \intxt{*w}, \intxt{*ˀw}, \intxt{*hw}, with some nuances (labiovelars are subject to full or secondary palatalization, depending on the environment and dialect; \sectref{ch-pal1}). As for \intxt{*h}, it becomes \intxt{hj}, realized as [hj] or [xj].

In Manjui, secondary palatalization ([Cʲ]) is often imperceptible or hardly perceptible, depending on the speaker, target, and phonological environment, as in [ʔiˈʔn(ʲ)oʔ] `man', [ʔiˈhl(ʲ)oʔ] `armadillo', which explains its frequent absence in \cits{AG83} transcriptions of that dialect. However, acoustic analysis shows that in most cases the secondary articulation does exist, as shown by the characteristic lowering of the second formant after the consonant \citep[364]{PLIM96}, although the lowering is much shorter than in Iyojwa'aja' and Iyo'awujwa'. The effects of the palatal articulation could be reflected in the following closed vowel [o] (instead of the otherwise expected [ɔ]), even though a different explanation for the closed vowel cannot be ruled out (\sectref{ch-vowel-lowering}). In other cases, no acoustic traces of palatalization are found, as is the case for /n/ before [e] derived from /a/ (\sectref{ch-v-r}): /i\mbox{-}najin/~>~[ʔiˈnejin] `s/he goes first'. An extensive account of the phonetic details of palatalization in Manjui is beyond the scope of the present book; see \citet{JC-amer} for details.

As a diachronic sound change, palatalization occurred at least four times in the history of the Chorote varieties. We dub these sound changes \conc{first}, \conc{second}, \conc{third palatalization}, and \conc{regressive palatalization}, keeping in mind that they were not shared by the extant varieties of Chorote but rather applied independently, with slightly differing results. The first palatalization is triggered by \sound{PCh}{*i} or \intxt{*(ˀ)j} (and, at least in Iyo’awujwa’ and Manjui, also by \intxt{*e}~>~\phonetic{i} in pretonic position). The second palatalization, which affects only coronal consonants (except /s/ and /ts'/ in Manjui) and does not apply in Iyo’awujwa’, is triggered by \phonetic{i}'s of different origins (including from \sound{PCh}{*ᵊ}), but also by \sound{PCh}{*u}, \sound{PCh}{*hw}, and, sporadically in Manjui, by \sound{PCh}{*e}. The third palatalization, triggered by \sound{PCh}{*i}, applies only in Iyo'awujwa' and Manjui and affects \sound{PCh}{*q(’)}, which had been immune to the first palatalization. The regressive palatalization is a marginal phenomenon whereby /s~ts’/ are palatalized to [ʃ~tʃ’] before an [i]; it is most common in Manjui. In what follows, we discuss in detail the first (\sectref{ch-pal1}), the second (\sectref{ch-pal2}), the third (\sectref{ch-pal3}), and the regressive (\sectref{ch-pal-regr}) palatalizations; the depalatalization process (\sectref{ch-depal}); as well as cases which we cannot explain at present (\sectref{ch-unexpl}).

\subsubsection{First palatalization}\label{ch-pal1}

The first (progressive) palatalization took place in all Chorote varieties. It affects all consonants in the onset position except \intxt{*(ˀ)j} and \intxt{*q(’)}. Arguably \intxt{*q} and \intxt{*q’} were still phonetically uvular in Proto-Chorote (though their reflexes are sometimes articulated as velar in the daughter languages), and palatalized uvulars are much more difficult to articulate than palatalized consonants with a more front place of articulation.\footnote{It is fairly common for a language to have a uvular series, a palatalized series, but no palatalized uvulars, as is the case in Xong (<~Hmongic < Hmong--Mien; \citnp{AS21}) and in Tsakhur (<~Lezgic < East Caucasian; \citnp{SK99}).} The triggers include \sound{PCh}{*i}, \intxt{*j}, and \intxt{*ˀj}, but also \intxt{*e}~>~\intxt{i} in pretonic position, suggesting that this latter change had taken place early enough. 
Despite the fact that the first palatalization affected all Chorote varieties, there is evidence suggesting that it took place (or remained active) after the split of Proto-Chorote. A case in point is the lack of the first palatalization in \REF{lesan} in Iyojwa'aja', where the stress retraction (\sectref{ch-ijw-prosody}) bled the change \intxt{*e}~>~\intxt{i}, necessary for the palatalization to occur; other dialects, where the stress retraction did not apply, do show both \intxt{*e}~>~\intxt{i} and the first palatalization.

\begin{exe}
    \ex \chox{ʔipǻk}{straw}{ʔipʲák}{ipʲék}{ʔipʲék}{---}
    \ex \chox{k’ihlóʔ}{armadillo}{k’ihlʲóʔ}{ihlʲóʔ}{\mbox{ʔihlʲóʔ}}{ʔihl(ʲ)óʔ}
    \ex \cho{ʔi-hlǻˀm}{s/he defecates}{ʔi-hlʲáˀm}{---}{ʔi-hl(ʲ)éˀm}
    \ex \cho{ʔi-mǻʔ}{s/he sleeps}{ʔi-mʲáʔ}{---}{ʔi-mʲéʔ \recind ʔi-máʔ\gloss{s/he camps}}
    \ex \chox{ʔihnáta-k}{tusca tree}{ʔihnʲéta-k}{ihnʲéta-k}{ʔihnʲéta-k}{ʔihn(ʲ)éta-k}
    \ex \chox{ʔiˀnóʔ}{man}{ʔiˀnʲóʔ}{inʲóʔ}{ʔiˀnʲóʔ}{ʔiˀn(ʲ)óʔ}
    \ex \cho{\mbox{-}selǻn\mbox{-}}{to prepare}{\mbox{-}lɛ́xsan\mbox{-}}{\mbox{-}silʲén\mbox{-}}{\mbox{-}ʃi(l)ʲén-}\label{lesan}
    \ex \cho{\mbox{-}ʔelǻk}{pus}{\mbox{-}ʔilʲák}{---}{---}
    \ex \cho{ʔi\mbox{-}nǻjin}{s/he goes first}{ʔi-nʲáˀn}{---}{ʔi-néjin}
\end{exe}

\sound{PCh}{*w}, \intxt{*ˀw}, and \intxt{*hw} palatalize to \intxt{j}, \intxt{ˀj}, and \intxt{hj}, respectively, before any vowel in Manjui, but only before rounded vowels in Iyojwa’aja’ \citep[64]{AG78} and Iyo’awujwa’ \citep[44]{AG83}. In these varieties they yield \intxt{w}, \intxt{ˀw}, and \intxt{hw} before [i], but \intxt{wʲ}, \intxt{ˀwʲ}, and \intxt{hwʲ} before [a] and [e].

\begin{exe}
    \ex \cho{ʔi-wún}{s/he burns}{ʔi-júˀn}{---}{ʔi-jún}
    \ex \cho{ʔi-ˀwén}{s/he sees}{ʔi-ˀwíˀn}{ʔi-ˀwín}{ʔi-ˀjín} 
    \ex \cho{ʔi-ˀwét}{my place}{ʔi-ˀwít}{ʔi-ˀwít}{ʔi-ˀjít}
    \ex \cho{ʔi-ˀwúɬ}{s/he climbs}{ʔi-ˀjúlh}{---}{ʔi-ˀjúɬ}
    \ex \cho{ʔi-hwéˀjåʔ}{s/he flies}{ʔi-hwíˀjaʔ}{---}{ʔi-hjíˀjeʔ}
    \ex \cho{ʔi-hwík}{s/he hides}{ʔi-hwík}{---}{ʔi-hjík}
    \ex \cho{ʔi-hwéhl-aˀm}{s/he tells}{ʔi-hwíhl-aˀm}{ʔi-hwíhl-aˀm}{ʔi-hjíhl-aˀm}
    \ex \chox{ʔi-hwáts’un-\APPL}{s/he spits}{ʔi-hwʲétsʲ’un-\APPL}{i-hjátsen-\APPL}{ʔi-hwʲáts’en-\APPL}{ʔi-hjéts’an-\APPL}
    \ex \cho{ʔi-ˀwǻåht-ij}{s/he shakes}{ʔi-ˀwʲátiʔ}{---}{ʔi-ˀjéehtijʔ}
    \ex \cho{ʔi-wǻqahl-\CAUS}{s/he prepares, brings up}{ʔi-wʲákahl-anit}{---}{ʔi-jákahl-at}
    \ex \cho{ʔi-hwán-hlih}{s/he is one}{ʔi-hwʲén-hli}{ʔi-hwʲén-hli}{ʔi-hjén-hiʔ}
\end{exe}

\sound{PCh}{*s} palatalizes to \intxt{(x)sʲ, (h)sʲ} in Iyojwa’aja’ except before [i], where one finds [(x)ʃ, (h)ʃ]. In Iyo’awujwa’ and Manjui, \sound{PCh}{*s} palatalizes to \intxt{(x)ʃ, (h)ʃ} or, less frequently, to \intxt{(x)sʲ, (h)sʲ}. But after \intxt{*(ˀ)j} the outcome \intxt{tʃ} is found in Iyojwa’aja’, as in \REF{jts-nietos} and \REF{jts-asco}. Here \sound{PCh}{*ts} (underlying */s/; see \sectref{ch-ts}) goes back to \sound{PM}{*ts}; we do not know if \sound{PM}{*s} yields the same outcome.

\begin{exe}
    \ex \cho{hwisúk}{palm \species{Copernicia alba}}{(h)wisʲúk}{(h)wisʲúk}{(h)wiʃúk}
    \ex \cho{ʔís-ijʔ}{it is clear/transparent}{ʔéʃ-iʔ}{---}{ʔéixʃ-iʔ}
    \ex \cho{\mbox{-}kéjtsås}{grandchildren}{\mbox{-}kítʃas}{---}{\mbox{-}kíxʃes}\label{jts-nietos}
    \ex \cho{\mbox{-}ʔǻjtsiʔ}{to feel disgust}{\mbox{-}ʔájtʃiʔ}{\mbox{-}ájsij-e}{\mbox{-}ʔájʃi(j)ʔ}\label{jts-asco}
\end{exe}

\sound{PCh}{*ts’} palatalized \intxt{tsʲ’} in Iyojwa’aja’, except before [i], where /ts’/ is found (typically realized as [tʃ’] in that position due to regressive palatalization, \sectref{ch-pal-regr}). In Iyo'awujwa' and Manjui, \sound{PCh}{*ts’} yields \intxt{tʃ’} and, less frequently, \intxt{tsʲ’}.

\begin{exe}
    \ex \cho{ʔi-ts’ú\mbox{-}}{s/he sucks}{ʔi-tsʲ’ú\mbox{-}}{ʔi-tsʲ’ú\mbox{-}}{ʔi-tʃ’ú-}
    \ex \cho{ʔi-ts’éʔ}{my belly}{ʔi-ts’íʔ \recind ʔi-tʃ’íʔ}{ʔi-tʃ’íʔ\recind ʔi-ts’í}{ʔi\mbox{-}tʃ’íʔ}
    \ex \cho{ʔi-ts’át}{s/he/it is wet}{ʔi-tsʲ’át}{---}{ʔi-tʃ’át}
\end{exe}

\sound{PCh}{*k} and \intxt{*k’} palatalize (or rather `dedorsalize') to \intxt{(x)sʲ \recind (h)sʲ \recind (x)ʃ \recind (h)ʃ \recind tʃ} and \intxt{tsʲ’ \recind tʃ’}, respectively, thus merging with \sound{PCh}{*s} and \intxt{*tsʲ’} in the same environment \citep[45]{AG83}. The postalveolar (or perhaps more precisely alveopalatal) allophones are typical of Manjui, but they have also been documented in Iyo’awujwa’ and Iyojwa’aja’ (especially before [i]; see \sectref{ch-pal-regr}).

\begin{exe}
    \ex \chox{ʔi-kúniʔ}{my sweat}{ʔi-sʲúniʔ}{i-sʲúniʔ}{ʔi-sʲúniʔ}{---}
    \ex \chox{ʔi-kéjås}{my grandson}{ʔi-síjas}{i-síjas}{ʔi-síjas}{ʔi-ʃíjes}
    \ex \chox{ʔi-kǻjuʔ \recvar ʔi-kǻjuh}{my back}{ʔi-sʲáji}{i-sʲáji}{ʔi-sʲáji}{ʔi-ʃéjuʔ}
    \ex \chox{ʔi-k(’)ásAmAh}{s/he scratches}{ʔi-tsʲ’éxsima}{i-sʲéxsama}{ʔi-sʲéxsama}{ʔi-ʃéxsama}
    \ex \chox{ʔi-kúˀm-eʔ}{s/he grabs}{ʔi-síˀm-eʔ}{i-síˀm-eʔ}{ʔi-síˀm-eʔ}{ʔi-ʃúˀm-eʔ}
    \ex \chox{ʔi-k’úu-ah}{s/he listens}{ʔi-tsʲ’ú-ji}{i-tsʲú-je}{ʔi-tsʲ’ú-je}{ʔi-tʃ’úuw-a}
    \ex \chox{ʔi-k’ókeʔ}{my waist}{ʔi-tsʲ’óki}{i-tsʲókiʔ}{ʔi-tsʲ’ókiʔ}{ʔi-tʃ’ókiʔ}
    \ex \chox{ʔi-k’élhwah}{my spouse}{\uncert~ʔi-tsʲ’émhla}{i-tsʲílfʷaʔ}{ʔi-tsʲ’ílhwa}{ʔi-tʃʲ’ílhwa}
    \ex \chox{ʔi-k’ésah}{s/he tears}{ʔi-ts’íxsa}{i-tsíxsa-ji}{ʔi-ts’íxsa-ji}{ʔi-tʃ’íxsa-haˀm}
    \ex \chox{ʔi-k’úu-ejʰ}{s/he waits}{ʔi-tsʲ’ú-je}{i-tsʲú-jije}{ʔi-tsʲ’ú-jejʰ}{ʔi-tʃ’úuw-ej\gloss{she listens to something distant}}
\end{exe}

After \sound{PCh}{*(ˀ)j} the outcome in Iyojwa'aja' is usually \intxt{tʃ} (best synchronically analyzed as a realization of /s/ in that environment); in one cognate set \REF{jky-kijlasip}, \citet[136]{ND09} documents ‹s› (‹kijlasip›), which we take to be a graphic representation of \intxt{ʃ}.\footnote{\citet{ND09} consistently uses ‹s› for both allophones of /s/, [s] and [ʃ].} In Manjui and Iyo'awujwa' the outcome is \intxt{ʃ}.

\begin{exe}
    \ex \cho{kéhla-jku-p}{fall season}{kíhla-ʃi-p]}{---}{kíhle-ʃe-p}\label{jky-kijlasip}
    \ex \chox{\mbox{-}péj-kejʔ}{to listen}{\mbox{-}pɛ́-tʃiʔ}{\mbox{-}péj-siʔ}{\mbox{-}pɛ́j-ʃiʔ}{\mbox{-}pɛ́j-ʃi(j)ʔ}
    \ex \cho{n̩k’á-jk-eʔ}{new (fem.)}{---}{---}{ʔinkʲ’é-jʃ-iʔ}
    \ex \chox{hwaʔáj-ku-jʰ}{white algarrobo trees}{hwaʔá-tʃu-ˀl}{\mbox{fʷaáj-si-ʔ}}{hwaʔáj-ʃi-j}{hwaʔáj-ʃi-j}
\end{exe}

The first palatalization also affected consonant clusters composed of two coronals, as well as those composed of a glottal and a supraglottal. In Iyojwa’aja’ only, palatalization of \intxt{kt} after \sound{PCh}{*i} is also subdialectally documented, as in \wordnl{jíktʲe \recind jíkta}{s/he would have left}.

\subsubsection{Second palatalization}\label{ch-pal2}

The second palatalization only occurs in Iyojwa’aja’ and Manjui. It only affects coronal consonants (except for /s/, /ts'/ in Manjui) as well as clusters of the shape /LL/, /hL/, where \intxt{L} stands for a coronal. It is triggered by most, but not all, surface [i]'s of diverse origins (notably from \sound{PCh}{*ᵊ} and \intxt{*u}, but not \intxt{*e}), as well as by /u/ and /hw/ and, in a few cases, by stressed /e/. Iyo’awujwa’ is notable for lacking the second palatalization \citep[41--42]{AG83}.

In the following examples the second palatalization applies both in Iyojwa’aja’ and Manjui.

\begin{exe}
    \ex \cho{hᵊ-túʍ}{you eat}{hi-tʲúʍ}{hi-tʊ́ʍ}{hi-tʲúʍ \recind hi\mbox{-}túʍ}
    \ex \cho{sᵊ-tójʔ}{I am tall}{si-tʲóˀjʔ}{ʃi-tɔ́jʔ}{ʃi-tʲójʔ}
    \ex \chox{ʔúlʔåh}{scaled dove}{---}{ólaha}{ʔʊ́laʔa}{ʔʊ́lʲ(e)ʔe \recind ʔʊ́l(a)ʔa}
    \ex \cho{sᵊʔúlah}{anteater}{soʔólʲe}{sʊʔʊ́la}{saʔʊ́la \recind saʔʊ́lʲeʔ}
    \ex \cho{túhw-naʔa}{eat it (later)}{tʊ́hw\mbox{-}nʲeʔe}{tʊ́hw-naʔa}{tʊ́hw\mbox{-}nʲeʔe \recind tʊ́hw-naʔa}
    \ex \cho{ʔᵊstúuˀn}{king vulture}{---}{ʔistʊ́ˀn}{ʔistʲúuˀn \recind ʔiʃtʲúuˀn}
    \ex \chox{ʔasétatah \recind *ʔåsétatah}{gualacate; armadillo}{ʔasɛ́tʲeta}{\mbox{ʔasétata}}{ʔasɛ́tata}{ʔasɛ́tʲeta}
\end{exe}

In the following examples, the second palatalization applies only in Iyojwa’aja’ but not in Manjui. In \REF{u-i-jisuun} and \REF{ints'eik}, an Iyojwa’aja’ cognate is lacking, but if such cognates existed one would expect them to show the second palatalization.

\begin{exe}
    \ex \chox{sᵊlǻhqajʔ \recind *sᵊlǻhqåjʔ}{wild cat}{silʲákaʔ}{siláhkaj}{siláhkajʔ}{ʃiláhkajʔ}
    \ex \cho{hᵊ-nǻʔ}{her/his father}{hi-nʲáʔ}{hi-náʔ}{hi-náʔ}
    \ex \cho{kulájʔ}{sun}{kilʲéʔ \recind kiliʔé}{kiláj}{kilájʔ}
    \ex \chox{kʼutáˀn}{thorn}{kʼitʲéˀn}{ʔitán}{ʔitáˀn}{ʔitáˀn}
    \ex \cho{p’ilusáh}{s/he is poor}{p’ilʲúxsʲe \recind p’élisʲe}{\mbox{-}pelíxsa}{p’ilisáh}\label{u-i-pilusah}
    \ex \chox{k’usáh}{cháguar}{k’isʲéh}{isáh}{ʔisáh}{ʔisáh}
    \ex \chox{túsah}{smoke}{tóxsʲe}{tóxsa}{tʊ́xsa}{tʊ́xsa}
    \ex \cho{hᵊ-sᵊʔún}{you love}{---}{hi-sʊʔʊ́n}{hi-sʊʔʊ́n}\label{u-i-jisuun}
    \ex \cho{hᵊ-sínån}{you roast}{hi-sínʲaˀn}{hi-sénʲan}{hi-séinʲan}
    \ex \cho{n̩ts'ik}{four}{---}{---}{ints'éik \recind ints'ɪ́k}\label{ints'eik}
    \ex \cho{hᵊ\mbox{-}nǻjin}{you go first}{hi-nʲáˀn}{---}{hi-nájin}
\end{exe}

The examples above show that second palatalization fails to apply in Manjui before a low vowel, and also when the target is /s, ts'/. This is quite puzzling, and we lack a convincing explanation for it. As for /s/, a typical realization in all Chorote varieties is [xs], and the velar articulation could be responsible for blocking the second palatalization.\footnote{In fact, this is our main reason to prefer [(x)s] over [(h)s] in our transcriptions.} However, [xs] (as well as [hs]) is usual after a stressed vowel, but less usual in other contexts, such as those shown above. Furthermore, no velar articulation is found in /ts'/.

The cluster \intxt{st} is immune to the second palatalization for some Manjui speakers, whereas for others it does palatalize to \intxt{ʃt(ʲ)}.

\begin{exe}
    \ex \chox{ʔᵊstǻhweʔ}{Chaco chachalaca}{ʔistʲáhwe}{istáfʷe}{ʔistáhweʔ}{\mbox{ʔistáhweʔ} \recind ʔiʃtáhweʔ}
    \ex \cho{ʔᵊstá\mbox{-}k}{cactus\species{Stetsonia coryne}}{ʔistʲé\mbox{-}k}{ʔistá\mbox{-}k}{ʔistá\mbox{-}k \recind ʔiʃtá\mbox{-}k}
    \ex \chox{ʔᵊsténiʔ / *ʔᵊsténi-k}{white quebracho}{ʔistíni-k}{isténi-k}{ʔistɛ́ni-k}{ʔistɛ́niʔ \recind ʔiʃtíniʔ}
    \ex \chox{k’ústah}{barn owl}{kʲ’ústa}{kʲústah}{kʲ’ústah}{ʔʲústa \recind ʔʲúʃta}
\end{exe}

A number of homophonous prefixes of the shape \intxt{ʔin\mbox{-}}, which go back to \sound{PCh}{*n̩\mbox{-}} (second-person inactive, indefinite possessor, and third-person nominative irrealis; see \sectref{ch-nn}), trigger the second palatalization in Manjui, but not in Iyojwa’aja’: compare \word{Mj}{ʔin\mbox{-}hlʲúk}{caraguatá bag} and \word{Ijw}{ʔin\mbox{-}hlók}{id.}. Interestingly, the palatalization is triggered even if [i] does not surface, as in \word{Mj}{ka-n-tʲún}{that s/he brings it} (underlying /ka\mbox{-}Vn\mbox{-}tún/).

The instances of [i] derived from \sound{PCh}{*e} by means of vowel raising (\sectref{ch-v-r}) fail to trigger the second palatalization even in coronals.

\begin{exe}
    \ex \cho{hwᵊkénah}{north wind, north}{wikína}{wikína}{hwikína}
    \ex \cho{kék’eh}{monk parakeet}{kík’i}{kík’ih}{kíʔih}
    \ex \cho{kéhla-juk}{red quebracho}{kíhla-jik}{kíhla-jik}{kíhlʲe\mbox{-}ek \recind kíhlʲa\mbox{-}jik \recind kíhli\mbox{-}jik}
    \ex \cho{kitéta-k}{tree\species{Prosopis elata}}{kitíta-k}{---}{kitíta-k}
\end{exe}

Finally, non-coronal consonants are not affected by the second palatalization. 

\begin{exe}
    \ex \chox{sᵊpúp}{Picui dove}{sipóp}{sipóp}{sipʊ́p}{ʃipʊ́p}
    \ex \cho{sᵊ-pǻsah}{I am quick}{si-pánsa}{si-páxsa \recind tsi-páxsa}{ʃi\mbox{-}páxsa}
    \ex \chox{k’uwáhlah}{puma}{k’iwáhla}{iwáhla}{ʔiwáhla}{ʔiwáhla}
    \ex \cho{tᵊkéhna-keʔ}{mountain}{tikíhna-kiʔ}{takíhna-kiʔ}{takíhnʲe\mbox{-}kiʔ}
    \ex \cho{hwᵊkénah}{north wind, north}{wikína}{wikína}{hwikína}
    \ex \chox{túkus}{ant}{tókis}{tókis}{tʊ́kis}{tʊ́kis}
\end{exe}

\subsubsection{Third palatalization}\label{ch-pal3}

As noted by \citet[43]{AG83} and \citet[100, fn. 36]{JC14b}, Iyo’awujwa’ and Manjui differ from Iyojwa’aja’ in that /k/ (from \sound{PCh}{*q}) does palatalize after /i/ in these varieties. This palatalization clearly occurred late enough, when the vowel raising after palatal(ized) consonants (\sectref{ch-v-r}) was no longer productive; the latter process, in turn, was fed by the first two palatalizations (\sectref{ch-pal1}--\sectref{ch-pal2}), as seen from the fact that the sequence \intxt{*iqa} yields \intxt{ikʲa} and not \intxt{*ikʲe} in Iyo’awujwa’ and Manjui. The sequence \intxt{*iqe}, however, yields \intxt{iki} at least in Manjui (probably through the stages \intxt{*ikʲe} and \intxt{*ikʲi}, with vowel raising followed by depalatalization), as in \REF{ikila}, suggesting that the raising of \intxt{*e} after palatalized consonants was still productive even after the third palatalization.

\begin{exe}
    \ex \chox{ʔi\mbox{-}qÁhlaˀm}{it is sharp}{ˀja\mbox{-}káhlaˀm}{i\mbox{-}kʲáhlam}{ʔi\mbox{-}kʲáhlaˀm}{ʔi\mbox{-}kʲáhlaˀm}
    \ex \cho{ʔi-qá-nt’ek}{my father-in-law}{ˀja-ká-nt’ek~\recind~ʔi-ká-nt’ek}{---}{ʔi-kʲá-nt’ek}
    \ex \chox{ʔi-qóhwah}{my enemy}{ʔi-kɔ́hwa \recind ja-kɔ́hwa}{i-kʲófʷah}{ʔi-kʲóhwah}{ʔi-kʲohwa}
    \ex \chox{ʔi-qÁhlek}{my liver}{ʔi-káhlik \recind ja-káhlik}{i-kʲáhlek}{ʔi-kʲáhlek}{ʔi\mbox{-}kʲáhlek}
    \ex \chox{ʔi-qÁsan}{my calf}{ʔi-káxsaˀn \recind ja-káxsaˀn}{i-kʲáxsan}{ʔi-kʲáxsan}{ʔi-kʲáxsen}
    \ex \chox{ʔi-qVjǻn}{s/he is used to}{ˀja-kájaˀn}{i-kʲojén-e}{ʔi-kʲojén-e}{---}
    \ex \cho{ʔi-qélAh}{s/he encourages}{ʔi-kɛ́la}{---}{ʔi-kíla}\label{ikila}
\end{exe}

\subsubsection{Regressive palatalization}\label{ch-pal-regr}

The regressive palatalization occurs systematically in Manjui and, less categorically, in Iyo’awujwa’. It palatalizes /s~ts’/ to [ʃ~tʃ’] before an [i] \citep[21]{AG83}. In Iyojwa’aja’ the allophones [ʃ~tʃ’] have also been documented, mostly (but not exclusively) when an /i/ precedes the consonant in question, probably conditioned by subdialectal variation. Notice that in Iyojwa'aja' this is the only environment in which [ʃ] is the usual realization of palatalized /s/, as seen in \REF{ch-isís}.

\begin{exe}
    \ex \cho{sᵊwǻlåk}{spider}{siwálak \recind ʃiwálak}{siwálak \recind ʃiwálak}{ʃiwálak}
    \ex \chox{tos-is}{snakes}{---}{tóxs-is}{tɔ́xs-is}{tɔ́xʃ-is}
    \ex \cho{\mbox{-}ǻås-ijʔ}{to sharpen}{\mbox{-}á(x)s-iʔ}{\mbox{-}áxs-iʔ \recind -áxʃ-iʔ}{\mbox{-}áaʃ-ijʔ}
    \ex\cho{ʔis-ís}{they are good}{ʔiʃ-ís}{ʔiʃ-ís}{ʔas-éis}\label{ch-isís}
\end{exe}

\subsubsection{Depalatalization}\label{ch-depal}
Consonants whose articulation involves a secondary articulation (i.e., [Cʲ]) -- this includes both /kʲ(’)/ and palatalized allophones derived by palatalization -- do not contrast with their non-palatal(ized) counterparts before [i] in any Chorote variety.\footnote{Although not strictly speaking a contrast between [Cʲi] and [Ci], there is a contrast in Iyojwa'aja' between [k̟] (the realization of /kʲ/ before [i]) and [k] in the environment \mbox{\_[i]}: \mbox{[ˈnak̟iwoʔ]} `moro bee honey (comb)' vs. the two-word expression [(ʔi)ˈnakiˈwoʔ] \recind [(ʔi)ˈnaqiˈwoʔ] `warehouseman' (underlying /Vn\mbox{-}ǻk hl\mbox{-}wó/), see \citet[79, fn. 6]{JC14b}. While it is true that the former probably contains a reflex of \sound{PCh}{*k} and the latter undoubtedly instantiates \sound{PCh}{*q}, one should keep in mind that, in the two-word expression, /k/ < \sound{PCh}{*q} is word-final, a position where the opposition between /kʲ/ and /k/ is neutralized (see \sectref{ch-k-q}).}
We represent the allophones that occur before [i] as non-palatalized in our transcriptions. In a number of cases, it is clear that these consonants were palatalized in the past, since they trigger raising in the following vowel (\sectref{ch-v-r}) and block the lowering of the following stressed vowel (\sectref{ch-vowel-lowering}). We attribute the fact that the consonants in question are no longer audibly palatalized to a sound change we dub \conc{depalatalization}, even though, strictly speaking, we cannot always ascertain there ever was a palatalization process which was later reversed. Indeed, this was not the case for /kʲ(’)/ before [i], where the pre-velar articulation of the contemporary varieties seems to continue that of Proto-Chorote, see \sectref{ch-k-q}.

\begin{exe}
    \ex \chox{\mbox{-}hwíhlek}{dream}{\mbox{-}hwéhlik}{\mbox{-}fʷéhlik}{\mbox{-}hwɪ́hlik}{\mbox{-}hwíhlik}
    \ex \cho{hwíneh}{crab}{hwíni}{---}{hwíni}
    \ex \cho{hwᵊkénah}{north wind, north}{wikína}{wikína}{hwikína}
    \ex \cho{ʔi-pén}{s/he cooks}{ʔi-píˀn}{ʔi-pín}{ʔi-pín}
\end{exe}

Consonants that were diachronically affected by palatalization but with an outcome that does not involve secondary palatalization do not undergo depalatalization. This includes \intxt{*w}~>~\intxt{j}, \intxt{*ˀw}~>~\intxt{ˀj}, \intxt{*hw}~>~\intxt{hj}, \intxt{*h}~>~\intxt{hj}, as well as sibilants. Recall that in Manjui, less categorically in Iyo’awujwa’, and even less frequently in Iyojwa’aja’, the palatalized counterparts of /kʲ/, /kʲ’/ (or /ʔʲ/ in dialects that show debuccalization), /s/, and /ts’/ are articulated as [tʃ], [tʃ’], [(x)ʃ]/[(h)ʃ], and [tʃ’], respectively.\footnote{\citet[79]{JC14b} actually describes these sounds as alveopalatal: [tɕ], [tɕ’], [(x)ɕ], and [tɕ’]. Such narrow transcription is not commonly employed in Chorote studies, and throughout this chapter we will use the symbols [tʃ], [tʃ’], [(x)ʃ], and [tʃ’].} These sounds do not depalatalize even before [i]: \word{Ijw}{ʔilʲúxʃina}{tijera net}, \wordnl{ʔéxʃ\mbox{-}ihiʔ}{it is good; thank you}, \wordnl{kasótʃ’i}{six-banded armadillo}; \word{Mj}{kaséiʃi}{snake’s rattle}, \wordnl{ʔi\mbox{-}ʃín}{s/he sends}.

In Manjui and maybe in Iyo'awujwa, the absence of a secondary palatal articulation has extended to \intxt{kʲ(’)} before [e], as in \word{PCh}{n̩k’áʔ}{new, recently} > Mj [ʔink̝̝’éʔ], cf. Ijw [ʔinkʲ’éʔ]. However, for simplicity we still represent it as kʲ(’) in our transcriptions.

It is possible that in Manjui the depalatalization has extended to other positions, as in \wordng{PCh}{*ʔi\mbox{-}nǻjin} > \intxt{*ʔi\mbox{-}nájin} > \intxt{*ʔi\mbox{-}nʲájin} > \intxt{*ʔi\mbox{-}nʲéjin} > \word{Mj}{ʔi\mbox{-}néjin}{s/he goes first}; see \sectref{ch-palat}. 

The process in question seems subject to variation and its conditions are still poorly understood, with multiple doublet forms in our corpus: \word{Mj}{ʔi\mbox{-}n(ʲ)éwetiijʔ}{cigarette}, \wordnl{ʔihn(ʲ)éta-k}{tusca tree}, \wordnl{ʔi-hl(ʲ)éˀm}{s/he defecates}. It is likewise possible that the variants with a non-palatalized consonant do not result from a diachronic depalatalization but rather from progressive vowel harmonization (\intxt{*iCa} or \intxt{*iCå} > \intxt{iCe}), on which matter see \citet{JC-amer}.

\subsubsection{Unexplained palatalization}\label{ch-unexpl}

Instances of palatalization of coronal consonants in the environment \intxt{á(ʔ)\_u} are documented in Iyojwa’aja’ and, less frequently, in Iyo'awujwa' and Manjui, which we cannot account for at present.

\begin{exe}
    \ex \cho{sátuk}{lecherón tree\species{Sapium haematospermum}}{sát(ʲ)uk}{sát(ʲ)uk}{sátuk}\label{ch-lecherón}
    \ex \cho{ʔáhluʔ}{iguana}{ʔáhlʲuʔ}{ʔáhluʔ}{ʔáhluʔ}\label{ch-iguana}
    \ex \cho{ʔalátuʔ}{hail}{ʔalátʲuʔ}{ʔalátʲuʔ}{ʔalátʊʔ}\label{ch-granizo}
    \ex \cho{\mbox{-}qáʔtuʔ}{yellow}{---}{káʔtsʲu<tʲuʔ>}{káʔatʲuʔ}\label{ch-amarillo}
\end{exe}

The PM reconstructed forms that gave rise to the cognate sets in \REF{ch-lecherón}, \REF{ch-iguana}, and \REF{ch-amarillo}, namely \intxt{*sátuˀk}, \intxt{*ʔáɬu(ʔ)}, and \intxt{*-qáʔtu(ʔ)}, respectively, do not contain the necessary environment for the palatalization processes described above. The word for `hail' is a possible borrowing, but the related forms in other languages do not explain palatalization, either (see `hail' in \sectref{wander}). It is even possible that we are dealing with a regular sound change, at least in Iyojwa’aja’.

\subsection{Consonants}\label{ch-c}

This section deals with the evolution of Proto-Chorote consonants in the contemporary varieties.

\subsubsection{\sound{PCh}{*q}}\label{ch-q}

\sound{PCh}{*q} is normally reflected as /k/ in all contemporary Chorote varieties. The phoneme in question is in fact still articulated as uvular between back vowels, as described by \citet[79]{JC14b} for Iyojwa'aja', but representing it as \intxt{k} in the modern Chorote lects is unproblematic, since the erstwhile velar stop \intxt{*k} has changed to \intxt{kʲ} in onsets (\sectref{ch-k}). \sound{PCh}{*q} is unequivocally reconstructed as a uvular stop based on two notable properties of this phoneme: it fails to undergo the first palatalization in the contemporary Chorote varieties (\sectref{ch-pal1}) and triggers a lowering effect in the preceding vowels (\sectref{ch-lowering}). Some examples of its development in the daughter lects follow.

\begin{exe}
    \ex \cho{qa}{in order to}{ka}{ka}{ka}
    \ex \cho{\mbox{-}qahlek \recind *-qǻhlek}{liver}{\mbox{-}káhlik}{\mbox{-}káhlik}{\mbox{-}káhlik}
    \ex \cho{qajáh}{Muscovy duck}{---}{kajé}{kajéh}
    \ex \cho{\mbox{-}qákaʔ}{medicine}{\mbox{-}kákʲeʔ}{\mbox{-}kákʲeʔ}{---}
    \ex \cho{\mbox{-}qákuʔ}{to distrust}{\mbox{-}kákʲuʔ}{---}{\mbox{-}kákʲuʔ}
    \ex \chox{\mbox{-}qaˀlǻʔ \recind *-qåˀlǻʔ}{leg}{---}{\mbox{-}kaláʔ}{\mbox{-}kaˀláʔ}{\mbox{-}kaˀláʔ}
    \ex \cho{qasíwoʔoh}{limpkin}{kaséwoʔo}{---}{kaséiwoʔo}
    \ex \cho{\mbox{-}qásit}{to stand}{\mbox{-}káxsit}{\mbox{-}ká(x)sit}{\mbox{-}káxʃit}
    \ex \chox{qatés}{star}{katɛ́s}{katés}{katɛ́s}{katɛ́s}
\ex \chox{\mbox{-}qatóʔ / -qató-keʔ}{elbow}{\mbox{-}káto-kiʔ}{\mbox{-}katóʔ / -kató-kiʔ}{\mbox{-}katɔ́ʔ / -katɔ́-kiʔ}{\mbox{-}katɔ́ʔ}
    \ex \cho{\mbox{-}qáwak}{belt}{\mbox{-}qáˀwak}{\mbox{-}káwak}{---}
    \ex \cho{\mbox{-}qǻhna-t}{fishhook}{\mbox{-}káhnat}{\mbox{-}káhnat}{---}
    \ex \cho{\mbox{-}qǻ-s}{foods}{\mbox{-}ká-s}{---}{\mbox{-}ká-s}
    \ex \cho{\mbox{-}qǻsile-jʰ}{guts}{\mbox{-}káxsili-∅}{\mbox{-}káxsili-∅}{\mbox{-}káxʃili-∅}
    \ex \cho{\mbox{-}qéjʔ}{costume}{\mbox{-}kɛ́ʔ}{---}{\mbox{-}kɛ́jʔ}
    \ex \chox{\mbox{-}qóso-keʔ}{node}{\mbox{-}kɔ́xso-ki}{\mbox{-}kóxso-kiʔ}{\mbox{-}kɔ́xso-kiʔ}{---}
    \ex \chox{sᵊlǻhqajʔ \recind *sᵊlǻhqåjʔ}{wild cat}{silʲákaʔ}{siláhkaj}{siláhkajʔ}{ʃiláhkajʔ}
    \ex \cho{taqám}{pacu fish}{takáˀm}{takám}{---}
    \ex \chox{t-’aq-ájʔ}{its ropes}{t-’ak-áʔ}{t-ak-áj}{t-’ak-ájʔ}{t-’ak-ájˈ}
    \ex \chox{\mbox{-}ʔaqús}{knee}{\mbox{-}ʔakós / -kós-ki}{\mbox{-}kós}{\mbox{-}kʊ́s}{\mbox{-}(ʔa)kʊ́s}
\end{exe}

Before a stressed low vowel, the Manjui reflex of \sound{PCh}{*q} has been documented dialectally (in the speech of the Jlimnájnas) as \phonetic{kx} or \phonetic{kh}: \phonetic{ˈkxaʔatiijʔ} `mate, tereré (drink)', \phonetic{ˈkxaawaʔ} `amount', \phonetic{waˈkxajʔ} `man who has sons/daughters', alongside \phonetic{ˈkaʔatiijʔ}, \phonetic{ˈkaawaʔ}, \phonetic{wahˈkajʔ}. Note that the feminine counterpart of the latter noun, where the stress shifts to the last syllable, shows only [k] in the speech of the same Jlimnájnas speaker: \phonetic{wakajéʔ} `woman who has sons/daughters'. We believe that the occurrence of \phonetic{kx} or \phonetic{kh} is purely allophonic as opposed to being a reflex of \sound{PCh}{*qh}, since \phonetic{ˈkxaʔatijʔ} is evidently related to \word{Paraguayan Guaraní}{kaʔa}{grass; mate}, where there is no reason to assume \intxt{*qh}. See the entry \hyperref[dic-qxan]{\word{PM}{*[t]qXǻn}{to dig}} in \sectref{chwonly} for a possible reflex of \intxt{*qh} in the Chorote varieties.

\subsubsection{\sound{PCh}{*k}}\label{ch-k}
\sound{PCh}{*k} is retained in the coda position in the daughter lects, but in onsets its default reflex is \intxt{kʲ} in all daughter varieties, unless palatalization (\sectref{ch-pal1}) or depalatalization (\sectref{ch-depal}) applies. In addition, the realization [k̟] is usual in Manjui and probably in Iyo'awujwa' before [e], and before [i] this is true for every Chorote lect. (As stated in \sectref{ch-depal}, we represent this sound conventionally as \intxt{kʲ} before [e] and as \intxt{k} before [i]; see also \sectref{ch-v-r} on surface [e] and [i] after \sound{PCh}{*k}.) We surmise that the sound change \sound{PCh}{*k}~>~\intxt{kʲ} took place after the disintegration of Proto-Chorote. The central piece of evidence for our claim is the fact that this sound change was bled by the first palatalization (\sectref{ch-pal1}).

\begin{exe}
    \ex \cho{\mbox{-}káʔ}{tool}{\mbox{-}kʲéʔ}{---}{---}
    \ex \cho{káˀlah}{lizard}{kʲéˀla}{kʲéˀla}{kʲéˀla}
    \ex \chox{\mbox{-}kánt’ijahaʔ}{kidney}{\mbox{-}kʲént’ijeʔ}{\mbox{-}kʲéntijeʔ}{\mbox{-}kʲént’ijeʔ}{\mbox{-}kʲéntijeeʔ}
    \ex \cho{kåhǻt-uk}{cactus\species{Cereus forbesii}}{kʲahátʲ-uk}{---}{kʲehét\mbox{-}uk}
    \ex \cho{\mbox{-}kǻnis}{testicle}{\mbox{-}kʲánis}{---}{\mbox{-}kʲénis}
    \ex \cho{\mbox{-}kǻs}{tail}{\mbox{-}kʲás}{\mbox{-}kʲés}{\mbox{-}kʲés}
    \ex \cho{\mbox{-}kǻʔ}{to be torn}{\mbox{-}kʲáʔ}{\mbox{-}kʲéʔe}{\mbox{-}kʲéʔ}
    \ex \cho{\mbox{-}kat}{collective of plants}{\mbox{-}kʲet}{\mbox{-}k(ʲ)et}{\mbox{-}kʲet}
    \ex \cho{\mbox{-}kóhjaht\mbox{-}ijʔ}{heavy}{\mbox{-}kʲóhjet\mbox{-}iʔ}{\mbox{-}kʲóhje(h)t\mbox{-}iʔ}{\mbox{-}kʲóhjiht\mbox{-}ijʔ}
    \ex \chox{\mbox{-}kójʔ}{hand}{\mbox{-}kʲoʔ}{\mbox{-}kʲój}{\mbox{-}kʲójʔ}{\mbox{-}kʲójʔ}
    \ex \chox{kóˀl}{locust}{kʲóˀl}{kʲól}{kʲóˀl}{kʲóˀl}
    \ex \cho{\mbox{-}kóweh}{middle, center}{\mbox{-}kʲówe}{\mbox{-}kʲówe}{\mbox{-}kʲówe}
    \ex \cho{\mbox{-}kúhl-\APPL}{to answer}{\mbox{-}kʲúhl-\APPL}{---}{\mbox{-}kʲúhl-\APPL}
    \ex \cho{kús-\APPL}{to be hot}{---}{kʲúxs-\APPL}{kʲús-\APPL}
    \ex \chox{\mbox{-}kút\mbox{-}eh}{to meet}{\mbox{-}kʲút\mbox{-}i}{\mbox{-}kʲút\mbox{-}eʔ}{\mbox{-}kʲút\mbox{-}e}{\mbox{-}kʲút\mbox{-}e}
    \ex \cho{\mbox{-}qákaʔ}{medicine}{\mbox{-}kákʲeʔ}{\mbox{-}kákʲeʔ}{---}
    \ex \cho{\mbox{-}qákuʔ}{to distrust}{\mbox{-}kákʲuʔ}{---}{\mbox{-}kákʲuʔ}
    \ex \chox{\mbox{-}tókoʔ}{face}{\mbox{-}tɔ́kʲoʔ}{\mbox{-}tókʲoʔ}{\mbox{-}tɔ́kʲoʔ}{\mbox{-}tɔ́kʲoʔ}
\end{exe}

In the following examples, \sound{PCh}{*k} yields \intxt{k} in the daughter varieties due to the depalatalization process (\sectref{ch-depal}); that is, we posit the following pathway of sound change: \intxt{*k}~>~\intxt{*kʲ}~>~\intxt{k}. The intermediate stage \intxt{*kʲ} is posited in order to account for the raising effect seen in the following vowel. In \REF{ch-depalat-hands}, the depalatalization is seen only in Iyo’awujwa’ and Manjui, but not in Iyojwa’aja’, which retained the Proto-Chorote vowel \intxt{o} due to accent retraction (\sectref{ch-ijw-prosody}) and no longer shows the context necessary for the depalatalization to occur. Similarly, in \REF{ch-depalat-yekiun} and \REF{ch-depalat-grab} the depalatalization applies only in those Chorote varieties where the reflex of \sound{PCh}{*k} is now followed by a high front vowel.

\begin{exe}
    \ex \cho{hwᵊkénah}{north wind, north}{wikína}{wikína}{hwikína}
    \ex \cho{tᵊ-ˀjákun}{s/he eats (intr.)}{ti-ˀjékʲuˀn}{\mbox{-}jékʲun}{ti\mbox{-}ˀjékin}\label{ch-depalat-yekiun}
    \ex \cho{kék’eh}{monk parakeet}{kík’i}{kík’ih}{kíʔi}
    \ex \cho{kéɬ}{nasal mucus, cold}{kílɬ}{---}{kíɬ}
    \ex \cho{kéhla-juk}{red quebracho}{kíhla-jik}{kíhla-jik}{kíhlʲe\mbox{-}ek \recind kíhlʲa\mbox{-}jik \recind kíhli\mbox{-}jik}
    \ex \cho{\mbox{-}kéjås}{grandson}{\mbox{-}kíjas}{\mbox{-}kíjas \recind -kíjes}{\mbox{-}kíjes}
    \ex \cho{\mbox{-}kén}{to send}{---}{---}{\mbox{-}kín}
    \ex \cho{kéteʔ}{squash}{---}{kítiʔ}{kítʲeʔ \recind kítiʔ}
    \ex \cho{\mbox{-}kilá-wot}{elder brothers}{\mbox{-}kílʲe-wot}{---}{\mbox{-}kilʲé-wat}
    \ex \chox{kitáˀnih}{Chaco tortoise}{---}{kitʲéneʔ}{kitʲéˀni}{kitíˀni \recind kitíˀnʲe}
    \ex \cho{\mbox{-}kitá-wot}{elder sisters}{\mbox{-}kítʲe-wot}{---}{\mbox{-}kitʲé-wat}
    \ex \cho{kitéta-k}{tree\species{Prosopis elata}}{kitíta-k}{---}{kitíta-k}
    \ex \cho{\mbox{-}koj-ájʰ}{hands}{\mbox{-}kʲój-e}{\mbox{-}kij-éj}{\mbox{-}kij-éjh}\label{ch-depalat-hands}
    \ex \chox{kulájʔ}{sun}{kilʲéʔ \recind kiliʔé}{kiláj}{kilájʔ}{kilájʔ}
    \ex \cho{\mbox{-}kúm-\APPL}{to grab}{\mbox{-}kím-\APPL}{\mbox{-}kʲúm-\APPL}{\mbox{-}kʲúm\mbox{-}\APPL}\label{ch-depalat-grab}
    \ex \cho{tᵊkéhna-keʔ}{mountain}{tikíhna-kiʔ}{takíhna-kiʔ}{takíhnʲe\mbox{-}kiʔ}
    \ex \cho{túkus}{ant}{tókis}{tókis}{tʊ́kis}
\end{exe}

\subsubsection{\sound{PCh}{*k(’)w}}\label{ch-kw}
\sound{PCh}{*kw} is reconstructed in order to account for the correspondence between \sound{Ijw}{kʲ} and \sound{I'w/Mj}{k}. Note the sound change \sound{PCh}{*e}~> \sound{Ijw}{o} in \intxt{j\mbox{-}ókʲos}.

\begin{exe}
    \ex \cho{j-ókwah}{s/he bites}{j-ókʲe}{\mbox{-}óka}{j-óka}
    \ex \cho{j-ókwes}{s/he frightens away}{j-ókʲos}{---}{j-ókes}
\end{exe}

The reconstruction of \sound{PCh}{*k’w} is tentative: in the only potential example, it appears to have merged with \intxt{*k’} in Manjui, yielding \sound{Mj}{ʔ} (or \intxt{tʃ’} in palatalizing contexts), whereas cognates in Iyojwa’aja’ or Iyo’awujwa’ are not presently known. The cluster is reconstructed for Proto-Chorote based on evidence from Wichí, but it is likewise possible that \sound{PCh}{*k’} should be reconstructed instead.

\begin{exe}
    \ex \cho{ʔi-k’(w)ós}{it is torn open}{---}{---}{ʔi-tʃ’ós}
\end{exe}

\subsubsection{\sound{PCh}{*q’}}\label{ch-q'}

\sound{PCh}{*q’} is normally reflected as \intxt{k’} in the contemporary Chorote varieties.

\begin{exe}
    \ex \chox{\mbox{-}q’áh}{tongue}{---}{\mbox{-}káh}{\mbox{-}k’áh}{\mbox{-}k’áh}
    \ex \chox{\mbox{-}sǻq’ålʰ}{soul}{\mbox{-}sák’al}{\mbox{-}sákal}{\mbox{-}sák’al}{---}
\end{exe}

In Manjui, \sound{PCh}{*q’} sometimes debuccalizes to \intxt{ʔ} between vowels.

\begin{exe}
    \ex \chox{\mbox{-}hnåq’ǻt}{to snore}{\mbox{-}hnák’at}{\mbox{-}hnakát}{\mbox{-}hnak’át}{\mbox{-}naʔát}
    \ex \cho{[ʔi]túq’ah}{to cook in ashes}{[ʔi]tʲók’a / -tɔ́k’a}{---}{[ʔi]tʲúʔu / -tʊ́ʔʊ}
\end{exe}

\subsubsection{\sound{PCh}{*k’}}\label{ch-k'}

Just like \sound{PCh}{*k} in onsets (\sectref{ch-k}), \sound{PCh}{*k’} acquired palatalization in non-palatalizing environments in the history of all Chorote lects, yielding \intxt{*kʲ’}, except that it yielded [k̟’] before /i/ and, at least in Manjui, also before /e/ derived from a low vowel. This  sound, however, was subject to further change in some varieties. In Manjui, \sound{PCh}{*k’} > \intxt{*kʲ’} was debuccalized to \intxt{ʔʲ} (and depalatalized to \intxt{ʔ} before \intxt{i}) in non-palatalizing environments, with very few exceptions. The same sound change often operated in Iyo’awujwa’, where \citet{AG83} attests the resulting sound as \intxt{ʔ(ʲ)} or \intxt{j}, but there are equally many cases where the original articulation remains; in this case, \citet{AG83} attests the sound in question as \intxt{kʲ’} or \intxt{kʲ}. In Iyojwa’aja’, \sound{PCh}{*kʲ’} is mostly retained, but in a few cases one finds a debuccalized variant with \intxt{ˀj} (these exceptions are probably best viewed as dialectal borrowings). Note that in Iyo’awujwa’ and Manjui \intxt{ʔʲ} contrasts both with \intxt{ʔ} and \intxt{ˀj}, whereas the product of debuccalization of \sound{PCh}{*k’} in Iyojwa’aja’ is not distinct from \intxt{ˀj} < \sound{PCh}{*ˀj}.

\begin{exe}\label{chorote-kj'-ex}
    \ex \chox{\mbox{-}k’alóʔ}{cheek}{\mbox{-}kʲ’óloʔ}{\mbox{-}kʲalóʔ}{\mbox{-}kʲ’alɔ́ʔ}{\mbox{-}ʔʲelɔ́ʔ}
    \ex \cho{\mbox{-}k’éhn-aˀm}{to extend}{\mbox{-}k’íhn-aˀm}{---}{\mbox{-}ʔíhn-aˀm}
    \ex \cho{\mbox{-}k’ésah}{to divide}{\mbox{-}k’íxsa}{---}{\mbox{-}ʔíxsah-\APPL}
    \ex \chox{k’ihlóʔ}{armadillo}{k’ihlʲóʔ}{ihlʲóʔ}{\mbox{ʔihlʲóʔ}}{ʔihl(ʲ)óʔ}
    \ex \chox{\mbox{-}k’íhnaʔ}{younger sister}{\mbox{-}kʼíhnʲa \recind -ˀjíhnʲa}{\mbox{-}kíhnʲeʔ}{\mbox{-}k’íhnʲeʔ}{\mbox{-}ʔíhnʲeʔ}
    \ex \chox{\mbox{-}k’ínih}{younger brother}{\mbox{-}kʼíni \recind -ˀjíni}{\mbox{-}jíni}{\mbox{-}ʔíni}{\mbox{-}ʔíni}
    \ex \cho{\mbox{-}k’ó-keʔ}{waist}{\mbox{-}kʲ’ó-kiʔ}{\mbox{-}kʲ’ó-kiʔ}{\mbox{-}ʔʲó-kiʔ}
    \ex \chox{\mbox{-}k’óoteʔ}{ear}{\mbox{-}kʲ’óteʔ}{\mbox{-}kʲóteʔ}{\mbox{-}kʲ’óteʔ}{\mbox{-}ʔʲóoteʔ}
    \ex \chox{k’újʔ}{cold}{---}{\mbox{-}júj-\APPL}{\mbox{-}ʔʲúj-\APPL}{ʔʲújʔ}
    \ex \chox{k’usáh}{cháguar}{k’isʲéh}{isáh}{ʔisáh}{ʔisáh}
    \ex \chox{k’ústah}{barn owl}{kʲ’ústa}{kʲústah}{kʲ’ústah}{ʔʲústa \recind ʔʲúʃta}
    \ex \chox{kʼutáˀn}{thorn}{kʼitʲéˀn}{ʔitán}{ʔitáˀn}{ʔitáˀn}
    \ex \chox{\mbox{-}k’úʔ}{horn}{\mbox{-}kʲ’úʔ}{\mbox{-}kʲúʔ}{\mbox{-}kʲ’úʔ}{\mbox{-}ʔʲúʔ}
    \ex \cho{kʲ’VlésAh}{\intxt{Jacaratia corumbensis}}{k’ilíxsah \recind ʔilíxsah}{ʔilíxsa}{ʔilíxsa}
    \ex \chox{\mbox{-}pók’oʔ}{foot}{\mbox{-}pɔ́kʲ’oʔ}{\mbox{-}pókʲ’oʔ}{\mbox{-}pɔ́kʲ’oʔ}{\mbox{-}pɔ́ʔoʔ}
    \ex \chox{\mbox{-}ték’uhluʔ}{brain, marrow}{\mbox{-}tɛ́k’ihliʔ}{\mbox{-}tékihlí}{\mbox{-}tɛ́k’ihliʔ}{\mbox{-}tɛ́ʔihlʲuʔ}
\end{exe}

One notable exception is the Manjui reflex of \word{PCh}{*n̩k’áʔ}{new} and its derivatives, where the velar articulation is preserved: \wordnl{ʔinkʲ’éʔ \recind kʲ’éʔ}{recently}, \wordnl{ʔinkʲ’é\mbox{-}jik}{new (masculine)}, \wordnl{ʔinkʲ’é\mbox{-}jʃ\mbox{-}iʔ}{new (feminine)}.

In palatalizing environments, the debuccalization does not apply, suggesting that by the time when the sound change \sound{PCh}{*k’}~>~\intxt{*kʲ’}~>~\intxt{ʔʲ} took place the first palatalization (\sectref{ch-pal1}) had already transformed \sound{PCh}{*kʲ’} into an affricate. For example, the Manjui reflexes of \word{PCh}{*n̩\mbox{-}k’óoteʔ} and \wordnl{*ʔi\mbox{-}k’óoteʔ}{my ear} are, respectively, \intxt{ʔin\mbox{-}ʔʲóoteʔ} and \intxt{ʔi\mbox{-}tʃʲ’óoteʔ}. For examples from Iyo’awujwa’, see \citet[45]{AG83}.

\subsubsection{Word-final sonorants in Iyojwa’aja’}

In Iyojwa’aja’, word-final sonorants receive obligatory glottalization \citep[87--88]{JC14b} and surface as sequences of the type \intxt{ʔC}. An intrusive vowel shows up optionally (dialectally?) if the last syllable is stressed. For example, the forms /A\mbox{-}lǻn/ `I kill' and /Vn\mbox{-}tate\mbox{-}l/ `one's eyes' surface as [ʔaˈlaʔan], [ʔinˈtateʔl]. In addition, the approximants /j/ and /w/ not only acquire glottalization but are themselves deleted in the coda position in Iyojwa’aja’: /A\mbox{-}kʲ’éw/ `I stick' surfaces as [ʔaˈkʲeʔ].

As a consequence, Iyojwa’aja’ no longer distinguishes between plain and glottalized sonorants in the word-final position, a contrast clearly present in Proto-Chorote. For example, the pronoun \wordnl{*j\mbox{-}ǻˀm}{I} and the third-person irrealis form \wordnl{*j\mbox{-}ǻm}{that s/he go away} are now homophonous in Iyojwa’aja’ and surface as \intxt{jáˀm} (phonetically \phonetic{ˈjaʔam}, underlying representation /jåm/). The erstwhile contrast is preserved in Manjui, where \wordnl{j\mbox{-}éˀm}{I} contrasts with \wordnl{j\mbox{-}ém}{that s/he go away}.

\subsubsection{Loss of \intxt{*h} word-finally}\label{ch-caida-de-h-final}

In all modern varieties of Chorote, /h/ is usually deleted word-finally in unstressed syllables, as in \word{Ijw}{ti\mbox{-}lʲákinʲ\mbox{-}e}{one dances} (underlying /t\mbox{-}lǻkʲVn\mbox{-}ah/). As argued in detail by \citet[85--89]{JC14b}, /h/ is still present in the underlying representation in such cases, since it prevents the insertion of [ʔ] before a pause (\sectref{ch-saltillo}). It sometimes appears in \cits{AG83} transcriptions of Iyo’awujwa’ in unstressed syllables (and, conversely, there are also unexpected instances of its absence even in stressed syllables in her transcriptions, as in \word{I’w}{ilí}{s/he washes}).

\begin{exe}
    \ex \cho{\mbox{-}ǻtah}{to be fat}{\mbox{-}áta}{\mbox{-}átah}{\mbox{-}áta}
    \ex \cho{hwíneh}{crab}{hwéni}{---}{hwéni}
    \ex \cho{hwᵊkénah}{north wind, north}{wikína}{wikína}{hwikína}
    \ex \cho{káˀlah}{lizard}{kʲéˀla}{kʲéˀla}{kʲéˀla}
    \ex \cho{kék’eh}{monk parakeet}{kík’i}{kík’ih}{kíʔi}
    \ex \cho{\mbox{-}koj-ájʰ}{hands}{\mbox{-}kʲój-e}{\mbox{-}kij-éj}{\mbox{-}kij-éjh}
    \ex \chox{k’uwáhlah}{puma}{k’iwáhla}{iwáhla}{ʔiwáhla}{ʔiwáhla}
    \ex \chox{pǻˀjih}{frog \species{Leptodactylus sp.}}{páˀji}{páji}{páˀji}{páˀji \recind páʔi}
    \ex \chox{\mbox{-}sǻq’ålʰ}{soul}{\mbox{-}sák’al}{\mbox{-}sákal}{\mbox{-}sák’al}{---}
    \ex \chox{túsah}{smoke}{tóxsʲe}{tóxsa}{tʊ́xsa}{tʊ́xsa}
    \ex \cho{wóp’ih}{snowy egret}{wóp’i}{---}{wóp’ih}
    \ex \cho{ʔáwusah}{peccary}{ʔáusʲe}{---}{ʔáwasa}
\end{exe}

\subsubsection{Loss of \intxt{*h} in Manjui}\label{ch-caida-de-h-mj}

In Manjui, \sound{PCh}{*h} is typically lost in unstressed syllables between vowels: compare \wordng{I’w}{ajéh\mbox{-}es} and \word{Mj}{ʔaˀjé\mbox{-}es}{jaguars}, \wordng{I’w}{wótaha} and \word{Mj}{wótaa}{chicken} (likely borrowed from \wordng{Ni}{βotåxåx}). In some cases Iyo’awujwa’ also undergoes this process.

\begin{exe}
    \ex \chox{ˀnǻhåteʔ}{Chacoan mara}{ˀnáhate}{náateʔ}{ˀnáateʔ}{ˀnáateʔ}
    \ex \cho{\mbox{-}ʔáhateʔ}{female breast}{\mbox{-}ʔáhate}{---}{\mbox{-}ʔáateʔ}
    \ex \cho{j-í-heˀn(eʔ)}{s/he sits}{j-í-hiˀn}{---}{j-í-iˀnʲeʔ}
\end{exe}

However, a sequence of /h/ and /h/ at morpheme boundaries always yields \intxt{h} in Manjui.

\ea
Manjui \citep{JC18}
    \begin{xlist}
        \ex \gll /i-ˀjas-eh-heˀneʔ/~[ʔiˈʔjeseheʔneʔ]\\
            3.\textsc{i.rls}-ask-\APPL-\PL\\
            \glt `s/he asks something to someone'
    \end{xlist}
\z

\subsubsection{Sequences of \sound{PCh}{*h} plus stop}

Proto-Chorote clusters of the type \intxt{*h}~+~stop are preserved in Manjui but are lost in Iyojwa’aja’. Iyo’awujwa’ usually preserves them, but some variation is attested.

\begin{exe}
    \ex \chox{sᵊlǻhqajʔ \recind *sᵊlǻhqåjʔ}{wild cat}{silʲákaʔ}{siláhkaj}{siláhkajʔ}{ʃiláhkajʔ}
    \ex \chox{\mbox{-}ʔóhtaleʔ \recind *-ʔóhtåleʔ}{heart}{\mbox{-}ʔɔ́tale}{\mbox{-}óhteleʔ \recind -óhtaleʔ}{\mbox{-}ʔɔ́hteleʔ \recind -ʔɔ́htaleʔ}{\mbox{-}ʔɔ́hteleʔ \recind -ʔɔ́htaleʔ}
    \ex \cho{wáhtuk}{plant sp.}{(h)wátok\gloss{\textit{Enterolobium contortisiliquum}}}{wáhtok\gloss{\textit{Albizia inundata}}}{wáhtuk\gloss{\textit{Albizia inundata}}}
    \ex \cho{kóhjat-ijʔ}{to be heavy}{kʲóhjet-iʔ}{kʲóhje(h)t-iʔ}{kʲóhjiht\mbox{-}ijʔ}
    \ex \chox{\mbox{-}héhte\mbox{-}}{head}{\mbox{-}hɛ́te\mbox{-}}{\mbox{-}héte\mbox{-}}{\mbox{-}hɛ́te\mbox{-}}{\mbox{-}hɛ́hte-} (vocalic stem)
    \ex \cho{tíhte\mbox{-}}{plate}{títe\mbox{-}}{téjti\mbox{-}}{téihti-} (vocalic stem)
\end{exe}

As a consequence of this sound change, Iyojwa’aja’ has a synchronically active alternation whereby the underlying sequences of a stop and /h/ do not yield /hC/ (as in other dialects) but rather /C/.

\ea
Iyojwa'aja' \citep{JC14a}
    \begin{xlist}
        \ex \gll /tát-hen/~[ˈtateʔn]\\
            throw-\APPL:downwards\\
            \glt `throw it to her/him!'
        \ex \gll /i-é-håp hA-ná Asíhnå/~[ˈjihapanaˈsehnʲaʔ]\\
            3.\textsc{i.rls}-be-\APPL:near {\textsc{fem}}-this woman\\
            \glt `s/he is next to the woman'
    \end{xlist}
\z

\subsubsection{Loss of \intxt{*h} in \sound{PCh}{*hw}, \intxt{*hl}}\label{ch-caida-de-h-hw-hl}

\sound{PCh}{*hw} sporadically yields \intxt{w} in pretonic syllables in all Chorote varieties.

\begin{exe}
    \ex \cho{(-)hwVhlek}{mortar}{(-)(h)wánhlek}{wihlík}{(h)wihlík}
    \ex \cho{hwisúk}{palm \species{Copernicia alba}}{(h)wisʲúk}{(h)wisʲúk}{(h)wiʃúk}
    \ex \cho{hwᵊkénah}{north wind, north}{wikína}{wikína}{hwikína}
    \ex \cho{hwiˀjét}{ice, frost}{wiˀjít}{---}{hwiˀjít}
\end{exe}

\citet[22--23]{AG83} documents a number of cases of synchronic variation of \intxt{fʷ} and \intxt{w}, \intxt{hl} ([ˣl] in her transcription) and \intxt{l} in Iyo’awujwa’, as in \wordnl{nafʷáxlek \recind nawáhlek}{wasp \species{Brachygastra lecheguana}}, \wordnl{\mbox{-}fʷésʲe}{bad} / \wordnl{si\mbox{-}wíxsʲe}{I am bad}, \wordnl{hlóxsa \recind lúxsa}{girl}, \wordnl{hlémiʔ \recind lémiʔ}{white}. She further states that the occurrence of [l] as a reflex of \sound{PCh}{*hl} is predominant in the third-person pronouns (\wordnl{l\mbox{-}ám}{s/he}, \wordnl{l\mbox{-}ám\mbox{-}is}{they}) and in the second-person active prefix (\wordnl{l\mbox{-}éj álsa\mbox{-}ham}{you are in the forest}). In Carol’s Iyo’awujwa’ records, /hl/ is systematically realized as [l] after a pause.

\subsubsection{\sound{PCh}{*s}}\label{pch-s}

In the contemporary varieties of Chorote, the pronunciation of /s/ varies between \phonetic{s}, \phonetic{xs}, and \phonetic{hs} intervocalically \citep[79]{JC18,JC14b}. This happens both in Iyojwa’aja’ ([ˈʔɔxsoʔ] \recind [ˈʔɔhsoʔ] for /óso/ `squash') and in Iyo’awujwa’ and Manjui ([ˈtaxsina] \recind [ˈtahsena] for /tásVnah/ `toad'). The realization \phonetic{xs} \recind \phonetic{hs} is especially frequent after a stressed syllable, and our transcriptions regularly reflect this.

For some speakers of Manjui, /s/ may surface as [ʃ] in the environments /i\_t/, /u\_t/, and /\_kʲ/: \wordnl{ʔiʃtáh}{cactus fruit\species{Stetsonia coryne}}, \wordnl{ʔʲúʃta}{barn owl}, \wordnl{hʊ́ʃkije}{be careful}, \wordnl{náaʃ kʲuʔ}{hello}.

Finally, we note that some speakers of Iyo’awujwa’ may articulate the reflex of \sound{PCh}{*s} as [ts] word-initially, at least in the 1\textsc{sg}.\textsc{inact} and 1\textsc{pl}.\textsc{poss} prefixes \citep[68–70, 76–77]{AG83}, as in \REF{v2a-tsipaxsa}. For the 1\textsc{pl}.\textsc{poss} prefix only, \citet{AG83} documents this realization not only for Iyo’awujwa’, but also for Manjui. In Carol’s data, [ts] does not occur in Manjui at all, and in Iyo’awujwa’ it is found in the speech of one speaker from La Merced. Even though it is tempting to speculate that Proto-Chorote could have actually retained the Proto-Mataguayan opposition between */s/ and */ts/ (contrary to our claim in \sectref{ch-ts}), the allophone [ts] in Iyo’awujwa’ is only marginally documented, and for the time being we contend that the evidence is insufficient to reconstruct the phoneme */ts/ for Proto-Chorote.

\subsubsection{Syllabic \intxt{*n̩}}\label{ch-nn}

\sound{PCh}{*n̩} is a straightforward retention from \sound{PM}{*n̩}. Most instances of this sound correspond to the allomorphs of three homophonous prefixes that occur word-initially before supraglottal consonants (but not after a particle that ends in a vowel): the second-person inactive prefix, the indefinite possessor prefix, or the third-person nominative irrealis prefix. It is reflected as \intxt{ʔin} in Iyojwa’aja’ and Manjui, whereas in Iyo’awujwa’ the attested reflexes include \intxt{in}, \intxt{en}, \intxt{n̩}, and \intxt{n}. The syllabic nasal is synchronically documented, for example, in \word{I’w}{n̩\mbox{-}tókʲoʔ}{one’s face} \citep[69]{AG83}. In addition, this sound assimilates its place of articulation to that of the following consonant, as in \word{Ijw}{ʔim\mbox{-}páˀn}{that s/he swim}, \wordnl{ʔim\mbox{-}pɛ́l\mbox{-}is}{movie} (literally `one's shadows'), \word{I’w}{im\mbox{-}pélisa}{you are poor}, \wordnl{m̩\mbox{-}póxse\mbox{-}j \recind im\mbox{-}póxse\mbox{-}j}{one’s beards}, and is deleted before a nasal, as in \word{Ijw}{ʔi\mbox{-}náhj\mbox{-}eˀn}{s/he gives you a bath}, \wordnl{ʔi\mbox{-}máʔ}{that s/he sleep}, \wordnl{ʔi\mbox{-}ní\mbox{-}ˀwɛ́ˀn}{s/he sees herself/himself}; \word{I’w}{i\mbox{-}nálen}{you are hungry}, \wordnl{i\mbox{-}mánisʲem}{you are the last}, \wordnl{i\mbox{-}má\mbox{-}juʔ}{you feel sleepy} \citep[75--79]{JC14a,AG83}.

The insertion of a vowel (documented as [i] in all three modern varieties, and sporadically as [e] in Iyo’awujwa’) must have occurred fairly late, when the first palatalization (\sectref{ch-pal1}) and the second palatalization (\sectref{ch-pal2}) were already complete. This is evident from the fact that the innovative vowel [i] fails to trigger palatalization of coronals in Iyojwa’aja’, as would be expected if one were to reconstruct \sound{PCh}{*ʔᵊn}, \intxt{*ʔin}, or \intxt{*ʔen}.

\begin{exe}
    \ex \chox{n̩-tójʔ}{you are tall}{ʔin-tɔ́ʔ}{in-tój}{ʔin-tɔ́jʔ}{ʔin-tʲójʔ}
    \ex \chox{n̩-pǻsat}{one’s lip}{ʔim-páxsat}{im-páxsat}{ʔim-páxsat}{ʔim-páxsat}
    \ex \chox{n̩-tóweh}{one’s belly}{ʔin-tɔ́we}{in-tówe}{ʔin-tɔ́we}{---}
    \ex \chox{n̩-púse-jʰ}{one’s beards}{ʔim-póxsi-ˀl}{im-póxse-j \recind m̩\mbox{-}póxse\mbox{-}j}{ʔim-pʊ́xse-j \recind m̩\mbox{-}pʊ́xse\mbox{-}j}{ʔim-pʊ́xse-j}
    \ex \chox{n̩-tókoʔ}{one’s face}{n̩-tɔ́kʲoʔ}{n̩-tókʲoʔ}{n̩-tɔ́kʲoʔ}{ʔin-tɔ́kʲoʔ}
    \ex \cho{n̩-ta-téʔ}{one’s eye}{ʔin-táteʔ}{---}{ʔin-ta-tɛ́ʔ}
\end{exe}

In Iyojwa’aja’ and Manjui, the allomorph \intxt{ʔin\mbox{-}} (or similar), originally found before supraglottal consonants only, has been extended to vowel-initial stems, as in \word{Ijw}{ʔin\mbox{-}ámtik}{one’s word}, \word{Mj}{ʔin\mbox{-}ɛ́j\mbox{-}is}{one’s names}.\footnote{\citet{JC14a} has also documented a variant with a geminate \intxt{n} in Iyojwa’aja’ in such cases, as in \wordnl{ʔinn\mbox{-}áh\mbox{ak}}{you were beaten} (as opposed to \wordng{PCh}{*n\mbox{-}áh\mbox{-}ak}). Our contention is that \intxt{ʔin\mbox{-}} was historically added to the etymological form with the allomorph \intxt{*n\mbox{-}} when the latter ceased to be productive.} This development has also occurred in many \intxt{ʔ}\mbox{-}initial stems, where it affected the second-person inactive prefix and the third-person nominative irrealis prefix, but not the indefinite possessor prefix, which retained its original allomorphy pattern (\word{Ijw/Mj}{ˀnɔ́t}{one's chest}, underlying /n\mbox{-}ʔot/). In Iyo’awujwa’, the development in question did not affect at least the second-person inactive prefix: \wordnl{n\mbox{-}éˀleʔ}{you are dry}, \wordnl{n\mbox{-}óppaleen}{you hiccup}, \wordnl{n\mbox{-}átah}{you are fat} \citep[77]{AG83}.

Another morpheme that may have contained a syllabic nasal in Proto-Chorote, albeit in a different position, is the pluractional suffix \intxt{*\mbox{-}ʔn̩}, with a probable cognate in Nivaĉle. In Iyo’awujwa’ and Manjui, it behaves as an independent phonological word: \wordng{I’w}{ʔen}, \wordng{Mj}{ʔɪn}. The Iyojwa’aja’ reflex is the unstressed enclitic or suffix \intxt{\mbox{-}ˀni} (underlying /\mbox{-}ˀnih/).

\subsubsection{Epenthetic glides}

A glide is inserted between vowels at base/suffix or base/enclitic boundary. The glide is /j/ in Iyojwa'aja' and /w/ in Iyo'awujwa' and Manjui.

\begin{exe}
    \ex \cho{ʔi-hlú-ah}{s/he orders}{ʔi-hlʲú-j-e}{---}{ʔi-hlʲú-w-a}
    \ex \cho{tᵊ-pó-eh}{it is full of}{ti-pɔ́-j-i}{ti-pɔ́-w-e}{ta-pɔ́-w-e}
    \ex \cho{ʔi-hó-ejʰ}{s/he goes to}{ʔi-hʲó-j-i}{ʔi-hʲó-w-ej}{ʔi\mbox{-}hʲo\mbox{-}w\mbox{-}ej}
\end{exe}

\subsubsection{Consonant clusters with \intxt{l} in Manjui}

In Manjui, several consonant clusters reconstructible to Proto-Chorote undergo a seemingly irregular change, whereby the initial consonant is replaced with /l/, often pronounced as [ɬ] in this environment \citep[26]{AG83}.

\begin{exe}
    \ex \chox{kempénah}{orphan}{kimpɛ́na}{kimpéna}{kimpɛ́na}{kilpɛ́na}
    \ex \chox{ʔaskúnaʔ}{spotted sorubim}{ʔaskʲúnʲeʔ}{askʲúnaʔ}{ʔaskʲúnaʔ}{ʔalkʲúnaʔ}
    \ex \chox{ʔa\mbox{-}skúhn\mbox{-}eˀn(eʔ)}{I wander}{ʔa\mbox{-}skʲúhn\mbox{-}iˀn}{a\mbox{-}skʲúhn\mbox{-en}}{ʔa\mbox{-}skʲúhn\mbox{-eˀn}}{ʔa\mbox{-}lkʲúhn\mbox{-}eˀneʔ}
\end{exe}

Yet in other cases, the change seems to be regular: \sound{PCh}{*ʍ} and \intxt{*ɬ} (allophones of PCh~*/hw/ and */hl/, respectively, in codas) are reflected as \sound{Manjui}{l} before a stop, dialectally realized as \intxt{*ɬ} in that position, whereas the other dialects show \intxt{h} in the same environment.

\begin{exe}
    \ex \cho{naɬqá-p \recind *-å\mbox{-}}{year}{nahkáp}{nahkáp}{nalkáp}
    \ex \cho{t-’aʍqós}{s/he crawls}{t-’ahkɔ́s-ˀn̩}{---}{t-’alkɔ́s}
\end{exe}

\subsubsection{Other consonantal changes}

Sporadic alternations are documented between nasal and oral labial sonorants. For example, \sound{PCh}{*ˀm} yielded \sound{Ijw}{ˀw} in \wordng{PCh}{*[ʔa]ˀmánhliʔ} > \word{Ijw}{ˀwán\mbox{-}hle\mbox{-}ʔe}{to stay}, whereas \sound{PCh}{*lhw} yielded \sound{Ijw}{mhl} in \wordnl{*\mbox{-}k’élhwah}{spouse} > \wordng{Ijw}{\mbox{-}kʲ’émhla}. Synchronic variation is attested in \word{Mj}{\mbox{-}kíˀwehnan \recind \mbox{-}kiˀmehnan}{to be pregnant} (compare \wordng{Ijw}{\mbox{-}kʲúʔuhnʲeˀn}).

\subsection{Vowels}\label{ch-v}

This section deals with the evolution of Proto-Chorote vowels in the contemporary varieties.

\subsubsection{Vowel raising after palatal and palatalized consonants}\label{ch-v-r}

In all three contemporary varieties of Chorote, the vowels \intxt{*a} and \intxt{*e} are raised to [e] and [i], respectively, after palatal consonants, as in \REF{ch-v-r-kj-1}--\REF{ch-v-r-kj-2}, and after palatalized consonants, derived through the first palatalization, as in \REF{ch-v-r-pal1-1}--\REF{ch-v-r-pal1-2}, or the second palatalization, as in \REF{ch-v-r-pal2-1}--\REF{ch-v-r-pal2-2}. Recall that palatalization is not perceptible before a surface \intxt{i} (except in consonants that change their place of articulation when palatalized, such as \intxt{*w}~>~\intxt{j}, \intxt{*ˀw}~>~\intxt{ˀj}, \intxt{*hw}~>~\intxt{hj}, \intxt{*h}~>~\intxt{hj}, \intxt{*ˀw}~>~\intxt{ˀj}, \intxt{*k}~>~\intxt{ʃ}, \intxt{*k’}~>~\intxt{tʃ’}, \intxt{*s}~>~\intxt{ʃ}, \intxt{*ts’}~>~\intxt{tʃ’}); this depalatalization process (\sectref{ch-depal}) is fed by the raising of \sound{PCh}{*e} after palatal(ized) consonants, resulting in the development \intxt{*Cʲe} > \intxt{*Cʲi} > \intxt{Ci}. Similarly, the depalatalization before \intxt{e} in Manjui was fed by the raising of \intxt{*a} and \intxt{*å} after palatal(ized) consonants, as in \wordng{PCh}{*ʔi\mbox{-}nǻjin} > \intxt{*ʔi\mbox{-}nájin} > \intxt{*ʔi\mbox{-}nʲájin} > \intxt{*ʔi\mbox{-}nʲéjin} > \word{Mj}{ʔi\mbox{-}néjin}{s/he goes first}.

\begin{exe}
    \ex \cho{hwiˀjét}{ice, frost}{wiˀjít}{---}{hwiˀjít}\label{ch-v-r-kj-1}
    \ex \cho{\mbox{-}jáɬ}{breath}{\mbox{-}jéɬ}{\mbox{-}jél}{\mbox{-}jéɬ}
    \ex \chox{\mbox{-}ˀjámuk}{feces}{\mbox{-}ˀjémuk}{\mbox{-}jémuk}{\mbox{-}ˀjémuk}{\mbox{-}ˀjémuk}
    \ex \chox{\mbox{-}ˀjákun}{to eat (intr.)}{\mbox{-}ˀjékʲuˀn}{\mbox{-}jékʲun}{\mbox{-}ˀjékʲun}{\mbox{-}ˀjékin}
    \ex \cho{j-é-ʔeʔ}{s/he is in}{j-íʔiʔ}{j-íʔiʔ}{j-íʔiʔ}
    \ex \cho{qajáh}{Muscovy duck}{---}{kajé}{kajéh}
    \ex \cho{kéɬ}{nasal mucus, cold}{kílɬ}{---}{kíɬ}
    \ex \cho{kék’eh}{monk parakeet}{kík’i}{kík’ih}{kíʔi}
    \ex \cho{kéhla-juk}{red quebracho}{kíhla-jik}{kíhla-jik}{kíhlʲe\mbox{-}ek \recind kíhlʲa\mbox{-}jik \recind kíhli\mbox{-}jik}
    \ex \cho{kéteʔ}{squash}{---}{kítiʔ}{kítʲeʔ \recind kítiʔ}
    \ex \cho{\mbox{-}koj-ájʰ}{hands}{\mbox{-}kʲój-e}{\mbox{-}kij-éj}{\mbox{-}kij-éjh}
    \ex \cho{káˀlah}{lizard}{kʲéˀla}{kʲéˀla}{kʲéˀla}
    \ex \chox{wósᵊk’at}{red-crested cardinal}{---}{wóxsijét}{wóxsiʔʲet}{wóxʃeʔet}
    \ex \chox{ʔéjaʔ}{mosquito}{ʔɛ́jeʔ}{ʔéjeʔ}{ʔɛ́jeʔ}{ʔɛ́jeʔ}
    \ex \cho{ʔijéstah}{dew}{jísta}{\mbox{-}jísta \recind -jíste}{ʔijísta \recind ʔajísta}\label{ch-v-r-kj-2}
    \ex \chox{ʔihnáta-k}{tusca tree}{ʔihnʲéta-k}{ihnʲéta-k}{ʔihnʲéta-k}{ʔihn(ʲ)éta-k}\label{ch-v-r-pal1-1}
    \ex \chox{\mbox{-}hwíhlek}{dream}{\mbox{-}hwéhlik}{\mbox{-}fʷéhlik}{\mbox{-}hwɪ́hlik}{\mbox{-}hwíhlik}
    \ex \cho{ʔi-ˀwén}{s/he sees}{ʔi-ˀwíˀn}{ʔi-ˀwín}{ʔi-ˀjín} 
    \ex \cho{ʔi-ˀwét}{my place}{ʔi-ˀwít}{ʔi-ˀwít}{ʔi-ˀjít}\label{ch-v-r-pal1-2}
    \ex \chox{ʔúlʔåh}{scaled dove}{---}{ólaha}{ʔʊ́laʔa}{ʔʊ́lʲ(e)ʔe \recind ʔʊ́l(a)ʔa}\label{ch-v-r-pal2-1}
    \ex \cho{sᵊʔúlah}{anteater}{soʔólʲe}{sʊʔʊ́la}{saʔʊ́la \recind saʔʊ́lʲe}
    \ex \cho{túhw-naʔa}{eat it (later)}{tʊ́hw\mbox{-}nʲeʔe}{tʊ́hw-naʔa}{tʊ́hw\mbox{-}nʲeʔe \recind tʊ́hw-naʔa}\label{ch-v-r-pal2-2}
\end{exe}

In Manjui and (somewhat less systematically) in Iyo’awujwa’, not only \sound{PCh}{*a}, but also \sound{PCh}{*å} is raised to [e] after palatal and palatalized consonants, on which see \sectref{ch-ao-a}.

\begin{exe}
    \ex \chox{\mbox{-}hwéˀjåʔ}{to fly}{\mbox{-}hwɛ́ˀjaʔ}{\mbox{-}fʷéjeʔ}{\mbox{-}hwɛ́ˀjeʔ}{\mbox{-}hwɛ́ˀjeʔ}
    \ex \cho{\mbox{-}kǻnis}{testicle}{\mbox{-}kʲánis}{---}{\mbox{-}kʲénis}
    \ex \cho{\mbox{-}kǻs}{tail}{\mbox{-}kʲás}{\mbox{-}kʲés}{\mbox{-}kʲés}
    \ex \cho{ʔi-hlǻˀm}{s/he defecates}{ʔi-hlʲáˀm}{---}{ʔi-hlʲéˀm}
    \ex \cho{\mbox{-}kéjås}{grandchildren}{\mbox{-}kíjas}{\mbox{-}kíjas \recind -kíjes}{\mbox{-}kíjes}
    \ex \cho{ʔi-kǻt}{it is red}{ʔi-sʲát}{ʔi-sʲát \recind [ʔi]sʲét}{ʔi-ʃét}
    \ex \cho{j-ǻs}{my son}{j-ás}{j-és}{j-és}
    \ex \cho{j-ǻp}{s/he cries}{j-áp}{j-ép}{j-ép}
    \ex \chox{ʔipǻk}{straw}{ʔipʲák}{ipʲék}{ʔipʲék}{---}
    \ex \cho{ʔi-hlǻˀm}{s/he defecates}{ʔi-hlʲáˀm}{---}{ʔi-hl(ʲ)éˀm}
    \ex \cho{ʔi-mǻʔ}{s/he sleeps}{ʔi-mʲáʔ}{---}{ ʔi-mʲéʔ \recind ʔi-máʔ\gloss{s/he camps}}
\end{exe}

The third palatalization (\sectref{ch-pal3}) occurred late enough to counterfeed the raising of \intxt{*a} to \intxt{e} in the varieties that undergo it (Iyo’awujwa’ and Manjui). That way, \sound{PCh}{*iqa} and \intxt{*iqå} are reflected as \intxt{ikʲa} and not as \intxt{*ikʲe} in these varieties. Interestingly, the sequence \intxt{*iqe} does yield \intxt{iki} at least in Manjui (probably through the stages \intxt{*ikʲe} and \intxt{*ikʲi}, with vowel raising followed by depalatalization), suggesting that the raising of \intxt{*e} after palatalized consonants was still productive even after the third palatalization, when the raising of \intxt{*a} no longer applied.

\begin{exe}
    \ex \chox{ʔi\mbox{-}qÁhlaˀm}{it is sharp}{ˀja\mbox{-}káhlaˀm}{i\mbox{-}kʲáhlam}{ʔi\mbox{-}kʲáhlaˀm}{ʔi\mbox{-}kʲáhlaˀm}
    \ex \cho{ʔi-qá-nt’ek}{my father-in-law}{ˀja-ká-nt’ek~\recind~ʔi-ká-nt’ek}{---}{ʔi-kʲá-nt’ek}
    \ex \chox{ʔi-qÁhlek}{my liver}{ʔi-káhlik \recind ja-káhlik}{i-kʲáhlek}{ʔi-kʲáhlek}{ʔi\mbox{-}kʲáhlek}
    \ex \chox{ʔi-qÁsan}{my calf}{ʔi-káxsaˀn \recind ja-káxsaˀn}{i-kʲáxsan}{ʔi-kʲáxsan}{ʔi-kʲáxsen}
    \ex \cho{ʔi-qélAh}{s/he encourages}{ʔi-kɛ́la}{---}{ʔi-kíla}
\end{exe}

\subsubsection{Stressed vowel lowering/laxing} \label{ch-vowel-lowering}

In Chorote, mid and high vowels have special lowered or diphthongized allophones, which occur in stressed syllables. The process is blocked following a [+high] segment: this includes palatalized allophones of consonants, underlying palatal consonants and, for back vowels, the labial consonants /w/, /hw/, /ˀw/.

The phenomenon is most clearly notable in Iyojwa'aja', where the open allophones of /i~u/ are \phonetic{e~o}, and thus overlap with the non-lowered allophones of /e~o/. Although no merger takes place -- since /e~o/ are lowered to \phonetic{ɛ~ɔ} in the same environments where /i~u/ are lowered to \phonetic{e~o} -- the vowels in question are not distinguished in the practical spelling.\footnote{This spelling is used, for example, in \cits{ND09} vocabulary, where the grapheme ‹e› stands for /i/~\phonetic{e}, /e/~\phonetic{ɛ}, and /e/~\phonetic{e}, whereas ‹o› stands for /u/~\phonetic{o}, /o/~\phonetic{ɔ}, and /o/~\phonetic{o}, though \citet[91]{ND09} does explicitly recognize that the language has ``a second \intxt{e}'' and ``a second \intxt{o}''. \citet{AG78,AG79} also confuses the lowered allophones of /i~u/ with /e~o/, though she acknowledges the existence of the allophone \phonetic{ow}, which she suspects to map to an independent phoneme.}

In Iyo’awujwa’, the open allophones of /i~u~e~o/ are, respectively, [ɪ~ʊ~ɛ~ɔ]. Note that \citet{AG83} does not employ the symbols in question in her study; instead, she variably represents [ɪ~ʊ] as ‹e~o› or as ‹i~u›, and consistently represents [ɛ~ɔ] as ‹e~o›. We retain her transcription when citing forms documented in \citet{AG83}, unless when explicitly stated otherwise, but it should be kept in mind that the characters \intxt{e} and \intxt{o} can each stand for two different sounds (and phonemes). In forms documented by Carol, on the other hand, we do use [ɪ, ʊ].

In Manjui, the lowered or lax allophones of /i~u~e~o/ are, respectively, [ei̯/ɪ], [ʊ/ou̯], [ɛ/ai̯], [ɔ]. Lowering is less frequent in /u/ in that variety (as in [ˈtuʍ] `eat!') and is not systematically reflected in our data. However, it does consistently occur after a glottal consonant: [saˈʔʊla] `anteater', [ˈhʊni] `bring it (here)'. In one of the subdialects of Manjui spoken in Santa Rosa (probably the Jlimnájnas subdialect), the realization [o] after \intxt{hw} was documented in /ahwú/ [ʔaˈhwóʔ] ‘woman’, which is quite unexpected, given that /hw/ behaves as [+high] in other Chorote varieties and does not trigger lowering of a following vowel.\footnote{In a couple of words, [u] alternates with [ʊ] or [o] after /hw/ in unstressed syllables: [ˈhlahwuʔ] alongside [ˈhlahwʊʔ] ‘strong wind’, [(ʔa)ˈjehwuʔ] alongside [ʔaˈjehwʊʔ] ‘jabiru’. This suggests that /hw/ is specified as [−high] in that subdialect, which could interestingly constitute a retention from Proto-Mataguayan, since \sound{PCh}{*hw} goes back to a fricative, \sound{PM}{*ɸ}. The unexpected behavior of /hw/ in the Jlimnájnas subdialect can hardly be attributed to language contact with a Mataguayan variety where Chorote /hw/ actually corresponds to a fricative, since Santa Rosa is located at the periphery of the Mataguayan-speaking area.} 

The monopthongized allophone of /i/ appears regularly in the Jlawá'a Wos subdialect in closed syllables, where the other subdialect shows a diphthong (as in \wordnl{ʔints'ɪ́k \recind ʔints'éik}{four}), but sometimes also in open syllables: \wordnl{lɪ́miʔ}{white}. The diphthongized realization of /e/ is frequent in the Jlimnájnas subdialect, also in open syllables, in contrast with a monophthongized realization in the other dialect, as in \wordnl{ʔáileʔ \recind ʔɛ́leʔ}{parrot}, \wordnl{ʔa\mbox{-}páin\mbox{-}a \recind ʔa\mbox{-}pɛ́n\mbox{-}a}{we cook it}. Our transcriptions do not usually reflect these diphthongued realizations of /e/. Preliminarly, the vowels in the Jlimnájnas subdialect seem more lax than those of the Jlawá'a Wos subdialect.

In the Jlimnájnas subdialect, \sound{PCh}{*ᵊCi} (where C is not a coronal) yields [iCi], whereas the other variety shows [iCei̯]: \wordnl{ʃi\mbox{-}hwíʃe \recind ʃi\mbox{-}hwéiʃe}{I am angry}, \wordnl{hi\mbox{-}p'ílisen \recind hi\mbox{-}p'éilisen}{you feel sorry for her/him}. By contrast, \sound{PCh}{*iCi} yields [iCi] in all subdialects of Manjui (\wordnl{ʔi\mbox{-}hwíʃe}{s/he is angry language}, \wordnl{ʔi\mbox{-}p'ílisen}{I feel sorry for her/him}), apparently not a retention but rather a combination of the first palatalization (\sectref{ch-pal1}) and depalatalization (\sectref{ch-depal}). The stressed vowel lowering must have postdated the former process and predated the latter.

\subsubsection{\sound{PCh}{*å} and \intxt{*a}}\label{ch-ao-a}

\sound{PCh}{*å} and \intxt{*a} were clearly distinct in Proto-Chorote, but no contemporary variety of Chorote preserves the opposition in question in all environments. After non-palatal(ized) consonants, both are reflected as \intxt{a} in all dialects (except when reduction in unstressed syllables applies, on which see \sectref{ch-unstr-red}).

After palatal(ized) consonants, however, the contrast between \sound{PCh}{*å} and \intxt{*a} is preserved in Iyojwa’aja’, where \sound{PCh}{*å} is reflected as \sound{Ijw}{a}, and \sound{PCh}{*a} is reflected as \sound{Ijw}{e}. Recall from \sectref{ch-v-r} that \sound{PCh}{*a} and \intxt{*e} after palatal and palatalized consonants are raised to [e] and [i], respectively, in all Chorote varieties. In Manjui and, somewhat less systematically, in Iyo’awujwa’, not only \sound{PCh}{*a}, but also \sound{PCh}{*å} is raised to [e] in that environment, whereas Iyojwa’aja’ reflects the vowel in question as [a]. That way, the underlying opposition between /a/ and /å/, posited by \citet[83]{JC14a} for Iyojwa’aja’, is non-existent in Manjui and virtually non-existent in Iyo’awujwa’.\footnote{\citet[83, fn. 12]{JC14a} states that [a] is exceedingly rare after palatal(ized) consonants in Iyo’awujwa’, but does occur, for example, in \wordnl{kʲaˈhwɪjh}{beneath}.}

\begin{exe}
    \ex \chox{\mbox{-}hwéˀjåʔ}{to fly}{\mbox{-}hwɛ́ˀjaʔ}{\mbox{-}fʷéjeʔ}{\mbox{-}hwɛ́ˀjeʔ}{\mbox{-}hwɛ́ˀjeʔ}
    \ex \cho{\mbox{-}kǻnis}{testicle}{\mbox{-}kʲánis}{---}{\mbox{-}kʲénis}
    \ex \cho{\mbox{-}kǻs}{tail}{\mbox{-}kʲás}{\mbox{-}kʲés}{\mbox{-}kʲés}
    \ex \cho{ʔi-hlǻˀm}{s/he defecates}{ʔi-hlʲáˀm}{---}{ʔi-hlʲéˀm}
    \ex \cho{\mbox{-}kéjås}{grandchildren}{\mbox{-}kíjas}{\mbox{-}kíjas \recind -kíjes}{\mbox{-}kíjes}
    \ex \cho{ʔi-kǻt}{it is red}{ʔi-sʲát}{ʔi-sʲát \recind ʔi-sʲét}{ʔi-ʃét}
    \ex \cho{j-ǻs}{my son}{j-ás}{j-és}{j-és}
    \ex \cho{j-ǻp}{s/he cries}{j-áp}{j-ép}{j-ép}
    \ex \chox{\mbox{-}k’alóʔ}{cheek}{\mbox{-}kʲ’óloʔ}{\mbox{-}kʲalóʔ}{\mbox{-}kʲ’alɔ́ʔ}{\mbox{-}ʔʲelɔ́ʔ}
\end{exe}

\subsubsection{\sound{PCh}{*ᵊ}}\label{ch-schwa-dial}

The emergence and the status of the intrusive vowel \intxt{*ᵊ} in Proto-Chorote is discussed in \sectref{pm-ch-ep-v}. In the contemporary varieties of Chorote, \intxt{*ᵊ} has mostly merged with \intxt{*i} as [i], but this latter merger took place independently in the varieties of Chorote: it fed the second palatalization, which occurred in Iyojwa’aja’ and, with some restrictions, in Manjui (\sectref{ch-pal2}), but not the first palatalization (\sectref{ch-pal1}). That way, \sound{PCh}{*ᵊ} differs from \sound{PCh}{*i} in not constituting the environment for the first palatalization. The default development of \sound{PCh}{*ᵊ} to \intxt{i} in all Chorote varieties is exemplified below.

\begin{exe}
    \ex \cho{hᵊ\mbox{-}nǻjin}{you go first}{hi-nʲáˀn}{---}{hi-nájin}
    \ex \cho{hᵊ-nǻʔ}{her/his father}{hi-nʲáʔ}{hi-náʔ}{hi-náʔ}
    \ex \cho{hᵊ-p'ot-és}{its lids}{hi-p'ɔ́t-is}{---}{hi-p’at-ɛ́s}
    \ex \cho{hᵊ-sínån}{you roast}{hi-sínʲaˀn}{hi-sénʲan}{hi-séinʲan}
    \ex \cho{hᵊ-túʍ}{you eat}{hi-tʲúʍ}{hi-tʊ́ʍ}{hi-tʲúʍ \recind hi\mbox{-}túʍ}
    \ex \cho{hwᵊkénah}{north wind, north}{wikína}{wikína}{hwikína}
    \ex \cho{pᵊhǻˀm}{I am tall}{pihjáˀm}{---}{---}
    \ex \chox{sᵊlǻhqajʔ \recind *sᵊlǻhqåjʔ}{wild cat}{silʲákaʔ}{siláhkaj}{siláhkajʔ}{ʃiláhkajʔ}
    \ex \cho{sᵊ-pǻsah}{I am quick}{si-pánsa}{si-páxsa \recind tsi-páxsa}{ʃi\mbox{-}páxsa}\label{v2a-tsipaxsa}
    \ex \chox{sᵊpúp}{Picui dove}{sipóp}{sipóp}{sipʊ́p}{ʃipʊ́p}
    \ex \cho{sᵊ-tójʔ}{I am tall}{si-tʲóˀjʔ}{ʃi-tɔ́jʔ}{ʃi-tʲójʔ}
    \ex \cho{sᵊwǻlåk}{spider}{siwálak \recind ʃiwálak}{siwálak \recind ʃiwálak}{ʃiwálak}
    \ex \cho{tᵊ-hwaˀjéjʔ}{s/he marries}{ti-hwáˀji}{---}{ti-hwaˀjíjʔ}
    \ex \cho{tᵊ-péj-kejʔ}{s/he hears}{ti-pɛ́-tʃiʔ}{---}{ti-pɛ́j-ʃi(j)ʔ}
    \ex \cho{tᵊ-ˀjákun}{s/he eats (intr.)}{ti-ˀjékʲuˀn}{---}{ti-ˀjékin}
    \ex \cho{wᵊkínah}{metal}{wikínʲe}{---}{---}
    \ex \cho{ʔᵊstǻhweʔ}{Chaco chachalaca}{ʔistʲáhwe}{istáfʷe}{\mbox{ʔistáhweʔ} \recind ʔiʃtáhweʔ}
    \ex \cho{ʔᵊstá\mbox{-}k}{cactus\species{Stetsonia coryne}}{ʔistʲé\mbox{-}k}{ʔistá\mbox{-}k}{ʔistá\mbox{-}k \recind ʔiʃtá\mbox{-}k}
    \ex \chox{ʔᵊsténiʔ / *ʔᵊsténi-k}{white quebracho}{ʔistíni-k}{isténi-k}{ʔistɛ́ni-k}{ʔistɛ́niʔ \recind ʔiʃtíniʔ}
    \ex \cho{ʔᵊstúuˀn}{king vulture}{---}{ʔistʊ́ˀn}{ʔistʲúuˀn \recind ʔiʃtʲúuˀn}
\end{exe}

Before a \intxt{ʔ}, including those resulting from debuccalization of an ejective dorsal consonant, \sound{PCh}{*ᵊ} typically assimilates to the following vowel, though in \REF{mj-saula} the reflex \intxt{a} is attested in Manjui.

\begin{exe}
    \ex \cho{hᵊ-sᵊʔún}{you love}{---}{hi-sʊʔʊ́n}{hi-sʊʔʊ́n}
    \ex \cho{sᵊʔúlah}{anteater}{soʔólʲe}{sʊʔʊ́la}{saʔʊ́la \recind saʔʊ́lʲeʔ}\label{mj-saula}
    \ex \chox{wósᵊk’at}{red-crested cardinal}{---}{wóxsijét}{wóxsiʔʲet}{wóxʃeʔet}
\end{exe}

In a handful of cases, \sound{PCh}{*tᵊ} yields \intxt{ta} instead of the expected \intxt{*ti} in Manjui and occasionally also in Iyo’awujwa’.

\begin{exe}
    \ex \cho{tᵊkénah}{precipice}{tikína\gloss{ravine}}{---}{takína}
    \ex \cho{tᵊkéhna-keʔ}{mountain}{tikíhna-kiʔ}{takíhna-kiʔ}{takíhnʲe\mbox{-}kiʔ}
    \ex \chox{tᵊlúk}{blind}{---}{talók}{talʊ́k}{---}
    \ex \cho{tᵊ-pó-eh}{it is full of}{ti-pɔ́-j-i}{ti-pɔ́-w-e}{ta-pɔ́-w-e}
\end{exe}

Finally, \sound{PCh}{*ᵊ} has distinct reflexes before uvular consonants. These are discussed in \sectref{ch-lowering}.

\subsubsection{Unstressed \sound{PCh}{*u} and \intxt{*o} after palatal and palatalized consonants} \label{ch-u-i}

In the unstressed position, \sound{PCh}{*u} and \intxt{*o} quite regularly yield \intxt{*i} after \sound{PCh}{*k(’)} > \intxt{*kʲ(’)} and \intxt{*j} in all contemporary varieties, with few exceptions, such as \REF{u-i-yekiun} in Iyojwa’aja’ and Iyo’awujwa’. \REF{u-i-kojaj} shows that this sound change was fed by the stress retraction in Iyojwa’aja’ (\sectref{ch-ijw-prosody}), suggesting that it occurred independently in different Chorote varieties.

\begin{exe}
    \ex \cho{tᵊ-ˀjákun}{s/he eats (intr.)}{ti-ˀjékʲuˀn}{\mbox{-}jékʲun}{ti-ˀjékin}\label{u-i-yekiun}
    \ex \cho{\mbox{-}koj-ájʰ}{hands}{\mbox{-}kʲój-e}{\mbox{-}kij-éj}{\mbox{-}kij-éjh}\label{u-i-kojaj}
    \ex \chox{kulájʔ}{sun}{kilʲéʔ \recind kiliʔé}{kiláj}{kilájʔ}{kilájʔ}
    \ex \chox{kʼutáˀn}{thorn}{kʼitʲéˀn}{ʔitán}{ʔitáˀn}{ʔitáˀn}
    \ex \chox{k’uwáhlah}{puma}{k’iwáhla}{iwáhla}{ʔiwáhla}{ʔiwáhla}
    \ex \cho{túkus}{ant}{tókis}{tókis}{tʊ́kis}
    \ex \cho{kéhla-juk}{red quebracho}{kíhla-jik}{kíhla-jik}{kíhlʲe\mbox{-}ek \recind kíhlʲa\mbox{-}jik \recind kíhli\mbox{-}jik}
\end{exe}

Unstressed \sound{PCh}{*u} may also sometimes change to \intxt{i} in the modern varieties after other consonants, but details are thus far unclear, and we consider this a sporadic change.

\begin{exe}
    \ex \chox{\mbox{-}hwétus}{root}{\mbox{-}hwɛ́tis}{fʷétis}{hwɛ́tis}{\mbox{-}hwɛ́tus}
    \ex \chox{p’ilusáh}{poor}{p’ilʲúxsʲe \recind p’élisʲe}{\mbox{-}pelíxsa}{\mbox{-}p’ilíxsa}{p’ilisáh}
\end{exe}

\subsubsection{Vowel lowering before \intxt{*q(’)}}\label{ch-lowering}

Chorote has a number of alternations that consist of vowel lowering before the consonant \intxt{*q(’)} (reflected as \intxt{k(’)} in the contemporary varieties). For example, the homophonous first-person singular inactive and first-person inclusive possessive prefixes (\wordng{PCh}{*sᵊ\mbox{-}}) usually surface as \intxt{ʃi\mbox{-}} before consonants in Manjui, but as \intxt{si\mbox{-}} (or, more rarely, \intxt{se\mbox{-}}) before /k(’)/. In Iyojwa’aja’, the cognate prefix has the allomorphs \intxt{si\mbox{-}} and \intxt{sa\mbox{-}} in the same respective contexts.

\newpage
\ea\label{mj-zi-se}
Manjui \citep{JC18}
    \begin{xlist}
        \ex \gll ʃi-táhwel-e\\
            1.\textsc{inact}-know-\APPL\\
            \glt `I know her/him'
        \ex \gll ʃi-ˀwɛ́t\\
            1+2.\textsc{poss}-place\\
            \glt `our (incl.) place'
        \ex \gll si-káaʔ\\
            1.\textsc{inact}-choke\\
            \glt `I choke'
        \ex \gll si-káˀmat\\
            1+2.\textsc{poss}-meat\\
            \glt `our (incl.) meat'
    \end{xlist}
\z

In Iyojwa’aja’, the first-person possessive prefix (\wordng{PCh}{*ʔi\mbox{-}}) and the third-person I-class verbal prefix (\wordng{PCh}{*ʔi\mbox{-}}) are usually reflected as \intxt{ʔi\mbox{-}} before consonants but as \intxt{ja\mbox{-} \recind ʔi\mbox{-}} before /k(’)/ \REF{ijw-i-ya}, whereas the third-person T-class verbal prefix (\wordng{PCh}{*tᵊ\mbox{-}}) is normally reflected as \intxt{ti\mbox{-}} before consonants but as \intxt{ta\mbox{-}} before /k(’)/ \REF{ijw-ti-ta}.

\ea\label{ijw-i-ya}
Iyojwa'aja' \citep{JC14a}
    \begin{xlist}
        \ex \gll ʔi-pʲáˀn\\
            3.\textsc{i.rls}-swim\\
            \glt `s/he swims'
        \ex \gll ʔi-hnʲétisʲeˀn\\
            3.\textsc{i.rls}-sneeze\\
            \glt `it makes her/him sneeze'
        \ex \gll ja-k’ɔ́hokoʔ\\
            3.\textsc{i.rls}-cough\\
            \glt `it makes her/him cough'
        \ex \gll ja-kɔ́hnʲeˀn\\
            3.\textsc{i.rls}-feed\\
            \glt `s/he feeds'
        \ex \gll ʔi-pʲúxsiʔ~(*ja-póxsiʔ)\\
            1\textsc{sg.poss}-beard\\
            \glt `my beard'
        \ex \gll ja-ká-nt’ek~\recind~ʔi-ká-nt’ek\\
            1{\textsc{sg.poss-alz}}-grandfather\\
            \glt `my father-in-law'
    \end{xlist}
\z

\ea\label{ijw-ti-ta}
Iyojwa'aja' \citep{JC14a}
    \begin{xlist}
        \ex \gll ti-lʲákiˀn\\
            3.\textsc{t.rls}-dance\\
            \glt `s/he dances'
        \ex \gll ti-més\\
            3.\textsc{t.rls}-be\_two\\
            \glt `they are two'
        \ex \gll ti-póxsiʔ\\
            3.\textsc{t.rls}-have\_beard\\
            \glt `he has a beard'
        \ex \gll ta-káxsit\\
            3.\textsc{t.rls}-stand\\
            \glt `s/he stands'
        \ex \gll ta-kɛ́lisʲeˀn\\
            3.\textsc{t.rls}-sing\\
            \glt `s/he sings'
        \ex \gll ta-k’ɔ́hokoʔ\\
            3.\textsc{t.rls}-cough\\
            \glt `s/he coughs'
        \ex \gll ta-kɔ́hnʲeˀn\\
            3.\textsc{t.rls}-feed\\
            \glt `s/he feeds someone'
    \end{xlist}
\z

At least in the case of the prefixes of the shape \wordng{PCh}{*ʔi\mbox{-}} in Iyojwa’aja’, one may suspect the influence of the neighboring dialects of Wichí, such as ’Weenhayek, which show an identical phenomenon (\sectref{wi-lowering}).

\subsubsection{Pretonic \sound{PCh}{*å}, \intxt{*o}}\label{ch-pret-ao-o}

Pretonic \sound{PCh}{*å}  yielded \intxt{i} in the contemporary varieties, late enough to counterfeed the second palatalization. It seems that this process is still underway: note that both variants have been synchronically attested in \word{Iyojwa'aja'}{pisáh \recind pitsáh \recind pasáh}{jabiru} \citep[143--144]{ND09}. The term \wordng{Ijw}{kiláji}, \word{Mj}{kilájiʔ \recind kilájuʔ}{non-indigenous person} is likely borrowed from some western Guaranian variety, from a form close to Ava Bolivian Guarani [kaˈɾai] \citep[76]{WD16}.

There are no clear examples of \sound{PM}{*o} in pretonic position, but \word{Ijw}{sihnát}{knife}, a possible early loanword from \sound{PW}{*tsonhat}, suggests that pretonic \intxt{*o} merged with \sound{PM}{*å} as \intxt{å}, since the Iyojwa’aja’ reflex of both vowels is an \intxt{i} that fails to palatalize a following coronal: \sound{PM}{påttséχ} > \cho{påtsáh}{jabiru}{pi(t)sáh \recind pasáh}{pisáh}{pisáh}; cf. also \sound{Ijw}{\mbox{-}<tɛ>́sahnat}\gloss{knife (relational)}. The Iyo’awujwa’ and Manjui term for woman, \intxt{ˀnikíʔ}, can be likely traced back to \word{PCh}{*ʔiˀno\mbox{-}kéʔ}, where a root meaning `man, person' is accompanied by a feminine suffix.

\subsubsection{Unstressed vowel reduction in Iyojwa'aja'}\label{ch-unstr-red}

In word-medial and word-initial unstressed syllables after a coronal or palatal(ized) sound, \sound{PCh}{*e}, \intxt{*a}, and \intxt{*å} are raised to \intxt{i} in Iyojwa’aja’, as in \word{Ijw}{táxsina}{toad} (compare \word{Mj}{táxsena}{id.}). After consonants that are not coronal or palatal(ized), the raising fails to occur, as in \word{Ijw}{pu\mbox{-}wáʔ}{those (unknown)} and \wordnl{ha\mbox{-}wáʔ}{those (absent)}, except that \sound{PCh}{*e} does get raised after non-coronals when it is preceded by a coronal \REF{ojwenni}. 

\booltrue{listing}
\ea
Iyojwa'aja'
    \begin{xlist}
        \ex \wordnl{ʔɛ́leʔ}{parrot} / \wordnl{ʔɛ́li-waʔ}{parrots}\\
        \ex \wordnl{sʲúnʲeʔ}{this} / \wordnl{sʲúni-waʔ}{these}\\
        \ex \wordnl{kʲaʔ \recind sʲu-kʲaʔ}{that (gone)} / \wordnl{ki-wáʔ \recind sʲú-ki-waʔ}{those (gone)}\\
        \ex \wordnl{ʔahwɛ́na}{bird} / \wordnl{ʔahwɛ́hni-kiʔ}{little bird}\\
        \ex \wordnl{t-’óhweˀn}{s/he wakes up} / \wordnl{t-’óhwin-ˀni}{s/he wakes up repeatedly}\label{ojwenni}
    \end{xlist}
\z
\boolfalse{listing}

Raising of \sound{PCh}{*e} to \intxt{i} may also occur in final syllables in Iyojwa’aja’ (and sometimes in Iyo’awujwa’) before, at least, \intxt{s} and \intxt{n}.

\begin{exe}
    \ex \chox{ʔahnát-es \recind *ʔåhnát-es}{lands}{ʔahnát-is}{ahnát-is}{ʔahnát-is}{ʔahnát-es}
    \ex \cho{\mbox{-}kǻhnat-es}{fishhooks}{\mbox{-}káhnat-is}{káhnat-es}{---}
    \ex \cho{\mbox{-}lǻkʲen}{to dance}{\mbox{-}lákiˀn}{\mbox{-}lákʲen}{\mbox{-}lákʲen}
\end{exe}

\subsubsection{Pretonic lowering in Manjui}\label{ch-pret-v-l-mj}

Pretonic vowels are sometimes lowered to \intxt{a} in Manjui (and, less frequently, also in Iyo’awujwa’).

\begin{exe}
    \ex \cho{ʔis-ís}{they are good}{ʔis-ís}{---}{ʔas-éis}
    \ex \chox{kates-él}{stars}{katɛ́s\mbox{-}eˀl}{kates-éj}{kates\mbox{-}ɛ́jh}{katas-ɛ́jh}
    \ex \chox{(hᵊ-)p'ot-és}{(its) lids}{hi-p'ɔ́t-is}{\mbox{-}pót-es}{\mbox{-}p’ɔ́t-es}{(hi-)p’at-ɛ́s}
    \ex \chox{ʔiˀnǻt}{water}{ʔiˀnʲát}{ʔanát}{ʔaˀnát}{ʔaˀnát}
    \ex \cho{ʔijéstah}{dew}{jísta}{\mbox{-}jísta \recind -jíste}{ʔijísta \recind ʔajísta}
\end{exe}

\subsubsection{Simplification of ``double'' vowels}

Proto-Chorote had heterosyllabic sequences of identical vowels that exceptionally were not separated by a glottal stop. These are retained in Manjui but simplified in Iyojwa’aja’ and Iyo’awujwa’.

\begin{exe}
    \ex \chox{\mbox{-}ˀjáan}{to watch}{\mbox{-}ˀjéˀn}{\mbox{-}jén\mbox{-}}{\mbox{-}ˀjén\mbox{-}}{\mbox{-}ˀjéen}
    \ex \chox{\mbox{-}áajʔ}{mouth}{---}{\mbox{-}áj}{\mbox{-}ájʔ}{\mbox{-}áajʔ}
    \ex \cho{\mbox{-}hǻåkeʔ}{ditch}{\mbox{-}hákiʔ}{\mbox{-}hákiʔ}{\mbox{-}háakiʔ}
    \ex \chox{\mbox{-}k’óoteʔ}{ear}{\mbox{-}kʲ’óteʔ}{\mbox{-}kʲóteʔ}{\mbox{-}kʲ’óteʔ}{\mbox{-}ʔʲóoteʔ}
    \ex \cho{ʔᵊstúuˀn}{king vulture}{---}{ʔistʊ́ˀn}{ʔiʃtʲúuˀn}
    \ex \chox{hl-úut}{scales}{hl-ót\gloss{placenta}}{hl-ót-is}{hl-ʊ́t-is}{hl-ʊ́ʊt}
\end{exe}

The Mataguayan background of such sequences is poorly understood at present. We assume that in some cases they result from loss of an intervocalic \intxt{*h}, yet in other cases they arose due to simplification of certain consonant clusters, as in \intxt{*stwV} > \intxt{*ʔᵊstVV}, \intxt{*qk} > \intxt{*Vk}. They are not in any way related to the long vowels of ’Weenhayek. 

\subsubsection{Other vowel changes}

This section describes other minor or subregular vowel changes in the Chorote varieties.

The alternation \intxt{a \recind o} includes environments other than those discussed in \sectref{ch-pret-v-l-mj}. Note that the variation in \REF{oa-tewok} has a parallel in Nivaĉle, where both \intxt{toβåk} and \wordnl{toβok}{river} are attested. The alternation in \REF{oa-punch} and \REF{oa-colaptes} could reflect the sound change \sound{PM}{*o} > \intxt{å} that might have been blocked in some varieties before a labiovelar, but in the absence of reliable cognates the directionality of the change cannot be ascertained.

\begin{exe}
 \ex \cho{mǻ(h)}{go!}{má(h)}{---}{mɔ́h}
 \ex \chox{téwok \recvar *téwåk}{river}{tɛ́wuk}{téwak}{tɛ́wak}{tɛ́wak}\label{oa-tewok}
 \ex \cho{ʔi-t'owás \recvar *ʔi-t'awás}{to punch}{ʔi-tʲ'ówas}{\mbox{-}t'awás}{ʔi-tʲ'owás \recind -t'awás}\label{oa-punch}
    \ex \cho{ts'ahwáʔ \recvar *ts'ohwáʔ}{woodpecker\species{Colaptes sp.}}{ts'ahwáʔa}{---}{ts'ahwáʔ \recind ts'ohwáʔ}\label{oa-colaptes}
 \end{exe}

Variation of this type is also attested in Manjui words that do not reconstruct to Proto-Chorote, such as \word{Mj}{[j]áwaset \recind [j]áwoset}{to address directly}. Curiously enough, the subdialectal variation in Manjui may also affect stressed vowels, as in \wordnl{Wónta \recind Wánta}{Santa Rosa}.

The sequence \intxt{*ji} after a stressed low vowel is deleted in Iyojwa’aja’.

\begin{exe}
    \ex \cho{\mbox{-}nǻjin}{to go first}{\mbox{-}náˀn}{\mbox{-}nájin}{\mbox{-}nájin}
\end{exe}

\sound{PCh}{*u} was lowered to /o/ ([o], [ɔ]) in Iyojwa’aja’ before \sound{PCh}{*q’}. Only one example is known.

\begin{exe}
    \ex \cho{\mbox{-}túk’ah}{to cook in ashes}{\mbox{-}tɔ́k’a}{---}{\mbox{-}tʊ́ʔʊ}
\end{exe}

\sound{PCh}{*e} has apparently yielded \intxt{o} in Iyojwa’aja’ after \sound{PCh}{*kw} > \sound{Ijw}{kʲ}, though only one example is known.

\begin{exe}
    \ex \cho{j-ókwes}{to frighten away}{j-ókʲos}{---}{j-ókes}\label{ch-e-o-yokwes}
\end{exe}

\subsection{Word-level prosody}\label{ch-ijw-prosody}

Iyo'awujwa' and Manjui quite faithfully retain the position of the stress reconstructed for Proto-Chorote. By contrast, Iyojwa'aja' innovated in that it no longer allows postpeninitial stress, licit in Proto-Chorote (and Proto-Mataguayan), and systematically retracts the stress to the peninitial syllable, as can be seen in the following examples.

\begin{exe}
\ex \chox{kates-él}{stars}{katɛ́s\mbox{-}eˀl}{kates-éj}{kates\mbox{-}ɛ́jh}{katas-ɛ́jh}
\ex \chox{\mbox{-}qatóʔ / -qató-keʔ}{elbow}{\mbox{-}káto-kiʔ}{\mbox{-}katóʔ / -kató-kiʔ}{\mbox{-}katɔ́ʔ / -katɔ́-kiʔ}{\mbox{-}katɔ́ʔ}
\ex \cho{\mbox{-}kiláʔ}{elder brother}{\mbox{-}kílʲa}{\mbox{-}kilʲéʔ}{\mbox{-}kilʲéʔ}
\ex \cho{\mbox{-}koj-ájʰ}{hands}{\mbox{-}kʲój-e}{\mbox{-}kij-éj}{\mbox{-}kij-éjh}
\ex \chox{\mbox{-}ta-téʔ}{eye}{\mbox{-}tá-teʔ}{\mbox{-}ta-téʔ}{\mbox{-}ta-tɛ́ʔ}{\mbox{-}ta-tɛ́ʔ}
\ex \cho{ʔi-t'owás \recvar ʔi-t'awás}{to punch}{ʔi-tʲ'ówas}{\mbox{-}t'awás}{ʔi-tʲ'owás \recind -t'awás}
\ex \cho{ʔi-selǻn}{to prepare}{ʔi-lɛ́xsan-e}{ʔi-silʲén\mbox{-}}{ʔi-ʃilʲén}
\end{exe}

As a consequence of this accent retraction, all stems that take obligatory syllabic prefixes (this includes all stems that start with a supraglottal consonant) and receive stress on their second syllable in Iyo’awujwa’/Manjui correspond to stems with initial stress in Iyojwa’aja’ \citep[91, fn. 22]{JC14b}. By contrast, stems that take non-syllabic prefixes -- such as \wordnl{\mbox{-}ʔahán}{to know} or \wordnl{\mbox{-}ʔahwɛ́lh}{to be ashamed} -- retain the original accent in Iyojwa’aja’, because the accretion of a prefix to the stem does not result in an illicit postpeninitial stress: \wordnl{ts\mbox{-}'ahán\mbox{-}e}{I know}, \wordnl{ts\mbox{-}'ahwɛ́lh}{I am ashamed}. Non-initial stress is likewise allowed in non-prefixed stems: \wordnl{ʔahwɛ́na}{bird}, \wordnl{ʔaˀláʔ}{tree}, etc.
\fussy
