\chapter{Nivaĉle} \label{ni}

This chapter deals with the historical phonology of Nivaĉle [niva1238] (\sectref{intro-ni}). \sectref{pm-to-ni} discusses the development of PM consonants, vowels, and prosody from the PM stage to Nivaĉle. \sectref{ni-dialects} is concerned with the diversification of the Nivaĉle dialects.

In what follows, we rely on \cits{JS16} dictionary, on \cits{AnG15} phonological description, and on \cits{NS87}, \cits{AF16}, and \cits{LC20} grammatical descriptions.

The consonantal inventory we assume for Nivaĉle is given in \tabref{ni-inv-cons}. We follow \cits{AnG15} analysis of the preglottalized codas as complex codas, and do not posit a set of preglottalized stops and fricatives; therefore, \word{Nivaĉle}{k͡loˀp}{winter} is analyzed as /k͡loʔp/. The inclusion of preglottalized segments – \intxt{ˀk͡l}, \intxt{ˀβ}, \intxt{ˀj}, \intxt{ˀm}, \intxt{ˀn} – is our addition, broadly inspired by \cits{AnG21} work. The vocalic inventory we assume for Nivaĉle includes six vowels, /i~e~a~å~o~u/.

\begin{table}
\footnotesize
\caption{Nivaĉle consonants}
\label{ni-inv-cons}
 \begin{tabularx}{\textwidth}{Yc@{~}ccc@{}c@{}c}
  \lsptoprule
            & labial & dental & alveolar & postalveolar & velar & glottal\\\midrule
  plain stops & p & t & ts & tʃ & k [k~\recind q] & ʔ\\
  \tablevspace
  ejective stops & p’ & t’ & ts’ & tʃ’ & k’ [k’~\recind q’] & \\
  \tablevspace
  laterally released stop & & & & & k͡l [k͡l~\recind q͡l] & \\
  \tablevspace
  preglottalized \mbox{laterally released stop} & & & & & ˀk͡l [ʔk͡l~\recind ʔq͡l] & \\
  \tablevspace
  plain fricatives & ɸ [ɸ~\recind f] & ɬ & s & ʃ & x [x~\recind χ \recind h] & \\
  \tablevspace
  plain approximants & β [β~\recind w] & & & j & &\\
  \tablevspace
  preglottalized approximants & ˀβ [ʔβ~\recind ʔw] & & & ˀj [ʔj] & &\\
  \tablevspace
  plain nasals & m & n & & & &\\
  \tablevspace
  preglottalized nasals & ˀm [ʔm] & ˀn [ʔn] & & & &\\
  \lspbottomrule
 \end{tabularx}
\end{table}

\section{From Proto-Mataguayan to Nivaĉle}\label{pm-to-ni}

This section describes the evolution of PM consonants (\sectref{ni-consonants}), vowels (\sectref{ni-vowels}), and prosody (\sectref{ni-prosody}) in Nivaĉle.

\subsection{Consonants}\label{ni-consonants}

The consonant system of Nivaĉle has undergone relatively little change since the Proto-Mataguayan stage. We start by discussing the phonetic (or even notational) change \sound{PM}{*w}~>~\sound{Ni}{β} (\sectref{ni-v}). Then we proceed to the major innovations that affected \sound{PM}{*l}, which changed to \intxt{k͡l} (\sectref{ni-kl}), as well as the consonants \intxt{*k(’)}, \intxt{*q(’)}, \intxt{*x}, \intxt{*χ}, \intxt{*h}, which are reflected as \sound{Ni}{k(’)}, \intxt{tʃ(’)}, \intxt{x}, \intxt{ʃ}, or \intxt{∅} depending on the environment (\sectref{ni-vel-uv}). We also describe two sound changes restricted to the coda position -- \intxt{*k͡l}~>~\sound{Ni}{k} (\sectref{ni-kl-k}) and \intxt{*ts}~>~\sound{Ni}{s} (\sectref{ni-ts-s}) -- and a number of changes involving glottalized consonants and the glottal stop (\sectref{ni-glott-fric}--\sectref{ni-glottal-insertion}). \sectref{ni-clusters} deals with the development of PM~consonant clusters in Nivaĉle.

\subsubsection{\sound{PM}{*w}}\label{ni-v}

In this book, we employ the symbol \intxt{β} for the labial approximant of Nivaĉle. It is the regular reflex of \sound{PM}{*w} (see \sectref{proto-w} for concrete examples). Note that even synchronically some authors still analyze the Nivaĉle phoneme in question as /w/, though all agree that its possible realizations include a bilabial approximant in addition to a labiovelar one. In this regard, \citet[4]{AnG16} states that in the Shichaam Lhavos variety ``[β] and [ʋ] appear to have replaced the use of /w/. However, the latter can still be found preceding back vowels''. \citet[44–45]{LC20} analyze the phoneme as question as /w/ and claim that it ``varies in pronunciation between [w] and [β]. In most cases, [β] is possible but one of these allophones is favored over the other in certain environments. It is typically pronounced as [β] before \intxt{i}, \intxt{e}, or \intxt{a}. This [β] is not a strong bilabial fricative, rather it is a bilabial approximant with very weak friction. It has the allophone [w] before \intxt{u}, \intxt{o}, and \intxt{ô} \anjc{our \intxt{å}}, sometimes alternating freely with [β] before these vowels''.

\subsubsection{\sound{PM}{*l}}\label{ni-kl}

\sound{PM}{*l} changed to \intxt{k͡l} in Nivaĉle. This cross-linguistically rare sound is described in great detail by \citet{AnG19-L}, who analyzes it as a complex segment. Its stop element is velar or uvular (IPA~[k] or [q]), whereas its release is a voiced velarized alveolar approximant (IPA~[ɫ]). The sound change from \sound{PM}{*l} to \sound{Ni}{k͡l} is argued by \citet[64–70]{AnG19-L} to have been a perception-driven one, whereby stop bursts were reinterpreted as emergent laterally released stops. In the coda position, \intxt{k͡l} further delateralized to \intxt{k}, as discussed in \sectref{ni-kl-k}. The following examples illustrate this process (for a more representative list, see \sectref{proto-l}).

\begin{exe}
    \ex \tell
    \ex \pll
    \ex \killv
    \ex \pet
    \ex \offspring
    \ex \wash
    \ex \winter
    \ex \bow
    \ex \cicada
    \ex \anteater
    \ex \parrot
\end{exe}

It must be noted that since the sound change in question Nivaĉle has innovated a new \intxt{l}, found in borrowings, such as \wordnl{alus}{rice}, \wordnl{palaβaj}{Paraguay}, \wordnl{kaletax}{cart}, \wordnl{ele}{German, missionary} \citep[252]{AnG15},\footnote{The former three loans ultimately come from Spanish \intxt{arroz}, \intxt{Paraguay}, and \intxt{carreta}, with identical meanings; the stem-final \intxt{x} in \intxt{kaletax} could be attributed to popular etymology, given the existence of the suffix \wordnl{\mbox{-}tax}{similar to}. The origin of the latter loan (identified by \citealt[8]{LC20} as a Shichaam Lhavos dialectism) is disputed: \citet[60]{NS87} and \citet[124]{JS16} claim it comes from Maká (we have been unable to identify a suitable etymon), whereas other believe it is a borrowing from \word{Spanish}{inglés}{Englishman} \citep{MF97}.} and in onomatopoeic words, such as \wordnl{sile sile}{a flute from old times}, \wordnl{ukuˈluku}{barn owl} \citep[60]{NS87}.

\subsubsection{Guttural stops and fricatives}\label{ni-vel-uv}

The guttural stops (\sound{PM}{*k}, \intxt{*k’}, \intxt{*q}, \intxt{*q’}) and fricatives (\intxt{*x}, \intxt{*χ}, \intxt{*h}) yielded velar segments in Nivaĉle (\sound{Ni}{k}, \intxt{k’}, and \intxt{x}), with two important exceptions: the velar consonants of Proto-Mataguayan -- but not the uvular and glottal consonants -- palatalized to \sound{Ni}{tʃ}, \intxt{tʃ’}, and \intxt{ʃ} in certain environments, and the glottal fricative \intxt{*h} is deleted in the coda position.

We start by presenting the reflexes of \sound{PM}{*q}, \intxt{*q’}, and \intxt{*χ}, which never palatalize in Nivaĉle. \sound{PM}{*q} and \intxt{*q’} yield \sound{Ni}{k} and \intxt{*k’}, respectively, in all positions:

\sloppy
\begin{exe}
    \ex \food
    \ex \elbow
    \ex \welln
    \ex \snore
    \ex \inorderto
    \ex \alienable
    \ex \distrust
    \ex \leg
    \ex \dwarf
    \ex \starn
    \ex \costume
    \ex \soul
    \ex \wildcat
    \ex \tsaqaq
    \ex \paralytic
    \ex \wildhoney
    \ex \knee
\end{exe}

Similarly, \sound{PM}{*χ} yielded \sound{Ni}{*x} in all environments:

\begin{exe}
    \ex \fatv
    \ex \najendup
    \ex \centipede
    \ex \crab
    \ex \north
    \ex \suncho
    \ex \killbird
    \ex \finger
    \ex \ocelot
    \ex \takeaway
    \ex \redquebracho
    \ex \runv
    \ex \barnowl
    \ex \oldn
    \ex \manysg
    \ex \quick
    \ex \jabiru
    \ex \deep
    \ex \longv
    \ex \siyaj
    \ex \anteater
    \ex \thunder
    \ex \pseudo
    \ex \shoot
    \ex \far
    \ex \burnvi
    \ex \fullriver
    \ex \tsofatajf
    \ex \tired
    \ex \largefat
    \ex \paloflojof
    \ex \piranhamn
    \ex \blackalgarrobof
    \ex \rhea
    \ex \night
    \ex \tuscaf
    \ex \caracara
    \ex \jararaca
    \ex \snakeatuj
    \ex \peccary
    \ex \mistolf
    \ex \hurt
    \ex \argentineboa
    \ex \chaguara
    \ex \wildbean
    \ex \widower
    \ex \mollef
    \ex \meat
    \ex \firei
    \ex \bro
    \ex \puma
\end{exe}

\sound{PM}{*h} also yielded \sound{Ni}{x}, but only in onsets.

\begin{exe}
    \ex \locustmn
    \ex \centipedepl
    \ex \demh
    \ex \acti
    \ex \welln
    \ex \coati
    \ex \cactus
    \ex \pacu
    \ex \oldpl
    \ex \powderpl
    \ex \ropepl
    \ex \wildcat
    \ex \platepl
    \ex \paloflojot
    \ex \headpl
    \ex \caracarapl
    \ex \knee
    \ex \wildbean
    \ex \mollef
    \ex \teach
    \ex \queenpalmf
\end{exe}

Word-finally, by contrast, \sound{PM}{*h} was lost in Nivaĉle (note that \sound{PM}{*h} is not known to have occurred in codas word-medially). The deletion of \sound{PM}{*h} also applied to \sound{PM}{*jʰ} and \intxt{*lʰ} (underlying clusters */jh/, */lh/), as in \REF{ni-yh-plaj}, \REF{ni-yh-distal}, \REF{ni-yh-soul}, \REF{ni-yh-recipient}.

\begin{exe}
    \ex \plaj \label{ni-yh-plaj}
    \ex \distal \label{ni-yh-distal}
    \ex \companion
    \ex \monkparakeet
    \ex \neighbor
    \ex \spouse
    \ex \snail
    \ex \goimp
    \ex \dog
    \ex \tapeti
    \ex \soul \label{ni-yh-soul}
    \ex \hornero
    \ex \recipient \label{ni-yh-recipient}
    \ex \moon
    \ex \waspaniti
    \ex \doveula
    \ex \lessergrison
\end{exe}

The velar consonants of Proto-Mataguayan followed a more complex evolution pathway: they clearly underwent a conditioned split, yielding velars (\sound{Ni}{k}, \intxt{k’}, \intxt{x}) in some environments and post-alveolars (\sound{Ni}{tʃ}, \intxt{tʃ’}, \intxt{ʃ}) in others. The environment for palatalization can be broadly defined as ``next to a non-back vowel (\sound{PM}{*i}, \intxt{*e}, \intxt{*ä}, \intxt{*a} > \sound{Ni}{i}, \intxt{e}, \intxt{a}), possibly with an intervening coronal consonant''. However, the palatalization did not occur if a back vowel (\intxt{u}, \intxt{o}, \intxt{å}) directly follows the target consonant or precedes it (either directly or with an intervening [+grave]~$=$~non-coronal consonant).

The following examples illustrate the palatalization of \sound{PM}{*k} and \intxt{*k’} to \sound{Ni}{tʃ} and \intxt{tʃ’}, respectively. Note that in each case there is a non-back vowel adjacent to the target consonant, and no back vowels adjacent to it. 

\begin{exe}
    \ex \honeycomb
    \ex \mortar
    \ex \hidev
    \ex \north
    \ex \tooln
    \ex \takeaway
    \ex \monkparakeet
    \ex \redquebracho
    \ex \sendv
    \ex \neighbor
    \ex \elderbro
    \ex \eldersis
    \ex \thread
    \ex \mesh
    \ex \plate
    \ex \allrcpr
    \ex \headn
    \ex \wildhoney
    \ex \queenpalmf
\end{exe}

The following examples illustrate the palatalization of \sound{PM}{*x} to \sound{Ni}{ʃ}. Note that in all cases there is a non-back vowel adjacent to the target consonant, and no back vowels adjacent to it. Note that back vowels fail to block the palatalization of \intxt{*x} in \REF{ni-sh-night}, \REF{ni-sh-tuscaf}--\REF{ni-sh-tuscag}, and in the plural forms in \REF{ni-sh-bow} and \REF{ni-sh-pathn}, since a coronal consonant intervenes. In \REF{ni-sh-spring}, the coronal consonant \intxt{n} is transparent for the palatalization triggered by the front vowel \intxt{*a} (the alternative reflex \intxt{å} is a late dialectal innovation, on which see \sectref{ni-a-ao-labials}).


\begin{exe}
    \ex \foodmn
    \ex \mouth
    \ex \cutdown
    \ex \rightn
    \ex \ameiva
    \ex \fieldn
    \ex \youngerbro
    \ex \wash
    \ex \bow \label{ni-sh-bow}
    \ex \languageword
    \ex \nose
    \ex \smelln
    \ex \pathn \label{ni-sh-pathn}
    \ex \leafmn
    \ex \thunder
    \ex \dig
    \ex \burrow
    \ex \saymn
    \ex \price
    \ex \night \label{ni-sh-night}
    \ex \egg
    \ex \headn
    \ex \jelayuk
    \ex \dirt
    \ex \spring \label{ni-sh-spring}
    \ex \tuscaf \label{ni-sh-tuscaf}
    \ex \tuscat \label{ni-sh-tuscat}
    \ex \tuscag \label{ni-sh-tuscag}
\end{exe}

The following examples illustrate \sound{PM}{*k} and \intxt{*k’} that fail to palatalize in Nivaĉle. In almost all cases there is a back vowel either directly following or preceding the target consonant. In \REF{ni-k-fry}, irregular vowel metathesis (\intxt{*å…a} > \intxt{*a…å}) must have counterfed the palatalization. In \REF{ni-k-cactus}, the non-back vowel in Nivaĉle is likewise irregular, but in this case it is not clear whether the irregular vowel change counterfed the palatalization or whether the palatalization simply did not apply in the environment \#\_C\textsubscript{[+grave]}V\textsubscript{[-back]}. \REF{ni-k-armadillo} and \REF{ni-k-willow} are genuine exceptions; the latter may turn out to be a late loan from Maká.

\begin{exe}
    \ex \tobacco
    \ex \palm
    \ex \arrowfok
    \ex \hunger
    \ex \fry \label{ni-k-fry}
    \ex \testicle
    \ex \tail
    \ex \fall
    \ex \sunn
    \ex \answer
    \ex \runv
    \ex \grabwork
    \ex \eatkun
    \ex \heat
    \ex \meet
    \ex \sweat
    \ex \cactus \label{ni-k-cactus}
    \ex \leniosapl
    \ex \armadillonomaka \label{ni-k-armadillo}
    \ex \willow \label{ni-k-willow}
    \ex \yicalhuk
    \ex \powder
    \ex \feces
    \ex \zorzal
    \ex \wildmanioc
    \ex \rope
    \ex \fence
    \ex \distrust
    \ex \cat
    \ex \river
    \ex \face
    \ex \blind
    \ex \ant
    \ex \uncle
    \ex \duraznillo
    \ex \leniosasg
    \ex \badmood
    \ex \palosanto
    \ex \firewoodhuk
    \ex \paralytic
\end{exe}

In the following examples, \sound{PM}{*x} does not palatalize in Nivaĉle. In almost all cases there is a back vowel either directly following or preceding the target consonant. The irregular change \sound{PM}{*å}~>~\sound{Ni}{a} in \REF{ni-x-teach} must have counterfed the palatalization of velars.

\begin{exe}
    \ex \finger
    \ex \truev
    \ex \ocelot
    \ex \arrowkaxe
    \ex \armadillonomaka
    \ex \sleepiness
    \ex \carrysh
    \ex \palocruzmn
    \ex \eatvt
    \ex \aunt
    \ex \uncle
    \ex \tuscaf
    \ex \tuscat
    \ex \tuscag
    \ex \skin
    \ex \teach \label{ni-x-teach}
\end{exe}

In a number of morphemes, the alternation between velar and postalveolar consonants is still synchronically active in Nivaĉle. This can happen when a Proto-Mataguayan consonant is found in different environments in the consonantal and vocalic allomorphs of the same stem (cf. \sectref{c-v-stems}). In the following example, \sound{PM}{*x} palatalizes in the singular, because there is no adjacent non-back vowel, but fails to palatalize in the plural, because the metathesis (\sectref{metathesis}) creates context for palatalization blocking: the back vowel \intxt{å} is now separated from \intxt{x} by a [+grave] (non-coronal) consonant.\footnote{In the speech of one of the co-authors of \citet{LC20}, representative of the Pilcomayeño subdialect of Chishamnee Lhavos, this pattern is also found in stems where the intervening consonant is coronal: \wordnl{k͡lutseʃ\plf{k͡lutsxe\mbox{-}}}{bow}. This must be a local innovation, since the regular form \intxt{k͡lutsʃe\mbox{-}s} is abundantly attested in the Central Paraguayan subdialect of Chishamnee Lhavos \citep[10]{LC20}, as well as in all other sources on the language, including \citet{NS87,AF16,JS16}.}

\begin{exe}
    \ex \abdcavity
\end{exe}

Another instance of an alternation between velars and postalveolars is seen in the verb\gloss{to go away}, where the root vowel varies throughout the paradigm (see \sectref{irrv}). In addition to the forms reconstructible to Proto-Mataguayan, the relation between the choice of \intxt{k} and \intxt{tʃ} and the backness of the adjacent vowel is seen in forms such as \wordnl{ʃn\mbox{-}åk}{we.{\textsc{incl}} go away}, \wordnl{n\mbox{-}åk}{(that) s/he go away},
and \wordnl{niʔ j\mbox{-}itʃ}{I don't go away} \citep[146]{AF16}.

\begin{exe}
    \ex \goaway
\end{exe}

Finally, and most importantly, velars and postalveolars alternate at the left edge of some suffixes, whose allomorphs are chosen depending on the final segment(s) of the stem. Quite expectedly, in all cases the initial segment of the suffix is followed by a non-back vowel. Proto-Mataguayan suffixes that start with a velar consonant followed by a back vowel have non-alternating reflexes in Nivaĉle (as in \wordnl{\mbox{-}xop}{next to, surrounding}), because velar consonants never palatalize in Nivaĉle if there is an adjacent back vowel.

\begin{exe}
    \ex \grove
    \ex \feminine
    \ex \vrbpl
    \ex \recipient
\end{exe}

\citet[64]{AnG15} and \citet[54–55]{LC20} document the alternation in question for suffixes such as \wordnl{\mbox{-}xam / \mbox{-}ʃam}{on top of, up, through}; \wordnl{\mbox{-}xaʔne / \mbox{-}ʃaʔne}{downwards}; \wordnl{\mbox{-}kiʃam / \mbox{-}tʃiʃam}{upward}; \wordnl{\mbox{-}xi / \mbox{-}ʃi}{indefinite location, indefinite direction; intensive}; \wordnl{\mbox{-}xij / \mbox{-}ʃij}{concave container}; \wordnl{\mbox{-}k’e / \mbox{-}tʃ’e}{along; distributive}; \wordnl{\mbox{-}ke / \mbox{-}tʃe}{feminine}; \wordnl{\mbox{-}kat / \mbox{-}tʃat}{group of plants}. In all these suffixes, the initial consonant (followed by a non-back vowel) surfaces as postalveolar if the preceding vowel is front, even if a consonant intervenes \REF{ex:jamsham:1:niv}, but as velar if the preceding vowel is back, if a [+grave] (non-coronal) consonant intervenes \REF{ex:jamsham:2:niv}.

\newpage
\ea\label{ex:jamsham:1:niv}
Nivaĉle \citep[64]{AnG15}\\
    \begin{xlist}
        \ex\gll ɬ-néˀt-ʃam \\
                2{\SG}-get-{\LOC}:up\\
                \glt `you get up'
        \ex\gll jitáʔ-ʃam \\
                scrub-{\LOC}:up\\
                \glt `very thick scrubland'
    \end{xlist}
\z

\ea\label{ex:jamsham:2:niv}
\gll xa-xúˀx-xam ɬa=tʼún \\
     1{\SG}-bite-{\LOC}:up {\textsc{f}}.{\DET}=cracker \\
    \glt `I bit the cracker'
\z

If a coronal consonant separates the suffix from a back vowel, the initial consonant of the suffix does palatalize, as in \REF{ex:jamsham:3:niv}.\footnote{Although \citet[54]{LC20} fail to note the key role of the [±grave] feature of the intervening consonants for the blocking of palatalization in Nivaĉle, they still give some interesting examples that shed light on the behavior of consonants such as \intxt{s}: compare \wordnl{tox=k’e}{far} and its plural \intxt{to\mbox{-}s=tʃ’e}, \wordnl{ʔakåx=xi=ʔin}{rich} and its plural \intxt{ʔakå\mbox{-}s=ʃi=ʔin}. The possible role of coronal consonants as an intervening factor had already been mentioned by \citet[66]{AnG15}.}

\ea\label{ex:jamsham:3:niv}
Nivaĉle \citep[66]{AnG15}\\
    \begin{xlist}
        \ex\gll -tǻˀɬ-ʃam \\
                come.from-{\LOC}:up \\
                \glt `to come from'
        \ex\gll ji-kxúˀs-ʃam \\
                1{\SG}-knee-{\LOC}:up \\
                \glt `on my knee'
    \end{xlist}
\z

Note that the instances of \intxt{k} derived from erstwhile \intxt{*k͡l}~<~\intxt{*l} (\sectref{ni-kl-k}) behave as coronal in what concerns palatalization blocking in Nivaĉle. This can be seen in words such as \wordnl{ɬa\intxt{-}ntåkʃitʃ\mbox{-}’a}{grandson (male ego)}, where the postalveolar fricative \intxt{ʃ} occurs despite being separated from a back vowel \intxt{å} by a prima facie [+grave] consonant \intxt{k}; the fact that this \intxt{k} goes back to \intxt{*k͡l} is clear from related forms such as \word{Ni}{ɬa\intxt{-}ntåk͡leʃtʃeʔe}{granddaughter (male ego)} (data from \citnp{LC20}: 89). The same explanation may account for \wordnl{t'ak͡låk\mbox{-}tʃat}{scrub}, derived from \wordnl{t'ak͡låk}{weed} and explicitly stated to be an exception by \citet[211]{NS87}.

Genuine exceptions from the palatalization rule are very rare in Nivaĉle. \citet[66–67]{AnG15} mentions the form \wordnl{tsanku\mbox{-}kat}{stand of duraznillo trees}; according to \citet[211]{NS87}, this form is typical of the Chishamnee Lhavos variety and may thus represent a late dialectal development. \citet[210]{NS87} also gives an unexplainable form \wordnl{ʃek͡lå\mbox{-}tʃat}{group of trees\species{Prosopis sp.}}.

\subsubsection{Delateralization of \sound{PM}{*l} > \intxt{*k͡l} > \intxt{k} in codas}\label{ni-kl-k}

The consonant \intxt{k͡l} cannot occur in Nivaĉle codas (except dialectally when followed by a glottal stop, see \sectref{ni-kl-glottal}). Instead, a productive rule delateralizes it to \intxt{k} in that position.

\ea\label{ex:delateralization:niv}
Nivaĉle \citep[225–226]{AnG15}\\
    \begin{xlist}
        \ex\gll ɬa-xpek͡l-is\\
                3-shadow-\PL\\
                \glt `her/his shades'
        \ex\gll ɬa-xpek\\
                3-shadow\\
                \glt `her/his shade'
        \ex\gll ∅-βak͡leˀtʃ\\
                3-walk\\
                \glt `her/his shades'
        \ex\gll ∅-βaktʃe-mat\\
                3-walk-defect\\
                \glt `s/he limps'
    \end{xlist}
\z

\citet[8–9]{LC-VG-07} ascribe this alternation to a positionally conditioned diachronic sound change \intxt{*k͡l}~>~\intxt{k} that must have occurred in the history of Nivaĉle. Comparative data show that this is indeed the case: \sound{PM}{*l} indeed evolved into \intxt{k} in the coda position in Nivaĉle, as also noted in \citet[253]{AnG15}.

\begin{exe}
    \ex \returnth
    \ex \pll
    \ex \tell
    \ex \returnh
    \ex \shadow
    \ex \orphanmn
    \ex \chaguara
\end{exe}

\subsubsection{Deaffrication of \sound{PM}{*ts} > \intxt{s} in codas}\label{ni-ts-s}

As discussed in \sectref{proto-ts}, the occurrence of \intxt{ts} is banned from codas in Nivaĉle, except when the onset of the next syllable is \intxt{x} or \intxt{ɸ} (see footnote \ref{tsx-heterosyllabic}). This restriction arose as a result of a diachronic deaffrication of \sound{PM}{*ts}~>~\intxt{s} in codas, shared with Wichí (\sectref{wi-ts-s}) and possibly Chorote (\sectref{ch-ts}).

\begin{exe}
    \ex \rootn
    \ex \dew
    \ex \offspring
    \ex \trunk
    \ex \starn
    \ex \knee
\end{exe}

In some etyma, the erstwhile presence of an affricate in certain forms is suggested by the synchronically active alternations in Nivaĉle: compare \word{Ni}{\mbox{-}fetats\mbox{-}ij}{roots}, \wordnl{\mbox{-}(ʔa)kxatsu\mbox{-}j}{knees}, \wordnl{\mbox{-}tats\mbox{-}uk}{trunk} vs. \wordnl{\mbox{-}fetas}{root}, \wordnl{\mbox{-}(ʔa)kxuˀs}{knee}, \wordnl{\mbox{-}tas\mbox{-}ku\mbox{-}j}{trunks} \citep[45]{AnG15}. \citet[50]{LC20} note that this alternation is restricted to nouns in Nivaĉle, whereas in verbs \intxt{ts} alternates with \intxt{t} instead: compare \word{Ni}{xa\mbox{-}nuts\mbox{-}xa\mbox{-}jan}{I cause him/her to be angry}, \wordnl{kuts\mbox{-}xanax}{thief, robber}, \wordnl{xa\mbox{-}taβkits\mbox{-}xat}{I make him/her/it dizzy} vs. \wordnl{xa\mbox{-}nut}{I get angry}, \wordnl{ɬa\mbox{-}t\mbox{-}kut}{you steal}, \wordnl{tsi\mbox{-}taβkit}{I am dizzy, I get dizzy} \citep[50]{LC20}. The diachronic origins of the latter alternation are unknown because the relevant roots do not reconstruct back to Proto-Mataguayan.

\subsubsection{\sound{PM}{*ɸ’}, \intxt{*ɬ’} > \sound{Ni}{p’}, \intxt{t’}}\label{ni-glott-fric}

Nivaĉle also participated in yet another sound change shared with Chorote and Wichí, but not with Maká, which consists of the fortition of the Proto-Mataguayan glottalized fricatives (phonologically possibly analyzable as tautosyllabic sequences of a fricative and a glottal stop) to glottalized stops: \sound{PM}{*ɸ’}, \intxt{*ɬ’}~>~\sound{Ni}{p’}, \intxt{t’}. The sequence \intxt{*kɸ’}, however, changed to \sound{Ni}{k’}, as in \REF{ni-kf'-torn}.

\begin{exe}
    \ex \foot
    \ex \arrowfok
    \ex \tornkf \label{ni-kf'-torn}
    \ex \femalebreastits
    \ex \skinits
    \ex \meatits
    \ex \juiceits
    \ex \urinateyou
    \ex \urineits
\end{exe}

As a result of the sound change \sound{PM}{*ɬ’}~>~\intxt{(*)t’}, Nivaĉle now displays a morphophonological rule which converts the underlying sequence /ɬ+ʔ/ into \intxt{t’} (rather than \intxt{ɬ’}, as in Maká). The rule is no longer entirely productive in Nivaĉle, since the sequence /ɬʔ/ may occur within a morpheme, as in \wordnl{ʃniɬʔå}{lizard\species{Teius teyou}}.

\subsubsection{Deglottalization of sonorants}\label{ni-deglottalization-sonorants}

Although the glottalized sonorants of Proto-Mataguayan (\intxt{*ˀw}, \intxt{*ˀl}, \intxt{*ˀj}, \intxt{*ˀm}, \intxt{*ˀn}) are normally preserved in Nivaĉle as sequences of the type ``\intxt{ʔ} + sonorant'' (\intxt{ˀC} in our notation), the glottalization fails to surface in some environments. Most notably, glottalized sonorants are deglottalized in word-initial position in Nivaĉle, merging with their plain counterparts. Note that in \REF{ni-'n-pathn}, \REF{ni-'w-walk}--\REF{ni-'w-blood}, and \REF{ni-'w-climb} the glottalization does surface after prefixes, even if not all of our sources on the language document it consistently: \wordnl{ʔa\mbox{-}ˀnåjiʃ}{your way} \citep[318]{AF16}, \wordnl{ɬa\mbox{-}ˀβak͡letʃ}{you walk} \citep[312]{JS16}, \wordnl{ja\mbox{-}ˀβéˀɬa}{I am alone} \citep[312]{JS16}, \wordnl{ji\mbox{-}ˀβoj\mbox{-}ej}{my blood.\PL} \citep[189]{AF16}, \wordnl{ʃta\mbox{-}ˀβåˀɬ}{we.{\textsc{incl}} climb, rise} \citep[234]{LC20}. The root in \REF{ni-'w-healthy}, by contrast, is attested with a plain \intxt{β} even after prefixes in all available sources \citep{NS87,AF16,JS16,LC20}.

\begin{exe}
    \ex \nightmonkey
    \ex \zorzal
    \ex \demnn
    \ex \dayworld
    \ex \pathn \label{ni-'n-pathn}
    \ex \healthy \label{ni-'w-healthy}
    \ex \rhea
    \ex \walk \label{ni-'w-walk}
    \ex \onemn \label{ni-'w-one}
    \ex \blood \label{ni-'w-blood}
    \ex \butterfly
    \ex \climb \label{ni-'w-climb}
\end{exe}

In the postconsonantal position, most of our sources (with the notable exception of \citnp{LC20}) rarely if ever indicate the glottalization of sonorants. \citet{AnG21} has recently described the phonetic realization of such clusters as involving creaky voice phonation either in the sonorant itself (\wordnl{ɬas\mbox{-}ˀβán \phonetic{ɬasˈβ̰an}}{you see me}, \wordnl{βát\mbox{-}ˀβan \phonetic{ˈβatʰβ̰an}}{s/he sees herself/himself}) or in the preceding segment, if it is also a sonorant (\wordnl{ʃin\mbox{-}ˀβán \phonetic{ʃin̰ˈβan}}{they see us}).

\begin{exe}
    \ex \wildcat
    \ex \likelove
\end{exe}

\subsubsection{Deglottalization in codas in ``weak" syllables}\label{ni-deglottalization-codas}

As described by \citet[183–184]{AnG16c}, Nivaĉle systematically deletes postvocalic instances of /ʔ/ whenever it does not get parsed to the head syllable of the foot; in other words, postvocalic /ʔ/ can only surface in syllables that carry primary or secondary stress in Nivaĉle. Importantly, in \cits{AnG16c} analysis /ʔ/ accounts not only for the occurrences of \phonetic{ʔ} in codas, but also for what we represent as preglottalized codas (\intxt{ˀC}) in this book: in Gutiérrez' account, these are analyzed as underlying sequences of the type /ʔC/, where /ʔ/ is parsed to the nucleus. This is clearly seen in some lexemes that either have or lack /ʔ/ in different inflected forms, where stress falls on different syllables (see \sectref{ni-prosody} on stress and prosodic feet in Nivaĉle).

\ea\label{ex:coda-deg:niv}
Nivaĉle \citep[183–184]{AnG16c}\\
    \begin{xlist}
        \ex\gll (taklóˀk)\\
                weed\\
                \glt `weed'
        \ex\gll ta(k͡lok-tʃát)\\
                weed-plant\_group\\
                \glt `scrub'
        \ex\gll (jijéʔ)\\
                caraguatá\\
                \glt `caraguatá'
        \ex\gll ji(je-tʃát)\\
                caraguatá-plant\_group\\
                \glt `a place where the caraguatá plant lives'
        \ex\gll (ʃinβóʔ)\\
                honey\\
                \glt `honey'
        \ex\gll ji-(ʃínβo)\\
                1.\textsc{poss}-honey\\
                \glt `my honey'
    \end{xlist}
\z

This rule is a direct consequence of a diachronic sound change that deleted the coda \intxt{*ʔ} and deglottalized erstwhile preglottalized codas in unaccented syllables in the history of Nivaĉle. Note that in some cases the position of the stress may have changed at least in some varieties of Nivaĉle (see \sectref{ni-prosody}); it is the position of the Proto-Mataguayan accent that matters. The following examples instantiate the loss of \intxt{*ʔ} in unaccented syllables, including the glottalization in preglottalized codas, as in \REF{deg-fryit} and \REF{deg-headits}.

\begin{exe}
    \ex \mouthits
    \ex \stingerits
    \ex \daughterits
    \ex \coalrel
    \ex \soninlaw
    \ex \welln
    \ex \hunger
    \ex \fryit \label{deg-fryit}
    \ex \arrowkaxe
    \ex \youngersis
    \ex \heartmn
    \ex \sleepiness
    \ex \cavy
    \ex \hear
    \ex \beard
    \ex \distrust
    \ex \eyelash
    \ex \spousewh
    \ex \headits \label{deg-headits}
    \ex \bat
    \ex \dirt
    \ex \pigeon
\end{exe}

One exception is given in \REF{ni-mosq-nodegl}, where the PM accent is reconstructed based on evidence from Chorote. Synchronically, the root in question has irregular final stress in Nivaĉle \perscomm{Analía Gutiérrez}{2023}. Consequently, the final glottal stop fails to be deleted.

\begin{exe}
    \ex \mosquito \label{ni-mosq-nodegl}
\end{exe}

As is clear from the discussion in \citet{AnG16c}, the deglottalization applies at a relatively shallow level in Nivaĉle and does not generally alter the underlying representation of the morphemes. The following examples show that in words with an established Mataguayan etymology the deglottalization applies word-finally only in forms where the accent is non-final.

\ea\label{ex:coda-deg-inherited:niv}
Nivaĉle \citep[129, 357, 382]{JS16}\\
    \begin{xlist}
        \ex\gll (ɬ̩-ɸáj)xo\\
                3.\textsc{poss}-charcoal\\
                \glt `its charcoal'
        \ex\gll (ɸajxóʔ)\\
                charcoal\\
                \glt `charcoal'
        \ex\gll (ʔa-jíp)ku\\
                2.\textsc{poss}-hunger\\
                \glt `your hunger'
        \ex\gll (jipkúʔ)\\
                hunger\\
                \glt `hunger'
        \ex\gll (ji-ʃá)tetʃ\\
                1.\textsc{poss}-head\\
                \glt `my head'
        \ex\gll (βàt)-(ʃatéˀtʃ)\\
                {\textsc{gnr}}-head\\
                \glt `one's head'
    \end{xlist}
\z

It is important to note that although PM~enclinomena (\sectref{corta-corta}) lacked an underlying accented syllable, they do not show the deletion of \intxt{*ʔ} in Nivaĉle. This entails that at the time when the deglottalization occurred in Nivaĉle, erstwhile enclinomena had already developed a default final stress, preserved to this day in Nivaĉle.

\begin{exe}
    \ex \coalabssg
    \ex \jaguar
    \ex \treensg
    \ex \vulturesg
    \ex \takeaway
    \ex \tailsg
    \ex \withstand
    \ex \petsg
    \ex \languagewordsg
    \ex \yicalhuksg
    \ex \smelln
    \ex \fence
    \ex \lid
    \ex \starnsg
    \ex \vein
    \ex \shoot
    \ex \carrysh
    \ex \bromelia
    \ex \walk
    \ex \placen
    \ex \pricesg
    \ex \earthsg
    \ex \nightnwsg
    \ex \firewoodhuksg
    \ex \meatitssg
\end{exe}

\subsubsection{Glottal insertion in monosyllables}\label{ni-glottal-insertion}

Synchronically, the minimal word in Nivaĉle is constituted by CVC \citep[118, 132ff.]{AnG15}. This is likely a result of an innovation whereby all monosyllabic roots of the shape CV underwent insertion of a word-final \intxt{ʔ}, a process shared with Maká. It is noteworthy that the epenthesis occurred even in monosyllabic roots that never constitute a morphological (or phonological) word by themselves, as seen in \REF{ni-glott-ep-tool}--\REF{ni-glott-ep-sleep}.

\begin{exe}
    \ex \thorne
    \ex \tooln \label{ni-glott-ep-tool}
    \ex \sleep \label{ni-glott-ep-sleep}
    \ex \penis
    \ex \worm
    \ex \belly
    \ex \price
    \ex \juice
\end{exe}

\subsubsection{Consonant clusters}\label{ni-clusters}

Nivaĉle is fairly conservative with regard to the consonant clusters of Proto-Mataguayan. Very few PM~clusters have apparently become illicit in Nivaĉle.

The sound change \intxt{*(ˀ)nj} > \intxt{n} is instantiated by two examples.

\begin{exe}
    \ex \smelln
    \ex \cavy
\end{exe}

The sound changes \intxt{*tts} > \intxt{ts} and \intxt{*qk} > \intxt{k} are found in one example each; the simplification /tts/~>~/ts/ does operate in Nivaĉle as a synchronically active process, as in \wordnl{βa\mbox{-}tseβte}{one's tooth}, from \intxt{βat\mbox{-}} and \intxt{\mbox{-}tseβte} \citep[294]{JS16}.

\begin{exe}
    \ex \welln
    \ex \willow
\end{exe}

The sound change \intxt{*wh} > \intxt{x} is known from only one root, presumably to the overall rarity of the cluster \intxt{*wh} in the Proto-Mataguayan lexicon.

\begin{exe}
    \ex \spousewh
    \ex \marry
\end{exe}

In one example, the cluster \intxt{*χw \recind *hw} yielded \sound{Ni}{xiβ}.

\begin{exe}
    \ex \moon
\end{exe}

The cluster \intxt{kɸ} is licit word-medially, as in \wordnl{ji\mbox{-}kɸij}{my shoe}, but not word-initially, where \sound{PM}{*kɸ} yielded \sound{Ni}{kx}.

\begin{exe}
    \ex \skunk
\end{exe}

Some clusters, including at least two triconsonantal clusters, underwent the insertion of an \intxt{a}. Known examples involve the clusters \intxt{*nxt}, \intxt{*stw}, and \intxt{*tl}, which yielded \intxt{nxat}, \intxt{staβ}, and \intxt{tak͡l}.

\begin{exe}
    \ex \cavy
    \ex \kingvulture
    \ex \blind
\end{exe}

Note that \intxt{a}-epenthesis is a synchronically active strategy for triconsonantal clusters in the language. The epenthesis of \sound{Ni}{a} is seen in the third-person possessive and the second-person active prefixes. Both surface as a syllabic \intxt{ɬ\mbox{-}} before simplex onsets \REF{ex:lhlha:1:niv} or as a regular \intxt{ɬ\mbox{-}} before vowels \REF{ex:lhlha:2:niv}, but as \intxt{ɬa\mbox{-}} before consonant clusters \REF{ex:lhlha:3:niv} \citep[59, 62, 230--231]{AnG15}.

\ea\label{ex:lhlha:1:niv}
    \begin{xlist}
        \ex\gll ɬ-t’óx \\
                3{\SG}-aunt \\
                \glt `his/her aunt'
        \ex\gll ɬ-k͡líˀʃ \\
                3{\SG}-word \\
                \glt `his/her word'
        \ex\gll ɬ-péˀja \\
                2{\SG}-listen \\
                \glt `you listen'
    \end{xlist}
\z

\ea\label{ex:lhlha:2:niv}
    \begin{xlist}
        \ex\gll ɬ-ǻse \\
                3{\SG}-daughter \\
                \glt `his/her daughter'
        \ex\gll ɬ-ám \\
                2{\SG}-come \\
                \glt `you come'
    \end{xlist}
\z

\ea\label{ex:lhlha:3:niv}
    \begin{xlist}
        \ex\gll ɬa-ktéˀtʃ \\
                3{\SG}-grandfather \\
                \glt `his/her grandfather'
        \ex\gll ɬa-ɸxúx \\
                3{\SG}-toe \\
                \glt `his/her toe'
        \ex\gll ɬa-ktʃáʔ \\
                2{\SG}-paddle \\
                \glt `you paddle'
    \end{xlist}
\z

Finally, there are further changes involving \intxt{x} and \intxt{ʃ} in the environment \#\_C in some Nivaĉle dialects. These will be discussed in greater detail in \sectref{ni-sc-shc}.

\subsection{Vowels}\label{ni-vowels}

Nivaĉle is quite conservative with regard to the vowels of Proto-Mataguayan, with the only major innovation being the unconditional merger of \intxt{*a} and \intxt{*ä} as \sound{Ni}{a} (see \sectref{pm-ae} for examples of the sound change \sound{PM}{*ä}~>~\sound{Ni}{a}). Before labials, \sound{PM}{*å} is sometimes reflected as \sound{Ni}{a}, though the inverse development is also found; as discussed in \sectref{ni-a-ao-labials} below, these apparently irregular correspondences may have in fact originated after the dialectal diversification of Nivaĉle as a result of dialectal borrowing.

\begin{exe}
    \ex \arrive
    \ex \cryao
    \ex \returnth
    \ex \burn
    \ex \shoulderblade
    \ex \spring
\end{exe}

Another minor innovation involving vowels is that the sequence \sound{PM}{*éwV} is reflected as \intxt{oβV} in Nivaĉle.

\begin{exe}
    \ex \wildmanioc
    \ex \river
\end{exe}

\subsection{Word-level prosody}\label{ni-prosody}

The stress system of Nivaĉle inherits some of the properties reconstructed for Proto-Mataguayan in \chapref{prosody}. A synchronic analysis of the Nivaĉle stress system is offered by \citet{AnG15}, who attributes the superficial patterns to systematic regularities of three types. Specifically, she argues that tautosyllabic sequences of the type \intxt{Vʔ} behave as heavy and attract stress; that the language has a number of edge-alignment constraints whereby prosodic foot domains align with the left edge or with the right edge, depending on the morphological category; and that syllables of the structure /CVC/ constitute degenerate feet. Let us examine the former two regularities in their relation with Proto-Mataguayan.

\subsubsection{Trochaic stress pattern as a remnant from Proto-Mataguayan}\label{ni-trochee}

The first generalization -- that tautosyllabic sequences of the type \intxt{Vʔ} are heavy in Nivaĉle -- is meant to account for the fact that although most disyllabic underived words receive final stress in the language (and are thus iambic), some receive initial stress (and are thus trochaic), and there is a strong correlation between the presence of a /ʔ/ in the initial syllable and the trochaic stress pattern. The following examples are from \citet[162–163, 168]{AnG15}.

\booltrue{listing}
\ea\label{ex:stress:1:niv}
    \begin{xlist}
        \ex \wordnl{såt’ǻ}{cactus fruit}
        \ex \wordnl{ʔitǻx}{fire}
        \ex \wordnl{k’akxó}{armadillo}
        \ex \wordnl{nuksítʃ}{manioc}
        \ex \wordnl{ʃinβóʔ}{honey}
        \ex \wordnl{k’utxáˀn}{thorn}
        \ex \wordnl{kúˀkten}{thunder}
        \ex \wordnl{tǻʔɬås}{pot}
        \ex \wordnl{jóʔnis}{fox}
        \ex \wordnl{βéʔɬa}{one}
    \end{xlist}
\z
\boolfalse{listing}

There are, however, several exceptions to this generalization, which are not explicitly discussed by \citet{AnG15}. In a handful of disyllabic roots, stress falls on the initial syllable despite the absence of /ʔ/, at least for some speakers.\footnote{\label{ni-stress-shift}\citet[150, 189, 205]{NS87} documents forms such as \wordnl{ʔoɸó}{dove}, \wordnl{nuʔú}{dog}, \intxt{=k’ojá} (no gloss), suggesting that the position of the stress may be different for some speakers. Analía Gutiérrez (personal communication, 2021) confirms that there is interspeaker variation in this regard. It is straightforward to assume that the less common trochaic pattern is conservative, and that the iambic pattern attested in \citet{NS87} is an innovation.} The following examples are from \citet[38, 267]{AnG15} and \citet[36]{LC20}.

\booltrue{listing}
\ea\label{ex:stress:trochaic:niv}
    \begin{xlist}
        \ex \wordnl{ʔóɸo}{dove}
        \ex \wordnl{ɬ\mbox{-}ǻse}{her/his daughter}
        \ex \wordnl{núʔu}{dog}
        \ex \wordnl{=k’ója}{for, before}
    \end{xlist}
\z
\boolfalse{listing}

The cognates of the former three stems in Chorote all have initial stress, reflecting the trochaic accent pattern of Proto-Mataguayan: \word{PCh}{*ʔóhwoʔ}{dove}, \wordnl{*hl-ǻseʔ}{her/his daughter}, \wordnl{*núʔuh}{dog}. The fourth one also occurs with initial stress when prefixed: \word{PCh}{*\mbox{-}kójaʔ}{for}. It is, therefore, tempting to assume the trochaic accent of PM is preserved in disyllables, but only in those ending in a vowel in Nivaĉle. By contrast, PM trochaic disyllables ending in a consonant appear to have innovated final stress in Nivaĉle: to the best of our knowledge, no variants with initial stress have been attested in any published source for nouns such as \wordnl{ʃnaβǻp}{spring}, \wordnl{ɸináx}{crab}, \wordnl{nåjíʃ}{path}, \wordnl{noβók}{wild manioc}, \wordnl{βosók}{butterfly}, \wordnl{ʔitǻx}{fire} \citep[40, 163, 271, 273, 304, 319]{AnG15}, even though their Proto-Mataguayan etyma are reconstructed as trochaic: \word{PM}{*xnáwåp}{spring}, \wordnl{*ɸínäχ}{crab}, \wordnl{*ˀnǻjix}{path}, \wordnl{*néwok}{wild manioc}, \wordnl{*ˀwósåq \recind *ˀwósåk}{butterfly}, \wordnl{*ʔítåχ}{fire}. Moreover, even some vowel-final roots are systematically documented with a final stress; examples include \wordnl{ɬ\mbox{-}aβǻ}{its flower}, \wordnl{tʃ’etʃé}{parrot}, \wordnl{ʔuk͡lʔǻ}{turtle dove} \citep[38, 68, 110]{AnG15}. We surmise that these nouns instantiate the type of variation discussed in footnote \ref{ni-stress-shift} and predict that they have trochaic variants at least in some dialects, something that can be tested in the future with native speakers of Nivaĉle.

As for the correlation between the presence of a postvocalic /ʔ/ and stress in Nivaĉle, one is left wondering whether that could not be an epiphenomenal consequence of deglottalization in unstressed syllables, discussed in \sectref{ni-deglottalization-codas} above. Indeed, if the language allows for disyllabic stems that are lexically specified as trochaic, one could expect some of them to contain a /ʔ/ after the vowel of the initial syllable (as in /kúʔkten/, or perhaps /kúʔk͡lten/ `thunder'). This glottal stop makes it to the surface, because it occurs in an accented syllable. On the other hand, disyllabic stem with final accent can also contain an underlying /ʔ/ after the vowel of the initial syllable, but the fact that it is located in the unaccented position is expected to prevent it from surfacing: compare \word{Ni}{ɸúˀx}{it smells} and \wordnl{ɸux\mbox{-}k’é}{it stinks} \citep[138]{JS16}. This possibility will need to be kept in mind in future descriptions of the stress system of Nivaĉle.

\subsubsection{Edge-aligned foot construction}

We have seen in \sectref{ni-trochee} that the disyllabic roots with initial stress (trochees) of Proto-Mataguayan show a tendency of shifting the stress rightwards in Nivaĉle, and in some dialects the erstwhile distinction may have been entirely erased in favor of the iambic pattern. This subsection presents additional evidence for an innovative pattern in Nivaĉle, where iambic feet are constructed from right to left.

\citet{AnG15} argues that different morphological categories are associated with different edge-alignment constraints in Nivaĉle. More specifically, prosodic foot domains align with the right edge in words composed of bare roots (\conc{Root} domain), or in words where roots are augmented by derivational suffixes (\conc{Morphological Stem 1}), in which case iambic feet are built from the right edge of word. The following examples are from \citet[165, 173]{AnG15}; note that non-final syllables of the structure \intxt{CVC} constitute a degenerate foot, and the grave accent indicates secondary stress.

\newpage
\booltrue{listing}
\ea
    \begin{xlist}
        \ex \wordnl{tʃa(xaní)}{wild boar}
        \ex \wordnl{ʔå(jintʃé)}{pepper}
        \ex \wordnl{(pùʔ)(xaná)}{three}
        \ex \wordnl{(ʔå̀k)(xek͡lǻ)}{woman}
        \ex \wordnl{(sisé)}{cane}
        \ex \wordnl{si(se-tʃát)}{cane field} \=\gloss{cane} +\gloss{plant group}
        \ex \wordnl{(samúk)}{feces}
        \ex \wordnl{(sàm)(ku-xíj)}{latrine} \=\gloss{feces} +\gloss{concave container}
    \end{xlist}
\z
\boolfalse{listing}

In words that contain prefixes and lack inflectional suffixes (\conc{Morphological Stem 2}), the iambic foot is instead aligned with the left edge of the word. The following data are from \citet[184, 186, 188–191, 195, 199–200]{AnG15}. Note the coda deglottalization in the unparsed syllables in \REF{yishinvo}, \REF{katsivlhi}, \REF{vatavlhi}, \REF{yic'utjan}, \REF{yicajuk}, \REF{yicatsos}, \REF{yicatjoc}, \REF{yicact'ech}, \REF{yicact'e}, \REF{ashatech}, \REF{atinish}. In \REF{yitjooc}, by contrast, deglottalization affects the coda of the weak syllable in an iambic foot.

\booltrue{listing}
 \ea
     \begin{xlist}
         \ex \wordnl{(ʃinβóʔ)}{honey}
         \ex \wordnl{(ji-ʃín)βo}{my honey} \label{yishinvo}
         \ex \wordnl{(ʔitǻx)}{fire}
         \ex \wordnl{(ʔa-β-í)tåx}{your fire}
         \ex \wordnl{(ji-βɬíʔ)}{my rib}
         \ex \wordnl{(katsí)-βɬi}{our rib} \label{katsivlhi}
         \ex \wordnl{(βatá)-βɬi}{one's rib} \label{vatavlhi}
         \ex \wordnl{(k’utxáˀn)}{thorn}
         \ex \wordnl{(ji-k’út)xan}{my needle} \label{yic'utjan}
         \ex \wordnl{(xúˀk)}{firewood}
         \ex \wordnl{(ji-ká)-xuk}{my firewood} \label{yicajuk}
         \ex \wordnl{(ji-tsóˀs)}{my milk (inalienable)}
         \ex \wordnl{(ji-ká)-tsos}{my milk (alienable)} \label{yicatsos}
         \ex \wordnl{(ji-txóˀk)}{my uncle} \label{yitjooc}
         \ex \wordnl{(ji-ká)-ˀtxok}{my brother-in-law} \label{yicatjoc}
         \newpage %longdistance
         \ex \wordnl{(ji-kt’éˀtʃ)}{my grandfather}
         \ex \wordnl{(ji-ká)-kt’etʃ}{my father-in-law} \label{yicact'ech}
         \ex \wordnl{(ji-kt’éʔ)}{my grandmother}
         \ex \wordnl{(ji-ká)-kt’e}{my mother-in-law} \label{yicact'e}
         \ex \wordnl{(βàt)-(ʃatéˀtʃ)}{one's head}
         \ex \wordnl{(ʔa-ʃá)tetʃ}{your head} \label{ashatech}
         \ex \wordnl{(kàs)-(tiníˀʃ)}{our necklace}
         \ex \wordnl{(ʔa-tí)niʃ}{your necklace} \label{atinish}
     \end{xlist}
\z
\boolfalse{listing}

Finally, the largest domain for stress assignment described in \citet{AnG15} -- the \conc{Morphological Word} -- is the one that encompasses inflectional suffixes, such as the nominal plural suffixes. The presence of such suffixes overrides the Morphological Stem 2 domain, defined by prefixes, and iambic feet are constructed, one again, from the right left edge of the word. The following examples are from \citet[202–206]{AnG15}.

\booltrue{listing}
\ea
    \begin{xlist}
        \ex \wordnl{(ɬ-åk-ǻs)}{her/his foods}
        \ex \wordnl{ji-(tat-ís)}{my thorns}
        \ex \wordnl{ji-(k͡liʃ-áj)}{my words}
        \ex \wordnl{ji-t’i(k͡l-éj)}{my tears}
        \ex \wordnl{(ji-kòˀts)(xat-ís)}{my lands}
        \ex \wordnl{(ji-ɸè)(tats-íj)}{my roots/medicines}
        \ex \wordnl{(ji-på̀ʔ)(kåt-ǻj)}{my hands}
    \end{xlist}
\z
\boolfalse{listing}

The right-aligned footing pattern, described by \citet[7]{AnG15} for the Root, Morphological Stem 1, and Morphological Word domains in Nivaĉle, constitutes an innovation with regard to the left-aligned accent pattern of Proto-Mataguayan, as reconstructed in \chapref{prosody}. For these morphological categories, the position of the Nivaĉle stress and of the PM accent coincide in a handful of cases (see \sectref{corta-corta}, \sectref{corta-larga}, \sectref{corta-corta-larga}) but differ in others. The innovative pattern erases the distinctions that may have been present in PM, and is thus of no use for comparative reconstruction, even if its similarity with the right-aligned stress of Maká (\sectref{mk-prosody}) and Wichí (\sectref{wi-stress}) is of note.

Conversely, the left-aligned stress in the Morphological Stem 2 domain must reflect directly the left-aligned accent of Proto-Mataguayan. Recall that this pattern obtains in prefixed words, and almost all known Nivaĉle prefixes go back to PM prefixes that lack an underlying accent.\footnote{We are aware of few exceptions. First of all, the PM etymon of the 1+2.\textsc{poss} prefix \intxt{kats(i)=} probably was not a canonical prefix at all. Its Chorote cognate has a different function (1+2.P/S\textsubscript{P}) and is invisible for the stress assignment rule, suggesting that \wordng{PM}{*qats} was an enclitic or even an independent word, possibly a pronoun rather than a person index (otherwise it would be difficult to account for the difference between the functions of its reflexes in Nivaĉle and Chorote). The second exception is the alienizing prefix \intxt{ka\mbox{-}}. It goes back to \wordng{PM}{*qá\mbox{-}}, an accented morpheme that must have been phonologically independent in Proto-Mataguayan, just like its Chorote reflex. Be it as it may, in Nivaĉle \intxt{ka\mbox{-}} is always preceded by a possessive person prefix; consequently, it is always stressed (just like its PM~etymon), thus posing no difficulties for our analysis. Finally, the reflexive/reciprocal \intxt{(\mbox{-})βa(ˀ)t(\mbox{-})} (as well as the indefinite possessor prefix \intxt{βat(-)}, which could be related to the reflexive/reciprocal prefix) is another possible candidate. Its Iyo’awujwa’ and Manjui cognates are not prefixes but rather roots of independent prosodical words; in absence of a Wichí cognate it is impossible to determine whether its PM etymon was accented (\intxt{*\mbox{-}wä́ˀt}) or not (\intxt{*\mbox{-}wäˀt}).} When such prefixes are followed by an unaccented monosyllabic root in Proto-Mataguayan, the word remains unaccented, as discussed in \chapref{corta-corta}, and its Nivaĉle reflex regularly receives default (final) stress: \wordnl{ji\mbox{-}ˀk͡líˀʃ}{my word}, \wordnl{n\mbox{-}átʃ}{(that) s/he go away}, \wordnl{ʔa\mbox{-}ʃáʔ}{your salary}.

\booltrue{listing}
\ea
    \begin{xlist}
        \ex \intxt{*ji-} + \intxt{*-ˀliˀx} → \wordnl{*ji-ˀliˀx}{my word}
        \ex \intxt{*n-} + \intxt{*-äk} → \wordnl{*n-äk}{(that) s/he go away}
        \ex \intxt{*ʔa-} + \intxt{*-xa} → \wordnl{*ʔa-xa}{your payment}
    \end{xlist}
\z
\boolfalse{listing}

When unaccented prefixes are followed by an accented monosyllabic consonant-initial root or by a trochaic or unaccented disyllabic consonant-initial root, the accent regularly falls on the peninitial syllable in Proto-Mataguayan, as discussed in \chapref{prosody}. In this case, Nivaĉle retains the peninitial stress of Proto-Mataguayan: \wordnl{ji\mbox{-}k͡lés}{my children}, \wordnl{ɬ̩\mbox{-}ɸájxo}{its charcoal}, \wordnl{ji\mbox{-}ɸétas}{my root/medicine}.

\booltrue{listing}
\ea
    \begin{xlist}
        \ex \intxt{*ji-} + \intxt{*-léts} → \wordnl{*ji-léts}{my children}
        \ex \intxt{*ɬ-} + \intxt{*ɸajxoʔ} → \wordnl{*ɬ̩-ɸájxoʔ}{its charcoal}
        \ex \intxt{*ji-} + \intxt{*ɸétäˀts} → \wordnl{*ji-ɸétäˀts}{my root}
    \end{xlist}
\z
\boolfalse{listing}

There are two combinations, however, where in our reconstruction prefixed words bear accent in a position other than non-peninitial in Proto-Mataguayan. One such combination arises when an unaccented prefix takes a non-moraic allomorph before a vowel-initial trochaic root \REF{pm-lhôse}, where the accent is initial. The second combination is when an unaccented prefix is followed by an underlyingly iambic consonant-initial root, as in \REF{pm-yichita'} or \REF{pm-acaclô'}, in which case the accent is postpeninitial.

\booltrue{listing}
\ea
    \begin{xlist}
        \ex \intxt{*ɬ-} + \intxt{*-ǻseʔ} → \wordnl{*ɬ-ǻseʔ}{her/his daughter} \label{pm-lhôse}
        \ex \intxt{*ji-} + \intxt{*-kitáʔ} → \wordnl{*ji-kitáʔ}{my elder sister} \label{pm-yichita'}
        \ex \intxt{*ʔa-} + \intxt{*-qalǻʔ} → \wordnl{*ʔa-qalǻʔ}{your leg} \label{pm-acaclô'}
    \end{xlist}
\z
\boolfalse{listing}

In each case, there is evidence that Nivaĉle might in fact retain the Proto-Mataguayan accent pattern, thus violating the left-aligned pattern posited by \citet{AnG15} for the Prosodic Word 2 domain. The Nivaĉle reflex of \word{PM}{*ɬ\mbox{-}ǻseʔ}{her/his daughter} is attested as \intxt{ɬ\mbox{-}ǻse} in \citet[38]{AnG15}, with initial stress. As for the postpeninitial accent pattern, although we have been unable to find the reflexes of forms such as \wordnl{*ji\mbox{-}kitáʔ}{my elder sister} or \wordnl{*ʔa\mbox{-}qalǻʔ}{your leg} in sources that indicate stress explicitly,\footnote{By saying this, we exclude \citet{NS87}, who attests final stress not only in the reflexes of these nouns, but also in multiple words where \citet{AnG15} has documented non-final stress. That way, the variety of Nivaĉle described by \citet{NS87} is not informative for the purposes of reconstructing PM prosody.} note that the final \intxt{ʔ} fails to deglottalize in Nivaĉle: \wordnl{ji\mbox{-}tʃitaʔ}{my elder sister}, \wordnl{ʔa\mbox{-}kak͡låʔ}{your leg} \citep[56, 103]{JS16}. This indicates that the Nivaĉle forms in question might retain the postpeninitial accent reconstructed for PM, a pattern unaccounted for by \citet{AnG15}: \intxt{ji\mbox{-}(tʃitáʔ)}, \intxt{ʔa\mbox{-}(kak͡lǻʔ)}. This point needs to be clarified in future fieldwork with native speakers of Nivaĉle.

\section{Innovations in Nivaĉle dialects}\label{ni-dialects}

\citet[7]{AnG15} reports at least three regional varieties of Nivaĉle as defined by linguistic criteria:
\begin{enumerate}
    \item \concl{Chishamnee Lhavos} (also known as the Arribeño, or Upriver dialect), spoken along the Pilcomayo River, from Fortín Magariños (to the west from Misión Esteros) in the southeast up to the Pedro P. Peña area (Paraguay) and Salta (Argentina) in the northwest \citep[21--22]{NS87};
    \item \concl{Shichaam Lhavos} (also known as the Abajeño, or Downriver dialect), spoken from Fortín Magariños up to the Missions of San José de Esteros and San Leonardo de Escalante/Fischat (Paraguay) \citep[21--22]{NS87};
    \item \concl{Yita’ Lhavos} (or the Bush dialect), whose zone lays to the north from the Chishamnee Lhavos area, entirely in Paraguay, reaching Mayor Infante Rivarola and approaching Mariscal Estigarribia, with speakers in the Mission of Santa Teresita.
\end{enumerate}

Little is known about the defining characteristics of the dialects spoken by the \concl{Jotoi Lhavos} (who live in the communities around Campo Loa, Paraguay) and the \concl{Tavashai Lhavos} (who live north of San José de Esteros, and southeast of Filadelfia, close to the Mennonites colonies, also in Paraguay).

In what follows, we outline the phonological evolution of the Nivaĉle dialects on which linguistic data are available.

\subsection{Reflexes of \intxt{*å} in Nivaĉle dialects}

The opposition between the back and non-back low vowels (\intxt{*å} and \intxt{*a}) is generally preserved in Nivaĉle, except for certain (sub)dialects, where \intxt{å} may merge with \intxt{a} or \intxt{o} in specific environments.

\subsubsection{Merger of \intxt{a} and \intxt{å}}\label{ni-a-ao-merger}

The merger of \intxt{å} and \intxt{a} is found in the speech of many speakers of Nivaĉle. Most notably, \sound{Ni}{å} and \intxt{a} are reported to have entirely merged as \intxt{a} in the variety spoken by the Yita’ Lhavos \citep[37]{AnG15}. According to one of \cits{AnG15} consultants, who works as a primary school teacher in Misión Santa Teresita (where the Yita’ Lhavos variety is spoken), ``the vowel [ɑ] is only produced when reading texts at school or during mass, otherwise the [a] has replaced the [ɑ] in everyday life''. The examples in \REF{ni-yita-ao-a}, taken from \citet[37--38]{AnG15}, illustrate.

\booltrue{listing}
\ea \label{ni-yita-ao-a}
    \begin{xlist}
        \ex \wordng{ShL}{x-ǻk} \recind \word{YL}{x-ák}{I go}
        \ex \wordng{ShL}{tåjéˀx} \recind \word{YL}{tajéˀx}{shaman}
        \ex \wordng{ShL}{ʔa-ɬǻn} \recind \word{YL}{ʔa-ɬán}{light!}
        \ex \wordng{ShL}{xa-k͡lǻˀp} \recind \word{YL}{xa-k͡láˀp}{I have (sb.) on my lap}
        \ex \wordng{ShL}{ʔinǻˀt} \recind \word{YL}{ʔináˀt}{water}
        \ex \wordng{ShL}{toβǻk} \recind \word{YL}{toβák}{river}
        \ex \wordng{ShL}{ɬ-ǻse} \recind \word{YL}{ɬ-ási}{his/her daughter}
    \end{xlist}
\z

In addition to the Yita' Lhavos variety, \citet[534--535]{NS87} reports that Shichaam Lhavos \intxt{å} corresponds to \intxt{a} in the speech of her Chishamnee Lhavos consultant from Las Vertientes (however, the same speaker is reported to produce \intxt{å} in some words where the Shichaam Lhavos tend to have \intxt{o}, on which see \sectref{ni-chishaam-rounding}). \citet[8]{LC20} also state that the merger is complete or ``very advanced'' for many (though not all) Chishamnee Lhavos. \citet[504, 507]{NS87} gives the following examples.

\ea\label{ni-chishamnee-ao-a}
    \begin{xlist}
        \ex \wordng{ShL}{t'ak͡låˀk} \recind \word{ChL}{t'ak͡laˀk}{weed}
        \ex \wordng{ShL}{-k͡lån} \recind \word{ChL}{-k͡lan}{to kill}
        \ex \wordng{ShL}{xokånåxå} \recind \word{ChL}{xokanaxa}{collared peccary}
    \end{xlist}
\z
\boolfalse{listing}

That way, \sound{Proto-Nivaĉle}{*å}, inherited from Proto-Mataguayan, is best preserved in Shichaam Lhavos and for some speakers of Chishamnee Lhavos in the default environment.

\subsubsection{Merger of \intxt{å} and \intxt{o}}\label{ni-chishaam-rounding}

Above we have seen that Shichaam Lhavos is generally conservative with regard to \sound{Proto-Nivaĉle}{*å}. In some words, however, it appears to be reflected as \intxt{o} in Shichaam Lhavos. In the same words, it fails to front to \intxt{a} in the Chishamnee Lhavos variety described by \citet{NS87}, as it usually does, on which see \REF{ni-chishamnee-ao-a} above. Consider the following examples from \citep[498, 504, 514, 517, 521]{NS87}, where \intxt{å} in the Chishamnee Lhavos dialect corresponds to \intxt{o} in Shichaam Lhavos.

\booltrue{listing}
\ea
    \begin{xlist}
        \ex \wordng{ShL}{βat\mbox{-}kåxoj\mbox{-}xajaʃ} \recind \word{ChL}{βat\mbox{-}kåxåj\mbox{-}xajaʃ}{one's game, prey} \label{vatcôjôyjayash}
        \ex \wordng{ShL}{xa\mbox{-}tʃetxoj} \recind \word{ChL}{xa\mbox{-}tʃetxåj}{I staked}
        \ex \wordng{ShL}{k\mbox{-}'oxeˀtʃ} \recind \word{ChL}{k\mbox{-}'åxeˀtʃ}{I skinned}
        \ex \wordng{ShL}{xa\mbox{-}tijox} \recind \word{ChL}{xa\mbox{-}tijåx}{I shoot}
        \ex \wordng{ShL}{tʃi\mbox{-}joʔ\mbox{-}xi} \recind \word{ChL}{tʃi\mbox{-}jåʔ\mbox{-}xi}{it is drunk} \label{chiyô'ji}
        \ex \wordng{ShL}{ʔinot} \recind \word{ChL}{ʔinåt}{water}
        \ex \wordng{ShL}{noke} \recind \word{ChL}{nåke}{this}
        \ex \wordng{ShL}{ʔopeˀʃ} \recind \word{ChL}{ʔåpeˀʃ}{therefore}
    \end{xlist}
\z
\boolfalse{listing}

Sources other than \citet{NS87} -- including \citet{AnG15}, who has worked with speakers of Shichaam Lhavos -- usually attest \intxt{å} in the cognates of these words (or \intxt{a}, for dialects that have lost \intxt{*å} altogether), suggesting that the reflex \intxt{o} is restricted to specific subdialects of Shichaam Lhavos. We have been unable to identify the exact conditioning environment, but note that the target vowel is adjacent to \intxt{x} in most examples, including \REF{vatcôjôyjayash}--\REF{chiyô'ji}. The same environment appears to have prevented \intxt{å} from fronting to \intxt{a} in the subdialect of Chishamnee Lhavos described by \citet{NS87}.

\subsubsection{Variation between \intxt{a} and \intxt{å} before labials}\label{ni-a-ao-labials}

The Proto-Mataguayan distinction between \intxt{*å} and \intxt{*a} appears to have blurred before labial consonants in Nivaĉle, with most varieties showing \intxt{a} as the reflex of both Proto-Mataguayan vowels. Consider the following examples of Proto-Mataguayan roots that are unequivocally reconstructed with \sound{PM}{*å}, yet most Nivaĉle varieties, including the conservative Shichaam Lhavos dialect, show \intxt{a} in its place according to our sources.\footnote{An anonymous reviewer notes that the vowel in question can be pronounced as [ɑ] in these examples, suggesting that extra documentation with special attention to the dialectal variation is needed in order to fully describe the reflexes of low vowels before labials in Nivaĉle.}

\begin{exe}
    \ex \arrive
    \ex \cryao
    \ex \returnth
\end{exe}

One exception is the Central Paraguayan subdialect of Chishamnee Lhavos, spoken by one of the co-authors of \citet{LC20}. In that variety, \intxt{*å} is the only low vowel found before labial consonants: \wordnl{n\mbox{-}åm}{s/he arrives}, \wordnl{x\mbox{-}åp=’in}{I cry},  \wordnl{β\mbox{-}åpek}{s/he returns thither}.

Yet in other cases, \sound{PM}{*å} before labials is reflected as \sound{Ni}{å}, sometimes in variation with \intxt{a}. The nature of variation in such cases is in all likelihood dialectal, though this is not explicitly stated in our sources. In the following examples, the Nivaĉle reflexes are cited as they most commonly appear in our sources, but note that the verb in \REF{ni-ao-a-burn} is attested not only as \intxt{\mbox{-}ap’aɬ}, but also as \intxt{\mbox{-}åp’aɬ}, as in the first-person reflexive \intxt{xa\mbox{-}βank\mbox{-}åp’aɬ} \citep[47]{JS16}, or even as \intxt{\mbox{-}åʔp’åɬ\mbox{-}}, as in \wordnl{t\mbox{-}åʔp’åɬ\mbox{-}xan}{s/he burns cháguar} \citep[111]{LC20}. Conversely, the noun in \REF{ni-ao-a-orphan} is usually attested with a back vowel \citep[254, 277]{AnG15}, but some sources give a form with a non-back vowel (\intxt{ʔaɸteˀk}), which is probably characteristic of the Pilcomayeño subdialect of Chishamnee Lhavos \citep[125]{LC20,NS87}. 

\begin{exe}
    \ex \burn \label{ni-ao-a-burn}
    \ex \snail
    \ex \defecate
    \ex \dinlaw
    \ex \abdcavity
    \ex \spring
    \ex \spin
    \ex \orphanmn \label{ni-ao-a-orphan}
\end{exe}

As for Proto-Mataguayan \intxt{*a} and \intxt{*ä} before labials, they are mostly reflected as \sound{Ni}{a}, sometimes in variation with \intxt{å}. In \REF{ni-a-ao-shoulderblade}, \intxt{å} is the only option attested. In \REF{ni-a-ao-spring}, the reflex \intxt{ʃnaβåp} is attested by \citet[111, 395]{NS87} and \citet[40, 64]{AnG15}, whereas the reflex \intxt{ʃnåβåp} is attested by \citet[180]{NS87}, \citet[118, 304]{AF16}, \citet[53]{AnG15}, \citet[244]{JS16}, and \citet[127]{LC20}. One can conclude that the former likely represents the Shichaam Lhavos variety, whereas the latter is typical of the Chishamnee Lhavos variety. In Yita' Lhavos, the reflex is expectedly \intxt{ʃnaβap} \citep[50]{AnG15}, because that variety lacks the phoneme /å/ altogether. By contrast, the Central Paraguayan subdialect of Chishamnee Lhavos is reported to display an \intxt{å} in such cases, as in \wordnl{ɬ\mbox{-}åβå}{its flower} \citep[73]{LC20}.

\begin{exe}
    \ex \wing
    \ex \flower
    \ex \lick
    \ex \shoulderblade \label{ni-a-ao-shoulderblade}
    \ex \smooth
    \ex \spring \label{ni-a-ao-spring}
    \ex \wolf
    \ex \rat
    \ex \jararaca
    \ex \peccary
\end{exe}

In conclusion, Nivaĉle preserves the distinction between \intxt{*å} and \intxt{*a} in a very unsystematic way before labial consonants, with the exceptions being too numerous to be ignored. We tentatively attribute them to interdialectal borrowing, but the issue clearly needs further research.

\subsection{Variation between \intxt{ji} and \intxt{i}}\label{ni-yi-i}

\citet[534--535]{NS87} states that the sequence \intxt{ji} may optionally lose the approximant \intxt{j} in the speech of most of her Shichaam Lhavos consultants (except for one consultant from San Leonardo/Fischat, who consistently has \intxt{ji}), whereas her consultant from the Chishamnee Lhavos group has only the \intxt{j\mbox{-}}less variant in his speech. More recently, \citet[49]{LC20} reported that the sequence \intxt{ji} -- not only word-initially, but in any position -- may optionally lose the approximant \intxt{j} in the Chishamnee Lhavos variety, especially in its riverine subdialect (spoken along the Pilcomayo River) and in non-careful speech.

\citet[173, 498, 514, 521, 531]{NS87} gives the following examples.

\booltrue{listing}
\ea
    \begin{xlist}
        \ex \wordng{ShL}{ji-} \recind \word{ChL}{i-}{1.\textsc{poss}}
        \ex \wordng{ShL}{jitaʔ} \recind \word{ChL}{itaʔ}{forest}
        \ex \wordng{ShL}{jitʃatxuɬ} \recind \word{ChL}{itʃatxuɬ}{four}
        \ex \wordng{ShL}{jiteˀx} \recind \word{ChL}{iteˀx}{grass}
        \ex \wordng{ShL}{ʔojintʃe-j} \recind \word{ChL}{ʔointʃe-j}{peppers}
    \end{xlist}
\z

No such variation concerns instances of \intxt{ʔi} that lack an underlying /j/, as in \wordnl{ʔitåx}{fire}, which never appears as \intxt{*jitåx}. 

\subsection{Variation between \intxt{Cˀβu} and \intxt{Cʔu}}\label{ni-wu-u}

\citet[50]{LC20} report that the sequence \intxt{Cˀβu} loses the approximant \intxt{β} (represented as \intxt{w} in the cited work) in the subdialect of Chishamnee Lhavos spoken in Central Paraguay:

\ea
    \begin{xlist}
        \ex \wordng{ChL-Pi}{sˀβuk͡lax} \recind \word{ChL-Py}{sʔuk͡lax}{anteater}
        \ex \wordng{ChL-Pi}{k\mbox{-}'a\mbox{-}sˀβun} \recind \word{ChL-Py}{k\mbox{-}'a\mbox{-}sʔun}{I love you, I want you}
    \end{xlist}
\z
\boolfalse{listing}

\subsection{Delateralization before \sound{Ni}{ʔ}}\label{ni-kl-glottal}

In all Nivaĉle dialects, an entirely productive rule delateralizes \intxt{k͡l} to \intxt{k} in codas as a result of a sound change (see \sectref{ni-kl-k}). Diachronically, the sound change in question also applied within morphemes, and consequently sequences of the type \intxt{*k͡lC} are not found anywhere in the lexicon of Nivaĉle with one exception: namely, the cluster \sound{Ni}{k͡lʔ} is licit in most dialects morpheme-internally, as in \wordnl{ʔuk͡lʔa}{dove} (from \wordng{PM}{*ʔúlʔåh}). At morpheme boundaries, \intxt{k͡l} is delateralized to \intxt{k} in all dialects even before a \intxt{ʔ}, with the resulting cluster \intxt{k+ʔ} expectedly yielding \intxt{k’}.

In the variety spoken by the Yitʼa Lhavos, however, the sequence \sound{Ni}{k͡lʔ} is entirely illicit. Erstwhile \intxt{*k͡lʔ} changes to \intxt{kʼ} both within morphemes and at morpheme boundaries in that dialect \citep[7, 227--228]{AnG15}, resulting in the sound correspondence between \intxt{k’} in Yitʼa Lhavos and \intxt{k͡lʔ} in other varieties, including Shichaam Lhavos and Chishamnee Lhavos. This is shown in \REF{Ni-kl-glottal-merger} \citep[227--228]{AnG15}.

\booltrue{listing}
\ea\label{Ni-kl-glottal-merger}
    \begin{xlist}
        \ex \wordng{YL}{ʔuk’á} \recind \word{ShL}{ʔuk͡lʔá}{dove}
        \ex \wordng{YL}{ji-ɸák’u} \recind \word{ShL}{ji-ɸák͡lʔu}{my brother-in-law}
        \ex \wordng{YL}{ji-ɸák’a} \recind \word{ShL}{ji-ɸák͡lʔa}{my nephew}
    \end{xlist}
\z
\boolfalse{listing}

\subsection{Variation before \sound{Ni}{sC\mbox{-}} and \intxt{ʃC\mbox{-}}}\label{ni-sc-shc}

\citet[534--535]{NS87} reports that the word-initial cluster \intxt{sC\mbox{-}} is found in the speech of her consultant from Las Vertientes (speaker of Chishamnee Lhavos) and – in variation with \intxt{ʃC\mbox{-}} – of one consultant from the Mission of San Leonardo/Fischat (speaker of Shichaam Lhavos), whereas her other Shichaam Lhavos-speaking consultants from San Leonardo/Fischat and San José de Esteros use exclusively \intxt{ʃC\mbox{-}}. This correspondence is found in items such as \wordnl{sk͡låkxaj \recind ʃk͡låkxaj}{wild cat} and \wordnl{st(a)\mbox{-} \recind ʃt(a)\mbox{-}}{1+2.A/S\textsubscript{A}}. The form \intxt{sk͡låkxaj} is attested as a variant alongside \intxt{ʃk͡låkxaj} in \citet[231]{AnG15}, who worked with speakers of Shichaam Lhavos and Yita' Lhavos. Only the forms \intxt{ʃk͡låkxaj} and \intxt{ʃt(a)\mbox{-}} are attested in \citet{LC20}, who deal with the Chishamnee Lhavos dialect.

From a diachronic point of view, the pattern discussed in this subsection is rather surprising: comparative data show that the variant with \intxt{s} is more conservative in words such as \wordnl{sk͡låkxaj \recind ʃk͡låkxaj}{wild cat}, but the variant with \intxt{ʃ} is apparently more conservative in \wordnl{st(a)\mbox{-} \recind ʃt(a)\mbox{-}}{1+2.A/S\textsubscript{A}}. It is therefore unclear whether the sound correspondence in question results from only one post-Proto-Nivaĉle sound change or whether various sound changes with different directionalities have occurred in different Nivaĉle dialects.

\subsection{Shichaam Lhavos \intxt{i} and Chishamnee Lhavos \intxt{e}} \label{ni-chishamnee-lowering}

\citet[124--125, 162, 498, 504, 514, 521, 526]{NS87} documents the correspondence between \intxt{i} in the Shichaam Lhavos dialect and \intxt{e} in the Chishamnee Lhavos dialect.

\booltrue{listing}
\ea
    \begin{xlist}
        \ex \wordng{ShL}{t\mbox{-}'åxi\mbox{-}tʃe} \recind \word{ChL}{t\mbox{-}'åxe\mbox{-}tʃe}{its scale}
        \ex \wordng{ShL}{\mbox{-}xpik} \recind \word{ChL}{\mbox{-}xpek}{shadow}\label{jpec}
        \ex \wordng{ShL}{t\mbox{-}pik} \recind \word{ChL}{t\mbox{-}pek}{s/he returns hither}\label{tpic}
        \ex \wordng{ShL}{nikxoˀk} \recind \word{ChL}{nekxoˀk}{boy}
        \ex \wordng{ShL}{kiʃam} \recind \word{ChL}{ketʃam}{upwards}
        \ex \wordng{ShL}{nijåtsitʃ} \recind \word{ChL}{niåtsetʃ}{maize chicha}
    \end{xlist}
\z

The same correspondence is found in the plural suffix \intxt{\mbox{-}is} in some nouns; \citet[276--277]{AnG15} considers the vowel in question epenthetic.

\ea
    \begin{xlist}
        \ex \wordng{ShL}{jinkåp\mbox{-}ís} \recind \word{ChL}{inkåp\mbox{-}és}{years}
        \ex \wordng{ShL}{kotsxat\mbox{-}ís} \recind \word{ChL}{kotsxat\mbox{-}és}{lands}
    \end{xlist}
\z
\boolfalse{listing}

In this case, too, it is difficult to establish whether the sound correspondence in question results from only one post-Proto-Nivaĉle sound change or whether various sound changes with different directionalities operated in different Nivaĉle dialects. Note that in \REF{jpec} it is the variant with \intxt{e} that seems to be archaic, judging by the cognates in other Mataguayan languages, whereas in \REF{tpic} and in the plural suffix \intxt{\mbox{-}is} it is the variant with \intxt{i} that must represent a retention. The issue requires further investigation. 

\subsection{Sporadic vowel raising in Yita' Lhavos}\label{ni-yita-raising}

\citet[38]{AnG15} reports that in some specific words Yita' Lhavos shows a high vowel where other varieties have a mid one:

\booltrue{listing}
\ea
    \begin{xlist}
        \ex \wordng{YL}{tʃ’itʃ’í} \recind \word{ShL}{tʃ’etʃ’é}{parrot}
        \ex \wordng{YL}{kek͡lejtʃí} \recind \word{ShL}{kek͡lejtʃé}{bean}
        \ex \wordng{YL}{nìkxaké} \recind \word{ShL}{nèkxåké}{girl}
        \ex \wordng{YL}{ʃijå} \recind \word{ShL}{ʃejå}{bat} (example from \citnp{JS16})
        \ex \wordng{YL}{kutsxáˀt} \recind \word{ShL}{kotsxáˀt}{earth}
    \end{xlist}
\z
\boolfalse{listing}

\subsection{Realization of /ij/}\label{ni-iy-ii}
In their description of the Shichamnee Lhavos variety of Nivaĉle, \citet[73]{LC20} state that the rhyme \intxt{ij} is pronounced as \phonetic{iː}, as in \wordnl{nijxåj}{ropes, strings}, \wordnl{ʔantʃ’anjij}{listen to me!} (phonetically \phonetic{niːxɑj}, \phonetic{ʔantʃ’anjiː}). This may account for the fact that some of our sources, such as \citet{JS16}, often represent the sequence in question simply as \intxt{i}. In this book we use only the representation \intxt{ij}.

\subsection{Intervocalic ejectives}\label{ni-eject}
\citet[54]{AnG15} explicitly states that, at least in her data, ``the glottal stop can occur before all consonants except before another glottal stop or an ejective''. However, in \cits{LC20} description one often finds sequences of the type \intxt{ʔC’} corresponding to ejective consonants in other sources:

\booltrue{listing}
\ea
    \begin{xlist}
        \ex \wordng{ChL}{n\mbox{-}aʔp’u} \recind \word{other}{n\mbox{-}ap’u}{s/he licks}
        \ex \wordng{ChL}{ʔaʔp’ax} \recind \word{other}{ʔap’ax}{jararaca}
        \ex \wordng{ChL}{naʔp’uk} \recind \word{other}{nap’uk}{salty}
        \ex \wordng{ChL}{-p’iʔk’o} \recind \word{other}{-p’ik’o}{heel}
    \end{xlist}
\z
\boolfalse{listing}

We believe that [ʔ] is hardly phonological in such cases: its presence more likely reflects a difference in the relative timing of the articulatory gestures involved in the production of intervocalic ejective, whereby the obstruction of the airflow in the glottis initiates before the supraglottal constriction reaches its maximum. We know of no clear minimal pairs involving the purported \intxt{ʔC’} sequences and ejective stops.

\subsection{Progressive vowel assimilation}\label{ni-translar}
\largerpage[2]
\citet[10, 317]{LC20} note that the Pilcomayeño subdialect of Chishamnee Lhavos lacks the progressive translaryngeal vowel assimilation process, which is pervasive in the Central Paraguayan subdialect of Chishamnee Lhavos and has also been attested by \citet{AnG16b}, who worked with speakers of Chishaam Lhavos and Yita' Lhavos. For example, the imperfective suffix \intxt{\mbox{-}ʔin} is reported to surface as \intxt{\mbox{-}ʔen}, \intxt{\mbox{-}ʔan}, \intxt{\mbox{-}ʔån} when preceded, respectively, by an \intxt{e}, \intxt{a}, or \intxt{å} in some varieties of the language \citep[339--340]{AnG16b}, whereas the Pilcomayeño subdialect of Chishamnee Lhavos knows no such process.\footnote{Note that in addition to the progressive translaryngeal vowel assimilation, which operates across an underlying glottal stop, Nivaĉle also has a process of regressive vowel assimilation, which operates across an epenthetic glottal stop \citep[340–341]{AnG16b}. The latter process apparently occurs in all dialects, including Chishamnee Lhavos \citep[167–168]{LC20,NS87}.}

\booltrue{listing}
\begin{exe}
        \ex \wordng{ChL-Pi}{xaˀj-at’o-ʔin} \recind \word{other}{xaˀj-at’o-ʔon}{I yawned}
        \ex \wordng{ChL-Pi}{ji-jaɬp’o-ʔin} \recind \word{other}{ji-jaɬp’o-ʔon}{s/he/it drowned}
\end{exe}
\boolfalse{listing}

The Pilcomayeño subdialect of Chishamnee Lhavos also lacks the allomorphy pattern whereby the antipassive suffix \intxt{\mbox{-}xan} surfaces as \intxt{\mbox{-}xun} after the vowel \intxt{u} \citep[10]{LC20}.

\booltrue{listing}
\ea
    \begin{xlist}
        \ex \wordng{ChL-Pi}{xaj-uˀk͡lu-xan} \recind \word{other}{xaj-uˀk͡lu-xun}{I roast}
        \ex \wordng{ChL-Pi}{xaj-ak͡lapxu-xan} \recind \word{other}{xaj-ak͡lapxu-xun}{I pile, I stack}
    \end{xlist}
\z
\boolfalse{listing}
\fussy

\citet{NS87} documents only the allomophs with \intxt{u} in such cases, whereas \citet[48]{AF16} claims the assimilation is optional.
