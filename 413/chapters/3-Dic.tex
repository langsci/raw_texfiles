\chapter{Dictionary} \label{etymdic}
\largerpage

This chapter contains a list of Mataguayan cognate sets (and contact etymologies that may be confused for cognate sets). We start by presenting reliable lexical cognate sets with reflexes at least in one of Maká or Nivaĉle, and at least in one of Chorote or Wichí (\sectref{bonafide}). We then proceed to nominal derivational affixes (\sectref{derafn}), valence and spatial suffixes (\sectref{vss}), demonstratives (\sectref{demons}), inflectional prefixes (\sectref{infpr}), and inflectional suffixes (\sectref{infsu}). After that, we list cognate sets restricted to Maká and Nivaĉle (\sectref{mnonly}), and those restricted to Chorote--Wichí (\sectref{chwonly}). The next section is devoted to roots present only in Iyojwa’aja’ and Wichí (these have been likely borrowed from Wichí to Iyojwa’aja’; \sectref{wiijw}). Finally, we list several etymologies that reunite material which is unlikely to go back to Proto-Mataguayan, but has rather been diffused between Mataguayan languages by direct or indirect contact (\sectref{wander}).

\hspace*{-3.1pt}When applicable, we include information on uncertainties regarding the phonological or semantic reconstruction, irregularities in specific languages or dialects, forms that we suspect to be ill-transcribed in our sources, and sources of each datum. For nouns, we indicate the plural suffix (or the entire plural forms), whereas verbs are listed with a third-person realis prefix (see \sectref{notation} for more details). We seek to systematically include information on formal lookalikes in the Guaicuruan family (which may turn out to be cognate with the Mataguayan forms if the Macro-Guaicuruan hypothesis is confirmed) and on previous works where the cognate sets in question had been identified.

\setlength{\normalparindent}{\parindent}
\setlength{\parindent}{0pt}
% <<<<<<< HEAD
% ATTENTION: Diacritics on the following phonetic characters might have been lost during conversion: {'ɑ', 'ɛ', 'ʊ', 'ɪ', 'ɔ'}

\sloppy
\section{\emph{Bona fide} PM etymologies} \label{bonafide}
\begin{adjustwidth}{6mm}{0pt}
\setlength{\parindent}{0pt}
\PMlemma{{\intxt{*\mbox{-}ajeˀk \recind *\mbox{-}ajéˀk}; \wordnl{*\mbox{-}q\mbox{-}ájeˀk}{honeycomb} [1]}}

\wordng{Ni}{\mbox{-}ajeˀtʃ\plf{\mbox{-}ajtʃe\mbox{-}j}}; \intxt{\mbox{-}k\mbox{-}ajetʃ} \citep[68, 379]{JS16} {\sep} \wordng{PCh}{*\mbox{-}q\mbox{-}ájek} > \word{Ijw}{ʔin\mbox{-}k\mbox{-}ájik}{honey} \citep[108]{ND09}

\dicnote{\word{PM}{*ʔaqǻjeˀk}{wild honey} is obviously derived from this root.}%1

\gc{\word{Mocoví}{\mbox{-}ɪʔjaːk}{load; honeycomb} \citep{LLB-ASB-RR-14} is somewhat similar to the Mataguayan forms, but this may be accidental.}

\lit{\citealt{EN84}: 12 (\intxt{*k’ajɛk’})}

\PMlemma{{\intxt{*n\mbox{-}ap’u \recind *n\mbox{-}aɸ’u} (\intxt{\recind *\mbox{-}á\mbox{-} \recind *\mbox{-}ú}) [1]\gloss{to lick}}}

\wordng{Ni}{n\mbox{-}ap’u} [2] \citep[181]{JS16} {\sep} \wordng{PCh}{*[ʔi]<n>áp’uʔ} [3] > \wordng{Ijw}{[ʔi]nʲép’uw\mbox{-}eʔ / \mbox{-}náp’uw\mbox{-}eʔ}; \wordng{I’w}{\mbox{-}nápuʔ}, \intxt{\mbox{-}nápu\mbox{-}un}, \intxt{\mbox{-}nápu\mbox{-}ʔweʔ} [4]; \wordng{Mj}{[ʔi]n(ʲ)ép’oʔ / \mbox{-}náp’oʔ} [5] (\citealt{JC14a}; \citealt{ND09}: 102; \citealt{AG83}: 150, 204; \citealt{JC18}) {\sep} \wordng{PW}{*<n>ap’u} (\intxt{\recind *\mbox{-}á\mbox{-} \recind *\mbox{-}úh}) [1 3] > \wordng{LB}{nap’e}; \wordng{’Wk}{[ʔi]náp’uʔ} (\citealt{VN14}: 278; \citealt{JB09}: 52; \citealt{KC16}: 257)

\dicnote{The prosodic properties of the root cannot be established because the ’Weenhayek cognate is not attested without extra prefixes (the forms with prefixes are not revealing because in trisyllabic words the vowel of the peninitial syllable is lengthened in any case).}%1

\dicnote{\citet[27, 43]{LC20} attest the variant \intxt{n\mbox{-}aʔp’u}, where [ʔp’] is likely an allophone of /p’/.}%2

\dicnote{The cislocative prefix \intxt{*n\mbox{-}} has been fossilized as a part of the stem in Chorote and Wichí.}%3

\dicnote{The plain stop \intxt{p} in \cits{AG83} data of Iyo’awujwa’ must be a mistranscription.}%4

\dicnote{The lowering of unstressed \wordng{PCh}{*u} to \wordng{Mj}{o} is irregular.}%5

\gc{Obviously related to \word{Proto-Guaicuruan}{*\mbox{-}ap’i}{to lick, to suck} (\citealt{PVB13b}, \#81; cf. \citealt{PVB13a}: 304).}

\lit{\citealt{EN84}: 9 (\intxt{*nap’u}); \citealt{PVB13a}: 304 (\intxt{*\mbox{-}n\mbox{-}ap’u})}

\PMlemma{{\intxt{*\mbox{-}á(\mbox{-}jʰ)\mbox{-}xiʔ\pla{l}} [1]\gloss{mouth, door}}}

\wordng{Mk}{\mbox{-}e<xiʔ>\pl{l}}, (Towothli doculect) ‹\mbox{-}aihe› (\citealt{AG99}: 168; \citealt{RJH15}: 244) {\sep} \wordng{Ni}{\mbox{-}a<ʃi>\pl{k}} \citep[49]{JS16} {\sep} (?) \wordng{PCh}{*\mbox{-}á<ajʔ>\pla{is}} [2] > \wordng{I’w}{\mbox{-}áj\pl{is}}; \wordng{Mj}{\mbox{-}áajʔ\pl{is}} (\citealt{AG83}: 117; \citealt{JC18}) {\sep} \wordng{PW}{*\mbox{-}ɬ\mbox{-}áj\mbox{-}hi\pla{lʰ}} > \wordng{LB}{\mbox{-}ɬ\mbox{-}aç\mbox{-}i}\gloss{word} (\intxt{\mbox{-}ɬ}); \wordng{Vej}{\mbox{-}ɬ\mbox{-}aɲ̊\mbox{-}i} [3]; \wordng{’Wk}{\mbox{-}ɬ\mbox{-}áç\mbox{-}iʔ\pl{ɬ}}\gloss{oral cavity; language; cutting edge} (\citealt{VN14}: 191, 209; \citealt{VU74}: 65; \citealt{KC16}: 73)

\dicnote{This root is a compound of an unidentified element \intxt{*\mbox{-}á\mbox{-}} (as suggested by modern Maká and Nivaĉle) or \intxt{*\mbox{-}áj\mbox{-}} (as suggested by the Towothli doculect of Maká and Wichí) and \wordnl{*\mbox{-}xiʔ}{inside a recipient}. It is possible that the Wichí reflex continues a compound with a pluralized first element: \wordng{PM}{*\mbox{-}á\mbox{-}jʰ\mbox{-}xiʔ}.}%1

\dicnote{It is unclear whether the Chorote form belongs here: the expected reflex would be \intxt{**\mbox{-}áhiʔ} (\intxt{**\mbox{-}l}), with subsequent translaryngeal assimilation to \intxt{*áha} or \intxt{*áhe} in the contemporary varieties, not \intxt{*\mbox{-}á<hajʔ>\pla{is}}.}%2

\dicnote{This normalized form is based on the attestations \intxt{\mbox{-}ɬajhni} (\citealt{VU74}: 65) and \intxt{ɬahɲi} (\citealt{AFG067}: 213).}%3

\lit{\citealt{EN84}: 32 (\wordnl{*hlahni}{mouth}); \citealt{PVB02}: 142 (\wordnl{*ɬaxi}{door})}

\PMlemma{{\wordnl{*n\mbox{-}át}{to fall on its own}}}

\word{Ni}{n\mbox{-}at}{to fall (of ripe fruits)} \citep[183]{JS16} {\sep} \wordng{PW}{*<n>át} [1] > \word{’Wk}{nát}{to fall on the ground (e.g. of leaves)} \citep[258]{KC16}

\dicnote{The cislocative prefix \intxt{*n\mbox{-}} has been fossilized as a part of the stem in Wichí.}%1

\gc{\citet[320]{PVB13a} compares the verb with \word{Proto-Qom}{*\mbox{-}ʔot}{downwards}, which could be spurious.}

\lit{\citealt{PVB13a}: 320 (\intxt{*\mbox{-}(n)\mbox{-}ʌt})}

\PMlemma{{\wordnl{*\mbox{-}áwå(ʔ)}{flower}}}

\wordng{Ni}{\mbox{-}aβå} (\wordng{ChL\mbox{-}Py}{\mbox{-}åβå}) (\intxt{\mbox{-}s}) (\citealt{JS16}: 51; \citealt{LC20}: 73) {\sep} \wordng{PCh}{\third{*hl\mbox{-}áwoʔ}} [1] > I’w~\third{hl\mbox{-}áwo \recind hl\mbox{-}áwu\pl{l}} [2]; Mj~\third{hl\mbox{-}áwoʔ} (\citealt{AG83}: 146, 198; \citealt{JC18}) {\sep} \wordng{PW}{*\mbox{-}ɬ\mbox{-}áwo} [1] > \wordng{LB}{ɬ\mbox{-}awu}; \wordng{Vej}{ɬ\mbox{-}awo}; \wordng{’Wk}{\mbox{-}ɬ\mbox{-}áwoʔ} (\citealt{VN14}: 161; \citealt{VU74}: 65; \citealt{KC16}: 140, 234)

\dicnote{The raising of \wordng{PM}{*å} to \wordng{PCh/PW}{*o} is not known to be regular.}%1

\dicnote{The absence of a final \intxt{ʔ} in \cits{AG83} data of Iyo’awujwa’ must be a mistranscription.}%2

\gc{Obviously related to \wordng{Proto-Guaicuruan}{*\mbox{-}awo<qó>}\gloss{flower} (the shorter root is preserved in \word{Proto-Pilagá–Toba}{*\mbox{-}awó}{to bloom}) (\citealt{PVB13b}, \#179; cf. \citealt{PVB13a}: 310).}

\lit{\citealt{PVB13a}: 310 (\intxt{*\mbox{-}ʌwo}); \citealt{AnG15}: 254}

\PMlemma{{\wordnl{*\mbox{-}áʔ\pla{jʰ}}{fruit}}}

\wordng{Mk}{\third{ɬ\mbox{-}eʔ\pl{j}}} \citep[252]{AG99} {\sep} \wordng{Ni}{\mbox{-}aʔ\pl{j}} \citep[35]{JS16} {\sep} \wordng{PCh}{\third{*hl\mbox{-}áʔ\pla{jʰ}}} > \wordng{Ijw}{\third{\textit{hl\mbox{-}áʔ}}\pl{j<is>}} [1]; I’w~\third{(h)l\mbox{-}áʔ\pl{j}}; Mj~\third{hl\mbox{-}áʔ\pl{jh}} (\citealt{JC14b}: 77; \citealt{ND09}: 130; \citealt{AG83}: 145, 199; \citealt{JC18}) {\sep} \wordng{PW}{*\mbox{-}ɬ\mbox{-}áʔ\pla{jʰ}} > \wordng{LB}{\mbox{-}ɬ\mbox{-}aʔ\pl{j}}; \wordng{Vej}{\mbox{-}ɬ\mbox{-}a\mbox{-}j}; \wordng{’Wk}{\mbox{-}ɬ\mbox{-}áʔ\pl{ç}} (\citealt{VN14}: 65, 170; \citealt{VU74}: 65; \citealt{KC16}: 73, 230)

\dicnote{The Iyojwa’aja’ plural suffix is innovative.}%1

\gc{Obviously related to \word{Proto-Guaicuruan}{*\mbox{-}a}{fruit (\textit{suffix})}, \wordnl{*<eˀl>á}{fruit} (with a fossilized third-person prefix) (\citealt{PVB13b}, \#705, \#212; cf. \citealt{PVB13a}: 310).}

\lit{\citealt{PVB13a}: 310 (\intxt{*\mbox{-}aʔ}); \citealt{AnG15}: 254}

\PMlemma{{\wordnl{*\mbox{-}ǻɸe(ʔ)}{tooth}}}

Mk (Lengua doculect) ‹hiafué› \citep[456]{AD60} {\sep} \wordng{PCh}{*\mbox{-}ǻhweʔ\pla{jʰ}} > \wordng{I’w}{\mbox{-}áfʷeʔ\pl{j}}; \wordng{Mj}{\mbox{-}áhweʔ\pl{j}} (\citealt{AG83}: 117; \citealt{JC18})

\gc{Obviously related to \word{Proto-Guaicuruan}{*\mbox{-}owe}{tooth} (\citealt{PVB13b}, \#463).}

\PMlemma{{\textit{*n\mbox{-}ǻjin}\gloss{to go first}}}

\wordng{Mk}{[wa]<th>ajin} [1] \citep[363]{AG99} {\sep} \wordng{Ni}{n\mbox{-}åjin} \citep[215]{JS16} {\sep} \wordng{PCh}{*[ʔi]<n>ǻjin} [2] > \wordng{Ijw}{[ʔi]nʲáˀn / \mbox{-}náˀn} [3]; \wordng{I’w}{\mbox{-}nájin}; \wordng{Mj}{[ʔi]néjin / \mbox{-}nájin} (\citealt{JC14b}: 77, fn. 4; \citealt{ND09}: 102; \citealt{AG83}: 149; \citealt{JC18})

\dicnote{We have no explanation for the occurrence of the sequence \intxt{\mbox{-}th\mbox{-}} in Maká.}%1

\dicnote{The cislocative prefix \intxt{*n\mbox{-}} has been fossilized as a part of the stem in Chorote.}%2

\dicnote{The sequence \textit{*\mbox{-}ji\mbox{-}} was irregularly lost in Iyojwa’aja’.}%3

\PMlemma{{PM 1~\intxt{*h\mbox{-}åk}, 2~\intxt{*ɬ\mbox{-}äk}, \third{*[j]ik}, 1\textsc{irr}~\intxt{*j\mbox{-}ik}, 2\textsc{irr}~\intxt{*\textmd{∅}\mbox{-}ʔåk}, 3\textsc{irr}~\wordnl{*n\mbox{-}äk}{to go away}; \textsc{cisl} \wordnl{*n\mbox{-}äk}{to~come, to walk}}}

Mk~1~\intxt{h\mbox{-}ak}, 2~\intxt{ɬ\mbox{-}ak} [1], \third{ik}, 2\textsc{irr}~\intxt{∅\mbox{-}ak}, 3\textsc{irr}~\intxt{n\mbox{-}ak} [1]; \textit{n\mbox{-}ek} (\citealt{AG94}: 92; \citealt{AG99}: 227, 268) {\sep} Ni 1~\intxt{x\mbox{-}åk}, 2~\intxt{ɬ\mbox{-}åk} [1], \third{[j]itʃ}, 1\textsc{irr}~\intxt{j\mbox{-}itʃ}, 3\textsc{irr}~\intxt{n\mbox{-}åk} [1]; \textit{n\mbox{-}atʃ} (\citealt{JS16}: 152, 380) {\sep} PCh~1~\intxt{*∅\mbox{-}ʔǻk}, 2~\intxt{*hl\mbox{-}ék}, 1\textsc{irr}~\intxt{*j\mbox{-}ík}, 2\textsc{irr}~\intxt{*∅\mbox{-}ʔǻk}, 3\textsc{irr}~\intxt{*n\mbox{-}ék} [2] > Ijw~1~\intxt{ʔá\mbox{-}k}, 2~\intxt{hl\mbox{-}έk}, 1\textsc{irr}~\intxt{j\mbox{-}ík}, 2\textsc{irr}~\intxt{∅\mbox{-}ʔák}, 3\textsc{irr}~\intxt{(ʔi)n\mbox{-}έk}; I’w~1~\intxt{á\mbox{-}k \recind a\mbox{-}ék}; 2~\intxt{hl\mbox{-}ék}; Mj~1~\intxt{ʔa\mbox{-}ʔέk}, 2~\intxt{hl\mbox{-}έk} (\citealt{JC14b}: 100; \citealt{ND09}: 158; \citealt{AG83}: 103; \citealt{JC18}) {\sep} PW~2~\intxt{*ɬ\mbox{-}eq}, \third{[j]iq}; \textit{*n\mbox{-}eq} > LB~2~\textit{ɬ\mbox{-}eq}, \third{[j]iq}; \textit{n\mbox{-}eq}; \wordng{Vej}{[j]ijk \recind [j]ik \recind [j]ek} [3]; \textit{n\mbox{-}ek}; ’Wk~2~\intxt{ɬ\mbox{-}ek}, \third{[j]ik}; \textit{n\mbox{-}ek} (\citealt{VN14}: 145, 226; \citealt{VU74}: 68, 84; \citealt{MG-MELO15}: 38; \citealt{KC16}: 261, 544)

\dicnote{Maká and Nivaĉle point to \word{PM}{*ɬ\mbox{-}åk}{you go} and \wordnl{*n\mbox{-}åk}{that s/he go} rather than \wordnl{*ɬ\mbox{-}äk}{you go}, \wordnl{*n\mbox{-}äk}{that s/he go}, possibly due to analogy with the first-person form. The same allomorph of the root is also found in the irrealis paradigm (Mk~1~\intxt{h\mbox{-}ak}, 2~\intxt{∅\mbox{-}ak}, \third{n\mbox{-}ak}, 1+2~\intxt{xin\mbox{-}ak\mbox{-}kij}; \wordng{Ni}{\third{n\mbox{-}åk}}, 1+2~\intxt{ʃn\mbox{-}åk}, but 1~\intxt{jitʃ}, 2~\intxt{må}) and, in Nivaĉle only, in the first-person inclusive realis (1+2~\intxt{ʃn\mbox{-}åk}).}%1

\dicnote{In Chorote, the third-person realis of this verb is suppletive: \wordng{PCh}{*[j]ǻˀm} > \wordng{Ijw}{[j]áˀm}; \wordng{I’w}{[j]ém}; \wordng{Mj}{[j]éˀm}.}%2

\dicnote{The variation attested in Vejoz is probably due to the fact that /ji/ surfaces as [jɪ] in Wichí.}%3

\gc{Obviously related to \word{Proto-Guaicuruan}{*\mbox{-}eko \recind *\mbox{-}iko}{to go} (\citealt{PVB13b}, \#202; cf. \citealt{PVB13a}: 305).}

\lit{\citealt{PVB13a}: 305 (\intxt{*\mbox{-}ʌk \recind *\mbox{-}ek \recind *\mbox{-}uk})}

\PMlemma{{\wordnl{*[j]ǻm}{to arrive} (MN),\gloss{to go away} (Ch); \textsc{cisl} \wordnl{*n\mbox{-}ǻm}{to arrive here} (MN),\gloss{to come here} (ChW)}}

\wordng{Mk}{n\mbox{-}am} \citep[118]{AG99} {\sep} \wordng{Ni}{[j]am}; \intxt{n\mbox{-}am} (\wordng{ChL\mbox{-}Py}{n\mbox{-}åm}) (\citealt{JS16}: 43, 180; \citealt{LC20}: 236) {\sep} \word{PCh}{*[j]ǻm}{to go away.3\textsc{irr}} > \wordng{Ijw}{[j]áˀm}; \wordng{I’w}{[j]ém}; \wordng{Mj}{[j]ém} [1]; \textit{*<n>ǻm} [2]\gloss{to come here} > \wordng{Ijw}{náˀm}; \wordng{Mj}{nám} (\citealt{JC14b}: 77, fn. 4; \citealt{ND09}: 141, 158; \citealt{AG83}: 103; \citealt{GH94}; \citealt{JC18}) {\sep} \wordng{PW}{*<n>ǻm} [2] > \wordng{LB}{nom}; \wordng{Vej}{nåm}; \wordng{’Wk}{nǻm̥} (\citealt{VN14}: 145; \citealt{JB09}: 53; \citealt{VU74}: 68; \citealt{KC16}: 252)

\dicnote{\citet{JC18} documents this Manjui form as \textit{[j]éˀm}, which could be a mistranscription.}%1

\dicnote{The cislocative prefix \intxt{*n\mbox{-}} has been fossilized as a part of the stem in Chorote and Wichí.}%2

\lit{\citealt{AF16}: 306}

\PMlemma{{\wordnl{*\mbox{-}ǻ-mmi-ˀs\plf{*\mbox{-}lé-mmi-ts}}{small, thin} [1]}}

\word{Mk}{\mbox{-}a\mbox{-}mmi\mbox{-}ˀs\plf{\mbox{-}li\mbox{-}mmi\mbox{-}s}}{small} [2] \citep[247]{AG99} {\sep} \wordng{Ni}{\mbox{-}<ɬ>amis\mbox{-}tʃ’e} (\citealt{JS16}: 163) {\sep} \wordng{PW}{*\mbox{-}<ɬ>ǻms<aχ>\plf{*\mbox{-}léms<a>\mbox{-}s}} > \word{LB}{\mbox{-}ɬo(m)saχ}{small} [3], \wordnl{\mbox{-}ɬemsas}{small} [4]; \wordng{Vej}{\mbox{-}ɬamsah\plf{\mbox{-}lemsa\mbox{-}s}} [5]; \word{Guisnay}{\mbox{-}ɬåmsah\plf{\mbox{-}<le>lemsa\mbox{-}s}}{small} (\citealt{JB09}: 51; \citealt{VN14}: 355, 374, 386; \citealt{VU74}: 65; \citealt{MG-MELO15}: 63; \citealt{RL16}: 57)

\dicnote{This term is evidently derived from \word{PM}{*\mbox{-}ǻˀs}{son}, \wordnl{*\mbox{-}léts}{offspring} by means of the infix \intxt{*\mbox{-}mmi\mbox{-}}. The derivation model is still morphologically transparent in Maká, where the masculine form \intxt{\mbox{-}a\mbox{-}mmi\mbox{-}ˀs} is derived from \wordnl{\mbox{-}aˀs}{son}, feminine form \intxt{\mbox{-}asi\mbox{-}mmi\mbox{-}ʔ} is derived from \wordnl{\mbox{-}asiʔ}{daughter}, and the plural form \intxt{\mbox{-}li\mbox{-}mmi\mbox{-}ts} is derived from \wordnl{\mbox{-}lits}{children}.}%1

\dicnote{The preglottalized coda in Maká is attested in the New Testament (e.g. James 3:4).}%2

\dicnote{The Lower Bermejeño reflex is attested as \intxt{\mbox{-}ɬomsaχ} by \citet{JB09} and as \intxt{\mbox{-}ɬosaχ} by \citet{VN14}. The irregular loss of \intxt{*m} is also documented in the Rivadavia subdialect by \citet[127, 199]{JT09-th}.}%3

\dicnote{In Lower Bermejeño, \intxt{\mbox{-}ɬemsas} (with an irregular \intxt{ɬ} instead of the expected \intxt{*l}) no longer behaves as the plural form, judging by the examples given in \citet[355, 374]{VN14}.}%4

\dicnote{The Vejoz singular reflex is unexpectedly documented as \intxt{\mbox{-}ɬamsah} rather than \intxt{*\mbox{-}ɬåmsah}.}%5

\PMlemma{{\textit{*[t](’)ǻn} [1]\gloss{to shout}}}

(?) \word{Mk}{[t]’an}{to win} \citep[121]{AG99} {\sep} \wordng{Ni}{[t]ån} \citep[104]{JS16} {\sep} \wordng{PCh}{*[t]ǻn} > \wordng{Ijw}{[t]áˀn}; \word{I’w}{\mbox{-}án\mbox{-}ej}{to call}; \wordng{Mj}{[t]án} (\citealt{ND09}: 149; \citealt{AG83}: 121; \citealt{JC18}) {\sep} \wordng{PW}{*[t]’ǻn} > \wordng{LB}{[t]’on}; \wordng{Vej}{[t]’ån}; \wordng{’Wk}{[t]’ǻn̥} (\citealt{VN14}: 42; \citealt{VU74}: 78; \citealt{KC16}: 428)

\dicnote{Nivaĉle and Chorote point to \wordng{PM}{*\mbox{-}ǻn}, Wichí and Maká (if the Maká word belongs to this cognate set) to \intxt{*\mbox{-}ʔǻn}.}%1

\lit{\citealt{EN84}: 21 (\third{*j\mbox{-}t’ån})}

\PMlemma{{\wordnl{*\mbox{-}ǻniˀs}{stinger}}}

\wordng{Mk}{\third{ɬ\mbox{-}aniˀs\plf{ɬ\mbox{-}ansi\mbox{-}ts}}}; \wordnl{\mbox{-}ansi\mbox{-}ʔi}{to sting} \citep[247]{AG99} {\sep} Ni~\third{ɬ\mbox{-}ånis\pl{ik}} \citep[170]{JS16} {\sep} PCh~\third{*hl\mbox{-}ǻnis} > Ijw~\third{hl\mbox{-}ánis}; Mj~\third{hl\mbox{-}ánis} (\citealt{ND09}: 129; \citealt{JC18}) {\sep} (?) PW~\third{*ɬ\mbox{-}ǻˀni} [1] > ’Wk~\third{ɬ\mbox{-}ǻˀniʔ\pl{lis}} \citep[70]{KC16}

\dicnote{The preglottalized coda in PM is reconstructed based on the Maká reflex, as attested in the New Testament (1 Corinthians 15:56).}%1

\dicnote{It is not clear that the ’Weenhayek word belongs here (the expected reflex would be \intxt{*ɬ\mbox{-}ǻnis}).}%2

\gc{\word{Mocoví}{\mbox{-}aʔna}{needle, stinger} \citep{LLB-ASB-RR-14} and \word{Abipón}{\mbox{-}aana}{thorn, needle} \citep[11]{EN66} are somewhat similar to the Mataguayan forms, but this may be accidental. \citet{PVB13a} traces the Mocoví form back to \word{Proto-Qom}{*\mbox{-}qaná}{needle}.}

\PMlemma{{\textit{*\mbox{-}åp}, \third{*ˀ[j]ip} [1]\gloss{to cry}}}

\wordng{Mk}{\mbox{-}ap}, \third{ip} \citep[122]{AG99} {\sep} \wordng{Ni}{\mbox{-}ap} (\wordng{ChL\mbox{-}Py}{\mbox{-}åp}), \third{[j]ip} \citep[46]{JS16} {\sep} \word{PCh}{*[j]ǻp}{to cry, to make noise (of animals)} > \wordng{Ijw}{[j]áp}; \wordng{I’w/Mj}{[j]ép / \mbox{-}áp} (\citealt{ND09}: 158; \citealt{AG83}: 43, 121; \citealt{JC18}) {\sep} \word{PW}{*ˀ[j]ip}{to make noise (of animals)} > \word{LB}{ˀ[j]ip\mbox{-}ɬi}{to chirp}; \word{Vej}{[j]ip}{to chirp} [2]; \wordng{’Wk}{ˀ[j]ip} (\citealt{VN14}: 186; \citealt{VU74}: 84; \citealt{KC16}: 125)

\dicnote{This verb evidently presented the same alternation as \wordng{PM}{*\mbox{-}ʔå(ˀ)l}, \third{*ˀ[j]i(ˀ)l}\gloss{to die} (ChW). Chorote and Wichí generalized the allomorphs with \intxt{*å} and \intxt{*i}, respectively.}%1

\dicnote{The absence of a glottal stop or glottalization in the root-initial position in \cits{VU74} attestation of the Vejoz reflex could result from mistranscription.}%2

\gc{Possibly related to \word{Proto-Guaicuruan}{*\mbox{-}ap’a}{to suffer} (\citealt{PVB13b}, \#65; cf. \citealt{PVB13a}: 304).}

\lit{\citealt{PVB13a}: 304 (\intxt{*\mbox{-}ap})}

\PMlemma{{\wordnl{*[w]ǻpil}{to return thither} [1]}}

\word{Mk}{[w]apil}{to return from an unspecified place} \citep[296]{AG99} {\sep} Ni ChL\mbox{-}Pi \textit{[β]apek}, \wordng{ChL\mbox{-}Py}{[β]åpek} [2] (\citealt{JS16}: 178; \citealt{LC20}: 238) {\sep} \word{PCh}{*[j]ǻpil}{to~return} > \wordng{Ijw}{[j]ápiʔ / \mbox{-}ápiʔ} [3], \textit{[j]ápil\mbox{-}i / \mbox{-}ápil\mbox{-}i}; \wordng{I’w}{\mbox{-}ápil\mbox{-}met}, \textit{\mbox{-}ápil\mbox{-}i}; \wordng{Mj}{[j]épil / \mbox{-}ápil} (\citealt{ND09}: 158; \citealt{AG83}: 121; \citealt{JC18}) {\sep} \wordng{PW}{*[j]ǻpilʰ} > \word{LB}{[j]opiɬ}{to return to one’s place of origin}; \wordng{Vej}{[j]apil}; \wordng{’Wk}{[j]ǻpiɬ / [j]ǻpn̥\mbox{-}} (\citealt{VN14}: 308; \citealt{VU74}: 85; \citealt{KC16}: 516–519)

\dicnote{Obviously derived from \word{PM}{*[t]píl}{to return hither} and related to \word{Proto-Guaicuruan}{*\mbox{-}op’il}{to return} (\citealt{PVB13b}, \#443).}%1

\dicnote{The irregular vowel \intxt{e} in Nivaĉle is likely a dialectal development in Chishamnee Lhavos (the verb is not attested in Shichaam Lhavos), just like in \wordnl{[t]pek \recind [t]pik}{to return hither} \citep[498]{NS87}.}%2

\dicnote{The loss of the stem-final \intxt{*l} in Iyojwa’aja’ is irregular. Cf. the form \wordnl{[j]ápilʲ\mbox{-}a\mbox{-}hahme}{it returned again} \citep{JC14a}, where \textit{l} resurfaces before the punctive suffix \intxt{\mbox{-}a}.}%3

\lit{\citealt{RJH15}: 239}

\PMlemma{{\wordnl{*[j]ǻp’ä(ˀ)ɬ \recind *[j]ǻɸ’ä(ˀ)ɬ}{to burn}}}

\wordng{Ni}{[j]ap’aɬ \recind \mbox{-}åp’aɬ} \citep[47]{JS16} {\sep} \wordng{PCh}{*[j]ǻp’eɬ} > \word{Ijw}{[j]áp’iɬ / \mbox{-}áp’iɬ}{to throw in a large fire} \citep[158]{ND09} {\sep} \wordng{PW}{*[j]ǻp’eɬ} > \wordng{’Wk}{[j]ǻp’eɬ} \citep[517]{KC16}

\PMlemma{{\wordnl{*\mbox{-}åq\plf{*\mbox{-}qǻ\mbox{-}ts}}{food}}}

\wordng{Mk}{\mbox{-}aq\plf{\mbox{-}(a)qa\mbox{-}ts}} [1] \citep[124]{AG99} {\sep} \wordng{Ni}{\mbox{-}åk\plf{\mbox{-}kå\mbox{-}s}} \citep[348]{JS16} {\sep} \wordng{PCh}{*\mbox{-}ǻk\plf{*\mbox{-}qǻ\mbox{-}s}} > \wordng{Ijw}{\mbox{-}ák\plf{\mbox{-}ká\mbox{-}s}}; \wordng{I’w}{\mbox{-}ák\plf{\mbox{-}ák\mbox{-}es}} [2]; \wordng{Mj}{\mbox{-}ák\plf{\mbox{-}ká\mbox{-}s}} (\citealt{JC14b}: 77, 79, fn. 6; \citealt{ND09}: 129; \citealt{AG83}: 118; \citealt{JC18}) {\sep} \wordng{PW}{*\mbox{-}ɬ\mbox{-}åq} > \wordng{LB}{\mbox{-}ɬ\mbox{-}oq}; \wordng{Vej}{\mbox{-}ɬ\mbox{-}åk}; \wordng{’Wk}{\mbox{-}ɬ\mbox{-}åq} (\citealt{VN14}: 166; \citealt{VU74}: 65; \citealt{KC16}: 71); \word{PW}{*\mbox{-}qǻ<s>}{cultivated plant (possessed)} [3] > \wordng{LB}{\mbox{-}qo<s>}; \word{’Wk}{\mbox{-}qǻ<s>}{plant (possessed), vegetable} (\citealt{VN14}: 215; \citealt{KC16}: 82, 220)

\dicnote{Both \intxt{\mbox{-}qa\mbox{-}ts} and \intxt{\mbox{-}aq\mbox{-}ats} are reported as the plural forms of \intxt{\mbox{-}aq} in Maká. Only \intxt{\mbox{-}qa\mbox{-}ts} appears to be etymological; the variant \intxt{\mbox{-}aq\mbox{-}ats} must have been analogically based on the singular form \intxt{\mbox{-}aq}.}%1

\dicnote{The plural Iyo’awujwa’ form attested by \citet{AG83} is not etymological.}%2

\dicnote{\word{PW}{*\mbox{-}qǻs}{cultivated plant (possessed)} is a phonologically regular (but semantically shifted) reflex of \word{PM}{*\mbox{-}qǻ\mbox{-}ts}{food (\intxt{plural})}; the erstwhile plural suffix is no longer segmentable.}%3

\gc{Obviously related to \word{Proto-Southern Guaicuruan}{*\mbox{-}oq}{food}, with reflexes in all daughter languages, including \wordng{Mocoví}{\mbox{-}oq} \citep{LLB-ASB-RR-14}, \wordng{Toba–Qom}{\mbox{-}oq} \citep[2]{ASB-LLB-13}, \wordng{Pilagá 3~}{hal\mbox{-}oq} \citep[31]{AV01}, \wordng{Abipón}{\mbox{-}ak} \citep[27]{EN66}.}

\lit{\citealt{LC-VG-07}: 15}

\PMlemma{{\wordnl{*\mbox{-}ǻˀs}{son}}}

\wordng{Mk}{\mbox{-}aˀs} [1] \citep[128]{AG99} {\sep} \wordng{Ni}{\mbox{-}åˀs} (\citealt{AnG15}: 36; \citealt{JS16}: 46) {\sep} \wordng{PCh}{*\mbox{-}ǻs} > \wordng{Ijw/I’w/Mj}{\mbox{-}ás} (\citealt{JC14b}: 94; \citealt{ND09}: 129; \citealt{AG83}: 122; \citealt{JC18}) {\sep} \wordng{PW}{*\mbox{-}ɬ\mbox{-}ǻs} > \wordng{LB}{\mbox{-}ɬ\mbox{-}os}; \wordng{Vej}{\mbox{-}ɬ\mbox{-}ås}; \wordng{’Wk}{\mbox{-}ɬ\mbox{-}ǻs} (\citealt{VN14}: 166; \citealt{VU74}: 65; \citealt{KC16}: 71, 400)

\dicnote{The preglottalized coda in Maká is attested in the New Testament (e.g. Matthew 1:7) as well as in \citet[62]{JB81}.}%1


\gc{\citet[312]{PVB13a} notes the similarity with \wordng{Proto-Guaicuruan}{*\mbox{-}etʃ’e\mbox{-}tʃi\mbox{-}k} (male), \intxt{*\mbox{-}etʃ’j\mbox{-}o} (female)\gloss{orphan; stepchild}, which could be spurious.}

\lit{\citealt{RJH15}: 240; \citealt{PVB13a}: 312 (\intxt{*\mbox{-}ʌs})}

\PMlemma{{\intxt{*\mbox{-}ǻseʔ} [1]\gloss{daughter}} \label{dic-aose}}

\wordng{Mk}{\mbox{-}asiʔ\pl{j}} [2] \citep[128]{AG99} {\sep} \wordng{Ni}{\mbox{-}åse} \citep[213]{JS16} {\sep} \wordng{PCh}{*\mbox{-}ǻseʔ} > \wordng{Ijw/I’w/Mj}{\mbox{-}áxseʔ} (\citealt{JC14b}: 79, fn. 7; \citealt{ND09}: 129; \citealt{AG83}: 124; \citealt{JC18}) {\sep} \wordng{PW}{*\mbox{-}ɬ\mbox{-}ǻse} > \wordng{LB}{\mbox{-}ɬ\mbox{-}ose}; \wordng{Vej}{\mbox{-}ɬ\mbox{-}åse}; \wordng{’Wk}{\mbox{-}ɬ\mbox{-}ǻseʔ} (\citealt{VN14}: 166; \citealt{VU74}: 65; \citealt{KC16}: 71)

\dicnote{The root is obviously derived from \word{PM}{*\mbox{-}ǻˀs}{son} by means of the non-productive feminine suffix \intxt{*\mbox{-}eʔ}.}%1

\dicnote{Maká has innovated in having a plural form of this noun; all other languages point to a suppletive plural \wordnl{*\mbox{-}léts}{offspring (sons and/or daughters)}.}%2

\lit{\citealt{RJH15}: 240; \citealt{EN84}: 11 (\intxt{*åhsɛ}, \third{*hl\mbox{-}ǻsɛ}); \citealt{PVB13a}: 312 (\intxt{*\mbox{-}ʌs\mbox{-}eʔ})}

\PMlemma{{\wordnl{*[n]åt \recind *[n]ǻt}{to bleed}}}

\wordng{Mk}{[n]at\mbox{-}xuʔ} [1] \citep[132]{AG99} {\sep} \wordng{Ni}{[n]åt} \citep[201]{JS16} {\sep} \wordng{PCh}{*<n>ǻt\mbox{-}} > \wordng{Mj}{náht\mbox{-}ijʔ}, \textsc{caus}~\intxt{[ʔi]n(ʲ)éht\mbox{-}it / \mbox{-}náht\mbox{-}it} {\sep} \wordng{PW}{*<n>åt\mbox{-} \recind *<n>ǻt\mbox{-}} > \word{Vej}{nåt\mbox{-}ɬi}{to bleed (of nose)} \citep[64]{RL16}

\dicnote{\word{Maká}{\mbox{-}ʔathi\mbox{-}ts}{blood}, \wordnl{[t]’athi\mbox{-}j}{to menstruate} \citep[131]{AG99} hardly belong here, since the stem-initial glottal stop lacks any correspondence in Manjui and Vejoz.}%1

\gc{\citet[309]{PVB13a} compares this suffix to \word{Proto-Southern Guaicuruan}{*\mbox{-}ʔet’otá}{vein} (\citealt{PVB13b}, \#684). We suggest that it could be compared to \word{Proto-Guaicuruan}{*\mbox{-}awot}{blood} (\citealt{PVB13b}, \#180) instead.}

\lit{\citealt{PVB13a}: 309 (\intxt{*\mbox{-}ʌt’})}

\PMlemma{{\wordnl{*\mbox{-}ǻˀt\plf{*\mbox{-}ǻt\mbox{-}its}}{drink}}}

\wordng{Ni}{\mbox{-}åˀt\plf{\mbox{-}åt\mbox{-}is}} \citep[356]{JS16} {\sep} \wordng{PCh}{*\mbox{-}át\pla{es}} > \wordng{Ijw}{\mbox{-}át}; \wordng{Mj}{\mbox{-}át\pl{es}} (\citealt{ND09}: 129; \citealt{JC18}) {\sep} \wordng{PW}{*\mbox{-}ɬ\mbox{-}ǻt} > \wordng{LB}{\mbox{-}ɬ\mbox{-}ot}; \wordng{Vej}{\mbox{-}ɬ\mbox{-}åt}; \wordng{’Wk}{\mbox{-}ɬ\mbox{-}ǻt} (\citealt{VN14}: 213; \citealt{VU74}: 66; \citealt{KC16}: 71)

\gc{\citet[300]{PVB13a} notes the similarity with \wordng{Proto-Guaicuruan}{*\mbox{-}Vtá\mbox{-}qa}\gloss{(alcoholic) drink} (\citealt{PVB13b}, \#611) and attributes it to language contact.}

\rej{\citet[46]{EN84} compares \word{Ni}{\mbox{-}åˀt}{drink} to the reflexes of \word{PM}{*ʔat’e(ˀ)(t)s \recind *ʔat’ä(ˀ)(t)s}{\textit{aloja} drink}.}

\PMlemma{{\wordnl{*[j]ǻte(ˀ)χ}{to be fat}}}

\wordng{Ni}{[j]åtex} \citep[389]{JS16} {\sep} \wordng{PCh}{*[j]ǻtah} > \wordng{Ijw}{[j]áta}; \wordng{I’w}{\mbox{-}átah}; \wordng{Mj}{[j]éta / \mbox{-}áta} (\citealt{ND09}: 158; \citealt{AG83}: 122; \citealt{JC18}) {\sep} \wordng{PW}{*[j]ǻtaχ} > \wordng{LB}{[j]otaχ}; \wordng{Vej}{[j]atah} [1]; \wordng{’Wk}{[j]ǻtax} (\citealt{VN14}: 224, 252; \citealt{VU74}: 83; \citealt{KC16}: 519)

\dicnote{The Vejoz form is likely mistranscribed in \citet[83]{VU74}; the expected reflex would be \textit{*[j]åtah}.}%1

\gc{Likely related to \word{Proto-Guaicuruan}{*\mbox{-}ot’jáqa}{to be fat} (\citealt{PVB13b}, \#454; cf. \citealt{PVB13a}: 308).}

\lit{\citealt{EN84}: 44 (\intxt{*(ja)åtha}); \citealt{PVB02}: 143 (\intxt{*\mbox{-}ʌtax}); \citealt{PVB13a}: 308 (\intxt{*\mbox{-}ʌtah})}

\PMlemma{{\wordnl{*\mbox{-}åˀw\mbox{-}\APPL}{to be} [1]}}

Mk~1~\intxt{h\mbox{-}aˀw\mbox{-}\APPL}, 2~\intxt{ɬ\mbox{-}aˀw\mbox{-}\APPL}, \textsc{1+2} \intxt{xu\mbox{-}uˀw\mbox{-}\APPL\mbox{-}kii}, 3\textsc{irr}~\intxt{n\mbox{-}aˀw\mbox{-}\APPL}, \textsc{1+2irr} \textit{xina\mbox{-}ˀw\mbox{-}\APPL\mbox{-}kii} (\citealt{AG94}: 92; \citealt{AG99}: 359) {\sep} Ni 1~\intxt{x\mbox{-}åˀβ\mbox{-}\APPL}, 2~\intxt{ɬ\mbox{-}åˀβ\mbox{-}\APPL}, \third{[j]i\mbox{-}\APPL}, 1+2~\intxt{ʃn\mbox{-}åˀβ\mbox{-}\APPL}, 1\textsc{irr}~\intxt{j\mbox{-}i\mbox{-}\APPL}, 3\textsc{irr}~\intxt{n\mbox{-}åˀβ\mbox{-}\APPL} (\citealt{AF16}: 146; \citealt{JS16}: 46) {\sep} PCh~1+2~\intxt{*ʔåw\mbox{-} \recind *ʔåw\mbox{-}} > \wordng{Ijw}{ʔáw<ak>}, \textsc{irr}~\intxt{ʔíwʲ<ek>}; I’w~1+2~\intxt{aw\mbox{-}áh}; Mj~1+2~\intxt{ʔáw\mbox{-}ah} (\citealt{JC14a}; \citealt{ND09}: 160; \citealt{AG83}: 103; \citealt{JC18})

\dicnote{This is a suppletive allomorph of the root \intxt{*\mbox{-}é\mbox{-} / *[j]í\mbox{-}}. Its distribution in Chorote (first person inclusive only) appears to be the original one, whereas in Maká and Nivaĉle it replaced the original allomorph \intxt{*\mbox{-}é\mbox{-}} throughout the paradigm.}%1

\PMlemma{{\wordnl{*n\mbox{-}åχ}{to end up}}}

\wordng{Mk}{n\mbox{-}aχ} \citep[128]{AG99} {\sep} \wordng{Ni}{n\mbox{-}åx} \citep[199]{JS16} {\sep} \wordng{PCh}{*<n>óhw\mbox{-}\APPL} > \word{Ijw}{<n>ɔ́hw\mbox{-}iʔ}{to be empty, to dry out}, \wordnl{<n>ɔ́hw\mbox{-}e} {to gather}, \wordnl{<n>ɔ́ʔw\mbox{-}eʔ}{to end up}; \wordng{I’w}{<n>ófʷ\mbox{-}ik}; \word{Mj}{<n>ɔ́hw\mbox{-}ijʔ}{to end up}, \wordnl{<n>ɔ́hw\mbox{-}e}{to be ready}, \wordnl{<n>ɔ́hʔw\mbox{-}eʔ}{to melt} (\citealt{JC14b}: 85; \citealt{ND09}: 141; \citealt{AG83}: 151; \citealt{JC18}) {\sep} \wordng{PW}{*<n>oxʷ} > \wordng{LB}{<n>ufʷ}; \wordng{Vej}{<n>oh}; \wordng{’Wk}{<n>oxʷ} (\citealt{VN14}: 272, 357; \citealt{VU74}: 68; \citealt{KC16}: 274)

\PMlemma{{\wordnl{*[j]ä́n}{to put}}}

\wordng{Mk}{[j]en\mbox{-}\APPL} (\citealt{AG99}: 153–154) {\sep} \wordng{Ni}{[j]an} \citep[105]{JS16} {\sep} \wordng{PCh}{*[j]én} > \wordng{Ijw}{[j]ín\mbox{-}{\APPL} / \mbox{-}έn\mbox{-}\APPL}; \wordng{I’w}{\mbox{-}én\mbox{-}\APPL}, \textit{\mbox{-}án}; \wordng{Mj}{[j]ín / \mbox{-}έn \recind \mbox{-}áin \recind \mbox{-}ä́in} [1] (\citealt{ND09}: 159; \citealt{AG83}: 126–127, 216; \citealt{JC18}) {\sep} \word{PW}{*[j]én}{to put a snare} > \word{LB}{[j]en}{to~fish}; \wordng{’Wk}{[j]én̥} (\citealt{VN14}: 226; \citealt{KC16}: 532)

\dicnote{In the Jlimnájnas subdialect of Manjui, [ai̯] \recind [æi̯] are allophones of /e/ before a sonorant.}%1

\gc{Obviously related to \word{Proto-Guaicuruan}{*\mbox{-}a(ˀ)n}{to put} (\citealt{PVB13b}, \#49; cf. \citealt{PVB13a}: 304).}

\lit{\citealt{PVB13a}: 304 (\intxt{*\mbox{-}an})}

\PMlemma{{\wordnl{{*\mbox{-}äɸ}\plf{*\mbox{-}ɸä́\mbox{-}ts}}{wing}}}

\wordng{Mk}{\third{ɬ\mbox{-}ef}\plf{ɬe\mbox{-}fe\mbox{-}ts}} \citep[249]{AG99} {\sep} \wordng{Ni}{\mbox{-}aɸ}, \textit{\mbox{-}<a>ɸa\mbox{-}s}\gloss{wing, feather} (\citealt{JS16}: 39, 162) {\sep} \wordng{PCh}{*\mbox{-}hw<és>\pla{is}} [1] > \wordng{Ijw}{\mbox{-}hwέs\pl{is}}; \wordng{I’w}{\mbox{-}fʷés\pl{is}}; \wordng{Mj}{\mbox{-}hwés} (\citealt{ND09}: 120; \citealt{AG83}: 129; \citealt{JC18}) {\sep} \wordng{PW}{*\mbox{-}ɬ\mbox{-}exʷ\pla{ís}} [2 3] > \wordng{LB}{\mbox{-}ɬ\mbox{-}efʷ\pl{is}}; \wordng{Vej}{\mbox{-}hʷ<is>\pl{eɬ}} [1]; \wordng{’Wk}{\mbox{-}ɬ\mbox{-}exʷ} (\intxt{\mbox{-}ís}) (\citealt{JB09}: 50; \citealt{VU74}: 59; \citealt{MG-MELO15}: 60; \citealt{KC16}: 73, 235)

\dicnote{In Chorote and Vejoz, the plural form of PM has been reanalyzed as a singular one, with the erstwhile plural suffix being reinterpreted as a part of the root.}%1

\dicnote{The plural suffix \intxt{\mbox{-}ís}, found in Wichí, is non-etymological: in all other languages, its vowel is a copy of the root vowel.}%2

\dicnote{\citet[56, 58]{RL16} documents the form \intxt{\mbox{-}ɬ\mbox{-}ahʷ\pl{is}} alongside \intxt{\mbox{-}ɬ\mbox{-}ehʷ\pl{is}}, but does not indicate whether it is representative of Vejoz or Guisnay. If it turns out to be a Guisnay form, it could be a Nivaĉle borrowing.}%3

\gc{Possibly related to \word{Proto-Guaicuruan}{*\mbox{-}aˀwá}{wing} (\citealt{PVB13b}, \#182; cf. \citealt{PVB13a}: 309).}

\lit{\citealt{EN84}: 27 (\intxt{*hlahw}); \citealt{PVB13a}: 309 (\intxt{*ɬ\mbox{-}ahʷ})}

\PMlemma{{\wordnl{*\mbox{-}ä́ˀj\plf{*\mbox{-}ä́j\mbox{-}its}}{\textit{yica} bag}}}

\wordng{Ni}{\mbox{-}aˀj\plf{\mbox{-}aj\mbox{-}is}} \citep[35]{JS16} {\sep} \wordng{PCh}{*\mbox{-}éjʔ\pla{is}} > \wordng{Ijw}{\mbox{-}έʔ}; \wordng{I’w}{\mbox{-}éj\pl{is}}; \wordng{Mj}{\mbox{-}έjʔ\pl{is}} (\citealt{ND09}: 131; \citealt{AG83}: 125; \citealt{JC18}) {\sep} \wordng{PW}{*\mbox{-}ɬ\mbox{-}éj\pla{is}} > \wordng{LB}{\mbox{-}ɬ\mbox{-}ej}; \wordng{’Wk}{\mbox{-}ɬ\mbox{-}éjʔ\pl{is}} (\citealt{VN14}: 174; \citealt{KC16}: 74)

\largerpage[2]
\empr{\citet[306]{AF16} notes the similarity with the Enlhet–Enenlhet term for\gloss{\textit{yica} bag}: \wordng{Enlhet}{aːjenˀ}, \wordng{Enenlhet-Toba}{ajenˀ}, \wordng{Enxet}{aːjen} (\citealt{EU-HK-97}: 12; \citealt{EU-HK-MR-03}: 304; \citealt{JE21}: 704).}

\lit{\citealt{AF16}: 306}\clearpage

\PMlemma{{\wordnl{*\mbox{-}e\plf{*\mbox{-}é\mbox{-}l}}{thorn}}}

\wordng{Mk}{\third{ɬ\mbox{-}iʔ}}, \textit{<t>iʔ} [1] \citep[341]{AG99} {\sep} \wordng{Ni}{\mbox{-}eʔ\pl{k}} (\citealt{JS16}: 123, 355) {\sep} PCh~3~\textit{*hl\mbox{-}éʔ\pla{l}} > Ijw~3\textsc{pl}~\textit{hl\mbox{-}έ\mbox{-}ˀl} [2]; Mj~\third{hl\mbox{-}έʔ\pl{l}} (\citealt{ND-SF-JG-00}: 74; \citealt{JC18}) {\sep} \wordng{PW}{*\mbox{-}ɬ\mbox{-}e} > (?) \wordng{LB}{\mbox{-}ɬ\mbox{-}e}\gloss{fishbone} [3], ’Wk~\third{ɬ\mbox{-}eʔ}, \textit{ɬ\mbox{-}é\mbox{-}ç} [4] (\citealt{VN14}: 170; \citealt{KC16}: 235)

\dicnote{The origin of the variant \intxt{tiʔ} in Maká is unclear. The alternation \intxt{t\mbox{-}/ɬ\mbox{-}} occurs at the left boundary of the stem in multiple Maká verbs of the so-called 7\textsuperscript{th} conjugation, but in that case it seems to continue \wordng{PM}{*t\mbox{-}}.}%1

\dicnote{\citet[130]{ND09} documents \textit{hlέɬ} instead, which could be a mistranscription.}%2

\dicnote{\wordng{LB}{ɬ\mbox{-}e} is attested only in the example \wordnl{ˀwahat ɬe}{fishbone} \citep[170]{VN14}. Despite the semantic divergence, it likely belongs to the cognate set under consideration; note that in Spanish both meanings (\enquote{thorn} and\gloss{fishbone}) are colexified as \intxt{espina}, which could also be the case in Lower Bermejeño.}%3

\largerpage
\dicnote{The plural suffix attested in ’Weenhayek does not correspond to what is found in Nivaĉle and Manjui.}%4

\gc{Obviously related to \word{Proto-Guaicuruan}{*\mbox{-}<ʔeˀl>é}{thorn}, with a fossilized third-person prefix (\citealt{PVB13b}, \#671).}

\PMlemma{{\intxt{*\mbox{-}é\mbox{-}}\textsc{appl,} \textsc{3/1irr} \intxt{*[j]í\mbox{-}\APPL}\gloss{to be} [1]}}

\wordng{Mk}{\third{i<ˀw>\mbox{-}}\APPL} [2] (\citealt{AG94}: 92; \citealt{AG99}: 359) {\sep} Ni~\third{[j]i\mbox{-}\APPL}, 1\textsc{irr}~\intxt{j\mbox{-}i\mbox{-}\APPL} (\citealt{AF16}: 146; \citealt{JS16}: 46) {\sep} PCh~1~\intxt{*ʔa\mbox{-}ʔé<j>ʔ}, 2~\intxt{*hl\mbox{-}é<j>ʔ}, \third{*[j]íʔ}, 1\textsc{irr}~\intxt{*j\mbox{-}éʔ}, 2\textsc{irr}~\textit{*ʔa\mbox{-}ʔé<j>ʔ}, 3\textsc{irr}~\intxt{*n\mbox{-}é<j>ʔ} [3] > Ijw~1~\intxt{ʔáʔ} [1], 2~\intxt{hl\mbox{-}έʔ}, \third{j\mbox{-}íʔ} [4], 1\textsc{irr}~\intxt{j\mbox{-}íʔ}, 2\textsc{irr}~\intxt{∅\mbox{-}ʔáʔ}, 3\textsc{irr}~\textit{n\mbox{-}έʔ \recind ʔinέʔ}; I’w~1~\intxt{∅\mbox{-}éj}, 2~\intxt{hl\mbox{-}éj}, \third{j\mbox{-}í}; Mj~1~\intxt{ʔa\mbox{-}ʔέjʔ}, 2~\intxt{hl\mbox{-}έjʔ}, \third{[j]íʔ}, 1\textsc{irr}~\intxt{j\mbox{-}íʔ}, 2\textsc{irr}~\intxt{ʔa\mbox{-}ʔέjʔ}, 3\textsc{irr}~\intxt{n\mbox{-}έjʔ} (\citealt{JC14a}; \citealt{ND09}: 160; \citealt{AG83}: 103; \citealt{JC18}) {\sep} PW~2~\intxt{*ɬ\mbox{-}é\mbox{-}\APPL}, \third{*ʔí\mbox{-}\APPL}, 3\textsc{irr}~\intxt{*n\mbox{-}é\mbox{-}\APPL} > LB~2~\intxt{ɬ\mbox{-}é\mbox{-}}\textsc{appl,}~\third{ʔi\mbox{-}\APPL}; Vej~\third{ʔi\mbox{-}}; ’Wk~2~\textit{ɬ\mbox{-}é\mbox{-}\APPL}, \third{ʔí\mbox{-}\APPL}, 3\textsc{irr}~\intxt{n\mbox{-}é\mbox{-}\APPL} (\citealt{VN14}: 226, 276; \citealt{VU74}: 60; \citealt{KC16}: 21–22)

\dicnote{In Maká, Nivaĉle, and (in the first person inclusive) in Chorote, this root alternates with its suppletive allomorph \textit{*\mbox{-}åˀw\mbox{-}}, which has replaced \textit{*\mbox{-}é(j)\mbox{-}} in the former two languages in the entire paradigm.}%1

\dicnote{The element \intxt{\mbox{-}ˀw\mbox{-}} in Maká is taken through intraparadigmatic analogy from the suppletive allomorph \textit{\mbox{-}aˀw\mbox{-}}. The preglottalization is attested in the New Testament (e.g. Mark 1:30).}%2

\dicnote{The allomorph \textit{*\mbox{-}éj} (instead of the expected \textit{**\mbox{-}é}) is seen in Chorote first\mbox{-} and second-person realis as well as in second\mbox{-} and third-person irrealis.}%3

\dicnote{For some speakers of Iyojwa’aja’, \third{[j]íʔ} behaves as /jé/, and for others as /jéj/. Both representations are unexpected.}%4

\PMlemma{{\textit{*\mbox{-}éj\pla{its}}\gloss{name}}}

\wordng{Mk}{\mbox{-}ij\pl{its}} \citep[190]{AG99} {\sep} \wordng{Ni}{\mbox{-}ej\pl{is}} \citep[345]{JS16} {\sep} \wordng{PCh}{*\mbox{-}éjʔ\pla{is}} > \wordng{Ijw}{\mbox{-}έʔ\pl{jis}}; \wordng{I’w}{\mbox{-}éj} [1]; \wordng{Mj}{\mbox{-}έjʔ\pl{is}} (\citealt{JC14b}: 88; \citealt{ND09}: 131; \citealt{AG83}: 125; \citealt{JC18}) {\sep} \wordng{PW}{*\mbox{-}ɬ\mbox{-}éj\pla{is}} > \wordng{LB}{\mbox{-}ɬ\mbox{-}ej\pl{is}}; \wordng{Vej}{\mbox{-}ɬ\mbox{-}ej}; \wordng{’Wk}{\mbox{-}ɬ\mbox{-}éjʔ\pl{is}} (\citealt{VN14}: 166, 394; \citealt{VU74}: 66; \citealt{MG-MELO15}: 66; \citealt{AFG067}: 220; \citealt{KC16}: 74)

\dicnote{The absence of a final \intxt{ʔ} in \cits{AG83} data of Iyo’awujwa’ must be a mistranscription.}%1

\dicnote{Likely related to \word{Proto-Guaicuruan}{*\mbox{-}ej}{to name, to call} (\citealt{PVB13b}, \#197).}

\PMlemma{{\textit{*[j]ékɸaˀx} [1]\gloss{to bite}}}

\wordng{Mk}{[j]ikfeˀx} [1] \citep[195]{AG99} {\sep} \wordng{PCh}{*[j]ókwah} [2] > \wordng{Ijw}{[j]ókʲe}; \wordng{I’w}{\mbox{-}óka}; \wordng{Mj}{[j]óka} (\citealt{ND09}: 161; \citealt{AG83}: 152; \citealt{JC18}) {\sep} \wordng{PW}{*[j]ókʷaχ} [2] > \wordng{LB}{[j]ukʷaχ}; \wordng{Vej}{[j]okʷah}; \wordng{’Wk}{[j]ókax} (\citealt{VN14}: 148; \citealt{VU74}: 84; \citealt{KC16}: 550)

\dicnote{The preglottalized coda in PM is reconstructed based on the Maká reflex, as attested in the New Testament (e.g. Revelations 16:10).}%1

\dicnote{\sound{PM}{*e} was apparently rounded to \intxt{*o} in PCh/PW before a \intxt{*kɸ} > \sound{PCh/PW}{*kʷ}. It may have been a regular sound change.}%2

\dicnote{Likely related to \word{Proto-Guaicuruan}{*\mbox{-}ewak}{to bite} (\citealt{PVB13b}, \#240).}

\PMlemma{{\wordnl{*\mbox{-}ɸájiˀx}{right (side)}}}

\wordng{Mk}{\mbox{-}fejiˀx} [1], \intxt{\mbox{-}fejix\mbox{-}ets}\gloss{left, left hand} \citep[174]{AG99} {\sep} \wordng{Ni}{\mbox{-}ɸajiˀʃ}, \textit{\mbox{-}ɸajiʃ\mbox{-}ik} \citep[131]{JS16} {\sep} \wordng{PCh}{*\mbox{-}hwíjah} [2] > \wordng{Ijw}{\mbox{-}hwéje}; \wordng{I’w}{\mbox{-}fʷéje\pl{j}} \textit{\recind \mbox{-}fʷéji}; \wordng{Mj}{\mbox{-}hwíji} (\citealt{ND09}: 120; \citealt{AG83}: 129, 194; \citealt{JC18})

\dicnote{The preglottalized coda in Maká is attested in the New Testament (e.g. Mark 15:27).}%1

\dicnote{Chorote shows an irregular metathesis.}%2

\gc{Possibly related to \word{Proto-Qom}{*\mbox{-}ojik}{right} \citep[309]{PVB13a}.}

\lit{\citealt{PVB02}: 143 (\intxt{*\mbox{-}xʷejix}); \citealt{PVB13a}: 309 (\intxt{*\mbox{-}hʷejih})\gloss{left/right}}

\PMlemma{{\intxt{*ɸajXoʔ\plf{*ɸajXó\mbox{-}l}} / \wordnl{*\mbox{-}ɸájXoʔ\pla{l}}{charcoal, ember}}}

\wordng{Ni}{ɸajxoʔ / \mbox{-}ɸajxo(ʔ)\pl{k}} \citep[129]{JS16} {\sep} \wordng{PCh}{*hwa(h)jó\mbox{-}keʔ} [1] > \wordng{I’w}{fʷajó\mbox{-}kiʔ}, \wordng{Mj}{hwajó\mbox{-}kiʔ} [1] (\citealt{AG83}: 128; \citealt{GH94}) {\sep} \wordng{PW}{*\mbox{xʷijho(ʔ)}\plf{*xʷijhó\mbox{-}lʰ} / *\mbox{-}xʷíjho\pla{lʰ}} [2] > \wordng{LB}{fʷiçu(ʔ)\pl{ɬ}} [3]; \wordng{Vej}{hʷiɲ̊o\pl{ɬ}} [4]; \wordng{’Wk}{xʷiçoʔ\plf{xʷiçó\mbox{-}ɬ} / \mbox{-}xʷíçoʔ\pl{ɬ}} (\citealt{VN14}: 53; \citealt{MG-MELO15}: 48; \citealt{KC16}: 61, 173)

\dicnote{The Iyo’awujwa’ and Manjui reflex has \textit{\mbox{-}j\mbox{-}} instead of the expected \textit{*\mbox{-}hj\mbox{-}}. It is unclear whether the irregular loss of \intxt{*h} occurred in Proto-Chorote or in Proto-Iyo’awujwa’–Manjui, as no cognates in Iyojwa’aja’ are known.}%1

\dicnote{The vowel raising \intxt{*a} > \intxt{*i} in Wichí is not known to be regular.}%2

\dicnote{The Lower Bermejeño Wichí form is attested as \intxt{hʷiɲ̊o} in \citet[43]{JB09}, with \intxt{o} (rather than the expected \intxt{u}) corresponding to \sound{PW}{*o}. It is possible that \wordng{LB}{u} is pronounced as a high-mid vowel by some speakers in Bazán.}%3

\dicnote{The Vejoz form is mistranscribed as \intxt{hʷino} in \citet[59]{VU74}.}%4

\lit{\citealt{EN84}: 10, 32 (\intxt{*hwajhnó})}

\PMlemma{{\intxt{*\mbox{-}ɸá\mbox{-}ˀmat} [1]\gloss{disease} \label{dic-famat}}}

\wordng{Mk}{<eq>fe\mbox{-}ˀmet} [2] \citep[157]{AG99} {\sep} \wordng{Ni}{\mbox{-}ɸa\mbox{-}ˀmat} \citep[130]{JS16} {\sep} \wordng{PCh}{*\mbox{-}hwá<ˀmat>} > \wordng{Mj}{\mbox{-}hwáˀmat \recind \mbox{-}hwóˀmat\pl{es}} [3] \citep{JC18}

\dicnote{Contains the PM suffix \wordnl{*\mbox{-}ˀmat}{negative quality, physical defect}.}%1

\dicnote{The Maká reflex contains an unidentified element \intxt{eq\mbox{-}}. The preglottalized coda is attested in the New Testament (e.g. Revelations 8:12).}%2

\dicnote{The variant \intxt{\mbox{-}hwó\mbox{-}ˀmat}, attested in Manjui, is irregular.}%3

\PMlemma{{\wordnl{*\mbox{-}ɸapǻ(ʔ)}{shoulder}, \wordnl{*\mbox{-}ɸapǻ\mbox{-}keʔ\pla{jʰ}}{shoulder blade}}}

\wordng{Ni}{\mbox{-}ɸåpå\mbox{-}ke\pl{j}} \citep[136]{JS16} {\sep} \wordng{PCh}{*\mbox{-}hwopóʔ}; \textit{*\mbox{-}hwopó\mbox{-}keʔ\pla{jʰ}} > \word{Ijw}{\mbox{-}hwópo\pl{ʔ}}{upper arm}; \word{I’w}{\mbox{-}fʷópo\mbox{-}kiʔ}{armpit}; \wordng{Mj}{\mbox{-}hwopó\mbox{-}kiʔ\pl{j}} (\citealt{ND09}: 120; \citealt{AG83}: 130; \citealt{JC18}) {\sep} \word{PW}{*\mbox{-}xʷápo}{shoulder} > LB~\textsc{pl}~\textit{\mbox{-}w̥apu\mbox{-}ɬ} [1]; \wordng{Vej}{\mbox{-}hʷap(ʰ)o\pl{ɬ}} [2]; \wordng{’Wk}{\mbox{-}xʷápoʔ\pl{ɬ}} (\citealt{VN14}: 249; \citealt{VU74}: 58; \citealt{MG-MELO15}: 60; \citealt{KC16}: 60)

\dicnote{Lower Bermejeño \textit{w̥}, as documented by \citet{VN14}, is entirely unexpected. The expected reflex, \textit{\mbox{-}fʷapuʔ\pl{ɬ}}, is attested by \citet[43]{JB09}.}%1

\dicnote{The non-etymological aspiration in the Vejoz reflex is attested by \citet{MG-MELO15}, but not by \citet{VU74}.}%2

\PMlemma{{\wordnl{*ɸaˀt \recind *ɸáˀt}{fire}}}

\wordng{Mk}{feˀt\plf{fet\mbox{-}ej}} (\citealt{AG99}: 173; \citealt{JB81}: 199) {\sep} \wordng{PCh}{*hwát} > \wordng{Ijw}{hwát} \citep[133]{ND09}

\PMlemma{{\wordnl{*ɸátsu(ˀ)χ\plf{*ɸátshu\mbox{-}ts}}{centipede}}}

\wordng{Ni}{ɸatsux\plf{ɸatsxu\mbox{-}s}} (\citealt{LC20}: 51) {\sep} \wordng{PCh}{*(h)wásuh\plf{*(h)wásu\mbox{-}s}} [1] > \wordng{Mj}{wáxsu}, \textit{wáxso\pl{s}} [1] (\citealt{JC18}; \citealt{GH94}) {\sep} \wordng{PW}{*xʷátsuxʷ} > \wordng{’Wk}{\mbox{xʷátsuxʷ}} \citep[164]{KC16}

\dicnote{It is unclear whether the irregular loss of \textit{h} had already occured in Proto-Chorote or in the independent history of Manjui.}%1

\rej{\citet[26]{EN84} lists \wordng{Chorote}{impesʲuk} under this etymology, a form incompatible with \textit{*ɸátsuχ} for phonological reasons. Moreover, we have been unable to identify the dialect to which it belongs.}

\lit{\citealt{EN84}: 26 (\intxt{*pawtshu})}

\PMlemma{{\wordnl{*[ji]ɸáˀx}{to cut down}}}

\wordng{Mk}{\mbox{-}fex\mbox{-}inet\mbox{-}kiʔ} (\intxt{\mbox{-}j \recind \mbox{-}l})\gloss{ax} \citep[174]{AG99} {\sep} \wordng{Ni}{[ji]ɸaˀʃ} \citep[127]{JS16} {\sep} \wordng{PCh}{*[ʔi]hwáh\mbox{-}\APPL} > \wordng{Ijw}{[ʔi]hwʲéh\mbox{-}{\APPL} / \mbox{-}hwáh\mbox{-}\APPL}; \wordng{I’w}{\mbox{-}fʷáh\mbox{-}aj}; \wordng{Mj}{[ʔi]hjéh\mbox{-}{\APPL} / \mbox{-}hwáh\mbox{-}\APPL} (\citealt{ND09}: 99; \citealt{AG83}: 129; \citealt{JC18}) {\sep} \wordng{PW}{*[ʔi]xʷáχ} > \wordng{LB}{[ʔi]fʷaχ}; \word{Vej}{\mbox{-}hʷah\mbox{-}o}{to nail down}; \wordng{’Wk}{[ʔi]xʷáx} (\citealt{VN14}: 351; \citealt{VU74}: 58; \citealt{KC16}: 162)

\lit{\citealt{EN84}: 29 (1~\intxt{*ahwa}); \citealt{PVB02}: 143 (\intxt{*\mbox{-}xʷex})}

\PMlemma{{\intxt{*ɸaʔáj} (fruit); \intxt{*ɸaʔáj\mbox{-}uˀk\plf{*ɸaʔáj\mbox{-}ku\mbox{-}jʰ}} (tree)\gloss{white algarrobo\species{Prosopis alba}}}}

\wordng{Ni}{ɸaʔaj}; \intxt{ɸaʔaj\mbox{-}<j>uk\plf{ɸaʔaj\mbox{-}ku\mbox{-}j}} \citep[127]{JS16} {\sep} \wordng{PCh}{*hwaʔájʔ}; \intxt{*hwaʔáj\mbox{-}uk\plf{*hwaʔáj\mbox{-}ku\mbox{-}jʰ}} > \wordng{Ijw}{hwaʔáʔ}; \textit{hwaʔáj\mbox{-}uk\plf{hwaʔá\mbox{-}tʃu\mbox{-}ˀl}}; \wordng{I’w}{fʷaáʔ\pl{j}} [1]; \intxt{fʷaáj\mbox{-}uk\plf{fʷaáj\mbox{-}si\mbox{-}ʔ}}; \wordng{Mj}{hwaʔájʔ}; \intxt{hwaʔáj\mbox{-}uk \recind \mbox{-}ik\plf{hwaʔáj\mbox{-}ʃi\mbox{-}j}} (\citealt{ND09}: 133; \citealt{AG83}: 128; \citealt{JC18}) {\sep} \wordng{PW}{*xʷaʔájʰ} [2], \intxt{*xʷaʔáj\mbox{-}ukʷ\plf{*xʷaʔá\mbox{-}kʲu\mbox{-}jʰ}} > \wordng{LB}{fʷaʔa\pl{j}} [1]; \intxt{fʷaʔaj\mbox{-}ekʷ\plf{fʷaʔa\mbox{-}tʃ\mbox{-}ej}}; \wordng{Vej}{hʷaʔaj}; \intxt{hʷaʔaj\mbox{-}uk\plf{hʷaʔa\mbox{-}tʃu\mbox{-}j}} [3]; \wordng{’Wk}{xʷaʔáç}; \intxt{xʷaʔáj\mbox{-}uk\plf{xʷaʔá\mbox{-}kʲu\mbox{-}ç}} (\citealt{VN14}: 192, 212, 245; \citealt{VU74}: 58; \citealt{MG-MELO15}: 17; \citealt{KC16}: 162)

\dicnote{In Iyo’awujwa’ and Lower Bermejeño Wichí, the form with a final \intxt{\mbox{-}j} has been attested as a plural form. These two varieties must have innovated by back-deriving a \intxt{j\mbox{-}}less singular from a reflex of \intxt{*ɸaʔáj}. Note that at least in Lower Bermejeño Wichí the form \intxt{fʷaʔa\mbox{-}j} is much more frequent than the singular \intxt{fʷaʔa} (attested only in the compound \wordnl{fʷaʔa muk}{\intxt{Prosopis alba} flour}), and the derivation processes take the plural form \intxt{fʷaʔa\mbox{-}j} as the base \citep[196]{VN14}. \wordng{LB}{fʷaʔaj} is also the only form attested in \citet[60]{CS08}.}%1

\dicnote{\wordng{PW}{*\mbox{-}ájʰ}, reconstructed based on the ’Weenhayek reflex with \intxt{\mbox{-}ç}, does not correspond to \wordng{PCh}{*\mbox{-}ájʔ} (underlying: */\mbox{-}áj/). The root must have been remodeled based on the plural suffix \intxt{*\mbox{-}jʰ}.}%2

\dicnote{In Vejoz, \citet[17]{MG-MELO15} document an irregular variant \intxt{hʷaʔatʃ\mbox{-}uk} alongside \textit{hʷaʔaj\mbox{-}uk}.}%3

\empr{\citet[300]{PVB13a} notes the similarity with \word{Lule}{waja}{green and black algarrobo} and \wordng{Proto-Guaicuruan}{*waˀjek} (\citealt{PVB13b}, \#619) > Mbayá ‹guayegi›\gloss{jasper-colored algarroba}, \word{Abipón}{oai\mbox{-}k}{\textit{Prosopis alba}} \citep[110]{EN66}, which is attributed to lexical diffusion.}

\lit{\citealt{EN84}: 12, 17, 27, 39, 46 (\intxt{*hwå(\mbox{-})á} (fruit); \intxt{*hwåajuk} (tree); \intxt{*hwajcat} (grove)); \citealt{LC-VG-07}: 19; \citealt{AnG15}: 77}

\PMlemma{{\wordnl{*[ji]ɸä́l}{to tell}}}

\wordng{Mk}{n(i)\mbox{-}fel\mbox{-}iˀm} \citep[172]{AG99} {\sep} \wordng{Ni}{n(i)\mbox{-}ɸak / n(i)\mbox{-}ɸak͡l\mbox{-}} \citep[189]{JS16} {\sep} \wordng{PCh}{*[ʔi]hwél} > \wordng{Ijw}{[ʔi]hwíˀl / \mbox{-}hwέˀl}; \wordng{I’w}{[i]híl\mbox{-}am / \mbox{-}fʷé(h)l\mbox{-}am}; \wordng{Mj}{[ʔi]hjíl / \mbox{-}hwél} (\citealt{ND09}: 100; \citealt{AG83}: 44, 130, 185; \citealt{JC18}) {\sep} \wordng{PW}{*[ʔi]xʷélʰ / *[ʔi]xʷél\mbox{-}} > \wordng{LB}{[ʔi]fʷeɬ / [ʔi]fʷel\mbox{-} / [ʔi]fʷen̥\mbox{-}}; \wordng{Vej}{\mbox{-}hʷen} [2]; \wordng{’Wk}{[ʔi]xʷéɬ} (\citealt{VN14}: 150, 184, 259; \citealt{VU74}: 59; \citealt{KC16}: 167)

\dicnote{The preglottalized coda in Maká is attested in the New Testament (e.g. John 17:8).}%1

\dicnote{The Vejoz reflex attested in \citet[58]{VU74} is not known to be regular.}%2

\PMlemma{{\wordnl{*\mbox{-}ɸä́lits}{sister-in-law; daughter-in-law}}}

\word{Mk}{\mbox{-}felits\plf{\mbox{-}feltsi\mbox{-}ʔ}}{daughter-in-law; brother-in-law’s wife} \citep[172]{AG99} {\sep} \word{Ni}{\mbox{-}ɸak͡lits<ʔa>\pl{k}}{sister-in-law} \citep[128]{JS16} {\sep} \word{PCh}{*\mbox{-}hwélis\plf{*\mbox{-}hwélsV\mbox{-}wot}}{daughter-in-law} > \wordng{Ijw}{\mbox{-}hwέlis\plf{\mbox{-}hwέlse \recind \mbox{-}hwέlse\mbox{-}wot}}; \wordng{I’w}{\mbox{-}fʷéles}; \wordng{Mj}{\mbox{-}hwéles\plf{\mbox{-}hwélsa\mbox{-}wot}} (\citealt{ND09}: 120; \citealt{AG83}: 129; \citealt{JC18})

\PMlemma{{\wordnl{*\mbox{-}ɸä́lʔuʔ\pla{ts}}{son-in-law; brother-in-law}}}

\word{Mk}{\mbox{-}feluʔ\pl{ts}}{son-in-law; sister-in-law’s husband} \citep[172]{AG99} {\sep} \word{Ni}{\mbox{-}ɸak͡lʔu\pl{s}}{brother-in-law} \citep[128]{JS16} {\sep} \wordng{PCh}{*\mbox{-}hwíluʔ \recvar *\mbox{-}hwéluʔ\pla{s}} [1]\gloss{son-in-law} > \wordng{Ijw}{\mbox{-}hwélʲuʔ\pl{s}}; \wordng{I’w}{\mbox{-}fʷéluʔ\pl{s}}; \wordng{Mj}{\mbox{-}hwílʲuʔ \recind hwéilʲuʔ\pl{s}} (\citealt{ND09}: 120; \citealt{AG83}: 129; \citealt{JC18})

\dicnote{\wordng{PCh}{*í} (whose reconstruction is supported by the Iyojwa’aja’ and Manjui cognates) is not the expected reflex of \wordng{PM}{*ä́}. By contrast, Iyo’awujwa’ points to \wordng{PCh}{*é} (as shown by the absence of palatalization in \textit{l}).}%1

\PMlemma{{\intxt{*\mbox{-}ɸät \recind *\mbox{-}ɸäˀt} [1]\gloss{belt}}}

\word{Mk}{(\mbox{-})fet<i(ˀ)ɬ>\plf{fet<iɬ>\mbox{-}its}}{men’s belt or skirt made of feathers worn at festivals} [2] \citep[174]{AG99} {\sep} \word{Ni}{\mbox{-}<nuk>ɸat\pl{es}}{belt, sash} [3] \citep[95]{LC20} {\sep} \wordng{PCh}{*\mbox{-}hwét} > \wordng{Mj}{\mbox{-}hwét\plf{\mbox{-}hwet\mbox{-}ájh}} (\citealt{JC18})

\dicnote{The vowel is reconstructed as unaccented based on the plural form attested in Manjui. It is unclear whether the coda should be reconstructed as preglottalized (\wordng{Ni}{\mbox{-}nukɸat} does not show any traces of preglottalization, but this could possible be the case due to deglottalization in unaccented syllables).}%1

\dicnote{We have no explanation for the element \intxt{\mbox{-}iɬ} or \intxt{\mbox{-}iˀɬ} in Maká (the term is not attested in our sources that distinguish between plain and preglottalized stops).}%2

\dicnote{We have no explanation for the element \intxt{\mbox{-}nuk\mbox{-}} in Nivaĉle.}%3

\PMlemma{{\wordnl{*ɸäˀx \recind *ɸä́ˀx}{field}}}

\word{Ni}{ɸaˀʃ\plf{ɸaʃ\mbox{-}ik}}{field, lowland} \citep[127]{JS16} {\sep} \wordng{PCh}{*hwéh} > \wordng{I’w}{fʷéh}; \wordng{Mj}{hwéh} (\citealt{AG83}: 129; \citealt{JC18})

\lit{\citealt{EN84}: 29 (\intxt{*hwɛhn})}

\PMlemma{{\wordnl{*[ji]ɸä́ˀjå \recvar *ɸäˀjå}{to fly}}}

\wordng{Ni}{[ji]ɸåˀjå} \citep[136]{JS16} {\sep} \wordng{PCh}{*[ʔi]hwéˀjåʔ} > \wordng{Ijw}{[ʔi]hwíˀjaʔ / \mbox{-}hwέˀjaʔ}; \wordng{I’w}{\mbox{-}fʷéjeʔ}; \wordng{Mj}{[ʔi]hjíˀjeʔ / \mbox{-}hwéˀjeʔ} (\citealt{ND09}: 100; \citealt{AG83}: 129; \citealt{JC18}) {\sep} \wordng{PW}{*xʷeˀjå \recvar *w\mbox{-} \recvar *\mbox{-}i\mbox{-}} [1] > \wordng{LB}{wiˀjo}; \wordng{Vej}{\mbox{-}hʷija}; \wordng{’Wk}{weˀjåʔ} (\citealt{VN14}: 258; \citealt{VU74}: 59; \citealt{KC16}: 481)

\dicnote{The correspondences between the Wichí varieties are entirely irregular. Only Vejoz points to \wordng{PW}{*xʷ} (which matches the evidence from Chorote and Nivaĉle), while other varieties point to \wordng{PW}{*w}. Only ’Weenhayek and the variety of Misión Santa María (\intxt{wejaʔ} in \citealt{SS07}) point to \wordng{PW}{*\mbox{-}e\mbox{-}} (which matches the evidence from Chorote), while other varieties point to \wordng{PW}{*\mbox{-}i\mbox{-}}.}%1

\gc{Possibly related to \word{Proto-Guaicuruan}{*\mbox{-}a(ˀ)jo}{to fly} (\citealt{PVB13b}, \#11; cf. \citealt{PVB13a}: 304), though a better comparandum for the Guaicuruan form is \word{Mk}{n\mbox{-}aˀjaʔ}{to fly} \citep[138]{AG99}.}

\lit{\citealt{PVB13a}: 304 (\intxt{*\mbox{-}(hʷ)ejʌʔ})}

\PMlemma{{\intxt{*(\mbox{-})ɸeɬek} (\intxt{\recind *\mbox{-}éɬe\mbox{-} \recind *\mbox{-}eɬé\mbox{-}}) [1]\gloss{mortar}}}

\word{Mk}{(\mbox{-})fiɬik\pl{i}}{drum} \citep[175]{AG99} {\sep} \wordng{Ni}{\mbox{-}ɸeɬetʃ\plf{\mbox{-}ɸeɬtʃe\mbox{-}j}} \citep[132]{JS16} {\sep} \wordng{PCh}{*(\mbox{-})hwVhlek} [2] > \wordng{Ijw}{(\mbox{-})(h)wánhlek\plf{(\mbox{-})(h)wánhle\mbox{-}ʔe}}; \wordng{I’w}{\mbox{wihlík}\pl{is}}; \wordng{Mj}{(h)wihlík} (\intxt{wihlík\mbox{-}is \recind wiɬk\mbox{-}íjh}) (\citealt{JC14b}: 78; \citealt{ND09}: 133; \citealt{AG83}: 170; \citealt{JC18}) {\sep} \wordng{PW}{*xʷéɬeq} > \wordng{LB}{fʷeɬeq}; \wordng{Vej}{hʷeɬek(\mbox{-}tʃ’o)}; \wordng{’Wk}{xʷéɬek} (\citealt{VN14}: 300; \citealt{VU74}: 59; \citealt{MG-MELO15}: 48; \citealt{AFG067}: 215; \citealt{KC16}: 167)

\dicnote{The prosodic properties of the root are difficult to reconstruct: Iyo’awujwa’ and Manjui point to \intxt{*ɸiɬek} or \intxt{*ɸiɬék}, ’Weenhayek to \intxt{*ɸéɬek}, and Iyojwa’aja’ to \intxt{*(\mbox{-})ɸánɬek} or \intxt{*(\mbox{-})ɸǻnɬek} (see below on the irregular segmental correspondences).}%1

\dicnote{Each Chorote variety presents some irregularity in the phonological development of this root. In Iyojwa’aja’, one finds the vowel \intxt{a} in the first syllable followed by a nasal consonant, with no parallels either in other Chorote varieties or in other Mataguayan languages; the expected outcome would be \intxt{*hwέhlek}. In Iyo’awujwa’ and Manjui, the first syllable contains the unexpected vowel \intxt{i}; furthermore, the initial consonant is \textit{w} (rather than \intxt{*hw}) in Iyo’awujwa’ (and optionally in Iyojwa’aja’ and Manjui).}%2

\PMlemma{{\wordnl{*(\mbox{-})ɸétäˀts}{root}}}

\word{Mk}{fitets\pl{its}}{\textit{Dorstenia sp.}}, \third{ɬe\mbox{-}fitets} [1] (\intxt{\mbox{-}its})\gloss{root} (\citealt{AG99}: 178, 249) {\sep} \wordng{Ni}{\mbox{-}ɸetaˀs\plf{\mbox{-}ɸetats\mbox{-}ij}} [2] \citep[132]{JS16} {\sep} \wordng{PCh}{*\mbox{-}hwétus} [3] > \wordng{Ijw}{\mbox{-}hwέtis}, \textit{\mbox{-}hwέtisʲ\mbox{-}uˀl}; \wordng{I’w}{fʷétis} (\intxt{\mbox{-}iʔ}); \wordng{Mj}{\mbox{-}hwétus} (\intxt{\mbox{-}ej \recind \mbox{-}uj}) (\citealt{ND09}: 120; \citealt{AG83}: 129; \citealt{JC18}) {\sep} \wordng{PW}{*(\mbox{-})xʷétes\plf{*xʷétes\mbox{-}elʰ / *\mbox{-}xʷéts\mbox{-}ilʰ}} [4] > LB~\textsc{pl}~\textit{\mbox{-}fʷets\mbox{-}il}; \wordng{Vej}{\mbox{-}hʷetes}; \wordng{’Wk}{(\mbox{-})xʷétes}, \textit{xʷétes\mbox{-}eɬ / \mbox{-}xʷéts\mbox{-}iɬ} (\citealt{VN14}: 324; \citealt{VU74}: 59; \citealt{KC16}: 61, 168)

\dicnote{The Maká reflex unexpectedly lacks preglottalization in the coda in the singular form, as attested in the New Testament (Romans 11:18; Luke 3:9).}%1

\dicnote{Nivaĉle also has \wordnl{ɸetåx\plf{ɸetx\mbox{-}ås}}{peel of a root} \citep[132]{JS16}, which is obviously related (cf. also \wordnl{\mbox{-}ʔåx\pl{is}}{skin, bark}), but the derivational relation is obscure.}%2

\dicnote{\sound{PM}{*ä} has undergone irregular change in Chorote and irregular syncope in the Wichí possessed plural form.}%3

\dicnote{The vowel syncope in the Wichí plural is irregular.}%4

\gc{Possibly related to \word{Proto-Guaicuruan}{*\mbox{-}pat’ád}{trunk, root} (\citealt{PVB13b}, \#479). \citet[313]{PVB13a} notes the similarity of the PM form with \word{Kadiwéu}{\mbox{-}itodi}{root}, which is likely spurious.}

\lit{\citealt{EN84}: 9, 19, 43 (\intxt{*hwɛtets}); \citealt{PVB13a}: 313 (\intxt{*hʷetets})}

\PMlemma{{\textit{*[ji]ɸiˀj \recind *[ji]ɸíˀj} [1]\gloss{not to be afraid}}}

\wordng{Ni}{[ji]ɸiˀj} \citep[133]{JS16} {\sep} \wordng{PCh}{*[ʔi]hwíjʔ} > \wordng{Ijw}{[ʔi]hwíj\mbox{-}e / \mbox{-}hwéj\mbox{-}e}; \word{I’w}{há fʷíj\mbox{-}in}{fearful}; \wordng{Mj}{[ʔi]hjíjʔ / \mbox{-}hwijʔ \recind \mbox{-}hwéiʔ} (\citealt{ND09}: 100; \citealt{AG83}: 172; \citealt{JC18}) {\sep} \wordng{PW}{*[ʔi]ˈxʷíj\mbox{-}eh} > \wordng{’Wk}{[ʔi]ˈxʷíj\mbox{-}eh} \citep[172]{KC16}

\dicnote{The prosodic properties of the root cannot be established because the ’Weenhayek cognate is not attested without applicative morphology (the form with an applicative suffix is not revealing because in trisyllabic words the vowel of the peninitial syllable is lengthened in any case).}%1

\PMlemma{{\wordnl{*ɸiˀjä́t}{cold weather, south wind}}}

\word{Ni}{ɸiˀjat\pl{is}}{south wind} \citep[134]{JS16} {\sep} \word{PCh}{*hwiˀjét}{ice, frost} > \wordng{Ijw}{wiˀjít}; \wordng{Mj}{hwiˀjít} (\citealt{ND09}: 157; \citealt{JC18}) {\sep} \word{PW}{*xʷiˀjét\pla{ilʰ}}{winter, cold weather} > \wordng{LB}{xʷiˀjét}; \wordng{Vej}{hʷiˀjet} (\intxt{\mbox{-}il \recind \mbox{-}iɬ}) [1]; \wordng{’Wk}{xʷiˀjét} (\intxt{\mbox{-}iɬ}) (\citealt{VN14}: 200, 212; \citealt{MG-MELO15}: 43; \citealt{KC16}: 61, 168)

\dicnote{\citet[59]{VU74} mistranscribes this word (possibly the plural form) as \textit{hʷijet til}.}%1

\PMlemma{{\wordnl{*[ji]ɸiˀk \recind *[ji]ɸíˀk}{to hide}}}

\wordng{Ni}{[ji]ɸiˀtʃ} \citep[133]{JS16} {\sep} \wordng{PCh}{*[ʔi]hwík} > \word{Ijw}{[ʔi]hwík / \mbox{-}hwék}{to keep in secret}, \wordnl{[ʔi]hwík\mbox{-}i / \mbox{-}hwék\mbox{-}i}{to hide}; \wordng{Mj}{[ʔi]hjík / \mbox{-}hwík} (\citealt{ND09}: 100; \citealt{JC18})

\PMlemma{{\wordnl{*ɸínä(ˀ)χ}{crab}}}

\wordng{Ni}{ɸinax}, \textit{ɸinxa\mbox{-}s} \citep[133]{JS16} {\sep} \wordng{PCh}{*hwíneh} > \wordng{Ijw}{hwéni}; \wordng{Mj}{hwɪ́ni} (\citealt{ND09}: 133; \citealt{JC18})

\PMlemma{{\wordnl{*ɸiˀs}{leech} [1]}}

\wordng{Ni}{ɸiˀs}, \textit{ɸis\mbox{-}ik} \citep[113]{JS16} {\sep} \wordng{PW}{*xʷis} > \wordng{’Wk}{xʷis} \citep[170]{KC16}

\gc{\word{Proto-Qom}{*pit}{leech} may have been borrowed from Mataguayan.}

\PMlemma{{\wordnl{*ɸít’i(ʔ) \recind *ɸít’ih}{dragonfly}}}

\wordng{Ni}{ɸit’i\pl{k}} \citep[134]{JS16} {\sep} \wordng{PCh}{*hwí(n)t’i…} [1] > \wordng{Ijw}{hwént’i<je>\pl{jis}} [1] (\citealt{ND09}: 133) {\sep} \wordng{PW}{*xʷit’i<s>} [2] > \wordng{Vejoz or Guisnay}{hʷit’i<s>} \citep[32]{RL16}

\dicnote{The Iyojwa’aja’ reflex is quite irregular: it contains an unexpected nasal consonant and an unidentified element fossilized to the erstwhile root.}%1

\dicnote{The Wichí reflex includes a non-etymological element \intxt{*s}. In \citet{diwica}, an irregular dialectal form ‹fwich’is› is also documented, it is attributed to the Pilcomayeño variety (corresponding to our Guisnay).}%2

\lit{\citealt{EN84}: 39 (\intxt{*hwethne})}

\PMlemma{{\wordnl{*ɸkéna(ˀ)χ}{north wind, north}}}

\wordng{Ni}{ɸtʃenax\plf{ɸtʃenxa\mbox{-}s}} \citep[132]{JS16} {\sep} \wordng{PCh}{*hwᵊkénah} > \wordng{Ijw/I’w}{wikína} [1]; \wordng{Mj}{hwikína} (\citealt{JC14b}: 74, fn. 1; \citealt{ND09}: 157; \citealt{AG83}: 170; \citealt{JC18})

\dicnote{\sound{Iyojwa’aja’ and Iyo’awujwa’}{w\mbox{-}} is not a regular reflex of \sound{PCh}{*hw\mbox{-}}.}%1

\rej{\citet[11]{EN84} compares \wordng{Ni}{ftʃenax} with Chorote and Wichí words meaning\gloss{mountain}, which are derived from \word{PM}{*tkénax}{precipice; hill, mountain} in our proposal.}

\PMlemma{{\wordnl{*\mbox{-}ɸo(ʔ) \recind *\mbox{-}ɸó(ʔ)}{foot} [1]}}

\word{Mk}{\mbox{-}fo<nxeʔ>\pl{j}}{ankle} [2] \citep[180]{AG99} {\sep} \word{Ni}{\mbox{-}ɸoʔ\pl{k}}{foot}, \wordnl{\mbox{-}fo<ˀk͡lå>\pl{s}}{ankle bracelet with white feathers} [3], \wordnl{\mbox{-}fo\mbox{-}xij\pl{is}}{stirrup} \citep[135]{JS16}

\dicnote{This root certainly reconstructs all the way to Proto-Mataguayan, since Chorote and Wichí reflect a likely derivative \wordnl{*\mbox{-}ɸólXaˀn}{ankle}.}%1

\dicnote{The formative \intxt{\mbox{-}nxeʔ} in Maká does not appear to be morphologically segmentable, but it is also found in \wordnl{\mbox{-}wonxeʔ}{neck} and other body-part terms.}%2

\dicnote{\wordng{Ni}{\mbox{-}foˀk͡lå} includes a fossilized reflex of \word{PM}{*\mbox{-}ˀlåʔ \recind *\mbox{-}ˀlǻʔ}{adornment}.}%3

\PMlemma{{\wordnl{*\mbox{-}ɸqató\pla{l}}{elbow}}}

\wordng{Ni}{\mbox{-}(ʔV)ɸkato\pl{k}} \citep[131]{JS16} {\sep} \wordng{PCh}{*\mbox{-}qatóʔ\pla{l}} > \wordng{Ijw}{\mbox{-}káto\mbox{-}kiʔ}, \wordng{I’w}{\mbox{-}katóʔ \recind \mbox{-}kató\mbox{-}kiʔ}, \wordng{Mj}{\mbox{-}katɔ́ʔ\pl{l}} (\citealt{JC14b}: 76, 91, fn. 22; \citealt{ND09}: 121; \citealt{AG83}: 137; \citealt{JC18}) {\sep} \wordng{PW}{*\mbox{-}qáto\pla{lʰ}} > \wordng{LB}{\mbox{-}qatu}; \wordng{Vej}{\mbox{-}kåto} [1]; \wordng{’Wk}{\mbox{-}qáto\pl{ɬ}} (\citealt{JB09}: 47; \citealt{VU74}: 62; \citealt{KC16}: 87)

\dicnote{The vowel \intxt{å} in the Vejoz reflex is unexpected and could be a mistranscription on \cits{VU74} part.}%1

\gc{Possibly related to \word{Proto-Guaicuruan}{*\mbox{-}q’oté}{elbow} (\citealt{PVB13b}, \#542).}

\lit{\citealt{EN84}: 10 (\intxt{*qatɔq}); \citealt{LC-VG-07}: 15}

\PMlemma{{\wordnl{*ɸtsǻna(ˀ)χ}{\textit{Baccharis sp.}}}}

\wordng{Ni}{ɸtsånax\plf{ɸtsåna\mbox{-}s}} \citep[137]{JS16} {\sep} \wordng{PCh}{*sǻnah} > \wordng{Ijw/Mj}{sána} (\citealt{ND09}: 144; \citealt{JC18}) {\sep} \wordng{PW}{*xʷitsǻnaχ} > \wordng{Vej}{hʷitsånah\plf{hʷitsån̥\mbox{-}as}} [1] (\citealt{MG-MELO15}: 18)

\dicnote{\citet[59]{VU74} mistranscribes the root as \intxt{hʷitsanah}.}%1

\lit{\citealt{EN84}: 29 (\intxt{*hwitsåhna})}

\PMlemma{{\intxt{*ɸts\mbox{-}uˀk}, collective \intxt{*ɸis\mbox{-}kat} [1]\gloss{\textit{Copernicia alba} palm}}}

\wordng{Mk}{fits\mbox{-}uk} [2], \textit{fis\mbox{-}kw\mbox{-}i}; \textit{fis\mbox{-}ket} \citep[178]{AG99} {\sep} \wordng{Ni}{ɸts\mbox{-}uˀk}; \textit{ɸis\mbox{-}tʃat}; stem used in derivatives: \textit{ɸts\mbox{-}uk\mbox{-}i\mbox{-}} (\citealt{JS16}: 133, 137–138) {\sep} \wordng{PCh}{*hwis<úk>} [3] > Ijw/\wordng{I’w}{(h)wisʲúk}; \wordng{Mj}{(h)wiʃúk} (\intxt{\mbox{-}ij}) [4] (\citealt{ND09}: 157; \citealt{AG83}: 170; \citealt{GS10}: 186; \citealt{JC18}) {\sep} \wordng{PW}{*xʷits<ukʷ>} > \word{LB}{fʷitsekʷ}{\textit{Ruprechtia triflora}}; \wordng{Vej}{hʷitsuk\pl{ɬajis}} [5]; \wordng{’Wk}{xʷitsuk} (\citealt{CS08}: 59; \citealt{JB09}: 43; \citealt{VU74}: 59; \citealt{MG-MELO15}: 17–18; \citealt{KC16}: 172)

\dicnote{Based on the Nivaĉle reflex, we reconstruct a non-productive alternation pattern, whereby the PM cluster \textit{*ɸts\mbox{-}} before vowels would have alternated with \textit{*ɸis\mbox{-}} before consonants (with an irregular deaffrication of \intxt{*ts} and epenthesis of \intxt{*i}, likely motivated by the necessity to avoid a tautosyllabic cluster \intxt{*ɸtsk}). We surmise that the epenthetic \intxt{*i} has been analogically extended to the prevocalic allomorph in all languages except Nivaĉle.}%1

\dicnote{The Maká reflex unexpectedly lacks preglottalization in the coda in the singular form, as attested in the New Testament (Revelations 7:9).}%2

\dicnote{In Chorote, \wordng{PM}{*ɸ} in this word is irregularly reflected as \intxt{w} alongside the expected reflex \textit{hw}. It is unclear why the vowel \intxt{*i} rather than \intxt{*ᵊ} was epenthesized.}%3

\dicnote{The plural form attested in Manjui is innovative.}%4

\dicnote{The absence of labialization in the reflex of \wordng{PW}{*\mbox{-}kʷ} in Vejoz is unexpected.}%5

\gc{\citet[310]{PVB13a} notes the similarity with \word{Proto-Guaicuruan}{*tsjáwa}{\textit{Copernicia alba} palm} (VB~2013b, \#584), which could be spurious.}

\lit{\citealt{EN84}: 16 (\intxt{*hwitsúk}); \citealt{LC-VG-07}: 15 (“diffused?”); \citealt{PVB13a}: 310 (\intxt{*hʷits\mbox{-}uk})}

\PMlemma{{\wordnl{*[ji]ɸúju}{to blow}}}

\wordng{Mk}{[ji]fuju} \citep[183]{AG99} {\sep} \word{Ni}{[ji]ɸuju}{to blow, to play a woodwind instrument} \citep[138]{JS16} {\sep} \wordng{PCh}{*[ʔi]hwúju\mbox{-}\APPL} > \word{Mj}{[ʔi]hjúji\mbox{-}iˀm \recind [ʔi]hjúju\mbox{-}uˀm / \mbox{-}hwúji\mbox{-}iˀm \recind \mbox{-}hwúju\mbox{-}uˀm}{to blow at}, \wordnl{[ʔi]hjúji\mbox{-}ʔiʔ \recind [ti]hwúji\mbox{-}ʔiʔ}{to blow} (\citealt{JC18})

\PMlemma{{\wordnl{*[ji]ɸún}{to be hesitant with, to respect}}}

\word{Ni}{[ji]ɸun\mbox{-}a}{to be delicate with, to respect} \citep[138]{JS16} {\sep} \wordng{PW}{*[ʔi]xʷún} > \word{Vejoz or Guisnay}{[i]hʷun}{to be timid, to be lazy, not to feel like doing something}; \wordng{’Wk}{[ʔi]xʷún̥} (\citealt{RL16}: 34; \citealt{KC16}: 177)

\PMlemma{{\intxt{*\mbox{-}ɸuˀt \recind *\mbox{-}ɸúˀt\plf{*\mbox{-}ɸtú\mbox{-}ts}} [1]\gloss{flatulence}}}

\wordng{Mk}{\mbox{-}ftuʔ\plf{\mbox{-}ftu\mbox{-}ts}} [2] \citep[141]{AG99} {\sep} \wordng{Ni}{\mbox{-}ɸuˀt\plf{\mbox{-}ɸtu\mbox{-}s}} \citep[138]{JS16} {\sep} \wordng{PCh}{*\mbox{-}hwút} > \wordng{Ijw/Mj}{\mbox{-}hwút} (\citealt{ND09}: 120; \citealt{JC18}) {\sep} \word{PW}{*[t]<’e>xʷtu\mbox{-}j \recind *[t]<’e>xʷtú\mbox{-}j}{to fart} > \wordng{LB}{[t]’efʷte\mbox{-}j} \citep[278]{VN-MA-21}

\dicnote{The plural form is reconstructed based on Maká and Nivaĉle; it is thus technically reconstructible only for Proto-Maká–Nivaĉle.}%1

\dicnote{The singular form in Maká has been reshaped based on the plural form. One would expect \intxt{*\mbox{-}fuˀt\plf{ftu\mbox{-}ts}}.}%2

\gc{\citet[310]{PVB13a} notes the similarity with \word{Proto-Guaicuruan}{*\mbox{-}wit’i}{flatulence, to fart} (\citealt{PVB13b}, \#632), which could be spurious.}

\lit{\citealt{PVB13a}: 309 (\intxt{*\mbox{-}ehʷutuʔ})}

\PMlemma{{\wordnl{*[ji]ɸχän\mbox{-} \recind *\mbox{-}ä́\mbox{-}}{to kill a bird}}}

\wordng{Ni}{[ji]ɸxan\mbox{-}\APPL} \citep[39]{JS16} {\sep} \word{PCh}{*<ʔa>hwén\mbox{-}(n)ah}{bird} [1] > \wordng{Ijw}{ʔahwέn\mbox{-}a\plf{<ʔa>hwέhn\mbox{-}a\mbox{-}s}}; \wordng{I’w}{afʷén\mbox{-}a\mbox{-}ki\pl{ji}}; \wordng{Mj}{ʔahwén\mbox{-}a\plf{ʔahwéhn\mbox{-}a\mbox{-}s}} (\citealt{ND09}: 93; \citealt{AG83}: 117; \citealt{JC18}) {\sep} \word{PW}{*<ʔa>xʷén\mbox{-}kʲe\pla{jʰ}}{bird} [1] > \wordng{LB}{ʔafʷen\mbox{-}tʃe\pl{j}}; \wordng{Vej}{ʔahʷen\mbox{-}tʃe\pl{j}}; \wordng{’Wk}{ʔaxʷén\mbox{-}kʲeʔ\pl{ç}} (\citealt{VN14}: 196, 253; \citealt{JB09}: 37; \citealt{VU74}: 50; \citealt{MG-MELO15}: 19; \citealt{KC16}: 10)

\dicnote{In Chorote and Wichí, the original verb is not preserved, but the term for\gloss{bird} appears to be its nominalization. The prefixed element \intxt{*ʔa\mbox{-}} is of unclear origin.}%1

\PMlemma{{\wordnl{*\mbox{-}ɸχúx\plf{*\mbox{-}ɸχú\mbox{-}ts}}{finger}}}

\wordng{Mk}{\mbox{-}fux\pl{uts}} [1] \citep[183]{AG99} {\sep} \word{Ni}{\mbox{-}ɸxux\plf{\mbox{-}ɸxu\mbox{-}s}}{toe} \citep[135]{JS16} {\sep} \wordng{PCh}{*\mbox{-}hwu\mbox{-}kéʔ} > \wordng{Ijw}{\mbox{-}hwú\mbox{-}kiʔ\pl{ˀl}}; \word{I’w}{\mbox{-}fʷi\mbox{-}kíʔ\plf{\mbox{-}ji}}{toe} [2] (\citealt{ND09}: 120; \citealt{AG83}: 130) {\sep} \wordng{PW}{*\mbox{-}xʷúxʷ}, \textit{*\mbox{-}xʷú\mbox{-}s} > \wordng{LB}{\mbox{-}fʷefʷ}, \textit{\mbox{-}fʷe\mbox{-}s}; \wordng{Vej}{\mbox{-}hʷuh}, \textit{\mbox{-}hʷu\mbox{-}s} [3]; \wordng{’Wk}{\mbox{-}xʷúxʷ} (\intxt{\mbox{-}xʷú\mbox{-}s}) (\citealt{VN14}: 191; \citealt{VU74}: 58; \citealt{MG-MELO15}: 32, 60; \citealt{KC16}: 62)

\dicnote{The Maká plural form is non-etymological.}%1

\dicnote{The vowel \intxt{i} as a reflex of \wordng{PCh}{*u} in \cits{AG83} data of Iyo’awujwa’ is irregular; alternatively, it could be a mistranscription.}%2

\dicnote{The singular form of the Vejoz reflex irregularly lacks labialization in the final consonant. It is mistranscribed as \intxt{\mbox{-}huh} in \citet[58]{VU74}.}%3

\lit{\citealt{EN84}: 15 (\textsc{pl}~\textit{*hwuq\mbox{-}ś})}

\PMlemma{{\textit{*(\mbox{-})ɸ’elxVtséχ\plf{*(\mbox{-})ɸ’elxVtsé\mbox{-}ts}} [1]\gloss{poor}}}

\word{Mk}{\mbox{-}f’ilxetsaχ\plf{\mbox{-}f’ilxetsi\mbox{-}ts}}{poor}; \wordnl{\mbox{-}f’ilxetsi\mbox{-}ʔ}{poverty} \citep[183]{AG99} {\sep} \wordng{PCh}{*p’ilusáh\plf{*p’ihlusé\mbox{-}s}} [2 3] > \wordng{Ijw}{p’ilʲúxse \recind p’élisʲe\plf{p’ihlʲúxsi\mbox{-}s}}; \wordng{I’w}{\mbox{-}pelíxsa}; \wordng{Mj}{p’ilisáh}, \textit{p’ilisέ\mbox{-}s} [2] (\citealt{JC14a}; \citealt{JC14b}: 92; \citealt{ND09}: 144; \citealt{AG83}: 155; \citealt{JC18}) {\sep} \wordng{PW}{*p’elítsaχ}, \textit{*p’elítse\mbox{-}s} [2] > \wordng{LB}{p’alitsaχ} [3]; \wordng{Vej}{p’elitsah}; \wordng{’Wk}{p’alítsax}, \textit{p’alítse\mbox{-}s} [4] (\citealt{JB09}: 54; \citealt{VU74}: 71; \citealt{MG-MELO15}: 52; \citealt{KC16}: 297)

\dicnote{Regarding the vowel of the medial syllable, Maká points to \wordng{PM}{*a} or \intxt{*ä}, Chorote to \intxt{*u}, and Wichí to \intxt{*i}.}%1

\dicnote{\sound{PM}{*x} is inexplicably lost in the Chorote singular form (in Manjui also in the plural) as well as in Wichí.}%2

\dicnote{The Proto-Chorote stress is unexpectedly retracted to the peninitial syllable in Iyo’awujwa’, and to the initial syllable in the Iyojwa’aja’ variant \intxt{p’élisʲe}.}%3

\dicnote{\sound{PW}{*e} is regularly reflected as \intxt{e} in Vejoz, whereas Lower Bermejeño and ’Weenhayek show the irregular reflex \intxt{a}.}%4

\PMlemma{{\wordnl{*(\mbox{-})hǻqkeʔ\pla{jʰ}}{well}}}

\wordng{Mk}{haqqiʔ\pl{l}} [1]\gloss{river} \citep[186]{AG99} {\sep} \word{Ni}{\mbox{-}xǻke\pl{j}}{dry well} \citep[153]{JS16} {\sep} \word{PCh}{*\mbox{-}hǻåkeʔ}{artificial well, ditch} > \wordng{Ijw}{\mbox{-}hákiʔ}; \wordng{I’w}{\mbox{-}hákiʔ\pl{ji}}; \word{Mj}{\mbox{-}háakiʔ\pl{j}}{artificial well, ditch} (\citealt{ND09}: 129; \citealt{AG83}: 173; \citealt{JC18})

\dicnote{The plural form in Maká is non-etymological.}%1

\lit{\citealt{EN84}: 14 (\intxt{*hnawq})}

\PMlemma{{\wordnl{*\mbox{-}í(t)s’i(ʔ)\pla{l}}{resin, sap}}}

\wordng{Ni}{\mbox{-}its’i\pl{k}} [1]\gloss{resin, earwax} \citep[142]{JS16} {\sep} PCh~\third{*hl\mbox{-}íts’iʔ\pla{l}} > \word{Ijw}{hl\mbox{-}éts’i}{resin, sap, wax}; Mj~\third{hl\mbox{-}éits’eʔ\pl{l}}\gloss{sap} [2] (\citealt{ND09}: 131; \citealt{JC18}) {\sep} \wordng{PW}{*\mbox{-}ɬ\mbox{-}íts’i} > \word{LB}{\mbox{-}ɬ\mbox{-}its’i}{wax}; \word{’Wk}{\mbox{-}ɬ\mbox{-}íts’iʔ}{resin, rubber} (\citealt{VN14}: 267; \citealt{KC16}: 75, 236)

\dicnote{\citet[142]{JS16} actually attests \textit{\mbox{-}iʔts’i}, where [ʔts’] is likely an allophone of /ts’/.}%1

\dicnote{\sound{Manjui}{e} is not known to be a regular reflex of unstressed \wordng{PCh}{*i}.}%2

\PMlemma{{\wordnl{*\mbox{-}jáɬ}{breath}}}

\wordng{Ni}{\mbox{-}jaɬ} (\intxt{\mbox{-}ij}) \citep[338]{JS16} {\sep} \wordng{PCh}{*\mbox{-}jáɬ} > \wordng{Ijw}{\mbox{-}jéɬ}; \wordng{I’w}{\mbox{-}jél}; \wordng{Mj}{\mbox{-}jéɬ} (\citealt{ND09}: 127; \citealt{AG83}: 133; \citealt{JC18}) {\sep} \wordng{PW}{*\mbox{-}jáɬ} > \wordng{LB/Vej}{\mbox{-}jaɬ}; \wordng{’Wk}{\mbox{-}jáɬ\pl{ɬajis}} (\citealt{JB09}: 60; \citealt{VU74}: 83; \citealt{KC16}: 104)

\lit{\citealt{EN84}: 46 (\intxt{*jahl})}

\PMlemma{{\wordnl{*[ji]jǻʔ}{to drink} [1]}}

\wordng{Mk}{<i>jaʔ} \citep[224]{AG99} {\sep} \wordng{Ni}{[ji]jåʔ / \mbox{-}(ʔi)jåʔ} \citep[387]{JS16} {\sep} \wordng{PCh}{*[ʔi]ˀjǻʔ}\gloss{to drink alcohol} [2] > \wordng{Ijw}{[ʔi]ˀjáʔ}; \wordng{I’w}{\mbox{-}jé} [3]; \wordng{Mj}{[ʔi]ˀjéʔ} (\citealt{ND09}: 118; \citealt{AG83}: 186; \citealt{JC18}) {\sep} \wordng{PW}{*[ʔi]jǻʔ} > \wordng{LB}{[ʔi]joʔ}\gloss{to drink water}; \wordng{Vej}{[hi]jå} [4]; \wordng{’Wk}{[ʔi]jǻʔ}\gloss{to drink alcohol} (\citealt{VN14}: 241, 251; \citealt{JB09}: 46; \citealt{MG-MELO15}: 41; \citealt{KC16}: 512)

\dicnote{The underived verb is intransitive. Applicative derivations are used for expressing the ingested substance.}%1

\dicnote{The glottalization in \wordng{PCh}{*ˀj} appears to be irregular (the seemingly plain reflex in Iyo’awujwa’ could be a mistranscription on Gerzenstein’s part).}%2

\dicnote{The absence of a final \intxt{ʔ} in \cits{AG83} data of Iyo’awujwa’ must be a mistranscription.}%3

\dicnote{In \citet[82]{VU74}, the root is mistranscribed as \intxt{\mbox{-}ja}.}%4

\lit{\citealt{EN84}: 15 (2~\intxt{*hl\mbox{-}jae})}

\PMlemma{{\wordnl{*\mbox{-}jáqsiʔ \recind *\mbox{-}jǻqsiʔ}{finger}}}

\word{Mk}{\mbox{-}jaqsiʔ\pl{j}}{finger, claw, ring} \citep[397]{AG99} {\sep} \wordng{PCh}{*\mbox{-}<ʔi>jási\mbox{-}keʔ \recind *\mbox{-}<ʔi>jǻsi\mbox{-}keʔ\pla{jʰ}} [1] > \wordng{I’w}{\mbox{-}jési\mbox{-}kiʔ\pl{ji}}; \wordng{Mj}{\mbox{-}(ʔi)jéxʃi\mbox{-}kiʔ\pl{jh}} [1] (\citealt{AG83}: 134; \citealt{JC18})

\dicnote{We have no explanation for the element \intxt{ʔi\mbox{-}} in the Manjui third-person form (\intxt{t\mbox{-}’ijéxʃi\mbox{-}kiʔ}), which disappears in other inflected forms and lacks a counterpart in Maká.}%1

\gc{Likely related to \word{Proto-Guaicuruan}{*\mbox{-}a(ˀ)jaqats’V}{finger} (\citealt{PVB13b}, \#9; cf. \citealt{PVB13a}: 308).}

\lit{\citealt{PVB13a}: 308 (\intxt{*\mbox{-}jaqsiʔ})}

\PMlemma{\wordnl{*(\mbox{-})jä́ja(ʔ)}{grandmother}}

\word{Ni}{jaja}{grandmother, old woman (possibly vocative)} \citep[495]{LC20} {\sep} \wordng{PCh}{*(\mbox{-})jéjaʔ} > \wordng{Mj}{(\mbox{-})jíjeʔ \recind jíjiʔ} \citep{JC18}

\PMlemma{{\wordnl{*jijáˀts}{dew}}}

\wordng{Mk}{ijeˀts} [1], \textit{ijets\mbox{-}its} \citep[225]{AG99} {\sep} \wordng{Ni}{jijaˀs} \citep[385]{JS16} {\sep} \wordng{PCh}{*ʔijés\mbox{-}tah} > \wordng{Ijw}{jís\mbox{-}ta} [2]; \wordng{I’w}{\mbox{-}jís\mbox{-}ta \recind \mbox{-}jís\mbox{-}te} [2]; Mj~‹ajísta, ijísta› [2] (\citealt{ND09}: 160; \citealt{AG83}: 33, 134; \citealt{GH94}) {\sep} \wordng{PW}{*ʔijás} > \wordng{LB}{\mbox{ʔijas}}; \wordng{’Wk}{ʔijás\pl{lis}} (\citealt{VN14}: 48; \citealt{KC16}: 43)

\dicnote{The presence of a preglottalized coda in Maká is inferred based on the Nivaĉle cognate; the singular form is not attested in our sources that distinguish between plain and preglottalized stops.}%1

\dicnote{The root-initial vowel has suffered irregular change or loss in all Chorote varieties (only in Manjui has the expected form been attested alongside an innovative one).}%2

\gc{\citet[312]{PVB13a} notes the similarity with \word{Proto-Guaicuruan}{*ewi}{dew} (\citealt{PVB13b}, \#245), which could be spurious.}

\lit{\citealt{PVB13a}: 312 (\intxt{*ija\mbox{-}ts})}

\PMlemma{{\wordnl{*jiˀjåˀX₁₂}{jaguar}}}

\wordng{Ni}{jiˀjåˀx\plf{jijxå\mbox{-}s}} (\citealt{JS16}: 386; \citealt{LC20}: 52) {\sep} \wordng{PCh}{*ʔaˀjǻh\pla{es}} > \wordng{I’w}{ajéh\pl{es}}; \wordng{Mj}{ʔaˀjéh}, \textit{ʔaˀjé\mbox{-}es} (\citealt{AG83}: 118; \citealt{JC18}) {\sep} \wordng{PW}{*haˀjåχ} > \wordng{LB}{haˀjoχ}; \wordng{Vej}{haˀjåh\pl{ɬajis}} [1]; \wordng{’Wk}{haˀjåx}, \textit{haˀjǻ\mbox{-}s} (\citealt{VN14}: 53; \citealt{MG-MELO15}: 20; \citealt{KC16}: 141)

\dicnote{\citet[57]{VU74} mistranscribes this word as \textit{hajoh}.}%1

\lit{\citealt{EN84}: 36, 41 (\intxt{*jåq}); \citealt{LC-VG-07}: 20}

\PMlemma{{\intxt{*jiˀlå \recvar jiˀlåʔ\plf{*jiˀlǻ\mbox{-}jʰ}} [1]\gloss{tree}}}

\wordng{Ni}{jiˀk͡låʔ\pl{j}} [2] (\citealt{LC20}: 58) {\sep} \wordng{PCh}{*ʔaˀlǻʔ\pla{jʰ}} > \wordng{Ijw}{ʔaˀláʔ}; \wordng{I’w}{aláʔ\pl{j}} [3]; \wordng{Mj}{ʔaˀláʔ\pl{jh}} (\citealt{JC14b}: 99; \citealt{ND09}: 95; \citealt{AG83}: 119; \citealt{JC18}) {\sep} \wordng{PW}{*haˀlå}, \textit{*haˀlǻ\mbox{-}jʰ} > \wordng{LB}{haˀlo}, \textit{haˀlo\mbox{-}j}; \wordng{Vej}{haˀlå}, \textit{haˀlå\mbox{-}j} [4]; \wordng{’Wk}{haˀlåʔ}, \textit{haˀlǻ\mbox{-}ç} (\citealt{VN14}: 191; \citealt{MG-MELO15}: 18; \citealt{KC16}: 139)

\dicnote{Nivaĉle points to \wordng{PM}{*ʔaˀlåʔ}, Lower Bermejeño Wichí to \intxt{*ʔaˀlå}.}%1

\dicnote{\citet[379]{JS16} documents \textit{jek͡låʔ\pl{j}}\gloss{wood, firewood}, which must be an irregular Shichaam Lhavos form. The basic term for\gloss{tree} in that variety is \textit{ʔaˀkxi\mbox{-}juk} \citep[35]{JS16}, of unknown origin.}%2

\dicnote{The absence of preglottalization in \wordng{I’w}{\mbox{-}l\mbox{-}} in this word is probably a mistranscription on \cits{AG83} part.}%3

\dicnote{\citet[56]{VU74} mistranscribes the Vejoz reflex as \intxt{haˀla \recind hala}.}%4

\lit{\citealt{RJH15}: 239; \citealt{EN84}: 36 (\intxt{*la}); \citealt{AnG15}: 253}

\PMlemma{{\wordnl{*jinǻˀt\plf{*jinǻt\mbox{-}its}}{water}}}

Mk (Guentusé doculect) ‹enaat› [1] \citep{JFA93} {\sep} \wordng{Ni}{jinåˀt}, \textit{jinåt\mbox{-}is / \mbox{-}ˀβ\mbox{-}inåt\pl{is}} (\citealt{JS16}: 361, 382) {\sep} \wordng{PCh}{*ʔiˀnǻt\pla{es}} [2] > \wordng{Ijw}{ʔiˀnʲát}; \wordng{I’w}{ʔanát} [3]; \wordng{Mj}{ʔaˀnát\pl{es}} [3] (\citealt{JC14b}: 99; \citealt{ND09}: 117; \citealt{AG83}: 127; \citealt{JC18}) {\sep} \wordng{PW}{*ʔinǻt\pla{es}} > \wordng{LB}{ʔinot}; \wordng{’Wk}{ʔinǻt\pl{es}} (\citealt{VN14}: 150; \citealt{JB09}: 45; \citealt{KC16}: 31)

\dicnote{In modern Maká, this root has been replaced by \textit{iweliʔ}\gloss{water} (in earlier sources \textit{ewaleʔ}; \citealt{RJH15}: 243).}%1

\dicnote{The glottalization in \wordng{PCh}{*ˀn} appears to be irregular (the seemingly plain reflex in Iyo’awujwa’ could be a mistranscription on Gerzenstein’s part). \wordng{PM}{*ji} evolves to \textit{ʔi} in Iyojwa’aja’, as if it were followed by a plain consonant, but to \textit{ʔa} in Iyo’awujwa’, as expected before an etymological glottalized consonant.}%2

\dicnote{The low vowel in the first syllable in Iyo’awujwa’ and Manjui could be due to the general dispreference for structures of the type \intxt{\#ʔiC’Á…}, where \intxt{C’} stands for a glottalized consonant and \intxt{Á} for a stressed low vowel (these sequences were eliminated in Chorote and Wichí by means of the sound change \textit{*ji\mbox{-} > *ʔi\mbox{-} > *ʔa\mbox{-}} before glottalized consonants followed by stressed vowels).}%3

\lit{\citealt{EN84}: 10, 28, 32, 44 (\intxt{*ihnǻt})}

\PMlemma{{\wordnl{*\{j/ʔ\}is\{a/å/e\}ˀχ \recind *\{j/ʔ\}is\{á/ǻ/é\}ˀχ}{sand}}}

\wordng{Mk}{isaˀχ} [1], \textit{isaχ\mbox{-}its} \citep[207]{AG99} {\sep} \wordng{PCh}{*ʔisáh \recind *ʔisǻh} > \wordng{I’w}{isʲé}; \wordng{Mj}{(ʔi)ʃéh} (\citealt{AG83}: 132; \citealt{JC18})

\dicnote{The preglottalized coda in the singular form in Maká is attested in the New Testament (Hebrews 11:12).}%1

\lit{\citealt{PVB02}: 144 (\intxt{*isʌχ})}

\PMlemma{{\wordnl{*jit’åʔ\plf{*jit’ǻ\mbox{-}l}}{turkey vulture}}}

\wordng{Ni}{jit’åʔ\pl{k}} \citep[384]{JS16} {\sep} \wordng{PCh}{*ʔat’ǻʔ\pla{l}} > \word{Ijw}{ʔat’áʔ\pl{ˀl}}{black vulture}; \word{Mj}{ʔat’áʔ}{turkey vulture; lesser yellow-headed vulture} (\citealt{ND09}: 95; \citealt{JC18}) {\sep} \wordng{PW}{*hat’å} > \wordng{LB}{hat’o}; \wordng{’Wk}{hat’åʔ} (\citealt{CS-FL-PR-VN13}; \citealt{KC16}: 147)

\PMlemma{{\wordnl{*jitsuˀx \recind *jitsúˀx\plf{*jitsx\mbox{-}ǻjʰ}}{male}}}

\wordng{Mk}{ɬe\mbox{-}∅\mbox{-}tsuˀx} [1], \intxt{ɬe\mbox{-}∅\mbox{-}tsux\mbox{-}its} [2] \citep[251]{AG99} {\sep} \word{Ni}{jitsuˀx\plf{jitsx\mbox{-}åj}}{male, man}, \wordnl{\mbox{-}ka\mbox{-}β\mbox{-}tsux\plf{\mbox{-}ka\mbox{-}βi\mbox{-}tsx\mbox{-}åj}}{male relative} \citep[101, 103]{LC20} {\sep} \word{PW}{*ˣtsh<ǻ><wet>\plf{*ˣtsh<ǻ><t>\mbox{-}åjʰ}}{animal} [3] > \wordng{LB}{tsʰowet\plf{tsʰot\mbox{-}oj}}; \wordng{’Wk}{ʔitsʰǻwet\plf{ʔitsʰǻt\mbox{-}åç}} (\citealt{VN14}: 193; \citealt{KC16}: 41)

\dicnote{The presence of a preglottalized coda in Maká is inferred based on the Nivaĉle cognate; the singular form is not attested in our sources that distinguish between plain and preglottalized stops. We assume this form contains a zero allomorph of the relationalizing prefix \intxt{\mbox{-}ˀw\mbox{-}}, parallel to \wordnl{ɬe\mbox{-}ˀw\mbox{-}efu}{female}; /ˀw/ is deleted before a consonant.}%1

\dicnote{The plural form in Maká is non-etymological.}%2

\dicnote{The identity of the element \intxt{\mbox{-}wet / \mbox{-}t\mbox{-}} in Wichí is unclear. It has been fossilized to what looks like an innovative vocalic stem \intxt{*jitsx<ǻ>\mbox{-}} > \intxt{*ˣtsʰ<ǻ>\mbox{-}}.}%3

\PMlemma{{\textit{*jixå \recind *jixǻ \recvar *jixåʔ \recind *jixǻʔ} [1]\gloss{true}}}

\wordng{Mk}{ixa} \citep[219]{AG99} {\sep} \wordng{Ni}{jixåʔ} \citep[381]{JS16} {\sep} \wordng{PCh}{*ʔihǻ<wet>} [2] > \wordng{Ijw}{ʔihját}; \wordng{I’w}{ihjét}; \wordng{Mj}{ʔihjéwet\mbox{-}e} (\citealt{JC14b}: 87; \citealt{ND09}: 96; \citealt{AG83}: 132; \citealt{JC18})

\dicnote{Maká points to the absence of a word-final \intxt{*ʔ} in PM, Nivaĉle to its presence.}%1

\dicnote{We have no explanation for the element \intxt{*\mbox{-}(we)t} in Chorote.}%2

\lit{\citealt{PVB02}: 143 (\intxt{*ixʌ})}

\PMlemma{{\wordnl{*\mbox{-}juˀs / *jijuˀs}{wax}}}

\wordng{Ni}{\mbox{-}juˀs}, \textit{\mbox{-}jus\mbox{-}ik / jijuˀs} (\citealt{JS16}: 69, 391) {\sep} \wordng{PCh}{*ʔijús} > \wordng{I’w}{ijús\pl{is}} \citep[130]{AG83}

\PMlemma{{\wordnl{*\mbox{-}ka\plf{*\mbox{-}ká\mbox{-}l}}{tool; person with skills for}}}

\wordng{Ni}{\mbox{-}tʃaʔ\pl{k}} \citep[94]{JS16} {\sep} \wordng{PCh}{*\mbox{-}káʔ\pla{l}} > \wordng{Ijw}{\mbox{-}kʲéʔ\pl{ˀl}}; \wordng{I’w}{\mbox{-}kʲéʔ\pl{l}}; \wordng{Mj}{\mbox{-}kʲéʔ\pl{ɬ}} (\citealt{JC14b}: 76; \citealt{ND09}: 122; \citealt{AG83}: 117; \citealt{JC18}) {\sep} \wordng{PW}{*\mbox{-}kʲa}, \textit{*\mbox{-}kʲá\mbox{-}lʰ} > \wordng{LB}{\mbox{-}tʃa\pl{ɬ}}; \wordng{Vej}{\mbox{-}tʃa}; \wordng{’Wk}{\mbox{-}kʲaʔ}, \textit{\mbox{-}kʲá\mbox{-}ɬ} (\citealt{VN14}: 150, 201; \citealt{VU74}: 51; \citealt{KC16}: 64)

\PMlemma{{\intxt{*[ji]kaˀχ \recvar *[ji]kåˀχ} [1]\gloss{to take away}}}

\word{Mk}{[j]<e>kaˀχ}{to take away}, \wordnl{[j]<e>\mbox{-}n\mbox{-}kaˀχ}{to bring} [2] \citep[143]{AG99} {\sep} \wordng{Ni}{[ji]tʃaˀx} \citep[94]{JS16} {\sep} \wordng{PW}{*[ʔi]kʲåχ} > \wordng{LB}{[ʔi]tʃoχ}; \word{Vej}{\mbox{-}tʃåh}{to take away, to buy} [3]; \word{’Wk}{[ʔi]kʲåχ}{to take away, to buy} (\citealt{VN14}: 225; \citealt{VU74}: 51; \citealt{MG-MELO15}: 33; \citealt{KC16}: 179)

\dicnote{The Nivaĉle form points to \intxt{*[ji]kaˀχ}, the Wichí one to \intxt{*[ji]kåˀχ}, and Maká is ambiguous, because \sound{PM}{*å}, \intxt{*a} and \intxt{*e} all merged before a \intxt{*χ} in that language.}%1

\dicnote{The function of the element \intxt{\mbox{-}e\mbox{-}} in Maká is unclear, but note that the cislocative prefix \intxt{\mbox{-}n\mbox{-}} comes between it and the (etymological) root in \textit{[j]e\mbox{-}n\mbox{-}kaχ}, showing that it must have originally been a separate morpheme. The preglottalized coda is documented in the New Testament (e.g. Mark 6:29).}%2

\dicnote{\citet{VU74} documents \textit{\mbox{-}tʃah}, which is more likely a mistranscription on Viñas Urquiza’s part rather than a retention from PM.}%3

\rej{\citet{EN84} lists \wordng{Chorote}{aki} and \wordnl{akahaj}{I buy} as cognates. In fact, \wordng{PCh}{*[tᵊ]qahájʔ} or \wordnl{*[tᵊ]qåhǻjʔ}{to buy} (> \wordng{I’w}{\mbox{-}kaháj\mbox{-}i}; \wordng{Mj}{[ti]kahájʔ} and \intxt{[ti]kaháj\mbox{-}e}) cannot be a reflex of \wordng{PM}{*[ji]kaˀχ \recind *[ji]kåˀχ}, because \wordng{PCh}{*q} cannot continue \wordng{PM}{*k} in the onset position. \word{Ijw}{∅\mbox{-}ák\mbox{-}i}{I take away/buy} is in fact a combination of the verb \wordnl{∅\mbox{-}ák}{I go} and the applicative /\mbox{-}eh/, as evidenced by the conjugated forms \wordnl{hl\mbox{-}έk\mbox{-}i}{you take away/buy}, \wordnl{j\mbox{-}ám\mbox{-}e}{he/she takes away/buys}.}

\lit{\citealt{EN84}: 24 (\intxt{*caq}); \citealt{AnG15}: 64}

\PMlemma{{\wordnl{*\mbox{-}kǻn\pla{its}}{testicle}}}

\wordng{Ni}{\mbox{-}kån\mbox{-}ʃij\pl{is}} (\citealt{JS16}: 75; \citealt{LC20}: 130) {\sep} \wordng{PCh}{*\mbox{-}kǻn<is>\pla{is}} [1] > \wordng{Ijw}{\mbox{-}kʲánis\pl{is}}; \wordng{Mj}{\mbox{-}kʲénis}, \textit{\mbox{-}kʲéniʃ\mbox{-}is} (\citealt{ND09}: 122; \citealt{JC18}) {\sep} \wordng{PW}{*\mbox{-}kʲǻn<is>} [1] > \wordng{LB}{\mbox{-}tʃonis}; \wordng{Vej}{\mbox{-}tʃanis} [1]; \wordng{’Wk}{\mbox{-}kʲǻnis}, \textit{\mbox{-}kʲǻhsi\mbox{-}lis} (\citealt{VN14}: 213; \citealt{VU74}: 52; \citealt{KC16}: 63)

\dicnote{In Chorote and Wichí, the PM plural suffix has been fossilized as a part of the root.}%1

\PMlemma{{\wordnl{{*\mbox{-}kåˀs}\plf{*\mbox{-}kås\mbox{-}él}}{tail}}}

\wordng{Ni}{\mbox{-}kåˀs\plf{\mbox{-}kås\mbox{-}ek}} \citep[75]{JS16} {\sep} \wordng{PCh}{*\mbox{-}kǻs} > \wordng{Ijw}{\mbox{-}kʲás}; \wordng{I’w}{\mbox{-}kʲés\plf{\mbox{-}kʲéxs\mbox{-}is}} [1]; \wordng{Mj}{\mbox{-}kʲés} (\citealt{JC14b}: 76; \citealt{ND09}: 122; \citealt{AG83}: 142; \citealt{JC18}) {\sep} \wordng{PW}{*\mbox{-}kʲås\plf{*\mbox{-}kʲǻs\mbox{-}elʰ}} > \wordng{LB}{\mbox{-}tʃos\pl{eɬ}}; \word{Vej}{\mbox{-}tʃås\pl{eɬ}}{tail; lower back} [2]; \wordng{’Wk}{\mbox{-}kʲås\plf{\mbox{-}kʲǻs\mbox{-}eɬ}} (\citealt{VN14}: 191; \citealt{VU74}: 52; \citealt{MG-MELO15}: 60; \citealt{KC16}: 63)

\largerpage
\dicnote{The plural suffix attested by \citet{AG83} for Iyo’awujwa’ does not match the Nivaĉle and Wichí data.}%1

\dicnote{The form is mistranscribed as \intxt{\mbox{-}tʃas} in \citet{VU74}.}%2

\lit{\citealt{EN84}: 27 (\intxt{*cåhs}); \citealt{LC-VG-07}: 17}\clearpage

\PMlemma{{\wordnl{*[ji]kǻˀt\mbox{-}\APPL}{to fall}}}

\wordng{Ni}{[ji]kåˀt\mbox{-}\APPL} \citep[75]{JS16} {\sep} \word{PW}{*[ni]kʲǻt(\mbox{-}\APPL)}{to fall, to be born} > \wordng{LB}{[ni]tʃot\mbox{-}tʃoʔ}; \wordng{Vej}{\mbox{-}tʃat}(\intxt{\mbox{-}}\textsc{appl)} \textsc{[1]}; \wordng{’Wk}{[ni]kʲǻt\mbox{-}\APPL} (\citealt{VN14}: 219, 333; \citealt{VU74}: 52; \citealt{KC16}: 183–184)

\dicnote{The vowel \intxt{a} (as opposed to \intxt{å}) in \citet{VU74} could be a mistranscription.}%1

\PMlemma{{\wordnl{*kéɬχa\mbox{-}juˀk\plf{*kéɬχa\mbox{-}jku\mbox{-}jʰ}}{red quebracho\species{Schinopsis balansae}}; \wordnl{*kéɬχa\mbox{-}jku\mbox{-}ˀp}{fall season}}}

\wordng{Mk}{keɬe\mbox{-}jku\mbox{-}te\mbox{-}ˀk}; \intxt{keɬe\mbox{-}jku\mbox{-}ˀp\pl{its}} (\citealt{AG99}: 229; \citealt{maka-etnomat}: 23–25) {\sep} \word{Ni}{tʃeɬxa\mbox{-}juk\plf{tʃeɬxa\mbox{-}ku\mbox{-}j}}{\textit{Myracrodruon balansae} tree} \citep[97]{JS16} {\sep} \wordng{PCh}{*kéhla\mbox{-}juk}; \textit{*kéhla\mbox{-}jku\mbox{-}p} > \wordng{Ijw}{kíhla\mbox{-}jik}; \textit{kíhla\mbox{-}si\mbox{-}p}; \wordng{I’w}{kíhla\mbox{-}jik}; \wordng{Mj}{kíhl(ʲ)e\mbox{-}ek \recind kíhla\mbox{-}jik \recind kíhli\mbox{-}jik}; \textit{kíhle\mbox{-}ʃe\mbox{-}p} (\citealt{JC14b}: 92; \citealt{ND09}: 136; \citealt{AG83}: 141; \citealt{JC18}) {\sep} \wordng{PW}{*kʲéɬ\mbox{-}jukʷ}, \textit{*kʲéɬ\mbox{-}kʲu\mbox{-}jʰ}; \textit{*kʲéɬ\mbox{-}kʲu\mbox{-}p} > \wordng{LB}{tʃeɬ\mbox{-}jekʷ}, \textit{tʃeɬ\mbox{-}tʃe\mbox{-}j}; \wordng{Vej}{tʃe(ˀ)ɬ\mbox{-}juk}; \textit{tʃeɬ\mbox{-}tʃu\mbox{-}p}; \wordng{’Wk}{kʲéɬ\mbox{-}juk}, \textit{kʲéɬ\mbox{-}kʲu\mbox{-}ç}; \textit{kʲéɬ\mbox{-}kʲu\mbox{-}p} (\citealt{VN14}: 192; \citealt{VU74}: 52; \citealt{MG-MELO15}: 17; \citealt{KC16}: 186, 187)

\lit{\citealt{EN84}: 51 (\intxt{*cɛhlaj(uk)}, \textsc{pl}~\textit{*cɛhlajuk\mbox{-}j}); \citealt{LC-VG-07}: 17}

\PMlemma{{\wordnl{*[ji]kén}{to send}}}

\wordng{Mk}{[j]<u>kin} (\citealt{AG99}: 227, 353) {\sep} \wordng{Ni}{[ji]tʃen} \citep[97]{JS16} {\sep} \wordng{PCh}{*[ʔi]kén} > \wordng{Mj}{[ʔi]ʃín / \mbox{-}kín} \citep{JC18} {\sep} \wordng{PW}{*[ʔi]kʲén} > \wordng{LB/Vej}{\mbox{-}tʃen}; \wordng{’Wk}{[ʔi]kʲén̥} (\citealt{JB09}: 39; \citealt{VU74}: 52; \citealt{KC16}: 188)

\PMlemma{{\wordnl{*kɸá(t)s'i(ʔ)}{Molina's hog-nosed skunk}}}

\wordng{Ni}{kxats’i \recind txats’i} [1] \citep[70]{JS16} {\sep} \wordng{PCh}{*kᵊhwáts’iʔ} > \word{I’w}{kiwáts’eʔ \recind kifʷáts’iʔ}{liar}; \wordng{Mj}{kihwáts’e\pl{s}} [2] (\citealt{AG83}; \citealt{JC18})

\dicnote{The variant \intxt{txats’i} is marked as ``T.~Lh.'' in \citet[70]{JS16}, which likely stands for ``Tavashai Lhavos'' (or maybe ``Tovôc Lhavos'').}%1

\dicnote{The absence of a stem-final \intxt{\mbox{-}ʔ} in the singular form in Manjui could be due to a mistranscription.}%2

\PMlemma{{\intxt{*\mbox{-}kɸe(ʔ)\plf{*\mbox{-}kɸé\mbox{-}jʰ}} [1]\gloss{ear} [2]}}

\word{Mk}{\mbox{-}kfiʔ\pl{j}}{ear; corner} (\citealt{AG99}: 143, 250) {\sep} \wordng{Ni}{\mbox{-}kɸeʔ\pl{j}} \citep[69]{JS16} {\sep} \wordng{PW}{*\mbox{-}(t\mbox{-})kʷe<j>}, \intxt{*\mbox{-}(t\mbox{-})kʷe\mbox{-}} (in compounds)\gloss{arm, hand} > \wordng{LB}{\mbox{-}t\mbox{-}kʷe<j>\pl{aj}}; \textit{\mbox{-}t\mbox{-}kʷe\mbox{-}} (in compounds); \wordng{Vej}{\mbox{-}kʷe<j>}; \wordng{’Wk}{\mbox{-}k(ʷ)e<j>ʔ\plf{\mbox{-}k(ʷ)é<j>\mbox{-}aç \recind \mbox{-}eç}}, \third{ta\mbox{-}ke<j>ʔ}; \textit{\mbox{-}ke\mbox{-}}, \third{ta\mbox{-}ké\mbox{-}} (in compounds) (\citealt{VN14}: 112, 154, 164; \citealt{VU74}: 63; \citealt{MG-MELO15}: 60, 61; \citealt{AFG067}: 214, 215; \citealt{KC16}: 62)

\dicnote{The uncertainty regarding the reconstruction of the word-final glottal stop is due to the fact that the Lower Bermejeño Wichí reflex never occurs without a suffix.}%1

\dicnote{Following \citet[29]{EN84}, we suggest that \wordng{PW}{*\mbox{-}(ta\mbox{-})kʷe<j>} (in compounds \intxt{*\mbox{-}(ta\mbox{-})kʷe\mbox{-}})\gloss{arm, hand} is a semantically shifted reflex of \word{PM}{*\mbox{-}kɸe(ʔ)\pla{jʰ}}{ear}. Despite the semantic difference, cases of colexification of the concepts such as\gloss{ear} and\gloss{shoulder} do exist (cf. \citealt{CLICS-ear-shoulder}). Also note that the Wichí word contains the prefix \intxt{\mbox{-}t(a)\mbox{-}}, found in a number of body part terms (\citealt{VN14}: 164–165) and absent in the proposed cognates in other languages; it is conceivable that the Wichí term for\gloss{arm, hand} arose as a compound whose original meaning was close to\gloss{ear of body}.}%2

\rej{\citet[15, 17]{LC-VG-07} claim the Wichí noun to be cognate with the Maká and Chorote reflexes of \word{PM}{*\mbox{-}ko(ˀ)j\pla{ájʰ}}{hand, arm}. This is impossible for phonological reasons.}

\lit{\citealt{EN84}: 29 (\intxt{*takhwɛj}); \citealt{AnG15}: 77}

\PMlemma{{\wordnl{*[ji]kɸ’äs \recind *[ji]kɸ’ä́s}{to be torn open} [1], \textsc{caus}~\intxt{*[ji]kɸ’ä́s\mbox{-}at} [2]}}

\word{Ni}{[ji]k’as\mbox{-}\APPL}{to break up into pieces}, \textsc{caus}~\intxt{[ji]k’as\mbox{-}at} \citep[304]{LC20} {\sep} \wordng{PCh}{*[ʔi]k’(w)ós}, \textsc{caus}~\intxt{*[ʔi]k’(w)ós\mbox{-}at} > \wordng{Mj}{[ʔi]tʃ’ós / \mbox{-}ʔɔ́s}, \textsc{caus}~\intxt{ˀ[j]óxs\mbox{-}at} \citep{JC18} {\sep} \wordng{PW}{*[hi]kʷ’es \recind *[hi]kʷ’és} [1] > \wordng{LB}{[hi]kʷ’es}; \wordng{’Wk}{[hi]k’és\mbox{-}kʲeʔ} (\citealt{VN14}: 49, 263; \citealt{KC16}: 179)

\dicnote{The prosodic properties of the root cannot be established because the ’Weenhayek cognate is not attested without applicative morphology (the form with an applicative suffix is not revealing because in trisyllabic words the vowel of the peninitial syllable is lengthened in any case).}%1

\dicnote{The reconstruction of the cluster \intxt{*kɸ’} is rather tentative. It aims to account for the unique vowel correspondence between Chorote and the remaining languages, and for the \sound{PW}{*kʷ’}, an extremely rare consonant. We do not exclude the possibility that \word{Mk}{\mbox{-}apk’as}{piece} \citep[248]{AG99} is also related, but \sound{Mk}{a} is not a regular reflex of \sound{PM}{*ä}.}%2

\PMlemma{{\intxt{*khǻt} (fruit); \intxt{*khǻt\mbox{-}uˀk\plf{*khǻt\mbox{-}ku\mbox{-}jʰ}} (plant)\gloss{cactus}}}

\wordng{Mk}{khat\mbox{-}uˀk} [1], \textit{khat\mbox{-}kw\mbox{-}i}\gloss{\textit{Cereus stenegonus}} \citep[230]{AG99} {\sep} \wordng{Ni}{kxat}; \textit{kxat\mbox{-}uk}, \textit{kxat\mbox{-}ku\mbox{-}j} [2] \citep[69]{JS16} {\sep} \wordng{PCh}{*kåhǻt}; \wordnl{*kåhǻt\mbox{-}uk\plf{*kåhǻt\mbox{-}ku\mbox{-}jʰ}}{\textit{Cereus forbesii}} > \wordng{Ijw}{kʲahátʲ\mbox{-}uk}; \wordng{Mj}{kʲehét}; \intxt{kʲehét\mbox{-}uk\plf{kʲehét\mbox{-}ki\mbox{-}j}} (\citealt{ND09}: 135; \citealt{JC18}) {\sep} \wordng{PW}{*kʲåhǻt}; \textit{*kʲåhǻt\mbox{-}ukʷ} > \wordng{LB}{tʃohot\mbox{-}ekʷ}; \wordng{Vej}{tʃåhåt\pl{ɬajis}}, \wordng{’Wk}{kʲåhǻt}; \textit{kʲåhǻt\mbox{-}uk} (\citealt{CS08}: 60; \citealt{MG-MELO15}: 17; \citealt{KC16}: 179)

\dicnote{The preglottalized coda in the Maká suffix for tree names is attested elsewhere \citep[7]{PMA}.}%1

\dicnote{\wordng{Ni}{a} is not the regular reflex of \wordng{PM}{*ǻ}.}%2

\lit{\citealt{LC-VG-07}: 16}

\PMlemma{{\textit{*\mbox{-}kíɸah\plf{*\mbox{-}kíɸa\mbox{-}ts}} (m.); \textit{*\mbox{-}kíɸa\mbox{-}keʔ\pla{jʰ}} (f.)\gloss{neighbor} [1]}}

\wordng{Mk}{\mbox{-}kife\pl{ts}}; \textit{\mbox{-}kife\mbox{-}kiʔ\pl{j}} \citep[230]{AG99} {\sep} \wordng{Ni}{\mbox{-}tʃiɸa\pl{s}}\gloss{fellow resident of the same village} \citep[101]{JS16} {\sep} \wordng{PCh}{*\mbox{-}kíhwah}, \textit{*\mbox{-}kíhwa\mbox{-}s}; \textit{*\mbox{-}kíhwa\mbox{-}keʔ} > \word{Ijw}{\mbox{-}kíhwa}{partner}; \wordng{Mj}{\mbox{-}kíhwa\pl{s}}; \textit{\mbox{-}kíhwa\mbox{-}kiʔ} (\citealt{ND09}: 122; \citealt{JC18})

\dicnote{This noun obviously contains the suffix \wordnl{*\mbox{-}ɸah\plf{*\mbox{-}ɸa\mbox{-}ts}}{companion}.}%1

\PMlemma{{\intxt{*kijápo(ˀ)p \recvar *k’ijápo(ˀ)p} [1]\gloss{common potoo\species{Nyctibius griseus}}}}

\wordng{Ni}{tʃ'ijapop\pl{is}} \citep[110]{JS16} {\sep} (?) \wordng{PCh}{*qalápop} [2] > \wordng{Ijw}{kalápap}; Mj~{kalápup} [3] (\citealt{ND09}: 134; \citealt{JC18}) {\sep} \wordng{PW}{*kʲijápop} > \wordng{LB}{tʃijapup}; \wordng{’Wk}{kʲijápop} (\citealt{VN14}: 157; \citealt{KC16}: 192)

\dicnote{The Nivaĉle reflex points to \sound{PM}{*k’}, and the Wichí one to \sound{PM}{*k}.}%1

\dicnote{The Chorote form is divergent, casting doubts on whether it is related to the Nivaĉle and Wichí forms.}%2

\dicnote{\citet{GH94} documents the Manjui form as \intxt{kalápap}.}%3

\gc{Toba–Qom shows a similar form, \wordnl{qapap \recind qopap}{common potoo} \citep[167]{ASB-LLB-13}.}

\PMlemma{{\wordnl{*\mbox{-}kiláʔ\pla{wot}}{elder brother}}}

\wordng{Ni}{\mbox{-}tʃek͡laʔ\pl{βot}} [1]; \wordnl{\mbox{-}tʃik͡la\mbox{-}jinxat}{deceased elder brother} \citep[100]{JS16} {\sep} \wordng{PCh}{*\mbox{-}kiláʔ\pla{wot}} > \wordng{Ijw}{\mbox{-}kílʲ<a>} [2], \textit{\mbox{-}kílʲe\mbox{-}wot}; \wordng{I’w}{\mbox{-}kilʲéʔ}; \wordng{Mj}{\mbox{-}kil(ʲ)éʔ\pl{wat}} (\citealt{ND09}: 122; \citealt{AG83}: 139; \citealt{JC18}) {\sep} \wordng{PW}{*\mbox{-}kʲíla\pla{lis}} [3] > \wordng{LB}{\mbox{-}tʃila}; \wordng{Vej}{\mbox{-}tʃila\pl{lis}};\wordng{’Wk}{\mbox{-}kʲílaʔ\pl{lis}} (\citealt{VN14}: 194; \citealt{VU74}: 53; \citealt{MG-MELO15}: 29; \citealt{KC16}: 65)

\dicnote{The vowel \intxt{e} in Nivaĉle is irregular. The expected vowel \intxt{i} shows up in \textit{\mbox{-}tʃik͡la\mbox{-}jinxat}\gloss{deceased elder brother}.}%1

\dicnote{The Iyojwa’aja’ reflex \intxt{\mbox{-}kílʲa} /\mbox{-}kʲílåh/ is irregular. One would expect \intxt{*\mbox{-}kilʲéʔ} /\mbox{-}kʲilá/. Maybe this noun contains an opaque suffix /\mbox{-}åh/ (not present in the plural form).}%2

\dicnote{The Wichí plural suffix does not match its Nivaĉle and Chorote counterparts and must be innovative.}%3

\lit{\citealt{EN84}: 50 (\intxt{*c’ɛjlǻ})}

\PMlemma{{\wordnl{*\mbox{-}kitáʔ\pla{wot}}{elder sister}}}

\wordng{Ni}{\mbox{-}tʃitaʔ\pl{βot}} (\citealt{JS16}: 103–104) {\sep} \wordng{PCh}{*\mbox{-}kitáʔ\pla{wot}} > \wordng{Ijw}{\mbox{-}kítʲ<a>} [1], \textit{\mbox{-}kítʲe\mbox{-}wot}; \wordng{Mj}{\mbox{-}kitéʔ} (\intxt{\mbox{-}wot}) (\citealt{ND09}: 122; \citealt{JC18}) {\sep} \wordng{PW}{*\mbox{-}kʲíta\pla{lis}} [2] > \wordng{LB}{\mbox{-}tʃita}; \wordng{Vej}{\mbox{-}tʃita\pl{lis}}; \wordng{’Wk}{\mbox{-}kʲítaʔ\pl{lis}} (\citealt{VN14}: 194; \citealt{VU74}: 53; \citealt{MG-MELO15}: 29; \citealt{KC16}: 65)

\dicnote{The Iyojwa’aja’ reflex \textit{\mbox{-}kítʲa} /\mbox{-}kʲílåh/ is irregular. One would expect \intxt{*\mbox{-}kitʲéʔ} /\mbox{-}kʲitá/. Maybe this noun contains an opaque suffix /\mbox{-}åh/ (not present in the plural form).}%1

\dicnote{The Wichí plural suffix does not match its Nivaĉle and Chorote counterparts and must be innovative.}%2

\lit{\citealt{EN84}: 50 (\intxt{*c’ɛjtǻ})}

\PMlemma{{\wordnl{*\mbox{-}ko(ˀ)j\plf{*\mbox{-}koj\mbox{-}ájʰ}}{hand, arm}}}

\word{Mk}{\mbox{-}koj\pl{ej}}{hand, arm, forearm} \citep[232]{AG99} {\sep} \wordng{PCh}{*\mbox{-}kójʔ}, \textit{*\mbox{-}koj\mbox{-}ájʰ} > \wordng{Ijw}{\mbox{-}kʲóʔ}, \textit{\mbox{-}kʲój\mbox{-}e}; \wordng{I’w}{\mbox{-}kʲój}, \textit{\mbox{-}kij\mbox{-}éj}; \wordng{Mj}{\mbox{-}kʲójʔ}, \textit{\mbox{-}kij\mbox{-}éjh} (\citealt{JC14b}: 77, 100; \citealt{ND09}: 122; \citealt{AG83}: 143; \citealt{JC18})

\rej{\citet[15, 17]{LC-VG-07} include the Wichí noun for\gloss{hand, arm} (\wordng{PW}{*\mbox{-}(t\mbox{-})kʷe<j> / *\mbox{-}(t\mbox{-})kʷe\mbox{-}}), which is impossible for phonological reasons. It is considered here to be a reflex of \word{PM}{*\mbox{-}kɸe(ʔ)}{ear} instead.}

\lit{\citealt{LC-VG-07}: 15}

\PMlemma{{\intxt{*kój\mbox{-} \recvar *k’ój\mbox{-}\APPL} [1]\gloss{round}}}

\wordng{Mk}{k’oːj\mbox{-}xiʔ}, \textit{k’oːj\mbox{-}om\mbox{-}xiʔ}\gloss{round (2D), disk\mbox{-}shaped} \citep[237]{AG99} {\sep} \wordng{PCh}{*kój<oj>\mbox{-}\APPL} > \word{Ijw}{kʲójo\mbox{-}ts’i}{cylindrical}, \wordnl{kʲójohj\mbox{-}iˀn}{round}; \wordng{I’w}{kʲójo\mbox{-}xiʔ}; \wordng{Mj}{ʔέti kʲójhjo\mbox{-}oj} (\citealt{ND09}: 136; \citealt{AG83}: 143; \citealt{JC18})

\dicnote{The Maká form points to \sound{PM}{*k’}, and the Chorote one to \sound{PM}{*k}.}%1

\PMlemma{{\wordnl{*[t]kúˀj\mbox{-}\APPL}{to vomit}; \intxt{*\mbox{-}kúj\mbox{-}hat \recvar *\mbox{-}kúj\mbox{-}et} [1]\gloss{vomit}}}

\wordng{Mk}{[t]<’e>kuj(i)\mbox{-}kij} [2]\gloss{to vomit} \citep[144]{AG99} {\sep} \wordng{Ni}{[t(’a)]kuˀj\mbox{-}\APPL}\gloss{to vomit} [2]; \textit{\mbox{-}kuj\mbox{-}et \recind \mbox{-}kuj\mbox{-}it}, \textit{kuj\mbox{-}te\mbox{-}s}\gloss{vomit} (\citealt{JS16}: 83, 282) {\sep} \wordng{PCh}{*[tᵊ]qúj\mbox{-}ˀn}, \textit{*[tᵊ]qúj\mbox{-}eh} [3]\gloss{to vomit} > \wordng{Ijw}{[ta]kó\mbox{-}ˀn<i>}, \textit{[ta]kój\mbox{-}i}; \wordng{I’w}{\mbox{-}kó\mbox{-}hin}; \wordng{Mj}{[ʔi]kʲúj\mbox{-}ʔin / \mbox{-}kʊ́j\mbox{-}ʔin}; \textit{*\mbox{-}qú<h>j<at>} [3]\gloss{vomit} > \wordng{Ijw}{\mbox{-}kóhjet}; \wordng{Mj}{\mbox{-}kʊ́hjet} (\citealt{ND09}: 123, 149; \citealt{AG83}: 144; \citealt{JC18}) {\sep} \wordng{PW}{*[t]kʲʼúj\mbox{-}\APPL} [4]\gloss{to vomit} > \wordng{LB}{[ta]tʃʼej\mbox{-}ɬin}; \wordng{Vej}{[ta]tʃʼuj\mbox{-}\APPL}; \wordng{’Wk}{[t(a)]kʲʼú\mbox{-}\APPL} [5]; \textit{*\mbox{-}kʲʼúj<hat>} [4]\gloss{vomit} > \wordng{Vej}{\mbox{-}tʃʼúçat}; \wordng{’Wk}{\mbox{-}kʲʼúçat} (\intxt{\mbox{-}kʲʼúçt\mbox{-}es}) (\citealt{JB09}: 56; \citealt{VU74}: 54; \citealt{MG-MELO15}: 33, 47; \citealt{KC16}: 69, 209, 365)

\dicnote{Nivaĉle points to \intxt{*\mbox{-}kúj\mbox{-}et \recind *\mbox{-}kúj\mbox{-}it}, and Chorote and Wichí to \intxt{*\mbox{-}kúj\mbox{-}hat}.}%1

\dicnote{We have no explanation for the element \intxt{\mbox{-}’e\mbox{-}} in Maká and its likely cognate \intxt{\mbox{-}’a\mbox{-}} in Nivaĉle (in the latter language, it disappears in some inflected forms).}%2

\dicnote{In the Chorote reflex, \sound{PM}{*k} unexpectedly yields \sound{PCh}{*q}.}%3

\dicnote{The glottalization in \sound{Wichí}{*kʲʼ} is irregular.}%4

\dicnote{The loss of \sound{PW}{*j} in the ’Weenhayek verb is irregular (compare \wordng{Vej}{le\mbox{-}ta\mbox{-}tʃ’uj\mbox{-}ɬi} and \wordng{’Wk}{la\mbox{-}tá\mbox{-}kʲʼu\mbox{-}ɬih}, both meaning\gloss{you vomit}).}%5

\PMlemma{{\wordnl{*kulaˀj \recind *kuláˀj}{sun}}}

\wordng{Ni}{<xum>kuk͡laˀj} [1] \citep[158]{JS16} {\sep} \wordng{PCh}{*kʲulájʔ} > \wordng{Ijw}{kilʲéʔ \recind kiliʔé}; \wordng{I’w}{kiláj}; \wordng{Mj}{kilájʔ} (\citealt{JC14b}: 92; \citealt{ND09}: 136; \citealt{AG83}: 139; \citealt{JC18})

\dicnote{The element \intxt{xum\mbox{-}} of unknown origin occurs in a number of Nivaĉle words whose cognates in other languages lack any counterpart thereof, suggesting that it was etymologically a prefix.}%1

\lit{\citealt{EN84}: 33, 38 (\intxt{*(hnu)culaj})}

\PMlemma{{\wordnl{*[ji]kúˀɬ}{to answer}}}

\wordng{Mk}{[j]<e>kuˀɬ} [1] \citep[144]{AG99} {\sep} \wordng{Ni}{[ji]kuˀɬ} \citep[82]{JS16} {\sep} \wordng{PCh}{*[ʔi]kúhl\mbox{-}\APPL} > \wordng{Ijw}{[ʔi]sʲúhl\mbox{-}i / \mbox{-}kʲúhl\mbox{-}i}; \wordng{Mj}{[ʔi]ʃúhl\mbox{-}{\APPL} / \mbox{-}kʲúhl\mbox{-}\APPL} (\citealt{ND09}: 112; \citealt{JC18}) {\sep} \wordng{PW}{*[ni]kʲúɬ} > \wordng{LB}{[ni]ˈtʃeɬ\mbox{-}u}; \wordng{Vej}{\mbox{-}tʃuɬ\mbox{-}o}; \wordng{’Wk}{[ni]kʲúɬ} (\citealt{VN14}: 402; \citealt{VU74}: 53; \citealt{KC16}: 198)

\dicnote{The preglottalized coda in Maká is attested in the New Testament (e.g. Luke 19:34).}%1

\PMlemma{{\wordnl{*[t]kúˀm\mbox{-}\APPL}{to grab; to work}}}

\wordng{Mk}{[t(’)e]kuˀm\mbox{-}\APPL} [1] \citep[144]{AG99} {\sep} \wordng{Ni}{[t’a]kuˀm\mbox{-}\APPL} \citep[282]{JS16} {\sep} \wordng{PCh}{*[ʔi]kúm\mbox{-}\APPL} > \word{Ijw}{[ʔi]síˀm / \mbox{-}kíˀm}{to grab}, \wordnl{[ʔi]síhm\mbox{-}eˀn / \mbox{-}kíhm\mbox{-}eˀn}{to work} [2]; \word{I’w}{\mbox{-}kíˀm\mbox{-}eʔ}{to grab}, \wordnl{\mbox{-}kíhm\mbox{-}en}{to work} [2]; \wordng{Mj}{[ʔi]ʃúm\mbox{-}{\APPL} / \mbox{-}kʲúm\mbox{-}\APPL} (\citealt{JC14b}: 90; \citealt{ND09}: 111–112; \citealt{AG83}: 140, 141; \citealt{JC18}) {\sep} \wordng{PW}{*[t]kʲú(ˀ)m\mbox{-}\APPL} > \wordng{LB}{[ta]tʃem\mbox{-}\APPL}; \wordng{Vej}{\mbox{-}tʃum\mbox{-}\APPL}; \wordng{’Wk}{[t(a)]kʲú(ˀ)m\mbox{-}\APPL} (\citealt{VN14}: 238; \citealt{VU74}: 53; \citealt{KC16}: 360–362)

\dicnote{The preglottalized coda in Maká is attested in the New Testament (e.g. Luke 7:41; Luke 24:43; Mark 14:46).}%1

\dicnote{Of all Chorote varieties, only Manjui preserves the etymological vowel \intxt{u}. Iyojwa’aja’ and Iyo’awujwa’ show a non-palatalizing \intxt{i} (underlying /e/).}%2

\lit{\citealt{EN84}: 16, 51 (\intxt{*cuhmɛ})}

\PMlemma{{\wordnl{*\mbox{-}kun \recind *\mbox{-}kún}{to eat (intr.)}; \textsc{caus}~\intxt{*[ʔi]kún\mbox{-}han}\gloss{to feed}}}

\wordng{Mk}{[j]<e>kun\mbox{-}hen} [1]\gloss{to feed} \citep[142]{AG99} {\sep} \wordng{Ni}{<tsak>kun} [2]; \textit{[ji]kun\mbox{-}xan} (\citealt{JS16}: 81, 291) {\sep} \wordng{PCh}{*[tᵊ]<ˀjá>kun}\gloss{to eat (intr.)} [3] > \wordng{Ijw}{[ti]ˀjékʲuˀn}; \wordng{I’w}{\mbox{-}jékʲun}; \wordng{Mj}{[ti]ˀjékin}; \textit{*[ʔi]qúhn\mbox{-}an}\gloss{to feed} [4] > \wordng{Ijw}{[ta]kóhnʲ\mbox{-}eˀn}; \wordng{I’w}{\mbox{-}kóhn\mbox{-}an}; \textit{[ʔi]kʲúhn\mbox{-}an / \mbox{-}kʊ́hn\mbox{-}an} (\citealt{ND09}: 149, 152; \citealt{AG83}: 144; \citealt{JC18}) {\sep} \wordng{PW}{*[ʔi]kʲún<han>}\gloss{to feed} > \wordng{LB}{[ʔi]tʃen̥an}; \wordng{Vej}{\mbox{-}tʃun̥en} [5]; \wordng{’Wk}{[ʔi]kʲún̥an̥} (\citealt{VN14}: 336; \citealt{VU74}: 53; \citealt{KC16}: 199)

\dicnote{We have no explanation for the element \intxt{e\mbox{-}} in Maká.}%1

\dicnote{We have no explanation for the element \intxt{tsak\mbox{-}} in Nivaĉle. Note that this verb belongs to the \textit{t\mbox{-}}class and thus contains the zero allomorph of the prefix \intxt{t\mbox{-}} in the third-person realis form.}%2

\dicnote{We have no explanation for the element \intxt{*\mbox{-}ˀjá\mbox{-}} in Chorote.}%3

\dicnote{In the Chorote causative, \wordng{PM}{*k} unexpectedly yields \wordng{PCh}{*q}.}%4

\dicnote{The vowel in the causative suffix is unexpectedly attested as \intxt{e} (rather than \intxt{a}) in the Vejoz reflex.}%5

\rej{\citet[28]{EN84} compares the Wichí causative with \wordng{Ijw}{kʲúnʲe}\gloss{jaguar} \citep[137]{ND09}, which is impossible both for phonological and semantic reasons.}

\PMlemma{{\textit{*kús}\gloss{heat}}}

(?) \wordng{Mk}{kus\pl{its}}\gloss{\textit{Pyrocephalus rubinus}} [1] \citep[233]{AG99} [1] {\sep} \wordng{Ni}{kus\pl{ik}} \citep[81]{JS16} {\sep} \wordng{PCh}{*kús\mbox{-}\APPL}\gloss{to be hot} > \wordng{I’w}{kʲúxs\mbox{-}\APPL}; \wordng{Mj}{kʲúxʃ\mbox{-}\APPL} (\citealt{AG83}: 144; \citealt{JC18})

\dicnote{The semantic relation between the Maká ornithonym \textit{kus} and the PM term for\gloss{heat} may have something to do with the seasonal migration pattern of \textit{Pyrocephalus rubinus}.}%1

\rej{\citet[12]{EN84} compares \wordng{Nivaĉle}{kus} with the Wichí and Chorote terms for\gloss{sweat} (\wordng{PW}{*kʲúxʷ}, \wordng{PCh}{*\mbox{-}kúniʔ}) and reconstructs \wordng{PM}{*cu}\gloss{heat}. This is impossible for phonological reasons.}

\lit{\citealt{LC-VG-07}: 15}

\PMlemma{{\wordnl{*\mbox{-}kút\mbox{-}ex}{to meet}}}

\wordng{Mk}{[w(e)]kut\mbox{-}ix\mbox{-}uˀɬ} [1] \citep[365]{AG99} {\sep} \wordng{Ni}{[βa]kut\mbox{-}eʃ} \citep[81]{JS16} {\sep} \wordng{PCh}{*[ʔi]kút\mbox{-}eh} > \wordng{Ijw}{[ʔi]sʲút\mbox{-}i / \mbox{-}kʲút\mbox{-}i}; \wordng{I’w}{\mbox{-}kʲút\mbox{-}eʔ} [2]; \wordng{Mj}{[ʔi]ʃút\mbox{-}e / \mbox{-}kʲút\mbox{-}e} (\citealt{ND09}: 112; \citealt{AG83}: 143; \citealt{JC18}) {\sep} \wordng{PW}{*\mbox{-}kʲút\mbox{-}eχ} > \wordng{Vej}{\mbox{-}tʃut\mbox{-}eh}; \wordng{’Wk}{[ni]kʲút\mbox{-}ex} (\citealt{VU74}: 54; \citealt{KC16}: 200)

\dicnote{The preglottalized coda in the Maká applicative suffix is attested in other verbs in the New Testament (e.g. \wordnl{[t]’ekuˀm\mbox{-}ixuˀɬ}{to grab something from one’s front} in Luke 24:43).}%1

\dicnote{The stem-final glottal stop in Iyo’awujwa’ must be a mistranscription on \cits{AG83} part.}%2

\PMlemma{{\wordnl{*kúˀX₁₂}{sweat}}}

\wordng{Ni}{\mbox{-}ˀβ\mbox{-}kuˀx}, \textit{\mbox{-}ˀβ\mbox{-}kux\mbox{-}is} \citep[336]{JS16} {\sep} \wordng{PW}{*kʲúxʷ} > \wordng{LB}{tʃefʷ ʔi\mbox{-}lon} X\gloss{X sweats} (literally\gloss{\textit{tʃefʷ} kills X}); \wordng{Vej}{tʃuhʷ} [1]; \wordng{’Wk}{kúˀx} (\citealt{JB09}: 39; \citealt{VU74}: 53; \citealt{KC16}: 196)

\dicnote{Attested without the labialization of the final consonant (\intxt{tʃuh}) in \citet[221]{AFG067}.}%1

\rej{\citet[12]{EN84} compares the Wichí word with \word{Nivaĉle}{kus}{heat} and with the Chorote term for\gloss{sweat} (\wordng{PCh}{*\mbox{-}kúniʔ}) and reconstructs \wordng{PM}{*cu}\gloss{heat}. This is impossible for phonological reasons.}

\PMlemma{{\wordnl{*k’alxó\plf{*k’alxó\mbox{-}ts}}{southern three-banded armadillo}}}

\wordng{Mk}{k’oloˀx\pl{its}} [1] \citep[237]{AG99} {\sep} \wordng{Ni}{k’akxo\pl{s}} [2] \citep[84]{JS16} {\sep} \wordng{PCh}{*k’ihlóʔ\pla{s}} [3] > \wordng{Ijw}{k’ihlʲóʔ}; \wordng{I’w}{ʔihlʲóʔ}, \textit{ʔihl\mbox{-}ís}; \wordng{Mj}{ʔihl(ʲ)óʔ\pl{s}} (\citealt{JC14b}: 82; \citealt{ND09}: 137; \citealt{AG83}: 132; \citealt{JC18}) {\sep} \wordng{PW}{*kʲ’anhóh} > \wordng{LB}{tʃ’an̥u}; \wordng{Vej}{tʃ’en̥o} [4]; \wordng{’Wk}{kʲ’an̥óh} (\citealt{VN14}: 51; \citealt{MG-MELO15}: 20; \citealt{KC16}: 204)

\dicnote{The singular form in Maká was first reshaped based on the PM plural form (\intxt{*k’alxóh\plf{*k’alxó\mbox{-}ts}} > \intxt{*k’oloˀx\plf{*k’olxó\mbox{-}ts}}); later the plural form was reshaped based on the innovative singular one (\intxt{k’oloˀx\plf{k’olox\mbox{-}its}}). One would expect \intxt{*k’olxo\pla{ts}}. The preglottalized coda in the singular form is attested in \citet[51]{JB81}.}%1

\dicnote{The failure of \wordng{PM}{*k’} to palatalize in Nivaĉle is unexpected.}%2

\dicnote{The development of \wordng{PM}{*a} to \wordng{Chorote}{i} is not known to be regular.}%3

\dicnote{Vejoz \textit{e} is not the regular reflex of \sound{PW}{*a}. The datum is mistranscribed as \intxt{tʃ’eno} in \citet[54]{VU74}.}%4

\rej{\citet{EN84} compares \wordng{Ni}{k’akxo} with the Wichí term for\gloss{big hairy armadillo} (\wordng{PW}{*hówanaχ}) and reconstructs \wordng{PM}{*qɔ}. The comparison is untenable.}

\lit{\citealt{EN84}: 48 (\intxt{*cɛhl(h)nɔ})}

\PMlemma{{\wordnl{*[t]k’aw\mbox{-}\APPL}{to hold in one’s arms, to hug} [1]}}

\wordng{Mk}{[t]<i>k’ej\mbox{-}ix} [2] \citep[196]{AG99} {\sep} \wordng{PCh}{*[ʔi]k’aw\mbox{-}(…)\mbox{-}hop} > \wordng{Ijw}{[ʔi]tsʲ’éhw\mbox{-}ap / \mbox{-}kʲ’éhw\mbox{-}ap}; \wordng{I’w}{\mbox{-}kʲafʷ<él>\mbox{-}ap} [3 4]; \wordng{Mj}{[ʔi]tʃ’e<h>w<έ>h<l>\mbox{-}ap / \mbox{-}ʔa<h>w<έ>h<l>\mbox{-}ap} [4]\gloss{to raise with one’s arms}, \intxt{[ʔi]tʃ’e<h>w<έl>\mbox{-}e / \mbox{-}ʔa<h>w<έl>\mbox{-}e} [4]\gloss{to raise or hold with one’s arms} (\citealt{ND09}: 114; \citealt{AG83}: 141; \citealt{JC18}) {\sep} \wordng{PW}{*[t]kʲ’áw\mbox{-}eχ} > \wordng{’Wk}{[t(a)]ˈkʲ’áw\mbox{-}ex} \citep[364]{KC16}

\dicnote{This constitutes one of the few cases of potential \sound{PM}{*w} in coda position. Since in Chorote this stem is documented without an applicative (with an NP followed by a postposition instead) it is reasonable to assume this also existed in PM.}%1

\dicnote{\sound{Maká}{j} is not the expected reflex of \sound{PM}{*w}. It is possible that \wordng{Mk}{[t]<i>k’aw}\gloss{to have sex} \citep[196]{AG99} is also related, with the expected consonant \intxt{w} but with an unexpected lowered vowel.}%2

\dicnote{\cits{AG83} attestation of the Iyo’awujwa’ reflex must be a mistranscription for \intxt{\mbox{-}kʲ’afʷéhlap}.}%3

\dicnote{The Iyo’awujwa’ and Manjui include the element /\mbox{-}hwél/, originally a reflex of \word{PM}{*\mbox{-}ɸél \recind *\mbox{-}ɸä́l}{to wrap, to hug}.}%4

\PMlemma{{\wordnl{*\mbox{-}k’ǻxeʔ\pla{l}}{arrow (made of wood)}}}

\wordng{Mk}{(\mbox{-})qaxiʔ\pl{l}} [1] \citep[304]{AG99} {\sep} \wordng{Ni}{<βat>k’åxe\pl{j}} [2]\gloss{diesel tree (\intxt{Copaifera langsdorffii}; wood used for making arrows)} \citep[343]{JS16} {\sep} \wordng{PCh}{*\mbox{-}k’ǻheʔ\pla{l}} > \wordng{Ijw}{\mbox{-}kʲ’áhaʔ\pl{ˀl}}; \wordng{Mj}{\mbox{-}éheʔ} [3] (\citealt{ND09}: 123; \citealt{AG83}: 198) {\sep} \wordng{PW}{*\mbox{-}kʲ’ǻhe\pla{lʰ}} > LB~\textsc{pl}~\textit{\mbox{-}tʃ’ohe\mbox{-}ɬ}; \wordng{Vej}{\mbox{-}tʃ’åhni} [4]; \wordng{’Wk}{\mbox{-}kʲ’ǻhaʔ\pl{ɬ}} [5] (\citealt{VN14}: 331; \citealt{VU74}: 54; \citealt{KC16}: 67)

\dicnote{The stem-initial consonant in Maká is irregularly reflected as \intxt{q} rather than the expected \textit{*k’}.}%1

\dicnote{The plural form in Nivaĉle is non-etymological.}%2

\dicnote{The Manjui form is attested in \citet[198]{AG83} as \textit{\mbox{-}éheʔ}. It must be a mistranscription for \textit{\mbox{-}ʔʲéheʔ}.}%3

\dicnote{The expected reflex in Vejoz would be \textit{*\mbox{-}tʃ’åhe} [tʃ’ɑhẽ]. It is possible that the representation \textit{hni} in \citet{VU74} results from a mistranscription of a phonetically nasalized vowel.}%4

\dicnote{’Weenhayek \textit{a} is not the regular reflex of \sound{PW}{*e}.}%5

\lit{\citealt{EN84}: 21 (\intxt{*c’ånhnɛ}); \citealt{LC-VG-07}: 16}

\PMlemma{{\wordnl{*k’å \recind *k’ǻ}{variable antshrike\species{Thamnophilus caerulescens}}}}

\word{Mk}{k’aʔ}{sibilant sirystes\species{Sirystes sibilator}} \citep[65]{JB81} {\sep} \wordng{Ni}{k’åʔ<å>\pl{k}} \citep[288]{LC20} {\sep} \wordng{PW}{*kʲ’å \recind *kʲ’ǻh} > \wordng{LB}{tʃ’o} (\citealt{CS-FL-PR-VN13})

\PMlemma{{\wordnl{*\mbox{-}k’ä́lɸah}{spouse} [1]}}

\wordng{Ni}{\mbox{-}tʃ’akɸa} (\citealt{LC20}: 191) {\sep} \wordng{PCh}{*\mbox{-}k’élhwah} > (?) \wordng{Ijw}{\mbox{-}kʲ’émhla\pl{jes}} [3 4]; \wordng{I’w}{\mbox{-}ʔílfʷaʔ\pl{jis}} [4]; \wordng{Mj}{\mbox{-}ʔílhwa} (\citealt{JC14b}: 100; \citealt{ND09}: 123; \citealt{AG83}: 130; \citealt{JC18}) {\sep} \wordng{PW}{*\mbox{-}kʲ’éxʷah} > \wordng{LB}{\mbox{-}tʃ’ehʷa\pl{j}}; \wordng{Vej}{\mbox{-}tʃ’ehʷa\pl{s}}; \wordng{’Wk}{\mbox{-}kʲ’éxʷah} (\citealt{VN14}: 163; \citealt{VU74}: 54; \citealt{MG-MELO15}: 29; \citealt{KC16}: 67)

\dicnote{This noun obviously contains the suffix \wordnl{*\mbox{-}ɸah\plf{*\mbox{-}ɸa\mbox{-}ts}}{companion}.}%1

\dicnote{The Nivaĉle reflex has an unexpected allomorph \intxt{\mbox{-}ktʃ’akɸa} when it combines with the indefinite possessor prefix \intxt{βat\mbox{-}} \citep[97]{LC20}. It is thus possible that the correct PM reconstruction is actually \intxt{*\mbox{-}lk’ä́lɸah}. However, \intxt{k} is not found in other possessed forms.}%3

\dicnote{If the Iyojwa’aja’ word belongs here, it must be considered quite irregular: one would expect \intxt{*\mbox{-}kʲ’ílhwa} and not \intxt{\mbox{-}kʲ’émhla}.}%3

\dicnote{In \cits{JC14b} and \cits{AG83} attestations of the reflexes in Iyojwa’aja’ and Iyo’awujwa’, there is an unexpected word-final glottal stop.}%4

\rej{\citet[37]{EN84} includes reflexes of \word{PCh}{*\mbox{-}nǻʔ}{father}, which are obviously unrelated.}

\lit{\citealt{EN84}: 37 (\intxt{*cɛlna})}

\PMlemma{{\textit{*[ji]k’ä́n}\gloss{to stretch out}}}

\wordng{Ni}{[ji]tʃ’an} \citep[109]{JS16} {\sep} \wordng{PCh}{*[ʔi]k’én\mbox{-}\APPL} > \wordng{Ijw}{[ʔi]ts’ín\mbox{-}{\APPL} / \mbox{-}k’ín\mbox{-}\APPL}; \wordng{Mj}{[ʔi]tʃ’íhn\mbox{-}aˀm / \mbox{-}ʔíhn\mbox{-}aˀm} (\citealt{ND09}: 115; \citealt{JC18}) {\sep} \wordng{PW}{*[hi]kʲ’én} > \wordng{Vej}{[hi]tʃ’en} [2]; \wordng{’Wk}{[hi]kʲ’én̥} (\citealt{MG-MELO15}: 32; \citealt{KC16}: 205)

\dicnote{\citet[103]{VU74} documents this root as \intxt{\mbox{-}tʃen<pa>}, which must be a mistranscription.}%1

\PMlemma{{\wordnl{*[ji]k’ä́saˀχ \recind *[ji]k’ä́seˀχ}{to divide}}}

\wordng{Mk}{[j]<a>k’esaˀχ} [1] (\citealt{AG99}: 115, 117) {\sep} \wordng{PCh}{*[ʔi]k’ésah} > \wordng{Ijw}{[ʔi]ts’íxsa / \mbox{-}k’íxsa}; \wordng{I’w}{[i]tsíxsa\mbox{-}ji / \mbox{-}ísa\mbox{-}ji} [2]; \wordng{Mj}{[ʔi]tʃ’íxsah\mbox{-}{\APPL} / \mbox{-}ʔíxsah\mbox{-}\APPL} (\citealt{ND09}: 115; \citealt{AG83}: 45; \citealt{JC18}) {\sep} \wordng{PW}{*[hi]kʲésaχ} > \wordng{LB}{[hi]tʃ’esaχ}; \wordng{Vej}{[hi]tʃ’esah}; \wordng{’Wk}{[hi]kʲésax} (\citealt{VN14}: 242; \citealt{MG-MELO15}: 32; \citealt{KC16}: 206)

\dicnote{The preglottalized coda in Maká is attested in the New Testament (e.g. 2 Corinthians 9:9).}%1

\dicnote{The Iyo’awujwa’ reflex is likely a mistranscription for \textit{[ʔi]ts’íxsa\mbox{-}ji / \mbox{-}ʔíxsa\mbox{-}ji}.}%2

\gc{\citet[305]{PVB13a} compares the verb to \wordng{Proto-Guaicuruan}{*\mbox{-}keʃ’óqo} (with reflexes in Mbayá\gloss{to peck}, Abipon\gloss{to cut wood}, Pilagá\gloss{to split wood, to axe}, Toba\mbox{-}Qom\gloss{to axe}), an etymology not mentioned in an updated work by the same author \citep{PVB13b}.}

\lit{\citealt{PVB13a}: 305 (\intxt{*\mbox{-}k’esah})\gloss{to split}}

\PMlemma{{\wordnl{*k’ék’eh}{monk parakeet}}}

\wordng{Ni}{tʃ’etʃ’e\pl{k}} [1 2] \citep[110]{JS16} {\sep} \wordng{PCh}{*kék’eh} > \wordng{Ijw}{kík’i\pl{wa}}; \wordng{I’w}{k’ík’ih\pl{jis}} [1]; \wordng{Mj}{kíʔih\pl{waʔ}} (\citealt{ND09}: 136; \citealt{AG83}: 139; \citealt{JC18}) {\sep} \wordng{PW}{*kʲékʲ’e} > \wordng{LB/Vej}{tʃetʃ’e}; \wordng{’Wk}{kʲékʲ’eʔ\pl{lis}} (\citealt{VN14}: 157; \citealt{MG-MELO15}: 20; \citealt{KC16}: 186)

\dicnote{In the Yita’ Lhavos dialect of Nivaĉle, this word is attested with a high vowel: \textit{tʃ’itʃ’i} \citep[38]{AnG15}.}%1

\dicnote{The glottalized stem-initial consonant in Iyo’awujwa’, as attested in \citet{AG83}, could be a mistranscription.}%2

\rej{\citet[64]{AnG15} compares the Nivaĉle word to \wordng{Maká}{k’ek’e\pl{l}}\gloss{white-winged parakeet} \citep[235]{AG99}, whose vowel cannot correspond to \wordng{Ni}{e} except before uvulars. Instead, we propose that the Maká term is an early borrowing from Nivaĉle.}

\PMlemma{{\intxt{*\mbox{-}kʼínix\plf{*\mbox{-}k’ínxi\mbox{-}ts}}[1]\gloss{younger brother}}}

\wordng{Mk}{\mbox{-}kʼinix\plf{\mbox{-}kʼinx\mbox{-}ats}} \citep[236]{AG99} {\sep} \wordng{Ni}{\mbox{-}tʃʼiniʃ / \mbox{-}tʃʼinʃi\mbox{-}k͡laj} (\citealt{JS16}: 110, 336) {\sep} \wordng{PCh}{*\mbox{-}kʼínih\plf{*\mbox{-}kʼíhni\mbox{-}s}} > \wordng{Ijw}{\mbox{-}kʼíni \recind \mbox{-}ˀjíni\plf{\mbox{-}ʔʲíhni\mbox{-}s}}; \wordng{I’w}{\mbox{-}jíni}; \wordng{Mj}{\mbox{-}ʔíni\plf{\mbox{-}ʔínʲa\mbox{-}wot}} (\citealt{ND09}: 123, 128; \citealt{AG83}: 134; \citealt{JC18}) {\sep} \wordng{PW}{*\mbox{-}kʲʼíniχ}, \textit{*\mbox{-}kʲʼínhi\mbox{-}s} > \wordng{LB}{\mbox{-}tʃiniχ} [2]; \wordng{Vej}{\mbox{-}tʃ’inih}, \textit{\mbox{-}tʃ’in̥i\mbox{-}s} [2]; \wordng{’Wk}{\mbox{-}kʲʼínix}, \textit{\mbox{-}kʲʼín̥i\mbox{-}s} (\citealt{VN14}: 194; \citealt{MG-MELO15}: 29; \citealt{KC16}: 68)

\dicnote{The plural form is reconstructed based on the evidence of Iyojwa’aja’ and Wichí. It is thus technically reconstructible only for Proto-Chorote–Wichí.}%1

\dicnote{The Lower Bermejeño Wichí form, as attested by \citet{VN14}, is irregular in having a plain initial consonant rather than the expected \intxt{*tʃʼ}. \citet{VU74} also documents plain \intxt{tʃ} in Vejoz, but this must be a mistranscription.}%2

\lit{\citealt{EN84}: 20, 50 (\intxt{*c’ihni}, \textit{*c’ɛjhni}); \citealt{LC-VG-07}: 16; \citealt{AnG15}: 255–256}

\PMlemma{{\textit{*\mbox{-}kʼínχåʔ \recvar *\mbox{-}kʼínxåʔ} [1] (\intxt{*\mbox{-}wot})\gloss{younger sister}}}

\wordng{Mk}{\mbox{-}kʼinχaʔ \recvar \mbox{-}kʼinxaʔ} [1] (\intxt{\mbox{-}j}) [2] \citep[236]{AG99} {\sep} \wordng{Ni}{\mbox{-}tʃʼinxå\pl{βot}} \citep[337]{JS16} {\sep} \wordng{PCh}{*\mbox{-}kʼíhnåʔ\pla{wot}} > \wordng{Ijw}{\mbox{-}kʼíhnʲa \recind \mbox{-}ˀjíhnʲa}, \textit{\mbox{-}ˀjíhn\mbox{-}is} [3]; \wordng{I’w}{\mbox{-}kíhnʲeʔ}, \textit{\mbox{-}kíhnʲa\mbox{-}wot} [4]; \wordng{Mj}{\mbox{-}ʔíhnʲeʔ\pl{wat}} (\citealt{ND09}: 123, 128; \citealt{AG83}: 141; \citealt{JC18}) {\sep} \wordng{PW}{*\mbox{-}kʲʼínhå\pla{lis}} [2] > \wordng{LB}{\mbox{-}tʃin̥o} [4]; \wordng{Vej}{\mbox{-}tʃ’in̥å\pl{lis}} [4]; \wordng{’Wk}{\mbox{-}kʲʼín̥åʔ\pl{lis}} (\citealt{VN14}: 194; \citealt{MG-MELO15}: 29; \citealt{KC16}: 68)

\dicnote{The Maká reflex is attested with \intxt{χ} in \citet{AG99} and with \intxt{x} in the New Testament (e.g. in Mark 3:35; Matthew 12:50; John 11:5).}%1

\dicnote{The plural forms in Maká and Wichí are innovations.}%2

\dicnote{The absence of a stem-final \intxt{\mbox{-}ʔ} in the singular form in Iyojwa’aja’ is unexpected.}%3

\dicnote{The Iyo’awujwa’ and Lower Bermejeño Wichí forms, as attested by \citet{AG83} and \citet{VN14}, are irregular in having a plain initial consonant rather than the expected \wordng{I’w}{*kʼ}, \wordng{LB}{*tʃʼ}. \citet[53]{VU74} also documents plain \intxt{tʃ} in Vejoz, but this could be a mistranscription.}%4

\lit{\citealt{LC-VG-07}: 16; \citealt{AnG15}: 64}

\PMlemma{{\wordnl{*\mbox{-}k’o\plf{*\mbox{-}k’ó\mbox{-}l}}{bottom}}}

\word{Ni}{\mbox{-}k’oʔ\pl{k}}{anus} \citep[86]{JS16} {\sep} \word{PCh}{*\mbox{-}k’óʔ}{bottom}, \wordnl{*\mbox{-}k’ó\mbox{-}keʔ}{waist} > \word{Ijw}{\mbox{-}kʲ’óʔ}{mount}, \wordnl{\mbox{-}kʲ’ó\mbox{-}ji}{bottom}, \textit{\mbox{-}kʲ’ó\mbox{-}kiʔ} [1]\gloss{waist}; \wordng{I’w}{\mbox{-}kʲ’ó\mbox{-}kiʔ}; \word{Mj}{\mbox{-}ʔʲó\mbox{-}kiʔ}{waist} (\citealt{JC14b}: 77; \citealt{ND09}: 123; \citealt{AG83}: 143, 190; \citealt{JC18}) {\sep} \wordng{PW}{*\mbox{-}kʲ’o}, \textit{*\mbox{-}kʲ’ó\mbox{-}lʰ} > \wordng{LB}{\mbox{-}tʃ’u\pl{ɬ}}; \wordng{Vej}{\mbox{-}tʃ’o}; \wordng{’Wk}{\mbox{-}kʲ’oʔ\plf{\mbox{-}kʲ’ó\mbox{-}ɬ}} (in compounds such as \wordnl{[ta]ké\mbox{-}kʲ’oʔ}{palm of hand}, \wordnl{\mbox{-}wílis\mbox{-}kʲ’oʔ}{armpit}) (\citealt{VN14}: 201; \citealt{VU74}: 54; \citealt{MG-MELO15}: 61, 66; \citealt{KC16}: 62, 102)

\dicnote{\citet[123]{ND09} mistranscribes the Iyojwa’aja’ form for\gloss{waist} as \textit{\mbox{-}kʲ’ó\mbox{-}ki}.}%1

\rej{\citet[44]{EN84} compares the Wichí word (glossed as\gloss{bark}) with the Nivaĉle and Chorote terms for\gloss{horn} (< \wordng{PM}{*\mbox{-}kʼu}, \textit{*\mbox{-}kʼú\mbox{-}l}\gloss{horn; club}).}

\PMlemma{{\wordnl{*\mbox{-}kʼu\plf{*\mbox{-}kʼú\mbox{-}l}}{horn; club}}}

\wordng{Mk}{\mbox{-}kʼuʔ} [1] (\intxt{\mbox{-}l})\gloss{club} \citep[237]{AG99} {\sep} \wordng{Ni}{\mbox{-}kʼuʔ\pl{k}}\gloss{weapon; digging stick} (\citealt{JS16}: 90; \citealt{AF16}: 83) {\sep} \wordng{PCh}{*\mbox{-}k’úʔ} (\intxt{*\mbox{-}l \recind *\mbox{-}l<is>}) > \word{Ijw}{\mbox{-}kʲ’úʔ\pl{l<is>}}{horn}, \wordnl{\mbox{-}kʲ’úʔ\pl{ˀl}}{stick, hammer for killing fish}; \word{I’w}{\mbox{-}kʲúʔ\pl{lis}}{horn, club}; \word{Mj}{\mbox{-}ʔʲúʔ\pl{l}}{horn} (\citealt{ND09}: 123; \citealt{AG83}: 143; \citealt{JC18}) {\sep} \word{PW}{*\mbox{-}kʲ’u\plf{*\mbox{-}kʲ’ú\mbox{-}lʰ}}{horn} > \wordng{LB}{\mbox{-}tʃ’e}; \wordng{’Wk}{\mbox{-}kʲ’uʔ\plf{\mbox{-}kʲʼú\mbox{-}ɬ}} (\citealt{VN14}: 48; \citealt{KC16}: 68)

\dicnote{The root-final \intxt{ʔ} in the Maká singular form is attested in the New Testament (Revelations 12:5; Revelations 19:15). \citet{AG99} attests \textit{\mbox{-}k’u}.}%1

\rej{\citet[44]{EN84} compares the Nivaĉle and Chorote terms for\gloss{horn} with reflexes of \wordng{PW}{*\mbox{-}kʲ’o\pla{lʰ}}\gloss{bottom} (glossed as\gloss{bark} in \citealt{EN84}).}

\lit{\citealt{EN84}: 16, 44 (\intxt{*co}\gloss{club}, \textit{*c’o}\gloss{horn}); \citealt{LC-VG-07}: 15 (\enquote{club}), 17 (\enquote{horn})}

\PMlemma{{\textit{*kʼuj \recind *k’új}\gloss{cold}}}

\wordng{Mk}{kʼwi / kʼuj\mbox{-}} \citep[238]{AG99} {\sep} \wordng{Ni}{k’uj\pl{jis}} \citep[91]{JS16} {\sep} \wordng{PCh}{*k’újʔ} > \wordng{I’w}{júj\mbox{-}\APPL}; \wordng{Mj}{ʔʲújʔ} (\citealt{AG83}: 135; \citealt{JC18})

\lit{\citealt{AF16}: 306}

\PMlemma{{\wordnl{*k’ú(t)sta(ˀ)χ\plf{*k’ú(t)sta\mbox{-}ts}}{barn owl\species{Tyto alba}}}}

(?) \wordng{Ni}{k’ustax}, \textit{k’usta\mbox{-}s}\gloss{chalk\mbox{-}browed mockingbird (\intxt{Mimus saturninus})} \citep[91]{JS16} [1] {\sep} \wordng{PCh}{*k’ústah}, \textit{*k’ústa\mbox{-}s} > \wordng{Ijw}{kʲ’ústa}; \wordng{I’w}{kʲú(h)stah\pl{as}} [2]; \wordng{Mj}{ʔʲústa \recind ʔʲúʃta\pl{s}} (\citealt{ND09}: 138; \citealt{AG83}: 205; \citealt{JC18}) {\sep} \wordng{PW}{*kʲ’ústaχ} > \wordng{LB}{tʃ’estaχ}; \wordng{’Wk}{kʲ’ústax} (\citealt{VN14}: 198; \citealt{KC16}: 209)

\dicnote{Phonologically, the Nivaĉle ornithonym is a perfect match with Chorote and Wichí, but the species denoted by it has nothing in common with the barn owl (\intxt{Tyto alba}). It is possible that the Nivaĉle term arose as a contamination of two similar-sounding PM roots, \wordnl{*k’ú(t)staχ}{barn owl} and \wordnl{*k’ǻ(t)staχ}{chalk-browed mockingbird} (whence \word{PCh}{*k’ǻstah\plf{*k’ǻsta\mbox{-}s}}{chalk-browed mockingbird} > \wordng{Ijw}{kʲ’ásta\pl{s}}, \wordng{Mj}{ʔésta\pl{s}}; see \citealt{ND09}: 138; \citealt{JC18}).}%1

\dicnote{The plain \intxt{kʲ} in \cits{AG83} data of Iyo’awujwa’ must be a mistranscription, and the plural form in that variety is non-etymological.}%2

\PMlemma{{\wordnl{*kʼutX₂₃áˀn\plf{*kʼutX₂₃án\mbox{-}its}}{thorn}}}

\wordng{Ni}{k’utxaˀn\plf{k’utxan\mbox{-}is}} \citep[91]{JS16} {\sep} \wordng{PCh}{*kʼutáˀn\plf{*kʼután\mbox{-}is}} > \wordng{Ijw}{kʼitʲéˀn}; \wordng{I’w}{ʔitán\plf{ʔitán\mbox{-}is}}; \wordng{Mj}{ʔitáˀn\plf{ʔiten\mbox{-}éis}} [1] (\citealt{ND09}: 137; \citealt{AG83}: 132) {\sep} \wordng{PW}{*kʲ’utháˀn\plf{*kʲ’uthán\mbox{-}is}} > \wordng{LB}{tʃ’itʰan} [2]; \wordng{Vej}{tʃ’utʰan}; \word{’Wk}{kʲ’utʰáˀn\plf{kʲ’utʰán\mbox{-}is}}{thistle sp.} (\citealt{CS08}: 60; \citealt{VN14}: 362; \citealt{MG-MELO15}: 17; \citealt{KC16}: 209)

\dicnote{The stress in the Manjui plural form is non-etymological.}%1

\dicnote{The expected form would be \textit{*tʃ’etʰan}.}%2

\lit{\citealt{LC-VG-07}: 17, 20}

\PMlemma{{\wordnl{*k’utsaˀχ\plf{*k’utshá\mbox{-}s} / *\mbox{-}k’útsaˀχ\plf{*\mbox{-}k’útsha\mbox{-}ts}}{old} [1]}}

\wordng{Mk}{k’utsaˀχ} [2], \intxt{k’utshe\mbox{-}ts} \citep[237]{AG99} {\sep} \wordng{Ni}{k’utsaˀx\plf{k’utsxa\mbox{-}s}} \citep[92]{JS16} {\sep} \wordng{PCh}{*\mbox{-}kʲ’úsah\plf{*\mbox{-}kʲ’úsa\mbox{-}s}} > \word{Ijw}{\mbox{-}kʲ’úxs\mbox{-}eʔ}{friend, boss}; \wordng{I’w}{\mbox{-}júxsa}; \wordng{Mj}{\mbox{-}ʔʲúxsa\plf{\mbox{-}ʔʲuxse\mbox{-}s}} (\citealt{ND09}: 123; \citealt{AG83}: 135; \citealt{JC18}) {\sep} \wordng{PW}{*\mbox{-}kʲ’útsaχ} > \word{’Wk}{ʔatsín̥a\mbox{-}kʲ’utsax}{old woman} \citep[18]{KC16}

\dicnote{In Maká, Nivaĉle, and ’Weenhayek, the reflex of this etymon refers to old humans; in Iyo’awujwa’ and Manjui, to old objects.}%1

\dicnote{The presence of a preglottalized coda in the singular form in Maká is inferred based on the Nivaĉle cognate; this form is otherwise not attested in our sources that distinguish between plain and preglottalized stops, such as \citet{PMA} and the New Testament.}%2

\rej{\citet[26]{EN84} and \citet[20]{LC-VG-07} includes also the reflexes of \word{PW}{*[hi]kʲ’út}{old} and \wordnl{*kʲutsáx}{cháguar\species{Bromelia hieronymi}} (only \citeauthor{EN84}), which cannot be related for phonological reasons.}

\lit{\citealt{EN84}: 27–28 (\intxt{*cutsha}); \citealt{LC-VG-07}: 20}

\PMlemma{{\textit{*lásåsi(ʔ) \recind *lǻsåsi(ʔ) \recvar *lasǻsi(ʔ) \recind *låsǻsi(ʔ)} [1]\gloss{slippery}}}

\wordng{Mk}{\mbox{-}<qa>lasasi<j>}\gloss{to slip} [2] \citep[124]{AG99} {\sep} \wordng{PCh}{*lásåsiʔ \recind *lǻsåsiʔ \recvar *lasǻsiʔ \recind *låsǻsiʔ} [1] > (?) \wordng{Ijw}{lálisiʔ} [3]; \wordng{I’w}{lasáxsiʔ}; \wordng{Mj}{láxsaʃiʔ} (\citealt{ND09}: 138; \citealt{AG83}: 146; \citealt{JC18})

\dicnote{The Iyo’awujwa’ reflex points to initial stress in PCh and PM. The Manjui reflex points to peninitial stress in PCh and PM.}%1

\dicnote{The Maká verb contains a fossilized alienizing prefix and verbalizing suffix (\enquote{to have slipperiness}).}%2

\dicnote{The Iyojwa’aja’ term is entirely irregular and might be noncognate. There is a similar term \wordnl{láxsasi}{blue}, \wordnl{láxsa apeʔe}{purple} \citep[138]{ND09}, cognate with \word{Iyo’awujwa’}{láxsa(sen) \recind laxsá}{blue}, which is hardly related for semantic reasons.}%3

\PMlemma{{\wordnl{*[ji]låˀj}{to withstand}}}

\wordng{Ni}{[ji]k͡låˀj} \citep[101]{JS16} {\sep} \wordng{PCh}{*[ʔi]lǻj\mbox{-}eh} > \wordng{Ijw}{[ʔi]lʲáj\mbox{-}i / \mbox{-}láj\mbox{-}i}; \wordng{Mj}{[ʔi]l(ʲ)éj\mbox{-}i / \mbox{-}láj\mbox{-}i} (\citealt{ND09}: 101; \citealt{JC18}) {\sep} \wordng{PW}{*[ʔi]låj} > \wordng{LB}{[ʔi]loj\mbox{-}eχ}; \wordng{Vej}{\mbox{-}laj} [1]; \wordng{’Wk}{[ʔi]låjʔ}\gloss{to be satisfied, to live} (\citealt{VN14}: 338; \citealt{VU74}: 63; \citealt{KC16}: 214)

\dicnote{The vowel \intxt{a} in the Vejoz reflex is likely a mistranscription on \cits{VU74} part.}%1

\PMlemma{{\wordnl{*[ji]lǻn}{to kill}}}

\wordng{Mk}{[ji]lan} \citep[239]{AG99} {\sep} \wordng{Ni}{[ji]k͡lån} \citep[246]{AF16} {\sep} \wordng{PCh}{*[ʔi]lǻn} > \wordng{Ijw}{[ʔi]lʲáˀn / \mbox{-}láˀn}; \wordng{I’w}{\mbox{-}lán}; \wordng{Mj}{[ʔi]lʲén / \mbox{-}lán} (\citealt{JC14b}: 77, 83; \citealt{ND09}: 101; \citealt{AG83}: 145; \citealt{JC18}) {\sep} \wordng{PW}{*[ʔi]lǻn} > \wordng{LB}{[ʔi]lon}; \wordng{Vej}{[i]lån}; \wordng{’Wk}{[ʔi]lǻn̥} (\citealt{VN14}: 241; \citealt{VU74}: 64; \citealt{MG-MELO15}: 34; \citealt{KC16}: 212)

\lit{\citealt{EN84}: 15 (3\textsc{pl}~\textit{*lån\mbox{-}hnέ}); \citealt{AnG15}: 253}

\PMlemma{{\wordnl{*lǻp’ih \recind *lǻɸ’ih}{snail}}}

\wordng{Ni}{k͡låp’i} (\citealt{LC20}: 27) {\sep} \wordng{PCh}{*lǻp’ih\pla{is}} > \wordng{Ijw}{láp’i\plf{láp’ih\mbox{-}is}}; \wordng{I’w}{láʔpih\plf{lápih\mbox{-}is}} [1]; \wordng{Mj}{láp’i}, \textit{láp’i\mbox{-}waʔ} [2] (\citealt{ND09}: 138; \citealt{AG83}: 146; \citealt{JC18})

\dicnote{The Iyo’awujwa’ form must be a mistranscription for \textit{lap’ih\pl{is}}.}%1

\dicnote{The Manjui plural does not match the form found in Iyojwa’aja’ and Iyo’awujwa’ and thus must be an innovation.}%2

\rej{\citet[48]{EN84} includes a reflex of \word{PW}{*móp’i}{white heron} into the comparison and reconstructs \wordnl{*p’i}{antenna, crest}. This is implausible for semantic, phonological, and morphological reasons.}

\lit{\citealt{EN84}: 48 (\wordnl{p’i}{antenna, crest})}

\PMlemma{{\textit{*[ji]låt \recind *[ji]lǻt \recvar *[ji]let \recind *[ji]lét} [1]\gloss{to flee}}}

\wordng{Mk}{<i>lat \recvar <i>lit} \citep[198]{AG99} {\sep} \wordng{Ni}{[ji]k͡låt} \citep[101]{JS16} {\sep} \wordng{PCh}{*<ˀ[j]í>lt<an> \recind [ʔi]<ˀjí>lt<an>} [2 3] > \wordng{Ijw}{ˀ[j]íltaˀn \recind [ʔi]ˀjíltaˀn}; \wordng{Mj}{[ʔi]ˀjíltan}\gloss{to separate from} (\citealt{ND09}: 118, 165; \citealt{JC18}) {\sep} \wordng{PW}{*[ʔi]lét<han>} [3] > \wordng{’Wk}{[ʔi]létʰan̥} \citep[225]{KC16}

\dicnote{Nivaĉle points to \wordng{PM}{*[ji]låt} or \intxt{*[ji]lǻt}, Wichí to \intxt{*[ji]let \recind *[ji]lét}, whereas Maká has reflexes of both variants.}%1

\dicnote{We have no explanation for the element \intxt{*ˀjí\mbox{-}} or \intxt{*ˀjí\mbox{-}} in the Chorote form. The loss of the root vowel is likewise irregular.}%2

\dicnote{The Chorote and Wichí reflexes contain a fossilized suffix (\wordng{PCh}{*\mbox{-}an}, \wordng{PW}{*\mbox{-}han}).}%3

\gc{Likely related to \word{Proto-Guaicuruan}{*\mbox{-}ʔi(ˀ)lote}{to flee} (\citealt{PVB13b}, \#688; cf. \citealt{PVB13a}: 306).}

\lit{\citealt{PVB13a}: 306 (\intxt{*\mbox{-}ilʌt})}

\PMlemma{{\wordnl{*\mbox{-}låʔ\plf{*\mbox{-}lǻ\mbox{-}jʰ}}{domestic animal}}}

\word{Ni}{\mbox{-}k͡låʔ\pl{j}}{domestic animal; one’s sport} \citep[337]{JS16} {\sep} \wordng{PCh}{*\mbox{-}lá<hwah>} [1] > \wordng{Ijw}{\mbox{-}láhwa\pl{s}}; \wordng{I’w}{\mbox{-}láfʷa\pl{j}}; \wordng{Mj}{\mbox{-}láhwa}, \textit{\mbox{-}láhwaa\mbox{-}j} (\citealt{ND09}: 123; \citealt{AG83}: 145; \citealt{JC18}) {\sep} \wordng{PW}{*\mbox{-}låʔ}, \textit{*\mbox{-}lǻ\mbox{-}jʰ} > \wordng{LB}{\mbox{-}loʔ\pl{j}}; \wordng{Vej}{\mbox{-}lå\mbox{-}j}; \wordng{’Wk}{\mbox{-}låʔ}, \textit{\mbox{-}lǻ\mbox{-}ç} (\citealt{VN14}: 195, 169; \citealt{VU74}: 64; \citealt{KC16}: 69)

\dicnote{In Chorote, the suffix \wordnl{*\mbox{-}hwah}{companion} has been fossilized to the root.}%1

\gc{Obviously related to \word{Proto-Southern Guaicuruan}{*\mbox{-}lo}{domestic animal} \citep[280, fn. 157]{PVB13b}.}

\lit{\citealt{EN84}: 35 (\intxt{*lå})}

\PMlemma{{\textit{*lä́tseni(ʔ)} (fruit); \textit{*lä́tsen\mbox{-}uˀk}, \textit{*lä́tsen\mbox{-}ku\mbox{-}jʰ} (tree)\gloss{chañar (\intxt{Geoffroea decorticans})}}}

\wordng{Mk}{<xu>letsin\mbox{-}uˀk}, \textit{<xu>letsin\mbox{-}kw\mbox{-}i} [1] \citep[393]{AG99} {\sep} \wordng{PCh}{*léseniʔ}; \textit{*léseni\mbox{-}k} > \wordng{Ijw}{lέsini}; \textit{lέsini\mbox{-}k} \citep[138]{ND09} {\sep} \wordng{PW}{*létseˀnih}; \textit{*létsen\mbox{-}ukʷ} > \wordng{LB}{lets’enekʷ} [2]; \wordng{Vej}{letseˀni}; \textit{letsen\mbox{-}uk}, \textit{letsen\mbox{-}ku\mbox{-}j} [3 4]; \wordng{’Wk}{létseˀnih}; \textit{létsen\mbox{-}uk}, \textit{létsen̥\mbox{-}u\mbox{-}ç} [4] (\citealt{VN14}: 193; \citealt{MG-MELO15}: 18; \citealt{KC16}: 225)

\dicnote{The origins of the element \intxt{xu\mbox{-}} in Maká are unclear. The preglottalized coda in the Maká suffix for tree names is attested elsewhere \citep[7]{PMA}.}%1

\dicnote{The glottalization in \wordng{LB}{ts’} is irregular.}%2

\dicnote{\citet[64]{VU74} documents \wordng{Vej}{letsenn\mbox{-}uk}, which must be a mistranscription.}%3

\dicnote{The plural forms \wordng{Vej}{letsen\mbox{-}ku\mbox{-}j} and \wordng{’Wk}{létsen̥\mbox{-}u\mbox{-}ç} do not correspond neither to each other nor to \wordng{Maká}{<xu>letsin\mbox{-}kw\mbox{-}i}. One would expect \wordng{Vej}{*letsen\mbox{-}tʃu\mbox{-}j}, \wordng{’Wk}{*létsen\mbox{-}kʲu\mbox{-}ç}.}%4

\lit{\citealt{EN84}: 36 (\intxt{*lɛtseni})}

\PMlemma{{\wordnl{*[ji]le\mbox{-}n \recind *[ʔi]lé\mbox{-}n}{to tattoo, to paint one’s face}; \wordnl{*\mbox{-}le\mbox{-}t \recind *\mbox{-}lé\mbox{-}t}{tattoo, face painting} [1]}}

\word{Mk}{[ji]lin\mbox{-}ix}{to oint, to paint} \citep[243]{AG99} {\sep} \wordng{PCh}{*[ʔi]ˀlé<n>} [2] > \word{Ijw}{[ʔi]ˀlíˀn / \mbox{-}ˀlέˀn}{to paint one’s face}; \word{I’w}{\mbox{-}lén}{to paint} [3]; \wordng{Mj}{[ʔi]ˀlín / \mbox{-}ˀlέn} (\citealt{ND09}: 117; \citealt{AG83}: 147; \citealt{JC18}) {\sep} \word{PW}{*\mbox{-}le<t> \recind *\mbox{-}lé<t>}{tattoo} [4] > \word{LB/Vej}{\mbox{-}let}{tattoo}; \word{’Wk}{\mbox{-}let \recind \mbox{-}lét}{face painting} [4] (\citealt{VN14}: 410; \citealt{VU74}: 64; \citealt{KC16}: 70)

\dicnote{\wordng{PM}{*\mbox{-}n} is a verbalizer and \intxt{*\mbox{-}t} is a nominalizer (\intxt{nomen instrumenti}). Neither suffix is synchronically productive in the contemporary languages. Maká and Chorote have preserved only the verb, and Wichí only the noun.}%1

\dicnote{The glottalization in \sound{PCh}{*ˀl} is irregular.}

\dicnote{The seemingly plain reflex of \sound{PCh}{*ˀl} in Iyo’awujwa’ could be a mistranscription on \cits{AG83} part.}%3

\dicnote{The ’Weenhayek reflex is only attested with the indefinite possessor prefix \intxt{ˀnó\mbox{-}}; for this reason, we do not know if it has an underlying short or long vowel.}%4

\PMlemma{{\wordnl{*\mbox{-}léts}{offspring (sons and/or daughters)} (\intxt{plurale tantum})}}

\wordng{Mk}{\mbox{-}lits} \citep[243]{AG99} {\sep} \word{Ni}{\mbox{-}k͡les}{offspring, sperm} \citep[337]{JS16} {\sep} \wordng{PCh}{*\mbox{-}lés} > \wordng{Ijw}{\mbox{-}lέs}; \wordng{I’w}{\mbox{-}lés}; \wordng{Mj}{\mbox{-}lέs} (\citealt{ND09}: 124; \citealt{AG83}: 122, 124; \citealt{JC18}) {\sep} \wordng{PW}{*\mbox{-}lés} > \wordng{LB/Vej}{\mbox{-}les}, \wordng{’Wk}{\mbox{-}lés} (\citealt{VN14}: 215; \citealt{VU74}: 64; \citealt{KC16}: 69)

\gc{\citet[312]{PVB13a} notes the similarity with \word{Proto-South Guaicuruan}{*\mbox{-}jalé}{daughter}, \wordnl{*\mbox{-}jalé\mbox{-}k}{son}, which could be spurious.}

\lit{\citealt{EN84}: 11 (\intxt{*lɛs}); \citealt{PVB13a}: 312 (\intxt{*\mbox{-}le\mbox{-}ts})}

\PMlemma{{\wordnl{*[ji]léˀx}{to wash}}}

\wordng{Mk}{[ji]liˀx\mbox{-}xuʔ} [1]\gloss{to clean} \citep[244]{AG99} {\sep} \wordng{Ni}{[ji]k͡leˀʃ} \citep[117]{JS16} {\sep} \wordng{PCh}{*[ʔi]léh} > \wordng{Ijw}{[ʔi]líh / \mbox{-}lέh}; \wordng{I’w}{[i]lí / \mbox{-}lé}; \wordng{Mj}{[ʔi]líh / \mbox{-}lέh} (\citealt{ND09}: 101; \citealt{AG83}: 41, 146; \citealt{JC18}) {\sep} \wordng{PW}{*[ʔi]léχ} > \wordng{LB}{[ʔi]leχ}; \wordng{Vej}{[hi]leh}; \wordng{’Wk}{[ʔi]léx} (\citealt{VN14}: 244; \citealt{JB09}: 44; \citealt{VU74}: 64; \citealt{MG-MELO15}: 36; \citealt{KC16}: 223)

\dicnote{The preglottalized coda and the presence of two \textit{x} is documented in the New Testament (\intxt{ne\mbox{-}n\mbox{-}liˀx\mbox{-}xuʔ} in Ephesians 5:26; Revelations 21:4).}%1

\lit{\citealt{AnG15}: 64, 253}

\PMlemma{{\wordnl{*lim \recind *lím}{white}}}

\wordng{Ni}{k͡lim} \citep[118]{JS16} {\sep} \wordng{PCh}{*lím\mbox{-}} > \wordng{Ijw}{lém<i>}, \textit{lém<ih>\mbox{-}ji}; \wordng{I’w}{lém<iʔ>} [1]; \wordng{Mj}{léim<iʔ>} (\citealt{ND09}: 138; \citealt{AG83}: 146; \citealt{JC18})

\dicnote{\citet[186]{AG83} also documents an irregular variant \intxt{hlém<iʔ>}, which must be a mistranscription. Elsewhere \citep[146]{AG83}, one finds multiple attestations with \textit{l\mbox{-}} (\intxt{lémiʔ}, \textit{lémi\mbox{-}tsiʔ}, \textit{lémi\mbox{-}jin}).}%1

\lit{\citealt{EN84}: 36 (\intxt{*lɛm}); \citealt{AnG15}: 253}

\PMlemma{{\textit{*(\mbox{-})lkä́(ˀ)ɬ}\gloss{nasal mucus, cold}}}

\wordng{Mk}{\mbox{-}leke(ˀ)ɬ\pl{its}} \citep[241]{AG99} {\sep} \wordng{PCh}{*kéɬ} > \wordng{Ijw}{kíɬ}; \wordng{Mj}{kíɬ\pl{es}} (\citealt{ND09}: 136; \citealt{JC18}) {\sep} \wordng{PW}{*kʲéɬ\mbox{-}taχ}, \textit{*kʲéɬ\mbox{-}ta\mbox{-}s} > \wordng{LB}{tʃeɬ\mbox{-}taχ}; \wordng{Vej}{tʃeɬ\mbox{-}tah}; \wordng{’Wk}{kʲéɬ\mbox{-}tax}, \textit{kʲéɬ\mbox{-}ta\mbox{-}s} (\citealt{JB09}: 39; \citealt{VU74}: 52; \citealt{MG-MELO15}: 47; \citealt{KC16}: 186)

\lit{\citealt{LC-VG-07}: 16}

\PMlemma{{\textit{*lkéte} (fruit); \textit{*lkéte\mbox{-}(ju)ˀk} (plant)\gloss{squash}}}

\wordng{Mk}{lekiti}; \textit{lekit\mbox{-}uˀk} [1], \textit{lekiti\mbox{-}kw\mbox{-}i} (\citealt{AG99}: 241; \citealt{JB81}: 85) {\sep} \wordng{PCh}{*kéteʔ}; \textit{*kéte\mbox{-}k} > \wordng{I’w}{kítiʔ}; \wordng{Mj}{kítʲeʔ \recind kítiʔ}; \textit{kítʲe\mbox{-}k} (\citealt{AG83}: 140; \citealt{JC18})

\dicnote{The preglottalized coda in the Maká suffix for tree names is attested elsewhere \citep[7]{PMA}.}%1

\lit{\citealt{LC-VG-07}: 16}

\PMlemma{{\wordnl{*(\mbox{-})lo(ʔ) \recind *(\mbox{-})ló(ʔ)}{ashes}}}

\wordng{Mk}{loʔ\pl{l}} \citep[235]{AG99} {\sep} \wordng{PCh}{*\mbox{-}lóʔ} > \word{Ijw}{\mbox{-}lɔ́ʔ}{burnt remains, ashes (of something)} \citep[124]{ND09}

\gc{Obviously related to \word{Proto-Guaicuruan}{*á(ˀ)lo}{ashes} (\citealt{PVB13b}, \#33; cf. \citealt{PVB13a}: 311).}

\lit{\citealt{PVB13a}: 311 (\intxt{*loʔ})}

\PMlemma{{\intxt{*loˀp \recind *lóˀp\plf{*lop\mbox{-}íts \recind *lóp\mbox{-}its}} [1]\gloss{winter}}}

\wordng{Mk}{loˀp} [2], \textit{lop\mbox{-}its} (\citealt{AG99}: 245; \citealt{maka-etnomat}: 23–25) {\sep} \wordng{Ni}{k͡loˀp}, \textit{k͡lop\mbox{-}is} \citep[119]{JS16} {\sep} \wordng{PCh}{*lóp} > \wordng{Ijw}{lɔ́p}\gloss{fall; hunger season} \citep[138]{ND09} {\sep} \wordng{PW}{*lop \recind *lóp} > \wordng{LB}{lup} \citep[49]{VN14}

\dicnote{The plural form is reconstructed based on \wordng{Maká}{lop\mbox{-}its} and \wordng{Nivaĉle}{k͡lop\mbox{-}is}; it is thus technically reconstructible only for Proto-Maká–Nivaĉle.}%1

\dicnote{The Maká reflex is mistranscribed as \intxt{lop} in the New Testament (John 10:22); the expected form \textit{loˀp} is otherwise documented (\citealt{maka-etnomat}: 23–25).}%2

\lit{\citealt{AnG15}: 253}

\PMlemma{{\wordnl{*lóta\mbox{-}(ju)ˀk}{\textit{iscayante} tree (for making bows)}}}

\wordng{Ni}{k͡lota<tʃ>} \citep[119]{JS16} {\sep} \word{PCh}{*lóta\mbox{-}juk}{\textit{Mimozyganthus carinatus}} > \wordng{Ijw}{lɔ́ta<k>\mbox{-}ik}; \wordng{I’w}{lóta<k>\mbox{-}ik \recind lóta\mbox{-}ʔik}; \wordng{Mj}{lɔ́ta\mbox{-}ʔik \recind lɔ́te\mbox{-}jik} (\citealt{ND09}: 138; \citealt{GS10}: 186; \citealt{JC18}) {\sep} \wordng{PW}{*lóte<q>}, \textit{*lót<h>\mbox{-}ajʰ} > \wordng{LB}{luteq}, \textit{lutʰ\mbox{-}aj}\gloss{arrow}; \wordng{Vej}{lotek}; \wordng{’Wk}{lótek}, \textit{lótʰ\mbox{-}aç \recind lótʰ\mbox{-}eç}\gloss{\textit{Prosopis abbreviata}; bow, arrow} (\citealt{VN14}: 192; \citealt{MG-MELO15}: 18, 57; \citealt{KC16}: 226)

\PMlemma{{\wordnl{*[ʔa]lóχ\plf{*[ʔa]ló\mbox{-}ts}}{many}}}

\wordng{Mk}{<o>lo<ts>} [1 2] \citep[281]{AG99} {\sep} \wordng{Ni}{<ʔa>k͡lox} \citep[38]{JS16} {\sep} \wordng{PCh}{*[ʔa]ˀlóh} [3] > \wordng{Ijw}{ˀlɔ́h}; \wordng{I’w}{[a]lóh}; \wordng{Mj}{[ʔa]ˀlɔ́h} (\citealt{JC14b}: 78; \citealt{ND09}: 162; \citealt{AG83}: 120; \citealt{JC18}) {\sep} \wordng{PW}{*[ʔa]ló<s>} [2 4] > \wordng{Vej}{los}; \wordng{’Wk}{<ʔa>lós} (\citealt{VU74}: 64; \citealt{KC16}: 11)

\dicnote{The third-person prefix \intxt{*ʔa\mbox{-}} has been fossilized in all languages except Chorote.}%1

\dicnote{The plural suffix \intxt{*\mbox{-}ts} has been fossilized in Maká and Wichí.}%2

\dicnote{The glottalization in \sound{PCh}{*ˀl} appears to be irregular (the seemingly plain reflex in Iyo’awujwa’ could be a mistranscription on \cits{AG83} part).}%3

\dicnote{In Southwestern Wichí, one finds \wordnl{lus}{two} (\citealt{JT09-th}: 93; \citealt{VN14}: 359; \citealt{JB09}: 50), which could be a phonologically regular reflex of \word{PW}{*\mbox{-}lós}{many}, but it is more probable that this number term is a recent loan from Spanish \textit{dos}.}%4

\lit{\citealt{RJH15}: 242; \citealt{AnG15}: 253}

\PMlemma{{\wordnl{*(\mbox{-})lútseˀx\plf{*(\mbox{-})lútsxe\mbox{-}s}}{bow}}}

\wordng{Ni}{k͡lutseʃ / \mbox{-}k͡lutseˀʃ} [1], \textit{(\mbox{-})k͡lutsʃe\mbox{-}s}\gloss{bow, gun} \citep[345]{JS16} {\sep} \wordng{PCh}{*(\mbox{-})lúseh\pl{es}} > \wordng{Ijw}{(\mbox{-})lóxse\pl{hes}} [1]; \wordng{I’w}{lóxseʔ} [2]; \wordng{Mj}{\mbox{-}lʊ́xse}, \textit{\mbox{-}lʊ́xʃi\mbox{-}is} (\citealt{ND09}: 124, 138; \citealt{AG83}: 147; \citealt{JC18}) {\sep} \wordng{PW}{*(\mbox{-})lútseχ\plf{*(\mbox{-})lúts\mbox{-}es}} > \wordng{LB}{\mbox{-}letseχ\plf{lets\mbox{-}es}}; \wordng{Vej}{\mbox{-}lutseh}; \wordng{’Wk}{(\mbox{-})lútsex\plf{(\mbox{-})lúts\mbox{-}es}} (\citealt{JB09}: 49; \citealt{VU74}: 64; \citealt{KC16}: 70, 228)

\dicnote{The allomorph \textit{\mbox{-}k͡lutseˀʃ} is attested in \citet[345]{JS16} in the form \intxt{βat\mbox{-}k͡lutseˀʃ}, yet the form \intxt{kas\mbox{-}k͡lutseʃ} is unexpectedly attested with a plain coda.}%1

\dicnote{The expected reflex in Iyojwa’aja’ would be \textit{*\mbox{-}lóxsi}. The failure of \intxt{*e} to raise is unclear.}%2

\dicnote{The expected reflex in Iyo’awujwa’ would be \textit{*\mbox{-}lʊ́xse}. \citet{AG83} systematically transcribes [ʊ] as \textit{o} in her data, but the word-final glottal stop must be a mistranscription.}%3

\lit{\citealt{RJH15}: 242; \citealt{EN84}: 11 (\intxt{*lutshɛ}); \citealt{LC-VG-07}: 19; \citealt{PVB02}: 143 (\intxt{*\mbox{-}lutsex})}

\PMlemma{{\wordnl{*[ji]lXón}{to roast}}}

\word{Ni}{[ji]kxon}{to cook in ashes} \citep[112]{JS16} {\sep} \wordng{PCh}{*[ʔi]hlón} > \wordng{Ijw}{[ʔi]hlʲóˀn / \mbox{-}hlɔ́ˀn}; \wordng{Mj}{[ʔi]hl(ʲ)ón / \mbox{-}hlɔ́n} (\citealt{ND09}: 99; \citealt{JC18}) {\sep} \wordng{PW}{*[t]nhón} > \wordng{LB}{[t]<i>n̥un}; \wordng{’Wk}{[t(a)]n̥ón̥} (\citealt{JB09}: 57; \citealt{KC16}: 369)

\PMlemma{{\wordnl{*\mbox{-}ˀlåʔ \recind *\mbox{-}ˀlǻʔ}{adornment} [1]}}

\word{Mk}{\mbox{-}<ʔeti>ˀlaʔ\pl{j}}{necklace} [2], \wordnl{\mbox{-}<qetsxiki>ˀlaʔ\pl{j}}{necklace} [2] \citep[160, 309]{AG99} {\sep} \word{Ni}{\mbox{-}<fo>ˀk͡lå\pl{s}}{ankle bracelet with white feathers} {\sep} \word{PCh}{*<kinú>ˀlaʔ \recind *<kenú>ˀlaʔ}{necklace} [3] > \wordng{Ijw}{kinʲúˀlʲeʔ}; \wordng{Mj}{kinʲúˀlaʔ} (\citealt{ND09}: 136; \citealt{JC18})

\dicnote{The possible derivative \wordnl{*\mbox{-}pǻˀlåʔ}{bracelet} is discussed in a separate entry.}%1

\dicnote{The presence of a preglottalized sonorant in Maká is inferred based on the Nivaĉle and Chorote cognates; the form is not attested in our sources that distinguish between plain and preglottalized codas, whereas \citet{AG99} gives simply \intxt{\mbox{-}ʔetilaʔ}, \intxt{\mbox{-}qetsxikilaʔ} (she does not otherwise distinguish between \intxt{l} and \intxt{ˀl}).}%2

\dicnote{Chorote unexpectedly shows \sound{PCh}{*a} instead of \intxt{*å} as the reflex of \sound{PM}{*å}, as shown by the vowel raising in Iyojwa’aja’.}%3

\PMlemma{{\intxt{*ˀlä́jX₂₃VnåX₁₃å} [1]\gloss{Azara’s night monkey}}}

\wordng{Ni}{k͡lajxenåxå\pl{k}} \citep[115]{JS16} {\sep} \wordng{PCh}{*ˀléhjanåhå\mbox{-}keʔ} > \wordng{Ijw}{<ʔa>ˀléhjena\mbox{-}kiʔ} [2]\gloss{Azara’s capuchin}; \wordng{I’w}{léhna\mbox{-}kiʔ\pl{ji}}; \wordng{Mj}{ˀlέhnaa\mbox{-}kiʔ\pl{j}} (\citealt{ND09}: 95; \citealt{AG83}: 147; \citealt{JC18})

\dicnote{Regarding the reconstruction of the vowel of the second syllable, the Nivaĉle reflex points to \intxt{*e}, whereas the Iyojwa’aja’ form points to \intxt{*a} or \intxt{*ä}.}%1

\dicnote{We have no explanation for the element \intxt{ʔa\mbox{-}} in Iyojwa’aja’.}%2

\lit{\citealt{EN84}: 15, 52 (\intxt{*laj\mbox{-}hnaq}, \textsc{pl}~\textit{*lajhnaqs})}

\PMlemma{{\wordnl{*\mbox{-}ˀliˀx\plf{*\mbox{-}ˀlix\mbox{-}ájʰ}}{language, word}}}

\wordng{Mk}{\mbox{-}ˀlix<eʔ>\pl{j}} [1] \citep[243]{AG99} {\sep} \word{Ni}{\mbox{-}ˀk͡liˀʃ\plf{\mbox{-}ˀk͡liʃ\mbox{-}aj}}{word} \citep[376]{JS16} {\sep} \wordng{PCh}{*\mbox{-}ˀlíh\plf{*\mbox{-}ˀlah\mbox{-}ájʰ}} > \wordng{Ijw}{\mbox{-}ˀléh}; \wordng{I’w}{\mbox{-}léh\pl{aj}} [2]; \word{Mj}{\mbox{-}ˀléih\plf{\mbox{-}ˀlah\mbox{-}ájh}}{language} (\citealt{ND09}: 127; \citealt{AG83}: 147; \citealt{JC18})

\dicnote{The glottalization in the stem-initial sonorant in Maká is attested in the New Testament (e.g. Matthew 2:23; Mark 4:14).}%1

\dicnote{The plain reflex of \wordng{PCh}{*ˀl} in Iyo’awujwa’ as attested by \citet{AG83} must be a mistranscription, and the plural form in that variety is leveled based on the singular form.}%2

\PMlemma{{\wordnl{*(\mbox{-})ɬaʔ\plf{*(\mbox{-})ɬá\mbox{-}ts}}{louse}}}

\wordng{Mk}{\mbox{-}<ij>ɬeʔ\pl{ts}} [1] \citep[193]{AG99} {\sep} \wordng{Ni}{\mbox{-}ɬaʔ\pl{s}} [2] \citep[161]{JS16} {\sep} \wordng{PCh}{*\mbox{-}hláʔ\pla{s}} > \wordng{Ijw}{\mbox{-}hláʔ\pl{s}}; \wordng{I’w}{<ʔi>hlʲéʔ\pl{s}}; \wordng{Mj}{\mbox{-}hláʔ\pl{s}} (\citealt{ND09}: 119; \citealt{AG83}: 132; \citealt{JC18}) {\sep} \wordng{PW}{*ɬaʔ} > \wordng{LB}{ɬaʔ}; \wordng{Vej}{ɬa}; \wordng{’Wk}{ɬaʔ} (\citealt{VN14}: 50; \citealt{VU74}: 64; \citealt{KC16}: 230)

\dicnote{We have no explanation for the element \intxt{\mbox{-}ij\mbox{-}} in Maká.}%1

\dicnote{\citet[84]{LC20} document the Nivaĉle reflex as \intxt{\mbox{-}ʔɬa}, a form that we cannot explain at this time.}%2

\lit{\citealt{EN84}: 28 (\intxt{*hla})}

\PMlemma{{\wordnl{*[ji]ɬǻˀm}{to defecate}}}

\wordng{Mk}{<i>ɬaˀm} \citep[199]{AG99} {\sep} \wordng{Ni}{[ji]ɬåˀm} \citep[170]{JS16} {\sep} \wordng{PCh}{*[ʔi]hlǻˀm} > \wordng{Ijw}{[ʔi]hlʲáˀm / \mbox{-}hláˀm}; \wordng{Mj}{[ʔi]hl(ʲ)éˀm / \mbox{-}hléˀm} (\citealt{ND09}: 99; \citealt{JC18}) {\sep} \wordng{PW}{*[t]<’a>ɬáˀm} [1] > \wordng{LB}{[t]<’a>ɬam}; \wordng{’Wk}{[t]<’a>ɬáˀm} (\citealt{JB09}: 59; \citealt{KC16}: 431)

\dicnote{The preglottalized coda in the Maká reflex is attested in the New Testament (\intxt{iɬaˀm\mbox{-}kij}\gloss{to have diarrhea} in Acts 28:8).}%1

\dicnote{The Wichí reflex is irregular: one would expect \wordng{PW}{**[t]ɬǻˀm} > \wordng{LB}{*[t]aɬom}; \wordng{’Wk}{*[t]aɬǻˀm}.}%2

\PMlemma{{\wordnl{*[ji]ɬǻn}{to light fire}}}

\word{Mk}{[ni]ɬan\mbox{-}iʔ}{to light fire}, \wordnl{[ni]ɬan\mbox{-}xiʔ}{to smoke in} \citep[248]{AG99} {\sep} \wordng{Ni}{[ji]ɬån} \citep[170]{JS16} {\sep} \word{PCh}{*[ʔi]hlǻn\mbox{-}eʔeʔ}{to fan the flame}, \wordnl{*[ti]hlǻhn\mbox{-}an}{to smoke (intr.)}, \wordnl{*[ʔi]hlǻhn\mbox{-}ijʔ}{to smoke (tr.)} > \wordng{Ijw}{[ʔi]hlʲán\mbox{-}eʔeʔ / \mbox{-}hlán\mbox{-}eʔeʔ} [1], \textit{[ti]hlʲáhn\mbox{-}aˀn}, \textit{[ʔi]hlʲáhn\mbox{-}ijʔ / \mbox{-}hláhn\mbox{-}ijʔ} [1]; \wordng{I’w}{\mbox{-}hlán\mbox{-}ee}, \textit{\mbox{-}hláhn\mbox{-}an}; \wordng{Mj}{[ʔi]hl(ʲ)én\mbox{-}eʔeʔ / \mbox{-}hlán\mbox{-}eʔeʔ}, \textit{[ti]hláhn\mbox{-}an}, \textit{[ʔi]hl(ʲ)éhn\mbox{-}ijʔ / \mbox{-}hláhn\mbox{-}ijʔ} (\citealt{ND09}: 98, 99, 150; \citealt{AG83}: 174; \citealt{JC18}) {\sep} \wordng{PW}{*[ʔi]ɬǻn\mbox{-}\APPL} > \wordng{’Wk}{[ʔi]ɬǻn\mbox{-}\APPL} \citep[229]{KC16}

\dicnote{\citet{ND09} mistranscribes \wordng{Ijw}{[ʔi]hlʲán\mbox{-}eʔeʔ / \mbox{-}hlán\mbox{-}eʔeʔ} and \intxt{[ʔi]hlʲáhn\mbox{-}ijʔ} as \textit{[ʔi]hlʲán\mbox{-}eʔe / \mbox{-}hlán\mbox{-}eʔe} and \textit{[ʔi]hlʲáhn\mbox{-}i}, respectively.}%1

\gc{Obviously related to \word{Proto-Qom}{*[j]alon}{to light fire} and \word{Kadiwéu}{[j]elo(n)\mbox{-}\APPL}{to light fire}. \citet{PVB13b} does not list this cognate set, but one may reconstruct \wordng{Proto-Guaicuruan}{*[j]alon \recvar *[j]elon}.}

\lit{\citealt{AnG15}: 254}

\PMlemma{{\wordnl{*ɬeɬ}{white snail}} [1]}

\wordng{Ni}{ɬeɬ} \citep[169]{JS16} {\sep} \wordng{PW}{*ɬeɬ} > \wordng{LB/’Wk}{ɬeɬ} (\citealt{VN14}: 51; \citealt{KC16}: 235)

\dicnote{\word{Ijw}{hlέhl\mbox{-}impe}{white monjita\species{Xolmis irupero}} \citep[130]{ND09} is ultimately related to this root, but it is likely a partial calque from \word{PW}{*ɬéɬ\mbox{-}t\mbox{-}’åχ}{snail shell; white monjita\species{Xolmis irupero}} > \wordng{LB}{ɬeɬ\mbox{-}t\mbox{-}’oχ}; \wordng{’Wk}{ɬéɬ\mbox{-}t\mbox{-}’åx} (\citealt{CS-FL-PR-VN13}; \citealt{KC16}: 235).}%1

\PMlemma{{\wordnl{*(\mbox{-})ɬé(ˀ)t}{firewood}}}

\wordng{Mk}{ɬit<uʔ>} [1]\gloss{half-burnt wood} \citep[254]{AG99} {\sep} \wordng{PCh}{*\mbox{-}<ʔa>hlét \recind *\mbox{-}<ʔå>hlét\pla{is}} [2] > \wordng{Ijw}{\mbox{-}ʔahlέt\pl{is}}; \wordng{I’w}{\mbox{-}ahlét\pl{is}}\gloss{burning firewood}; \wordng{Mj}{\mbox{-}ʔahlέt\pl{es \recind \mbox{-}is}} (\citealt{ND09}: 154; \citealt{AG83}: 123; \citealt{JC18}) {\sep} \wordng{PW}{*\mbox{-}ɬét} > \wordng{Vej}{\mbox{-}ɬet}; \wordng{’Wk}{\mbox{-}ɬét} (\citealt{VU74}: 66; \citealt{KC16}: 74)

\dicnote{We have no explanation for the element \intxt{\mbox{-}uʔ} in Maká.}%1

\dicnote{We have no explanation for the element \intxt{*\mbox{-}ʔa\mbox{-}} or \intxt{*\mbox{-}ʔå\mbox{-}} in Chorote.}%2

\largerpage
\gc{Obviously related to \word{Proto-Guaicuruan}{*\mbox{-}oˀlet}{fire} (\citealt{PVB13b}, \#439; cf. \citealt{PVB13a}: 311).}

\lit{\citealt{PVB13a}: 311 (\intxt{*\mbox{-}VɬetVʔ})}\clearpage

\PMlemma{{\wordnl{*\mbox{-}ɬiˀk \recind *\mbox{-}ɬíˀk\plf{*\mbox{-}ɬí\mbox{-}jʰ}}{thread}}}

\wordng{Ni}{\mbox{-}ɬiˀtʃ\plf{\mbox{-}ɬi\mbox{-}j<is>}} \citep[169]{JS16} {\sep} \wordng{PCh}{*\mbox{-}hlík\plf{*\mbox{-}hlí\mbox{-}jʰ}} > \wordng{I’w}{\mbox{-}hlék\plf{\mbox{-}hlé\mbox{-}j}}; \wordng{Mj}{\mbox{-}hlɪ́k} (\citealt{AG83}: 174; \citealt{JC18})

\PMlemma{{\wordnl{*\mbox{-}ɬuˀk\plf{*\mbox{-}ɬú\mbox{-}jʰ}}{\textit{yica} bag, load}}}

\wordng{Mk}{\mbox{-}ɬuˀk} [1], \intxt{\mbox{-}ɬu\mbox{-}j} \citep[255]{AG99} {\sep} \wordng{Ni}{\mbox{-}ɬuˀk} \citep[171]{JS16} {\sep} \wordng{PCh}{*\mbox{-}hlúk\plf{*\mbox{-}hlúj\mbox{-}…}} > \wordng{Ijw}{\mbox{-}hlók\plf{\mbox{-}hló\mbox{-}j<eʔ>}}; \wordng{Mj}{\mbox{-}hlʊ́k} (\citealt{ND09}: 119; \citealt{JC18}) {\sep} \wordng{PW}{*\mbox{-}ɬukʷ}, \textit{*\mbox{-}ɬú\mbox{-}j<is>}\gloss{bag, load} > \wordng{LB}{\mbox{-}ɬekʷ}; \wordng{Vej}{\mbox{-}ɬuk} [2]; \wordng{’Wk}{\mbox{-}ɬuk\plf{\mbox{-}ɬú\mbox{-}j<is>}} (\citealt{VN14}: 418; \citealt{VU74}: 66; \citealt{KC16}: 76)

\dicnote{The presence of a preglottalized coda in the singular form in Maká is inferred based on the Nivaĉle cognate; it is not attested in our sources that distinguish between plain and preglottalized stops.}%1

\dicnote{The absence of labialization in the final consonant in the Vejoz reflex might be a mistranscription on \cits{VU74} part.}%2

\PMlemma{{\wordnl{*ɬúmʔa}{day}}}

\word{Ni}{ɬumʔa\mbox{-}ʃi}{tomorrow}; \intxt{ɬumʔa\mbox{-}kɸinuk \recind ɬumå\mbox{-}kxinuk\pl{its}} [1]\gloss{east} \citep[171]{JS16} {\sep} \wordng{PCh}{*hlúmaʔ\pla{s}} > \wordng{Ijw}{hlóma\pl{s}} [1]\gloss{day, air, east}; \wordng{I’w}{hlóma\pl{s}} [1]; \wordng{Mj}{hlʊ́maʔ\pl{s}} (\citealt{ND09}: 131; \citealt{AG83}: 175; \citealt{JC18})

\dicnote{The variant \intxt{ɬumå\mbox{-}kxinuk} is irregular.}%1

\dicnote{The absence of a final glottal stop in \cits{ND09} and \cits{AG83} attestations of the Iyojwa’aja’ and Iyo’awujwa’ reflexes must be a mistranscription.}%2

\rej{\citet[38, 51]{EN84} includes the Wichí term for\gloss{east} (cf. \wordng{Vej}{hʷoma} in \citealt{VU74}: 60) into the comparison. By contrast, \citet[254]{AnG15} compares the Chorote noun with the reflexes of \word{PM}{*ˀnáɬu(h)\plf{ˀnáɬu\mbox{-}ts}}{day, world}. Both proposals are untenable for phonological reasons.}

\lit{\citealt{EN84}: 38, 51 (\intxt{*hlɔwmahn})}

\PMlemma{{\wordnl{*ɬútsX₂₃a(ʔ)\pla{jek}}{girl}}}

\wordng{Ni}{ɬutsxa\pl{jetʃ}} \citep[171]{JS16} {\sep} \wordng{PCh}{*hlúsaʔ\pla{jek}} > \wordng{Ijw}{hlóxse} [1]; \wordng{I’w}{hlóxsa \recind lúxsa\plf{lúxsa\mbox{-}ji}} [2]; \wordng{Mj}{(ʔa)hlʊ́xsaʔ}, \textit{hlʊ́xse\mbox{-}jik} (\citealt{ND09}: 132; \citealt{AG83}: 147, 203; \citealt{JC18}) {\sep} \wordng{PW}{*ɬútsha\pla{jʰ}} [3] > \wordng{LB}{ɬetsʰa}; \wordng{Vej}{ɬutsʰa} (\intxt{\mbox{-}j \recind ɬutsa\mbox{-}j}); \wordng{’Wk}{ɬútsʰaʔ\pl{ç}} (\citealt{VN14}: 182; \citealt{MG-MELO15}: 51; \citealt{KC16}: 239)

\dicnote{The expected Iyojwa’aja’ form would be \textit{*hlóxseʔ} */hlúsa/, not \textit{hlóxse} /hlúsah/.}%1

\dicnote{The variant with an \textit{l\mbox{-}}, given by \citet{AG83}, is irregular. The plural suffix \intxt{\mbox{-}ji} (as opposed to the expected \textit{\mbox{-}jik}) could be a mistranscription.}%2

\dicnote{The plural form attested in Wichí does not match those seen in Nivaĉle and Manjui.}%3

\lit{\citealt{EN84}: 26 (\intxt{*hlutsh\mbox{-}a}); \citealt{AnG15}: 254}

\PMlemma{{\wordnl{*ma}{interrogative particle (heads polar interrogatives)}}}

\wordng{Mk}{me} \citep[195]{AG94} {\sep} \wordng{PCh}{*ma} > \wordng{Ijw}{ma / =mi}; \wordng{Mj}{ma} (\intxt{mi} before \textit{i}, \textit{hi}) (\citealt{JC14a}; \citealt{ND09}: 139; \citealt{JC18})

\gc{\citet[318]{PVB13a} compares the Mataguayan particle with \word{Abipón}{m\mbox{-}}{polar question marker} \citep[103]{EN66}.}

\lit{\citealt{RJH15}: 241; \citealt{PVB13a}: 318 (\intxt{*me})}

\PMlemma{{\wordnl{*[ji]må}{to sleep}}}

\wordng{Mk}{[i]maʔ} \citep[260]{AG99} {\sep} \wordng{Ni}{[ji]måʔ} \citep[175]{JS16} {\sep} \wordng{PCh}{*[ʔi]mǻʔ} > \wordng{Ijw}{[ʔi]mʲáʔ}; \word{I’w}{\mbox{-}máʔa}{to sleep}; \word{Mj}{[ʔi]m(ʲ)éʔ / \mbox{-}máʔ}{to roam through the forest for game or honey hunting}, \wordnl{[ʔi]m(ʲ)é\mbox{-}ʔeʔ / \mbox{-}má\mbox{-}ʔaʔ}{to sleep} (\citealt{ND09}: 102; \citealt{AG83}: 148; \citealt{JC18}) {\sep} \wordng{PW}{*[ʔi]må} > \wordng{LB}{[ʔi]mo}; \wordng{Vej}{[hi]må} [1]; \wordng{’Wk}{[ʔi]måʔ} (\citealt{VN14}: 209; \citealt{MG-MELO15}: 34; \citealt{KC16}: 239)

\dicnote{\citet[66]{VU74} mistranscribes the root as \intxt{\mbox{-}ma}.}%1

\gc{\citet[306]{PVB13a} notes the similarity with \word{Proto-Guaicuruan}{*\mbox{-}oma}{to lie (with)} (\citealt{PVB13b}, \#440), which could be spurious.}

\lit{\citealt{EN84}: 10, 18, 41 (\intxt{*må}, 2~\intxt{*hl\mbox{-}må}); \citealt{PVB13a}: 306 (\intxt{*\mbox{-}maʔ})}

\PMlemma{{\wordnl{*mǻh}{go!}}}

\wordng{Mk}{ma} \citep[259]{AG99} {\sep} \wordng{Ni}{må} \citep[175]{JS16} {\sep} \wordng{PCh}{*mǻʰ} > \wordng{Ijw}{má(h)}; \wordng{Mj}{mɔ́h} [1] (\citealt{JC14b}: 86; \citealt{ND09}: 139; \citealt{JC18}) {\sep} \wordng{PW}{*mǻh} > \word{LB}{mo}{go ahead!}; \wordng{Vej}{mä(h)} [2]; \wordng{’Wk}{mǻh} (\citealt{VN14}: 284; \citealt{VU74}: 66; \citealt{MG-MELO15}: 25; \citealt{KC16}: 239)

\dicnote{The vowel \intxt{ɔ} in Manjui is an irregular reflex of \intxt{*ǻ}.}%1

\dicnote{\citet[66]{VU74} mistranscribes the Vejoz form as \intxt{ma}.}%2

\gc{Obviously related to \word{Proto-Guaicuruan}{*mo}{you go; go!} (\citealt{PVB13b}, \#385; cf. \citealt{PVB13a}: 305).}

\lit{\citealt{PVB13a}: 305 (\intxt{*mʌ})}

\PMlemma{{\wordnl{*\mbox{-}mǻˀk\plf{*\mbox{-}mhǻ\mbox{-}jʰ}}{powder, flour}}}

\wordng{Ni}{\mbox{-}måˀk\plf{\mbox{-}mxå\mbox{-}j}} \citep[175]{JS16} {\sep} \wordng{PCh}{*\mbox{-}mǻk} > \wordng{Ijw}{\mbox{-}mák}; \wordng{I’w}{wátso\mbox{-}hl\mbox{-}<a>mák} [1]; Mj~\third{hl\mbox{-}<a>mák} [1] (\citealt{ND09}: 124; \citealt{AG83}: 168; \citealt{JC18}) {\sep} \wordng{PW}{*\mbox{-}mókʷ}, \textit{*\mbox{-}mhó\mbox{-}jʰ} [2] > \wordng{LB}{\mbox{-}muq} [3]; \wordng{Vej}{\mbox{-}mok’} [4]; \wordng{’Wk}{\mbox{-}mók}, \textit{\mbox{-}m̥ó\mbox{-}ç} (\citealt{VN14}: 212; \citealt{VU74}: 67; \citealt{KC16}: 76)

\dicnote{The element \intxt{\mbox{-}a\mbox{-}} in Iyo’awujwa’ and Manjui is plausibly the same root as \word{PM}{*\mbox{-}áʔ}{fruit}. The Chorote make use of two plant species, \textit{Prosopis alba} and \textit{Ziziphus mistol}, whose fruit are commonly “ground into flour and sometimes molded into dough to make small cakes or biscuits, which are then cooked” (\citealt{PA-FGS-07}: 77, 84, 85).}%1

\dicnote{\sound{PW}{*o} is not a known regular reflex of \sound{PM}{*å}.}%2

\dicnote{The final \intxt{q} instead of \intxt{kʷ} in the Lower Bermejeño form could be a mistranscription on \cits{VN14} part.}%3

\dicnote{Final \intxt{\mbox{-}k’} in \cits{VU74} attestation of the Vejoz reflex could be a mistranscription for \intxt{\mbox{-}kʷ}.}%4

\gc{Likely related to \word{Proto-Guaicuruan}{*áˀmoqo}{powder} (\citealt{PVB13b}, \#47; cf. \citealt{PVB13a}: 311). \word{LB}{ʔamuqu}{manioc} \citep[52]{VN14} is clearly borrowed from an unidentified Guaicuruan language, with the semantic development *\enquote{powder} > *\enquote{(manioc) flour} >\gloss{manioc}.}

\lit{\citealt{EN84}: 21, 45 (\intxt{*hmåk’}); \citealt{LC-VG-07}: 15; \citealt{PVB13a}: 311 (\intxt{*\mbox{-}mʌq’})}

\PMlemma{{\wordnl{*mǻxå \recind *máxå}{yellow}}}

\wordng{Mk}{maːxa\plf{maxa-m}} \citep[259]{AG99} {\sep} \wordng{PCh}{*mǻhåʔ \recind *máhåʔ} > \wordng{Ijw}{máhaʔ} (\citealt{ND09}: 139)

\PMlemma{{\textit{*mät} [1]\gloss{hither; nearby}}}

\wordng{Mk}{met} [1]\gloss{nearby} \citep[260]{AG99} {\sep} \word{PCh}{*mét}{hither} > \wordng{Ijw}{mέt}; \wordng{I’w}{\mbox{-}met}; \wordng{Mj}{mέt} [2] (\citealt{ND09}: 139; \citealt{AG83}: 121; \citealt{JC18})

\dicnote{The absence of preglottalization in the coda in PM and in Maká is shown by the attestations of the Maká reflex in the New Testament (e.g. Matthew 14:18).}%1

\dicnote{The Manjui reflex is mistranscribed as \intxt{mɪ́t} in \citet{JC18}.}%2

\PMlemma{{\textit{*me(ʔ) \recind *mé(ʔ)} [1]\gloss{otter}}}

\wordng{Mk}{miʔ\pl{l}} \citep[261]{AG99} {\sep} \wordng{Ni}{meʔ} \citep[174]{JS16} {\sep} \wordng{PCh}{*méʔ} > \wordng{Ijw}{mέʔ} \citep[139]{ND09}

\dicnote{The dubious status of the word-final glottal stop and of the prosodical properties of the root are due to the absence of a known cognate in Wichí.}%1

\PMlemma{{\wordnl{*mijó\pla{l}}{savannah hawk}}}

\wordng{Mk}{mijo\pl{l}} \citep[261]{AG99} {\sep} \word{Ni}{mijo\pl{k}}{black-collared hawk} \citep[174]{JS16} {\sep} \wordng{PCh}{*mijóʔ\pla{l}} > \wordng{Ijw}{mijóʔ}; \wordng{Mj}{ˀmijóʔ\pl{l}} [1] (\citealt{ND09}: 139; \citealt{JC18}) {\sep} \wordng{PW}{*mijóh} > \wordng{LB}{miju}; \wordng{Vej}{mijo}\gloss{eagle}; \word{’Wk}{mijóh}{bird sp.} (\citealt{CS-FL-PR-VN13}; \citealt{MG-MELO15}: 21; \citealt{KC16}: 250)

\dicnote{The glottalized nasal \intxt{ˀm} in Manjui is irregular.}%1

\gc{Possibly related to \word{Proto-Pilagá–Toba}{*májo}{large bird} (\citealt{PVB13b}, \#114; cf. \citealt{PVB13a}: 310).}

\lit{\citealt{PVB13a}: 310 (\intxt{*mijo})}

\PMlemma{{\intxt{*\mbox{-}muk\plf{*\mbox{-}mhu\mbox{-}jʰ}} [1]\gloss{feces}}}

\wordng{Mk}{\mbox{-}<i>muk\plf{\mbox{-}<i>mhu\mbox{-}j}} (\citealt{AG99}: 201, 253) {\sep} \wordng{Ni}{(\mbox{-})<sa>muk\plf{(\mbox{-})<sa>mxu\mbox{-}j}} \citep[230]{JS16} {\sep} \wordng{PCh}{*\mbox{-}<ˀjá>muk} > \wordng{Ijw}{\mbox{-}ˀjémuk\plf{\mbox{-}ˀjému\mbox{-}s}} [2]; \wordng{I’w}{\mbox{-}jémuk} [3]; \wordng{Mj}{\mbox{-}ˀjémuk\plf{\mbox{-}ˀjéhmoo\mbox{-}j}} [2 4] (\citealt{ND09}: 128; \citealt{AG83}: 134; \citealt{JC18}) {\sep} \wordng{PW}{*\mbox{-}<ˀjá>mukʷ\plf{*\mbox{-}<ˀjá>mhu\mbox{-}jʰ}} > \wordng{Vej}{\mbox{-}jamok} [3 4]; \wordng{’Wk}{\mbox{-}ˀjámuk\plf{\mbox{-}ˀjám̥u\mbox{-}ç}} (\citealt{VU74}: 83; \citealt{KC16}: 57)

\dicnote{In all daughter languages, this root occurs in what looks like obscure, non-analyzable compounds, with the elements \wordng{Mk}{\mbox{-}i\mbox{-}}, \wordng{Ni}{\mbox{-}sa\mbox{-}}, and \wordng{PCh/PW}{*\mbox{-}ˀjá\mbox{-}}.}%1

\dicnote{The plural forms in Iyojwa’aja’ and Manjui are non-etymological.}%2

\dicnote{The lack of glottalization in \intxt{j} in the Iyo’awujwa’ and Vejoz reflexes could be a mistranscription on our sources’ part.}%3

\dicnote{The vowel \intxt{o}, attested in the Manjui (plural only) and Vejoz reflexes, may be attributed to contamination with reflexes of \word{PM}{*\mbox{-}mǻˀk\plf{*\mbox{-}mhǻ\mbox{-}jʰ}}{powder, flour}. The absence of labialization in the stem-final consonant in Vejoz is irregular.}%4

\gc{\word{Toba–Qom}{jamok}{feces} \citep[187]{ASB-LLB-13} lacks known cognates in other Guaicuruan languages and is thus likely to be a Wichí loan.}

\lit{\citealt{LC-VG-07}: 15}

\PMlemma{{\wordnl{*[ʔa]ˀmån \recind *[ʔa]ˀmǻn}{to stay, to be alive}}}

\wordng{Mk}{<a>man} [1]\gloss{to stay, to stop} (\citealt{AG99}: 119–120) {\sep} \wordng{Ni}{mån<ɬa> / \mbox{-}ˀmån<ɬa>} [2] \citep[175]{JS16} {\sep} \wordng{PCh}{*[ʔa]ˀmán<hliʔ>} [2 3] > \word{Ijw}{ˀwán\mbox{-}hle\mbox{-}ʔe}{to stay} [4]; \wordng{I’w}{\mbox{-}mánni\mbox{-}ji}\gloss{to live} [5]; \word{Mj}{[ʔa]ˀmán\mbox{-}hiʔ}{to be alive}, \wordnl{[ʔa]ˀmánhi\mbox{-}ʔiʔ}{to stay}; \textsc{caus~}\textit{*[ʔi]ˀmǻn\mbox{-}it} > \word{Ijw}{[ʔi]ˀmʲén\mbox{-}it/ \mbox{-}ˀmán\mbox{-}it}{to defend, to cure}; \word{Mj}{[ʔi]ˀm(ʲ)én\mbox{-}it/ \mbox{-}ˀmán\mbox{-}it}{to save} (\citealt{ND09}: 163; \citealt{AG83}: 148; \citealt{JC18}) {\sep} \word{PW}{*[ʔi]mǻ<ɬ>\mbox{-}\APPL}{to stay} [2] > \word{LB}{[ʔi]moɬ\mbox{-}i}{to be the last}; \wordng{Vej}{\mbox{-}maɬ\mbox{-}e} [6]; \wordng{’Wk}{[ʔi]mǻɬ\mbox{-}\APPL}; \textsc{caus}~\intxt{*[ʔi]mǻɬ\mbox{-}t\mbox{-}\APPL} > \word{LB}{[ʔi]moɬ\mbox{-}t\mbox{-}ʰi}{to leave, to extract}; \wordng{’Wk}{[ʔi]mǻɬ\mbox{-}t\mbox{-}\APPL} (\citealt{VN14}: 154, 203, 351; \citealt{VU74}: 67; \citealt{KC16}: 240–243)

\dicnote{The Maká reflex unexpectedly lacks preglottalization in the root-initial nasal, as attested in the New Testament (Hebrews 4:9; 2 Peter 2:6; John 7:37; John 8:44; 1 John 3:14; Revelations 10:6).}%1

\dicnote{All languages except Maká (and Chorote, in the case of the causative) have fossilized a suffix or a sequence of suffixes starting with \textit{*ɬ}.}%2

\dicnote{\wordng{PCh}{*a} is not the regular reflex of \wordng{PM}{*å}. }%3

\dicnote{\wordng{Ijw}{ˀw} is not the regular reflex of \wordng{PCh}{*ˀm}.}%4

\dicnote{The Iyo’awujwa’ form in \citet[148]{AG83} is likely a mistranscription for \intxt{\mbox{-}ˀmánhi\mbox{-}ijʔ}.}%5

\dicnote{The vowel \intxt{a} in the Vejoz reflex is likely a mistranscription on \cits{VU74} part.}%6

\PMlemma{{\wordnl{*ˀmók\pla{its}}{creamy-bellied thrush\species{Turdus amaurochalinus}}}}

\word{Mk}{mok\pl{its}}{kind of \textit{zorzal}\species{Turdus sp.}} [1] \citep[261]{AG99} {\sep} \wordng{Ni}{mok\pl{is}} \citep[174]{JS16} {\sep} \wordng{PCh}{*ˀmók\pla{is}} > \word{Mj}{ˀmɔ́k\pl{is}}{kind of \textit{zorzal}\species{Turdus sp.}} \citep{JC18}

\dicnote{\word{Mk}{maq\mbox{-}itaχ\plf{maq\mbox{-}ite\mbox{-}ts}}{creamy-bellied thrush\species{Turdus amaurochalinus}} \citep[259]{AG99} is obviously indirectly related to this root. It may have been borrowed from \word{Ni}{mok\mbox{-}itax\plf{mok\mbox{-}ita\mbox{-}s}}{creamy-bellied thrush\species{Turdus amaurochalinus}}, though the phonological adaptation pattern remains unaccounted for.}%1

\rej{\citet[13]{EN84} compares the Nivaĉle reflex to \word{Vej}{woktak’ak}{\textit{cochapoye} bird} (\citealt{VU74}: 81) and reconstructs \wordng{PM}{*mɔk \recind *wɔk}, which is problematic from a phonological point of view.}

\gc{Compare \word{Toba–Qom}{mok}{\textit{Podager facunda}; \textit{Nyctibius griseus}; \textit{Turdus amaurochalinus}} \citep[248]{PC-AP-09}, which does not reconstruct to Proto-Guaicuruan and is thus a probable loan from a Mataguayan language.}

\PMlemma{{\wordnl{*\mbox{-}nájʰ}{to bathe}}}

\wordng{Ni}{[βa]naj} \citep[184]{JS16} {\sep} \wordng{PCh}{*[ʔi]náj\mbox{-}\APPL} > \wordng{Ijw}{[ʔi]nʲéhj\mbox{-}iʔ / \mbox{-}náhj\mbox{-}iʔ} [1]; \wordng{I’w}{\mbox{-}náj\mbox{-}i\mbox{-}náhtiʔ}; \wordng{Mj}{[ʔi]néhj\mbox{-}ijʔ / \mbox{-}náhj\mbox{-}ijʔ} (\citealt{JC14b}: 93; \citealt{AG83}: 149; \citealt{JC18}) {\sep} \wordng{PW}{*[ʔi]nájʰ} > \wordng{LB}{[ʔi]naj}; \wordng{Vej}{\mbox{-}naj}; \wordng{’Wk}{[ʔi]náç} (\citealt{VN14}: 251; \citealt{VU74}: 67; \citealt{KC16}: 259)

\dicnote{\citet[102]{ND09} mistranscribes this form as \textit{[ʔi]nʲéhj\mbox{-}i / \mbox{-}náhj\mbox{-}i}.}%1

\gc{\citet[306]{PVB13a} notes the similarity with \word{Proto-Guaicuruan}{*\mbox{-}n\mbox{-}ij’ó}{to wash oneself}, which could be spurious.}

\lit{\citealt{PVB13a}: 306 (\intxt{*\mbox{-}naj})}

\PMlemma{{\wordnl{*náwa(ˀ)j(\mbox{-}xiʔ)}{to boil}}}

\wordng{Ni}{naβaj\mbox{-}ʃi} \citep[183]{JS16} {\sep} \wordng{PCh}{*náwahj\mbox{-}ijʔ} > \wordng{Ijw}{náwahj\mbox{-}iʔ}; \wordng{Mj}{náwohj\mbox{-}ijʔ} [1] (\citealt{ND09}: 140; \citealt{JC18}) {\sep} \wordng{PW}{*náwaj}, \intxt{*náˈwaj-hi} > \wordng{LB}{nawaç\mbox{-}i}; \wordng{’Wk}{náwajʔ}, \intxt{náˈwaç-iʔ} (\citealt{VN14}: 48; \citealt{KC16}: 259)

\dicnote{The unstressed vowel rounding in Manjui is not known to be regular, though it does sometimes happen next to a \intxt{w}.}%1

\PMlemma{{\wordnl{*náwa(ˀ)x}{cactus sp.}}}

\word{Ni}{naβaʃ\pl{ik}}{cactus fruit (ca.~5~cm in diameter and height, its pulp is very good for killing one’s thirst)} \citep[183]{JS16} {\sep} \word{PW}{*náwaχ}{cactus\species{Echinopsis rhodotricha}} > \wordng{Southeastern (Salta)}{nawaχ}; \wordng{’Wk}{náwax} (\citealt{MS14}: 234; \citealt{KC16}: 259)

\PMlemma{{\wordnl{*\mbox{-}naˀx \recind *\mbox{-}náˀx\plf{*\mbox{-}nxá\mbox{-}ts}}{nose} [1]}}

\wordng{Mk}{\mbox{-}neˀx\plf{\mbox{-}nex\mbox{-}its}} [1] \textit{/ \mbox{-}nxe\mbox{-}} (\citealt{AG99}: 151; \citealt{JB81}: 202) {\sep} \wordng{Ni}{\mbox{-}naˀʃ\plf{\mbox{-}nʃa\mbox{-}s}} \citep[177]{JS16} {\sep} \wordng{PCh}{*\mbox{-}hná<tVwoh>} [2] > \wordng{Ijw}{\mbox{-}hnátawo\pl{s}}; \wordng{I’w}{\mbox{-}hnátowu \recind \mbox{-}hnátawo \recind \mbox{-}hnátowe\mbox{-}} (\intxt{\mbox{-}hnátowe\mbox{-}j}); \wordng{Mj}{\mbox{-}hnátowo} (\citealt{JC14b}: 98; \citealt{ND09}: 119; \citealt{AG83}: 175, 210; \citealt{JC18}) {\sep} \wordng{PW}{*\mbox{-}nh<us>} [1] > \wordng{LB}{\mbox{-}n̥es\pl{ej}}; \wordng{Vej}{\mbox{-}n̥us\pl{eɬ}} [3]; \wordng{’Wk}{\mbox{-}n̥us}, \textit{\mbox{-}n̥ús\mbox{-}eɬ} (\citealt{VN14}: 161; \citealt{MG-MELO15}: 60; \citealt{KC16}: 79)

\dicnote{The Maká plural is non-etymological. The presence of a preglottalized coda in the singular form is inferred based on the Nivaĉle cognate; this form is otherwise not attested in our sources that distinguish between plain and preglottalized stops, such as \citet{PMA} and the New Testament.}%1

\dicnote{The Chorote and Wichí words are obscure compounds involving \wordng{PM}{*\mbox{-}nxa\mbox{-}}.}%2

\dicnote{\citet[69]{VU74} documents this root as \intxt{\mbox{-}nus} in Vejoz, which must be a mistranscription on her part.}%3

\PMlemma{{\wordnl{*\mbox{-}nå(ʔ) \recind *\mbox{-}nǻ(ʔ)\pla{wot}}{father}}}

Mk (Lengua doculect) ‹inà›\gloss{my father}, ‹sanã›\gloss{father} \citep[488]{EP98} {\sep} \word{Ni}{\mbox{-}nå\mbox{-}βot}{parents} \citep[202]{JS16} {\sep} \wordng{PCh}{*\mbox{-}nåʔ}, \textit{*\mbox{-}ná\mbox{-}wot} > \wordng{Ijw}{\mbox{-}náʔ\plf{\mbox{-}wot}}, \textit{\mbox{-}jis}); \wordng{I’w}{\mbox{-}náʔ\pl{wot}}; \wordng{Mj}{\mbox{-}náʔ} (\citealt{JC14b}: 101; \citealt{ND09}: 124; \citealt{AG83}: 149; \citealt{JC18})

\PMlemma{{\wordnl{*néwo(ˀ)k}{wild manioc\species{Marsdenia castillonii}} [1]}}

\wordng{Ni}{noβok\plf{noβxo\mbox{-}j}} \citep[198]{JS16} {\sep} (?) \wordng{PCh}{*nᵊwák} [2] > \wordng{Ijw}{niwák}, \textit{\mbox{-}iwa}; (?)~\wordng{I’w}{náwasʲuk \recind náwisʲuk}; (?) \wordng{Mj}{náwasuk \recind náwasek \recind náwosuk} (\citealt{ND09}: 141; \citealt{GS10}: 189; \citealt{JC18}) {\sep} \wordng{PW}{*néwokʷ} > \wordng{LB}{newukʷ}; \wordng{Southeastern (Salta)}{newuk}; \wordng{Vej}{newok}; \wordng{’Wk}{néwok} (\citealt{CS08}: 60; \citealt{MS14}: 189; \citealt{MG-MELO15}: 18; \citealt{KC16}: 265)

\dicnote{\word{Maká}{jowek}{wild manioc} \citep[80]{JB81} is hardly related.}%1

\dicnote{The Chorote forms are entirely irregular and are probably a result of horizontal transmission by the way of non-Mataguayan languages. The Proto-Chorote form is tentatively reconstructed here based on the Iyojwa’aja’ datum; the other two varieties point rather to \intxt{*náwV(i)s\mbox{-}uk \recind *\mbox{-}ǻ\mbox{-}}.}%2

\empr{\citet[300]{PVB13a} notes the similarity with \word{Proto-Guaicuruan}{*nawjék}{kind of tuber (similar to manioc)} (\citealt{PVB13b}, \#396) and attributes it to language contact.}

\lit{\citealt{PVB13a}: 300}

\PMlemma{{\wordnl{*(\mbox{-})níjåk\plf{*(\mbox{-})níjhå\mbox{-}jʰ}}{rope, cord}}}

\wordng{Mk}{(\mbox{-})nijak\plf{(\mbox{-})nijha\mbox{-}j}} \citep[275]{AG99} {\sep} \wordng{Ni}{\mbox{-}nijåk\plf{\mbox{-}nijxå\mbox{-}j}} \citep[198]{JS16} {\sep} \wordng{PCh}{*níjåk\plf{*níhjå\mbox{-}jʰ}} > \wordng{Ijw}{néjak\plf{néhja\mbox{-}ʔ \recind néhja\mbox{-}ˀl}} [1]; (?) \wordng{I’w}{\mbox{-}jék\plf{\mbox{-}hjé\mbox{-}j}} [2]; (?) \wordng{Mj}{\mbox{-}(ʔi)jík\plf{\mbox{-}ʔihjí\mbox{-}jh}} [2] (\citealt{ND09}: 141; \citealt{AG83}: 133; \citealt{JC18}) {\sep} \wordng{PW}{*níjåkʷ\plf{*níjhå\mbox{-}jʰ}} > \wordng{LB}{nijokʷ\plf{niço\mbox{-}j}}; \wordng{Vej}{nijak}; \wordng{’Wk}{níjåk\plf{níçå\mbox{-}ç}} (\citealt{VN14}: 192; \citealt{VU74}: 68; \citealt{KC16}: 273)

\dicnote{The plural variant \intxt{néhja\mbox{-}ˀl}, attested in \citet[141]{ND09}, is non-etymological. The word-final glottal stop in the variant \intxt{néhja\mbox{-}ʔ} is likewise irregular, but there are other cases where the plural suffix \intxt{*\mbox{-}(a)jʰ} yielded Iyojwa’aja’ \textit{\mbox{-}(a)ʔ} (e.g. in the participles).}%1

\dicnote{The Iyo’awujwa’ and Manjui forms are not the expected reflexes of \wordng{PM}{*(\mbox{-})níjåk\plf{*(\mbox{-})níjhå\mbox{-}j}}.}%2

\dicnote{The vowel \intxt{a} (as opposed to \intxt{å}) in Vejoz must be a mistranscription on \cits{VU74} part.}%3

\lit{\citealt{EN84}: 18 (\intxt{*nejåwk}); \citealt{LC-VG-07}: 15 (“diffused”), 21}

\PMlemma{{\wordnl{*\mbox{-}njiˀx}{smell}}}

\wordng{Mk}{\mbox{-}njiˀx} [1], \intxt{\mbox{-}njix\mbox{-}its} \citep[151]{AG99} {\sep} \wordng{Ni}{\mbox{-}niˀʃ} \citep[190]{JS16} {\sep} \wordng{PCh}{*\mbox{-}níh} > \wordng{Ijw}{\mbox{-}néh}; \wordng{I’w}{\mbox{-}né\pl{hes}}; \wordng{Mj}{\mbox{-}néih} (\citealt{JC14b}: 71; \citealt{ND09}: 124; \citealt{AG83}: 150; \citealt{JC18}) {\sep} \wordng{PW}{*\mbox{-}niχ\plf{*\mbox{-}nh\mbox{-}ís}} > \wordng{LB}{\mbox{-}niχ}; \wordng{’Wk}{\mbox{-}nix\plf{\mbox{-}n̥\mbox{-}ís}} (\citealt{VN14}: 202; \citealt{KC16}: 78)

\dicnote{The presence of a preglottalized coda in the Maká singular form is inferred based on the Nivaĉle cognate; the singular form is not attested in our sources that distinguish between plain and preglottalized stops.}%1

\gc{Obviously related to \word{Proto-Guaicuruan}{*\mbox{-}(ˀ)nik}{smell; bad smell} (\citealt{PVB13b}, \#405; cf. \citealt{PVB13a}: 311).}

\lit{\citealt{EN84}: 31 (\intxt{*nehn}); \citealt{PVB02}: 143 (\intxt{*(V)nix}); \citealt{PVB13a}: 311 (\intxt{*\mbox{-}(a)nih})}

\PMlemma{{\wordnl{*n̩k’a}{new, recently}; \intxt{*n̩k’a\mbox{-}jik\plf{*n̩k’a\mbox{-}jh\mbox{-}its}} (fem. \intxt{*n̩k’a\mbox{-}jk\mbox{-}eʔ})\gloss{new}}}

\word{Mk}{iˀnk’a}{recently}; \textit{iˀnk’a\mbox{-}jik}, \textit{iˀnk’a\mbox{-}jh\mbox{-}its} (fem. \textit{iˀnk’a\mbox{-}jk\mbox{-}iʔ\plf{iˀnk’a\mbox{-}jk\mbox{-}i\mbox{-}j}})\gloss{new} [1] (\citealt{AG99}: 203–204) {\sep} \word{Ni}{nitʃ’a\pl{k}}{new}; \wordnl{nitʃ’a\mbox{-}jik}{young, boy} (fem. \textit{nitʃ’a\mbox{-}jik\mbox{-}eʔ, nitʃ’a\mbox{-}jik\mbox{-}ej}) (\citealt{JS16}: 188–189; \citealt{AF16}: 110) {\sep} \wordng{PCh}{*n̩k’áʔ} > \word{Ijw}{ʔinkʲ’éʔ}{new}; \word{I’w}{inkʲéʔ}{new}; \wordng{Mj}{(ʔin)kʲ’éʔ}; \wordng{PCh}{*n̩k’á\mbox{-}jik\plf{*n̩k’á\mbox{-}hj\mbox{-}is}} (fem. \intxt{*n̩kʲ’á\mbox{-}jk\mbox{-}eʔ}) > Ijw~\textsc{pl}~\intxt{ʔinkʲ’é\mbox{-}hj\mbox{-}is}; \wordng{Mj}{ʔinkʲ’é\mbox{-}jik\plf{ʔinkʲ’é\mbox{-}hj\mbox{-}is}} (fem. \intxt{ʔinkʲ’é\mbox{-}jʃ\mbox{-}iʔ}) (\citealt{ND09}: 109; \citealt{AG83}: 131; \citealt{JC18}) {\sep} \wordng{PW}{*nekʲ’a / *nékʲ’a \recind *nekʲ’e / *nékʲ’e} [2]\gloss{recently, just now} > \wordng{LB}{netʃ’a \recind netʃ’e}; \wordng{Vej}{netʃ’e} [3]\gloss{already}; \word{’Wk}{nekʲ’eʔ / nékʲ’eʔ}{new, recently, just now}; \intxt{*nékʲ’a\mbox{-}jik\plf{*nékʲ’a\mbox{-}jh\mbox{-}is} \recind *nékʲ’e\mbox{-}jik\plf{*nékʲ’e\mbox{-}jh\mbox{-}is}} [2]\gloss{new} > \wordng{LB}{netʃ’a\mbox{-}jik}; \wordng{Vej}{netʃ’a\mbox{-}jek} [3]\gloss{new}; \wordng{’Wk}{nékʲ’e\mbox{-}jik\plf{nékʲ’e\mbox{-}ç\mbox{-}is}} (\citealt{VN14}: 297; \citealt{VU74}: 68; \citealt{MG-MELO15}: 8; \citealt{KC16}: 263–264)

\dicnote{\sound{Maká}{a} is not the expected reflex of \sound{PM}{*a}. The preglottalization in \textit{ˀn} is attested in the New Testament (e.g. Galatians 6:15).}%1

\dicnote{The Wichí reflex shows an irregular reflex of the vowel of the initial syllable and an irregular dialectal variation in the second syllable (\intxt{a \recind e}).}%2

\dicnote{The plain \intxt{tʃ} in \cits{VU74} attestation of the Vejoz root must be a mistranscription.}%3

\PMlemma{{\wordnl{*n̩näˀk / *\mbox{-}nnäˀk}{spoon}}}

\wordng{Mk}{neneˀk} [1], \intxt{nenek\mbox{-}its}\gloss{spoon, bivalve} \citep[272]{AG99} {\sep} \wordng{PW}{*<ɬ̩>nnek} [2] > \wordng{LB}{lanek}; \wordng{Vej}{lenek}; \wordng{’Wk}{la(n)nek\plf{la(n)nék\mbox{-}is}}; \textit{*\mbox{-}<qá>nnek} [3] > \wordng{Vej}{\mbox{-}kanek}; \wordng{’Wk}{qannek\plf{qann̥\mbox{-}aç} / \mbox{-}qá\mbox{-}nnek\plf{\mbox{-}qá\mbox{-}nn̥\mbox{-}aç}} (\citealt{VN14}: 40; \citealt{VU74}: 61; \citealt{MG-MELO15}: 51; \citealt{KC16}: 86, 218)

\dicnote{The preglottalized coda in the Maká reflex is attested in \citet[69]{JB81}.}%1

\dicnote{\wordng{PW}{*<ɬ̩>nnek} is a fossilized third-person form of the erstwhile relational stem (\wordng{PM}{*\mbox{-}nnäk}).}%2

\dicnote{\wordng{PW}{\mbox{-}<qá>nnek} is a fossilized alienized form of the erstwhile absolute stem (\wordng{PM}{*n̩näk}).}%3

\empr{\citet[301]{PVB13a} claims that \word{Abipón}{\mbox{-}enenk}{spoon} \citep[65]{EN66} is a Mataguayan loanword.}

\lit{\citealt{PVB13a}: 301 (\intxt{*\mbox{-}anek})}

\PMlemma{{\textit{*(\mbox{-})nú(ʔ)\pla{ts}} [1]\gloss{bone}}}

\wordng{Mk}{\mbox{-}nu} [2] (\intxt{\mbox{-}ts})\gloss{bone, stalk} (\citealt{AG99}: 152, 250) {\sep} \wordng{Ni}{\mbox{-}nuʔ\pl{s}} [3] \citep[203]{JS16} {\sep} \wordng{PW}{*nú(ʔ)} > \wordng{LB}{ne(ʔ)}; \wordng{Vej}{nu}; \wordng{’Wk}{núʔ\pl{lis}} (\citealt{JB09}: 52; \citealt{VU74}: 69; \citealt{KC16}: 278)

\dicnote{The plural form is reconstructed based on \wordng{Maká}{\mbox{-}nu\mbox{-}ts} and \wordng{Nivaĉle}{\mbox{-}nu\mbox{-}s}; it is thus technically reconstructible only for Proto-Maká–Nivaĉle (if one accepts the binary split hypothesis). The ’Weenhayek reflex does not match it.}%1

\dicnote{The absence of a final \intxt{ʔ} in the Maká singular form is unexpected.}%2

\dicnote{\citet[515]{LC20} document absolute \intxt{nuʔ} and relational \intxt{\mbox{-}β\mbox{-}nuʔ} for Nivaĉle.}%3

\lit{\citealt{EN84}: 33 (\wordnl{*hnu}{shoulderblade})}

\PMlemma{{\wordnl{*núʔuh\plf{*núʔu\mbox{-}ts}}{dog}}}

\word{Ni}{nuʔu\pl{s}}{dog; black-winged stilt} \citep[205]{JS16} {\sep} \wordng{PCh}{*núʔuh\pla{s}} > \wordng{I’w}{nówu \recind nóo \recind núu\pl{s}}; \wordng{Mj}{nʊ́ʔu\pl{s}} (\citealt{AG83}: 151, 214; \citealt{JC18})

\rej{\citet[38]{EN84} includes reflexes of \word{Wichí}{*hóʔoh}{rooster} (mistranscribed as \intxt{õo \recind õu}), which is impossible both for semantic and phonological reasons.}

\lit{\citealt{EN84}: 18, 38 (\intxt{*nu\mbox{-}o \recind *nɔo})}

\PMlemma{{\intxt{*n̩\mbox{-}xǻteʔ\pla{l}} \recvar \intxt{*n̩\mbox{-}xátiʔ} [1]\gloss{dream, sleepiness}}}

\word{Mk}{\mbox{-}nixatiʔ\pl{l}}{dream}; \wordnl{[ni]xati\mbox{-}juʔ}{to be sleepy} \citep[385]{AG99} {\sep} \word{Ni}{\mbox{-}nxåte\pl{k}}{dream} (\citealt{JS16}: 191–192) {\sep} \word{PCh}{*ʔihnátiʔ}{dream} > \wordng{Ijw}{ʔihnʲéti} [2]; \wordng{I’w}{ihnʲétiʔ} (\citealt{ND09}: 98; \citealt{AG83}: 133) {\sep} \word{PW}{*naháti}{dream; sleepiness} > \wordng{Vej}{nahate}, \wordnl{nehatʰi\mbox{-}ʔilån}{to be very sleepy} [3]; \wordng{’Wk}{nahátiʔ} (\citealt{VU74}: 67; \citealt{MG-MELO15}: 38; \citealt{KC16}: 253)

\dicnote{Maká and Nivaĉle point to \intxt{*n\mbox{-}xǻteʔ\pla{l}}, Chorote and Wichí to \intxt{*n\mbox{-}xátiʔ}. The stem-initial \intxt{*n\mbox{-}} must have been a prefix; its reflex \textit{ni\mbox{-}} is still segmentable in Maká.}%1

\dicnote{The absence of the stem-final glottal stop in Iyojwa’aja’ is unexpected.}%2

\dicnote{The Vejoz reflex is attested as \textit{nahate} by \citet{VU74} and as \textit{nehatʰi\mbox{-}ʔilån} by \citet{MG-MELO15}. The expected form would be \textit{*nahati}.}%3

\gc{Possibly related to \word{Proto-Guaicuruan}{*\mbox{-}eʔot’é}{to sleep} (\citealt{PVB13b}, \#256; cf. \citealt{PVB13a}: 305).}

\lit{\citealt{PVB13a}: 305 (\wordnl{*\mbox{-}hʌteʔ \recind *\mbox{-}hʌtiʔ}{to be sleepy})}

\PMlemma{{\wordnl{*[ji]nxíˀwän}{to smell} [1]}}

\wordng{Mk}{[ji]nxiˀwen} [2] \citep[152]{AG99} {\sep} \wordng{PCh}{*[ʔi]hníˀwen} > \wordng{Ijw}{[ʔi]hníˀwiˀn / \mbox{-}hnéˀwiˀn}; \wordng{I’w}{\mbox{-}hnéwin\mbox{-}e}; \wordng{Mj}{[ʔi]hníˀwen / \mbox{-}hnéiˀwen} (\citealt{ND09}: 98; \citealt{AG83}: 175; \citealt{JC18})

\dicnote{This verb is probably a compound of \wordnl{*\mbox{-}njiˀx}{smell} and \wordnl{[ji]ˀwä́n}{to see}.}%1

\dicnote{The preglottalized onset of the root-final syllable in Maká is attested in the New Testament (e.g. 1~Corinthians 12:17).}%2

\PMlemma{{\textit{*\mbox{-}nX₂₃aqǻt \recvar *\mbox{-}nX₂₃aq’ǻt} [1]\gloss{to snore} [2]}}

\wordng{Ni}{[ta]nxakåt} (\citealt{LC20}: 242) {\sep} \wordng{PCh}{*[ʔi]hnåq’ǻt} [2] > \wordng{Ijw}{[ʔi]hnʲák’at / \mbox{-}hnák’at}; \wordng{I’w}{\mbox{-}hnakát} [1]; \wordng{Mj}{[ʔi]n(ʲ)éʔát / \mbox{-}naʔát} [3] (\citealt{ND09}: 98; \citealt{AG83}: 175; \citealt{JC18})

\dicnote{Nivaĉle points to \wordng{PM}{*q} and Chorote to \intxt{*q’} (except for the Iyo’awujwa’ form as attested by \citealt{AG83}, but this must be a mistranscription).}%1

\dicnote{This etymon is obviously derived from \wordng{PM}{*\mbox{-}naˀx / *\mbox{-}nxa\mbox{-}}\gloss{nose}.}%2

\dicnote{The Manjui form in \citet{JC18} is attested with a root-initial \intxt{n\mbox{-}} and not the expected \textit{*hn\mbox{-}}. This is also the case in \cits{GH94} vocabulary). However, the expected form with \textit{hn\mbox{-}} is found in early unpublished Carol’s field notes.}%3

\PMlemma{{\wordnl{*\mbox{-}nX₂₃átåʔ}{nasal mucus} [1]}}

\wordng{Ni}{\mbox{-}nxatåʔ\pl{j}} \citep[190]{JS16} {\sep} \wordng{PCh}{*\mbox{-}hnát<ijah\mbox{-}}\textsc{pl>} \textsc{[1]} > \wordng{Ijw}{\mbox{-}hnátihje\mbox{-}s}; \wordng{I’w}{\mbox{-}hnátije\mbox{-}j}; \wordng{Mj}{\mbox{-}hnátije\mbox{-}el} (\citealt{ND09}: 119; \citealt{AG83}: 175; \citealt{JC18})

\dicnote{This etymon is obviously derived from \word{PM}{*\mbox{-}naˀx / *\mbox{-}nxa\mbox{-}}{nose}.}%1

\dicnote{Chorote appears to have fossilized a nonproductive suffix here.}%2

\largerpage
\PMlemma{{\wordnl{*ˀnáɬu(h)\plf{*ˀnáɬu\mbox{-}ts}}{day, world}}}

\wordng{Mk}{neɬu\pl{ts}} \citep[271]{AG99} {\sep} \wordng{Ni}{naɬu\pl{s}} \citep[179]{JS16} {\sep} \word{PCh}{*ˀnáhl<ekis> \recind *ˀnáhl<ekes>}{midday} [1] > \wordng{Ijw}{ˀnáhlikis}; \wordng{Mj}{ˀnáhlekis} (\citealt{ND09}: 162; \citealt{JC18})

\dicnote{Chorote appears to have fossilized a nonproductive suffix here.}%1

\rej{\citet[254]{AnG15} includes \wordng{Ijw/I’w}{hlóma} into the comparison, which is better understood as a reflex of \wordng{PM}{*ɬúmʔa}.}

\gc{Likely related to \word{Proto-Guaicuruan}{*nalóʔ}{natural light, day, sun} (\citealt{PVB13b}, \#388). \citet[312]{PVB13a} compares it to \word{Proto-Guaicuruan}{*ʔal’éwa}{earth} instead, which is hardly convincing.}

\lit{\citealt{PVB13a}: 312 (\intxt{*aɬu}); \citealt{AnG15}: 254}

\PMlemma{{\wordnl{*(\mbox{-})ˀnǻjiˀx\plf{*(\mbox{-})ˀnǻjx\mbox{-}ajʰ}}{path}}}

\wordng{Ni}{nåjiʃ\plf{(\mbox{-})nåjʃ\mbox{-}aj / \mbox{-}ˀnåjiˀʃ}} (\citealt{AF16}: 318; \citealt{JS16}: 202) {\sep} \wordng{PCh}{*(\mbox{-})ˀnǻjih}, \textit{*(\mbox{-})ˀnǻhj\mbox{-}ajʰ} > \wordng{Ijw}{(\mbox{-})ˀnáji\plf{(\mbox{-})ˀnáhj\mbox{-}a(ʔ)}} [1]; \wordng{I’w}{náji\plf{nahj\mbox{-}éh}} [2]; \wordng{Mj}{ˀnáji\plf{ˀnáhj\mbox{-}eej}} [3] (\citealt{ND09}: 162; \citealt{AG83}: 149; \citealt{JC18}) {\sep} \wordng{PW}{*(\mbox{-})ˀnǻjiχ\plf{*(\mbox{-})ˀnǻjh\mbox{-}ajʰ}} > \wordng{LB}{(\mbox{-})ˀnojiχ}; \wordng{Vej}{nåjh \recind najih\plf{nåjhåj \recind najhaj}} [4]; \wordng{’Wk}{(\mbox{-})ˀnǻjix\plf{(\mbox{-})ˀnǻç\mbox{-}aç}} (\citealt{VN14}: 40, 164; \citealt{VU74}: 68; \citealt{MG-MELO15}: 43; \citealt{KC16}: 53, 55)

\dicnote{The plural form \intxt{\mbox{-}ˀnáhj\mbox{-}a} is attested by \citet[162]{ND09}, whereas in our data the irregular reflex \intxt{(\mbox{-})ˀnáhj\mbox{-}aʔ} is attested. There are other cases where the plural suffix \intxt{*\mbox{-}(a)jʰ} yielded \wordng{Iyojwa’aja’}{\mbox{-}(a)ʔ} (e.g. in the participles and in \word{Ijw}{néhja\mbox{-}ʔ}{cords, ropes}).}%1

\dicnote{The plain \intxt{n} in \cits{AG83} attestation of the Iyo’awujwa’ reflex must be a mistranscription. The stress on the suffix in the plural form does not match what is found in other Chorote varieties and ’Weenhayek.}%2

\dicnote{The plural suffix found in Manjui is irregular (one would expect \intxt{*ˀnáhj\mbox{-}ej}).}%3

\dicnote{The forms attested in Vejoz are somewhat unexpected. The regular reflex would be \textit{*ˀnåjih}, \textit{*ˀnåjh\mbox{-}aj}.}%4

\gc{This root resembles \word{Proto-Qom}{*<n>aˀdíg}{path}, whose initial consonant is claimed by \citet{PVB13b} to have been fossilized to the root after the split of Proto-Guaicuruan (compare \word{Proto-Guaicuruan}{*\mbox{-}aˀdíko}{path}; \citealt{PVB13b}, \#4). If \wordng{PM}{*(\mbox{-})ˀnǻjix\plf{*(\mbox{-})ˀnǻjx\mbox{-}ajʰ}} is related to the Guaicuruan root, it should be explained as a borrowing from Southern Guaicuruan; alternatively, \wordng{PM}{*ˀn} could continue an erstwhile fossilized prefix (in this case, the Mataguayan and Guaicuruan material could be cognate).}

\lit{\citealt{EN84}: 10, 31, 48 (\intxt{*najehn}); \citealt{PVB02}: 143 (\intxt{*nʌjix})}

\PMlemma{{\wordnl{*ˀnjǻnxteʔ}{chacoan mara (cavy), tapeti}}}

\wordng{Mk}{nijaxtiʔ\pl{l}} \citep[278]{AG99} {\sep} \word{Ni}{nånxate\pl{j}}{chacoan cavy, tapeti, (?) guinea pig} \citep[200]{JS16} {\sep} \wordng{PCh}{*ˀnǻhåteʔ\pla{waʔ}} > \wordng{Ijw}{ˀnáhate}, \textit{ˀnáhati\mbox{-}waʔ} [1]; \wordng{I’w}{náateʔ\pl{j}}; \wordng{Mj}{ˀnáateʔ\pl{waʔ}} (\citealt{ND09}: 162; \citealt{AG83}: 149; \citealt{JC18}) {\sep} \wordng{PW}{*ˣnǻte} > \wordng{LB}{note}; \wordng{Vej}{nåte \recind inåte \recind hnåte\pl{ɬajis}}; \word{’Wk}{ʔinǻteʔ}{tapeti} (\citealt{VN14}: 48; \citealt{VU74}: 57; \citealt{MG-MELO15}: 20, 22; \citealt{KC16}: 31)

\dicnote{The absence of a word-final glottal stop in \cits{ND09} attestation of this noun must be a mistranscription.}%1

\PMlemma{{\wordnl{*\mbox{-}ó\pla{l}}{penis}}}

\word{Ni}{\mbox{-}oʔ\pl{k}}{glans} \citep[206]{JS16} {\sep} \wordng{PCh}{*\mbox{-}óʔ\pla{l}} > \wordng{Ijw}{\mbox{-}ɔ́ʔ}; \wordng{Mj}{\mbox{-}ɔ́ʔ\pl{l}}\gloss{penis} (\citealt{ND09}: 132; \citealt{JC18}) {\sep} \wordng{PW}{*\mbox{-}ɬ\mbox{-}ó\pla{lʰ}} > \wordng{LB}{\mbox{-}ɬ\mbox{-}u}; \wordng{Vej}{\mbox{-}ɬ\mbox{-}o}; \wordng{’Wk}{\mbox{-}ɬ\mbox{-}óʔ\pl{ɬ}} (\citealt{VN14}: 213; \citealt{VU74}: 66; \citealt{KC16}: 75)

\PMlemma{{\wordnl{*\mbox{-}óʔ\pla{jʰ}}{seed} [1]}}

\wordng{Mk}{\third{ɬ\mbox{-}oʔ\pl{j}}} \citep[255]{AG99} {\sep} \wordng{PCh}{*\mbox{-}óʔ} > \wordng{Ijw}{\mbox{-}ɔ́ʔ} \citep[132]{ND09} {\sep} \wordng{PW}{*\mbox{-}ɬ\mbox{-}óʔ\pla{jʰ}} > \wordng{LB}{\mbox{-}ɬ\mbox{-}uʔ}; \wordng{Vej}{\mbox{-}ɬ\mbox{-}o\mbox{-}j}; \wordng{’Wk}{\mbox{-}ɬ\mbox{-}óʔ\pl{ç}} (\citealt{VN14}: 212; \citealt{VU74}: 66; \citealt{KC16}: 75, 236)

\dicnote{In Maká, Iyojwa’aja’, and in the ’Weenhayek compound \intxt{ɬútsex\mbox{-}ɬ\mbox{-}oʔ\pl{ç}}, this stem also means\gloss{bullet}, which must be a postcolonial semantic extension.}%1

\lit{\citealt{LC-VG-07}: 19}

\PMlemma{{\wordnl{*[t]pǻˀj}{to be bitter}}}

\wordng{Ni}{[t’a]påˀj} \citep[284]{JS16} {\sep} \wordng{PCh}{*pǻhj\mbox{-}iʔ / *\mbox{-}pǻj\mbox{-}} > \wordng{Ijw}{páhj\mbox{-}iʔ},\textsc{caus} \intxt{ʔi\mbox{-}pʲáhj\mbox{-}et\mbox{-}iʔ}; \wordng{I’w}{\mbox{-}páhj\mbox{-}i} [1] (\citealt{ND09}: 109, 143; \citealt{AG83}: 154) {\sep} \wordng{PW}{*[t]páj} [2] > \wordng{LB}{[ta]paj}\gloss{bitter, sour}; \wordng{Vej}{\mbox{-}paj}; \wordng{’Wk}{[t(a)]pájʔ} (\citealt{VN14}: 98; \citealt{JB09}: 56; \citealt{VU74}: 70; \citealt{KC16}: 370)

\dicnote{The absence of a final \intxt{ʔ} in \cits{AG83} data of Iyo’awujwa’ must be a mistranscription.}%1

\dicnote{\sound{PW}{*a} is not a regular reflex of \sound{PM}{*å} (the reconstruction of \intxt{*å} is unequivocally supported by the Nivaĉle reflex and by the Iyojwa’aja’ causative \wordnl{[ʔi]pʲáhj\mbox{-}eti}{makes bitter}, as opposed to \intxt{*[ʔi]pʲéhj\mbox{-}eti}; \citealt{ND09}: 109).}%2

\lit{\citealt{EN84}: 17 (\intxt{*på\mbox{-}åj})}

\PMlemma{{\wordnl{*\mbox{-}pǻˀlåʔ}{bracelet} [1]}}

\wordng{Mk}{(\mbox{-})paˀlaʔ\pl{j}} [2] \citep[293]{AG99} {\sep} \wordng{Ni}{\mbox{-}påˀk͡lå\pl{s}} \citep[221]{JS16} {\sep} \wordng{PCh}{*\mbox{-}pǻˀlåʔ} > \wordng{Ijw}{\mbox{-}páˀlaʔ} [3]; \wordng{I’w}{\mbox{-}páˀlaʔ} (\citealt{ND09}: 124; \citealt{AG83}: 154)

\dicnote{This etymology has been first identified by \citet{LC-subm}. The stem is obviously derived from \word{PM}{*\mbox{-}ˀlåʔ \recind *\mbox{-}ˀlǻʔ}{adornment}.}%1

\dicnote{The presence of a preglottalized sonorant in Maká is inferred based on the Nivaĉle and Iyojwa’aja’ cognates; the form is not attested in our sources that distinguish between plain and preglottalized codas, whereas \citet{AG99} gives simply \textit{palaʔ} (she does not otherwise distinguish between \textit{l} and \textit{ˀl}).}%2

\dicnote{\citet{ND09} actually gives the form \intxt{\mbox{-}páˀla}, which we assume to be a mistranscription.}%3

\lit{\citealt{LC-subm} (\intxt{*\mbox{-}paʔla})}

\PMlemma{{\textit{*pǻnhajeχ \recind *pånhájeχ \recind *pånhajéχ} [1]\gloss{neotropic cormorant}}}

\wordng{Mk}{panhejaχ\plf{panheji\mbox{-}ts}} (\citealt{JB81}: 54; \citealt{AG99}: 294) {\sep} \wordng{PCh}{*pǻnhajah \recind *pånhájah \recind *pånhajáh} [1] > \wordng{Ijw}{pahnaji} [1] (\citealt{ND09}: 124)

\dicnote{The position of the stress in PM and PCh is unknown, since the Iyojwa’aja’ reflex is unattested in our data, and \citet{ND09} does not indicate the position of the stress.}%1


\PMlemma{{\textit{*\mbox{-}pǻˀs \recind *\mbox{-}pǻseˀt} [1]\gloss{lip}}}

\wordng{Mk}{\mbox{-}paˀs} [2], \textit{\mbox{-}p(a)s\mbox{-}its} \citep[294]{AG99} {\sep} \word{Ni}{\mbox{-}påseˀt\plf{\mbox{-}påste\mbox{-}s}}{upper lip} \citep[222]{JS16} {\sep} \word{PCh}{*\mbox{-}pǻsat \recind *\mbox{-}pǻsåt}{lip, beak} > \wordng{Ijw}{\mbox{-}páxsat}, \textit{\mbox{-}pásta\mbox{-}∅}; \wordng{I’w}{\mbox{-}páxsat}, \textit{\mbox{-}páxsat\mbox{-}ej \recind \mbox{-}pásta\mbox{-}j}; \wordng{Mj}{\mbox{-}páxsat} (\citealt{ND09}: 124; \citealt{AG83}: 155; \citealt{JC18}) {\sep} \wordng{PW}{*\mbox{-}pǻset\plf{*\mbox{-}pǻste\mbox{-}jʰ}} > \word{LB}{\mbox{-}poset}{lip, beak}; \wordng{Vej}{\mbox{-}pǻset\plf{\mbox{-}pǻste\mbox{-}j}} [3]; \wordng{’Wk}{\mbox{-}pǻset\plf{\mbox{-}pǻste\mbox{-}ç}} (\citealt{VN14}: 132; \citealt{MG-MELO15}: 61; \citealt{KC16}: 79)

\dicnote{The original root must have been \textit{*\mbox{-}pǻˀs} (preserved only in Maká). \wordng{PM}{*\mbox{-}pǻseˀt} is an opaque derivative reflected in all languages other than Maká.}%1

\dicnote{The preglottalized coda in the Maká reflex is attested in the New Testament in the form \wordnl{ɬa\mbox{-}paˀs}{ship’s bow} (Acts 27:30; Acts 27:41).}%2

\dicnote{\citet[70]{VU74} mistranscribes the Vejoz reflex as \intxt{\mbox{-}paset}.}%3

\lit{\citealt{LC-VG-07}: 19}

\PMlemma{{\wordnl{*\mbox{-}påt \recind *\mbox{-}pǻt}{to shuck}}}

\wordng{Ni}{[t]påt\mbox{-}xan}, \intxt{[n(i)]påt\mbox{-}aʔ} (\citealt{JS16}: 194, 279) {\sep} \word{PCh}{*[ʔi]pǻt}{to shake off} > \wordng{Ijw}{[ʔi]pʲát / \mbox{-}pát}; \wordng{Mj}{[ʔi]p(ʲ)ét / \mbox{-}pát}; \wordnl{*[ʔi]pǻt\mbox{-}ʔeʔ}{to shuck} > \wordng{Ijw}{[ʔi]pʲát\mbox{-}’e / \mbox{-}pát\mbox{-}’e}; \wordng{Mj}{[ʔi]p(ʲ)ét\mbox{-}’eʔ / \mbox{-}pát\mbox{-}’eʔ} (\citealt{ND09}: 109, 110; \citealt{JC18})

\gc{\citet[310]{PVB13a} compares the Mataguayan term to \word{Proto-South Guaicuruan}{*\mbox{-}petá}{grain, seed}. We find the comparison with \word{Proto-Qom}{*[ʔi]pot}{to touch}, with reflexes in Mocoví and Qom, more promising.}

\lit{\citealt{PVB13a}: 310 (\intxt{*\mbox{-}pʌtaʔ})}

\PMlemma{{\wordnl{*pǻtse(ˀ)χ}{fast, quick}}}

\wordng{Ni}{påtsex\plf{påtse\mbox{-}s}} \citep[222]{JS16} {\sep} \wordng{PCh}{*(\mbox{-})pǻsah} > \wordng{Ijw}{pánsa\plf{páns\mbox{-}is}} [1]; \wordng{I’w}{[a]páxsa}; \wordng{Mj}{[ʔa]páxsa} (\citealt{ND09}: 143; \citealt{AG83}: 78, 155; \citealt{JC18})

\dicnote{The nasal consonant in the Iyojwa’aja’ reflex is entirely irregular.}%1

\PMlemma{{\textit{*påttséχ} [1]\gloss{jabiru}}}

\wordng{Ni}{påtséx\pl{is}} (\citealt{JS16}: 222–223) {\sep} \wordng{PCh}{*påtsáh} [1] > \wordng{Ijw}{pi(t)sáh \recind pasáh} [1]; \wordng{I’w}{pisáh\pl{as}}; \wordng{Mj}{pisáh\plf{pisá\mbox{-}as}} (\citealt{JC14b}: 99; \citealt{ND09}: 143, 144; \citealt{AG83}: 155; \citealt{JC18}) {\sep} \wordng{PW}{*påtsáχ} > \wordng{LB}{putsaχ} [2]; \wordng{’Wk}{påtsáx} (\citealt{VN14}: 41, 47; \citealt{CS-FL-PR-VN13} [2011]; \citealt{KC16}: 286)

\dicnote{The cluster \wordng{PM}{*tts} > \wordng{PCh}{*ts} is reconstructed based on the Iyojwa’aja’ subdialectal variant \intxt{pitsáh}. Note that Chorote has no affricate /ts/, suggesting that we are dealing here with a cluster composed of /t/ and /s/.}%1

\dicnote{The vowel of the first syllable is reflected irregularly in Lower Bermejeño Wichí as \textit{u}, a development also seen in \word{LB}{pulaχ}{brown cachalote}.}%2

\lit{\citealt{EN84}: 28, 49 (\intxt{*pajtsha}); \citealt{PVB02}: 143 (\intxt{*pʌjtsax})}

\PMlemma{{\wordnl{*pätóχ}{to be deep}}}

\wordng{Ni}{[ʔa]patox} \citep[46]{JS16} {\sep} \wordng{PCh}{*\mbox{-}pítohw<ijʔ>} > \wordng{I’w}{\mbox{-}pétʲofʷiʔ}; \wordng{Mj}{\mbox{-}péitihwijʔ} (\citealt{AG83}: 155; \citealt{JC18}) {\sep} \wordng{PW}{*pitóxʷ} > \wordng{LB}{pitufʷ} [1]; \wordng{Vej}{pitoh} [1]; \wordng{’Wk}{pitóxʷ} (\citealt{VN14}: 335; \citealt{VU74}: 70; \citealt{KC16}: 293)

\dicnote{The final consonant is documented as a non-labialized \textit{χ} in Lower Bermejeño \citep[54]{JB09} and Vejoz (\citealt{VU74}: 70), possibly as a result of mistranscription.}%1

\lit{\citealt{EN84}: 19 (\intxt{*pajtho})}

\PMlemma{{\textit{*\mbox{-}pe(ʔ)}, \textit{*\mbox{-}pé\mbox{-}l}\gloss{fat, oil}}}

\wordng{Ni}{\mbox{-}<a>peʔ\pl{k}}\gloss{oil} \citep[164]{JS16} {\sep} \wordng{PCh}{*\mbox{-}péʔ\pla{l}} > \wordng{Ijw}{\mbox{-}pέʔ}; \wordng{I’w}{\mbox{-}péʔ}; \wordng{Mj}{\mbox{-}<i>pέʔ\pl{l}}\gloss{fat, oil} (\citealt{ND09}: 124; \citealt{AG83}: 155; \citealt{JC18}) {\sep} \wordng{PW}{*\mbox{-}pe(ʔ)} > \wordng{LB}{\mbox{-}pe(ʔ)}; \wordng{Vej}{<a>pe}; \wordng{’Wk}{\mbox{-}peʔ} (\citealt{JB09}: 54; \citealt{VU74}: 51; \citealt{KC16}: 219)

\gc{Possibly related to \word{Proto-Guaicuruan}{*\mbox{-}apijó}{fat} (\citealt{PVB13b}, \#60; cf. \citealt{PVB13a}: 308).}

\empr{\citet[307]{AF16} compares the Nivaĉle reflex to \word{Enlhet, Enenlhet-Toba, Angaité, Enxet, Guaná}{peɬmok}{fat} (\citealt{EU-HK-97}: 550; \citealt{EU-HK-MR-03}: 335; \citealt{PW20}: 46; \citealt{JE21}: 193; \citealt{HK-23}: 51), but this is likely an accidental similarity.}

\lit{\citealt{PVB13a}: 308 (\intxt{*\mbox{-}apeʔ})}

\PMlemma{{\textit{*[ji]péˀj\mbox{-}aʔ} (antipassive: \textit{*[t]péˀj\mbox{-}käj})\gloss{to hear, to understand}}}

\wordng{Mk}{[j]<e>piˀj<eʔ>} [1] \citep[154]{AG99} {\sep} \wordng{Ni}{[ji]peˀj\mbox{-}a} (\intxt{[t]peˀj\mbox{-}tʃaj}) (\citealt{JS16}: 278, 349) {\sep} \wordng{PCh}{*[ʔi]péˀj\mbox{-}aʔ} (\intxt{*[tᵊ]péj\mbox{-}kejʔ}) > \wordng{Ijw}{[ʔi]píˀj\mbox{-}aʔ / \mbox{-}pέˀj\mbox{-}aʔ} [2] (\intxt{[ti]pέ\mbox{-}tʃiʔ}); \wordng{I’w}{\mbox{-}péˀj\mbox{-}eʔ \recind \mbox{-}péj\mbox{-}iʔ} (\intxt{\mbox{-}péj\mbox{-}siʔ}); \wordng{Mj}{[ʔi]píˀj\mbox{-}aʔ / \mbox{-}pέˀj\mbox{-}aʔ} (\intxt{[ti]pέj\mbox{-}ʃi(j)ʔ}) (\citealt{ND09}: 110; \citealt{AG83}: 155, 197; \citealt{JC18})

\dicnote{The glottalized palatal approximant in the Maká reflex is attested in the New Testament (e.g. John 3:32).}%1

\dicnote{Mistranscribed as \intxt{[ʔi]píˀj\mbox{-}a / \mbox{-}pέˀj\mbox{-}a} in \citet[110]{ND09}.}%2

\PMlemma{{\intxt{*péɬa(ˀ)j\plf{*peɬaj\mbox{-}its}} [1]\gloss{rain}}}

\wordng{Mk}{piɬej\pl{its}} \citep[297]{AG99} {\sep} \wordng{PCh}{*péhlajʔ} > \word{Ijw}{pέhlaʔ}{rain season}, \wordnl{pέhla}{rainstorm, rain}; \wordng{I’w}{péhlaj<i>\pl{s}}; \wordng{Mj}{pέhlijʔ} (\citealt{ND09}: 143; \citealt{AG83}: 155; \citealt{JC18}) {\sep} \wordng{PW}{*péɬajʰ\pla{is}} [1] > \word{LB}{peɬaj\pl{is}}{rainstorm}; \word{Vej}{peɬaj\plf{peɬaɲ̊\mbox{-}is}}{rainstorm, rain}; \wordng{’Wk}{péɬaç\pl{is \recind péɬaj\mbox{-}is}} (\citealt{VN14}: 161, 343; \citealt{MG-MELO15}: 44; \citealt{KC16}: 292)

\dicnote{\sound{PW}{*\mbox{-}ajʰ}, reconstructed based on the Vejoz and ’Weenhayek reflexes, does not correspond to \sound{PCh}{*\mbox{-}ajʔ} (underlying: */\mbox{-}aj/). The root must have been remodeled based on the plural suffix \intxt{*\mbox{-}jʰ}.}%1

\PMlemma{{\intxt{*\mbox{-}phaˀɬ} [1]\gloss{to wrap, to bind, to tie}}}

\word{Mk}{[ji]<xu>pheˀɬ}{to wrap}, \wordnl{[j]<o>pheˀɬ / \mbox{-}<ʔo>pheˀɬ}{to tie} [2] \citep[283, 394]{AG99} {\sep} \word{Ni}{[ji]<k͡lå>pxaɬ}{to wrap up, to roll up}, \wordnl{[j]ako\mbox{-}pxaɬ}{to embrace with one’s legs around}, \wordnl{[ji]<ta>pxaɬ}{to hobble legs, to bind hands}, \wordnl{[ji]<tse>pxaɬ}{to sew}, \wordnl{[j]<etʃe>pxaɬ}{to hug} (\citealt[36, 120, 122, 257, 293]{JS16}; \citealt[320]{LC20}) {\sep} \wordng{PCh}{*[ˀja]<qa>paɬ\mbox{-}\APPL} > \word{Ijw}{[ˀja]qapahl\mbox{-}a\mbox{-}ˀni}{to wrap}, \wordnl{[ˀja]qapahl\mbox{-}at\mbox{-}k’iʔ}{to wrap, to fold}, \wordnl{[ˀja]qapahl\mbox{-}e}{to gather}~[3]

\dicnote{This morpheme can be alternatively described as a verbal root that requires an incorporated object or as a suffix with a highly lexical meaning. \citet[320]{LC20} identify its reflex as a suffix that ``appears to involve, loosely, a sense of `binding'.}%1

\dicnote{The morpheme-final consonant in Maká is attested as preglottalized in the New Testament (Acts 1:16; Acts 5:6; Acts 21:33; Acts 25:14; Matthew 14:3; Matthew 18:30; Matthew 23:4; Matthew 27:2; John 18:12; Luke 3:20; 2 Corinthians 3:17).}%1

\dicnote{We are unsure which syllable in the Iyojwa’aja’ reflex is stressed. We cannot exclude at present that \wordnl{pahlát}{all} is related; the semantic link would be\gloss{to bind} > \enquote{to gather} > \enquote{together} > \enquote{all}.}%3

\PMlemma{{\wordnl{*phåˀm}{up}}}

\wordng{Mk}{\mbox{-}phaˀm} (\citealt{AG94}: 118; \citealt{PMA}: 7) {\sep} \wordng{PCh}{*pᵊhǻˀm} > \wordng{Ijw}{pihjáˀm}; \word{I’w}{\mbox{-}én\mbox{-}<i>fʷóm}{to hang}; \wordng{Mj}{<ʔa>húˀm / \mbox{-}<ʔá>huˀm / húˀm} [1] (\citealt{ND09}: 144; \citealt{AG83}: 127; \citealt{JC18}; own field notes) {\sep} \wordng{PW}{*\mbox{-}pʰå} [2] > \wordng{LB}{\mbox{-}pʰo}; \wordng{Vej}{\mbox{-}pʰå}; \wordng{’Wk}{\mbox{-}pʰåʔ}; \wordnl{*pʰåm\mbox{-}ɬéle\pla{jʰ}}{the one from upriver} > \wordng{LB}{pʰom\mbox{-}ɬele\mbox{-}j}; \wordng{’Wk}{pʰåm\mbox{-}ɬéleʔ\pl{ç}} (\citealt{VN14}: 27, 149; \citealt{MG-MELO15}: 34; \citealt{KC16}: 302)

\dicnote{The Iyo’awujwa’ and Manjui reflexes are entirely irregular.}%1

\dicnote{The loss of \intxt{*m} in the Wichí directional suffix is irregular. It resurfaces in the derivative for\gloss{the one from upriver}.}%2

\PMlemma{{\intxt{*[t]píl} [1]\gloss{to return hither}}}

\word{Mk}{[t(e)]pil}{to return from a specified place} \citep[296]{AG99} {\sep} \wordng{Ni~ChL}{[t(a)]pek} [1], ShL \textit{[t(a)]pik} (\citealt{NS87}: 498; \citealt{JS16}: 178) {\sep} \wordng{PW}{*[t]pílʰ} > \word{LB}{[t(a)]piɬ}{to return to one’s destination}; \wordng{Vej}{\mbox{-}pil \recind \mbox{-}piɬ}; \wordng{’Wk}{[t(a)]píɬ / [t(a)]píl\mbox{-}{\APPL} / [t(a)]pín̥\mbox{-}\APPL} (\citealt{VN14}: 289, 308; \citealt{VU74}: 70; \citealt{MG-MELO15}: 39; \citealt{KC16}: 371)

\dicnote{The Chishamnee Lhavos Nivaĉle form with \intxt{e} is irregular. Shichaam Lhavos preserves the etymological vowel \intxt{i}.}%1

\dicnote{\word{PM}{*[w]ǻpil}{to return thither} is an obvious derivative of this root.}%2

\gc{Obviously related to \word{Proto-Guaicuruan}{*\mbox{-}op’il}{to return} (\citealt{PVB13b}, \#443).}

\lit{\citealt{LC-VG-07}: 22; \citealt{AnG15}: 253}

\PMlemma{{\wordnl{*pínuʔ}{kind of honey} [1]}}

\word{Mk}{pinuʔ\pl{l}}{small black bee, stings lightly, makes its nest inside tree trunks, produces small amounts of edible honey}; \wordnl{ɬe\mbox{-}qe\mbox{-}pinuʔ\pl{l}}{sugar, sugarcane} (\citealt{AG99}: 250, 297) {\sep} \wordng{PW}{*pínu} > \word{LB}{pini}{\textit{llana} bee, honey} [2]; \wordng{Vej}{pinu} [2]\gloss{sugarcane}, \wordnl{pinu ˀwet\mbox{-}es}{apiary; sugar mill}; \wordng{’Wk}{pínuʔ} (\citealt{VN14}: 41, 178; \citealt{VU74}: 70; \citealt{MG-MELO15}: 52; \citealt{KC16}: 292)

\rej{\word{Iyojwa’aja’}{piníʔ\pl{ˀl}}{kind of insect} (metaphorically also\gloss{spirit}, since the Chorote believe that the \textit{piníʔ} gets inside humans and possesses them) does not regularly correspond to the reflexes of \wordng{PM}{*pínuʔ}. From a phonological point of view, it could be a loan from Southeastern Wichí, but this possibility is unlikely for geographic reasons, and the semantic discrepancy does not speak in favor of the loan etymology either.}

\dicnote{Both in Maká and Wichí, reflexes of \wordng{PM}{*pínuʔ} or their derivatives are used to designate a kind of bee (or its honey) and sugarcane. Since sugarcane is not native to the Americas and therefore cannot have been known to the speakers of Proto-Mataguayan, we assume that Maká and Wichí have extended the name of a type of honey to sugar.}%1

\dicnote{\sound{Lower Bermejeño}{i} is not the regular reflex of \sound{PW}{*e}; \intxt{*pine} would be expected.}%2

\dicnote{\citet[70]{VU74} mistranscribes the Vejoz reflex as \intxt{pinnu}.}%3

\lit{\citealt{RJH15}: 239}

\PMlemma{{\wordnl{*pí(t)staʔ}{masked gnatcatcher}}}

\wordng{Ni}{pistaʔ\pl{k}} [1] \citep[219]{JS16} {\sep} \wordng{PCh}{*pístV\mbox{-}keʔ} [2 3] > \wordng{Ijw}{péstʲo\mbox{-}kiʔ} [3 4] \citep[143]{ND09} {\sep} \wordng{PW}{*písta} > \wordng{LB}{pista}; \wordng{’Wk}{pístaʔ} (\citealt{CS-FL-PR-VN13}; \citealt{KC16}: 293)

\dicnote{The Nivaĉle reflex is irregular in that deglottalization failed to apply to the stem-final \intxt{ʔ}.}%1

\dicnote{The Chorote form seems to contain a feminine suffix.}%2

\dicnote{The vowel \intxt{o} in Iyojwa’aja’ is not the regular reflex of \sound{PM}{*a}. It is unknown whether the irregular change occurred in the individual history of Iyojwa’aja’ or before the desintegration of Proto-Chorote.}%3

\dicnote{\citet{ND09} transcribes this as \textit{péstʲoki}; we assume that this is a mistranscription for \textit{péstʲokiʔ}.}%4

\PMlemma{{\wordnl{*pitéχ\plf{*pité\mbox{-}ts}}{long}}}

\wordng{Ni}{pitex\plf{pite\mbox{-}s}} \citep[219]{JS16} {\sep} \wordng{PW}{*pitáχ\plf{*pité\mbox{-}s}} > \wordng{LB}{pitaχ}; \wordng{’Wk}{pitáx\plf{pité\mbox{-}s}} (\citealt{VN14}: 312; \citealt{VU74}: 70; \citealt{KC16}: 293)

\PMlemma{{\intxt{*[t]póʔ}, \wordnl{*[t]póʔ\mbox{-}ex}{to be full}}}

\wordng{Mk}{[to]poʔ\mbox{-}ox}, \textsc{pl}~\textit{[to]po\mbox{-}l\mbox{-}ix} \citep[284]{AG99} {\sep} \wordng{Ni}{[ta]poˀ\mbox{-}x}, \textit{[ta]poʔ\mbox{-}in}; \wordnl{[ji]ka\mbox{-}po}{to have one’s container full} \citep[257]{JS16} {\sep} \wordng{PCh}{*[tᵊ]póʔ}, \textit{*[tᵊ]pó\mbox{-}eh} > \wordng{Ijw}{[ti]pɔ́\mbox{-}ji}; \wordng{Mj}{[ta]pɔ́ʔ}, \textit{[ta]pɔ́w\mbox{-}e} (\citealt{ND09}: 151; \citealt{JC18}) {\sep} \wordng{PW}{*[t]ˈpó\mbox{-}jeχ} > \wordng{LB}{[ta]ˈpu\mbox{-}jeχ}; \wordng{Vej}{\mbox{-}po\mbox{-}jeh}; \wordng{’Wk}{[t(a)]ˈpó\mbox{-}jex}, \textsc{pl}~\textit{[t(a)]ˈpó\mbox{-}keʔ} (\citealt{JB09}: 56; \citealt{VN14}: 56; \citealt{VU74}: 70; \citealt{KC16}: 372)

\PMlemma{{\wordnl{*[ji]pónit\mbox{-}ex}{to fill} [1]}}

\wordng{Mk}{[j]<o>pon\mbox{-}het\mbox{-}ix} [2] (\citealt{AG99}: 283–284) {\sep} \wordng{Ni}{[ji]pont\mbox{-}eʃ} [3] \citep[103]{JS16} {\sep} \wordng{PCh}{*[ʔi]pónit\mbox{-}eh} > \wordng{Ijw}{[ʔi]pʲónit\mbox{-}i / \mbox{-}pɔ́nit\mbox{-}i}; \wordng{I’w}{\mbox{-}ta\mbox{-}pónit\mbox{-}i} [4]; \wordng{Mj}{[ʔi]t(ʲ)e\mbox{-}pɔ́nit(ʲ)\mbox{-}e / \mbox{-}ta\mbox{-}pɔ́nitʲ\mbox{-}e} [4 5] (\citealt{JC14b}: 77; \citealt{ND09}: 110; \citealt{AG83}: 163; \citealt{JC18}) {\sep} \word{PW}{*[ʔi]tá\mbox{-}ponit\mbox{-}eχ}{to fill with} [4] > \wordng{’Wk}{[ʔi]tá\mbox{-}ponit\mbox{-}ex} \citep[372]{KC16}

\dicnote{This verb is obviously related to \word{PM}{*[t]póʔ}{to be full}, but \textit{*\mbox{-}nit\mbox{-}} is not known to have been a productive causative suffix in PM.}%1

\dicnote{We have no explanation for the element \intxt{\mbox{-}o\mbox{-}} in Maká. The causative suffix \intxt{\mbox{-}het\mbox{-}} has replaced the etymological sequence \textit{*\mbox{-}it\mbox{-}}, which must have functioned as a part of the root in PM, due to a morphological change.}%2

\dicnote{The loss of the stem-medial vowel \intxt{*i} in Nivaĉle is irregular.}%3

\dicnote{Iyo’awujwa’, Manjui, and Wichí have innovated in inserting the reflex of the prefix \intxt{*t\mbox{-}} by analogy with \wordnl{*[t]póʔ}{to be full}.}%4

\dicnote{The applicative suffix is unexpectedly reflected as \intxt{\mbox{-}e} and not \textit{*\mbox{-}it} in Manjui.}%5

\PMlemma{{\wordnl{*pútäh}{tapeti}}}

\wordng{Ni}{puta\pl{k}} \citep[223]{JS16} {\sep} \wordng{PCh}{*púteh} > \wordng{I’w}{póʔtih}, \textit{pótih\mbox{-}is} [1]; \wordng{Mj}{pʊ́ti\pl{is}} (\citealt{AG83}: 156; \citealt{JC18})

\dicnote{\textit{ʔ} in \cits{AG83} attestation of the Iyo’awujwa’ reflex must be a mistranscription.}%1

\PMlemma{{\wordnl{*\mbox{-}pxúseʔ\pla{jʰ}}{beard}; \wordnl{*pxúse\mbox{-}naˀχ}{bearded; gilded catfish}}}

\word{Mk}{(\mbox{-})<a>pxusiʔ\pl{j}}{beard, moustache} \citep[124]{AG99} {\sep} \wordng{Ni}{\mbox{-}påse\pl{j}} [1]; \intxt{påse<nxa>\pl{j}} [1 2]\gloss{gilded catfish} (\citealt{JS16}: 222, 350) {\sep} \wordng{PCh}{*\mbox{-}púseʔ\pla{jʰ}} > \wordng{Ijw}{\mbox{-}póxsiʔ\pl{ˀl}}; \wordng{I’w}{\mbox{-}póxsiʔ\plf{\mbox{-}póxse\mbox{-}j}}; \wordng{Mj}{\mbox{-}pʊ́xseʔ\pl{j}}; \wordnl{*púse<nah>\plf{*púse<hna>\mbox{-}s}}{bearded} > \wordng{Mj}{pʊ́xsena}, \textit{pʊ́xsehna\mbox{-}s} (\citealt{JC14b}: 76; \citealt{ND09}: 125; \citealt{AG83}: 156; \citealt{JC18}) {\sep} \wordng{PW}{*\mbox{-}pǻse\pl{jʰ}} [1] > \wordng{LB}{\mbox{-}pose}; \word{Vej}{\mbox{-}påse\pl{j}}{moustache}; \wordng{’Wk}{\mbox{-}pǻse\mbox{-}ç}; \wordnl{*pǻsenaχ\plf{*pǻsenha\mbox{-}s}}{gilded catfish} [1] > \wordng{Vej}{\mbox{påsenah}}; \wordng{’Wk}{pǻsenax\plf{pǻsen̥a\mbox{-}s}} (\citealt{VN14}: 148; \citealt{MG-MELO15}: 22, 61; \citealt{KC16}: 79, 286)

\dicnote{The Nivaĉle and Wichí forms are entirely irregular: one would expect \wordng{Ni}{*\mbox{-}pxuse}, \wordng{PW}{**\mbox{-}phúse}. The stem has obviously suffered contamination with \word{PM}{*\mbox{-}pǻs}{lower lip} in these languages. Wichí also has a similar root, \word{PW}{*\mbox{-}púse(\mbox{-})jʰ}{bodily hair} > \wordng{LB}{\mbox{-}pesej}; \wordng{’Wk}{\mbox{-}púseç} (\citealt{VN14}: 406; \citealt{KC16}: 296), which could be related or unrelated to the PM etymon.}%1

\dicnote{The Nivaĉle reflex could be a back-formation from the plural form (\wordng{PM}{*pǻsenha\mbox{-}ts} or \intxt{*pǻsenha\mbox{-}jʰ}).}%2

\PMlemma{{\textit{*[ji]p’o(ʔ) \recind *[ji]p’ó(ʔ)} [1]\gloss{to cover}}}

\wordng{Ni}{[ji]p’o} \citep[103]{JS16} {\sep} \wordng{PCh}{*[ʔi]p’ó\mbox{-}\APPL} > \wordng{Ijw}{[ʔi]pʲ’ó<n>\mbox{-}e / \mbox{-}p’ɔ́<n>\mbox{-}e}; \wordng{I’w}{\mbox{-}pó\mbox{-}\APPL} [2]; \wordng{Mj}{[ʔi]p(ʲ)’ó\mbox{-}{\APPL} / \mbox{-}p’ɔ́\mbox{-}\APPL} (\citealt{JC14b}: 77; \citealt{ND09}: 110; \citealt{AG83}: 156; \citealt{JC18}) {\sep} \wordng{PW}{*[hi]p’ó\mbox{-}\APPL} > \wordng{LB}{[hi]p’u\mbox{-}\APPL}; \wordng{Vej}{\mbox{-}p’o(ʔ)\mbox{-}pe}; \wordng{’Wk}{[hi]p’ó\mbox{-}\APPL} (\citealt{VN14}: 117; \citealt{VU74}: 71; \citealt{MG-MELO15}: 39; \citealt{KC16}: 300)

\dicnote{We reconstruct \intxt{*p’} rather than \intxt{*ɸ’}, because the root is obviously related to \word{PM}{*\mbox{-}p’oˀt}{lid}.}%1

\dicnote{The absence of glottalization in \cits{AG83} attestation of the Iyo’awujwa’ reflex must be a mistranscription.}%2

\gc{Likely related to \word{Proto-Guaicuruan}{*\mbox{-}ap’o}{to cover, to wear} (\citealt{PVB13b}, \#89; cf. \citealt{PVB13a}: 305).}

\lit{\citealt{EN84}: 33 (\wordnl{*p’ɔhni}{to lock up}); \citealt{PVB13a}: 305 (\wordnl{*\mbox{-}p’o(\mbox{-}hi)}{to close}})

\PMlemma{{\wordnl{*\mbox{-}p’oˀk \recind *\mbox{-}ɸ’oˀk}{fence}}}

\wordng{Ni}{\mbox{-}p’oˀk\plf{\mbox{-}pok͡l\mbox{-}is}} [1]\gloss{beehive marked as one’s own by its discoverer} (\citealt{NS87}: 125; \citealt{JS16}: 225, 351) {\sep} \wordng{PCh}{*\mbox{-}p’ók} > \word{Ijw}{\mbox{-}p’ɔ́k}{fence for fishing} \citep[125]{ND09} {\sep} \word{PW}{*\mbox{-}p’okʷ}{fence, earthenware field bottle (\textit{caramayola})} [2] > \word{Vej}{\mbox{-}p’okʷ}{earthenware field bottle}; \wordng{’Wk}{\mbox{-}p’ok\plf{\mbox{-}p’óho\mbox{-}ç}} (\citealt{VU74}: 71; \citealt{KC16}: 80; \citealt{JAA12a}: 71–72)

\dicnote{The Nivaĉle plural form must be non-etymological.}%1

\dicnote{The semantic relation between\gloss{fence} and\gloss{earthenware field bottle} is attributed to the circular shape of the bottle by \citep[71–72]{JAA12a}.}%2

\rej{\citet[38]{EN84} compares the Wichí term for\gloss{earthenware field bottle} with \word{Nivaĉle}{(\mbox{-})p’ok}{arrow} and reconstructs \wordng{PM}{*p’ɔwk’}. This is implausible for semantic reasons.}

\PMlemma{{\wordnl{*(\mbox{-})p’oˀt\plf{*(\mbox{-})p’ot\mbox{-}ots \recvar *\mbox{-}p’ot\mbox{-}ets}}{lid}}}

\word{Mk}{p’ot<oʔ>\pl{l}}{recipient with a lid for storing objects} \citep[299]{AG99} {\sep} \wordng{Ni}{\mbox{-}p’oˀt\plf{\mbox{-}p’ot\mbox{-}os}} \citep[225]{JS16} {\sep} \wordng{PCh}{*\mbox{-}p’ót\plf{*\mbox{-}p’ot\mbox{-}és}} > \wordng{Ijw}{\mbox{-}p’ɔ́t\pl{is}}; \wordng{I’w}{\mbox{-}pót\pl{es}} [2]; \wordng{Mj}{(\mbox{-})p’ɔ́t\plf{(\mbox{-})p’at\mbox{-}έs}} [3] (\citealt{ND09}: 125; \citealt{AG83}: 156; \citealt{JC18}) {\sep} \wordng{PW}{*\mbox{-}p’ot\plf{*\mbox{-}p’ót\mbox{-}es}} > \wordng{’Wk}{\mbox{-}p’ot\plf{\mbox{-}p’ót\mbox{-}es}} \citep[85]{KC16}

\dicnote{The noun is obviously derived from \word{PM}{*[ji]p’o(ʔ) \recind *[ji]p’ó(ʔ)}{to cover}.}%1

\dicnote{The absence of glottalization in the initial consonant in the Iyo’awujwa’ reflex must be a mistranscription on \cits{AG83} part. The stress in the plural form appears to be an innnovation.}%2

\dicnote{The unrounding and lowering of \intxt{*o} in the Manjui plural form is irregular.}%3

\lit{\citealt{PVB13a}: 304 (\intxt{*\mbox{-}(a)p’o\mbox{-}t})}

\PMlemma{{\wordnl{*qa}{in order to (\textit{irrealis subordinator})}}}

\word{Mk}{qe}{in order to, because} (\citealt{AG94}: 210; \citealt{AG99}: 305) {\sep} \wordng{Ni}{ka} (\citealt{AF16}: 275; \citealt{JS16}: 53) {\sep} \wordng{PCh}{*qa} > \wordng{Ijw/I’w/Mj}{ka} (\citealt{ND09}: 133; \citealt{AG83}: 81; \citealt{JC18})

\PMlemma{{\wordnl{*[ji]qákuʔ}{to distrust}}}

\wordng{Mk}{[je]qekuʔ} \citep[155]{AG99} {\sep} \wordng{Ni}{[ji]kaku} \citep[55]{JS16} {\sep} \wordng{PCh}{*[ʔi]qákuʔ} > \wordng{Ijw}{[ʔi]kákʲuʔ} [1]; \wordng{Mj}{[ʔi]kʲákʲuʔ / \mbox{-}kákʲuʔ} (\citealt{ND09}: 100; \citealt{JC18}) {\sep} \wordng{PW}{*[ji]qákʲu\mbox{-}\APPL} > \wordng{’Wk}{[ja]qákʲu\mbox{-}\APPL} \citep[306]{KC16}

\dicnote{The Iyojwa’aja’ reflex is mistranscribed as \intxt{[ʔi]kákʲu} in \citet[100]{ND09}.}%1

\PMlemma{{\textit{*\mbox{-}qák\mbox{-}xiʔ \recind *\mbox{-}qak\mbox{-}xíʔ \recvar *\mbox{-}qák\mbox{-}xijʰ \recind *\mbox{-}qak\mbox{-}xíjʰ} [1]\gloss{lap; calf}}}

\word{Mk}{\mbox{-}qek\mbox{-}xiʔ}{calf} \citep[305]{AG99} {\sep} \wordng{PW}{*\mbox{-}qák\mbox{-}hih} [2] > \word{’Wk}{\mbox{-}qák\mbox{-}ʰih}{lap} \citep[84]{KC16}

\dicnote{Maká points to a compound with \wordnl{*\mbox{-}xiʔ}{inside a recipient}, and Wichí to a compound with \wordnl{*\mbox{-}xíjʰ}{recipient}.}%1

\dicnote{The PW reflex \intxt{*kh} of \sound{PM}{*kX} may be regular, as Wichí does not otherwise have \intxt{*kʲh}.}%2

\PMlemma{{\wordnl{*\mbox{-}qalǻʔ\pla{jʰ}}{leg} [1]}}

\wordng{Ni}{\mbox{-}kak͡låʔ\pl{j}} \citep[56]{JS16} {\sep} \wordng{PCh}{*\mbox{-}qaˀlǻʔ \recind *\mbox{-}qåˀlǻʔ\pla{jʰ}} [2] > \wordng{I’w}{\mbox{-}kaláʔ\pl{j}}\gloss{foot}; \wordng{Mj}{\mbox{-}kaˀláʔ\pl{jh}} (\citealt{AG83}: 136; \citealt{JC18}) {\sep} \wordng{PW}{*\mbox{-}qǻlå\pla{jʰ}} [3] > \wordng{LB}{\mbox{-}(t\mbox{-})qolo}; \wordng{Vej}{\mbox{-}kåla} [4]; \wordng{’Wk}{\mbox{-}qǻlåʔ}, \third{ta\mbox{-}qǻlåʔ\pl{ç}} (\citealt{VN14}: 55, fn. 17, 164–165; \citealt{VU74}: 61; \citealt{KC16}: 82)

\dicnote{The body part denoted by this term canonically encompasses one’s shank and foot.}%1

\dicnote{The glottalization in \sound{PCh}{*ˀl} appears to be irregular (the seemingly plain reflex in Iyo’awujwa’ could be a mistranscription on Gerzenstein’s part). It is impossible to determine whether the PCh form contained an \intxt{*a} or an \intxt{*å}, because this opposition is neutralized following a \intxt{*q} (even in Iyojwa’aja’, though a cognate in that variety is lacking anyway). One possible explanation for the occurrence of \sound{PCh}{*ˀl} is contamination with \word{PCh}{*ʔaˀlǻʔ}{tree}, as if it were a derivation thereof containing the alienizer \intxt{*\mbox{-}qá\mbox{-}} (compare \word{Maká}{naxak}{stick, (fire)wood} and \wordnl{\mbox{-}qa\mbox{-}naxak}{leg}; \citealt{AG94}: 266, 302).}%2

\dicnote{The loss of \wordng{PM}{*ʔ} in Wichí is not known to be regular.}%3

\dicnote{The final vowel \intxt{a} in the Vejoz form as documented by \citet{VU74} must be a mistranscription.}%4

\gc{Possibly related to \word{Proto-Guaicuruan}{*qoˀná}{leg (lower part)} (\citealt{PVB13b}, \#530).}

\lit{\citealt{EN84}: 12, 18 (\intxt{*qala}, \textsc{pl}~\intxt{*qala\mbox{-}\'{j}}); \citealt{LC-VG-07}: 15; \citealt{AnG15}: 253}

\PMlemma{{\wordnl{*qatiˀts\plf{*qatits\mbox{-}él}}{star}}}

\wordng{Ni}{katiˀs} \citep[112]{JS16} {\sep} \wordng{PCh}{*qatés\plf{*qates\mbox{-}él}} > \wordng{Ijw}{katέs\pl{eˀl}}; \wordng{I’w}{katés\pl{éj}} [1]; \wordng{Mj}{katέs\plf{katas\mbox{-}έjh \recind katis\mbox{-}έjh}} [1] (\citealt{JC14b}: 77; \citealt{ND09}: 134; \citealt{AG83}: 137; \citealt{JC18}; \citealt{GH94}) {\sep} \wordng{PW}{*qates\plf{*qatéts\mbox{-}elʰ}} > \wordng{LB}{qates\plf{qatets\mbox{-}eɬ}}; \wordng{Vej}{kates\plf{katets\mbox{-}eɬ \recind katets\mbox{-}el}}; \wordng{’Wk}{qates\plf{qatéts\mbox{-}eɬ}} (\citealt{VN14}: 191; \citealt{VU74}: 61; \citealt{MG-MELO15}: 43; \citealt{AFG067}: 214; \citealt{KC16}: 316)

\dicnote{Iyo’awujwa’ and Manjui use a non-etymological plural suffix, having replaced \intxt{*\mbox{-}él} with \intxt{*\mbox{-}éjʰ}.}%1

\gc{Possibly related to \word{Proto-Guaicuruan}{*aqat’í}{star} (\citealt{PVB13b}, \#99; cf. \citealt{PVB13a}: 311).}

\lit{\citealt{EN84}: 18 (\intxt{*qatέts}); \citealt{LC-VG-07}: 16; \citealt{PVB13a}: 311 (\intxt{*qate\mbox{-}ts})}

\PMlemma{{\wordnl{*[t]qǻnhan}{to fish with a hook}}}

\wordng{Mk}{[ta]<qa>qanhen} \citep[302]{AG99} {\sep} \wordng{PCh}{*[tᵊ]qǻhnan} > \wordng{Ijw}{[ta]káhnaˀn}; \wordnl{*\mbox{-}qǻhna\mbox{-}t}{fishhook} > \wordng{Ijw}{\mbox{-}káhnat\pl{is}}; \wordng{I’w}{\mbox{-}káhnat\pl{es}} (\citealt{ND09}: 120, 148; \citealt{AG83}: 138) {\sep} \wordng{PW}{*[t]qǻnhan} > \wordng{’Wk}{[t(a)]qǻn̥an̥} \citep[373]{KC16}

\gc{Possibly cognate with \word{Proto-Qom}{*[do]qojna\mbox{-}ʁan}{to fish with a hook, to trap}, itself a derivative of \wordnl{\mbox{-}qojna}{trap}.}

\PMlemma{{\intxt{*\mbox{-}q(Á)xɬek\plf{*\mbox{-}q(Á)xɬe\mbox{-}jʰ}} [1]\gloss{liver}}}

\wordng{Mk}{\mbox{-}<ʔa>qɬik\plf{\mbox{-}<ʔa>qɬi\mbox{-}j}} [2] (\citealt{AG99}: 127; \citealt{JB81}: 202) {\sep} \wordng{Ni}{\mbox{-}(<ʔa>)kåxɬåk\pl{is}} [1 2 3] \citep[36]{JS16} {\sep} \wordng{PCh}{*\mbox{-}qÁhlek\plf{*\mbox{-}qÁhle\mbox{-}jʰ}} > \wordng{Ijw}{\mbox{-}káhlik\plf{káhle\mbox{-}ʔ}}; \wordng{I’w}{\mbox{-}káhlik\plf{\mbox{-}káhle\mbox{-}j}}; \wordng{Mj}{(\mbox{-})káhlek\plf{káhle\mbox{-}j}} (\citealt{JC14a}; \citealt{ND09}: 120; \citealt{AG83}: 138; \citealt{JC18}) {\sep} \wordng{PW}{*\mbox{-}ˈqáɬeq} > \word{’Wk}{\mbox{-}ˈqáɬek}{stomach} \citep[86]{KC16}

\dicnote{Maká points to \wordng{PM}{*\mbox{-}…q(x)ɬek}; Nivaĉle to \intxt{*\mbox{-}…qǻxɬåk}; Chorote to \intxt{*\mbox{-}qǻxɬek} or \intxt{*\mbox{-}qáxɬek}; Wichí to \intxt{*\mbox{-}qáɬek}.}%1

\dicnote{We have no explanation for the elements \wordng{Mk/Ni}{\mbox{-}ʔa\mbox{-}}. The stem-initial glottal stop is attested only in \citet[202]{JB81}, who gives the form \intxt{wit\mbox{-}’oqɬik} with the unexpected vowel \intxt{o}, but is left untranscribed by \citet{AG99}.}%2

\dicnote{The vowel in the final syllable in Nivaĉle must be a product of progressive vowel harmonization, and the plural form is non-etymological in that language.}%3

\lit{\citealt{LC-VG-07}: 15}

\PMlemma{{\wordnl{*\mbox{-}qéj\pla{its}}{custom} [1]}}

\wordng{Ni}{\mbox{-}kej\pl{is}} \citep[226]{JS16} {\sep} \wordng{PCh}{*\mbox{-}qéjʔ\pla{is}} > \wordng{Ijw}{\mbox{-}kέʔ\pl{jis}}; \wordng{Mj}{\mbox{-}kέjʔ\pl{is}} (\citealt{JC14b}: 76; \citealt{ND09}: 121; \citealt{JC18}) {\sep} \wordng{PW}{*\mbox{-}qéj\pl{is}} > \wordng{LB}{\mbox{-}qej\pl{is}}; \wordng{Vej}{\mbox{-}kej}; \wordng{’Wk}{\mbox{-}qéjʔ\pl{is}} (\citealt{VN14}: 191; \citealt{VU74}: 62; \citealt{KC16}: 88)

\dicnote{Possibly from \wordng{PM}{*\mbox{-}qá\mbox{-}} (alienable possession) + \wordnl{*\mbox{-}ej}{name}.}%1

\PMlemma{{\wordnl{*sát\mbox{-}uˀk\plf{*sát\mbox{-}ku\mbox{-}jʰ}}{\textit{lecherón} tree\species{Sapium haematospermum}}}}

\wordng{Mk}{setuˀk} [1], \textit{setkw\mbox{-}i} \citep[324]{AG99} {\sep} \wordng{PCh}{*sátuk} > \wordng{Ijw/I’w}{sát(ʲ)uk}; \wordng{Mj}{sátuk} (\citealt{ND09}: 145; \citealt{GS10}: 187; \citealt{JC18}) {\sep} \wordng{PW}{*sátukʷ} > \wordng{Southeastern (Salta)}{satekʷ}; \wordng{’Wk}{sátuk} (\citealt{MS14}: 263; \citealt{KC16}: 326)

\dicnote{The presence of a preglottalized coda in Maká is presumed based on the fact that the suffix \intxt{\mbox{-}uˀk} is otherwise attested with \intxt{ˀk}. The Maká datum is not attested in our sources that distinguish between plain and preglottalized codas.}%1

\PMlemma{{\wordnl{*sát’a(ˀ)(t)s}{parakeet sp.}}}

\word{Ni}{sát’as}{white-eyed parakeet} \citep[231]{JS16} {\sep} \word{PCh}{*sát’as}{blue-crowned parakeet} > \wordng{Ijw}{sát’as}; \wordng{I’w}{sáˀtas\pl{is}}; \wordng{Mj}{sát’as} (\citealt{ND09}: 145; \citealt{AG83}: 157; \citealt{JC18}) {\sep} \wordng{PW}{*sát’as} > \word{LB}{sat’as}{blue-crowned parakeet}; \wordng{Vej}{sat’as}; \wordng{’Wk}{sát’is} [1] (\citealt{VN14}: 157; \citealt{VU74}: 72; \citealt{MG-MELO15}: 22; \citealt{KC16}: 327)

\dicnote{The vowel \intxt{i} in the ’Weenhayek reflex is not the expected outcome of \sound{PW}{*a}.}%1

\PMlemma{{\wordnl{*\mbox{-}sǻq’ålʰ\plf{*\mbox{-}sǻq’ål\mbox{-}its}}{soul, spirit}}}

(?) \wordng{Mk}{\mbox{-}siˀnq’al\pl{its}} [1] \citep[326]{AG99} {\sep} \wordng{Ni}{\mbox{-}såk’åk͡l<it>\plf{\mbox{-}såk’åk͡l<ti>\mbox{-}s}} \citep[358]{JS16} {\sep} \wordng{PCh}{*\mbox{-}sǻq’ålʰ\plf{*\mbox{-}sǻq’ål\mbox{-}is}} > \wordng{Ijw}{\mbox{-}sák’al\plf{\mbox{-}sák’al\mbox{-}is}}; \wordng{I’w}{\mbox{-}sákal} [2] (\citealt{ND09}: 125; \citealt{AG83}: 157)

\dicnote{The Maká form is attested in the New Testament (e.g. Luke 20:24); \citet[326]{AG99} actually mistranscribes it as \intxt{\mbox{-}sinqal\pl{its}}. The Maká word is tentatively included under this etymology, but the sound correspondences are entirely irregular: one would expect \wordng{Maká}{*\mbox{-}saq’al\pl{its}}.}%1

\dicnote{The plain \intxt{k} in \cits{AG83} attestation of the Iyo’awujwa’ reflex must be a mistranscription.}%2

\rej{\citet[47]{EN84} lists reflexes of \word{PW}{*\mbox{-}húsek\plf{*\mbox{-}húse\mbox{-}jʰ}}{temperance, soul} under this etymology.}

\lit{\citealt{EN84}: 47 (\intxt{*sakål}); \citealt{AnG15}: 253}

\PMlemma{{\wordnl{*\mbox{-}såˀt}{vein, tendon}}}

\wordng{Mk}{\mbox{-}<ʔa>saˀt\plf{\mbox{-}<ʔa>sta\mbox{-}j}} [1] \citep[129]{AG99} {\sep} \wordng{Ni}{\mbox{-}såˀt\plf{\mbox{-}såt\mbox{-}åj}} \citep[383]{JS16} {\sep} \wordng{PCh}{*\mbox{-}såt\mbox{-}ǻ…} > \wordng{Ijw}{\mbox{-}sát<aki>}; \wordng{I’w}{\mbox{-}sat<ájik>\plf{sat<áje>\mbox{-}j}}; \word{Mj}{\mbox{-}sat<ájik>\plf{sat<áje>\mbox{-}ej}}{vein} (\citealt{ND09}: 125; \citealt{AG83}: 157; \citealt{JC18}) {\sep} \word{PW}{*\mbox{-}såt}{tendon, heel} > \word{Vej}{\mbox{-}såt}{muscle, tendon}; \word{’Wk}{\mbox{-}såt\plf{\mbox{-}sǻt\mbox{-}aç}}{tendon, heel} [2] (\citealt{VU74}: 72; \citealt{KC16}: 90)

\dicnote{The element \intxt{ʔa\mbox{-}} in Maká has no parallels in other Mataguayan languages and is probably a fossilized morpheme. The presence of a preglottalized coda in Maká is inferred based on the Nivaĉle cognate; the singular form is not attested in our sources that distinguish between plain and preglottalized codas. The plural form is attested in the New Testament (Colossians 2:19), but it is not revealing.}%1

\dicnote{’Weenhayek shows contamination of \word{PW}{*\mbox{-}sat}{heel} and \wordnl{*\mbox{-}såt}{tendon}, which has resulted in a polysemic noun \wordnl{\mbox{-}såt}{tendon, heel}.}%2

\lit{\citealt{LC-VG-07}: 20}

\PMlemma{{\wordnl{*[ji]selǻn}{to spank} [1]}}

\wordng{Mk}{[j]<eq>silan} [2]\gloss{to spank with something flexible} \citep[157]{AG99} {\sep} \word{PCh}{*[ʔi]selǻn}{to prepare} [1] > \wordng{Ijw}{[ʔi]líxsaˀn / \mbox{-}lέxsaˀn} [3]; \word{Mj}{[ʔi]ʃilʲén}{to store}; \wordnl{*[ʔi]selǻn\mbox{-}eh}{to make, to prepare} [1] > \wordng{Ijw}{[ʔi]líxsan\mbox{-}e / \mbox{-}lέxsan\mbox{-}e} [3]; \wordng{I’w}{\mbox{-}silʲén\mbox{-}}; \wordng{Mj}{[ʔi]ʃilʲén\mbox{-}e} (\citealt{ND09}: 102; \citealt{AG83}: 158; \citealt{JC18})

\dicnote{Despite the semantic discrepancy between the Maká and Chorote verbs, we believe them to be cognate. Spanking cháguar (raw caraguatá fiber) against one’s leg is a very important part of making it ready for textile production among the peoples of Chaco.}%1

\dicnote{We have no explanation for the element \intxt{\mbox{-}eq\mbox{-}} in Maká.}%2

\dicnote{Iyojwa’aja’ shows an irregular metathesis of \sound{PCh}{*s} and \intxt{*l} and a regular stress retraction.}%3

\PMlemma{{\wordnl{*\mbox{-}seʔ\plf{*\mbox{-}sé\mbox{-}jʰ}}{bodily hair}}} → \wordnl{*\mbox{-}pxúseʔ}{beard}, \wordnl{*\mbox{-}t(á)ko\mbox{-}seʔ}{eyebrow}, (?) \wordnl{*\mbox{-}tǻtseʔ}{eyelash}

\PMlemma{{\wordnl{*(\mbox{-})skäˀt}{mesh}}}

\wordng{Ni}{\mbox{-}stʃaˀt\plf{\mbox{-}stʃat\mbox{-}is}} \citep[232]{JS16} {\sep} \word{PW}{*sikʲet}{mesh purse} > \wordng{LB}{sitʃet}; \wordng{’Wk}{sikʲet} (\citealt{VN14}: 418; \citealt{KC16}: 329)

\lit{\citealt{EN84}: 41, 47 (\intxt{*s\mbox{-}cɛt’})}

\PMlemma{{\wordnl{*slǻqha(ˀ)j\plf{*slǻqhaj\mbox{-}its}}{wild cat}}}

\wordng{Ni}{ʃk͡låkxaj \recind sk͡låkxaj\pl{is}} [1] (\citealt{NS87}: 498, 535; \citealt{AnG15}: 231; \citealt{JS16}: 239; \citealt{LC20}: 95) {\sep} \wordng{PCh}{*sᵊlǻhqajʔ \recind *sᵊlǻhqåjʔ\pla{is}} [2] > \wordng{Ijw}{silʲákaʔ}; \wordng{I’w}{siláhkaj\pl{is}}; \wordng{Mj}{ʃiláhkajʔ\pl{is}} (\citealt{JC14b}: 91; \citealt{ND09}: 145; \citealt{AG83}: 153; \citealt{JC18}) {\sep} \wordng{PW}{*silǻqhåj} > \wordng{Vej}{silåkåj} [3]; \wordng{’Wk}{silǻqʰåʔ} [4] (\citealt{MG-MELO15}: 22; \citealt{KC16}: 329)

\dicnote{The form \intxt{sk͡låkxaj} is attested as a variant alongside \intxt{ʃk͡låkxaj} in \citet[498, 535]{Stell1987} and \citet[231]{AnG15}. In her discussion of the variation of the type \intxt{sC\mbox{-} \recind ʃC\mbox{-}}, \citet[534–535]{NS87} observes that \intxt{sC\mbox{-}} is found in the speech of her consultant from Las Vertientes (speaker of Chishamnee Lhavos) and – in variation with \intxt{ʃC\mbox{-}} – of one consultant from the Mission of San Leonardo/Fischat (speaker of Shichaam Lhavos), whereas her other consultants from San Leonardo/Fischat and San José de Esteros use exclusively \intxt{ʃC\mbox{-}}. Only the form \intxt{ʃk͡låkxaj} is attested in \citet[95]{LC20}, who deal with the Chishamnee Lhavos dialect, and in \citet[239]{JS16}.}%1

\dicnote{It is impossible to determine whether the PCh form contained an \intxt{*a} or an \intxt{*å} in the last syllable; other Mataguayan languages offer conflicting evidence.}%2

\dicnote{The loss of the aspiration of \wordng{PW}{*qh} in Vejoz is irregular. \citet[72]{VU74} gives \textit{silokaj}, which must be a mistranscription.}%3

\dicnote{The expected reflex in ’Weenhayek would in fact be \intxt{*silǻqʰåjʔ}.}%4

\rej{Despite a superficial similarity to the aforementioned forms, \word{Maká}{xunkhaj\pl{its}}{wild cat} \citep[393]{AG99} shows no regular correspondence with \wordng{PM}{*slǻqhaj\pla{its}}. It must be a borrowing from Nivaĉle instead, whose form was probably influenced by that of \word{Mk}{xunkhaj}{fog}, another likely loan from Nivaĉle (\wordng{Ni}{ʃnakxaj}). \citet[48]{JB81} documents the Maká form as \intxt{xunqaj}.}

\lit{\citealt{EN84}: 11, 37 (\intxt{*slåqaj}); \citealt{LC-VG-07}: 16}

\PMlemma{{\wordnl{*sóp’wa(\mbox{-}ta)\mbox{-}juˀk\plf{*sóp’wa(\mbox{-}ta)\mbox{-}jku\mbox{-}jʰ}}{caspi zapallo\species{Pisonia zapallo}}}}

\wordng{Ni}{sop’a\mbox{-}ta<tʃ>\plf{sop’a\mbox{-}ta<ku>\mbox{-}j}} \citep[235]{JS16} {\sep} \wordng{PCh}{*sóp’wa\mbox{-}juk} > \wordng{Ijw}{sɔ́p’ajik \recind sɔ́p’uwa\mbox{-}jik}; \wordng{I’w}{sóp’(w)a\mbox{-}jik}; \wordng{Mj}{sɔ́p’a\mbox{-}jik} (\intxt{\mbox{-}ij}) (\citealt{ND09}: 147; \citealt{GS10}: 187; \citealt{JC18}) {\sep} \wordng{PW}{*sop’wa\mbox{-}jukʷ} > \wordng{LB}{supfʷa\mbox{-}jekʷ}; \wordng{Southeastern (Salta)}{sup’wajuk \recind so\mbox{-} \recind \mbox{-}pfʷ\mbox{-}} (\citealt{CS08}: 59; \citealt{MS14}: 313)

\PMlemma{{\intxt{*sténi(ʔ)} (fruit); \intxt{*stén\mbox{-}uˀk} (tree)\gloss{white quebracho\species{Aspidosperma quebracho-blanco}}}}

\wordng{Mk}{sitin\mbox{-}uˀk} [1], \intxt{sitin\mbox{-}kw\mbox{-}i} \citep[327]{AG99} {\sep} \wordng{PCh}{*ʔᵊsténiʔ}; \intxt{*ʔᵊsténi\mbox{-}k} > \wordng{Ijw}{ʔistíni\mbox{-}k}; \intxt{ʔistín\mbox{-}kʲet}; \wordng{I’w}{isténi\mbox{-}k}; \wordng{Mj}{ʔistέniʔ\plf{ʔistέni\mbox{-}wal} \recind ʔiʃtínʲeʔ\plf{ʔiʃtínʲe\mbox{-}l}} (\citealt{ND09}: 112; \citealt{AG83}: 132; \citealt{JC18}) {\sep} \wordng{PW}{*ʔistéˀnih} > \wordng{Southeastern (Salta)}{ʔisteˀni} [2]; \wordng{Vej}{isteˀni}; \wordng{’Wk}{ʔistéˀnih} (\citealt{MS14}: 184; \citealt{VU74}: 61; \citealt{MG-MELO15}: 18; \citealt{KC16}: 37)

\dicnote{The preglottalized coda in the Maká suffix for tree names is attested elsewhere \citep[7]{PMA}.}%1

\dicnote{\citet[184]{MS14} actually gives \intxt{isteni}, but note that she consistently fails to transcribe glottalized consonants as such. \citet[59]{CS08} gives the unexpected form \textit{siten̥i}.}%2

\lit{\citealt{EN84}: 39 (\intxt{*s\mbox{-}teni}); \citealt{LC-VG-07}: 20}

\PMlemma{{\wordnl{*stwúˀn\plf{*stwún\mbox{-}its}}{king vulture}}}

\word{Ni}{staβuˀn\plf{staβun\mbox{-}is}}{king vulture; Milky Way} \citep[236]{JS16} {\sep} \wordng{PCh}{*ʔᵊstúuˀn\plf{*ʔᵊstúun\mbox{-}is}} > \wordng{I’w}{ʔistʊ́ˀn}; \wordng{Mj}{ʔistúuˀn\plf{ʔistúun\mbox{-}is}} \recind \intxt{\mbox{-}ʃt\mbox{-}} (own field notes; \citealt{JC18}) {\sep} \wordng{PW}{*ʔistíwin} [1] > \wordng{LB}{ʔistiwin}; \wordng{Vej}{istiwin̥<i>\mbox{-}tah} [2]; \wordng{’Wk}{ʔitsíwin\mbox{-}tax \recind stíwin\mbox{-}tax} (\citealt{CS-FL-PR-VN13}; \citealt{MG-MELO15}: 21; \citealt{KC16}: 40, 334)

\dicnote{The Wichí reflex is entirely irregular.}%1

\dicnote{The Vejoz reflex is mistranscribed as \intxt{istiwin<i>\mbox{-}tah} in \citet[61]{VU74}.}%2

\PMlemma{{\wordnl{*\mbox{-}su(ʔ)\pla{l}}{vagina}}}

\wordng{Mk}{\mbox{-}suʔ\pl{l}} \citep[328]{AG99} {\sep} \wordng{Ni}{\mbox{-}suʔ\pl{k}} \citep[236]{JS16} {\sep} \wordng{PCh}{*\mbox{-}<í>suʔ\pla{l}} [1] > \wordng{Ijw}{\mbox{-}<é>sʲu\pl{ˀl}} [2]; \wordng{I’w}{\mbox{-}<é>sʲuʔ}; \wordng{Mj}{\mbox{-}<éi>ʃuʔ\pl{l}} (\citealt{ND09}: 131; \citealt{AG83}: 127; \citealt{JC18}) {\sep} \wordng{PW}{*\mbox{-}su(ʔ)} > \wordng{Vej}{\mbox{-}su}; \wordng{’Wk}{\mbox{-}suʔ} (\citealt{VU74}: 73; \citealt{KC16}: 221)

\dicnote{The Chorote reflex contains an extra vowel (\wordng{PCh}{*i}) before the root, which appears to continue a fossilized unidentified morpheme.}%1

\dicnote{The absence of a final \intxt{ʔ} in the Iyojwa’aja’ form is unexpected. The regular outcome of \wordng{PCh}{*\mbox{-}ísuʔ} in this variety would be \textit{*\mbox{-}ésʲuʔ} */\mbox{-}ísuʔ/ rather than the attested \textit{\mbox{-}ésʲu} /\mbox{-}ísuh/.}%2

\lit{\citealt{EN84}: 26, 28 (\intxt{*ahs\mbox{-}u \recind *achu})}

\PMlemma{{\wordnl{*sˀwúla(ˀ)χ\plf{*sˀwúla\mbox{-}ts}}{anteater}}}

\wordng{Ni}{sˀβuk͡lax\plf{sˀβuk͡la\mbox{-}s}} [1]\gloss{anteater; rayfish} (\citealt{AnG15}: 53; \citealt{JS16}: 237; \citealt{LC20}: 80) {\sep} \wordng{PCh}{*sᵊʔúlah\plf{*sᵊʔúla\mbox{-}s}} [2] > \wordng{Ijw}{soʔólʲe\pl{s}}; \wordng{I’w}{sʊʔʊ́la \recind soólah\plf{soóla\mbox{-}s}}; \wordng{Mj}{saʔʊ́la\pl{s}} (\citealt{JC14b}: 76, fn. 2, 91; \citealt{ND09}: 147; \citealt{AG83}: 161; \citealt{JC18}) {\sep} \wordng{PW}{*súlaχ} > \wordng{LB}{selaχ}; \wordng{Vej}{\mbox{sulah}} (\intxt{\mbox{-}ɬajis}); \wordng{’Wk}{súlax} (\citealt{VN14}: 213; \citealt{VU74}: 73; \citealt{MG-MELO15}: 22; \citealt{KC16}: 332)

\dicnote{The glottalization in \sound{Nivaĉle}{sˀβ} is attested only in \citet[80]{LC20}, who also report that the speakers of Chishamnee Lhavos from Central Paraguay lose the \intxt{β} and produce \intxt{sʔ\mbox{-}} instead \citep[83]{LC20}.}%1

\dicnote{The correspondence between the vowels of the first syllable in Iyojwa’aja’/Iyo’awujwa’ and Manjui is irregular.}%2

\lit{\citealt{EN84}: 50 (\intxt{*sɛwhla}); \citealt{PVB02}: 144 (\intxt{*seulaχ})}

\PMlemma{{\wordnl{*[ji]sˀwun \recind *[ji]sˀwún}{to like, to love}}}

\wordng{Mk}{[ji]suʔun} \citep[329]{AG99} {\sep} \wordng{Ni}{[ji]sˀβun} [1] \citep[237]{JS16} {\sep} \wordng{PCh}{*[ʔi]sᵊʔún} > \wordng{Mj}{[ʔi]ʃǝʔʊ́n / \mbox{-}saʔʊ́n \recind [ʔi]ʃʊʔʊ́n / \mbox{-}sʊʔʊ́n} \citep{JC18}

\dicnote{In the Chishamnee Lhavos dialect, the verb \textit{[j]en} is used instead of \textit{[ji]sˀβun} \citep[9]{LC20}.}%1

\PMlemma{{\wordnl{*s’ǻm\pla{its}}{frog sp.}}}

\word{Mk}{s’am\mbox{-}s’am\pl{its}}{frog\species{Leptodactylus macrosternum}} (\citealt{AG99}: 329; \citealt{JB81}: 70) {\sep} \wordng{PCh}{*ts’ǻm\pla{its}} > \word{Mj}{ts’ám\pl{is}}{\textit{ju’i} frog\species{Pseudis platensis}} \citep{JC18}

\PMlemma{{\wordnl{*táxχan}{to thunder}}}

\wordng{Mk}{texen} \citep[336]{AG99} {\sep} \wordng{Ni}{taʃxen} [1] \citep[258]{JS16} {\sep} \wordng{PW}{*t’áχan} [2] > \wordng{’Wk}{t’áxan̥} [2] \citep[431]{KC16}

\dicnote{In Nivaĉle, \intxt{e} is not the expected reflex of \wordng{PM}{*a}.}%1

\dicnote{The glottalization of the initial consonant in the Wichí reflex is irregular.}%2

\dicnote{Concerning the final consonant, \citet[431]{KC16} explicitly notes that it is uncertain whether it is glottalized (\intxt{t’áxaˀn}) or voiceless (\intxt{t’áxan̥}); only the voiceless one matches the Nivaĉle cognate.}%3

\PMlemma{{\wordnl{*[ni]tǻɸä(ˀ)l\mbox{-}\APPL}{to know, to be acquainted} [1]}}

\wordng{Ni}{[ni]tåɸak͡l\mbox{-}\APPL} \citep[274]{JS16} {\sep} \wordng{PCh}{*[ʔi]tǻhwel\mbox{-}\APPL} > \wordng{I’w}{[i]tʲéfʷel\mbox{-}eʔ / \mbox{-}táfʷel\mbox{-}eʔ}\gloss{to know, to know how to} [2]; \wordng{Mj}{[ʔi]t(ʲ)éhwel\mbox{-}e / \mbox{-}táhwel\mbox{-}e} (\citealt{AG83}: 42, 162; \citealt{JC18}) {\sep} \wordng{PW}{*\mbox{-}tǻxʷel\mbox{-}{\APPL} / *\mbox{-}tǻxʷnh\mbox{-}\APPL} > \wordng{LB}{\mbox{-}tofʷel\mbox{-}eχ / \mbox{-}tofʷn̥\mbox{-}…\mbox{-}eχ}\textsc{;} \wordng{Vej}{\mbox{-}tahʷel\mbox{-}eh} [3]; \wordng{’Wk}{[ni]tǻxʷel\mbox{-}{\APPL} / [ni]tǻxʷn̥\mbox{-}\APPL} (\citealt{VN14}: 342; \citealt{VU74}: 74; \citealt{KC16}: 337–339)

\dicnote{This could be an ancient compound involving a root for\gloss{eye, sight} (as \wordng{Ni}{tå\mbox{-}} in \wordnl{[ji]tå<ɸaɬ>}{to get something in one’s eye}, \wordnl{tå\mbox{-}ˀmat}{to have bad sight}, \wordnl{tå<sex>}{eye, seed}) and\gloss{to tell} (\wordng{PM}{*[ji]ɸä́l}). Compare \word{Maká}{[n]ikfeˀl\mbox{-}\APPL}{to know, to be acquainted} \citep[195]{AG99}, whose element \intxt{\mbox{-}feˀl\mbox{-}} might be cognate with \wordng{PM}{*\mbox{-}ɸä(ˀ)l\mbox{-}} in \intxt{*[ni]tǻɸä(ˀ)l\mbox{-}\APPL}.}%1

\dicnote{\citet[191]{AG83} also documents the irregular forms \textit{\mbox{-}táwel\mbox{-}eʔ} and \textit{\mbox{-}táfʷeʔ}, which could result from mistranscription.}%2

\dicnote{The vowel \intxt{a} (as opposed to \textit{å}) in Vejoz must be a mistranscription on \cits{VU74} part.}%3

\empr{\citet[308]{AF16} compares the Mataguayan verb with the Enlhet–Enelhet verb with the same meaning – \wordng{Enlhet, Enenlhet-Toba}{\mbox{-}jekpelk\mbox{-}}, \wordng{Enxet}{\mbox{-}jekpeltʃ\mbox{-}}, \wordng{Sanapaná}{\mbox{-}jepeɬ\mbox{-}}, \wordng{Guaná}{\mbox{-}jekpeɬk\mbox{-}} (\citealt{EU-HK-97}: 459; \citealt{EU-HK-MR-03}: 323; \citealt{ASG12}: 349; \citealt{JE21}: 618; \citealt{HK-23}: 164) – but this could be spurious.}

\lit{\citealt{AF16}: 308}

\PMlemma{{\wordnl{*tåˀɬ}{to sprout, to come out}}}

\wordng{Mk}{taˀɬ} [1] \citep[331]{AG99} {\sep} \wordng{Ni}{tåˀɬ} \citep[276]{JS16} {\sep} \wordng{PCh}{*tåɬ} > \wordng{Ijw}{taɬ}; \wordng{I’w}{\mbox{-}tál}; \wordng{Mj}{táɬ} (\citealt{JC14b}: 87; \citealt{ND09}: 149; \citealt{AG83}: 162; \citealt{JC18}) {\sep} \wordng{PW}{*tåɬ} > \wordng{LB}{toɬ\mbox{-}\APPL}\gloss{to come from}; \wordng{Vej}{\mbox{-}tåɬ\mbox{-}e}\gloss{sprout, descendant}; \wordng{’Wk}{tåɬ} (\citealt{VN14}: 230, 263; \citealt{VU74}: 75; \citealt{KC16}: 339)

\dicnote{The preglottalized coda in the Maká reflex is attested in the New Testament (e.g. John 17:7).}%1

\PMlemma{{\wordnl{*\mbox{-}tǻmteʔ}{daughter-in-law}; \wordnl{*\mbox{-}tǻmte\mbox{-}ts}{children-in-law} [1]}}

\wordng{Ni}{\mbox{-}tåmit’a}, \textit{\mbox{-}tåmte\mbox{-}s}\gloss{son-in-law}; \textit{\mbox{-}tåmte<ʔe>\pl{j}}\gloss{daughter-in-law}; \textit{\mbox{-}tåmk͡låˀji\pl{k}}\gloss{child-in-law responsible for a funerary ritual} \citep[276]{JS16} {\sep} \wordng{PCh}{*\mbox{-}tǻmteʔ}; \textit{*\mbox{-}tǻmte\mbox{-}ts} > \wordng{Mj}{\mbox{-}támet}\gloss{son-in-law}; \textit{\mbox{-}támteʔ}\gloss{daughter-in-law}; \textit{\mbox{-}támte\mbox{-}s}\gloss{children-in-law} \citep{JC18}

\dicnote{It is possible to reconstruct the root \wordnl{*\mbox{-}tǻm\mbox{-}}{child-in-law}, but other derivatives cannot be reconstructed at this time.}%1

\lit{\citealt{EN84}: 47 (\wordnl{*tɛmɛt}{son-in-law})}

\PMlemma{{\wordnl{*\mbox{-}tǻtseʔ\pla{jʰ}}{eyelash}}}

\wordng{Mk}{\mbox{-}tetsiʔ\pl{j}} [1] \citep[336]{AG99} {\sep} \wordng{Ni}{\mbox{-}tåtse\pl{j}} \citep[384]{JS16} {\sep} \wordng{PCh}{*\mbox{-}tǻseʔ\pla{jʰ}} > \wordng{Ijw}{\mbox{-}táxseʔ\pl{ˀl}} [2]; \wordng{I’w/Mj}{\mbox{-}táxseʔ\pl{j}} (\citealt{JC14b}: 93; \citealt{ND09}: 125; \citealt{AG83}: 162; \citealt{JC18})

\dicnote{The vowel \intxt{e} in the Maká word is unexpected and does not match either \wordng{Ni}{å} or \wordng{Chorote}{*å} (it is certain that PCh had \textit{*å} and not \textit{*a} in this word, cf. Iyojwa’aja’ \textit{hitʲáseʔ} /hl\mbox{-}tɑ́se/\gloss{his/her eyelash}, \textit{ʔitʲáseʔ} /j\mbox{-}tɑ́se/\gloss{my eyelash}).}%1

\dicnote{The Iyojwa’aja’ plural form, as attested by \citet{ND09}, is non-etymological.}%2

\gc{\citet[308]{PVB13a} suggests that this is a compound (with its first element meaning\gloss{eye}) and compares the second element with \wordng{Proto-Guaicuruan}{*\mbox{-}ad’e}\gloss{eyelash}.}

\lit{\citealt{PVB13a}: 308 (\intxt{*\mbox{-}tʌ\mbox{-}tsiʔ})}

\PMlemma{{\intxt{*\mbox{-}tǻwäˀx\plf{*\mbox{-}tǻwxä\mbox{-}ts}} [1]\gloss{cavity, abdominal cavity} [2]}}

\wordng{Mk}{\mbox{-}taweˀx} [3], \textit{\mbox{-}tawxe\mbox{-}ts} \citep[333]{AG99} {\sep} \wordng{Ni}{\mbox{-}tåβa(ˀ)ʃ}, \textit{\mbox{-}tåβxa\mbox{-}s} \citep[277]{JS16} {\sep} \wordng{PCh}{*\mbox{-}tóweh} [4] > \wordng{Ijw}{\mbox{-}tɔ́we}, \textit{\mbox{-}tɔ́waˀl}; \wordng{I’w}{\mbox{-}tówe\pl{j}} [1]; \wordng{Mj}{\mbox{-}tɔ́we} (\citealt{ND09}: 126; \citealt{AG83}: 166; \citealt{JC18}) {\sep} \wordng{PW}{*toweχ}, \textit{*towhá\mbox{-}jʰ} [1 4 5]\gloss{vessel} > \wordng{LB}{tuweχ}, \textit{tuʍa\mbox{-}j}; \wordng{Vej}{toweh}; \wordng{’Wk}{towex}, \textit{toʍá\mbox{-}ç}; \textit{*\mbox{-}tóweχ}, \textit{*\mbox{-}tówha\mbox{-}jʰ} [1 4]\gloss{opening} > \wordng{Vej}{toweh}; \wordng{’Wk}{\mbox{-}tówex}, \textit{\mbox{-}tóʍa\mbox{-}ç} (\citealt{VN14}: 58; \citealt{VU74}: 77; \citealt{MG-MELO15}: 52; \citealt{KC16}: 94, 420)

\dicnote{The plural form is reconstructed based on the evidence from Maká and Nivaĉle. Chorote and Wichí show noncognate plural forms.}%1

\dicnote{This term is likely an obscure compound, with \wordng{PM}{*\mbox{-}wä́ˀx} as its second part.}%2

\dicnote{The preglottalized coda in the Maká reflex is attested in the New Testament (e.g. Luke 1:46).}%3

\dicnote{The raising of \wordng{PM}{*å} to \wordng{PCh/PW}{*o} is not known to be regular.}%4

\dicnote{The absolute form is only documented in Wichí and might not be reconstructible all the way to PM.}%5

\lit{\citealt{EN84}: 27, 56 (\intxt{*thɔwɛhn \recind *tåwɛhn}\gloss{opening}); \citealt{PVB02}: 143 (\intxt{*towex})\gloss{hole}; \citealt{PVB13a}: 311 (\intxt{*\mbox{-}to\mbox{-}weh})}

\PMlemma{{\textit{*tänúk\pla{its}}\gloss{feline} (\enquote{cat} in the contemporary languages) [1]}}

\wordng{Mk}{tenuk\pl{its}} \citep[335]{AG99} {\sep} \wordng{Ni}{tanuk\pl{is}} \citep[255]{JS16} {\sep} \wordng{PCh}{*tinúk\pla{is}} > \wordng{Ijw/I’w}{tinʲúk\pl{is}}; \wordng{Mj}{tinʲúk\pl{is}} (\citealt{ND09}: 151; \citealt{AG83}: 165; \citealt{JC18})

\dicnote{The reflexes of this term in the contemporary varieties designate \textit{Felis catus} (the domestic cat). In the protolanguage, the root in question must have designated an unidentified feline species native to South America, possibly the jaguarundi (\intxt{Herpailurus yagouaroundi}), still designated by a derivative of the same root in Manjui (\intxt{tin(ʲ)úk\mbox{-}ite}, literally\gloss{similar to a \textit{tin(ʲ)úk}}).}%1

\empr{\citet[308]{AF16} observes that this root is obviously related via borrowing to an Enlhet–Enenlhet term with the same meaning, \word{Enenlhet-Toba, Sanapaná, Guaná}{tenok}{cat} (\citealt{EU-HK-MR-03}: 337; \citealt{ASG12}: 149; \citealt{HK-23}: 188).}

\lit{\citealt{EN84}: 12, 49 (\intxt{*tajn\mbox{-}(j)úk}); \citealt{LC-VG-07}: 15; \citealt{AF16}: 308}

\PMlemma{{\intxt{*\mbox{-}tä(ˀ)ts\plf{*\mbox{-}täts\mbox{-}él}} [1]\gloss{trunk; base; origin, fault}; \wordnl{*\mbox{-}tä́ts\mbox{-}uˀk\plf{*\mbox{-}tä́ts\mbox{-}ku\mbox{-}jʰ}}{trunk}}}

\wordng{Ni}{\mbox{-}tats\mbox{-}uk\plf{\mbox{-}tas\mbox{-}ku\mbox{-}j}} \citep[259]{JS16} {\sep} \wordng{PCh}{*\mbox{-}tés\plf{*\mbox{-}tes\mbox{-}él}}; \intxt{*(\mbox{-})tés\mbox{-}uk\plf{*\mbox{-}tés\mbox{-}ku\mbox{-}jʰ}} > Ijw~3~\textit{hi\mbox{-}tís} (\intxt{\mbox{-}eˀl})\gloss{root; procedence; fault};~\third{hi\mbox{-}tísʲ\mbox{-}uk}, \textit{hi\mbox{-}tís\mbox{-}kʲu\mbox{-}ˀl} [1]\gloss{trunk}; \wordng{I’w}{tésʲ\mbox{-}uk}, \textit{\mbox{-}tés\mbox{-}ki\mbox{-}ʔ}; Mj~\third{hi\mbox{-}tés\mbox{-}uk \recind hi\mbox{-}tés\mbox{-}kiʔ}, \textit{hi\mbox{-}tés\mbox{-}ki\mbox{-}j}\gloss{stump} (\citealt{ND09}: 126; \citealt{AG83}: 164; \citealt{JC18}) {\sep} \wordng{PW}{*\mbox{-}tes}, \textit{\mbox{-}téts\mbox{-}elʰ} > \wordng{LB}{\mbox{-}tes}, \textit{\mbox{-}tets\mbox{-}eɬ}; \wordng{Vej}{\mbox{-}tes}\gloss{fault, debt}; \wordng{’Wk}{\mbox{-}tes}, \textit{\mbox{-}téts\mbox{-}eɬ} (\citealt{VN14}: 114, 154, 215; \citealt{JB09}: 49; \citealt{VU74}: 75; \citealt{MG-MELO15}: 57; \citealt{KC16}: 93, 221)

\dicnote{The plural form \intxt{hi\mbox{-}tis\mbox{-}kʲu\mbox{-}ˀl}, attested in Iyojwa’aja’, is non-etymological.}%1

\lit{\citealt{LC-VG-07}: 16}

\PMlemma{{\textit{*\mbox{-}teʔ} (\intxt{*\mbox{-}té\mbox{-}jʰ})\gloss{eye}}}

\wordng{Mk}{\mbox{-}t<oʔ>\pl{j}} [1] \citep[343]{AG99} {\sep} \wordng{PCh}{*\mbox{-}ta\mbox{-}téʔ\pla{jʰ}} > \wordng{Ijw}{\mbox{-}tá\mbox{-}teʔ\pl{ˀl}} [2]; \wordng{I’w}{\mbox{-}ta\mbox{-}téʔ\pl{j}}; \wordng{Mj}{\mbox{-}ta\mbox{-}tέʔ\pl{jh}} (\citealt{JC14b}: 87; \citealt{ND09}: 126; \citealt{AG83}: 163; \citealt{JC18}) {\sep} \wordng{PW}{*\mbox{-}t(a)\mbox{-}teʔ\pla{jʰ}} > \wordng{LB}{\mbox{-}t\mbox{-}te\mbox{-}j}\gloss{face}, \textit{\mbox{-}t\mbox{-}te\mbox{-}ɬu}\gloss{eye} [3]; \wordng{Vej}{\mbox{-}te\pl{j}}, \textit{\mbox{-}te\mbox{-}ɬo}; \wordng{’Wk}{\mbox{-}t(a)\mbox{-}teʔ} (\intxt{\mbox{-}t(a)\mbox{-}té\mbox{-}ç}) (\citealt{VN14}: 161, 165; \citealt{VU74}: 75; \citealt{AFG067}: 219; \citealt{KC16}: 99)

\dicnote{The Maká word is apparently an ancient compound of \wordnl{*\mbox{-}teʔ}{eye} and \wordnl{*\mbox{-}oʔ\pla{jʰ}}{seed}.}%1

\dicnote{The plural form attested in Iyojwa’aja’ does not match the one found in Manjui and Wichí and is thus non-etymological.}%2

\dicnote{In Lower Bermejeño Wichí, the erstwhile plural form of\gloss{eye} is now used in the meaning\gloss{face}; a compound (\enquote{eye} +\gloss{seed}) is now used for the meaning\gloss{eye} (compare ’Weenhayek \textit{\mbox{-}t(a)\mbox{-}té\mbox{-}ɬoʔ\pl{ç}}\gloss{eye globe}, attested in \citealt{KC16}: 100). Note, however, that \citet[57]{JB09} documents \wordng{LB}{\mbox{-}teʔ}\gloss{eye}.}%3

\lit{\citealt{PVB13a}: 308, fn. 20 (\intxt{*\mbox{-}tʌʔ})}

\PMlemma{{\textit{*téwo(ˀ)k \recvar *téwå(ˀ)k} [1]\gloss{river}}}

\wordng{Ni}{toβok\plf{toβxo\mbox{-}j}}; \wordng{ChL/ShL}{toβåk\plf{toβxå\mbox{-}j}}; \wordng{YL}{toβak} (\citealt{AnG15}: 38; \citealt{JS16}: 274; \citealt{LC20}: 99) {\sep} \wordng{PCh}{*téwok \recind *téwåk} [1] > \wordng{Ijw}{tέwuk\pl{is}}; \wordng{I’w}{téwak}; \wordng{Mj}{tέwak} (\citealt{JC14b}: 90; \citealt{ND09}: 150; \citealt{AG83}: 164; \citealt{JC18}) {\sep} \wordng{PW}{*téwokʷ} > \wordng{LB}{tewukʷ}; \word{Vej}{tek\mbox{-}tah}{river}, \wordnl{tewokʷ\mbox{-}tah}{Pilcomayo River}; \wordng{’Wk}{téwok} (\intxt{\mbox{-}is \recind \mbox{-}lis \recind \mbox{-}ɬajis}) (\citealt{VN14}: 161; \citealt{VU74}: 75; \citealt{MG-MELO15}: 44; \citealt{KC16}: 397)

\largerpage
\dicnote{The variant \intxt{*téwok} is suggested by the reflexes in Iyojwa’aja’, Wichí, and by the Nivaĉle reflex \textit{toβok}, attested in \citet{AF16} and \citet{JS16}. The latter is likely a dialectal reflex, though our sources do not specify the dialect to which it belongs. The variant \intxt{*téwåk} is suggested by the reflexes in Iyo’awujwa’, Manjui, and all major varieties of Nivaĉle, such as Chishamnee Lhavos (\citealt{LC20}), Shichaam Lhavos \citep{AnG15}, and Yita’ Lhavos \citep{AnG15}. It is unclear which variant is more conservative.}%1

\lit{\citealt{LC-VG-07}: 15, 21}\clearpage

\PMlemma{{\wordnl{*tiɸ \recind *tíɸ}{to spend}}}

\wordng{Ni}{tiɸ} \citep[268]{JS16} {\sep} \wordng{PCh}{*[ʔi]tíʍ} [1] > \wordng{Ijw}{[ʔi]tíʍ / \mbox{-}téʍ}; \wordng{Mj}{[ʔi]tíʍ / \mbox{-}téiʍ} (\citealt{ND09}: 113; \citealt{JC18})

\dicnote{In Chorote, this verb now receives a non-etymological third-person prefix \intxt{ʔi\mbox{-}} (rather than zero).}%1

\PMlemma{{\wordnl{*tiˀɸ}{to suckle (at)}}}

\wordng{Mk}{tuˀf / \mbox{-}ɬuˀf} [1] \citep[343]{AG99} {\sep} \wordng{Ni}{tiˀɸ} \citep[268]{JS16} {\sep} \wordng{PCh}{*[ʔi]tíʍ} [2] > \wordng{Mj}{[ʔi]tíʍ / \mbox{-}téiʍ} \citep{JC18} {\sep} \wordng{PW}{*tip} [3] > \wordng{Vej}{\mbox{-}tip\mbox{-}eh}; \wordng{’Wk}{tip} (\citealt{VU74}: 76; \citealt{MG-MELO15}: 36; \citealt{KC16}: 407)

\dicnote{The rounded vowel in the Maká reflex is unexpected. The preglottalized coda is attested in the New Testament (e.g. Matthew 21:16).}%1

\dicnote{In Chorote, this verb now receives a non-etymological third-person prefix \intxt{ʔi\mbox{-}} (rather than zero).}%2

\dicnote{It is unclear whether the development \sound{PM}{*ˀɸ} > \sound{PW}{*p} is regular, as no supporting examples are known. Compare the causative \wordng{PW}{*[ʔi]ˈtíx\mbox{-}qat}\gloss{to breastfeed} > \wordng{Vej}{\mbox{-}tih\mbox{-}kat}; \wordng{’Wk}{[ʔi]ˈtíx\mbox{-}qat} (\citealt{VU74}: 76; \citealt{KC16}: 400).}%3

\gc{Possibly related to \word{Proto-Guaicuruan}{*\mbox{-}ˀlip}{to suck} (\citealt{PVB13b}, \#376).}

\PMlemma{{\wordnl{*tijåˀχ}{to shoot, to throw}}}

\wordng{Mk}{tijaˀχ / \mbox{-}ɬijaˀχ} \citep[340]{AG99} {\sep} \wordng{Ni}{tijåˀx} (ShL \textit{tijoˀx}) (\citealt{NS87}: 504; \citealt{JS16}: 270) {\sep} \wordng{PCh}{*[ʔi]tíjåh} [1] > \wordng{Ijw}{[ʔi]tíja / \mbox{-}téja}; \wordng{Mj}{[ʔi]tíje / \mbox{-}téije} (\citealt{ND09}: 114; \citealt{JC18}) {\sep} \wordng{PW}{*tijåχ} > \wordng{LB}{tijoχ}; \wordng{’Wk}{tijåx} (\citealt{VN14}: 145; \citealt{KC16}: 409)

\dicnote{The presence of a preglottalized coda in Maká is inferred based on the Nivaĉle cognate; the verb is not attested in our sources that distinguish between plain and preglottalized stops.}%1

\dicnote{In Chorote, this verb now receives a non-etymological third-person prefix \intxt{ʔi\mbox{-}} (rather than zero).}%2

\PMlemma{{\intxt{*tilVχ \recind *tílVχ \recind *tilV́χ} [1]`gloss{white woodpecker}}}

\wordng{Mk}{tilaχ} \citep[62]{JB81} {\sep} {\sep} \wordng{PW}{*tiliχ \recind *tíliχ \recind *tilíχ} > \wordng{LB}{tiliχ} (\citealt{CS-FL-PR-VN13})

\dicnote{The vowel of the second syllable cannot be reconstructed with certainty: Maká points to \sound{PM}{*å}, \intxt{*a}, or \intxt{*e}, whereas Lower Bermejeño Wichí points to \intxt{*i}.}%1

\PMlemma{{\wordnl{*\mbox{-}tiˀɬ}{to spin a thread, to sew}}}

\wordng{Mk}{[ji]tiɬ} [1]\gloss{to sew} \citep[337]{AG99} {\sep} \wordng{Ni}{tiˀɬ} \citep[269]{JS16} {\sep} \word{PCh}{*[j]<á>tiɬ}{to sew} > \wordng{Ijw}{[j]étiɬ / \mbox{-}átiɬ}; \wordng{I’w}{\mbox{-}átel\mbox{-}jiʔ}; \wordng{Mj}{[j]étiɬ / \mbox{-}átiɬ}; \wordnl{*[ʔi]tíl\mbox{-}kʲ’eʔ}{to spin a thread} > \wordng{Ijw}{[ʔi]tíl\mbox{-}k’i / \mbox{-}tél\mbox{-}k’i}; \wordng{Mj}{[ʔi]tíl\mbox{-}ʔiʔ / \mbox{-}téil\mbox{-}ʔiʔ} (\citealt{ND09}: 113, 159; \citealt{AG83}: 122; \citealt{JC18})

\dicnote{The Maká reflex unexpectedly lacks preglottalization in the coda, as attested in the New Testament (Mark 1:19; Matthew 4:21).}%1

\PMlemma{{\wordnl{*tiɬåˀx}{to carry on one’s shoulders}}}

\wordng{Mk}{tiɬoˀx / \mbox{-}ɬiɬoˀx} [1] \citep[337]{AG99} {\sep} \wordng{Ni}{tiɬåˀx} \citep[269]{JS16} {\sep} \wordng{PCh}{*[ʔi]tíhlåh} [2] > \wordng{Ijw}{[ʔi]tíhlʲa / \mbox{-}téhlʲa}; \wordng{I’w}{\mbox{-}té(h)li \recind \mbox{-}téjhli}; \wordng{Mj}{[ʔi]tíhlʲe / \mbox{-}téihlʲe} (\citealt{ND09}: 113; \citealt{AG83}: 164, 189; \citealt{JC18}) {\sep} \wordng{PW}{*tiɬåχ} > \wordng{LB}{tiɬoχ}; \wordng{Vej}{tiɬåh}; \wordng{’Wk}{tiɬåx} (\citealt{VN14}: 145; \citealt{VU74}: 76; \citealt{KC16}: 404)

\dicnote{The vowel \intxt{o} in the Maká reflex is entirely unexpected. The presence of a preglottalized coda in Maká is inferred based on the Nivaĉle cognate; the verb is not attested in our sources that distinguish between plain and preglottalized stops.}%1

\dicnote{In Chorote, this verb now receives a non-etymological third-person prefix \intxt{ʔi\mbox{-}} (rather than zero).}%2

\gc{Possibly related to \word{Proto-Guaicuruan}{*\mbox{-}i(ˀ)lak}{shoulder}, whence Mbayá ‹\mbox{-}ilacate›\gloss{to carry on one’s shoulders} (\citealt{PVB13b}, \#276). \citet[309]{PVB13a} compares it to \wordng{Proto-Guaicuruan}{*\mbox{-}iˀlaqa}\gloss{back (of body)} instead.}

\lit{\citealt{PVB02}: 144 (\intxt{*\mbox{-}tiɬʌχ}, misglossed as\gloss{to dig}); \citealt{PVB13a}: 309 (\intxt{*\mbox{-}t\mbox{-}iɬʌh})}

\PMlemma{{\wordnl{*tim}{to swallow}}}

\wordng{Mk}{tim\mbox{-}xuʔ / \mbox{-}ɬim\mbox{-}xuʔ} \citep[338]{AG99} {\sep} \wordng{Ni}{tim} \citep[269]{JS16} {\sep} \wordng{PCh}{*[ʔi]tím} [1] > \wordng{Ijw}{[ʔi]tíˀm / \mbox{-}téˀm}; \wordng{I’w}{\mbox{-}tém}; \wordng{Mj}{[ʔi]tím / \mbox{-}téim} (\citealt{ND09}: 114; \citealt{AG83}: 164; \citealt{JC18}) {\sep} \wordng{PW}{*tim} > \wordng{LB/Vej}{tim}; \wordng{’Wk}{tim̥} (\citealt{VN14}: 349; \citealt{VU74}: 76; \citealt{KC16}: 407)

\dicnote{In Chorote, this verb now receives a non-etymological third-person prefix \intxt{ʔi\mbox{-}} (rather than zero).}%1

\PMlemma{{\wordnl{*tis}{to invite, to pay}}}

\word{Mk}{tis\mbox{-}ix / \mbox{-}ɬis\mbox{-}ix}{to give} \citep[339]{AG99} {\sep} \wordng{Ni}{tis} \citep[270]{JS16} {\sep} \wordng{PCh}{*[ʔi]tís} [1] > \wordng{Ijw}{[ʔi]tís / \mbox{-}tés}; \wordng{I’w}{\mbox{-}tés}; \wordng{Mj}{[ʔi]tís / \mbox{-}téis} (\citealt{ND09}: 114; \citealt{AG83}: 164; \citealt{JC18}) {\sep} \wordng{PW}{*tis} > \wordng{Vej/’Wk}{tis} (\citealt{VU74}: 76; \citealt{KC16}: 408)

\dicnote{In Chorote, this verb now receives a non-etymological third-person prefix \intxt{ʔi\mbox{-}} (rather than zero).}%1

\PMlemma{{\wordnl{*títe(ˀ)k\plf{*títhe\mbox{-}jʰ}}{plate}}}

\wordng{Ni}{(\mbox{-})titetʃ\plf{(\mbox{-})titxe\mbox{-}j}} \citep[270]{JS16} {\sep} \wordng{PCh}{*títek\plf{*tíhte\mbox{-}jʰ}} > \wordng{Ijw}{tétik\plf{téti\mbox{-}ˀl}} [1]\gloss{recipient for food}; \wordng{I’w}{téitik\plf{téjti\mbox{-}ji}} [1]; \wordng{Mj}{téitik\plf{téihti\mbox{-}j}} (\citealt{ND09}: 150; \citealt{AG83}: 163; \citealt{JC18})

\dicnote{The plural forms in Iyojwa’aja’ and Iyo’awujwa’ are non-etymological.}%1

\lit{\citealt{LC-VG-07}: 16, 22; \citealt{AnG15}: 64}

\PMlemma{{\wordnl{*tiˀx}{to dig} [1]}}

\wordng{Mk}{ti(ˀ)x\mbox{-}{\APPL} / \mbox{-}ɬi(ˀ)x\mbox{-}\APPL} [2] \citep[339]{AG99} {\sep} \wordng{Ni}{tiˀʃ} \citep[269]{JS16} {\sep} \wordng{PCh}{*[ʔi]tíh\mbox{-}ijʔ} [3] > \wordng{Ijw}{[ʔi]tíh\mbox{-}iʔ / \mbox{-}téh\mbox{-}eʔ}; \wordng{I’w}{\mbox{-}téh\mbox{-}iʔ}; \wordng{Mj}{[ʔi]tíh\mbox{-}ijʔ / \mbox{-}tɪ́h\mbox{-}ijʔ} (\citealt{JC14b}: 90; \citealt{ND09}: 113; \citealt{AG83}: 165; \citealt{JC18}) {\sep} \wordng{PW}{*tiχ} > \wordng{LB}{tiʃ\mbox{-}i hohnat} (lit.\gloss{to dig\textit{\mbox{-}\APPL} earth}); \wordng{Vej}{tih\mbox{-}\APPL}; \wordng{’Wk}{tix} (\citealt{JB09}: 57; \citealt{VU74}: 76; \citealt{KC16}: 399)

\dicnote{The underived verb is intransitive. Applicative derivations are used for expressing an object.}%1

\dicnote{The root-final consonant in Maká is attested as preglottalized in the New Testament in the forms \textit{tiˀx\mbox{-}ik’wi}\gloss{to bury, to dig} (Acts 5:6; Acts 5:9; Acts 8:2; Luke 6:48; Mark 6:29; Matthew 25:18), \textit{tiˀx\mbox{-}ifiʔ}\gloss{to row} (John 6:19; Mark 6:48). However, the forms \textit{tix\mbox{-}xuʔ}\gloss{to dig} (Matthew 21:33; Mark 12:1) and \textit{wi\mbox{-}tix\mbox{-}kiʔ}\gloss{well} (e.g. Revelations 9:2) are attested with a plain coda.}%2

\dicnote{In Chorote, this verb now receives a non-etymological third-person prefix \intxt{ʔi\mbox{-}} (rather than zero).}%3

\lit{\citealt{PVB02}: 143 (\intxt{*tix}; glossed as Spanish\gloss{lavar}, a typo for\gloss{cavar})}

\PMlemma{{\wordnl{*\mbox{-}t(á)koʔ\pla{l}}{face}; \wordnl{*\mbox{-}t(á)ko\mbox{-}seʔ\pla{jʰ}}{eyebrow} [1]}}

\wordng{Mk}{\mbox{-}tko<jek>}, \textit{\mbox{-}tko<jeh>\mbox{-}ej}; \textit{\mbox{-}tko\mbox{-}siʔ\pl{j}} \citep[286]{AG99} {\sep} \wordng{Ni}{\mbox{-}takoʔ\pl{l}} \citep[246]{JS16} {\sep} \wordng{PCh}{*\mbox{-}tókoʔ\pla{l}} > \wordng{Ijw}{\mbox{-}tɔ́kʲoʔ\pl{ˀl}}; \wordng{I’w}{\mbox{-}tókʲoʔ\pl{l}}; \wordng{Mj}{\mbox{-}tɔ́kʲoʔ}; \textit{*\mbox{-}tóko\mbox{-}seʔ\pla{jʰ}} > \wordng{Ijw}{\mbox{-}tɔ́kʲo\mbox{-}seʔ}; \wordng{I’w}{\mbox{-}tókʲo\mbox{-}seʔ\pl{j}}; Mj~\textsc{pl}~\intxt{\mbox{-}tɔ́kʲo\mbox{-}se\mbox{-}j} (\citealt{ND09}: 126; \citealt{AG83}: 166; \citealt{JC18}) {\sep} \wordng{PW}{*\mbox{-}tákʲo\pla{lʰ}}\gloss{forehead} > \wordng{Vej}{\mbox{-}tatʃo\pl{ɬ}}; \wordng{’Wk}{\mbox{-}tákʲoʔ}; \intxt{*\mbox{-}tákʲo\mbox{-}se\pla{jʰ}} > LB~\textsc{pl}~\intxt{\mbox{-}tatʃu\mbox{-}se\mbox{-}j}; \wordng{’Wk}{\mbox{-}tákʲo\mbox{-}seʔ\pl{ç}} (\citealt{JB09}: 56; \citealt{VU74}: 73; \citealt{MG-MELO15}: 61; \citealt{KC16}: 92)

\dicnote{It is unclear whether a consonant cluster should be reconstructed in this case (assuming vowel insertion in Nivaĉle, Chorote, and Wichí) or whether the vowel was already there in Proto-Mataguayan (assuming an irregular syncope in Maká).}%1

\lit{\citealt{EN84}: 22 (\wordnl{*tåçɔ}{face}); \citealt{PVB13a}: 308, fn. 21 (\intxt{*\mbox{-}tʌkoʔ}\gloss{forehead}, \textit{*\mbox{-}tʌko\mbox{-}siʔ}\gloss{eyebrow}); \citealt{LC-VG-07}: 16 (\enquote{forehead})}

\PMlemma{{\wordnl{*tlúˀk}{blind}}}

\wordng{Ni}{tak͡luˀk}, \textit{tak͡lux\mbox{-}uj}\gloss{blind; greater pichiciego} \citep[248]{JS16} {\sep} \wordng{PCh}{*tᵊlúk} > \wordng{I’w}{talók} \citep[162]{AG83} {\sep} \wordng{PW}{*tilúkʷ} > \wordng{’Wk}{tilúk\pl{is}} \citep[404]{KC16}

\lit{\citealt{EN84}: 24 (\textsc{pl}~\textit{*taluk\mbox{-}j}); \citealt{LC-VG-07}: 15; \citealt{AnG15}: 253}

\PMlemma{{\wordnl{*tós\pla{its}}{snake}}}

\wordng{Ni}{tos\pl{is}} (\citealt{LC20}: 95) {\sep} \wordng{PCh}{*tós\pla{is}} > \wordng{I’w}{tóxs\pl{is}}; \wordng{Mj}{tɔ́s}, \textit{tɔ́xʃ\mbox{-}is} (\citealt{AG83}: 166; \citealt{JC18})

\PMlemma{{\intxt{*tóχ\mbox{-}ejʰ\plf{*tó\mbox{-}ts\mbox{-}ejʰ}}; \wordnl{*tóχ\mbox{-}\APPL\plf{*tó\mbox{-}ts\mbox{-}\APPL}}{far}}}

\wordng{Mk}{toχ\mbox{-}ij\plf{to\mbox{-}ts\mbox{-}ij}} \citep[342]{AG99} {\sep} \wordng{Ni}{tox\mbox{-}ej\plf{tox\mbox{-}\APPL}} \citep[273]{JS16} {\sep} \wordng{PCh}{*tóhw\mbox{-}ejʰ\plf{*tó\mbox{-}ts\mbox{-}ejʰ}}; \intxt{*tóh\mbox{-}\APPL\plf{*tó\mbox{-}ts\mbox{-}\APPL}} > \wordng{Ijw}{tɔ́hw\mbox{-}e\plf{tɔ́\mbox{-}s\mbox{-}e}}; \intxt{tɔ́hw\mbox{-}\APPL\plf{tɔ́\mbox{-}s\mbox{-}\APPL}}; \wordng{I’w}{tófʷ<en>}; \wordng{Mj}{[ʔa]tɔ́hw\mbox{-}ej}; \intxt{[ʔa]tɔ́h\mbox{-}\APPL} (\citealt{ND09}: 152; \citealt{AG83}: 165; \citealt{JC18}) {\sep} \wordng{PW}{*tóxʷ\mbox{-}ejʰ} > \wordng{LB}{tufʷ\mbox{-}ej}; \wordng{Vej}{tohʷ\mbox{-}ej} [1]; \wordng{’Wk}{\mbox{-}<ʔa>tóxʷ\mbox{-}eʔ} [1] (\citealt{VN14}: 327; \citealt{AFG067}: 215; \citealt{KC16}: 16)

\dicnote{The loss of the word-final \intxt{*\mbox{-}jʰ} in ’Weenhayek is irregular. A \textit{j\mbox{-}}less form is also attested for Vejoz by \citeauthor{ViñasUrquiza1974} (1974:108, \textit{tohʷ\mbox{-}e}), which could be a mistranscription.}%1

\lit{\citealt{RJH15}: 240; \citealt{PVB02}: 145 (no reconstruction)}

\PMlemma{{\wordnl{*túku(ˀ)(t)s}{ant}}}

\word{Ni}{tukus}{ant; Bolivian} \citep[279]{JS16} {\sep} \wordng{PCh}{*túkus} > \word{Ijw}{tókis}{ant; soldier}; \wordng{I’w}{tókis}; \word{Mj}{tʊ́kis}{ant; soldier} (\citealt{JC14b}: 94, fn. 25; \citealt{ND09}: 152; \citealt{AG83}: 165; \citealt{JC18})

\lit{\citealt{EN84}: 42, 43 (\intxt{*thus}); \citealt{LC-VG-07}: 15}

\PMlemma{{\wordnl{*túsu(ˀ)(t)s}{lesser yellowlegs}}}

\word{Ni}{tusus}{lesser yellowlegs; solitary sandpiper} \citep[281]{JS16} {\sep} \wordng{PCh}{*\mbox{túsus}} > \wordng{Ijw}{tóxsus} \citep[153]{ND09} {\sep} \wordng{PW}{*túsus} > \wordng{LB}{teses}; \word{’Wk}{túsus}{kind of bird (small, white)} (\citealt{CS-FL-PR-VN13}; \citealt{KC16}: 426)

\PMlemma{{\wordnl{*tux}{to eat (vt.)}}}

\wordng{Mk}{tux / \mbox{-}ɬux} \citep[344]{AG99} {\sep} \wordng{Ni}{tux} \citep[280]{JS16} {\sep} \wordng{PCh}{*[ʔi]túʍ} > \wordng{Ijw}{[ʔi]tʲúʍ / \mbox{-}tóʍ}; \wordng{I’w}{[i]tʲúh / \mbox{-}tóh}; \wordng{Mj}{[ʔi]tʲúʍ / \mbox{-}tʊ́ʍ} [1] (\citealt{JC14b}: 87; \citealt{ND09}: 114; \citealt{AG83}: 42, 166; \citealt{JC18}) {\sep} \wordng{PW}{*tuxʷ} > \wordng{LB}{tefʷ}; \wordng{Vej}{tuhʷ}; \wordng{’Wk}{tuxʷ} (\citealt{VN14}: 237; \citealt{VU74}: 77; \citealt{KC16}: 420)

\dicnote{In Chorote, this verb now receives a non-etymological third-person prefix \intxt{ʔi\mbox{-}} (rather than zero).}%1

\gc{Possibly related to \word{Proto-Guaicuruan}{*\mbox{-}eˀlíko}{to eat} (\citealt{PVB13b}, \#214).}

\lit{\citealt{EN84}: 39 (\intxt{*thu}); \citealt{PVB02}: 143 (\intxt{*\mbox{-}tux})}

\PMlemma{{\wordnl{*\mbox{-}ˀtxoˀk \recind *\mbox{-}ˀtxóˀk\plf{*\mbox{-}ˀtxók\mbox{-}owot}}{uncle}}}

\wordng{Mk}{\mbox{-}txoˀk} [1], \textit{\mbox{-}txok\mbox{-}its} [2] \citep[287]{AG99} {\sep} \wordng{Ni}{\mbox{-}ˀtxoˀk\plf{\mbox{-}ˀtxok\mbox{-}oβot}} [3 4] \citep[271]{JS16} {\sep} \wordng{PCh}{*\mbox{-}<i>tók\plf{*\mbox{-}<i>tók\mbox{-}owot}} [5] > \wordng{Ijw}{\mbox{-}tʲók\plf{\mbox{-}tʲókʲ\mbox{-}owot}} [6]; \wordng{Mj}{\mbox{-}(<i>)t(ʲ)ók\plf{\mbox{-}tɔ́ʔ\mbox{-}oj}} [2 7] (\citealt{ND09}: 126; \citealt{JC18}) {\sep} \wordng{PW}{*\mbox{-}<wi>thokʷ} [5] > \wordng{LB}{\mbox{-}<wi>tʰuq} [8]; \wordng{’Wk}{\mbox{-}<wí>tʰok} (\citealt{VN14}: 194; \citealt{KC16}: 102)

\dicnote{The presence of a preglottalized coda in the Maká reflex is inferred based on the Nivaĉle cognate; it is not attested in our sources that distinguish between plain and preglottalized codas.}%1

\dicnote{The plural forms attested in Maká and Manjui are non-etymological.}%2

\dicnote{In the Chishamnee Lhavos dialect, \textit{x} is lost: \textit{\mbox{-}toˀk}. }%3

\dicnote{The onset of the Nivaĉle nouns carries the feature [+constricted glottis], as it induces glottalization in the preceding vowel \citep[193]{AnG15}.}%4

\dicnote{The origin of the elements \textit{*\mbox{-}<i>\mbox{-}} in Chorote and \intxt{*\mbox{-}<wi>\mbox{-}} in Wichí is unclear.}%5

\dicnote{\citet[126]{ND09} claims the Iyojwa’aja’ form to be a Iyo’awujwa’ loan, but it is not clear on what grounds.}%6

\dicnote{The Manjui plural form is non-etymological.}%7

\dicnote{Lower Bermejeño Wichí appears to have irregularly lost labialization of the final consonant. Alternatively, it could be a mistranscription or a typo on \cits{VN14} part, as only one attestation of this word is available.}%8

\gc{Compare \word{Proto-Qom}{*\mbox{-}tesóqoʔ}{uncle} (cf. \citealt{PVB13b}, \#567).}

\lit{\citealt{EN84}: 10, 25 (\intxt{*ithɔ́uk}); \citealt{LC-VG-07}: 16}

\PMlemma{{\intxt{*\mbox{-}t’é\mbox{-}l} [1]\gloss{tears} (\intxt{plurale tantum})}}

\wordng{Mk}{\mbox{-}t’i\mbox{-}l} \citep[345]{AG99} {\sep} \wordng{Ni}{\mbox{-}t’e<k͡l>\mbox{-}is} \citep[286]{JS16} {\sep} \wordng{PCh}{*\mbox{-}t’é<l>\mbox{-}is} [1] > \wordng{Ijw}{\mbox{-}t’έl\mbox{-}is} \citep[126]{ND09}

\dicnote{This word appears to be an ancient compound of \word{PM}{*\mbox{-}teʔ}{eye} and \wordnl{*\mbox{-}ʔí\pla{l}}{liquid}. Chorote and Wichí also use a more transparent compound of the reflexes of these roots, cf. Iyojwa’aja’ \textit{\mbox{-}tá\mbox{-}te t’éʔ\pl{ˀl}}, ’Weenhayek \textit{\mbox{-}t\mbox{-}té\mbox{-}t’iʔ\pl{ɬ}} (lit.\gloss{liquid of the eye}; \citealt{ND09}: 155; \citealt{KC16}: 100). Note that these compounds go back to PChW \textit{*\mbox{-}t(a)\mbox{-}te t’\mbox{-}íʔ\pla{l}} and thus cannot reflect \wordng{PM}{*\mbox{-}t’e\mbox{-}l \recind *\mbox{-}t’é\mbox{-}l}.}%1

\dicnote{Nivaĉle and Chorote have fossilized the erstwhile plural suffix \intxt{*\mbox{-}l >} \wordng{Ni}{\mbox{-}k͡l}, \wordng{Ijw}{\mbox{-}l} as a part of the stem.}%2

\gc{Possibly related to \word{Proto-Guaicuruan}{*\mbox{-}át’iʔ}{tear} (\citealt{PVB13b}, \#128), if only the Proto-Guaicuruan reconstruction is correct. However, there is evidence that the Proto-Guaicuruan form should be reconstructed as \intxt{*\mbox{-}át’it} instead. The stem-final stop would account for \intxt{di} in the Kadiwéu reflex \intxt{\mbox{-}atːiːdi} and for the stem-final consonant in Mocoví, seen in the 2\textsc{sg} form \intxt{ɾ\mbox{-}atʃitʃ\mbox{-}iʔ} and in the 2\textsc{pl} form \intxt{ɾ\mbox{-}atʃiɾ\mbox{-}i}.}

\lit{\citealt{AnG15}: 253}

\PMlemma{{\wordnl{*\mbox{-}Ct’éh}{grandmother} / \wordnl{*\mbox{-}qá\mbox{-}Ct’éh}{mother-in-law}; \wordnl{*\mbox{-}Ct’éˀk}{grandfather} / \wordnl{*\mbox{-}qá\mbox{-}Ct’eˀk}{father-in-law} [1 2]}}

\wordng{Ni}{\mbox{-}kt’e\pl{j} / \mbox{-}ka\mbox{-}kt’e\pl{j}}; \intxt{\mbox{-}kt’eˀtʃ\plf{\mbox{-}ktʃe\mbox{-}βot} / \mbox{-}ka\mbox{-}kt’etʃ\plf{\mbox{-}ka\mbox{-}ktʃe\mbox{-}βot}} \citep[90, 182, 495]{LC20} {\sep} \wordng{PCh}{*\mbox{-}nt’éh\plf{*\mbox{-}nt’é\mbox{-}ewot} / *\mbox{-}qá\mbox{-}nt’eh}; \intxt{*\mbox{-}nt’ék\pla{åwot} / *\mbox{-}qá\mbox{-}nt’ek} > \wordng{Ijw}{\mbox{-}nt’ɛ́h\plf{\mbox{-}nt’ɛ́\mbox{-}wot} / \mbox{-}ká\mbox{-}nt’e}; \intxt{\mbox{-}nt’ɛ́k\plf{\mbox{-}nt’ɛ́kʲ\mbox{-}awot} / \mbox{-}ká\mbox{-}nt’ek}; \wordng{Mj}{\mbox{-}(i)nt’ɛ́ʔ\plf{\mbox{-}(i)nt’ɛ́\mbox{-}(ɛ)wat} / \mbox{-}ká\mbox{-}nt’eʔ\pl{wot \recind \mbox{-}wat}}; \intxt{\mbox{-}(i)nt'ɛ́k\plf{\mbox{-}(i)nt'ɛ́kʲ\mbox{-}ewat \recind \mbox{-}(i)nt'ɛ́kʲ\mbox{-}owat} / \mbox{-}ká\mbox{-}nt’ek\plf{\mbox{-}ká\mbox{-}nt’ekʲ\mbox{-}ewot \recind \mbox{-}ká\mbox{-}nt’ekʲ\mbox{-}ewat}} (\citealt{JC14a}; \citealt{JC18})

\dicnote{The root-initial consonant cannot be reconstructed at present: Nivaĉle points to \intxt{*l}, \intxt{*k}, or \intxt{*q}, whereas Chorote points to \intxt{*n}. In Chorote, this is the only relational noun that starts with a consonant cluster, suggesting that it may have undergone a unique sound change due to the position being unparalleled.}%1

\dicnote{Maká and Wichí have similar but obviously unrelated roots: \word{Mk}{\mbox{-}wket\pl{its}}{grandfather} / \wordnl{\mbox{-}qe\mbox{-}wket\pl{its}}{father-in-law}, \wordnl{\mbox{-}wket\mbox{-}iʔ\pl{j}}{grandmother} / \wordnl{\mbox{-}qe\mbox{-}wket\mbox{-}iʔ\pl{j}}{mother-in-law} \citep[165, 310]{AG99}; \word{PW}{*\mbox{-}kʲǻtih}{grandfather} / \wordnl{\mbox{-}qá\mbox{-}kʲåtih}{father-in-law} > \word{LB}{\mbox{-}tʃoti}{grandfather, father-in-law}; \wordng{’Wk}{\mbox{-}kʲǻtih / \mbox{-}qá\mbox{-}kʲåtih} (\citealt{VN14}: 194; \citealt{KC16}: 63, 84). It is possible that the Maká and Wichí forms are partial cognates between themselves, but the vowels do not match.}%2

\lit{\citealt{EN84}: 23 (\intxt{*theuk}); \citealt{LC-VG-07}: 15}

\PMlemma{{\wordnl{*\mbox{-}t’íleʔ\pla{jʰ}}{rheum} [1]}}

\wordng{Mk}{\mbox{-}t’iliʔ\pl{j}} \citep[345]{AG99} {\sep} \wordng{Ni}{\mbox{-}t’ik͡le\pl{j}} \citep[287]{JS16} {\sep} \wordng{PCh}{*\mbox{-}t’íle\mbox{-}} > \wordng{Ijw}{\mbox{-}t’élʲ<ak>\pl{is}} [2]; \wordng{I’w}{\mbox{-}téli<jes>}; \wordng{Mj}{\mbox{-}t’éili<jees>} (\citealt{ND09}: 126; \citealt{AG83}: 164; \citealt{JC18})

\dicnote{This is likely a compound of the root \wordnl{*\mbox{-}t’i\mbox{-} \recind *\mbox{-}t’í\mbox{-}}{eye (in compounds)}, preserved in \word{Nivaĉle}{\mbox{-}t’i\mbox{-}påk͡lå\pl{s}}{eyebrow}, \wordnl{\mbox{-}t’i\mbox{-}βaʃ\plf{\mbox{-}t’i\mbox{-}βʃa\mbox{-}s}}{inner corner of the eye} (\citealt{JS16}: 287, 288).}%1

\dicnote{The Iyojwa’aja’ reflex seems to have been influenced by \wordnl{\mbox{-}ʔilʲák}{pus}.}%2

\PMlemma{{\intxt{*\mbox{-}t’ij \recind *\mbox{-}t’íj} [1]\gloss{to move (intr.), to infect}, \textsc{caus} \intxt{*[ji]t’ij\mbox{-}hat}}}

\wordng{Ni}{[βa]t’ij}, \intxt{[ji]t’ij\mbox{-}xat} \citep[288]{JS16} {\sep} \wordng{PCh}{*[ʔi]t’ijʔ}, \intxt{*[ʔi]t’ihj\mbox{-}at} > \wordng{I’w}{\mbox{-}téj} [2], —; \wordng{Mj}{[ʔi]t’ijʔ / \mbox{-}t’eiʔ}, \intxt{[ʔi]t’ihj\mbox{-}et / \mbox{-}t’eihj\mbox{-}et} (\citealt{AG83}: 163; \citealt{JC18})

\dicnote{The correspondence \sound{Ni}{t’} \recind \sound{PCh}{*t’} could in principle also go back to \sound{PM}{*ɬ’}. We reconstruct \sound{PM}{*t’} because \sound{PM}{*ɬ’} is not known to have occurred tautomorphemically.}%1

\dicnote{The plain stop in the Iyo’awujwa’ form attested by \citet{AG83} must be a mistranscription.}%2

\PMlemma{{\wordnl{*t’isåʔ \recind *t’isǻʔ\pla{l}}{cream-backed woodpecker\species{Campephilus leucopogon}}}}

\wordng{Mk}{t’isaʔ\pl{l}} \citep[345]{AG99} {\sep} \word{Ni}{t’isåʔ\pl{k}}{woodpecker sp.} \citep[287]{JS16} {\sep} \wordng{PCh}{*t’isǻʔ\pl{l}} > \wordng{Ijw}{t’isʲáʔ\pl{ˀl}} \citep[155]{ND09}

\PMlemma{{\textit{*\mbox{-}t’ox \recind *\mbox{-}t’óx} [1]\gloss{aunt}}}

\wordng{Ni}{\mbox{-}t’ox}, \textit{\mbox{-}t’ox\mbox{-}oβot} \citep[288]{JS16} {\sep} \wordng{PCh}{*\mbox{-}<i>t’óh} [2] > \wordng{Mj}{\mbox{-}(<i>)t(ʲ)’óh} \citep{JC18} {\sep} \wordng{PW}{*\mbox{-}<wi>t’oχ} [2] > \wordng{LB}{\mbox{-}<wi>t’uχ}; \wordng{Vej}{\mbox{-}<wi>t’oh(ʷ)}, \textit{\mbox{-}<wi>t’oh\mbox{-}ɬajis}; \wordng{’Wk}{\mbox{-}<wí>t’oxʷ} (\citealt{VN14}: 194; \citealt{VU74}: 81; \citealt{MG-MELO15}: 69; \citealt{KC16}: 102)

\dicnote{The correspondence \wordng{Ni}{t’} \recind \wordng{PCh/PW}{*t’} could in principle also go back to \wordng{PM}{*ɬ’}. We reconstruct \wordng{PM}{*t’} because the root is evidently related to \word{PM}{*\mbox{-}ˀtxoˀk \recind *\mbox{-}ˀtxóˀk}{uncle} and because \wordng{PM}{*ɬ’} is not known to have occurred tautomorphemically.}%1

\dicnote{The origin of the elements \textit{*\mbox{-}<i>\mbox{-}} in Chorote and \intxt{*\mbox{-}<wi>\mbox{-}} in Wichí is unclear.}%2

\lit{\citealt{EN84}: 10, 40 (\intxt{*ithɔ́})}

\PMlemma{{\wordnl{*tʼún}{hard}}}

\wordng{Mk}{t’un\pl{its}} \citep[346]{AG99} {\sep} \wordng{Ni}{tʼun}\gloss{hard; cookie} \citep[290]{JS16} {\sep} \wordng{PCh}{*tʼún} > \wordng{Ijw}{t’óˀn} \citep[156]{ND09} {\sep} \wordng{PW}{*t’ún} > \wordng{LB}{t’en}; \wordng{Vej}{t’un}; \wordng{’Wk}{t’ún̥} (\citealt{VN14}: 178; \citealt{VU74}: 78; \citealt{KC16}: 450)

\PMlemma{{\textit{*tsåhǻq} [1] (\intxt{*\mbox{-}its})\gloss{\textit{chajá} bird}}}

\wordng{Mk}{tsahaq} [1] (\intxt{\mbox{-}its}) \citep[347]{AG99} {\sep} \wordng{PCh}{*såhǻk\plf{*såhǻq\mbox{-}es \recvar *såhǻq\mbox{-}is}} > \wordng{Ijw}{sahák}; \wordng{I’w}{sahák\pl{is}}; \wordng{Mj}{sahák\pl{es \recind \mbox{-}ɪs}} (\citealt{ND09}: 144; \citealt{AG83}: 157; \citealt{JC18}) {\sep} \wordng{PW}{*tsåhǻq} > \wordng{LB}{tsohoq}; \wordng{’Wk}{tsåhǻq} (\citealt{VN14}: 50; \citealt{KC16}: 463)

\dicnote{The reconstruction \textit{*tsåhǻ(ˀ)q} is ruled out because the Maká reflex is attested with a plain coda in \citet[55]{JB81}.}%1

\gc{Likely related to \word{Proto-Guaicuruan}{*t’aqaqa}{\textit{chajá} bird} (\citealt{PVB13b}, \#553), whence \word{Toba–Qom}{taqaq}{id.} (\citealt{PC-AP-09}: 251).}

\lit{\citealt{PVB02}: 144 (\intxt{*tsʌχʌq})}

\PMlemma{{\wordnl{*tsänúˀk}{duraznillo\species{Ruprechtia triflora}}}}

\wordng{Ni}{tsanuˀk\plf{tsanku\mbox{-}j}} \citep[292]{JS16} {\sep} \wordng{PCh}{*sinúk} > \word{Ijw}{sinʲúk}{a tree similar to \textit{Ziziphus mistol} but thinner}; \wordng{Mj}{ʃinʲúk} (\intxt{\mbox{-}ij}) (\citealt{ND09}: 145; \citealt{JC18}) {\sep} \wordng{PW}{*tsinúkʷ} > \wordng{LB}{tsinekʷ}; \wordng{Southeastern (Salta)}{tʃinekʷ} [1]; \wordng{Vej}{tsinuk}; \wordng{’Wk}{tsinúk} ( \citealt{CS08}: 59; \citealt{MS14}: 320; \citealt{VU74}: 55; \citealt{MG-MELO15}: 19; \citealt{KC16}: 465)

\dicnote{The affricate \intxt{tʃ} in Southeastern Wichí, as attested by \citet[320]{MS14}, is irregular.}%1

\lit{\citealt{EN84}: 14, 49 (\intxt{*tsajn\mbox{-}úk}); \citealt{LC-VG-07}: 21}

\PMlemma{{\textit{*\mbox{-}tséwte(ʔ)\pla{jʰ}}\gloss{tooth}}}

\wordng{Ni}{\mbox{-}tseβte\pl{j}} \citep[294]{JS16} {\sep} \wordng{PW}{*\mbox{-}tsóte\pla{jʰ}} > \wordng{LB}{\mbox{-}tsute}; \wordng{Vej}{\mbox{-}tsote}; \wordng{’Wk}{\mbox{-}tsóteʔ\pl{ç}} (\citealt{JB09}: 39; \citealt{VU74}: 55; \citealt{KC16}: 100)

\PMlemma{{\wordnl{*tséχ\mbox{-}\APPL}{full (e.g. a river)}}}

\word{Ni}{tsex\mbox{-}\APPL}{full, abundant} \citep[293]{JS16} {\sep} \wordng{PCh}{*\mbox{-}sáh} [1]\gloss{to rise (of water)} > \wordng{Ijw}{[ʔi]sʲéh / \mbox{-}sáh}; \wordng{Mj}{[ʔa]sáh} (\citealt{ND09}: 111; \citealt{JC18}) {\sep} \wordng{PW}{*tsáχ\mbox{-}\APPL} > \wordng{’Wk}{tsáx\mbox{-}\APPL}\gloss{voluminous} \citep[63]{KC16}

\dicnote{In Chorote, this verb now receives non-etymological third-person prefixes \textit{ʔi\mbox{-}} or \textit{ʔa\mbox{-}} (rather than zero).}%1

\PMlemma{{\intxt{*tsijáʔ \recvar *ts’ijáʔ} [1]\gloss{caracara\species{Milvago sp.}}}}

\word{Mk}{tsijeʔ}{chimango caracara\species{Milvago chimango}; yellow-headed caracara\species{Milvago chimachima}; black-collared hawk\species{Busarellus nigricollis}} (\citealt{JB81}: 58) {\sep} \word{PW}{*ts’ijáʔ}{chimango caracara\species{Milvago chimango}} > \wordng{LB}{ts’ija} [2]; \wordng{LB}{ts’ijáʔ} (\citealt{VN14}: 157; \citealt{CS-FL-PR-VN13}; \citealt{KC16}: 470)

\dicnote{The Maká reflex points to \wordng{PM}{*tsijáʔ}, the Wichí one to \intxt{*ts’ijáʔ}.}%1

\dicnote{The Lower Bermejeño Wichí reflex unexpectedly lacks the root-final glottal stop.}%2

\PMlemma{{\wordnl{*tsiwáɬqoɬ}{little nightjar\species{Setopagis parvula}}}}

\wordng{Mk}{tsiwoɬqoɬ} (\citealt{JB81}: 61) {\sep} \wordng{PW}{*tsiwáɬqoɬ} > \wordng{LB}{tsiwaɬkʷuɬ} [1]; \wordng{’Wk}{\mbox{siwáɬqoɬ}} [2] (\citealt{CS-FL-PR-VN13}; \citealt{KC16}: 330)

\dicnote{The Lower Bermejeño Wichí reflex, as attested by \citet{CS-FL-PR-VN13}, unexpectedly shows \intxt{kʷ} instead of \intxt{q}.}%1

\dicnote{The root-initial fricative in the ’Weenhayek reflex is irregular. It is also seen in the dialectal reflexes attested in \citet[79, 80]{RL16}, \intxt{siwaɬkoɬ \recind suwaɬkoɬ}; it is unknown whether these forms are representative of Guisnay or Vejoz.}%2

\PMlemma{{\textit{*tsóɸa} (fruit)\gloss{\textit{Maytenus vitis\mbox{-}idaea}}; \textit{*tsóɸa\mbox{-}taχ} (fruit); \textit{*tsóɸa\mbox{-}ta\mbox{-}(ju)ˀk} (tree)\gloss{\textit{Lycium americanum}}}}

\wordng{Mk}{tsofe\mbox{-}taχ}; \textit{tsofe\mbox{-}te\mbox{-}ˀk\plf{tsofe\mbox{-}te\mbox{-}ket}} \citep[349]{AG99} {\sep} \wordng{Ni}{tsoɸ\mbox{-}tax\plf{tsoɸ\mbox{-}ta\mbox{-}s}}; \intxt{tsoɸ\mbox{-}ta\mbox{-}juk\plf{tsoɸta\mbox{-}ku\mbox{-}j}} [2]\gloss{bush sp.} \citep[297]{JS16} {\sep} \wordng{PCh}{\mbox{*sóhwaʔ}}\gloss{\textit{Maytenus vitis\mbox{-}idaea}} > \wordng{Ijw}{sɔ́hwaʔ}; \wordng{I’w}{sóhwaʔ}; \wordng{Mj}{sɔ́hwaʔ \recind sɔ́hwoʔ} (\citealt{ND09}: 147; \citealt{GS10}: 187; \citealt{JC18}) {\sep} \wordng{PW}{*tsóxʷa}; \textit{*tsóxʷa\mbox{-}t\mbox{-}ukʷ}\gloss{\textit{Lycium nodosum}} > \wordng{Southeastern (Salta)}{tsufʷa}; \wordng{Vej}{tsohʷa} (no gloss); \wordng{’Wk}{tsóxʷaʔ}; \textit{tsóxʷa\mbox{-}t\mbox{-}uk} (\citealt{MS14}: 343; \citealt{MG-MELO15}: 74; \citealt{KC16}: 466)

\dicnote{The preglottalized coda in the Maká suffix for tree names is attested elsewhere \citep[7]{PMA}.}%1

\dicnote{The syncope of the vowel of the medial syllable is irregular in Nivaĉle.}%2

\PMlemma{{\wordnl{*tsoˀm \recind *tsóˀm}{plush-crested jay\species{Cyanocorax chrysops}} [1]}}

\wordng{Mk}{tsoˀm\plf{tsom\mbox{-}its}} (\citealt{AG99}: 349; \citealt{JB81}: 64) {\sep} \wordng{PCh}{*sóˀm} > \wordng{Mj}{sɔ́ˀm} (\citealt{JC18})

\dicnote{\word{Ni}{tsum}{plush-crested jay\species{Cyanocorax chrysops}} \citep[506]{LC20} is similar to these forms, but its initial consonant is the only segment that shows a regular correspondence with the Maká and Manjui forms.}%1

\PMlemma{\wordnl{*(\mbox{-})tsútsuh}{grandfather}}

\word{Ni}{tsutsu}{grandfather, old man (possibly vocative)} \citep[495]{LC20} {\sep} \wordng{PCh}{*\mbox{-}súsuh} > \wordng{Mj}{\mbox{-}sʊ́su} [2] \citep{JC18}

\dicnote{There is also a similar form \wordng{Ni}{tʃutʃu}, used in the children’s language \citet[493]{LC20}.}%1

\dicnote{There is also an absolute form \wordng{Mj}{tʊ́tʲu \recind tútʲu}, possibly associated with the children’s language.}%1

\PMlemma{{\wordnl{*ts’áts’ih\plf{*ts’áts’i\mbox{-}l}}{rufous hornero}}}

\wordng{Mk}{ts’its’i\pl{l}} [2] \citep[351]{AG99} {\sep} \wordng{Ni}{ts’ats’i\pl{k}} \citep[301]{JS16} {\sep} \wordng{PCh}{*sát’ih} [3] > \wordng{Ijw}{sát’i} (\intxt{\mbox{-}his}); \wordng{Mj}{sát’i\pl{waʔ}} (\citealt{JC14b}: 90; \citealt{ND09}: 145; \citealt{JC18}) {\sep} \wordng{PW}{*tats’i} [4] > \wordng{LB/Vej}{tats’i} [5]; \wordng{’Wk}{táts’iʔ} (\citealt{VN14}: 50; \citealt{VU74}: 77; \citealt{MG-MELO15}: 22; \citealt{KC16}: 386)

\dicnote{The plural form is reconstructed based on the evidence of Maká and Nivaĉle. It is thus technically reconstructible only for Proto-Maká–Nivaĉle.}%1

\dicnote{The expected reflex in Maká would be \textit{*ts’ets’i}.}%2

\dicnote{The Chorote reflex shows an irregular dissimilation: \textit{*ts’…ts’ > *ts…ts’ > *s…t’}.}%3

\dicnote{The Wichí reflex shows an irregular dissimilation: \textit{*ts’…ts’ > *ts…ts’ > *t…ts’}.}%4

\dicnote{\citet[77]{VU74} attests \wordng{Vej}{t’ats’i}, whose initial glottalized consonant may be a mistranscription.}%5

\PMlemma{{\wordnl{*(t)s’óˀts}{milk}}}

\wordng{Ni}{(\mbox{-})ts’oˀs\plf{(\mbox{-})ts’os\mbox{-}ik}} [1]; \wordnl{ts’ots\mbox{-}i}{to have milk} \citep[303]{JS16} {\sep} \wordng{PCh}{*\mbox{-}qá <i>t’ós} [2 3] > \wordng{Ijw}{\mbox{-}ká\mbox{-}tʲ’ós}; \wordng{Mj}{\mbox{-}ká\mbox{-}itʲ’ós\plf{\mbox{-}ká\mbox{-}itʲ’óʃ\mbox{-}is}} (\citealt{ND09}: 121; \citealt{JC18}) {\sep} \wordng{PW}{*ts’ós} > \wordng{Guisnay}{t’os} [2]; \wordng{’Wk}{ts’ós} (\citealt{RL16}: 94, 100; \citealt{KC16}: 470)

\dicnote{The Nivaĉle plural form is non-etymological, since it does not preserve the root-final /ts/, seen in the verb \wordnl{ts’ots\mbox{-}i}{to have milk}.}%1

\dicnote{The Chorote and Guisnay reflexes show an irregular dissimilation: \textit{*ts’…ts > *t’…ts > *t’…s}.}%2

\dicnote{We have no explanation for the element \intxt{*i} in the Chorote reflex.}%3

\PMlemma{{\wordnl{*[j]úɬå(ˀ)χ}{to be tired}}}

\wordng{Mk}{\mbox{-}uɬa(ˀ)χ} [1], \intxt{\mbox{-}uɬaχ\mbox{-}its}\gloss{breath} \citep[354]{AG99} {\sep} \wordng{Ni}{[j]uɬåx} \citep[306]{JS16} {\sep} \wordng{PCh}{*[j]úhlåh} > \wordng{I’w}{\mbox{-}óhula / \mbox{-}ó(h)la\mbox{-}}; \wordng{Mj}{[j]úhla} (\citealt{AG83}: 154, 188; \citealt{JC18})

\dicnote{The uncertainty regarding the coda in Maká is due to the fact that the singular form is not attested in our sources that distinguish between plain and preglottalized codas. The plural form is attested in the New Testament (Acts 17:25), but it is not revealing.}%1

\rej{\citet[307]{PVB13a} compares the Nivaĉle and Chorote terms to \wordng{Maká}{walχal}\gloss{idler} \citep[360]{AG99} and the Wichí term for\gloss{slow} (\wordng{PW}{*[j]íwaɬ}, whence \wordng{LB}{[j]iwaɬ}, \wordng{’Wk}{[j]íwaɬ\mbox{-}\APPL}\gloss{slow}; cf. \citealt{JB09}: 63; \citealt{KC16}: 549). This is untenable both phonologically and semantically.}

\largerpage[2]
\gc{\citet[307]{PVB13a} compares the Mataguayan root to \wordng{Proto-Guaicuruan}{*\mbox{-}ewe(ˀ)la}\gloss{to be tired} (VB~2013b, \#243), which is likely a spurious comparison.}

\lit{\citealt{PVB13a}: 307 (\wordnl{*\mbox{-}wʌɬʌh}{slow, tired})}\clearpage

\PMlemma{{\wordnl{*\mbox{-}úˀp\plf{*\mbox{-}úp\mbox{-}its}}{nest}}}

\wordng{Mk}{\third{ɬ\mbox{-}up}} [1] (\intxt{\mbox{-}its}) (\citealt{AG99}: 255; \citealt{PMA}: 22) {\sep} \wordng{Ni}{\mbox{-}uˀp}, \textit{\mbox{-}up\mbox{-}is} \citep[308]{JS16} {\sep} \wordng{PCh}{*\mbox{-}úp\pla{is}} > \wordng{Ijw}{\mbox{-}óp\pl{is}}; I’w~\third{hl\mbox{-}úp\pl{is}}; Mj~\third{hl\mbox{-}ʊ́p\pl{is}} (\citealt{ND09}: 132; \citealt{AG83}: 175; \citealt{JC18}) {\sep} \wordng{PW}{*\mbox{-}ɬ\mbox{-}úp\pla{is}} > \wordng{LB}{\mbox{-}ɬ\mbox{-}ep}; \wordng{Vej}{\mbox{-}ɬ\mbox{-}up}; \wordng{’Wk}{\mbox{-}ɬ\mbox{-}úp\pl{is}} (\citealt{VN14}: 170; \citealt{VU74}: 66; \citealt{KC16}: 76)

\dicnote{The Maká reflex unexpectedly lacks preglottalization in the coda in the singular form, as attested in \citet[22]{PMA}.}%1

\empr{\citet[306]{AF16} notes the similarity with \wordng{Enlhet}{ɬoːp}\gloss{pipe} (\citealt{EU-HK-97}: 230), but the similarity is obviously accidental.}

\lit{\citealt{EN84}: 21 (\intxt{*hlhnup’}); \citealt{LC-VG-07}: 20; \citealt{AF16}: 306; \citealt{AnG15}:~254}

\PMlemma{{\wordnl{*\mbox{-}uwa}{termite house}}}

\wordng{Ni}{\mbox{-}uβa\pl{k}} \citep[308]{JS16} {\sep} \wordng{PW}{*<ɬ>uwa} > \wordng{Vej}{ɬuwa}; \wordng{’Wk}{ɬuwaʔ} (\citealt{MG-MELO15}: 66; \citealt{KC16}: 239)

\gc{\citet[311]{PVB13a} compares the root with \wordng{Proto-Guaicuruan}{*a(ˀ)lo}\gloss{termite house} (\citealt{PVB13b}, \#119), which could be spurious.}

\lit{\citealt{EN84}: 50 (\intxt{*hlsewa}); \citealt{PVB13a}: 311 (\intxt{*ɬuwa})}

\PMlemma{{\wordnl{*\mbox{-}u(ʔ) \recind *\mbox{-}ú(ʔ)}{to throw, to push}; \wordnl{*n\mbox{-}u(ʔ) \recind *n\mbox{-}ú(ʔ)}{to throw oneself, to pass}}}

\word{Ni}{[j]uʔ}{to throw, to push}; \textit{n\mbox{-}uʔ}\gloss{to throw oneself} \citep[305]{JS16} {\sep} \word{PCh}{*[ʔi]<n>úʔ}{to pass} > \wordng{Ijw}{[ʔi]nʲúʔ / \mbox{-}nóʔ}; \word{I’w}{\mbox{-}nó}{to exit, to walk}; \wordng{Mj}{[ʔi]nʲúʔ / \mbox{-}nʊ́ʔ} (\citealt{JC14b}: 95; \citealt{ND09}: 106; \citealt{AG83}: 151; \citealt{JC18}) {\sep} \wordng{PW}{*[ʔi]<n>ú\mbox{-}\APPL} > \wordng{LB}{[ʔi]ne\mbox{-}\APPL}; \wordng{Vej}{\mbox{-}nu\mbox{-}\APPL}; \wordng{’Wk}{[ʔi]nú\mbox{-}\APPL} (\citealt{VN14}: 177; \citealt{VU74}: 69; \citealt{MG-MELO15}: 36; \citealt{KC16}: 279–282)

\lit{\citealt{EN84}: 13 (\wordnl{*nu}{to walk fast})}

\PMlemma{{\intxt{*wák’a(ʔ)} (fruit); \intxt{*wák’a\mbox{-}juˀk\plf{*wák’a\mbox{-}jku\mbox{-}jʰ}} (tree)\gloss{guayacán\species{Libidibia paraguariensis}}}}

\wordng{Mk}{wek’e\mbox{-}juˀk} [1], \textit{wek’e\mbox{-}jkw\mbox{-}i} \citep[366]{AG99} {\sep} \wordng{PCh}{*wák’a\mbox{-}juk}, \textit{*wák’a\mbox{-}jku\mbox{-}jʰ} > \wordng{Ijw}{(h)wákʲ’e\mbox{-}k} [2]; \wordng{I’w}{áe\mbox{-}jik \recind áʔa\mbox{-}jik \recind aʔi\mbox{-}jík}, \textit{áe\mbox{-}si\mbox{-}ʔ} [3]; \wordng{Mj}{ʔáʔa\mbox{-}jik} [3] (\citealt{ND09}: 133; \citealt{AG83}: 117; \citealt{GS10}: 187; \citealt{JC18}) {\sep} \wordng{PW}{*wákʲ’a(ʔ)}; \textit{*wákʲ’a\mbox{-}jukʷ}, \textit{*wákʲ’a\mbox{-}kʲu\mbox{-}jʰ} > \wordng{LB}{watʃa\mbox{-}jekʷ}, \textit{watʃa\mbox{-}tʃe\mbox{-}j} [4]; \wordng{Vej}{wåtʃ’a\mbox{-}juk} [5]; \wordng{’Wk}{wǻkʲ’åʔ}; \textit{wǻkʲ’å\mbox{-}juk} [5] (\citealt{VN14}: 192; \citealt{MG-MELO15}: 19; \citealt{KC16}: 475)

\dicnote{The preglottalized coda in the Maká suffix for tree names is attested elsewhere \citep[7]{PMA}.}%1

\dicnote{The Iyojwa’aja’ variant with \textit{hw\mbox{-}} is attested in \citet[133]{ND09}.}%2

\dicnote{In Iyo’awujwa’ and Manjui, \wordng{PCh}{*w} was irregularly lost.}%3

\dicnote{In Lower Bermejeño, the glottalization in \wordng{PW}{*kʲ’} is unexpectedly lost. \citet[61]{CS08} documents the unexpected form \textit{wotʃo\mbox{-}jekʷ}.}%4

\dicnote{Vejoz and ’Weenhayek \textit{å} is not the expected reflex of \sound{PW}{*a}.}%5

\PMlemma{{\wordnl{*wáqa(ˀ)ɬ}{to be fruitful, ready, ripe}, \textsc{caus~}\intxt{*[ʔi]wáq(a)ɬ\mbox{-}Vt}}}

\wordng{Ni}{βakaɬ / \mbox{-}βkaɬ}, \textsc{caus~}\intxt{[ji]βakɬ\mbox{-}it} (\citealt{JS16}: 311, 312; \citealt{LC20}: 316, 390) {\sep} \wordng{PCh}{*wǻqaɬ} [1] > \wordng{Ijw}{wákaɬ}; \wordng{Mj}{wákaɬ}; \textsc{caus~}\textit{*[ʔi]wǻqahl\mbox{-}at} > \wordng{Ijw}{[ʔi]wʲákahl\mbox{-}<an\mbox{-}it>}; \word{Mj}{[ʔi]jákahl\mbox{-}at / \mbox{-}wákahl\mbox{-}at}{to bring up, to adopt} (\citealt{ND09}: 116, 156; \citealt{JC18}) {\sep} \wordng{PW}{*wáq’aɬ} [2] > \wordng{LB}{waq’aɬ}; \wordng{Vej}{wak’aɬ}; \wordng{’Wk}{wáq’aɬ}; \textsc{caus~}\textit{*[ʔi]wáq’ɬ\mbox{-}at} [2] > \wordng{Vej}{\mbox{-}wakɬat}; \wordng{’Wk}{[ʔi]wáq’ɬat} (\citealt{VN14}: 50; \citealt{VU74}: 79; \citealt{KC16}: 477, 478)

\dicnote{The back vowel \intxt{*ǻ} in Chorote (reconstructed based on the Iyojwa’aja’ causative \intxt{[ʔi]wʲákahl\mbox{-}an\mbox{-}it}) does not match the evidence from Nivaĉle and Wichí.}%1

\dicnote{The glottalization in \sound{PW}{*q’} is irregular.}%2

\PMlemma{{\intxt{*wátå(ˀ)χ} (fruit); \intxt{*wáth(å\mbox{-}j)uˀk} (tree)\gloss{palo flojo (\intxt{Albizia inundata} or \textit{Enterolobium contortisiliquum})}}}

\wordng{Ni}{βåtåx}; \intxt{βåtxå\mbox{-}juk\plf{βåtxå\mbox{-}ku\mbox{-}j}} \citep[372]{JS16} {\sep} \wordng{PCh}{*wáht<uk>} > \wordng{Ijw}{(h)wátok} [1]\gloss{\textit{Enterolobium contortisiliquum}}; \word{I’w}{wáhtok}{\textit{Albizia inundata}}; \word{Mj}{wáhtuk\pl{ij}}{\textit{Albizia inundata}} (\citealt{ND09}: 133; \citealt{GS10}: 187; \citealt{JC18}) {\sep} \wordng{PW}{*wátoxʷ} > \wordng{Southeastern (Salta)}{watux}; \wordng{’Wk}{xʷátoxʷ} [1]\gloss{pacará} (\citealt{MS14}: 270; \citealt{KC16}: 164)

\dicnote{Iyojwa’aja’ and ’Weenhayek show reflexes of \intxt{*ɸ} instead of the expected \intxt{*w}.}%1

\PMlemma{{\wordnl{*\mbox{-}wǻˀk}{bad mood}}}

\wordng{Mk}{\mbox{-}wak\plf{\mbox{-}wah\mbox{-}aj}} \citep[360]{AG99} {\sep} \wordng{Ni}{\mbox{-}βåˀk} \citep[371]{JS16} {\sep} \wordng{PCh}{*\mbox{-}wǻk} > \wordng{Ijw}{\mbox{-}wák} \citep[127]{ND09} {\sep} \wordng{PW}{*\mbox{-}wǻkʷ} > \wordng{LB}{\mbox{-}wokʷ}; \word{Vej}{[te]wakʷ\mbox{-}aje}{to be in mad mood}; \wordng{’Wk}{\mbox{-}wǻk} (\citealt{VN14}: 161; \citealt{VU74}: 75; \citealt{KC16}: 101)

\dicnote{The Maká reflex unexpectedly lacks preglottalization in the coda in the singular form, as attested in the New Testament (e.g. Romans 9:22).}%1

\PMlemma{{\wordnl{*wäk}{all, each other}}}

\word{Mk}{weːk}{all} \citep[365]{AG99} {\sep} \word{Ni}{=βatʃ}{reciprocal}; \wordnl{\mbox{-}βatʃ}{reflexive} (\citealt{JS16}: 311; \citealt{LC20}: 172–173, 299) {\sep} \wordng{PCh}{*(\mbox{-})wék / *(\mbox{-})wek\mbox{-}áʔa…} > \word{Ijw}{wikʲ<éʔeji>}{all}, \wordnl{<hi>wέk \recind <hi>wέkʲ<eʔe>}{finally}; \word{I’w}{kʲ<éehe>}{all} [1] (\citealt{JC14b}: 83; \citealt{ND09}: 127, 157; \citealt{AG83}: 142) {\sep} \word{PW}{*\mbox{-}wek}{each other, completely} > \word{LB}{=wek}{each other}; \word{Vej}{\mbox{-}wek}{completely}; \wordng{’Wk}{\mbox{-}wek}; \wordnl{*[ʔí]wek}{to be together, close to each other} > \wordng{LB}{[ʔi]wek}; \wordng{’Wk}{[ʔí]wek} (\citealt{VN14}: 247; \citealt{VU74}: 80; \citealt{KC16}: 482)

\dicnote{The loss of \wordng{PCh}{*we\mbox{-}} in Iyo’awujwa’ is irregular.}%1

\gc{\citet[319]{PVB13a} compares it to the Proto-Guaicuruan “total quantifier” \textit{*\mbox{-}ʔawéʔke \recind *\mbox{-}t’awéʔke} (VB~2013b, \#720; a suffix found in demonstratives). Alternatively, it could be related to \wordng{Proto-Guaicuruan}{*\mbox{-}ʔake}\gloss{each other} (\citealt{PVB13b}, \#722).}

\lit{\citealt{PVB13a}: 319 (\intxt{*wek})}

\PMlemma{{\wordnl{*\mbox{-}wä́ˀx\plf{*\mbox{-}w(ä)x\mbox{-}ájʰ}}{burrow; anus} [1]}}

\word{Ni}{\mbox{-}βaˀʃ\plf{\mbox{-}βaʃ\mbox{-}ajʰ}}{burrow} \citep[309]{JS16} {\sep} \wordng{PCh}{*\mbox{-}wéh}; \wordnl{*\mbox{-}wéh\mbox{-}k’alóʔ\pla{s}}{buttock} > \word{Ijw}{\mbox{-}wέh}{anus; container; cave}; \textit{\mbox{-}wέ\mbox{-}kʲ’oloʔ\pl{s}}; \wordng{I’w}{\mbox{-}wé\mbox{-}kʲalóʔ\pl{s}}\gloss{buttock}; \word{Mj}{\mbox{-}wéh\plf{\mbox{-}weh\mbox{-}éjh}}{anus}; \wordnl{\mbox{-}wé ʔelɔ́ʔ \recind \mbox{-}wé\mbox{-}ˀloʔ}{buttock} (\citealt{ND09}: 127; \citealt{AG83}: 169; \citealt{JC18}) {\sep} \wordng{PW}{*\mbox{-}wéχ}, \textit{*\mbox{-}wh\mbox{-}ájʰ}; \wordnl{*\mbox{-}wéχ\mbox{-}kʲ’alo\pla{s}}{buttock} > \word{LB}{\mbox{-}weχ}{back part, butt}; \wordnl{\mbox{-}wéχ\mbox{-}tʃ’alu}{buttock}; \word{Vej}{\mbox{-}weh}{opening, anus}; \textit{\mbox{-}weh tʃ’alo\pl{s}} [3]\gloss{buttock}; \wordng{’Wk}{\mbox{-}wéx\plf{\mbox{-}ʍ\mbox{-}áç}}; \intxt{\mbox{-}wéx\mbox{-}kʲaloʔ\pl{s}} (\citealt{VN14}: 153, 312; \citealt{JB09}: 61; \citealt{VU74}: 80; \citealt{KC16}: 102)

\dicnote{The original semantics of this root must have been that of\gloss{hole, opening}. It is likely that \wordng{PM}{*\mbox{-}wä́ˀx} is etymologically the second part of the opaque compounds \wordnl{*\mbox{-}tǻwäˀx}{cavity, abdominal cavity} and \wordnl{*kowäˀx / *\mbox{-}kówäˀx}{hole} (ChW).}%1

\dicnote{The term for\gloss{buttock} in Chorote and Wichí is a compound of \intxt{*\mbox{-}wä́ˀx} and \wordnl{*\mbox{-}k’alo(ʔ) \recind *\mbox{-}k’aló(ʔ)}{cheek}.}%2

\dicnote{\citet[60]{MG-MELO15} mistranscribe \intxt{tʃ’} as \intxt{tʃ} in the Vejoz reflex.}%3

\gc{Obviously related to \word{Proto-Guaicuruan}{*\mbox{-}ˀwVˀg}{hole} (\citealt{PVB13b}, \#644; cf. \citealt{PVB13a}: 311).}

\lit{\citealt{EN84}: 34 (\intxt{*wɛhn})}

\PMlemma{{\wordnl{*wé\mbox{-}\APPL}{be!}}}

\wordng{Ni}{βe\mbox{-}\APPL} \citep[146]{AF16} {\sep} \wordng{PCh}{*wé\mbox{-}\APPL} > \wordng{Ijw}{wέ\mbox{-}\APPL} \citep{JC14a}

\PMlemma{{\wordnl{*wijeʔ}{cactus\species{Bromelia serra}}}}

\wordng{Ni}{βijeʔ \recind jijeʔ\pl{k}} [1] (\citealt{JS16}: 363, 386) {\sep} \wordng{PCh}{*wijéʔ} > \wordng{Ijw}{(h)wijíʔ} [2]; \wordng{I’w}{fʷijíʔ \recind wijíʔ} [2]; \wordng{Mj}{wijíʔ} (\citealt{ND09}: 157; \citealt{AG83}: 130; \citealt{GS10}: 190; \citealt{JC18}) {\sep} \wordng{PW}{*ˀwuje(ʔ)} [3] > \wordng{LB}{huje} [4]; \wordng{Southeastern (Salta)}{wije} [5]; \wordng{Vej}{ˀwuje}; \wordng{’Wk}{ˀwujeʔ} (\citealt{CS08}: 60; \citealt{MS14}: 223–224; \citealt{MG-MELO15}: 19; \citealt{KC16}: 115)

\dicnote{The regular reflex \textit{βijeʔ\pl{k}} is used in the Chishamnee Lhavos dialect of Nivaĉle; in other dialects, the irregular variant with \textit{j\mbox{-}} is attested.}%1

\dicnote{In Iyojwa’aja’ and Iyo’awujwa’, the initial consonant has an irregular variant \intxt{hw/fʷ}. The absence of a final \intxt{ʔ} in \cits{AG83} attestation of the Iyo’awujwa’ reflex as \intxt{fʷijí} must be a mistranscription.}%2

\dicnote{The Wichí reflex is entirely irregular: the initial consonant is unexpectedly glottalized, and the vowel of the first syllable is reflected as \sound{PW}{*u}. The term may have been influenced by \word{PW}{*ˀwujés}{guinea pig}.}%3

\dicnote{The Lower Bermejeño reflex is entirely irregular. One would expect \intxt{*ˀweje}.}%4

\dicnote{The form \intxt{wije} is attested by \citet{MS14}, whose ethnobotanical fieldwork was carried out in Salta with speakers of the Southeastern dialect of Wichí. Although it formally matches the Nivaĉle and Chorote cognates (it could go back to \wordng{PW}{*wijeʔ}), it should probably be considered a slightly irregular reflex of \wordng{PW}{*ˀwujeʔ} (\intxt{*ˀweje} would be expected). Note that Suárez does not represent either glottalization in sonorants or word-final glottal stops in her transcription system, so the only irregularity is \intxt{i} instead of the expected \intxt{*e}.}%5

\lit{\citealt{EN84}: 48 (\intxt{*hwijéj})}

\PMlemma{{\wordnl{*\mbox{-}wháˀjaʔ}{spouse}; \wordnl{*[t]whaˀjä́\mbox{-}ˀj}{to marry} [1]}}

\wordng{Mk}{\mbox{-}wheˀjeʔ} (\intxt{\mbox{-}l \recind \mbox{-}ts}); \textit{[te]wheˀje\mbox{-}j} [1] \citep[164]{AG99} {\sep} \word{Ni}{\mbox{-}xaˀja\pl{s}}{spouse (before one has children)}; \intxt{[t]xaˀja\mbox{-}ˀj} (\citealt{AF16}: 133; \citealt{JS16}: 147, 271) {\sep} \wordng{PCh}{*\mbox{-}hwáˀjaʔ} > \word{Ijw}{\mbox{-}hwáˀje\mbox{-}hwa}{co-sibling-in-law}; \wordnl{*[tᵊ]hwaˀjé<jʔ>}{to marry} > \wordng{Ijw}{[ti]hwáˀji} [2]; \wordng{I’w}{\mbox{-}fʷají} [2 3]; \wordng{Mj}{[ti]hwaˀjíjʔ} (\citealt{JC14a}; \citealt{ND09}: 151; \citealt{AG83}: 128; \citealt{JC18}) {\sep} \wordng{PW}{*[t]wháje<j>} [3]\gloss{to marry} > \wordng{LB}{[t(a)]ʍajej}; \wordng{’Wk}{[t(a)]ʍájeʔ} [4] (\citealt{VN14}: 209, 272, 296; \citealt{KC16}: 388)

\dicnote{The glottalized palatal approximant in the Maká reflex is attested in the New Testament (e.g. Luke 2:36; Romans 16:3).}%1

\dicnote{The word-final \intxt{ʔ} is unexpectedly missing in Iyojwa’aja’ and Iyo’awujwa’.}%2

\dicnote{Wichí has irregularly lost the glottalization in \sound{PM}{*ˀj} > \sound{PW}{*j}. In Iyo’awujwa’, the corresponding consonant is also attested as \textit{j}, but this is likely a mistranscription.}%3

\dicnote{The expected reflex in ’Weenhayek would actually be \intxt{*[t(a)]ʍájejʔ}.}%4

\PMlemma{{\wordnl{*[ji]wó}{to do (\intxt{light verb})}; \wordnl{*wóʔ\mbox{-}ojʰ / *wó\mbox{-}…\mbox{-}ejʰ}{to look for}}}

\wordng{Mk}{woʔ\mbox{-}oj / wo\mbox{-}…\mbox{-}ij >}\gloss{to look for} \citep[380]{AG99} {\sep} \wordng{Ni}{βoʔ<oj>}\gloss{to look for} \citep[366]{JS16} {\sep} \wordng{PCh}{*[ʔi]wó / *\mbox{-}wó}\gloss{to do, to say so}, \wordnl{*[ʔi]wóʔ\mbox{-}ojʰ / *\mbox{-}wóʔ\mbox{-}ojʰ / *\mbox{-}wó\mbox{-}…\mbox{-}ejʰ}{to say, to want} > \wordng{Ijw}{[ʔi]jó / \mbox{-}wó}; \wordng{Mj}{[ʔi]jó / \mbox{-}wó}, \textit{[ʔi]jóʔ\mbox{-}oj / \mbox{-}wóʔ\mbox{-}oj}\gloss{to say, to want} (\citealt{JC14b}: 78; \citealt{ND09}: 116; \citealt{JC18}) {\sep} \wordng{PW}{*[ʔi]wó\mbox{-}} > \wordng{LB}{[ʔi]wu\mbox{-}}; \wordng{’Wk}{[ʔi]wó\mbox{-}} (\citealt{VN14}: 155; \citealt{JB09}: 46; \citealt{KC16}: 486–508)

\gc{\citet[305]{PVB13a} compares the Mataguayan verb for\gloss{to look for} with \wordng{Proto-Guaicuruan}{*\mbox{-}awiʔa}\gloss{to hunt} (absent from \citealt{PVB13b}), which is likely a spurious comparison.}

\lit{\citealt{PVB13a}: 305 (\intxt{*\mbox{-}woʔj})\gloss{to look for}}

\PMlemma{{\textit{*\mbox{-}wó\pla{ts}}\gloss{worm};~\third{*ɬ̩\mbox{-}wó}\gloss{mythological snake}}}

\wordng{Ni}{\mbox{-}βoʔ\pl{s}}; \textit{la\mbox{-}βoʔ} (\citealt{JS16}: 166, 363) {\sep} \wordng{PCh}{*\mbox{-}wóʔ\pla{s}} > \wordng{Ijw}{<ʔa>wóʔ\pl{s}}; \wordng{I’w/Mj}{\mbox{-}wóʔ\pl{s}} (\citealt{ND09}: 95; \citealt{AG83}: 170; \citealt{JC18}) {\sep} \wordng{PW}{*\mbox{-}wó\pla{s}}; \textit{*ɬ̩\mbox{-}wó}\gloss{mythological snake; rainbow} > \wordng{LB}{lawu}; \wordng{Vej}{<i>wo}\gloss{worm}; \textit{le\mbox{-}wo} [1]; \wordng{’Wk}{\mbox{-}woʔ}\gloss{wart}; \textit{<ʔi>wó\mbox{-}s}\gloss{worms}; \textit{la\mbox{-}wóʔ} (\intxt{\mbox{-}lis \recind \mbox{-}ɬajis}) (\citealt{MS14}: 77; \citealt{VN14}: 47; \citealt{VU74}: 61; \citealt{MG-MELO15}: 43; \citealt{KC16}: 43, 103, 222)

\dicnote{The noun is misprinted as \textit{le\mbox{-}we} in \citet[43]{MG-MELO15}.}%1

\lit{\citealt{AnG15}: 77}

\PMlemma{{\wordnl{*[ji]woˀm}{to throw}}}

\word{Mk}{[i]wuˀm}{to push, to throw} [1] (\citealt{AG99}: 380–381) {\sep} \word{PCh}{*[ʔi]wóm\mbox{-}\APPL}{to add} > \wordng{Ijw/Mj}{[ʔi]jóm\mbox{-}{\APPL} / \mbox{-}wóm\mbox{-}\APPL} (\citealt{ND09}: 116; \citealt{JC18}) {\sep} \wordng{PW}{*[ʔi]woˀm} > \word{LB}{[ʔi]wum\mbox{-}ɬi}{to share}; \word{Vej}{\mbox{-}wom}{to distribute}; \word{’Wk}{[ʔi]woˀm}{to throw, to abandon} (\citealt{VN14}: 402; \citealt{VU74}: 81; \citealt{MG-MELO15}: 37; \citealt{KC16}: 496)

\dicnote{The glottalized coda in the Maká reflex is attested in the New Testament (e.g. Luke 6:42; Matthew 7:5).}%1

\PMlemma{{\intxt{*wósitseχ} (fruit); \wordnl{*wósits\mbox{-}uˀk\plf{*wósits(e)\mbox{-}ku\mbox{-}jʰ}}{\textit{Prosopis nigra}}}}

\wordng{Mk}{ositsaχ}; \intxt{osits\mbox{-}uˀk\plf{osits\mbox{-}ik\mbox{-}wi}} [1 2] \citep[284]{AG99} {\sep} \wordng{Ni}{βaitsex}; \intxt{βaitse\mbox{-}juk\plf{βaitse\mbox{-}ku\mbox{-}j}} [3] \citep[313]{JS16} {\sep} \wordng{PCh}{*wósis\mbox{-}uk\plf{*wósis\mbox{-}ku\mbox{-}jʰ}} > \wordng{Ijw}{ʔisʲóxso}; \intxt{ʔisʲóxs\mbox{-}ok\pl{is}} [2 4]; \wordng{I’w}{wóxsisʲ\mbox{-}uk\plf{wóxsis\mbox{-}ki\mbox{-}ʔ}}; \wordng{Mj}{wóxʃiʃ\mbox{-}uk \recind wóxʃuʃ\mbox{-}uk} [5] (\citealt{ND09}: 111; \citealt{AG83}: 172; \citealt{JC18}) {\sep} \wordng{PW}{*\mbox{wósotsaχ}}; \intxt{*wósots\mbox{-}ukʷ} [5] > \wordng{LB}{wusutsaχ}, \intxt{wusuts\mbox{-}ekʷ} [6]; \wordng{Vej}{wosotsax}, \intxt{wosots\mbox{-}uk}; \wordng{’Wk}{\mbox{wósotsax}}; \intxt{wósots\mbox{-}uk} (\citealt{CS08}: 60; \citealt{VU74}: 81; \citealt{MG-MELO15}: 19; \citealt{KC16}: 503)

\dicnote{The absence of preglottalization in the term for the fruit in Maká is attested in a narrative by \citet[17]{unuuneiki}. The preglottalized coda in the Maká suffix for tree names is attested elsewhere \citep[7]{PMA}.}%1

\dicnote{The loss of \intxt{*w} in Maká and Iyojwa’aja’ is irregular.}%2

\dicnote{The Nivaĉle reflex is irregular: one would expect \intxt{*βositsex} and not \intxt{βaitsex}.}%3

\dicnote{The Iyojwa’aja’ reflex shows an irregular metathesis of \intxt{*o} and \intxt{*i}. The plural form is also not etymological.}%4

\dicnote{In Wichí and optionally in Manjui, the vowel of the second syllable irregularly becomes rounded.}%5

\dicnote{\citet[60]{CS08} actually gives \intxt{wusutasaχ}, \intxt{wusuts\mbox{-}ewk}, which look like typos.}%6

\PMlemma{{\textit{*\mbox{-}woʔ \recind *\mbox{-}wóʔ\pla{ts}}\gloss{expert, professional, owner; related to}}}

\wordng{Mk}{\mbox{-}woʔ\pl{ts}}\gloss{object that serves for X} \citep[221]{AG94} {\sep} \wordng{Ni}{\mbox{-}βoʔ\pl{s}} (\citealt{JS16}: 166, 363) {\sep} \wordng{PCh}{*\mbox{-}wóʔ\pla{s}} > \wordng{Ijw}{\mbox{-}wó\pl{s}} [1]; \wordng{Mj}{\mbox{-}wóʔ\pl{s}} (\citealt{JC14b}: 79, fn. 6; \citealt{ND09}: 127; \citealt{JC18}) {\sep} \wordng{PW}{*\mbox{-}woʔ \recind *\mbox{-}wóʔ\pl{s}} > \wordng{LB}{\mbox{-}wu\pl{s}}; \wordng{Vej}{\mbox{-}wo}; \wordng{’Wk}{\mbox{-}woʔ \recind \mbox{-}wóʔ\pl{s}} (\citealt{VN14}: 199; \citealt{VU74}: 81; \citealt{MG-MELO15}: 51; \citealt{KC16}: 103)

\dicnote{The absence of a word-final glottal stop in \cits{ND09} attestation of this noun must be a mistranscription.}%1

\PMlemma{{\textit{*\mbox{-}w(t)s’é\pla{l}}\gloss{belly}}}

\wordng{Ni}{\mbox{-}βtsʼeʔ\pl{k}} \citep[338]{JS16} {\sep} \wordng{PCh}{*\mbox{-}ts’éʔ\pla{l}} > \wordng{Ijw}{\mbox{-}ts’έʔ\pl{ˀl}}; \wordng{I’w}{\mbox{-}tséʔ\pl{l}}; \wordng{Mj}{\mbox{-}ts’έʔ\pl{l}} (\citealt{ND09}: 126; \citealt{AG83}: 167; \citealt{JC18}) {\sep} \wordng{PW}{*\mbox{-}tsʼé\pla{lʰ}} > \wordng{LB/Vej}{\mbox{-}tsʼe\pl{ɬ}}; \wordng{’Wk}{\mbox{-}tsʼéʔ\pl{ɬ}} (\citealt{VN14}: 147, 191; \citealt{VU74}: 56; \citealt{MG-MELO15}: 60, 61; \citealt{KC16}: 101)

\PMlemma{{\intxt{*wV́ˀχ\plf{*wV́\mbox{-}ts}} [1 2]\gloss{large, fat}}}

\word{Ni}{[ɬa]βåˀx}{to be of a size} \citep[371]{JS16} {\sep} \wordng{PCh}{*wúh}, \textit{*wú\mbox{-}s} > \wordng{Ijw}{wúh}, \textit{wú\mbox{-}s}; \wordng{I’w}{(\mbox{-})wúh}; \wordng{Mj}{wúh}, \textit{wú\mbox{-}s} (\citealt{ND09}: 157; \citealt{AG83}: 172; \citealt{JC18}) {\sep} \wordng{PW}{*wúxʷ}, \textit{*wú\mbox{-}s} > \wordng{LB}{wefʷ}; \wordng{Vej}{wúh}; \wordng{’Wk}{wúxʷ}, \textit{wú\mbox{-}s} (\citealt{VN14}: 357; \citealt{VU74}: 82; \citealt{KC16}: 509)

\dicnote{The vowel cannot be securely reconstructed at this time. Nivaĉle points to \sound{PM}{*å}, Chorote and Wichí to \intxt{*u}. The correspondence is similar to the one in \word{PM}{*\mbox{-}ˀwVˀɬ \recind *\mbox{-}ˀwV́ˀɬ}{to climb}.}%1

\dicnote{The plural form is reconstructed based on the evidence of Iyo’awujwa’, Manjui, and Wichí. It is thus technically reconstructible only for Proto-Chorote–Wichí.}%2

\empr{\citet[308]{AF16} compares the Mataguayan root with \word{Enlhet}{wah}{big} (\citealt{EU-HK-97}: 659).}

\PMlemma{{\wordnl{*ˀwátshan \recind *ˀwátsχan}{to be healthy, alive}}}

\word{Ni}{βatsxan}{to be healthy} \citep[357]{JS16} {\sep} \wordng{PCh}{*ˀwásaˀn} [1]\gloss{to be alive} > \wordng{Ijw}{ˀwáxsaˀn}; \word{Mj}{ˀwáxsaˀn}{to be green, living (plant)} (\citealt{ND09}: 163; \citealt{JC18}) {\sep} \wordng{PW}{*ˀwátshan}\gloss{to be green, blue, alive} > \wordng{LB}{watsʰan} [2 3]; \wordng{Vej}{ˀwatsʰan \recind ˀwatsan} [3 4]; \wordng{’Wk}{ˀwátsʰan̥} (\citealt{VN14}: 106, 262; \citealt{MG-MELO15}: 8, 42; \citealt{KC16}: 106)

\dicnote{The glottalization of the final consonant in Chorote is irregular (both Nivaĉle and Wichí point to its absence in PM). A superficially similar yet distinct root is \word{PCh}{*\mbox{-}wáts’oh}{green, raw} > \word{Ijw}{\mbox{-}wáts’o}{green, alive}; \wordng{I’w}{\mbox{-}wátso} (probably a mistranscription for \intxt{\mbox{-}wats’o})\gloss{green}; \word{Mj}{[ʔi]jéts’o\mbox{-}one / \mbox{-}wáts’o\mbox{-}one}{to eat raw} (\citealt{ND09}: 127; \citealt{AG83}: 168; \citealt{JC18}). In principle, it is conceivable that \intxt{*\mbox{-}wáts’oh} and \intxt{*ˀwásaˀn} ultimately go back to \intxt{**\mbox{-}ˀwáts\mbox{-}ʔo(ˀ)X} (with irregular dissimilation) and \intxt{**\mbox{-}ˀwáts\mbox{-}han}.}%1

\dicnote{The absence of glottalization in the initial consonant in Lower Bermejeño is irregular.}%2

\dicnote{Both in Lower Bermejeño \citep[60]{JB09} and in Vejoz (\citealt{VU74}: 79; \citealt{AFG067}: 212) this form has been documented as \textit{watsan}, which could be a mistranscription.}%3

\dicnote{\citet[8, 42]{MG-MELO15} attest both the expected form \intxt{ˀwatsʰan} and the apparently irregular \intxt{ˀwatsan}.}%4

\lit{\citealt{EN84}: 28 (\intxt{*wåtshan})}

\PMlemma{{\wordnl{*ˀwǻnXåɬåχ\plf{*ˀwǻnXåɬå\mbox{-}ts}}{rhea}}}

\wordng{Mk}{waaɬaχ} (\intxt{\mbox{-}its \recind waaɬe\mbox{-}ts}) [1] (\citealt{AG99}: 360; \citealt{PMA}: 20) {\sep} \wordng{Ni}{βånxåɬåx}, \textit{βånxåɬå\mbox{-}s} \citep[370]{JS16} {\sep} \wordng{PCh}{*ˀwǻnhlåh} (\intxt{*\mbox{-}ås \recind *ˀwǻnhlå\mbox{-}s}) [2 3] > \wordng{Ijw}{ˀwánhla} (\intxt{\mbox{-}has \recind \mbox{-}s}); \wordng{I’w}{ámhla\pl{s}} [4]; \wordng{Mj}{ʔámhla\pl{as}} [4] (\citealt{ND09}: 163; \citealt{AG83}: 121; \citealt{JC18}) {\sep} \wordng{PW}{*wǻˀnɬåχ\plf{*wǻˀnɬå\mbox{-}s}} [2 5] > \wordng{LB}{wonɬoχ}; \wordng{’Wk}{wǻˀ(n)ɬåx\plf{wǻˀ(n)ɬå\mbox{-}s}} (\citealt{VN14}: 170; \citealt{KC16}: 475)

\dicnote{The loss of \sound{PM}{*nX} in Maká is unprecedented. The plural variant \intxt{waaɬe\mbox{-}ts} is in all likelihood innovative, its shape having been influenced by the Maká nouns whose PM etymon ended of \intxt{*\mbox{-}aχ} (plural \textit{*\mbox{-}a\mbox{-}ts}), which regularly yielded \wordng{Maká}{\mbox{-}aχ}, plural \textit{\mbox{-}e\mbox{-}ts}. The word-initial sonorant is attested as non-glottalized in the sources that distinguish between plain and glottalized sonorants \citep[20]{PMA}.}%1

\dicnote{The vowel of the medial syllable was irregularly lost in Chorote and Wichí.}%2

\dicnote{The plural variant \intxt{*ˀwǻnhlåh\mbox{-}ås} in Chorote is likely innovative. The original plural is preserved as a variant in Iyojwa’ja’.}%3

\dicnote{The Iyo’awujwa’ and Manjui reflexes are irregular; one would expect \intxt{*ˀwánhla\plf{*ˀwánhlah\mbox{-}as}}.}%4

\dicnote{In Wichí, the preglottalization has apparently moved from the initial segment to \intxt{*n} and was later lost in Lower Bermejeño and retained in ’Weenhayek (with an optional loss of the nasal consonant).}%5

\lit{\citealt{EN84}: 42 (\intxt{*wahnhlå}); \citealt{PVB02}: 144 (\intxt{*wam(xa)ɬʌχ})}

\PMlemma{{\wordnl{*ˀwäleˀk}{to walk}; \wordnl{*ˀwälke\mbox{-}ˀmat}{to limp}}}

\wordng{Mk}{\mbox{-}<i>ˀwelki\mbox{-}ˀmet} [1]\gloss{to limp} \citep[216]{AG99} {\sep} \word{Ni}{βak͡leˀtʃ}{to walk}, \textit{βaktʃe\mbox{-}mat}\gloss{to limp} \citep[312]{JS16} {\sep} \wordng{PCh}{*[ʔi]ˀwélek} > \wordng{Mj}{[ʔi]ˀjílek / \mbox{-}ˀwélek} \citep{JC18} {\sep} \wordng{PW}{*ˀweleq} > \wordng{LB}{ˀwileq} [2]; \wordng{Vej}{ˀwelek} [3]; \word{’Wk}{ˀwelek}{to camp} (\citealt{VN14}: 311; \citealt{MG-MELO15}: 37; \citealt{KC16}: 109)

\dicnote{The preglottalization in the root-initial consonant in Maká is inferred based on the Chorote and Wichí cognates; the suffix is attested with a glottalized nasal, for example, in the New Testament (\wordnl{eqfe\mbox{-}ˀmet}{ill}; Revelations 8:12). }%1

\dicnote{The vowel \intxt{i} in the Lower Bermejeño reflex, as attested by \citet[311]{VN14}, is entirely unexpected. The etymological vowel \intxt{e} is documented by \citet[61]{JB09} in \wordnl{welek\mbox{-}ɬi}{to walk}, but that source fails to transcribe the glottalization in the stem-initial consonant.}%2

\dicnote{\citet[80]{VU74} documents the verb as \wordnl{welek}{to travel}, with no glottalization in \intxt{w}.}%3

\gc{Obviously related to \word{Proto-Guaicuruan}{*\mbox{-}awalek}{to walk} (\citealt{PVB13b}, \#163; cf. \citealt{PVB13a}: 306).}

\lit{\citealt{PVB13a}: 306 (\intxt{*\mbox{-}welek})}

\PMlemma{{\textit{*[ji]ˀwä́n}\gloss{to see}}}

\wordng{Mk}{[ji]ˀwen} (\citealt{AG99}: 366; \citealt{JB81}: 203) {\sep} \wordng{Ni}{[ji]ˀβan} \citep[314]{JS16} {\sep} \wordng{PCh}{*[ʔi]ˀwén} > \wordng{Ijw}{[ʔi]ˀwíˀn / \mbox{-}ˀwέˀn}; \wordng{I’w}{[i]ín / \mbox{-}wén}; \wordng{Mj}{[ʔi]ˀjín / \mbox{-}ˀwén} (\citealt{JC14b}: 77; \citealt{ND09}: 117; \citealt{AG83}: 44, 169; \citealt{JC18}) {\sep} \wordng{PW}{*[hi]ˀwén} > \wordng{LB}{[hi]ˀwen}\gloss{to see; to have}; \wordng{Vej}{[hi]ˀwen} [1]; \wordng{’Wk}{[hi]ˀwén̥} (\citealt{VN14}: 172, fn. 31, 339; \citealt{MG-MELO15}: 41; \citealt{KC16}: 110)

\dicnote{The Vejoz root is attested as \intxt{\mbox{-}wen} in \citet[80]{VU74} and \citet[212]{AFG067}.}

\gc{Obviously related to \wordng{Proto-Guaicuruan}{*\mbox{-}wen}\gloss{to see; to look} (\citealt{PVB13b}, \#626; cf. \citealt{PVB13a}: 306).}

\lit{\citealt{PVB13a}: 306 (\intxt{*\mbox{-}wen})}

\PMlemma{{\textit{*\mbox{-}ˀwät}\gloss{place}}}

\wordng{Mk}{\mbox{-}ˀwet} [1] (\intxt{\mbox{-}its}) \citep[221]{AG94} {\sep} \wordng{Ni}{\mbox{-}ˀβat}, \textit{\mbox{-}βt\mbox{-}es} (\citealt{AF16}: 113–114) {\sep} \wordng{PCh}{*\mbox{-}ˀwét} > \wordng{Ijw}{\mbox{-}ˀwέt\pl{is}}; \wordng{I’w}{\mbox{-}wét\pl{is}}; \wordng{Mj}{\mbox{-}ˀwét\pl{es}} (\citealt{ND09}: 127; \citealt{AG83}: 169; \citealt{JC18}) {\sep} \wordng{PW}{*\mbox{-}ˀwet} > \wordng{LB/Vej}{\mbox{-}ˀwet\pl{es}} [2]\gloss{place; house}; \wordng{’Wk}{\mbox{-}ˀwet} (\citealt{VN14}: 153, 154, 191; \citealt{MG-MELO15}: 52; \citealt{KC16}: 56)

\dicnote{The Maká reflex functions as a derivational suffix. The glottalization in its initial sonorant is attested in the New Testament in forms such as \textit{ɬ\mbox{-}’exinqa\mbox{-}ˀwet}\gloss{field} (Mark 13:16) or \textit{ɬe\mbox{-}wenq’en\mbox{-}he\mbox{-}ˀwet}\gloss{her/his plantation} (Matthew 13:3), though not in \textit{wit\mbox{-}aqha\mbox{-}wet}\gloss{market} (John 2:16).}%1

\dicnote{The Vejoz root is attested as \textit{\mbox{-}wet} in \citet[80]{VU74} and \citet[212, 219]{AFG067}.}%2

\gc{\citet[318]{PVB13a} compares this root to the Proto-Guaicuruan root for\gloss{home} (*\textit{\mbox{-}ˀwat’a}\gloss{home, camp, family}; \citealt{PVB13b}, \#642).}

\lit{\citealt{EN84}: 48 (\intxt{wɛt}); \citealt{PVB13a}: 319 (\intxt{*\mbox{-}wet})}

\PMlemma{{\textit{*\mbox{-}ˀwɬiʔ \recind *\mbox{-}ˀwɬíʔ}, \textit{*\mbox{-}ˀwɬí\mbox{-}ts}\gloss{rib}}}

\wordng{Mk}{\mbox{-}ˀweɬiʔ\pl{ts}} [1] \citep[366]{AG99} {\sep} \wordng{Ni}{\mbox{-}ˀβɬi / \mbox{-}βɬiʔ\pl{s}} \citep[336]{JS16} {\sep} \wordng{PCh}{*\mbox{-}hlí<s>} [2], \textit{*\mbox{-}hlís\mbox{-}is} > \wordng{Ijw}{\mbox{-}hlés}, \textit{\mbox{-}hlés\mbox{-}is}; \wordng{I’w}{\mbox{-}hlés}, \textit{\mbox{-}hlés\mbox{-}is}; \wordng{Mj}{\mbox{-}hléis}, \textit{\mbox{-}hléiʃ\mbox{-}is} (\citealt{ND09}: 119; \citealt{AG83}: 174; \citealt{JC18})

\dicnote{The glottalization in the root-initial sonorant is attested in \citet[17]{unuuneiki} and in the New Testament (Acts 12:7; John 19:34; John 20:20).}%1

\dicnote{The PM plural form has been reanalyzed as singular in Chorote.}%2

\PMlemma{{\wordnl{*\mbox{-}ˀwo\plf{*\mbox{-}ˀwó\mbox{-}l}}{neck}}}

\wordng{Mk}{\mbox{-}wo<nxeʔ>} (\intxt{\mbox{-}l \recind \mbox{-}ts}) [1] \citep[379]{AG99} {\sep} \wordng{Ni}{\mbox{-}ˀβoʔ\pl{k}} [2]\gloss{neck, nape} (\citealt{LC20}: 80) {\sep} \wordng{PCh}{*\mbox{-}ˀwóʔ\pla{l}} > \wordng{Ijw}{\mbox{-}ˀwóʔ\pl{ˀl}} \citep[128]{ND09} {\sep} \wordng{PW}{*\mbox{-}ˀwo}, \textit{*\mbox{-}ˀwó\mbox{-}lʰ} > \wordng{LB}{\mbox{-}ˀwu\pl{j}} [3]; \wordng{’Wk}{\mbox{-}ˀwo} [4]; \wordng{’Wk}{\mbox{-}ˀwoʔ\pl{ɬ}} (\citealt{VN14}: 163; \citealt{MG-MELO15}: 60; \citealt{KC16}: 57)

\dicnote{The formative \intxt{\mbox{-}nxeʔ} in Maká does not appear to be morphologically segmentable, but it is also found in \wordnl{\mbox{-}fonxeʔ}{ankle} and other body-part terms. The root-initial consonant unexpectedly lacks glottalization, as attested in the New Testament (Luke 15:5).}%1

\dicnote{\citet[353]{JS16} documents the initial consonant of this stem as \textit{β}.}%2

\dicnote{The Lower Bermejeño plural suffix does not match the evidence from Nivaĉle and ’Weenhayek.}%3

\dicnote{The Vejoz root is attested as \textit{\mbox{-}wo} in \citet[81]{VU74}.}%4

\lit{\citealt{EN84}: 9, 18 (\intxt{*wo}, 2~\intxt{*a\mbox{-}wo}); \citealt{AnG15}: 255}

\PMlemma{{\wordnl{*(\mbox{-})ˀwoˀj}{blood}}}

\wordng{Ni}{βoˀj\plf{\mbox{-}ˀβoj\mbox{-}ej}} [1] (\citealt{JS16}: 366, 368; \citealt{LC20}: 71, 515) {\sep} \wordng{PCh}{*(\mbox{-})ˀwój\mbox{-}is} (\intxt{plurale tantum}) > \wordng{Ijw}{\mbox{-}ˀwój\mbox{-}is}; \wordng{I’w}{\mbox{-}wój\mbox{-}is}, \wordng{Mj}{(\mbox{-})wój\mbox{-}is} (\citealt{ND09}: 128; \citealt{AG83}: 170; \citealt{JC18}) {\sep} \wordng{PW}{*ˀwoj\mbox{-}ís / *\mbox{-}ˀwój\mbox{-}is} (\intxt{plurale tantum}) > \wordng{LB}{\mbox{-}ˀwuj\mbox{-}is \recind \mbox{-}(ˀ)wij\mbox{-}is} [2]; \wordng{Vej}{\mbox{-}woj\mbox{-}is \recind \mbox{-}ˀwoj\mbox{-}s}; \wordng{’Wk}{\mbox{-}ˀwój\mbox{-}is / ˀwoj\mbox{-}ís} (\citealt{VN14}: 48, 152, 164; \citealt{VU74}: 82; \citealt{MG-MELO15}: 69; \citealt{KC16}: 54, 114)

\dicnote{\citet[366, 368]{Seelwische2016} documents the initial consonant as \textit{β} not only in the singular (absolute) form, but also in the plural (relational) form of this stem.}%1

\dicnote{The variants \textit{\mbox{-}ˀwij\mbox{-}is \recind \mbox{-}wij\mbox{-}is}, attested in Lower Bermejeño Wichí, are irregular.}%2

\PMlemma{{\wordnl{*ˀwóså(ˀ)q \recind *ˀwóså(ˀ)k}{butterfly}}}

\wordng{Ni}{βosåk\plf{βosåk͡l\mbox{-}is \recind βosåk͡l\mbox{-}ij}} (\wordng{ShL}{βosok\plf{βosok͡l\mbox{-}is}}) [1] (\citealt{NS87}: 125; \citealt{AnG15}: 119; \citealt{JS16}: 367; \citealt{LC20}: 99) {\sep} \wordng{PCh}{*ˀwósak} > \wordng{Ijw}{ˀwóxsak\pl{is}} \citep[163]{ND09}

\dicnote{The Nivaĉle plural form must be an analogical development because it points to a stem-final \intxt{*l} in PM, which is incompatible with the Chorote datum. Alternatively, the Iyojwa’aja’ word could be a Nivaĉle loan.}%1

\empr{\citet[308]{AF16} compares the Nivaĉle reflex to \word{Enlhet, Enxet, Angaité, Sanapaná, Guaná}{\mbox{seleklek}}{butterfly} (\citealt{EU-HK-97}: 603; \citealt{PW20}: 23, 92; \citealt{JE21}: 559; \citealt{HK-23}: 184), which is obviously a spurious comparison.}

\lit{\citealt{EN84}: 45 (\intxt{*wohsåk})}

\PMlemma{{\intxt{*\mbox{-}ˀwut \recind *\mbox{-}ˀwút} (fem. \intxt{*\mbox{-}ˀwút\mbox{-}eʔ})\gloss{riding animal}}}

\wordng{Mk}{\mbox{-}ˀwut\pl{its}} (fem. \intxt{\mbox{-}ˀwut\mbox{-}iʔ\pl{j}}) [1] (\citealt{AG99}: 382) {\sep} \wordng{PW}{*\mbox{-}ˀwút<e>\pla{jʰ}} [2] > \word{LB}{[ʔi]wu\mbox{-}ˀwete\mbox{-}j\mbox{-}a}{to ride an animal}; \word{Vejoz or Guisnay}{\mbox{-}ˀwute\pl{j}}{mount, bicycle}; \wordng{’Wk}{\mbox{-}ˀwúteʔ} (\citealt{VN14}: 267; \citealt{RL16}: 109; \citealt{KC16}: 57)

\dicnote{The preglottalization in \textit{ˀw} is attested in the New Testament (e.g. Luke 10:34).}%1

\dicnote{The Wichí reflex continues the erstwhile feminine form. It is formally possible to include \word{PW}{*\mbox{-}ˀwut\plf{*\mbox{-}ˀwút\mbox{-}es}}{pole, log, bar, crossbar, crossbeam, handle} > \wordng{Vejoz or Guisnay}{\mbox{-}ˀwut\pl{es}}; \wordng{’Wk}{\mbox{-}ˀwut\plf{\mbox{-}ˀwút\mbox{-}es}} (\citealt{RL16}: 109; \citealt{KC16}: 57), which would reflect the erstwhile masculine form, but this runs into semantic difficulties. If these etyma are shown to be related, the PM masculine form should be reconstructed with an unaccented vowel.}%1

\PMlemma{{\intxt{*\mbox{-}ˀwVˀɬ \recind *\mbox{-}ˀwV́ˀɬ} [1]\gloss{to climb}}}

\wordng{Mk}{weˀɬ} (\citealt{AG99}: 366; \citealt{PMA}: 3) {\sep} \wordng{Ni}{βåˀɬ} \citep[371]{JS16} {\sep} \wordng{PCh}{*[ʔi]ˀwúɬ} > \wordng{Ijw}{[ʔi]ˀjúɬ / \mbox{-}ˀwúɬ}; \wordng{I’w}{\mbox{-}wúl}; \wordng{Mj}{[ʔi]ˀjúɬ / \mbox{-}ˀwúɬ} (\citealt{ND09}: 118; \citealt{AG83}: 172; \citealt{JC18}) {\sep} \wordng{PW}{*[t]ˀwuɬ \recind *[t]ˀwúɬ} > \wordng{LB}{[t(a)]ˀweɬ}; \wordng{Vej}{\mbox{-}wuɬ\mbox{-}o}; \wordng{’Wk}{[t(a)]ˀwuɬ \recind [t(a)]ˀwúɬ} (\citealt{VN14}: 128, 258; \citealt{VU74}: 82; \citealt{KC16}: 347)

\dicnote{The vowel cannot be securely reconstructed at this time. Maká points to \sound{PM}{*a}, Nivaĉle to \intxt{*å}, Chorote and Wichí to \intxt{*u}. The correspondence is similar to the one in \word{PM}{*wV́x}{large}.}%1

\lit{\citealt{EN84}: 24 (\intxt{*wulq}); \citealt{AnG15}: 254}

\PMlemma{{\textit{*\mbox{-}xa}, \textit{*\mbox{-}xá\mbox{-}l}\gloss{price}}}

\wordng{Ni}{\mbox{-}ʃaʔ\pl{k}} \citep[238]{JS16} {\sep} \wordng{PW}{*\mbox{-}ha}, \textit{*\mbox{-}há\mbox{-}lʰ} > \wordng{LB}{\mbox{-}ha}, \wordng{’Wk}{\mbox{-}haʔ}, \textit{\mbox{-}há\mbox{-}ɬ} (\citealt{VN14}: 273, 291; \citealt{KC16}: 57)

\PMlemma{{\intxt{*…xaˀχ\plf{*…xáh\mbox{-}ajʰ}} [1] \wordnl{\recind *Xon\mbox{-}xaˀχ\plf{*Xon\mbox{-}xáh\mbox{-}ajʰ}}{night}}}

\wordng{Mk}{<na>xaˀχ} [2], \intxt{<na>xa\mbox{-}j} \citep[266]{AG99} {\sep} \word{Ni}{<xon>ʃaˀx}{midnight}, \wordnl{<xon>ʃax\mbox{-}aj}{every night} [3] \citep[150]{JS16} {\sep} \wordng{PCh}{*<ʔa>h<n>áh \recind *<ʔå>h<n>áh\pla{as}} [5] > \wordng{Ijw}{ʔahnáh\pl{as}}; \wordng{I’w}{ahnáh\plf{ahná\mbox{-}as}}; \wordng{Mj}{ʔahnáh\plf{ʔahná\mbox{-}as}} (\citealt{JC14b}: 91; \citealt{ND09}: 93; \citealt{AG83}: 124; \citealt{JC18}) {\sep} \word{PW}{*<hon>aχ\plf{*<hon>áh\mbox{-}ajʰ}}{afternoon, night} [5] > \word{LB}{hunaχ}{afternoon}; \word{Vej}{honax\plf{honah\mbox{-}aj}}{afternoon}; \wordng{’Wk}{honax\plf{honáh\mbox{-}aç}}; \wordnl{*honá<tsi>}{night} > \wordng{LB}{hunatsi}; \wordng{Vej}{honatsi}; \wordng{’Wk}{honátsiʔ\pl{s}} (\citealt{VN14}: 344; \citealt{VU74}: 57; \citealt{MG-MELO15}: 43, 70; \citealt{KC16}: 153)

\dicnote{We speculate that this was a suffix in PM. In individual languages, it is attached to otherwise unattested roots: \wordng{Maká}{na\mbox{-}}, \wordng{Nivaĉle}{xon\mbox{-}}, \wordng{Chorote}{*ʔan\mbox{-}} or \intxt{*ʔån\mbox{-}}, and \wordng{Wichí}{*hon\mbox{-}} (the latter two prefixes are also found in the word for\gloss{earth}). \wordng{Chorote}{*ʔan\mbox{-} \recind *ʔån\mbox{-}} might be cognate with \wordng{Nivaĉle}{xon\mbox{-}}, \wordng{Wichí}{*hon\mbox{-}}.}%1

\dicnote{The preglottalized coda in the Maká singular form is attested in the New Testament (e.g. John 11:10).}%2

\dicnote{This expression goes back to a PM plural form.}%3

\dicnote{The Chorote plural form is non-etymological.}%4

\dicnote{The development \wordng{PM}{*nx} > \wordng{PW}{*n} is irregular.}%5

\lit{\citealt{EN84}: 10, 27, 41 (\intxt{*hnahn})}

\PMlemma{{\textit{*\mbox{-}xä́jk’u(ʔ)\pla{l}}\gloss{egg}}}

\wordng{Ni}{\mbox{-}ʃajk’u} (\mbox{-}\textit{k}) \citep[357]{JS16} {\sep} PCh~\third{*hl\mbox{-}éjk’uʔ\pla{l}} > Ijw~\third{hl\mbox{-}έtsʲuʔ\pl{ˀl}}; I’w~3~\textit{l\mbox{-}éˀkʲuʔ\pl{l}}; Mj~\third{hl\mbox{-}έʔʲuʔ\pl{l}}\gloss{egg, pulp, tree heart} (\citealt{ND09}: 131; \citealt{AG83}: 146; \citealt{JC18}) {\sep} \wordng{PW}{*\mbox{-}ɬ\mbox{-}ɪ́kʲ’u\pla{lʰ}} [1] > \wordng{LB}{ɬ\mbox{-}etʃ’e\pl{ɬ}}; \wordng{Vej}{\mbox{-}ɬ\mbox{-}itʃ’u}; \wordng{’Wk}{\mbox{-}ɬ\mbox{-}íkʲ’uʔ\pl{ɬ}} (\citealt{VN14}: 191; \citealt{VU74}: 66; \citealt{KC16}: 75)

\dicnote{It is uncertain whether \wordng{PW}{*ɪ} is the regular outcome of \wordng{PM}{*äj}.}%1

\rej{Despite a superficial similarity to the aforementioned forms, \wordng{Maká}{ɬihiʔ\pl{j}} shows no regular correspondence with \wordng{PM}{*\mbox{-}xéjk’u\pla{l}}, whose expected reflex in Maká would be \textit{*\mbox{-}xijk’u\pla{l}}.}

\lit{\citealt{EN84}: 22, 48 (\intxt{*hlec’u}); \citealt{LC-VG-07}: 16}

\PMlemma{{\textit{*\mbox{-}xä́teˀk\plf{*\mbox{-}xä́the\mbox{-}jʰ}} [1]\gloss{head}}}

\wordng{Ni}{\mbox{-}ʃateˀtʃ\plf{\mbox{-}ʃatxe\mbox{-}s}} (ShL \textit{\mbox{-}ʃatitʃ\plf{\mbox{-}ʃatxi\mbox{-}s}}) \citep[357]{JS16} {\sep} \wordng{PCh}{*\mbox{-}hétek\plf{*\mbox{-}héhte\mbox{-}jʰ}} > \wordng{Ijw}{\mbox{-}hέtik\plf{\mbox{-}hέte\mbox{-}ˀl}} [2]; \wordng{I’w}{\mbox{-}hétik\plf{\mbox{-}héte\mbox{-}j}} [2]; \wordng{Mj}{\mbox{-}hέtek\plf{\mbox{-}hέhte\mbox{-}j}} (\citealt{JC14b}: 90, 98; \citealt{AG83}: 146; \citealt{JC18}) {\sep} \wordng{PW}{*\mbox{-}ɬ\mbox{-}éteq\plf{*\mbox{-}ɬ\mbox{-}éthe\mbox{-}jʰ}} > \wordng{LB}{\mbox{-}ɬ\mbox{-}eteq\plf{\mbox{-}ɬ\mbox{-}etʰe\mbox{-}j}}; \wordng{Vej}{\mbox{-}ɬ\mbox{-}etek}; \wordng{’Wk}{\mbox{-}ɬ\mbox{-}étek\plf{\mbox{-}ɬ\mbox{-}étʰe\mbox{-}ç}} (\citealt{VN14}: 166, 192; \citealt{VU74}: 66; \citealt{MG-MELO15}: 60, 61; \citealt{AFG067}: 217; \citealt{KC16}: 74, 204, 300)

\dicnote{The plural form is reconstructed based on the evidence of Iyo’awujwa’, Manjui, and Wichí. It is thus technically reconstructible only for Proto-Chorote–Wichí.}%1

\dicnote{The vowel \intxt{i} in the singular form in Iyojwa’aja’ and Iyo’awujwa’ is not etymological, as is the choice of the suffix in the plural form in Iyojwa’aja’.}%2

\empr{\citet[308]{AF16} compares the Mataguayan root with the Enlhet–Enenlhet term for\gloss{head}: \wordng{Enlhet}{\mbox{-}paʔtek / \mbox{-}kaːtek}, \wordng{Enxet}{\mbox{-}paːtek / \mbox{-}qaːtek}, \wordng{Enenlhet-Toba}{\mbox{-}patek / \mbox{-}qatek}, \wordng{Sanapaná}{\mbox{-}patek / \mbox{-}katek}, \wordng{Angaité}{\mbox{-}paʔtek}, \wordng{Guaná}{\mbox{-}paʔtek / \mbox{-}(p)qatek} (\citealt{EU-HK-97}: 144; \citealt{EU-HK-MR-03}: 186, 308; \citealt{ASG12}: 168, 173; \citealt{PW20}: 92; \citealt{JE21}: 125, 677; \citealt{HK-23}: 84). The root is also similar to \word{Proto-Guaicuruan}{*\mbox{-}t’ek}{hair; to brush one’s hair}, (?) \wordnl{*\mbox{-}(a)t’ek}{head, hair} (\citealt{PVB13b}, \#558).}

\lit{\citealt{EN84}: 23, 34, 48 (\intxt{*ɛthe}, \textsc{pl}~\intxt{*ɛthe\mbox{-}j \recind *ɛthe\mbox{-}s}); \citealt{PVB02}: 142 (\intxt{*\mbox{-}xetik}); \citealt{LC-VG-07}: 16, 22; \citealt{AF16}: 308; \citealt{AnG15}: 64}

\PMlemma{{\wordnl{*xéjåʔ\pla{l}}{bat}}}

\wordng{Mk}{xajaʔ\pl{l}} [1] (\citealt{AG99}: 386; \citealt{PMA}: 7) {\sep} \wordng{Ni}{ʃejå\pl{k}} [2] \citep[240]{JS16} {\sep} \wordng{PCh}{*<ʔa>héjaʔ\pla{l}} [3] > \wordng{Ijw}{ʔehέjeʔ\pl{jis}} [4]; \wordng{I’w}{\mbox{ahéjeʔ}\pl{l}}; \wordng{Mj}{ʔahέjeʔ\pl{l}} (\citealt{ND09}: 96; \citealt{AG83}: 123; \citealt{JC18})

\dicnote{The reflex of the vowel of the initial sylable in Maká is entirely irregular.}%1

\dicnote{In the Yita’ Lhavos dialect of Nivaĉle, the vowel of the initial syllable is irregularly raised to \intxt{i}.}%2

\dicnote{In Chorote, an element \intxt{*ʔa\mbox{-}} of unclear origin was appended to the root, and \wordng{PM}{*å} is unexpectedly reflected as \intxt{*a}.}%3

\dicnote{The Iyojwa’aja’ plural form is non-etymological.}%4

\lit{\citealt{PVB02}: 142 (\intxt{*(V)xejʌʔ})}

\PMlemma{{\textit{*xélå(ˀ)X₁₂} (fruit), \textit{*xélå\mbox{-}juˀk} (tree)\gloss{plant sp.}}}

\word{Ni}{ʃek͡låx}{\textit{sutia} fruit\species{Solanaceae}}; \wordnl{ʃek͡lå\mbox{-}juk\plf{ʃek͡lå\mbox{-}ku\mbox{-}j}}{\textit{Prosopis sp.} tree} \citep[240]{JS16} {\sep} \word{PCh}{*hél<ek>\plf{*hél<ke>\mbox{-}jʰ}}{\textit{Tabebuia nodosa}} > \wordng{Ijw}{hέlik}, \textit{hέlikʲ\mbox{-}et \recind hέlki\mbox{-}ʔ} [1]; \wordng{I’w}{hélik\plf{hélki\mbox{-}ʔ}}; \wordng{Mj}{hέlek\plf{hέlki\mbox{-}j}} (\citealt{ND09}: 119; \citealt{AG83}: 173; \citealt{JC18}) {\sep} \wordng{PW}{*hél<ekʷ>} > \wordng{LB}{helekʷ}; \wordng{Vej}{helek}; \wordng{’Wk}{hélek} (\citealt{CS08}: 59; \citealt{MS14}: 205; \citealt{VU74}: 57; \citealt{KC16}: 148)

\dicnote{The final glottal stop in \wordng{Ijw}{hέlki\mbox{-}ʔ} is unexpected.}%1

\PMlemma{{\wordnl{*\mbox{-}xíjʰ}{recipient}}}

\wordng{Mk}{\mbox{-}xij} \citep[221]{AG94} {\sep} \wordng{Ni}{\mbox{-}ʃij / \mbox{-}xij} (after \textit{V\textsubscript{[+back]}}\textit{(C\textsubscript{[+grave]}})) (\intxt{\mbox{-}is}) (\citealt{AF16}: 99–100; \citealt{LC20}: 129) {\sep} \wordng{PW}{*\mbox{-}híh\plf{*\mbox{-}hí\mbox{-}s}} > \wordng{LB}{\mbox{-}hi\pl{s}}; \wordng{’Wk}{\mbox{-}híh\plf{\mbox{-}hí\mbox{-}s}} (\citealt{VN14}: 215, 393; \citealt{KC16}: 58)

\gc{\citet[316]{PVB13a} compares it to the Proto-Guaicuruan locative suffix \intxt{*\mbox{-}ˀgi} (\citealt{PVB13b}, \#790).}

\lit{\citealt{PVB02}: 143 (\intxt{*\mbox{-}xij}); \citealt{PVB13a}: 316 (\intxt{*\mbox{-}hij})}

\PMlemma{{\wordnl{*xnáqha(ˀ)j\pla{its}}{fog}}}

\wordng{Ni}{ʃnakxaj \recind snakxaj\pl{is}} (\citealt{NS87}: 110; \citealt{JS16}: 244) {\sep} \wordng{PCh}{*\mbox{ʔihnáhqajʔ}\pla{is}} [1] > \wordng{Mj}{ʔihn(ʲ)éhkajʔ\pl{is}} \citep{GH94}

\dicnote{It is not clear why Chorote reflects \wordng{PM}{*xn\mbox{-}} as \textit{*ʔihn\mbox{-}} here (cf. the reflex \textit{*n\mbox{-}} in \wordng{PM}{*xnáwåp}).}%1

\rej{Despite superficial similarity, \wordng{Maká}{xunkhaj}\gloss{fog} \citep[393]{AG99} and Iyojwa’aja’ \textit{sinʲákaʔ}\gloss{fog} \citep[145]{ND09} show no regular correspondence with \wordng{PM}{*xnáqhaj}. They must have been borrowed from \wordng{Nivaĉle}{ʃnakxaj}, just like \wordng{Mk}{xunkhaj} < \word{Ni}{ʃk͡låkxaj \recind sk͡låkxaj}{wild cat}. A problematic fact for our hypothesis is that the Iyojwa’aja’ (unlike Iyo’awujwa’ and Manjui) have not been demonstrably in contact with the Nivaĉle until recently. Alternatively, one could view the Iyojwa’aja’ form as inherited from \wordng{PM}{*snáqhaj}, in which case the Manjui form would have to be explained as an early loan from Nivaĉle (however, it would be more difficult to account for its phonological adaptation pattern than if the Manjui datum is considered cognate with the Nivaĉle one).}

\lit{\citealt{EN84}: 12, 25, 38 (\intxt{*snaqaj}); \citealt{LC-VG-07}: 15}

\PMlemma{{\wordnl{*xnáwåˀp}{spring} [1]}}

\wordng{Mk}{xinawaˀp\plf{xinawap\mbox{-}its}} (\citealt{AG99}: 389; \citealt{maka-etnomat}: 23–25) {\sep} \wordng{Ni}{ʃnaβåp \recind ʃnåβåp} (\citealt{AnG15}: 64; \citealt{JS16}: 244) {\sep} \wordng{PCh}{*náwop} [2] > \wordng{Ijw/I’w}{náwop} (\citealt{ND09}: 140; \citealt{AG83}: 150) {\sep} \wordng{PW}{*ˣnáwop} [2] > \wordng{LB}{nawup}; \wordng{Vej}{nawop \recind inawop}; \wordng{’Wk}{ʔináwop} (\citealt{VN14}: 47; \citealt{VU74}: 67; \citealt{MG-MELO15}: 43; \citealt{KC16}: 32)

\dicnote{This noun is obviously related to \wordng{PM}{*\mbox{-}áwå}\gloss{flower} and literally means\gloss{flower season}.}%1

\dicnote{The raising of \wordng{PM}{*å} to \wordng{PCh/PW}{*o} is not known to be regular.}%2

\lit{\citealt{EN84}: 33 (\intxt{*hnawɔp}); \citealt{PVB02}: 142 (\intxt{*xinawʌp}); \citealt{AnG15}:~64}

\PMlemma{{\wordnl{*xókhajeχ}{Muscovy duck}}}

\wordng{Mk}{xokhejaχ} [1], \textit{xokheji\mbox{-}ts} (\citealt{AG99}: 390; \citealt{PMA}: 5) {\sep} \word{Ni}{\mbox{xokxajex}\pl{is}}{Muscovy duck; canoe} \citep[149]{JS16} {\sep} \wordng{PCh}{*qajáh\pla{Vs}} [2] > \wordng{I’w}{kajé\pl{es}}; \word{Mj}{kajéh\plf{kajé\mbox{-}es}}{Muscovy duck; canoe} (\citealt{AG83}: 136; \citealt{JC18}) {\sep} \wordng{PW}{*xʷóqˀjaχ} [3] > \wordng{LB}{fʷuˀjaχ} [4]; \wordng{’Wk}{xʷóqˀjax}; \intxt{*xʷóqˀje\mbox{-}taχ} [3] > \wordng{Vej}{hʷok(j)e\mbox{-}tah} (\citealt{CS-FL-PR-VN13}; \citealt{MG-MELO15}: 20; \citealt{KC16}: 174)

\dicnote{The absence of preglottalization in Maká is attested in a narrative by \citet[17]{unuuneiki}, as well as in \citet[67]{JB81}.}%1

\dicnote{The Chorote reflex is irregular. One would expect \wordng{PCh}{**hóhqajah}.}%2

\dicnote{The Wichí reflex is irregular. One would expect \wordng{PW}{**xókhajaχ}.}%3

\dicnote{\citet[51]{VN14} mistranscribes the Lower Bermejeño reflex as \intxt{fʷujaχ}.}%4

\lit{\citealt{EN84}: 44 (\intxt{*hwokajɛhn})}

\PMlemma{{\wordnl{*xpåˀk \recind *xpǻˀk}{straw}}}

Mk ‹hupak› \citep[62]{JBe31}, \wordnl{xupek}{\textit{Imperata sp.}} [1] \citep[83]{JB81} {\sep} \wordng{Ni}{xpåˀk}, \textit{xpåk\mbox{-}uj} \citep[156]{JS16} {\sep} \wordng{PCh}{*ʔipǻk} > \wordng{Ijw}{ʔipʲák}, \textit{ʔipʲákʲ\mbox{-}et}; \wordng{I’w}{ipʲék} (\citealt{ND09}: 109; \citealt{AG83}: 131)

\dicnote{The Maká form attested by \citet{JB81} is surprising; one would expect \intxt{*xupaˀk}.}%1

\empr{\citet[306]{AF16} suggests that the Nivaĉle reflex is related to the Enlhet–Enenlhet term for\gloss{grass} – Enlhet, Enenlhet–Toba, Guaná \wordnl{paʔat}{grass, house}, Enxet, Sanapaná \wordnl{paʔat}{grass} (\citealt{EU-HK-97}: 536; \citealt{EU-HK-MR-03}: 334; \citealt{ASG12}: 140; \citealt{JE21}: 210; \citealt{HK-23}: 78) – via borrowing. This possibility seems unlikely to us.}

\lit{\citealt{EN84}: 9, 18, 25, 28 (\intxt{*ipʰǻk’})}

\PMlemma{{\intxt{*xunxátaχ} (fruit); \intxt{*xunxáta\mbox{-}(ju)ˀk} (tree); \intxt{*xunxáta\mbox{-}kat} (grove)\gloss{tusca (\intxt{Acacia aroma})}}}

\wordng{Mk}{xunxetaχ}; \intxt{xunxete\mbox{-}ˀk}; \intxt{xunxete\mbox{-}ket} [1] \citep[394]{AG99} {\sep} \wordng{Ni}{xunʃatax}; \intxt{xunʃata\mbox{-}juk}; \intxt{xunʃata\mbox{-}tʃat} \citep[159]{JS16} {\sep} \wordng{PCh}{*ʔihnátah}; \intxt{*ʔihnáta\mbox{-}k}; \intxt{*ʔihnáta\mbox{-}kat} > \wordng{Ijw}{ʔihnʲétah}; \intxt{ʔihnʲéta\mbox{-}k}; \textit{—}; \wordng{I’w}{—}; \intxt{ihnʲéta\mbox{-}k}; \intxt{ihnʲéta\mbox{-}ket}; \wordng{Mj}{—}; \intxt{ʔihn(ʲ)éta\mbox{-}k}; \intxt{—} (\citealt{ND09}: 98; \citealt{AG83}: 133; \citealt{JC18}) {\sep} \wordng{PW}{\mbox{*ˣnhátaχ}}; \intxt{*ˣnháte\mbox{-}q} > \wordng{LB}{n̥ataχ}; \intxt{—}; \wordng{Southeastern (Salta)}{ʔin̥ataχ \recind nataχ}; \intxt{ʔin̥ate\mbox{-}q \recind nate\mbox{-}q}; \wordng{Vej}{—}; \intxt{n̥ate\mbox{-}k} [2]; \wordng{’Wk}{ʔin̥átax}; \intxt{ʔin̥áte\mbox{-}k} (\citealt{CS08}: 60; \citealt{VN14}: 52; \citealt{MS14}: 265; \citealt{MG-MELO15}: 18; \citealt{KC16}: 32, 33)

\dicnote{The absence of preglottalization in the term for the fruit in Maká is attested in \citet[77]{JB81}. The preglottalized coda in the Maká suffix for tree names is attested elsewhere \citep[7]{PMA}.}%1

\dicnote{The Vejoz reflex is mistranscribed as \intxt{nate\mbox{-}k} in \citet[125]{VU74}.}%2

\lit{\citealt{EN84}: 34, 47 (\intxt{*(hnu)hnɛtak \recind *hnatak}); \citealt{PVB02}: 142 (\intxt{*xunxetek}); \citealt{LC-VG-07}: 16, 22; \citealt{AnG15}: 64}

\PMlemma{{\wordnl{*xu(ˀ)p}{grass}}}

\wordng{Mk}{xup<’el>} [1] \citep[158]{AG99} {\sep} \wordng{PCh}{*húp\plf{*hup\mbox{-}ájʰ}} > \wordng{Ijw}{hóp}; \word{I’w}{hóp}{maize}, \wordnl{hup\mbox{-}áj}{grass}; \word{Mj}{hʊ́p\plf{hup\mbox{-}ájh}}{maize} (in plural also\gloss{grass}) (\citealt{ND09}: 128; \citealt{AG83}: 176; \citealt{JC18}) {\sep} \word{PW}{*hup\pla{újʰ}}{grass; house made of hay} > \wordng{LB}{hep\pl{ej}}; \wordng{Vej}{hup} (\intxt{\mbox{-}uj}); \wordng{’Wk}{hup} (\intxt{\mbox{-}úç}) (\citealt{VN14}: 161, 327; \citealt{VU74}: 58; \citealt{KC16}: 158)

\dicnote{We have no explanation for the element \intxt{\mbox{-}’el} in Maká. \citet[83]{JB81} gives the form \intxt{xupeɬ}.}%1

\rej{\citet[33]{EN84} includes \wordng{Ni}{ɬ\mbox{-}uˀp}\gloss{its nest} under this etymology, which is obviously incorrect.}

\lit{\citealt{EN84}: 33 (\intxt{*hnup’}); \citealt{PVB02}: 143 (\intxt{*xup’})}

\PMlemma{{\wordnl{*[ji]X₁₃o(ʔ) \recind *[ji]X₁₃ó(ʔ)}{to go}; \wordnl{*[ji]X₁₃óʔ\mbox{-}xäˀneʔ}{to lie down}}}

\word{Ni}{[ji]xoʔ}{to advance}; \wordnl{[ji]xoʔ\mbox{-}xane}{to lie down} \citep[149]{JS16} {\sep} \wordng{PCh}{*[ʔi]hóʔ} > \wordng{Ijw}{[ʔi]hjóʔ / \mbox{-}hɔ́ʔ}; \wordng{I’w}{\mbox{-}hó\mbox{-}\APPL}; \wordng{Mj}{[ʔi]hjóʔ / \mbox{-}hɔ́ʔ}; \textit{*[ʔi]hó\mbox{-}heˀn(eʔ)}\gloss{to lie down} > \wordng{Ijw}{[ʔi]hjó\mbox{-}hweˀn / \mbox{-}hɔ́\mbox{-}hweˀn}; \wordng{I’w}{\mbox{-}hó\mbox{-}ʔneʔ}; \wordng{Mj}{[ʔi]hjó\mbox{-}oˀneʔ / \mbox{-}hɔ́\mbox{-}oˀneʔ} (\citealt{JC14a}; \citealt{ND09}: 97; \citealt{AG83}: 176; \citealt{JC18}) {\sep} \wordng{PW}{*[ji]ho(ʔ) \recind *[ji]hó(ʔ)} > \wordng{LB}{[ji]hu\mbox{-}\APPL}; \wordng{Vej}{\mbox{-}ho}; \wordng{’Wk}{[ja]hó\mbox{-}\APPL} (\citealt{VN14}: 265, 329; \citealt{VU74}: 57; \citealt{KC16}: 151–156)

\lit{\citealt{EN84}: 32 (\wordnl{*hnowet}{bed})}

\PMlemma{{\wordnl{*X₁₃óˀk}{\textit{Bulnesia sarmientoi}}}}

\wordng{Ni}{xoˀk}, \textit{xok\mbox{-}is} \citep[150]{JS16} {\sep} \wordng{PCh}{*hók} > \wordng{I’w}{hók}, \textit{\mbox{-}iʔ}; \wordng{Mj}{hɔ́k\pl{ej}} (\citealt{AG83}: 176; \citealt{JC18}) {\sep} \wordng{PW}{*hókʷ} > \wordng{LB}{hukʷ}; \wordng{Vej}{hok} [1]; \wordng{’Wk}{hók} (\citealt{VN14}: 193; \citealt{VU74}: 57; \citealt{MG-MELO15}: 18; \citealt{AFG067}: 218; \citealt{KC16}: 152)

\dicnote{The absence of labialization in the reflex of \wordng{PW}{*\mbox{-}kʷ} in Vejoz is unexpected.}%1

\lit{\citealt{EN84}: 17 (\intxt{*hno\mbox{-}uk}); \citealt{LC-VG-07}: 19 (\enquote{lapacho tree}, with the suffix \intxt{*\mbox{-}taχ})}

\PMlemma{{\wordnl{*X₁₃óˀt}{sandy place}}}

\wordng{Ni}{xoˀt\plf{xot\mbox{-}oj}} \citep[151]{JS16} {\sep} \wordng{PCh}{*hót} > \wordng{Ijw}{hɔ́t}; \word{Mj}{hɔ́t\pl{ej}}{sand} (\citealt{ND09}: 128; \citealt{JC18}) {\sep} \wordng{PW}{*hót} > \wordng{’Wk}{hót} \citep[154]{KC16}

\PMlemma{{\wordnl{*\mbox{-}X₁₃uˀk\plf{*\mbox{-}X₁₃ú\mbox{-}jʰ}}{firewood}}}

\wordng{Ni}{\mbox{-}xuˀk\plf{\mbox{-}xu\mbox{-}j}} \citep[160]{JS16} {\sep} \wordng{PCh}{*(ʔítåh)\mbox{-}huk} > \wordng{I’w}{éjti\mbox{-}fʷuk} [1] \citep[126]{AG83} {\sep} \wordng{PW}{*\mbox{-}hukʷ\plf{*\mbox{-}hú\mbox{-}j<is>}} > \wordng{’Wk}{\mbox{-}huk\plf{\mbox{-}hú\mbox{-}jis}} (\citealt{KC16}: 38, 59)

\dicnote{\sound{Iyo’awujwa’}{fʷ} could be a mistranscription (\textit{pro} the expected reflex \textit{h}) on \cits{AG83} part.}%1

\PMlemma{{\wordnl{*[ji]X₁₃út}{to push}}}

\word{Ni}{[ji]xut}{to give} \citep[159]{JS16} {\sep} \wordng{PCh}{*[ʔi]hút} > \wordng{Ijw}{[ʔi]hjút / \mbox{-}hót}; \wordng{Mj}{[ʔi]hjút / \mbox{-}hʊ́t} (\citealt{ND09}: 97; \citealt{JC18}) {\sep} \wordng{PW}{*[ji]hút} > \wordng{LB}{[ji]het\mbox{-}tsi}; \wordng{Vej}{\mbox{-}hut}; \wordng{’Wk}{[ja]hút} (\citealt{JB09}: 63; \citealt{VU74}: 58; \citealt{KC16}: 159)

\PMlemma{{\intxt{*(ʔa)X₁₃útsa(ˀ)χ\plf{*(ʔa)X₁₃útsha\mbox{-}ts}} [1]\gloss{crested caracara}}}

\wordng{Ni}{xutsax\plf{xutsxa\mbox{-}s}} \citep[159]{JS16} {\sep} \wordng{PCh}{*(ʔa)húsah}, \textit{*(ʔa)húsa\mbox{-}s} > \wordng{Ijw}{ʔawúxse\pl{jis}} [2]; \wordng{I’w}{ohúxsa}, \textit{ohúxse\mbox{-}s} [3]; \wordng{Mj}{ʔahʊ́xsa \recind hʊ́xsa\pl{s}} (\citealt{ND09}: 95; \citealt{AG83}: 154; \citealt{JC18}) {\sep} \word{PW}{*ʔahútsaχ\plf{*ʔahútsha\mbox{-}s}}{crested caracara; kind of dance} > \word{LB}{ʔahetsaχ}{crested caracara}; \word{Vej}{ahutsah}{dance}; \wordng{’Wk}{ʔahútsax\plf{ʔahútsʰa\mbox{-}s}} (\citealt{VN14}: 66; \citealt{VU74}: 50; \citealt{KC16}: 10)

\dicnote{The form without \textit{*ʔa\mbox{-}} is reflected in Nivaĉle and Manjui. In Chorote and Wichí, a reflex of \intxt{*ʔa\mbox{-}} is found.}%1

\dicnote{The reflex \textit{w} (< \wordng{PCh}{*h}) and the plural suffix in Iyojwa’aja’ are irregular.}%2

\dicnote{The Iyo’awujwa’ reflex is somewhat irregular: one would expect \intxt{*ahʊ́xsa\pla{s}}.}%3

\lit{\citealt{LC-VG-07}: 19}

\PMlemma{{\wordnl{*…X₂₃aˀt\plf{*…X₂₃át\mbox{-}its}}{earth, land} [1]}}

\wordng{Ni}{<kots>xaˀt\plf{<kots>xat\mbox{-}is}} (\wordng{YL}{<kuts>xaˀt} [2]) (\citealt{AnG15}: 38, fn. 19; \citealt{JS16}: 155) {\sep} \wordng{PCh}{*<ʔa>h<n>át \recind *<ʔå>h<n>át\pla{es}} > \wordng{Ijw}{\mbox{ʔahnát}\pl{is}}; \wordng{I’w}{ahnát\pl{is}}; \wordng{Mj}{ʔahnát\pl{es}} (\citealt{ND09}: 93; \citealt{AG83}: 124; \citealt{JC18}) {\sep} \wordng{PW}{*<hon>hat\plf{*<hon>hát\mbox{-}es}} > \wordng{LB}{hun̥at}; \wordng{Vej}{hon̥at\pl{es}}; \wordng{’Wk}{hon̥at\plf{hon̥át\mbox{-}es \recind hon̥át\mbox{-}iɬ}} (\citealt{VN14}: 48; \citealt{MG-MELO15}: 43; \citealt{KC16}: 154)

\dicnote{We speculate that this was a suffix in PM. In individual languages, it is attached to otherwise unattested roots: \wordng{Nivaĉle}{kots\mbox{-}}, \wordng{Chorote}{*ʔan\mbox{-}} or \intxt{*ʔån\mbox{-}}, and \wordng{Wichí}{*hon\mbox{-}} (the latter two morphemes are also found in the word for\gloss{night}). \wordng{Chorote}{*ʔan\mbox{-} \recind *ʔån\mbox{-}} is likely cognate with \wordng{Wichí}{*hon\mbox{-}} and goes back to Proto-Chorote–Wichí \textit{*X₁₃on\mbox{-}}.}%1

\dicnote{In the Yita’ Lhavos dialect, \textit{o} is unexpectedly raised to \textit{u} in this word.}%2

\dicnote{The Vejoz reflex is mistranscribed as \intxt{honat} in \citet[57]{VU74}.}%3

\lit{\citealt{EN84}: 32 (\intxt{*hnat})}

\PMlemma{{\wordnl{*(\mbox{-})X₂₃pél\pla{its}}{shadow, image}}}

\wordng{Ni}{\mbox{-}xpek}, \textit{\mbox{-}xpek͡l\mbox{-}es} (ShL \textit{\mbox{-}xpik}, \textit{\mbox{-}xpik͡l\mbox{-}is}) [1] (\citealt{NS87}: 124–125; \citealt{JS16}: 155) {\sep} \wordng{PCh}{*\mbox{-}pél\pla{is}} > \wordng{Ijw}{\mbox{-}pέˀl}, \textit{\mbox{-}pέl\mbox{-}is}; \wordng{I’w}{\mbox{-}pél<uk>\pl{is}}; \wordng{Mj}{\mbox{-}péil<ik>}, \textit{\mbox{-}péihl<i>\mbox{-}j} [2] (\citealt{ND09}: 124; \citealt{AG83}: 155; \citealt{JC18}) {\sep} \wordng{PW}{*hpélʰ / *\mbox{-}hpelʰ} > \wordng{LB}{hipeɬ / \mbox{-}peɬ}; \wordng{Vej}{hupel \recind hupeɬ}; \wordng{’Wk}{\mbox{-}húpeɬ / hupéɬ\plf{hupél\mbox{-}is}} (\citealt{VN14}: 278; \citealt{JB09}: 41; \citealt{VU74}: 58; \citealt{MG-MELO15}: 57; \citealt{KC16}: 59, 158)

\dicnote{In Nivaĉle, the Chishamnee Lhavos has innovated with regard to the vowel in the plural suffix, whereas the Shichaam Lhavos has lowered the root vowel.}%1

\dicnote{The Iyo’awujwa’ and Manjui reflexes contain a fossilized suffix (\intxt{\mbox{-}ik}); at least Manjui shows an irregular reflex of \wordng{PCh}{*e} (one would expect \intxt{*\mbox{-}pέl}).}%2

\empr{\citet[306]{AF16} notes the similarity with the Enlhet–Enenlhet term for\gloss{shadow} – \wordng{Enlhet, Enenlhet-Toba, Sanapaná}{peskeska}; \wordng{Guaná}{(m)peskeska} (\citealt{EU-HK-97}: 555; \citealt{EU-HK-MR-03}: 335; \citealt{ASG12}: 130; \citealt{HK-23}: 110) – but this could be accidental.}

\lit{\citealt{EN84}: 10, 25, 28, 36, 53 (\intxt{*phɛl}); \citealt{PVB02}: 144 (\intxt{*χupel}); \citealt{AF16}: 306; \citealt{AnG15}: 253}

\PMlemma{{\wordnl{*X₂₃wéˀlah\plf{*X₂₃wéˀla\mbox{-}ts}}{moon} [1]}}

\wordng{Ni}{xiβeˀk͡la\pl{s}} \citep[148]{JS16} {\sep} \wordng{PCh}{*wéˀlah\plf{*wéˀla\mbox{-}s}} > \wordng{Ijw}{wέˀla\pl{s}}; \wordng{I’w}{wéˀla\pl{s}}; \wordng{Mj}{wέˀla\pl{s}} (\citealt{ND09}: 157; \citealt{AG83}: 169; \citealt{JC18}) {\sep} \wordng{PW}{*ˣwéˀlah\plf{*ˣwéˀla\mbox{-}s}} > \wordng{LB}{weˀla\pl{lis}} [2]; \wordng{Vej}{iwela \recind wela\pl{s}}; \wordng{’Wk}{ʔiwéˀlah\plf{ʔiwéˀla\mbox{-}lis}} [2] (\citealt{VN14}: 48, 334; \citealt{VU74}: 61; \citealt{MG-MELO15}: 44; \citealt{KC16}: 41)

\dicnote{\word{Maká}{xuwel\pl{its}}{moon} (\citealt{AG99}: 395; \citealt{PMA}: 3, 9) is suspiciously similar to the reflexes of \wordng{PM}{*X₂₃wéˀlah} but the sound correspondences do not follow any regular pattern. It could be an early borrowing from \wordng{pre-Nivaĉle}{*xwéˀla}.}%1

\dicnote{The LB and ’Wk plural allomorph does not match the Nivaĉle and Chorote data and is thus considered non-etymological.}%2

\lit{\citealt{EN84}: 35 (\intxt{*iwɛla}); \citealt{PVB02}: 142 (\intxt{*xuweʔla}); \citealt{AnG15}: 253}

\PMlemma{{\wordnl{*ʔaɸqó(t)s}{to crawl} [1]}}

\wordng{Ni}{[t]’aɸkos} \citep[283]{JS16} {\sep} \wordng{PCh}{*[t]’aʍqós} > \wordng{Ijw}{[t]’ahkɔ́xs\mbox{-}ˀn̩}; \wordng{Mj}{[t]’alkɔ́s} [2] (\citealt{ND09}: 153; \citealt{JC18}) {\sep} \wordng{PW}{*[t]qhós} [3] > \wordng{LB}{[ta]qʰus}; \wordng{’Wk}{[t(a)]qʰós} (\citealt{VN14}: 48; \citealt{KC16}: 378)

\dicnote{This verb is semantically and formally similar to \word{PM}{*\mbox{-}ʔaqhuˀts \recind *\mbox{-}ʔaqhúˀts}{knee}, and we believe they may be ultimately etymologically related, but the relation had become opaque by the Proto-Mataguayan stage. The verb \intxt{*ʔaɸqó(t)s} might involve an allomorph of the locative verb \wordng{PM}{*\mbox{-}åˀw\mbox{-}} plus the root for\gloss{knee}. A parallel is seen in Chorote, where the verb for\gloss{to sit (down)} consists of the locative verb plus the locative suffix \word{PCh}{*\mbox{-}heˀn(eʔ)}{downwards}.}%1

\dicnote{\wordng{Mj}{lk} is not known to be the regular reflex of \wordng{PCh}{*ʍq}.}%2

\dicnote{\wordng{PW}{*qh} is not known to be the regular reflex of \wordng{PM}{*ɸq}.}%3

\PMlemma{{\wordnl{*ʔaɸu \recind *ʔaɸú}{woman}}}

\wordng{Mk}{efu\pl{ts}} \citep[141]{AG99} {\sep} \wordng{PCh}{*ʔahwúʔ} > \wordng{I’w}{ʔah(w)úʔ \recind ʔahó\mbox{-} \recind ʔohó\mbox{-}\plf{ʔahó\mbox{-}wet}}; \wordng{Mj}{ʔahwúʔ \recind ʔahwóʔ\plf{ʔahʊ́\mbox{-}wet}} (\citealt{AG83}: 125, 209; \citealt{JC18})

\gc{\citet[314]{PVB13a} notes the similarity with \word{Pilagá}{awó}{woman}.}

\lit{\citealt{PVB13a}: 314 (\intxt{*ahʷu})}

\PMlemma{{\intxt{*\mbox{-}ʔá(j)k’i(h) \recind *\mbox{-}ʔá(j)k’iʔ\plf{*\mbox{-}ʔá(j)k’i\mbox{-}l}} [1]\gloss{saliva}}}

\wordng{Ni}{\mbox{-}ʔatʃ’i\pl{k}} \citep[37]{JS16} {\sep} \wordng{PCh}{*\mbox{-}ájk’i<l><is>} [2] > \wordng{Ijw}{\mbox{-}áts’ilis} [2]; \wordng{I’w}{\mbox{-}átsilis\pl{is}} [3]; \wordng{Mj}{\mbox{-}áʔalis} (\citealt{ND09}: 129; \citealt{AG83}: 123; \citealt{JC18}) {\sep} \wordng{PW}{*\mbox{-}ɬ\mbox{-}ák’i<lʰ>} > \wordng{LB}{\mbox{-}ɬ\mbox{-}atʃ’iɬ}; \wordng{’Wk}{\mbox{-}ɬ\mbox{-}ákʲ’iɬ} (\citealt{JB09}: 73; \citealt{KC16}: 72)

\dicnote{Nivaĉle and Wichí point to \intxt{*\mbox{-}ʔák’i(h) \recind *\mbox{-}ʔák’iʔ}, and Chorote to \intxt{*\mbox{-}ʔájk’i(h) \recind *\mbox{-}ʔájk’iʔ}.}%1

\dicnote{In Chorote and Wichí, the plural form of PM has been reanalyzed as a singular one, with the erstwhile plural suffix being reinterpreted as a part of the root. In Chorote, the process occurred even twice, with the innovative plural suffix \intxt{*\mbox{-}is} being fossilized to the root.}%2

\dicnote{The plain (non-ejective) \intxt{ts} in \cits{AG83} attestations of the Iyo’awujwa’ reflex must be a mistranscription.}%3

\PMlemma{{\wordnl{*[t]’áˀɬ}{to ask}}}

\wordng{Ni}{[t]’aˀɬ} \citep[282]{JS16} {\sep} \wordng{PCh}{*[t]’ǻɬ} [1] > \wordng{Ijw}{[t]’aɬ\mbox{-}\APPL}; \wordng{I’w}{\mbox{-}áhl\mbox{-}am}; \wordng{Mj}{[t]’aɬ} (\citealt{JC14b}: 80; \citealt{ND09}: 154; \citealt{AG83}: 123; \citealt{JC18}) {\sep} \wordng{PW}{*[t]’áɬ} > \wordng{LB}{[t]’aɬ\mbox{-}a}; \wordng{Vej}{[t]’áɬ}; \wordng{’Wk}{[t]’áɬ} (\citealt{VN14}: 239; \citealt{VU74}: 77; \citealt{KC16}: 431)

\dicnote{\wordng{PCh}{*å} (as opposed to \intxt{*a}) is reconstructed based on the behavior of its reflex in Iyojwa’aja’: in forms such as \textit{hitʲ\mbox{-}’áhl\mbox{-}e}\gloss{you ask} \citep[154]{ND09} it fails to undergo raising to [e], as is typical of \wordng{PCh}{*a}. \wordng{PCh}{*å} is not the regular reflex of \wordng{PM}{*a}.}%1

\PMlemma{{\textit{*ʔáɬu(ʔ)\pla{ts}}\gloss{iguana}; \textit{*ʔáɬu\mbox{-}taχ}, \textit{*ʔáɬu\mbox{-}ta\mbox{-}ts}\gloss{alligator}}}

\wordng{Ni}{ʔaɬu\pl{s}}; \textit{ʔaɬu\mbox{-}tax}, \textit{ʔaɬu\mbox{-}ta\mbox{-}s} \citep[43]{JS16} {\sep} \wordng{PCh}{*ʔáhluʔ\pla{s}}; \textit{*ʔáhlu\mbox{-}tah}, \textit{*ʔáhlu\mbox{-}ta\mbox{-}s} > \wordng{Ijw}{ʔáhlʲuʔ\pl{s}}; \textit{ʔáhlʲu\mbox{-}tʲe\pl{hes}} [1]; \wordng{I’w}{ʔáhluʔ\pl{s}}; \textit{ʔáhlu\mbox{-}tah\pl{as}} [1]; \wordng{Mj}{ʔáhluʔ\pl{s}}; \textit{ʔáhlu\mbox{-}ta \recind ʔáhlu\mbox{-}t(ʲ)e\pl{s}} (\citealt{JC14b}: 100, fn. 35; \citealt{ND09}: 93; \citealt{AG83}: 123–124; \citealt{JC18}) {\sep} \wordng{PW}{*ʔáɬu}; \textit{*ʔáɬu\mbox{-}taχ}, \textit{*ʔáɬu\mbox{-}t\mbox{-}as} > \wordng{LB}{ʔaɬe}; \textit{ʔaɬe\mbox{-}taχ}; \wordng{Vej}{aɬu\pl{ɬajs}}; \textit{aɬu\mbox{-}tah}, \textit{aɬu\mbox{-}tas} [2]; \wordng{’Wk}{ʔáɬuʔ\pl{lis}}; \textit{ʔáɬu\mbox{-}tax}, \textit{ʔáɬu\mbox{-}t\mbox{-}as} (\citealt{VN14}: 197; \citealt{VU74}: 50; \citealt{MG-MELO15}: 20; \citealt{AFG067}: 221; \citealt{KC16}: 11)

\dicnote{The plurals \intxt{ʔáhlʲu\mbox{-}tʲeh\mbox{-}es} (Iyojwa’aja’), \intxt{ʔáhlu\mbox{-}tah\mbox{-}as} (Iyo’awujwa’)\gloss{alligators} are non-etymological; all other languages and varieties point to \wordng{PM}{*ʔáɬu\mbox{-}ta\mbox{-}ts}, which would yield \wordng{Iyojwa’aja’}{*ʔáhlu\mbox{-}tʲe\mbox{-}s}, \wordng{Iyo’awujwa’}{*ʔáhlu\mbox{-}ta\mbox{-}s}.}%1

\dicnote{\citet[50]{VU74} mistranscribes the Vejoz term for\gloss{iguana} as \textit{aʔɬu}.}%2

\lit{\citealt{EN84}: 10, 27 (\intxt{*ahlu}; \textit{*ahlutha}); \citealt{AnG15}: 254}

\PMlemma{{\wordnl{*ʔámʔåh\plf{*ʔámʔå\mbox{-}ts}}{rat}}}

\wordng{Ni}{ʔamʔå\pl{s}} \citep[43]{JS16} {\sep} \wordng{PCh}{*ʔámʔah \recind *ʔámʔåh\plf{*ʔámʔa\mbox{-}s \recind *ʔámʔå\mbox{-}s}} > \wordng{I’w}{ʔámaa\pl{s}}; \wordng{Mj}{ʔám(a)ʔa\pl{s}} (\citealt{AG83}: 120; \citealt{JC18}) {\sep} \wordng{PW}{*ʔáma} [1] > \wordng{LB}{ʔama}; \wordng{Vej}{ama\pl{ɬajis}}; \wordng{’Wk}{ʔámaʔ} (\citealt{VN14}: 161; \citealt{VU74}: 50; \citealt{MG-MELO15}: 20; \citealt{KC16}: 12)

\dicnote{Wichí must have undergone irregular vowel harmony (\intxt{*a…å > *a…a}). Chorote may have also participated in this sound change, but it is not recoverable whether this is the case.}%1

\lit{\citealt{EN84}: 10 (\intxt{*hmaa})}

\PMlemma{{\wordnl{*ʔáp’a(ˀ)χ \recind *ʔáɸ’a(ˀ)χ}{jararaca}}}

\wordng{Ni}{ʔap’ax} [1], \textit{ʔapx\mbox{-}as} (\citealt{AnG20}: 286–287) {\sep} \wordng{PCh}{*ʔáp’ah} > \wordng{Ijw}{ʔáp’a\mbox{-}ki\pl{jis}}; \wordng{I’w}{ʔáˀpah\pl{as}}; \wordng{Mj}{ʔáp’a\pl{s}} (\citealt{ND09}: 94; \citealt{AG83}: 121; \citealt{JC18})

\dicnote{\citet[27]{LC20} attest the variant \intxt{ʔaʔp’ax}, where [ʔp’] is likely an allophone of /p’/.}%1

\lit{\citealt{EN84}: 9 (\intxt{*ap’áq})}

\PMlemma{{\wordnl{*ʔaqǻjeˀk}{wild honey} [1]}}

\wordng{Ni}{ʔakåjetʃ\plf{ʔakåjxe\mbox{-}s} / \mbox{-}ˀβ\mbox{-}åkåjetʃ} \citep[36]{JS16} {\sep} \wordng{PW}{*ʔaqǻjeq} > \wordng{LB}{ʔaqojeq}; \wordng{Vej}{k’åjek} [2]; \wordng{’Wk}{ʔaqǻjek} (\citealt{VN14}: 350; \citealt{VU74}: 63; \citealt{KC16}: 14)

\dicnote{This is obviously a derivative from \word{PM}{*\mbox{-}ajeˀk \recind *\mbox{-}ajéˀk}{honey comb}.}%1

\dicnote{Vejoz \textit{k’åjek} is not a regular reflex of \wordng{PW}{*ʔaqǻjeq}.}%2

\PMlemma{{\wordnl{*ʔáqåtse(ˀ)χ}{kind of armadillo}}}

(?) \word{Mk}{enqetsaχ<hiɬehus>}{six-banded armadillo} [1] \citep[51]{JB81} {\sep} \word{Ni}{ʔakåtse\mbox{-}tax\plf{ʔakåtse\mbox{-}ta\mbox{-}s}}{six-banded armadillo} \citep[36]{JS16} {\sep} \word{PCh}{*ʔáqåsah}{nine-banded armadillo} > \wordng{Ijw}{ʔákasa}; \wordng{Mj}{ʔɔ́kasa} [2] (\citealt[93]{ND09}; \citealt{JC18})

\dicnote{The Maká reflex shows a number of irregularities, provided it is related at all. The expected reflex would be \intxt{*aqatsaχ}.}%1

\dicnote{The Manjui reflex has irregularly rounded the stressed vowel.}%2

\PMlemma{{\textit{*ʔa(C)qǻχ}, \textit{*ʔa(C)qǻ\mbox{-}ts} [1]\gloss{rich, pleasant, tasty}}}

\wordng{Ni}{ʔakåx\plf{ʔakå\mbox{-}s}} \citep[36]{JS16} {\sep} \wordng{PCh}{*\mbox{-}(ʔaC)qǻh\mbox{-}\plf{*\mbox{-}(ʔaC)qǻ\mbox{-}s\mbox{-}}} [1] > \wordng{Ijw}{\mbox{-}(ʔah)káh\mbox{-}eʔ}, \textit{\mbox{-}(ʔah)ká\mbox{-}s\mbox{-}iʔ}; \wordng{I’w}{\mbox{-}káh\mbox{-}ej \recind \mbox{-}káh\mbox{-}aj}; \wordng{Mj}{\mbox{-}(ʔam)káh\mbox{-}(…) in}\gloss{happy, rich} (\citealt{ND09}: 108; \citealt{AG83}: 138; \citealt{JC18}) {\sep} \wordng{PW}{*ʔaqǻχ}, \textit{*ʔaqǻ\mbox{-}s}\gloss{pleasant, tasty} > \wordng{LB}{ʔaqoχ}; \wordng{’Wk}{ʔaqǻx}, \textit{\mbox{-}ʔáqå\mbox{-}s} (\citealt{VN14}: 197; \citealt{KC16}: 13)

\dicnote{Chorote suggests that there was a consonant between \wordng{PM}{*a} and \intxt{*q}, but Iyojwa’aja’ and Manjui point to different consonants (the former to \wordng{PM}{*ɸ} or \intxt{*ɬ} > \wordng{PCh}{*ʍ} or \intxt{*ɬ}, the latter to \intxt{*m}).}%1

\PMlemma{{\wordnl{*\mbox{-}ʔaqhuˀts \recind *\mbox{-}ʔaqhúˀts}{knee}}}

\wordng{Mk}{\mbox{-}aqhuˀts} [1] (\intxt{\mbox{-}ij}) \citep[127]{AG99} {\sep} \wordng{Ni}{\mbox{-}(ʔa)kxuˀs}, \textit{\mbox{-}(ʔa)kxatsu\mbox{-}j} (\citealt{JS16}: 70, 354) {\sep} \wordng{PCh}{*\mbox{-}ʔaqús} > \wordng{Ijw}{\mbox{-}ʔakós / \mbox{-}kós\mbox{-}ki}; \wordng{I’w}{\mbox{-}kós(\mbox{-}hl\mbox{-}étik\mbox{-}iʔ)}; \wordng{Mj}{\mbox{-}(ʔa)kʊ́s}, \textit{\mbox{-}ʔakʊ́ʃ\mbox{-}is} (\citealt{ND09}: 123, 154; \citealt{AG83}: 144, 219; \citealt{JC18})

\dicnote{The Maká noun is not attested in \citet{unuuneiki}, \citet{maka-etnomat,PMA}, or the New Testament, where only the verb \textit{[wo]nokokʼen}\gloss{to kneel} is found (Mark 15:19); the presence of a preglottalized coda in Maká is thus inferred based on the Nivaĉle cognate. The absence of a stem-initial \intxt{ʔ} in Maká could be a mistranscription.}%1

\lit{\citealt{EN84}: 24 (\intxt{*t’aqawsq}); \citealt{LC-VG-07}: 15}

\PMlemma{{\wordnl{*\mbox{-}ʔaqaˀt \recind *\mbox{-}ʔaqáˀt}{chin}}}

\word{Ni}{\mbox{-}(ʔa)kaˀt\plf{\mbox{-}(ʔa)kat\mbox{-}is}}{chin, barbel} \citep[152]{LC20} {\sep} \wordng{PCh}{*\mbox{-}ʔakát} > \wordng{Ijw}{\mbox{-}ʔakát} \citep[154]{ND09}

\gc{Obviously related to \word{Proto-Guaicuruan}{*\mbox{-}aq’ád}{chin} (\citealt{PVB13b}, \#101).}

\PMlemma{{\wordnl{*ʔatuˀχ \recind *ʔatúˀχ}{snake sp.}}}

\word{Ni}{ʔatuˀx\plf{ʔatux\mbox{-}is}}{Argentine boa} \citep[50]{JS16} {\sep} \wordng{PCh}{*ʔatúh} > \word{Ijw}{ʔatóh}{a kind of snake (yellow, large, agressive when it eats)} \citep[95]{ND09}

\PMlemma{{\textit{*ʔáwu(C)tseχ} [1]\gloss{Chacoan peccary; collared peccary}}}

\wordng{Ni}{ʔaβuktsex\plf{ʔaβuktse\mbox{-}s} \recind ʔaβoktsex\plf{ʔaβoktse\mbox{-}s}} [2]\gloss{Chacoan peccary} (\citealt{JS16}: 51; \citealt{LC20}: 23) {\sep} \wordng{PCh}{*ʔáwusah} > \word{Ijw}{ʔáʊxse\plf{ʔáʊxseh\mbox{-}es}}{collared peccary}; \word{Mj}{ʔáwaxsa}{Chacoan peccary} (\citealt{ND09}: 95; \citealt{JC18}) {\sep} \wordng{PW}{*ʔáwutsaχ} > \word{LB}{ʔawetsaχ}{collared peccary}; \wordng{Vej}{awutsah}, \wordng{’Wk}{\mbox{ʔáwutsax}\plf{ʔáwutsʰ\mbox{-}as}} (\citealt{JB09}: 38; \citealt{VU74}: 51; \citealt{MG-MELO15}: 20; \citealt{KC16}: 19)

\dicnote{Nivaĉle points to \wordng{PM}{*ʔáwoltseχ} or \intxt{*ʔáwoktseχ}, whereas Chorote and Wichí point to \intxt{*ʔáwutseχ}.}%1

\dicnote{The form \intxt{ʔaβoktsex\plf{ʔaβoktse\mbox{-}s}} with the unexpected vowel \intxt{o} is attested in \citet[51]{JS16}, whereas \citet[23]{LC20} give \intxt{ʔaβuktsex\plf{ʔaβuktse\mbox{-}s}}.}%2

\PMlemma{{\wordnl{*ʔáxaʔ}{stork}}}

\word{Mk}{exeʔ\pl{l}}{maguari stork} (\citealt{AG99}: 167; \citealt{PMA}: 5) {\sep} \wordng{PCh}{*ʔáhaʔ} > \word{Ijw}{ʔáhaʔ}{jabiru} \citep[93]{ND09}

\lit{\citealt{PVB02}: 142 (\intxt{*axaʔ})}

\PMlemma{{\intxt{*ʔaX₁₃ǻje(ˀ)χ} (fruit); \intxt{*ʔaX₁₃ǻj\mbox{-}uˀk\plf{*ʔaX₁₃ǻj\mbox{-}ku\mbox{-}jʰ}} (tree)\gloss{mistol\species{Ziziphus \mbox{mistol}}}}}

\wordng{Ni}{ʔaxåjex}; \intxt{ʔaxåj\mbox{-}uk\plf{ʔaxåj\mbox{-}ku\mbox{-}j}} (\citealt{JS16}: 41–42) {\sep} \wordng{PCh}{*ʔahǻjah}; \intxt{*ʔahǻj\mbox{-}uk\plf{*ʔahǻj\mbox{-}ku\mbox{-}jʰ}} > I’w —; \intxt{aháj\mbox{-}ik\plf{aháj\mbox{-}si\mbox{-}ʔ}}; \wordng{Mj}{ʔaháje\pl{l}}; \intxt{ʔaháj\mbox{-}uk} (\citealt{AG83}: 123; \citealt{JC18}) {\sep} \wordng{PW}{*ʔahǻjaχ}; \intxt{*ʔahǻj\mbox{-}ukʷ} > \wordng{LB}{(ʔa)hojaχ}; \intxt{(ʔa)hojekʷ} [1]; \wordng{Vej}{ahåjak}; \intxt{ahåj\mbox{-}uk} [2]; \wordng{’Wk}{ʔahǻjax}; \intxt{ʔahǻj\mbox{-}uk} (\citealt{CS08}: 60; \citealt{VN14}: 192, 340; \citealt{MG-MELO15}: 16; \citealt{KC16}: 9)

\dicnote{In Lower Bermejeño, there appears to be a variant with an irregular loss of the initial vowel. \citet{VN14} gives the forms \intxt{ʔahojaχ}, \intxt{hojekʷ}. \citet{CS08}, by contrast, documents the \intxt{hojaχ}, \intxt{ʔahojekʷ}.}%1

\dicnote{The final \intxt{\mbox{-}k} in the name of the fruit in Vejoz is irregular. \citet[50]{VU74} mistranscribes the name of the tree as \textit{aha\mbox{-}juk}.}%2

\lit{\citealt{LC-VG-07}: 19}

\PMlemma{{\wordnl{*\mbox{-}ʔáX₂₃te(ʔ)\pla{jʰ}}{female breast}}}

\wordng{Ni}{\mbox{-}ʔaxte\pl{j}} \citep[42]{JS16} {\sep} \wordng{PCh}{*\mbox{-}ʔáhateʔ\pla{jʰ}} > \wordng{Ijw}{\mbox{-}ʔáhate} [1]; \wordng{Mj}{\mbox{-}ʔáateʔ\pl{j}} (\citealt{ND09}: 153; \citealt{JC18}) {\sep} \wordng{PW}{*\mbox{-}t’\mbox{-}áte\pla{jʰ}} > \wordng{LB}{\mbox{-}t\mbox{-}’ate}; \wordng{Vej}{\mbox{-}t\mbox{-}’ate}; \wordng{’Wk}{\mbox{-}t\mbox{-}’áteʔ\pl{ç}} (\citealt{VN14}: 164; \citealt{JB09}: 59; \citealt{VU74}: 78; \citealt{KC16}: 96)

\dicnote{The absence of a word-final glottal stop in \cits{ND09} attestation of this noun must be a mistranscription.}%1

\PMlemma{{\wordnl{*ʔǻˀjteχ\plf{*ʔǻˀjte\mbox{-}ts}}{to hurt}}}

\wordng{Mk}{aʔtaχ\plf{aʔti\mbox{-}ts}} [1] \citep[130]{AG99} {\sep} \wordng{Ni}{ʔåˀβteχ\plf{ʔåˀβte\mbox{-}s} \recind ʔåˀjteχ} [2] (\citealt{AnG15}: 27; \citealt{JS16}: 45; \citealt{LC20}: 102, 166) {\sep} \wordng{PCh}{*ʔǻˀjtah\mbox{-}}\textsc{appl}, \textit{*\mbox{-}ʔǻˀjte\mbox{-}s\mbox{-}\APPL} > \wordng{Ijw}{ʔáʔtʲeh\mbox{-}eʔ \recind ʔáʔtih\mbox{-}iʔ\plf{\mbox{-}ʔáʔti\mbox{-}s\mbox{-}iʔ}} [3]; \wordng{I’w}{átih\mbox{-}iʔ}; \wordng{Mj}{ʔátih\mbox{-}\APPL} [4] (\citealt{JC14b}: 90; \citealt{ND09}: 96; \citealt{AG83}: 122; \citealt{JC18}) {\sep} \wordng{PW}{*ʔǻjtaχ\plf{*ʔǻjte\mbox{-}s}} > \wordng{LB}{ʔojtaχ}; \wordng{Vej}{ʔåjtah} [5]; \wordng{’Wk}{\mbox{ʔǻjtax}\plf{ʔǻjte\mbox{-}s}} (\citealt{VN14}: 403; \citealt{MG-MELO15}: 32; \citealt{KC16}: 8)

\dicnote{\citet[130]{AG99} documents this as \intxt{a(ʔ)taχ\plf{ati\mbox{-}ts}}. In the New Testament, only \intxt{aʔtaχ\plf{aʔti\mbox{-}ts}} is attested (1 Corinthians 13:7; Romans 3:16).}%1

\dicnote{The Nivaĉle variant with \textit{j} is attested in \citet[45]{JS16} only. Note that the rhyme \textit{åˀβ} is phonetically realized as [ɑɔ̯β] \citep[27]{AnG15} or [aʔɑw] \citep{LC20}.}%2

\dicnote{\citet[96]{ND09} mistranscribed the plural form of Iyojwa’aja’ as \textit{\mbox{-}ʔáʔti\mbox{-}s\mbox{-}i}.}%3

\dicnote{The loss of \intxt{*ˀj} in Manjui is irregular.}%4

\dicnote{\citet[51]{VU74} mistranscribes the Vejoz reflex \textit{ajtah}.}%5

\lit{\citealt{RJH15}: 240}

\PMlemma{{\wordnl{*ˀ[n]åˀl, \textsc{caus}~\intxt{*ˀ[n]ål\mbox{-}it \recind [ji]ˀn\mbox{-}ǻl\mbox{-}it}}{to be visible}}}

\word{Mk}{[n]aˀl / \mbox{-}aˀl}{to be present, to exist} [1], \wordnl{[n]aˀl(\mbox{-}\APPL)\mbox{-}kij}{to be (of light)} [1], \wordnl{[n]aˀl\mbox{-}ip\mbox{-}xiʔ}{to be illuminated from above} [1], \textsc{caus}~\wordnl{[n]\mbox{-}al\mbox{-}it\mbox{-}ik’i}{to illuminate} \citep[117]{AG99} {\sep} \wordng{Ni}{[n]åˀk / \mbox{-}ʔåˀk}, \textsc{caus}~\intxt{[ji]n\mbox{-}åk͡l\mbox{-}it}, \intxt{[ta]n\mbox{-}åˀk\mbox{-}ɬanit}; \wordng{ChL}{[n]åˀk / \mbox{-}åˀk} [2], \textsc{caus}~\intxt{n\mbox{-}åk͡l\mbox{-}it / \mbox{-}ˀn\mbox{-}åk͡l\mbox{-}it} [2] (\citealt{JS16}: 199, 200; \citealt{LC20}: 79) {\sep} \wordng{PCh}{*ˀ<n>ǻl} > \word{Mj}{ˀnál}{to be visible, to appear nitidly} (\citealt{ND09}: 162; \citealt{JC18}) {\sep} \wordng{PW}{*ˀ<n>åˀl / *ˀ<n>ål-{\APPL} / *ˀ<n>ån-\APPL} [3], \textsc{caus}~\intxt{*[hi]ˀ<n>ǻl\mbox{-}it / *[hi]ˀ<n>ǻl\mbox{-}t\mbox{-}} > \word{LB}{ˀnol<eχ> \recind ˀno<χ>}{apparently} [4]; \wordng{Vejoz or Guisnay}{ˀnål / ˀnål\mbox{-}{\APPL} / ˀnån̥\mbox{-}{\APPL} / ˀnån\mbox{-}\APPL}, \textsc{caus}~\intxt{\mbox{-}ˀnǻl\mbox{-}it / \mbox{-}ˀnǻl\mbox{-}t\mbox{-}}; \wordng{’Wk}{ˀnåˀl \recind ˀnåɬ / ˀnål\mbox{-}{\APPL} / ˀnån̥\mbox{-}\APPL}, \textsc{caus}~\intxt{[hi]ˀnǻl\mbox{-}it / [hi]ˀnǻl\mbox{-}t\mbox{-}} (\citealt{VN14}: 334–335; \citealt{RL16}: 69; \citealt{KC16}: 50–52)

\dicnote{The preglottalized coda in Maká is attested in the New Testament (e.g. Juan 8:58; Mark 8:18; Revelations 16:18). The loss of the stem-initial glottal stop is irregular, except in third-person forms with the prefix \intxt{n\mbox{-}}, where it is expected. It is possible that the stem was remodeled based on the third-person forms.}%1

\dicnote{The forms attested in \citet{LC20} (presumably representative of the Chishamnee Lhavos dialect) show an irregular loss of the stem-initial glottal stop in the underived verb (as seen in \wordnl{ts\mbox{-}åˀk}{I appear}); the expected reflex is documented in \citet{JS16}. Conversely, when the root is preceded by the prefix \intxt{\mbox{-}n\mbox{-}}, the underlying glottal stop shows up in Chishamnee Lhavos, but not in \cits{JS16} data.}%2

\dicnote{The allomorph \intxt{*ˀnån-} in Wichí expectedly appears before \intxt{*h}\mbox{-}initial suffixes.}%3

\dicnote{The Lower Bermejeño particle \intxt{ˀnoleχ \recind ˀnoχ}, with an optional irregular loss of two segments, goes back to \word{PW}{*ˈˀnål\mbox{-}eχ}{to look like, to appear as}.}%4

\PMlemma{{\intxt{*\mbox{-}ʔå(ˀ)l}, \third{*ˀ[j]i(ˀ)l} [1]\gloss{to die}}}

Mk (Lengua doculect) ‹al›, ‹il› \citep[496]{EP98} {\sep} \wordng{PCh}{*ˀ[j]ǻ(ˀ)l} > \wordng{Ijw}{ˀ[j]áˀl}; \wordng{I’w}{[j]él / \mbox{-}ál / \mbox{-}áhl\mbox{-}} (\citealt{JC14b}: 78, 79, fn. 8; \citealt{ND09}: 165; \citealt{AG83}: 78, 119, 208) {\sep} \wordng{PW}{*ˀ[j]ilʰ} > \wordng{LB}{ˀ[j]iɬ}; \wordng{Vej}{[j]iɬ} [2]; \wordng{’Wk}{ˀ[j]iɬ} (\citealt{VN14}: 292; \citealt{AFG067}: 218, 219; \citealt{KC16}: 124)11

\dicnote{This verb evidently presented the same alternation as \wordng{PM}{*\mbox{-}åp}, \third{*ˀ[j]ip}\gloss{to cry}. Chorote and Wichí generalized the allomorphs with \textit{*å} and \intxt{*i}, respectively. The reconstruction of the presence or absence of glottalization in the final consonant is uncertain because diagnostic cognates in modern Maká, Manjui, and Nivaĉle are lacking.}%1

\dicnote{The absence of a glottal stop or glottalization in the root-initial position in Vejoz could result from mistranscription. \citet[84]{VU74} documents the verb as \textit{[j]ijl}.}%2

\PMlemma{{\wordnl{*ʔåˀlå}{South American rattlesnake; caninana}; \wordnl{*ʔåˀlǻ\mbox{-}taχ}{Argentine boa}}}

\word{Ni}{ʔåˀk͡lå\pl{s}}{South American rattlesnake; caninana}; \wordnl{ʔåˀk͡lå\mbox{-}tax\plf{ʔåˀk͡lå\mbox{-}ta\mbox{-}s}}{jararaca or similar snake\species{Bothrops alternatus; Xenodon merremii; Bothrops neuwedi meridionalis; Lystrophis dorbignyi}} \citep[210]{JS16} {\sep} \wordng{PCh}{*ʔåˀlǻ<tah> \recind *ʔåˀlá<tah>\plf{*ʔåˀlǻ<ta>\mbox{-}s \recind *ʔåˀlá<ta>\mbox{-}s}} > \wordng{Ijw}{ʔaˀlátah\pl{as}} [1]; \wordng{I’w}{alátah\plf{aláta\mbox{-}s}}; \wordng{Mj}{ʔaˀláta\pl{s}} (\citealt{ND09}: 95; \citealt{AG83}: 119; \citealt{JC18}) {\sep} (?) \wordng{PW}{*lá<taχ>} [2] > \wordng{LB}{lataχ} \citep[368]{VN14}

\dicnote{The Iyojwa’aja’ plural form is non-etymological.}%1

\dicnote{Lower Bermejeño \textit{lataχ} is not the expected reflex of \wordng{PM}{*ʔåˀlǻ\mbox{-}taχ}; one would rather expect \intxt{*ʔoˀlotaχ}. It is possible that the Wichí term does not belong to this etymology altogether.}%2

\PMlemma{{\textit{*ʔǻl(V)tse(ˀ)χ}, \textit{*ʔǻl(V)tse\mbox{-}ts} [1]\gloss{cháguar (\intxt{Bromelia urbaniana = Deinacanthon urbanianum})}}}

\wordng{Ni}{ʔåktsex}, \textit{ʔåktse\mbox{-}s}\gloss{\textit{Dyckia chaguar}} \citep[209]{JS16} {\sep} \wordng{PCh}{*ʔǻlVsah}, \textit{*ʔǻlVse\mbox{-}s} [2] > \wordng{Ijw}{ʔálisa / \mbox{-}ˀw\mbox{-}álisa}; \wordng{I’w}{álisa}, \textit{álisi\mbox{-}s}; \wordng{Mj}{ʔálasa / \mbox{-}w\mbox{-}álasa} (\citealt{JC14b}: 99; \citealt{ND09}: 94, 127; \citealt{AG83}: 120; \citealt{JC18}) {\sep} \wordng{PW}{*ʔǻletsaχ} > \wordng{LB}{ʔoletsaχ} (\citealt{CS08}: 59; \citealt{VN14}: 48; \citealt{MS14}: 225)

\dicnote{The Nivaĉle form points to \wordng{PM}{*ʔǻltseχ}, the Chorote one to \wordng{PM}{*ʔǻlVtseχ}, and the Wichí one to \wordng{PM}{*ʔǻletseχ}.}%1

\dicnote{\wordng{PCh}{*V} can stand for any vowel that fails to cause both the first and the second palatalization in Chorote (such as \textit{*a} or \intxt{*å}).}%2

\PMlemma{{\textit{*ʔǻnhajeχ} (bean); \textit{*ʔǻnhaj\mbox{-}uˀk} (plant); \textit{*ʔǻnhaje\mbox{-}ˀp} (season)\gloss{\textit{Capparis retusa}}}}

\wordng{Mk}{anhejaχ}; \intxt{anhej\mbox{-}uˀk}; \textit{anheji\mbox{-}ˀp} (\citealt{JB81}: 77; \citealt{AG99}: 121; \citealt{maka-etnomat}: 23–25, \citeyear{PMA}: 7) {\sep} \wordng{Ni}{ʔånxajex}; \intxt{ʔånxaj\mbox{-}uk}; \intxt{ʔånxaje\mbox{-}p} \citep[212]{JS16} {\sep} \wordng{PCh}{*ʔóhnajah}; \intxt{*ʔóhnaj\mbox{-}uk\plf{*ʔóhnaj\mbox{-}ku\mbox{-}jʰ}} [1] > \wordng{Ijw}{ʔɔ́hnajeʔ} [2]; \wordng{Mj}{ʔɔ́hnaje \recind ʔɔ́hnaji \recind ʔɔ́hneje} [3]; \intxt{ʔɔ́hnaj\mbox{-}ik\plf{ʔɔ́hnaj\mbox{-}ʃi\mbox{-}j}} (\citealt{ND09}: 142; \citealt{JC18}) {\sep} \wordng{PW}{*ʔǻnhjaχ}; \intxt{*ʔǻnhj\mbox{-}ukʷ} [4] > \wordng{LB}{ʔonjaχ}, \intxt{ʔonj\mbox{-}ekʷ} [5]; \wordng{Vej}{ån̥jax}; \intxt{ån̥j\mbox{-}uk}; \wordng{’Wk}{ʔǻn̥jax} (\citealt{CS08}: 60; \citealt{VN14}: 324, 403; \citealt{MG-MELO15}: 17; \citealt{KC16}: 7)

\dicnote{We surmise that the vowel of the first syllable is irregularly reflected in Chorote as \wordng{PCh}{*o} due to the contamination with \wordng{PCh}{*ʔóhnaʔ}\gloss{\textit{Capparis salicifolia} fruit}.}%1

\dicnote{The word-final \intxt{\mbox{-}ʔ} in the Iyojwa’aja’ form is irregular.}%2

\dicnote{The Manjui variant \intxt{ʔɔ́hneje} is irregular.}%3

\dicnote{The loss of \wordng{PM}{*a} in the Wichí form is irregular.}%4

\dicnote{The voiced nasal \textit{n} in the Lower Bermejeño Wichí form is irregular.}%5

\PMlemma{{\wordnl{*ʔǻnitih}{wasp sp.}}}

\wordng{Ni}{ʔåniti\pl{s}}\gloss{red paper wasp swarm} \citep[211]{JS16} {\sep} \wordng{PCh}{*ʔǻnitih} > \wordng{Ijw}{ʔániti\pl{jis}}\gloss{black wasp} \citep[94]{ND09}

\citealt{EN84}: 16 (\intxt{*ånthi})

\PMlemma{{\wordnl{*[t]’ås}{to step}}}

\wordng{Ni}{[t]’ås} \citep[289]{JS16} {\sep} \wordng{PCh}{*[t]’ǻs} > \wordng{Ijw}{[t]’ás}; \wordng{I’w}{[t]áts\mbox{-}eʔ / \mbox{-}áhts\mbox{-}eʔ} [1]; \wordng{Mj}{[t]’as} (\citealt{ND09}: 154; \citealt{AG83}: 124, 215; \citealt{JC18}) {\sep} \wordng{PW}{*[t]’ås\mbox{-}\APPL} > \wordng{LB}{[t]’os\mbox{-}\APPL}; \wordng{Vej}{[t]’ås\mbox{-}\APPL}; \wordng{’Wk}{[t]’ås\mbox{-}\APPL} (\citealt{VN14}: 239; \citealt{VU74}: 78; \citealt{KC16}: 429–430)

\dicnote{The Iyo’awujwa’ reflex is attested as \textit{[t]áts\mbox{-}eʔ / \mbox{-}áhts\mbox{-}eʔ} in \citet{AG83}, which is likely a mistranscription for \textit{[t]’á(h)ts\mbox{-}’eʔ / \mbox{-}á(h)ts\mbox{-}’eʔ} (where the initial glottal stop of the applicative \intxt{\mbox{-}ʔeʔ} fuses with the underlying /s/ as \textit{(h)ts’}). The underived verb most probably exists in the language but is not documented in the cited work.}%1

\PMlemma{{\wordnl{*ʔåsk’ä́la(ˀ)χ}{widower}; \wordnl{*ʔåsk’ä́l(a)\mbox{-}keʔ}{widow}}}

\wordng{Ni}{ʔåstʃ’ak͡lax\pl{is}}; \intxt{ʔåstʃ’ak\mbox{-}tʃe\pl{j}} \citep[213]{JS16} {\sep} \wordng{PCh}{*ʔåsk’élah}; \intxt{*ʔåsk’éla\mbox{-}keʔ\pla{jʰ}} > \wordng{Ijw}{ʔask’ílʲe}; \intxt{ʔask’ílʲe\mbox{-}ki} [1]; \wordng{I’w}{astʲéla\pl{s}}; \intxt{astʲéla\mbox{-}kiʔ}; \wordng{Mj}{ʃiʔéla\pl{s}}; \intxt{ʃiʔéla\mbox{-}kiʔ\pl{j}} [2] (\citealt{ND09}: 94; \citealt{AG83}: 122; \citealt{JC18})

\dicnote{The absence of a word-final glottal stop in \cits{ND09} attestation of this noun must be a mistranscription.}%1

\dicnote{The development of the initial syllable in Manjui is entirely irregular.}%2

\lit{\citealt{LC-VG-07}: 22}

\PMlemma{{\textit{*ʔåtits \recind *ʔåtíts \recind *ʔåtets \recind *ʔåtéts} [1]\gloss{wild pepper}}}

\wordng{Mk}{atits} [1] (\intxt{\mbox{-}ket}) \citep[132]{AG99} {\sep} \wordng{PCh}{*ʔåtés} > \wordng{I’w}{ʔatés}; \wordng{Mj}{ʔatέs}, \textit{ʔatέs \recind ʔatέ(h)ʃ\mbox{-}is} (\citealt{AG83}: 122; \citealt{JC18})

\dicnote{The reconstructions \textit{*ʔåtiˀts \recind *ʔåtíˀts \recind *ʔåteˀts \recind *ʔåtéˀts} are ruled out because the Maká reflex is attested with a plain coda in \citet[80]{JB81}.}%1

\PMlemma{{\textit{*\mbox{-}ʔåx\pla{íts}}\gloss{skin, bark}}}

\wordng{Mk}{\mbox{-}ʔax\pl{its}} \citep[135]{AG99} {\sep} \wordng{Ni}{\mbox{-}ʔåx\pl{is}} \citep[355]{JS16} {\sep} \wordng{PCh}{*\mbox{-}ʔǻh}, \textit{*\mbox{-}ʔåh\mbox{-}és} > \wordng{Ijw}{\mbox{-}ʔáh\plf{\mbox{-}ʔeh\mbox{-}έs}}; \wordng{I’w}{\mbox{-}áh\pl{as}} [1]; Mj~\third{t\mbox{-}’áh\plf{\mbox{-}(ʔa)h\mbox{-}έkiʔ}} (\citealt{JC14b}: 86, 92; \citealt{ND09}: 153; \citealt{AG83}: 123; \citealt{JC18}) {\sep} \wordng{PW}{*\mbox{-}t\mbox{-}’åχ\plf{*\mbox{-}t\mbox{-}’åh\mbox{-}és}} > \wordng{LB}{\mbox{-}t\mbox{-}’oχ\plf{\mbox{-}t\mbox{-}’oh\mbox{-}es}}; Vej~\third{t\mbox{-}’åh}; \wordng{’Wk}{\mbox{-}t\mbox{-}’åx\plf{\mbox{-}t\mbox{-}’åh\mbox{-}és}} (\citealt{VN14}: 191; \citealt{VU74}: 78; \citealt{KC16}: 7, 95)

\dicnote{The plural form attested in Iyo’awujwa’ is non-etymological.}%1

\gc{Likely related to \word{Proto-Guaicuruan}{*\mbox{-}ʔáko}{leather, skin} (\citealt{PVB13b}, \#650; cf. \citealt{PVB13a}: 309).}

\lit{\citealt{EN84}: 10, 19 (\intxt{*t’åhn}, 1~\intxt{*j\mbox{-}t’åhn}, 2~\intxt{*a\mbox{-}t’åhn}); \citealt{PVB02}: 143 (\intxt{*\mbox{-}ʔʌx}); \citealt{PVB13a}: 309 (\intxt{*\mbox{-}ʌh})}

\PMlemma{{\textit{*ˀ[n]ǻCtsiʔ} [1]\gloss{to feel disgust}}}

\wordng{Ni}{[n]åxtsi / \mbox{-}ʔåxtsi} \citep[211]{JS16} {\sep} \wordng{PCh}{*ˀ[n]ǻjtsiʔ} [2] > \wordng{Ijw}{ˀ[n]átʃiʔ \recind ˀ<n>átʃiʔ} [3]; \wordng{I’w}{\mbox{-}ájsij\mbox{-}e}; \wordng{Mj}{ˀ[n]ájʃi(j)ʔ} (\citealt{JC14a}; \citealt{ND09}: 162; \citealt{AG83}: 118; \citealt{JC18}) {\sep} \wordng{PW}{*ˀ<n>ǻxʷˈts<ej>\mbox{-}eh} > \wordng{Vejoz or Guisnay}{ˀnåhʷtsej\mbox{-}e}; \wordng{’Wk}{ˀnǻxʷˈtsej\mbox{-}eh} (\citealt{RL16}: 69; \citealt{KC16}: 49)

\dicnote{Nivaĉle points to \wordng{PM}{*xts} or \intxt{*χts}, Chorote to \intxt{*jts}, and Wichí to \intxt{*ɸts}.}%1

\dicnote{The cluster \wordng{PCh}{*ts} is reconstructed based on the Iyojwa’aja’ reflex with an affricate. Note that Chorote has no affricate /ts/, suggesting that we are dealing here with a cluster composed of /t/ and /s/.}%2

\dicnote{\citet[162]{ND09} mistranscribes this as \intxt{ˀ<n>átʃi}.}%3

\PMlemma{{\textit{*[t]’äk} [1]\gloss{to eat (intr.)}}}

\wordng{Mk}{[t]’ek} [1] (\citealt{AG99}: 142, 267) {\sep} \wordng{PW}{*[t]’eq} > \wordng{LB}{[t]’eq}; \wordng{Vej/’Wk}{[t]’ek} (\citealt{VN14}: 237, 239; \citealt{JB09}: 56; \citealt{VU74}: 78; \citealt{AFG067}: 213; \citealt{KC16}: 438)

\dicnote{The reconstruction \textit{*[t]’äˀk} is ruled out because the Maká reflex, as attested in the New Testament (e.g. Luke 18:12), shows a coda with no glottalization.}%1

\gc{\citet[305]{PVB13a} compares this verb to \wordng{Proto-Guaicuruan}{*\mbox{-}ekéʔe}, but the updated reconstruction \wordnl{*\mbox{-}kége}{to eat} (\citealt{PVB13b}, \#326) appears to be incompatible with the Mataguayan datum.}

\lit{\citealt{PVB13a}: 305 (\wordnl{*\mbox{-}ek}{to eat})}

\PMlemma{{\wordnl{*[t]’äskäj}{to laugh}}}

\word{Ni}{[t]’astʃaj / \mbox{-}ʔistʃaj}{to smile}, \wordnl{[t]’astʃaj=ʔin / \mbox{-}ʔistʃaj=ʔin}{to laugh} \citep[242, 317]{LC20} {\sep} \wordng{PCh}{*[t]’iskéjʔ} > \wordng{Ijw}{[t]’iskíʔ / \mbox{-}skíʔ}; \wordng{I’w}{\mbox{-}skíj=(ʔ)in}; \wordng{Mj}{[t]’iskíʔ / \mbox{-}skíjʔ}\gloss{to laugh, to smile (of a baby)}; \textit{[t]’iskí\mbox{-}hiˀneʔ}\gloss{to laugh} (\citealt{ND09}: 155; \citealt{AG83}: 161; \citealt{JC18}) {\sep} \wordng{PW}{*[t]’iskʲej} > \wordng{LB}{[t]’istʃej}; \wordng{Vej}{\mbox{-}stʃej\mbox{-}ɬi}, \wordng{’Wk}{[t]’iskʲejʔ} (\citealt{VN14}: 149; \citealt{VU74}: 72; \citealt{KC16}: 445)

\dicnote{This etymology has been first identified by \citet{LC-subm}.}%1

\lit{\citealt{LC-subm} (\intxt{*\mbox{-}iskey})}

\PMlemma{{\textit{*\mbox{-}ʔäsχaˀn}, \textit{*\mbox{-}ʔäsχán\mbox{-}its}\gloss{meat}}}

\wordng{Mk}{\mbox{-}ʔeseˀn} [1] (\intxt{\mbox{-}its}) (\citealt{AG99}: 158, 257) {\sep} \wordng{Ni}{\mbox{-}(ʔa)sxaˀn}, \textit{\mbox{-}(ʔa)sxan\mbox{-}is} (\citealt{JS16}: 234, 354) {\sep} \wordng{PCh}{*\mbox{-}ʔisáˀn}, \textit{*\mbox{-}ʔisán\mbox{-}is} > \wordng{Ijw}{\mbox{-}(ʔi)sʲéˀn}; \wordng{I’w}{\mbox{-}sʲén}; \wordng{Mj}{\mbox{-}(ʔi)ʃéˀn}, \textit{\mbox{-}ʔiʃén\mbox{-}is} (\citealt{ND09}: 155; \citealt{AG83}: 159; \citealt{JC18}) {\sep} \wordng{PW}{*\mbox{-}t\mbox{-}’isaˀn}, \textit{*\mbox{-}t\mbox{-}’isán\mbox{-}is} > \wordng{LB/Vej}{\mbox{-}t\mbox{-}’isan}; \wordng{’Wk}{\mbox{-}t\mbox{-}’isaˀn}, \textit{\mbox{-}t\mbox{-}’isán\mbox{-}is} (\citealt{VN14}: 291; \citealt{VU74}: 78; \citealt{KC16}: 97)

\dicnote{The preglottalized coda in the singular form in Maká is attested in the New Testament (e.g. Colossians 2:19; Mark 10:8).}%1

\lit{\citealt{EN84}: 28, 41 (\intxt{*tshan})}

\PMlemma{{\wordnl{*ʔéjaʔ\pla{l}}{mosquito}}}

\wordng{Mk}{ijeʔ\pl{l}}, (Towothli doculect) ‹eya› (\citealt{AG99}: 225; \citealt{RJH15}: 251) {\sep} \wordng{Ni}{jijaʔ} [1] \citep[385]{JS16} {\sep} \wordng{PCh}{*ʔéjaʔ\pla{l}} > \wordng{Ijw}{ʔέjeʔ\pl{waʔ}} [2]; \wordng{I’w}{ʔéjeʔ}; \wordng{Mj}{ʔέjeʔ\pl{l}} (\citealt{ND09}: 96; \citealt{AG83}: 125; \citealt{JC18})

\dicnote{The Nivaĉle reflex is entirely irregular: one would expect \intxt{*ʔeja}.}%1

\dicnote{The plural form attested in Iyojwa’aja’ is non-etymological.}%2

\PMlemma{{\textit{*ˀ[j]éjxåts\mbox{-}han}\gloss{to teach} [1]}}

\wordng{Mk}{[j]ixats<hen>} [2] (\citealt{AG99}: 219–220) {\sep} \wordng{Ni}{[j]ejxats\mbox{-}xan / \mbox{-}ʔejxats\mbox{-}xan} [3] \citep[123]{JS16} {\sep} \wordng{PCh}{*ˀ[j]éjåhås<an>} [4] > \wordng{Ijw}{ˀ[j]íjasaˀn / \mbox{-}ʔέjasaˀn} [5]; \wordng{I’w}{\mbox{-}éjesan} [5]; \wordng{Mj}{ˀ[j]íjeesän / \mbox{-}ʔέjeesän} (\citealt{ND09}: 166; \citealt{AG83}: 125; \citealt{JC18})

\dicnote{The PM verb is obviously derived from the etymon of \word{Ni}{\mbox{-}k\mbox{-}’eˀjxat}{news} (\citealt{JS16}: 123, 227).}%1

\dicnote{The expected reflex in Maká would be \intxt{*[j]ijxats<hen> / *\mbox{-}ʔijxats<hen>}.}%2

\dicnote{The expected reflex in Nivaĉle would be \intxt{*[j]ejxåts\mbox{-}xan / *\mbox{-}ʔejxåts\mbox{-}xan}. The irregular change \textit{*å > a} must have counterfed the palatalization of velars.}%3

\dicnote{In Chorote, \intxt{*å} was unexpectedly epenthesized between \intxt{*j} and \intxt{*h}.}%4

\dicnote{\wordng{PCh}{*åhå} was simplified to a single vowel in all dialects except Manjui (\wordng{Ijw}{a}, \wordng{I’w}{e}).}%5

\gc{Possibly related to \word{Proto-Guaicuruan}{*\mbox{-}iʔats’én}{to know, to understand} (\citealt{PVB13b}, \#306; cf. \citealt{PVB13a}: 305).}

\lit{\citealt{PVB13a}: 305 (\intxt{*\mbox{-}ejhats\mbox{-}han}\gloss{to know})}

\PMlemma{{\textit{*\mbox{-}ʔelå(ˀ)k \recind *\mbox{-}ʔelǻ(ˀ)k / *\mbox{-}ʔelkå\mbox{-} \recind *\mbox{-}ʔelkǻ\mbox{-}} [1]\gloss{pus}}}

\wordng{Mk}{\mbox{-}(i)lka\pl{l}} \citep[199]{AG99} {\sep} \wordng{Ni}{\mbox{-}(ʔe)kkå<ʔ>\pl{s}} \citep[355]{JS16} {\sep} \wordng{PCh}{*\mbox{-}ʔelǻk} > \wordng{Ijw}{\mbox{-}ʔilʲák / \mbox{-}lák\pl{is}} \citep[155]{ND09}

\dicnote{Maká and Nivaĉle would appear to have generalized the vocalic stem, and Chorote the consonantal one.}%1

\PMlemma{{\textit{*ʔéle(ʔ)}\gloss{parrot}}}

\wordng{Ni}{ʔek͡le\pl{s}} \citep[122]{JS16} {\sep} \wordng{PCh}{*ʔéleʔ\pla{waʔ}} > \wordng{Ijw}{ʔέleʔ}, \textit{ʔέl\mbox{-}iwaʔ}; \wordng{I’w}{ʔéleʔ}, \textit{ʔále\mbox{-}waʔ} [1]; \wordng{Mj}{ʔέleʔ\pl{waʔ}} (\citealt{ND09}: 96; \citealt{AG83}: 126; \citealt{JC18}) {\sep} \wordng{PW}{*ʔéle} > \wordng{LB}{ʔele}; \wordng{Vej}{ele}; \wordng{’Wk}{ʔéleʔ\pl{lis}} (\citealt{VN14}: 152; \citealt{VU74}: 56; \citealt{KC16}: 20)

\rej{\wordng{Maká}{eheʔ\pl{l}}\gloss{parrot} (\citealt{AG99}: 142; \citealt{PMA}: 5) cannot be related to \wordng{PM}{*ʔele} for phonological reasons.}

\empr{Compare \wordng{Proto-Qom}{*elé} (>~\wordng{Mocoví}{elé}, \wordng{Pilagá}{ele}, \wordng{Toba–Qom}{ele})\gloss{parrot}, which does not reconstruct to Proto-Guaicuruan and is thus a probable loan from a Mataguayan language, as well as \word{Lule}{ele}{parrot}, which is also obviously related \citep[300]{PVB13a}.}

\lit{\citealt{EN84}: 16, 35 (\intxt{*ɛlɛ}); \citealt{AnG15}: 253}

\PMlemma{{\textit{*\mbox{-}ʔeɬ \recind *\mbox{-}ʔéɬ}\gloss{other}}}

\wordng{Ni}{\mbox{-}ʔeɬ} \citep[490]{JS16} {\sep} \wordng{PW}{*\mbox{-}ʔeɬ \recind *\mbox{-}ʔéɬ} > \wordng{LB}{\mbox{-}ʔeɬ}; \wordng{Vej}{\mbox{-}eɬ}; \wordng{’Wk}{\mbox{-}ʔeɬ \recind \mbox{-}ʔéɬ} (\citealt{VN14}: 42; \citealt{VU74}: 56; \citealt{KC16}: 20)

\gc{\citet[314]{PVB13a} compares the Wichí form with \word{Kadiwéu}{eːlːe}{other}.}

\lit{\citealt{EN84}: 40 (\intxt{*ahl})}

\PMlemma{{\wordnl{*\mbox{-}ʔí\pla{l}}{liquid, juice}}}

\word{Mk}{\third{ɬ\mbox{-}’iʔ\pl{l}}}{juice} \citep[258]{AG99} {\sep} \wordng{Ni}{\mbox{-}ʔiʔ\pl{k}}\gloss{liquid, juice, broth, sap} (\citealt{JS16}: 139, 287) {\sep} \wordng{PCh}{*\mbox{-}ʔíʔ\pla{l}} > \wordng{Ijw}{\mbox{-}ʔéʔ\pl{ˀl}}; I’w~\third{t\mbox{-}’é}, \textit{t\mbox{-}é\mbox{-}j} [1]; Mj~\third{t\mbox{-}’éiʔ} (\citealt{ND09}: 155; \citealt{AG83}: 163; \citealt{JC18}) {\sep} \wordng{PW}{*\mbox{-}t\mbox{-}’í\pla{lʰ}} > \wordng{LB/Vej}{\mbox{-}t\mbox{-}’i}; \wordng{’Wk}{\mbox{-}t’íʔ\pl{ɬ}} (\citealt{VN14}: 197, 212; \citealt{VU74}: 107; \citealt{KC16}: 97)

\dicnote{The plain \intxt{t} in \cits{AG83} attestation of the Iyo’awujwa’ plural form must be a mistranscription.}%1

\gc{Possibly related to \word{Proto-Guaicuruan}{*\mbox{-}ʔegi}{juice} (\citealt{PVB13b}, \#669).}

\lit{\citealt{EN84}: 16, 48 (\intxt{*t’e \recind *t’ɛ})}

\PMlemma{{\wordnl{*ˀ[j]im}{to dry out, to be low (of water)}}}

\word{Mk}{[j]im}{to go low (of rivers)} \citep[186]{AG99} {\sep} \wordng{Ni}{[j]im} \citep[382]{JS16} {\sep} \wordng{PCh}{*ˀ[j]ím\mbox{-}\APPL} \textsc{/} \textsc{\mbox{-}}\textit{ʔím\mbox{-}\APPL} > \wordng{Ijw}{ˀ[j]ím\mbox{-}\APPL} \textsc{/} \textsc{\mbox{-}}\textit{ʔém\mbox{-}\APPL}; \wordng{Mj}{ˀ[j]ím\mbox{-}\APPL} \textsc{/} \textsc{\mbox{-}}\textit{ʔéim\mbox{-}\APPL} (\citealt{ND09}: 165, 166; \citealt{JC18}) {\sep} \wordng{PW}{*ˀ[j]im} > \wordng{Vej}{[j]im}; \wordng{’Wk}{ˀ[j]im̥} (\citealt{VU74}: 84; \citealt{KC16}: 125)

\gc{\citet[308]{PVB13a} notes the similarity with \word{Proto-Qom}{*ʔim}{to be dry}.}

\lit{\citealt{PVB13a}: 308 (\intxt{*\mbox{-}(j)im})}

\PMlemma{{\intxt{*ʔis\pla{íts}} [1]\gloss{good}}}

\wordng{Ni}{ʔis\plf{\mbox{-}ʔis\mbox{-}is}} \citep[140]{JS16} {\sep} \wordng{PCh}{*ʔís} > \wordng{Ijw}{ʔés\plf{ʔixʃ\mbox{-}ís}}; \wordng{I’w}{ʔés}; \wordng{Mj}{ʔéis\plf{ʔas\mbox{-}éis}} (\citealt{JC14b}: 84; \citealt{ND09}: 112, 161; \citealt{AG83}: 127; \citealt{JC18}) {\sep} \wordng{PW}{*ʔis\pla{ís}} > \wordng{LB}{ʔis}; \wordng{Vej}{is}; \wordng{’Wk}{ʔis\pl{ís}} (\citealt{VN14}: 312; \citealt{VU74}: 60; \citealt{MG-MELO15}: 34; \citealt{KC16}: 34)

\dicnote{In absence of a known cognate in Maká, one could wonder whether this stem could be reconstructed as \intxt{*ʔits}, with a regular \intxt{*ts} > \intxt{s} in coda. This seems unlikely, given that the daughter languages maintain the fricative \intxt{s} even before vowel-initial suffixes, as in the Lower Bermejeño inchoative derivate \wordnl{ʔis\mbox{-}eχ}{to become good} \citep[262]{VN14}. This contrasts with the behavior of the roots which reflect \textit{bona fide} \wordng{PM}{*ts}\mbox{-}final roots: compare \word{LB}{qates\plf{qatets\mbox{-}eɬ}}{star} \citep[112]{VN14}.}%1

\PMlemma{{\wordnl{*ʔítå(ˀ)χ\plf{*ʔítå\mbox{-}ts}}{fire}}}

\wordng{Ni}{ʔitåx\plf{ʔitå\mbox{-}s} / \mbox{-}β\mbox{-}itåx\plf{\mbox{-}β\mbox{-}itå\mbox{-}s}} (\citealt{JS16}: 141, 362) {\sep} \wordng{PCh}{*ʔítåh\plf{*ʔítå\mbox{-}s}} > \wordng{I’w}{ʔéjtʲeʔ \recind ʔéjtiʔ\pl{s}} [1]; \wordng{Mj}{ʔéit(ʲ)e\pl{s}} (\citealt{AG83}: 126, 199; \citealt{JC18}) {\sep} \wordng{PW}{*ʔítåχ\plf{*ʔítå\mbox{-}s}} > \wordng{LB}{ʔitoχ}; \word{Vej}{itåh\plf{itå\mbox{-}s}}{fire, match}; \wordng{’Wk}{ʔítåx\plf{ʔítå\mbox{-}s}} (\citealt{VN14}: 295; \citealt{VU74}: 61; \citealt{MG-MELO15}: 48; \citealt{AFG067}: 213; \citealt{KC16}: 38)

\dicnote{\cits{AG83} attestation of a word-final glottal stop in the Iyo’awujwa’ reflex must be a mistranscription.}%1

\lit{\citealt{EN84}: 16, 19 (\intxt{ithǻ}); \citealt{PVB02}: 144 (\intxt{*itʌχ})}

\PMlemma{{\wordnl{*ˀ[n]ixowáj / *\mbox{-}ʔixowáj}{to be afraid}}}

\wordng{Mk}{[n]ixiwej / \mbox{-}ʔixiwej} [1] \citep[221]{AG99} {\sep} \wordng{Ni}{[n(i)]xoβaj / \mbox{-}ʔixoβaj} \citep[259]{LC20} {\sep} \wordng{PW}{*<n>owáj} [2] > \wordng{LB}{nuwaj}; \wordng{’Wk}{nowájʔ} (\citealt{VN14}: 149; \citealt{KC16}: 278)

\dicnote{Maká shows an irregular change \intxt{*o} > \intxt{i}.}%1

\dicnote{We assume an irregular loss of the initial syllable in Wichí. It is also possible that \intxt{*[n]owáj} was the original Proto-Mataguayan root, with Maká and Nivaĉle showing an extra prefix.}%2

\PMlemma{{\wordnl{*\mbox{-}ʔo(ʔ)\plf{*\mbox{-}ʔó\mbox{-}l}}{grave}}}

Ni~\third{t\mbox{-}’oʔ} (\citealt{LC20}: 39) {\sep} \wordng{PCh}{*\mbox{-}ʔóʔ\pla{l}} > \wordng{Ijw}{\mbox{-}ʔɔ́ʔ\pl{ˀl}} \citep[156]{ND09} {\sep} \wordng{PW}{*\mbox{-}t\mbox{-}’o(ʔ)} > \wordng{LB}{\mbox{-}t\mbox{-}’u(ʔ)}; \wordng{’Wk}{\mbox{-}t\mbox{-}’oʔ\plf{\mbox{-}t\mbox{-}’o\mbox{-}lis}} (\citealt{JB09}: 60; \citealt{KC16}: 98)

\PMlemma{{\wordnl{*ʔóɸoʔ\pla{ts}}{picazuro pigeon\species{Patagioenas picazuro}}}}

\wordng{Mk}{ofoʔ\pl{l}} [1] \citep[281]{AG99} {\sep} \wordng{Ni}{ʔoɸo\pl{s}} \citep[206]{JS16} {\sep} \wordng{PCh}{*ʔóhwoʔ\pla{s}} > \wordng{Ijw}{ʔɔ́hwoʔ}; \wordng{I’w}{ófʷoʔ\pl{s}} [2]; \wordng{Mj}{ʔɔ́hwoʔ\pl{s}} (\citealt{JC14b}: 142; \citealt{ND09}: 142; \citealt{AG83}: 152; \citealt{JC18})

\dicnote{The Maká plural form with \textit{\mbox{-}l} does not match the Nivaĉle and Chorote data.}%1

\dicnote{\citet[213]{AG83} documents also the phonetic variant \intxt{óxuʔ}.}%2

\PMlemma{{\wordnl{*ˀ[j]om}{to be extinguished}, \textsc{caus}~\wordnl{*ˀ[j]om\mbox{-}hat}{to extinguish}}}

\wordng{Mk}{[j]om}, \intxt{[j]om\mbox{-}het} \citep[282]{AG99} {\sep} \wordng{PCh}{*ˀ[j]óm\mbox{-}\APPL}, \intxt{*ˀ[j]óhm\mbox{-}at\mbox{-}\APPL} > \wordng{Ijw}{ˀ[j]óˀm\mbox{-}e}, \intxt{ˀ[j]óhm\mbox{-}at\mbox{-}\APPL}; I’w~—, \intxt{\mbox{-}ohm\mbox{-}at\mbox{-}eʔ \recind \mbox{-}owm\mbox{-}at\mbox{-}eʔ}; Mj~—, \intxt{ˀ[j]óhm\mbox{-}at\mbox{-}\APPL} (\citealt{JC14b}: 78; \citealt{ND09}: 166; \citealt{AG83}: 153, 183; \citealt{JC18}) {\sep} \wordng{PW}{*ˀ[j]om}, \intxt{*ˀ[j]om\mbox{-}ét} [1] > LB~—, \intxt{ˀ[j]um\mbox{-}et}; \wordng{Vej}{[j]om} [2], —; \wordng{’Wk}{ˀ[j]om̥}, \intxt{ˀ[j]om\mbox{-}ét} (\citealt{VN14}: 295; \citealt{VU74}: 84; \citealt{KC16}: 128)

\dicnote{The Wichí causative \intxt{*ˀ[j]om\mbox{-}ét} is not a reflex of \wordng{PM}{*ˀ[j]om\mbox{-}hat}, but rather an independent formation.}%1

\dicnote{The absence of a glottal stop or glottalization in the root-initial position in \cits{VU74} attestation of the Vejoz reflex could result from mistranscription.}%2

\gc{\citet[307]{PVB13a} compares this to \word{Proto-Guaicuruan}{*\mbox{-}ʔem}{to be extinguished} (\citealt{PVB13b}, \#672).}

\lit{\citealt{RJH15}: 239; \citealt{PVB13a}: 307 (\intxt{*\mbox{-}om}, \textsc{caus}~\intxt{*\mbox{-}om\mbox{-}hate})}

\PMlemma{{\wordnl{*ˀ[n]om}{to wake up} [1]}}

\wordng{Mk}{[n]om\mbox{-}phaˀm} [1] (\citealt{AG99}: 222, 282; \citealt{CM15}: 138) {\sep} \wordng{PW}{*ˀ<n>om} > \wordng{LB}{ˀnum}; \wordng{’Wk}{ˀnom̥} (\citealt{diwica}; \citealt{KC16}: 76)

\dicnote{Morphologically, this verbs looks like a middle voice derivation from the verb \wordnl{*ˀ[j]om}{to be extinguished}.}%1

\dicnote{The absence of an underlying glottal stop in Maká, as seen in inflected forms such as \intxt{ts\mbox{-}om\mbox{-}phaˀm} (as opposed to the expected form \intxt{*ts\mbox{-}’om\mbox{-}phaˀm}), must have come about through analogy with the third-person form \intxt{[n]om\mbox{-}phaˀm}, where glottalization is regularly lost in the word-initial position.}%2

\PMlemma{\wordnl{*ʔóna(ˀ)χ}{my brother}}

\word{Ni}{ʔonax}{my younger brother} \citep[207]{JS16} {\sep} \wordng{PCh}{*ʔónah} > \word{Mj}{ʔɔ́na\pl{wat}}{my elder brother} \citep{JC18}

\rej{\citet[20]{EN84} considers the Nivaĉle term related to \word{Ni}{\mbox{-}sunxa}{younger sister} and reflexes of \word{PW}{*\mbox{-}púhxʷa}{brother}, which are all derived from \word{PM}{*p’unhwa}{sibling} in her reconstruction. This is obviously a spurious comparison.}

\PMlemma{{\wordnl{*ˀ[j]óp’ale(ʔ)}{to hiccup}}}

\wordng{Ni}{[j]op’ak͡le / \mbox{-}ʔóp’ak͡le}\gloss{to choke} \citep[212]{JS16} {\sep} \wordng{PCh}{*[j]óp’ale\mbox{-}ˀn} > \wordng{Ijw}{[j]óp’aleʔ} [1]; \wordng{I’w}{\mbox{-}óppali\mbox{-}en} [2]; \wordng{Mj}{[j]óp’ele\mbox{-}ʔɪn / \mbox{-}ɔ́p’ele\mbox{-}ʔɪn} [3] (\citealt{ND09}: 161; \citealt{AG83}: 153; \citealt{JC18}) {\sep} \wordng{PW}{*[j]ópˀle} [1] > \wordng{LB}{\mbox{-}juˀle}; \wordng{Vej}{[j]ople}; \wordng{’Wk}{ˀ[j]ople<j>ʔ} [4] (\citealt{VN14}: 53; \citealt{RJH13a}: 67, 113, 177; \citealt{KC16}: 128)

\dicnote{\citet[161]{ND09} transcribes this as \intxt{[j]óp’ali\mbox{-}ˀn}, which does not match our field data.}%1

\dicnote{The geminate \intxt{pp} in the Iyo’awujwa’ reflex is probably a mistranscription of \intxt{p’}.}%2

\dicnote{In Manjui, unstressed \wordng{PCh}{*a} irregularly yielded \textit{e}.}%3

\dicnote{The ’Weenhayek reflex is likely ill-transcribed, as \citet[218]{KC16} marks the respective entry as an “early note” (apparently meaning that the form was documented when his knowledge of the language was suboptimal). The expected form would be \intxt{*[j]opˀleʔ}.}%4

\gc{\citet[306]{PVB13a} compares this to \word{Proto-Guaicuruan}{*\mbox{-}t’ap’ela}{to choke} (\citealt{PVB13b}, \#550).}

\lit{\citealt{PVB13a}: 306 (\intxt{*\mbox{-}op’ale})}

\PMlemma{{\wordnl{*\mbox{-}ʔoˀt \recind *\mbox{-}ʔóˀt}{chest}}}

\wordng{Ni}{\mbox{-}ʔoˀt}, \textit{\mbox{-}ʔot\mbox{-}is} \citep[355]{JS16} {\sep} \wordng{PCh}{*\mbox{-}ʔót} > \wordng{Ijw}{\mbox{-}ʔɔ́t}; \wordng{I’w}{\mbox{-}ót\pl{es}} [1]; \wordng{Mj}{\mbox{-}ʔɔ́t} (\citealt{JC14b}: 77, 85; \citealt{ND09}: 156; \citealt{AG83}: 153; \citealt{JC18})

\dicnote{The absence of a \textit{ʔ} in \citet{AG83} must be a mistranscription.}%1

\rej{\citet[38, 42]{EN84} compares the Chorote reflex to \word{Ni}{\mbox{-}ɬiˀβte}{heart} and reflexes of \word{PW}{*\mbox{-}t’ókʷe}{chest}, but this is absolutely impossible for phonological reasons.}

\PMlemma{{\wordnl{*ˀ[j]uj}{to enter, to sink, to set (of sun)}}}

\word{Mk}{[j]uj / \mbox{-}ʔwi}{to enter, to sink} \citep[374]{AG99} {\sep} \wordng{Ni}{[j]uj / \mbox{-}ʔuj} \citep[390]{JS16} {\sep} \wordng{PCh}{*ˀ[j]újʔ}\gloss{to enter} > \wordng{Ijw}{ˀ[j]úʔ / \mbox{-}ʔóʔ} [1]; \wordng{I’w}{\mbox{-}oj\mbox{-}i} [2]; \wordng{Mj}{ˀ[j]újʔ / \mbox{-}ʔʊ́jʔ} (\citealt{JC14a}; \citealt{JC14b}: 77, fn. 4; \citealt{ND09}: 166; \citealt{AG83}: 152; \citealt{JC18}) {\sep} \wordng{PW}{*ˀ[j]uj}\gloss{to sink, to set (of sun)} > \wordng{Vej}{ˀ[j]uj} [3]; \wordng{’Wk}{ˀ[j]ujʔ}\gloss{to set (of sun)}; \textit{ˀ[j]új\mbox{-}\APPL}\gloss{to enter}; \textit{*ˀ[j]ú\mbox{-}kʲe}\gloss{to enter, to wear}, \textit{*ˀ<j>ú<kʲe>\pla{lis}}\gloss{shirt} > \wordng{LB}{ˀ[j]e\mbox{-}tʃe}; \wordng{Vej}{ˀ[j]u\mbox{-}tʃe} [3]; \textit{ˀjutʃe\pl{lis}}; \wordng{’Wk}{ˀ[j]ú\mbox{-}kʲeʔ}; \textit{ˀjúkʲeʔ\pl{lis}} (\citealt{VN14}: 152; \citealt{VU74}: 84; \citealt{MG-MELO15}: 51, 66; \citealt{KC16}: 129–131)

\dicnote{\citet{ND09} mistranscribes this as \textit{ˀ[j]ú}.}%1

\dicnote{The absence of a \textit{ʔ} in \citet{AG83} must be a mistranscription.}%2

\dicnote{\citet[84]{VU74} mistranscribes \textit{ˀ[j]\mbox{-}} as \textit{[j]\mbox{-}}.}%3

\PMlemma{{\wordnl{*ʔúlʔåh\plf{*ʔúlʔå\mbox{-}ts}}{dove\species{Columbina sp.}}}}

\word{Ni}{ʔuk͡lʔå\pl{s}}{Picui dove} \citep[306]{JS16} {\sep} \wordng{PCh}{*ʔúlʔåh\plf{*ʔúlʔå\mbox{-}s}} > \wordng{I’w}{ólaha\pl{s}}; \word{Mj}{ʔúl(a)ʔa\pl{s}}{scaled dove} (\citealt{AG83}: 152; \citealt{JC18})

\PMlemma{{\wordnl{*\mbox{-}ʔuka}{to swell}}}

\word{Ni}{[t]’uka<ˀn>}{to swell}, \wordnl{\mbox{-}ʔuka<ˀx>\plf{\mbox{-}ʔuka<x>\mbox{-}is}}{swelling} [1] \citep[247]{LC20} {\sep} \word{PCh}{*[t]’ᵊká<ˀn>}{to swell} [1 2] > \wordng{Ijw}{[t]’ikʲéˀn} (\citealt{ND09}: 155) {\sep} \word{PW}{*<t>’ukʷa}{to swell} [3] > \wordng{LB}{t’ikʷa} [2]; \wordng{’Wk}{t’ukaʔ} (\citealt{diwica}; \citealt{KC16}: 449)

\dicnote{Nivaĉle and Chorote have fossilized a verbalizing suffix; in addition, Nivaĉle reflects a nominalization of the erstwhile verb.}%1

\dicnote{Chorote and Lower Bermejeño Wichí show unusual reflexes of the root-initial vowel; one would expect to find \intxt{u} in Iyojwa’aja’ and \intxt{e} in Lower Bermejeño Wichí.}%2

\dicnote{Wichí, or at least ’Weenhayek, has fossilized the erstwhile third-person prefix as a part of the root \citep[99]{KC16}.}%2

\PMlemma{{\wordnl{*\mbox{-}ʔúɬ}{to urinate}}}

\wordng{Mk}{uɬ / \mbox{-}ʔuɬ} \citep[354]{AG99} {\sep} \wordng{Ni}{[j]uɬ / \mbox{-}ʔuɬ} \citep[306]{JS16} {\sep} \wordng{PCh}{*[t]’úɬ} > \wordng{Ijw}{[t]’óɬ}; \wordng{I’w}{\mbox{-}ól} [1]; \wordng{Mj}{[t]’úɬ} (\citealt{ND09}: 155; \citealt{AG83}: 152; \citealt{JC18}) {\sep} \wordng{PW}{*[t]’úɬ} > \wordng{LB}{[t]’eɬ}; \wordng{Vej}{[t]uɬ} [2]; \wordng{’Wk}{[t]’úɬ} (\citealt{VN14}: 238; \citealt{JB09}: 59; \citealt{VU74}: 77; \citealt{KC16}: 449)

\dicnote{The absence of an initial glottal stop in \cits{AG83} attestation of the word could result from mistranscription.}%1

\dicnote{The plain stop \intxt{t} in \cits{VU74} attestation of the Vejoz reflex must be a mistranscription.}%2

\lit{\citealt{AnG15}: 254–255}

\PMlemma{{\wordnl{*\mbox{-}ʔúɬu(ʔ)}{urine}}}

\wordng{Ni}{\mbox{-}ʔuɬu} \citep[307]{JS16} {\sep} \wordng{PCh}{*\mbox{-}ʔúhluʔ} > \wordng{Ijw}{\mbox{-}ʔéhlʲuʔ} [1]; \wordng{I’w}{\mbox{-}óhluʔ\pl{s}} [2]; Mj~‹tsojliu› \recind ‹sojliu› (\citealt{ND09}: 155; \citealt{AG83}: 153; \citealt{RLN10}: 118) {\sep} \wordng{PW}{*\mbox{-}t\mbox{-}’úɬu} > \wordng{Vej}{\mbox{-}t\mbox{-}uɬu} [3]; \wordng{’Wk}{\mbox{-}t\mbox{-}’úɬuʔ} (\citealt{VU74}: 77; \citealt{KC16}: 99)

\dicnote{\sound{Iyojwa’aja’}{e} (underlying /i/) is not a regular reflex of \wordng{PCh}{*u}.}%1

\dicnote{The absence of an initial glottal stop in \cits{AG83} attestation of the word could result from mistranscription.}%2

\dicnote{The plain stop \intxt{t} in \cits{VU74} attestation of the Vejoz reflex must be a mistranscription.}%3

\lit{\citealt{EN84}: 21 (\intxt{t’uhlu})}

\PMlemma{{\intxt{*ʔuwáɬe(ˀ)χ \recvar *C’uwáɬe(ˀ)χ} [1]\gloss{puma}}}

\wordng{Ni}{<xum>p’uβaɬex\plf{<xum>p’uβaɬxe\mbox{-}s}} \citep[158]{JS16} {\sep} \wordng{PCh}{*k’uwáhlah}, \textit{*k’uwáhla\mbox{-}s} > \wordng{Ijw}{k’iwáhla}; \wordng{I’w}{iwáhla\pl{s}}; \wordng{Mj}{ʔiwáhla\pl{s}} (\citealt{JC14b}: 99; \citealt{ND09}: 138; \citealt{AG83}: 132; \citealt{JC18}) {\sep} \wordng{PW}{*ʔowáɬaχ \recind *C’owáɬaχ\plf{*ʔowáɬa\mbox{-}s \recvar *C’owáɬa\mbox{-}s}} [1 2] > \wordng{LB}{p’uwaɬaχ}; \wordng{Southeastern (Pozo Yacaré)}{puwaɬoχ}; \wordng{Guisnay (Alto de la Sierra)}{powaɬah}; \wordng{Vej}{owaɬah}; \wordng{’Wk}{t’owáɬax}, \textit{t’owáɬa\mbox{-}s} (\citealt{JB09}: 55; \citealt{RL16}: 71; \citealt{VU74}: 69; \citealt{MG-MELO15}: 22; \citealt{KC16}: 448)

\dicnote{Nivaĉle and Lower Bermejeño point to \wordng{PM}{*p’uwáɬeχ} > \wordng{PW}{*p’owáɬaχ}; ’Weenhayek to \wordng{PM}{*t’owáɬeχ} > \wordng{PW}{*t’owáɬaχ}, Vejoz to \wordng{PM}{*ʔowáɬeχ} > \wordng{PW}{*ʔowáɬaχ}, and Chorote to \wordng{PM}{*k’uwáɬeχ}.}%1

\dicnote{The lowering of \wordng{PM}{*u} to \sound{PW}{*o} is irregular.}%2

\lit{\citealt{EN84}: 20 (\intxt{*t’ɔahla}); \citealt{LC-VG-07}: 19}

\PMlemma{{\intxt{*ʔVláʔah\plf{*ʔVláʔa\mbox{-}ts}} [1]\gloss{lesser grison}}}

\wordng{Mk}{ile\pl{j}} \citep[198]{AG99} {\sep} \wordng{Ni}{ʔak͡laʔa\pl{s}} \citep[38]{JS16} {\sep} \wordng{PCh}{*ʔeláʔah \recvar *ʔaláʔah\plf{*ʔaláʔa\mbox{-}s}} > \wordng{Ijw}{ʔeláʔa\plf{ʔeláh\mbox{-}as}}; \wordng{I’w}{aláah\pl{as}}; \wordng{Mj}{ʔaláʔa\pl{s}} (\citealt{ND09}: 96; \citealt{AG83}: 119; \citealt{JC18}) {\sep} \wordng{PW}{*ʔiláʔah} > \wordng{Vej}{ilaʔa\mbox{-}tah}; \word{’Wk}{ʔiláʔah}{southern river otter} (\citealt{VU74}: 60; \citealt{KC16}: 29)

\dicnote{Maká points to \wordng{PM}{*ʔeláʔah\plf{*ʔeláʔa\mbox{-}ts}} or \intxt{*ʔiláʔah\plf{*ʔiláʔa\mbox{-}ts}}, Iyojwa’aja’ to \wordng{PM}{*ʔeláʔah\plf{*ʔeláʔa\mbox{-}ts}}, Wichí to \intxt{*ʔiláʔah\plf{*ʔiláʔa\mbox{-}ts}}, whereas Nivaĉle, Iyo’awujwa’, and Manjui point to \wordng{PM}{*ʔaláʔah\plf{*ʔaláʔa\mbox{-}ts}}.}%1

\lit{\citealt{EN84}: 36 (\intxt{*elaatha}\gloss{neotropical otter})}
\end{adjustwidth}
\section{Derivational affixes (nouns)} \label{derafn}
\begin{adjustwidth}{6mm}{0pt}

\PMlemma{{\wordnl{*\mbox{-}äk\plf{*\mbox{-}h\mbox{-}ajʰ}}{participle, resultative nominalization}}}

\wordng{Mk}{wit\mbox{-}…\mbox{-}ek} \citep[225]{AG94} {\sep} \wordng{Ni}{\mbox{-}atʃ} [1] \citep[37]{JS16} {\sep} \wordng{PCh}{*\mbox{-}ek\plf{*\mbox{-}h\mbox{-}ajʰ}} > \wordng{Ijw}{\mbox{-}ik\plf{\mbox{-}h\mbox{-}aʔ}} [1]; \wordng{Mj}{\mbox{-}ek\plf{\mbox{-}h\mbox{-}aj}} (\citealt{JC14a,JC14b,JC18}) {\sep} \wordng{PW}{*\mbox{-}eq\plf{*\mbox{-}h\mbox{-}ajʰ}} > \wordng{LB}{\mbox{-}eq\plf{\mbox{-}h\mbox{-}aç}}; \wordng{’Wk}{\mbox{-}ek\plf{\mbox{-}h\mbox{-}aç}} (\citealt{VN14}: 150, 192; \citealt{JAA-KC-14}: 444)

\dicnote{\sound{Iyojwa’aja’}{\mbox{-}ʔ} in the plural form is not the regular reflex of \wordng{PCh}{*\mbox{-}jʰ}.}%1

\gc{Obviously related to \word{Proto-Guaicuruan}{*\mbox{-}ek}{result or action nominalizer} (\citealt{PVB13b}, \#719; cf. \citealt{PVB13a}: 317).}

\lit{\citealt{PVB13a}: 317 (\intxt{*\mbox{-}ek \recind *\mbox{-}ik})}

\PMlemma{{\wordnl{*\mbox{-}aχ}{nominalizer (abstract nouns)} [1]}}

\wordng{Mk}{\mbox{-}aχ\pl{its}} (\citealt{AG94}: 219; \citealt{AG99}: 194, 221, 368) {\sep} \wordng{Ni}{\mbox{-}ax} (\citealt{LC20}: 108)

\dicnote{\citet[317]{PVB13a} reconstructs this nominalizer as \intxt{*\mbox{-}tsah \recind *\mbox{-}ah}, as if these were two allomorphs of the same suffix. In our reconstruction, these two morphemes have different vowels (\intxt{*\intxt{-}aχ} vs. \intxt{*\mbox{-}tseχ}) and are hardly related to each other. }%1

\lit{\citealt{PVB13a}: 317 (\wordnl{*\mbox{-}ah}{nominalizer})}

\PMlemma{{\wordnl{*\mbox{-}eʔ}{feminine} (not productive)}}

\wordng{Mk}{\mbox{-}iʔ} \citep[152]{AG94} {\sep} \wordng{Ni}{\mbox{-}eʔ} (\citealt{LC20}: 107) {\sep} \wordng{PCh}{*\mbox{-}eʔ} > \wordng{Ijw/I’w/Mj}{\mbox{-}eʔ} (\citealt{JC14a,JC14b,JC18}) {\sep} \wordng{PW}{*\mbox{-}e} > \wordng{LB/Vej}{\mbox{-}e}; \wordng{’Wk}{\mbox{-}eʔ} (see \hyperref[dic-aose]{\word{PM}{*\mbox{-}ǻseʔ}{daughter}})

\gc{Possibly related to \word{Proto-Guaicuruan}{*\mbox{-}ʔé}{feminine} (\citealt{PVB13b}, \#741; cf. \citealt{PVB13a}: 317).}

\lit{\citealt{PVB13a}: 317 (\intxt{*\mbox{-}e})}

\PMlemma{{\wordnl{*\mbox{-}ɸah\plf{*\mbox{-}ɸa\mbox{-}ts}}{companion}}}

\wordng{Mk}{\mbox{-}fe} [1] (\intxt{\mbox{-}ts}) (\citealt{AG99}: 142, 162, 210, 230, 286, 302–303, 386, 393) {\sep} \wordng{Ni}{\mbox{-}ɸa\pl{s}} (\citealt{JS16}: 127; \citealt{AF16}: 105) {\sep} \wordng{PCh}{*\mbox{-}hwah}, \textit{*\mbox{-}hwa\mbox{-}s} > \wordng{Ijw}{\mbox{-}hwa\pl{s}}; \wordng{I’w}{\mbox{-}fʷa\pl{j}} [1]; \wordng{Mj}{\mbox{-}hwa}, \textit{\mbox{-}hwaa\mbox{-}j} [1] (\citealt{ND09}: 132; \citealt{AG83}; \citealt{JC18}) {\sep} \wordng{PW}{*\mbox{-}xʷah}, \textit{*\mbox{-}xʷa\mbox{-}s} > \wordng{LB}{\mbox{-}fʷa\pl{j}} in \wordnl{\mbox{-}tʃ’e<fʷa>\pl{j}}{spouse} [2]; \wordng{’Wk}{\mbox{-}xʷah}, \textit{\mbox{-}xʷa\mbox{-}s} (\citealt{VN14}: 163; \citealt{KC16}: 162)

\dicnote{\citet{AG99} documents two variants of this suffix, \textit{\mbox{-}fe} (in \wordnl{\mbox{-}xefe}{compatriot, fellow Indigenous person}, \wordnl{\mbox{-}kife}{neighbor}) and \textit{\mbox{-}feʔ} (\wordnl{\mbox{-}eku\mbox{-}feʔ}{eating companion}, \wordnl{\mbox{-}tseti\mbox{-}feʔ}{compatriot}, \wordnl{\mbox{-}ʔexujhi\mbox{-}feʔ}{enemy}). In the New Testament, this suffix is always attested as \textit{\mbox{-}fe}: \wordnl{j\mbox{-}eku\mbox{-}fe}{the one who eats with me} (Mark 14:18), \wordnl{ji\mbox{-}tseti\mbox{-}fe}{my compatriot} (Romans 16:11), \wordnl{ɬ\mbox{-}’exujhi\mbox{-}fe}{his enemy} (1 Corinthians 15:26).}%1

\dicnote{The plural form in Lower Bermejeño Wichí is non-etymological.}%2

\gc{Possibly related to \word{Proto-Guaicuruan}{*\mbox{-}awa \recind *\mbox{-}aqawa}{companion} (\citealt{PVB13b}, \#711).}

\lit{\citealt{EN84}: 15 (\wordnl{*cɛ(h)l\mbox{-}hwa}{spouse})}

\PMlemma{{\textit{*\mbox{-}(ha\mbox{-})jaˀx} [1]\gloss{nominalizer (abstract nouns)}}}

\wordng{Mk}{\mbox{-}(he\mbox{-})jeˀx / \mbox{-}eˀx} [2] \textit{/ \mbox{-}he\mbox{-}ji(ˀ)x} [3] \citep[220]{AG94} {\sep} \wordng{Ni}{\mbox{-}(xa\mbox{-})jaʃ / \mbox{-}aʃ} [4] (\citealt{LC20}: 136–137) {\sep} \wordng{PCh}{*\mbox{-}(ha\mbox{-})jah} > \wordng{Ijw/Mj}{\mbox{-}(ha\mbox{-})je} (\citealt{JC14a,JC18}) {\sep} \wordng{PW}{*\mbox{-}(ha\mbox{-})jaχ} > \wordng{LB}{\mbox{-}(ha\mbox{-})jaχ\pl{aj}}; \wordng{’Wk}{\mbox{-}(ha\mbox{-})jax}, \textit{\mbox{-}(ha\mbox{-})jah\mbox{-}aj} (\citealt{VN14}: 161, 204–205, 421–422; \citealt{JAA-KC-14}: 442)

\dicnote{The element \intxt{*\mbox{-}ha\mbox{-}} occurs in some nominalizations but not in others. At least in Chorote, it is possible that the allomorph \intxt{*\mbox{-}jah} is phonologically conditioned, occurring after stems that end in low vowels. This allomorphy pattern awaits further study.}%1

\dicnote{The allomorph \intxt{\mbox{-}eˀx} in Maká is found after \intxt{j}.}%2

\dicnote{The preglottalized coda in Maká is attested in the New Testament: \wordnl{wit\mbox{-}’ijin\mbox{-}hejeˀx}{demand} (1 Timothy 4:5), \wordnl{wit\mbox{-}’ik\mbox{-}hejiˀx}{path} (Romans 3:17). The latter noun is also attested as \intxt{\mbox{-}ʔik\mbox{-}hejix}, though (Luke 13:33; cf. also \citealt{unuuneiki}: 17).}%3

\dicnote{The allomorph \intxt{\mbox{-}aʃ} in Nivaĉle occurs after consonants.}%4

\PMlemma{{\wordnl{*\mbox{-}haˀt\plf{*\mbox{-}hat\mbox{-}ets \recind *\mbox{-}hat\mbox{-}its}}{instrument nominalizer}}}

\wordng{Mk}{\mbox{-}heˀt} [1], \textit{\mbox{-}het\mbox{-}its} (\citealt{AG99}: 362, 363, …) {\sep} \wordng{Ni}{\mbox{-}xat\pl{es \recind \mbox{-}is}} (\citealt{AF16}: 100–101; \citealt{LC20}: 118) {\sep} \wordng{PCh}{*\mbox{-}hat\pla{is}} > \wordng{Ijw}{\mbox{-}hat\pl{is}}; \wordng{I’w}{\mbox{-}hat\pl{es}}; \wordng{Mj}{\mbox{-}hat\pl{es \recind \mbox{-}is}} (\citealt{JC14a}; \citealt{AG83}: 135, 147; \citealt{JC18})

\dicnote{The preglottalized coda in the Maká singular form is attested in the New Testament in derivatives such as \textit{wit\mbox{-}eqhun\mbox{-}heˀt}\gloss{medicine} (Revelations 3:18).}%1

\gc{Obviously related to \word{Proto-Guaicuruan}{*\mbox{-}aqate}{instrument nominalizer} (\citealt{PVB13b}, \#714; cf. \citealt{PVB13a}: 317).}

\lit{\citealt{PVB13a}: 317 (\intxt{*\mbox{-}hate})}

\PMlemma{{\wordnl{*\mbox{-}kat}{collective of plants}}}

\wordng{Mk}{\mbox{-}ket}, \textit{\mbox{-}et} (after \textit{k}) (\citealt{AG94}: 151–152) {\sep} \wordng{Ni}{\mbox{-}tʃat / \mbox{-}kat} (after \textit{V\textsubscript{[+back]}}\textit{(C\textsubscript{[+grave]}})) \citep[77]{AF16} {\sep} \wordng{PCh}{*\mbox{-}kat} > \wordng{Ijw}{\mbox{-}kʲet}; \wordng{I’w}{\mbox{-}ket \recind \mbox{-}kʲet}; \wordng{Mj}{\mbox{-}kʲet} (\citealt{JC14a}; \citealt{AG83}: 119–120, 145, 151, 158, 173; \citealt{JC18}) {\sep} \wordng{PW}{*\mbox{-}kʲat}, \textit{*\mbox{-}at} (after \textit{*kʷ}, \textit{*q}) > \wordng{LB}{\mbox{-}tʃat}, \textit{\mbox{-}at} (after \textit{kʷ}, \textit{q}); \wordng{’Wk}{\mbox{-}kʲat}, \textit{\mbox{-}at} (after \textit{k}) (\citealt{VN14}: 193; \citealt{KC16}: 19, 139, 152, 186, 225, 326, 466)

\gc{Possibly related to \word{Proto-Guaicuruan}{*\mbox{-}tʃate}{collective of trees (\intxt{suffix})} (\citealt{PVB13b}, \#751).}

\PMlemma{{\wordnl{*\mbox{-}keʔ\pla{jʰ}}{feminine}}}

\wordng{Mk}{\mbox{-}kiʔ\pl{j}} (\citealt{AG94}: 152; \citealt{AG99}: 137, 142) {\sep} \wordng{Ni}{\mbox{-}tʃe / \mbox{-}ke} (after \textit{V\textsubscript{[+back]}}\textit{(C\textsubscript{[+grave]}})) (\intxt{\mbox{-}j}) (\citealt{AF16}: 104–105) {\sep} \wordng{PCh}{*\mbox{-}keʔ\pla{jʰ}} > \wordng{Ijw}{\mbox{-}kiʔ\pl{wa}}, \textit{\mbox{-}jis}, \textit{\mbox{-}ˀl}) [1]; \wordng{I’w}{\mbox{-}kiʔ}, \textit{\mbox{-}ki\mbox{-}jh}; \wordng{Mj}{\mbox{-}kiʔ\pl{jh}} (\citealt{JC14a}; own field notes; \citealt{JC18}) {\sep} \wordng{PW}{*\mbox{-}kʲe\pla{jʰ}} > \wordng{LB}{\mbox{-}tʃe\pl{j}} in \textit{ʔafʷen<tʃe>\pl{j}}\gloss{bird}; \wordng{’Wk}{\mbox{-}kʲeʔ\pl{ç}} in \textit{ʔaxʷén<kʲe>\pl{ç}}\gloss{bird} (\citealt{VN14}: 196, 253; \citealt{KC16}: 10)

\dicnote{The plural allomorphs in Iyojwa’aja’ are non-etymological.}%1

\lit{\citealt{LC-VG-07}: 16; \citealt{AnG15}: 64}

\PMlemma{{\wordnl{*\mbox{-}ˀmat}{negative quality, physical defect}}}

\wordng{Mk}{\mbox{-}ˀmet} [1]\gloss{physical defect} (\citealt{AG99}: 216, 328) {\sep} \wordng{Ni}{\mbox{-}ˀmat} \citep[226]{AF16} {\sep} \wordng{PCh}{*\mbox{-}ˀmat} in \wordnl{*\mbox{-}<hwá>ˀmat}{disease} (see \hyperref[dic-famat]{\wordng{PM}{*\mbox{-}ɸá\mbox{-}ˀmat}{disease}})

\dicnote{The preglottalization in the initial consonant of the Maká reflex is attested in the New Testament in derivatives such as \textit{eqfe\mbox{-}ˀmet}\gloss{ill} (Revelations 8:12), \textit{[i]tawxe\mbox{-}ˀmet}\gloss{to worry} (literally\gloss{to be bellyless/spiritless}).}%1

\PMlemma{{\wordnl{*\mbox{-}(ha\mbox{-})naˀχ\plf{*\mbox{-}(ha\mbox{-})nha\mbox{-}ts}}{agent nominalizer} (\enquote{the one who typically does~X})}}

\wordng{Mk}{\mbox{-}(he\mbox{-})naˀχ} [1], \textit{\mbox{-}(he\mbox{-})nhe\mbox{-}ts} \citep[222]{AG94} {\sep} \wordng{Ni}{\mbox{-}(xa\mbox{-})nax\plf{\mbox{-}(xa\mbox{-})nxa\mbox{-}s}} (fem. \intxt{\mbox{-}(xa\mbox{-})nxa\plf{\mbox{-}(xa\mbox{-})nxa\mbox{-}j}}) (\citealt{AF16}: 111; \citealt{LC20}: 116–117)

\dicnote{The preglottalized coda in the Maká singular form is attested in the New Testament in derivatives such as \wordnl{eku\mbox{-}naˀχ}{glutton} (Luke 7:34).}%1

\gc{\citet[315]{PVB13a} compares this prefix to \wordng{Proto-Guaicuruan}{*\mbox{-}(ˀ)naqa}\gloss{the one who has a lot of X} (\citealt{PVB13b}, \#709).}

\lit{\citealt{PVB13a}: 317 (\intxt{*\mbox{-}nah \recind *\mbox{-}hanah})}

\PMlemma{{\wordnl{*\mbox{-}ˀp}{season}}}

\wordng{Mk}{\mbox{-}ˀp} [1], \textit{\mbox{-}p\mbox{-}its} (\citealt{AG99}: 121, 202, 389; \citealt{maka-etnomat}: 23–25) {\sep} \wordng{Ni}{\mbox{-}(ˀ)p} \citep[118]{AF16} {\sep} \wordng{PCh}{*\mbox{-}p} > \wordng{Ijw}{\mbox{-}(i)p}; \wordng{Mj}{\mbox{-}(e)p} (\citealt{JC14a,JC14b,JC18}) {\sep} \wordng{PW}{*\mbox{-}p} in \wordnl{*kʲéɬ\mbox{-}kʲu\mbox{-}p}{fall season}, \wordnl{*ˣnáwo<p>}{spring}

\dicnote{In the New Testament, the coda in the Maká singular form is attested as preglottalized in \wordnl{xinawa\mbox{-}ˀp}{spring} (e.g. Mark 13:28), but not in \wordnl{ininqa\mbox{-}p}{summer, year} (e.g. Acts 18:11) and \wordnl{lo\mbox{-}p}{winter} (John 10:22). This must be a mistranscription, as the forms \intxt{xinawa\mbox{-}ˀp}, \intxt{ininqa\mbox{-}ˀp}, \intxt{lo\mbox{-}ˀp}, \intxt{anheji\mbox{-}ˀp}, \intxt{keɬe\mbox{-}jku\mbox{-}ˀp} (misspelt as ‹keleiku’p›) are documented in \citet[23–25]{maka-etnomat}.}%1

\PMlemma{{\intxt{*\mbox{-}qá\mbox{-}} (before C) \intxt{/ *\mbox{-}q\mbox{-}} (before V)\gloss{indirect possession}}}

\wordng{Mk}{\mbox{-}qe\mbox{-} / \mbox{-}qa\mbox{-} / \mbox{-}qo\mbox{-} / \mbox{-}q\mbox{-}} \citep[149]{AG94} {\sep} \wordng{Ni}{\mbox{-}ka\mbox{-} / \mbox{-}k\mbox{-}} (\citealt{AF16}: 86–88; \citealt{JS16}: 53) {\sep} \wordng{PCh}{*\mbox{-}qá\mbox{-} / *\mbox{-}q\mbox{-}} > \wordng{Ijw/I’w/Mj}{\mbox{-}ká\mbox{-} / \mbox{-}k\mbox{-}} (\citealt{JC14a}; \citealt{AG83}: 136–137; \citealt{JC18}) {\sep} \wordng{PW}{*\mbox{-}qá\mbox{-} / *\mbox{-}q\mbox{-}} > \wordng{LB}{\mbox{-}qa\mbox{-}}; \wordng{’Wk}{\mbox{-}qá\mbox{-} / \mbox{-}q\mbox{-}} (\citealt{VN14}: 168; \citealt{KC16}: 88, 305)

\gc{\citet[315]{PVB13a} compares this prefix to \wordng{Proto-Guaicuruan}{*q’o(ˀm)}\gloss{person} (\citealt{PVB13b}, \#540).}

\lit{\citealt{PVB13a}: 317 (\intxt{*q’a\mbox{-}})}

\PMlemma{{\wordnl{*\mbox{-}taχ\plf{*\mbox{-}ta\mbox{-}ts}}{pseudo-, augmentative}}}

\wordng{Mk}{\mbox{-}taχ\plf{\mbox{-}te\mbox{-}ts}} (\citealt{AG99}: 142, 174, 236, 278, 281, 294, 331, 386) {\sep} \wordng{Ni}{\mbox{-}tax\plf{\mbox{-}ta\mbox{-}s}} (\citealt{AF16}: 103–104; \citealt{JS16}: 249) {\sep} \wordng{PCh}{*\mbox{-}tah\plf{*\mbox{-}ta\mbox{-}s}} > \wordng{Ijw/I’w/Mj}{\mbox{-}ta\pl{s}} (\citealt{JC14b}: 99; \citealt{AG83}: 120, 161; \citealt{JC18}) {\sep} \wordng{PW}{*\mbox{-}taχ\plf{*\mbox{-}ta\mbox{-}s}} > \wordng{LB}{\mbox{-}taχ\plf{\mbox{-}ta\mbox{-}s}}; \wordng{’Wk}{\mbox{-}tax\plf{\mbox{-}ta\mbox{-}s}} (\citealt{VN14}: 196; \citealt{JAA-KC-14}: 441)

\lit{\citealt{PVB02}: 144 (\intxt{*\mbox{-}taχ})}

\PMlemma{{\wordnl{*\mbox{-}tseχ\plf{*\mbox{-}tse\mbox{-}ts}}{notable quality}}}

\wordng{Mk}{\mbox{-}tsaχ\plf{\mbox{-}tsi\mbox{-}ts}} (\citealt{AG94}: 223; \citealt{AG99}: 122, 223, 225, 307) {\sep} \wordng{Ni}{\mbox{-}tsex}, \textit{\mbox{-}tse\mbox{-}s} (\citealt{AF16}: 223–224) {\sep} \wordng{PW}{*\mbox{-}tsaχ}, \textit{*\mbox{-}tse\mbox{-}s} > \wordng{LB}{\mbox{-}tsaχ}, \textit{\mbox{-}tse\mbox{-}s}; \wordng{’Wk}{\mbox{-}tsax}, \textit{\mbox{-}tse\mbox{-}s} (\citealt{VN14}: 210–211; \citealt{JAA-KC-14}: 441)

\dicnote{\citet[317]{PVB13a} reconstructs this nominalizer as \textit{*\mbox{-}tsah \recind *\mbox{-}ah}, as if these were two allomorphs of the same suffix. In our reconstruction, these two morphemes have different vowels (\intxt{*\mbox{-}aχ} vs. \textit{*\mbox{-}tseχ}) and are hardly related to each other.}%1

\gc{Possibly related to \word{Proto-Guaicuruan}{*\mbox{-}ts’aqa}{the one who has or does X a lot} (\citealt{PVB13b}, \#770; cf. \citealt{PVB13a}: 317).}

\lit{\citealt{PVB13a}: 317 (\wordnl{*\mbox{-}tsah \recind *\mbox{-}ah}{nominalizer})}

\PMlemma{{\wordnl{*\mbox{-}(j)uˀk\plf{*\mbox{-}(j)ku\mbox{-}jʰ}}{tree (\intxt{suffix})} [1]}}

\wordng{Mk}{\mbox{-}(j)uˀk\plf{\mbox{-}(j)kw\mbox{-}i}} (\citealt{AG99}; \citealt{PMA}: 7) {\sep} \wordng{Ni}{\mbox{-}(j)uk\plf{\mbox{-}ku\mbox{-}j}} \citep[116]{AF16} {\sep} \wordng{PCh}{*\mbox{-}(j)uk\plf{*\mbox{-}(j)ku\mbox{-}jʰ}} > \wordng{Ijw}{\mbox{-}uk / \mbox{-}(j)ik\plf{\mbox{-}kʲuʔ / \mbox{-}tʃuʔ}}; \wordng{I’w}{\mbox{-}uk / \mbox{-}(j)ik\plf{\mbox{-}kiʔ / \mbox{-}siʔ}}; \wordng{Mj}{\mbox{-}uk / \mbox{-}(j)ik\plf{\mbox{-}kiʔ / \mbox{-}ʃiʔ}} (\citealt{JC14a}; \citealt{AG83}; \citealt{JC18}) {\sep} \wordng{PW}{*\mbox{-}(j)ukʷ\plf{*\mbox{-}kʲu\mbox{-}jʰ}} > \wordng{LB}{\mbox{-}jekʷ\plf{\mbox{-}tʃe\mbox{-}j}}; \wordng{’Wk}{\mbox{-}(j)uk\plf\mbox{-}kʲu\mbox{-}ç} (\citealt{VN14}: 192; \citealt{KC16}: 162, 187)

\dicnote{In most languages, the PM sequence \intxt{*…a\mbox{-}juk} suffers contraction of \intxt{*aju} into \intxt{*e}.}%1

\largerpage
\gc{Obviously related to \word{Proto-Guaicuruan}{*\mbox{-}iko}{tree (\intxt{suffix})} (\citealt{PVB13b}, \#706; cf. \citealt{PVB13a}: 317).}

\lit{\citealt{PVB13a}: 317 (\intxt{*\mbox{-}uk})}\clearpage

\PMlemma{{\wordnl{*\mbox{-}ˀw\mbox{-}}{relationalizing prefix}}}

\wordng{Mk}{\mbox{-}ˀw\mbox{-}} [1] (\citealt{AG99}: 251, 370) {\sep} \wordng{Ni}{\mbox{-}ˀβ\mbox{-}} (\citealt{AF16}: 89–90) {\sep} \wordng{PCh}{*\mbox{-}ˀw\mbox{-}} > \wordng{Ijw}{\mbox{-}ˀw\mbox{-}}; \wordng{Mj}{\mbox{-}w\mbox{-}} [3] (\citealt{ND09}: 94, 127; \citealt{JC18})

\dicnote{Identifiable in the pair \wordnl{efu}{woman} / \wordnl{\mbox{-}ˀw\mbox{-}efu}{female} and possibly in \wordnl{\mbox{-}ˀw\mbox{-}extits\mbox{-}iʔ}{lie} / \wordnl{\mbox{extitsaχ}}{liar}. The preglottalization is attested in the New Testament (\intxt{ɬe\mbox{-}ˀw\mbox{-}extits\mbox{-}iʔ}; Ephesians 6:11).}%1

\dicnote{The prefix can be seen in the pair \wordnl{ʔálisa}{caraguatá} / \wordnl{\mbox{-}ˀw\mbox{-}álisa}{caraguatá of}.}%2

\dicnote{The absence of glottalization in Manjui is irregular. The prefix can be seen in the pair \wordnl{ʔálasa}{caraguatá (of an unspecified plant)} / \wordnl{\mbox{-}w\mbox{-}álasa}{caraguatá of}.}%3
\end{adjustwidth}
\section{Valence and spatial suffixes or clitics} \label{vss}

\begin{adjustwidth}{6mm}{0pt}

\PMlemma{{\wordnl{*\mbox{-}ah}{towards (often metaphoric)}}}

\wordng{Ni}{\mbox{-}a} (\citealt{AF16}: 159–161) {\sep} \wordng{PCh}{*\mbox{-}ah} > \wordng{Ijw/I’w/Mj}{\mbox{-}ah} (\citealt{JC14a}; own field notes; \citealt{JC18}) {\sep} \wordng{PW}{*\mbox{-}ah} > \wordng{LB}{\mbox{-}a}\gloss{near}; \wordng{’Wk}{\mbox{-}eh} [1] (\citealt{VN14}: 249; \citealt{JAA-KC-14}: 450)

\dicnote{The reflex in ’Weenhayek is irregular; one would expect \intxt{*\mbox{-}ah}.}%1

\PMlemma{{\wordnl{*\mbox{-}(a)ˀm \recind *\mbox{-}(ä)ˀm}{for (benefactive)}}}

\wordng{Mk}{\mbox{-}(e)ˀm} [1] \citep[126]{AG94} {\sep} \wordng{Ni}{\mbox{-}(a)m} (\citealt{AF16}: 179–180)

\dicnote{The preglottalized coda in the Maká reflex is documented in the New Testament, as in the forms of the verb\gloss{to tell}: \textit{ni\mbox{-}fel\mbox{-}i\mbox{-}ˀm}, \textit{he\mbox{-}n\mbox{-}fel\mbox{-}eˀm} (Luke 1:73; Luke 4:18).}%1

\gc{\citet[316]{PVB13a} compares this suffix to \wordng{Proto-Guaicuruan}{*\mbox{-}ma}\gloss{benefactive} (\citealt{PVB13b}, \#337).}

\lit{\citealt{PVB13a}: 316 (\intxt{*\mbox{-}m})}

\PMlemma{{\intxt{*\mbox{-}ejʰ} [1]\gloss{far (distal)}}}

\wordng{Mk}{\mbox{-}ij} \citep[342]{AG99} {\sep} \wordng{Ni}{\mbox{-}ej} (\citealt{LC20}: 281) {\sep} \wordng{PCh}{*\mbox{-}ejʰ} > \wordng{Ijw}{\mbox{-}e}, \wordng{Mj/I’w}{\mbox{-}ejʰ} (\citealt{JC11}: 55, \citeyear{JC14a}, \citeyear{JC18}) {\sep} \wordng{PW}{*\mbox{-}ejʰ} > \wordng{LB}{\mbox{-}ej} \citep[276]{VN14}

\dicnote{The vowel \intxt{*\mbox{-}e\mbox{-}} in this suffix is likely a third-person suffix.}%1

\PMlemma{{\intxt{*\mbox{-}ex} [1]\gloss{instrumental}}}

\wordng{Mk}{\mbox{-}ix} (\citealt{AG99}: 127–128) {\sep} \wordng{Ni}{\mbox{-}eʃ} (\citealt{LC20}: 386–391) {\sep} \wordng{PCh}{*\mbox{-}eh} > \wordng{Ijw/I’w/Mj}{\mbox{-}e} (\citealt{JC11}: 55, \citeyear{JC14a}; own field notes) {\sep} \wordng{PW}{*\mbox{-}eχ} > \wordng{LB}{\mbox{-}eχ}; \wordng{’Wk}{\mbox{-}ex} (\citealt{VN14}: 134; \citealt{JAA-KC-14}: 450)

\dicnote{The vowel \intxt{*\mbox{-}e\mbox{-}} in this suffix is likely a third-person suffix.}%1

\PMlemma{{\intxt{*\mbox{-}ɸíh / *\mbox{-}qɸíh / *\mbox{-}kåɸíh} [1]\gloss{below, beneath}}}

\wordng{Mk}{\mbox{-}fi} \citep[123]{AG99} {\sep} \wordng{Ni}{\mbox{-}<ʔa>kɸi \recind \mbox{-}<ʔa>kxi \recind \mbox{-}<ʔå>kɸi \recind \mbox{-}<ʔå>kxi} [2] (\citealt{AF16}: 169; \citealt{JS16}: 36; \citealt{LC20}: 8) {\sep} \word{PCh}{*\mbox{kåhwíh} / *\mbox{-}kǻhwih}{inside, below, beneath} > \wordng{Ijw}{kʲahwéh / \mbox{-}kʲáhwe}; \textit{*qihwíh / *\mbox{-}qíhwih} > \wordng{I’w}{\mbox{-}kifʷí}; \wordng{Mj}{kihwíjh / \mbox{-}kéihwi} (\citealt{JC14a}; \citealt{ND09}: 135; \citealt{AG83}: 127; \citealt{JC18}; \citealt{GH94}) {\sep} \wordng{PW}{*=qxʷíh / *=kʲåxʷíh} > \wordng{LB}{[ʔi]qfʷi / =qfʷi} [3] \textit{/ =tʃefʷi} [4]; \wordng{Vej}{tʃuhʷi} [4]\gloss{inside}; \wordng{’Wk}{\mbox{-}kʲåxʷíh} (\citealt{VN14}: 249, 276; \citealt{VU74}: 53; \citealt{KC16}: 218; \citealt{JAA-KC-14}: 450)

\dicnote{\wordng{PM}{*\mbox{-}ɸíh} is preserved in Maká, \intxt{*\mbox{-}qɸíh} in Nivaĉle and Lower Bermejeño Wichí, \intxt{*\mbox{-}kåɸíh} in Chorote and Wichí.}%1

\dicnote{The variants with \textit{ɸ} are found in the Shichaam Lhavos dialect; \textit{\mbox{-}ʔåkxi} (\intxt{\recind \mbox{-}βåkxi}) is documented by \citet[285–286]{LC20} for the Chishamnee Lhavos dialect; \textit{\mbox{-}ʔakxi} is attested by \citet[36]{JS16} for Yita’ Lhavos.}%2

\dicnote{\citet{VN14} actually gives \wordng{LB}{=fʷi}, but in all her examples the clitic is preceded by a \textit{q}.}%3

\dicnote{\wordng{LB}{e} and \wordng{Vej}{u} are not the regular reflex of \wordng{PW}{*å}.}%4

\PMlemma{{\textit{*\mbox{-}ɸVk’e(ʔ)} [1]\gloss{outside}}}

\wordng{Mk}{\mbox{-}fik’i} \citep[117]{AG94} {\sep} \wordng{Ni}{\mbox{-}ɸatʃ’eʔ} \citep[169]{AF16}

\dicnote{Maká points to \intxt{*\mbox{-}ɸek’e} or \intxt{*\mbox{-}ɸik’e}; Nivaĉle to \intxt{*\mbox{-}ɸak’eʔ} or \intxt{*\mbox{-}ɸäk’eʔ}.}%1

\gc{\citet[316]{PVB13a} compares this suffix to \wordng{Proto-Guaicuruan}{*\mbox{-}ek’e}\gloss{outwards} (\citealt{PVB13b}, \#725).}

\lit{\citealt{PVB13a}: 316 (\intxt{*(\mbox{-})hʷek’e})}

\PMlemma{{\wordnl{*\mbox{-}hat}{(direct) causative}}}

\wordng{Mk}{\mbox{-}het} \citep[107]{AG99} {\sep} \wordng{Ni}{\mbox{-}xat} (\citealt{AF16}: 216–217; \citealt{JS16}: 146) {\sep} \wordng{PCh}{*\mbox{-}hat} > \wordng{Ijw/I’w/Mj}{\mbox{-}hat} (\citealt{JC14a}; own field notes) {\sep} \wordng{PW}{*\mbox{-}hat} > \wordng{LB}{\mbox{-}hat}; \wordng{’Wk}{\mbox{-}hat} (\citealt{VN14}: 253–254; \citealt{KC16}: 146)

\gc{\citet[317]{PVB13a} compares this suffix to \wordng{Proto-Guaicuruan}{*\mbox{-}aq\mbox{-}atV \recind *\mbox{-}atV}\gloss{instrumental transitivizer}.}

\lit{\citealt{PVB13a}: 318 (\intxt{*\mbox{-}qVt \recind *\mbox{-}hVt \recind *\mbox{-}Vt})}

\PMlemma{{\wordnl{*\mbox{-}han}{(indirect) causative; antipassive} [1 2]}}

\wordng{Mk}{\mbox{-}hen<in>}; \textit{\mbox{-}<ts>hen}\gloss{causative} \citep[106]{AG99} {\sep} \wordng{Ni}{\mbox{-}xan} \citep[310]{AF16} {\sep} PCh~*\textit{\mbox{-}han} > \wordng{Ijw/Mj}{\mbox{-}han} (\citealt{JC14a}; own field notes)

\dicnote{This suffix is preserved in Wichí only in fossilized derivations (as in \wordng{PW}{*[ʔi]kʲún<han>}\gloss{to feed}, which goes back to \wordng{PM}{*[ʔi]kún\mbox{-}han} but is no longer analyzable).}%1

\dicnote{It is possible that \wordnl{*\mbox{-}han}{(indirect) causative} and \wordnl{*\mbox{-}han}{antipassive} were originally two distinct morphemes. Only the former, but not the latter, might have been a reduced allomorph of a longer suffix \intxt{*\mbox{-}hajin}, with reflexes in Nivaĉle \citep[217]{AF16} and Chorote (after low vowels, with \intxt{*…å\mbox{-}ha…/…*a\mbox{-}ha…} yielding \intxt{*å / *a}, as in \word{PCh}{*[ʔi]mǻ\mbox{-}jin}{to make sleep} and \wordnl{*\mbox{-}ˀjǻ\mbox{-}jin\mbox{-}\APPL}{to give to drink}).}%2

\gc{This suffix could be related to \word{Proto-Guaicuruan}{*\mbox{-}aqen}{agentive transitivizer} (\citealt{PVB13b}, \#727).}

\PMlemma{{\wordnl{*=hajuʔ}{prospective; desiderative}}}

\wordng{Mk}{\mbox{-}hijuʔ / \mbox{-}hejuʔ} [1] (\citealt{AG94}: 109–111) {\sep} \wordng{Ni}{=xaju} (\citealt{AF16}: 219; \citealt{LC20}: 313–314) {\sep} \wordng{PCh}{*\mbox{-}hajuʔ} > \wordng{I’w}{\mbox{-}má\mbox{-}juʔ}\gloss{to want to sleep}; \wordng{Mj}{\mbox{-}hajuʔ \recind \mbox{-}hajiʔ \recind \mbox{-}heeʔ} (\citealt{JC14a}; \citealt{AG83}: 105; \citealt{JC18})

\dicnote{The suffix-final glottal stop is not represented in \citet{AG94}, but it found in most available examples in \citet{AG99}. After some consonants the \textit{h} is lost. After vowels other than \textit{i}, one finds the allomorphs \intxt{\mbox{-}ju / \mbox{-}jo}, and after \intxt{j} the suffix may be simply \textit{\mbox{-}u} in Maká. The alternation \intxt{i / e} is irregular.}%1

\PMlemma{{\textit{*\mbox{-}käj}\gloss{antipassive}}}

\wordng{Mk}{\mbox{-}kij} [1] \citep[119]{AG94} {\sep} \wordng{Ni}{\mbox{-}tʃaj} (\citealt{AF16}: 198–199) {\sep} \wordng{PCh}{*\mbox{-}kejʔ} > \wordng{Ijw}{[ta]k(á)\mbox{-}…\mbox{-}kiʔ}; \wordng{Mj}{[ti]k(á)\mbox{-}…\mbox{-}kijʔ} (\citealt{JC14a,JC18})

\dicnote{The Maká reflex is irregular; one would expect \intxt{\mbox{-}kej}.}%1

\PMlemma{{\textit{*\mbox{-}kʲ’e}\gloss{along; distributive, plural object} [1]}}

\wordng{Mk}{\mbox{-}k’i} \citep[125]{AG94} {\sep} \wordng{Ni}{\mbox{-}tʃ’e(ʔ) / \mbox{-}k’e(ʔ)} (\citealt{AF16}: 165–167; \citealt{LC20}: 112, 278–279) {\sep} \wordng{PCh}{*\mbox{-}k’eʔ} > \wordng{Ijw}{\mbox{-}k’iʔ}; \wordng{Mj}{\mbox{-}ʔiʔ} (\citealt{JC14a,JC14b,JC18}) {\sep} \wordng{PW}{*\mbox{-}kʲe} [2] > \wordng{LB}{\mbox{-}tʃe}; \wordng{’Wk}{\mbox{-}kʲeʔ} (\citealt{VN14}: 134; \citealt{JAA-KC-14}: 439; \citealt{KC16}: 186)

\dicnote{We refer the reader to \cits{AF18} study on the functions of this suffix.}%1

\dicnote{The initial consonant irregularly deglottalized in Wichí.}%2

\PMlemma{{\wordnl{*k’oja(ʔ) / *\mbox{-}k’ója(ʔ)}{before, for}}}

\wordng{Ni}{\mbox{-}k’ója}\gloss{before, for, than} (\citealt{AF16}: 184–186; \citealt{JS16}: 88; \citealt{LC20}: 284) {\sep} \wordng{PCh}{*k’ojáʔ / *\mbox{-}k’ójaʔ}\gloss{for} > \wordng{Ijw}{k’ijé / \mbox{-}kʲ’óje} [1]; \wordng{Mj}{\mbox{ʔijéʔ} / \mbox{-}ʔʲójeʔ} (\citealt{JC14b}: 90; \citealt{ND09}: 138; \citealt{JC18}) {\sep} \wordng{PW}{*\mbox{-}kʲ’ója} > \wordng{LB}{\mbox{-}tʃ’uja}\gloss{sensorially}; \word{Vej}{\mbox{-}tʃ’oje}{inside} [2]; \word{’Wk}{\mbox{-}kʲ’ojeʔ}{invisible, absent} [2] (\citealt{VN14}: 313–315; \citealt{VU74}: 53; \citealt{MG-MELO15}: 35; \citealt{JAA-KC-14}: 450)

\dicnote{\wordng{Ijw}{k’ijé / \mbox{-}kʲ’óje}, which lacks a word-final glottal stop and thus ends in an underlying /h/, is irregular. One would expect \intxt{*k’ijéʔ / *\mbox{-}kʲ’ójeʔ}.}%1

\dicnote{\sound{Vej/’Wk}{e} is not the regular reflex of \sound{PW}{*a}.}%2

\PMlemma{{\wordnl{*\mbox{-}taxam \recind *\mbox{-}ä\mbox{-}}{into, entering}}}

\wordng{Mk}{\mbox{-}texem} \citep[118]{AG94} {\sep} \wordng{Ni}{\mbox{-}taʃam} (\citealt{AF16}: 176–177)

\PMlemma{{\wordnl{*\mbox{-}wäˀt}{reflexive} [1]}}

\wordng{Mk}{\mbox{-}wet\mbox{-} \recind \mbox{-}t\mbox{-}} [2] \citep[117]{AG94} {\sep} \wordng{Ni}{\mbox{-}βat\mbox{-} / \mbox{-}βaˀt\mbox{-}} (\citealt{LC20}: 297–298) {\sep} \wordng{PCh}{*\mbox{-}wét}\gloss{reflexive/reciprocal} > \wordng{Ijw}{wit\mbox{-}áˀm} [3]\gloss{reciprocal (with an object as the antecedent)}; \wordng{I’w}{\mbox{-}wét}; \wordng{Mj}{\mbox{-}wέt}\gloss{reflexive/reciprocal} (\citealt{JC14a}; \citealt{ND09}: 157; \citealt{AG83}: 169–170; \citealt{JC18})

\dicnote{At least in Iyo’awujwa’ and Manjui the reflexes of this marker (which precedes the verb) are phonologically independent from the verb. The hyphen on the left indicates the slot that corresponds to the subject (agent), not to the verb.}%1

\dicnote{The Maká reflex unexpectedly lacks a preglottalized coda, as attested in the New Testament (e.g. \textit{wet\mbox{-}fel}\gloss{to greet}; Philemon 1:23).}%2

\dicnote{\wordng{Ijw}{\mbox{-}áˀm} corresponds to the applicative \wordng{PCh}{*\mbox{-}håm}\gloss{through}. The lack of palatalization in \textit{t} is unexpected after a pretonic \wordng{PCh}{*e > i}. The palatalization process may have been inactive when *\textit{wét} lost its stress and changed to \textit{wit}, or maybe both morphemes merged when palatalization was inactive.}%3

\PMlemma{{\intxt{*\mbox{-}xAˀm} [1]\gloss{general locative}}}

\wordng{Mk}{\mbox{-}xeˀm} [2]\gloss{through} (\citealt{AG94}: 119–120) {\sep} \wordng{Ni}{\mbox{-}ʃaˀm / \mbox{-}xaˀm} (after \textit{V\textsubscript{[+back]}}\textit{(C\textsubscript{[+grave]}})) (\citealt{AF16}: 169–170; \citealt{LC20}: 286–288) {\sep} \wordng{PCh}{*\mbox{-}håˀm} > \wordng{Ijw/I’w/Mj}{\mbox{-}haˀm} (\citealt{JC14a}; own field notes)

\dicnote{Maká points to \wordng{PM}{*\mbox{-}xaˀm} or \intxt{*\mbox{-}xäˀm}; Iyojwa’aja’ to \wordng{PM}{*\mbox{-}xåˀm}, whereas Nivaĉle, Iyo’awujwa’, and Manjui are ambiguous in this sense.}%1

\dicnote{The preglottalized coda in the Maká reflex is attested in the New Testament (e.g. \wordnl{tux\mbox{-}xeˀm}{to burn}; Ephesians 6:16).}%2

\lit{\citealt{AnG15}: 64}

\PMlemma{{\wordnl{*\mbox{-}xiʔ}{inside a recipient}}}

\wordng{Mk}{\mbox{-}xiʔ} \citep[119]{AG94} {\sep} \wordng{Ni}{\mbox{-}ʃi / \mbox{-}xi} (after \intxt{V\textsubscript{[+back]}(C\textsubscript{[+grave]})}) (\citealt{LC20}: 289–290) {\sep} \wordng{PCh}{*\mbox{-}hiʔ} > \wordng{Ijw/I’w}{\mbox{-}hiʔ}; \wordng{Mj}{\mbox{-}hijʔ} (\citealt{JC11}: 55, \citeyear{JC14a}; own field notes; \citealt{JC18}) {\sep} \wordng{PW}{*\mbox{-}hi} > \wordng{LB}{\mbox{-}hi}; \wordng{’Wk}{\mbox{-}hiʔ} (\citealt{VN14}: 148; \citealt{KC16}: 148)

\gc{\citet[316]{PVB13a} compares it to the Proto-Guaicuruan locative suffix \intxt{*\mbox{-}ˀgi} (\citealt{PVB13b}, \#790).}

\lit{\citealt{PVB02}: 143 (\intxt{*\mbox{-}xij}); \citealt{PVB13a}: 316 (\intxt{*\mbox{-}hij}); \citealt{AnG15}: 64}

\PMlemma{{\wordnl{*\mbox{-}xop}{next to, surrounding}}}

\wordng{Mk}{\mbox{-}xup} \citep[129]{AG94} {\sep} \wordng{Ni}{\mbox{-}xop} (\citealt{AF16}: 174–175) {\sep} PCh *\textit{\mbox{-}hop} [1] > \wordng{Ijw}{\mbox{-}hap}; \wordng{I’w}{\mbox{-}hop}; \wordng{Mj}{\mbox{-}hap} (own field notes)

\dicnote{We reconstruct \wordng{PCh}{*\mbox{-}hop} based on the regular correspondence between \wordng{I’w}{\mbox{-}hop} (attested in our field notes with person prefixes) and Nivaĉle. The Ijw/Mj reflex \textit{\mbox{-}hap} (underlying /\mbox{-}håp/ in Ijw) is irregular.}%1

\gc{\citet[320]{PVB13a} compares this to \wordng{Proto-Guaicuruan}{*\mbox{-}atʃ’ap}\gloss{near, next to} (\citealt{PVB13b}, \#154), which could be spurious.}

\lit{\citealt{PVB02}: 142 (\intxt{*\mbox{-}xop}); \citealt{PVB13a}: 320 (\intxt{*\mbox{-}hVp}\gloss{near})}

\PMlemma{{\textit{*\mbox{-}xoʔ}\gloss{down / inwards}}}

\wordng{Mk}{\mbox{-}xuʔ \recind \mbox{-}xoʔ}\gloss{down} \citep[118]{AG94} {\sep} \wordng{PW}{*\mbox{-}ho} > \wordng{LB}{\mbox{-}hu}\gloss{inwards, entering, for}; \wordng{’Wk}{\mbox{-}hoʔ}\gloss{entering, exiting, for} (\citealt{VN14}: 249, 259; \citealt{KC16}: 151; \citealt{JAA-KC-14}: 450)

\PMlemma{{\textit{*\mbox{-}xuˀɬ}\gloss{in front of, approaching}}}

\wordng{Mk}{\mbox{-}xuˀɬ}\gloss{in front of} [1] \citep[128]{AG94} {\sep} \wordng{Ni}{\mbox{-}xuˀɬ}\gloss{approaching; same as} (\citealt{AF16}: 182–184)

\dicnote{The preglottalized coda in the Maká applicative suffix is attested in the New Testament (e.g. \wordnl{[t]’ekuˀm\mbox{-}ixuˀɬ}{to grab something from one’s front}; Luke 24:43).}%1

\PMlemma{{\textit{*ʔapé(\mbox{-}ʔeʔ) / *\mbox{-}tápe(\mbox{-}ʔeʔ)}\gloss{on, on top of}}}

\wordng{Ni}{=ʔape<ʔe> / \mbox{-}tape<ʔe>} (\citealt{AF16}: 167–168; \citealt{JS16}: 47; \citealt{LC20}: 337–338) {\sep} \wordng{PCh}{*ʔapé<ʔeʔ> / *\mbox{-}tépe<ʔeʔ>} [1] > \wordng{Ijw}{ʔapέʔɛ / \mbox{-}tέpeʔe} [2]; \wordng{I’w}{apéʔe} [2]; \wordng{Mj}{ʔapέʔɛʔ / \mbox{-}tέpeʔeʔ} (\citealt{JC14a}; \citealt{ND09}: 94; \citealt{AG83}: 126; \citealt{JC18}) {\sep} \wordng{PW}{*\mbox{-}ʔpeʔ / *\mbox{-}t(a)peʔ} [3] > \wordng{LB}{=peʔ}; \wordng{Vej}{\mbox{-}nu\mbox{-}pe}\gloss{to surpass}; \wordng{’Wk}{\mbox{-}ʔpeʔ / \mbox{-}t(a)peʔ} (\citealt{VN14}: 276; \citealt{VU74}: 69; \citealt{JAA-KC-14}: 450)

\dicnote{Chorote appears to have undergone some sort of vowel harmonization.}%1

\dicnote{\wordng{Ijw}{ʔapέʔɛ / \mbox{-}tέpeʔe} and \wordng{I’w}{apéʔe}, which lack a word-final glottal stop and thus end in an underlying /h/, are irregular (in fact, this could be a mistranscription for \textit{ʔapέʔɛʔ / \mbox{-}tέpeʔeʔ}).}%2

\dicnote{\wordng{PW}{*\mbox{-}ʔpeʔ} unexpectedly lack a vowel between \intxt{*ʔ} and \intxt{*p}.}%3
\end{adjustwidth}
\section{Demonstratives} \label{demons}

\begin{adjustwidth}{6mm}{0pt}

\PMlemma{{\wordnl{*h\mbox{-}}{that (outside the speaker’s sight)}}}

Mk~\textsc{m}~\textit{haʔ}, \textsc{pl}~\textit{heʔ} \citep[166]{AG94} {\sep} Ni~\textsc{m}~\textit{xaʔ}, \textsc{f}~\textit{ɬ\mbox{-}xaʔ}, \textsc{pl.h}~\textit{xa\mbox{-}piʔ}, \textsc{pl.nh}~\textit{xa\mbox{-}βaʔ}\gloss{absent at utterance time; firsthand evidence available} (\citealt{AnG15-evid}: 415; \citealt{LC20}: 175) {\sep} PCh~\textsc{m}~\textit{*háʔ \recind *hǻʔ}, \textsc{f}~\textsc{*}\textit{hla\mbox{-}háʔ \recind hlå\mbox{-}hǻʔ}, \textsc{pl.h}~\textit{*ha\mbox{-}púʔ \recind *hå\mbox{-}púʔ}, \textsc{pl.nh}~\textit{*ha\mbox{-}wáʔ \recind *hå\mbox{-}wáʔ} > Ijw~\textsc{m}~\textit{háʔ}, \textsc{f}~\textit{hla\mbox{-}háʔ}, \textsc{pl.h}~\textit{ha\mbox{-}póʔ}, \textsc{pl.nh}~\wordnl{ha\mbox{-}wáʔ}{that (outside the speaker’s sight but seen before)}; Mj~\textsc{m}~\textit{ha}, \textsc{f}~\textit{la\mbox{-}ha}, \textsc{pl.h}~\textit{ha\mbox{-}pʊ}, \textsc{pl.nh}~\textit{ho\mbox{-}wa} (\citealt{JC14b}: 78, \citeyear{JC14a}; \citealt{ND09}: 169; \citealt{JC18})

\PMlemma{{\wordnl{*k\mbox{-}}{that (outside the speaker’s sight)}}}

Mk~\textsc{m}~\textit{kaʔ}, \textsc{f}~\textit{keʔ}, \textsc{pl}~\textit{ke\mbox{-}kheweʔ \recind keʔ}\gloss{that (outside the speaker’s sight but seen before)} \citep[166]{AG94} {\sep} Ni~\textsc{m}~\textit{kaʔ}, \textsc{f}~\textit{ɬ\mbox{-}kaʔ}, \textsc{pl.h}~\textit{ka\mbox{-}piʔ}, \textsc{pl.nh}~\textit{ka\mbox{-}βaʔ} [1]\gloss{no longer in existence, deceased, or moving across one’s field of vision about to move out of sight; firsthand evidence available} (\citealt{AnG15-evid}: 415; \citealt{LC20}: 175) {\sep} PCh~\textsc{m}~\textit{*kǻʔ}, \textsc{f}~\textsc{*}\textit{ha\mbox{-}kǻʔ \recind *hå\mbox{-}kǻʔ}, \textsc{pl.h}~\textit{*kå\mbox{-}púʔ}, \textsc{pl.nh}~\textit{*ko\mbox{-}wáʔ} > Ijw~\textsc{m}~\textit{kʲáʔ \recind k<íʔ>}, \textsc{f}~\textit{ha\mbox{-}kʲáʔ}, \textsc{pl.h}~\textit{kʲa\mbox{-}póʔ}, \textsc{pl.nh}~\textit{ki\mbox{-}wáʔ \recind kʲu\mbox{-}wáʔ}; Mj~\textsc{m}~\textit{kʲé}, \textsc{f}~\textit{ha\mbox{-}kʲé}, \textsc{pl.h}~\textit{kʲe\mbox{-}pʊ́}, \textsc{pl.nh}~\textit{kʲo\mbox{-}wá} (\citealt{JC14a}; \citealt{ND09}: 169; \citealt{JC18})

\dicnote{The failure of \wordng{PM}{*k} to palatalize in Nivaĉle before an \textit{a} is unexpected. If the gender distinction seen in Maká goes back to Proto-Mataguayan, we might be dealing with contamination of \wordng{PM}{*kåʔ} (masculine) and \intxt{*kaʔ} (feminine), whose expected reflexes in Nivaĉle would be \intxt{*kåʔ} and \intxt{*tʃaʔ}, respectively.}%1

\dicnote{Possibly related to \word{Proto-Guaicuruan}{*k’a}{absent, [−visible]} (\citealt{PVB13b}, \#337; cf. \citealt{PVB13a}: 313), though the semantic match is imperfect.}

\lit{\citealt{PVB13a}: 313 (\wordnl{*kaʔ}{this})}

\PMlemma{{\textit{*\mbox{-}khaʔ}\gloss{emphatic/pronominal base} [1 2], as in \intxt{*ˀn\mbox{-}V\mbox{-}khaʔ}; \intxt{*n\mbox{-}V\mbox{-}khaʔ}; \intxt{*ts\mbox{-}V\mbox{-}khaʔ}; \intxt{*h\mbox{-}V\mbox{-}khaʔ}; \intxt{*k\mbox{-}V\mbox{-}khaʔ}; \intxt{*p\mbox{-}V\mbox{-}khaʔ}}}

Mk~\textsc{m}~\textit{n\mbox{-}a\mbox{-}khaʔ}, \textsc{f}~\textit{n\mbox{-}e\mbox{-}kheʔ}, \textsc{pl}~\textit{n\mbox{-}e\mbox{-}khe\mbox{-}weʔ}; \textsc{m}~\textit{tsa\mbox{-}kha\mbox{-}}, \textsc{f}~\textit{tse\mbox{-}khe\mbox{-}}; \textsc{m}~\textit{ha\mbox{-}khaʔ}, \textsc{f}~\textit{ki\mbox{-}kheʔ}, \textsc{pl~}\textit{he\mbox{-}khe\mbox{-}weʔ}; \textsc{m}~\textit{ka\mbox{-}khaʔ}, \textsc{f}~\textit{ke\mbox{-}kheʔ}, \textsc{pl}~\textit{ke\mbox{-}khe\mbox{-}weʔ}; \textsc{m}~\textit{pa\mbox{-}khaʔ}, \textsc{f}~\textit{pe\mbox{-}kheʔ}, \textsc{pl}~\textit{pe\mbox{-}khe\mbox{-}weʔ}; (\citealt{AG94}: 170–172) {\sep} \wordng{PCh}{*\mbox{-}hqa} [3] > \wordng{Ijw}{ˀná\mbox{-}ka}; \intxt{ná\mbox{-}ka}; \intxt{sé\mbox{-}ka}; \intxt{há\mbox{-}ka}; \intxt{kʲá\mbox{-}ka}; \intxt{pá\mbox{-}ka}; \wordng{I’w/Mj}{ˀná\mbox{-}hak}; \intxt{ná\mbox{-}hak}; (\intxt{sɪ́\mbox{-}hɪk}) [4]; \intxt{há\mbox{-}hak}; \intxt{kʲé\mbox{-}hek}; \intxt{pá\mbox{-}hak} [5 6] (\citealt{JC14a}, own field notes, 2018)

\dicnote{In Chorote, demonstratives with this suffix are usually translated into Spanish as adnominal or pronominal demonstratives, whereas the corresponding forms without this suffix tend to be translated as articles. In Maká, this suffix is added to demonstrative bases to form emphatic and indefinite demonstratives (\citealt{AG94}: 170–172). Furthermore, a form \textit{\mbox{-}akhaʔ}, probably related, forms personal and possessive pronouns with personal prefixes, e.g. \wordnl{j\mbox{-}akhaʔ}{I, mine} \citep[174\mbox{-}177]{AG94}.}%1

\dicnote{The Chorote reflex of the vowel does not allow to decide between \sound{PM}{*a} and \intxt{*å}, and the Maká reflexes \textsc{f}~\intxt{\mbox{-}khe\mbox{-}} alongside \textsc{m}~\intxt{\mbox{-}kha\mbox{-}} suggest both. The vowel of the Maká suffix seems to copy the gender vowel of the base. However, in the Maká plural, where no gender distinction is involved, only the allomorph \intxt{\mbox{-}khe\mbox{-}} occurs, which suggests this is the basic one. Therefore, we reconstruct \intxt{*\mbox{-}kha} rather than \intxt{*\mbox{-}khå.}}%2

\dicnote{For simplicity, in Chorote only masculine singular forms are given. Notice, however, that, the plural suffix precedes the emphatic one, unlike in Maká: \wordng{Ijw}{ni\mbox{-}wá\mbox{-}ka}, \word{I’w/Mj}{nu\mbox{-}wá\mbox{-}hak}{these ones (non-human)}, etc.}%3

\dicnote{The form \textit{sɪ́\mbox{-}hɪk} is not attested for Iyo’awujwa’ in our material.}%4

\dicnote{Iyo’awujwa’ and Manjui show an irregular metathesis: \textit{*\mbox{-}hqa > \mbox{-}hak}.}%5

\dicnote{Most probably related are Manjui forms \textit{Cá\mbox{-}hka\mbox{-}ta} such as \textit{ná\mbox{-}hka\mbox{-}ta}\gloss{this only one}.}%6

\PMlemma{{\textit{*ɬ̩\mbox{-}}\gloss{\textsc{f}~(in demonstratives)}, as of \intxt{*ɬ̩\mbox{-}n\mbox{-}…}; \textit{*ɬ̩\mbox{-}ts\mbox{-}…}; \textit{*ɬ̩\mbox{-}h\mbox{-}…}; \textit{*ɬ̩\mbox{-}k\mbox{-}…}; \textit{*ɬ̩\mbox{-}p\mbox{-}…}}}

Ni –; –; \textit{ɬ\mbox{-}xaʔ}; \textit{ɬ\mbox{-}kaʔ}; \textit{ɬ\mbox{-}paʔ} (\citealt{AnG15-evid}: 414–415; \citealt{LC20}: 175) {\sep} \wordng{PCh}{*ha\mbox{-}náʔ \recind *hå\mbox{-}náʔ}; \textit{*ha\mbox{-}séʔ \recind *hå\mbox{-}séʔ}; \textit{*hla\mbox{-}háʔ \recind *hlå\mbox{-}hǻʔ}; \textit{*ha\mbox{-}kǻʔ \recind *hå\mbox{-}kǻʔ}; \textit{*ha\mbox{-}páʔ \recind *ha\mbox{-}pǻʔ \recind *hå\mbox{-}páʔ \recind *hå\mbox{-}pǻʔ} [1] > \wordng{Ijw}{ha\mbox{-}náʔ}; \textit{ha\mbox{-}sέʔ}; \textit{hla\mbox{-}háʔ}; \textit{ha\mbox{-}kʲáʔ}; \textit{ha\mbox{-}páʔ}; \wordng{Mj}{ha\mbox{-}na}; \textit{ha\mbox{-}sɪ́ʔ \recind ha\mbox{-}sɪ}; \textit{la\mbox{-}ha}; \textit{ha\mbox{-}kʲé}; \textit{ha\mbox{-}pá} (\citealt{JC14a}; \citealt{ND09}: 169; \citealt{JC18})

\dicnote{We have no convincing explanation for the fact that all contemporary Chorote varieties have \intxt{a} instead of the expected \intxt{*i} in this prefix (except for \intxt{*hla\mbox{-}háʔ \recind *hlå\mbox{-}hǻʔ}, where a low vowel was copied from \textit{*háʔ \recind *hǻʔ} by means of translaryngeal harmony early enough so as to prevent \intxt{*hl\mbox{-}} from changing to \intxt{*hᵊ\mbox{-}}).}%1

\PMlemma{{\wordnl{*ɬaʔ}{this.\textsc{f}~(within one’s hands’ reach)}}}

\word{Ni}{ɬaʔ}{present at utterance time; firsthand evidence available (feminine)} (\citealt{AnG15-evid}: 415; \citealt{LC20}: 175) {\sep} \wordng{PCh}{*hlaʔ<ah>} > \wordng{Ijw}{hlaʔa}; \wordng{I’w}{sʲú\mbox{-}hla}; \wordng{Mj}{hlaʔa} (\citealt{JC14a}; \citealt{ND09}: 169; \citealt{AG83}: 160; \citealt{JC18})

\PMlemma{{\wordnl{*n\mbox{-}}{this (outside one’s hands’ reach)}}}

Mk~\textsc{m}~\textit{naʔ}, \textsc{f}~\textit{neʔ}, \textsc{pl}~\textit{ne\mbox{-}khe\mbox{-}weʔ \recind neʔ} \citep[166]{AG99} {\sep} PCh~\textsc{m}~\textit{*náʔ}, \textsc{f}~\textsc{*}\textit{ha\mbox{-}náʔ}, \textsc{pl.h}~\textit{*na\mbox{-}púʔ}, \textsc{pl.nh}~\textit{*no\mbox{-}wáʔ} > Ijw~\textsc{m}~\textit{náʔ \recind n<íʔ>}, \textsc{f}~\textit{ha\mbox{-}náʔ}, \textsc{pl.h}~\textit{na\mbox{-}póʔ}, \textsc{pl.nh}~\textit{ni\mbox{-}wáʔ \recind nʲu\mbox{-}wáʔ}; Mj~\textsc{m}~\textit{ná}, \textsc{f}~\textit{ha\mbox{-}ná}, \textsc{pl.h}~\textit{na\mbox{-}pʊ́}, \textsc{pl.nh}~\textit{no\mbox{-}wá} (\citealt{JC14a}; \citealt{ND09}: 169; \citealt{JC18}) {\sep} \word{PW}{*=nah}{this (within one’s hands’ reach)} > \wordng{LB/Vej}{=na}; \wordng{’Wk}{\mbox{-}nah}; (?) \wordnl{*=n<ih>}{this (outside one’s hands’ reach, vertical)} > \wordng{LB}{=ni}; \wordng{’Wk}{\mbox{-}nih \recind \mbox{-}nåh \recind \mbox{-}noh} (\citealt{VN14}: 177–178; \citealt{MG-MELO15}: 70; \citealt{JAA-KC-14}: 446)

\gc{Possibly related to \word{Proto-Guaicuruan}{*ˀna}{proximal, in movement} (\citealt{PVB13b}, \#420; cf. \citealt{PVB13a}: 313).}

\lit{\citealt{PVB13a}: 313 (\wordnl{*nʌʔ \recind *naʔ}{this})}

\PMlemma{{\wordnl{*ˀnaʔ}{this.\textsc{m}~(within one’s hands’ reach)}}}

Mk~\textsc{m}~\textit{haʔ\mbox{-}neʔ}, \textsc{f}~\textit{e\mbox{-}neʔ}, \textsc{pl}~\textit{e\mbox{-}ne\mbox{-}weʔ} \citep[166]{AG94} {\sep} Ni~\textsc{m}~\textit{naʔ}, \textsc{pl.h}~\textit{na\mbox{-}piʔ}, \textsc{pl.nh}~\textit{na\mbox{-}βaʔ}\gloss{present at utterance time; firsthand evidence available (masculine)} (\citealt{AnG15-evid}: 415; \citealt{LC20}: 175) {\sep} PCh~\textsc{m}~\textit{*ˀnáʔ}, \textsc{pl.h}~\textit{*ˀna\mbox{-}púʔ}, \textsc{pl.nh}~\textit{*ˀno\mbox{-}wáʔ} > Ijw~\textsc{m~}\textit{ˀnáʔ \recind ˀn<íʔ>}, \textsc{pl.h}~\textit{ˀna\mbox{-}póʔ}, \textsc{pl.nh}~\textit{ˀni\mbox{-}wáʔ \recind ˀnʲu\mbox{-}wáʔ}; I’w~\textsc{m}~\textit{sʲúh\mbox{-}na}, \textsc{pl.h}~\textit{sʲúh\mbox{-}na\mbox{-}po}, \textsc{pl.nh}~\textit{sʲúh\mbox{-}nu\mbox{-}wa}; Mj~\textsc{m}~\textit{ˀná}, \textsc{pl.h}~\textit{ˀna\mbox{-}pʊ́}, \textsc{pl.nh}~\textit{ˀno\mbox{-}wá} (\citealt{JC14a}; \citealt{ND09}: 169; \citealt{AG83}: 161; \citealt{JC18})

\PMlemma{{\wordnl{*paʔ}{that (outside the speaker’s sight and never seen before)}}}

Mk~\textsc{m}~\textit{paʔ}, \textsc{f}~\textit{peʔ}, \textsc{pl}~\textit{pe\mbox{-}khe\mbox{-}weʔ \recind peʔ} \citep[166]{AG94} {\sep} Ni~\textsc{m}~\textit{paʔ}, \textsc{f}~\textit{ɬ\mbox{-}paʔ}, \textsc{pl.h}~\textit{pa\mbox{-}piʔ}, \textsc{pl.nh}~\textit{pa\mbox{-}βaʔ}\gloss{absent at utterance time; firsthand evidence unavailable} (\citealt{AnG15-evid}: 415; \citealt{LC20}: 175) {\sep} PCh~\textsc{m}~\textit{*páʔ \recind *pǻʔ}, \textsc{f}~\textit{*ha\mbox{-}páʔ \recind *ha\mbox{-}pǻʔ \recind *hå\mbox{-}páʔ \recind *hå\mbox{-}pǻʔ}, \textsc{pl.h}~\textit{*pa\mbox{-}púʔ \recind *på\mbox{-}púʔ}, \textsc{pl.nh}~\textit{*po\mbox{-}wáʔ} > Ijw~\textsc{m}~\textit{páʔ \recind p<íʔ>}, \textsc{f}~\textit{ha\mbox{-}páʔ}, \textsc{pl.h}~\textit{pa\mbox{-}póʔ}, \textsc{pl.nh}~\textit{pu\mbox{-}wáʔ}; Mj~\textsc{m}~\textit{pá(ʔ)}, \textsc{f}~\textit{ha\mbox{-}pá}, \textsc{pl.h}~\textit{pa\mbox{-}pʊ́}, \textsc{pl.nh}~\textit{po\mbox{-}wá} (\citealt{JC14a}; \citealt{ND09}: 169; \citealt{JC18}) {\sep} \wordng{PW}{*=pa<h>} > \wordng{LB}{=pa}; \word{’Wk}{=pah}{hearsay evidential} (\citealt{VN14}: 186; \citealt{KC16}: 287)

\PMlemma{{\wordnl{*ts\mbox{-}}{that (within the speaker’s sight)}}}

Mk~\textsc{m}~\textit{tsaʔ}, \textsc{f}~\textit{tseʔ}, \textsc{pl}~\textit{e\mbox{-}tsi\mbox{-}weʔ} \citep[166]{AG94} {\sep} PCh~\textsc{m}~\textit{*séʔ}, \textsc{f}~\textit{*ha\mbox{-}séʔ \recind *hå\mbox{-}séʔ}, \textsc{pl.h}~\textit{*se\mbox{-}púʔ}, \textsc{pl.nh}~\textit{*so\mbox{-}wáʔ} > Ijw~\textsc{m}~\textit{sέʔ \recind sʲéʔ}, \textsc{f}~\textit{ha\mbox{-}sέʔ}, \textsc{pl.h}~\textit{sʲa\mbox{-}póʔ}, \textsc{pl.nh}~\textit{sʲu\mbox{-}wáʔ}; I’w~\textsc{m}~\textit{sʲú\mbox{-}xsʲeʔ}, \textsc{f}~\textit{sʲó\mbox{-}ho\mbox{-}seʔ}, \textsc{pl.h}~\textit{sʲú\mbox{-}xsa\mbox{-}po}, \textsc{pl.nh}~\textit{sʲú\mbox{-}xsu\mbox{-}wa}; Mj~\textsc{m}~\textit{sɪ́ʔ \recind sɪ}, \textsc{f}~\textit{ha\mbox{-}sɪ́ʔ \recind ha\mbox{-}sɪ}, \textsc{pl.h}~\textit{se\mbox{-}pʊ́}, \textsc{pl.nh}~\textit{so\mbox{-}wá} (\citealt{JC14a}; \citealt{ND09}: 169; \citealt{AG83}: 160–161; \citealt{JC18}) {\sep} (?) \wordng{PW}{*=ts<oh>}\gloss{that (moving away); the one just mentioned} > \wordng{LB}{=tsu}; \wordng{’Wk}{\mbox{-}tsoh}; (?)~\wordnl{*=ts<ih>}{this (outside one’s hands’ reach, horizontal)} > \wordng{LB}{=tsi}; \wordng{’Wk}{\mbox{-}tsih \recind \mbox{-}tsåh} (\citealt{VN14}: 180; \citealt{JAA-KC-14}: 446)

\PMlemma{{\wordnl{*\mbox{-}waʔ}{plural (non-human, demonstratives)}}}

\wordng{Mk}{\mbox{-}weʔ} (\citealt{AG94}: 165–166) {\sep} \wordng{Ni}{\mbox{-}βaʔ} (\citealt{AnG15-evid}: 414–415; \citealt{LC20}: 184) {\sep} \wordng{PCh}{*\mbox{-}wáʔ} > \wordng{Ijw}{\mbox{-}wáʔ}; \wordng{I’w}{sʲúhnu\mbox{-}wa}\gloss{these}; \wordng{Mj}{\mbox{-}wáʔ} (\citealt{JC14a}; \citealt{ND09}: 169; \citealt{AG83}: 160; \citealt{JC18})

\dicnote{The absence of a word-final glottal stop in \cits{AG83} attestation of this suffix must be a mistranscription.}%1

\gc{Obviously related to \word{Proto-Guaicuruan}{*\mbox{-}wa}{dual} (\citealt{PVB13b}, \#754; cf. \citealt{PVB13a}: 316).}

\lit{\citealt{PVB13a}: 316 (\intxt{*\mbox{-}wa})}
\end{adjustwidth}
\section{Inflectional prefixes} \label{infpr}

\begin{adjustwidth}{6mm}{0pt}

\PMlemma{{\textit{*ha\mbox{-}} (before C) \textit{/ *h\mbox{-}} (before V) \textit{/} \textit{*k’\mbox{-}} (coalescing with \textit{*ʔ…})\gloss{1.A/S\textsubscript{A} (realis)}}}

\wordng{Mk}{he\mbox{-} / ha\mbox{-} / ho\mbox{-} / h\mbox{-} / k\mbox{-}’…} (\citealt{AG94}: 98; \citealt{CM15}: 132) {\sep} \wordng{Ni}{xa\mbox{-} / x\mbox{-} / k\mbox{-}’…} (\citealt{AF16}: 145; \citealt{JS16}: 143) {\sep} \wordng{PCh}{*ʔa\mbox{-} / *∅\mbox{-}} > \wordng{Ijw}{ʔa\mbox{-} / ∅\mbox{-}}; \wordng{I’w}{a\mbox{-} / a\mbox{-} \recind ∅\mbox{-}}; \wordng{Mj}{ʔa\mbox{-} / ∅\mbox{-}} (\citealt{JC14a}; \citealt{ND09}: 168; \citealt{AG83}: 73; \citealt{JC18}) {\sep} \wordng{PW}{*ʔa\mbox{-}} > \wordng{’Wk}{ʔa\mbox{-}} (“informal sociolect”) \citep[58]{JAA12b}

\gc{\citet[314]{PVB13a} compares this prefix to \wordng{Proto-Guaicuruan}{*tʃV\mbox{-} \recind *tʃ\mbox{-}}\gloss{1.A/S\textsubscript{A}}.}

\lit{\citealt{PVB02}: 144 (\intxt{*χa\mbox{-}}); \citealt{PVB13a}: 314 (\intxt{*ha\mbox{-}}).}

\PMlemma{{\textit{*ji\mbox{-}} (before C) \textit{/ *j\mbox{-}} (before V) \textit{/} \textit{*ˀj\mbox{-}} (coalescing with \textit{*ʔ…})\gloss{1.Poss} (also\gloss{1.A/S\textsubscript{A}.\textsc{irr}}) [1]}}

\wordng{Mk}{ji\mbox{-} / j\mbox{-}} \citep[142]{AG99} {\sep} \wordng{Ni}{ji\mbox{-} / j\mbox{-}} (\citealt{AF16}: 80; \citealt{JS16}: 379) {\sep} \wordng{PCh}{*ʔi\mbox{-} / *j\mbox{-} / *ˀj\mbox{-}} [2] > \wordng{Ijw}{ʔi\mbox{-} / j\mbox{-} / ˀj\mbox{-}}; \wordng{I’w}{i\mbox{-} / j\mbox{-}}; \wordng{Mj}{ʔi\mbox{-} / j\mbox{-} / ˀj\mbox{-}} (\citealt{JC14a}; \citealt{ND09}: 168; \citealt{AG83}: 65; \citealt{JC18}) {\sep} \wordng{PW}{*ʔi\mbox{-} / *ji\mbox{-}} [2] \textit{/ *j\mbox{-}} > \wordng{’Wk}{ʔi\mbox{-} / ja\mbox{-} / j\mbox{-}}\gloss{vocative prefix} (\citealt{JAA-KC-14}: 445)

\dicnote{This affix can also occur before applicatives to express a first-person singular participant in Maká \citep[136]{CM15}, Nivaĉle \citep[194]{AF16}, and Chorote (with a subset of applicatives; cf. \citealt{JC14a}).}%1

\dicnote{The allomorph \sound{PW}{*ji\mbox{-}} > \sound{’Wk}{ja\mbox{-}} is found preceding uvular and glottal consonants.}%2

\gc{Obviously related to \word{Proto-Guaicuruan}{*j\mbox{-} \recind *ej\mbox{-} \recind *ji\mbox{-}}{1.Poss}, \wordnl{*i\mbox{-}}{1.S (stative and middle diathesis)} \citep[314]{PVB13a}.}

\lit{\citealt{RJH15}: 241; \citealt{PVB13a}: 314 (\intxt{*j(i)\mbox{-}}), 315 (\intxt{*jV\mbox{-}}\gloss{1.S\textsubscript{P}})}

\PMlemma{{\textit{*ji\mbox{-}} (before C) \textit{/ *j\mbox{-}} (before V) \textit{/} \textit{*ˀj\mbox{-}} (coalescing with \textit{*ʔ…})\gloss{3.A/S\textsubscript{I} (realis)}}}

\wordng{Mk}{(j)i\mbox{-} / j\mbox{-}} (\citealt{AG94}: 98; \citealt{CM15}: 132) {\sep} \wordng{Ni}{ji\mbox{-} / j\mbox{-}} (\citealt{AF16}: 145; \citealt{JS16}: 375) {\sep} \wordng{PCh}{*ʔi\mbox{-} / *j\mbox{-} / *ˀj\mbox{-}} > \wordng{Ijw}{ʔi\mbox{-} / ja\mbox{-}} [1] \textit{/ j\mbox{-} / ˀj\mbox{-}}; \wordng{I’w}{i\mbox{-} / j\mbox{-}}; \wordng{Mj}{ʔi\mbox{-} / j\mbox{-} / ˀj\mbox{-}} (\citealt{JC14a}; \citealt{ND09}: 168; \citealt{AG83}: 75; \citealt{JC18}) {\sep} \wordng{PW}{*ʔi\mbox{-} / *ji\mbox{-}} [1] \intxt{/ *hi\mbox{-}} [2] \intxt{/ *j\mbox{-} / *ˀj\mbox{-}} [3] > \wordng{LB}{ʔi\mbox{-} / ji\mbox{-} / hi\mbox{-} / j\mbox{-} / ˀj\mbox{-}}; \wordng{’Wk}{ʔi\mbox{-} / ja\mbox{-} / hi\mbox{-} / j\mbox{-} / ˀj\mbox{-}} (\citealt{VN14}: 241–242; \citealt{JAA-KC-14}: 449)

\dicnote{The allomorph \wordng{Ijw}{ja\mbox{-}} is found before Ijw~/k/, LB~/q/ (< \wordng{PM}{*q}). Similarly, the allomorphs \wordng{LB}{ji\mbox{-}} / \wordng{’Wk}{ja\mbox{-}} are found before uvular and glottal consonants. In the Rivadavia variety of Southeastern Wichí, verbs that took \intxt{*ji\mbox{-}} in Proto-Wichí may now take \intxt{ja\mbox{-}} (if the agent acts with low intensity) or \intxt{ʔi\mbox{-}} (if the agent acts with high intensity), according to \citet[135]{JT09-th}.}%1

\dicnote{The allomorph \intxt{hi\mbox{-}} is found before glottalized consonants in Wichí.}%2

\dicnote{As a result of Watkins’ Law, the prefix in question is now found in persons other that the third person in Wichí and is now best analyzed as a verb class marker.}%3

\gc{Obviously related to \word{Proto-Guaicuruan}{*j(i)\mbox{-}}{3.A/S\textsubscript{A}} and \wordnl{*\mbox{-}i}{1\textsc{sg} indirect object} (\citealt{PVB13b}, \#779; see \citealt{PVB13a}: 315).}

\lit{\citealt{PVB13a}: 315 (\intxt{*j\mbox{-} \recind *i\mbox{-}} (person prefix); \textit{*\mbox{-}ji} (with applicatives))}

\PMlemma{{\textit{*ɬ̩\mbox{-}} (before C) \textit{/ *ɬ\mbox{-}} (before V) \textit{/} \textit{*ɬ’\mbox{-}} (coalescing with \textit{*ʔ…})\gloss{3.Poss}}}

\wordng{Mk}{ɬe\mbox{-} / ɬa\mbox{-} / ɬo\mbox{-}} [1] \textit{/ ɬ\mbox{-} / ɬ’\mbox{-}} \citep[147]{AG94} {\sep} \wordng{Ni}{ɬ̩\mbox{-} / ɬ\mbox{-} / t’\mbox{-}} (\citealt{AF16}: 80; \citealt{JS16}: 161) {\sep} \wordng{PCh}{*hᵊ\mbox{-} / *hl\mbox{-} / *t’\mbox{-}} > \wordng{Ijw}{hi\mbox{-} / hl\mbox{-} / t’\mbox{-}}; \wordng{I’w}{hi\mbox{-} / hl\mbox{-} / t’\mbox{-}}; \wordng{Mj}{hi\mbox{-} / hl\mbox{-} / t’\mbox{-}} (\citealt{JC14a}; \citealt{ND09}: 168; \citealt{AG83}: 66; \citealt{JC18}) {\sep} \wordng{PW}{*ɬ̩\mbox{-} / *ɬ\mbox{-} / *t’\mbox{-}} > \wordng{LB}{la\mbox{-} / ɬ\mbox{-} / <t’>\mbox{-}} [2]; \wordng{’Wk}{la\mbox{-} / ɬ\mbox{-} / t’\mbox{-}} (\citealt{VN14}: 163–166; \citealt{JAA-KC-14}: 444–445)

\dicnote{The allomorphs \textit{ɬe\mbox{-} / ɬa\mbox{-} / ɬo\mbox{-}} in Maká are conditioned by vowel harmony.}%1

\dicnote{In Lower Bermejeño, the erstwhile allomorph \textit{t’\mbox{-}} has been reanalyzed as part of the stems.}%2

\gc{\citet[315]{PVB13a} compares this prefix to \wordng{Proto-Guaicuruan}{*(e)ˀl\mbox{-}}\gloss{3.Poss}.}

\lit{\citealt{RJH15}: 241; \citealt{PVB13a}: 315 (\intxt{*ɬ(V)\mbox{-}}); \citealt{AnG15}: 255}

\PMlemma{{\textit{*ɬ̩\mbox{-}} (before C) \textit{/ *ɬ\mbox{-}} (before V) \textit{/} \textit{*ɬ’\mbox{-}} (coalescing with \textit{*ʔ…})\gloss{2.A/S\textsubscript{A} (realis)}}}

\wordng{Mk}{ɬe\mbox{-} / ɬa\mbox{-} / ɬo\mbox{-}} [1] \textit{/ ɬ\mbox{-}} (\citealt{AG94}: 98; \citealt{CM15}: 132) {\sep} \wordng{Ni}{ɬ(a)\mbox{-} / ɬ\mbox{-} / t’\mbox{-}} (\citealt{AF16}: 145; \citealt{JS16}: 161) {\sep} \wordng{PCh}{*hᵊ\mbox{-} / *hl\mbox{-} / *<hᵊ>t’\mbox{-}} [2] > \wordng{Ijw}{hi\mbox{-} / hl\mbox{-} / hitʲ’\mbox{-}}; \wordng{I’w}{hi\mbox{-} / hl\mbox{-} / —}; \wordng{Mj}{hi\mbox{-} / hl\mbox{-} / <hi>t’\mbox{-}} (\citealt{JC14a}; \citealt{ND09}: 168; \citealt{AG83}: 74; \citealt{JC18}) {\sep} \wordng{PW}{*ɬ̩\mbox{-} / *ɬ\mbox{-} / *<ɬ̩>t’\mbox{-}} [2] > \wordng{LB}{la\mbox{-} / ɬ\mbox{-}} [3]; \wordng{’Wk}{la\mbox{-}}´ \textit{/} \textit{ɬ\mbox{-}´ / lat’\mbox{-}}´ [4] (\citealt{VN14}: 241; \citealt{JAA-KC-14}: 449)

\dicnote{The allomorphs \textit{ɬe\mbox{-} / ɬa\mbox{-} / ɬo\mbox{-}} in Maká are conditioned by vowel harmony.}%1

\dicnote{In Chorote and Wichí, one finds reflexes of \intxt{*ɬ̩ɬ’\mbox{-}} instead of \intxt{*ɬ’\mbox{-}} before \intxt{ʔ\mbox{-}}initial stems, possibly as a result of analogical extension (see \citealt{JC14a}).}%2

\dicnote{In Lower Bermejeño, erstwhile \intxt{ʔ\mbox{-}}initial roots of transitive verbs extended the occurrence of a \intxt{j’\mbox{-}}initial allomorph (originally restricted to the third person) to the entire realis paradigm (Watkins’ Law), and forms such as \word{PW}{*ɬ̩t\mbox{-}’áχ}{you beat} were replaced by the non-etymological \wordng{LB}{la\mbox{-}ˀj\mbox{-}aχ} \citep[241]{VN14}, as opposed to \wordng{’Wk}{lat\mbox{-}’áx} \citep[116]{KC16}.}%3

\dicnote{In ’Weenhayek, this prefix is unique in triggering vowel lengthening in the subsequent syllable.}%3

\lit{\citealt{EN84}: 9, 15, 53 (\intxt{*hl\mbox{-}})}

\PMlemma{{\textit{*n̩\mbox{-}} (before C) \textit{/ *n\mbox{-}} (before V) \textit{/} \textit{*ˀn\mbox{-}} (coalescing with \textit{*ʔ…})\gloss{2.S\textsubscript{P}/P (realis)}}}

\wordng{Mk}{<ɬe>n\mbox{-} / <ɬa>n\mbox{-} / <ɬo>n\mbox{-}} [1] (\citealt{AG94}: 89; \citealt{CM15}: 132) {\sep} \wordng{Ni}{na\mbox{-} / n\mbox{-}} (\citealt{AF16}: 141–142, 148; \citealt{JS16}: 177) {\sep} \wordng{PCh}{*n̩\mbox{-} / *n\mbox{-} / *ˀn\mbox{-}} > \wordng{Ijw}{ʔin\mbox{-} / <ʔi(n)>n\mbox{-} / (<ʔi>)ˀn\mbox{-}} [2]; \wordng{I’w}{in\mbox{-} / n\mbox{-} / —}; \wordng{Mj}{ʔin\mbox{-} / <ʔi>n\mbox{-} / (<ʔi>)ˀn\mbox{-}} [2] (\citealt{JC14a}; \citealt{ND09}: 167, 169; \citealt{AG83}: 77–78; \citealt{JC18})

\dicnote{The element \intxt{ɬe\mbox{-} / ɬa\mbox{-} / ɬo\mbox{-}} in Maká (with allomorphy conditioned by vowel harmony) is likely etymologically related to the 2.A/S\textsubscript{A} prefix.}%1

\dicnote{In Iyojwa’aja’ and Manjui, one finds both \textit{ˀn\mbox{-}} and \textit{ʔiˀn\mbox{-}} before \textit{ʔ\mbox{-}}initial stems, and \textit{ʔin\mbox{-}} before vowel-initial stems. The choice most likely depends on the position of the stress (\intxt{ˀn\mbox{-}} is found in roots where the stress falls on the second syllable, and \textit{ʔiˀn\mbox{-}} is predominant in roots with initial stress), though there is some variation (and in Iyojwa’aja’ this variation is apparently of subdialectal nature). Iyo’awujwa’ preserves the more archaic pattern here.}%2

\PMlemma{{\textit{*n̩\mbox{-}} (before C) \textit{/ *n\mbox{-}} (before V) \textit{/} \textit{*ˀn\mbox{-}} (coalescing with \textit{*ʔ…})\gloss{indefinite possessor}}}

\wordng{Mk}{n\mbox{-}} (\citealt{AG94}: 147, fn. 41) {\sep} \wordng{Ni}{na\mbox{-} / n\mbox{-}} \citep[83]{AF16} {\sep} \wordng{PCh}{*n̩\mbox{-} / *n\mbox{-} / *ˀn\mbox{-}} > \wordng{Ijw}{ʔin\mbox{-} / <ʔi>n\mbox{-} / ˀn\mbox{-}} [1]; \wordng{I’w}{in\mbox{-} \recind n̩\mbox{-} / — / n\mbox{-}} [2]; \wordng{Mj}{ʔin\mbox{-} / <ʔi>n\mbox{-} / ˀn\mbox{-}} [1] (\citealt{JC14b}: 77, \citeyear{JC14a}; \citealt{ND09}: 168; \citealt{AG83}: 69; \citealt{JC18})

\dicnote{In Iyojwa’aja’ and Manjui, one finds \textit{ʔin\mbox{-}} before vowel\textit{\mbox{-}}initial stems. No relevant data on Iyo’awujwa’ have been attested for this specific environment.}%1

\dicnote{With stems that are known to start with a glottal stop, the prefix in question is attested as \intxt{n\mbox{-}} in \citet[69]{AG83}, which must be a mistranscription for \intxt{ˀn\mbox{-}}.}%2

\gc{Obviously related to \word{Proto-Guaicuruan}{*en\mbox{-} \recind *n\mbox{-}}{indefinite possessor} (\citealt{PVB13b}, \#735).}

\PMlemma{{\textit{*n̩\mbox{-}} (before C) \textit{/ *n\mbox{-}} (before V) \textit{/} \textit{*ˀn\mbox{-}} (coalescing with \textit{*ʔ…})\gloss{3.A/S.\textsc{irr}}}}

\wordng{Mk}{ne\mbox{-} / na\mbox{-} / no\mbox{-}} [1] \textit{/ n\mbox{-}} (\citealt{AG94}: 85–98) {\sep} \wordng{Ni}{na\mbox{-} / n\mbox{-}} \citep[145]{AF16} {\sep} \wordng{PCh}{*n̩\mbox{-} / *n\mbox{-} / *ˀn\mbox{-}} > \wordng{Ijw}{ʔin\mbox{-} / <ʔi(n)>n\mbox{-} / <ʔi>ˀn\mbox{-} \recind ˀn\mbox{-}} [2]; \wordng{I’w}{(e)n\mbox{-} / <i>n\mbox{-} / —}; \wordng{Mj}{ʔin\mbox{-} / <ʔi>n\mbox{-} / —} (\citealt{JC14b}: 89, \citeyear{JC14a}; \citealt{ND09}: 168; \citealt{AG83}: 75–76; \citealt{JC18}) {\sep} \wordng{PW}{*ní\mbox{-}…\mbox{-}aʔ / *n\mbox{-}´…\mbox{-}aʔ / *ˀn\mbox{-}´…\mbox{-}aʔ} > \wordng{LB}{ni\mbox{-}…\mbox{-}a} / — / —; \wordng{’Wk}{ní\mbox{-}…\mbox{-}aʔ / n\mbox{-}´…\mbox{-}aʔ / ˀn\mbox{-}´…\mbox{-}aʔ} (\citealt{VN14}: 316; \citealt{JAA-KC-14}: 458, fn. 36)

\dicnote{The allomorphs \textit{ne\mbox{-} / na\mbox{-} / no\mbox{-}} in Maká are conditioned by vowel harmony.}%1

\dicnote{In Iyojwa’aja’, the third-person irrealis prefix usually coalesces with the stem-initial glottal stop as \intxt{ʔiˀn\mbox{-}}, but in some verbs \intxt{ˀn\mbox{-}} is found instead: \wordnl{ka ˀnaháne}{so that s/he knows}. The sequence \intxt{ʔi\mbox{-}} is also often omitted after particles that end in a low vowel.}%2

\PMlemma{{\intxt{*ni\mbox{-} / *n\mbox{-}} (next to a vowel)\gloss{cislocative}}}

\wordng{Mk}{ni\mbox{-} / \mbox{-}n\mbox{-}} \citep[94]{AG94} {\sep} \wordng{Ni}{ni\mbox{-} / n\mbox{-}} (\citealt{AF16}: 191–192) {\sep} \wordng{PCh}{*n\mbox{-}} in \wordnl{*<n>ǻm}{to come here} (cf. \wordnl{[j]ǻm}{to go away.3\textsc{irr}}) > \wordng{Ijw}{náˀm}; \wordng{Mj}{nám} (\citealt{JC14a,JC14b,JC18}) {\sep} \wordng{PW}{*n\mbox{-}} in \wordnl{*<n>ǻm}{to come here} > \wordng{LB}{nom}; \wordng{Vej}{nåm}; \wordng{’Wk}{nǻm̥} (\citealt{VN14}: 145; \citealt{JB09}: 53; \citealt{VU74}: 68; \citealt{KC16}: 252)


\gc{\citet[317]{PVB13a} compares this to \wordng{Proto-Guaicuruan}{*n\mbox{-}}\gloss{middle diathesis} (\citealt{PVB13b}, \#774).}

\lit{\citealt{PVB13a}: 317 (\intxt{*n\mbox{-}})\gloss{cislocative, middle voice}}

\PMlemma{{\intxt{*ni\mbox{-} / *n\mbox{-}} (next to a vowel)\gloss{middle voice}}}

\wordng{Ni}{n\mbox{-}} [1] \citep[192]{AF16} {\sep} (?) \wordng{PCh}{*\mbox{-}n…\mbox{-}} [2] > \wordng{Ijw}{\mbox{-}ní\mbox{-}}\gloss{reflexive} \citep{JC14a} {\sep} \wordng{PW}{*ni\mbox{-} / *n\mbox{-}} > \wordng{LB}{ni\mbox{-} / —}; \wordng{’Wk}{ni\mbox{-} / n\mbox{-}} (\citealt{JT09-th}: 192–194; \citealt{JAA-KC-14}: 449)

\dicnote{\citet[297]{LC20} state that this prefix only occurs before vowel-initial stems. \citet{AF16} considers it to be a metaphorical extension of the cislocative prefix.}%1

\dicnote{We can think of no convincing way of accounting for an instance of [i] in a stressed syllable after a non-palatalized consonant in Iyojwa’aja’, which in addition fails to trigger palatalization of following segments (even coronal ones). We have considered the possibility of positing a stressed syllabic \textit{*n̩} for Proto-Chorote, but this is problematic because the reflexive prefix surfaces as \textit{\mbox{-}ní\mbox{-}} even after vowels in Iyojwa’aja’.}%2

\largerpage
\gc{\citet[317]{PVB13a} compares this to \wordng{Proto-Guaicuruan}{*n\mbox{-}}\gloss{middle diathesis} (\citealt{PVB13b}, \#774).}

\lit{\citealt{PVB13a}: 317 (\intxt{*n\mbox{-}})\gloss{cislocative, middle voice}}\clearpage

\PMlemma{{\intxt{*ni\mbox{-} / *n\mbox{-}} (next to a vowel)\gloss{3.S\textsubscript{N} (realis)} [1]}}

\wordng{Mk}{ni\mbox{-} / \mbox{-}n\mbox{-}} \citep[89]{AG94} {\sep} \wordng{Ni}{ni\mbox{-} / n\mbox{-}} \citep[142]{AF16} {\sep} \wordng{PCh}{*n̩\mbox{-} / *n\mbox{-} / *ˀn\mbox{-}} > \wordng{Ijw}{ʔin\mbox{-} / n\mbox{-} / ˀn\mbox{-}}; \wordng{I’w}{in\mbox{-} / — / —}; \wordng{Mj}{ʔin\mbox{-} / — / ˀn\mbox{-}} (\citealt{JC14a}; \citealt{AG83}: 79; \citealt{JC18}) {\sep} see \word{PW}{*ni\mbox{-} / *n\mbox{-}}{middle voice}

\dicnote{This is probably the same prefix as\gloss{cislocative} and/or\gloss{middle voice}, which has become obligatory with some verbs and is no longer analyzable as a direction or voice marker.}%1

\PMlemma{{\intxt{*qats=} (before C) \intxt{/ *qats=} (before V) \intxt{/ *qats’=} (coalescing with \intxt{*ʔ…})\gloss{1\textsc{pl}.S\textsubscript{P}/P} or\gloss{1\textsc{pl}.Poss}}}

\word{Ni}{kas\mbox{-} \recind katsi\mbox{-} / kats\mbox{-} / kats’\mbox{-}}{1\textsc{pl}.Poss} \citep[82]{AF16} {\sep} \wordng{PCh}{*qas=sᵊ\mbox{-} / *qas=s\mbox{-} / *qas=ts’\mbox{-}}\gloss{1\textsc{pl}.S\textsubscript{P}/P} > \wordng{Ijw}{kas=∅\mbox{-} / kas=…\mbox{-}s\mbox{-} / kas=…\mbox{-}ts’\mbox{-}}; \wordng{I’w}{kasi\mbox{-} / kas\mbox{-} / kats\mbox{-}}; \wordng{Mj}{ka\mbox{-}ʃi\mbox{-} / ka\mbox{-}si\mbox{-} / ka\mbox{-}se\mbox{-}} [1] \intxt{/ kas\mbox{-}s\mbox{-} / kas\mbox{-}ts’\mbox{-}} (\citealt{JC14b}: 89, \citeyear{JC14a}; \citealt{ND09}: 167, 169; \citealt{AG83}: 79–80; \citealt{JC18})

\dicnote{The allomorph \intxt{kasi\mbox{-} \recind kase\mbox{-}} appears in Manjui before a non-palatalized \intxt{k} < \sound{PCh}{*q.}}%1

\gc{Obviously related to \word{Proto-Guaicuruan}{*qoˀd\mbox{-} / *qo\mbox{-}}{1\textsc{pl}.Poss}, \wordnl{*qod\mbox{-} / *qo\mbox{-}}{1\textsc{pl}.S\textsubscript{P}/P} (\citealt{PVB13b}, \#732, \#764).}

\lit{\citealt{PVB13a}: 315 (\wordnl{*kats’\mbox{-}}{1+2.Poss}, \wordnl{*kats\mbox{-}}{1+2.S\textsubscript{P}})}

\PMlemma{{\intxt{*t̩\mbox{-}} (before C) \intxt{/ *t\mbox{-}} (before V) \intxt{/ *t’\mbox{-}} (coalescing with \intxt{*ʔ…})\gloss{3.S\textsubscript{T}}}}

\wordng{Mk}{te\mbox{-} / ta\mbox{-} / to\mbox{-}} [1] \textit{/ t\mbox{-} / t’\mbox{-}} \citep[85]{AG94} {\sep} \wordng{Ni}{t(a)\mbox{-} / t\mbox{-} / t’\mbox{-}} [2] \citep[135]{AF16} {\sep} \wordng{PCh}{*tᵊ\mbox{-} / *t\mbox{-} / *t’\mbox{-}} > \wordng{Ijw}{ti\mbox{-} / ta\mbox{-}} [3] \textit{/ t\mbox{-} / t’\mbox{-}}; \wordng{I’w}{ti\mbox{-} \recind te\mbox{-} / t\mbox{-}}; \wordng{Mj}{ti\mbox{-} / t\mbox{-} / t’\mbox{-}} (\citealt{JC14b}: 86–86, 91, 98, \citeyear{JC14a}; \citealt{AG83}: 75; \citealt{JC18}) {\sep} \wordng{PW}{*ta\mbox{-} / \mbox{-}t(á)\mbox{-}} [4] > \wordng{LB}{ta\mbox{-} / \mbox{-}t(a)\mbox{-}}; \wordng{Vej}{ta\mbox{-} / \mbox{-}t(a)\mbox{-}}; \wordng{’Wk}{ta\mbox{-} / \mbox{-}t(á)\mbox{-}} (\citealt{VN14}: 120–121, 237–240; \citealt{MG-MELO15}: 14; \citealt{JAA-KC-14}: 448)

\dicnote{The allomorphs \intxt{te\mbox{-} / ta\mbox{-} / to\mbox{-}} in Maká are conditioned by vowel harmony.}%1

\dicnote{In Nivaĉle, the morpheme in question is also found in the second-person form (between the person prefix and the root) and is now best analyzed as a verb class marker, though it is absent from the first-person form. The allomorph \intxt{ta\mbox{-}} in Nivaĉle is only found before \intxt{tʃ\mbox{-}}initial stems.}%2

\dicnote{The allomorph \intxt{ta\mbox{-}} appears in Iyojwa’aja’ before /k/ < \wordng{PM}{*q}.}%3

\dicnote{As a result of Watkins’ Law, the prefix in question is now found in persons other that the third person in Wichí and is now best analyzed as a verb class marker.}%4

\PMlemma{{\intxt{*tsi\mbox{-}} (before C) \intxt{/ *ts\mbox{-}} (before V) \intxt{/ *ts’\mbox{-}} (coalescing with \intxt{*ʔ…})\gloss{1.S\textsubscript{P}/P (realis)}}}

\wordng{Mk}{ts(’)i\mbox{-} / ts(’)\mbox{-}} (\citealt{AG94}: 89; \citealt{CM15}: 132) {\sep} \wordng{Ni}{tsi\mbox{-} / ts\mbox{-} / ts’\mbox{-}} (\citealt{AF16}: 141–142, 148; \citealt{JS16}: 300) {\sep} \wordng{PCh}{*sᵊ\mbox{-} / *s\mbox{-} / *ts’\mbox{-}} > \wordng{Ijw}{si\mbox{-} / s\mbox{-} / ts’\mbox{-}}; \wordng{I’w}{si\mbox{-} \recind tsi\mbox{-} / s\mbox{-} / ts\mbox{-}}; \wordng{Mj}{ʃi\mbox{-} / si\mbox{-} \recind se\mbox{-}} [1] \textit{/ s\mbox{-} / s’\mbox{-}} (\citealt{JC14b}: 79, fn. 7, \citeyear{JC14a}; \citealt{ND09}: 167, 169; \citealt{AG83}: 76–77; \citealt{JC18})

\dicnote{The allomorph \textit{si\mbox{-} \recind se\mbox{-}} appears in Manjui before a non-palatalized \textit{k}.}%1

\gc{\citet[315]{PVB13a} compares this prefix to \wordng{Proto-Guaicuruan}{*i\mbox{-}d\mbox{-}}\gloss{1.S\textsubscript{P}/P} (\citealt{PVB13b}, \#763).}

\lit{\citealt{PVB13a}: 315 (\intxt{*ts(’)i\mbox{-}})}

\PMlemma{{\intxt{*wa\mbox{-}} (before C) \intxt{/ *w\mbox{-}} (before V)\gloss{3.S\textsubscript{WA}}}}

\wordng{Mk}{we\mbox{-}} \citep[85]{TT15} {\sep} \wordng{Ni}{βa\mbox{-} / β\mbox{-}} \citep[236–238]{LC20}

\PMlemma{{\wordnl{*xi\mbox{-}}{1+2 (realis)}}}

\word{Mk}{xi\mbox{-} / x\mbox{-}}{1+2.A/S\textsubscript{A}/P (realis)}; \wordnl{xi\mbox{-}n(i)\mbox{-} / xi\mbox{-}j(i)\mbox{-}}{1+2.S\textsubscript{P} (realis)} (\citealt{AG94}: 86–91, 100–102) {\sep} \word{Ni}{ʃi<n(a)>\mbox{-} / ʃi<ˀn>\mbox{-}}{1+2.P/S\textsubscript{P} (realis)} \citep[148]{AF16}

\lit{\citealt{PVB02}: 142 (\wordnl{*xina\mbox{-}}{1+2})}

\PMlemma{{\textit{*xt̩\mbox{-}} (before C) \textit{/ *xt\mbox{-}} (before V) \textit{/ *xt’\mbox{-}} (coalescing with \textit{*ʔ…})\gloss{1+2.A/S\textsubscript{A} (realis)}}}

\wordng{Mk}{xite\mbox{-} / xita\mbox{-} / xito\mbox{-} / xit\mbox{-} / xit’\mbox{-}} (\citealt{AG94}: 85–86, 93, 96) {\sep} \wordng{Ni}{ʃta\mbox{-} / ʃt\mbox{-} / ʃt’\mbox{-}} (ShL~\textit{sta\mbox{-} / st\mbox{-} / st’\mbox{-}}) \citep[145]{AF16}

\dicnote{The allomorphs \textit{xite\mbox{-} / xita\mbox{-} / xito\mbox{-}} in Maká are conditioned by vowel harmony.}%1

\lit{\citealt{PVB02}: 142 (\wordnl{*xita\mbox{-}}{1+2.S}})

\PMlemma{{\textit{*ʔa\mbox{-}} (before C) \textit{/ *\textmd{∅}\mbox{-}} (before V or \intxt{*ʔ})\gloss{2.Poss} (also\gloss{2.A/S\textsubscript{A}.\textsc{irr}}) [1]}}

\wordng{Mk}{e\mbox{-} / a\mbox{-} / o\mbox{-} / ∅\mbox{-}} \citep[147]{AG94} {\sep} \wordng{Ni}{ʔa\mbox{-} / ∅\mbox{-}} (\citealt{AF16}: 80; \citealt{JS16}: 35) {\sep} \wordng{PCh}{*ʔa\mbox{-} / *∅\mbox{-}} > \wordng{Ijw}{ʔa\mbox{-} / ∅\mbox{-}}; \wordng{I’w}{a\mbox{-} / ∅\mbox{-}}; \wordng{Mj}{ʔa\mbox{-} / ∅\mbox{-}} (\citealt{JC14b}: 85, 100, \citeyear{JC14a}; \citealt{ND09}: 168; \citealt{AG83}: 65–66; \citealt{JC18}) {\sep} \wordng{PW}{*a\mbox{-} / *ha\mbox{-}} [2] \textit{/ *∅\mbox{-}} > \wordng{LB/’Wk}{ʔa\mbox{-} / ha\mbox{-} / ∅\mbox{-}} (\citealt{VN14}: 163–166; \citealt{JAA-KC-14}: 444–445)

\dicnote{This affix can also occur before applicatives to express a second-person participant in Maká \citep[136]{CM15}, Nivaĉle \citep[194]{AF16}, Chorote \citep{JC14a}, and Wichí (variants \textit{\mbox{-}ʔam\mbox{-}} and \textit{\mbox{-}ʔa\mbox{-}}) (\citealt{VN14}: 223; \citealt{JAA-KC-14}: 433, 449).}%1

\dicnote{The allomorph \intxt{ha\mbox{-}} is found before glottalized consonants in Wichí.}%2

\gc{\citet[315]{PVB13a} compares this prefix to \word{Proto-Guaicuruan}{*ʔa\mbox{-}}{2.A/S\textsubscript{A}}.}

\lit{\citealt{RJH15}: 241; \citealt{EN84}: 9, 17, 18 (\intxt{*a\mbox{-}}); \citealt{PVB13a}: 315 (\intxt{*ʔa\mbox{-} \recind *∅\mbox{-}}\gloss{2.\textsc{irr}})}

\PMlemma{{\wordnl{*ʔin\mbox{-}}{1+2.S\textsubscript{P}/P} or\gloss{1+2.Poss}}}

\word{Mk}{in\mbox{-}}{1+2.Poss} \citep[137]{CM15} {\sep} \word{PW}{*ˣn<á>\mbox{-}}{1+2.S\textsubscript{P}/P}, \wordnl{*ˣn\mbox{-}ám\mbox{-}elʰ}{we (inclusive)} > \word{LB}{n\mbox{-}am\mbox{-}iɬ}{we (hortative)} [1]; \wordng{Vej}{ˀn\mbox{-}am\mbox{-}el} [2]; \word{’Wk}{ʔin<á>\mbox{-}}{1+2.S\textsubscript{P}/P; hortative}, \wordnl{ʔin\mbox{-}ám\mbox{-}eɬ}{we (inclusive)} (\citealt{VN14}: 120–121, 237–240; \citealt{MG-MELO15}: 26; \citealt{JAA-KC-14}: 437, 445, 447)

\dicnote{Southeastern Wichí has irregularly raised the vowel of the plural suffix. Lower Bermejeño Wichí does not preserve the pronoun in question in non-hortative usages, having replaced \intxt{*ˣn\mbox{-}ám\mbox{-}elʰ} with \intxt{to\mbox{-}ɬam\mbox{-}iɬ}; the Rivadavia subdialect shows a more conservative picture, where \intxt{n\mbox{-}am\mbox{-}iɬ} varies with \intxt{tɔ\mbox{-}ɬam\mbox{-}iɬ} \citep[100, 116]{JT09-th}.}%1

\dicnote{The Vejoz reflex is attested with a plain nasal, that is, as \intxt{n\mbox{-}am\mbox{-}el} in \citet[67]{VU74}, which must be a mistranscription.}%2
\end{adjustwidth}
\section{Inflectional suffixes} \label{infsu}

\begin{adjustwidth}{6mm}{0pt}
\PMlemma{{\wordnl{*\mbox{-}a}{punctual, momentary}}}

\wordng{Ni}{\mbox{-}a} (\citealt{AF16}: 159–161) {\sep} \wordng{PCh}{*\mbox{-}aʔ} > \wordng{Ijw/I’w/Mj}{\mbox{-}aʔ} (\citealt{JC14a}; own field notes; \citealt{JC18})

\PMlemma{{\wordnl{*\mbox{-}ájʰ / *\mbox{-}jʰ}{\textsc{pl}}}} → see examples in the main corpus (\sectref{bonafide})

\PMlemma{{\wordnl{*\mbox{-}él / *\mbox{-}l}{\textsc{pl}}}} → see examples in the main corpus (\sectref{bonafide})

\gc{\citet[316]{PVB13a} compares this suffix to \word{Proto-Guaicuruan}{*\mbox{-}ʔaˀl}{distributive plural} (\citealt{PVB13b}, \#749).}

\lit{\citealt{PVB13a}: 316 (\intxt{*\mbox{-}(V)l})}

\PMlemma{{\wordnl{*\mbox{-}eˀɬ}{pronominal plural}}}

\word{Mk}{j\mbox{-}e\mbox{-}khewe\mbox{-}l\mbox{-}iˀl}{we (exclusive)}, \wordnl{∅\mbox{-}e\mbox{-}khewe\mbox{-}l\mbox{-}iˀl}{you (plural)} [1] (\citealt{AG99}: 143, 398) {\sep} \wordng{Ni}{\mbox{-}eˀɬ / \mbox{-}eɬ} (\citealt{LC20}: 69, 149–150, 262–263) {\sep} \wordng{PCh}{*\mbox{-}eɬ} [2] > \wordng{I’w}{\mbox{-}el / \mbox{-}Vl / \mbox{-}<w>el} [3]\gloss{2\textsc{pl}}; \wordng{Mj}{\mbox{-}eɬ / \mbox{-}iɬ / \mbox{-}Vɬ / \mbox{-}<w>eɬ} [2] (\citealt{AG83}: 105; \citealt{JC18}) {\sep} \word{PW}{*ˣn\mbox{-}ám\mbox{-}elʰ}{we (inclusive)}; \wordnl{*ʔõ\mbox{-}ɬ\mbox{-}ám\mbox{-}elʰ / *j\mbox{-}ám\mbox{-}elʰ}{we (exclusive)}; \wordnl{*∅\mbox{-}ʔám\mbox{-}elʰ}{you (plural)}; \wordnl{*ɬ\mbox{-}ám\mbox{-}elʰ}{they} [4] > \word{LB}{n\mbox{-}am\mbox{-}iɬ}{we (hortative)} (\wordnl{to\mbox{-}ɬ\mbox{-}am\mbox{-}iɬ}{we (exclusive)}); \intxt{n̩\mbox{-}ɬ\mbox{-}am\mbox{-}iɬ}; \intxt{∅\mbox{-}am\mbox{-}iɬ}; \intxt{ɬ\mbox{-}am\mbox{-}iɬ} [5]; \wordng{Vej}{ˀn\mbox{-}am\mbox{-}el} [6]; \intxt{ʔo\mbox{-}ɬ\mbox{-}am\mbox{-}el}; \intxt{∅\mbox{-}ʔam{-}el}; \intxt{ɬ\mbox{-}am\mbox{-}el}; \wordng{’Wk}{ʔin\mbox{-}ám\mbox{-}eɬ}; \intxt{ʔõ\mbox{-}ɬ\mbox{-}ám\mbox{-}eɬ} (“formal sociolect”) \intxt{/ j\mbox{-}ám\mbox{-}eɬ} (“informal sociolect”); \intxt{∅\mbox{-}ʔám\mbox{-}eɬ}; \intxt{ɬ\mbox{-}ám\mbox{-}eɬ} (\citealt{VN14}: 335; \citealt{VU74}: 50, 65, 67, 69; \citealt{MG-MELO15}: 26; \citealt{JAA-KC-14}: 437)

\dicnote{The preglottalized coda in Maká is attested in the New Testament (e.g. John 7:34, 2 Corinthians 13:6).}%1

\dicnote{In Chorote, the suffix in question expresses extended plural of possessors and clause participants, except in the third person.}%2

\dicnote{The allomorph \intxt{\mbox{-}Vɬ} (\intxt{\mbox{-}Vl}) in Chorote results from translaryngeal harmony. The allomorph \intxt{\mbox{-}weɬ} (\intxt{\mbox{-}wel}) occurs after vowels. The allomorph \intxt{\mbox{-}iɬ} in Manjui occurs after \intxt{k} and \intxt{j}.}%3

\dicnote{Wichí irregularly reflects \sound{PM}{*ɬ} as \intxt{*lʰ} (this innovation may in fact be restricted to Vejoz and Guisnay, given that ’Weenhayek and Southeastern Wichí reflect \sound{PW}{*lʰ} and \intxt{*ɬ} as \intxt{ɬ} anyway).}%4

\dicnote{Southeastern Wichí has irregularly raised the vowel of the suffix.}%5

\dicnote{The Vejoz reflex of the first-person inclusive pronoun is attested with a plain nasal, that is, as \intxt{n\mbox{-}am\mbox{-}el} in \citet[67]{VU74}, which must be a mistranscription.}%6

\PMlemma{{\wordnl{*\mbox{-}íts / *\mbox{-}ts}{\textsc{pl}}}} → see examples in the main corpus (\sectref{bonafide})

\gc{\citet[316]{PVB13a} compares this suffix to \word{Proto-Guaicuruan}{*\mbox{-}Vdi / *\mbox{-}di}{\textsc{pl}} (\citealt{PVB13b}, \#745).}

\lit{\citealt{PVB13a}: 316 (\intxt{*\mbox{-}(V)ts})}

\PMlemma{{\wordnl{*\mbox{-}xäˀn(eʔ)}{downwards; verbal plural}}}

\wordng{Ni}{\mbox{-}ʃaˀneʔ / \mbox{-}xaˀneʔ} (after \textit{V\textsubscript{[+back]}}\textit{(C\textsubscript{[−coronal]}})) (\citealt{AF16}: 173–174, 208–210) {\sep} \wordng{PCh}{*\mbox{-}heˀn(eʔ)} > \wordng{Ijw}{\mbox{-}heˀn}; \wordng{I’w}{\mbox{-}hen}, \textit{\mbox{-}ˀneʔ}; \wordng{Mj}{\mbox{-}heˀneʔ} (\citealt{JC14b}: 78, \citeyear{JC14a}; \citeyear{JC18}) {\sep} \wordng{PW}{*\mbox{-}heˀn} > \wordng{LB}{\mbox{-}hen}; \wordng{’Wk}{\mbox{-}heˀn} (\citealt{VN14}: 228–232; \citealt{KC16}: 148; \citealt{JAA-KC-14}: 449)

\lit{\citealt{EN84}: 42 (\intxt{*\mbox{-}hnɛ}); \citealt{PVB02}: 142 (\intxt{*\mbox{-}xe(ne)})}

\PMlemma{{\wordnl{*\mbox{-}ʔeʔ}{\textsc{loc}}}}

\wordng{Mk}{\mbox{-}ʔiʔ} [1] (\citealt{AG94}: 123–124) {\sep} \wordng{Ni}{\mbox{-}ʔeʔ}\gloss{proximal locative} (\citealt{AF16}: 157–159) {\sep} \wordng{PCh}{*\mbox{-}ʔeʔ} > \wordng{Ijw/I’w/Mj}{\mbox{-}ʔeʔ}\gloss{punctual locative} (\citealt{JC14a}; own field data; \citealt{JC18}) {\sep} (?)~\wordng{PW}{*\mbox{-}e} [2] > \wordng{LB}{\mbox{-}e}\gloss{distal locative}; \wordng{’Wk}{\mbox{-}eʔ} (\citealt{VN14}: 255; \citealt{JAA-KC-14}: 460)

\dicnote{This applicative is actually represented as \intxt{\mbox{-}i} in \citet{AG94}. We assume this is a mistranscription for \intxt{\mbox{-}ʔiʔ}, as in the Wycliffe Bible translations one finds forms such as \intxt{iˀniʔ} (from \wordnl{in + \mbox{-}ʔiʔ}{s/he, it is in}).}%1

\dicnote{We are unsure whether the Wichí applicative \intxt{*\mbox{-}e} is related to \wordng{PM}{*\mbox{-}ʔeʔ}.}%2
\end{adjustwidth}
\section{MN only} \label{mnonly}

\begin{adjustwidth}{6mm}{0pt}
In this section, we list the cognate sets with reflexes only in Maká and Nivaĉle. Due to the absence of the diagnostic reflexes in Chorote and ’Weenhayek, it is often impossible to reconstruct the prosodic properties of the etyma listed in this section. For this reason, the reconstructions in this section are strictly segmental (for example, \wordng{PM}{*sålål} should be read as \wordng{PM}{*sålål \recind *sǻlål \recind *sålǻl}), unless specified otherwise.

\PMlemma{{\wordnl{*\mbox{-}aˀɬ \recind *\mbox{-}äˀɬ}{to burn} (MN)}}

\wordng{Mk}{[n]eˀɬ\mbox{-}xuʔ} [1] \citep[151]{AG99} {\sep} \word{Ni}{[ji]<n>aˀɬ}{to burn}; \wordnl{t\mbox{-}aɬ\mbox{-}xen}{to burn a field}; \wordnl{\mbox{-}aɬ\mbox{-}etʃ\plf{\mbox{-}aɬ\mbox{-}xe\mbox{-}s}}{burnt field} (\citealt{JS16}: 42, 177, 250)

\dicnote{The preglottalized coda in the Maká reflex is attested in the New Testament (e.g. Luke 1:10).}%1

\gc{\citet[304]{PVB13a} compares this root to \word{Proto-Guaicuruan}{*\mbox{-}a(ˀ)leg}{to burn} (\citealt{PVB13b}, \#28).}

\lit{\citealt{PVB13a}: 304 (\intxt{*\mbox{-}aɬ})}

\PMlemma{{\textit{*\mbox{-}ata(ˀ)x \recind *\mbox{-}ä\mbox{-}} [1]\gloss{food} (MN)}}

\wordng{Mk}{\mbox{-}ete(ˀ)x} [1], \textit{\mbox{-}etex\mbox{-}its} \citep[159]{AG99} {\sep} \wordng{Ni}{\mbox{-}ataʃ}, \textit{\mbox{-}ata\mbox{-}k} \citep[50]{JS16}

\dicnote{The uncertainty regarding the coda is due to the fact that the form is not attested in our sources on Maká that distinguish between plain and preglottalized codas. In PM, the reconstruction of a preglottalized coda is possible only if the root has initial accent (in this case the deglottalization in Nivaĉle would be regular).}%1

\PMlemma{{\wordnl{*ʔåɸínaˀχ\plf{*ʔåɸínha\mbox{-}ts}}{black howler} (MN)}}

\wordng{Mk}{afinaˀχ\plf{afinhe\mbox{-}ts}} (\citealt{AG99}: 113; \citealt{PMA}: 2) {\sep} \wordng{Ni}{ʔåɸinax\plf{ʔåɸinxa\mbox{-}s}} \citep[210]{JS16}

\PMlemma{{\textit{*[j]åɸti(ˀ)ɬ} [1]\gloss{to spin a thread} [2] (MN)}}

\wordng{Mk}{[j]afti(ˀ)ɬ} [1] \citep[113]{AG99} {\sep} \wordng{Ni}{[j]åɸtiɬ} \citep[107]{JS16}

\dicnote{The uncertainty regarding the coda is due to the fact that the form is not attested in our sources on Maká that distinguish between plain and preglottalized codas. In PM, the reconstruction of a preglottalized coda is possible only if the root has initial accent (in this case the deglottalization in Nivaĉle would be regular).}%1

\dicnote{This verb is likely derived from \wordng{PM}{*tiˀɬ}\gloss{to sew}.}%2

\PMlemma{{\textit{*[j]åtsi(ˀ)j} [1]\gloss{to spill} (MN)}}

\wordng{Mk}{[j]atsij\mbox{-}xuʔ} \citep[134]{AG99} {\sep} \wordng{Ni}{[j]åtsij} (\citealt{LC20}: 236; \citealt{JS16}: 154)

\dicnote{In PM, the reconstruction of a preglottalized coda is possible only if the root has initial accent (in this case the deglottalization in Nivaĉle would be regular).}%1

\gc{Possibly related to \word{Proto-Guaicuruan}{*\mbox{-}ʔotsi(\mbox{-}t’\mbox{-}iˀni)}{to fall} (\citealt{PVB13b}, \#699; cf. \citealt{PVB13a}: 307).}

\lit{\citealt{PVB13a}: 307 (\intxt{*\mbox{-}ʌtsi})}

\PMlemma{{\wordnl{*ɸánhaʔ \recind *ɸä́nhaʔ\pla{jʰ}}{locust} (MN)}}

\wordng{Mk}{<e>fenheʔ\pl{j}} [1] \citep[141]{AG99} {\sep} \wordng{Ni}{ɸanxa\pl{j}} \citep[130]{JS16}

\dicnote{The identity of the element \intxt{e\mbox{-}} in Maká is unclear.}%1

\PMlemma{{\textit{*ɸaxi(ˀ)j \recind *ɸäxi(ˀ)j} [1]\gloss{green ameiva (\intxt{Ameiva ameiva})} (MN)}}

\wordng{Mk}{fexij\pl{its}} \citep[174]{AG99} {\sep} \wordng{Ni}{ɸaʃij\pl{k}} (\citealt{LC20}: 468; \citealt{JS16}: 131)

\dicnote{In PM, the reconstruction of a preglottalized coda is possible only if the root has initial accent (in this case the deglottalization in Nivaĉle would be regular).}%1

\PMlemma{{\wordnl{*ɸinåk\plf{*ɸinhå\mbox{-}jʰ}}{tobacco} (MN) [1]}}

\wordng{Mk}{finak\plf{finha\mbox{-}j}} (\citealt{AG99}: 176; \citealt{JB81}: 85) {\sep} \wordng{Ni}{ɸinåk\plf{ɸinxå\mbox{-}j}} \citep[133]{JS16}

\dicnote{This noun could be derived from a verb meaning\gloss{to suck, to kiss} (cf. \wordng{Ni}{[ji]ɸin}), but the hypothetical verb \intxt{*[ji]ɸin} is not reconstructible. \citet[16]{LC-VG-07} suggest that the Maká and Nivaĉle words could have been diffused from one language to another rather than inherited, though there appears to be no valid reason to believe so.}%1

\lit{\citealt{LC-VG-07}: 16 (``diffused?''), 21}

\PMlemma{{\wordnl{*\mbox{-}ɸ’i(ʔ)}{foot} (MN)}}

\wordng{Mk}{\mbox{-}f’iʔ\pl{jej}} [1] \citep[183]{AG99} {\sep} \word{Ni}{\mbox{-}p’i<k’o>}{heel} [2] \citep[224]{JS16}

\dicnote{The Maká plural is mistranscribed as \intxt{\mbox{-}fi\mbox{-}jej} in \citet[183]{AG99}; the expected form \textit{\mbox{-}f’i\mbox{-}jej} is found in the Maká version of the New Testament (e.g. Luke 24:40).}%1

\dicnote{\wordng{Nivaĉle}{\mbox{-}k’o} is a fossilized reflex of \word{PM}{*\mbox{-}k’o\plf{*{-}k’ó\mbox{-}l}}{bottom}.}%2

\rej{\citet[55]{EN84} claims that \wordng{Ni}{\mbox{-}p’ik’o} is a cognate of the reflexes of \wordng{PM}{*\mbox{-}pák’o\pla{l}}\gloss{heel}. Only the element \intxt{*\mbox{-}k’o} could actually be cognate across Mataguayan in this case.}

\PMlemma{{\wordnl{*(\mbox{-})ɸ’ok\pla{its}}{arrow} (MN)}}

\word{Mk}{(\mbox{-})f’ok\pl{its}}{blunt-pointed arrow} \citep[184]{AG99} {\sep} \wordng{Ni}{(\mbox{-})p’ok\pl{is}} \citep[225]{JS16}

\rej{\citet[38]{EN84} compares the Nivaĉle reflex with a Wichí term for\gloss{earthenware field bottle} (\wordng{PW}{*\mbox{-}p’okʷ}) and reconstructs \wordng{PM}{*p’ɔwk’}. This is implausible for semantic reasons.}

\PMlemma{{\wordnl{*him\pla{its}}{coati} (MN)}}

\wordng{Mk}{him\pl{its}} \citep[188]{AG99} {\sep} \wordng{Ni}{xim\pl{is}} \citep[148]{JS16}

\citealt{PVB02}: 143 (\intxt{*χim})

\PMlemma{{\intxt{*jinqå\mbox{-}(ju)ˀk\plf{*jinqå\mbox{-}ku\mbox{-}jʰ}} (tree); \intxt{*jinqåˀ\mbox{-}p\plf{*jinqå\mbox{-}p\mbox{-}its}} (season)\gloss{white algarrobo (\intxt{Prosopis alba})} (MN)}}

\wordng{Mk}{<in>inqa\mbox{-}ˀk} (\intxt{\mbox{-}wi}); \textit{<in>inqa\mbox{-}ˀp\pl{its}}\gloss{summer, year} [1] (\citealt{AG99}: 202; \citealt{maka-etnomat}: 23–25) {\sep} \wordng{Ni}{jinkåˀp}, \textit{jinkåp\mbox{-}is}\gloss{algarrobo season, year} \citep[382]{JS16}

\dicnote{The coda is documented as plain (without preglottalization) in the New Testament (e.g. in Acts 18:11), which must be a mistranscription.}%1

\rej{\citet[16, 20]{LC-VG-07} and \citet[311]{PVB13a} include reflexes of \word{PCh}{*naɬqá\mbox{-}p \recind *\mbox{-}å\mbox{-}\pla{is}}{year} > \wordng{Ijw/I’w}{nahkáp\pl{is}}; \wordng{Mj}{nalkáp\pl{is}} (\citealt{ND09}: 140; \citealt{AG83}: 150; \citealt{JC18}), but this must be derived from an unrelated root with the same suffix. \citet{LC-VG-07} also include reflexes of \wordng{PW}{*neqkʲåm}\gloss{year} > \wordng{LB}{nektʃom}; \wordng{Vej}{nektʃam}; \wordng{’Wk}{nekkʲåʔ} (\intxt{\mbox{-}lis \recind nekkʲåm\mbox{-}is}) (\citealt{JB09}: 52; \citealt{VU74}: 68; \citealt{KC16}: 262), which is obviously a spurious match.}

\largerpage
\gc{\citet[311]{PVB13a} compares this root to \wordng{Proto-Guaicuruan}{*inaqa}\gloss{algarrobo tree} (\citealt{PVB13b}, \#288), \textit{*inaqá}\gloss{year} (\citealt{PVB13b}, \#289).}

\lit{\citealt{LC-VG-07}: 16, 20; \citealt{PVB13a}: 311 (\intxt{*in(a)qʌ\mbox{-}p})\gloss{year}}\clearpage

\PMlemma{{\wordnl{*(\mbox{-})jipkuʔ\pla{l}}{hunger} (MN)}}

\wordng{Mk}{(\mbox{-})jipkuʔ\pl{l}} \citep[399]{AG99} {\sep} \wordng{Ni}{jipkuʔ / \mbox{-}jipku\pl{k}} \citep[382]{JS16}

\PMlemma{{\wordnl{*jiʔixåtaχ\plf{*jiʔixåta\mbox{-}ts}}{ocelot} (MN)}}

\wordng{Mk}{iʔihataχ}, \textit{iʔihate\mbox{-}ts} \citep[226]{AG99} {\sep} \wordng{Ni}{jixåtax}, \textit{jixåta\mbox{-}s} \citep[382]{JS16}

\lit{\citealt{LC-VG-07}: 20}

\PMlemma{{\wordnl{*[ji]kålaˀɬ}{to fry} (MN)}}

\wordng{Mk}{[j]<a>kaleˀɬ} [1] \citep[114]{AG99} {\sep} \wordng{Ni}{[ji]kak͡låɬ / \mbox{-}kak͡låˀɬ} [2] \citep[56]{JS16}

\dicnote{The presence of a preglottalized coda in Maká is inferred based on the Nivaĉle cognate; the verb is not attested in our sources that distinguish between plain and preglottalized codas.}%1

\dicnote{In Nivaĉle, the vowels \textit{a} and \textit{å} were historically metathesized, but not before the palatalization of velars.}%2

\PMlemma{{\wordnl{*kómiʔ}{Chilean flamingo\species{Phoenicopterus chilensis}} (MN)}}

\wordng{Mk}{kómiʔ\pl{l}} [1] \citep[231]{AG99} {\sep} \wordng{Ni}{komi\pl{s}} \citep[71]{JS16}

\dicnote{The Maká reflex is attested as \intxt{qomi} in \citet[55]{JB81}, suggesting the reconstruction \intxt{*qomi} instead.}%2

\PMlemma{{\wordnl{*\mbox{-}ku(ʔ)}{cheek} (MN)}}

\wordng{Mk}{\mbox{-}ku\mbox{-}kiʔ\pl{j}} \citep[233]{AG99} {\sep} \wordng{Ni}{\mbox{-}kuʔ\pl{l}} \citep[341]{JS16}

\PMlemma{{\wordnl{*[wa]kumaˀχ}{to run} (MN)}}

\wordng{Mk}{[we]kumaˀχ}, \textsc{caus}~\intxt{[ji]kumk\mbox{-}et} \citep[233]{AG99} {\sep} \wordng{Ni}{[βa]kumaˀx} \citep[79]{JS16}

\dicnote{The preglottalized coda in the Maká reflex is attested in the New Testament (e.g. Luke 19:4).}%1

\PMlemma{{\wordnl{*[t]k’an \recind *[t]k’än}{to obey} (MN)}}

\wordng{Mk}{[te]k’en}\gloss{to believe, to respect} \citep[235]{AG99} {\sep} \wordng{Ni}{[t(a)]tʃ’an} \citep[248]{JS16}

\PMlemma{{\wordnl{*[t]k’ij}{to spit} (MN)}}

\wordng{Mk}{[te]k’ij} \citep[236]{AG99} {\sep} \wordng{Ni}{[t]<’a>tʃ’ij \recind [t]<’a>tʃ’i} \citep[282]{JS16}

\PMlemma{{\wordnl{*\mbox{-}k’unhateʔ}{tooth}; \wordnl{*k’unhate\mbox{-}nhaʔ\pla{jʰ}}{pacu fish} [1] (MN)}}

\wordng{Mk}{\mbox{-}k’unhetiʔ\pl{j}}; \textit{<i>k’unheti\mbox{-}nheʔ\pl{j}} \citep[196]{AG99} {\sep} \wordng{Ni}{k’unxate<nxa>\pl{j}}\gloss{pacu fish} \citep[237]{JS16}

\dicnote{It is tempting to analyze this root as a \textit{nomen instrumenti} of \wordng{PM}{*\mbox{-}kun \recind *\mbox{-}kún}\gloss{to eat (intr.)}, but the discrepancy in the glottalization of the root-initial consonant would be problematic for such analysis.}%1

\lit{\citealt{LC-VG-07}: 17 (\enquote{pacu fish})}

\PMlemma{{\wordnl{*lama(h) \recind *läma(h)\pla{m}}{to be smooth} (MN)}}

\wordng{Mk}{leːme\plf{leme\mbox{-}m}} \citep[241]{AG99} {\sep} \wordng{Ni}{k͡lama<m>} [1] \citep[115]{JS16}

\dicnote{Nivaĉle appears to have generalized the erstwhile plural form.}%1

\gc{\citet[307]{PVB13a} compares the root with \word{Proto-Guaicuruan}{*\mbox{-}ʔa(ˀ)le(ˀ)m}{to be bald}, which seems semantically far-fetched.}

\lit{\citealt{PVB13a}: 307 (\intxt{*leme(m)}); \citealt{AnG15}: 253}

\PMlemma{{\wordnl{*lasa(h) \recind *läsa(h) \recind *lasaʔ \recind *läsaʔ}{to be thin} (MN)}}

\wordng{Mk}{<e>lese\mbox{-}j} \citep[145]{AG99} {\sep} \wordng{Ni}{k͡lasa-tʃ’e} \citep[116]{JS16}

\PMlemma{{\wordnl{*låttsiki\mbox{-}juˀk\plf{*låttsiki\mbox{-}ku\mbox{-}jʰ}}{willow} (MN)}}

\wordng{Mk}{lattsiki\mbox{-}juˀk} [1] \citep[240]{AG99} {\sep} \wordng{Ni}{k͡låtsiki\mbox{-}juk\plf{k͡låtsiki\mbox{-}ku\mbox{-}j}} [2] \citep[120]{JS16}

\dicnote{The preglottalized coda in the Maká suffix for tree names is attested elsewhere \citep[7]{PMA}.}%1

\dicnote{The failure of \sound{PM}{*k} to palatalize in Nivaĉle is unexpected.}%2

\lit{\citealt{LC-VG-07}: 16}

\PMlemma{{\wordnl{*\mbox{-}ɬíˀwteʔ}{heart} (MN)}}

\wordng{Mk}{\mbox{-}ɬitiʔ\pl{j \recind \mbox{-}l}} \citep[254]{AG99} {\sep} \wordng{Ni}{\mbox{-}ɬiˀβte} (\citealt{AF16}: 303; \citealt{LC20}: 119)

\rej{\citet[38, 42]{EN84} compares the Nivaĉle reflex with reflexes of \word{PCh}{*\mbox{-}ʔót}{chest} and \word{PW}{*\mbox{-}t’ókʷe}{chest}, but this is absolutely impossible for phonological reasons.}

\PMlemma{{\wordnl{*maˀlaˀl \recind *mä\mbox{-} \recind *\mbox{-}ˀläˀl}{agile} (MN)}}

\word{Mk}{meˀleˀl}{to move (intr.)} [1], \textsc{caus}~\wordnl{\mbox{-}meˀleˀl\mbox{-}hit}{to move} \citep[260]{AG99} {\sep} \wordng{Ni}{mak͡laˀk} \citep[172]{JS16}

\dicnote{The intransitive verb is documented in the New Testament (Hebrews 12:27; Matthew 28:2; Revelations 6:12; Revelations 8:4; Revelations 16:8). It could be etymologically identical to \textit{melel\pl{its}}\gloss{deer} \citep[260]{AG99}, which is, however, attested with no glottalization in \citet[49]{JB81}.}%1

\PMlemma{{\wordnl{*(\mbox{-})nawan \recind *\mbox{-}ä\mbox{-}}{hook} (MN)}}

\wordng{Mk}{newen\pl{its}} \citep[273]{AG99} {\sep} \wordng{Ni}{\mbox{-}naβan} (\intxt{\mbox{-}ij}) \citep[183]{JS16}

\PMlemma{{\wordnl{*nijåtsek\plf{*nijåtshe\mbox{-}jʰ}}{fermented drink} (MN)}}

\wordng{Mk}{nijatsik} [1], \intxt{nijatshi\mbox{-}j} (\citealt{AG99}: 224; \citealt{unuuneiki}: 18) {\sep} \wordng{Ni}{(\mbox{-})nijåtsetʃ\plf{(\mbox{-})nijåtsxe\mbox{-}j}} \citep[198]{JS16}

\dicnote{The singular form is attested both as \textit{nijatsik} and \textit{nijatshik} in Maká by \citet[224]{AG99}, of which only the former is etymological.}%1

\PMlemma{{\wordnl{*[n]xt’oʔ}{to wake up}, \textsc{caus}~\intxt{[n]xt’o\mbox{-}tshan} [1]}}

Mk~—, \intxt{[n]<i>xt’o\mbox{-}tshen} (\citealt{AG99}: 222) {\sep} \wordng{Ni}{[n(i)]xat’oʔ}, \intxt{[n(i)]xat’o\mbox{-}tsxan} (\citealt{LC20}: 114; \citealt{JS16})

\PMlemma{{\wordnl{*\mbox{-}pas \recind *\mbox{-}päs}{hand / finger} (MN)}}

Mk (Lengua doculect) ‹hipès›\gloss{hand}, ‹hipecé›\gloss{fingers} \citep[456]{AD60} {\sep} \word{Ni}{\mbox{-}pas\mbox{-}tʃe\pl{j}}{finger} (\citealt{JS16}: 218; \citealt{LC20}: 129)

\PMlemma{{\textit{*qapa(ˀ)p \recind *\mbox{-}ä\mbox{-}} [1], \textit{*qapap\mbox{-}its \recind *\mbox{-}ä\mbox{-}}\gloss{dwarf} (MN)}}

\wordng{Mk}{qep<ep>e(ˀ)p} [1], \textit{qep<ep>ep\mbox{-}its} \citep[308]{AG99} {\sep} \word{Ni}{kapap\pl{is}}{dwarf dog} \citep[61]{JS16}

\dicnote{The uncertainty regarding the coda is due to the fact that the form is not attested in our sources on Maká that distinguish between plain and preglottalized codas. In PM, the reconstruction of a preglottalized coda is possible only if the root has initial accent (in this case the deglottalization in Nivaĉle would be regular).}%1

\dicnote{The extra element \intxt{\mbox{-}ep\mbox{-}} in Maká appears to be an instance of partial reduplication.}%2

\gc{\citet[308]{PVB13a} notes the similarity with \word{Proto-Pilagá–Toba}{*qapí}{small}, which could be spurious.}

\lit{\citealt{PVB13a}: 308 (\intxt{*qapap})}

\PMlemma{{\wordnl{*\mbox{-}q’åχtåχ}{palate} (MN) [1]}}

\wordng{Mk}{\mbox{-}q’ataχ\plf{\mbox{-}q’ate\mbox{-}ts}} \citep[319]{AG99} {\sep} \wordng{Ni}{\mbox{-}k’åxtåx\pl{is}} \citep[89]{JS16}

\dicnote{The root could be related to \word{PM}{\mbox{-}q’á(ˀ)X₁₂}{tongue} (ChW), but the vowels do not match.}%1

\gc{\citet[309]{PVB13a} notes the similarity with \wordng{Proto-Guaicuruan}{*\mbox{-}qot’e}\gloss{palate} (absent from \citealt{PVB13b}).}

\lit{\citealt{PVB13a}: 309 (\intxt{*\mbox{-}q’ʌtʌh})}

\PMlemma{{\wordnl{*\mbox{-}saˀx \recind *\mbox{-}säˀx}{leaf} (MN)}}

\wordng{Mk}{\third{ɬe\mbox{-}seˀx}} [1], \intxt{ɬe\mbox{-}sex\mbox{-}ets} \citep[251]{AG99} {\sep} \word{Ni}{\mbox{-}saˀʃ\plf{\mbox{-}saʃ\mbox{-}aj}}{leaf, hair} \citep[63]{JS16}

\dicnote{The presence of a preglottalized coda in Maká is inferred based on the Nivaĉle cognate; the singular form is not attested in our sources that distinguish between plain and preglottalized codas. The plural form is attested in the New Testament (e.g. Mark 11:13), but it is not revealing.}%1

\lit{\citealt{PVB02}: 143 (\intxt{*sex})}

\PMlemma{{\wordnl{*sámtoʔ}{foreigner} (MN)}}

\word{Mk}{sontoʔ}{non-indigenous person} \citep[327]{AG99} {\sep} \word{Ni}{samto}{Argentine criollo} \citep[230]{JS16}

\PMlemma{{\wordnl{*samto\mbox{-}ˀk}{bamboo\species{Guadua angustifolia}} (MN)}}

\wordng{Mk}{sontok} [1] (\citealt{AG99}: 327; \citealt{JB81}: 82) {\sep} \wordng{Ni}{samtoˀk} \citep[230]{JS16}

\dicnote{The loss of preglottalization in the coda in Maká is unexpected.}%1

\PMlemma{{\textit{*sålå(ˀ)l} [1], \textit{*sålål\mbox{-}its}\gloss{middle-sized cicada} (MN) [2]}}

\wordng{Mk}{sala(ˀ)l} [1], \textit{salal\mbox{-}its} \citep[323]{AG99} {\sep} \wordng{Ni}{såk͡l<åk͡l>åk\pl{is}} [3] \citep[235]{JS16}

\dicnote{The uncertainty regarding the coda is due to the fact that the form is not attested in our sources on Maká that distinguish between plain and preglottalized codas. In PM, the reconstruction of a preglottalized coda is possible only if the root has initial accent (in this case the deglottalization in Nivaĉle would be regular).}%1

\dicnote{\wordng{Iyojwa’aja’}{sʲáhla}, \word{Iyo’awujwa’}{sʲáhlala\plf{sʲáhlala\mbox{-}l \recind sʲáhlal\mbox{-}is \recind sʲéhlala\mbox{-}as}}{cicada} (\citealt{JC14b}: 100; \citealt{AG83}: 159) cannot be cognate for phonological reasons; it must be a borrowing instead.}%2

\dicnote{The extra element \intxt{\mbox{-}åk͡l\mbox{-}} in Nivaĉle appears to be an instance of partial reduplication.}%3

\PMlemma{{\textit{*sijå(ˀ)χ} [1], \textit{*sijåχ\mbox{-}its}\gloss{fish sp.} (MN)}}

\wordng{Mk}{sija(ˀ)χ} [1], \textit{sijaχ\mbox{-}its}\gloss{fish sp. (small, unedible, with a black stripe)} \citep[327]{AG99} {\sep} \wordng{Ni}{sijåx\pl{is}} \citep[234]{JS16}

\dicnote{The uncertainty regarding the coda is due to the fact that the form is not attested in our sources on Maká that distinguish between plain and preglottalized codas. In PM, the reconstruction of a preglottalized coda is possible only if the root has initial accent (in this case the deglottalization in Nivaĉle would be regular).}%1

\PMlemma{{\wordnl{*(\mbox{-})tak’o(h) \recind *(\mbox{-})täk’o(h)}{kind of utensil} (MN)}}

\word{Mk}{tok’o\pl{l}}{plate, bucket, jar} \citep[341]{AG99} {\sep} \word{Ni}{\mbox{-}tak’o\mbox{-}tax\plf{\mbox{-}tak’o\mbox{-}txa\mbox{-}s}}{piece of knife} \citep[247]{JS16}

\PMlemma{{\wordnl{*tana(h) \recind *täna(h)}{standing, vertical} (MN)}}

\wordng{Mk}{teːne\plf{tene\mbox{-}m}} \citep[333]{AG99} {\sep} \wordng{Ni}{tana} \citep[251]{JS16}

\PMlemma{{\wordnl{*teχ\pla{its}}{parrot sp.} (MN)}}

\word{Mk}{taχ\pl{its}}{nanday parakeet\species{Aratinga nenday}} (\citealt{AG99}: 333; \citealt{JB81}: 60) {\sep} \word{Ni}{tex\pl{is}}{scaly-headed parrot\species{Pionus maximiliani}} \citep[96, 506]{LC20}

\PMlemma{{\wordnl{*tiˀj}{to weave} (MN)}}

\wordng{Mk}{tij / \mbox{-}ɬij} \citep[336]{AG99} {\sep} \word{Ni}{tiˀj}{to weave; to model (with clay)} \citep[269]{JS16}

\PMlemma{{\wordnl{*(\mbox{-})tiˀnåx\pl{its}}{object made of leather} [1] (MN)}}

\wordng{Mk}{tiˀnax\pl{its}} [2]\gloss{leather bag for travel} (formerly\gloss{traditional bag made of rhea skin}) \citep[338]{AG99} {\sep} \word{Ni}{tiˀnåx\plf{tinåx\mbox{-}is}}{leather strap, lash} (\citealt{AnG15}: 57, fn. 22; \citealt{JS16}: 269; \citealt{LC20}: 95)

\dicnote{This noun is likely derived from \word{PM}{\mbox{-}ʔåx\pla{íts}}{skin, bark} by means of an absolutizing prefix.}%1

\dicnote{The preglottalization in the stem-medial nasal in Maká is attested in the New Testament (e.g. Luke 10:4).}%2

\PMlemma{{\wordnl{*tuχ\mbox{-}\APPL}{to burn (intr.)} (MN)}}

\wordng{Mk}{tuχ\mbox{-}xeˀm \recind tux\mbox{-}xeˀm} [1], \textit{tuχ\mbox{-}eʔ} \citep[344]{AG99} {\sep} \wordng{Ni}{tux\mbox{-}aˀm}, \textit{tux\mbox{-}ej} \citep[280]{JS16}

\dicnote{The root-final consonant is attested as \textit{χ} in \citet{AG99} and as \textit{x} in the New Testament (e.g. Ephesians 6:16).}%1

\gc{Possibly related to \wordng{Proto-Guaicuruan}{*\mbox{-}a(ˀ)leg}{to burn} (\citealt{PVB13b}, \#28).}

\PMlemma{{\wordnl{*[n]t’å}{to gather fruit} (MN)}}

\wordng{Mk}{[n]<a>t’a<ʔa>\mbox{-}kii / \mbox{-}t’a<ʔa>\mbox{-}kii} \citep[133]{AG99} {\sep} \wordng{Ni}{[n(i)]t’å} \citep[196]{JS16}

\gc{\citet[306]{PVB13a} compares this verb to \word{Proto-Pilagá–Toba}{*\mbox{-}n\mbox{-}áto}{to gather, to collect}.}

\lit{\citealt{PVB13a}: 306 (\intxt{*\mbox{-}at’ʌʔ})}

\PMlemma{{\wordnl{*t’åˀj}{to sound, to have voice} (MN)}}

\word{Mk}{t’aj}{to sound} \citep[345]{AG99} {\sep} \word{Ni}{t’åˀj}{to have voice} \citep[289]{JS16}

\PMlemma{{\wordnl{*[ji]t’ex}{to say} (MN)}}

\wordng{Mk}{[ji]t’ix} \citep[212]{AG99} {\sep} \wordng{Ni}{[ji]t’eʃ / \mbox{-}eˀʃ} [1] \citep[384]{JS16}

\dicnote{The allomorph \textit{\mbox{-}eˀʃ} is irregular and has no counterpart in Maká. It might have an entirely different origin.}%1

\PMlemma{{\wordnl{*tsaqaq \recind *\mbox{-}ä\mbox{-}}{plant sp.} [1] (MN)}}

\word{Mk}{tseqeq}{\textit{Cissus palmata}} (\citealt{AG99}: 348; \citealt{JB81}: 79) {\sep} \word{Ni}{tsakak\pl{is}}{São Caetano melon\species{Cayaponia espelina}} \citep[291]{JS16}

\dicnote{\textit{Cissus palmata} and \textit{Cayaponia espelina} have in common the trait that while their fruits are unsuitable for human consumption, they are eaten by animals (toucans and maned wolfs, respectively).}%1

\PMlemma{{\intxt{*(\mbox{-})tsaˀt\plf{*(\mbox{-})tsat\mbox{-}its}} (\intxt{\recind *\mbox{-}ä\mbox{-}})\gloss{village} [1] (MN)}}

\wordng{Mk}{\mbox{-}tset} [2], \intxt{\mbox{-}tset\mbox{-}its} \citep[161]{AG99} {\sep} \wordng{Ni}{<ji>tsaˀt\plf{<ji>tsat\mbox{-}is}} [3] \intxt{/ \mbox{-}β\mbox{-}tsaˀt\plf{\mbox{-}β\mbox{-}tsat\mbox{-}its}} (\citealt{JS16}: 338, 385)

\dicnote{This etymology has been first identified by \citet{LC-subm}.}%1

\dicnote{The Maká reflex unexpectedly lacks preglottalization in the coda in the singular form, as attested in the New Testament (John 1:44).}%2

\dicnote{We have no explanation for the element \intxt{*ji\mbox{-}} in the absolute form in Nivaĉle.}%3

\lit{\citealt{LC-subm} (\intxt{*(w)itset})}

\PMlemma{{\wordnl{*\mbox{-}xéleʔ}{dirt} (MN)}}

\wordng{Mk}{\mbox{-}xiliʔ\pl{j}} \citep[389]{AG99} {\sep} \wordng{Ni}{\mbox{-}ʃek͡le\pl{k}} \citep[357]{JS16}

\citealt{PVB02}: 142 (\intxt{*xele})

\PMlemma{{\wordnl{*waɸ \recind *wäɸ}{to be tired, to die} (MN) [1]}}

\word{Mk}{[ji]wef}{to be tired} \citep[365]{AG99} {\sep} \word{Ni}{βaɸ}{to die} \citep[313]{JS16}

\dicnote{\citet[29]{EN84} claims to have discovered a cognate in Chorote (\wordnl{wax}{dead}), but we are unaware of the existence of any similar lexeme in Chorote.}%1

\gc{\citet[314]{PVB13a} compares the root to \word{Abipón}{\mbox{-}oaoa}{to die} \citep[113]{EN66}, but this could be spurious.}

\lit{\citealt{EN84}: 29 (\intxt{*wahw}); \citealt{PVB13a}: 314 (\intxt{*\mbox{-}wahʷ})}

\PMlemma{{\wordnl{*waˀj \recind *wäˀj}{to be wet, to get wet} (MN)}}

\wordng{Mk}{wej\mbox{-}xuʔ} \citep[373]{AG99} {\sep} \wordng{Ni}{βaˀj} \citep[259]{LC20}

\PMlemma{{\wordnl{*wapen \recind *wäpen}{to be ashamed; “shame plant” [1]} (MN)}}

\word{Mk}{wepin}{to be ashamed; \textit{Cassia patellaria}, \textit{Mimosa chacoensis}} \citep[367]{AG99} {\sep} \word{Ni}{βapen}{to be ashamed; \textit{Bauhinia langdorffiana}, \textit{Cassia flexuosa}} (\citealt{JS16}: 334–335)

\dicnote{The plants designated by reflexes of this etymon are species whose leaves close when touched. Both the Maká and the Nivaĉle rub their leaves against children’s faces so as to prevent them from being shameless.}%1

\PMlemma{{\wordnl{*(ˀ)wawo(h)\pla{l}}{maned wolf} (MN) [1]}}

\wordng{Mk}{wowo\pl{l}} \citep[380]{AG99} {\sep} \wordng{Ni}{βaβo\pl{k}} \citep[358]{JS16}

\dicnote{This etymology is very similar to \wordnl{*Xmáwoh}{fox} (ChW), but the root-initial consonants do not match. \citet{EN84} lumps these etymologies together.}%1

\lit{\citealt{EN84}: 13, 44 (\intxt{*mawo \recind *wawo})}

\PMlemma{{\wordnl{*wåˀm}{to disappear} (MN)}}

\word{Mk}{waˀm}{to die} (\citealt{AG99}: 360; \citealt{JB81}: 203) {\sep} \word{Ni}{βåˀm}{to disappear} \citep[371]{JS16}

\PMlemma{{\intxt{*wåˀmqåʔ} [1]\gloss{to wash oneself} (MN)}}

\wordng{Mk}{waˀnqaʔ} \citep[361]{AG99} {\sep} \wordng{Ni}{βåmqåʔ / \mbox{-}βåˀmqå} \citep[371]{JS16}

\dicnote{The Maká form is attested as such in the New Testament (e.g. Matthew 15:2). \citet[361]{AG99} gives simply \textit{wanqa}.}%1

\PMlemma{{\wordnl{*(ˀ)wǻnaˀχ\plf{*(ˀ)wǻnha\mbox{-}ts}}{piranha} (MN)}}

\wordng{Mk}{wanaˀχ\plf{wanhe\mbox{-}ts}} (\citealt{AG99}: 361; \citealt{JB81}: 67) {\sep} \word{Ni}{βånax\plf{βånxa\mbox{-}s}}{piranha; barn owl} \citep[370]{JS16}

\PMlemma{{\intxt{*wåpi(ˀ)j} [1]\gloss{to unload} (MN)}}

\wordng{Mk}{wapij} [2]\gloss{to have a rest} \citep[362]{AG99} {\sep} \wordng{Ni}{βåpij} \citep[372]{JS16}

\dicnote{In PM, the reconstruction of a preglottalized coda is possible only if the root has initial accent (in this case the deglottalization in Nivaĉle would be regular).}%1

\dicnote{The Maká form is attested as such in the New Testament (e.g. Hebrews 4:10). \citet[362]{AG99} gives \textit{wapiʔi}.}%2

\PMlemma{{\wordnl{*(ˀ)wåˀs}{sky} (MN)}}

\wordng{Mk}{waˀs\plf{was\mbox{-}its}} (\citealt{AG99}: 363; \citealt{JB81}: 198) {\sep} \wordng{Ni}{βåˀs} \citep[371]{JS16}

\PMlemma{{\intxt{*(ˀ)wåseʔ} [1]\gloss{cloud} (MN)}}

\wordng{Mk}{wasiʔ\pl{l}} \citep[363]{AG99} {\sep} \wordng{Ni}{βåseʔ\pl{j}} \citep[372]{JS16}

\dicnote{The stem is evidently derived from \wordnl{*(ˀ)wåˀs \recind *(ˀ)wǻˀs}{sky}, but the identity of the second element is unknown.}%1

\PMlemma{{\intxt{*\mbox{-}wåˀt}; \wordnl{*\mbox{-}wåt\mbox{-}hajeχ}{birthmark} (MN)}}

\wordng{Mk}{\mbox{-}wat<hejɑχ>} \citep[363]{AG99} {\sep} \word{Ni}{\mbox{-}βåˀt}{birthmark}; \wordnl{\mbox{-}βåt\mbox{-}xajex}{mole} (\citealt{JS16}: 372–373)

\PMlemma{{\wordnl{*(ˀ)wq’am \recind *(ˀ)wq’äm}{white-eared opossum} (MN)}}

\wordng{Mk}{weq’em\pl{its}} (\citealt{AG99}: 368; \citealt{JB81}: 49) {\sep} \wordng{Ni}{k’am<i>\pl{k}} \citep[85]{JS16}

\PMlemma{{\wordnl{*(ˀ)wut}{a bushy leguminous plant} (MN)}}

\wordng{Mk}{wut}\gloss{\textit{Sesbania exasperata}} \citep[382]{AG99} {\sep} \wordng{Ni}{βut}\gloss{\textit{Acacia sp.}} \citep[374]{JS16}

\PMlemma{{\intxt{*ˀwéˀɬ}; \wordnl{*ˀwéˀɬ=aʔ}{one} (MN)}}

\wordng{Mk}{<e>wiˀɬ}\gloss{one}; \textit{<e>wiˀɬ\mbox{-}eʔ}\gloss{alone} (\citealt{AG99}: 165; \citealt{JB81}: 197) {\sep} \wordng{Ni}{βéʔɬ<a> / \mbox{-}ˀβéʔɬ<a>} \citep[359]{JS16}

\dicnote{The Maká forms are attested as such in \citet[197]{JB81} and in the New Testament (e.g. John 3:1; John 15:13). \citet[165]{AG99} gives simply \textit{ewiɬ}, \textit{ewiɬe}.}%1

\empr{\citet[308]{AF16} compares the Nivaĉle word to the Wichí term for\gloss{one, only one} (\wordng{LB}{\mbox{ʔiwenjaɬa}}; \wordng{Vej}{wenjaɬa}; \wordng{’Wk}{ʔiwehˈjáɬah}, \wordng{Güisnay}{weihaɬa \recind un̥jaɬa} (\citealt{VN14}: 358; \citealt{VU74}: 80; \citealt{MG-MELO15}: 27; \citealt{KC16}: 41) and with the Enlhet–Enenlhet term for\gloss{only, just, just that} – \wordng{Enlhet}{waːmɬa}, \wordng{Enxet}{wanɬa}, \wordng{Enenlhet-Toba, Guaná}{wanɬaʔ} (\citealt{EU-HK-97}: 655; \citealt{EU-HK-MR-03}: 338; \citealt{JE21}: 245; \citealt{HK-23}: 191) – but that is likely a spurious comparison.}

\PMlemma{{\intxt{*xoxaw\mbox{-}uˀk \recvar *xoxi\mbox{-}juˀk\plf{*\mbox{-}ku\mbox{-}jʰ}} [1]\gloss{\textit{Tabebuia nodosa} tree} (MN)}}

\wordng{Mk}{xoxew\mbox{-}uˀk} [2], \textit{xoxew\mbox{-}kw\mbox{-}i} \citep[392]{AG99} {\sep} \wordng{Ni}{xoxi\mbox{-}juk}, \textit{xoxi\mbox{-}ku\mbox{-}j} \citep[149]{JS16}

\dicnote{The Maká form points to \intxt{*xoxaw\mbox{-}uˀk}, the Nivaĉle one to \intxt{*xoxi\mbox{-}juˀk}.}%1

\dicnote{The preglottalized coda in the Maká suffix for tree names is attested elsewhere \citep[7]{PMA}.}%2

\lit{\citealt{PVB02}: 142 (\intxt{*xoxewuk})}

\PMlemma{{\wordnl{*\mbox{-}ʔåɸk’uˀt}{bile} (MN)}}

\wordng{Mk}{\mbox{-}ʔaftuk\plf{\mbox{-}ʔafthu\mbox{-}j}} [1] \citep[114]{AG99} {\sep} \wordng{Ni}{\mbox{-}ʔaɸk’uˀt\plf{\mbox{-}ʔaɸk’ut\mbox{-}es}} (\citealt{LC20}: 143, 154)

\dicnote{Maká suffered an irregular metathesis of \sound{PM}{*k’} and \intxt{*t} and loss of glottalization in both consonants. The coda is attested as plain (with no glottalization) in the New Testament (Matthew 27:34).}%1

\rej{\citet[15]{LC-VG-07} list reflexes of \wordng{PCh}{*\mbox{-}témek}, \wordng{PW}{*\mbox{-}témeq} under this etymology, an obviously false comparison.}

\lit{\citealt{LC-VG-07}: 15}

\PMlemma{{\wordnl{*ʔaˀnqoˀk}{paralytic} (MN)}}

\wordng{Mk}{onqok\pl{its}} [1] \citep[283]{AG99} {\sep} \word{Ni}{ʔaˀnkoˀk\plf{\mbox{-}ʔankoxo\mbox{-}j}}{limp, paralytic} (\citealt{AF16}: 207; \citealt{JS16}: 44)

\dicnote{The Maká reflex unexpectedly lacks preglottalization in both codas, as attested in the New Testament (Mark 2:3).}%1

\lit{\citealt{AF16}: 43, fn. 27}

\PMlemma{{\intxt{*[t]’aqsin \recvar *[t]’aq’asin} [1]\gloss{to sneeze} (MN)}}

\wordng{Mk}{[t]’aqsin\mbox{-}kij} [1] \citep[128]{AG99} {\sep} \wordng{Ni}{[t]’ak’asin} (\citealt{LC20}: 241, 250)

\dicnote{The Maká reflex points to \intxt{*[t]’aqsin}, the Nivaĉle one to \intxt{*[t]’aq’asin}. A similar root is found in Chorote and Wichí (see \intxt{*[tᵊ]nxát’itsaXan} in \sectref{chwonly}), but the correspondences are entirely irregular.}%1

\PMlemma{{\wordnl{*[t]’at’o}{to yawn} (MN)}}

\wordng{Mk}{[t]ot’o\mbox{-}kij} \citep[287]{AG99} {\sep} \wordng{Ni}{[t]’at’o} \citep[378]{JS16}

\gc{Obviously related to \word{Proto-Guaicuruan}{*\mbox{-}at’ó}{to yawn} (\citealt{PVB13b}, \#132; cf. \citealt{PVB13a}: 305).}

\lit{\citealt{PVB13a}: 305 (\intxt{*\mbox{-}at’o})}

\PMlemma{{\wordnl{*ʔåɸteˀl}{orphan} (MN)}}

\wordng{Mk}{(\mbox{-})aftiˀl} [1], \textit{(\mbox{-})aftil\mbox{-}ets} \citep[113]{AG99} {\sep} \wordng{Ni}{ʔåɸteˀk}, \textit{ʔåɸtek͡l\mbox{-}es \recind ʔåɸtek͡l\mbox{-}ej} (ChL\mbox{-}Pi) (\citealt{AnG15}: 254, 277)

\dicnote{The presence of a preglottalized coda in the Maká singular form is inferred based on the Nivaĉle cognate; the noun is not attested in our sources that distinguish between plain and preglottalized stops.}%1

\lit{\citealt{LC-VG-07}: 22; \citealt{AnG15}: 253}

\PMlemma{{\intxt{*ʔåthajeχ} (fruit); \intxt{*ʔåthaj\mbox{-}uˀk\plf{*ʔåthaj\mbox{-}ku\mbox{-}jʰ}} (tree) (\intxt{*\mbox{-}hä\mbox{-}})\gloss{molle plant} (MN)}}

\wordng{Mk}{athejaχ}; \textit{athej\mbox{-}uˀk} [1] (\intxt{\mbox{-}kw\mbox{-}i \recind \mbox{-}ku\mbox{-}ket})\gloss{\textit{Sideroxylon obtusifolium}} \citep[131]{AG99} {\sep} \wordng{Ni}{ʔåtxajex\pl{s}}; \textit{ʔåtxaj\mbox{-}uk}, \textit{ʔåtxa\mbox{-}ku\mbox{-}j}\gloss{\textit{Schinus molle}} \citep[214]{JS16}

\dicnote{The preglottalized coda in the Maká suffix for tree names is attested elsewhere \citep[7]{PMA}.}%1

\PMlemma{{\wordnl{*ʔåχtinaˀχ\plf{*ʔåχtinha\mbox{-}ts}}{marsh deer\species{Blastocerus dichotomus}} (MN)}}

\wordng{Mk}{aχtinaχ\plf{aχtinhe\mbox{-}ts}} [1] (\citealt{AG99}: 138; \citealt{unuuneiki}: 16, 17) {\sep} \wordng{Ni}{ʔåxtinaˀx\plf{ʔåxtinxa\mbox{-}s}} \citep[211]{JS16}

\dicnote{The preglottalization in the singular form in Maká is attested in a narrative by \citet[16, 17]{unuuneiki}.}%1

\PMlemma{{\intxt{*ʔomhatäk} (fruit); \intxt{*ʔomhatä\mbox{-}(ju)ˀk\plf{*ʔomhatä\mbox{-}ku\mbox{-}jʰ}} (tree) (\intxt{\recind *\mbox{-}hä\mbox{-}})\gloss{queen palm\species{Syagrus romanzoffiana}} (MN)}}

\wordng{Mk}{omhetek}; \textsc{pl}~\textit{omhet\mbox{-}kw\mbox{-}i} (\citealt{AG99}: 282; \citealt{unuuneiki}: 17) {\sep} \wordng{Ni}{ʔomxatatʃ}; \textit{ʔomxata\mbox{-}juk}, \textit{ʔomxata\mbox{-}ku\mbox{-}j} \citep[207]{JS16}

\PMlemma{{\intxt{*ʔujhVl} [1]\gloss{otter sp.} (MN)}}

\word{Mk}{wihil\pl{ets}}{\textit{lobo pirí} otter}, \wordnl{wihil\mbox{-}te\mbox{-}kiʔ\pl{j}}{\textit{lobo pe} otter\species{Lontra longicaudis}} (\citealt{AG99}: 375; \citealt{JB81}: 48) {\sep} \word{Ni}{ʔujxak͡l<å>\pl{j}}{\textit{lobo pe} otter\species{Lontra longicaudis}} \citep[306]{JS16}

\dicnote{The Maká reflex points to \intxt{*ʔujhel} or \intxt{ʔujhil}; the Nivaĉle one to \intxt{*ʔujhal} or \intxt{ʔujhäl}.}%1

\PMlemma{{\wordnl{*ʔutsi(h)\pla{l}}{marbled swamp eel} (MN)}}

\wordng{Mk}{utsi\pl{l}} \citep[356]{AG99} {\sep} \wordng{Ni}{ʔutsi\pl{k}} \citep[308]{JS16}
\end{adjustwidth}
\section{ChW only}\label{chwonly}

\begin{adjustwidth}{6mm}{0pt}
\largerpage
In this section, we list the cognate sets with reflexes only in Chorote and Wichí. Despite being technically reconstructible only for Proto-Chorote–Wichí, the reconstructions given in this section correspond to the Proto-Mataguayan level. This is done in order to facilitate the future search of cognates in other languages, but also because a detailed reconstruction of the Proto-Chorote–Wichí phonology is yet to be worked out.

\PMlemma{{\wordnl{*\mbox{-}áˀl}{light, brightness} (ChW)}}

PCh~\third{*hl\mbox{-}áˀl} > Ijw/Mj~\third{hl\mbox{-}áˀl} (\citealt{ND09}: 130; \citealt{JC18}) {\sep} \wordng{PW}{*\mbox{-}ɬ\mbox{-}álʰ} > \wordng{’Wk}{\mbox{-}<ɬ>áɬ} \citep[72]{KC16}

\PMlemma{{\wordnl{*\mbox{-}ǻˀm}{pronominal formative} (ChW)}}

PCh~1~\intxt{*j\mbox{-}ǻˀm}; 2~\intxt{*∅\mbox{-}ʔǻˀm}; 1+2~\intxt{*s\mbox{-}ǻˀm};~\third{*hl\mbox{-}ǻˀm}, \textit{*hl\mbox{-}ǻm\mbox{-}is} > Ijw~1~\intxt{j\mbox{-}áˀm}; 2~\intxt{∅\mbox{-}ʔáˀm}; 1+2~\textit{s\mbox{-}áˀm};~\third{hl\mbox{-}áˀm}, \textit{hl\mbox{-}ám\mbox{-}is}; I’w~1~\intxt{j\mbox{-}ém}; 2~\intxt{∅\mbox{-}ám}; 1+2~\intxt{s\mbox{-}ám};~\third{hl\mbox{-}ám\pl{is}}; Mj~1~\intxt{j\mbox{-}éˀm}, \textit{j\mbox{-}ém\mbox{-}eɬ}; 2~\intxt{∅\mbox{-}áˀm}, \textit{∅\mbox{-}ám\mbox{-}eɬ}; 1+2~\intxt{s\mbox{-}áˀm}, \textit{sám\mbox{-}eɬ};~\third{hl\mbox{-}áˀm}, \textit{hl\mbox{-}ám\mbox{-}is} (\citealt{JC14b}: 90, fn. 20; \citealt{ND09}: 95, 130, 145, 158; \citealt{AG83}: 120, 134, 157, 174; \citealt{JC18}) {\sep} PW~1~\intxt{*j\mbox{-}áˀm}; 2~\intxt{*∅\mbox{-}ʔáˀm\plf{*∅\mbox{-}ʔám\mbox{-}elʰ}}; 1+2~\intxt{*ˣn\mbox{-}ám\mbox{-}elʰ}; \third{*ɬ\mbox{-}áˀm}, \textit{*ɬ\mbox{-}ám\mbox{-}elʰ} > LB~1~\intxt{n̩\mbox{-}ɬ\mbox{-}am} (\intxt{\mbox{-}iɬ}); 2~\intxt{∅\mbox{-}ʔam} (\intxt{\mbox{-}iɬ}); 1+2~\intxt{to\mbox{-}ɬ\mbox{-}am\mbox{-}iɬ};~\third{ɬ\mbox{-}am} (\intxt{\mbox{-}iɬ}); \textsc{hort} \textit{n\mbox{-}am\mbox{-}iɬ} [1 2]; Vej~1~\intxt{ʔo\mbox{-}ɬ\mbox{-}am} (\intxt{\mbox{-}el}); 2~\intxt{∅\mbox{-}ʔam} (\intxt{\mbox{-}el}); 1+2~\intxt{(ˀ)n\mbox{-}am\mbox{-}el};~\third{ɬ\mbox{-}am} (\intxt{\mbox{-}el}); ’Wk~1~\intxt{ʔõ\mbox{-}ɬ\mbox{-}áˀm\plf{ʔõ\mbox{-}ɬ\mbox{-}ám\mbox{-}eɬ}} (“formal sociolect”) \intxt{/ j\mbox{-}áˀm\plf{j\mbox{-}ám\mbox{-}eɬ}} (“informal sociolect”); 2~\intxt{∅\mbox{-}ʔáˀm\plf{∅\mbox{-}ʔám\mbox{-}eç \recind ∅\mbox{-}ʔám\mbox{-}ejaç \recind ∅\mbox{-}ʔám\mbox{-}eɬ}}; 1+2~\intxt{ʔin\mbox{-}ám\mbox{-}eɬ}; \third{ɬ\mbox{-}áˀm \recind ɬ\mbox{-}ám\plf{ɬ\mbox{-}ám\mbox{-}eɬ}} (\citealt{VN14}: 335; \citealt{VU74}: 50, 65, 67, 69; \citealt{MG-MELO15}: 13; \citealt{JAA12b}: 57; \citealt{KC16}: 12, 32, 45, 231)

\dicnote{Lower Bermejeño Wichí and Vejoz have irregularly lost glottalization in the final nasal (\sound{PW}{*ˀm} > \intxt{m}).}%1

\dicnote{Lower Bermejeño Wichí has irregularly raised \sound{PW}{*e} to \intxt{i} in the plural suffix.}%2

\gc{Likely related to \wordng{Proto-Guaicuruan}{*\mbox{-}ˀm}, as in \wordnl{*ejé\mbox{-}ˀm}{I}, \wordnl{*ʔa\mbox{-}ˀm}{thou}, \wordnl{*q’o\mbox{-}ˀm}{we}, \wordnl{*aq’a\mbox{-}ˀm\mbox{-}ʔi}{you all} (\citealt{PVB13b}, \#103, 198, \#541, \#660; cf. \citealt{PVB13a}: 312).}

\lit{\citealt{PVB13a}: 312 (1~\intxt{*j\mbox{-}am}; 2~\intxt{*am};~\third{*ɬ\mbox{-}am})}

\PMlemma{{\wordnl{*\mbox{-}ǻme(ˀ)t / *\mbox{-}ǻmte\mbox{-}ts}{word} (ChW)}}

\wordng{PCh}{*\mbox{-}ǻmt\mbox{-}} > \wordng{Ijw}{\mbox{-}ámt\mbox{-}ik\plf{\mbox{-}ámt\mbox{-}i\mbox{-}s}}; \wordng{I’w}{\mbox{-}ámt\mbox{-}ik\plf{\mbox{-}ámt\mbox{-}e\mbox{-}s}}; \word{Mj}{\mbox{-}ámt\mbox{-}eʔ\pl{s}}{word}, \wordnl{\mbox{-}ámti(j)\mbox{-}ik}{discourse, meeting} (\citealt{ND09}: 129; \citealt{AG83}: 121; \citealt{JC18}) {\sep} \wordng{PW}{*\mbox{-}ɬ\mbox{-}ǻmet}, \textit{*\mbox{-}ɬ\mbox{-}ǻmt\mbox{-}es} > \wordng{LB}{\mbox{-}ɬ\mbox{-}omet}, \textit{\mbox{-}ɬ\mbox{-}omt\mbox{-}es}; \wordng{Vej}{\mbox{-}ɬ\mbox{-}åmet}\gloss{word}, \textit{\mbox{-}ɬ\mbox{-}åmt\mbox{-}es}\gloss{language} [1]; \wordng{’Wk}{\mbox{-}ɬ\mbox{-}ǻmet\plf{\mbox{-}ɬ\mbox{-}ǻmt\mbox{-}es}} (\citealt{VN14}: 166; \citealt{MG-MELO15}: 15; \citealt{KC16}: 70)

\dicnote{The Vejoz reflex is mistranscribed as \wordnl{\mbox{-}ɬ\mbox{-}amet}{word}, \wordnl{\mbox{-}ɬ\mbox{-}amt\mbox{-}es}{language} in \citet[65]{VU74}.}%1

\lit{\citealt{EN84}: 17, 23 (\intxt{*amthe}, 2~\intxt{*a\mbox{-}amthe})}\clearpage

\PMlemma{{\wordnl{*\mbox{-}ǻte(ʔ)}{jar} (ChW)}}

\wordng{PCh}{*\mbox{-}ǻteʔ\pla{jʰ}} > \wordng{Ijw}{\mbox{-}ate\plf{\mbox{-}ati\mbox{-}wa}} [1]; \wordng{I’w}{\mbox{-}ateʔ\pl{j}}; \wordng{Mj}{\mbox{-}ateʔ\pl{j}} (\citealt{ND09}: 129; \citealt{AG83}: 122; \citealt{JC18}) {\sep} \wordng{PW}{*<ˣj>ǻte\pla{jʰ}} [2] > \wordng{LB}{jote}; \wordng{Vej}{jate} [3]; \wordng{’Wk}{ʔijǻteʔ\pl{ç}} (\citealt{VN14}: 161, 163; \citealt{VU74}: 83; \citealt{KC16}: 43)

\dicnote{The absence of the stem-final glottal stop in the Iyojwa’aja’ reflex could be a mistranscription on \cits{ND09} part. The plural form in Iyojwa’aja’ is non-etymological.}%1

\dicnote{We have no explanation for the element \intxt{*ˣj\mbox{-}} in Wichí.}%2

\dicnote{The vowel \intxt{a} in \cits{VU74} attestation of the Vejoz reflex must be a mistranscription.}%3

\PMlemma{{\wordnl{*\mbox{-}éle(ʔ) \recind *\mbox{-}ä́le(ʔ)\pla{jʰ}}{inhabitant, inner} (ChW)}}

\word{PCh}{*\mbox{-}éleʔ\pla{jʰ}}{inhabitant, intestine} > \wordng{Ijw}{\mbox{-}έleʔ} [1]; \word{Mj}{\mbox{-}έle\mbox{-}j}{guts} (\citealt{ND09}: 130; \citealt{JC18}) {\sep} \wordng{PW}{*\mbox{-}ɬ\mbox{-}éle\pla{jʰ}} > \wordng{LB}{\mbox{-}ɬ\mbox{-}ele\pl{j}}; \word{Vej}{\mbox{-}ɬ\mbox{-}ele\pl{j}}{inhabitant}; \word{’Wk}{\mbox{-}ɬ\mbox{-}éleʔ\pl{ç}}{inhabitant, inner, tumor, sprout} (\citealt{VN14}: 154; \citealt{VU74}: 66; \citealt{MG-MELO15}: 15; \citealt{KC16}: 73)

\dicnote{In \citet[130]{ND09}, a word-final glottal stop is missing from the Iyojwa’aja’ term.}%1

\PMlemma{{\wordnl{*ɸálawuˀk}{strangler vine\species{Morrenia odorata}} (ChW)}}

\word{PCh}{*hwálok}{\textit{Morrenia odorata}, \textit{Morrenia variegata}} > \wordng{Ijw/I’w}{hwálok} (\citealt{ND09}: 133; \citealt{GS10}: 189) {\sep} \wordng{PW}{*xʷálawukʷ} > \wordng{LB}{fʷalawekʷ}; \wordng{Vej}{hʷalak} [1]; \wordng{’Wk}{xʷálawuk} (\citealt{CS08}: 60; \citealt{MS14}: 189; \citealt{MG-MELO15}: 17; \citealt{KC16}: 164)

\dicnote{The loss of the sequence \intxt{\mbox{-}wu\mbox{-}} in Vejoz is irregular. \citet[17]{MG-MELO15} state explicitly that \intxt{\mbox{-}wu\mbox{-}} is preserved in the Pilcomayeño variety.}%1

\PMlemma{{\wordnl{*[ji]ɸá(t)s’un}{to spit} (ChW)}}

\wordng{PCh}{*[ʔi]hwáts’un\mbox{-}\APPL} > \wordng{Ijw}{[ʔi]hwʲétsʲ’uhn\mbox{-}eˀn / \mbox{-}hwátsʲ’uhn\mbox{-}eˀn} [1]; \wordng{I’w}{[i]hjátsen\mbox{-} / \mbox{-}fʷátsuhn\mbox{-}en \recind \mbox{-}fʷatsen\mbox{-}} [2]; \wordng{Mj}{[ʔi]hjéts’an\mbox{-}\APPL} \textit{\recind [ʔi]hjéts’on\mbox{-}{\APPL} / \mbox{-}hwáts’an\mbox{-}\APPL} \textit{\recind \mbox{-}hwáts’on\mbox{-}\APPL} [3] (\citealt{ND09}: 99; \citealt{AG83}: 44, 129; \citealt{JC18}) {\sep} \wordng{PW}{*[ʔi]xʷáts’un} > \wordng{LB}{fʷatsen\mbox{-}katsi} [2]; \wordng{Vej}{\mbox{-}hʷats’un}; \wordng{’Wk}{[ʔi]xʷátsʼun̥} (\citealt{JB09}: 42; \citealt{VU74}: 58; \citealt{KC16}: 164)

\dicnote{The palatalization in Iyojwa’aja’ \textit{tsʲ’} is irregular.}%1

\dicnote{The plain (non-ejective) \textit{ts} in \cits{AG83} and \cits{JB09} attestations of the Iyo’awujwa’ and Lower Bermejeño forms must be a mistranscription.}%2

\dicnote{The vowel of the second syllable of the stem is unexpectedly lowered in Manjui.}%3

\PMlemma{{\wordnl{*\mbox{-}ɸél \recind *\mbox{-}ɸä́l}{to wrap, to hug, to fold, to bend} \label{dic-fel} [1] (ChW)}}

\word{PCh}{*[ʔi]k’aw\mbox{-}hwél\mbox{-}(…)\mbox{-}hop}{to hug} [2] > \wordng{I’w}{\mbox{-}<kʲa>fʷél\mbox{-}ap} [3]; \word{Mj}{[ʔi]<tʃ’e>hwέhl\mbox{-}ap / \mbox{-}<ʔa>hwέhl\mbox{-}ap}{to raise with one’s arms}; \intxt{*[ʔi]k’aw\mbox{-}hwél\mbox{-}(…)\mbox{-}eh} > \word{Mj}{[ʔi]<tʃ’e>hwέl\mbox{-}e / \mbox{-}<ʔa>hwέl\mbox{-}e}{to raise or hold with one’s arms} (\citealt{AG83}: 141; \citealt{JC18}) {\sep} \word{PW}{*[t]<tsu>xʷelʰ}{to hug, to contract one's muscles involuntarily} [4] > \word{LB}{[ta]tsefʷel}{to hug}; \word{’Wk}{[t(a)]tsúxʷel\mbox{-}\APPL}{to hug, to fight}; \intxt{*[ʔi]<qǻ>xʷ(e)l\mbox{-}{\APPL} / *[ʔi]<qǻ>xʷnh\mbox{-}\APPL} > \word{’Wk}{[ja]qǻxʷ(e)l\mbox{-}{\APPL} / [ja]qǻxʷeɬ\mbox{-}{\APPL} / [ja]qǻxʷn̥\mbox{-}\APPL}{to wrap, to fold}; \intxt{*[t]<kʲó>xʷel\mbox{-}{\APPL} / *[t]<kʲó>xʷnh\mbox{-}\APPL} > \word{’Wk}{[t(a)]kʲóxʷeɬ\mbox{-}kʲåʔ / [t(a)]kʲóxʷn̥\mbox{-}\APPL}{to be bent, curved, tortuous} (\citealt{VN14}: 248; \citealt{KC16}: 193, 303–304, 359, 386)

\dicnote{This morpheme can be alternatively described as a verbal root that requires an incorporated object or as a suffix with a highly lexical meaning.}%1

\dicnote{The Chorote reflex is a compound whose initial element is a reflex of the Proto-Mataguayan verb \wordnl{*[t]k’aw\mbox{-}\APPL}{to hold in one’s arms, to hug}.}%2

\dicnote{\cits{AG83} attestation of the Iyo’awujwa’ reflex must be a mistranscription for \intxt{\mbox{-}kʲ’afʷéhlap}.}%3

\dicnote{The dialectal reflexes with different applicatives attested in \citet[98]{RL16} show the following meanings:\gloss{to feel pain in the muscles},\gloss{to shrink when feeling cold},\gloss{to limp},\gloss{to have brucellosis}.}%4

\PMlemma{{\wordnl{*ɸílå(ˀ)X₁₂}{\textit{Solanum sp.}} (ChW)}}

\wordng{PCh}{*hwílåh} > \wordng{Ijw}{hwélʲeʔ} [1]\gloss{\textit{Solanum sp.}; \textit{Argemone subfusiformis}}; \wordng{Mj}{hwíl(ʲ)e \recind hwéil(ʲ)e}\gloss{\textit{Solanum sisymbrifolium}} (\citealt{ND09}: 133; \citealt{JC18}) {\sep} \wordng{PW}{*xʷílåχ} > \wordng{’Wk}{xʷílåx} \citep[169]{KC16}

\dicnote{The Iyojwa’aja’ reflex is entirely irregular; one would expect \intxt{*\mbox{-}hwélʲa}.}%1

\PMlemma{{\wordnl{*\mbox{-}ɸíɬä(ˀ)k}{dream}; \wordnl{*\mbox{-}ɸíɬan}{to dream} (ChW)}}

\wordng{PCh}{*\mbox{-}hwíhlek}; \intxt{*[ʔi]hwíhlan} > \wordng{Ijw}{\mbox{-}hwéhlik}, \intxt{\mbox{-}hwéhl\mbox{-}∅\mbox{-}aʔ \recind \mbox{-}hwéhl\mbox{-}ik\mbox{-}is}; \intxt{[ʔi]hwíhlʲaˀn / \mbox{-}hwéhlʲaˀn}; \wordng{I’w}{\mbox{-}fʷéhlik}; \intxt{\mbox{-}fʷéhlʲen}; \wordng{Mj}{\mbox{-}hwíhlik}; \intxt{[ʔi]hjíhlan / \mbox{-}hwíhlan} (\citealt{ND09}: 100, 119; \citealt{AG83}: 130; \citealt{GH94}; \citealt{JC18}) {\sep} \wordng{PW}{*\mbox{-}xʷíɬeq}; \intxt{*[t]xʷíɬan} > \wordng{LB}{\mbox{-}fʷiɬeq}; \intxt{[t(a)]fʷiɬan}; \wordng{Vej}{\mbox{-}hʷiɬek}, \textit{\mbox{-}hʷiɬ\mbox{-}ej}; \wordng{’Wk}{\mbox{-}xʷíɬek}, \intxt{\mbox{-}xʷíɬ\mbox{-}aç \recind \mbox{-}xʷíɬ\mbox{-}eç}; \textit{[t(a)]xʷíɬan̥} (\citealt{VN14}: 150; \citealt{VU74}: 123; \citealt{MG-MELO15}: 35; \citealt{KC16}: 61, 356)

\lit{\citealt{EN84}: 48 (\intxt{*hwehle})}

\PMlemma{{\wordnl{*\mbox{-}ɸom}{to throw, to push} (ChW)}}

\word{PCh}{*\mbox{-}hwóm\mbox{-}ah}{to push} > \wordng{I’w}{\mbox{-}fʷóm\mbox{-}a}; \wordng{Mj}{[ʔi]hjóm\mbox{-}a / \mbox{-}hwóm\mbox{-}a} (\citealt{AG83}: 130; \citealt{JC18}) {\sep} \word{PW}{*[t]xʷom}{to throw} > \wordng{LB}{[ta]fʷum\mbox{-}eχ}; \wordng{Vej}{\mbox{-}hʷom}; \wordng{’Wk}{[t(a)]xʷom̥} (\citealt{VN14}: 47; \citealt{VU74}: 59; \citealt{KC16}: 357)

\largerpage
\gc{\citet[304]{PVB13a} compares the verb with \word{Proto-Guaicuruan}{*\mbox{-}aˀm\mbox{-}áqa}{to push} (\citealt{PVB13b}, \#46), which could be spurious.}

\lit{\citealt{PVB13a}: 304 (\wordnl{*\mbox{-}hʷʌm}{to push})}\clearpage

\PMlemma{{\wordnl{*\mbox{-}ɸólXaˀn}{ankle} [1] (ChW only)}}

\wordng{PCh}{*\mbox{-}hwóhlaˀn} > \wordng{Mj}{\mbox{-}hwóhlaˀn} (\citealt{JC18}) {\sep} \wordng{PW}{*\mbox{-}xʷónhaˀn} > \wordng{Guisnay}{\mbox{-}hʷon̥an\plf{\mbox{-}hʷon\mbox{-}lis}} \citep[33]{RL16}

\dicnote{This is a likely derivative of \word{PM}{*\mbox{-}ɸo(ʔ) \recind *\mbox{-}ɸó(ʔ)}{foot}.}%1

\PMlemma{{\intxt{*ɸ(u)nájXV(ˀ)j} [1]\gloss{earthworm, amphisbaenian} (ChW)}}

\wordng{PCh}{*ʔᵊhnáhjåjʔ} > \word{Ijw}{ʔihnáhjaʔ\plf{ʔihnáhjaj\mbox{-}is}}{earthworm\species{Pheretima hawayana}}; \wordng{Mj}{̀ʔihnʲéhejʔ} [2] (\citealt{ND09}: 98; \citealt{GH94}) {\sep} \wordng{PW}{*xʷunájxij} > \wordng{LB}{fʷinan̥ij \recind fʷinaɲ̊aj} [3]\gloss{earthworm}; \wordng{Vejoz or Guisnay}{hunaçi\pl{lis}} [4]\gloss{earthworm}; \wordng{’Wk}{xʷunáhiʔ} [5] (\citealt{diwica}; \citealt{RL16}: 39; \citealt{KC16}: 176)

\dicnote{It is unclear whether this etymon should be reconstructed with a stem-initial consonant cluster (assuming epenthesis in Wichí) or with \sound{PM}{*u} (assuming syncope in Chorote). Regarding the vowel of the stem-final syllable, Iyojwa'aja' points to \sound{PM}{*å}, most Wichí varieties to \intxt{*i}, and one dialectal reflex to \intxt{*a}.}%1

\dicnote{\sound{Manjui}{h} is not the expected reflex of \sound{PCh}{*hj}.}%2

\dicnote{The forms attested in \citet{diwica} are irregular reflexes of \wordng{PW}{*xʷunájxij}. One would expect \intxt{*fʷenaɲ̊ij}.}%3

\dicnote{The Vejoz or Guisnay form attested in \citet{RL16} shows an irregular development of \sound{PW}{*xʷ} and an irregular loss of the stem-final \intxt{*j}. One would expect the reflex \intxt{*hʷunaçij}.}%4

\dicnote{The ’Weenhayek reflex attested in \citet{KC16} shows an irregular loss of both instances of \sound{PW}{*j}. One would expect the reflex \intxt{*xʷunáçijʔ}.}%5

\PMlemma{{\wordnl{*[ji]ˀjáXin}{to watch} (ChW)}}

\wordng{PCh}{*[ʔi]ˀjáan} > \wordng{Ijw}{[ʔi]ˀjéˀn}; \wordng{I’w}{\mbox{-}jén\mbox{-}a} [1]\gloss{to look after}, \wordnl{\mbox{-}jén\mbox{-}e}{to spy}; \wordng{Mj}{[ʔi]ˀjéen} (\citealt{JAA12b}: 89; \citealt{ND09}: 118; \citealt{AG83}: 134; \citealt{JC18}) {\sep} \wordng{PW}{*[ʔi]jáhin}, imp. \textit{jáhin} > \wordng{LB}{[ʔi]jahin}, imp. \textit{jahin}; \wordng{Vej}{\mbox{-}jahen} [2]; \wordng{’Wk}{[ʔi]jáhin̥}, imp. \textit{jáhin̥} (\citealt{VN14}: 148, 177; \citealt{VU74}: 82; \citealt{KC16}: 521)

\dicnote{The seemingly plain \intxt{j} in Iyo’awujwa’ could be a mistranscription on \cits{AG83} part.}%1

\dicnote{\sound{Vejoz}{e} is not the regular reflex of \sound{PW}{*i}.}%2

\PMlemma{{\wordnl{*jiˀno\plf{*jiˀnó\mbox{-}l}}{man} (ChW)}}

\wordng{PCh}{*ʔiˀnóʔ\pla{l}}\gloss{man, person} > \wordng{Ijw}{ʔiˀnʲóʔ\pl{ˀl}}; \wordng{I’w}{inʲóʔ}\gloss{person}; \wordng{Mj}{ʔiˀn(ʲ)óʔ\pl{l}} (\citealt{JC10}: 100; \citealt{ND09}: 117; \citealt{AG83}: 131; \citealt{JC18}) {\sep} \wordng{PW}{*hiˀno}, \textit{*hiˀnó\mbox{-}lʰ} > \wordng{LB}{hiˀnu\pl{ɬ}}; \wordng{Vej}{hiˀno} [1]; \wordng{’Wk}{hiˀno}, \textit{hiˀnó\mbox{-}ɬ} (\citealt{VN14}: 191, 196; \citealt{MG-MELO15}: 12; \citealt{KC16}: 148)

\largerpage
\dicnote{\citet[57]{VU74} mistranscribes the word as \textit{hino}.}%1

\lit{\citealt{EN84}: 13, 16 (\intxt{*iˀhnɔ}); \citealt{PVB02}: 144 (\intxt{*χinoʔ})}
\clearpage

\PMlemma{{\wordnl{*káˀlah\plf{*káˀla\mbox{-}ts}}{lizard} (ChW)}}

\wordng{PCh}{*káˀlah}, \textit{*káˀla\mbox{-}s} > \wordng{Ijw}{kʲéˀla}; \wordng{I’w/Mj}{kʲéˀla\pl{s}} (\citealt{ND09}: 135; \citealt{AG83}: 142; \citealt{JC18}) {\sep} \wordng{PW}{*kʲáˀlah}, \textit{*kʲáˀla\mbox{-}s} > \wordng{LB}{tʃaˀla}; \wordng{Vej}{tʃala} [1]; \wordng{’Wk}{kʲáˀlah}, \textit{kʲáˀla\mbox{-}s} (\citealt{VN14}: 123; \citealt{VU74}: 51; \citealt{MG-MELO15}: 20; \citealt{KC16}: 185)

\dicnote{The sound change \intxt{*ˀl} > \intxt{l} in Vejoz is irregular.}%1

\rej{\word{Ni}{kak͡låˀmatax}{gray iguana} \citep[57]{JS16} is very similar to \wordng{PM}{*káˀlah}, but cannot be a reflex thereof for phonological reasons (one would expect \intxt{*kaˀk͡la}). Formally, it could be a compound of \wordnl{\mbox{-}kak͡låʔ}{leg}, \wordnl{\mbox{-}ˀmat}{physical defect}, and \wordnl{\mbox{-}tax}{similar to}.}

\lit{\citealt{EN84}: 47 (\intxt{*cɛla}); \citealt{LC-VG-07}: 17}

\PMlemma{{\wordnl{*[ji]kǻ(ˀ)t}{to be red} (ChW)}}

\wordng{PCh}{*[ʔi]kǻt} > \wordng{Ijw}{[ʔi]sʲát}; \wordng{I’w}{[ʔi]sʲát \recind [ʔi]sʲét}; \wordng{Mj}{[ʔi]ʃét / \mbox{-}kʲét} (\citealt{JC14b}: 76; \citealt{ND09}: 110; \citealt{AG83}: 132; \citealt{JC18}) {\sep} \wordng{PW}{*[ʔi]kʲǻt} > \wordng{LB}{[ʔi]tʃot}; \wordng{Vej}{\mbox{-}tʃåt}; \wordng{’Wk}{<ʔi>kʲǻt} [1] (\citealt{VN14}: 312; \citealt{JB09}: 40; \citealt{VU74}: 52; \citealt{MG-MELO15}: 42; \citealt{KC16}: 27)

\dicnote{The third-person prefix \intxt{ʔi\mbox{-}} has fossilized to the root in ’Weenhayek.}%1

\lit{\citealt{EN84}: 22 (\third{*j\mbox{-}cåt})}

\PMlemma{{\wordnl{*[ji]kåʔ}{to be torn} (ChW)}}

\wordng{PCh}{*[ʔi]kǻʔ} > \wordng{Ijw}{[ʔi]sʲáʔ / \mbox{-}kʲáʔ}; \wordng{I’w}{\mbox{-}kʲéʔe}; \wordng{Mj}{[ʔi]ʃéʔ / \mbox{-}kʲéʔ} (\citealt{ND09}: 110; \citealt{AG83}: 141; \citealt{JC18}) {\sep} \wordng{PW}{*[ʔi]kʲåʔ} > \wordng{LB}{[ʔi]tʃoʔ}; \wordng{’Wk}{[ʔi]kʲåʔ} (\citealt{VN14}: 237; \citealt{KC16}: 27)

\PMlemma{{\wordnl{*\mbox{-}kéjå(ʔ)}{granddaughter}; \wordnl{*\mbox{-}kéjåts}{grandson}; \wordnl{*\mbox{-}ké(j)tså\mbox{-}ts}{grandchildren} (ChW)}}

\wordng{PCh}{*\mbox{-}kéjåʔ}; \intxt{*\mbox{-}kéjås}; \intxt{*\mbox{-}kéjtsås} [1] > \wordng{Ijw}{\mbox{-}kíjaʔ}; \intxt{\mbox{-}kíjas}; \intxt{\mbox{-}kítʃas}; \wordng{I’w}{—}; \intxt{\mbox{-}kíjas \recind \mbox{-}kíjes}; \intxt{—}; \wordng{Mj}{\mbox{-}kíjeʔ}; \intxt{\mbox{-}kíjes}; \intxt{\mbox{-}kíʃes} (\citealt{JC14b}: 122; \citealt{ND09}: 122; \citealt{AG83}: 139, 210; \citealt{JC18}) {\sep} \wordng{PW}{*\mbox{-}kʲéjå}; \intxt{*\mbox{-}kʲéjås}; \intxt{*\mbox{-}kʲétsås} > \wordng{LB}{\mbox{-}tʃejo}; \intxt{\mbox{-}tʃejos}; —; \wordng{Vej}{\mbox{-}tʃejå}; \intxt{\mbox{-}tʃejås}; \intxt{\mbox{-}tʃetsos} [1]; \wordng{’Wk}{\mbox{-}kʲéjåʔ}; \intxt{\mbox{-}kʲéjås}; \intxt{\mbox{-}kʲétsås} (\citealt{VN14}: 194; \citealt{MG-MELO15}: 29; \citealt{KC16}: 64, 65)

\dicnote{The cluster \wordng{PCh}{*ts} is reconstructed based on the Iyojwa’aja’ reflex with an affricate. Note that Chorote has no affricate /ts/, suggesting that we are dealing here with a cluster composed of /t/ and /s/.}%1

\dicnote{The Vejoz reflexes are mistranscribed in \citet[52]{VU74}, who gives \intxt{\mbox{-}tʃeja} and \intxt{\mbox{-}tʃejas} for the former two items (the plural is not attested). Note that the vowel \intxt{o} in \intxt{\mbox{-}tʃetsos} is not the regular reflex of \wordng{PW}{*å}.}%2

\lit{\citealt{EN84}: 49 (\intxt{*c’ɛjås}\gloss{grandson})}

\PMlemma{{\wordnl{*(\mbox{-})késoj \recind *(\mbox{-})kä́soj}{skin disease} (ChW)}}

\wordng{PCh}{*\mbox{-}késoj} > \word{Ijw}{\mbox{-}kíso\pl{ˀl}}{acne}; \wordng{I’w}{\mbox{-}kíxsoj} (\citealt{ND09}: 122; \citealt{AG83}) {\sep} \wordng{PW}{*\mbox{-}kʲésoj} > \word{Vejoz or Guisnay}{tʃesoj}{scabies; kind of leguminous plant with edible roots whose leaves burn one’s skin} (\citealt{RL16}: 21)

\PMlemma{{\wordnl{*kójXa(ˀ)t}{to be heavy} (ChW)}}

\wordng{PCh}{*kóhjat\mbox{-}\APPL} > \wordng{Ijw}{kʲóhjet\mbox{-}i}; \wordng{I’w}{[a]kʲówiht\mbox{-}iʔ \recind kʲóhje(h)t\mbox{-}iʔ}; \wordng{Mj}{kʲóhjiht\mbox{-}ijʔ} (\citealt{ND09}: 136; \citealt{AG83}: 78, 143, 214; \citealt{JC18}) {\sep} \wordng{PW}{*kʲójhat} > \wordng{LB}{ni\mbox{-}tʃuɲ̊at}; \wordng{Vej}{\mbox{-}tʃoɲ̊at} [1]; \wordng{’Wk}{kʲóçet} [2] (\citealt{JB09}: 53; \citealt{MG-MELO15}: 62; \citealt{KC16}: 196)

\dicnote{\citet[115]{VU74} mistranscribes the Vejoz reflex as \intxt{tʃojnjat}.}%1

\dicnote{’Weenhayek \intxt{e} is not the regular reflex of \sound{PW}{*a}.}%2

\PMlemma{{\wordnl{*kóˀl}{locust} (ChW)}}

\wordng{PCh}{*kóˀl} > \wordng{Ijw}{kʲóˀl}; \wordng{I’w}{kʲól}; \wordng{Mj}{kʲóˀl\plf{kʲól\mbox{-}is}} (\citealt{ND09}: 136; \citealt{AG83}: 143; \citealt{JC18}) {\sep} \wordng{PW}{*kʲólʰ} > \wordng{LB}{tʃuɬ}; \wordng{Vej}{tʃoɬ}; \wordng{’Wk}{kʲóɬ} (\citealt{VN14}: 51; \citealt{VU74}: 53; \citealt{MG-MELO15}: 20; \citealt{KC16}: 193)

\citealt{EN84}: 52 (\textsc{pl} \intxt{*cɔl\mbox{-}s})

\PMlemma{{\intxt{*kowäˀx / *\mbox{-}kówäˀx} [1]\gloss{hole} (ChW)}}

\wordng{PCh}{*kowéh / *\mbox{-}kóweh} > \wordng{Ijw}{\mbox{-}kʲówe\plf{\mbox{-}kʲóhw\mbox{-}aˀl}} [2]\gloss{center, inner part}; \word{I’w}{\mbox{-}kʲówe}{in the middle of}; \wordng{Mj}{kʲowéh\plf{kʲowé\mbox{-}jh} / \mbox{-}kʲówe} (\citealt{ND09}: 122; \citealt{AG83}: 143; \citealt{JC18}) {\sep} \wordng{PW}{*kʲoweχ / *\mbox{-}kʲóweχ} > \word{LB}{tʃuweχ}{in the middle of}; \word{Vej}{tʃoweh}{well}; \wordng{’Wk}{kʲowex\plf{kʲoʍ\mbox{-}áç} / {-}kʲówex\plf{\mbox{-}kʲóʍ\mbox{-}aç}} (\citealt{VN14}: 276; \citealt{MG-MELO15}: 48; \citealt{KC16}: 194)

\dicnote{This term is likely an obscure compound, with \wordng{PM}{*\mbox{-}wä́ˀx} as its second part.}%1

\dicnote{The Iyojwa’aja’ plural form is non-etymological.}%2

\PMlemma{{\wordnl{*kpéna(ˀ)X₁₂ \recind *kpä́na(ˀ)X₁₂\plf{*kpénX₁₃a\mbox{-}ts \recind *kpä́nX₁₃a\mbox{-}ts}}{orphan} (ChW)}}

\wordng{PCh}{*k<em>pénah\plf{*k<em>pénha\mbox{-}s}} [1] > \wordng{Ijw}{kimpέna\plf{kimpέhna\mbox{-}s}}; \wordng{I’w}{\mbox{kimpéna}\pl{s}}; \wordng{Mj}{kilpέna} [2] (\citealt{ND09}: 136; \citealt{AG83}: 140, 202) {\sep} \wordng{PW}{*kpénaχ\plf{*kpénha\mbox{-}s}} > \wordng{Guisnay}{tʃipenah} [2]; \wordng{’Wk}{pénax\plf{pén̥a\mbox{-}s}} \citep[292]{KC16}

\dicnote{We have no explanation for the element \intxt{*\mbox{-}em\mbox{-}} in Chorote (which irregularly yields \intxt{\mbox{-}il\mbox{-}} in Manjui).}%1

\dicnote{\citet[21, 73]{RL16} documents the variant \intxt{penah} alongside \intxt{tʃipenah} in Wichí, but does not indicate the dialectal procedence of these variants (his dictionary includes Vejoz and Guisnay forms). Since Vejoz is otherwise known to simplify word-initial consonant clusters composed of two stops, we surmise that the variant \intxt{tʃipenah} is Guisnay.}%2

\PMlemma{{\wordnl{*ktáˀnih\plf{*ktáˀni\mbox{-}ts}}{Chaco tortoise} (ChW)}}

\wordng{PCh}{*kitáˀnih\plf{*kitáˀni\mbox{-}s}} > \wordng{I’w}{kitʲéneʔ\plf{kitʲéni\mbox{-}s}} [1]; \wordng{Mj}{kitíˀni \recind kitíˀnʲe\pl{s}} (\citealt{AG83}: 140; \citealt{JC18}) {\sep} \wordng{PW}{*kʲtáˀnih} > \wordng{LB}{tʃitaˀni}; \wordng{Vej}{taˀni\pl{ɬajis}}; \wordng{’Wk}{táˀnih} (\citealt{VN14}: 52, 231; \citealt{MG-MELO15}: 22; \citealt{KC16}: 346)

\dicnote{The plain reflex of \sound{PCh}{*ˀn} in Iyo’awujwa’ as attested by \citet{AG83} must be a mistranscription.}%1

\rej{\citet[22, 51]{EN84} compares the Chorote word with \word{Ni}{tʃ’at’a\pl{s}}{Chaco tortoise} \citep[110]{JS16} and reconstructs \intxt{*cɛthán}. We reject this possibility; the expected reflex of \wordng{PM}{*ktáˀnih} in that language would actually be \intxt{*ktaˀni}.}

\PMlemma{{\intxt{*ktéta(ʔ) \recind *ktä́ta(ʔ)} (fruit); \intxt{*ktéta\mbox{-}(ju)k \recind *ktä́ta\mbox{-}juk} (tree)\gloss{\textit{Prosopis elata}} (ChW)}}

\wordng{PCh}{*kitétaʔ}; \intxt{*kitéta\mbox{-}k\plf{*kitéta\mbox{-}kʲu\mbox{-}jʰ}} > \wordng{Ijw}{kitíta\mbox{-}k}; \wordng{Mj}{kitítaʔa\pl{s}}; \intxt{kitíta\mbox{-}k\plf{kitíta\mbox{-}ku\mbox{-}j}} (\citealt{ND09}: 136; \citealt{JC18}) {\sep} \wordng{PW}{*kʲtéta}; \intxt{*kʲtéta\mbox{-}k} > \wordng{Southeastern (Salta)}{tʃiteta}; \intxt{tʃitete\mbox{-}k}; \wordng{’Wk}{tétaʔ}; \intxt{téta\mbox{-}k} (\citealt{MS14}: 291; \citealt{KC16}: 396)

\PMlemma{{\intxt{*kutsá(ˀ)X₁₂ \recind *kutsé(ˀ)χ \recvar *k’utsá(ˀ)X₁₂ \recind *k’utsé(ˀ)χ} [1]\gloss{cháguar\species{Bromelia hieronymi}} (ChW)}}

\wordng{PCh}{*k’usáh} > \wordng{Ijw}{k’iséh}; \wordng{I’w}{isáh\pl{as}}; \wordng{Mj}{ʔisáh} (\citealt{ND09}: 137; \citealt{AG83}: 131; \citealt{JC18}) {\sep} \wordng{PW}{*kʲutsáχ} > \wordng{LB}{tʃitsaχ} [2]; \wordng{Vej}{tʃutsah}; \wordng{’Wk}{\mbox{kutsáx}} [3] (\citealt{CS08}: 59; \citealt{VU74}: 53; \citealt{MG-MELO15}: 17; \citealt{KC16}: 178)

\dicnote{The Chorote form points to \sound{PM}{*k’}, and the Wichí one to \sound{PM}{*k}.}%1

\dicnote{\sound{LB}{i} is not the expected reflex of \sound{PW}{*u}.}%2

\dicnote{The unpalatalized \intxt{k} in the ’Weenhayek form is entirely irregular.}%3

\rej{\citet[26]{EN84} compares the Wichí reflex with the reflexes of \word{PW}{*[hi]kʲ’út}{old}, \word{Ni}{k’utsaˀx}{old}, and \word{Ijw}{kʲút}{little owl} \citep[90]{RJH15}, which cannot be related for phonological and/or semantic reasons.}

\PMlemma{{\intxt{*\mbox{-}kV́nt(’)…} [1]\gloss{kidney}}}

\wordng{PCh}{*\mbox{-}kánt’ijaaʔ} > \wordng{Ijw}{\mbox{-}kʲént’ijeʔ\pl{jis}}; \wordng{I’w}{\mbox{-}kʲéntijeʔ\pl{jis}}; \wordng{Mj}{\mbox{-}kʲént’ijeeʔ\pl{l}} (\citealt{ND09}: 122; \citealt{AG83}: 142; \citealt{JC18}) {\sep} \wordng{PW}{*\mbox{-}kʲóntowaj}\gloss{kidney} > \wordng{Vej}{\mbox{-}tʃontowaj}; \wordng{’Wk}{\mbox{-}kʲóntowajʔ\plf{\mbox{-}kʲóntowa\mbox{-}lis}} (\citealt{VU74}: 53; \citealt{KC16}: 65)

\dicnote{The correspondences between Chorote and Wichí are so irregular that it is impossible to reconstruct the protoform.}%1

\PMlemma{{\wordnl{*\mbox{-}k’aló(ʔ)\pla{ts}}{cheek} (ChW)}}

\wordng{PCh}{*\mbox{-}k’alóʔ\pla{ts}} > \wordng{Ijw}{\mbox{-}kʲ’óloʔ\pl{s}} [1]; \wordng{I’w}{\mbox{-}kʲalóʔ\pl{s}} [2]; \wordng{Mj}{\mbox{-}(ʔʲe)lɔ́ʔ\pl{s}} (\citealt{ND09}: 123; \citealt{AG83}: 141; \citealt{JC18}) {\sep} \wordng{PW}{*\mbox{-}kʲ’álo\pla{s}} > \wordng{LB}{\mbox{-}tʃ’alu}; \wordng{Vej}{\mbox{-}tʃ(’)alo\pl{s}}; \wordng{’Wk}{\mbox{-}kʲ’áloʔ\pl{s}} (\citealt{VN14}: 48; \citealt{VU74}: 54; \citealt{MG-MELO15}: 60; \citealt{KC16}: 67)

\dicnote{The Iyojwa’aja’ reflex is entirely irregular; one would expect \intxt{*\mbox{-}kʲ’elɔ́ʔ\pla{s}}.}%1

\dicnote{The plain reflex of \wordng{PCh/PW}{*kʲ’} in Iyo’awujwa’ and Vejoz as attested by \citet{AG83} and \citet[60]{MG-MELO15} is unexpected.}%2

\rej{\citet[35, 45]{EN84} lists \word{Ni}{\mbox{-}kuʔ\pl{l}}{cheek} as a member of this cognate set, but not a single segment of this root shows any regular correspondence with the Chorote and Wichí roots listed here.}

\lit{\citealt{EN84}: 35, 37, 45 (\intxt{*cǻlɔ}; \wordnl{*cålɔncɛ}{jaw}); \citealt{LC-VG-07}: 16}

\PMlemma{{\wordnl{*\mbox{-}k’óX₂₃te(ʔ)\pla{jʰ}}{ear} (ChW)}}

\wordng{PCh}{*\mbox{-}k’óoteʔ\pla{jʰ}} > \wordng{Ijw}{\mbox{-}kʲ’óteʔ} [1]; \wordng{I’w}{\mbox{-}kʲóteʔ\pl{j}} [2]; \wordng{Mj}{\mbox{-}ʔʲóoteʔ\pl{jh}} (\citealt{ND09}: 123; \citealt{AG83}: 143, 211; \citealt{JC18}) {\sep} \wordng{PW}{*\mbox{-}kʲ’óte\pla{jʰ}} > \wordng{LB}{\mbox{-}tʃ’ute\pl{j}}; \wordng{Vej}{\mbox{-}tʃ’ote}; \wordng{’Wk}{\mbox{-}kʲ’óteʔ\pl{ç}} (\citealt{VN14}: 112, 164; \citealt{JB09}: 40; \citealt{VU74}: 54; \citealt{MG-MELO15}: 29; \citealt{KC16}: 68)

\dicnote{The Iyojwa’aja’ word is mistranscribed as \intxt{\mbox{-}kʲ’óte} in \citet{ND09}.}%1

\dicnote{The plain reflex of \sound{PCh}{*k’} in Iyo’awujwa’ as attested by \citet{AG83} must be a mistranscription.}%2

\gc{Possibly related to \word{Proto-Guaicuruan}{*\mbox{-}k’et’élV}{ear} (\citealt{PVB13b}, \#341; cf. \citealt{PVB13a}: 309).}

\lit{\citealt{EN84}: 16, 44 (\intxt{*c’otɛ}); \citealt{PVB13a}: 309 (\intxt{*\mbox{-}k’ote})\gloss{ear}}

\PMlemma{{\wordnl{*[ji]lǻ(ˀ)t}{to feel} (ChW)}}

\wordng{PCh}{*[ʔi]lǻt\mbox{-}ejʰ} > \wordng{Ijw}{[ʔi]lʲát\mbox{-}e / \mbox{-}lát\mbox{-}e}; \wordng{Mj}{[ʔi]lʲét\mbox{-}ej / \mbox{-}lát\mbox{-}ej} (\citealt{ND09}: 101; \citealt{JC18}) {\sep} \wordng{PW}{*[ʔi]lǻt} > \wordng{LB}{[ʔi]lot}; \word{Vej}{\mbox{-}låt}{to hear}; \wordnl{[hi]låt\mbox{-}e}{to smell}; \wordng{’Wk}{[ʔi]lǻt} (\citealt{VN14}: 315; \citealt{VU74}: 64; \citealt{MG-MELO15}: 35; \citealt{KC16}: 213)

\PMlemma{{\wordnl{*níltsa(ˀ)X₁₂\plf{*níltsX₁₃a\mbox{-}ts}}{white-lipped peccary} (ChW)}}

\wordng{PCh}{*<ʔih>nílsah\plf{*<ʔih>nílsa\mbox{-}s}} [1] > \wordng{Ijw}{ʔihnílsʲe}; \wordng{I’w}{ihníxsa\mbox{-}tók\plf{ihníxsa\mbox{-}s\mbox{-}tó\mbox{-}ji}}; \wordng{Mj}{ʔihnílsa} (\intxt{\mbox{-}s \recind \mbox{-}∅}) (\citealt{ND09}: 98; \citealt{AG83}: 132; \citealt{JC18}) {\sep} \wordng{PW}{*nítsaχ\plf{*nítsha\mbox{-}s}} > \wordng{LB}{nitsaχ}; \wordng{Vej}{nitsah}; \wordng{’Wk}{nítsax\plf{nítsʰa\mbox{-}s}} (\citealt{JB09}: 52; \citealt{VU74}: 68; \citealt{MG-MELO15}: 21; \citealt{KC16}: 273)

\dicnote{We have no explanation for the element \intxt{*ʔih\mbox{-}} in Chorote.}%1

\rej{\citet{EN84} compares the reflexes of \wordng{PW}{*nítsaχ} with the Nivaĉle term for\gloss{wild cavy} (\intxt{tʃaxani}) and the Chorote term for\gloss{Chacoan peccary} or\gloss{collared peccary} (\wordng{Ijw}{kíhnʲe}, \wordng{I’w}{kíhnije\pl{s}}, \wordng{Mj}{kíhnʲeʔe\pl{s}}), which are poor matches from both the phonological and semantic points of view.}

\PMlemma{{\wordnl{*n̩tå(ˀ)k}{two} (ChW)}}

\wordng{PCh}{*n̩tǻk} > \wordng{I’w}{n̩ták}; \wordng{Mj}{inták} (\citealt{AG83}: 152; \citealt{JC18}) {\sep} \word{PW}{*nitåkʷ}{two, many} > \wordng{LB}{nitokʷ}\gloss{many}; \word{Vej}{nitåkʷ}{many} (> 4)’ [1]; \word{’Wk}{nitåkʷ}{two, many} (\citealt{VN14}: 356; \citealt{MG-MELO15}: 27; \citealt{KC16}: 271)

\dicnote{\citet[74]{VU74} documents \word{Vej}{\mbox{-}takʷ}{two}, which must be the same word.}%1

\lit{\citealt{EN84}: 39 (\intxt{*tawk})}

\PMlemma{{\intxt{*[tᵊ]nxát’itsaXan} [1]\gloss{to sneeze} (ChW)}}

\wordng{PCh}{*[tᵊ]hnát’isaan} > \wordng{Ijw}{[ti]hnʲét’isʲeˀn / \mbox{-}hnát’isʲeˀn}; \wordng{I’w}{\mbox{-}hnátisʲen} [2]; \wordng{Mj}{[ʔi]hnʲéʔiʃeen / \mbox{-}hnáʔiʃeen \recind \mbox{-}hnáʔaʃeen} [3] (\citealt{JC14a}; \citealt{AG83}; \citealt{JC18}) {\sep} \wordng{PW}{*[t]náʔtsan \recvar *[t]náʔtshan} [4] > \wordng{LB}{[ta]naʔtsan}; \wordng{’Wk}{náʔtsʰan̥} (\citealt{VN14}: 157; \citealt{KC16}: 253)

\dicnote{The reconstruction is tentative. We assume that the element \intxt{*\mbox{-}nxá\mbox{-}} is identical to the PM root \wordnl{*\mbox{-}naˀx \recind *\mbox{-}náˀx\plf{*\mbox{-}nxá\mbox{-}ts}}{nose}. A similar root is found in Maká and Nivaĉle (see \intxt{*[t]’aqsin \recvar *[t]’aq’asin} in \sectref{mnonly}), but the correspondences are entirely irregular. We have also contemplated the possibility that the correct reconstruction is \intxt{*[tᵊ]nxáq’isaXan}, which would be more similar to the Maká and Nivaĉle forms and could account for the otherwise irregular Manjui reflex, but such a decision would require to posit additional irregular developments for Iyojwa’aja’, Iyo’awujwa’, and Wichí.}%1

\dicnote{The plain stop \intxt{t} in \cits{AG83} attestation of the Iyo’awujwa’ reflex, as opposed to an ejective stop \intxt{t’}, must be a mistranscription.}%2

\dicnote{Manjui has irregularly debuccalized the ejective stop \intxt{*t’} and shows an optional translaryngeal harmony.}%3

\dicnote{Wichí has irregularly lost the PM guttural fricative. It also shows an irregular syncope of the vowels in the medial syllables. The Lower Bermejeño Wichí reflex points to \intxt{*[t]náʔtsan}, the ’Weenhayek one to \intxt{*\mbox{-}náʔtshan}. \citet[65]{RL16} attested the surprising forms \intxt{\mbox{-}nektsʰan} and \intxt{\mbox{-}naktsʰan}, but does not indicate their dialectal procedence.}%4

\PMlemma{{\wordnl{*[j]ókɸe(ˀ)(t)s \recind *[j]ókɸä(ˀ)(t)s \recind *[j]ékɸe(ˀ)(t)s \recind *[j]ékɸä(ˀ)(t)s}{frighten away [animals]} (ChW)}}

\wordng{PCh}{*[j]ókwes} > \wordng{Ijw}{[j]ókʲos / \mbox{-}ɔ́kʲos}; \wordng{Mj}{[j]ókes / \mbox{-}ɔ́kes} (\citealt{ND09}: 161; \citealt{JC18}) {\sep} \wordng{PW}{*[j]ókʷes} > \wordng{’Wk}{[j]ókes} \citep[551]{KC16}

\PMlemma{{\wordnl{*\mbox{-}pák’o\pla{l}}{heel} (ChW) [1]}}

\wordng{PCh}{*\mbox{-}pók’oʔ\pla{l}} [2] > \wordng{Ijw}{\mbox{-}pɔ́kʲ’oʔ\pl{ˀl}}; \wordng{I’w}{\mbox{-}pókʲ’oʔ\pl{l}}; \wordng{Mj}{\mbox{-}pɔ́ʔoʔ\pl{l}} (\citealt{ND09}: 125; \citealt{AG83}: 156; \citealt{JC18}) {\sep} \word{PW}{*\mbox{-}pákʲ’o\pla{lʰ}}{foot} > \wordng{LB}{\mbox{-}patʃ’u\pl{ɬ}}; \wordng{Vej}{\mbox{-}patʃ’o\pl{ɬ}} [3]; \wordng{’Wk}{\mbox{-}pákʲ’oʔ\pl{ɬ}} (\citealt{VN14}: 201; \citealt{VU74}: 69; \citealt{MG-MELO15}: 61; \citealt{KC16}: 79)

\dicnote{This is obviously a fossilized compound of an unidentified root \intxt{*\mbox{-}pa\mbox{-}} and \word{PM}{*\mbox{-}k’o\plf{*\mbox{-}k’ó\mbox{-}l}}{bottom}.}%1

\dicnote{Chorote has apparently undergone irregular vowel harmony.}%2

\dicnote{The glottalization of the stem-medial consonant is missing in \citet[61]{MG-MELO15}.}%3

\rej{\citet[55]{EN84} lists \word{Ni}{\mbox{-}p’ik’o}{heel} under this etymology. We regard it as a fossilized compound whose second element is also \word{PM}{*\mbox{-}k’o\plf{*\mbox{-}k’ó\mbox{-}l}}{bottom}, but whose first element is a cognate of \word{Maká}{\mbox{-}f’iʔ}{foot} (thus \intxt{\mbox{-}p’i\mbox{-}k’o} < \intxt{*\mbox{-}ɸ’i\mbox{-}k’o}).}

\lit{\citealt{EN84}: 36, 45, 55 (\intxt{*pácɔ}, 2~\intxt{*a\mbox{-}pácɔ}, \textsc{pl}~\textit{*pac’ɔl})}

\PMlemma{{\intxt{*pǻ(ˀ)x \recind *pǻ(ˀ)χ \recvar *pá(ˀ)x \recind *pä́(ˀ)x \recind *pá(ˀ)χ \recind *pé(ˀ)χ} [1]\gloss{to pass (of time), to be soon} (ChW)}}

\wordng{PCh}{*pǻh} > \wordng{Ijw}{páh}, \textsc{caus}~\intxt{[ʔi]pʲáh\mbox{-}anit}; \word{Mj}{[ʔa]páh}{to be ancient, to spend a lot of time doing something} (\citealt{ND09}: 109, 142; \citealt{JC18}) {\sep} \word{PW}{*páχ}{to take time}, \wordnl{*(\mbox{-})paχ(\mbox{-})}{deictic root found in temporal adverbs} > \word{LB}{paχ}{later}; \word{’Wk}{páx}{to take time} \textsc{caus}~\intxt{[ʔi]pá\mbox{-}nit\mbox{-}ex}, \intxt{(\mbox{-})pax(\mbox{-})} (\citealt{VN14}: 342–343; \citealt{KC16}: 288–289)

\dicnote{The Iyojwa’aja’ causative points to \sound{PM}{*ǻ}, and the Wichí reflex points to \intxt{*á}, \intxt{*ä́}, or \intxt{*é}. If \word{Mk}{paʔax}{a long time ago} \citep[294]{AG99} is shown to be related, the original vowel should be reconstructed as \intxt{*ǻ}, with an irregular evolution in Wichí.}%1

\PMlemma{{\wordnl{*pǻˀjih}{frog\species{Leptodactylus sp.}} (ChW)}}

\wordng{PCh}{*pǻˀjih} > \wordng{Ijw}{páˀji} (\intxt{\mbox{-}his}); \wordng{I’w}{páji} [1]; \wordng{Mj}{páʔi \recind páˀji} (\intxt{\mbox{-}wa \recind \mbox{-}∅}) (\citealt{ND09}: 143; \citealt{AG83}: 154; \citealt{JC18}) {\sep} \wordng{PW}{*pǻˀjih} > \wordng{LB}{poˀji}; \wordng{Vej}{påˀji} [2]; \wordng{’Wk}{pǻˀjih} (\intxt{\mbox{-}lis \recind \mbox{-}ɬajis}) (\citealt{VN14}: 47; \citealt{MG-MELO15}: 22; \citealt{KC16}: 284)

\dicnote{The plain reflex of \wordng{PCh}{*ˀj} in Iyo’awujwa’ as attested by \citet{AG83} must be a mistranscription.}%1

\dicnote{\citet[70]{VU74} mistranscribes the root as \intxt{paˀji}.}%2

\rej{It is tempting to include \word{Mk}{paχpajeʔ\pl{l}}{a tiny frog\species{Melanophryniscus fulvoguttatus}} in this cognate set, but the expected reflex of \wordng{PM}{*pǻˀjih} in Maká would be \intxt{*paˀjiʔ}, making the comparison dubious.}

\lit{\citealt{EN84}: 12, 17 (\intxt{*pa(\mbox{-})i})}

\PMlemma{{\wordnl{*på(ˀ)q}{kind of \textit{zorzal}\species{Turdus sp.}} (ChW)}}

\wordng{PCh}{*pǻq} > \word{Ijw}{pák\mbox{-}hitʲok}{creamy-bellied thrush\species{Turdus amaurochalinus}}; \word{Mj}{pák}{bird sp.} (\citealt{ND09}: 143; \citealt{GH94}) {\sep} \word{PW}{*påq}{creamy-bellied thrush\species{Turdus amaurochalinus}} > \wordng{Vejoz or Guisnay}{påk}; \wordng{’Wk}{påq} (\citealt{RL16}:
72; \citealt{KC16}: 286); \wordnl{*påq\mbox{-}taχ} > \word{LB}{poq\mbox{-}taχ}{creamy-bellied thrush\species{Turdus amaurochalinus}}; \word{Vejoz or Guisnay}{påk\mbox{-}t’åh}{rufous-bellied thrush\species{Turdus rufiventris}} [1] (\citealt{CS-FL-PR-VN13}; \citealt{RL16}: 72)

\dicnote{The form \intxt{påk\mbox{-}t’åh}, attested in \citet{RL16}, is quite unexpected. The regular reflex would be \intxt{påq\mbox{-}tah}. It is unknown whether this form should be attributed to the Vejoz or to the Guisnay variety.}%2

\PMlemma{{\wordnl{*[ji]på(ˀ)x \recind *[ji]på(ˀ)χ \recind *[ji]pǻ(ˀ)x \recind *[ji]pǻ(ˀ)χ}{to hit} (ChW)}}

\wordng{PCh}{*[ʔi]pǻh} > \wordng{Ijw}{[ʔi]pʲáh / \mbox{-}páh}; \word{Mj}{[ʔi]pé\mbox{-}e / \mbox{-}pá\mbox{-}a}{to slap with one’s palm} (\citealt{ND09}: 109; \citealt{JC18}) {\sep} \word{PW}{*[ʔi]pǻχ\mbox{-}\APPL}{to beat}, \wordnl{*[ʔi]<nhǻ>påχ}{to punch} > \word{LB}{[ʔi]poχ\mbox{-}ɬi}{to punch}, \wordnl{[ʔi]n̥opoχ\mbox{-}ɬi}{to punch (iteratively)}, \wordnl{\mbox{-}poχ\mbox{-}hek}{blow (noun)}; \word{’Wk}{[ʔi]pǻx(\mbox{-}\APPL)\mbox{-}ɬih}{to beat (iteratively)}, \wordnl{[ʔi]n̥ǻpåx}{to punch} (\citealt{VN14}: 161, 224, 298, 365; \citealt{KC16}: 285)

\PMlemma{{\wordnl{*[ji]pén \recind *[ʔi]pä́n}{to cook} (ChW)}}

\wordng{PCh}{*[ʔi]pén} > \wordng{Ijw}{[ʔi]píˀn / \mbox{-}pέˀn}; \wordng{I’w}{\mbox{-}pén}; \wordng{Mj}{[ʔi]pín / \mbox{-}pέn} (\citealt{ND09}: 110; \citealt{AG83}: 155; \citealt{JC18}) {\sep} \wordng{PW}{*[ʔi]pén} > \wordng{LB}{[ta]pen<ek>}; \wordng{Vej}{\mbox{-}pen}; \wordng{’Wk}{[ʔi]pén̥} (\citealt{JB09}: 56; \citealt{VU74}: 70; \citealt{KC16}: 292)

\lit{\citealt{EN84}: 9 (2~\intxt{*hl\mbox{-}pέn})}

\PMlemma{{\wordnl{*pex \recind *päx}{each time, every time} (ChW) [1]}}

\wordng{PCh}{*péh} > \wordng{Ijw}{pέh} (\citealt{JC14a}) {\sep} \wordng{PW}{*=peχ} > \wordng{LB}{=peχ}; \wordng{Vej}{-peh}; \wordng{’Wk}{-pex} (\citealt{VN14}: 304; \citealt{VU74}: 70; \citealt{KC16}: 291)

\dicnote{Even though we have not found cognates in Iyo’wujwa’ or Manjui, we find the Iyojwa’aja’ form unlikely to be a Wichí borrowing because it shows a greater degree of autonomy (it is always stressed and does not behave like an enclitic or suffix). The putative Guaicuruan cognates listed above yield further support to the possibility that the etymon in question is old enough.}%1

\gc{Likely related to \word{Proto-Guaicuruan}{*\mbox{-}pek'e}{each (distributive)} (\citealt{PVB13b}, \#721).}

\PMlemma{{\wordnl{*púle(ʔ)\pla{ts}}{sky, cloud} (ChW)}}

\wordng{PCh}{*puleʔ\pla{s}} > \word{Ijw}{póliʔ\pl{s}}{cloud}, \wordnl{póliʔ\pl{jis}}{sky} [1]; \wordng{I’w}{púleʔ \recind \mbox{-}ó\mbox{-} \recind \mbox{-}iʔ}; \wordng{Mj}{pʊ́leʔ\pl{s}} (\citealt{ND09}: 144; \citealt{AG83}: 156, 189, 211; \citealt{JC18}) {\sep} \wordng{PW}{*púle} (\intxt{*\mbox{-}s \recind *\mbox{-}ɬajis}) > \wordng{LB}{pele}; \wordng{Vej}{pule\pl{ɬajis}}; \wordng{’Wk}{púleʔ} (\intxt{\mbox{-}s \recind \mbox{-}ɬajis}) (\citealt{VN14}: 161; \citealt{VU74}: 70, 112; \citealt{MG-MELO15}: 44; \citealt{AFG067}: 213; \citealt{KC16}: 296)

\dicnote{The Iyojwa’aja’ form is mistranscribed as \intxt{póli} in \citet{ND09}.}%1

\lit{\citealt{EN84}: 9, 43 (\intxt{*pule})}

\PMlemma{{\wordnl{*púm}{drum} (ChW)}}

\wordng{PCh}{*púm\pla{is}} > \wordng{Ijw}{póʔom}, \textit{póm\mbox{-}is}; \wordng{I’w}{póm\mbox{-}itók}, \textit{póm\mbox{-}is\mbox{-}itó\mbox{-}ji}; \wordng{Mj}{pʊ́m}, \textit{\mbox{-}pʊ́m\mbox{-}is} (\citealt{ND09}: 144; \citealt{AG83}: 156; \citealt{JC18}) {\sep} \wordng{PW}{*púm} > \wordng{LB}{pem}; \wordng{Vej}{pum}; \wordng{’Wk}{púm\mbox{-}tax} (\citealt{JB09}: 54; \citealt{VU74}: 70; \citealt{KC16}: 296)

\PMlemma{{\wordnl{*qaka\pla{l / *\mbox{-}qáka\pla{l}}}{medicine} (ChW)}}

\wordng{PCh}{*\mbox{-}qákaʔ\pla{l}} > \wordng{Ijw}{\mbox{-}kákʲeʔ} [1]; \wordng{I’w}{\mbox{-}kákʲeʔ\pl{l}} (\citealt{ND09}: 120; \citealt{AG83}: 136) {\sep} \wordng{PW}{*qakʲa\plf{*qakʲá\mbox{-}ɬ} *\mbox{-}qákʲa\pla{lʰ}} > \wordng{LB}{qatʃa}; \wordng{Vej}{\mbox{-}katʃa\pl{l}} [2]; \wordng{’Wk}{qakʲaʔ}, \intxt{qakʲá\mbox{-}ɬ / \mbox{-}qákʲaʔ\pl{ɬ}} (\citealt{VN14}: 199; \citealt{VU74}: 61; \citealt{MG-MELO15}: 47; \citealt{KC16}: 85, 306)

\dicnote{The Iyojwa’aja’ form is mistranscribed as \intxt{\mbox{-}kákʲe} in \citet{ND09}.}%1

\dicnote{The Vejoz reflex is attested with an aspirated velar in \citet[47]{MG-MELO15}: \textit{\mbox{-}kʰatʃa\pl{l}}.}%2

\PMlemma{{\wordnl{*[t]qási(ˀ)t / *\mbox{-}qasí(ˀ)t}{to stand} (ChW)}}

\wordng{PCh}{*[tᵊ]qásit} > \wordng{Ijw}{[ta]káxsit}; \wordng{I’w}{\mbox{-}ká(x)sit}; \wordng{Mj}{[ti]káxʃit} (\citealt{ND09}: 148; \citealt{AG83}: 139, 213; \citealt{JC18}) {\sep} \wordng{PW}{*[t]qásit}, imp. \textit{*qasít} > \wordng{LB}{[ta]qasit}; \wordng{Vej}{[ta]kasit}; \wordng{’Wk}{[t(a)]qásit}, imp. \textit{qasít} (\citealt{VN14}: 275; \citealt{JB09}: 55; \citealt{VU74}: 61; \citealt{MG-MELO15}: 35; \citealt{AFG067}: 217; \citealt{KC16}: 375)

\lit{\citealt{EN84}: 46 (\intxt{*qahsit})}

\PMlemma{{\wordnl{*qatsíwo(ʔ)}{limpkin} (ChW)}}

\wordng{PCh}{*qasíwo<ʔoh>} [1] > \wordng{Ijw}{kaséwoʔo}; \wordng{Mj}{kaséiwoʔo}, \textit{kasɪ́woʔo\pl{s}} (\citealt{ND09}: 134; \citealt{JC18}) {\sep} \wordng{PW}{*qatsíwo} > \wordng{LB}{tsiwu} [2]; \wordng{’Wk}{qatsíwoʔ} (\citealt{CS-FL-PR-VN13}; \citealt{KC16}: 317)

\dicnote{We have no explanation for the element \intxt{*\mbox{-}ʔoh} in Chorote.}%1

\dicnote{The root-initial syllable was irregularly lost in Lower Bermejeño Wichí.}%2

\PMlemma{{\wordnl{*qawa(ˀ)q / \mbox{-}qáwa(ˀ)q}{belt, band} (ChW)}}

\wordng{PCh}{*\mbox{-}qáwak} > \wordng{Ijw}{\mbox{-}káˀwak}, \textit{\mbox{-}káˀwakʲ\mbox{-}awa} [1]; \wordng{I’w}{\mbox{-}káwak} (\citealt{ND09}: 121; \citealt{AG83}: 138) {\sep} \wordng{PW}{*\mbox{-}qáwaq} > \wordng{LB}{\mbox{-}qawaq}; \wordng{’Wk}{qawaq\plf{qawáq\mbox{-}aç} / \mbox{-}qáwaq\pl{aç}} (\citealt{JB09}: 47; \citealt{KC16}: 317)

\dicnote{The glottalization in \textit{ˀw} in Iyojwa’aja’ is unexpected.}%1

\PMlemma{{\wordnl{*\mbox{-}qáʔtu(ʔ)}{yellow} (ChW)}}

\wordng{PCh}{*\mbox{-}qáʔtuʔ} > \wordng{I’w}{káʔtsʲu<tʲuʔ>}; \wordng{Mj}{\mbox{-}káʔatʲuʔ} (\citealt{AG83}: 138; \citealt{JC18}) {\sep} \wordng{PW}{*qáʔtu} > \wordng{LB}{qaʔte}; \wordng{Vej}{kaʔtu} [1]; \wordng{’Wk}{<ja>qáʔtuʔ} (\citealt{JB09}: 47; \citealt{MG-MELO15}: 42; \citealt{KC16}: 527)

\largerpage
\dicnote{\citet[62]{VU74} mistranscribes the root as \intxt{\mbox{-}kåtu}.}%1

\lit{\citealt{EN84}: 25 (\intxt{*qatu})}\clearpage

\PMlemma{{\wordnl{*\mbox{-}qǻtsile(ʔ)\pla{jʰ}}{guts} [1] (ChW)}}

\wordng{PCh}{*\mbox{-}qǻsile\mbox{-}jʰ} > \wordng{Ijw}{\mbox{-}káxsili\pl{wa}}\gloss{intestine, umbilical cord}; \wordng{I’w}{\mbox{-}káxsili}; \wordng{Mj}{\mbox{-}káxʃili} (\citealt{ND09}: 121; \citealt{AG83}: 139; \citealt{JC18}) {\sep} \wordng{PW}{*\mbox{-}qǻsle\pla{jʰ}} > \wordng{LB}{\mbox{-}t(a)\mbox{-}qosle\mbox{-}j}; \wordng{Vej}{\mbox{-}kǻsle}; \wordng{’Wk}{\mbox{-}qǻsle\mbox{-}jʰ} (\citealt{VN14}: 164, 339; \citealt{VU74}: 62; \citealt{KC16}: 83)

\dicnote{This is likely an opaque compound of \wordnl{*\mbox{-}qǻ\mbox{-}ts}{food (pl.)} and \wordnl{*\mbox{-}éle(ʔ) \recind *\mbox{-}ä́le(ʔ)\pla{jʰ}}{inhabitant, inner} (in Chorote also\gloss{intestine}).}%1

\lit{\citealt{EN84}: 16 (\intxt{*qatsle}); \citealt{LC-VG-07}: 15}

\PMlemma{{\wordnl{*\mbox{-}qótso(ʔ)}{node} (ChW)}}

\wordng{PCh}{*\mbox{-}qóso\mbox{-}keʔ} > \wordng{Ijw}{\mbox{-}kɔ́xso\mbox{-}ki\pl{jis}} [1]; \wordng{I’w}{\mbox{-}kóxso\mbox{-}kiʔ\pl{waʔ}} (\citealt{ND09}: 123; \citealt{AG83}: 144) {\sep} \wordng{PW}{*\mbox{-}qótso} > \wordng{LB}{\mbox{-}qutsu}; \wordng{’Wk}{\mbox{-}qótsoʔ \recind [ta]qótsoʔ\pl{ɬ}} (\citealt{JB09}: 48; \citealt{KC16}: 89)

\dicnote{The absence of a word-final glottal stop in \cits{ND09} attestation of this noun must be a mistranscription.}%1

\lit{\citealt{EN84}: 24 (\intxt{*kɔtshɔq})}

\PMlemma{{\wordnl{*[t]qXǻn}{to dig} [1] (ChW) \label{dic-qxan}}}

\wordng{PCh}{*[tᵊ]q(h)ǻn} > \wordng{Ijw}{[ta]káˀn}; \wordng{Mj}{[ti]k(x)án}, \textit{[ti]k(h)án} (\citealt{ND09}: 148; \citealt{GH94}; \citealt{JC18}) {\sep} \wordng{PW}{*[t]χhǻn} > \wordng{’Wk}{[t(a)]xhǻn̥} \citep[352]{KC16}

\dicnote{The reconstruction \intxt{*qX} is highly tentative. Note that the cluster \intxt{xh} in ’Weenhayek is unique and occurs only in this root. In Manjui, the verb is attested as \textit{[ti]khán} in \citet{JC18} but with a plain \intxt{\mbox{-}k\mbox{-}} in \citet{GH94}; we concede that [kh], [kx] could be simply allophones of /k/ before a low vowel in Manjui; see \sectref{ch-q} \textit{in fine}.}%1

\PMlemma{{\wordnl{*\mbox{-}q’á(ˀ)X₁₂}{tongue} (ChW)}}

\wordng{PCh}{*\mbox{-}q’áh} > \wordng{I’w}{\mbox{-}káh\pl{es}} [1]; \wordng{Mj}{\mbox{-}k’áh\pl{as}} (\citealt{AG83}: 138; \citealt{JC18}) {\sep} \word{PW}{*\mbox{-}q’áχ}{mouth} > \wordng{LB}{\mbox{-}q’aχ}; \wordng{Vej}{\mbox{-}kah} [1]; \wordng{’Wk}{\mbox{-}q’áx} (\citealt{VN14}: 121; \citealt{MG-MELO15}: 60; \citealt{KC16}: 89)

\dicnote{The plain reflex of the stem-initial stop in Iyo’awujwa’ and Vejoz as attested in \citet{AG83} and \citet[60]{MG-MELO15} must be a mistranscription.}%1

\rej{\citet[23]{EN84} compares the Wichí word with \wordng{Ni}{\mbox{-}tʃ’ak͡letʃ}, \textit{\mbox{-}tʃ’akxe\mbox{-}s}\gloss{tongue} \citep[109]{JS16} and reconstructs \textit{*k’ahn hle}. \citet[16]{LC-VG-07} and \citet[309]{PVB13a}, in turn, compare the Nivaĉle word with the Wichí compound \textit{*\mbox{-}q’áχ\mbox{-}ɬ\mbox{-}ɪkʲ’u}\gloss{tongue} (literally\gloss{the egg of the mouth}); \citeauthor{PVB13a} reconstructs \wordng{PM}{*\mbox{-}kahlik’u}. The comparisons are untenable; the Nivaĉle word must go back to \intxt{*\mbox{-}k’álek}, \textit{*\mbox{-}k’álhe\mbox{-}ts}.}

\lit{\citealt{LC-VG-07}: 16}

\PMlemma{{\intxt{*silóʔtåɸV(ʔ) \recvar *siwóʔtåɸe(ʔ)} [1 2]\gloss{Caatinga puffbird} (ChW)}}

\wordng{PCh}{*silóʔtåhwVʔ} [2] > \wordng{Ijw}{silʲóʔtʲohwaʔ} [1]; \wordng{Mj}{ʃilóʔtahwej} (\citealt{ND09}: 145; \citealt{JC18}) {\sep} \wordng{PW}{*siwótåxʷe} > \wordng{LB}{siwutofʷe}; \wordng{’Wk}{siwótåxʷeʔ} (\citealt{CS-FL-PR-VN13}; \citealt{KC16}: 330)

\dicnote{Chorote points to \wordng{PM}{*l} and Wichí to \wordng{PM}{*w}.}%1

\dicnote{Wichí points to \wordng{PM}{*\mbox{-}e(ʔ)}, whereas in Chorote one finds different endings in Iyojwa’aja’ and Manjui, neither of which matches the evidence from Wichí.}%2

\PMlemma{{\wordnl{*spú(ˀ)p}{dove} (ChW) [1]}}

\wordng{PCh}{*sᵊpúp} > \wordng{Ijw}{sipóp} [2]\gloss{Picui dove}; \wordng{I’w}{sipóp\pl{is}}; \wordng{Mj}{ʃipʊ́p\pl{is}} (\citealt{ND09}: 146; \citealt{AG83}: 159; \citealt{JC18}) {\sep} \wordng{PW}{*spúp} > \wordng{LB}{sipep}\gloss{white-tipped dove}; \wordng{Vej}{sipup}\gloss{white-tipped dove}; \wordng{’Wk}{supúp} [3] (\citealt{CS-FL-PR-VN13}; \citealt{MG-MELO15}: 22; \citealt{KC16}: 332)

\dicnote{Maká has a similar root, \wordnl{sapip\pl{its}}{white-tipped dove} \citep[323]{AG99}, but the vowels are very different from those found in Chorote and Wichí.}%1

\dicnote{The Iyojwa’aja’ reflex is attested as \intxt{sipɔ́p} in \citet[99]{JC14b}, which is most likely a mistranscription.}%2

\dicnote{The ’Weenhayek reflex shows an irregular sound change \textit{*i > u}.}%3

\PMlemma{{\intxt{*stá(ˀ)X} (fruit); \intxt{*stá\mbox{-}ˀq} (plant)\gloss{\textit{Stetsonia coryne} cactus} (ChW)}}

\wordng{PCh}{*ʔᵊstáh}; \textit{*ʔᵊstá\mbox{-}k} > \wordng{Ijw}{ʔistʲé}; \textit{ʔistʲé\mbox{-}k}, \textit{ʔistʲé\mbox{-}kʲet}; \wordng{I’w}{ʔistá\mbox{-}k}, \textit{ʔistá\mbox{-}ki\mbox{-}ʔ}; \wordng{Mj}{ʔistáh \recind ʔiʃtáh}; \textit{ʔiʃtá\mbox{-}k \recind ʔiʃtá\mbox{-}k} (\citealt{ND09}: 112; \citealt{AG83}: 132; \citealt{JC18}) {\sep} \wordng{PW}{*ʔistá\mbox{-}q} > \word{LB}{ʔista\mbox{-}q}{white cactus}; \wordng{Southeastern (Salta)}{ʔista\mbox{-}q}; \word{Vej}{ista\mbox{-}k}{Mataco tree}; \wordng{’Wk}{ʔistá\mbox{-}k} [1]\gloss{\textit{Cereus giganteus}} (\citealt{VN14}: 339; \citealt{MS14}: 242; \citealt{MG-MELO15}: 18; \citealt{KC16}: 37)

\dicnote{The velar consonant \intxt{\mbox{-}k} in ’Weenhayek is explained as a result of analogical leveling (the suffix for trees \intxt{\mbox{-}(u)k} ends in a velar consonant). Note that in \wordng{PM}{*k} was banned following the vowel \intxt{*a}, which is why the compound of \intxt{*stáX} and \intxt{*\mbox{-}uk} has the shape \intxt{*stá\mbox{-}q} and not \intxt{**stá\mbox{-}k}.}%1

\lit{\citealt{EN84}: 39 (\intxt{*s\mbox{-}thɛk} (plant))}

\PMlemma{{\textit{*stǻɸe(ʔ)}\gloss{Chaco chachalaca} (ChW)}}

\wordng{PCh}{*ʔᵊstǻhweʔ\pla{waʔ}} > \wordng{Ijw}{ʔistʲáhwe}, \textit{ʔistʲáhwi\mbox{-}waʔ}; \wordng{I’w}{istáfʷe\pl{waʔ}}; \wordng{Mj}{\mbox{ʔistáhweʔ} \recind ʔiʃtáhweʔ} (\intxt{\mbox{-}l \recind \mbox{-}waʔ}) (\citealt{ND09}: 112; \citealt{AG83}: 132; \citealt{JC18}) {\sep} \wordng{PW}{*ʔistǻxʷe} > \wordng{Southeastern (Salta)}{sitofʷe \recind ʔistofʷe}; \wordng{Vej}{iståhʷe}; \wordng{’Wk}{ʔistǻxʷeʔ} (\citealt{MS14}: 178; \citealt{MG-MELO15}: 20; \citealt{KC16}: 37)

\dicnote{The Vejoz reflex is mistranscribed as \intxt{istahʷe} in \citet[61]{VU74}.}%1

\lit{\citealt{EN84}: 39 (\intxt{*s\mbox{-}thåhwɛ})}

\PMlemma{{\wordnl{*tǻtsna(ˀ)X₁₂ \recind *tǻtsne(ˀ)χ}{toad} (ChW)}}

\wordng{PCh}{*tǻsVnah} > \wordng{Ijw}{táxsina}\gloss{\textit{Rhinella arenarum}}; \wordng{I’w}{táxsina \recind táxsena\pl{s}}; \word{Mj}{táxsena\pl{as}}{cururu toad} (\citealt{JC14b}: 99; \citealt{ND09}: 149; \citealt{AG83}: 163; own field data; \citealt{JC18}) {\sep} \wordng{PW}{*tǻtnaχ} [2] > \wordng{LB}{totnaχ}; \wordng{Vej}{\mbox{tåtnah}}; \wordng{’Wk}{tǻtnax\plf{tǻtn̥a\mbox{-}s}} (\citealt{JB09}: 58; \citealt{VU74}: 121; \citealt{KC16}: 344)

\dicnote{\wordng{PCh}{*V} can stand for any vowel that fails to cause both the first and the second palatalization in Chorote (such as \textit{*a} or \intxt{*å}).}%1

\dicnote{\citet[84]{RL16} documents the form \intxt{tåtsinah} alongside \intxt{tåtnah}, but does not indicate whether it is representative of Vejoz or Guisnay. If it turns out to be a Guisnay form, it could be an old Chorote borrowing.}%1

\lit{\citealt{PVB02}: 144 (\intxt{*tʌtsinaχ})}

\PMlemma{{\wordnl{*\mbox{-}témä(ˀ)k\plf{*\mbox{-}témha\mbox{-}jʰ \recind *\mbox{-}ä́\mbox{-}}}{bile} (ChW)}}

\wordng{PCh}{*\mbox{-}témek\plf{*\mbox{-}téhma\mbox{-}jʰ}} > \wordng{Ijw}{\mbox{-}tέmik\plf{\mbox{-}tέhma\mbox{-}ˀl}} [1]; \wordng{I’w}{\mbox{-}témak\plf{\mbox{-}téma\mbox{-}j}} [2]; \wordng{Mj}{\mbox{-}tέmak} (\citealt{JC14b}: 93; \citealt{ND09}: 126; \citealt{AG83}: 164; \citealt{JC18}) {\sep} \wordng{PW}{*\mbox{-}témeq\plf{*\mbox{-}témha\mbox{-}jʰ}} > \word{LB}{\mbox{-}temeq\plf{\mbox{-}tem̥a\mbox{-}j}}{an organ of a fish}; \wordng{Vej}{\mbox{-}temek}; \wordng{’Wk}{\mbox{-}témek} (\citealt{VN14}: 192; \citealt{VU74}: 75; \citealt{MG-MELO15}: 57; \citealt{KC16}: 93)

\dicnote{The plural form in Iyojwa’aja’ is non-etymological.}%1

\dicnote{The consonant \intxt{m} (rather than \intxt{*hm}) in the plural form in Iyo’awujwa’ is unexpected and could result from mistranscription.}%2

\rej{\citet[15]{LC-VG-07} list \wordng{Ni}{\mbox{-}ʔaɸk’uˀt}, \wordng{Mk}{\mbox{-}ʔaftuk} under this etymology, an obviously false comparison.}

\lit{\citealt{LC-VG-07}: 15}

\PMlemma{{\wordnl{*tkéna(ˀ)X₁₂ \recind *tkä́na(ˀ)X₁₂\plf{*tkénX₁₃a\mbox{-}ts \recind *tkä́nX₁₃a\mbox{-}ts}}{precipice; hill, mountain} (ChW)}}

\word{PCh}{*tᵊkénah\plf{*tᵊkéhna\mbox{-}s}}{precipice}; \wordnl{*tᵊkéhna\mbox{-}kʲeʔ}{mountain} > \word{Ijw}{tikína}{ravine}, \intxt{tikíhna\mbox{-}kiʔ\pl{s}} [1]\gloss{mountain}; \wordng{I’w}{takíhna\mbox{-}kiʔ\pl{ji}}\gloss{mountain}; \word{Mj}{takína\plf{takíhna\mbox{-}s}}{precipice}, \wordnl{takíhnʲe\mbox{-}kiʔ\pl{j}}{mountain} (\citealt{ND09}: 151; \citealt{AG83}: 162; \citealt{JC18}) {\sep} \wordng{PW}{*tkʲénaχ}, \textit{*tkʲénha\mbox{-}s}\gloss{mountain, hill} > \wordng{LB}{\mbox{tatʃenaχ}}; \wordng{Vej}{tʃenah}, \textit{tʃen̥a\mbox{-}s}; \wordng{’Wk}{kʲénax}, \textit{kʲén̥a\mbox{-}s} (\citealt{VN14}: 51; \citealt{JB09}: 56; \citealt{VU74}: 72; \citealt{MG-MELO15}: 43; \citealt{KC16}: 187)

\dicnote{The Iyojwa’aja’ word is mistranscribed as \intxt{tikíhna\mbox{-}ki} in \citet[151]{ND09}.}%1

\largerpage
\rej{\citet[11]{EN84} lists \word{Ni}{ɸtʃenax}{north wind} as a member of this cognate set. We derive it from \word{PM}{*ɸkénaχ}{north wind, north} instead. \citet[15]{LC-VG-07} compare the Chorote word with \word{Ni}{\mbox{-}takoʔ} {forehead}, \wordnl{\mbox{-}tako\mbox{-}jiʃ}{ravine}, an obviously spurious comparison.}

\lit{\citealt{EN84}: 11, 41 (\intxt{*cɛnaq \recind *t\mbox{-}cɛnaq})}\clearpage

\PMlemma{{\wordnl{*\mbox{-}tk’úɬu(ʔ)}{marrow} (ChW)}}

\word{PCh}{*\mbox{-}<té>k’uhluʔ}{brain, marrow} > \wordng{Ijw}{\mbox{-}tέk’ihli} [1]\gloss{brain}; \wordng{I’w}{\mbox{-}tékihlí}, \textit{\mbox{-}tékihlé\mbox{-}j} [1]\gloss{marrow}; \wordng{Mj}{\mbox{-}tέʔihlʲuʔ} [2] (\citealt{ND09}: 126; \citealt{AG83}: 164; \citealt{JC18}) {\sep} \wordng{PW}{*\mbox{-}tkʲ’úɬu} > \wordng{’Wk}{\mbox{-}kʲ’úɬuʔ} \citep[68]{KC16}

\dicnote{The absence of a word-final glottal stop in \cits{ND09} attestation of this noun must be a mistranscription.}%1

\dicnote{This is mistranscribed as \intxt{\mbox{-}téiʔihlʲuʔ} in \citet{JC18}.}%2

\rej{\citealt{PVB13a}: 313 compares the Chorote term with \word{Maká}{\mbox{-}xkitiɬa}{brain, marrow} and reconstructs \wordng{PM}{*hetekiɬV}, an obviously false comparison. He also includes Mbayá ‹\mbox{-}atiquelo›, ‹\mbox{-}atiquilo›\gloss{brain, marrow} as possible Guaicuruan cognates.}

\PMlemma{{\intxt{*(\mbox{-})tútse(ˀ)χ} [1]\gloss{smoke} (ChW)}}

\wordng{PCh}{*(\mbox{-})túsah} > \wordng{Ijw}{tóxse\pl{hes}}; \wordng{I’w}{tóxsa}, \textit{tóxsi\mbox{-}s}; \wordng{Mj}{(\mbox{-})tʊ́xsa} (\citealt{ND09}: 153; \citealt{AG83}: 166; \citealt{JC18}) {\sep} \wordng{PW}{*(\mbox{-})tútsaχ} > \wordng{LB}{tetsaχ}; \wordng{Vej}{tutsah}; \wordng{’Wk}{(\mbox{-})tútsax}, \wordnl{tútse\mbox{-}tax}{mist} (\citealt{VN14}: 47; \citealt{VU74}: 77; \citealt{KC16}: 95, 426)

\dicnote{\sound{PM}{*\mbox{-}eχ} (rather than \intxt{**\mbox{-}aχ} or \intxt{**\mbox{-}ax}) is reconstructed based on \word{’Wk}{tútse\mbox{-}tax}{mist} and \wordng{I’w}{tóxsi\mbox{-}s}, which show that the root had the allomorph \textit{*tútse\mbox{-}} before suffixes.}%1

\gc{Possibly related to \word{Proto-Guaicuruan}{*\mbox{-}á(ˀ)lodqa}{smoke} (\citealt{PVB13b}, \#35).}

\rej{\citet[43]{EN84} includes \word{Nivaĉle}{ʃtutax}{soot} into the comparison, which is implausible for phonological reasons.}

\lit{\citealt{EN84}: 16, 43 (\intxt{*tutsha}); \citealt{PVB02}: 144 (\intxt{*tutsaχ})}

\PMlemma{{\textit{*\mbox{-}tXá(ˀ)t}\gloss{to throw, to put} (ChW)}}

\wordng{PCh}{*[ʔi]tát\mbox{-}\APPL} > \wordng{Ijw}{[ʔi]tʲét\mbox{-}{\APPL} / \mbox{-}tát\mbox{-}\APPL}; \wordng{I’w}{\mbox{-}tát\mbox{-}e}; \wordng{Mj}{[ʔi]t(ʲ)ét\mbox{-}{\APPL} / \mbox{-}tát\mbox{-}\APPL} (\citealt{JC14b}: 76; \citealt{ND09}: 113; \citealt{AG83}: 163; \citealt{JC18}) {\sep} \wordng{PW}{*[ʔi]thát} > \wordng{LB}{[ʔi]tʰat}; \wordng{Vej}{\mbox{-}tat} [1]; \wordng{’Wk}{[ʔi]tʰát} (\citealt{VN14}: 255, 280; \citealt{JB09}: 45; \citealt{VU74}: 74; \citealt{KC16}: 455)

\dicnote{The absence of aspiration in \wordng{Vej}{\mbox{-}tat}, as attested by \citet[74]{VU74}, could be a mistranscription.}%1

\lit{\citealt{EN84}: 52 (1\textsc{pl} \textit{*a\mbox{-}tat\mbox{-}ehne})}

\PMlemma{{\textit{*[ji]tså(ˀ)j}\gloss{to spill} (ChW)}}

\wordng{PCh}{*[ʔi]sǻjʔ} > \wordng{Ijw}{[ʔi]sʲá(j)\mbox{-}{\APPL} / \mbox{-}sá(j)\mbox{-}\APPL}; \wordng{I’w}{\mbox{-}sáj\mbox{-}\APPL}; \wordng{Mj}{[ʔi]ʃéjʔ / \mbox{-}sájʔ} (\citealt{ND09}: 110; \citealt{AG83}: 157; \citealt{JC18}) {\sep} \wordng{PW}{*[ʔi]tsåj} > \wordng{LB}{[ʔi]tsoj\mbox{-}ka}; \wordng{Vej}{\mbox{-}tsaj}; \wordng{’Wk}{[ʔi]tsåjʔ} (\citealt{JB09}: 43; \citealt{VU74}: 55; \citealt{KC16}: 462)

\lit{\citealt{EN84}: 11 (\intxt{*tsaj})}

\PMlemma{{\wordnl{*\mbox{-}tséɬå(ʔ) \recind *\mbox{-}ä́\mbox{-}}{sharp corner, tip}; \wordnl{*\mbox{-}tséɬå\mbox{-}(ˀ)χ \recind *\mbox{-}ä́\mbox{-}\plf{*\mbox{-}tséɬå\mbox{-}ts \recind *\mbox{-}ä́\mbox{-}}}{sharp}; \wordnl{*\mbox{-}tséɬå\mbox{-}(ˀ)t \recind *\mbox{-}ä́\mbox{-}}{to sharpen} (ChW)}}

\word{PCh}{*\mbox{-}séhlå\mbox{-}h\mbox{-}iʔ\plf{*\mbox{-}séhlå\mbox{-}s\mbox{-}iʔ}}{to be sharp}; \wordnl{*\mbox{-}séhlå\mbox{-}ht\mbox{-}iʔ}{to sharpen} > \wordng{Ijw}{[ʔi]síhla\mbox{-}h\mbox{-}e}, \intxt{[ʔi]síhla\mbox{-}s\mbox{-}its’iʔn} [1]; \intxt{[ʔi]síhla\mbox{-}t\mbox{-}i / \mbox{-}sέhla\mbox{-}t\mbox{-}i}; \wordng{Mj}{[ʔa]sέhleh\mbox{-}ijʔ}; \intxt{[ʔi]ʃíhle\mbox{-}ht\mbox{-}ijʔ / \mbox{-}sέhle\mbox{-}ht\mbox{-}ijʔ} (\citealt{ND09}: 111; \citealt{JC18}) {\sep} \wordng{PW}{*\mbox{-}tséɬå(ʔ)}; \intxt{*\mbox{-}tséɬå\mbox{-}(ˀ)χ}; \intxt{*\mbox{-}tséɬå\mbox{-}(ˀ)t} > \wordng{LB}{\mbox{-}tseɬo(ʔ)} [2]; \intxt{ʔi\mbox{-}tseɬoχ}; \wordng{’Wk}{\mbox{-}tséɬåʔ\pl{s}}; \intxt{ʔi\mbox{-}tséɬåχ\plf{\mbox{-}ʔi\mbox{-}tséɬå\mbox{-}s}}; \intxt{[ni]tséɬå\mbox{-}t} (\citealt{JB09}: 43, 48; \citealt{KC16}: 40, 110, 464)

\dicnote{The absence of a word-final glottal stop in \cits{ND09} attestation of the singular form must be a mistranscription.}%1

\dicnote{This root is not attested in \citet{VN14}, hence the uncertainty regarding the presence of a word-final glottal stop. \citet{JB09} documents a word-final glottal stop in this form, but since he is otherwise known to document one where \citet{VN14} documents none, the datum is considered unreliable.}%2

\PMlemma{{\wordnl{*tsémɬå(ˀ)k \recind *tsä́mɬå(ˀ)k}{silk floss tree} (ChW)}}

\wordng{PCh}{*sémhlåk} > \wordng{Ijw}{sέmhlak}; \wordng{I’w}{sémlak\pl{is}} [1]; \wordng{Mj}{sέmhlak} (\intxt{\mbox{-}ij}) (\citealt{ND09}: 145; \citealt{AG83}: 158; \citealt{JC18}) {\sep} \wordng{PW}{*tsémɬåkʷ} > \wordng{LB}{tsemɬokʷ} [2]; \wordng{Vej}{tsemɬåkʷ}, \textit{tsemɬåk\mbox{-}uj}; \wordng{’Wk}{tsémɬåk} (\intxt{\mbox{-}uç}) (\citealt{CS08}: 59; \citealt{MG-MELO15}: 18; \citealt{KC16}: 464)

\dicnote{The absence of \intxt{h} in the Iyo’awujwa’ form attested in \citet{AG83} must be a mistranscription.}%1

\dicnote{\citet[384]{VN14} gives the form \intxt{tsemɬoq}, which could be a mistranscription.}%2

\rej{\citet[37]{EN84} includes \word{Chorote}{sel}{thorn} (probably a mistranscription \word{PCh}{*hl\mbox{-}é\mbox{-}l}{its thorns}, since the first-person plural form \wordnl{*s\mbox{-}é\mbox{-}l}{our thorns} cannot seem to be pragmatically felicitous) as a possible cognate, which is absolutely impossible for phonological and semantic reasons.}

\lit{\citealt{EN84}: 17, 37 (\intxt{*sɛmhla\mbox{-}uk \recind *sɛlnauk})}

\PMlemma{{\wordnl{*tsóna(ʔ)}{red brocket} (ChW)}}

\wordng{PCh}{*sónaʔ} > \wordng{Ijw}{sɔ́naʔ\pl{jis}}; \word{I’w}{són\mbox{-}ta\pl{s}}{sheep}; \word{Mj}{sɔ́n(a)\mbox{-}ta\pl{s}}{sheep} (\citealt{ND09}: 147; \citealt{AG83}: 161; \citealt{JC18}) {\sep} \wordng{PW}{*tsóˀnah} > \wordng{LB}{tsuˀna}; \wordng{Vej}{tsoˀna} [1]; \wordng{’Wk}{tsóˀnah}, \textit{tsóˀna\mbox{-}lis} (\citealt{VN14}: 197; \citealt{VU74}: 55; \citealt{MG-MELO15}: 23; \citealt{KC16}: 466)

\dicnote{\cits{VU74} attestation of the Vejoz reflex as \intxt{tsona} (with no glottalization) must be a mistranscription.}%1

\lit{\citealt{EN84}: 28 (\intxt{*sonatha}\gloss{sheep})}

\PMlemma{{\intxt{*tsu(ˀ)X \recvar *ts’u(ˀ)X} (fruit); \intxt{*tsuX\mbox{-}uk \recvar *ts’uX\mbox{-}uk} (tree)\gloss{sachamembrillo\species{Capparis tweediana}} (ChW)}}

\wordng{PCh}{*ts’úh}; \textit{*ts’úh\mbox{-}uk} > \wordng{Ijw}{<mé>tsʲ’u}; \textit{<mé>tsʲ’u\mbox{-}k \recind ts’ówk \recind ts’éwk}; \wordng{I’w}{ts’ów<k> \recind ts’éw<k>} [2]; \wordng{Mj}{s’ʊ́u<k>} (\citealt{ND09}: 139; \citealt{AG83}: 167; \citealt{GS10}: 187; \citealt{JC18}) {\sep} \wordng{PW}{*tsúhukʷ} [3] > \wordng{LB}{tsehekʷ}; \wordng{’Wk}{tsúhuk} (\citealt{CS08}: 62; \citealt{KC16}: 467)

\dicnote{Chorote points to \sound{PM}{*ts’} (or \intxt{*s’}), and Wichí to \intxt{*ts}.}%1

\dicnote{The Iyo’awujwa’ reflex is mistranscribed as \intxt{tsok} in \citet[167]{AG83}. }%2

\dicnote{\citet[247]{MS14} documents the reflex \textit{tʃjuhuk \recind ʔitʃjuhuk} without specifying the location where this name was attested.}%3

\PMlemma{{\wordnl{*[ji](t)s’u(ʔ)}{to suck} (ChW)}}

\wordng{PCh}{*[ʔi]ts’ú\mbox{-}\APPL} > \wordng{Ijw}{[ʔi]tsʲ’ú\mbox{-}{\APPL} / \mbox{-}ts’ó\mbox{-}\APPL}; \wordng{I’w}{[i]tsʲú\mbox{-}fʷeʔ / \mbox{-}tsó\mbox{-}fʷeʔ \recind \mbox{-}tsó\mbox{-}wej}; \wordng{Mj}{[ʔi]tʃ’ú\mbox{-}ujʔ / \mbox{-}ts’ʊ́\mbox{-}ujʔ} (\citealt{ND09}: 115; \citealt{AG83}: 42, 167, 194; \citealt{JC18}) {\sep} \wordng{PW}{*[hi]ts’u(ʔ)} > \wordng{Vej}{\mbox{-}ts’u}\gloss{to absorb}; \wordng{’Wk}{[hi]ts’uʔ} (\citealt{VU74}: 56; \citealt{KC16}: 470)

\lit{\citealt{EN84}: 11 (\intxt{*ts’o})}

\PMlemma{{\wordnl{*(\mbox{-})(t)s’u\mbox{-}k}{\textit{añapa} drink} [1] (ChW)}}

\wordng{PCh}{*ts’ú<k>} > \wordng{I’w}{tsók} [2] \citep[167]{AG83} {\sep} \wordng{PW}{*\mbox{-}ts’u<kʷ>} > \word{LB}{\mbox{-}ts’ekʷ}{suction}; \wordng{Southeastern (Salta)}{\mbox{-}tʃ’ekʷ}; \wordng{’Wk}{\mbox{-}ts’uk\plf{\mbox{-}ts’úh\mbox{-}uç}} (\citealt{VN14}: 268; \citealt{MS14}: 247; \citealt{KC16}: 101)

\dicnote{This is transparently analyzable as a participle of \wordnl{*[ji](t)s’u(ʔ)}{to suck}.}%1

\dicnote{The non-glottalized affricate in the Iyo’awujwa’ reflex must be a mistranscription on \cits{AG83} part.}%2

\PMlemma{{\wordnl{*wkína(ˀ)X₁₂\plf{*wkínX₁₃a\mbox{-}ts}}{metal} (ChW) [1]}}

\wordng{PCh}{*wᵊkínah\plf{*wᵊkínha\mbox{-}s}} > \wordng{Ijw}{wikínʲe\plf{wikíhnʲe\mbox{-}s}} (\citealt{JC14b}: 74, fn. 1; \citealt{ND09}: 157) {\sep} \wordng{PW}{*kʲínaχ\plf{*kʲínha\mbox{-}ts}} > \word{LB}{\mbox{-}tʃinaχ}{knife}; \wordnl{tʃinaχ\mbox{-}t’\mbox{-}oχ}{money}; \wordng{Vej}{tʃinah}; \wordng{’Wk}{kʲínax\plf{kʲín̥a\mbox{-}s}} (\citealt{VN14}: 326, 447; \citealt{VU74}: 53; \citealt{MG-MELO15}: 47; \citealt{KC16}: 191)

\dicnote{Despite the suspiciously narrow distribution of this etymology (only Iyojwa’aja’ and Wichí), the possibility of a Wichí borrowing in Iyojwa’aja’ is excluded because of the correspondence between \wordng{Ijw}{wik} and \sound{PW}{*kʲ}.}%1

\lit{\citealt{EN84}: 28 (\intxt{*wcihna})}

\PMlemma{{\wordnl{*wóna(ʔ)}{\textit{bala} wasp\species{Polybia ruficeps} honey(comb); hat} (ChW)}}

\word{PCh}{*wónaʔ\pla{l}}{\textit{bala} wasp\species{Polybia ruficeps} honey(comb)} > \wordng{Ijw}{wónaʔ}; \wordng{I’w/Mj}{wónaʔ\pl{l}}; \wordnl{*wón(a)\mbox{-}tah\plf{*wón(a)\mbox{-}ta\mbox{-}s}}{hat} > \wordng{Ijw}{\mbox{-}ka\mbox{-}wónta\pl{s}}; \wordng{I’w}{wónta\pl{s}}; \wordng{Mj}{\mbox{-}ka wón(a)\mbox{-}ta\pl{s}} (\citealt{ND09}: 157; \citealt{AG83}: 171; \citealt{JC18}) {\sep} \wordng{PW}{*wóˀnah} > \wordng{LB}{wuˀna}; \wordng{Vej}{wona}\gloss{bee} [1]; \wordng{’Wk}{wóˀnah} (\citealt{VN14}: 173; \citealt{JB09}: 62; \citealt{VU74}: 81; \citealt{KC16}: 488)

\dicnote{The absence of glottalization in \cits{VU74} attestation of the Vejoz reflex must be a mistranscription.}%1

\PMlemma{{\intxt{*wóp’ih \recind *wóɸ’ih \recvar *móp’ih \recind *móɸ’ih} [1]\gloss{snowy egret, great egret} (ChW)}}

\wordng{PCh}{*wóp’ih} > \word{Ijw}{wóp’i}{snowy egret}; \wordng{Mj}{wóp’i\pl{is}} (\citealt{ND09}: 157; \citealt{JC18}) {\sep} \wordng{PW}{*móp’i} > \wordng{LB}{mup’i}\gloss{great egret}; \wordng{’Wk}{móp’iʔ\pl{ɬajis}} (\citealt{CS-FL-PR-VN13}; \citealt{KC16}: 250)

\dicnote{Chorote points to \intxt{*w\mbox{-}} and Wichí to \intxt{*m\mbox{-}}.}%1

\PMlemma{{\intxt{*wósak’V(ˀ)t} [1]\gloss{red-crested cardinal} (ChW)}}

\wordng{PCh}{*wósᵊk’at\pl{is}} > \wordng{I’w}{wóxsijét\pl{is}}; \wordng{Mj}{wóxʃeʔet\pl{is}} (\citealt{AG83}: 172; \citealt{JC18}) {\sep} \wordng{PW}{*wósakʲ’it \recvar *wósakʲ’ut} [1] > \wordng{LB}{wusatʃ’it}; \wordng{Vej}{wos(a)tʃ’ut} [1]; \wordng{’Wk}{wósakʲ’it} (\citealt{CS-FL-PR-VN13}; \citealt{VU74}: 81; \citealt{MG-MELO15}: 23; \citealt{KC16}: 503)

\dicnote{Regarding the vowel of the final syllable, Chorote points to \wordng{PM}{*a}, Lower Bermejeño and ’Weenhayek to PM and \sound{PW}{*i}, whereas Vejoz \textit{wosatʃ’ut} (\citealt{VU74}) or \textit{wostʃ’ut} (\citealt{MG-MELO15}, with an irregular syncope) point to PW and \wordng{PM}{*u}.}%1

\PMlemma{{\wordnl{*[ji]wún}{to burn (vt.)} (ChW)}}

\wordng{PCh}{*[ʔi]wún} > \wordng{Ijw}{[ʔi]júˀn / \mbox{-}wúˀn}; \wordng{I’w}{\mbox{-}wún}; \wordng{Mj}{[ʔi]jún / \mbox{-}wún} (\citealt{ND09}: 117; \citealt{AG83}: 172; \citealt{JC18}) {\sep} \wordng{PW}{*[ʔi]wún} > \word{LB}{[ʔi]wen\mbox{-}eχ}{to set on fire}; \wordng{’Wk}{[ʔi]wún̥} (\citealt{JB09}: 46; \citealt{KC16}: 511)

\lit{\citealt{EN84}: 53 (2~\intxt{*hl\mbox{-}wún})}

\PMlemma{{\wordnl{*ˀwá(ˀ)x\plf{*ˀwáx\mbox{-}ajʰ}}{stagnant water}(ChW)}}

PCh~\third{*hl\mbox{-}<a>ˀwáh\pla{ajʰ}} > \wordng{Mj}{hlaˀwáh\plf{hlaˀwá\mbox{-}aj}} \citep{JC18} {\sep} \wordng{PW}{*ˀwáχ\plf{*ˀwáh\mbox{-}ajʰ}} > \wordng{Vej}{wah\pl{aj}} [1]\gloss{water}; \wordng{’Wk}{ˀwáx\plf{ˀwáh\mbox{-}aç}} (\citealt{VU74}: 79; \citealt{MG-MELO15}: 44; \citealt{KC16}: 105)

\dicnote{The semantically shifted Vejoz reflex has irregularly lost the glottalization in the initial consonant.}%1

\PMlemma{{\wordnl{*\mbox{-}ˀwóle(ʔ)\pla{jʰ}}{leaf, hair, feather} (ChW)}}

\wordng{PCh}{*\mbox{-}ˀwóleʔ\pla{jʰ}} > \wordng{Ijw}{\mbox{-}ˀwóleʔ} [1]; \wordng{I’w}{\mbox{-}wóleʔ\pl{j}}; \wordng{Mj}{\mbox{-}ˀwóleʔ\pl{j}} (\citealt{ND09}: 128; \citealt{AG83}: 171; \citealt{JC18}) {\sep} \wordng{PW}{*\mbox{-}ˀwóle\pla{jʰ}} > \wordng{LB}{\mbox{-}ˀwule \recind \mbox{-}wuˀle \recind \mbox{-}wule\pl{j}} [2]; \wordng{Vej}{\mbox{-}ˀwole\pl{j}}; \wordng{’Wk}{\mbox{-}ˀwóleʔ\pl{ç}} (\citealt{VN14}: 170, 233, 294, 321; \citealt{JB09}: 61; \citealt{MG-MELO15}: 61; \citealt{KC16}: 57)

\dicnote{The Iyojwa’aja’ form is mistranscribed as \intxt{\mbox{-}ˀwóle} in \citet{ND09}.}%1

\dicnote{The variants \textit{\mbox{-}wuˀle\mbox{-}j \recind \mbox{-}wule\mbox{-}j}, attested in \citet{VN14}, are irregular.}%2

\dicnote{\citet[81]{VU74} mistranscribes the root as \intxt{\mbox{-}wole}.}%3

\PMlemma{{\wordnl{*\mbox{-}ˀwu(ˀ)j}{clothes, blanket} (ChW)}}

\wordng{PCh}{*\mbox{-}ˀwújʔ} > \wordng{Ijw}{\mbox{-}ˀwúʔ\plf{\mbox{-}ˀwúj\mbox{-}e}}; \wordng{I’w}{\mbox{-}wúj} [1] (\citealt{ND09}: 128; \citealt{AG83}: 172) {\sep} \wordng{PW}{*\mbox{-}ˀwuj} > \wordng{LB}{(\mbox{-})ˀwej} [2]; \wordng{Vej}{\mbox{-}ˀwuj} [2]; \wordng{’Wk}{\mbox{-}ˀwujʔ} (\citealt{VN14}: 132; \citealt{JB09}: 61; \citealt{VU74}: 82; \citealt{MG-MELO15}: 69; \citealt{KC16}: 57)

\dicnote{The absence of glottalization in the initial consonant in Iyo’awujwa’ and Vejoz must be a mistranscription on \cits{AG83} part.}%1

\dicnote{\citet{JB09} and \citet{VU74} fail to attest the glottalization in the initial consonant in Lower Bermejeño.}%2

\PMlemma{{\intxt{*X₁₃ajáˀwu(ʔ) \recvar *X₁₃ajáwu(ʔ)\pla{l}} [1]\gloss{shaman} (ChW) }}

\wordng{PCh}{*ʔajáˀwuʔ\pla{l}} > \wordng{Ijw}{ʔajéˀwuʔ} (\intxt{\mbox{-}ˀl \recind \mbox{-}lis}); \wordng{I’w}{ajéwuʔ\pl{l}} [2]; \wordng{Mj}{\mbox{ʔajéˀwuʔ}\pl{l}} (\citealt{JC14a}; \citealt{ND09}: 95; \citealt{AG83}: 117; \citealt{JC18}) {\sep} \wordng{PW}{\mbox{*hajáwu(ʔ)}\pla{lʰ}} > \wordng{LB}{hajawe(ʔ)}; \wordng{’Wk}{hijáwuʔ\pl{ɬ}} [3] (\citealt{JB09}: 41; \citealt{KC16}: 151)

\dicnote{Chorote points to \wordng{PM}{*X₁₃ajáˀwu(ʔ)}, whereas Wichí points to \intxt{*X₁₃ajáwu(ʔ)}. The Towothli doculect of Maká has a similar root, \intxt{ejawin} (\citealt{RJH15}: 245–251), but it cannot correspond to the Chorote and Wichí forms.}%1

\dicnote{The absence of glottalization in \cits{AG83} attestation of the Iyo’awujwa’ reflex must be a mistranscription.}%2

\dicnote{\sound{’Weenhayek}{i} is not the regular reflex of \sound{PW}{*a}.}%3

\lit{\citealt{RJH15}: 240; \citealt{EN84}: 41, 43, 48 (\intxt{*jɛwu}); \citealt{PVB02}: 144 (\intxt{*χajawu})}

\PMlemma{{\wordnl{*[ji]X₁₃án\mbox{-}ex}{to know} (ChW)}}

\wordng{PCh}{*<ˀ[j]a>hán\mbox{-}eh} [1] > \wordng{Ijw}{ˀ[j]ihén\mbox{-}e / \mbox{-}ʔahán\mbox{-}e}; \wordng{I’w}{\mbox{-}hán\mbox{-}eʔ}; \wordng{Mj}{ˀ[j]ehέn\mbox{-}e / \mbox{-}ʔahán\mbox{-}e} (\citealt{JC14b}: 91; \citealt{ND09}: 165; \citealt{AG83}: 173; \citealt{JC18}) {\sep} \wordng{PW}{*[ji]hán\mbox{-}eχ} > \wordng{LB}{[ji]han\mbox{-}eχ}; \wordng{Vej}{\mbox{-}han\mbox{-}eh}; \wordng{’Wk}{[ja]hán\mbox{-}ex} (\citealt{VN14}: 308; \citealt{VU74}: 56; \citealt{KC16}: 141)

\dicnote{We have no explanation for the element \intxt{*\mbox{-}ʔa\mbox{-}} in Chorote.}%1

\PMlemma{{\intxt{*Xmáwoh}; \wordnl{*Xmáwo\mbox{-}taχ\plf{*Xmáwo\mbox{-}ta\mbox{-}ts}}{fox} (ChW)}}

\wordng{PCh}{*máwo\mbox{-}tah\pla{as}} > \wordng{I’w}{máwo\mbox{-}ta\pl{s}}; \wordng{Mj}{máwo\mbox{-}ta \recind máwa\mbox{-}ta\pl{as}}\gloss{crab-eating fox} (\citealt{AG83}: 148; \citealt{JC18}) {\sep} \word{PW}{*ˣmáwoh}{fox}; \wordnl{*ˣmáwo\mbox{-}taχ\plf{*ˣmáwo\mbox{-}ta\mbox{-}s}}{maned wolf} > \wordng{LB}{mawu}; \textit{mawu\mbox{-}taχ}; \wordng{Vej}{ˀmawo} (\intxt{\mbox{-}ˀlajis}); \textit{ˀmawo\mbox{-}tah\plf{ˀmawo\mbox{-}ta\mbox{-}s}} [2]; \word{’Wk}{ʔimáwoh\plf{ʔimáwo\mbox{-}lis}}{South American gray fox; culpeo}; \intxt{ʔimáwo\mbox{-}tax\plf{ʔimáwo\mbox{-}ta\mbox{-}s}} (\citealt{VN14}: 197; \citealt{MG-MELO15}: 21; \citealt{KC16}: 31)

\dicnote{This etymology is very similar to \wordnl{*wawo\pla{l}}{maned wolf} (MN), but the root-initial consonants do not match. \citet{EN84} lumps these etymologies together.}%1

\dicnote{\citet[67]{VU74} documents \textit{maˀwo}; \textit{mawo\mbox{-}tah}, which must be a mistranscription.}%2

\lit{\citealt{EN84}: 13, 44 (\intxt{*mawo \recind *wawo})}

\PMlemma{{\wordnl{*\mbox{-}X₁₃úsek \recind *\mbox{-}X₁₃úsäk}{temperance} (ChW)}}

\wordng{PCh}{*\mbox{-}húsek} > \wordng{Ijw}{\mbox{-}hóxsik} [1]; \wordng{Mj}{\mbox{-}hʊ́xsek} (\citealt{ND09}: 113; \citealt{JC18}) {\sep} \word{PW}{*\mbox{-}húsek\plf{*\mbox{-}húse\mbox{-}jʰ}}{temperance, soul} > \wordng{LB}{\mbox{-}hesek\plf{\mbox{-}hese\mbox{-}j}}; \wordng{Vej}{\mbox{-}husek}; \wordng{’Wk}{\mbox{-}húsek\plf{\mbox{-}húse\mbox{-}ç}} (\citealt{VN14}: 191; \citealt{JB09}: 41; \citealt{VU74}: 58; \citealt{KC16}: 60)

\dicnote{The raising of \wordng{PCh}{*e} to \wordng{Ijw}{i} is not known to be regular.}%1

\rej{\citet[47]{EN84} compares the Wichí reflex to those of \word{PM}{*\mbox{-}såq’ål \recind *\mbox{-}sǻq’ål}{soul, spirit}.}

\PMlemma{{\wordnl{*\mbox{-}ʔaɬå(ʔ)}{fat} (ChW)}}

\wordng{PCh}{*\mbox{-}ʔahlǻʔ} > \word{Ijw}{\mbox{-}ʔahláʔ}{honey, liquid, fat}; \word{Mj}{\mbox{-}ʔihláʔ\pl{s}}{fat (while on one’s body)} (\citealt{ND09}: 154; \citealt{JC18}) {\sep} \wordng{PW}{*\mbox{-}t\mbox{-}’aɬå(ʔ)} > \wordng{’Wk}{\mbox{-}t\mbox{-}’aɬåʔ} \citep[96]{KC16}

\PMlemma{{\wordnl{*\mbox{-}ʔa(ˀ)q}{rope, cord} (ChW)}}

\wordng{PCh}{*\mbox{-}ʔák\plf{*\mbox{-}ʔaq\mbox{-}ájʔ}} > \wordng{Ijw}{\mbox{-}ʔák\plf{\mbox{-}ʔak\mbox{-}áˀl \recind \mbox{-}ʔak\mbox{-}áʔ}} [1]; I’w~\third{t\mbox{-}ák\plf{t\mbox{-}ak\mbox{-}áj}} [2]; Mj~\third{t\mbox{-}’ák\plf{t\mbox{-}’ak\mbox{-}ájʔ}}\gloss{rope, cable, shoe lace} (\citealt{JC14b}: 92; \citealt{ND09}: 154; \citealt{AG83}: 162; \citealt{JC18}) {\sep} \wordng{PW}{*\mbox{-}t\mbox{-}’aq\plf{*\mbox{-}t\mbox{-}’aq\mbox{-}ájʰ}} > \wordng{LB}{\mbox{-}t\mbox{-}’aq}; Vej~\third{t\mbox{-}’ak}\gloss{band, rope, headband}; \word{’Wk}{\mbox{-}t\mbox{-}’aq\pl{áç}}{object for tying, chain} (\citealt{VN14}: 212; \citealt{VU74}: 77; \citealt{KC16}: 96)

\dicnote{The plural form \intxt{\mbox{-}ʔak\mbox{-}áˀl} in Iyojwa’aja’ is non-etymological.}%1

\dicnote{The plain \intxt{t} in \cits{AG83} attestation of the Iyo’awujwa’ reflex must be a mistranscription.}%2

\PMlemma{{\wordnl{*ʔaté(ˀ)k \recind *ʔatä́(ˀ)k}{cebil\species{Anadenanthera colubrina} or \textit{vinal}\species{Prosopis ruscifolia}} (ChW)}}

\wordng{PCh}{*ʔátek} > \wordng{Ijw/I’w}{ʔaték} (\citealt{ND09}: 94; \citealt{GS10}: 185) {\sep} \wordng{PW}{*ʔatéq} > \wordng{LB}{ʔateq}; \wordng{Vej}{atek}; \wordng{’Wk}{tek \recind ték} [1] (\citealt{CS08}: 62; \citealt{VN14}: 193; \citealt{VU74}: 51; \citealt{KC16}: 391)

\dicnote{The absence of any trace of \sound{PW}{*ʔa\mbox{-}} in the ’Weenhayek reflex is unexpected. \citet[391]{KC16} is unsure whether the vowel \intxt{e} is short or long in this noun.}%1

\PMlemma{{\wordnl{*ʔat’e(ˀ)(t)s \recind *ʔat’ä(ˀ)(t)s}{\textit{aloja} drink} (ChW)}}

\wordng{PCh}{*ʔat’és} > \wordng{Ijw}{ʔat’έs}; \word{I’w}{ʔatés}{drink}; \wordng{Mj}{ʔat’έs\plf{ʔat’έʃ\mbox{-}is}} (\citealt{JC14b}: 77; \citealt{AG83}: 122; \citealt{JC18}) {\sep} \wordng{PW}{*hat’es} > \wordng{LB}{hat’es}; \wordng{Vej}{hates} [1]; \wordng{’Wk}{hat’es} (\citealt{VN14}: 230; \citealt{VU74}: 57; \citealt{KC16}: 147)

\dicnote{The plain \intxt{t} in \cits{VU74} attestation of the Vejoz reflex must be a mistranscription.}%1

\rej{\citet[46]{EN84} lists \wordng{Ni}{\mbox{-}åˀt}\gloss{drink} under this etymology, which instead goes back to \wordng{PM}{*\mbox{-}ǻˀt}.}

\lit{\citealt{RJH15}: 240; \citealt{EN84}: 46 (\intxt{*åtetsh})}

\PMlemma{{\wordnl{*ʔatsXa(ʔ)\plf{*ʔatsXá\mbox{-}l}}{dorado} (ChW)}}

\wordng{PCh}{*ʔasáʔ\pla{l}} > \wordng{Ijw}{ʔasáʔ\pl{ˀl}}; \wordng{I’w}{asáʔa\pl{l}} (\citealt{ND09}: 94; \citealt{AG83}: 122) {\sep} \wordng{PW}{*ʔatsha(ʔ)\plf{*ʔatshá\mbox{-}lʰ}} > \wordng{Vej}{atsʰa\pl{l}}; \wordng{’Wk}{ʔatsʰaʔ\plf{ʔatsʰá\mbox{-}ɬ}} (\citealt{MG-MELO15}: 20; \citealt{KC16}: 19)

\lit{\citealt{EN84}: 11, 17 (\intxt{*atsá \recind *atsa\mbox{-}a})}

\PMlemma{{\wordnl{*ˀ[n]åɸé(ˀ)ɬ \recind *ˀ[n]åɸä́(ˀ)ɬ}{to be ashamed} (ChW)}}

\wordng{PCh}{*ˀ[n]åhwéɬ} > \wordng{Ijw}{ˀ[n]ahwέɬ / \mbox{-}ʔahwέɬ}; \wordng{Mj}{ˀ[n]ahwéɬ / \mbox{-}ʔahwéɬ} (\citealt{JC14b}: 91; \citealt{ND09}: 162; \citealt{JC18}) {\sep} \wordng{PW}{*ˀ[n]åxʷéɬ \recvar *ˀ[n]åxʷélʰ} [1] > \wordng{LB}{[n]ohʷeˀl} [2]; \wordng{’Wk}{ˀ<n>åxʷéɬ / [hi]ˀ<n>ǻxʷɬ\mbox{-} / [hi]ˀ<n>ǻxʷen̥\mbox{-}} (\citealt{JB09}: 53; \citealt{KC16}: 48–49)

\dicnote{The variant \intxt{*[n]åxʷélʰ}, which does not match the Chorote cognate, is reconstructed based on the ’Weenhayek allomorph with \textit{n̥}, as in \wordnl{[hi]ˀnǻˈxʷen̥\mbox{-}oʔ}{s/he feels ashamed in front of}.}%1

\dicnote{The Lower Bermejeño reflex is attested as \intxt{nohʷeˀl} in \citet[53]{JB09}, but this must be a mistranscription for \intxt{ˀnohʷeɬ}.}%2

\PMlemma{{\wordnl{*ˀ[j]o}{ripe} (ChW)}}

\wordng{PCh}{*ˀ[j]ó\mbox{-}ʔeʔ} > \wordng{Ijw}{ˀ[j]ó\mbox{-}ˀweʔ}; \wordng{I’w}{jó\mbox{-}weʔ} [1]; \wordng{Mj}{ˀ[j]ó\mbox{-}ˀweʔ} (\citealt{ND09}: 166; \citealt{AG83}: 135; \citealt{JC18}) {\sep} \wordng{PW}{*ˀ[j]o} > \wordng{LB}{ˀ[j]u}; \wordng{’Wk}{ˀ[j]oʔ} (\citealt{VN14}: 349; \citealt{KC16}: 127)

\dicnote{\citet[166]{ND09} mistranscribes the Iyojwa’aja’ reflex as \intxt{ˀ[j]ó\mbox{-}ˀwe}.}%1

\dicnote{The absence of glottalization in \intxt{j} and \intxt{w} in the Iyo’awujwa’ reflex must be a mistranscription on \cits{AG83} part.}%2

\rej{\citet[307]{PVB13a} lists \word{Nivaĉle}{[j]iˀj / \mbox{-}ʔiˀj}{to be vigorous, ripe} \citep[139]{JS16} under this etymology, an impossible comparison from a phonological point of view.}

\gc{\citet[307]{PVB13a} compares the Mataguayan root with \word{Proto-Guaicuruan}{*\mbox{-}eji}{to become ripe, to bear fruit, to be ripe} (\citealt{PVB13b}, \#199), which could be spurious.}

\lit{\citealt{EN84}: 12 (\intxt{*jɔ}); \citealt{PVB13a}: 307 (\intxt{*\mbox{-}juʔ})}

\PMlemma{{\wordnl{*\mbox{-}ʔóˀthale(ʔ) \recind *\mbox{-}ʔóˀthåle(ʔ)}{heart} [1] (ChW)}}

\wordng{PCh}{*\mbox{-}ʔóhtaleʔ \recind *\mbox{-}ʔóhtåleʔ} > \wordng{Ijw}{\mbox{-}ʔɔ́tale} [2], \textit{\mbox{-}ʔɔ́tahl\mbox{-}aʔ}; \wordng{I’w}{\mbox{-}óhteleʔ \recind \mbox{-}óhtaleʔ\plf{\mbox{-}óhtale\mbox{-}j}}; \wordng{Mj}{\mbox{-}ʔɔ́hteleʔ \recind \mbox{-}ʔɔ́htaleʔ\pl{l}} (\citealt{ND09}: 156; \citealt{AG83}: 154, 191; \citealt{JC18}) {\sep} \wordng{PW}{*\mbox{-}t\mbox{-}’otle} > \wordng{LB}{\mbox{-}t\mbox{-}’utle}; \wordng{Vej}{\mbox{-}t\mbox{-}’otle} [3]; \wordng{’Wk}{\mbox{-}t\mbox{-}’ótleʔ\pl{lis}} (\citealt{VN14}: 97; \citealt{VU74}: 78; \citealt{KC16}: 99)

\dicnote{This stem is likely derived from \wordng{PM}{*\mbox{-}ʔoˀt \recind *\mbox{-}ʔóˀt}\gloss{chest}.}%1

\dicnote{The absence of a final glottal stop in \wordng{Ijw}{\mbox{-}ʔɔ́tale} is unexpected.}%2

\dicnote{\citet[61]{MG-MELO15} document \wordng{Vej}{\mbox{-}t\mbox{-}’oltle}, which could be a typo.}%3

\rej{\citet[42]{EN84} includes \wordng{Ni}{\mbox{-}ɬiˀβte}\gloss{heart} under this etymology, but this is absolutely impossible for phonological reasons.}

\lit{\citealt{EN84}: 42 (\intxt{*t’ɔwtlɛ})}
\end{adjustwidth}
\section{Wichí and Iyojwa’aja’} \label{wiijw}

\begin{adjustwidth}{6mm}{0pt}
The etymologies listed in this section have a very restricted distribution, limited to Wichí and the Iyojwa’aja’ variety of Chorote. It is highly likely that in most or all of these cases, Iyojwa’aja’ borrowed from Wichí (and in a couple of cases, it is probable that both Iyojwa’aja’ and Wichí borrowed from a common third source). In fact, it is often possible to show that such loans replaced Proto-Chorote terms with a \textit{bona fide} Mataguayan etymology (\wordng{PCh}{*ʔisáh} or \wordnl{*ʔisǻh}{sand}, \wordnl{*kʲús\mbox{-}\APPL}{to be hot}, \wordnl{*kʲújʔ}{cold}, \wordnl{*núʔuh}{dog}, \wordnl{*ʔahwúʔ}{woman} vs. \wordng{Ijw}{hɔ́loʔ}, \textit{kʲóˀjo}, \textit{tέtʃah\mbox{-}aʔ}, \textit{ʔaléna}, \textit{ʔaséhnʲaʔ}). The fact that in most cases Iyojwa’aja’ and Wichí terms display regular sound correspondences is hardly surprising given that the correspondences are largely trivial.

\Ijwlemma{\wordng{Ijw}{[j]éhwut} /\mbox{-}áhwut/\gloss{to fan, to blow} \citep[159]{ND09}}

← \word{PW}{*[j]áxʷut / *[j]áxʷ(u)t\mbox{-}\APPL}{to blow} > \wordng{LB}{[j]afʷit} [1]; \word{Guisnay}{j\mbox{-}ahʷt\mbox{-}ʰi\mbox{-}tah\plf{j\mbox{-}ahʷt\mbox{-}ʰi\mbox{-}ta\mbox{-}s}}{wind}; \wordng{’Wk}{[j]áxʷ(u)t\mbox{-}\APPL}, \wordnl{j\mbox{-}axʷt\mbox{-}ʰi\mbox{-}tax}{North; north wind} (\citealt{JB09}: 62; \citealt{MG-MELO15}: 44; \citealt{KC16}: 524–525)

\dicnote{Lower Bermejeño unexpectedly reflects \sound{PW}{*u} as \intxt{i} rather than \intxt{e} here.}%1

\Ijwlemma{\wordng{Ijw}{\mbox{-}éliʔ} /\mbox{-}íle/\gloss{bone} \citep[130]{ND09}}

← \word{PW}{*\mbox{-}ɬ\mbox{-}íle\pla{jʰ}}{bone, branch} > \wordng{LB}{\mbox{-}ɬ\mbox{-}ile}; \wordng{Vej}{\mbox{-}ɬ\mbox{-}ile\pl{j}}; \wordng{’Wk}{\mbox{-}ɬ\mbox{-}íle\pl{ç}} (\citealt{VN14}: 348; \citealt{VU74}: 66; \citealt{KC16}: 75)

\lit{\citealt{EN84}: 36 (\intxt{*ele})}

\Ijwlemma{\wordng{Ijw}{\mbox{-}ép} /\mbox{-}íp/\gloss{side} \citep[130]{ND09}}

← \word{PW}{*\mbox{-}ɬ\mbox{-}íp\pla{ejʰ}}{side, part} > \wordng{LB}{\mbox{-}ɬ\mbox{-}ip\pl{ej}}; \word{Vej}{\mbox{-}ɬ\mbox{-}ip}{some, few}; \wordng{’Wk}{\mbox{-}ɬ\mbox{-}íp\pl{eç}} (\citealt{VN14}: 213, 414; \citealt{VU74}: 66; \citealt{MG-MELO15}: 8, 9; \citealt{KC16}: 75)

\lit{\citealt{EN84}: 17 (2~\intxt{*a\mbox{-}ep})}

\Ijwlemma{\wordng{Ijw}{hɔ́loʔ} /hólo/\gloss{sand} \citep[128]{ND09}}

← \wordng{PW}{*hólo} > \wordng{LB}{hulu}; \wordng{Vej}{holo\mbox{-}tah}; \wordng{’Wk}{hóloʔ\pl{lis}} (\citealt{VN14}: 161; \citealt{VU74}: 57; \citealt{MG-MELO15}: 43; \citealt{KC16}: 152)

\lit{\citealt{EN84}: 33 (\intxt{*hnɔlo}); \citealt{PVB02}: 144 (\intxt{*χolo})}

\Ijwlemma{\wordng{Ijw}{hwatέˀn} /hwatén/\gloss{sachapera\species{Acanthosyris falcata}} \citep[133]{ND09}}

← \wordng{PW}{*xʷitén} > \wordng{LB}{fʷiχten \recind fʷisten} [1]; \wordng{Southeastern (Salta)}{fʷiten}; \wordng{’Wk}{xʷitén̥}\gloss{kind of wild fruit}, \wordnl{xʷitén̥\mbox{-}tax}{sachapera} (\citealt{CS08}: 60; \citealt{MS14}: 334; \citealt{KC16}: 171)

\dicnote{The Lower Bermejeño reflex is irregular.}%1

\Ijwlemma{\wordng{Ijw}{hwék\mbox{-}hwék} /hwík\mbox{-}hwík/\gloss{red\mbox{-}billed scythebill} \citep[133]{ND09}}

← \wordng{PW}{*wíq\mbox{-}wiq} > \wordng{LB}{wiq\mbox{-}wiq}; \wordng{’Wk}{wík\mbox{-}wik\mbox{-}tax} (\citealt{CS-FL-PR-VN13}; \citealt{KC16}: 485)

\Ijwlemma{\wordng{Ijw}{hwétina\plf{hwétihna\mbox{-}s}} /hwítenah/\gloss{firefly} \citep[133]{ND09}}

← \wordng{PW}{*xʷítånax}, *\textit{xʷítånha\mbox{-}s} > \wordng{LB}{fʷitonaχ}; \wordng{’Wk}{xʷítånax}, \textit{xʷítån̥a\mbox{-}s} (\citealt{JB09}: 43; \citealt{KC16}: 170)

\lit{\citealt{EN84}: 42 (\intxt{*hwethna}); \citealt{PVB02}: 144 (\intxt{*xʷetenaχ})}

\Ijwlemma{\wordng{Ijw}{[ʔi]hwíˀn\mbox{-}i / \mbox{-}hwéˀn\mbox{-}i} /\mbox{-}hwín+ʔeh/\gloss{to braid} \citep[100]{ND09}}

← \wordng{PW}{*[ʔi]xʷin} > (?) \word{LB}{\mbox{-}fʷin\mbox{-}aχ}{line}; \word{Vej}{\mbox{-}hʷin}{to line up}; \word{’Wk}{[ʔi]xʷin̥}{to interweave, to intertwine} (\citealt{JB09}: 43; \citealt{VU74}: 59; \citealt{KC16}: 170)

\Ijwlemma{\wordng{Ijw}{[j]ímiˀn / \mbox{-}émiˀn} /\mbox{-}ímin/\gloss{to love} \citep[159]{ND09}}

← \wordng{PW}{*[ji]húmin} > \wordng{LB}{[ji]hemin}; \wordng{Vej}{\mbox{-}humin}; \wordng{’Wk}{[ja]húmin̥} (\citealt{VN14}: 308; \citealt{VU74}: 58; \citealt{KC16}: 156)

\lit{\citealt{EN84}: 10, 40 (\intxt{*hmi})}

\Ijwlemma{\wordng{Ijw}{[j]íp’is / \mbox{-}ép’is} /\mbox{-}íp’is/\gloss{to be full, satisfied} \citep[160]{ND09}}

← \wordng{PW}{*[j]íp’is} > \wordng{LB}{[j]ip’is} \citep[49]{VN14}

\Ijwlemma{\wordng{Ijw}{[j]íxsit / \mbox{-}éxsit} /\mbox{-}ísit/\gloss{to cut} \citep[160]{ND09}}

← \wordng{PW}{*[j]ísit \recvar *[j]íset / *[j]íst\mbox{-}} [1] > \wordng{LB}{[j]iset / [j]ist\mbox{-}}; \wordng{Vej}{[j]isit}; \wordng{’Wk}{[j]ísit / [j]íst\mbox{-}} (\citealt{VN14}: 234, 406; \citealt{VU74}: 84; \citealt{KC16}: 548)

\dicnote{The Lower Bermejeño form points to \wordng{PW}{*[j]íset}; Vejoz and ’Weenhayek to \intxt{*[j]ísit}.}%1

\Ijwlemma{\wordng{Ijw}{kaláp’i<tʲe>\plf{kaláp’i<tʲeh>\mbox{-}es}} /kaláp’i<tah>/\gloss{plumbeous ibis} \citep[134]{ND09}}

← \wordng{PW}{*qalá(q)p’ih} [1] > \wordng{LB}{qalaqp’i}; \wordng{’Wk}{qaláp’ih} (\citealt{CS-FL-PR-VN13}; \citealt{KC16}: 307)

\dicnote{The Lower Bermejeño form points to \wordng{PW}{*\mbox{-}qp’\mbox{-}} and ’Weenhayek to \intxt{*\mbox{-}p’\mbox{-}}.}%1

\Ijwlemma{\wordng{Ijw}{[ʔi]síˀm / \mbox{-}kíˀm} /\mbox{-}kʲím/\gloss{to be thirsty} \citep[112]{ND09}}

← \wordng{PW}{*[ʔi]kʲím} > \wordng{LB}{[ʔi]tʃim}; \wordng{’Wk}{[ʔi]kʲím̥} (\citealt{VN14}: 108; \citealt{JB09}: 86; \citealt{KC16}: 191)

\gc{Possibly related to \word{Proto-Guaicuruan}{*\mbox{-}ák’ip}{thirst} (\citealt{PVB13b}, \#23).}

\Ijwlemma{\wordng{Ijw}{kʲóˀjo} /kʲóʔjoh/\gloss{hot} \citep[136]{ND09}}

← \wordng{PW}{*kʲájo} > \wordng{LB}{[ni]tʃaju}; \wordng{Vej}{tʃajo}; \wordng{’Wk}{kʲájoʔ} (\citealt{VN14}: 217; \citealt{VU74}: 52; \citealt{KC16}: 185)

\Ijwlemma{\wordng{Ijw}{páhnaʔ} /pǻhnå/\gloss{pepper} \citep[143]{ND09}}

← \wordng{PW}{*pǻnhån} > \wordng{LB}{pon̥on}; \wordng{Vej}{pånån} [1]; \wordng{’Wk}{pǻn̥ån̥} (\citealt{CS08}: 60; \citealt{VN14}: 197; \citealt{VU74}: 70; \citealt{KC16}: 285)

\dicnote{The voiced \textit{n} in \cits{VU74} attestation of the Vejoz reflex must be a mistranscription.}%1

\rej{\citet[17, 49]{EN84} includes \wordng{Ni}{ojintʃe\pl{j}} \citep[208]{JS16}, but there are no regular correspondences between Nivaĉle and the other languages.}

\lit{\citealt{EN84}: 17, 49 (\intxt{*på\mbox{-}ahn\mbox{-}åjn})}

\Ijwlemma{\wordng{Ijw}{palak} /pålak/\gloss{brown cachalote\species{Pseudoseisura lophotes}} \citep[143]{ND09}}

← \wordng{PW}{*pålaχ \recind *pǻlaχ \recind *påláχ} > \wordng{LB}{pulaχ} [1]; \word{Vejoz or Guisnay}{pålah}{hoopoe} [2] (\citealt{CS08}: 60; \citealt{VN14}: 197; \citealt{VU74}: 70; \citealt{KC16}: 285)

\dicnote{The vowel of the first syllable is reflected irregularly in Lower Bermejeño Wichí as \textit{u}, a development also seen in \word{LB}{putsaχ}{jabiru}.}%1

\dicnote{The gloss\gloss{hoopoe} (Spanish\gloss{abubilla}) in \citet{RL16} is obviously incorrect, since hoopoes are not natively found in South America.}%2

\Ijwlemma{\wordng{Ijw}{pɔ́p} /pop/\gloss{eared dove} \citep[144]{ND09}}

← \wordng{PW}{*póp} > \wordng{LB}{pup}; \wordng{Vej}{pop}; \wordng{’Wk}{póp} (\citealt{VN14}: 157; \citealt{MG-MELO15}: 22; \citealt{KC16}: 295)

\Ijwlemma{\wordng{Ijw}{\mbox{-}sát\pl{is}} /\mbox{-}sat/\gloss{foot} \citep[125]{ND09}}

← \wordng{PW}{*\mbox{-}sat}\gloss{heel} > \wordng{Vej}{\mbox{-}sat}\gloss{heel}; \wordng{’Wk}{\mbox{-}såt}, \textit{\mbox{-}sǻt\mbox{-}aç}\gloss{tendon, heel} [1] (\citealt{VU74}: 72; \citealt{KC16}: 90)

\dicnote{’Weenhayek shows contamination of \word{PW}{*\mbox{-}sat}{heel} and \wordnl{*\mbox{-}såt}{tendon}, which has resulted in a polysemic noun \textit{\mbox{-}såt}\gloss{tendon, heel}.}%1

\Ijwlemma{\wordng{Ijw}{tέtʃah\mbox{-}aʔ} [1]\gloss{cold} \citep[149]{ND09}}

← \wordng{PW}{*tékʲåχ} > \wordng{LB}{[ni]tetʃoχ(\mbox{-}tʃe/=hi)}; \wordng{Vej}{\mbox{-}tetʃah\mbox{-}tʃe}; \wordng{’Wk}{tékʲåx} (\citealt{VN14}: 283; \citealt{VU74}: 75; \citealt{KC16}: 392)

\dicnote{\intxt{tʃ} is not a native phoneme of Iyojwa’aja’.}%1

\Ijwlemma{\wordng{Ijw}{tihwána} /tᵊhwánah/\gloss{Molina's hog-nosed skunk} \citep[150]{ND09}}

← \wordng{PW}{*túxʷanaχ} > \wordng{Vejoz or Guisnay}{tuhwanah \recind tuhwenah}; \wordng{’Wk}{túxʷanax} (\citealt{RL16}: 90; \citealt{KC16}: 420)

\Ijwlemma{\wordng{Ijw}{sihnát} /sᵊhnát/\gloss{knife} (\citealt{JC14b}: 99; \citealt{ND09}: 145)}

← \wordng{PW}{*tsonhat} > \wordng{Vej}{tsonat}; \wordng{’Wk}{tson̥at\plf{tson̥át\mbox{-}es}} (\citealt{VU74}: 55; \citealt{KC16}: 466)

\dicnote{The voiced \intxt{n} in \cits{VU74} attestation of the Vejoz reflex must be a mistranscription.}%1

\Ijwlemma{\wordng{Ijw}{wóna wúmki\mbox{-}na} /wónah wúmkʲV-nah/\gloss{crane hawk\species{Geranospiza caerulescens}} (\citealt{ND09}: 157)}

← \wordng{PW}{*wóˀnah wúmeq} [1] > \wordng{LB}{wuˀna wemek}; \wordng{Vejoz or Guisnay}{woˀna wumek}; \wordng{’Wk}{wóˀna\mbox{-}wumek} (\citealt{CS-FL-PR-VN13}; \citealt{RL16}: 105; \citealt{KC16}: 488)

\dicnote{In Wichí, this is a transparent compound of \word{PW}{*wóˀnah}{\textit{bala} wasp\species{Polybia ruficeps} honey(comb); hat} and \wordnl{*\mbox{-}wúmeq\plf{\mbox{-}wumh\mbox{-}ajʰ}}{old}.}%1

\Ijwlemma{\wordng{Ijw}{\mbox{-}ˀwúk}, \textit{\mbox{-}ˀwúk\mbox{-}iˀl} /\mbox{-}ʔwúk/\gloss{house} (\citealt{JC14b}: 96; \citealt{ND09}: 128)}

← \word{PW}{*\mbox{-}wúkʷ\plf{*\mbox{-}wuh\mbox{-}ujʰ}}{owner} > \wordng{LB}{\mbox{-}wekʷ\plf{\mbox{-}wehe\mbox{-}j}}; \wordng{Vej}{\mbox{-}wuk\plf{\mbox{-}wuh\mbox{-}uj}}; \wordng{’Wk}{\mbox{-}wuk},\textit{\mbox{-}wuh\mbox{-}uç}; \textit{*\mbox{-}wúkʷ\mbox{-}e\pla{jʰ}}\gloss{house} > \wordng{LB}{\mbox{-}wekʷ\mbox{-}e}; \wordng{Vej}{\mbox{-}wuk(ʷ)\mbox{-}e}; \wordng{’Wk}{\mbox{-}wúk\mbox{-}eʔ\pl{ç}} (\citealt{VN14}: 192; \citealt{JB09}: 61; \citealt{VU74}: 82; \citealt{MG-MELO15}: 152; \citealt{KC16}: 103)

\Ijwlemma{\wordng{Ijw}{ʔahwijeta\plf{ʔahwihjeta\mbox{-}}} /ahwihatah/\gloss{\textit{mojarra} fish\species{Cheirodon interruptus}} (\citealt{JC14b}: 91; \citealt{ND09}: 94)}

← \wordng{PW}{*ʔáxʷetaχ} > \wordng{Vej}{ahwetah} (\citealt{RL16}: 15)

\Ijwlemma{\wordng{Ijw}{ʔaléna\pl{s}} /alínah/ [1]\gloss{dog} (\citealt{JC14b}: 999; \citealt{ND09}: 94)}

Possibly borrowed from a source identical or close to that of \wordng{PW}{*ʔasínåχ\plf{*ʔasínhå\mbox{-}s}} > \wordng{LB}{ʔasinoχ\plf{ʔasin̥o\mbox{-}s}}; \wordng{Vej}{asinåh\plf{asin̥å\mbox{-}s}}; \wordng{’Wk}{ʔasínåx\plf{ʔasín̥å\mbox{-}s}} (\citealt{VN14}: 191; \citealt{MG-MELO15}: 20; \citealt{KC16}: 15).

\dicnote{The absence of palatalization in \wordng{Ijw}{\mbox{-}n\mbox{-}} in this word is synchronically irregular.}%1

\dicnote{\citet[51]{VU74} documents \textit{asinah}, which must be a mistranscription.}%2

\Ijwlemma{\wordng{Ijw}{ʔaséhnʲaʔ} /asíhna/\gloss{woman} (\citealt{JC14b}: 91; \citealt{ND09}: 94)}

← \wordng{PW}{*ʔatsínha\pla{jʰ}} [1] > \wordng{LB}{ʔatsin̥a\pl{j}}; \wordng{Vej}{atsin̥a} [2]; \wordng{’Wk}{ʔatsín̥aʔ\pl{ç}} (\citealt{VN14}: 285, 303; \citealt{MG-MELO15}: 29; \citealt{KC16}: 18)

\dicnote{The Wichí noun itself is likely derived from \intxt{*ʔásnaq} (if from \intxt{*ʔátsinak}, vocalic stem \intxt{*ʔátsinha\mbox{-}}) > \word{LB}{ʔasnaq}{male} \citep[197]{VN14}.}%1

\dicnote{\citet[50]{VU74} documents \textit{atsina}, which must be a mistranscription.}%2

\Ijwlemma{\wordng{Ijw}{ʔáxseˀni\pl{wa}} /ǻseˀnih/\gloss{guira cuckoo} \citep[94]{ND09}}

← \wordng{PW}{*hǻtseˀnih} > \wordng{LB}{hotseˀni}; \wordng{’Wk}{hǻtsaˀnih \recind hǻtseˀnih} [1] (\citealt{CS-FL-PR-VN13}; \citealt{KC16}: 139)

\dicnote{The variant \intxt{hǻtsaˀnih} in ’Weenhayek is irregular.}%1

\rej{\wordng{I’w}{áxsina\pl{s}}, \wordng{Mj}{ʔáxsena\pl{s}}\gloss{quebracho crested tinamou} (\citealt{AG83}: 124; \citealt{JC18}) must be unrelated, despite apparent formal similarity. The only thing guira cuckoos and quebracho crested tinamous have in common is that both species are crested, but otherwise these birds are quite different.}

\Ijwlemma{\wordng{Ijw}{ʔipʲáta} /ipǻtah/ \citep[109]{ND09}}

← \wordng{PW}{*ʔixpát} > \wordng{Vej}{ihpat\pl{ɬajis}}; \wordng{’Wk}{ʔixpát} (\citealt{VU74}: 60; \citealt{MG-MELO15}: 18; \citealt{KC16}: 24)

\lit{\citealt{EN84}: 9, 26 (\intxt{*iphåtha})}

\Ijwlemma{\wordng{Ijw}{ʔisʲáˀni\pl{wa}} /isǻʔnih/\gloss{narrow-billed woodcreeper} \citep[111]{ND09}}

← \wordng{PW}{*xʷitsǻˀnih} > \wordng{LB}{fʷitsoˀni}; \wordng{’Wk}{xʷitsǻˀnih} (\citealt{CS-FL-PR-VN13}; \citealt{KC16}: 171)

\Ijwlemma{\wordng{Ijw}{ʔɔ́hnaʔ} /óhna/\gloss{sachasandía (\intxt{Capparis salicifolia}) fruit}; \textit{ʔɔ́hna\mbox{-}k} /óhna\mbox{-}k/\gloss{sachasandía (\intxt{Capparis salicifolia}) tree} \citep[142]{ND09}}

← \wordng{PW}{*ʔónhaʔ}; \textit{*ʔónha\mbox{-}q \recvar *ʔónha\mbox{-}kʷ} [1] > \wordng{LB}{ʔun̥a\mbox{-}q}; \wordng{Vej}{on̥a\mbox{-}j}\gloss{sachapera}, \textit{on̥a\mbox{-}ɬile}\gloss{sachasandía}; \wordng{’Wk}{ʔón̥aʔ}; \textit{ʔón̥a\mbox{-}k} (\citealt{CS08}: 61; \citealt{VN14}: 348; \citealt{MG-MELO15}: 18; \citealt{KC16}: 46)

\dicnote{\wordng{LB}{ʔun̥a\mbox{-}q} points to \wordng{PW}{*ʔónha\mbox{-}q}, \wordng{’Wk}{ʔón̥a\mbox{-}k} to \intxt{*ʔónha\mbox{-}kʷ}.}%1

\rej{\word{Maká}{inhek}{vinal\species{Prosopis ruscifolia}} \citep[202]{AG83} cannot be related, as \sound{Mk}{i} cannot correspond to \sound{PW}{*o}.}
\end{adjustwidth}
\section{Possible borrowings and Wanderwörter} \label{wander}

\begin{adjustwidth}{6mm}{0pt}
The etymologies listed in this section show too irregular correspondences to allow for a reconstruction of a Proto-Mataguayan etymon. In some cases, evidence from neighboring languages suggests that horizontal transmission, as opposed to cognation, may account for the similarity between the forms.

\wwort{\enquote{to help}:}

\wordng{Mk}{[ji]fen} \citep[173]{AG99} {\sep} \wordng{Ni}{[j]eɸen / \mbox{-}ʔeɸen} \citep[123]{JS16}

\wwort{\enquote{seven- or nine-banded armadillo}:}

\word{Ni}{βokotsex\plf{βokotse\mbox{-}s}}{seven-banded armadillo} (\citealt{JS16}: 364; \citealt{LC20}: 131) {\sep} \wordng{PW}{*xʷóq(’)atsaχ} > \wordng{LB}{fʷuq’atsaχ}; \wordng{Vej}{hʷok’åtsah} [1]; \word{’Wk}{xʷóq(’)atsax}{nine-banded armadillo} (\citealt{VN14}: 231; \citealt{VU74}: 59; \citealt{KC16}: 174)

\dicnote{\wordng{Vej}{hʷok’åtsah} (\citealt{VU74}: 59) is likely a mistranscription for \textit{hʷok’atsah}.}%1

\empr{Nivaĉle points to \intxt{*wóqotseχ} and Wichí to \intxt{*ɸóq(’)atseχ}.}

\lit{\citealt{EN84}: 13 (\intxt{*hwɔqɔtsha \recind *wɔqɔtsha}); \citealt{PVB02}: 144 (\intxt{*xʷoqotsaχ})}

\wwort{\enquote{Azara’s capuchin\species{Sapajus cay paraguayanus}}:}

\wordng{Mk}{k’ateni} (\citealt{AG99}: 235) {\sep} \wordng{PW}{*hǻtåˀnih \recvar *hǻtaˀnih} [1] > \wordng{LB}{\mbox{hotoˀni}}; \wordng{Vejoz or Guisnay}{hǻtåˀni}; \wordng{’Wk}{hǻtaˀnih\plf{hǻtaˀni\mbox{-}lis}} [4] (\citealt{EM-MM-19}; \citealt{RL16}: 36; \citealt{VU74}: 59, 63; \citealt{KC16}: 138)

\dicnote{Different Wichí dialects point to different root-medial vowels: ’Weenhayek suggests the reconstruction \intxt{*hǻtaˀnih}, which matches the Maká form somewhat better, whereas other varieties point to \intxt{*hǻtåˀnih}.}%1

\lit{\citealt{PVB02}: 146 (\intxt{*k’ʌtʌni \recind *χʌtʌni})}

\wwort{\enquote{bare-faced curassow\species{Crax fasciolata}}:}

\wordng{Mk}{hehe} \citep[58]{JB81} {\sep} \wordng{Ni}{xexe\pl{k}} \citep[148]{JS16}

\wwort{\enquote{\textit{yica} bag}:}

\wordng{PCh}{*\mbox{-}hílijʔ \recind *\mbox{-}hílujʔ\pla{is}} > \wordng{Ijw}{<hl>éliʔ\pl{jis}}; \wordng{I’w}{\mbox{-}éliʔ\pl{jis}}; Mj~\third{hl\mbox{-}éilijʔ} (\citealt{ND09}: 130; \citealt{AG83}: 126; \citealt{GH94}) {\sep} \wordng{PW}{*(\mbox{-})hɪ́lu\pla{lis}} > \wordng{LB}{hele\pl{lis}}; \wordng{Vej}{\mbox{-}hilu}; \wordng{’Wk}{híluʔ\pl{lis}} (\citealt{VN14}: 191; \citealt{VU74}: 57; \citealt{KC16}: 150)

\lit{\citealt{EN84}: 33 (\intxt{*hnelu})}

\wwort{\enquote{tapir}:}

\wordng{Ni}{jiˀjek͡le\pl{k}} {\sep} \wordng{PW}{*ˣjéˀlah} > \wordng{LB}{jeˀla\pl{lis}}; \wordng{’Wk}{ʔijéˀlah} (\citealt{VN14}: 191; \citealt{KC16}: 43)

\wwort{\enquote{fly} /\gloss{mosquito}:}

\word{Ni}{ɬaɸ\mbox{-}katax\plf{ɬaɸ\mbox{-}kata\mbox{-}s}}{fly}, \wordnl{ɸisin\mbox{-}katax\plf{ɸisin\mbox{-}kata\mbox{-}s}}{gnat} (\citealt{JS16}: 134, 162) {\sep} \wordng{PCh}{*qatá\mbox{-}keʔ \recind *qáta\mbox{-}keʔ\pla{jʰ}} [1] > \wordng{Ijw}{káta\mbox{-}kiʔ} [2]; \wordng{I’w}{katákiʔ\pl{ji}}; \wordng{Mj}{katákiʔ\pl{j}}; cf. also \wordng{Ijw}{hatak’i} [3]\gloss{mosquito} (\citealt{JC14b}: 91, fn. 22; \citealt{ND09}: 118, 134; \citealt{AG83}: 137; \citealt{JC18}) {\sep} \wordng{PW}{*q’átaq \recvar *\mbox{ʔátaq}} [4]\gloss{fly} > \wordng{LB}{ʔataq}; \wordng{Vej}{k’atak}; \wordng{’Wk}{q’átaq}; \wordnl{*xʷinátaq}{gnat, mosquito} > \wordng{LB}{fʷinataq}; \wordng{Vej}{hʷinatak}; \wordng{’Wk}{xʷunátaq} [4] (\citealt{JB09}: 38, 43; \citealt{VN14}: 47; \citealt{VU74}: 59, 63; \citealt{KC16}: 322)

\dicnote{Iyojwa’aja’ points to \wordng{PCh}{*qáta\mbox{-}keʔ}, and the other varieties to \intxt{*qatá\mbox{-}keʔ}, suggesting that these terms are not necessarily inherited from Proto-Chorote.}%1

\dicnote{This is mistranscribed as \intxt{káta\mbox{-}ki\pl{ʔ}} in \citet[134]{ND09}.}%2

\dicnote{\wordng{Ijw}{hatak’i} is attested only in \citet{ND09} but is absent from our corpus, making it impossible for us to decide which syllable is stressed in this noun.}%3

\dicnote{Lower Bermejeño points to \wordng{PW}{*ʔátaq}, and the other varieties to \intxt{*q’átaq}, suggesting that these terms are not necessarily inherited from Proto-Wichí.}%4

\dicnote{\sound{’Weenhayek}{u} is not the regular reflex of \sound{PW}{*i}.}%5

\rej{\citet[15]{LC-VG-07} also include \word{Maká}{qaχtets\pl{its}}{horsefly} \citep[305]{AG99}, which is hardly related.}

\lit{\citealt{EN84}: 23,34 (\wordnl{*qataq}{fly}, \wordnl{*hwinhnatha}{mosquito}); \citealt{LC-VG-07}: 15}

\wwort{\enquote{ray (fish)}:}

\wordng{Mk}{k’ejejkiʔ\pl{l}} \citep[236]{AG99} {\sep} \wordng{Ni}{k’ijejke\pl{k}} \citep[228]{JS16}

\wwort{\enquote{smooth-billed ani\species{Crotophaga ani}}:}

\wordng{Ni}{k'onxaʔ} \citep[118]{LC20} {\sep} \wordng{PW}{*kʲ’inhå \recind *kʲ’ínhå \recind *kʲ’inhǻ} > \wordng{LB}{tʃ’in̥o} \citep{CS-FL-PR-VN13}

\wwort{\enquote{black-legged seriema\species{Chunga burmeisteri}}:}

\wordng{Ijw}{nókʲ’u\pl{s}} [1]; \wordng{I’w}{ohónʲukʲuʔ \recind ohónʲukʲuh\pl{us}}\gloss{red\mbox{-}legged seriema} [1]; \wordng{Mj}{hʊ́n(i)ʔi \recind hʊ́niʔu}, \textit{hʊ́nʔi\mbox{-}is} [1] (\citealt{ND09}: 141; \citealt{AG83}: 153, 194; \citealt{JC18}) {\sep} \wordng{PW}{*ˣnɪ́kʲ’u} > \wordng{LB}{netʃ’e}; \wordng{’Wk}{ʔiníkʲ’uʔ} (\citealt{VN14}: 170; \citealt{KC16}: 32)

\dicnote{Iyojwa’aja’ points to \wordng{PCh}{*núk’uh}, Iyo’awujwa’ to \intxt{*uhújnukuh \recind *uhújnukuʔ}, and Manjui to \intxt{*húnk’uh}, suggesting that these terms are not necessarily inherited from Proto-Chorote. It is admittedly possible to reconstruct a PChW form similar to \intxt{*Xúnjuk’uh} or maybe \intxt{*Xunjúk’uh}, but in this case it is not clear how to reconstruct the hypothetical PCh form.}%1

\wwort{\enquote{sweet potato} (MN) /\gloss{manioc} (W):}

\wordng{Mk}{peχejeʔ}; \textit{peχeje\mbox{-}k}, \textit{peχeje\mbox{-}ket} \citep[295]{AG99} {\sep} \wordng{Ni}{pexaja\pl{k}}; \textit{pexaja\mbox{-}juk}, \textit{pexaja\mbox{-}ku\mbox{-}j} \citep[218]{JS16} {\sep} \wordng{PW}{*piˀjókʷ} > \wordng{’Wk}{piˀjók} \citep[292]{KC16}

\empr{The Maká and Nivaĉle forms cannot be cognate because the expected reflex of \wordng{PM}{*e} before a uvular is \wordng{Maká}{a}, not \textit{e}. \citet[300]{PVB13a} and \citet[307]{AF16} note the similarity with \wordng{Proto-Guaicuruan}{*pijóko}\gloss{manioc} (\citealt{PVB13b}, \#487), \word{Ayoreo}{peheei}{manioc}, and the Enlhet–Enenlhet term for\gloss{sweet potato} – \wordng{Enlhet, Angaité}{pehejaʔ}, \wordng{Enxet}{peheːje \recind pehejeʔ \recind peheʔ}, Enenlhet-Toba, Sanapaná, \wordng{Guaná}{pejaʔ}\gloss{sweet potato} (\citealt{EU-HK-97}: 549; \citealt{EU-HK-MR-03}: 334; \citealt{PW20}: 48; \citealt{JE21}: 33, 97, 730; \citealt{HK-23}: 180) – which is attributed to language contact. Of these, the ’Weenhayek word is most similar to the Guaicuruan forms, whereas Maká and Nivaĉle display more similarity with the data of Ayoreo and Enlhet–Enenlhet languages.}

\rej{\citet[38]{EN84} derives \wordng{Ni}{pexaja} from \wordng{PM}{*pɛwhla}, which is claimed to be the etymon of \word{Chorote}{hwélʲe\mbox{-}tʲ’o}{potato} (a reflex of \wordng{PM}{*ɸílå(ˀ)X₁₂} in our account), \word{Ni}{ʃek͡låx}{\textit{sutia} fruit\species{Solanaceae}} (a reflex of \wordng{PM}{*xélåX₁₂} in our account), and \word{Wichí}{weltsitax}{tobacco (in old times)}, a term we were unable to locate in other published sourced on Wichí.}

\lit{\citealt{PVB02}: 145; \citealt{PVB13a}: 300}

\wwort{\enquote{kind of jay\species{Cyanocorax sp.}}:}

\word{Mk}{qolom\mbox{-}qolom\pl{its}}{a kind of jay larger than the plush-crested jay\species{Cyanocorax chrysops}; makes elongated hanging nests} (\citealt{JB81}: 64; \citealt{AG99}: 233) {\sep} \word{Ni}{kok͡lop\pl{is}}{kind of a black weaving bird}; \wordnl{kok͡lop\mbox{-}itax\plf{kok͡lop\mbox{-}ita\mbox{-}s}}{purplish jay\species{Cyanocorax cyanomelas}} (\citealt{JS16}: 70; \citealt{LC20}: 506)

\wwort{\enquote{cane\species{Arundo donax}}:}

\wordng{Ni}{sise\pl{k}} \citep[233]{JS16} {\sep} \wordng{Ijw}{sisέh\pl{ˀl}}; \wordng{I’w}{sisé} (\intxt{\mbox{-}jis \recind \mbox{-}hes}) [1]; \wordng{Mj}{ʃisέh\pl{k}} [1] (\citealt{ND09}: 146; \citealt{AG83}: 159; \citealt{JC18})

\dicnote{The plural forms attested in Iyo’awujwa’ and Manjui do not match the Iyojwa’aja’ and Nivaĉle data (nor do they match each other).}%1

\empr{The Chorote form is likely a recent Nivaĉle loan, as suggested by the fact that the vowel \intxt{i} in the first syllable fails to trigger the first and the second palatalizations, as well as by the Manjui plural form.}

\lit{\citealt{EN84}: 41 (\intxt{*s\mbox{-}sɛ})}

\wwort{\enquote{spider}:}

\wordng{Mk}{siˀwalaχ\pl{its}} (\citealt{AG99}: 327; \citealt{PMA}: 15) {\sep} \wordng{Ni}{siβåk͡låk}, \textit{siβåk͡låk͡l\mbox{-}is} (\citealt{JS16}: 233–234) {\sep} \wordng{PCh}{*sᵊwǻlåk}, \textit{*sᵊwǻlåq\mbox{-}is} > \wordng{Ijw}{\mbox{siwálak}}; \wordng{I’w}{siwálak \recind ʃiwálak\pl{es}}; \wordng{Mj}{ʃiwálak\pl{is}} (\citealt{ND09}: 146; \citealt{AG83}: 21, 159; \citealt{JC18})

\empr{Based on the Nivaĉle and Chorote forms, it could be possible to reconstruct \wordng{PM}{*siwǻlåq}, but the Maká form cannot be derived from this reconstruction. The discrepancy in the final consonant suggests independent borrowings from a source close to \wordng{Enlhet}{sawaːlak}, \wordng{Enxet}{sawaːlaq}, \wordng{Sanapaná}{sewaːlak}, \word{Guaná}{sewalaq}{spider} (\citealt{EU-HK-97}: 595; \citealt{PW20}: 92; \citealt{JE21}: 33; \citealt{HK-23}: 184), as suggested by \citet[307]{AF16} for Enlhet.}

\lit{\citealt{EN84}: 41 (\intxt{*s\mbox{-}wålåk}); \citealt{PVB02}: 146; \citealt{LC-VG-07}: 21; \citealt{AnG15}: 253}

\wwort{\enquote{fish, \textit{sábalo} fish}:}

\wordng{Mk}{sehets} (\citealt{AG99}: 323; \citealt{PMA}: 5) {\sep} \wordng{Ni}{saxetʃ} \citep[229]{JS16} {\sep} \wordng{PCh}{*sik’ús} > \wordng{Ijw}{siʔjús}; \wordng{I’w}{sijús} [1]\gloss{fish}; \wordng{Mj}{ʃiʔʲús \recind ʃiˀjús} (\citealt{JC14b}: 90; \citealt{ND09}: 147; \citealt{AG83}: 158; \citealt{JC18}) {\sep} \wordng{PW}{*sikʲ’ús}\gloss{\textit{sábalo} fish} > \wordng{Guisnay}{sitʃ’us}; \wordng{’Wk}{sikʲ’ús} (\intxt{\mbox{-}ɬajis}) (\citealt{RL16}: 78; \citealt{MG-MELO15}: 22; \citealt{KC16}: 329)

\dicnote{The seemingly plain \intxt{j} in Iyo’awujwa’ could be a mistranscription on \cits{AG83} part.}%1

\empr{Based on the Chorote and Wichí forms, it could be possible to reconstruct \wordng{PM}{*sik’ú(t)s}, but the Maká and Nivaĉle forms cannot be derived from this reconstruction.}

\lit{\citealt{EN84}: 43 (\intxt{*scutsh}); \citealt{PVB02}: 144 (\intxt{*saχets})}

\wwort{\enquote{\textit{anco} squash}:}

\wordng{Mk}{koːsinheʔ\pl{j}} \citep[232]{AG99} {\sep} \wordng{Ni}{sinxeja\mbox{-}tax}, \textit{sinxeja\mbox{-}ta\mbox{-}s} \citep[232]{JS16} {\sep} \wordng{Ijw}{ʔósinʲe}, \textit{ʔósini\mbox{-}s}; \wordng{I’w}{sihnájeʔ}; \wordng{Mj}{ʃihnájeʔ}\gloss{\textit{andaí} squash} (\citealt{ND09}: 142; \citealt{AG83}: 159; \citealt{JC18}) {\sep} \wordng{PW}{*ʔúsenha\pla{jʰ}} > \wordng{’Wk}{ʔúsen̥aʔ\pl{ç}} (\citealt{MG-MELO15}: 19; \citealt{KC16}: 46)

\empr{Maká points to \wordng{PM}{*koosenhaʔ} or \intxt{*koosinhaʔ}; Nivaĉle to \intxt{*sinheja(ʔ)}; Iyojwa’aja’ to \intxt{*\mbox{ʔúsenah}} or \intxt{*ʔúsinah}; Iyo’awujwa’ and Manjui to \intxt{*senhája(ʔ)} or \intxt{*senhája(ʔ)} (though the failure of \intxt{*n} to palatalize would remain unexplained); Wichí to \intxt{*ʔúsenha(ʔ)}. \citet{AF16} suggests that these are independent borrowings from a source close to \wordng{Enlhet}{semheːjaʔ}, \wordng{Enenlhet-Toba, Angaité, Guaná}{semhejaʔ} (\citealt{EU-HK-97}: 604; \citealt{EU-HK-MR-03}: 336; \citealt{PW20}: 38; \citealt{HK-23}: 184).}

\rej{\citet[26, 31]{EN84} includes \word{Vejoz}{amjo\mbox{-}tah}{\textit{anco} squash} (\citealt{VU74}: 50; \citealt{MG-MELO15}: 17) into the comparison, but this is impossible for phonological reasons.}

\lit{\citealt{EN84}: 26, 31 (\intxt{*(ɔtsh)ajhmɛtha})}

\wwort{\enquote{wax} [1]:}

\wordng{Ni}{\mbox{-}sup’ax\pl{is}} \citep[237]{JS16} {\sep} \wordng{PW}{*sóp’a} > \wordng{Vej}{sop’a}; \wordng{’Wk}{sóp’aʔ\plf{sóp’l\mbox{-}is}}; \wordnl{*[ʔi]sóp’a\mbox{-}n}{to stick} > \word{LB}{sup’an̥\mbox{-}i}{stew}; \word{Vej}{sop’an̥\mbox{-}i}{paste}; \wordng{’Wk}{[ʔi]sóp’an̥} (\citealt{VN14}: 310; \citealt{VU74}: 73; \citealt{KC16}: 330)

\dicnote{\citet[18]{EN84} adds \wordng{Chorote}{sóʔpa}\gloss{wax} to the comparison. We have been unable to identify any similar word either in our corpus or in published works.}%1

\lit{\citealt{EN84}: 18 (\intxt{*sɔwp’a})}

\wwort{\enquote{\textit{moro} bee honey(comb)}:}

\wordng{Ni}{(\mbox{-})ʃnakuβax\pl{is}} \citep[243]{JS16} {\sep} \wordng{PCh}{*nákowoʔ \recind *nákuwoʔ} > \wordng{Ijw}{nákiwoʔ} [1]; \wordng{I’w}{nákiwoʔ\pl{l}} (\citealt{JC14b}: 79; \citealt{ND09}: 140; \citealt{AG83}: 149) {\sep} \wordng{LB}{naquwu\mbox{-}taχ} \citep[52]{JB09}

\dicnote{This is mistranscribed as \intxt{nákiwo} in \citet[40]{ND09}.}%1

\lit{\citealt{EN84}: 34, 42 (\intxt{*hnawko(tha)}); \citealt{LC-VG-07}: 15}

\wwort{\enquote{pacu fish}:}

\wordng{PCh}{*taqám} > \wordng{Ijw}{takáˀm\pl{is}}; \wordng{I’w}{takám\pl{is}} (\citealt{ND09}: 148; \citealt{AG83}: 162) {\sep} \wordng{PW}{*tákʲam} > \wordng{Guisnay}{tatʃam}; \wordng{’Wk}{tákʲ’am̥} [1] (\citealt{RL16}: 80; \citealt{KC16}: 363)

\dicnote{The glottalization of the root-medial consonant in the ’Weenhayek reflex is unexpected.}%1

\empr{The Chorote form can only go back to \intxt{*taqam} or \intxt{*taqám}, the Wichí one to \intxt{*tákam}.}

\lit{\citealt{LC-VG-07}: 17}

\wwort{\enquote{garabato\species{Acacia praecox}}:}

\wordng{Mk}{t’okonok} \citep[346]{AG99} {\sep} \wordng{PCh}{*kútunuk} > \wordng{Ijw}{kʲút(ʲ)unʲuk \recind \mbox{kʲútinik} \recind \mbox{kʲútunuk}}; \wordng{I’w}{kʲútʲunuk \recind kʲútanuk}; \wordng{Mj}{kʲútenek \recind \mbox{kʲútunuk} \recind \mbox{kʲútanuk} \recind kʲútanek}, \textit{kʲútenki\mbox{-}j} (\citealt{ND09}: 137; \citealt{JC18}) {\sep} \wordng{PW}{*hútunukʷ} [1] > \wordng{LB}{hetenekʷ} (\citealt{CS08}: 63; \citealt{MS14}: 269)

\dicnote{\citet[269]{MS14} documents the forms \intxt{hutunuk}, \intxt{hutunekʷ}, and \intxt{hetenekʷ} in Wichí, without specifying the respective dialects.}%1

\wwort{\enquote{salt}:}

\wordng{Ni~ChL}{tsiɸoni\pl{k}} \citep[295]{JS16} {\sep} \wordng{Ijw}{sihwónʲeʔ}; \wordng{I’w}{sifʷóniʔ\pl{l}}; \wordng{Mj}{ʃihwóniʔ \recind ʃihwóneʔ} (\citealt{JC14b}: 100; \citealt{ND09}: 145; \citealt{AG83}: 158; \citealt{JC18})

\empr{\citet[295]{JS16} states that the Nivaĉle word is a Chorote loan. However, the Chorote word itself does not look native, as in Iyojwa’aja’ [nʲ] does not normally occur following a non-high vowel /o/ (unless the underlying representation is /sᵊhwójna/). The term in question could be related to \wordng{PM}{*tsóɸa}\gloss{\textit{Maytenus vitis-idaea}} (whose ashes are used for making salt) via indirect borrowing by means of unidentified languages.}

\wwort{\enquote{roseate spoonbill}:}

\wordng{Ni}{tsinɬetsex\plf{tsinɬetse\mbox{-}s}} \citep[295]{JS16} {\sep} \wordng{PCh}{*kin(al)Vsah} > \wordng{Ijw}{kinʲélisa}; \wordng{Mj}{kíniʃe} (\citealt{AG79}: 38; \citealt{ND09}: 136; \citealt{JC18}) {\sep} \wordng{PW}{*\mbox{níletsaχ}} > \wordng{LB}{niletsaχ}; \wordng{’Wk}{níletsax}, \textit{níletsa\mbox{-}s} (\citealt{VN14}: 170; \citealt{KC16}: 269)

\empr{The correspondences are too irregular to consider the aforementioned terms cognate. Nivaĉle points to \intxt{*tsinɬetseχ}, Chorote to \intxt{*kin(a)lVtseχ}, and Wichí to \intxt{*níletseχ}.}

\lit{\citealt{EN84}: 46 (\intxt{*cihnilitsha}); \citealt{PVB02}: 144 (\intxt{*kineɬitsaχ})}

\wwort{\enquote{dorado fish}:}

\wordng{Mk}{tsiwanaq\pl{its}} (\citealt{AG99}: 349; \citealt{PMA}: 5; \citealt{JB81}: 68) {\sep} \wordng{Ni}{siβånåk\plf{siβånåk͡l\mbox{-}is}} \citep[234]{JS16}

\empr{Obviously related to \word{Proto-Guaicuruan}{*ats’iwanaqa}{dorado fish} (\citealt{PVB13b}, \#143). Note that \sound{Maká}{ts} cannot regularly correspond to \sound{Nivaĉle}{s}.}

\lit{\citealt{LC-VG-07}: 22}

\wwort{\enquote{tinamou}:}

\word{Mk}{wextsoxoxo\pl{l}}{solitary tinamou\species{Tinamus solitarius}; red-winged tinamou\species{Rhynchotus rufescens}; elegant crested tinamou\species{Eudromia elegans}} (\citealt{AG99}: 371; \citealt{JB81}: 54) {\sep} \word{Ni}{tʃoxoxo\pl{xis}}{red-winged tinamou\species{Rhynchotus rufescens}} \citep[108]{JS16}

\wwort{\enquote{a Chacoan game; stick used in that game}:}

\wordng{Mk}{\mbox{-}tsukaʔ\pl{l}} \citep[350]{AG99} {\sep} Ni ‹tsukoc› \citep[157]{EN19} {\sep} \word{Mj}{ʃúkʲeʔ}{the stick}, \wordnl{ʃúkʲe\mbox{-}l}{the game} \citep{JC18} {\sep} \wordng{’Wk}{sokaʔ \recind sukaʔ}, \intxt{soká\mbox{-}lis \recind suká\mbox{-}lis} \citep[330]{KC16}

\empr{A similar game is played by many other peoples of the Chaco (cf. \wordng{Tapiete}{ʃuka}, \citealt{HG05}: 359), and is ultimately of Andean origin. \citet[157]{EN19} suggests that its name derives from \word{Quechua}{tʃunka}{ten; a game of chance}.}

\wwort{\enquote{white-barred piculet\species{Picumnus cirratus}}:}

\wordng{Mk}{tsxini(ˀ)n\plf{tsxinin\mbox{-}its}} \citep[350]{AG99} {\sep} \wordng{Ni}{tsiniˀni\pl{k}} [1] (\citealt{JS16}: 295; \citealt{LC20}: 502) {\sep} \wordng{Ijw}{ʔéskiniˀni \recvar ʔέskiniˀni} [1] \citep[96]{ND09}

\dicnote{The Iyojwa’aja’ term is not documented in our data, and \citet{ND09} does not distinguish between /i/ [e] and /e/ [ɛ], hence the uncertainty.}%1

\wwort{\enquote{great antshrike\species{Taraba major}}:}

\wordng{Ni}{ts’iˀjok͡lok͡lo} (\citealt{LC20}: 506) \recind \intxt{ts’ijoxok͡lå \recind ts’ijokåk͡lå \recind ts’ijokåk͡lo} \citep[303]{JS16} {\sep} \wordng{PW}{*ts’ólo\mbox{-}taχ} > \wordng{LB}{ts’ulu\mbox{-}taχ}; \wordng{’Wk}{ts’ólo\mbox{-}tax} (\citealt{CS-FL-PR-VN13}; \citealt{KC16}: 470)

\wwort{\enquote{wood rail\species{Aramides sp.}}:}

\word{Mk}{wuqaʔaʔ\pl{l}}{giant wood rail\species{Aramides ypecaha}} [1] \citep[350]{AG99} {\sep} \word{Ni}{βotåxåx\pl{is}}{chicken}; \wordnl{βotåxåx\mbox{-}itax}{giant wood rail\species{Aramides ypecaha}} \citep[95]{LC20} {\sep} \word{I’w}{wótaha}{chicken}; \word{Mj}{ˀwótaa}{chicken} (\citealt{LC-VG-12}: 345; \citealt{JC18})
{\sep} \word{LB}{wutqaq}{grey-necked wood rail\species{Aramides cajanea}}; \word{Vejoz or Guisnay}{wotaqa}{giant wood rail\species{Aramides ypecaha}} (\citealt{CS-FL-PR-VN13}; \citealt{RL16}: 105)

\dicnote{The Maká form is documented as \intxt{wuq’aʔa}, with an ejective stop, in \citet[58]{JB81}.}%1

\empr{The Iyo’awujwa’ and Manjui forms are likely borrowed from Nivaĉle, but before word-initial glottalized sonorants were deglottalized. The relation between other forms is unclear. Compare also the Guachí term ‹wokaaké›\gloss{chicken} \citep[280]{FrC51}.}

\wwort{\enquote{ibis sp.}:}

\word{Ni}{βakåk\pl{is}}{plumbeous ibis\species{Harpiprion caerulescens}} \citep[504]{LC20} {\sep} \wordng{PW}{*woqáq} > \word{LB}{wuqaq}{black-faced ibis\species{Theristicus melanopis}}; \wordng{’Wk}{woqák} [1] (\citealt{CS-FL-PR-VN13}; \citealt{KC16}: 500)

\dicnote{The stem-final velar stop (rather than uvular) in the ’Weenhayek reflex is unexpected.}%1

\wwort{\enquote{catfish sp.}:}

\word{Ijw}{ʔawánhleʔ}{\textit{Pimelodus clarias}}; \wordng{I’w}{wánhle\pl{jis}} (\citealt{JC14b}: 76; \citealt{ND09}: 95; \citealt{AG83}: 168) {\sep} \wordng{Vej}{wahnoɬi} [1] (\citealt{VU74}: 79)

\dicnote{\citet{RL16} gives the form \intxt{wahnoɬå} for Wichí, but does not indicate whether it is representative of Vejoz or Guisnay. In \citep{diwica}, the form is given as ‹wajnulha› without any dialectal attribution; judging by the root-medial vowel, it could be representative of the Southeastern dialect, in which case it should be phonologized as \intxt{waχnuɬa}.}%1

\lit{\citealt{EN84}: 42 (\intxt{*wahnhle})}

\wwort{\enquote{hail}:}

\wordng{Ni}{xak͡latu} (\citealt{LC20}: 100) {\sep} \wordng{PCh}{*ʔalátuʔ} > \wordng{Ijw}{ʔalátʲuʔ}; \wordng{I’w}{alátʲuʔ}; \wordng{Mj}{ʔalátʊʔ} (\citealt{ND09}: 94; \citealt{AG83}: 119; \citealt{JC18}) {\sep} \wordng{PW}{*qalátu} > \wordng{’Wk}{qalátuʔ} \citep[307]{KC16}

\empr{Based on the Nivaĉle and Chorote forms, it could be possible to reconstruct \wordng{PM}{*halátu(ʔ)}, but the ’Weenhayek form cannot be derived from this reconstruction. Obviously related to \word{Proto-Guaicuruan}{*qa(ˀ)lat’i}{hail} (\citealt{PVB13b}, \#513). The Lower Bermejeño form \intxt{qalati} \citep{diwica}, in turn, is perhaps a late borrowing from the Qom languages.}

\lit{\citealt{EN84}: 16 (\intxt{*(q)alathu}); \citealt{PVB02}: 146; \citealt{PVB13a}: 312}

\wwort{\enquote{spotted sorubim}:}

\wordng{Ijw}{ʔaskʲúnʲeʔ}; \wordng{I’w}{askʲúnaʔ\pl{l}}; \wordng{Mj}{ʔalkʲúnaʔ} (\citealt{ND09}: 94; \citealt{AG83}: 122, 221) {\sep} \wordng{’Wk}{ʔaxʷúkn̥aʔ\pl{lis}} \citep[10]{KC16}

\lit{\citealt{LC-VG-07}: 16 (\enquote{suruví (fish)})}

\wwort{\enquote{marbled swamp eel}:}

\wordng{Ijw}{ʔahjeʔ} [1]; \wordng{Mj}{ʔihn(ʲ)éeʔ\pl{l}} (\citealt{ND09}: 93; \citealt{JC18}) {\sep} \wordng{PW}{*ʔijhá(ʔ)} > \wordng{LB}{ʔiçá(ʔ)}; \wordng{’Wk}{ʔiçáʔ} (\citealt{JB09}: 44; \citealt{KC16}: 45)

\dicnote{The position of the stress in \wordng{Ijw}{ʔahjeʔ} is unknown to us.}%1

\empr{The Iyojwa’aja’ and Manjui forms cannot be cognate with each other, and neither of them corresponds to Wichí. The expected cognate of \wordng{Wichí}{*ʔijhá(ʔ)} in Chorote would be \wordng{PCh}{**ʔihjáʔ} > \wordng{Ijw/I’w/Mj}{*ʔihjéʔ}.}

\wwort{\enquote{clay}}

\wordng{Ijw}{ʔisát}; \wordng{I’w}{isát}; \wordng{Mj}{ʔisát} (\citealt{ND09}: 110; \citealt{AG83}: 131; \citealt{JC18}) {\sep} \wordng{PW}{*ʔijhåt} > \wordng{LB}{ʔiçåt}; \wordng{Vej}{injåt} [1]; \wordng{’Wk}{ʔiçåt\plf{ʔiçǻt\mbox{-}es}} (\citealt{JB09}: 44; \citealt{VU74}: 60; \citealt{KC16}: 45)

\dicnote{The sequence \intxt{nj} in the Vejoz form, as given by \citet{VU74}, must represent [ɲ̥], the realization of the underlying sequence /jh/ (where /j/ undergoes devoicing and nasalization).}%1

\empr{It is unclear whether the Chorote forms are even reconstructible to Proto-Chorote. Note that \intxt{i} of whichever origin is expected to induce progressive palatalization in coronals, unless it goes back to a Proto-Chorote low vowel, but PCh low vowels do not yield \intxt{i} in the word-initial position. That way, the Chorote form is best viewed as a Wichí borrowing.}

\rej{\citet[11]{EN84} includes \word{Ni}{ʔajisxan}{clay} into the comparison, which is hardly related.}

\lit{\citealt{EN84}: 11 (\intxt{*ihsá})}

\wwort{\enquote{stone}:}

\wordng{Mk}{ute\pl{l}} \citep[356]{AG99} {\sep} \wordng{Ni}{ʔutex\plf{ʔute\mbox{-}s}} \citep[307]{JS16}

\empr{Note that \sound{Maká}{e} cannot regularly correspond to \sound{Nivaĉle}{e}, and Maká zero cannot match \sound{Nivaĉle}{x}.}

\end{adjustwidth}

\setlength{\parindent}{\normalparindent}
\fussy
