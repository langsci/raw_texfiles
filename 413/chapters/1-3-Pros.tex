\chapter{Word-level prosody} \label{prosody}
This chapter deals with the reconstruction of the Proto-Mataguayan word-level prosody. We reconstruct word-level accent for Proto-Mataguayan based on evidence from the ’Weenhayek variety of Wichí and from Chorote; additional indirect evidence comes from Nivaĉle.

Our proposal is based on the observation that long vowels in ’Weenhayek regularly correspond to stressed syllables in Chorote. In our reconstruction of Proto-Mataguayan, at most one syllable in a phonological word is contrastively \conc{prominent}. A phonological word may also lack a prominent syllable; compare this to the so-called \conc{enclinomena} in languages such as Old Russian, where words with a stress (``orthotonic words'') are opposed to words without a stress, or enclinomena \citep{RJ63}.

In ’Weenhayek, the prominent syllables of Proto-Mataguayan are typically reflected as syllables with a long nucleus, whereas all other syllables have a short nucleus in ’Weenhayek. In Chorote, the prominent syllables of Proto-Mataguayan are typically reflected as stressed. The acoustic cues of stress in Chorote await further study; they may include an increase in intensity (\figref{fig-ch-acc-int}) and pitch (\figref{fig-ch-acc-pitch}) and, at least in some cases, increased vowel duration. Proto-Mataguayan words that lacked a prominent syllable receive a default stress in Chorote.

\begin{figure}[ht]
\hspace*{1cm}\includegraphics[width=10 cm]{figures/ch-acc-int.png}
\caption{Intensity in \word{Ijw}{sihwéhlʲanʔnih}{I’m dreaming}}
\label{fig-ch-acc-int}
\end{figure}

\begin{figure}[ht]
\hspace*{1cm}\includegraphics[width=10 cm]{figures/ch-acc-pitch.png}
\caption{Pitch in \word{Ijw}{sihwéhlʲanʔnih}{I’m dreaming}}
\label{fig-ch-acc-pitch}
\end{figure}

It is not yet clear what the acoustic correlates are of what we call prominence in Proto-Mataguayan; in this book, we speak of ``accented'' (¯) and ``unaccented'' (˘) syllables for ease of reference but this is purely a terminological convention, and we do not insist on any particular interpretation of PM~prominence. We indicate PM~prominence, ’Weenhayek vowel length, and Chorote stress by means of an acute accent in this book. ’Weenhayek (as well as other Wichí varieties) also has stress, whose position is mostly predictable; its placement is indicated by means of the dedicated IPA symbol~\intxt{ˈ}~unless the stress is final (see \sectref{wi-stress}).

The prosodic pattern of Proto-Mataguayan is not preserved in Maká and in most Wichí varieties, which have innovated final stress; Nivaĉle is somewhat more conservative in this regard but less so than Chorote and ’Weenhayek. Innovative final stress is found even in ’Weenhayek, though it does not interact with the more archaic vowel length system in any way. Nevertheless, there are indirect vestiges of the Proto-Mataguayan prosodic system in Nivaĉle and in Lower Bermejeño Wichí: in these varieties, \sound{PM}{*ʔ} is diachronically deleted when it occurs as a coda in posttonic syllables, but preserved in enclinomena and in accented syllables.

In our proposal, Proto-Mataguayan morphemes are underlyingly specified as accented or unaccented, and within a word only the leftmost underlying accent makes it to the surface. In addition, unaccented words of more than two syllables are not permitted; polysyllabic words composed of unaccented morphemes take a default peninitial accent.

\sectref{monosyllabic} presents the distinction between unaccented (``enclinomena'') and accented (``orthotonic'') monosyllables of Proto-Mataguayan, with clearly distinct reflexes found in ’Weenhayek. \sectref{disyllabic} shows all three possible configurations for disyllabic words: enclinomena (unaccented--unaccented), iambs (unaccented--accented), and trochees (accented--unaccented). \sectref{polysyllabic} shows the possible patterns in words with more than two syllables. Our findings are summarized in \sectref{prosody-conclusions}.

\section{Monosyllabic words} \label{monosyllabic}
This section discusses the distinction between unaccented (``enclinomena'') and accented (``orthotonic'') monosyllables of Proto-Mataguayan. They have clearly distinct reflexes in ’Weenhayek (and, consequently, in Proto-Wichí). No distinctions are found in other languages.

Note that this section covers monosyllabic \emph{words} and not \emph{stems}. This is important because monosyllabic consonant-initial stems of certain classes (such as relational nouns) always show up with a moraic prefix, and are thus considered in \sectref{disyllabic}. However, monosyllabic vowel-initial stems of these same classes usually take non-moraic prefixes, and are thus discussed in this section.

\subsection{˘} \label{corta}
Monosyllabic enclinomena are reflected as monosyllables with a short vowel in ’Weenhayek and, consequently, in Proto-Wichí. In \REF{corta-palm} and \REF{corta-mesh}, the word-initial consonant cluster is broken up by an epenthetic \sound{PW}{*i}; in this case, both vowels remain short.

\begin{exe}
    \ex \goaway
    \ex \cry
    \ex \fooditssg
    \ex \najendup
    \ex \wingitssg
    \ex \thorneitssg
    \ex \leech
    \ex \palm \label{corta-palm}
    \ex \lousesg
    \ex \whitesnail
    \ex \mesh \label{corta-mesh}
    \ex \sprout
    \ex \suckb
    \ex \swallow
    \ex \invite
    \ex \dig
    \ex \eatvt
    \ex \spillcwimp
    \ex \grass
    \ex \cordits
    \ex \diecw
    \ex \stepv
    \ex \skinits
    \ex \eatvi
    \ex \dryout
    \ex \good
    \ex \extinguished
    \ex \ripe
\end{exe}

The accretion of a plural suffix to an unaccented monosyllabic noun invariably results in an orthotonic form. Suffixes of the shape \intxt{\mbox{-}VC} are stressed in Chorote in such cases, and in ’Weenhayek they surface with a long vowel (recall that we indicate the long vowels of ’Weenhayek and Proto-Wichí by means of an acute accent).

\newpage
\booltrue{listing}
\ea \label{ex:corta-plural:ijw}
    Iyojwa’aja’ \citep[92]{JC14b}
    \begin{xlist}
        \ex \intxt{ʔés}\gloss{it is good} → \intxt{ʔiʃ-ís}\gloss{they are good}
        \ex \intxt{t-’ák}\gloss{its rope, cord} → \intxt{t-’ak-áʔ \recind t-’ak-áʔl}\gloss{its ropes, cords}
        \ex \intxt{t-’áx}\gloss{its skin} → \intxt{t-’ɛh-ɛ́s}\gloss{its skins}
    \end{xlist}
\z
\ea \label{ex:corta-plural:i'w}
    Iyo’awujwa’ \citep[176]{AG83}
    \begin{xlist}
        \ex \intxt{hóp}\gloss{maize} (etymologically\gloss{grass.\SG}) → \intxt{hup-áj}\gloss{grass} (etymologically\gloss{grass.\PL})
    \end{xlist}
\z
\ea  \label{ex:corta-plural:mj}
    Manjui \citep{JC18}
    \begin{xlist}
        \ex \intxt{hʊ́p}\gloss{maize.\SG} → \intxt{hup-ájh}\gloss{maize.\PL, grass}
        \ex \intxt{ʔéis}\gloss{it is good} → \intxt{ʔas-éis}\gloss{they are good}
    \end{xlist}
\z
\ea \label{ex:corta-plural:whk}
    ’Weenhayek \citep[95, 96, 158, 235]{KC16}\\
    \begin{xlist}
        \ex \intxt{hup}\gloss{grass.\SG; house made of hay} → \intxt{hup-úç}\gloss{grass.\PL; houses made of hay}
        \ex \intxt{ɬ-exʷ}\gloss{its wing} → \intxt{ɬ-exʷ-ís}\gloss{its wings}
        \ex \intxt{t-’aq}\gloss{its tie} → \intxt{t-’aq-áç}\gloss{its ties}
        \ex \intxt{t-’åx}\gloss{its skin} → \intxt{t-’åh-és}\gloss{its skins}
    \end{xlist}
\z
\boolfalse{listing}

If the plural suffix takes a non-moraic allomorph, the resulting plural form becomes orthotonic (as shown by the ’Weenhayek reflexes), even though the plural suffix does not constitute a syllable on its own.

\ea
    \begin{xlist}
    \ex \fooditssg
    \ex \foodpl
    \ex \toolnsg
    \ex \toolnpl
    \end{xlist}
\z

We propose that the suffixes \wordng{PM}{*\mbox{-}l}, \intxt{*\mbox{-}jʰ}, and \intxt{*\mbox{-}ts} contain an underlyingly accented vowel, which surfaces in the allomorphs \intxt{*\mbox{-}él}, \intxt{*\mbox{-}ájʰ}, \intxt{*\mbox{-}íts} (see \sectref{c-v-stems}). The accent is preserved even when the underlying vowel is elided, as can also be seen in the plural forms of disyllabic enclinomena (\sectref{corta-corta}).

\subsection{¯} \label{larga}

Monosyllabic orthotonic words are reflected as monosyllables with a long vowel in ’Weenhayek and, consequently, in Proto-Wichí, as shown below. In \REF{larga-cactus}, \REF{larga-cardon}, and \REF{larga-blind}, the word-initial consonant cluster is resolved by means of inserting an unstressed vowel in Chorote and a short vowel in Wichí, respectively. Recall that we indicate long vowels of Proto-Wichí by means of an acute accent.

\begin{exe}
    \ex \brightnessits
    \ex \fallonitsown
    \ex \fruitits
    \ex \arrive
    \ex \pronominal
    \ex \shout
    \ex \putv
    \ex \sonits
    \ex \drinknitssg
    \ex \yicaayitssg
    \ex \namenitssg
    \ex \locustcw
    \ex \heat
    \ex \sweat
    \ex \cactus \label{larga-cactus}
    \ex \flu
    \ex \goimp
    \ex \zorzalsg
    \ex \bonenu
    \ex \penisits
    \ex \seedits
    \ex \drum
    \ex \cardon \label{larga-cardon}
    \ex \blind \label{larga-blind}
    \ex \snakesg
    \ex \hardv
    \ex \nestitssg
    \ex \largefat
    \ex \stagnantsg
    \ex \palosanto
    \ex \sandyplace
    \ex \ask
    \ex \juiceits
    \ex \urinate
\end{exe}

Evidence for the ancient opposition between unaccented and accented monosyllables comes not only from ’Weenhayek, but also from Chorote: in PM~orthotonic monosyllables, the stress never moves to the suffix in Chorote, as in \REF{ex:larga-larga:ijw}--\REF{ex:larga-larga:mj}, unlike what happens in enclinomena in examples such as \REF{ex:corta-plural:ijw}--\REF{ex:corta-plural:mj}.

\booltrue{listing}
\ea \label{ex:larga-larga:ijw}
    Iyojwa’aja’ \citep[131, 132]{ND09}
    \begin{xlist}
        \ex \intxt{hl-ɛ́ʔ}\gloss{her/his/its name} → \intxt{hl-ɛ́j-is}\gloss{her/his/its names}
        \ex \intxt{hl-óp}\gloss{its nest} → \intxt{hl-óp-is}\gloss{its nests}
    \end{xlist}
\z
\ea \label{ex:larga-larga:i'w}
    Iyo’awujwa’ \citep[125, 176, 176, 183]{AG83}
    \begin{xlist}
        \ex \intxt{-éj}\gloss{yica bag} → \intxt{-éj-is}\gloss{yica bags}
        \ex \intxt{hl-úp}\gloss{its nest} → \intxt{hl-úp-is}\gloss{its nests}
        \ex \intxt{hók}\gloss{palo santo tree} → \intxt{hók-iʔ}\gloss{palo santo trees}
        \ex \intxt{tóxs}\gloss{snake} → \intxt{tóxs-is}\gloss{snakes}
    \end{xlist}
\z
\newpage
\ea \label{ex:larga-larga:mj}
    Manjui \citep{JC18}
    \begin{xlist}
        \ex \intxt{-át}\gloss{drink.\SG} → \intxt{-át-es}\gloss{drink.\PL}
        \ex \intxt{-ɛ́jʔ}\gloss{name} → \intxt{-ɛ́j-is}\gloss{names}
        \ex \intxt{-ɛ́jʔ}\gloss{yica bag} → \intxt{-ɛ́j-is}\gloss{yica bags}
        \ex \intxt{ˀmɔ́k}\gloss{zorzal bird} → \intxt{ˀmɔ́k-is}\gloss{zorzal birds}
        \ex \intxt{hɔ́k}\gloss{palo santo tree} → \intxt{hɔ́k-ej}\gloss{palo santo trees}
        \ex \intxt{hɔ́t}\gloss{sand.\SG (small quantity of sand)} → \intxt{hɔ́t-ej}\gloss{sand.\PL (large patch of sand)}
        \ex \intxt{hl-ʊ́p}\gloss{its nest} → \intxt{hl-ʊ́p-is}\gloss{its nests}
        \ex \intxt{tɔ́s}\gloss{snake} → \intxt{tɔ́xʃ-is}\gloss{snakes}
    \end{xlist}
\z
\boolfalse{listing}

\section{Disyllabic words} \label{disyllabic}

This section discusses the distinction between unaccented (``enclinomena'') and two types of accented (``iambic'' and ``trochaic'') disyllables of Proto-Mataguayan. All three types have clearly distinct reflexes in ’Weenhayek (and, consequently, in Proto-Wichí): the reflexes of disyllabic enclinomena have two short vowels in that variety, iambic disyllables are reflected as words with a short vowel followed by a long one, and trochaic disyllables are reflected as words with a long vowel followed by a short one. In Chorote, the former two types (enclinomena and iambic disyllables) merge: both are reflected as disyllables with stress falling on the final syllable. PM~trochaic disyllables remain distinct in Chorote (and possibly in Nivaĉle): they receive stress on the initial syllable.

\subsection{˘˘} \label{corta-corta}

Disyllabic enclinomena are reconstructed based on evidence from ’Weenhayek: in that variety, a disyllabic word may lack long vowels altogether. The cognates in Chorote and Nivaĉle have default (final) stress.

\begin{exe}
    \ex \coalabssg
    \ex \jaguar
    \ex \treensg
    \ex \mancwsg
    \ex \vulturesg
    \ex \holeabs
    \ex \two
    \ex \starnsg
    \ex \shoot
    \ex \carrysh
    \ex \termitehouseits
    \ex \bromelia
    \ex \walk
    \ex \nightncw
    \ex \fatalhaitssg
    \ex \aloja
    \ex \doradocwsg
    \ex \meatitssg
\end{exe}

The same combination occurs when an unaccented moraic prefix is added to an unaccented monosyllabic root. The following roots typically show up with a moraic prefix:

\begin{exe}
    \ex \throwpush
    \ex \takeaway
    \ex \tailsg
    \ex \torn
    \ex \earkfesg
    \ex \handsg
    \ex \hornclubsg
    \ex \withstand
    \ex \petsg
    \ex \languagewordsg
    \ex \toolnsg
    \ex \yicalhuksg
    \ex \sleep
    \ex \smelln
    \ex \fatpesg
    \ex \fence
    \ex \lid
    \ex \vein
    \ex \vaginasg
    \ex \basetrunk
    \ex \eyesg
    \ex \spillcw
    \ex \throwv
    \ex \suckcw
    \ex \placen
    \ex \necksg
    \ex \clothes
    \ex \pricesg
    \ex \earthsg
    \ex \firewoodhuksg
\end{exe}

The following roots can occur with a zero 3.{\textsc{rls}} prefix and form monosyllabic words, but they may also take a moraic unaccented prefix, and in this case they behave just like any other disyllabic enclinomena.
 
\begin{exe}
    \ex \sprout
    \ex \suckb
    \ex \swallow
    \ex \invite
    \ex \dig
    \ex \eatvt
\end{exe}

Note that disyllabic unaccented nouns become orthotonic in the plural form, even if the plural form has the same amount of syllables as the singular one. This can be seen most clearly in 'Weenhayek pairs of singular and plural nouns \REF{ex:disspl:whk}.

\booltrue{listing}
\ea\label{ex:disspl:whk}
'Weenhayek \citep{KC16}
    \begin{xlist}
        \ex \intxt{hiˀnoʔ}\gloss{man} → \intxt{hiˀnó-ɬ}\gloss{men}
        \ex \intxt{xʷiçoʔ}\gloss{coal} → \intxt{xʷiçó-ɬ}\gloss{coals}
        \ex \intxt{la-kʲ’uʔ}\gloss{its horn} → \intxt{la-kʲ’ú-ɬ}\gloss{its horns}
        \ex \intxt{haˀlåʔ}\gloss{tree} → \intxt{haˀlǻ-ç}\gloss{trees}
        \ex \intxt{qakʲaʔ}\gloss{medicine} → \intxt{qakʲá-ɬ}\gloss{medicines}
        \ex \intxt{ʔatsʰaʔ}\gloss{dorado} → \intxt{ʔatsʰá-ç}\gloss{dorados}
        \ex \intxt{la-låʔ}\gloss{her/his pet} → \intxt{la-lǻ-ç}\gloss{her/his pets}
        \ex \intxt{ta-teʔ}\gloss{her/his eye} → \intxt{ta-té-ç}\gloss{her/his eyes}
        \ex \intxt{kʲowex}\gloss{hole} → \intxt{kʲow̥-áç}\gloss{holes}
        \ex \intxt{towex}\gloss{pan; kind of drum} → \intxt{tow̥-áç}\gloss{pans; drums}
    \end{xlist}
\z
\boolfalse{listing}

We propose that the suffixes \wordng{PM}{*\mbox{-}l}, \intxt{*\mbox{-}jʰ}, and \intxt{*\mbox{-}ts} contain an underlyingly accented vowel, which surfaces in the allomorphs \intxt{*\mbox{-}él}, \intxt{*\mbox{-}ájʰ}, \intxt{*\mbox{-}íts} (see \sectref{c-v-stems}). The accent is preserved even when the underlying vowel is elided.

\subsection{˘¯} \label{corta-larga}

Iambic disyllables are reconstructed based on evidence from ’Weenhayek. Their reflexes in Chorote and Nivaĉle also have default (final) stress and are thus indistinguishable from the reflexes of enclinomena.

\begin{exe}
    \ex \algarrobof
    \ex \coldweather
    \ex \dew
    \ex \watersg
    \ex \armadillo
    \ex \thorncutjansg
    \ex \savannahhawk
    \ex \jabiru
    \ex \deep
    \ex \longv
    \ex \chajasg
    \ex \duraznillo
    \ex \ashamedcw
\end{exe}

The same combination occurs when an unaccented moraic prefix is added to an accented monosyllabic root. The following roots typically show up with a moraic prefix:

\begin{exe}
    \ex \cutdown
    \ex \tell
    \ex \finger
    \ex \breath
    \ex \drinkv
    \ex \redv
    \ex \sendv
    \ex \answer
    \ex \chaguark
    \ex \stretchout
    \ex \hornclubpl
    \ex \killv
    \ex \feel
    \ex \offspring
    \ex \wash
    \ex \many
    \ex \roast
    \ex \defecate
    \ex \lightfire
    \ex \firewoodlhet
    \ex \yicalhukpl
    \ex \powder
    \ex \bathe
    \ex \bitter
    \ex \cook
    \ex \returnh
    \ex \foodpl
    \ex \costume
    \ex \tongue
    \ex \dovesipup
    \ex \throwcw
    \ex \tears
    \ex \hardvpr
    \ex \badmood
    \ex \burrow
    \ex \worm
    \ex \belly
    \ex \seev
    \ex \recipient
    \ex \burnvt
    \ex \pushv
\end{exe}

The same combination arises when an unaccented monosyllabic root takes an accented plural suffix, as in \word{'Wk}{ˀwojís}{blood (plurale tantum)}, derived from \word{PM}{*(\mbox{-})ˀwoˀj}{blood} by means of the plural suffix \intxt{\mbox{-}ís}. For more examples, see \REF{ex:corta-plural:ijw}--\REF{ex:corta-plural:whk} above.

\subsection{¯˘} \label{larga-corta}

Trochaic disyllables are reflected in the following way. In ’Weenhayek, they have a long vowel in the initial syllable and a short one in the final syllable. In Chorote, they have initial stress. In Nivaĉle, they sometimes also have initial stress, which is not typical for the language \citep{AnG15}; so, for example, in \wordnl{ʔóɸo\pl{s}}{dove} \citep[267]{AnG15}, \wordnl{ɬ\mbox{-}ǻse}{her/his daughter}, \wordnl{ɬútsxa}{girl}, \wordnl{púta}{tapeti rabbit}, \wordnl{títetʃ}{plate}, \wordnl{ʃnáβåp \recind ʃnǻβåp}{spring}, \wordnl{ʔék͡le}{parrot} \perscomm{Analía Gutiérrez}{2023}, or \wordnl{núʔu}{dog} \citep[34]{LC20}, though variation has been attested. In addition, final PM glottal stop is lost in trochees in Nivaĉle and Wichí -- at least in its Lower Bermejeño variety, as documented by \citet{VN14} -- as in \REF{tr-mouthits}, \REF{tr-daughterits}, \REF{tr-cavy}, \REF{tr-pigeon}, further described in \sectref{ni-deglottalization-codas} and \sectref{wi-posttonic-deglottalization}. \REF{tr-mosquito} has an irregular reflex in Nivaĉle: not only does it irregularly reflect \sound{PM}{*ʔe} as \intxt{ji}, but it also has final stress \perscomm{Analía Gutiérrez}{2023}, which does not match the evidence from Chorote.

\begin{exe}
    \ex \mouthits \label{tr-mouthits}
    \ex \flowerits
    \ex \wordametits
    \ex \stingerits
    \ex \returnth
    \ex \burn
    \ex \daughterits \label{tr-daughterits}
    \ex \drinknitssg
    \ex \jarits
    \ex \fatv
    \ex \bite
    \ex \inhabitantits
    \ex \centipede
    \ex \pocote
    \ex \rootnabssg
    \ex \crab
    \ex \north
    \ex \suncho
    \ex \resinits
    \ex \lizard
    \ex \monkparakeet
    \ex \heavyv
    \ex \tortoise
    \ex \whitealgarrobof
    \ex \barnowl
    \ex \snail
    \ex \squash
    \ex \bowabssg
    \ex \daylhuma
    \ex \girlsg
    \ex \majan
    \ex \ropeabs
    \ex \dog
    \ex \pathnabs
    \ex \cavy \label{tr-cavy}
    \ex \frighten
    \ex \frog
    \ex \quick
    \ex \rainsg
    \ex \orphancw
    \ex \skycloud
    \ex \tapeti
    \ex \limpkin
    \ex \parakeet
    \ex \chachalaca
    \ex \wildcatsg
    \ex \whitequebracho
    \ex \anteater
    \ex \thunder
    \ex \toad
    \ex \river
    \ex \plate
    \ex \precipice
    \ex \far
    \ex \ant
    \ex \yellowlegs
    \ex \smokeabs
    \ex \silkfloss
    \ex \tsofa
    \ex \redbrocket
    \ex \hornero
    \ex \tired
    \ex \paloflojof
    \ex \metal
    \ex \balawasp
    \ex \whiteegret
    \ex \healthy
    \ex \butterfly
    \ex \spring
    \ex \fox
    \ex \moon
    \ex \iguana
    \ex \rat
    \ex \jararaca
    \ex \maguari
    \ex \femalebreastits
    \ex \hurt
    \ex \mosquito \label{tr-mosquito}
    \ex \parrot
    \ex \firei
    \ex \pigeon \label{tr-pigeon}
    \ex \bro
    \ex \doveula
    \ex \urineits
\end{exe}

Words of this structure occur whenever a monosyllabic morpheme with underlying accent (either a prefix or a root) is combined with another monosyllabic morpheme regardless of the underlying accentual properties of the latter. The examples in \REF{ex:larga-corta:whk} from 'Weenhayek instantiate the combination of a prefix with a long vowel (\word{'Wk}{ˀnó\mbox{-}}{{\textsc{gnr}}}, \wordnl{ʔá\mbox{-}}{2.S\textsubscript{{P}}}) with a root with an underlying short vowel.\footnote{Note that forms that arose due to Watkins' Law (\sectref{wi-watkins}) do not conform to these regularities in Wichí, since the domain for accent assignment excludes any material that precedes the erstwhile third-person prefix. Consequently, prefixes such as \word{'Wk}{ˀnó\mbox{-}}{{\textsc{gnr}}} surface with a short vowel in forms such as \wordnl{ˀno\mbox{-}ɬ\mbox{-}åq}{one's food}.\label{proswatkfn}}

\booltrue{listing}
\ea\label{ex:larga-corta:whk}
'Weenhayek \citep{KC16}
    \begin{xlist}
        \ex \intxt{-låʔ}\gloss{domestic animal} → \intxt{ˀnó-låʔ}\gloss{one's domestic animal}
        \ex \intxt{-ɬuk}\gloss{load, bag} → \intxt{ˀnó-ɬuk}\gloss{one's load, bag}
        \ex \intxt{-kʲ’uʔ}\gloss{horn, club} → \intxt{ˀnó-kʲ’uʔ}\gloss{one's horn, club}
        \ex \intxt{-kʲås}\gloss{tail} → \intxt{ˀnó-kʲås}\gloss{one's tail}
        \ex \intxt{-nix}\gloss{smell} → \intxt{ˀnó-nix}\gloss{one's smell}
        \ex \intxt{-p’ot}\gloss{lid} → \intxt{ˀnó-p’ot}\gloss{one's lid}
        \ex \intxt{-kejʔ}\gloss{hand} → \intxt{ˀnó-kejʔ}\gloss{one's hand}
        \ex \intxt{-haʔ}\gloss{price} → \intxt{ˀnó-haʔ}\gloss{one's price}
        \ex \intxt{-ˀwet}\gloss{place, home} → \intxt{ˀnó-ˀwet}\gloss{one's place, home}
        \ex \intxt{-huk}\gloss{firewood} → \intxt{ˀnó-huk}\gloss{one's firewood}
        \ex \intxt{ʔis}\gloss{good} → \intxt{ʔá-ʔis}\gloss{you are good}
        \ex \intxt{noxʷ}\gloss{good} → \intxt{ʔá-noxʷ}\gloss{you end up}
    \end{xlist}
\z
\boolfalse{listing}

The following examples from 'Weenhayek instantiate the combination of a prefix with a long vowel (\word{'Wk}{ˀnó\mbox{-}}{{\textsc{gnr}}}, \wordnl{ʔá\mbox{-}}{2.S\textsubscript{{P}}}) with a root with an underlying long vowel.\footnote{The generalization in footnote \ref{proswatkfn} applies to roots with underlying long vowels as well: once again, the domain for accent assignment excludes anything that precedes the erstwhile third-person prefix, fossilized due to Watkins' Law. Therefore, prefixes such as \word{'Wk}{ˀnó\mbox{-}}{{\textsc{gnr}}} surface with a short vowel in forms such as \wordnl{ˀno\mbox{-}ɬ\mbox{-}ǻs}{one's son}.}

\booltrue{listing}
\ea\label{ex:larga-larga:whk}
'Weenhayek \citep{KC16}
    \begin{xlist}
        \ex \intxt{-mók}\gloss{powder} → \intxt{ˀnó-mok}\gloss{one's powder}
        \ex \intxt{-ts’éʔ}\gloss{belly} → \intxt{ˀnó-ts’eʔ}\gloss{one's belly}
        \ex \intxt{-qéjʔ}\gloss{custom} → \intxt{ˀnó-qejʔ}\gloss{one's custom}
        \ex \intxt{-lés}\gloss{children} → \intxt{ˀnó-les}\gloss{one's children}
        \ex \intxt{-jáɬ}\gloss{breath} → \intxt{ˀnó-jaɬ}\gloss{one's breath}
        \ex \intxt{-q’áx}\gloss{mouth} → \intxt{ˀnó-q’ax}\gloss{one's mouth}
        \ex \intxt{-wǻk}\gloss{rage} → \intxt{ˀnó-wåk}\gloss{one's rage}
        \ex \intxt{-ɬét}\gloss{fire} → \intxt{ˀnó-ɬet}\gloss{one's fire}
        \ex \intxt{wúxʷ}\gloss{big} → \intxt{ʔá-wuxʷ}\gloss{you are big}
        \ex \intxt{t’ún̥}\gloss{hard} → \intxt{ʔá-t’un̥}\gloss{you are hard}
        \ex \intxt{ʔím̥}\gloss{swollen} → \intxt{ʔá-ʔim̥}\gloss{you are swollen}
        \ex \intxt{ˀjújʔ}\gloss{sharp} → \intxt{ʔá-ˀjujʔ}\gloss{you are sharp}
        \ex \intxt{tilúk}\gloss{blind} → \intxt{ʔá-tiluk}\gloss{you are blind}
    \end{xlist}
\z
\boolfalse{listing}

The examples in \REF{ex:larga-larga-pl:whk} from 'Weenhayek instantiate the combination of a root with a long vowel with the plural suffix \intxt{\mbox{-}ís}, whose vowel is underlyingly long, as seen in \REF{ex:corta-plural:ijw}--\REF{ex:corta-plural:whk}. For analogous examples from Chorote, see \REF{ex:larga-larga:ijw}--\REF{ex:larga-larga:mj}.

\booltrue{listing}
\ea\label{ex:larga-larga-pl:whk}
'Weenhayek \citep{KC16}
    \begin{xlist}
        \ex \intxt{ɬ-éjʔ}\gloss{her/his name} → \intxt{ɬ-éj-is}\gloss{their names}
        \ex \intxt{ɬ-úp}\gloss{her/his nest} → \intxt{ɬ-úp-is}\gloss{their nests}
    \end{xlist}
\z
\boolfalse{listing}

Therefore, we conclude that PM~words composed of two (or more) morphemes with underlying accent preserve only the leftmost accent in the surface realization, whereas all accents to the right are deleted: \intxt{*ɬ\mbox{-}} + \intxt{*\mbox{-}úˀp} + \intxt{*\mbox{-}íts} results in \wordnl{*ɬ\mbox{-}úp\mbox{-}its}{their nests}, as opposed to \intxt{*ˀwoˀj} + \intxt{*\mbox{-}íts} → \wordnl{*ˀwoj\mbox{-}íts}{blood.\PL}, \intxt{*ɬ\mbox{-}} + \intxt{*\mbox{-}ʔåx} + \intxt{*\mbox{-}íts} → \wordnl{*ɬ\mbox{-}’åx\mbox{-}íts}{their skins}.

\section{Words with three or more syllables} \label{polysyllabic}

In the surface representation of PM words composed of three or more syllables, there must be an accent, and it must fall within the first three syllables of the stem.\footnote{It is theoretically possible that in some exceptional cases the stress could be moved even farther from the left edge of the stem, as in \word{Manjui}{ʃi\mbox{-}p’ilisáh}{I am poor}, where a trisyllabic root with a final accent receives an unaccented prefix. However, this combination is exceedingly rare, and we have been unable to identify evidence from other Mataguayan varieties that would support the antiquity of the pattern in question.} There is no evidence supporting the reconstruction of trisyllabic (or longer) enclinomena. If a word is composed of morphemes with no underlying accents, a default accent is assigned to the peninitial syllable of the word. We start by discussing words with the accent falling on the postpeninitial syllable, or the third one counting from the left edge (\sectref{corta-corta-larga}), then words with the accent on the peninitial syllable (\sectref{corta-larga-corta}), and finally words with initial stress (\sectref{larga-corta-corta}).

\subsection{˘˘¯} \label{corta-corta-larga}

Most likely, postpeninitial accent in Proto-Mataguayan was restricted to morphologically complex words. It is reconstructed primarily based on evidence from Iyo'awujwa' and Manjui, whereas Iyojwa'aja' and Wichí have innovated by retracting the accent to the peninitial syllable. As a consequence of that innovation, the stress in Iyojwa'aja' can synchronically fall on either syllable within the disyllabic -- and not trisyllabic -- window at the left edge of the word \citep[91–2]{JC14b}.\footnote{Apparent violations of this restriction are observed in forms such as \word{Ijw}{kasts’aháne}{we know it}, \wordnl{kasts’iʃís}{we are good}, \wordnl{ʔiˀnahwɛ́l}{you are ashamed}. This entails that when stress retraction applied in Iyojwa'aja, the first-person plural proclitic \intxt{kas$=$} was outside the respective domain, and that the insertion of \intxt{ʔi} in the prefixes of the shape \intxt{ʔin\mbox{-}} before vowels and glottal consonants had not yet occurred. The Proto-Chorote reconstructions of the aforementioned forms are as follows: \word{PCh}{*kas ts-’ahán-eh}{we know it}, \wordnl{*kas ts-’is-ís}{we are good}, \wordnl{*ˀ<n>ahwéɬ}{you are ashamed}.} Likewise, in 'Weenhayek long vowels usually occur within the disyllabic window at the left edge of the word, except for instances of noun incorporation \citep[9]{KC94} and forms that arose due to Watkins' Law (\sectref{wi-watkins}), such as \wordnl{ˀno\mbox{-}t\mbox{-}’åx\mbox{-}kʲá\mbox{-}tax}{one's chickenpox} or \wordnl{ˀno\mbox{-}ɬ\mbox{-}exʷ\mbox{-}ís}{one's wings}, where the domain for accent assignment excludes any material that precedes the erstwhile third-person prefix \intxt{ɬ\mbox{-} / t\mbox{-}’}.

Words with postpeninitial accent are most commonly composed of an unaccented prefix and a root with an underlying accent on the second syllable, as in \REF{anap-shoulder}–\REF{anap-spank}. Note that the accent retraction fed the deletion of the word-final glottal stop in unaccented syllables in Wichí (cf. \sectref{wi-posttonic-deglottalization}), whereas in Nivaĉle no accent retraction occurred, and the word-final glottal stop (if present in PM) remained, as in \REF{anap-elderbro}, \REF{anap-eldersis}, \REF{anap-leg}. The preservation of the word-final glottal stop in Nivaĉle contrasts with its loss in trochees, as in \REF{tr-mouthits}, \REF{tr-daughterits}, \REF{tr-cavy}, \REF{tr-pigeon}, further described in \sectref{ni-deglottalization-codas}.

\begin{exe}
    \ex \shoulder \label{anap-shoulder}
    \ex \shoulderblade
    \ex \elbow
    \ex \elderbro \label{anap-elderbro}
    \ex \eldersis \label{anap-eldersis}
    \ex \cheek
    \ex \snore
    \ex \leg \label{anap-leg}
    \ex \spank \label{anap-spank}
\end{exe}

The same stress pattern is found when an unaccented prefix is combined with an unaccented monosyllabic root and an accented suffix, as in the plural forms in \REF{prostailpl}–\REF{prosbasetrunkpl}.

\begin{exe}
    \ex \tailpl \label{prostailpl}
    \ex \handpl
    \ex \languagewordpl
    \ex \lidpl
    \ex \basetrunkpl \label{prosbasetrunkpl}
\end{exe}

Finally, postpeninitial accent is found when a disyllabic enclinomenon receives a suffix with an underlying accent, as in \REF{prosstarnpl}.\footnote{A few forms remain problematic for our proposal. First of all, the plural form of \word{Mj}{\mbox{-}(ʔi)ʃéˀn}{meat} is \intxt{\mbox{-}ʔiʃén\mbox{-}is} and not *\intxt{\mbox{-}ʔiʃen\mbox{-}éis}, despite the fact that its PM~etymon is reconstructed as an enclinomenon: \wordng{PM}{*-ʔäsχaˀn}, expected plural form \intxt{**\mbox{-}ʔäsχan\mbox{-}ís}. Second, the root for\gloss{to stand} behaves as iambic in 'Weenhayek, as seen in the imperative \word{'Wk}{qasít}{stand!}, but consistently has stem-initial stress in Iyo'awujwa' and Manjui, as in \word{Mj}{ti\mbox{-}káʃit}{s/he stands}. Since this is observed in only two lexemes, it is not currently possible to decide whether we are dealing with a true exception or with some sort of an additional restriction whereby the accent is retracted in inflected forms with person prefixes.}

\begin{exe}
    \ex \starnpl \label{prosstarnpl}
\end{exe}

As noted above, the postpeninitial accent pattern is reconstructed based on evidence from Iyo'awujwa' and Manjui, and indirect evidence for its antiquity comes from the failure of the final \intxt{ʔ} to be lost in Nivaĉle, as in \wordnl{ji\mbox{-}tʃitaʔ}{my elder sister}, \wordnl{ʔa\mbox{-}kak͡låʔ}{your leg} \citep[56, 103]{JS16}. This counters the pattern established by \citet[182–194]{AnG15}, whereby in unsuffixed nouns with a (possessive) person index iambic feet are normally built from the left edge of the word. Consequently, the second syllable of the root is expected to undergo deglottalization in weak prosodic positions, as in \wordnl{(ʃinβóʔ)}{honey} → \wordnl{(ji\mbox{-}ʃín)βo}{my honey} \citep[186]{AnG15}. Although we have no information on the position of the stress in forms such as \intxt{ji\mbox{-}tʃitaʔ} and \intxt{ʔa\mbox{-}kak͡låʔ} in the variety of Nivaĉle studied by Gutiérrez, the consistent presence of the word-final glottal stop in all inflected forms of these nouns indicates that they retain the final stress pattern of PM, quite atypically for Nivaĉle: \intxt{ji\mbox{-}(tʃitáʔ)}, \intxt{ʔa\mbox{-}(kak͡lǻʔ)}. This prediction will need to be tested with native speakers of Nivaĉle. At least in plurals, which in our account contain an accented suffix in PM, Nivaĉle is explicitly reported to receive final stress, as in \wordnl{ji\mbox{-}(k͡liʃ\mbox{-}áj)}{my words} \citep[204]{AnG15}. This fully conforms with our expectations.

\subsection{˘¯˘} \label{corta-larga-corta}

Peninitial accent is the most frequent pattern in polysyllabic words. It arises whenever the initial syllable lacks an underlying accent and the peninitial syllable carries one, regardless of the properties of all subsequent syllables. In addition, it comes about as the default accent pattern in words that lack any underlying accent within the trisyllabic window at the left edge.

Peninitial accent often arises when an unaccented prefix is attached to a disyllabic or longer stem (unless the stem itself carries an underlying accent on its second syllable, on which see \sectref{corta-corta-larga}). In order to recover the underlying accentual properties of any given stem, one needs to examine its behavior in absence of prefixes. However, many verbs and relational nouns never occur without prefixes, and it is therefore not always possible to determine whether a given stem carries any underlying accent at all.

In a handful of cases, we can be fairly certain that the initial syllable of the stem carried an underlying accent. This can be seen in prefixless forms such as \word{Ijw}{lóxsʲe}{bow}, \wordnl{ˀnáji}{path}, \wordnl{tóxsʲe}{smoke}; \word{I'w}{fʷétis}{root}, \wordnl{lóxseʔ}{bow}, \wordnl{náji}{path}, \wordnl{tóxsa}{smoke}; \word{Mj}{tʊ́xsa}{smoke}, \wordnl{pʊ́xsena}{bearded}; \word{'Wk}{xʷétes}{root}, \wordnl{lútsex}{bow}, \wordnl{ˀnǻjix}{path}, \wordnl{pǻsenax}{gilded catfish}, \wordnl{tútsax}{smoke}, all of which show initial accent. The accent does not shift upon accretion of an unaccented prefix:

\begin{exe}
    \ex \rootn
    \ex \bow
    \ex \pathn
    \ex \beard
    \ex \smoke
\end{exe}

In yet other cases, there is evidence that the stem itself lacks an underlying accent. The stems listed below behave as enclinomena when used without a prefix: in Chorote they carry final stress (\word{Ijw}{k’ijé}{for}; \word{Mj}{kʲowɛ́h}{hole}, \wordnl{ʔijéʔ}{for}), in 'Weenhayek they lack long vowels (\word{'Wk}{xʷiçoʔ}{coal}, \wordnl{kʲowex}{hole}, \wordnl{qakʲaʔ}{medicine}, \wordnl{towex}{pan, kind of drum}), and in Nivaĉle they fail to undergo deglottalization in the stem-final position, which suggests final stress (\word{Ni}{ɸajxóʔ}{coal}, \wordnl{k’utsáˀx}{old}). Note that when such stems combine with a monomoraic prefix in Nivaĉle, the coda of the stem-final syllable deglottalizes, suggesting peninitial stress (\word{Ni}{ɬ̩\mbox{-}ɸájxo}{its charcoal}, \wordnl{ji\mbox{-}tǻβaʃ}{my abdominal cavity}). In \REF{encl-coal} and \REF{encl-abd}, we list the allomorphs without the deglottalization effect in Nivaĉle, which occur with the prefixes \wordnl{βat\mbox{-}}{indefinite possessor} and \wordnl{kas\mbox{-}}{our}.

\begin{exe}
    \ex \coalrel \label{encl-coal}
    \ex \holerel
    \ex \kojarel
    \ex \oldnrel
    \ex \medicine
    \ex \abdcavity \label{encl-abd}
\end{exe}

This strongly suggests that Proto-Mataguayan did not tolerate enclinomena of more than two syllables: if an unaccented prefix was added to a disyllabic enclinomenon, a default accent was assigned to the initial syllable of the stem (the peninitial syllable of the word). In fact, Nivaĉle, Chorote, and Wichí still show synchronically active alternations in accent placement, exemplified in \REF{ex:mobileaccent:niv}–\REF{ex:mobileaccent:whk}.

\booltrue{listing}
\ea \label{ex:mobileaccent:niv}
    Nivaĉle \citep[184, 186, 211–212, 272]{AnG15}
    \begin{xlist}
        \ex \intxt{samúk}\gloss{excrement} → \intxt{ji-sámuk}\gloss{my excrement}
        \ex \intxt{k͡lesá}\gloss{knife} → \intxt{ji-k͡lésa}\gloss{my knife}
        \ex \intxt{ʃinβóʔ}\gloss{honey} → \intxt{ji-ʃínβo}\gloss{my honey}
        \ex \intxt{jiktsúˀk}\gloss{silk floss tree} → \intxt{ʔa-β-íktsuk}\gloss{your canoe (made of the wood of a silk floss tree)}
        \ex \intxt{tiɬóˀx}\gloss{s/he carries it on her/his shoulders} → \intxt{xa-tíɬox}\gloss{I carry it on my shoulders}
        \ex \intxt{βak͡léˀtʃ}\gloss{s/he walks} → \intxt{xa-βák͡letʃ}\gloss{I walk}
        \ex \intxt{βåmkǻʔ}\gloss{s/he washes} → \intxt{xa-βǻˀmkå}\gloss{I wash}
        \ex \intxt{ɸajxóʔ}\gloss{charcoal} → \intxt{ɬ̩-ɸájxo}\gloss{its charcoal}
    \end{xlist}
\z

\ea \label{ex:mobileaccent:ijw}
    Iyojwa’aja’ \citep[92]{JC14b}
    \begin{xlist}
        \ex \intxt{k’ijé}\gloss{for} → \intxt{si-kʲ’óje}\gloss{for us}
        \ex \intxt{ʔapɛ́ʔɛ}\gloss{above} → \intxt{si-típeʔe}\gloss{above us}
        \ex \intxt{kʲahwéh}\gloss{below} → \intxt{si-kʲáhwe}\gloss{below us}
    \end{xlist}
\z

\ea \label{ex:mobileaccent:mj}
   Manjui \citep{JC18,GH94}
    \begin{xlist}
        \ex \intxt{ʔijéʔ}\gloss{for} → \intxt{hi-ʔʲójeʔ}\gloss{for her/him}
        \ex \intxt{ʔapɛ́ʔɛʔ}\gloss{above} → \intxt{hi-tɛ́peʔeʔ}\gloss{on top of it}
        \ex \intxt{kihwíjh}\gloss{below} → \intxt{ʃi-kéihwi}\gloss{below us}
    \end{xlist}
\z

\ea \label{ex:mobileaccent:whk}
    ’Weenhayek \citep[65, 85, 94, 124, 173, 306, 317, 420, 472]{KC16}
    \begin{xlist}
        \ex \intxt{towex}\gloss{pan; kind of drum} → \intxt{la-tówex}\gloss{its hole}
        \ex \intxt{kʲowex}\gloss{hole} → \intxt{la-kʲówex}\gloss{its center}
        \ex \intxt{qawaq}\gloss{belt} → \intxt{la-qáwaq}\gloss{its belt}
        \ex \intxt{xʷiçoʔ}\gloss{coal} → \intxt{la-xʷíçoʔ}\gloss{its coal}
        \ex \intxt{qakʲaʔ}\gloss{medicine} → \intxt{la-qákʲaʔ}\gloss{its medicine}
    \end{xlist}
\z
\boolfalse{listing}

In a great number of disyllabic stems, it is impossible to determine whether their initial syllable carries an underlying accent or not, since these stems never occur without a prefix. Some examples are shown below. Note the loss of the word-final glottal stop in an unaccented syllable in Nivaĉle and Wichí in \REF{amph-soninlaw}, \REF{amph-welln}, \REF{amph-arrow}, \REF{amph-youngersis}, \REF{amph-hear}, \REF{amph-distrust}, \REF{amph-eyelash}, as described in \sectref{ni-deglottalization-codas} and \sectref{wi-posttonic-deglottalization}.

\begin{exe}
    \ex \rightn
    \ex \disease
    \ex \spitcw
    \ex \sisinlaw
    \ex \soninlaw \label{amph-soninlaw}
    \ex \dreamn
    \ex \dreamv
    \ex \welln \label{amph-welln}
    \ex \grandchild
    \ex \neighbor
    \ex \arrowkaxe \label{amph-arrow}
    \ex \spouse
    \ex \dividev
    \ex \youngerbro
    \ex \youngersis \label{amph-youngersis}
    \ex \smellv
    \ex \heel
    \ex \lippaset
    \ex \hear \label{amph-hear}
    \ex \full
    \ex \fillv
    \ex \distrust \label{amph-distrust}
    \ex \belt
    \ex \yellowv
    \ex \fishwithhook
    \ex \noden
    \ex \dinlaw
    \ex \acquainted
    \ex \eyelash \label{amph-eyelash}
    \ex \bilecw
    \ex \rheum
    \ex \tooth
    \ex \leafhaircw
    \ex \egg
    \ex \headn
    \ex \temperance
\end{exe}

Peninitial accent also occurs when an unaccented prefix is attached to an accented monosyllabic stem followed by a suffix (either accented or not).

\begin{exe}
    \ex \fall
    \ex \grabwork
    \ex \meet
    \ex \coverappl
    \ex \costumepl
    \ex \bloodpl
    \ex \saber
\end{exe}

A combination of an (unprefixed) iambic root and a suffix is also expected to result in peninitial accent. Note that the Chorote reflex in \REF{iteneis} is reconstructed based on the Iyo'awujwa' reflex \intxt{itán\mbox{-}is}, attested in \citet[132]{AG83}, whereas Manjui shows an irregular rightward stress shift: \wordnl{ʔiten\mbox{-}éis}{thorns}. The Iyo'awujwa' datum is considered more conservative because it fits better with the rest of the comparative data.

\begin{exe}
    \ex \algarrobot
    \ex \waterpl
    \ex \thorncutjanpl \label{iteneis}
    \ex \chajapl
\end{exe}

Finally, peninitial stress is found in a number of unprefixed trisyllabic roots. It is preserved in all derivatives and inflected forms.

\begin{exe}
    \ex \durmili
    \ex \tuscaf
    \ex \tuscat
    \ex \tuscag
    \ex \caracara
    \ex \wildhoney
    \ex \mistolf
    \ex \mistolt
    \ex \widower
    \ex \puma
    \ex \lessergrison
\end{exe}

\subsection{¯˘˘}  \label{larga-corta-corta}

Initial accent in polysyllabic words occurs whenever the initial syllable is lexically specified as accented. This is especially common in roots. In such cases, Chorote retains initial accent, and 'Weenhayek has a long vowel in the initial syllable and short vowels in all other syllables.\footnote{The position of stress in the Nivaĉle reflexes of the words of this type is not documented in \citet{AnG15}. Since the language requires a primary stress within a disyllabic window at the right edge of a prosodic word, we predict that PM polysyllabic words with initial stress are reflected with a final (default) stress in Nivaĉle, as described for trisyllabic nouns by \citet[165]{AnG15}. \perscommp{Analía Gutiérrez}{2023} reports that our prediction is in fact borne out for many of these forms, though not all of them are documented in her corpus, with the proviso that the examples with an initial heavy (CVC) syllable carry a secondary initial stress \citep[34, 55]{AnG19}.} The peninitial vowel is sometimes syncopated in Wichí and less commonly in other languages, as in \REF{sync-quebracho}, \REF{sync-tsoftaj}, \REF{sync-rhea}, \REF{sync-wildbean}, \REF{sync-hiccup}, \REF{sync-heartcwits}.

\begin{exe}
    \ex \redquebracho \label{sync-quebracho}
    \ex \chaniarf
    \ex \chaniart
    \ex \iscayante
    \ex \nightmonkey
    \ex \bearded
    \ex \teach
    \ex \tsofatajf \label{sync-tsoftaj}
    \ex \tsofatajt
    \ex \guayacan
    \ex \blackalgarrobof
    \ex \blackalgarrobot
    \ex \cardinal
    \ex \rhea \label{sync-rhea}
    \ex \jelayuk
    \ex \alligator
    \ex \peccary
    \ex \chaguara
    \ex \wildbean \label{sync-wildbean}
    \ex \waspaniti
    \ex \teach
    \ex \hiccup \label{sync-hiccup}
    \ex \heartcwits \label{sync-heartcwits}
\end{exe}

At present, we have found no evidence for reconstructing accented prefixes for Proto-Mataguayan, though prefixes with an underlying long vowel do exist in 'Weenhayek (for example, \word{'Wk}{ˀnó\mbox{-}}{{\textsc{gnr}}}, \wordnl{ʔá\mbox{-}}{2.S\textsubscript{{P}}}). In this language, such prefixes always keep their long vowel and shorten all subsequent vowels in a given phonological word (except in innovative forms that arose due to Watkins' Law and that are therefore not reconstructible to Proto-Mataguayan), regardless of whether the stem is underlyingly unaccented, as in \REF{'nookyowej}--\REF{'aathalàk}, trochaic, as in \REF{'noo'nàyij}--\REF{'aakyoyhet}, or iambic, as in \REF{'noojwi'yet}--\REF{'aapitaj}.

\booltrue{listing}
\ea \label{ex:longprefixes:whk}
    ’Weenhayek \citep{KC16}
    \begin{xlist}
        \ex \intxt{kʲowex}\gloss{hole} → \intxt{ˀnó-kʲowex}\gloss{one's center}\label{'nookyowej}
        \ex \intxt{xʷiçoʔ}\gloss{coal} → \intxt{ˀnó-xʷiçoʔ}\gloss{one's coal}
        \ex \intxt{qakʲaʔ}\gloss{medicine} → \intxt{ˀnó-qakʲaʔ}\gloss{one's medicine}\label{'nooqakya'}
        \ex \intxt{tʰalåk}\gloss{old} → \intxt{ʔá-tʰalåk}\gloss{you are old}\label{'aathalàk}
        \ex \intxt{ˀnǻjix}\gloss{path} → \intxt{ˀnó-ˀnåjix}\gloss{one's path}\label{'noo'nàyij}
        \ex \intxt{xʷétes}\gloss{root} → \intxt{ˀnó-xʷetes}\gloss{one's root}
        \ex \intxt{tútsax}\gloss{smoke} → \intxt{ˀnó-tutsax}\gloss{one's smoke}\label{'nootutsaj}
        \ex \intxt{kʲóçet}\gloss{heavy} → \intxt{ʔá-kʲoçet}\gloss{you are heavy}\label{'aakyoyhet}
        \ex \intxt{xʷiˀjét}\gloss{cold} → \intxt{ˀnó-xʷiˀjet}\gloss{one's cold}\label{'noojwi'yet}
        \ex \intxt{ˀwoj-ís}\gloss{blood} → \intxt{ˀnó-ˀwoj-is}\gloss{one's blood}\label{'noo'woyis}
        \ex \intxt{pitáx}\gloss{long, tall} → \intxt{ʔá-pitax}\gloss{you are tall}\label{'aapitaj}
    \end{xlist}
\z
\boolfalse{listing}

One can therefore conclude that if Proto-Mataguayan had prefixes with an underlying accent, the accent of the prefix most likely overrode any underlying accents located further to the right.

\section{Conclusions} \label{prosody-conclusions}

We have seen that the position of stress in Chorote and the distribution of long vowels in 'Weenhayek can be rather neatly explained by positing word-level accent for Proto-Mataguayan. Nivaĉle and Lower Bermejeño Wichí also show traces of an erstwhile word-level accent, whereby word-final glottal stops are lost if there is an accent in a non-final syllable (this deglottalization process is fed by accent retraction in Wichí, but not in Nivaĉle). It is quite likely that some of the reconstructed PM~patterns actually survive in some varieties of Nivaĉle, a topic worthy of further research.

We have also seen that the position of the word-level accent in Proto-Mataguayan can be determined by examining the underlying accentual properties of individual morphemes. Any morpheme can have or lack an underlying accent. The leftmost underlying accent is the one that appears in the surface realization, whereas all subsequent accents are deleted. If no morpheme in a given mono- or disyllabic word contains an underlying accent, the entire word surfaces as unaccented. Longer words cannot surface as unaccented, and if all morphemes in a given polysyllabic word are specified as unaccented, a default accent is inserted in the peninitial syllable.

The derivation of the surface accent in PM from the underlying accentual properties of its morphemes, as well as the reflexes of the PM~accentual patterns in the contemporary languages, are shown in \tabref{PM-accent-patterns}.

\begin{table}
\caption{PM accent patterns and their reflexes}
\label{PM-accent-patterns}
\fittable{
 \begin{tabular}{ccccccc}
  \lsptoprule
           PM (underlying) & PM (surface) & Ni & I'w/Mj & Ijw & 'Wk & LB\\\midrule
  ˘ & ˘ & ¯ & ¯ & ¯ & ˘ &\\
  ¯ & ¯ & ¯ & ¯ & ¯ & ¯ &\\
  ˘˘ & ˘˘ & ˘¯ & ˘¯ & ˘¯ & ˘˘ &\\
  ˘¯ & ˘¯ & ˘¯ & ˘¯ & ˘¯ & ˘¯ &\\
  ¯˘ / ¯¯ & ¯˘ & ¯˘ (-ʔ → ∅) \recind ˘¯ & ¯˘ & ¯˘ & ¯˘ & -ʔ → ∅\\
  ˘˘¯ & ˘˘¯ & ˘˘¯ (?) & ˘˘¯ & ˘¯˘ & ˘¯˘ & -ʔ → ∅\\
  ˘˘˘ / ˘¯˘ / ˘¯¯ & ˘¯˘ & ˘¯˘ (-ʔ → ∅) & ˘¯˘ & ˘¯˘ & ˘¯˘ & -ʔ → ∅\\
  ¯˘˘ / ¯˘¯ / ¯¯˘ / ¯¯¯ & ¯˘˘ & ˘˘¯ (?) & ¯˘˘ & ¯˘˘ & ¯˘˘ & -ʔ → ∅\\
  \lspbottomrule
 \end{tabular}
 }
\end{table}

The pattern whereby the surface accent (ictus) placement is determined based on the underlying accentual properties of individual morphemes by means of a rule (or a set or rules) is by no means exclusive to Mataguayan. Similar systems, where morphemes are underlyingly specified as dominant (underlyingly accented) or recessive (lacking an underlying accent) – among other possibilities, such as preaccenting or postaccenting – are documented in a diverse set of languages, including the Uto-Aztecan languages Cupeño \citep{JHH-KH-68,JDA99} and Choguita Rarámuri \citep{GC11,GC-LC-15}; the Salishan language Nłeʔkepmxcín (also known as Thompson) and other closely related languages \citep{LCT-MTT-92,GC02}; the Saharan language Dazaga \citep{VD95}; the Northwest Caucasian languages Abkhaz, Abaza, and Ubykh \citep{AS85,VD00,LB21}; the Macro-Jê language Chiquitano \citep{AN22}; and are possibly best known from a number of Indo-European languages \citep{PK-MH-77}, particularly those of the Balto-Slavic branch (Lithuanian, Old Prussian, Slovincian, Slovene, Bosnian–Croatian–Serbian, Bulgarian, Ukrainian, Belarusian, Russian, and some Rusyn dialects), as analyzed by a number of authors \citep{AZ85,JLM89,VD00,YK19}. Proto-Mataguayan is similar to languages such as Dazaga and Old Russian in that the stress falls on the leftmost underlyingly accented mora, overriding all subsequent underlying accents (unlike in Chiquitano, where the rule operates from right to left, or in Abkhaz, where the final accent in the leftmost sequence of accented morphemes makes it to the surface). However, it differs from these languages in that enclinomena (words where all morphemes are underlyingly unaccented) do not receive a default initial stress, but rather acquire a default peninitial accent in polysyllabic words (and, in Chorote and Nivaĉle, also in disyllabic ones), like in Choguita Rarámuri. This combination of features makes Mataguayan particularly interesting from a cross-linguistic perspective.
