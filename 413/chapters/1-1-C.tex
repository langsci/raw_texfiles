\chapter{Consonants}\label{pm-consonants}
This chapter deals with the reconstruction of the Proto-Mataguayan (PM) consonants. We reconstruct an inventory composed of seventeen plain (non-glottal\-ized) consonants, including six voiceless stops,\footnote{\sound{PM}{*ts} is reconstructed as an affricate, but it fits phonologically into the stop series (see \citnp{JR94}, \citnp{GNC99} on the possibility of analyzing affricates as strident stops).} six voiceless fricatives, three approximants, and two nasals, in addition to a series of glottalized consonants, as shown in \tabref{PM-inv-cons}. Note that the phonemic status of \sound{PM}{*ɬ’} is dubious; this sound arose when an underlying */ɬ/ coalesced with an underlying heteromorphemic */ʔ/ (\sectref{glott-status}).

\begin{table}
\caption{PM consonants}
\label{PM-inv-cons}
 \begin{tabular}{rcccccc}
  \lsptoprule
            & labial & dental & alveolar & velar & uvular & glottal\\\midrule
  stops & *p *p’ & *t *t’ & *ts *ts’ & *k *k’ & *q *q’ & *ʔ\\
  fricatives & *ɸ *ɸ’ & *ɬ (*ɬ’) & *s *s’ & *x & *χ & *h\\
  approximants & *w *ˀw & *l *ˀl & \multicolumn{2}{c}{*j *ˀj} & &\\
  nasals & *m *ˀm & *n *ˀn &  & & &\\
  \lspbottomrule
 \end{tabular}
\end{table}

We depart from earlier proposals in reconstructing */ɸ/ (based on the reflexes in Maká and Nivaĉle) instead of */xʷ/ (a reconstruction based on the Wichí reflex) and show that this segment was related to */p/ in the same way that */ɬ~s~x~χ/ were related to */t~ts~k~q/. Although in most Mataguayan varieties the glottal stop is automatically inserted in onsetless syllables (at least word-initially), we show that in Proto-Mataguayan vowel-initial stems clearly contrasted with *\intxt{ʔ}\mbox{-}initial stems, as shown by the alternations in prefixes which attached to such stems.

We follow \citet{PVB02} in reconstructing an opposition between velar, uvular, and glottal stops and fricatives. The opposition in question is relatively well preserved in Maká, whereas in other languages it has been subject to partial, conditioned mergers.

The reconstruction of a glottalization feature in consonants is somewhat controversial: at least in some cases it is possible to show, via internal reconstruction, that glottalized onsets in contemporary Mataguayan languages go back to earlier clusters of the type */Cʔ/. No such evidence is available for tautomorphemic glottalized onsets. It is unproblematic to derive most glottalized consonants from Proto-Mataguayan */Cʔ/ clusters, given that there is independent evidence for the sound change */Cʔ/ > */C’/ and that clusters of the type */Cʔ/ are otherwise not reconstructed. However, this solution is not available for the consonants \intxt{*ˀl} and \intxt{*ˀm}, which synchronically contrast with the clusters \intxt{*lʔ} and \intxt{*mʔ}.

The basic sound correspondences for plain onsets and codas are discussed in \sectref{simplex-C}. \sectref{glott-ons} deals with the status of the glottalized onsets in PM. \sectref{glott-codas} is dedicated to the glottalized codas. In \sectref{cx}, we discuss the reconstruction of the consonant clusters of the type \intxt{*CX} (where \intxt{C} stands for a consonant and \intxt{X} for a velar or postvelar fricative). Tautosyllabic consonant clusters of other shapes are dealt with in \sectref{clusters}. In \sectref{syllabic-C}, we show that some affixes formed a syllable on their own despite containing a single consonant in PM.

\section{Plain onsets and codas}\label{simplex-C}
\sloppy
In this subsection, we present our reconstruction of the PM~consonants in the most basic environment, i.e., when they occur as simplex onsets or codas. \tabref{PM-refl-cons} shows the basic reflexes of the PM~consonants in individual Mataguayan languages.

\begin{table}[p]
\caption{PM consonants and their reflexes}
\label{PM-refl-cons}
\small
 \begin{tabularx}{\textwidth}{lcCCcc}
  \lsptoprule
            \multicolumn{2}{l}{Proto-Mataguayan} & Maká & Nivaĉle & Proto-Chorote & Proto-Wichí\\
            \midrule
  *p & & p & p & *p & *p\\
  *t & & t & t & *t & *t\\
  \multirow{2}*{*ts} & onset & \multirow{2}*{ts} & ts & \multirow{2}*{*s} & *ts\\
  & coda & & s & & *s\\
  \tablevspace
  \multirow{2}*{*k} & onset & \multirow{2}*{k} & \multirow{2}*{k, tʃ\textsuperscript{A}} & \multirow{2}*{k} & *kʲ\\
  & coda & & & & *q, *kʷ\textsuperscript{B}\\
  \tablevspace
  *q & & q & k & *q & *q\\
  *ʔ & & ʔ & ʔ, ∅\textsuperscript{C} & *ʔ & *ʔ, ∅\textsuperscript{C}, *h\textsuperscript{D}\\
  \tablevspace
  \multirow{2}*{*ɸ} & onset & \multirow{2}*{f} & \multirow{2}*{ɸ} & *hw & \multirow{2}*{*xʷ}\\
  & coda & & & *ʍ & \\
  \tablevspace
  \multirow{2}*{*ɬ} & onset & \multirow{2}*{ɬ} & \multirow{2}*{ɬ} & *hl & \multirow{2}*{*ɬ}\\
  & coda & & & *ɬ & \\
  \tablevspace
  *s & & s & s & *s & *s\\
  \tablevspace
  \multirow{2}*{*x} & onset & \multirow{2}*{x} & \multirow{2}*{x, ʃ\textsuperscript{A}} & *h, *hw\textsuperscript{E} & *h\\
  & coda & & & *h, *ʍ\textsuperscript{E} & *χ, *xʷ\textsuperscript{E}\\
  \tablevspace
  \multirow{2}*{*χ} & onset & \multirow{2}*{χ} & \multirow{2}*{x} & *h, *hw\textsuperscript{F} & *xʷ\textsuperscript{F}\\
  & coda & & & *h & *χ, *xʷ\textsuperscript{F}\\
  \tablevspace
  \multirow{2}*{*h} & onset & h & h & *h, *∅\textsuperscript{G} & *h\\
  & coda & ∅ & ∅ & *h & *h, *∅\textsuperscript{H}\\
  \tablevspace
  *w & & w & β & *w & *w\\
  \tablevspace
  \multirow{2}*{*l} & onset & \multirow{2}*{l} & k͡l &  \multirow{2}*{*l} & *l\\
  & coda & & k & & *l, *lʰ\textsuperscript{I}\\
  \tablevspace
  *j & & j & j & *j & *j\\
  *m & & m & m & *m & *m\\
  *n & & n & n & *n & *n, *ˀn\textsuperscript{J}\\
  \lspbottomrule
 \end{tabularx}
 \legendbox{\textsuperscript{A}before or after non-back vowels, except when preceded by a back vowel, possibly with an intervening [+grave] consonant (\sectref{ni-vel-uv});\\
  \textsuperscript{B}after a back vowel (\sectref{wi-q-k});\\
  \textsuperscript{C}word-finally in posttonic syllables (\sectref{ni-deglottalization-codas}, \sectref{wi-posttonic-deglottalization});\\
  \textsuperscript{D}preceding a syllable with a glottalized onset (\sectref{wi-glottal-dissim});\\
  \textsuperscript{E}following   \intxt{*u} (\sectref{ch-j-jj}, \sectref{wi-jj-j-h});\\
  \textsuperscript{F}following   \intxt{*o} or \intxt{*u} (\sectref{ch-j-jj}, \sectref{wi-jj-j-h});\\
  \textsuperscript{G}in onsets of unstressed syllables (\sectref{ch-j-jj});\\
  \textsuperscript{H}following a syllable with a glottalized sonorant onset (\sectref{wi-h-loss});\\
  \textsuperscript{I}word-finally (\sectref{wi-l-lh});\\
  \textsuperscript{J}as an onset of a word-final open syllable (\sectref{wi-nv-nvh})
  }

\end{table}

\subsection{\sound{PM}{*p}}\label{proto-p}
\sound{PM}{*p} is a stable phoneme: it is preserved in all daughter languages as \intxt{p}.

\begin{exe}
    \ex \cry
    \ex \returnth
    \ex \shoulder
    \ex \shoulderblade
    \ex \winter
    \ex \demp
    \ex \bitter
    \ex \lip
    \ex \shuck
    \ex \quick
    \ex \jabiru
    \ex \deep
    \ex \hear
    \ex \rain
    \ex \up
    \ex \returnh
    \ex \longv
    \ex \full
    \ex \fillv
    \ex \tapeti
    \ex \beard
    \ex \nest
    \ex \spring
    \ex \grass
    \ex \straw
    \ex \shadow
\end{exe}

The very same correspondence is observed in etymologies with a limited distribution (Maká and Nivaĉle, Chorote and Wichí), whose PM~age is thus questionable.

\begin{exe}
    \ex \hunger
    \ex \heel
    \ex \frog
    \ex \cook
    \ex \orphancw
    \ex \skycloud
    \ex \drum
    \ex \dwarf
    \ex \dovesipup
    \ex \ashamedmn
\end{exe}

\subsection{\sound{PM}{*t}}\label{proto-t}
\sound{PM}{*t} is a stable phoneme: it is preserved in all daughter languages as \intxt{t}. An irregular glottalized reflex is found in Wichí in \REF{t-thunder}.

\begin{exe}
    \ex \fallonitsown
    \ex \wordamet
    \ex \bleedv
    \ex \drinkn
    \ex \jar
    \ex \fatv
    \ex \firef
    \ex \rootn
    \ex \coldweather
    \ex \elbow
    \ex \fart
    \ex \water
    \ex \grove
    \ex \fall
    \ex \cactus
    \ex \eldersis
    \ex \meet
    \ex \barnowl
    \ex \thorncutjan
    \ex \flee
    \ex \squash
    \ex \iscayante
    \ex \firewoodlhet
    \ex \defect
    \ex \hither
    \ex \sleepiness
    \ex \mucus
    \ex \snore
    \ex \cavy
    \ex \lippaset
    \ex \shuck
    \ex \deep
    \ex \longv
    \ex \fillv
    \ex \tapeti
    \ex \lid
    \ex \starn
    \ex \vein
    \ex \mesh
    \ex \whitequebracho
    \ex \kingvulture
    \ex \cat
    \ex \thunder \label{t-thunder}
    \ex \pseudo
    \ex \acquainted
    \ex \sprout
    \ex \dinlaw
    \ex \eyelash
    \ex \abdcavity
    \ex \basetrunk
    \ex \trunk
    \ex \eye
    \ex \river
    \ex \spinsew
    \ex \plate
    \ex \face
    \ex \eyebrow
    \ex \blind
    \ex \snake
    \ex \far
    \ex \ant
    \ex \yellowlegs
    \ex \uncle
    \ex \tooth
    \ex \tsofatajf
    \ex \tsofatajt
    \ex \paloflojof
    \ex \placen
    \ex \headn
    \ex \tuscaf
    \ex \tuscat
    \ex \tuscag
    \ex \earth
    \ex \sandyplace
    \ex \pushv
    \ex \snakeatuj
    \ex \femalebreast
    \ex \hurt
    \ex \argentineboa
    \ex \waspaniti
    \ex \wildpepper
    \ex \firei
    \ex \chest
\end{exe}

The very same correspondence is observed in etymologies with a limited distribution (Maká and Nivaĉle, Chorote and Wichí), whose PM~age is thus questionable. The correspondence between a plain stop in Wichí and a glottalized stop in Chorote in \REF{t-kidney} is irregular.

\begin{exe}
    \ex \foodmn
    \ex \spin
    \ex \ocelot
    \ex \redv
    \ex \heavyv
    \ex \tortoise
    \ex \whitealgarrobof
    \ex \kidney \label{t-kidney}
    \ex \earcw
    \ex \pacu
    \ex \feel
    \ex \heartmn
    \ex \standv
    \ex \yellowv
    \ex \durmili
    \ex \cardon
    \ex \chachalaca
    \ex \utensil
    \ex \vertical
    \ex \bilecw
    \ex \precipice
    \ex \smoke
    \ex \burnvi
    \ex \throwcw
    \ex \cardinal
    \ex \wut
    \ex \cebil
    \ex \orphanmn
    \ex \queenpalmf
    \ex \heartcw
\end{exe}

In a number of \intxt{t}-initial verbs in Maká, which belong to the 7\textsuperscript{th} conjugation in \cits{AG94} classification, the initial consonant changes to \intxt{ɬ\mbox{-}} after the prefixes \intxt{xite\mbox{-}} 1\INCL.\IND, \intxt{xinte\mbox{-}/qinte\mbox{-}} 1\INCL.\NIND, \intxt{k’e\mbox{-}} 1>2, \intxt{tse\mbox{-}} 3>1, \intxt{ne\mbox{-}} 3>2, \intxt{∅-} 2\IMP \citep[96, 100, 145]{AG94}. Their cognates in Nivaĉle present a similar alternation: their citation form starts with a \intxt{t\mbox{-}}, which changes to \intxt{ɬ\mbox{-}} after the reflexive prefix \intxt{βat\mbox{-}} \citep[191, fn. 163]{AF16}. All such verbs select for a zero third-person prefix in Nivaĉle, which is also true of their cognates in Maká and Wichí (but not in Chorote, where they take the allomorph \intxt{ʔi\mbox{-}}). The origins of the alternation between \intxt{t\mbox{-}} and \intxt{ɬ\mbox{-}} are as of yet unclear.

\begin{exe}
    \ex \spend
    \ex \suckb
    \ex \weave
    \ex \shoot
    \ex \carrysh
    \ex \swallow
    \ex \invite
    \ex \dig
    \ex \eatvt
\end{exe}

\subsection{\sound{PM}{*ts}}\label{proto-ts}
\sound{PM}{*ts} is preserved as a distinct segment in all Mataguayan languages except Chorote, which merges it with \sound{PM}{*s} as \sound{PCh}{*s} in all positions (\sectref{ch-ts}, but see \sectref{pch-s} for possible remnants of \intxt{*ts} in the Iyo’awujwa’ variety of Chorote).

\begin{exe}
    \ex \centipede
    \ex \suncho
    \ex \palm
    \ex \oldn
    \ex \chaniart
    \ex \bow
    \ex \quick
    \ex \jabiru
    \ex \eyelash
    \ex \demts
    \ex \chaja
    \ex \duraznillo
    \ex \fullriver
    \ex \tooth
    \ex \tsofa
    \ex \tsofatajf
    \ex \tsofatajt
    \ex \healthy
    \ex \peccary
    \ex \caracara
    \ex \chaguara
\end{exe}

The very same correspondence is observed in etymologies with a limited distribution (Maká and Nivaĉle, Chorote and Wichí), whose PM~age is thus questionable.

\begin{exe}
    \ex \spillmn
    \ex \grandchildmpl
    \ex \chaguark
    \ex \willow
    \ex \majan
    \ex \limpkin
    \ex \smoke
    \ex \tsaqaq
    \ex \spillcw
    \ex \silkfloss
    \ex \redbrocket
    \ex \eel
\end{exe}

However, the occurrence of \intxt{ts} is synchronically limited to the onset position in Nivaĉle \citep[45]{AnG15} and Wichí (\cites[15]{KC94}[42]{JT09-th}[50]{VN14}).\footnote{As an exception, in Nivaĉle \intxt{ts} can occur in codas when followed by \intxt{x} or \intxt{ɸ}. Although it could be tempting to assume that the sequences \intxt{tsx} and \intxt{tsɸ} are always tautosyllabic in Nivaĉle, \citet{AnG15} reports that \intxt{ts} does syllabify as a coda in such cases.\label{tsx-heterosyllabic}} This restriction arose as a result of a diachronic deaffrication of \sound{PM}{*ts} > \intxt{s} in codas in these languages. Of all Mataguayan languages, only Maká preserves \sound{PM}{*ts} in the coda position.

\begin{exe}
    \ex \sisinlaw
    \ex \rootn
    \ex \dew
    \ex \offspring
    \ex \basetrunk
    \ex \trunk
    \ex \plits
    \ex \starn
    \ex \gutscw
    \ex \knee
    \ex \wildpepper
\end{exe}

In some etyma, the erstwhile presence of an affricate in certain forms is suggested by the synchronically active alternations in Nivaĉle and Wichí: compare \word{Ni}{\mbox{-}fetats\mbox{-}ij}{roots}, \wordnl{\mbox{-}(ʔa)kxatsu\mbox{-}j}{knees}, \wordnl{\mbox{-}tats\mbox{-}uk}{trunk} (where \intxt{ts} is syllabified as an onset and thus fails to deaffricate) vs. \wordnl{\mbox{-}fetas}{root}, \wordnl{\mbox{-}(ʔa)kxuˀs}{knee}, \wordnl{\mbox{-}tas\mbox{-}ku\mbox{-}j}{trunks}; \word{PW}{*\mbox{-}téts\mbox{-}elʰ}{trunks, bases}, \wordnl{*qatéts\mbox{-}elʰ}{stars} vs. \wordnl{*\mbox{-}tes}{trunk, base}, \wordnl{*qates}{star}.

Both in Nivaĉle and Wichí, underlying \intxt{ts} can also alternate with \intxt{t} in the coda position: compare \word{Ni}{xa\mbox{-}nuts\mbox{-}xa\mbox{-}jan}{I cause him/her to be angry}, \wordnl{kuts\mbox{-}xanax}{thief, robber}, \wordnl{xa\mbox{-}taβkits\mbox{-}xat}{I make him/her/it dizzy} (see footnote \ref{tsx-heterosyllabic} on the status of \intxt{tsx}) vs. \wordnl{xa-nut}{I get angry}, \wordnl{ɬa\mbox{-}t\mbox{-}kut}{you steal}, \wordnl{tsi\mbox{-}taβkit}{I am dizzy, I get dizzy} \citep[50]{LC20}; \word{LB}{mati\mbox{-}qut}{the one who always drinks mate} vs. \wordnl{mati\mbox{-}quts\mbox{-}es}{the ones who always drink mate} \citep[200]{VN14}. These data suggest that in some cases \sound{PM}{*ts} could deaffricate to \intxt{t} in the coda position in Nivaĉle and Wichí. However, we have been unable to identify Mataguayan etymologies for morphemes that undergo the alternation in question, and the question regarding its diachronic origins thus remains unresolved.

\subsection{\sound{PM}{*k}}\label{proto-k}
\sound{PM}{*k} is preserved as a velar stop in Maká, whereas in other languages it has suffered a number of splits. In Nivaĉle, it palatalizes to \intxt{tʃ} before or after non-back vowels (\sound{PM}{*i}, \intxt{*e}, \intxt{*ä}, \intxt{*a} > \sound{Ni}{i}, \intxt{e}, \intxt{a}), except when preceded by a back vowel, possibly with an intervening [+grave] consonant (see \sectref{ni-vel-uv} for more details). In Chorote, it is usually reflected as \sound{PCh}{*k} (typically reflected as \intxt{kʲ} in the contemporary Chorote lects); however, in several cases it is reflected as \sound{PCh}{*q} in onsets when next to the vowel \intxt{*u}. In Wichí, \sound{PM}{*k} always palatalizes to \sound{PW}{*kʲ} in the onset position, whereas in codas it is reflected as \sound{PW}{*q} (phonetically *[k]) following front vowels and as \sound{PW}{*kʷ} following back vowels. The tendency of \sound{PM}{*k} to palatalize in the daughter languages suggests that it may have had a palatalized allophone (at least in onsets when next to front vowels) already in Proto-Mataguayan, as is still the case in Maká \citep[24]{AG89}.

The following examples show the development of \sound{PM}{*k} in the onset position, where it is reflected as \sound{Mk}{k}, \sound{Ni}{k} or \intxt{tʃ}, \sound{PCh}{*k}, \sound{PW}{*kʲ}. The correspondence between a glottalized stop in Maká and a plain stop in Chorote in \REF{k-roundkoy} is irregular. The failure of \sound{PM}{*k} to palatalize in Nivaĉle before an \intxt{a} in \REF{l-demk} is unexpected; if the gender distinction seen in Maká goes back to Proto-Mataguayan, we might be dealing with a contamination of \wordng{PM}{*kåʔ} (masculine) and \intxt{*kaʔ} (feminine), whose expected reflexes in Nivaĉle would be \intxt{*kåʔ} and \intxt{*tʃaʔ}, respectively.

\begin{exe}
    \ex \north
    \ex \demk \label{l-demk}
    \ex \tooln
    \ex \grove
    \ex \takeaway
    \ex \testicle
    \ex \tail
    \ex \fall
    \ex \redquebracho
    \ex \sendv
    \ex \feminine
    \ex \cactus
    \ex \neighbor
    \ex \elderbro
    \ex \eldersis
    \ex \hand
    \ex \roundkoy \label{k-roundkoy}
    \ex \leniosapl
    \ex \sunn
    \ex \answer
    \ex \grabwork
    \ex \eatkun
    \ex \heat
    \ex \meet
    \ex \sweat
    \ex \flu
    \ex \squash
    \ex \distrust
    \ex \mesh
    \ex \face
    \ex \eyebrow
    \ex \ant
    \ex \unclepl
\end{exe}

The very same correspondence is observed in etymologies with a limited distribution (Maká and Nivaĉle, Chorote and Wichí), whose PM~age is thus questionable. The correspondence between a plain stop in Wichí and a glottalized stop in Chorote in \REF{k-chaguark} is irregular.

\begin{exe}
    \ex \hunger
    \ex \lizard
    \ex \fry
    \ex \redv
    \ex \torn
    \ex \grandchild
    \ex \heavyv
    \ex \locustcw
    \ex \hole
    \ex \tortoise
    \ex \whitealgarrobof
    \ex \runv
    \ex \chaguark \label{k-chaguark}
    \ex \kidney
    \ex \willow
    \ex \medicine
    \ex \precipice
    \ex \metal
\end{exe}

In the coda position, \sound{PM}{*k} is reflected as \sound{Mk}{k}, \sound{Ni}{k} or \intxt{tʃ}, \sound{PCh}{*k}, \sound{PW}{*q} or \intxt{kʷ} (see \sectref{wi-q-k}). Note that this consonant never occurs in codas following the vowel \sound{PM}{*a}.

\begin{exe}
    \ex \honeycomb
    \ex \goaway
    \ex \mortar
    \ex \hidev
    \ex \palm
    \ex \thread
    \ex \yicalhuk
    \ex \powder
    \ex \feces
    \ex \zorzal
    \ex \wildmanioc
    \ex \rope
    \ex \fence
    \ex \cat
    \ex \river
    \ex \plate
    \ex \blind
    \ex \unclesg
    \ex \duraznillo
    \ex \leniosasg
    \ex \badmood
    \ex \allrcpr
    \ex \headn
    \ex \straw
    \ex \palosanto
    \ex \firewoodhuk
    \ex \wildhoney
    \ex \eatvi
\end{exe}

The very same correspondence is observed in etymologies with a limited distribution (Maká and Nivaĉle, Chorote and Wichí), whose PM~age is thus questionable.

\begin{exe}
    \ex \tobacco
    \ex \arrowfok
    \ex \two
    \ex \bilecw
    \ex \silkfloss
    \ex \temperance
    \ex \cebil
    \ex \paralytic
    \ex \queenpalmf
\end{exe}

As we will see in \sectref{velar-weakening}, in some cases stem-final \sound{PM}{*k} may alternate with \sound{PM}{*h} (or zero after fricatives).

\subsection{\sound{PM}{*q}}\label{proto-q}
\sound{PM}{*q} is preserved as a distinct segment in Maká, Proto-Chorote, and Wichí, but not in Nivaĉle, where it yields \intxt{k} (phonetically, it can still be pronounced as uvular in some environments, but there is no longer an opposition between velars and uvulars in Nivaĉle). In codas, it merges with \sound{PM}{*k} as \sound{PCh}{*k} in Chorote. Note that when \sound{PM}{*q} occurs in a coda position, it can only be preceded by a low vowel (\sound{PM}{*a} or \intxt{å}). In one cognate set, there is an irregular correspondence between a plain stop in Nivaĉle and a glottalized stop in Chorote \REF{q-snore}.

\begin{exe}
    \ex \food
    \ex \elbow
    \ex \welln
    \ex \snore \label{q-snore}
    \ex \inorderto
    \ex \alienable
    \ex \distrust
    \ex \leg
    \ex \starn
    \ex \fishwithhook
    \ex \gutscw
    \ex \costume
    \ex \wildcat
    \ex \chaja
    \ex \wildhoney
    \ex \knee
\end{exe}

The very same correspondence is observed in etymologies with a limited distribution (Maká and Nivaĉle, Chorote and Wichí), whose PM~age is thus questionable.

\begin{exe}
    \ex \dwarf
    \ex \medicine
    \ex \standv
    \ex \limpkin
    \ex \belt
    \ex \yellowv
    \ex \noden
    \ex \cardon
    \ex \tsaqaq
    \ex \paralytic
    \ex \cord
\end{exe}

\subsection{\sound{PM}{*ʔ}}\label{proto-glottal}

In Proto-Mataguayan, as in most contemporary Mataguayan varieties, all syllables are required to have an onset, unless the nucleus is a syllabic consonant (see \sectref{syllabic-C}). The default consonant inserted in order to satisfy this requirement is \sound{PM}{*ʔ}. For example, underlying vowel-initial stems such as \word{PM}{*\mbox{-}ǻseʔ}{daughter} (which contrast with underlying \sound{PM}{*ʔ}\mbox{-}initial stems, such as \wordnl{*\mbox{-}ʔúɬu}{urine}) take a zero allomorph of the second-person prefix, and a glottal stop is inserted in order to prevent the resulting word from starting with an onsetless syllable: compare \word{PM}{*ʔǻseʔ}{your daughter} (with an inserted glottal stop) and \wordnl{*ʔúɬu}{your urine} (with an underlying glottal stop). For similar rules in the contemporary Mataguayan languages, see \citet[43, 67, 102--105]{AnG15} for Nivaĉle, \citet[90]{JC14b} for Iyojwa’aja’ (word-initially only).

If a stem that starts with \sound{PM}{*ʔ} is incompatible with prefixes, it is impossible to determine whether the glottal stop is inserted or underlying. This is also the case with intervocalic occurrences of \sound{PM}{*ʔ} within a morpheme. Whether one analyzes them as underlying or epenthetic is, therefore, a matter of one's theoretical preferences. In the contemporary languages, \sound{PM}{*ʔ} in onsets is preserved at least in Nivaĉle, Iyojwa’aja’, Manjui, ’Weenhayek, Lower Bermejeño Wichí, and possibly other varieties, except that in Wichí it dissimilates to \sound{PW}{*h} whenever the onset of the following syllable is a glottalized consonant. In Maká, \sound{PM}{*ʔ} is preserved between vowels, but not word-initially. Some examples follow; note that in \REF{gl-argentineboa} the initial syllable is irregularly lost in Wichí (provided that the Wichí datum belongs to the cognate set in question).

\begin{exe}
    \ex \algarrobof
    \ex \dog
    \ex \woman
    \ex \iguana
    \ex \rat
    \ex \jararaca
    \ex \wildhoney
    \ex \snakeatuj
    \ex \peccary
    \ex \maguari
    \ex \mistolf
    \ex \mistolt
    \ex \chaguara
    \ex \argentineboa \label{gl-argentineboa}
    \ex \wildbean
    \ex \waspaniti
    \ex \widower
    \ex \wildpepper
    \ex \parrot
    \ex \firei
    \ex \bro
    \ex \pigeon
    \ex \doveula
    \ex \lessergrison
\end{exe}

The very same correspondence is observed in etymologies with a limited distribution (Maká and Nivaĉle, Chorote and Wichí), whose PM~age is thus questionable. The correspondence in \REF{gl-ocelot} seems somewhat irregular.

\begin{exe}
    \ex \ocelot \label{gl-ocelot}
    \ex \paralytic
    \ex \cebil
    \ex \aloja
    \ex \doradocw
    \ex \orphanmn
    \ex \mollef
    \ex \queenpalmf
    \ex \eel
\end{exe}

In \REF{hiatusappr}, \sound{PM}{*ʔ} occurs between vowels at a root–suffix boundary. This was preserved in Maká; note that intervocalic glottal stops must be flanked by identical vowels in that language due to translaryngeal harmony \citep[62]{AG94}. Nivaĉle has eliminated the second vowel altogether. In Chorote and Wichí, one finds hiatus-filling approximants in place of \sound{PM}{*ʔ}, as in \wordng{Ijw}{[ti]pɔ́ji}, \wordng{Mj}{[ta]pɔ́we}, \wordng{PW}{*[t]ˈpójeχ} (since different hiatus-filling approximants are found in different Chorote varieties, we assume that the glide insertion occurred there independently and reconstruct a vowel sequence for  Proto-Chorote).

\begin{exe}
    \ex \full\label{hiatusappr}
\end{exe}

\sound{PM}{*ʔ} is clearly contrastive at the left edge of stems which are compatible with prefixes. After a prefix that ends in a consonant, the stem-initial glottal stop surfaces as glottalization on that consonant, something that does not occur in vowel-initial stems. For example, underlying vowel-initial stems such as \word{PM}{*\mbox{-}ǻseʔ}{daughter} and \intxt{*ʔ}\mbox{-}initial stems such as \wordnl{*\mbox{-}ʔúɬu}{urine} behave differently when they combine with the third-person prefix \intxt{*ɬ\mbox{-}}: compare \word{PM}{*ɬǻseʔ}{her/his daughter} and \wordnl{*ɬ’úɬu}{her/his urine}. The distinction is systematically maintained in all contemporary Mataguayan languages.

\begin{exe}
    \ex \ask
    \ex \knee
    \ex \femalebreast
    \ex \hurt
    \ex \stepv
    \ex \skin
    \ex \eatvi
    \ex \meat
    \ex \teach
    \ex \othern
    \ex \juice
    \ex \dryout
    \ex \good
    \ex \extinguished
    \ex \ripe
    \ex \chest
    \ex \urinate
    \ex \urine
\end{exe}

The very same correspondence is observed in etymologies with a limited distribution (Maká and Nivaĉle, Chorote and Wichí), whose PM~age is thus questionable.

\begin{exe}
    \ex \fatalha
    \ex \cord
    \ex \yawn
    \ex \ashamedcw
    \ex \diecw
    \ex \heartcw
\end{exe}

In \REF{glottalshout}, the correspondence is irregular: Nivaĉle and Chorote point to an underlying vowel-initial stem, whereas Wichí and Maká point to a \intxt{*ʔ}\mbox{-}initial stem. Furthermore, the Maká verb is semantically off, and may turn out to be noncognate.

\begin{exe}
    \ex \shout\label{glottalshout}
\end{exe}

\sound{PM}{*ʔ} is also contrastive in the word-final position, where it is best preserved in Maká. In Nivaĉle and Wichí, it is usually preserved, but it is deleted in posttonic syllables in both languages (see \sectref{ni-deglottalization-codas}, \sectref{wi-posttonic-deglottalization}). Note that the loss of word-final \sound{PM}{*ʔ} occurred independently in Nivaĉle and Wichí, given that in the latter language it was fed by the accentual retraction process (\sectref{wi-prosody}). In Chorote, \sound{PM}{*ʔ} was preserved, but the erstwhile contrast between its presence and absence was lost because \intxt{*ʔ} was inserted at the end of \emph{all} words that ended in a vowel or in \sound{PCh}{*j} (in fact, \citnp{JC14b} synchronically analyzes all word-final instances of [ʔ] as automatic in the Iyojwa’aja’ variety of Chorote); see \sectref{ch-saltillo} for details.

\begin{exe}
    \ex \fruit
    \ex \mouth
    \ex \daughter
    \ex \coal
    \ex \soninlaw
    \ex \welln
    \ex \drinkv
    \ex \treen
    \ex \vulture
    \ex \arrowkaxe
    \ex \feminine
    \ex \elderbro
    \ex \eldersis
    \ex \youngersis
    \ex \pet
    \ex \louse
    \ex \mucus
    \ex \sleepiness
    \ex \cavy
    \ex \seed
    \ex \fatpe
    \ex \beard
    \ex \leg
    \ex \dinlaw
    \ex \eyelash
    \ex \eye
    \ex \face
    \ex \eyebrow
    \ex \rheum
    \ex \woodpecker
    \ex \demwa
    \ex \bromelia
    \ex \expert
    \ex \rib
    \ex \bat
    \ex \vrbpl
    \ex \maguari
    \ex \mosquito
    \ex \pigeon
\end{exe}

The very same correspondence is observed in etymologies with a limited distribution (Maká and Nivaĉle, Chorote and Wichí), whose PM~age is thus questionable.

\begin{exe}
    \ex \locustmn
    \ex \hunger
    \ex \torn
    \ex \pacu
    \ex \heartmn
    \ex \yellowv
    \ex \durmili
    \ex \dirt
\end{exe}

In some cases, word-final glottal stops in Maká and Nivaĉle appear not to reconstruct to Proto-Mataguayan, as evidenced by the Lower Bermejeño Wichí cognates (where no glottal stop is found). We suggest that Maká and Nivaĉle underwent \intxt{ʔ}\mbox{-}epenthesis in roots of the shape \intxt{(C)V} (see \sectref{mk-glottal-insertion}, \sectref{ni-glottal-insertion}).

\begin{exe}
    \ex \thorne
    \ex \sleep
    \ex \penis
    \ex \worm
    \ex \belly
    \ex \price
    \ex \juice
\end{exe}

\subsection{\sound{PM}{*ɸ}}\label{proto-f}
\sound{PM}{*ɸ} is preserved as a bilabial fricative only in Nivaĉle, at least in the Chishamnee Lhavos dialect.\footnote{\citet[29, 81]{LC20} state emphatically that this consonant is articulated as bilabial and not labiodental, at least in their data. In \cits{AnG15} work, [ɸ] is said to be an allophone of /f/. An anonymous reviewer reports that the labiodental fricative is now the most extended realization in Nivaĉle, according to their field data.} In other languages, its reflexes are \sound{Mk}{f}, \sound{PCh}{*hw} (in onsets) or \intxt{*ʍ} (in codas), and \sound{PW}{*xʷ}. Note the irregular reflexes in Wichí in two examples: \intxt{*w} in \REF{f-flyv} and \intxt{*p} in \REF{f-suckb} (unless it turns out to be the regular outcome of the preglottalized coda \intxt{*ˀɸ}, see \sectref{glott-codas}).

\begin{exe}
    \ex \wing
    \ex \companion
    \ex \coal
    \ex \disease
    \ex \shoulder
    \ex \shoulderblade
    \ex \firef
    \ex \centipede
    \ex \cutdown
    \ex \algarrobof
    \ex \algarrobot
    \ex \rightn
    \ex \tell
    \ex \sisinlaw
    \ex \soninlaw
    \ex \fieldn
    \ex \flyv \label{f-flyv}
    \ex \mortar
    \ex \rootn
    \ex \notafraid
    \ex \coldweather
    \ex \hidev
    \ex \crab
    \ex \palmg
    \ex \leech
    \ex \fart
    \ex \neighbor
    \ex \spouse
    \ex \acquainted
    \ex \spend
    \ex \suckb \label{f-suckb}
    \ex \tsofa
    \ex \tsofatajf
    \ex \tsofatajt
    \ex \woman
    \ex \pigeon
\end{exe}

The very same correspondence is observed in etymologies with a limited distribution (Maká and Nivaĉle, Chorote and Wichí), whose PM~age is thus questionable.

\begin{exe}
    \ex \spin
    \ex \locustmn
    \ex \spitcw
    \ex \ameiva
    \ex \pocote
    \ex \dreamv
    \ex \dreamn
    \ex \tobacco
    \ex \throwpush
    \ex \durmili
    \ex \chachalaca
    \ex \tireddie
    \ex \ashamedcw
    \ex \orphanmn
\end{exe}

\subsection{\sound{PM}{*ɬ}}\label{proto-lh}
\sound{PM}{*ɬ} is preserved as \intxt{ɬ} in all daughter languages except Chorote, where it unpacks to \sound{PCh}{*hl} in onsets (its allophone in codas is represented as \sound{PCh}{*ɬ} in this book, with the realizations in the contemporary varieties including [ll̥] alongside [ɬ]). 

\begin{exe}
    \ex \burn
    \ex \mortar
    \ex \breath
    \ex \redquebracho
    \ex \answer
    \ex \flu
    \ex \demlh
    \ex \louse
    \ex \defecate
    \ex \lightfire
    \ex \whitesnail
    \ex \firewoodlhet
    \ex \thread
    \ex \yicalhuk
    \ex \daylhuma
    \ex \girl
    \ex \dayworld
    \ex \rain
    \ex \sprout
    \ex \carrysh
    \ex \spinsew
    \ex \tired
    \ex \rhea
    \ex \rib
    \ex \climb
    \ex \ask
    \ex \iguana
    \ex \othern
    \ex \urinate
    \ex \urine
    \ex \puma
\end{exe}

The very same correspondence is observed in etymologies with a limited distribution (Maká and Nivaĉle, Chorote and Wichí), whose PM~age is thus questionable.

\begin{exe}
    \ex \burnalh
    \ex \spin
    \ex \dreamv
    \ex \dreamn
    \ex \fry
    \ex \heartmn
    \ex \silkfloss
    \ex \onemn
    \ex \fatalha
    \ex \ashamedcw
\end{exe}

\subsection{\sound{PM}{*s}}\label{proto-s}
\sound{PM}{*s} is a stable phoneme: it is preserved in all daughter languages as \intxt{s}. Note the irregular loss of \sound{PM}{*s} in Wichí in \REF{s-stinger} and in Nivaĉle in \REF{s-blackalgarrobof}--\REF{s-blackalgarrobot}.

\begin{exe}
    \ex \stinger \label{s-stinger}
    \ex \son
    \ex \daughter
    \ex \palmg
    \ex \leech
    \ex \wax
    \ex \sandisaj
    \ex \tail
    \ex \dividev
    \ex \lip
    \ex \beard
    \ex \parakeet
    \ex \soul
    \ex \vein
    \ex \spank
    \ex \mesh
    \ex \whitequebracho
    \ex \kingvulture
    \ex \vagina
    \ex \anteater
    \ex \likelove
    \ex \invite
    \ex \eyebrow
    \ex \snake
    \ex \yellowlegs
    \ex \woodpecker
    \ex \blackalgarrobof \label{s-blackalgarrobof}
    \ex \blackalgarrobot \label{s-blackalgarrobot}
    \ex \butterfly
    \ex \stepv
    \ex \good
    \ex \widower
    \ex \meat
\end{exe}

The very same correspondence is observed in etymologies with a limited distribution (Maká and Nivaĉle, Chorote and Wichí), whose PM~age is thus questionable.

\begin{exe}
    \ex \standv
    \ex \leafmn
    \ex \cicada
    \ex \siyaj
    \ex \durmili
    \ex \dovesipup
    \ex \cardon
    \ex \chachalaca
    \ex \skymn
    \ex \cloudmn
    \ex \cardinal
    \ex \temperance
    \ex \aloja
\end{exe}

\subsection{\sound{PM}{*x}}\label{proto-j}
\sound{PM}{*x} is preserved as a velar fricative in Maká, whereas in other languages it has suffered a split or a merger. In Nivaĉle, it palatalizes to \intxt{ʃ} before or after non-back vowels (\sound{PM}{*i}, \intxt{*e}, \intxt{*ä}, \intxt{*a} > \sound{Ni}{i}, \intxt{e}, \intxt{a}), except when preceded or followed by a back vowel, possibly with an intervening [+grave] consonant (see \sectref{ni-vel-uv} for more details). In Chorote, it yields \sound{PCh}{*h} except when it follows the vowel \intxt{*u}, in which case it is reflected as PCh~*/hw/. In Wichí, \sound{PM}{*x} always changes to \sound{PW}{*h} in the onset position, whereas in codas it is reflected as \sound{PW}{*χ} (except after the vowel \intxt{*u}, in which case it yields \sound{PW}{*xʷ}). The following examples show the development of \sound{PM}{*x} in the onset position, where it is reflected as \sound{Mk}{x}, \sound{Ni}{x} or \intxt{ʃ}, \sound{PCh}{*h}, \sound{PW}{*h}. The Chorote and Wichí reflexes in \REF{x-tuscaf}--\REF{x-tuscag} may turn out to be regular if one recognizes the regularity of deletion of \intxt{*x} in word-initial unaccented syllables.

\begin{exe}
    \ex \mouth
    \ex \truev
    \ex \arrowkaxe
    \ex \price
    \ex \night
    \ex \egg
    \ex \headn
    \ex \jelayuk
    \ex \vrbpl
    \ex \recipient
    \ex \tuscaf \label{x-tuscaf}
    \ex \tuscat
    \ex \tuscag \label{x-tuscag}
    \ex \grass
    \ex \maguari
\end{exe}

The very same correspondence is observed in etymologies with a limited distribution (Maká and Nivaĉle, Chorote and Wichí), whose PM~age is thus questionable.

\begin{exe}
    \ex \ameiva
    \ex \ocelot
    \ex \dirt
    \ex \palocruzmn
\end{exe}

The following examples show the development of \sound{PM}{*x} in the coda position, where it is reflected as \sound{Mk}{x}; \sound{Ni}{x} or \intxt{ʃ}; \sound{PCh}{*h}, but \intxt{*hw} after \intxt{*u} \REF{eatvt-xw}; \sound{PW}{*χ}, but \intxt{*xʷ} after \intxt{*u}, as in \REF{finger-xw}, \REF{eatvt-xw}. Note that in \REF{nose-x} the suffixless form has not been preserved in Chorote and Wichí, and the velar fricative evolves there as detailed in \sectref{cx}.

\begin{exe}
    \ex \bite
    \ex \cutdown
    \ex \rightn
    \ex \fieldn
    \ex \finger \label{finger-xw}
    \ex \youngerbro
    \ex \wash
    \ex \bow
    \ex \languageword
    \ex \nose \label{nose-x}
    \ex \smelln
    \ex \pathn
    \ex \thunder
    \ex \abdcavity
    \ex \carrysh
    \ex \dig
    \ex \eatvt \label{eatvt-xw}
    \ex \aunt
    \ex \burrow
    \ex \skin
\end{exe}

The very same correspondence is observed in etymologies with a limited distribution (Maká and Nivaĉle, Chorote and Wichí), whose PM~age is thus questionable.

\begin{exe}
    \ex \foodmn
    \ex \hole
    \ex \leafmn
    \ex \saymn
    \ex \stagnant
\end{exe}

\subsection{\sound{PM}{*χ}}\label{proto-jj}
\sound{PM}{*χ} occurs predominantly in the coda position, though it can resyllabify as an onset if a \intxt{*χ}\mbox{-}final stem takes a vowel-initial suffix, as in \REF{deep-resyll}, \REF{far-resyll}, \REF{fullriver-resyll}; it also occurs in consonant clusters. It is consistently preserved as a uvular fricative only in Maká, where it still contrasts with the velar fricative \intxt{x}. In other languages, its reflexes are \sound{Ni}{x}, \sound{PCh}{*h} (but \intxt{*hw} in onsets after a rounded vowel), and \sound{PW}{*h} (in onsets), \intxt{*χ} (in codas), or \intxt{*xʷ} (in onsets or codas after a rounded vowel). Note that \sound{PW}{*χ} does not contrast with a velar fricative, unlike in Maká.

\begin{exe}
    \ex \fatv
    \ex \najendup
    \ex \centipede
    \ex \crab
    \ex \north
    \ex \suncho
    \ex \sandisaj
    \ex \takeaway
    \ex \dividev
    \ex \barnowl
    \ex \oldn
    \ex \manysg
    \ex \quick
    \ex \jabiru
    \ex \deep \label{deep-resyll}
    \ex \longv
    \ex \anteater
    \ex \pseudo
    \ex \shoot
    \ex \far \label{far-resyll}
    \ex \fullriver \label{fullriver-resyll}
    \ex \tsofatajf
    \ex \tired
    \ex \largefat
    \ex \paloflojof
    \ex \blackalgarrobof
    \ex \rhea
    \ex \night
    \ex \tuscaf
    \ex \caracara
    \ex \jararaca
    \ex \snakeatuj
    \ex \peccary
    \ex \mistolf
    \ex \hurt
    \ex \argentineboa
    \ex \chaguara
    \ex \wildbean
    \ex \widower
    \ex \firei
    \ex \bro
    \ex \puma
\end{exe}

The very same correspondence is observed in etymologies with a limited distribution (Maká and Nivaĉle, Chorote and Wichí), whose PM~age is thus questionable.

\begin{exe}
    \ex \ocelot
    \ex \runv
    \ex \siyaj
    \ex \smoke
    \ex \burnvi
    \ex \piranhamn
    \ex \mollef
\end{exe}

As we will see in \sectref{jj-suff}, in some cases stem-final \sound{PM}{*χ} may be deleted or alternate with \sound{PM}{*h}.

\subsection{\sound{PM}{*h}}\label{proto-h}
\sound{PM}{*h} does not occur very frequently in onsets, and it contrasts only marginally with \sound{PM}{*χ} in that position (recall that \sound{PM}{*χ} typically occurs in codas except at root–suffix boundaries). In onsets, it is reflected as \intxt{h} in all daughter languages except Nivaĉle, where \intxt{x} is found (Nivaĉle has no \intxt{h} in its inventory). Word-initially it is apparently reflected as zero in Chorote and Wichí \REF{h-acti}, but in the distal [−visible] [+firsthand] demonstrative it is exceptionally preserved in Chorote as \sound{PCh}{*h} \REF{h-demh}.

\begin{exe}
    \ex \demh \label{h-demh}
    \ex \acti \label{h-acti}
    \ex \welln
    \ex \chaja
\end{exe}

The very same correspondence is observed in an etymology with a limited distribution (Maká and Nivaĉle), whose PM~age is thus questionable.

\begin{exe}
    \ex \coati
\end{exe}

By contrast, word-finally in codas \sound{PM}{*h} clearly contrasts with \sound{PM}{*χ}. It is lost altogether in Maká and Nivaĉle in that position, but is usually preserved as \intxt{*h} in Proto-Chorote and Proto-Wichí (in the only example of a monosyllabic root, given in \REF{h-goimp}, it is reflected as a so-called \conc{unstable \intxt{ʰ}} in Chorote). Note that all contemporary Chorote and Wichí dialects except ’Weenhayek have lost word-final \intxt{*h} in some or all environments, but \intxt{*h} is clearly reconstructible to Proto-Chorote and Proto-Wichí based on evidence internal to Chorote and Wichí, respectively.

\begin{exe}
    \ex \companion \label{h-companion}
    \ex \neighbor \label{h-neighbor}
    \ex \spouse \label{h-spouse}
    \ex \snail
    \ex \goimp \label{h-goimp}
    \ex \dog
    \ex \tapeti
    \ex \moon
    \ex \waspaniti
    \ex \doveula
    \ex \lessergrison
\end{exe}

The very same correspondence is observed in etymologies with a limited distribution (Chorote and Wichí), whose PM~age is thus questionable.

\begin{exe}
    \ex \lizard
    \ex \frog
    \ex \fox
\end{exe}

An additional quirk comes from the fact that in Wichí word-final \intxt{*h} is lost if the onset of the syllable in question is a glottalized stop or affricate (as well as in one unclear exception shown in \REF{rat-h-loss}, where the loss of \intxt{*h} may have something to do with the sequence \intxt{*\mbox{-}mʔ\mbox{-}}). In this case only Chorote, of all Mataguayan languages, preserves any trace of \sound{PM}{*h}.

\begin{exe}
    \ex \monkparakeet
    \ex \hornero
    \ex \whiteegret
    \ex \rat \label{rat-h-loss}
\end{exe}

\subsection{\sound{PM}{*w}}\label{proto-w}
\sound{PM}{*w} is preserved as a distinct segment in all Mataguayan languages. In Nivaĉle, its reflex is often articulated as bilabial ([β]), but [w] is also a possible realization (see \sectref{ni-v} for details); in this book we consistently represent the phoneme in question as \sound{Ni}{β}. The distribution of \sound{PM}{*w} is defective: it is the only consonant that is hardly ever reconstructed in the coda position in Proto-Mataguayan.\footnote{The possible exceptions to this generalization include \word{PM}{*[t]k’áw\mbox{-}\APPL}{to hold in one’s arms, to hug} and \wordnl{*\mbox{-}åˀw\mbox{-}\APPL}{to be}, but these are typically followed by applicative suffixes. Word-internally, clusters such as \intxt{*wts’} or \intxt{*ˀwt} are securely reconstructed in Proto-Mataguayan, but it is not clear whether they were necessarily heterosyllabic.} Some examples follow; note the irregular reflexes in Nivaĉle (in dialects other than Chishamnee Lhavos) and Wichí in \REF{w-bromelia} as well as the irregular loss of \sound{PM}{*w} in Maká in \REF{w-blackalgarrobof}--\REF{w-blackalgarrobot}.

\begin{exe}
    \ex \flower
    \ex \wildmanioc
    \ex \abdcavity
    \ex \river
    \ex \termitehouse
    \ex \demwa
    \ex \guayacan
    \ex \paloflojof
    \ex \paloflojot
    \ex \badmood
    \ex \allrcpr
    \ex \burrow
    \ex \bromelia \label{w-bromelia}
    \ex \dov
    \ex \worm
    \ex \throwv
    \ex \cardinal
    \ex \blackalgarrobof \label{w-blackalgarrobof}
    \ex \blackalgarrobot \label{w-blackalgarrobot}
    \ex \expert
    \ex \largefat
    \ex \spring
    \ex \moon
    \ex \peccary
    \ex \puma
\end{exe}

The very same correspondence is observed in etymologies with a limited distribution (Maká and Nivaĉle, Chorote and Wichí), whose PM~age is thus questionable.

\begin{exe}
    \ex \hole
    \ex \hookmn
    \ex \limpkin
    \ex \ashamedmn
    \ex \tireddie
    \ex \wolf
    \ex \piranhamn
    \ex \skymn
    \ex \cloudmn
    \ex \balawasp
    \ex \burnvt
    \ex \wut
    \ex \fox
\end{exe}

\subsection{\sound{PM}{*l}}\label{proto-l}
\sound{PM}{*l} is preserved as a distinct segment in all Mataguayan languages except Nivaĉle, where it yields \intxt{k͡l} (\sectref{ni-kl}) or -- in the coda position -- \intxt{k} (\sectref{ni-kl-k}). In Wichí, it changes to \sound{PW}{*lʰ} word-finally (\sectref{wi-l-lh}). Some examples follow; note the irregular glottalized reflexes in Chorote in \REF{l-many} and \REF{l-leg}.

\begin{exe}
    \ex \returnth
    \ex \pll
    \ex \inhabitant
    \ex \tell
    \ex \sisinlaw
    \ex \soninlaw
    \ex \elderbro
    \ex \sunn
    \ex \withstand
    \ex \killv
    \ex \snail
    \ex \flee
    \ex \pet
    \ex \chaniarf
    \ex \chaniart
    \ex \spank
    \ex \offspring
    \ex \wash
    \ex \whitelim
    \ex \ashes
    \ex \winter
    \ex \iscayante
    \ex \many \label{l-many}
    \ex \bow
    \ex \returnh
    \ex \leg \label{l-leg}
    \ex \wildcat
    \ex \anteater
    \ex \acquainted
    \ex \blind
    \ex \rheum
    \ex \walk
    \ex \jelayuk
    \ex \shadow
    \ex \chaguara
    \ex \widower
    \ex \parrot
    \ex \doveula
    \ex \lessergrison
\end{exe}

The very same correspondence is observed in etymologies with a limited distribution (Maká and Nivaĉle, Chorote and Wichí), whose PM~age is thus questionable.

\begin{exe}
    \ex \brightness
    \ex \pocote
    \ex \fry
    \ex \locustcw
    \ex \cheek
    \ex \smooth
    \ex \feel
    \ex \willow
    \ex \agile
    \ex \skycloud
    \ex \gutscw
    \ex \cicada
    \ex \leafhaircw
    \ex \dirt
    \ex \orphanmn
    \ex \diecw
    \ex \hiccup
    \ex \heartcw
\end{exe}

\subsection{\sound{PM}{*j}}\label{proto-y}
\sound{PM}{*j} is a stable phoneme: it is preserved in all daughter languages as \intxt{j} (except in the sequence \sound{PM}{*ji}, on which see below). In \REF{y-algarrobof} and \REF{y-rain}, Wichí shows an irregular reflex (\sound{PW}{*jʰ}) word-finally, possibly due to analogy with the plural suffix \wordng{PW}{*\mbox{-}(á)jʰ}. Also note the irregular glottalized reflex in Chorote in \REF{y-drinkv}.

\begin{exe}
    \ex \honeycomb
    \ex \gofirst
    \ex \yicaay
    \ex \namen
    \ex \algarrobof \label{y-algarrobof}
    \ex \rightn
    \ex \notafraid
    \ex \breath
    \ex \drinkv \label{y-drinkv}
    \ex \dew
    \ex \wax
    \ex \leniosa
    \ex \hand
    \ex \sunn
    \ex \coldn
    \ex \barnowl
    \ex \nightmonkey
    \ex \savannahhawk
    \ex \rope
    \ex \pathn
    \ex \bitter
    \ex \hear
    \ex \rain \label{y-rain}
    \ex \costume
    \ex \wildcat
    \ex \shoot
    \ex \movev
    \ex \bromelia
    \ex \bat
    \ex \wildhoney
    \ex \mistolf
    \ex \mistolt
    \ex \wildbean
    \ex \mosquito
\end{exe}

The very same correspondence is observed in etymologies with a limited distribution (Maká and Nivaĉle, Chorote and Wichí), whose PM~age is thus questionable.

\begin{exe}
    \ex \spillmn
    \ex \ameiva
    \ex \grandchild
    \ex \spitmn
    \ex \siyaj
    \ex \weave
    \ex \soundv
    \ex \spillcw
    \ex \clothes
    \ex \mollef
\end{exe}

In the sequence \sound{PM}{*ji}, all languages show some tendency for eliminating the palatal approximant. It is most consistently preserved in Nivaĉle, though even there \intxt{ji} varies with \intxt{i} depending on the dialect and on the speech rate (see \sectref{ni-yi-i}). In Maká, it yields either \intxt{ji} or \intxt{i}, with no clear distribution. In Chorote, it is consistently reflected as \sound{PCh}{*ʔi} (or as \intxt{*ˀja} before \intxt{*q}). In Wichí, it is usually reflected as \sound{PW}{*ʔi} (or \sound{PW}{*hi} before a glottalized consonant due to a general glottal dissimilation rule, \sectref{wi-glottal-dissim}), but is retained as \sound{PW}{*ji} when followed by a uvular or glottal consonant, as evident from alternations in the third-person prefix \citep[241–242]{VN14}.

\begin{exe}
    \ex \dew
    \ex \wax
    \ex \water
    \ex \mancw
    \ex \hunger
    \ex \truev
    \ex \ocelot
\end{exe}

When followed by a glottalized consonant and a low vowel (\sound{PM}{*a} or \intxt{*å}, but not \intxt{*ä}), \sound{PM}{*ji} evolved to \intxt{*ʔi} > \intxt{*ʔa} in Chorote, and to \intxt{*ʔi} > \intxt{*ʔa} > \intxt{*ha} in Wichí.

\begin{exe}
    \ex \jaguar
    \ex \treen
    \ex \vulture
\end{exe}

\subsection{\sound{PM}{*m}}\label{proto-m}
\sound{PM}{*m} is a stable phoneme: it is preserved in all daughter languages as \intxt{m}. Note the irregular loss of \sound{PM}{*m} in Wichí in \sectref{m-up}.

\begin{exe}
    \ex \arrive
    \ex \wordamet
    \ex \grabwork
    \ex \whitelim
    \ex \defecate
    \ex \daylhuma
    \ex \interr
    \ex \sleep
    \ex \goimp
    \ex \powder
    \ex \hither
    \ex \otter
    \ex \savannahhawk
    \ex \feces
    \ex \up \label{m-up}
    \ex \dinlaw
    \ex \swallow
    \ex \rat
    \ex \dryout
    \ex \extinguished
\end{exe}

The very same correspondence is observed in etymologies with a limited distribution (Maká and Nivaĉle, Chorote and Wichí), whose PM~age is thus questionable.

\begin{exe}
    \ex \throwpush
    \ex \coati
    \ex \runv
    \ex \smooth
    \ex \agile
    \ex \drum
    \ex \bilecw
    \ex \silkfloss
    \ex \fox
\end{exe}

\subsection{\sound{PM}{*n}}\label{proto-n}
\sound{PM}{*n} is a stable phoneme: it is preserved in all daughter languages as \intxt{n}, except that in Wichí the word-final sequence \intxt{*\mbox{-}nV} changes to \intxt{*\mbox{-}ˀnVh}, as in \REF{n-chaniarf}, \REF{n-whitequebracho}, \REF{n-kingvulture}, \REF{n-redbrocket}, \REF{n-balawasp} (see \sectref{wi-nv-nvh}). An irregular glottalized reflex of \sound{PM}{*n} in other environments is occasionally found in Chorote, as in \REF{n-water} and \REF{n-healthy}, and Wichí \REF{n-stinger}.

\begin{exe}
    \ex \gofirst
    \ex \shout
    \ex \stinger \label{n-stinger}
    \ex \putv
    \ex \killbird
    \ex \crab
    \ex \north
    \ex \suncho
    \ex \water \label{n-water}
    \ex \testicle
    \ex \sendv
    \ex \eatkun
    \ex \stretchout
    \ex \youngerbro
    \ex \thorncutjan
    \ex \killv
    \ex \chaniarf \label{n-chaniarf}
    \ex \chaniart
    \ex \roast
    \ex \nightmonkey
    \ex \lightfire
    \ex \demn
    \ex \bathe
    \ex \nose
    \ex \father
    \ex \wildmanioc
    \ex \rope
    \ex \smelln
    \ex \bonenu
    \ex \smellv
    \ex \dayworld
    \ex \fillv
    \ex \fishwithhook
    \ex \spank
    \ex \whitequebracho \label{n-whitequebracho}
    \ex \kingvulture \label{n-kingvulture}
    \ex \likelove
    \ex \cat
    \ex \thunder
    \ex \hardv
    \ex \duraznillo
    \ex \healthy \label{n-healthy}
    \ex \seev
    \ex \spring
    \ex \night
    \ex \wildbean
    \ex \waspaniti
    \ex \meat
    \ex \teach
    \ex \bro
\end{exe}

The very same correspondence is observed in etymologies with a limited distribution (Maká and Nivaĉle, Chorote and Wichí), whose PM~age is thus questionable.

\begin{exe}
    \ex \spitcw
    \ex \dreamv
    \ex \tobacco
    \ex \kidney
    \ex \obey
    \ex \hookmn
    \ex \majan
    \ex \cook
    \ex \orphancw
    \ex \vertical
    \ex \toad
    \ex \precipice
    \ex \redbrocket \label{n-redbrocket}
    \ex \ashamedmn
    \ex \piranhamn
    \ex \metal
    \ex \balawasp \label{n-balawasp}
    \ex \burnvt
    \ex \saber
    \ex \paralytic
\end{exe}

\subsection{Underdifferentiated consonants}\label{proto-underdif}

Since some pairs of PM~consonants suffered similar mergers in the daughter languages, it is at times impossible to ascertain whether a given cognate set contained one or another consonant in Proto-Mataguayan. For example, the fricatives \sound{PM}{*x}, \intxt{*χ}, and \intxt{*h} are most consistently distinguished in Maká, and when a Maká cognate is absent two or three alternatives must be reconstructed. We use the symbols \intxt{*X₁₂} for ``\sound{PM}{*x} or \intxt{*χ}''; \intxt{*X₁₃} for ``\sound{PM}{*x} or \intxt{*h}''; \intxt{*X₂₃} for ``\sound{PM}{*χ} or \intxt{*h}'', and \intxt{*X} for ``\sound{PM}{*x}, \intxt{*χ}, or \intxt{*h}''.

The following examples illustrate the reconstruction of \sound{PM}{*X₁₂} (for ``\intxt{*x} or \intxt{*χ}'') in codas. Note that \sound{PM}{*x} and \intxt{*χ} merge in codas in Nivaĉle, Chorote, and Wichí (except in palatalizing environments in Nivaĉle, after the vowel \intxt{*u} in Chorote, and after the vowel \intxt{*o} in Wichí).

\begin{exe}
    \ex \pocote
    \ex \chaguark
    \ex \sweat
    \ex \majan
    \ex \orphancw
    \ex \tongue
    \ex \toad
    \ex \precipice
    \ex \metal
    \ex \jaguar
\end{exe}

The following examples illustrate the reconstruction of \sound{PM}{*X₁₃} (for ``\intxt{*x} or \intxt{*h}'') in onsets. Note that \sound{PM}{*x} and \intxt{*h} merge in onsets in Chorote, Wichí, and -- in non-palatalizing environments -- in Nivaĉle.

\begin{exe}
    \ex \nightmonkey
    \ex \saber
    \ex \gov
    \ex \palosanto
    \ex \nightnw
    \ex \sandyplace
    \ex \firewoodhuk
    \ex \temperance
    \ex \pushv
    \ex \caracara
    \ex \mistolf
    \ex \mistolt
\end{exe}

The following examples illustrate the reconstruction of \sound{PM}{*X₂₃} (for ``\intxt{*χ} or \intxt{*h}'').

\begin{exe}
    \ex \nightmonkey
    \ex \girl
    \ex \earth
\end{exe}

Finally, in some cases it is impossible to choose between \sound{PM}{*k} and \intxt{*q}. This happens when the consonant in question occurs in the coda position following \sound{PM}{*å}, and diagnostic cognates in Maká and Wichí are lacking.

\begin{exe}
    \ex \butterfly
\end{exe}

\section{Glottalized onsets}\label{glott-ons}
All Mataguayan languages have a series of glottalized stops, and at least Chorote and Wichí have a series of glottalized sonorants \citep{AnG-VN-21}. These are usually granted phonemic status in synchronic descriptions (for a dissenting view, see \citnp{KC94}; the issue is further discussed in \sectref{glott-status}), and their occurrence is restricted to onsets. In addition, Nivaĉle has sequences of the type ``\intxt{ʔ} + sonorant'' at the surface, which are usually analyzed as consonant clusters; however, these sequences display phonotactic properties typical of phonemes, such as the possibility to occur at the left edge of a morpheme, as in /\mbox{-}ʔβan/\gloss{to see}, or after a consonant, as in /\mbox{-}sʔβun/\gloss{to like, to love} \citep{AnG21}. In our notation, we symbolize such sequences as preglottalized sonorants (e.g.~\word{Ni}{\mbox{-}ˀβan}{to see}, \wordnl{\mbox{-}sˀβun}{to like, to love}).

Across the Mataguayan language family, glottalized stops are typically articulated as ejective plosives or affricates; for acoustic studies, see \citet{AnG-GE-23} for Nivaĉle and \citet[79–82]{VN14} for Lower Bermejeño Wichí. Chorote is an exception, where glottalized stops surface as preglottalized after stressed syllables \citep[80–81]{JC14b}. In addition, in some Wichí dialects glottalized stops have been described as implosive (\sectref{wi-gl-cons}). By contrast, glottalized sonorants typically surface as preglottalized in the onset position in the contemporary Mataguayan languages. This is in agreement with the cross-linguistic timing tendency identified by \citet[394–396]{MGPL01}, among others, whereby prevocalic glottalized sonorants tend to realize their nonmodal phonation early in the consonant in order to enhance the acoustic cues associated with the consonant-to-vowel transition.

For Proto-Mataguayan, we reconstruct a series of glottalized stops (\sound{PM}{*p’}, \intxt{*t’}, \intxt{*ts’}, \intxt{*k’}, \intxt{*q’}) and a series of glottalized sonorants (\sound{PM}{*ˀw}, \intxt{*ˀl}, \intxt{*ˀj}, \intxt{*ˀm}, \intxt{*ˀn}), in addition to a series of glottalized fricatives (at least \sound{PM}{*ɸ’}, \intxt{*ɬ’}, \intxt{*s’}). As we will see, there is ample evidence that some of these segments result from a combination of a plain (non-glottalized) consonant and a glottal stop.

\subsection{Glottalized stops}\label{glott-obstr}
Glottalized stops are preserved in all daughter languages, where their reflexes are typically realized as ejective (less frequently as preglottalized or implosive) stops. Other than for the [constricted glottis] feature, they evolve just like their plain counterparts. In just one cognate set, \sound{Mk}{q} shows up instead of the expected \intxt{*k’} \REF{gl-arrowkaxe}. When two consecutive syllables have glottalized stops as their onsets, Chorote and Wichí deglottalize the onset of the first syllable, as in \REF{gl-monkparakeet} and – with further irregularities regarding the place of articulation – \REF{gl-hornero}.\footnote{We owe this observation to an anonymous reviewer, who questioned our earlier attempt to account for this sound correspondence by positing irregular sound changes.}

\begin{exe}
    \ex \vulture
    \ex \monkparakeet \label{gl-monkparakeet}
    \ex \armadillo
    \ex \arrowkaxe \label{gl-arrowkaxe}
    \ex \spouse
    \ex \stretchout
    \ex \dividev
    \ex \youngerbro
    \ex \youngersis
    \ex \bottomn
    \ex \hornclub
    \ex \coldn
    \ex \thorncutjan
    \ex \oldn
    \ex \newadj
    \ex \cover
    \ex \lid
    \ex \parakeet
    \ex \soul
    \ex \utensil
    \ex \tears
    \ex \movev
    \ex \rheum
    \ex \woodpecker
    \ex \aunt
    \ex \hardv
    \ex \hornero \label{gl-hornero}
    \ex \guayacan
    \ex \egg
    \ex \widower
\end{exe}

The very same correspondence is observed in etymologies with a limited distribution (Maká and Nivaĉle, Chorote and Wichí), whose PM~age is thus questionable.

\begin{exe}
    \ex \cheek
    \ex \obey
    \ex \spitmn
    \ex \earcw
    \ex \pacu
    \ex \heel
    \ex \tongue
    \ex \soundv
    \ex \saymn
    \ex \cardinal
    \ex \aloja
    \ex \yawn
    \ex \hiccup
\end{exe}

\subsection{Glottalized sonorants}\label{glott-son}
Glottalized sonorants are best preserved in Chorote and Wichí; in Maká and Nivaĉle, they surface as sequences of the type ``\intxt{ʔ} + sonorant'' (\intxt{ˀC} in our notation) except word-initially, where they merge with the respective plain sonorants.

Some examples follow; note that in Wichí the glottalization irregularly migrated to another sonorant in \REF{'w-rhea} and disappeared completely in \REF{'w-argentineboa} (if the word belongs to the cognate set in question at all).

\begin{exe}
    \ex \flyv
    \ex \coldweather
    \ex \jaguar
    \ex \treen
    \ex \nightmonkey
    \ex \languageword
    \ex \defect
    \ex \zorzal
    \ex \smellv
    \ex \demnn
    \ex \dayworld
    \ex \pathn
    \ex \hear
    \ex \likelove
    \ex \healthy
    \ex \rhea \label{'w-rhea}
    \ex \walk
    \ex \seev
    \ex \placen
    \ex \rib
    \ex \neck
    \ex \blood
    \ex \butterfly
    \ex \climb
    \ex \vrbpl
    \ex \argentineboa \label{'w-argentineboa}
\end{exe}

The very same correspondence is observed in etymologies with a limited distribution (Maká and Nivaĉle, Chorote and Wichí), whose PM~age is thus questionable.

\begin{exe}
    \ex \lizard
    \ex \heartmn
    \ex \frog
    \ex \stagnant
    \ex \onemn
    \ex \leafhaircw
    \ex \clothes
\end{exe}

\subsection{Glottalized fricatives}\label{glott-fric}
Synchronically, phonemic glottalized fricatives are not attested in any Mataguayan language. In Maká, \citet[46, 67--68]{AG94} documents sequences of a fricative and a glottal stop, of which at least \sound{Mk}{fʔ} and \intxt{sʔ} may occur within a morpheme: \wordnl{lefʔef}{ant}, \wordnl{sʔotot}{tailless}. Other possible combinations are \sound{Mk}{ɬʔ}, which occurs at morpheme boundaries only (as in \wordnl{ɬ-ʔiʔ}{its liquid, its juice}), and the exceedingly rare \sound{Mk}{xʔ}. At least \sound{Mk}{fʔ} and \intxt{sʔ} correspond to glottalized stops \intxt{p’} and \intxt{ts’}, respectively, in other Mataguayan languages; in this book we tentatively treat them as single segments and transcribe them as \sound{Mk}{f’}, \intxt{s’} in our notation. We suggest that they go back to \sound{PM}{*ɸ’}, \intxt{*s’} (possibly articulated as ejective fricatives), which remained fricatives in Maká but merged with \sound{PM}{*p’}, \intxt{*ts’} as \intxt{(*)p’}, \intxt{(*)ts’} in all other languages.

\begin{exe}
    \ex \poor
    \ex \foot
    \ex \arrowfok
    \ex \samsam
\end{exe}

There is extra evidence that clearly shows that other Mataguayan languages (that is, Nivaĉle, Chorote, and Wichí) have eliminated glottalized fricatives by means of converting them to homorganic glottalized stops. In addition to the sound changes \sound{PM}{*ɸ’}~>~\intxt{(*)p’}, \sound{PM}{*s’}~>~\intxt{(*)ts’}, these languages also underwent the sound change \sound{PM}{*ɬ’}~>~\intxt{(*)t’}. Judging by the non-existence of words with a tautomorphemic \intxt{ɬ’} in Maká, the occurrence of \sound{PM}{*ɬ’} must have been restricted to morpheme boundaries in the protolanguage, notably when a \intxt{*ʔ}\mbox{-}initial stem combines with the third-person prefix \intxt{*ɬ\mbox{-}}.

\begin{exe}
    \ex \fatalhaitssg
    \ex \cordits
    \ex \femalebreastits
    \ex \skinits
    \ex \meatits
    \ex \juiceits
    \ex \urineits
\end{exe}

The underlying form of the third-person prefix is undoubtedly \sound{PM}{*ɬ\mbox{-}}, as seen in stems that begin with a vowel (or with a consonant other than a glottal stop; see \sectref{syllabic-lh}).

\begin{exe}
    \ex \mouthits
    \ex \flowerits
    \ex \wordametits
    \ex \stingerits   
    \ex \fooditssg
    \ex \sonits
    \ex \daughterits
    \ex \drinknitssg
    \ex \jarits
    \ex \wingitssg
    \ex \yicaayitssg
    \ex \thorneitssg
    \ex \namenitssg
    \ex \inhabitantits
    \ex \resinits
    \ex \penisits
    \ex \seedits
    \ex \nestits
\end{exe}

As a result of the sound change \sound{PM}{*ɬ’}~>~\intxt{(*)t’}, Nivaĉle, Chorote, and Wichí now display a morphophonological rule which converts the underlying sequence /ɬ+ʔ/ or /hl+ʔ/ into \intxt{t’} (rather than \intxt{ɬ’}, as in Maká). The rule is no longer entirely productive in the contemporary languages. In Nivaĉle, the sequence /ɬʔ/ may occur within a morpheme, as in \wordnl{ʃniɬʔå}{lizard\species{Teius teyou}}. In Chorote, a combination of /hl/ and /ʔ/ at the stem--suffix/enclitic boundary results in \intxt{hˀl}, as in Ijw~/táhl+ʔe/ → \wordnl{táh.ˀleʔ}{exits from}, often pronounced with an intrusive echo vowel (see \sectref{ch-f}), i.e., [ˈtahăˀleʔ]. In Wichí, \intxt{ɬ} and \intxt{ʔ} suffer no changes at the morpheme boundaries at least in ’Weenhayek, as in \wordnl{tåɬ\mbox{-}ˈʔúxʷ$=$eh}{comes from the riverside}.

We have until now seen that Proto-Mataguayan must have had \intxt{*ɸ’} and \intxt{*s’} (occurring within morphemes) and \sound{PM}{*ɬ’} (occurring at morpheme boundaries only). The possibility of reconstructing \intxt{*x’} cannot be ruled out at this time, since \intxt{x’} does occur morpheme-internally in Maká; we would expect it to correspond to \intxt{k(ʲ)’} in other Mataguayan languages, though no clear cases have been identified thus far.\footnote{In Maká, \intxt{x’} has been attested in only one lexeme, \wordnl{ts’ix’ix\pl{its}}{mid-sized bee (gray, stings strongly, makes a hanging nest, produces small amounts of edible honey)} \citep[352]{AG99}.} We have found no evidence for reconstructing a glottalized uvular fricative \intxt{*χ’} in Proto-Mataguayan. The glottal fricative (or approximant) \intxt{*h}, of course, also lacks a glottalized equivalent.

In the following cognate sets, a cognate in Maká is lacking, and it is therefore impossible to determine whether they should be reconstructed with a glottalized stop or fricative in Proto-Mataguayan.

\begin{exe}
    \ex \lick
    \ex \burn
    \ex \resin
    \ex \snail
    \ex \fence
    \ex \belly
    \ex \jararaca
\end{exe}

The same situation is observed in etymologies with a limited distribution (Chorote and Wichí), whose PM~age is thus questionable.

\begin{exe}
    \ex \spitcw
    \ex \suckcw
    \ex \whiteegret
\end{exe}

\subsection{Status of glottalized consonants}\label{glott-status}
\citet{WK-CG-04} show that a contrast between a postglottalized consonant, a preglottalized consonant, and a sequence of a consonant and a /ʔ/ (in any order) is impossible within an onset or a coda in any language, suggesting that outputs such as [ʔm] or [t’] can be modeled in a variety of ways (i.e., by positing glottalized segments, sequences of a modal segment and a /ʔ/, or a prosodic feature [constricted glottis]). Throughout this book, we follow \citet{AG94}, \citet{VN14}, \citet{JC14b}, and \citet{AnG15} in analyzing glottalized onsets as complex segments rather than clusters of the type /Cʔ/ or /ʔC/. The two-segment analysis, posited by \citet[28–30]{KC94} for ’Weenhayek, assumes that glottalized consonants are sequences of underlying plain consonants and /ʔ/. This could technically also be applied to Proto-Mataguayan, which otherwise allows complex onsets. A third possibility, following \cits{WK-CG-04} reasoning, would be to consider that /ʔ/ and glottalization could be a property of the onset rather than of a given segment; that is, these elements could be associated with a laryngeal node (unordered with respect to the segments) dominated directly by the onset. The choice between these possibilities is a theory-internal one.

Synchronically, in all Mataguayan languages glottalized consonants may result when a plain consonant (stop, sonorant, or even fricative) coalesces with a heteromorphemic glottal stop. This has been described for Nivaĉle by \citet[29]{AnG15} and \citet[57]{LC20}, who dub the phenomenon in question \conc{secondary glottalization}, for Iyojwa’aja’ by \citet[78]{JC14b}, for ’Weenhayek by \citet[30]{KC94}, among others. \REF{ex:obst-plus-glott:niv}--\REF{ex:obst-plus-glott:lbw} illustrate this for stops.

\ea\label{ex:obst-plus-glott:niv}
Nivaĉle \citep[29]{AnG15}\\
        \gll x-åk-ʔín~\phonetic{xɑˈk’in}\\
                1\SG.\textsc{act}-go\_away-\textsc{ipfv}\\
                \glt `I am leaving'
\z
\ea\label{ex:obst-plus-glott:ijw}
Iyojwa’aja’ \citep[77--78]{JC14b}\\
\begin{xlist}
        \ex \gll t-ʔú-hat-ah-hen~\phonetic{ˈt’ohwatahaʔn}\\
                \textsc{imprs}-wake\_up-\CAUS-\textsc{imprs}/1\PL-\textsc{hen}\\
                \glt `someone wakes her/him up'
        \ex \gll i-ˀwét-ʔe~\phonetic{ʔiˈʔwit’eʔ}\\
                1\SG-place-\LOC\\
                \glt `at my place'
\end{xlist}
\z
\ea\label{ex:obst-plus-glott:whk}
’Weenhayek \citep[30]{KC94}\\
        \gll ʔimák-ʔis-hitʔah~\phonetic{ʔimaːɠ̊isiˈɗ̥ah}\\
                thing-good-\NEG\\
                \glt `it is insignificant'
\z
\ea\label{ex:obst-plus-glott:lbw}
Lower Bermejeño Wichí \citep[239]{VN14}\\
        \gll ∅-t-ʔeq~\phonetic{ˈt’ek}\\
             3-\textsc{t}-eat\\
             \glt `s/he eats'
\z

As for fricatives, the process in question is less productive, but still occurs at the prefix--root boundary \REF{ex:fric-plus-glott:niv}--\REF{ex:fric-plus-glott:whk}.

\newpage
\ea\label{ex:fric-plus-glott:niv}
Nivaĉle \citep[139]{JS16}\\
\begin{xlist}
        \ex \gll x-ʔíˀs~\phonetic{ˈk’iʔis}\\
                1\SG.\textsc{act}-write\\
                \glt `I write'
        \ex \gll ɬ-ʔíˀs~\phonetic{ˈt’iʔis}\\
                2.\textsc{act}-write\\
                \glt `you write'
\end{xlist}
\z
\ea\label{ex:fric-plus-glott:ijw}
Iyojwa’aja’ \citep[78, 91]{JC14b}\\
\begin{xlist}
        \ex \gll hl-ʔǻh~\phonetic{ˈt’ah}\\
                3.\textsc{poss}-skin/bark\\
                \glt `its skin/bark'
        \ex \gll s-ʔú-hat-hen~\phonetic{ˈts’ohwateʔn}\\
                1\SG.\textsc{inact}-wake\_up-\CAUS-\textsc{hen}\\
                \glt `you/s/he wake(s) me up'
        \ex \gll s-ʔahán-eh~\phonetic{ts’aˈhane}\\
                1\SG.\textsc{inact}-know-\APPL\\
                \glt `I know'
        \ex \gll s-ʔåhwéhl~\phonetic{ts’aˈhwɛl}\\
                1\SG.\textsc{inact}-be\_ashamed\\
                \glt `I am ashamed'
\end{xlist}
\z
\ea\label{ex:fric-plus-glott:whk}
    ’Weenhayek \citep[96]{KC16}
    \begin{xlist}
        \ex \gll ɬ-ʔisaˀn~\phonetic{t’iˈsanʔ}\\
                3.{\textsc{poss}}-meat/flesh\\
                \glt `its meat/flesh'
    \end{xlist}
\z

Finally, coalescence of sonorants with a glottal stop has been described for Chorote, and traces of this process are found in Nivaĉle and Wichí. Phonetically, an underlying sequence of a sonorant and a glottal stop yields a preglottalized sonorant in Chorote, analyzed as a two-phase segment by \citet[81]{JC14b}.\footnote{If one adopts a two-segment analysis for glottalized sonorants, the phenomenon in question should be viewed as an instance of metathesis. Throughout this book, glottalized sonorants are rather analyzed as complex segments.}

\newpage
\ea\label{ex:son-plus-glott:ijw}
Iyojwa’aja’ \citep[77--78]{JC14b}\\
\begin{xlist}
        \ex \gll n-ʔót~\phonetic{ˈʔnɔt}\\
                \textsc{gnr}-chest\\
                \glt `chest (indefinite possessor)'
        \ex \gll j-ʔǻl-hen~\phonetic{ˈʔjahleʔn}\\
                3.\textsc{act}-die-\PL\\
                \glt `they died'
\end{xlist}
\z

In Nivaĉle, the absolutizing prefix \intxt{t(i)n\mbox{-}} fuses with the stem-initial glottal stop as \intxt{tiˀn\mbox{-}}.

\ea\label{ex:son-plus-glott:niv}
Nivaĉle \citep[159]{LC20}\\
\begin{xlist}
        \ex \gll t(i)n-ʔåx~\phonetic{tiˈˀnɑx}\\
                \textsc{gnr}-skin\\
                \glt `leather strap'
\end{xlist}
\z

In Wichí, at least the palatal approximant \intxt{j} systematically coalesces with a glottal stop in the verbs that take the prefix \intxt{j\mbox{-}} (allomorph of \intxt{ji\mbox{-}}).

\ea\label{ex:son-plus-glott:lbw}
Lower Bermejeño Wichí \citep[237--238]{VN14}\\
\begin{xlist}
        \ex \gll n̩-j-ʔaχ-ʔam~\phonetic{n̩ˀjɑχˈʔam}\\
                1\SG-\textsc{i}-hit-2\SG.P\\
                \glt `I hit you'
        \ex \gll n̩-j-ʔeɬ-jen n̩-ɬ-os~\phonetic{n̩ˀjeɬˈjen n̩ˈɬos}\\
                1\SG-\textsc{i}-urinate-\textsc{caus} 1\SG-\textsc{th}-son\\
                \glt `I make my son urinate'
\end{xlist}
\z

\ea\label{ex:son-plus-glott:whk}
'Weenhayek \citep[124, 128]{KC16}\\
\begin{xlist}
        \ex \gll ∅-j-ʔót~\phonetic{ˈˀjoːt}\\
                3-\textsc{i}-hit-2\SG.P\\
                \glt `I hit you'
        \ex \gll n(í)-ʔíl-a~\phonetic{ˀnĩːˈlaʔ}\\
                3.\textsc{neg.irr}-die-\textsc{neg.irr}\\
                \glt `lest s/he die'
\end{xlist}
\z

In light of these alternations, which were certainly active already in Proto-Mataguayan, one is tempted to ask whether all instances of glottalized consonants in Proto-Mataguayan must be synchronically analyzed as sequences of a plain consonant and a glottal stop. The answer is negative at least for combinations of a sonorant and a glottal stop: both \sound{PM}{*lʔ} and \intxt{*mʔ} were licit clusters in Proto-Mataguayan. No examples have been found for \sound{PM}{*nʔ}, \intxt{*jʔ}, or \intxt{*wʔ}.

\begin{exe}
    \ex \soninlaw
    \ex \daylhuma
    \ex \doveula
    \ex \rat
\end{exe}

Since contrasts such as /lʔ/ vs. /ˀl/ are predicted by \citet{WK-CG-04} to be impossible within an onset or a coda, we conclude that \sound{PM}{*lʔ} and \intxt{*mʔ} were heterosyllabic, and that the process transforming sequences of a plain consonant and a glottal stop into glottalized segments was not fully active in Proto-Mataguayan.

\section{Preglottalized codas}\label{glott-codas}
Most complex codas in Proto-Mataguayan are of the type */ʔC/.\footnote{In this book, we follow \cits{AnG16b} analysis of the Nivaĉle reflexes of such codas as sequences of the type /ʔC/. Alternatively, one could follow \cits{WK-CG-04} idea, whereby glottalization in a coda is represented by means of the feature [constricted glottis] in the laryngeal node dominated directly by the coda. The choice between these possibilities is a theory-internal one.} (In addition, there is evidence for reconstructing */jh/ and */lh/, for which see \sectref{cx}.) We dub */ʔC/ codas ``preglottalized'' and represent them as \intxt{*ˀC}. They are best preserved in Maká. In Nivaĉle, they are preserved only in stressed syllables; in unstressed syllables, these codas are deglottalized, as discussed in \sectref{ni-deglottalization-codas}. In Wichí, Manjui, and possibly Iyo’awujwa’, the preglottalized codas \intxt{*ˀm}, \intxt{*ˀn}, \intxt{*ˀl} are preserved word-finally (the latter only in Manjui), whereas other preglottalized codas merge with their plain counterparts.\footnote{In ’Weenhayek, the reflexes of these codas are in fact articulated as postglottalized rather than preglottalized \citep[33–35]{KC94}. This is in line with well-known cross-linguistic tendency of word-final or preconsonantal glottalized sonorants to realize their creak toward the end of the sonorant, attributed by \citet[394–396]{MGPL01} to the necessity to enhance the acoustic cues associated with the vowel-to-consonant transition. However, the glottalized sonorant codas are clearly preglottalized rather than postglottalized in Nivaĉle and Chorote. In \cits{CG-WK-13} terms, ’Weenhayek follows the so-called \conc{prosodic pattern}, whereas Nivaĉle and Chorote conform to the so-called \conc{onset pattern} of laryngeal timing, both of which are cross-linguistically attested.} In Iyojwa’aja’, all preglottalized codas merge with their plain counterparts. Already in Proto-Mataguayan, a process exists whereby preglottalized codas are deglottalized when the coda resyllabifies as the onset of the next syllable before certain types of affixes (for example, the plural form of \wordnl{*kʼutX₂₃áˀn}{thorn} is reconstructed as \intxt{*kʼutX₂₃án\mbox{-}its}). Other affixes fail to trigger this process, however, as seen in \word{PM}{*ji\mbox{-}péˀj\mbox{-}aʔ}{s/he hears}.

The following examples show that preglottalized obstruent codas are preserved as such in Maká and (in stressed syllables) in Nivaĉle, but merge with their plain counterparts in all other languages. One possible exception to this generalization is that \sound{PM}{*ˀɸ} may have regularly yielded \sound{PW}{*p} rather than \intxt{*xʷ}, even though only one example is known \REF{pregl-suckb}. The unexpected loss of preglottalization in Maká is seen in \REF{degl-rootn}, \REF{degl-tilh}, and \REF{degl-lhup}.

\begin{exe}
    \ex \honeycomb
    \ex \drinkn
    \ex \son
    \ex \firef
    \ex \cutdown
    \ex \fieldn
    \ex \rootn \label{degl-rootn}
    \ex \hidev
    \ex \leech
    \ex \fart
    \ex \dew
    \ex \wax
    \ex \jaguar
    \ex \water
    \ex \sandisaj
    \ex \tail
    \ex \fall
    \ex \dividev
    \ex \answer
    \ex \sweat
    \ex \oldn
    \ex \wash
    \ex \winter
    \ex \languageword
    \ex \thread
    \ex \yicalhuk
    \ex \powder
    \ex \nose
    \ex \smelln
    \ex \lippaset
    \ex \fence
    \ex \lid
    \ex \starn
    \ex \vein
    \ex \mesh
    \ex \sprout
    \ex \abdcavity
    \ex \suckb \label{pregl-suckb}
    \ex \shoot
    \ex \spinsew \label{degl-tilh}
    \ex \carrysh
    \ex \dig
    \ex \blind
    \ex \uncle
    \ex \duraznillo
    \ex \nest \label{degl-lhup}
    \ex \badmood
    \ex \burrow
    \ex \walk
    \ex \climb
    \ex \night
    \ex \headits
    \ex \spring
    \ex \earth
    \ex \palosanto
    \ex \sandyplace
    \ex \firewoodhuk
    \ex \knee
    \ex \snakeatuj
    \ex \chest
\end{exe}

The very same correspondence is observed in etymologies with a limited distribution (Maká and Nivaĉle, Chorote and Wichí), whose PM~age is thus questionable.

\begin{exe}
    \ex \fry
    \ex \hole
    \ex \leafmn
    \ex \onemn
\end{exe}

\sound{PM}{*ˀj} also merges with its plain counterpart (\sound{PM}{*j}) in all languages except Nivaĉle in the coda position. Note that \sound{PCh}{*jʔ} is the regular reflex not only of \sound{PM}{*ˀj}, but also of \sound{PM}{*j} word-finally due to the process of \intxt{ʔ}\mbox{-}epenthesis in Chorote.

\begin{exe}
    \ex \yicaay
    \ex \notafraid
    \ex \sunn
    \ex \withstand
    \ex \bitter
    \ex \blood
\end{exe}

The very same correspondence is observed in two etymologies with a limited distribution (Maká and Nivaĉle), whose PM~age is thus questionable.

\begin{exe}
    \ex \weave
    \ex \soundv
\end{exe}

By contrast, the examples below show that \sound{PM}{*ˀm} and \sound{PM}{*ˀn} are preserved as contrastive units not only in Maká and Nivaĉle, but also in Chorote and Wichí, at least word-finally. The Wichí reflexes in \REF{'m-up} and \REF{'n-kingvulture} are irregular: the former shows an irregular loss of the word-final consonant; the latter is deviant in a number of respects and lacks the expected glottalization.

\begin{exe}
    \ex \pronominal
    \ex \grabwork
    \ex \thorncutjan
    \ex \defecate
    \ex \up \label{'m-up}
    \ex \kingvulture \label{'n-kingvulture}
    \ex \throwv
    \ex \meat
\end{exe}

Finally, \sound{PM}{*ˀl} is reconstructed in order to account for three cognate sets with a limited distribution (Maká and Nivaĉle, Chorote and Wichí), whose PM~age is thus questionable. In these cases, the glottalization in preserved in Maká, Nivaĉle, and Chorote, but not in Wichí (due to a process that converted word-final \sound{PM}{*l} and \intxt{*ˀl} into \sound{PW}{*lʰ}, see \sectref{wi-l-lh}).

\begin{exe}
    \ex \brightness
    \ex \locustcw
    \ex \agile
\end{exe}

\section{\intxt{*CX}-clusters (consonant + a guttural fricative)}\label{cx}
There is ample evidence supporting the reconstruction of consonant clusters of the structure */CX/, where \intxt{X} stands for a velar, uvular, or glottal fricative. Their development is shown in \tabref{PM-CX}. Note that \sound{PM}{*h} does not occur after fricatives (\sectref{fricative-h}). Conversely, \sound{PM}{*χ} is only securely reconstructed after fricatives (it may have also occurred after stops and/or sonorants, but the evidence is inconclusive). It is unclear how these clusters were syllabified in Proto-Mataguayan; their reflexes are typically tautosyllabic in Chorote and Wichí, but not in Nivaĉle and Maká. We find it more likely that Chorote and Wichí retain the original situation, since */CX/ clusters are particularly common morpheme-initially.

\begin{table}
\caption{PM clusters with a guttural fricative as the second element}
\label{PM-CX}
\fittable{
 \begin{tabular}{ccccc}
  \lsptoprule
            Proto-Mataguayan & Maká & Nivaĉle & Proto-Chorote & Proto-Wichí\\\midrule
            *Px & Px & Px / Pʃ & *P & *Ph\\
            (*Pχ) & (Pχ) & (Px) & (*P) & (*Ph)\\
            *Ph & Ph & Px & *hP / *P & *Ph\\
            *Fx & Fx & Fx / Fʃ & *F & *F\\
            *Fχ & F & Fx & *F & *F\\
            *Mx & Mx & Mx / Mʃ & *hM & *Mh\\
            (*Mχ) & (Mχ) & (Mx) & (*hM) & (*Mh)\\
            *Mh & Mh & Mx & *hM & *Mh\\
  \lspbottomrule
  \multicolumn{5}{p{\textwidth}}{P~=~stop, F~=~fricative, M~=~sonorant}\\
 \end{tabular}
 }
\end{table}

The examples below show the evolution of PM~clusters with \intxt{*x} as the second element. These are preserved in Maká and Nivaĉle (with \sound{PM}{*x} yielding \sound{Ni}{ʃ} in palatalizing environments, as discussed in \sectref{proto-j} and \sectref{ni-vel-uv}). In Chorote, they yield \sound{PCh}{*hC} if the consonant is a sonorant and \sound{PCh}{*C} otherwise; the vowel epenthesis in \REF{jx-teach} is irregular (see more on the developement of \sound{PM}{*Px} > \sound{PCh}{*P} in \sectref{ch-consonant-dorsal}). In Wichí, they yield \sound{PW}{*Ch} unless the consonant is a fricative, in which case one finds the reflex \sound{PW}{*C}. Note that the reflexes in \REF{px-beard} in Nivaĉle and Wichí are entirely irregular due to contamination with those of \word{PM}{*\mbox{-}pǻs(\mbox{-}eˀt)}{lip}; the regular reflexes are found in Maká and Chorote. \REF{nx-sleepiness} shows vowel epenthesis in Maká and Wichí, presumably due to the fact that the consonant cluster occurs word-initially.

\begin{exe}
    \ex \armadillonomaka
    \ex \noseobl
    \ex \sleepiness \label{nx-sleepiness}
    \ex \smellv
    \ex \pathpl
    \ex \beard \label{px-beard}
    \ex \uncle \label{tx-uncle}
    \ex \eggits
    \ex \headits
    \ex \tuscaf
    \ex \tuscat
    \ex \tuscag
    \ex \teach \label{jx-teach}
\end{exe}

The following examples show the evolution of PM~clusters with \intxt{*χ} as the second element. All clear cases involve a fricative as the first element. In Maká, Chorote, and Wichí, \sound{PM}{*χ} is lost after a fricative. In Nivaĉle, one finds the reflex \intxt{Cx}. We believe \intxt{*χ} could also occur after other kinds of consonants, as is still the case in Maká, and we predict its reflexes to be as detailed in \tabref{PM-CX}; however, all putative cases of \intxt{*Pχ} and \intxt{*Mχ} that we have considered allow for alternative reconstructions as well.

\begin{exe}
    \ex \killbird
    \ex \finger
    \ex \redquebracho
    \ex \thunder
    \ex \meat
\end{exe}

The examples below show the evolution of PM~clusters with \intxt{*h} as the second element. In Maká, \sound{PM}{*h} is preserved. In Nivaĉle, one finds \intxt{Cx} (except that \intxt{*wh} yields \sound{Ni}{x}). In Chorote, such clusters always yield \sound{PCh}{*hC} after a stressed vowel except if the consonant in question is \sound{PCh}{*s} < \sound{PM}{*ts} (phonetically, /s/ in Chorote often does surface as \phonetic{hs} or \phonetic{xs}, but there is no contrast between /s/ and /hs/). After an unstressed vowel, the reflex is \sound{PCh}{*C} \REF{qh-knee}, and word-initially one finds an inserted vowel, as in \REF{ph-cactus} and \REF{ph-up}. In Wichí, these same clusters yield \sound{PW}{*Ch}, with vowel insertion applying word-initially at least in the cluster \intxt{*kh} \REF{ph-cactus}.

\begin{exe}
    \ex \centipedepl
    \ex \cactus \label{ph-cactus}
    \ex \oldpl
    \ex \powderpl
    \ex \ropepl
    \ex \up \label{ph-up}
    \ex \fishwithhook
    \ex \wildcat
    \ex \platepl
    \ex \paloflojot
    \ex \spousewh
    \ex \marry
    \ex \healthy
    \ex \headpl
    \ex \caracarapl
    \ex \knee \label{qh-knee}
    \ex \wildbean
    \ex \teach
\end{exe}

The same correspondences are observed in etymologies with a limited distribution (Maká and Nivaĉle, Chorote and Wichí), whose PM~age is thus questionable.

\begin{exe}
    \ex \locustmn
    \ex \pacu
    \ex \bilecwpl
    \ex \mollef
    \ex \queenpalmf
    \ex \heartcw
\end{exe}

In some cases crucial cognates in Maká are either lacking or attested with different consonants in different sources, making it impossible to ascertain which guttural fricative is to be reconstructed to Proto-Mataguayan.

\begin{exe}
    \ex \coal
    \ex \youngersis
    \ex \thorncutjan
    \ex \roast
    \ex \nightmonkey
    \ex \girl
    \ex \mucus
    \ex \snore
    \ex \rhea
\end{exe}

The same correspondences are observed in etymologies with a limited distribution (Maká and Nivaĉle, Chorote and Wichí), whose PM~age is thus questionable.

\begin{exe}
    \ex \heavyv
    \ex \orphancwpl
    \ex \throwcw
    \ex \doradocw
\end{exe}

Quite exceptionally for Mataguayan languages, in a handful of morphemes, the clusters \intxt{*jh} and possibly \intxt{*lh} are reconstructed in the coda position (word-finally only). For aesthetic reasons, we represent them as \intxt{*jʰ} and \intxt{*lʰ}. The evidence for this comes from Chorote and 'Weenhayek. In both lects, /h/ occurs in word-final position, thus bleeding *\intxt{ʔ}\mbox{-}insertion. However, it does not surface when the morpheme is not word-final, as described by \citet{KC94} and \citet{JC14b}. For instance, the indirect evidential in Iyo'awujwa' and Manjui surfaces as \intxt{\mbox{-}{t'ejʰ}} when it is word-final, but as \intxt{\mbox{-}{t'ej}\mbox{-}} or \intxt{\mbox{-}{t'ij}\mbox{-}} when an enclitic or suffix follows. The phonetic realization of the reflexes of \intxt{*lʰ} does not differ in Chorote and ’Weenhayek from that of the reflexes of \sound{PM}{*ɬ}. In Wichí, one finds the reflex \intxt{*h} rather than \intxt{**jʰ} after the vowel \intxt{*i} \REF{recipient-yh-h}.

\begin{exe}
    \ex \plaj
    \ex \distal
    \ex \bathe
    \ex \soul
    \ex \recipient \label{recipient-yh-h}
\end{exe}

\section{Other consonant clusters}\label{clusters}
Other types of consonant clusters are reconstructed primarily based on evidence from Nivaĉle.

The Proto-Mataguayan sequence \intxt{*kɸ} develops normally in Maká and Nivaĉle, but yields \sound{Proto-Chorote}{*kw} (>~\sound{Ijw}{kʲ}, \sound{I’w/Mj}{k}) and \sound{Proto-Wichí}{*kʷ}. The preceding vowel (if there is one) apparently becomes rounded in the latter two languages, though it is unknown whether this is regular, since only one example has been found.

\begin{exe}
    \ex \bite
    \ex \earkfe
    \ex \frighten
\end{exe}

The Proto-Mataguayan sequence \intxt{*nj} or \intxt{*ˀnj} preserves its palatal approximant in Maká (with \sound{PM}{*ˀnj} > \sound{Mk}{nij} at least word-initially), but loses it in Chorote and Wichí (in the latter language, \sound{PM}{*ˀnj} > \sound{PW}{*ˣn} at least word-initially).

\begin{exe}
    \ex \smelln
    \ex \cavy
\end{exe}

The Proto-Mataguayan onset \intxt{*st} is preserved in Nivaĉle. It is resolved by means of \intxt{i}\mbox{-}insertion in Maká, whereas in Chorote and Wichí a vowel (\sound{PCh}{*ᵊ}, \sound{PW}{*i}) is inserted before the cluster (at least word-initially).

\begin{exe}
    \ex \whitequebracho
    \ex \kingvulture
    \ex \cardon
    \ex \chachalaca
\end{exe}

Most clusters involving two voiceless segments are typically preserved in Nivaĉle and Wichí, whereas in Chorote they are resolved by means of vowel insertion (the inserted vowel is \sound{PCh}{*ᵊ}, or \sound{PCh}{*i} after \sound{PCh}{*k}). Note the sound change \sound{PM}{*tsn} > \sound{PW}{*tn} in Wichí in \REF{ts-toad}.

\begin{exe}
    \ex \north
    \ex \tortoise
    \ex \whitealgarrobof
    \ex \dovesipup
    \ex \toad \label{ts-toad}
    \ex \precipice
\end{exe}

In one root, a cluster involving two voiceless segments occurs in the beginning of a relational stem in Maká, whereas other languages show a reflex of \sound{PM}{*á} between the consonants in question. It is unclear whether a consonant cluster should be reconstructed in this case (assuming vowel insertion in Nivaĉle, Chorote, and Wichí) or whether the vowel was already there in Proto-Mataguayan (assuming an irregular syncope in Maká).

\begin{exe}
    \ex \face
    \ex \eyebrow
\end{exe}

Clusters involving \sound{PM}{*l}, \intxt{*w}, or \intxt{*ˀw} as the first member develop normally in Nivaĉle. In Maká, they are resolved by means of \intxt{e}\mbox{-}insertion if the cluster occurs stem-initially; in the middle of the stem the sonorant is simply lost \REF{wt-heartmn}. In Chorote, \sound{PM}{*l} as a first member of a consonant cluster is deleted word-initially, but is preserved word-medially; \sound{PM}{*(ˀ)w}, by contrast, is preserved word-initially (with an intrusive \sound{PCh}{*ᵊ} breaking the cluster) but lost word-medially. In Wichí, the first element of the cluster is lost, but a deleted \sound{PM}{*w} can trigger rounding of \sound{PM}{*e} to \sound{PW}{*o} in \REF{wt-tooth}.

\begin{exe}
    \ex \spouse
    \ex \flu
    \ex \squash
    \ex \heartmn \label{wt-heartmn}
    \ex \majan
    \ex \tooth \label{wt-tooth}
    \ex \rib
    \ex \belly
    \ex \metal
\end{exe}

Only one word is reconstructed with a cluster whose initial element is \sound{PM}{*ˀj}. In Maká, \sound{PM}{*ˀjt} yields \intxt{ʔt} in variation with \intxt{t} \citep[130]{AG99}; in Nivaĉle, one finds \intxt{ˀβt} varying with \intxt{ˀjt}; in Chorote, the reflex is \intxt{*jʔt}; in Wichí, \intxt{*jt}.

\begin{exe}
    \ex \hurt
\end{exe}

Clusters with a PM~guttural fricative followed by another consonant evolve normally in Maká and Nivaĉle, with an epenthetic \sound{Mk}{i} breaking apart the PM~cluster \intxt{*xn} \REF{xn-spring} and an epenthetic \sound{Ni}{a} resolving the triconsonantal cluster in \REF{nxt-cavy}. In Chorote, the guttural consonant disappears stem-initially, as in \REF{xn-spring}, \REF{xm-fox}–\REF{xw-moon}, except in \REF{xp-straw}, where \sound{PM}{*Xp} yields \sound{PCh}{*ʔip}. Word-medially (at least before a stop), the guttural consonant yields \sound{PCh}{*h}, and a vowel (a copy of the preceding vowel) is inserted to break the cluster apart, as in \REF{xt-earcw}--\REF{nxt-cavy}, \REF{xt-femalebreast}. In Wichí, the guttural consonant is lost stem-medially, at least preceding a stop, as in \REF{nxt-cavy} and \REF{xt-femalebreast}; stem-initial clusters of a guttural consonant and a sonorant yields \sound{PW}{*ˣC}, as in \REF{xn-spring}, \REF{xm-fox}, \REF{xw-moon}, whereas in the only example of a stem-initial cluster of a guttural consonant and a stop one finds \sound{PW}{*hp} as the reflex \REF{xp-shadow}.

\begin{exe}
    \ex \earcw \label{xt-earcw}
    \ex \cavy \label{nxt-cavy}
    \ex \spring \label{xn-spring}
    \ex \straw \label{xp-straw}
    \ex \fox \label{xm-fox}
    \ex \shadow \label{xp-shadow}
    \ex \moon \label{xw-moon}
    \ex \femalebreast \label{xt-femalebreast}
\end{exe}

Clusters with \sound{PM}{*(ˀ)w} as the last element are followed by \sound{PM}{*u} in all known examples. These evolve normally in Nivaĉle, with an epenthetic \sound{Ni}{a} resolving the triconsonantal cluster in \REF{stw-kingvulture}. The cluster \sound{PM}{*s(ˀ)w} yields \sound{Mk}{suʔ}, \sound{PCh}{*sᵊʔ}, \sound{PW}{*s}, whereas \sound{PM}{*stw} is found in one example \REF{stw-kingvulture}, where it evolves in an idiosyncratic way in Chorote and Wichí.

\begin{exe}
    \ex \kingvulture \label{stw-kingvulture}
    \ex \anteater
    \ex \likelove
\end{exe}

The PM~clusters \intxt{*sk},  \intxt{*sl}, and  \intxt{*tl} are resolved by vowel insertion in Chorote (\sound{PCh}{*ᵊ}) and Wichí (\sound{PW}{*i}) when tautosyllabic. In the only example, a heterosyllabic instance of \intxt{*sk’} develops normally in Chorote. In Nivaĉle, an epenthetic \intxt{a} breaks apart the cluster \intxt{tk͡l}, and in most dialects the PM~sequence \intxt{*sl} is reflected as \intxt{ʃk͡l} rather than \intxt{sk͡l}.

\begin{exe}
    \ex \mesh
    \ex \wildcat
    \ex \blind
    \ex \widower
\end{exe}

The PM~clusters \intxt{*qk} and \intxt{*tts} occur in one etymology each. In Maká, they yield \intxt{qq} and \intxt{tts}. In Nivaĉle, they are reflected as \intxt{k} and \intxt{ts}. In Chorote, \intxt{*qk} is reflected as \intxt{*Vk}, with the doubling of the preceding vowel.

\begin{exe}
    \ex \welln
    \ex \willow
\end{exe}

Finally, the PM~clusters \intxt{*ɸq} and \intxt{*ɸts} occur in one or two etymologies each and are reconstructed based on evidence from Nivaĉle. In other languages, \sound{PM}{*ɸ} is either lost or separated from the following consonant by an epenthetic \intxt{*i}. Due to the scarcity of examples, it is difficult to formulate a generalization.

\begin{exe}
    \ex \elbow
    \ex \suncho
    \ex \palm
\end{exe}

\section{Syllabic consonants}\label{syllabic-C}
Some coronal consonants could apparently occur as syllabic nuclei. They are reconstructed only at the left margin of words in grammatical prefixes, with very few exceptions. This distribution aligns well with one’s typological expectations: cross-linguistically, syllabic consonants are known to be preferred in grammatical affixes and at word edges \citep[159–161]{AB78}. The inventory of syllabic consonants in our reconstruction is, however, rather surprising from a typological point of view: alongside the cross-linguistically common syllabic nasal \intxt{*n̩} we posit two syllabic obstruents, \intxt{*ɬ̩} and \intxt{t̩}. This counters \cits{AB78} generalization whereby ``[i]f a language possesses syllabic obstruents, it possesses syllabic \intxt{s} or \intxt{š̩} \anjc{IPA~[ʃ̩]}, given that it has nonsyllabic \intxt{s} or \intxt{š} \anjc{IPA~[ʃ]}'': note that Proto-Mataguayan clearly had a \intxt{*s}, but we have found no solid evidence to support the reconstruction of \intxt{*s̩}.\footnote{It is technically possible that some of the \intxt{*sC} sequences that we reconstruct for Proto-Mataguayan, as in \word{PM}{*skäˀt}{mesh} or \wordnl{*stwúˀn}{king vulture}, could have in fact involved a syllabic \intxt{*s̩}, as suggested by the fact that Maká, Chorote, and Wichí typically insert a vowel before or after the \intxt{*s} in such words. However, it is equally possible to account for the evolution of these cognate sets by positing a non-syllabic \intxt{*s} for Proto-Mataguayan, as done in this book.} Be that as it may, at this time we are unable to ascertain the details of phonetic implementation of the phonologically syllabic obstruents in Proto-Mataguayan. At least \intxt{*t̩} must have been articulated with an audible release or with a transitional (intrusive) vowel, as syllabic voiceless stops must be released in order to be audible before another obstruent \citep[185]{AB78}. This is indirectly supported by the reflexes in the daughter languages, where one frequently finds an epenthetic vowel continuing what may have been a PM~intrusive vowel (that way, an erstwhile syllabic consonant is unpacked into a sequence of a consonant and a vowel, with the preservation of the mora associated with the consonant in PM). The insertion of a segment in these cases must have occurred independently in the daughter languages, because the individual languages differ regarding the exact conditions and quality of the inserted vowels.

\subsection{Syllabic \intxt{*ɬ}}\label{syllabic-lh}
Syllabic \intxt{*ɬ} occurs in a number of homophonous prefixes when they precede consonant-initial stems. These include the 3.{\textsc{poss}} prefix, the 2.{\textsc{act}} prefix, and the feminine prefix in demonstratives. Before vowels, all of these prefixes surface as a regular (non-syllabic) \intxt{*ɬ\mbox{-}}. Before consonants, these prefixes constitute a syllable on their own in PM, as evidenced by their reflexes in the daughter languages (this does not include the position before a glottal stop, as \sound{PM}{*ɬ\mbox{-}ʔ\mbox{-}} coalesces into \intxt{*ɬ’\mbox{-}}).

\begin{table}
\caption{PM prefixes of the shape \intxt{*ɬ-} and their reflexes}
\label{PM-lh-syll-refl}
 \begin{tabularx}{\textwidth}{cccXXXX}
  \lsptoprule
            PM & function & position & Maká & Nivaĉle & PCh & PW\\\midrule
  *ɬ-V… & 3.{\textsc{poss}} & before V & ɬ-V… & ɬ-V… & *hl-V… & *ɬ-V…\\
  *ɬ-V… & 2.{\textsc{act}} & before V & ɬ-V… & ɬ-V… & *hl-V… & *ɬ-V…\\
  *ɬ-V… & {\textsc{f.dem}} & before V & --- & --- & *hl- & ---\\
  *ɬ̩-C… & 3.{\textsc{poss}} & before C & ɬe-C… /\newline ɬa-Ca… /\newline ɬo-Co… & ɬ-C… /\newline ɬa-CC… & *hᵊ-C… & *ɬ̩-C…\\
  *ɬ̩-C… & 2.{\textsc{act}} & before C & ɬe-C… /\newline ɬa-Ca… /\newline ɬo-Co… & ɬ-C… /\newline ɬa-CC… & *hᵊ-C… & *ɬ̩-C…\\
  *ɬ̩-C… & {\textsc{f.dem}} & before C & --- & ɬ-C… & *ha-C… & ---\\
  *ɬ-’… & 3.{\textsc{poss}} & before ʔ & ɬ-’… & t-’… & *t-’… & *t-’…\\
  *ɬ-’… & 2.{\textsc{act}} & before ʔ & ? & t-’… &  *<hᵊ>t-’… &  *<ɬ̩>t-’…\\
  \lspbottomrule
 \end{tabularx}
\end{table}

In Maká, the third-person possessive and the second-person active prefixes both surface as \intxt{ɬ\mbox{-}} before vowels \REF{ex:sylllh:1:mak}, whereas before consonants \intxt{ɬe\mbox{-}} is found; in the latter case the prefix vowel harmonizes to \intxt{a} or \intxt{o} if the next syllable contains a low vowel \citep[106--107]{AG-BG03}, as in \REF{ex:sylllh:2:mak}. Before \sound{Mk}{ʔ}, the third-person possessive prefix surfaces as \intxt{ɬ\mbox{-}}, a combination claimed to involve a syllabic \intxt{ɬ} by \citet[67]{AG89} and transcribed as \intxt{*ɬ’\mbox{-}} in this book \REF{ex:sylllh:3:mak}. The feminine prefix in demonstratives is not preserved in Maká.

\ea\label{ex:sylllh:1:mak}
    Maká \citep[85, 91, 148]{AG94}
    \begin{xlist}
        \ex\gll ɬ-uk\\
                3.{\textsc{poss}}-grandson\\
                \glt `his/her grandson'
        \ex\gll ɬ-exiʔ\\
                3.{\textsc{poss}}-mouth\\
                \glt `his/her mouth'
        \ex\gll ɬ-otoj\\
                2.{\textsc{act}}-dance\\
                \glt `you dance'
        \ex\gll ɬ-ija\\
                2.{\textsc{act}}-drink\\
                \glt `you drink'
    \end{xlist}
\z

\ea\label{ex:sylllh:2:mak}
    Maká \citep[85, 88, 148]{AG94}
    \begin{xlist}
        \ex\gll ɬe-k’inix\\
                3.{\textsc{poss}}-younger\_brother\\
                \glt `his/her younger brother'
        \ex\gll ɬo-nokiʔ\\
                3.{\textsc{poss}}-elbow\\
                \glt `his/her elbow'
        \ex\gll ɬe-fejejkiʔ\\
                2.{\textsc{act}}-rotate\\
                \glt `you rotate'
        \ex\gll ɬa-maʔ\\
                2.{\textsc{act}}-sleep\\
                \glt `you sleep'
    \end{xlist}
\z

\ea\label{ex:sylllh:3:mak}
    Maká \citep[68]{AG94}
    \begin{xlist}
        \ex\gll ɬ-’iʔ\\
                3.{\textsc{poss}}-juice\\
                \glt `its juice'
    \end{xlist}
\z

In Nivaĉle, according to \citet[59, 62, 230--231]{AnG15}, the third-person possessive and the second-person active prefixes surface as \intxt{ɬ\mbox{-}} before vowels \REF{ex:sylllh:1:niv} and before simplex onsets, a position where the prefixes in question are likely to form a syllable on their own \REF{ex:sylllh:2:niv}. (The feminine prefix in demonstratives, which only occurs before consonants, also surfaces as \intxt{ɬ\mbox{-}}.) Before consonant clusters, \intxt{ɬa\mbox{-}} is found \REF{ex:sylllh:3:niv}. If the stem starts with a glottal stop, the prefixes in question coalesce with them as \intxt{t\mbox{-}’} \REF{ex:sylllh:4:niv}.

\ea\label{ex:sylllh:1:niv}
    Nivaĉle \citep[59, 62]{AnG15}
    \begin{xlist}
        \ex\gll ɬ-ǻse \\
                3.{\textsc{poss}}-daughter \\
                \glt `his/her daughter'
        \ex\gll ɬ-ám \\
                2.{\textsc{act}}-come \\
                \glt `you come'
    \end{xlist}
\z

\ea\label{ex:sylllh:2:niv}
    Nivaĉle \citep[59, 62, 99, 231]{AnG15}
    \begin{xlist}
        \ex\gll ɬ-t’óx \\
                3.{\textsc{poss}}-aunt \\
                \glt `his/her aunt'
        \ex\gll ɬ-k͡líˀʃ \\
                3.{\textsc{poss}}-word \\
                \glt `his/her word'
        \ex\gll ɬ-péˀja \\
                2.{\textsc{act}}-listen \\
                \glt `you listen'
        \ex\gll ɬ-pa \\
                {\textsc{f}}-{\textsc{dem.nfh}} \\
                \glt `that (feminine, never seen by the speaker)'
    \end{xlist}
\z

\ea\label{ex:sylllh:3:niv}
    Nivaĉle \citep[59, 62, 231]{AnG15}
    \begin{xlist}
        \ex\gll ɬa-ktéˀtʃ \\
                3.{\textsc{poss}}-grandfather \\
                \glt `his/her grandfather'
        \ex\gll ɬa-ɸxúx \\
                3.{\textsc{poss}}-toe \\
                \glt `his/her toe'
        \ex\gll ɬa-ktʃáʔ \\
                2.{\textsc{act}}-paddle \\
                \glt `you paddle'
    \end{xlist}
\z

\ea\label{ex:sylllh:4:niv}
    Nivaĉle \citep[123]{AnG15,JS16}
    \begin{xlist}
        \ex\gll t-’íʔ \\
                3.{\textsc{poss}}-liquid \\
                \glt `its broth'
        \ex\gll t-’eɸén\\
                2.{\textsc{act}}-help\\
                \glt `you help'
    \end{xlist}
\z

In Chorote, the third-person possessive, the second-person active prefixes, and the feminine prefix in demonstratives surface as \intxt{hl\mbox{-}} before vowels or \intxt{h}\mbox{-}initial stems \REF{ex:sylllh:1:ijw} but as \intxt{hi\mbox{-}} before supraglottal consonants \REF{ex:sylllh:2:ijw}. The \intxt{i} in the latter case goes back to the intrusive vowel \intxt{*ᵊ}, as it causes the second palatalization but not the first palatalization in Chorote (see \sectref{ch-palat}). If the stem starts with a glottal stop, the third-person possessive prefix coalesces with it as \intxt{t\mbox{-}’} and the second-person active prefix as \intxt{hit\mbox{-}’} \REF{ex:sylllh:3:ijw}.

\ea\label{ex:sylllh:1:ijw}
    Iyojwa’aja’ \citep[132, 161, 169]{ND09}
    \begin{xlist}
        \ex\gll hl-ɔ́t\\
                3.{\textsc{poss}}-scales\\
                \glt `its scales'
        \ex\gll hl-ɔ́h \\
                2.{\textsc{act}}-shovel \\
                \glt `you shovel'
        \ex\gll hl-aha \\
                {\textsc{f}}-{\textsc{dem}}:not\_visible \\
                \glt `that.\textsc{f} (not visible)'
    \end{xlist}
\z

\ea\label{ex:sylllh:2:ijw}
    Iyojwa’aja’ \citep[113, 122, 169]{ND09}
    \begin{xlist}
        \ex\gll hi-kʲóʔ \\
                3.{\textsc{poss}}-hand \\
                \glt `his/her hand'
        \ex\gll hi-tʲét-e \\
                2.{\textsc{act}}-throw-\APPL \\
                \glt `you throw it for her/him'
        \ex\gll ha-na \\
                {\textsc{f}}-{\textsc{dem}}:outside\_hands'\_reach \\
                \glt `this.\textsc{f} (outside one’s hands’ reach)'
    \end{xlist}
\z

\ea\label{ex:sylllh:3:ijw}
    Iyojwa’aja’ \citep[156]{ND09}
    \begin{xlist}
        \ex\gll t-’ɔ́t\\
                3.{\textsc{poss}}-breast\\
                \glt `her/his breast'
        \ex\gll hit-’íjasaˀn\\
                2.{\textsc{act}}-teach\\
                \glt `you teach'
    \end{xlist}
\z

In Proto-Wichí, the third-person possessive and the second-person active prefixes surface as \intxt{*ɬ\mbox{-}} before vowels, as in \REF{ex:sylllh:1:whk}--\REF{ex:sylllh:1:lbw}, but as \intxt{*ɬ̩\mbox{-}} before supraglottal consonants, as in \REF{ex:sylllh:2:whk}--\REF{ex:sylllh:2:lbw}. If the stem starts with a glottal stop, the third-person possessive prefix coalesces with it as \intxt{*t\mbox{-}’} and the second-person active prefix as \intxt{*ɬ̩t\mbox{-}’} \REF{ex:sylllh:3:whk}. In the contemporary Wichí dialects, \sound{PW}{*ɬ̩} is variously reflected as \intxt{la}, \intxt{le}, or \intxt{ha} (see \sectref{wi-syll-lh}). The feminine prefix in demonstratives is not preserved in Wichí.

\ea\label{ex:sylllh:1:whk}
    ’Weenhayek \citep[234, 550]{KC16}
    \begin{xlist}
        \ex\gll ɬ-áwoʔ\\
                3.{\textsc{poss}}-flower\\
                \glt `its flower'
        \ex\gll ɬ-ok \\
                2.{\textsc{act}}-say \\
                \glt `you say'
    \end{xlist}
\z

\ea\label{ex:sylllh:1:lbw}
    Lower Bermejeño Wichí \citep[166, 226]{VN14}
    \begin{xlist}
        \ex\gll ɬ-omet\\
                3.{\textsc{poss}}-word\\
                \glt `her/his word'
        \ex\gll ɬ-otaχ \\
                2.{\textsc{act}}-be\_fat \\
                \glt `you are fat'
    \end{xlist}
\z

\ea\label{ex:sylllh:2:whk}
    ’Weenhayek \citep[220, 438]{KC16}
    \begin{xlist}
        \ex\gll la-p’ot\\
                3.{\textsc{poss}}-lid\\
                \glt `its lid'
        \ex\gll la-t-’ek \\
                2.{\textsc{act}}-\textsc{t}-eat \\
                \glt `you eat'
    \end{xlist}
\z

\newpage
\ea\label{ex:sylllh:2:lbw}
    Lower Bermejeño Wichí \citep[163, 237]{VN14}
    \begin{xlist}
        \ex\gll la-n̥es\\
                3.{\textsc{poss}}-nose\\
                \glt `her/his/its nose'
        \ex\gll la-ta-qatay \\
                2.{\textsc{act}}-\textsc{t}-cook \\
                \glt `you cook'
    \end{xlist}
\z

\ea\label{ex:sylllh:3:whk}
    ’Weenhayek \citep[96, 123]{KC16}
    \begin{xlist}
        \ex\gll t-’áteʔ\\
                3.{\textsc{poss}}-breast\\
                \glt `her breast'
        \ex\gll lat-’éˀl \\
                2.{\textsc{act}}-be\_tired \\
                \glt `you are tired'
    \end{xlist}
\z

The allomorphs of the 2.{\textsc{act}} prefix before a \intxt{ʔ}\mbox{-}initial stem in Chorote (\wordng{Ijw/Mk}{hit\mbox{-}’…} < \wordng{PCh}{*hᵊt\mbox{-}’…}) and ’Weenhayek (\wordng{’Wk}{lat-’…} < \wordng{PCh}{*ɬ̩t\mbox{-}’…}) likely result from a morphological innovation whereby the inherited reflex \intxt{*t\mbox{-}’…} was augmented by \intxt{*ɬ̩\mbox{-}}, the allomorph of the same morpheme found in consonant-initial stems.
  
\subsection{Syllabic \intxt{*n}}\label{syllabic-n}

The reconstruction of a syllabic \intxt{*n} for Proto-Mataguayan remains rather tentative. The first piece of evidence comes from the allomorphy patterns of several homophonous prefixes.

\begin{exe}
    \ex \thirr
    \ex \imppssr
    \ex \secrls
\end{exe}

The 3.A/S\textsubscript{A}.{\textsc{irr}} and indefinite possessor prefixes both surface as \intxt{n\mbox{-}} before vowel-initial stems in all contemporary Mataguayan languages (except Iyojwa'aja' and Manjui), but a moraic allomorph is found before supraglottal consonants (\wordng{Mk}{ne\mbox{-}}; \wordng{Ni}{na\mbox{-}}; \wordng{I'w}{in\mbox{-} \recind n̩\mbox{-}}; \wordng{’Wk}{ní\mbox{-}}, \wordng{LB}{ni\mbox{-}} <~\wordng{PW}{*ní\mbox{-}}). The 2.P/S\textsubscript{P}.{\textsc{rls}} follows a similar pattern, except that in Maká the prefix was augmented by the element \intxt{ɬe\mbox{-} / ɬa\mbox{-} / ɬo\mbox{-}} and is never moraic. At least in Chorote and 'Weenhayek, the prefixes in question fuse with the initial glottal stop of stems that start with a \intxt{ʔ} as \intxt{ˀn}.

The following examples show \sound{Mk}{n\mbox{-}} occurring before vowel-initial \REF{ex:sylln:1:mak} and consonant-initial \REF{ex:sylln:2:mak} stems.

\ea\label{ex:sylln:1:mak}
    Maká \citep[90--91, 147, fn. 41]{AG94}
    \begin{xlist}
        \ex\gll n-aqfinet\\
                {\textsc{gnr}}-pestle\\
                \glt `pestle'
        \ex\gll n-ija\\
                3.A/S\textsubscript{A}.{\textsc{irr}}-drink\\
                \glt `(that) s/he drink'
        \ex\gll n-ek'uwet\\
                3.A/S\textsubscript{A}.{\textsc{irr}}-get\_drunk\\
                \glt `(that) s/he get drunk'
    \end{xlist}
\z

\ea\label{ex:sylln:2:mak}
    Maká \citep[85--86, 96]{AG94}
    \begin{xlist}
        \ex\gll ne-tux\\
                3.A/S\textsubscript{A}.{\textsc{irr}}-eat.{\textsc{tr}}\\
                \glt `(that) s/he eat it'
        \ex\gll no-t-otoj\\
                3.A/S\textsubscript{A}.{\textsc{irr}}-3.{\textsc{intr}}-dance\\
                \glt `(that) s/he dance'
        \ex\gll na-wanqa\\
                3.A/S\textsubscript{A}.{\textsc{irr}}-wash\_hands\\
                \glt `(that) s/he wash their hands'
    \end{xlist}
\z

The following examples from Nivaĉle show the allomorph \sound{Ni}{n\mbox{-}} occurring before vowel\mbox{-}initial (or \intxt{ʔ}\mbox{-}initial) stems \REF{ex:sylln:1:niv} and the allomorph \intxt{na\mbox{-}} preceding stems that begin with supraglottal consonants \REF{ex:sylln:2:niv}.\footnote{Even before consonants, the  3.A/S\textsubscript{A}.{\textsc{irr}} prefix can surface as \intxt{n}; in this case it syllabifies as a coda of the irrealis conjunction \intxt{kaʔ}.}

\ea\label{ex:sylln:1:niv}
    Nivaĉle \citep[159, 256, 414]{LC20}
    \begin{xlist}
        \ex\gll n-ʔaˀkɸij~\phonetic{nakˈɸiː}\\
                {\textsc{gnr}}-shoe\\
                \glt `shoe'
        \ex\gll n-uɬåx\\
                2.P/S\textsubscript{P}.{\textsc{rls}}-be\_tired\\
                \glt `you are tired'
        \ex\gll n-åk\\
                3.A/S\textsubscript{A}.{\textsc{irr}}-go\\
                \glt `(that) s/he go'
    \end{xlist}
\z

\ea\label{ex:sylln:2:niv}
    Nivaĉle \citep[255, 527]{LC20}
    \begin{xlist}
        \ex\gll na-pånt'ax\\
                2.P/S\textsubscript{P}.{\textsc{rls}}-jump\_well\\
                \glt `you can jump high'
        \ex\gll na-n-tʃaˀx\\
                3.A/S\textsubscript{A}.{\textsc{irr-cisl}}-carry\\
                \glt `(that) s/he bring'
    \end{xlist}
\z

Of the Chorote varieties, Iyo'awujwa' is the one that best preserves the archaic allomorphy patterns. The following examples show the allomorph \sound{I'w}{n\mbox{-}} occurring before vowel\mbox{-}initial stems \REF{ex:sylln:1:i'w}, \sound{I'w}{ˀn\mbox{-}} before \intxt{ʔ}\mbox{-}initial stems \REF{ex:sylln:1b:i'w}, and the allomorph \intxt{in\mbox{-} \recind n̩\mbox{-}} preceding stems that begin with supraglottal consonants \REF{ex:sylln:2:i'w}, with the alveolar nasal assimilating to \intxt{m} before the labial stop \intxt{p}.\footnote{Other Chorote varieties have innovated in that the moraic allomorph \intxt{ʔin\mbox{-}} is now used there before vowel-initial stems. With \intxt{ʔ\mbox{-}}initial stems, however, one finds the non-moraic allomorph of the indefinite possessor prefix and, in some cases, of the 2.P/S\textsubscript{P}.{\textsc{rls}} and 3.A/S\textsubscript{A}.{\textsc{irr}} prefixes both in Iyojwa'aja' and Manjui.} The examples below are mostly from \citet{AG83}, but we have altered her transcriptions in order to match our conventions. \REF{i'w-youhiccup} and \REF{ex:sylln:1b:i'w} are from Carol’s field data; note that \citet[77]{AG83} mistranscribes \sound{Iyo'awujwa'}{ˀn} as \intxt{n} (\wordnl{nóxteleʔ}{heart}, \wordnl{nafʷés}{body}).

\ea\label{ex:sylln:1:i'w}
    Iyo'awujwa' \citep[77]{AG83}
    \begin{xlist}
        \ex\gll n-ɔ́p'aleʔ\label{i'w-youhiccup}\\
                2.P/S\textsubscript{P}.{\textsc{rls}}-hiccup\\
                \glt `you hiccup'
        \ex\gll n-ɛ́ˀleʔ\\
                2.P/S\textsubscript{P}.{\textsc{rls}}-be\_dry\\
                \glt `you are dry'
        \ex\gll n-átah\\
                2.P/S\textsubscript{P}.{\textsc{rls}}-be\_fat\\
                \glt `you are fat'
    \end{xlist}
\z

\ea\label{ex:sylln:1b:i'w}
    Iyo'awujwa'
    \begin{xlist}
        \ex\gll n-ʔɔ́htele~[ˈˀnɔhteleʔ]\\
                {\textsc{gnr}}-heart\\
                \glt `heart'        
        \ex\gll n-ʔahwís~[ˀnaˈhwɪs]\\
                {\textsc{gnr}}-body\\
                \glt `body'
    \end{xlist}
\z

\ea\label{ex:sylln:2:i'w}
    Iyo'awujwa' \citep[69, 77]{AG83}
    \begin{xlist}
        \ex\gll ʔin-tɔ́we\\
                {\textsc{gnr}}-belly\\
                \glt `belly'
        \ex\gll n̩-tɔ́kʲoʔ\\
                {\textsc{gnr}}-face\\
                \glt `face'
        \ex\gll m̩-pʊ́xs-ej\\
                {\textsc{gnr}}-beard-\PL\\
                \glt `beards'
        \ex\gll ʔim-páxsat\\
                {\textsc{gnr}}-lip\\
                \glt `lip'
        \ex\gll ʔin-káhej\\
                2.P/S\textsubscript{P}.{\textsc{rls}}-be\_rich\\
                \glt `you are rich'
        \ex\gll ʔin-tɔ́jʔ\\
                2.P/S\textsubscript{P}.{\textsc{rls}}-be\_tall\\
                \glt `you are tall'
        \ex\gll ʔin-hwɪ́hlʲen\\
                2.P/S\textsubscript{P}.{\textsc{rls}}-dream\\
                \glt `you dream'
    \end{xlist}
\z

In Wichí, the 3.A/S\textsubscript{A}.{\textsc{irr}} prefix is reflected as \sound{PW}{*n\mbox{-}} before vowel-initial stems, as \sound{PW}{*ní\mbox{-}} before stems that start with a supraglottal consonant, and as \sound{PW}{*ˀn\mbox{-}} before \intxt{ʔ}\mbox{-}initial stems.

\newpage
\ea\label{ex:sylln:whk}
    'Weenhayek \citep[125, 544]{KC16}
    \begin{xlist}
        \ex \gll n(í)-ekʷ-a~\phonetic{nẽːˈk(ʷ)aʔ}\\
                3.\textsc{neg.irr}-go-\textsc{neg.irr}\\
                \glt `lest s/he go'
        \ex \gll n(í)-t(a)-áhuj-a~\phonetic{nĩːtaˈhũjaʔ}\\
                3.\textsc{neg.irr-t}-speak-\textsc{neg.irr}\\
                \glt `lest s/he speak'
        \ex \gll n(í)-ʔip-a~\phonetic{ˀnĩːˈpaʔ}\\
                3.\textsc{neg.irr}-cry-\textsc{neg.irr}\\
                \glt `lest s/he cry'
    \end{xlist}
\z

Finally, syllabic \intxt{*n} may have also apparently occurred as part of roots, as in the following example.

\begin{exe}
    \ex \spoon
\end{exe}

\subsection{Syllabic \intxt{*t}}\label{syllabic-t}
Syllabic \intxt{*t} is reconstructed for one morpheme, the T-class third-person prefix \intxt{*t\intxt{-}} (in Nivaĉle and Wichí, its reflex is also found in some other inflected forms and is best analyzed as a T-class marker rather than a person index). Before vowels, its surfaces as regular (non-syllabic) \intxt{*t\mbox{-}} in Proto-Mataguayan and in all contemporary languages (this is also the allomorph used in Chorote with \intxt{h}\mbox{-}initial stems). Before supraglottal consonants, it has a moraic allomorph in almost all contemporary languages (which we reconstruct as \sound{PM}{*t̩\mbox{-})}, unless it can syllabify as a coda to a preceding morpheme. Nivaĉle is an exception in that the moraic allomorph shows up only before \intxt{tʃ(')}, but not before other consonants. In stems that start with a glottal stop, \sound{PM}{*t\mbox{-}ʔ\mbox{-}} coalesces into \intxt{*t’\mbox{-}}.

\ea\label{ex:syllt:mak}
Maká \citep[118, 121, 244, 329]{AG99}\\
    \begin{xlist}
        \ex \gll t-altsaj\\
             3.\textsc{t}-beget\\
             \glt `she begets'
        \ex \gll te-lixtsij\\
             3.\textsc{t}-sing\\
             \glt `s/he sings'
        \ex \gll ne-t-lixtsij\\
             3.A/S\textsubscript{A}.{\textsc{irr}}-3.\textsc{t}-sing\\
             \glt `s/he snores'
        \ex \gll t-’an\\
             3.\textsc{t}-win\\
             \glt `s/he wins'
    \end{xlist}
\z

\ea\label{ex:syllt:niv}
Nivaĉle \citep[248, 266, 270, 282]{JS16}
    \begin{xlist}
        \ex \gll t-itsin\\
             3.\textsc{t}-get\_cured\\
             \glt `s/he gets cured'
        \ex \gll t-k͡låˀj\\
             3.\textsc{t}-play\\
             \glt `s/he plays'
        \ex \gll ta-tʃ'an\\
             3.\textsc{t}-obey\\
             \glt `s/he obeys'
        \ex \gll ɬa-t-tʃ'an\\
             2.\textsc{act-t}-obey\\
             \glt `you obey'
        \ex \gll ∅-t-’akut\\
             3-\textsc{t}-steal\\
             \glt `s/he steals'
    \end{xlist}
\z

\ea\label{ex:syllt:ijw}
Iyojwa'aja' \citep{JC14a}
    \begin{xlist}
        \ex \gll t-ámtiʔ\\
             3.\textsc{t.rls}-speak\\
             \glt `s/he speaks'
        \ex \gll ti-més\\
             3.\textsc{t.rls}-be\_two\\
             \glt `they are two'
        \ex \gll ti-lʲákiˀn\\
             3.\textsc{t.rls}-play/dance\\
             \glt `s/he plays/dances'
        \ex \gll ta-kásit\\
             3.\textsc{t.rls}-stand\\
             \glt `s/he stands'
        \ex \gll t-'ɔ́siʔ\\
             3.\textsc{t.rls}-run\\
             \glt `s/he runs'
    \end{xlist}
\z

\ea\label{ex:syllt:i'w}
Iyo'awujwa' \citep[75]{AG83}
    \begin{xlist}
        \ex \gll t-ákihnan\\
             3.\textsc{t.rls}-hunt\\
             \glt `s/he hunts'
        \ex \gll ti-lákʲen\\
             3.\textsc{t.rls}-play\\
             \glt `s/he plays'
        \ex \gll te-kénisʲen\\
             3.\textsc{t.rls}-sing\\
             \glt `s/he sings'
    \end{xlist}
\z

\ea\label{ex:syllt:mj}
Manjui \citep{JC18}
    \begin{xlist}
        \ex \gll t-án\\
             3.\textsc{t.rls}-shout\\
             \glt `s/he shouts'
        \ex \gll t-hʊ́jʔ\\
             3.\textsc{t.rls}-return\_home\\
             \glt `s/he returns home'
        \ex \gll ti-khán\\
             3.\textsc{t.rls}-dig\\
             \glt `s/he digs'
        \ex \gll t-'as\\
             3.\textsc{t.rls}-step\\
             \glt `s/he steps'
    \end{xlist}
\z

\ea\label{ex:syllt:whk}
'Weenhayek \citep[375, 426, 431]{KC16}
    \begin{xlist}
        \ex \gll ∅-t-útkʲejʔ\\
             3-\textsc{t}-sow\\
             \glt `s/he sows'
        \ex \gll ∅-ta-qásit\\
             3-\textsc{t}-stand\_up\\
             \glt `s/he stands up'
        \ex \gll ʔõ-t-qásit\\
             1\textsc{sg-t}-stand\_up\\
             \glt `I stand up'
        \ex \gll ∅-t-’áɬ\\
             3-\textsc{t}-ask\\
             \glt `s/he asks'
    \end{xlist}
\z

\ea\label{ex:syllt:lbw}
Lower Bermejeño Wichí \citep[239--240]{VN14}
    \begin{xlist}
        \ex \gll ∅-t-afʷɬi\\
             3-\textsc{t}-cry\\
             \glt `s/he cries'
        \ex \gll ∅-ta-qatin\\
             3-\textsc{t}-jump\\
             \glt `s/he jumps'
        \ex \gll n̩-t-qatin\\
             1\textsc{sg-t}-jump\\
             \glt `I jump'
        \ex \gll ∅-t-’on\\
             3-\textsc{t}-shout\\
             \glt `s/he shouts'
    \end{xlist}
\z

In Chorote and Wichí, there are prefixes of the same shape that present an identical allomorphy pattern. In Chorote, \intxt{t\mbox{-} / ti\mbox{-} / t'\mbox{-}} (in Iyojwa'aja' also \intxt{ta\mbox{-}} before /k/) is used in the impersonal forms of verbs. In Wichí, the prefix \intxt{t\mbox{-} / ta\mbox{-}} is found in a closed set of nouns that denote body parts \citep[164–165]{VN14}. It is, however, unclear whether they are related to the 3.\textsc{t} prefix of Proto-Mataguayan and whether they represent retentions or innovations.

\subsection{Syllabic consonants as opposed to consonant clusters}

An anonymous reviewer inquires whether what we reconstruct as syllabic consonants could be replaced with plain consonants as first members of consonant clusters. In this regard, it should be noted that the reflexes of syllabic consonants often contrast with those of word-initial non-syllabic consonants followed by another consonant.

For examples, \sound{PM}{*tk} and \intxt{*t̩k} have distinct reflexes in varieties such as ’Weenhayek or Vejoz. \sound{PM}{*tk} is reflected as \sound{’Wk}{kʲ} word-initially, as in \wordng{PM}{*tkénaX₁₂ \recind *tkä́naX₁₂} > \word{’Wk}{kʲénax}{mountain, hill}. Conversely, \sound{PM}{*t̩k} is reflected as \sound{’Wk}{takʲ}, as in \wordng{PM}{*t̩\mbox{-}kúm$=$ex} > \word{’Wk}{ta\mbox{-}ˈkʲúm$=$ex}{s/he grabs it}.

Similarly, the reflexes of \sound{PM}{*tl} contrast with those of \sound{PM}{*t̩l}. Word-initially the Proto-Mataguayan sequence \intxt{*tl} evolves into \sound{Ni}{tak͡l} and \sound{’Wk}{til}, as in \wordng{PM}{*tlúˀk} > \wordng{Ni}{tak͡luˀk}, \word{’Wk}{tilúk}{blind}. By contrast, when \sound{PM}{*t̩} combines with an \intxt{*l}\mbox{-}initial verbal root, one finds the reflexes \sound{Ni}{tk͡l} (with loss of syllabicity), as in \word{Ni}{t\mbox{-}k͡låˀj}{s/he dances}, and \sound{’Wk}{tal}, as in \word{’Wk}{ta\mbox{-}líkʲ’iʔ}{in good condition, not shabby}. Unfortunately, we do not know of any \intxt{*l}\mbox{-}initial T\mbox{-}class verb reconstructible to Proto-Mataguayan.

The reconstruction of \sound{PM}{*n̩} and \intxt{*ɬ̩} is less questionable than that of \sound{PM}{*t̩}, since these sounds are preserved even synchronically in some cases, as in \word{I’w}{n̩\mbox{-}tókʲoʔ}{face} or \word{Ni}{ɬ̩\mbox{-}k͡líˀʃ}{his/her word}.
