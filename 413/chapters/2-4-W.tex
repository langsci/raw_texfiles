\chapter{Wichí} \label{wi}

This chapter deals with the historical phonology of Wichí [wich1261] (\sectref{intro-wi}). \sectref{pm-to-wi} discusses the development of PM~consonants, vowels, and prosody from the PM stage to Proto-Wichí. \sectref{wi-dialects} is concerned with the diversification of the Wichí varieties.

For the ’Weenhayek variety, we rely on \citet{KC16}'s dictionary as well as \cits{JAA-KC-14} grammatical description and \cits{KC94,KCnd} phonological descriptions. For Vejoz, we have consulted the vocabularies by \citet{VU74} and \citet{MG-MELO15}. For the Lower Bermejeño variety, we rely on \citet{VN14}'s grammar and on \cits{JB09} vocabulary as a secondary source; in addition, many flora and avifauna terms have been extracted from \citet{CS08} and \citet{CS-FL-PR-VN13}. \citet{MS14} is a useful source on plant names in the Southeastern variety as spoken in Salta.

The consonantal inventory we assume for Proto-Wichí is given in \tabref{pw-inv-cons}. The vocalic inventory we assume for Proto-Wichí includes six or seven vowels, */i~(ɪ)~e~a~å~o~u/.

\begin{table}
\caption{Proto-Wichí consonants}
\label{pw-inv-cons}
\fittable{
 \begin{tabular}{rccccccc}
  \lsptoprule
            & labial & dental & alveolar & palatal & dorsal & dorsal labialized & glottal\\\midrule
  plain stops & *p & *t & *ts & *kʲ & *q *[q~\recind k] & *kʷ *[kʷ~\recind qʷ] & *ʔ\\
  ejective stops & *p’ & *t’ & *ts’ & *kʲ’ & *q’ & *kʷ’ & \\
  fricatives & & *ɬ & *s & & *χ *[χ~\recind x] & *xʷ & *h \\
  plain approximants & *w & *l & & *j & & &\\
  gl. approximants & *ˀw & *ˀl & & *ˀj & & &\\
  plain nasals & *m & *n & & & & &\\
  glottalized nasals & *ˀm & *ˀn & & & & &\\
  \lspbottomrule
 \end{tabular}
 }
\end{table}

Individual Wichí lects depart from this scheme in a number of ways. Regarding the consonant system, in a number of (sub)dialects \sound{PW}{*kʲ} and \intxt{*kʲ’} are replaced with /tʃ/ and /tʃ’/, whereas \sound{PW}{*xʷ} is often replaced with /fʷ/ or /hʷ/. Contemporary Wichí lects also have aspirated stops, voiceless approximants, and voiceless nasals, though their phonological status is debated. No contemporary Wichí lect is known to retain the hypothetical phoneme \sound{PW}{*ɪ}, and many varieties also lack \intxt{*å}.

\section{From Proto-Mataguayan to Proto-Wichí} \label{pm-to-wi}

This section deals with the development of PM~consonants (\sectref{wi-cons}), vowels (\sectref{wi-vow}), and prosody (\sectref{wi-prosody}) from the Proto-Mataguayan stage to Proto-Wichí. \sectref{wi-watkins} presents evidence for the regular operation of Watkins' Law in the historical development of Proto-Wichí, whereby forms with third-person inflection were reanalyzed as uninflected forms.

\subsection{Consonants} \label{wi-cons}

The historical development of the PM consonants in Wichí includes the following sound changes: the sound change \sound{PM}{*ɸ} > \sound{PW}{*xʷ} (\sectref{wi-f}), the palatalization of \sound{PM}{*k(’)} to \sound{PW}{*kʲ(’)} in the onset position and the labialization of \sound{PM}{*k} to \sound{PW}{*kʷ} in the coda position after a back vowel (\sectref{wi-q-k}), the merger of the fricatives \sound{PM}{*x} and \intxt{*χ} > \sound{PW}{*χ} (in codas, except that \sound{PM}{*oχ}, \intxt{*uχ}, \intxt{*ux} > \sound{PW}{*oxʷ}, \intxt{*uxʷ}, \intxt{*uxʷ}) or \intxt{*h} (in onsets, merging with PM~\intxt{*h}) (\sectref{wi-jj-j-h}), the deaffrication of \sound{PM}{*ts} to \sound{PW}{*s} in the coda position (\sectref{wi-ts-s}), the loss of contrastive glottalization in non-nasal codas (\sectref{wi-deglottalization-non-nasal-codas}), the fortition of glottalized fricatives (\sectref{wi-glott-fric}), the change of word-initial \sound{PM}{*ji\mbox{-}} to \sound{PW}{*ʔi\mbox{-}} preceding non-dorsal consonants (\sectref{wi-yi}), the sound change PM~*[ʔ] > \sound{PW}{*h} in onset of syllables followed by a syllable with a glottalized consonant (\sectref{wi-glottal-dissim}), the deglottalization of glottalized onsets of syllables followed by a syllable with a glottalized consonant (\sectref{wi-glot-dissim}), the loss of word-final \sound{PM}{*h} following syllables with a glottalized obstruent (\sectref{wi-h-loss}), the insertion of a word-final \sound{PW}{*h} following an accented vowel (\sectref{wi-h-insertion}), the change of word-final \sound{PM}{*\mbox{-}nV} to \sound{PW}{*\mbox{-}ˀnVh} (\sectref{wi-nv-nvh}), the change of word-final \sound{PM}{*(ˀ)l} to \sound{PW}{*lʰ} (\sectref{wi-l-lh}), the loss of word-final \sound{PM}{*ʔ} in posttonic syllables (\sectref{wi-posttonic-deglottalization}), and the change of syllabic \sound{PM}{*n̩}, \intxt{*t̩} to \sound{PW}{*ni}, \intxt{*ta} (\sectref{wi-t-n}). The evolution of Proto-Mataguayan consonant clusters is described in \sectref{wi-consonant-dorsal} (for clusters whose second element is a guttural fricative) and \sectref{wi-other-clusters} (for all other clusters).

\subsubsection{\sound{PM}{*ɸ}}\label{wi-f}

Proto-Mataguayan \intxt{*ɸ} yielded \sound{PW}{*xʷ} (in the contemporary varieties of Wichí, the pronunciation of its default reflex varies from [xʷ] to [fʷ], as detailed in \sectref{wi-xw}) in both onsets and codas. For a representative sample of examples, see \sectref{proto-f}.

Two cognate sets show irregular reflexes of \sound{PM}{*ɸ} in Wichí: \intxt{*xʷ \recvar *w} in \REF{fw:flyv}, \intxt{*p} in \REF{fp:suckle}.

\begin{exe}
    \ex \flyv\label{fw:flyv}
    \ex \suckb\label{fp:suckle}
\end{exe}

At least in the latter example, the reflex \intxt{*p} may turn out to be the regular outcome of the preglottalized coda \sound{PM}{*ˀɸ} (see \sectref{glott-codas} on the preglottalized codas of Proto-Mataguayan). We have not identified any other example of \sound{PM}{*ˀɸ} in our comparative corpus. Note that the causative of \word{PW}{*tip}{to suckle} is \word{PW}{*[ʔi]tíxʷ\mbox{-}qat}{to breastfeed}, with a regular reflex of \sound{*PM}{*ɸ}.

\subsubsection{\sound{PM}{*q}, \intxt{*k}, and their glottalized counterparts}\label{wi-q-k}

This subsection describes the evolution of \sound{PM}{*q} and \intxt{*k} (and their glottalized counterparts) in Wichí. Already in Proto-Mataguayan, the distribution of these segments appears to have been subject to some restrictions: \sound{PM}{*q} is not reconstructed following non-low vowels (that is, the sequences \intxt{*uq}, \intxt{*oq}, \intxt{*eq}, \intxt{*iq} are not known to have been possible in~PM), whereas \intxt{*k} was apparently banned following \sound{PM}{*a}. Both \sound{PM}{*q} and \intxt{*k} could occur stem-initially (as in \wordnl{*qatiˀts}{star} vs. \wordnl{*\mbox{-}kåˀs}{tail}) and following an \intxt{*å} (as in \wordnl{*tsåhǻq}{chajá bird} vs. \wordnl{*níjåk}{cord, rope}); data regarding \intxt{*q} and \intxt{*k} in post-consonantal position are scarce. Stem-final \intxt{*k} could also alternate with \intxt{*h} in plural formation, as in \intxt{*\mbox{-}mǻˀk}, plural \wordnl{*\mbox{-}mhǻ\mbox{-}j}{powder, flour} (\sectref{velar-weakening}).

In Proto-Wichí, \sound{PM}{*q} and \intxt{*q’} remained intact in all positions.

\sloppy
\begin{exe}
    \ex \food
    \ex \elbow
    \ex \alienable
    \ex \medicine
    \ex \distrust
    \ex \leg
    \ex \standv
    \ex \starn
    \ex \limpkin
    \ex \belt
    \ex \yellowv
    \ex \fishwithhook
    \ex \costume
    \ex \noden
    \ex \soul
    \ex \wildcat
    \ex \cardon
    \ex \chaja
    \ex \cord
    \ex \wildhoney
\end{exe}

By contrast, \sound{PM}{*k} changed in most positions. In onsets, it became palatalized, yielding \sound{PW}{*kʲ}. Likewise, \sound{PM}{*k’} yielded \sound{PW}{*kʲ’}. This sound change is shared with the contemporary Chorote varieties, though not with Proto-Chorote (see \sectref{ch-k} and \sectref{ch-k'}).

\begin{exe}
    \ex \grove
    \ex \takeaway
    \ex \lizard
    \ex \testicle
    \ex \tail
    \ex \redv
    \ex \fall
    \ex \torn
    \ex \grandchild
    \ex \monkparakeet
    \ex \redquebracho
    \ex \sendv
    \ex \feminine
    \ex \elderbro
    \ex \eldersis
    \ex \heavyv
    \ex \locustcw
    \ex \hole
    \ex \tortoise
    \ex \whitealgarrobof
    \ex \answer
    \ex \grabwork
    \ex \meet
    \ex \chaguark
    \ex \sweat
    \ex \leniosapl
    \ex \cactus
    \ex \kidney
    \ex \armadillo
    \ex \arrowkaxe
    \ex \spouse
    \ex \stretchout
    \ex \dividev
    \ex \youngerbro
    \ex \youngersis
    \ex \bottomn
    \ex \hornclub
    \ex \thorncutjan
    \ex \oldn
    \ex \flu
    \ex \newadj
    \ex \medicine
    \ex \distrust
    \ex \mesh
    \ex \precipice
    \ex \face
    \ex \eyebrow
    \ex \guayacan
    \ex \metal
    \ex \egg
\end{exe}

In intervocalic clusters composed of a \intxt{*k} and a guttural fricative, \sound{PM}{*k} failed to palatalize, possibly because it was still syllabified as a coda in that position when the sound change \sound{PM}{*k} > \sound{PW}{*kʲ} took place. The outcome is \sound{PW}{*kh}, reflected as \intxt{kʰ} in most contemporary varieties of Wichí.

\begin{exe}
    \ex \lapcalf
\end{exe}

In codas, \sound{PM}{*k} acquired labialization following back vowels, yielding \sound{PW}{*kʷ}. (Proto-Wichí also innovated \intxt{*kʷ} and \intxt{*kʷ’} in onsets from \sound{PM}{*kɸ} and \intxt{*kɸ’}, as in \wordng{PW}{*[j]ókʷaχ} < \word{PM}{[j]ékɸaˀx}{to bite}; see \sectref{wi-other-clusters}.)

\begin{exe}
    \ex \palm
    \ex \yicalhuk
    \ex \powder
    \ex \feces
    \ex \wildmanioc
    \ex \rope
    \ex \two
    \ex \fence
    \ex \river
    \ex \blind
    \ex \unclesg
    \ex \duraznillo
    \ex \silkfloss
    \ex \leniosasg
    \ex \badmood
    \ex \palosanto
    \ex \firewoodhuk
\end{exe}

Following front vowels, \sound{PM}{*k} kept its velar articulation in Wichí: \word{PM}{*[j]ik}{goes away}, \wordnl{*ɬ\mbox{-}xä́teˀk}{her/his head} > \wordng{PW}{*[j]i\phonetic{k}}, \intxt{*ɬ-éte\phonetic{k}}.\footnote{\citet[49]{VN14} reports that in the Lower Bermejeño subdialect of Southeastern Wichí /q/ surfaces as [k] in the coda position when preceded by a front vowel as well as in the onset position after a coronal consonant: [ˈjik] `goes away', [ɬeˈtek] `her/his head', [tetˈkal] `vine'. In the ’Weenhayek variety, too, /q/ surfaces as [k] in the coda position when preceded by a front vowel: [ˈjik] `goes away', [ˌɬeːˈtek] `her/his head', though [q] is found in onsets after coronal consonants: [laˌteːnˈqaç] `her/his songs' \citep[16--17]{KC94}. Since ’Weenhayek and Lower Bermejeño are on the opposite ends of the Wichí-speaking area (both geographically and linguistically), the allophony pattern whereby /q/ surfaces as [k] in codas following front vowels must be reconstructed for Proto-Wichí. \citet[25]{JT09-th} reports only the uvular realization for the Rivadavia subdialect of Southeastern Wichí, even after front vowels ([ˈjiq] `goes away', [ɬeˈteq] `her/his head'), which must be a local innovation.} However, synchronically in Proto-Wichí \intxt{*\phonetic{k}} does not contrast either with \intxt{*}/q/ (recall that \sound{PM}{*}/q/ > \sound{PW}{*}/q/ does not occur after front vowels other than \intxt{*}/a/) or with \intxt{*}/kʷ/ (which is also impossible following front vowels). In this book, we follow \cits{KC94} and \cits{VN14} analysis and represent all instances of \sound{PW}{*\phonetic{k}} as \intxt{q}.

\begin{exe}
    \ex \goaway
    \ex \mortar
    \ex \bilecw
    \ex \allrcpr
    \ex \headn
    \ex \temperance
    \ex \wildhoney
    \ex \eatvi
\end{exe}

The fact that \sound{PM}{*k} evolved differently in onsets and codas gave rise to synchronically active alternations in Wichí. As we have seen, following front vowels, stem-final \sound{PM}{*k} yielded \sound{PW}{*q} when syllabified as a coda, and \sound{PW}{*kʲ} when syllabified as an onset. The resulting alternation is still productive in varieties such as Lower Bermejeño Wichí \REF{ex:q-ch:lbw}, where /q/~[k] alternates with /tʃ/~[tʃ], at least if a front vowel precedes it (see also \cites[123]{MC09}).\footnote{In varieties such as 'Weenhayek, it fails to occur even after front vowels: \word{'Wk}{ʔi\mbox{-}wo\mbox{-}la\mbox{-}tén\mbox{-}ek\mbox{-}aʔ}{s/he performs her/his song}, \wordnl{j\intxt{-}ik(ʷ)-eh}{s/he goes for it} \citep[17]{KC94}.}

\ea\label{ex:q-ch:lbw}
Lower Bermejeño Wichí \citep{VN14}\\
    \begin{xlist}
        \ex\gll -teneq~\phonetic{-teˈnẽk}\\
                -song\\
                \glt `song'
        \ex\gll ʔi-wu-tenetʃ-a~\phonetic{ʔiˌwuˈtenẽtʃa}\\
                3.{\textsc{i}}-do-song-\textsc{incorp}\\
                \glt `s/he prays, praises'
        \ex\gll ∅-neq~\phonetic{ˈnẽk}\\
                3-walk\\
                \glt `s/he walks'
        \ex\gll ∅-netʃ-hen \phonetic{nẽˈtʃʰẽn}\\
                3-walk-\PL\\
                \glt `they walk'
        \ex\gll j-iq~\phonetic{ˈjɪk}\\
                3.{\textsc{i}}-go\_away\\
                \glt `s/he goes away'
        \ex\gll j-itʃ-hila~\phonetic{jɪˌtʃʰĩˈla}\\
                3.{\textsc{i}}-go\_away-\FUT\\
                \glt `s/he will go away'
        \ex\gll j-itʃ-hen~\phonetic{jɪˈtʃʰẽn}\\
                3.{\textsc{i}}-go\_away-\PL\\
                \glt `they go away'
        \ex\gll j-itʃ-hu~\phonetic{ˈjɪtʃʰũ}\\
                3.{\textsc{i}}-go\_away-\APPL\\
                \glt `s/he goes away from inside'
        \ex\gll t-ʔeq~\phonetic{ˈt’ek}\\
                3.{\textsc{t}}-eat\\
                \glt `s/he eats'
        \ex\gll ha-n̩-t-ʔetʃ-hi~\phonetic{hãˌn̩t’eˈtʃʰĩ}\\
                \NEG-1\SG-\textsc{t}-eat-\NEG\\
                \glt `I don't eat'
        \ex\gll ʔi-tseq=mathi~\phonetic{nˌˈtsekmatʰĩ}\\
                3.{\textsc{i}}-sew=\textsc{dp}\\
                \glt `s/he sewed'
        \ex\gll n̩-tsetʃ-eq pujelu~\phonetic{n̩ˌtseˈtʃek puˌjeˈlu}\\
                1\SG-sew-{\PTCP} skirt\\
                \glt `a skirt sewn by me'
    \end{xlist}
\z

\largerpage
Following front vowels, stem-final \sound{PM}{*k} yielded \sound{PW}{*kʷ} when syllabified as a coda, and \sound{PW}{*kʲ} when syllabified as an onset. The resulting alternation is still present in varieties such as Lower Bermejeño Wichí \REF{ex:kw-ch:lbw}, where /kʷ/~[kʷ] alternates with /tʃ/~[tʃ], and ’Weenhayek \REF{ex:kw-ch:whk}, where /kʷ/~[k] alternates with /kʲ/~[kʲ] at least in the suffix for woody plants (\wordng{PW}{*\mbox{-}ukʷ\plf{*\mbox{-}kʲu\mbox{-}jʰ}}).

\booltrue{listing}
\ea\label{ex:kw-ch:lbw}
Lower Bermejeño Wichí \citep[192]{VN14}\\
    \begin{xlist}
        \ex \intxt{fʷaʔaj-ekʷ} (\SG), \intxt{fʷaʔa-tʃe-j} (\PL)\gloss{algarrobo tree}
        \ex \intxt{tʃeɬj-ekʷ} (\SG), \intxt{tʃeɬ-tʃe-j} (\PL)\gloss{red quebracho tree}
        \ex \intxt{watʃaj-ekʷ} (\SG), \intxt{watʃa-tʃe-j} (\PL)\gloss{guayacán tree}
        \ex \intxt{tsuwaj-ekʷ} (\SG), \intxt{tsuwa-tʃe-j} (\PL)\gloss{kiscarolo tree}
        \ex \intxt{hoj-ekʷ} (\SG), \intxt{ho-tʃe-j} (\PL)\gloss{mistol tree}
    \end{xlist}
\z

\ea\label{ex:kw-ch:whk}
’Weenhayek \citep{KC16}\\
    \begin{xlist}
        \ex \intxt{xʷaʔáj-uk} (\SG), \intxt{xʷaʔá-kʲu-ç} (\PL)\gloss{algarrobo tree}
        \ex \intxt{tʃéɬj-uk} (\SG), \intxt{tʃéɬ-kʲu-ç} (\PL)\gloss{red quebracho tree}
    \end{xlist}
\z
\boolfalse{listing}

\subsubsection{\sound{PM}{*x}, \intxt{*χ}, and \intxt{*h}}\label{wi-jj-j-h}

This subsection describes the evolution of the Proto-Mataguayan ``guttural'' fricatives -- \sound{PM}{*x}, \intxt{*χ}, and \intxt{*h} -- in Wichí. In onsets, \sound{PM}{*x} and and \intxt{*h} fell together and yielded \sound{PW}{*h}, a glottal fricative notable for triggering automatic nasalization in the following vowel in all varieties of Wichí due to a rhinoglottophilia effect (not represented in our broad transcriptions; see \cites[13]{KC94}[51--52]{JT09-th}[41--42]{VN14}). It is likely that the merger in question had \intxt{*χ} as its intermediate stage, as discussed in Footnote \ref{prepwjj}. The examples below show the evolution of \sound{PM}{*x} and \intxt{*h} in simplex onsets. (When guttural fricatives occur as parts of complex onsets, they also yield \sound{PW}{*h} except after a fricative, where they are deleted; see \sectref{wi-consonant-dorsal} for examples and details.)

\begin{exe}
    \ex \mouth
    \ex \arrowkaxe
    \ex \chaja
    \ex \price
    \ex \jelayuk
    \ex \vrbpl
    \ex \grass
    \ex \gov
    \ex \palosanto
    \ex \nightnw
    \ex \sandyplace
    \ex \firewoodhuk
    \ex \temperance
    \ex \pushv
    \ex \caracara
    \ex \mistolf
    \ex \mistolt
\end{exe}

In codas, however, \sound{PM}{*x} and \intxt{*χ} never change to \sound{PW}{*h}. Instead, \sound{PM}{*x} and \intxt{*χ} typically merge as \sound{PW}{*χ} after unrounded vowels (note that \intxt{*å} is unrounded).

\begin{exe}
    \ex \fatv
    \ex \bite
    \ex \cutdown
    \ex \suncho
    \ex \takeaway
    \ex \hole
    \ex \dividev
    \ex \youngerbro
    \ex \barnowl
    \ex \oldn
    \ex \wash
    \ex \bow
    \ex \smelln
    \ex \pathn
    \ex \jabiru
    \ex \longv
    \ex \anteater
    \ex \pseudo
    \ex \abdcavity
    \ex \shoot
    \ex \carrysh
    \ex \dig
    \ex \smoke
    \ex \fullriver
    \ex \tsofatajf
    \ex \paloflojof
    \ex \burrow
    \ex \blackalgarrobof
    \ex \stagnant
    \ex \rhea
    \ex \night
    \ex \tuscaf
    \ex \caracara
    \ex \peccary
    \ex \mistolf
    \ex \argentineboa
    \ex \chaguara
    \ex \wildbean
    \ex \skin
    \ex \firei
    \ex \puma
\end{exe}

After \sound{PW}{*u}, \sound{PM}{*x} and \intxt{*χ} merge as \sound{PW}{*xʷ} in the coda position.

\begin{exe}
    \ex \centipede
    \ex \finger
    \ex \sweat
    \ex \eatvt
    \ex \largefat
\end{exe}

The contrast between \sound{PM}{*x} and \intxt{*χ} is maintained after \sound{PW}{*o}: \sound{PM}{*χ} labializes to \sound{PW}{*xʷ} in that environment, whereas \sound{PM}{*x} changes to \sound{PW}{*χ}.

\begin{exe}
    \ex \najendup
    \ex \deep
    \ex \far
    \ex \aunt
\end{exe}

As for \sound{PM}{*h} in the coda position, it is usually retained as \sound{PW}{*h} > \sound{’Weenhayek}{h} (except that it is lost if there is a glottalized obstruent in the onset of the same syllable, as discussed in \sectref{wi-h-loss}).

\begin{exe}
    \ex \companion
    \ex \lizard
    \ex \spouse
    \ex \goimp
    \ex \frog
    \ex \fox
    \ex \moon
    \ex \lessergrison
\end{exe}

The fact that \sound{PM}{*x/*χ} yielded \sound{PW}{*h} in onsets but not in codas gave rise to synchronically active alternations in Wichí, as shown below (see also \cites[21]{KC94}).

\booltrue{listing}
\ea\label{ex:j-h:lbw}
Lower Bermejeño Wichí \citep[191]{VN14}\\
    \begin{xlist}
        \ex \intxt{t-’oχ} (\SG), \intxt{t-’oh-es} (\PL)\gloss{its skin}
        \ex \intxt{nisoχ} (\SG), \intxt{nisoh-es} (\PL)\gloss{shoe}
    \end{xlist}
\z

\newpage
\ea\label{ex:j-h:riv}
Rivadavia Wichí \citep[44]{JT09-th}\\
    \begin{xlist}
        \ex \intxt{\mbox{-}te-t-’ɔx} (\SG), \intxt{\mbox{-}te-t-’ɔh-es} (\PL)\gloss{eyelid}
        \ex \intxt{nisɔx} (\SG), \intxt{nisɔh-es} (\PL)\gloss{shoe}
    \end{xlist}
\z

\ea\label{ex:j-h:whk}
’Weenhayek \citep[95, 271]{KC16}\\
    \begin{xlist}
        \ex \intxt{t-’åx} (\SG), \intxt{t-’åh-és} (\PL)\gloss{skin}
        \ex \intxt{nísåx} (\SG), \intxt{nísåh-es} (\PL)\gloss{shoe}
    \end{xlist}
\z
\boolfalse{listing}

Synchronically, \sound{PW}{*χ} can occur in the onset position as a result of evolution of \sound{PM}{*xχ} (possibly also \sound{PM}{*χx}, \sound{PM}{*xx}), as will be shown in \sectref{wi-consonant-dorsal}.

\subsubsection{Deaffrication of \sound{PM}{*ts} > \intxt{*s} in codas}\label{wi-ts-s}

As discussed in \sectref{proto-ts}, the occurrence of \intxt{ts} is synchronically limited to the onset position in Wichí (\cites[15]{KC94}[42]{JT09-th}[50]{VN14}). This restriction arose as a result of a diachronic deaffrication of \sound{PM}{*ts} > \sound{PW}{*s} in codas (shared with Nivaĉle and possibly Chorote).

\begin{exe}
    \ex \rootn
    \ex \dew
    \ex \offspring
    \ex \basetrunk
    \ex \trunk
    \ex \plits
    \ex \starn
\end{exe}

In some etyma, the erstwhile presence of an affricate in certain forms is suggested by the synchronically active alternations in Wichí. In the plural forms given below, \intxt{ts} is syllabified as an onset and thus fails to deaffricate, whereas the respective singular forms show \intxt{s} in its place.

\booltrue{listing}
\ea\label{ex:s-ts:lbw}
Lower Bermejeño Wichí \citep[191]{VN14}\\
    \begin{xlist}
        \ex \intxt{qates} (\SG), \intxt{qatets-eɬ} (\PL)\gloss{star}
        \ex \intxt{la-tes} (\SG), \intxt{la-tets-eɬ} (\PL)\gloss{its trunk}
    \end{xlist}
\z

\ea\label{ex:s-ts:riv}
Rivadavia Wichí \citep[87]{JT09-th}\\
    \begin{xlist}
        \ex \intxt{qates} (\SG), \intxt{qatets-el} (\PL)\gloss{star}
        \ex \intxt{\mbox{-}tes} (\SG), \intxt{\mbox{-}tets-el} (\PL)\gloss{ancestor, trunk}
    \end{xlist}
\z

\ea\label{ex:s-ts:whk}
’Weenhayek \citep[316]{KC16}\\
    \begin{xlist}
        \ex \intxt{qates} (\SG), \intxt{qatéts-eɬ} (\PL)\gloss{star}
        \ex \intxt{\mbox{-}tes} (\SG), \intxt{\mbox{-}téts-eɬ} (\PL)\gloss{fault, origin, cause, ancestor}
    \end{xlist}
\z
\boolfalse{listing}

\subsubsection{Deglottalization of preglottalized codas}\label{wi-deglottalization-non-nasal-codas}

Most preglottalized codas of Proto-Mataguayan merge with their plain counterparts in Wichí by means of deglottalization. This includes the codas \intxt{*ˀp}, \intxt{*ˀt}, \intxt{*ˀts}, \intxt{*ˀk}, \intxt{*ˀɬ}, \intxt{*ˀs}, \intxt{*ˀx}, \intxt{*ˀχ}, \intxt{*ˀl}, and \intxt{*ˀj}. The coda \intxt{*ˀl} not only deglottalizes, but also changes to \intxt{*lʰ}, as in \REF{wi-deg-brightness} and \REF{wi-deg-locustcw}, thus merging with \intxt{*l} (see \sectref{wi-l-lh}).

\begin{exe}
    \ex \brightness \label{wi-deg-brightness}
    \ex \drinkn
    \ex \son
    \ex \yicaay
    \ex \cutdown
    \ex \leech
    \ex \dew
    \ex \jaguar
    \ex \water
    \ex \tail
    \ex \locustcw \label{wi-deg-locustcw}
    \ex \hole
    \ex \answer
    \ex \sweat
    \ex \oldn
    \ex \withstand
    \ex \wash
    \ex \winter
    \ex \yicalhuk
    \ex \powder
    \ex \smelln
    \ex \bitter
    \ex \lippaset
    \ex \fence
    \ex \lid
    \ex \starn
    \ex \vein
    \ex \mesh
    \ex \sprout
    \ex \abdcavity
    \ex \shoot
    \ex \carrysh
    \ex \dig
    \ex \blind
    \ex \uncle
    \ex \duraznillo
    \ex \nest
    \ex \badmood
    \ex \burrow
    \ex \walk
    \ex \climb
    \ex \night
    \ex \headn
    \ex \palosanto
    \ex \sandyplace
    \ex \firewoodhuk
    \ex \earth
\end{exe}

Three preglottalized codas do not merge with their plain counterpart in Wichí: \sound{PM}{*ˀm} and \intxt{*ˀn} keep their glottalization, whereas \intxt{*ˀɸ} is apparently reflected as \sound{PW}{*p} rather than \intxt{*xʷ}, even though only one example is known \REF{wi-deg-suckb}. The Wichí reflex in \REF{wi-deg-kingvulture} is irregular in a number of respects and lacks the expected glottalization.

\begin{exe}
    \ex \pronominal
    \ex \grabwork
    \ex \thorncutjan
    \ex \defecate
    \ex \kingvulture \label{wi-deg-kingvulture}
    \ex \suckb \label{wi-deg-suckb}
    \ex \throwv
    \ex \meat
\end{exe}

\subsubsection{\sound{PM}{*ɸ’}, \intxt{*ɬ’} > \sound{PW}{*p’}, \intxt{*t’}}\label{wi-glott-fric}

Another sound change in Wichí, shared with Chorote and Nivaĉle but not with Maká, consists of the fortition of the Proto-Mataguayan glottalized fricatives (phonologically possibly analyzable as tautosyllabic sequences of a fricative and a glottal stop) to glottalized stops: \sound{PM}{*ɸ’}, \intxt{*ɬ’}~>~\sound{PW}{*p’}, \intxt{*t’}. (The sequence \intxt{*kɸ’}, however, changed to \sound{PW}{*kʷ’}.)

\begin{exe}
    \ex \poor
    \ex \femalebreastits
    \ex \skinits
    \ex \meatits
    \ex \juiceits
    \ex \urinateyou
    \ex \urineits
\end{exe}

As a result of the sound change \sound{PM}{*ɬ’}~>~\intxt{(*)t’}, Proto-Wichí now displays a morphophonological rule which converts the underlying sequence */ɬ+ʔ/ into \intxt{*t’} (rather than \intxt{ɬ’}, as in Maká). The rule is no longer entirely productive in Wichí, since the sequence /ɬʔ/ may occur at the root--suffix boundary, as in \word{’Weenhayek}{tåɬ\mbox{-}ˈʔúxʷ=eh}{comes from the riverside}.

\subsubsection{\sound{PM}{*ji\mbox{-}}}\label{wi-yi}

The sequence \sound{PM}{*ji} is usually reflected as \sound{PW}{*ʔi} (or \sound{PW}{*hi} before a glottalized consonant due to a general glottal dissimilation rule, \sectref{wi-glottal-dissim}). It is especially common in the high-frequency 3.A/S\textsubscript{A} prefix, but also found in some roots, as in \REF{wi-ji-dew}--\REF{wi-ji-mancw}.

\begin{exe}
    \ex \cutdown
    \ex \tell
    \ex \notafraid
    \ex \dew \label{wi-ji-dew}
    \ex \water \label{wi-ji-water}
    \ex \mancw \label{wi-ji-mancw}
    \ex \redv
    \ex \takeaway
    \ex \torn
    \ex \sendv
    \ex \feed
    \ex \withstand
    \ex \killv
    \ex \feel
    \ex \flee
    \ex \wash
    \ex \sleep
    \ex \bathe
    \ex \cook
    \ex \throwcw
    \ex \spillcw
    \ex \throwv
    \ex \burnvt
\end{exe}

When followed by a glottalized consonant and a low vowel (\sound{PM}{*a} or \intxt{*å}, but not \intxt{*ä}), \sound{PM}{*ji} > \intxt{*ʔi} changed to \intxt{*ʔa} > \sound{PW}{*ha} word-initially (\sectref{pm-wi-ji-ha}).

\begin{exe}
    \ex \jaguar
    \ex \treen
    \ex \vulture
\end{exe}

However, \sound{PM}{*ji} is retained as \sound{PW}{*ji} when followed by a uvular consonant or \intxt{*h}, as evident synchronically from alternations in the third-person prefix \citep[241–242]{VN14}.\footnote{\label{prepwjj}In fact, the fact that \sound{PW}{*h} patterns with uvulars suggests that \intxt{*h} in onsets goes back to a pre-Proto-Wichí uvular fricative. However, in Proto-Wichí \intxt{*χ} and \intxt{*h} clearly contrasted in onsets due to the sound change \sound{PM}{*xχ} > \sound{PW}{*χ}, discussed in \sectref{wi-consonant-dorsal}. Therefore, the change \intxt{*χ} > \intxt{*h} must have been complete by the Proto-Wichí stage.} It is likely that the vowel \intxt{*i} in such cases had a somewhat lowered allophone (for example, [ɪ]), conditioned by a following uvular/glottal, thus bleeding the sound change \sound{PM}{*ji} > \sound{PW}{*ʔi} (i.e., \intxt{*jiq} > \intxt{*j}[ɪ]\intxt{q} > \intxt{*jiq}).

\begin{exe}
    \ex \distrust
    \ex \saber
    \ex \gov
    \ex \pushv
\end{exe}

In the latter case, 'Weenhayek consistently reflects \sound{PW}{*ji\mbox{-}} as \intxt{ja\mbox{-}} (see \sectref{ch-lowering} for a similar outcome in Iyojwa’aja’). In Lower Bermejeño, the sequence /ji/ is articulated as [jɪ]. In the Rivadavia variety of Southeastern Wichí, verbs that took \intxt{*ji\mbox{-}} in Proto-Wichí may now take either \intxt{ja\mbox{-}} (if the agent acts with low intensity) or \intxt{ʔi\mbox{-}} (if the agent acts with high intensity), according to \citet[135]{JT09-th}. For more details, see \sectref{wi-lowering}.

\subsubsection{Glottal dissimilation affecting glottal stops}\label{wi-glottal-dissim}

A dissimilatory process has transformed \sound{PM}{*ʔ} (and the instances of \intxt{*ʔ} originating from \sound{PM}{*j} by means of the sound change \sound{PM}{*ji} > \intxt{*ʔi} word-initially) into \sound{PW}{*h} if the next syllable contained a glottalized consonant. Although unique to Wichí within Mataguayan, a similar process has been identified as a defining innovation of the Guaranian subbranch of the Tupi--Guaranian branch (Tupian family), where \intxt{*ʔVʔ} evolved into \intxt{hVʔ} \citep{FC-guarani}. Yet another language where \intxt{h} was inserted in erstwhile vowel-initial words that contain a glottalized (ejective) consonant is Cuzco Quechua, though in that variety the glottalized trigger need not be located in an adjacent syllable \citep[170]{GP13}.

\begin{exe}
    \ex \jaguar
    \ex \vulture
    \ex \treen
    \ex \stretchout
    \ex \dividev
    \ex \cover
    \ex \suckcw
    \ex \seev
    \ex \aloja
    \ex \mancw
\end{exe}

The glottal dissimilation rule has resulted in synchronically active alternations in Wichí. For example, in the Lower Bermejeño dialect the second-person possessive index usually surfaces as \intxt{ʔa\mbox{-}} before consonants, but if the stem starts with a glottalized consonant, the allomorph \intxt{ha\mbox{-}} shows up instead \REF{ex:aha:1:lb}.

\booltrue{listing}
\ea\label{ex:aha:1:lb}
    Lower Bermejeño Wichí \citep[163--164]{VN14}
    \begin{xlist}
        \ex \intxt{ha-ˀnojiχ}\gloss{your path}
        \ex \intxt{ha-t’alaχ}\gloss{your pillow}
        \ex \intxt{ha-t’ate}\gloss{your breast}
        \ex \intxt{ha-tʃ’efʷa}\gloss{your spouse}
        \ex \intxt{ha-tʃ’ute}\gloss{your ear}
        \ex \intxt{ha-ˀwet}\gloss{your place}
        \ex \intxt{ha-ˀwu}\gloss{your neck}\\
        compare:
        \ex \intxt{ʔa-fʷtʃa}\gloss{your father}
        \ex \intxt{ʔa-n̥es}\gloss{your nose}
        \ex \intxt{ʔa-pʰi}\gloss{your pocket}
        \ex \intxt{ʔa-tset}\gloss{your walking stick}
    \end{xlist}
\z

Similarly, the prefix found in transitive verbs with a third-person subject take the prefix \intxt{ʔi\mbox{-}} before consonants in Lower Bermejeño (\intxt{ji\mbox{-}} before uvulars and glottals), but if the stem starts with a glottalized consonant, the allomorph \intxt{hi\mbox{-}} shows up instead \REF{ex:ihi:1:lb}.

\ea\label{ex:ihi:1:lb}
    Lower Bermejeño Wichí \citep[241--242]{VN14}
    \begin{xlist}
        \ex \intxt{hi-p’aɬtsen}\gloss{s/he forgives}
        \ex \intxt{hi-p’aq}\gloss{s/he dyes}
        \ex \intxt{hi-p’ethat}\gloss{s/he forgets}
        \ex \intxt{hi-p’u}\gloss{s/he burns}
        \ex \intxt{hi-ts’efʷi-hu}\gloss{s/he twists}
        \ex \intxt{hi-ts’ifʷin}\gloss{s/he pinches}
        \ex \intxt{hi-tʃ’esaχ}\gloss{s/he divides}
        \ex \intxt{hi-ˀwen}\gloss{s/he sees}\\
        compare:
        \ex \intxt{ʔi-jo-jeχ}\gloss{s/he drinks}
        \ex \intxt{ʔi-leχ}\gloss{s/he washes}
        \ex \intxt{ʔi-lon}\gloss{s/he kills}
        \ex \intxt{ʔi-tʰat-ʰu}\gloss{s/he puts inside}
        \ex \intxt{ʔi-tʃefʷen}\gloss{s/he teaches}
        \ex \intxt{ʔi-tʃoχ}\gloss{s/he takes away}
        \ex \intxt{ji-haneχ}\gloss{s/he knows}
        \ex \intxt{ji-hemin}\gloss{s/he likes}
        \ex \intxt{ji-hon}\gloss{s/he follows}
        \ex \intxt{ji-qontʃi}\gloss{s/he destroys}
        \ex \intxt{ji-qun}\gloss{s/he plays}
    \end{xlist}
\z
\boolfalse{listing}

\subsubsection{Glottal dissimilation affecting glottalized consonants}\label{wi-glot-dissim}

When two consecutive syllables have glottalized consonants as their onsets in PM, Wichí deglottalizes the onset of the first syllable in a development shared with Chorote (\sectref{ch-glot-dissim}). Example \REF{wi-degl-hornero} shows further irregularities regarding the place of articulation of the dissimilating consonants.

\begin{exe}
    \ex \monkparakeet
    \ex \hornero \label{wi-degl-hornero}
    \ex \saliva
    \ex \hiccup
\end{exe}

\subsubsection{\intxt{*h}-loss after glottalized stops and affricates}\label{wi-h-loss}

In Wichí, word-final \sound{PM}{*h} is lost word-finally if the onset of the syllable in question is a glottalized stop or affricate (as well as in one unclear exception shown in \REF{wi-rat-h-loss}, where the loss of \intxt{*h} may have something to do with the sequence \intxt{*\mbox{-}mʔ\mbox{-}}).

\begin{exe}
    \ex \monkparakeet
    \ex \hornero
    \ex \whiteegret
    \ex \rat \label{wi-rat-h-loss}
\end{exe}

The same kind of sound change must underlie \word{PW}{*ˣnɪ́kʲ’u}{black-legged seriema\species{Chunga burmeisteri}}, whose Chorote counterparts (\wordng{Ijw}{nókʲ’u} /núkʲ'uh/, \wordng{Mj}{hʊ́n(i)ʔi \recind hʊ́niʔu} /hún(i)k'uh/) point to a word-final \intxt{*h}. However, the Chorote and Wichí forms show no regular correspondences and are probably related by horizontal transmission rather than by cognation.

\subsubsection{\intxt{*h}-insertion after word-final accented vowels}\label{wi-h-insertion}
In Proto-Wichí, polysyllabic words cannot end in a long vowel. We account for this restriction by positing a process whereby a \sound{PW}{*h} was inserted word-finally whenever the Proto-Mataguayan etymon ended in an accented vowel (>~PW~long vowel; see \sectref{wi-length} on vowel length in Wichí).

\begin{exe}
    \ex \armadillo
    \ex \savannahhawk
\end{exe}

This sound change does not apply to monosyllabic stems.

\begin{exe}
    \ex \penisits
    \ex \bellyits
    \ex \juiceits
\end{exe}

\subsubsection{\sound{PM}{*-nV} > \sound{PW}{*-ˀnVh}} \label{wi-nv-nvh}
The word-final sequence \intxt{*\mbox{-}nV} changes to \intxt{*\mbox{-}ˀnVh} in Wichí.

\begin{exe}
    \ex \chaniarf
    \ex \whitequebracho
    \ex \redbrocket
    \ex \balawasp
\end{exe}

As a result of this sound change, words ending in \intxt{*\mbox{-}nV} are practically non-existent in the lexicon of Wichí. One exception is \word{PW}{*qáno\pla{lʰ}}{needle}, but this is a likely borrowing from Guaicuruan: compare \word{Toba-Qom (Cerriteño dialect)}{qana}{needle} \citep[263]{CM09}.

\subsubsection{Destiny of word-final \sound{PM}{*l} and \intxt{*ˀl}} \label{wi-l-lh}

Word-finally, \sound{PM}{*l} and \intxt{*ˀl} yield \sound{PW}{*lʰ} (this can be analyzed as a consonant cluster or a marginal phoneme of Proto-Wichí, alongside \sound{PW}{*jʰ}).

\begin{exe}
    \ex \brightness
    \ex \diecw
    \ex \returnth
    \ex \tell
    \ex \locustcw
    \ex \pll
    \ex \returnh
    \ex \shadow
\end{exe}

\subsubsection{Loss of posttonic \sound{PM}{*ʔ} word-finally}\label{wi-posttonic-deglottalization}

\sound{PM}{*ʔ} is lost word-finally in Wichí after short vowels if a long vowel (\sectref{wi-length}) occurs somewhere to the left in the same word.\footnote{Note that the only source that systematically reflects the contrast between \intxt{ʔ}\mbox{-}final and vowel-final words is \citet{VN14} in her description of Lower Bermejeño Wichí. \citet{JB09} does not systematically document the distinction in the same variety. In all other varieties of Wichí, the contrast is lost word-finally: in 'Weenhayek \citep[25]{KC94} and in the Rivadavia subdialect of Southeastern Wichí \citep[31–34]{JT09-th}, \phonetic{ʔ} is automatically inserted in the clause-final position after stressed vowels (in 'Weenhayek also after \intxt{j} and unstressed vowels), whereas in Vejoz \intxt{ʔ} is not reported in the word-final position at all \citep{VU74,MG-MELO15}.}

\begin{exe}
    \ex \mouthits
    \ex \daughterits
    \ex \coalrel
    \ex \arrowkaxe
    \ex \youngersis
    \ex \cavy
\end{exe}

The word-final deglottalization in Wichí is similar to an analogous process known from Nivaĉle, but must have occurred independently. Note that it was fed by accent retraction in words with postpeninitial PM accent, a process unique to Wichí and Iyojwa'aja'; in this case deglottalization occurs in Wichí, but not in Nivaĉle, leading to different outcomes. Note that the Nivaĉle cognates in \REF{wini-degl-elderbro}–\REF{wini-degl-leg} have stress in the final syllable, which is why deglottalization fails to occur in them \perscomm{Analía Gutiérrez}{2023}.

\begin{exe}
    \ex \elderbro \label{wini-degl-elderbro}
    \ex \eldersis
    \ex \leg \label{wini-degl-leg}
\end{exe}

\subsubsection{Syllabic consonants}\label{wi-t-n}

The Proto-Mataguayan consonants \intxt{*n̩} and \intxt{*t̩} are reflected in Wichí as \sound{PW}{*ni}, \intxt{*ta}. This is seen in the allomorphy pattern of the 3.\textsc{neg.irr} prefix (\sound{PW}{*ni\mbox{-}} before supraglottal consonants, \sound{PW}{*n\mbox{-}} before vowels or \intxt{*ʔ}), of the T-class verbal prefix (\sound{PW}{*ta\mbox{-}} word-initially before supraglottal consonants, \sound{PW}{*t\mbox{-}} elsewhere), and of the homophonous third-person prefix found in a closed set of terms for body parts.

\booltrue{listing}
\ea\label{ex:syllnt:whk}
'Weenhayek \citep[62, 76, 82, 99, 349, 375--376]{KC16}
    \begin{xlist}
        \ex \wordnl{ní-t-ahuj-aʔ}{lest s/he speak}
        \ex \wordnl{ní-ˀnom-aʔ}{lest s/he wake up}
        \ex \wordnl{∅-ta-qásit}{s/he stands up}
        \ex \wordnl{∅-ta-qátin̥}{s/he dances}
        \ex \wordnl{ta-kejʔ}{her/his hand}
        \ex \wordnl{ta-qǻlåʔ}{her/his leg}
        \ex \wordnl{ta-teʔ}{her/his eye}
    \end{xlist}
\z
\boolfalse{listing}

\ea\label{ex:syllnt:lbw}
Lower Bermejeño Wichí \citep[239, 289, 320--321]{VN14}\footnote{In Lower Bermejeño, the Proto-Wichí third-person prefix found in a closed set of terms for body parts has been reanalyzed as a part of the stem, and is now always preceded by an overt person index. Since it never occurs word-initially, it does not have a moraic allomorph: \wordnl{n̩\mbox{-}t\mbox{-}kʷej}{my hand}, \wordnl{ʔa\mbox{-}t\mbox{-}kʷej}{your hand}, \wordnl{la\mbox{-}t\mbox{-}kʷej}{her/his hand}, \wordnl{ɬa\mbox{-}t\mbox{-}kʷej}{our hand}, \wordnl{to\mbox{-}t\mbox{-}kʷej}{one's hand} \citep[147]{VN14}. This is obviously an innovation when compared to the situation in 'Weenhayek, where the prefix in question shows up only in the third person: \wordnl{ʔṍ\mbox{-}kejʔ}{my hand}, \wordnl{ʔa\mbox{-}kejʔ}{your hand}, \wordnl{ta\mbox{-}kejʔ}{her/his hand}, \wordnl{ˀnó\mbox{-}kejʔ}{one's hand}.}
\booltrue{listing}
    \begin{xlist}
        \ex \wordnl{ni-tamtʃoj-a}{lest it dry}
        \ex \wordnl{ni-ˀwatsʰan-a}{lest it be green}
        \ex \wordnl{ni-fʷit-a}{lest s/he reach}
        \ex \wordnl{∅-ta-qásit}{s/he stands up}
        \ex \wordnl{∅-ta-qatin}{s/he jumps} 
    \end{xlist}
\z
\boolfalse{listing}

\subsubsection{Consonant + guttural fricative}\label{wi-consonant-dorsal}

Proto-Mataguayan clusters of the shape */Cx/, */Cχ/, */Ch/ largely yield aspirated consonants or voiceless nasals (phonologically */Ch/) except if the consonant is a fricative, in which case the subsequent guttural fricative is lost.

The examples below show the evolution of PM~clusters of the shape */Cx/, */Cχ/, */Ch/ whose first element is not a fricative or the lateral approximant */l/. This includes the word-final cluster */jh/ (represented as \intxt{*jʰ} in this book), which is generally preserved in Proto-Wichí except that after the vowel \intxt{*i} it is simplified to \intxt{*h} \REF{wi-recipient-yh-h}. \REF{wi-kx-cactus} and \REF{wi-nx-sleepiness} show vowel epenthesis, presumably due to the fact that the consonant cluster occurs word-initially. In \REF{wi-nx-rhea} and \REF{wi-th-heart}, vowel syncope (\sectref{wi-vow}) probably had originally resulted in triconsonantal clusters of the shape \intxt{*ChC}, which were subsequently simplified to \intxt{*CC}. The reflex in \REF{wi-px-beard} is entirely irregular due to contamination with that of \word{PM}{*\mbox{-}pǻs(\mbox{-}eˀt)}{lip}. 

\begin{exe}
    \ex \plaj
    \ex \distal
    \ex \coal
    \ex \cactus \label{wi-kx-cactus}
    \ex \heavyv
    \ex \youngersis
    \ex \thorncutjan
    \ex \girl
    \ex \powderpl
    \ex \ropepl
    \ex \noseobl
    \ex \sleepiness \label{wi-nx-sleepiness}
    \ex \pathpl
    \ex \orphancwpl
    \ex \up
    \ex \beard \label{wi-px-beard}
    \ex \fishwithhook
    \ex \wildcat
    \ex \bilecwpl
    \ex \uncle
    \ex \throwcw
    \ex \marry
    \ex \healthy
    \ex \rhea \label{wi-nx-rhea}
    \ex \headpl
    \ex \recipient \label{wi-recipient-yh-h}
    \ex \tuscaf
    \ex \tuscat
    \ex \caracarapl
    \ex \doradocw
    \ex \wildbean
    \ex \heartcw \label{wi-th-heart}
\end{exe}

Interestingly, the clusters involving \sound{PM}{*l} as the first element did not yield \sound{PW}{**lh}, as one could expect, but rather \intxt{*nh}, possibly as a rhinoglottophilia effect (see \sectref{wi-jj-h} on rhinoglottophilia in Wichí).

\begin{exe}
    \ex \ankle
    \ex \armadillo
    \ex \roast
\end{exe}

The latter sound change has resulted in a synchronically active alternation in Wichí, where the underlying cluster /lh/ (in some analyses, /lh̃/) surfaces as \phonetic{n̥}.\footnote{At least in some dialects, this rule is no longer entirely productive. For example, in the Rivadavia subdialect of Southeastern Wichí forms such as \wordnl{ʔitsel\mbox{-}hat}{to sharpen}, \wordnl{qalel\mbox{-}hit’e}{not to know}, \wordnl{totajal\mbox{-}hu}{next year} are attested \citep[47]{JT09-th}.}

\booltrue{listing}
\ea
'Weenhayek \citep[337–338, 454, 516]{KC16}\\
    \begin{xlist}
        \ex \wordnl{ni-tǻˈxʷel-ex}{s/he is known} → \wordnl{ʔi-tǻxʷˈn̥-at-ex}{s/he makes aware}
        \ex \wordnl{tʰalåk}{s/he is old} → \wordnl{ʔiná-tʰan̥å-ç}{we are old}
        \ex \wordnl{ʔõ-j-ǻpiɬ}{I return there} → \wordnl{ʔi-j-ǻpn̥-eˀn}{we return there}
    \end{xlist}
\z
\boolfalse{listing}

\newpage
\ea
Southeastern Wichí (Ingeniero Juárez) \citep[102–103]{LCB-MBC09}\\
    \begin{xlist}
        \ex\gll j-el-h̃en~\phonetic{jɛˈn̥ɛ̃n}\\
                3.{\textsc{i}}-be\_tired-\PL\\
                \glt `they are tired'
        \ex\gll j-opil-h̃it’e~\phonetic{jɔpˈn̥ɪ̃ɗɛ}\\
                3.{\textsc{i}}-return\_thither-\NEG\\
                \glt `s/he does not come back'
        \ex\gll to-ʔoxʷel-h̃en~\phonetic{tɔfʷɛˈn̥ɛ̃n}\\
                {\textsc{gnr}}-be\_ashamed-\PL\\
                \glt `we are ashamed'
    \end{xlist}
\z

The following examples show the evolution of PM~clusters of the shape */Cx/ or */Cχ/ where the first element is a fricative (*/Ch/ was not a licit sequence in PM, as discussed in \sectref{fricative-h}). Such clusters simply lose the second element in Wichí. 

\begin{exe}
    \ex \killbird
    \ex \finger
    \ex \redquebracho
    \ex \thunder
    \ex \eggits
    \ex \headits
    \ex \meat
\end{exe}

This sound change accounts for the fact that /h/ is synchronically banned after fricatives in all Wichí varieties, including ’Weenhayek \citep[28]{KC94},\footnote{The only exception is the root \wordnl{\mbox{-}xhån}{to bury}, whose Chorote cognate \intxt{*-qhǻn} has a stop.} and Southeastern Wichí. Whenever an \intxt{h}\mbox{-}initial morpheme is preceded by a fricative, the glottal fricative is deleted.

\ea
'Weenhayek \citep[28, fn. 34]{KC94}\\
    \begin{xlist}
        \ex\gll ʔis-heˀn~\phonetic{ʔiˈsenʔ}\\
                good-\PL\\
                \glt `they are well'
        \ex\gll ʔi-kʲåx-heˀn~\phonetic{ʔikʲɑˈxenʔ}\\
                3.{\textsc{i}}-buy-\PL\\
                \glt `s/he buys them'
    \end{xlist}
\z

\ea
Southeastern Wichí (Rivadavia) \citep[43–44]{JT09-th}\\
    \begin{xlist}
        \ex\gll pite-s-hit’e~\phonetic{pitesiˈt’e}\\
                long-\PL-\NEG\\
                \glt `they are short'
        \ex\gll i-kʲes-hen~\phonetic{ikʲeˈsen}\\
                3.{\textsc{i}}-heal-\PL\\
                \glt `they are in good health'
        \ex\gll n̩-kʲɔx-hu~\phonetic{n̩ˈkʲɔxu}\\
                1-buy-\APPL:for\\
                \glt `I buy for'
        \ex\gll la-sax-hi~\phonetic{lasaˈxi}\\
                2.{\textsc{act}}-cut-\APPL:in\\
                \glt `you work'
    \end{xlist}
\z

\ea
Southeastern Wichí (Lower Bermejeño) \citep[108–109]{VN14}\\
    \begin{xlist}
        \ex\gll ha-n̩-tefʷ-hi~\phonetic{hãˌn̩teˈfʷi}\\
                \NEG-1-eat-\NEG\\
                \glt `I don't eat it'
        \ex\gll n̩-kʷes-hen~\phonetic{n̩ˌkʷeˈsen}\\
                1-cut\_oneself-\PL\\
                \glt `we cut ourselves'
        \ex\gll ʔi-tʃes-hat~\phonetic{ʔiˌtʃeˈsat}\\
                3.{\textsc{i}}-heal-{\textsc{caus}}\\
                \glt `s/he heals her/him/it'
        \ex\gll ∅-toɬ-hu~\phonetic{ˈtoɬu}\\
                3-come\_from-\APPL:for\\
                \glt `s/he comes from'
        \ex\gll j-ukʷaχ-hi~\phonetic{juˌkʷaˈχɪ}\\
                3.{\textsc{i}}-bite-\APPL:in\\
                \glt `s/he chews something'
        \ex\gll n̩-tijoχ-hila~\phonetic{n̩ˌtijoˌχɪˈla}\\
                1-throw-{\textsc{fut}}\\
                \glt `I will throw it'
        \ex\gll j-ʔaχ-hu~\phonetic{ˈˀjaχu}\\
                3.{\textsc{i}}-hit-\APPL:for\\
                \glt `s/he breaks it'
    \end{xlist}
\z

\subsubsection{Other consonant clusters} \label{wi-other-clusters}

Though some consonant clusters of Proto-Mataguayan have been preserved in Wichí, many underwent considerable change.

The following examples instantiate retentions; note that although the tautosyllabic clusters \intxt{*kʲt} and \intxt{*tkʲ} have subsequently changed in all Wichí dialects (\sectref{wi-cc}), they are clearly reconstructible to Proto-Wichí.

\begin{exe}
    \ex \tortoise
    \ex \whitealgarrobof
    \ex \dovesipup
    \ex \precipice
\end{exe}

The Proto-Mataguayan sequences \intxt{*kɸ} and \intxt{*kɸ’} yield \sound{Proto-Wichí}{*kʷ}, \intxt{*kʷ’}. The preceding vowel (if there is one) apparently becomes rounded, though it is unknown whether this is regular, since only one example has been found.

\begin{exe}
    \ex \bite
    \ex \earkfe
    \ex \tornkf
    \ex \frighten
\end{exe}

Several clusters, such as \sound{PM}{*ɸts}, \intxt{*sk}, \intxt{*sl}, and \intxt{*tl}, are resolved by \intxt{*i}\mbox{-}epenthesis, at least word-initially.

\begin{exe}
    \ex \suncho
    \ex \palm
    \ex \mesh
    \ex \wildcat
    \ex \blind
\end{exe}

The cluster \sound{PM}{*st} undergoes \intxt{*i}\mbox{-}prothesis in the word-initial position.

\begin{exe}
    \ex \whitequebracho
    \ex \kingvulture
    \ex \cardon
    \ex \chachalaca
\end{exe}

In clusters whose first member is any of \intxt{*l}, \intxt{*w}, or \intxt{*ˀw}, only the last member survives in Wichí, but a deleted \sound{PM}{*w} can trigger rounding of a preceding vowel (\sound{PM}{*e} > \sound{PM}{*o}). Other clusters where only the last member survives include \intxt{*ɸq}, \intxt{*nxt}, and \intxt{*X₂₃t}.

\begin{exe}
    \ex \elbow
    \ex \spouse
    \ex \flu
    \ex \majan
    \ex \cavy
    \ex \tooth
    \ex \belly
    \ex \metal
    \ex \femalebreast
\end{exe}

In clusters that involve an approximant as their final element -- such as \intxt{*sw}, \intxt{*nj}, and \intxt{*ˀnj} -- the approximant is lost in Wichí; \sound{PM}{*ˀnj} is reflected as \sound{PW}{*ˣn} at least word-initially. The Wichí reflex in \REF{wi-stw-kingvulture} is in any case irregular.

\begin{exe}
    \ex \smelln
    \ex \cavy
    \ex \kingvulture \label{wi-stw-kingvulture}
    \ex \anteater
\end{exe}

\sound{PM}{*tsn} yielded \sound{PW}{*tn}.

\begin{exe}
    \ex \toad
\end{exe}

Stem-initial clusters of a guttural fricative and a sonorant yield \sound{PW}{*ˣC}, whereas in the only example of a stem-initial cluster of a guttural fricative and an obstruent one finds \sound{PW}{*hp} as the reflex.

\begin{exe}
    \ex \spring
    \ex \fox
    \ex \shadow
    \ex \moon
\end{exe}

\subsection{Vowels} \label{wi-vow}

Wichí shows more or less the same reflexes of PM vowels as Chorote: most vowels are preserved intact except for \sound{PM}{*ä}, which merges with \intxt{*e} (or with \intxt{*i}, if an accented syllable follows; \sectref{pm-wi-ae}). Three minor innovations shared with Chorote are the lowering of \intxt{*e} to \intxt{*a} before a \intxt{*χ} in the coda position (\sectref{pm-wi-ejj-ajj}; also shared with Maká), the lowering of \intxt{*i} to \intxt{*e} in the environment \intxt{*At/x…ts} (\sectref{pm-wi-atits-atets}) and to \intxt{*a} in the environment \intxt{*\#ʔ…C'Á} (\sectref{pm-wi-ji-ha}), and the rounding of \intxt{*e} before clusters with a labial (\sectref{pm-wi-ew-ow}). Other minor innovations, not shared with Chorote, are the fronting of \intxt{*å} before \intxt{*ˀm} (\sectref{aom-am}) and word-medial syncope in words with initial accent (\sectref{wi-syncope}).

\subsubsection{Reflexes of \sound{PM}{*ä}} \label{pm-wi-ae}
\sound{PM}{*ä} is most commonly reflected as \sound{PW}{*e}. The reflex \sound{PW}{*ɪ} in \REF{wi-ae-egg} is apparently the regular continuation of \sound{PM}{*äj}. In \REF{wi-ae-flyv}, only 'Weenhayek shows the expected reflex \intxt{e}, whereas other varieties have an irregular reflex \intxt{i}.

\begin{exe}
    \ex \burn
    \ex \wing
    \ex \yicaay
    \ex \goawayyou
    \ex \goawaycisl
    \ex \putv
    \ex \flyv \label{wi-ae-flyv}
    \ex \tell
    \ex \rootn
    \ex \coldweather
    \ex \dreamn
    \ex \killbird
    \ex \hole
    \ex \spouse
    \ex \stretchout
    \ex \dividev
    \ex \chaniarf
    \ex \chaniart
    \ex \flu
    \ex \mesh
    \ex \acquainted
    \ex \abdcavity
    \ex \basetrunk
    \ex \bilecw
    \ex \allrcpr
    \ex \burrow
    \ex \walk
    \ex \seev
    \ex \placen
    \ex \egg \label{wi-ae-egg}
    \ex \vrbpl
    \ex \headn
    \ex \eatvi
\end{exe}

In syllables that precede the accented one, however, the regular reflex of \sound{PM}{*ä} seems to be \sound{PW}{*i} rather than \intxt{*e}, though the conditioning environment is not entirely clear at present.

\begin{exe}
    \ex \deep
    \ex \duraznillo
    \ex \meat
\end{exe}

\subsubsection{Lowering of \intxt{*e} before \intxt{*χ}}\label{pm-wi-ejj-ajj}

Before the uvular fricative \sound{PM}{*χ}, the vowel \intxt{*e} has a special lowered reflex, \sound{PW}{*a}. This is shared with Maká (\sectref{mk-uvul-retr}) and Chorote (\sectref{pm-ch-ejj-ah}).

\begin{exe}
    \ex \fatv
    \ex \jabiru
    \ex \longv
    \ex \smoke
    \ex \fullriver
    \ex \blackalgarrobof
    \ex \peccary
    \ex \hurt
    \ex \chaguara
    \ex \wildbean
    \ex \mistolf
    \ex \puma
\end{exe}

The lowering induced by the uvular fricative left behind a synchronically active alternation in Wichí. In forms that go back to PM~etyma with a \intxt{*χ}, the lowering applies, and one finds \sound{PW}{*a}. By contrast, the reflexes of PM~forms derived from the vocalic stems of the same etyma (see \sectref{jj-suff}) show no lowering, because \sound{PM}{*χ} was absent in the respective protoforms. Consequently, one finds \sound{PW}{*e}.

\booltrue{listing}
\ea
   ’Weenhayek \citep[8, 92, 293,  297, 426]{KC16}
    \begin{xlist}
        \ex \intxt{pitáx}\gloss{long.\SG} → \intxt{pité-s}\gloss{long.\PL}
        \ex \intxt{p’alítsax}\gloss{poor.\SG} → \intxt{p’alítse-s}\gloss{poor.\PL}
        \ex \intxt{(-)tútsax}\gloss{smoke} → \intxt{tútse-tax}\gloss{mist}
        \ex \intxt{ʔǻjtax}\gloss{it hurts} → \intxt{ʔǻjte-s}\gloss{they hurt}
    \end{xlist}
\z

\ea
   Southeastern Wichí (Lower Bermejeño) \citep[210–211]{VN14}
    \begin{xlist}
        \ex \intxt{-tsax}\gloss{\textsc{nmlz.sg}} → \intxt{-tse-s}\gloss{\textsc{nmlz.pl}}
    \end{xlist}
\z
\boolfalse{listing}

\subsubsection{Lowering of \intxt{*i} in the environment \intxt{*At/x…ts}}\label{pm-wi-atits-atets}

In Wichí, \sound{PM}{*i} lowers to \intxt{*e} before \intxt{*ts}, provided that there is a low vowel (\intxt{*a} or \intxt{*å}) in the preceding syllable. This most regularly happens when the syllable has \intxt{*t} as the onset, but one example with \sound{PM}{*x} > \sound{PW}{*h} has also been identified. As a consequence, the nominal plural suffix \intxt{*\mbox{-}is} shows the allomorph \intxt{*\mbox{-}es} in Proto-Wichí, an alternation best described as an instance of progressive height harmony. This innovation is shared with Chorote (\sectref{pm-ch-atits-ates}); in addition, a similar process operates dialectally in Nivaĉle (\sectref{ni-chishamnee-lowering}).

\begin{exe}
    \ex \waterpl
    \ex \starn
    \ex \earthpl
    \ex \skinpl
\end{exe}

\subsubsection{Lowering of \intxt{*i} before glottalized consonants followed by a low vowel}\label{pm-wi-ji-ha}

We have already seen that the sequence \sound{PM}{*ji} changed to \intxt{*ʔi} word-initially in Proto-Wichí (\sectref{wi-yi}). However, when followed by a glottalized consonant and a low vowel (\sound{PM}{*a} or \intxt{*å}, but not \intxt{*ä}), it underwent further change: the vowel was lowered, yielding \intxt{*ʔa}, and then glottal dissimilation applied, with \sound{PW}{*ha} as the outcome (\sectref{wi-glottal-dissim}). The development \sound{PM}{*ji} > \intxt{*ʔi} > \intxt{*ʔa} in this environment is shared with Chorote (\sectref{pm-ch-ji-a}), but the change \intxt{*ʔa} > \intxt{*ha} is exclusive to Wichí.

\begin{exe}
    \ex \jaguar
    \ex \treen
    \ex \vulture
\end{exe}

\subsubsection{Rounding of \intxt{*e} before clusters with a labial}\label{pm-wi-ew-ow}

In two examples, \sound{PM}{*e} appears to have acquired rounding in Wichí before a cluster with a labial consonant, yielding Proto-Wichí~\intxt{*o}.

\begin{exe}
    \ex \bite
    \ex \tooth
\end{exe}

\subsubsection{Fronting of \intxt{*å} before \intxt{*ˀm}}\label{aom-am}
\sound{PM}{*å} is fronted to \sound{PW}{*a} before the coda \intxt{*ˀm}, as the following two examples show.

\begin{exe}
    \ex \pronominal
    \ex \defecate
\end{exe}

\subsubsection{Syncope}\label{wi-syncope}

In polysyllabic words, a vowel is sometimes syncopated in a medial open syllable if there is an accented syllable to the left.

\begin{exe}
    \ex \redquebracho
    \ex \gutscw
    \ex \rhea
    \ex \wildbean
    \ex \hiccup
    \ex \heartcwits
\end{exe}

However, there are many words with the same prosodic structure where the syncope fails to occur, such as \word{PW}{*tsóxʷa\mbox{-t-}ukʷ}{shrub\species{Lycium americanum}}, \wordnl{*wósakʲVt}{red-crested cardinal}, \wordnl{*wákʲa\mbox{-}jukʷ}{guayacán}. The exact conditions for syncope in Wichí require further study.

The syncope left behind a number of alternations in Wichí, as exemplified below.

\booltrue{listing}
\ea
Southeastern Wichí (Rivadavia) \citep[27–29, 40, 53]{JT09-th}\\
    \begin{xlist}
        \ex \wordnl{j-iˈset}{s/he cuts} → \wordnl{ji-sˈt-ex}{s/he cuts with}, \wordnl{ji-st-ʰiˈt’e}{s/he does not cut}
        \ex \wordnl{n̩-lesaˈjen}{I write} → \wordnl{ja-lesajˈn̥-en}{we write}
        \ex \wordnl{huˈsan}{ax} → \wordnl{husˈn-is}{axes}
    \end{xlist}
\z
\boolfalse{listing}

\subsection{Word-level prosody}\label{wi-prosody}

Two phenomena should be distinguished in Wichí at the suprasegmental level: vowel length (symbolized here with the acute accent) and stress (marked with the sign ˈ). The distribution of the vowel length (\sectref{wi-length}) follows a complex left-aligned pattern, with different morphemes (including lexical roots) having different underlying specifications; in \chapref{prosody} we argued that this pattern is the direct continuation of the Proto-Mataguayan accent. 'Weenhayek is the only variety known to systematically preserve the vowel length distinctions of Proto-Wichí. By contrast, the stress (\sectref{wi-stress}) is right-aligned in Wichí, its function is to signal the right edge of the word, and the only complication is that a few suffixes are specified as extrametrical. Although the right-aligned stress in Wichí is superficially similar to the right-aligned stress in Maká and Nivaĉle, the pattern is so trivial that it could very well result from independent innovations, and we do not reconstruct it to Proto-Mataguayan.

\subsubsection{Vowel length}\label{wi-length}

The long vowels of Proto-Wichí are reconstructed based on evidence from only one variety, 'Weenhayek, where vowel length is contrastive to this day: consider the pairs \wordnl{ʔõjik}{I go} and \wordnl{ʔõjík}{my scar}, \wordnl{ɬaʔ}{louse} and \wordnl{ɬáʔ}{its fruit}, \wordnl{lapaq}{her/his voice} and \wordnl{lapáq}{you paint}, \wordnl{ʔeɬ}{another} and \wordnl{ʔéɬ}{her/his relative} \citep[24]{KCnd}; recall that the acute accent in our notation denotes vowel length and not stress. As for the varieties of Wichí spoken in Argentina, the erstwhile vowel length opposition appears to have been lost, at least according to our reference sources. In what follows, we rely exclusively on 'Weenhayek in our discussion of the Proto-Wichí vowel length.

In 'Weenhayek (and Proto-Wichí), there may be at most one long vowel per word, and which vowel surfaces as long depends on the morphological composition of the word and on the lexical specifications of individual morphemes \citep[24--30]{KCnd}. An inspection of the 'Weenhayek data in \citet{KC16} shows that the language has three kinds of morphemes with regard to vowel length:

\begin{enumerate}
  \item some morphemes contain an underlying long vowel;
  \item some morphemes lack underlying long vowels;
  \item one prefix (\wordnl{la\mbox{-} / lat’\mbox{-} / ɬ\mbox{-}}{2.{\textsc{act}}}) is exceptional in that it triggers vowel length in the initial syllable of the stem.
\end{enumerate}

Typically, only the leftmost underlying long vowel surfaces as long, whereas all subsequent underlying long vowels are shortened \citep[25--26]{KCnd}. The syllable that contains a long vowel receives secondary stress, unless when primary stress (\sectref{wi-stress}) happens to fall on that syllable.

\ea
'Weenhayek \citep[25--26]{KCnd}\\
    \begin{xlist}
        \ex\gll /tájhi-ɬ-éle/~\phonetic{ˌtaːɲ̥jĩɬeˈleʔ}\\
                forest-3.\textsc{poss}-inhabitant\\
                \glt `forest dweller'
        \ex\gll /ˀnó-híh-wúk/~\phonetic{ˌˀnõːhĩˈwuk}\\
                {\textsc{gnr}}-boat-owner\\
                \glt `boat owner'
        \ex\gll /nijáte-(á)jh-lés-(ʔa)tsính(a)-ájh/~\phonetic{nĩˌjaːtelestsiˈn̥ãç}\\
                {chief-\PL-children-woman-\PL}\\
                \glt `kings' daughters'
    \end{xlist}
\z

Exceptionally, in incorporation constructions, where a verbal stem and a nominal stem are combined in one phonological word, it is always the long vowel in the nominal stem that makes it to the surface, and any long vowels in the verbal stem are shortened (even though they are located to the left), as in the example \word{'Wk}{ni\mbox{-}kʲåt\mbox{-}p’ante\mbox{-}ˈʔúxʷ\mbox{=}eh}{s/he came to the other side of the river a long time ago (without my witnessing it)}, where the verb \wordnl{ni\mbox{-}kʲǻt\mbox{=}eh}{s/he came to} loses its long vowel before an incorporated noun \wordnl{ʔúxʷ}{side of the river, shore} \citep[9]{KC94}.

An additional rule applies to trisyllabic (or longer) words that lack an underlying long vowel within the disyllabic windows at their left edge: in this case the vowel of the peninitial syllable (underlyingly short) surfaces as long, and any subsequent underlying long vowels are shortened \citep[27--29]{KCnd}. In forms that arose due to Watkins' Law (\sectref{wi-watkins}), the domain for the application of this rule excludes any material that precedes the erstwhile third-person prefix \citep[11]{KC94}; this includes all forms inflected for the first person singular \REF{'otilhààjlhih}, and all other forms where an erstwhile third-person prefix intervenes between a prefix and a vowel-initial or a \intxt{ʔ}-initial stem \REF{'not'àjwkyaataj}. In the following examples, which instantiate the lengthening rule, the location of the disyllabic window is shown by means of parentheses.

\ea
'Weenhayek \citep[65, 95, 109, 140, 173, 405]{KC16}\\
    \begin{xlist}
        \ex\gll /(la-kʲo)wex/~\phonetic{(laˌkʲoː)ˈwex}\\
                3.\textsc{poss}-hole\\
                \glt `its hole'
        \ex\gll /(la-xʷi)jho/~\phonetic{(laˌxʷiː)ˈɲ̥jõʔ}\\
                3.\textsc{poss}-charcoal\\
                \glt `its charcoal'
        \ex\gll /(ˀwelek)-ɬih/~\phonetic{(ʔweˌleːk)ˈɬih}\\
                3.glean-{\textsc{hab}}\\
                \glt `s/he routinely gleans'
        \ex\gll /(haˀlå)-ɬ-áwoʔ/~\phonetic{(hãˌʔlɑː)ɬaˈwoʔ}\\
                tree-3.\textsc{poss}-flower\\
                \glt `tree flower'
        \ex\gll /(haˀlå)-towh-ájh/~\phonetic{(hãˌʔlɑː)toˈŋ̥wãç}\\
                tree-hole-\PL\\
                \glt `tree holes'
        \ex\gll /ʔõ-(tiɬåx)-ɬih/~\phonetic{ʔõ(tiˌɬɑːx)ˈɬih}\label{'otilhààjlhih}\\
                1\SG-3.carry\_on\_shoulders-{\textsc{hab}}\\
                \glt `I routinely carry it on my shoulders'
        \ex\gll /ˀnó-(ɬ-ʔåxʷ-kʲa)-tax/~\phonetic{ʔnõ(t’ɑxʷˌkʲaː)ˈtax}\label{'not'àjwkyaataj}\\
                {\textsc{gnr}}-3.\textsc{poss}-skin-illness\_spirit-pseudo\\
                \glt `one's chickenpox'
    \end{xlist}
\z

We suggest that in most cases the long vowels of 'Weenhayek (and Proto-Wichí) straightforwardly continue the accented vowels of Proto-Mataguayan, and that the underlying accentual properties of specific morphemes were also inherited from PM (though we currently have no explanation for the behavior of the prefix \wordnl{la\mbox{-} / lat’\mbox{-} / ɬ\mbox{-}}{2.{\textsc{act}}}). As discussed in \chapref{prosody}, already in Proto-Mataguayan only the leftmost underlying accent in any given word made it to the surface, whereas all subsequent underlying accents were eliminated; this rule (mutatis mutandis) is still active in Proto-Wichí and 'Weenhayek. In addition, as shown in \sectref{corta-larga-corta}, Proto-Mataguayan had a rule whereby a default peninitial accent is inserted in words without an underlying accent within the trisyllabic window at the left edge: ˘˘˘(…)~→~˘¯˘(…). This rule is also preserved in Proto-Wichí and ’Weenhayek, but with an important change regarding the rule conditioning: in Wichí, the peninitial lengthening now occurs not only in the unaccented left-aligned trisyllabic window, but also in the unaccented left-aligned \emph{disyllabic} window (provided that the word is trisyllabic or longer): ˘˘…~→~˘¯…. In particular, the sequence ˘˘¯, reconstructible for Proto-Mataguayan, is no longer licit in Wichí, where it yields ˘¯˘, a change that can be seen in the following examples.

\begin{exe}
    \ex \shoulder
    \ex \elbow
    \ex \elderbro
    \ex \eldersis
    \ex \cheek
    \ex \lidpl
    \ex \leg
    \ex \starnpl
    \ex \basetrunkpl
\end{exe}

\tabref{PM-accent-patterns-Wi} summarizes the evolution of the Proto-Mataguayan accent patterns in Wichí.

\begin{table}
\caption{PM accent patterns}
\label{PM-accent-patterns-Wi}
 \begin{tabular}{ccccccc}
  \lsptoprule
           PM (underlying) & PM (surface) & PW and 'Wk (surface)\\\midrule
  ˘ & ˘ & ˘\\
  ¯ & ¯ & ¯\\
  ˘˘ & ˘˘ & ˘˘\\
  ˘¯ & ˘¯ & ˘¯\\
  ¯˘ / ¯¯ & ¯˘ & ¯˘\\
  ˘˘¯ & ˘˘¯ & ˘¯˘\\
  ˘˘˘ / ˘¯˘ / ˘¯¯ & ˘¯˘ & ˘¯˘\\
  ¯˘˘ / ¯˘¯ / ¯¯˘ / ¯¯¯ & ¯˘˘ & ¯˘˘\\
  \lspbottomrule
 \end{tabular}
\end{table}

\subsubsection{Stress}\label{wi-stress}

If the complex rules that determine the distribution of long vowels in Proto-Wichí are inherited from Proto-Mataguayan, the same cannot be said of the distribution of \conc{stress} in Proto-Wichí. Stress in Wichí has a low contrastive load, and is typically assigned to the rightmost syllable in a word, unless it belongs to a verbal suffix lexically specified as extrametrical. There appears to be some dialectal variation regarding whether a given suffix is specified as extrametrical or not, as \tabref{wi-extr-suff} shows. The data are from \citet[134--136]{VN14}, \citet[54--56]{JT09-th}, \citet[22--23]{KCnd}, and \citet{KC16}.

\begin{table}
\caption{Extrametrical and metrical suffixes in Wichí lects}
\label{wi-extr-suff}
\fittable{
 \begin{tabular}{lllll}
  \lsptoprule
            PWi & gloss & Lower Bermejeño & Rivadavia & ’Weenhayek\\\midrule
            *-kʲe & \gloss{along; distributive; plural object} & extrametrical & extrametrical & usually extrametrical\\
            *-kʲåʔ & \gloss{downwards} & metrical & metrical & metrical\\
            *-peʔ & \gloss{above} & metrical & metrical & metrical\\
            *-h(i)låʔ & \gloss{to the front} & ? & metrical & metrical\\
            *-ho & \gloss{towards} & extrametrical & metrical & usually extrametrical\\
            *-ej & \gloss{far} & extrametrical & lexical variation & metrical\\
            *-eχ & \gloss{by means of} & extrametrical & lexical variation & extrametrical\\
            *-ah & \gloss{towards, near} & extrametrical & lexical variation & usually extrametrical\\
            *-hi & \gloss{in} & metrical & lexical variation & usually extrametrical\\
            *-phå & \gloss{upwards} & metrical & lexical variation & usually extrametrical\\
    \lspbottomrule
 \end{tabular}
 }
\end{table}

In the examples below, extrametrical suffixes are segmented using the equal sign.

\newpage
\ea
Southeastern Wichí (Lower Bermejeño) \citep[396--397]{VN14}\\
    \begin{xlist}
        \ex\gll la-nuwaj~\phonetic{laˌnũˈwaj}\\
                2.{\textsc{act}}-be\_afraid\\
                \glt `you are afraid'
        \ex\gll la-nuwaj$=$a~\phonetic{laˌnũˈwaja}\\
                2.{\textsc{act}}-be\_afraid-\APPL:near\\
                \glt `you are afraid of'
        \ex \gll n̩-t-qatin~\phonetic{n̩tˌqaˈtin}\\
             1\textsc{sg-t}-jump\\
             \glt `I jump'
        \ex \gll n̩-t-qatin-hi~\phonetic{n̩tˌqatiˈn̥ĩ}\\
             1\textsc{sg-t}-jump-\APPL:in\\
             \glt `I jump in'
    \end{xlist}
\z
\ea
'Weenhayek \citep[22--23, 33]{KC16}\\
    \begin{xlist}
        \ex\gll /∅-í-phå/~\phonetic{ʔiːˈpʰɑ̃ʔ}\\
                3-be-\APPL:up\\
                \glt `it is up'
        \ex\gll /∅-í$=$hi/~\phonetic{ˈʔiːhĩʔ}\\
                3-be-\APPL:in\\
                \glt `it exists'
        \ex\gll /∅-í$=$hi$=$kʲe/~\phonetic{ˈʔiːhĩkʲeʔ}\\
                3-be-\APPL:in-\PL\\
                \glt `they exist'
        \ex\gll /∅-ipélax/~\phonetic{ʔipeːˈlax}\\
                3-be\_white\\
                \glt `it is white'
        \ex\gll /∅-ipélax-pe/~\phonetic{ʔipeːlaxˈpeʔ}\\
                3-be\_white-\APPL:above\\
                \glt `it dawns'
        \ex\gll /∅-ipélax$=$kʲe/~\phonetic{ʔipeːˈlaxkʲeʔ}\\
                3-be\_white-\APPL:along\\
                \glt `it is white along'
    \end{xlist}
\z

\newpage
It is beyond the scope of this book to provide a coherent account for suffixes with a variable behavior in the Rivadavia subdialect of Southeastern Wichí and in ’Weenhayek. In the latter variety, for example, it is possible that at least some of these are actually pairs of homophonous suffixes with different underlying stress properties: compare \word{’Wk}{[j]ik\mbox{-}ˈpʰåʔ}{s/he goes away upriver} and \wordnl{ˈ[j]ik\mbox{-}pʰåʔ}{s/he goes away upwards (in the air)}, \wordnl{tåɬ\mbox{-}ˈpʰåʔ}{s/he comes from upriver} and \wordnl{ˈtåɬ\mbox{-}pʰåʔ}{s/he comes from above (in the air)} \citep[18]{KCnd}.

There are some further exceptions from the general rule regarding stress assignment in the Wichí varieties, none of which has known parallels elsewhere in Mataguayan. For example, the roots \wordnl{ʔi\mbox{-}}{to be} and \wordnl{hu\mbox{-}}{to go} are reported to exceptionally attract stress in the Rivadavia subdialect of Southeastern Wichí, even when they are followed by metrical material \citep[56]{JT09-th}. Exceptional non-final stress is found in the reflexes of \word{PW}{*ˣˈxʷála}{sun, day}, reflected as ’Weenhayek [ʔiˈxʷálaʔ] in free variation with [ʔixʷáˈlaʔ] \citep[25]{KC16}, Rivadavia [iˈxʷala] \citep[36]{JT09-th}, Misión El Carmen [ˈxʷala], Colonia Muñiz [ˈfʷala] \citep[138]{MC09}, among others. Other nouns with an exceptional stress pattern include \word{’Wk}{ʔaˈxʷúmaq}{corpse} and the Spanish loan \wordnl{ˈmósoʔ}{young man} \citep[19, fn. 16]{KCnd}. In ’Weenhayek, syllables with a long vowel receive secondary stress when the primary stress falls elsewhere \citep[20, 25]{KCnd}.

Since the Wichí stress pattern lacks known counterparts in other Mataguayan languages, we consider it an innovation.

\subsection{Watkins' Law as a regular morphological change in Wichí} \label{wi-watkins}
\largerpage[-1]

Watkins’ Law is the name given to a process whereby the form inflected for the third person singular is diachronically reanalyzed as a ``base'' form of a stem. This kind of morphological change has been originally identified in a number of Indo-European languages by \citet[90–96]{CW62}.

In Wichí, the operation of Watkins’ Law is most clearly seen in vowel-initial and \intxt{*ʔ}\mbox{-}initial obligatorily possessed nouns. In such nouns, the erstwhile third-person prefix \intxt{*ɬ\mbox{-}} (before vowels, as in \word{PM}{*ɬ\mbox{-}ǻˀs}{her/his son}) or \intxt{*t\mbox{-}’…} (in \intxt{*ʔ}\mbox{-}initial stems, as in \word{PM}{*t\mbox{-}’áte}{her breast}) is now found not only in the form inflected for the third person, but also in the uninflected form (\word{PW}{*{\upshape{NP}}~ɬ\mbox{-}ǻs}{NP’s son}, \wordnl{*{\upshape{NP}}~t\mbox{-}’áte}{NP’s breast}), in the form inflected for the first person singular (\word{PW}{*n̩\mbox{-}ɬ\mbox{-}ǻs}{my son}, \wordnl{*n̩\mbox{-}t\mbox{-}’áte}{my breast}), for the first person inclusive (\word{PW}{*ɬá\mbox{-}ɬ\mbox{-}ås}{our son}, \wordnl{*ɬá\mbox{-}t\mbox{-}’áte}{our breast}), and in the form with a generic possessor (\word{PW}{*ˀno\mbox{-}ɬ\mbox{-}ǻs}{one’s son}, \wordnl{*ˀno\mbox{-}t\mbox{-}’áte}{one’s breast}).\footnote{The generic possessor prefix is reconstructed as \wordng{PW}{*ˀnó\mbox{-}} based on its reflexes in ’Weenhayek, Vejoz, and Guisnay. In Southeastern Wichi, the prefix \intxt{to\mbox{-}} of unknown origin is found instead \citep[163]{VN14}; this prefix also requires the occurrence of \intxt{ɬ\mbox{-}} (as in \wordnl{to\mbox{-}ɬ\mbox{-}os}{one’s son}) or \intxt{t\mbox{-}’…} (as in \wordnl{to\mbox{-}t\mbox{-}’ate}{one’s breast}) in stems that were historically subject to the operation of Watkins’ Law.} This includes all forms that are not inherited from Proto-Mataguayan but rather result from recent grammaticalization restricted to Wichí. The elements \intxt{*ɬ\mbox{-}} and \intxt{*t\mbox{-}’…} do not show up in the forms inherited from Proto-Wichí, such as the second-person form (\word{PW}{*∅\mbox{-}ʔǻs}{your son}, \wordnl{*∅\mbox{-}ʔáte}{your breast}\footnote{In Southeastern Wichí, erstwhile \intxt{*ʔ}\mbox{-}initial nouns no longer preserve the archaic second-person forms with a zero allomorph of the person prefix, but rather attach the second-person prefix \intxt{ha\mbox{-}} (allomorph of \intxt{ʔa\mbox{-}} before glottalized consonants) to the stem augmented by Watkins’ Law, as in \word{LB}{ha\mbox{-}t\mbox{-}’ate}{your breast} \citep[164]{VN14}.}) or the vocative form, a relic of the Proto-Mataguayan first-person form, preserved only in ’Weenhayek (\word{PW}{*j\mbox{-}ås}{son!}).

\begin{exe}
    \ex \mouth
    \ex \brightness
    \ex \flower
    \ex \fruit
    \ex \food
    \ex \son
    \ex \daughter
    \ex \drinkn
    \ex \wing
    \ex \yicaay
    \ex \thorne
    \ex \namen
    \ex \inhabitant
    \ex \resin
    \ex \penis
    \ex \seed
    \ex \nest
    \ex \cord
    \ex \femalebreast
    \ex \skin
    \ex \meat
    \ex \juice
    \ex \urine
\end{exe}

Watkins’ Law also operates in disyllabic stems whose Proto-Mataguayan etyma begin with \intxt{*x}, possibly due to the fact that the sequence \intxt{*ɬx} evolved into \intxt{*ɬh}~>~\intxt{*ɬ} in the history of Wichí (\sectref{wi-consonant-dorsal}), leading to the emergence of third-person forms starting with \wordng{PW}{*ɬ\mbox{-}V…}. The respective stems were subsequently reanalyzed as vowel-initial, as in \REF{lhx-egg} and \REF{lhx-headn}. In the only example involving a monosyllabic stem, Watkins’ Law failed to apply \REF{lhx-price}.

\begin{exe}
    \ex \price \label{lhx-price}
    \ex \egg \label{lhx-egg}
    \ex \headn \label{lhx-headn}
\end{exe}

In addition to nouns, Watkins’ Law altered the distribution of two extremely frequent verbal prefixes, reconstructed as third-person prefixes in Proto-Mataguayan: \word{PM}{*ji\mbox{-} / *j\mbox{-}}{3.A/S\textsubscript{I}} and \wordnl{*t̩\mbox{-} / *t\mbox{-}}{3.S\textsubscript{T}}. Their Wichí reflexes, \wordng{PW}{*ʔi\mbox{-} / *ji\mbox{-} / *hi\mbox{-} / *j\mbox{-}} and \intxt{*ta\mbox{-} / *t\mbox{-}}, are no longer entirely restricted to the third-person form; their distribution is described below.

In I-class verbs, the prefix in question surfaces as \wordng{PW}{*ʔi\mbox{-}} before most consonants, as \intxt{*ji\mbox{-}} before uvulars and \intxt{*h}, as \intxt{*hi\mbox{-}} before glottalized consonants, and as \intxt{*j\mbox{-}} before vowels or \intxt{*ʔ} (in the latter case the sequence \intxt{*j\mbox{-}ʔ…} fuses as \intxt{*ˀj…}). The allomorphs \intxt{*ʔi\mbox{-}} and \intxt{*hi\mbox{-}} are conservative in that they are still restricted to the third person in Proto-Wichí, though in the Southeastern dialect they appear as \intxt{i\mbox{-}} after the dialectal 1{\textsc{incl}} or impersonal prefix \intxt{to\mbox{-}}, yielding \intxt{t\mbox{-}i\mbox{-}}, as in \word{LB}{t\mbox{-}i\mbox{-}potsin}{we build, one builds}, \wordnl{t\mbox{-}i\mbox{-}ˀwen}{we see, one sees} \citep[241]{VN14}. At least in the Southeastern dialect, the reflex of the allomorph \intxt{*ji\mbox{-}} has a reduced variant \intxt{j\mbox{-}}, which appears in the first-person form and in the dialectal 1{\textsc{incl}}/impersonal form: \word{LB}{n̩\mbox{-}j\mbox{-}qon}{I like}, \wordnl{to\mbox{-}j\mbox{-}qon}{we like, one likes} \citep[241]{VN14}, though no trace of \intxt{j\mbox{-}} is seen in the ’Weenhayek verbs of the same class, as in \wordnl{ʔõ\mbox{-}qǻx}{I crush} \citep[302]{KC16}. Finally, the allomorph \intxt{*j\mbox{-}}, found in vowel-initial and \intxt{*ʔ}-initial stems, has clearly been extended to the first-person form already in Proto-Wichí: \word{PW}{*n̩\mbox{-}j\mbox{-}én}{I set a trap}, \wordnl{*n̩\mbox{-}ˀj\mbox{-}áχ}{I beat} > \wordng{’Wk}{ʔõ\mbox{-}j\mbox{-}én̥}, \intxt{ʔõ\mbox{-}ˀj\mbox{-}áx} \citep[116, 532]{KC16}; \word{LB}{n̩\mbox{-}j\mbox{-}en}{I fish}, \wordnl{n̩\mbox{-}ˀj\mbox{-}aχ}{I beat} \citep[241]{VN14}. In the Southeastern dialect, the allomorph \intxt{j\mbox{-}} has been further extended to the dialectal 1{\textsc{incl}}/impersonal form (\word{LB}{to\mbox{-}j\mbox{-}en}{we fish, one fishes}, \wordnl{to\mbox{-}ˀj\mbox{-}aχ}{we beat, one beats}) and, in the case of \intxt{ʔ}-initial verbal stems but not of vowel-initial ones, to the second-person form, as in \word{LB}{la\mbox{-}ˀj\mbox{-}aχ}{you beat} \citep[241]{VN14}.

As for T-class verbs, the erstwhile third-person prefix has the shape \intxt{*ta\mbox{-} / *t\mbox{-}} in Proto-Wichí, and it is now used in all persons in that language except in imperatives, as documented by \citet[448]{JAA-KC-14} for ’Weenhayek, by \citet[237]{JT09-th} for the Rivadavia subdialect of Southeastern Wichí, and by \citet[120, 239–240]{VN14} for the Lower Bermejeño subdialect of Southeastern Wichí.

Watkins’ Law continued to operate after the diversification of Proto-Wichí. For exSciample, the prefix \intxt{*ta\mbox{-} / *t\mbox{-}} that encoded a third-person possessor in a handful of nouns in Proto-Wichí retains its original distribution in ’Weenhayek, as in \wordnl{ʔṍ\mbox{-}kejʔ}{my hand/arm}, \wordnl{ʔa\mbox{-}kejʔ}{your hand/arm}, \wordnl{ta\mbox{-}kejʔ}{her/his hand/arm} \citep[62, 294, 331]{KC16}. In the Rivadavia subdialect of Southeastern Wichí, its occurrence was extended to the first-person singular form but not to any other form: \wordnl{n̩\mbox{-}t\mbox{-}kʷej}{my hand/arm}, \wordnl{a\mbox{-}kʷej}{your hand/arm}, \wordnl{ta\mbox{-}kʷej}{her/his hand/arm}, \wordnl{ɬa\mbox{-}kʷej}{our hand/arm}, \wordnl{to\mbox{-}kʷej}{one’s hand/arm} \citep[69]{JT09-th}. In the Lower Bermejeño subdialect of Southeastern Wichí, the prefix in question is found in all inflected forms: \wordnl{n̩\mbox{-}t\mbox{-}kʷej}{my hand/arm}, \wordnl{ʔa\mbox{-}t\mbox{-}kʷej}{your hand/arm}, \wordnl{la\mbox{-}t\mbox{-}kʷej}{her/his hand/arm}, \wordnl{ɬa\mbox{-}t\mbox{-}kʷej}{our hand/arm}, \wordnl{to\mbox{-}t\mbox{-}kʷej}{one’s hand/arm} \citep[147]{VN14}, and is thus no longer identifiable as a person prefix in that specific subdialect. Another instance of a sporadic morphological change involving Watkins’ Law is the emergence of forms such as \word{’Wk}{ʔõ\mbox{-}lates}{my origin}, \wordnl{ʔá\mbox{-}lates}{your origin} \citep[221]{KC16}, where \intxt{la\mbox{-}} is a fossilized third-person prefix attached to the stem \wordnl{\mbox{-}tes}{origin, fault, trunk, founding father} \citep[93]{KC16}. The lack of vowel lengthening in the peninitial syllable in \intxt{ʔõ\mbox{-}lates} betrays the recent formation of the aforementioned forms in ’Weenhayek (see \sectref{wi-length} for more details). 

\section{From Proto-Wichí to the contemporary Wichí varieties}\label{wi-dialects}

The dialectal division of Wichí presents considerable complexity and remains insufficiently studied. Early works include \citet[36]{AT61}, who identifies three major dialects (Vejoz, Guisnay, and Noctén), and \citet{EN68}, who adds two dialects to that list (Forest and Mataco Proper). Based on the speakers’ own assessment of mutual intelligibility, \citet[27]{VN14} identifies a basic distinction between the Pilcomayeño and the Bermejeño dialect groups, spoken on the Pilcomayo and Bermejo Rivers, respectively; in turn, each of these dialect groups is divided in a binary fashion into an Upper and a Lower dialect. Further evidence supporting \cits{VN14} classification can be found in \citet{VN20} and \citet{VN-MA-21}. Our own examination of the published data has revealed the existence of a clear primary split of Wichí into two dialect clusters, as suggested by the distribution of certain phonological innovations.

\conc{Northwestern Wichí} (Nercesian’s \concl{Pilcomayeño}\footnote{We do not adopt Nercesian’s label in this book in order to avoid potential confusion: note that the Vejoz variety (classified as Pilcomayeño by Nercesian) is actually spoken on the Bermejo River.}) is a diverse group of dialects which are characterized by the simplification of word-initial consonant clusters (as in \word{PW}{*tkʲénaχ}{mountain}, \wordnl{*kʲtáˀnih}{Chaco tortoise} >~\intxt{*kʲénaχ}, \intxt{*táˀnih}) and by the merger of \sound{Proto-Wichí}{*i} and \intxt{*ɪ} (as in \word{PW}{*ɬ\mbox{-}ɪ́kʲ’u}{its egg}, \wordnl{*hɪ́lu}{yica bag} >~\intxt{*ɬ\mbox{-}íkʲ’u}, \intxt{*hílu}). The most well-described dialects are ’Weenhayek and Vejoz.

\begin{itemize}
    \item \concl{’Weenhayek} (=~Tovar’s and Najlis’ Noctén, Nercesian’s Upper Pilcomayeño), spoken in the Bolivian department of Tarija, is characterized by the devoicing of all non-glottalized sonorants before a pause \citep[33–35]{KC94}, among other innovations; it is also the only Wichí variety known to retain the Proto-Wichí vowel length contrast. The phonology and lexicon of ’Weenhayek are known fairly well thanks to the contributions of \citet{KC94,KC16}.
    \item \concl{Vejoz} (=~a fraction of Nercesian’s \concl{Lower Pilcomayeño}), spoken in the Argentine province of Salta, is represented in our study by the subdialects of Misión Chaqueña \citep{VU74,MG-MELO15} and Paraje La Paz \citep{AFG067}. A salient innovation exclusive to Vejoz is the semantic shift which transformed \word{PW}{*ˀwáχ}{stagnant water} into the basic term for ‘water’, thus replacing \wordng{PW}{*ʔinǻt}.\footnote{\citet[280--282]{VN-MA-21} suggest that \word{Vejoz}{ˀwáχ}{water} could be a retention, whereas other Wichí varieties would have replaced it with reflexes of \wordng{PW}{*ʔinǻt}, claimed to be an innovation by Nercesian and Amarilla. This seems quite unlikely to us, since Nivaĉle and Chorote use cognates of \wordng{PW}{*ʔinǻt} -- and not of \wordng{PW}{*ˀwáχ} -- for `water'. In any case, the Vejoz innovation must be quite old, because the earliest known record of that variety (a 1795 manuscript by Esteban Primo de Ayala) has ‹guag› `water' \citep[507]{IC-RM-20}, which we tentatively phonologize as \intxt{ˀwáh}.}
    \item As for the dialectal zone referred to as \concl{Guisnay} (by \citnp{AT61} and \citnp{EN68}, from Wichí \intxt{W’enhayey} [w’en̥ãjej])) or \concl{Lower Pilcomayeño} \citep{VN14}, we have as of yet been unable to verify its validity by means of identifying its precise limits and defining innovations. In part, this is due to the scarcity of the available data. We dispose only of a basic phonological description of the variety spoken in Misión La Paz \citep{MA08}.
    \item For other lects, which could be suspected on geographical grounds to belong to the purported Guisnay/Lower Pilcomayeño dialect zone, only some isolated words have been documented (\citnp{SS07} and \citnp{AFG-SS-09} for Misión Santa María; \citnp{AFG-SS-09} for Santa Victoria Este, Las Vertientes, Lapacho Mocho; \citnp{MC09} for Misión El Carmen; \citnp{VU74} and \citnp{LCB15} for Tartagal). In fact, at least the Lapacho Mocho lect shows some features typical of Vejoz (such as the third-person prefix \intxt{le\mbox{-}}, as opposed to \intxt{ha\mbox{-}} in Tartagal and \intxt{la\mbox{-}} in Misión La Paz). The Tartagal lect shares with Vejoz the irregular reflex \intxt{e} (<~\sound{PW}{*a}) in [tɕeˈno] `armadillo'. As for the other lects, we provisionally do not include them into any dialect group; throughout this section, we always specify the community where a given phenomenon was reported when referring to the data of such varieties.
\end{itemize}

\conc{Southeastern Wichí} (Nercesian’s \concl{Bermejeño}, roughly corresponding to Najlis’ Mataco Proper; not mentioned by Tovar) encompasses the variety spoken in Rivadavia, Salta (classified by \citnp{VN14} as Upper Bermejeño) as well as Nercesian’s Lower Bermejeño, spoken in the Argentine provinces of Formosa and Chaco to the south from the town of Ingeniero Suárez. Its most notable phonological feature is the Southeastern Wichí vowel shift (\sectref{wi-vowel-shift}). There are small differences between the varieties spoken in Rivadavia \citep{JT09-th}, Ingeniero Suárez \citep{LCB-MBC09,LCB15}, and the communities located to the east of El Sauzalito (Lower Bermejeño stricto sensu), including Misión Nueva Pompeya, Laguna Yema, Pozo del Mortero, Juan G. Bazán, Las Lomitas, and Pozo del Tigre (see \citnp{VN14} for a comprehensive description, \citnp{JB09} for a vocabulary based on data from Bazán, and \citnp{MC09} for some fragmentary data from specific communities).

In most cases, the Lower Bermejeño (as documented by \citnp{VN14}) and ’Weenhayek (as documented by \citnp{KC16}) reflexes suffice to reconstruct a Proto-Wichí form. These varieties, spoken in the extreme southeast and in the extreme north of the Wichí territory, respectively, differ phonologically in all possible dimensions: there are almost no innovations shared by Lower Bermejeño with ’Weenhayek to the exclusion of some other Wichí variety (one exception is \intxt{*ɬ̩\mbox{-}}~>~\intxt{la\mbox{-}}; see \sectref{wi-syll-lh}). The etymological dictionary in \chapref{etymdic} systematically lists reflexes in these two varieties as well as in Vejoz (Misión Chaqueña subdialect).

In what follows, we examine the reflexes of PW segments and suprasegmental units in the contemporary dialects. As a detailed phonological analysis is available only for a handful of Wichí lects, we make no attempt at differentiating between sound changes with and without phonological significance.

In \figref{fig-wi-map}, the numbers correspond to \cits{JB09} numeration of the Wichí groups. Four main dialects (’Weenhayek, Guisnay, Vejoz, and Southeastern) are colored in light blue, green, magenta, and orange, respectively. Gray means that we lack conclusive information on the dialectal appurtenance of a given group.

\begin{figure}
\includegraphics[width=\textwidth]{figures/wichi_map.pdf}
\caption{Map of the Wichí-speaking area}
\label{fig-wi-map}
\end{figure}

The lects examined in this section and the sources of linguistic data on each of them are shown in \tabref{wi-lects}.\footnote{Two of our sources – \citet{MA08} and \citet{VN14} – are possibly based on multiple lects. Of \cits{MA08} consultants, one is from Las Vertientes -- a community identified with the Santa Teresa group in \citet[3]{JB09}, another one is born to a father from Las Vertientes and a non-Wichí mother, and the third one is reported to have moved to Misión La Paz from the province of Formosa. That way, the variety described by \citet{MA08} may be in fact representative of a region located to the southeast from Misión La Paz. In turn, \cits{VN14} grammar is based on data collected in multiple communities located within the triangle delimited by Pozo del Tigre, Misión Nueva Pompeya, and Ingeniero Juárez. She does not indicate the exact provenance of the data she cites and does not report any diatopic variation.}

\begin{table}[t]
\caption{Sources on Wichí lects}
\label{wi-lects}
 \begin{tabularx}{\textwidth}{lQp{2.2cm}}
  \lsptoprule
            community & sources & Wichí group according to \citet{JB09}\\
            \midrule
            ’Weenhayek & \citet{KC94,KC16} & Villa Montes\\
            Misión Santa María & \citet{SS07,AFG-SS-09} & Villa Montes\\
            Santa Victoria Este & \citet{AFG-SS-09} & La Paz\\
            Misión La Paz & \citet{MA08} & La Paz\\
            Las Vertientes & \citet{AFG-SS-09} & Santa Teresa\\
            Misión El Carmen & \citet{MC09} & Pozo de Maza\\
            Juárez (Barrio Viejo) & \citet{LCB-MBC09,LCB15,VN14} & Ingeniero Juárez\\
            El Sauzalito & \citet{MC09,VN14} & El Sauzalito\\
            Bazán & \citet{JB09,MC09,VN14} & Bazán\\
            Colonia Muñiz & \citet{MC09,VN14} & Pozo del Tigre\\
            Teniente Fraga & \citet{MC09} & Ruta 81\\
            Rivadavia & \citet{JT09-th} & Rivadavia\\
            Paraje La Paz & \citet{AFG067,AFG-SS-09} & (?) Ruta 81\\
            Misión Chaqueña & \citet{MG-MELO15} & Embarcación\\
            Embarcación & \citet{VU74} & Embarcación\\
            Lapacho Mocho & \citet{AFG-SS-09} & Mosconi\\
            Tartagal & \citet{VU74,LCB15} & Mosconi\\
    \lspbottomrule
 \end{tabularx}
\end{table}

\subsection{Consonants}

This section describes the evolution of the Proto-Wichí consonants in the contemporary varieties of Wichí.

\subsubsection{\sound{PW}{*kʲ} and \intxt{*kʲ’}}

The Proto-Wichí reflexes of \sound{PM}{*k} and \intxt{*k’} in onsets are reconstructed as palatalized velar stops (IPA~*\phonetic{kʲ}, *\phonetic{kʲ’}). Their original articulation is faithfully retained both in the Rivadavia subdialect of Southeastern Wichí \REF{wi:kj:riv}, and in some speakers in El Sauzalito \REF{wi:kj:elsauz}, as well as in the Misión La Paz subdialect of Guisnay \REF{wi:kj:mlp}. This is also the predominant realization in ’Weenhayek and in Misión El Carmen \REF{wi:kj:mec}, but in these varieties an affricate realization ([tʃ(’)] or [tɕ(’)]) is increasingly frequent in younger speakers’ speech \citep[14]{KC94}.

\booltrue{listing}
\ea\label{wi:kj:riv}
Rivadavia Wichí \citep[36]{JT09-th}\\
    \begin{xlist}
        \ex \phonetic{kʲeˈjɔʔ}\gloss{granddaughter} < \wordng{PW}{*\mbox{-}kʲéjå}
        \ex \phonetic{kʲaˈlaʔ}\gloss{lizard} < \wordng{PW}{*kʲáˀlah}
        \ex \phonetic{kʲul}\gloss{locust} < \wordng{PW}{*kʲólʰ}
    \end{xlist}
\z

\ea\label{wi:kj:elsauz}
Sauzalito Wichí \citep[132]{MC09}\\
    \begin{xlist}
        \ex \phonetic{iˈkʲot}\gloss{it is red} < \wordng{PW}{*ʔikʲǻt}
        \ex \phonetic{kʲɛˈɬekʷ}\gloss{quebracho tree} < \wordng{PW}{*kʲéɬjukʷ}
    \end{xlist}
\z

\ea\label{wi:kj:mlp}
Misión La Paz Wichí \citep[44–45]{MA08}\\
    \begin{xlist}
        \ex \phonetic{otkʲumɬi}\gloss{I work} < \wordng{PW}{*n̩\mbox{-}t\mbox{-}kʲúm\mbox{-}ɬih}
        \ex \phonetic{kʲajohĩ}\gloss{hot} < \wordng{PW}{*kʲáˈjo\mbox{-}hi}
    \end{xlist}
\z

\ea\label{wi:kj:mec}
Misión El Carmen Wichí \citep[131–132, 138]{MC09}\\
    \begin{xlist}
        \ex \phonetic{kʲaˈlaʔ}\gloss{lizard} < \wordng{PW}{*kʲáˀlah}
        \ex \phonetic{niˈkʲim}\gloss{I am thirsty} < \wordng{PW}{*n̩-kʲím}
        \ex \phonetic{kʲɯˈkɯk}\gloss{butterfly} < \wordng{PW}{*kʲókʷokʷ}
        \ex \phonetic{kʲeˈʔʲe} (older) \recind \phonetic{tɕeˀˈeʔ} (younger)\gloss{parakeet sp.} < \wordng{PW}{*kʲékʲ’e}
        \ex \phonetic{n̩tʃemˈɬi}\gloss{I work} < \wordng{PW}{*n̩\mbox{-}t\mbox{-}kʲúm\mbox{-}ɬih}
    \end{xlist}
\z
\boolfalse{listing}

In some varieties, the occurrence of [kʲ] is positionally restricted. In Santa Victoria Este and in the Vejoz community of Paraje La Paz, [kʲ] may occur only before front vowels in younger speakers’ speech, and even then it is reported to freely vary with [tʃ] \recind [tɕ] \REF{wi:kj:sve-plp}. Before non-front vowels, [kʲ] is not documented; at least in Paraje La Paz, [tɕ] is the most common realization, but [tʃ] and [tʲ] are also possible \citep{AFG067}. In Southeastern Wichí as spoken in Teniente Fraga, [tɕ] is predominant, as in \REF{wi:kj:tf1}–\REF{wi:kj:tf3}, but one also finds [kʲ] \REF{wi:kj:tf4}.

\booltrue{listing}
\ea\label{wi:kj:sve-plp}
Santa Victoria Este or Paraje La Paz Wichí \citep[160]{AFG-SS-09}\\
    \begin{xlist}
        \ex \phonetic{kʲiliˈtɕuk} \recind \phonetic{tʃiliˈtɕuk}\gloss{owl} < \wordng{PW}{*kʲilúkʲukʷ}
        \ex \phonetic{kʲiˈnax} \recind \phonetic{tʃiˈnax}\gloss{metal, iron} < \wordng{PW}{*kʲínaχ}
        \ex \phonetic{oɬetʃeˈɦis} \recind \phonetic{oɬekʲeˈɦis}\gloss{my trousers} < \wordng{PW}{*n̩\mbox{-}ɬ\mbox{-}ékʲe\mbox{-}hi\mbox{-}s}
    \end{xlist}
\z

\ea\label{wi:kj:tf}
Teniente Fraga Wichí \citep[130–132]{MC09}\\
    \begin{xlist}
        \ex \phonetic{tɕoʔˈhẽt}\gloss{arrow} < \wordng{PW}{*\mbox{-}kʲ’ǻhe} \label{wi:kj:tf1}
        \ex \phonetic{tɕeˈtɕ’eʔ} \recind \phonetic{tʃeˈtʃeʔ}\gloss{parakeet sp.} < \wordng{PW}{*kʲékʲ’e}
        \ex \phonetic{tɕɛˈɬekʷ}\gloss{quebracho tree} < \wordng{PW}{*kʲéɬjukʷ} \label{wi:kj:tf3}
        \ex \phonetic{kʲuˀˈte}\gloss{ear} < \wordng{PW}{*\mbox{-}kʲ’óte} \label{wi:kj:tf4}
    \end{xlist}
\z
\boolfalse{listing}

Elsewhere, the affrication of \sound{PW}{*kʲ(’)} has progressed to a point where the palatalized velar realization is no longer available, giving rise to [tʃ(’)] \recind [tɕ(’)]. For example, in Tartagal, [tɕ] \recind [tʃ] have been documented as exclusive realizations of the phoneme in question \REF{wi:kj:tar}. In Lapacho Mocho and Misión Santa María, [tʃ] and [tɕ] occur in free variation with other, but also with [tʲ] before [e] and [o] \REF{wi:kj:lmmsm}. Regarding Lower Bermejeño, \citet[51]{VN14} characterizes the sound in question as a “palatal affricate” (likely [tɕ]), though she also reports that two of her consultants -- both young women -- produced [tʃ] instead. Based on data recorded in Bazán, \citet{MC09} transcribes mostly [tʃ], as in \REF{wi:kj:baz1}–\REF{wi:kj:baz5}; in one instance its glottalized equivalent is transcribed as alveopalatal \REF{wi:kj:baz6}; \citet{JB09} also characterizes the affricate in question as “palatal” in the Bazán subdialect. Similarly, [tʃ] is usually found in Colonia Muñiz, a community located between Las Lomitas and Pozo del Tigre, as in \REF{wi:kj:cm1}–\REF{wi:kj:cm5}; this segment is transcribed as alveopalatal in one example \REF{wi:kj:cm6}. Only [tʃ] is documented in Ingeniero Juárez (Barrio Viejo), as shown in \REF{wi:kj:ijbv}. Finally, only an affricate realization is reported in Vejoz as spoken in Misión Chaqueña and in the variety of Las Vertientes \citep[160]{VU74,MG-MELO15,AFG-SS-09}, but the data we dispose of are not accompanied by narrow transcriptions.

\booltrue{listing}
\ea\label{wi:kj:tar}
Tartagal Wichí \citep[359--360, 366]{LCB15}\\
    \begin{xlist}
        \ex \phonetic{ɬiˈtɕu}\gloss{its egg} < \wordng{PW}{*ɬ\mbox{-}ɪ́kʲu}
        \ex \phonetic{tɕeˈno}\gloss{armadillo} < \wordng{PW}{*kʲanhóh}
        \ex \phonetic{tʃiˈnax}\gloss{iron} < \wordng{PW}{*kʲínaχ}
        \ex \phonetic{tʃeˈnax}\gloss{mountain} < \wordng{PW}{*tkʲénaχ}
    \end{xlist}
\z

\ea\label{wi:kj:lmmsm}
Lapacho Mocho or Misión Santa María Wichí \citep[160]{AFG-SS-09}\\
    \begin{xlist}
        \ex \phonetic{siˈtʲet} \recind \phonetic{siˈtɕet} \recind \phonetic{siˈtʃet}\gloss{large bag} < \word{PW}{*sikʲet}{mesh purse}
        \ex \phonetic{leˈtʲo} \recind \phonetic{leˈtʃo}\gloss{short} <~\word{PW}{*ɬ̩\mbox{-}kʲo}{its bottom, depth}
    \end{xlist}
\z

\ea\label{wi:kj:baz}
Bazán Wichí \citep[130--132, 134, 138, 140]{MC09}\\
    \begin{xlist}
        \ex \phonetic{tʃeˈtʃ’e(ʔ)}\gloss{parakeet sp.} < \wordng{PW}{*kʲékʲ’e} \label{wi:kj:baz1}
        \ex \phonetic{tʃeˈɬekʷ}\gloss{quebracho tree} < \wordng{PW}{*kʲéɬjukʷ}
        \ex \phonetic{nitʃottʃeʔ}\gloss{similar} < \wordng{PW}{*ni\mbox{-}ˈkʲǻt\mbox{-}kʲe}
        \ex \phonetic{tʃefʷ}\gloss{sweat} < \wordng{PW}{*kʲúxʷ}
        \ex \phonetic{laxəˈtʃa} (older) \recind \phonetic{lawxˈtʃa} (younger)\gloss{her/his father} < \wordng{PW}{*ɬ̩-xʷkʲah} \label{wi:kj:baz5}
        \ex \phonetic{tɕ’oˈhẽt}\gloss{arrow} < \wordng{PW}{*\mbox{-}kʲ’ǻhe} \label{wi:kj:baz6}
    \end{xlist}
\z

\ea\label{wi:kj:cm}
Colonia Muñiz Wichí \citep[130, 132, 138]{MC09}\\
    \begin{xlist}
        \ex \phonetic{iˈtʃot}\gloss{it is red} < \wordng{PW}{*ʔikʲǻt} \label{wi:kj:cm1}
        \ex \phonetic{tʃ’uˈte}\gloss{ear} < \wordng{PW}{*\mbox{-}kʲóte}
        \ex \phonetic{n̩ˈtʃim}\gloss{I am thirsty} < \wordng{PW}{*n̩\mbox{-}kʲím}
        \ex \phonetic{tʃuˈkuk}\gloss{butterfly} < \wordng{PW}{*kʲókʷokʷ}
        \ex \phonetic{n̩tʃemˈxli}\gloss{I work} < \wordng{PW}{*n̩\mbox{-}t\mbox{-}kʲúm\mbox{-}ɬih} \label{wi:kj:cm5}
        \ex \phonetic{tɕaˈlaʔ}\gloss{lizard} < \wordng{PW}{*kʲáˀlah} \label{wi:kj:cm6}
    \end{xlist}
\z

\ea\label{wi:kj:ijbv}
Ingeniero Juárez (Barrio Viejo) Wichí \citep[360, 366]{LCB15}\\
    \begin{xlist}
        \ex \phonetic{ɬɛˈtʃɛ}\gloss{its egg} < \wordng{PW}{*ɬ\mbox{-}ɪ́kʲu}
        \ex \phonetic{tʃaˈnɔ̃}\gloss{armadillo} < \wordng{PW}{*kʲanhóh}
    \end{xlist}
\z
\boolfalse{listing}

\subsubsection{\sound{PW}{*q} and \intxt{*kʷ}}

In \sectref{wi-q-k} and \sectref{wi-other-clusters}, we saw that \sound{PW}{*kʷ(’)} goes back either to \sound{PM}{*kɸ(’)} (when it occurs in onsets) or to \sound{PM}{*k} (when it occurs in codas following a back vowel). By contrast, as discussed in \sectref{wi-q-k}, \sound{PW}{*q(’)} goes back to \sound{PM}{*q(’)} in onsets or codas (note that \sound{PM}{*q} is not known to have occurred following non-low vowels in codas). \sound{PW}{*q} can also continue \sound{PM}{*k} when it occurs in the coda position following a front vowel; in this case, it actually still surfaces as [k] in most contemporary Wichí varieties. In fact, one could simply say that PW~*/q/ and */kʷ/ are neutralized as [k] in the coda position following front vowels, as in \word{PW}{*ji\phonetic{k}}{s/he goes away}, \wordnl{*\mbox{-}whájene\phonetic{k}}{son-in-law}, \wordnl{*xʷéɬe\phonetic{k}}{mortar}, \wordnl{*\mbox{-}téme\phonetic{k}}{bile}. We follow \citet[49--50]{VN14} in analyzing \sound{PW}{*\phonetic{k}} as a positional allophone of */q/, which occurs in the coda position after a front vowel. 

As a result, the synchronic distribution of the consonants \intxt{*q} and \intxt{*kʷ} was asymmetrical in the coda position in Proto-Wichí: only \intxt{*q} could occur following the vowels \intxt{*a}, \intxt{*e}, and \intxt{*i} (note the allophony: */aq/~*[aq], */eq/~*[ek], */iq/~*[ik]), and only \intxt{*kʷ} was found following the vowels \intxt{*o} and \intxt{*u}. \sound{PW}{*q} and \intxt{*kʷ} contrasted following the vowel \intxt{*å} (compare \wordnl{*ɬ\mbox{-}åq}{her/his food}, \wordnl{*tsåhǻq}{chajá bird}, but \wordnl{*níjåkʷ}{rope, cord}, \wordnl{*nitåkʷ}{two}) and in the onset position (\wordnl{*ɬ̩\mbox{-}qéj}{her/his custom} vs. \wordnl{*ta\mbox{-}kʷej}{her/his hand}).

As anticipated above, the Wichí sounds [q] and [k] are most commonly analyzed as allophones of the same phoneme, represented as /q/ by \citet[49--50]{VN14} and as /k/ by \citet[43]{MA08} and \citet[136]{MC09}. The original distribution of the allophones (uvular in onsets and in codas following the low vowels \intxt{a} and \intxt{å}; velar in codas following the front vowels \intxt{e} and \intxt{i}) is preserved in varieties such as 'Weenhayek and the Lower Bermejeño subdialect of Southeastern Wichí. Other varieties, however, may display innovations. For example, in the Rivadavia subdialect of Southeastern Wichí only the allophone [q] is reported, even after front vowels, as in \wordnl{jiq}{s/he goes away} \citep[48]{JT09-cap}. By contrast, in the variety of Paraje La Paz, /q/ and /k/ are synchronically analyzed as phonemes \citep{AFG067}: while in many cases the distribution of these consonants matches fairly well the allophony pattern reconstructed for Proto-Wichí, as in \REF{plp-grayheron}--\REF{plp-bile}, in several words one finds [k] or [k’] in onsets, as in \REF{plp-all}--\REF{plp-flies}, or in codas following non-front vowels, as in \REF{plp-thing}. Note that in the last case the PW etymon did contain a front vowel.

\booltrue{listing}
\ea
Paraje La Paz Wichí \citep{AFG067}\\
    \begin{xlist}
        \ex \phonetic{qaˈlaq}\gloss{gray heron} < \wordng{PW}{*qaláq} \label{plp-grayheron}
        \ex \phonetic{oˈqoj}\gloss{I put clothes on} < \word{PW}{*ń̩\mbox{-}qhå\mbox{-}jʰ}{my clothes}
        \ex \phonetic{laˈqɛ} \recind \phonetic{lɑˈqɛ}\gloss{it shines} < \wordng{PW}{*laq’e}
        \ex \phonetic{ˈqej}\gloss{custom} < \wordng{PW}{*\mbox{-}qéj}
        \ex \phonetic{oˈpaq}\gloss{I paint} < \wordng{PW}{*n̩\mbox{-}páq}
        \ex \phonetic{xʷeˈɬek}\gloss{mortar} < \wordng{PW}{*xʷéɬe\phonetic{k}}
        \ex \phonetic{teˈmek}\gloss{bile} < \wordng{PW}{*\mbox{-}téme\phonetic{k}} \label{plp-bile}
        \ex \phonetic{iɬoˈkex}\gloss{all} < \wordng{PW}{*ni\mbox{-}ˈɬóq-eχ} \label{plp-all}
        \ex \phonetic{kalaˈtu}\gloss{hail} < \wordng{PW}{*qalátu}
        \ex \phonetic{oˈkoj}\gloss{I play} < \wordng{PW}{*n̩\mbox{-}qój}
        \ex \phonetic{isˈkat}\gloss{s/he hides} < \wordng{PW}{*ʔi\mbox{-}sqat}
        \ex \phonetic{k’aˈtas}\gloss{flies} < \wordng{PW}{*q’áta\mbox{-}s} \label{plp-flies}
        \ex \phonetic{ˈmak}\gloss{thing} < \wordng{PW}{*ˣmáje\phonetic{k}} \label{plp-thing}
    \end{xlist}
\z

\citet{MC09} documents the velar allophone in onsets in the varieties spoken in Misión El Carmen \REF{mec-kq}, Teniente Fraga \REF{tfw-kq}, and (variably) El Sauzalito \REF{ezw-kq}. The uvular allophone is attested in Colonia Muñiz \REF{cmw-kq} and Bazán \REF{bzw-kq}.

\ea
Misión El Carmen Wichí \citep[130, 137]{MC09} \label{mec-kq}\\
    \begin{xlist}
        \ex \phonetic{kɑˈnu}\gloss{needle} < \wordng{PW}{*qáno}
        \ex \phonetic{n̩kɑˈʰɲi}\gloss{my pocket} < \wordng{PW}{*ń̩\mbox{-}qhå\mbox{-}j\mbox{-}hih}
        \ex \phonetic{isˈkɑt}\gloss{s/he steals} < \wordng{PW}{*ʔi\mbox{-}sqat}
    \end{xlist}
\z
\ea
Teniente Fraga Wichí \citep[137]{MC09} \label{tfw-kq}\\
    \begin{xlist}
        \ex \phonetic{laˈk’ɑx}\gloss{her/his mouth} < \wordng{PW}{*ɬ̩\mbox{-}q’áχ} \label{tfw-mouth}
    \end{xlist}
\z
\ea
El Sauzalito Wichí \citep[137]{MC09} \label{ezw-kq}\\
    \begin{xlist}
        \ex \phonetic{n̩kɑˈʰɲi}\gloss{my pocket} < \wordng{PW}{*ń̩\mbox{-}qhå\mbox{-}j\mbox{-}hih}
        \ex \phonetic{laˈk’ɑx}\gloss{her/his mouth} < \wordng{PW}{*ɬ̩\mbox{-}q’áχ}
        \ex \phonetic{isˈqɑt}\gloss{s/he steals} < \wordng{PW}{*ʔi\mbox{-}sqat}
    \end{xlist}
\z
\ea
Colonia Muñiz Wichí \citep[137]{MC09} \label{cmw-kq}\\
    \begin{xlist}
        \ex \phonetic{qaˈnu}\gloss{needle} < \wordng{PW}{*qáno}
        \ex \phonetic{toˈq’ax}\gloss{one’s mouth} < \word{PW}{*\mbox{-}q’áχ}{mouth}
        \ex \phonetic{isˈqat}\gloss{s/he steals} < \wordng{PW}{*ʔi\mbox{-}sqat}
    \end{xlist}
\z
\ea
Bazan Wichí \citep[137]{MC09} \label{bzw-kq}\\
    \begin{xlist}
        \ex \phonetic{n̩qohˈɲi}\gloss{my pocket} < \wordng{PW}{*ń̩\mbox{-}qhå\mbox{-}j\mbox{-}hih}
    \end{xlist}
\z
\boolfalse{listing}

In her description of the Guisnay dialect as spoken in Misión La Paz, \citet[43--44]{MA08} posits a phoneme /k/ (a reflex of PW~*/q/) and states that ``[t]here are no minimal pairs to justify the existence of both /k/ and /q/ as phonemes. It is difficult to determine the exact environment for the allophones, so it appears they occur in free variation''. An inspection of the examples given in the cited work, however, shows that the distribution of [k] and [q] in Misión La Paz is similar to the one reconstructed for Proto-Wichí: [q] occurs in onsets as well as in codas following low vowels, and [k] occurs in codas following front vowels \REF{mlp-kq}.\footnote{\citet[43, 82]{MA08} gives some possible counterexamples to this distribution: \wordnl{katetsek}{star}, \wordnl{owukeʔ}{my house}, \wordnl{kamionwoʔ}{truck driver}. The former word is highly anomalous, and contains what looks like a Nivaĉle plural suffix attached to a Wichí root (compare \word{PW}{*qates}{star}, \wordnl{*qatéts\mbox{-}elʰ}{stars}); note that all of \cits{MA08} consultants understand Nivaĉle to some extent, and one of them was born to a Nivaĉle mother. The Proto-Wichí etymon of \wordnl{owukeʔ}{my house} is \wordng{PW}{*ń̩\mbox{-}wukʷ\mbox{-}e}, and thus instantiates loss of labialization in \intxt{*kʷ} and not the alleged change *[q]~>~*[k]. Finally, \wordnl{kamionwoʔ}{truck driver} is derived from the Spanish loan \intxt{kamion} (<~\wordnl{camión}{truck}).} The glottalized counterpart of the consonant in question, which occurs only in onsets, is always uvular in Misión La Paz \citep[44]{MA08}, as shown in \REF{mlp-qajtaj}–\REF{mlp-woqo}.

\booltrue{listing}
\ea
Misión La Paz Guisnay \citep[43--44, 50--51, 54--55, 64--65, 69, 87--88, 95, 99--100]{MA08} \label{mlp-kq}\\
    \begin{xlist}
        \ex \phonetic{qɑl̥qɑl̥tɑx}\gloss{turkey} < \wordng{PW}{*qáɬqaɬ\mbox{-}taχ}
        \ex \phonetic{hãpqit’a}\gloss{it is not} < \wordng{PW}{*håpqhit’ah}
        \ex \phonetic{jaqɑʔtuʔ}\gloss{yellow} < \wordng{PW}{*qáʔtu}
        \ex \phonetic{laqɑs}\gloss{horsefly} < \wordng{PW}{*laqas}
        \ex \phonetic{qɑtes}\gloss{star} < \wordng{PW}{*qates}
        \ex \phonetic{qɑtetsel}\gloss{start} < \wordng{PW}{*qatéts\mbox{-}elʰ}
        \ex \phonetic{ˈts’ilɑq}\gloss{only} < \wordng{PW}{*ˈts’ilaq}
        \ex \phonetic{tsoʔnatɑq}\gloss{deer} < \word{PW}{*tsóˀna\mbox{-}taq}{marsh deer\species{Blastocerus dichotomus}}
        \ex \phonetic{on̥ɑq}\gloss{sachasandía\species{Capparis salicifolia}} < \wordng{PW}{*ʔónha\mbox{-}q \recvar *ʔónha\mbox{-}kʷ}
        \ex \phonetic{ol̥etek}\gloss{my head} < \wordng{PW}{*n̩\mbox{-}ɬ\mbox{-}éte\phonetic{k}}
        \ex \phonetic{notsek}\gloss{to sew} < \wordng{PW}{*ˀnó\mbox{-}tse\phonetic{k}}
        \ex \phonetic{nosek}\gloss{to sweep} < \wordng{PW}{*ˀnó\mbox{-}se\phonetic{k}}
        \ex \phonetic{hʷel̥ek}\gloss{mortar} < \wordng{PW}{*xʷéɬe\phonetic{k}}
        \ex \phonetic{nowal̥ek}\gloss{wasp} < \word{PW}{*nówaɬe\phonetic{k}}{kind of wasp\species{lechiguana}}
        \ex \phonetic{nekkʲaʔ}\gloss{year} < \wordng{PW}{*ne\phonetic{k}kʲåm}
        \ex \phonetic{nekkʲeʔ}\gloss{s/he comes with her/him} < \wordng{PW}{*n\mbox{-}e\phonetic{k}\mbox{-}kʲe}
        \ex \phonetic{tsiliklik}\gloss{kind of eagle} < \word{PW}{*tsíli\phonetic{k}li\phonetic{k} \recind *tsilí\phonetic{k}lik}{snail kite\species{Rostrhamus sociabilis}}
        \ex \phonetic{q’ɑxtax}\gloss{person with a big mouth} < \wordng{PW}{\mbox{-}q’áχ\mbox{-}taχ} \label{mlp-qajtaj}
        \ex \phonetic{laq’ɑs}\gloss{their mouths} < \wordng{PW}{*ɬ̩\mbox{-}q’á\mbox{-}s}
        \ex \phonetic{ihʷaq’ɑn}\gloss{it is blue} < \wordng{PW}{*ʔixʷáq’an}
        \ex \phonetic{sɑq’i}\gloss{Argentine boa}
        \ex \phonetic{woq’o}\gloss{owl} < \wordng{PW}{wóq’oh} \label{mlp-woqo}
    \end{xlist}
\z

\citet{SS07} reports that [q] occurs as a free variant of /k/ in the variety of Misión Santa María \REF{bzw-msm}.

\ea
Misión Santa María Wichí \citep{SS07} \label{bzw-msm}\\
    \begin{xlist}
        \ex \phonetic{qaˈtaq} \recind \phonetic{kaˈtak}\gloss{fly} < \wordng{PW}{*q’átaq}
        \ex \phonetic{oqaˈla} \recind \phonetic{okaˈla}\gloss{my thigh} < \word{PW}{*ń̩\mbox{-}qålå}{my leg}
    \end{xlist}
\z
\boolfalse{listing}

The data above show that the original distribution of the allophones [k] and [q] is preserved to a great extent at least in the northern (’Weenhayek, Misión La Paz) and southeastern (Lower Bermejeño) extremes of the Wichí territory. Deviations are found in the central part of the Wichí territory: in Paraje La Paz, Teniente Fraga, Misión El Carmen, and El Sauzalito at least some instances of [q] have changed to [k]; in Misión Santa María [q] and [k] occur in free variation; whereas in Rivadavia, by contrast, the allophone [k] no longer exists, and /q/ now surfaces as [q] even in codas following front vowels.
    
We now turn to the evolution of \sound{Proto-Wichí}{*kʷ}. While it is regularly preserved in varieties such as Lower Bermejeño Wichí, it may delabialize to [k] in some other dialects.\footnote{A significantly less common development is the change of \sound{PW}{*kʷ} to [qʷ]. This allophone is reported in the word-final position in the Misión La Paz subdialect of Guisnay, as in \wordnl{hʷitsuqʷ}{palm} \citep[44]{MA08}. Between non-front vowels, \intxt{*kʷ} > \intxt{*qʷ} can be further delabialized to [q], as in \wordnl{hʷitsuqat}{group of palm trees}, \wordnl{atsuqat}{group of bola verde trees} < \wordng{PW}{*xʷitsúkʷ\mbox{-}at}, \intxt{*ʔátsukʷ\mbox{-}at}. The allophone [qʷ] also occurs in free variation with [kʷ] following back vowels in Lower Bermejeño Wichí \citep[49]{VN14}.} In his description of the phonology of 'Weenhayek, \citet[19]{KC94} states that ``in the current phonetic development, the loss of labialization is an increasing phenomenon and has reached a level where the phoneme is affected in all positions, except those adjacent to rounded vowels within the syllable'' (that is, forms such as \wordnl{kʷútsax}{caraguatá\species{Bromelia serra}}, \wordnl{tokʷ}{not}, and \wordnl{xʷitsukʷ}{palm} are unaffected by the delabialization). \citet[19]{KC94} also observes that forms such as \wordnl{yi\phonetic{k}eh}{she/goes for it}, \wordnl{ʔṍwu\phonetic{k}eʔ}{my house}, and \wordnl{ʔṍ\phonetic{k}ejʔ}{my hand} are nowadays ``more popular'' than the more conservative variants \intxt{yi[kʷ]eh}, \intxt{ʔṍwu[kʷ]eʔ}, \intxt{ʔṍ[kʷ]ejʔ}. Delabialization can also be seen to various extent in some other varieties. For example, \citet{AFG067} documents the Paraje La Paz reflex of \word{PW}{*kʲókʷokʷ}{butterfly} as [tʃoˈkok]. According to \citet[139]{MC09}, /kʷ/ may optionally lose labialization in the coda position, especially in the speech of younger speakers at least in the Bazán and Teniente Fraga varieties, as in Bazán /tewukʷ/ [teˈwukʷ \recind teˈwuk̚] ‘river’, Teniente Fraga /atsekʷ/ [aˈtsekʷ \recind aˈtsek̚] ‘bola verde tree’ (< \wordng{PW}{*téwokʷ}, \intxt{*ʔátsukʷ}). At least in Lower Bermejeño Wichí as spoken in Bazán (younger speakers), /kʷ/ may surface as prelabialized rather than postlabialized: Bazán /tselekʷ/ [tseˈleʷk] ‘entangled’ \citep[140]{MC09}. Yet another process involving the delabialization of /kʷ/ is proposed in the literature: \citet[63]{JT09-cap} states that /kʷ/ delabializes to [k] before a high rounded vowel /u/ in the Rivadavia subdialect of Southeastern Wichí, as in /nkʷuxʷa/ [nkuˈxʷa] `I feel cold', /jukʷus/ [jukus] (no gloss provided). Note, however, that at least the former datum goes back to \word{PW}{*n̩\mbox{-}qóxʷa}{I feel cold}, and thus does not involve a reflex of \sound{PW}{*kʷ} at all. Further dialectological research is needed to clarify the patterns of */kʷ/ delabialization throughout the Wichí-speaking territory.

\subsubsection{\sound{PW}{*χ} and \intxt{*h}} \label{wi-jj-h}

As we saw in \sectref{wi-jj-j-h}, the Proto-Mataguayan system of three guttural fricatives (\sound{PM}{*x}, \intxt{*χ}, and \intxt{*h}) was reduced to a system composed of only two consonants, represented in this book as \sound{PW}{*χ} and \intxt{*h}. \sound{PW}{*χ} (< \sound{PM}{*x} or \intxt{*χ}) is found almost exclusively in the coda position; its occurrences in onsets are rare and always result from late (post-PM) resyllabification (\intxt{*Vχ.CV} > \intxt{*V.χV}). \sound{PW}{*h} is very common in simplex or complex onsets, where it goes back to \sound{PM}{*x}, \intxt{*χ}, or \intxt{*h}, but it is also sometimes found in the coda position (word-finally only), reflecting \sound{PM}{*h} or zero (\sectref{wi-h-insertion}).

The only known variety that preserves the original (Proto-Wichí) distribution of \sound{PW}{*χ} and \intxt{*h} is 'Weenhayek \citep[19--25]{KC94}, where its reflexes are represented as \intxt{x} and \intxt{h} in this book. Note, however, that /x/ appears to surface as uvular ([χ]) at least after an \intxt{å}, as can be inferred from the following statement by \citet[19--20, fn. 23]{KC94}: ``the reader should be aware that there is a clear phonetic difference between the fricative sound of, for example, instance, \intxt{ʔàj}  \anjc{our \intxt{ʔåx}} `your skin' and the corresponding one in \intxt{tij} \anjc{our \intxt{tix}} `(s)he digs it'. In the groups of
sounds represented in this paper by [x] and [xw], the nonlabialized fricative produced after [ɑ] seems to come nearest to the uvular position''.

Most other varieties of Wichí retain the PW opposition between two guttural fricatives in the onset position only (the reflex of \sound{PW}{*χ} is variably represented as \intxt{x} or \intxt{χ}; the reflex of \sound{PW}{*h} is variably represented as \intxt{h} or \intxt{h̃}), but not in codas: all dialects except 'Weenhayek share the loss of word-final \sound{PW}{*h}.

\booltrue{listing}
\ea
Loss of word-final \sound{PW}{*h} in the Wichí dialects \citep{JT09-cap,JT09-th,MC09,VN14,KC16} \label{wi-h-0}\\
    \begin{xlist}
        \ex \word{PW}{*kʲáˀlah}{lizard} > \wordng{’Wk}{kʲáˀlah}, but Rivadavia~[kʲaˈla], \wordng{LB}{tʃaˀla}
        \ex \word{PW}{*kʲ’anhóh}{armadillo} > \wordng{’Wk}{kʲ’an̥óh}, but Rivadavia~[kʲaˈn̥u], \wordng{LB}{tʃ’an̥u}
        \ex \word{PW}{*-qoh}{mother} > \wordng{’Wk}{-qoh}, but Rivadavia~[\mbox{-}ˈqu], \wordng{LB}{\mbox{-}qu}
        \ex \word{PW}{*\mbox{-}qhǻ\mbox{-}j\mbox{-}hih}{pocket} > \wordng{’Wk}{-qʰǻçih}, but Bazán~[n̩qohˈɲi], El Sauzalito~[n̩kɑˈʰɲi], Misión El Carmen~[n̩kɑˈʰɲi], \wordng{LB}{\mbox{-}qʰoçi}
        \ex \word{PW}{*tsóˀnah}{brocket} > \wordng{’Wk}{tsóˀnah}, but Rivadavia~[tsuˈna], \wordng{LB}{tsuˀna}
    \end{xlist}
\z
\boolfalse{listing}

Some authors report only one guttural fricative for certain Wichí varieties, suggesting a merger of \sound{PW}{*χ} and \intxt{*h} (except where \sound{PW}{*h} was lost word-finally). A case in point is the Misión Chaqueña subdialect of Vejoz, where \citet{VU74} symbolizes the reflexes of both fricatives as \intxt{h}, and the symbols \intxt{x} and \intxt{χ} are not even employed. For the Paraje La Paz subdialect of Vejoz, \citet{AFG067} reports only one guttural fricative, which is claimed to surface as [h] preceding vowels \REF{plp-hupel}–\REF{plp-ahayuk}, as [h] \recind [x] preceding a consonant \REF{plp-opajtit}, and as [x] \recind [χ] before a pause \REF{plp-asinaj}. In the variety of Misión Santa María, \citet{SS07} reports that [h] and [x] freely vary between vowels \REF{msm-ochahuye} and before a consonant \REF{msm-tsonhat}; word-finally, [x] is reported to freely vary with [χ] \REF{msm-lakoj};\footnote{Note that in this example we are in fact dealing with a reflex of \intxt{*xʷ}.} word-initially, only [h] is attested, as in \REF{msm-hosan}–\REF{msm-hokinaj}. In her description of the Guisnay dialect as spoken in Misión La Paz, \citet{MA08} posits a phoneme /h/ with two allophones, [h] (word-initially and word-medially, as in [ahãt] `devil') and [x] (word-finally and -- rarely -- word-medially, as in [isaxije] `handsome'), though she gives no explanation for the fact that both allophones occur between vowels.

\booltrue{listing}
\ea
Paraje La Paz Wichí \citep{AFG067}\\
    \begin{xlist}
        \ex \phonetic{huˈpel}\gloss{shadow} < \wordng{PW}{*hpélʰ} \label{plp-hupel}
        \ex \phonetic{ahaˈjuk}\gloss{mistol\species{Ziziphus mistol} tree} < \wordng{PW}{*ʔahǻj\mbox{-}ukʷ} \label{plp-ahayuk}
        \ex \phonetic{opahˈtit} \recind \phonetic{opaxˈtit}\gloss{I squeeze} < \word{PW}{*n̩\mbox{-}ˈpáχ\mbox{-}tit}{I fix, I join (tr.), I crush} \label{plp-opajtit}
        \ex \phonetic{asiˈnɑx} \recind \phonetic{asiˈnɑχ}\gloss{dog} < \wordng{PW}{*ʔasínåχ} \label{plp-asinaj}
    \end{xlist}
\z
\ea
Misión Santa María Wichí \citep{SS07}\\
    \begin{xlist}
        \ex \phonetic{otʃaˈhuje} \recind \phonetic{otʃaˈxuje}\gloss{I listen} < \wordng{PW}{*n̩\mbox{-}t\mbox{-}kʲ’áˈhu\mbox{-}jeh} \label{msm-ochahuye}
        \ex \phonetic{tsohˈnat} \recind \phonetic{tsoxˈnat}\gloss{knife} < \wordng{PW}{*tsonhat} \label{msm-tsonhat}
        \ex \phonetic{laˈkoχ} \recind \phonetic{laˈkox}\gloss{its foam} < \wordng{PW}{*ɬ̩\mbox{-}qhóxʷ} \label{msm-lakoj}
        \ex \phonetic{hoˈsan}\gloss{ax} < \wordng{PW}{*hósaˀn} \label{msm-hosan}
        \ex \phonetic{hokiˈnax}\gloss{dove} < \wordng{PW}{*hókʷinaχ} \label{msm-hokinaj}
    \end{xlist}
\z
\boolfalse{listing}

We are not convinced that \sound{PW}{*χ} and \intxt{*h} actually merged in Guisnay and Vejoz: recall that \sound{PW}{*χ} in onsets was a low-frequency segment in the protolanguage, and it is thus possible that \citet{VU74} and \citet{AFG067} simply missed the opposition in question, whose functional load is in any case expected to be very low. This is confirmed by an inspection of another source on Vejoz, \citet{MG-MELO15}, which systematically employs the grapheme ‹h› where we reconstruct \intxt{*h} (except, of course, when \intxt{*h} is lost word-finally), and the grapheme ‹j› where we reconstruct \intxt{*χ}, including the onset position, as in ‹lew’ijiyej› `to be startled' \citep[37]{MG-MELO15}, from \wordng{PW}{*\mbox{-}ˀwíˈχij\mbox{-}eχ}. It remains to be established whether any Wichí dialect has effectively merged \sound{PW}{*χ} and \intxt{*h}.

In Lower Bermejeño Wichí, there is a further process involving morpheme-final instances of /χ/ preceded by front vowels. In such cases, /χ/ surfaces as [x] when it occurs in the coda position, as we have seen above. However, when it resyllabifies as an onset before a vowel\mbox{-} or a /h/-initial morpheme, /χ/ palatalizes to [ʃ] in Lower Bermejeño \citep[109--111]{VN14}, as shown in \REF{lb-esh}.

\ea
Southeastern Wichí (Lower Bermejeño) \citep[110--111]{VN14} \label{lb-esh}\\
    \begin{xlist}
        \ex\gll ʔi-leχ-eχ~\phonetic{ʔiˈleʃeχ}\\
                3.{\textsc{i}}-wash-\APPL\\
                \glt `s/he washes it with'
        \ex\gll n̩-leχ-hen~\phonetic{n̩ˌleˈʃen}\\
                1-wash-{\textsc{hen}}\\
                \glt `I wash them'
        \ex\gll ʔi-leχ-hu~\phonetic{ʔiˈleʃu}\\
                3.{\textsc{i}}-wash-\APPL\\
                \glt `s/he washes it from inside'
        \ex\gll ʔi-that-eχ-hu~\phonetic{ʔiˌtʰãˈteʃu}\\
                3.{\textsc{i}}-throw-\APPL-\APPL\\
                \glt `s/he washes it from inside'
        \ex\gll n̩-tʃoχ-eχ-e~\phonetic{n̩ˈtʃoχeʃe}\\
                1-bring-\APPL-\LOC\\
                \glt `s/he barters it'
        \ex\gll ˀnojiχ=na~\phonetic{ˀnoˈjiʃa}\\
                path=this\\
                \glt `this path'
        \ex\gll ha-ʔa-qa-tuweχ-hi~\phonetic{hãˌʔaqaˌtuweˈʃi}\\
                \NEG-2.\textsc{poss}-{\textsc{alz}}-jug-\NEG\\
                \glt `it is not your jug'
    \end{xlist}
\z

We suggest that the positionally conditioned palatalization of /χ/ arose in Lower Bermejeño due to an overgeneralization of an inherited process, whereby /q/ in codas alternates with /tʃ/ in onsets, as in \REF{ex:q-ch:lbw} above. \citet[138--139]{MC09} also documents the palatalization of /χ/ (/x/ in her notation) in the varieties spoken in Colonia Muñiz (\wordnl{ˌtiʃeˈlis}{scissors}) and Bazán (\wordnl{ˌtiʃeˈlis}{scissors}, \wordnl{iˌtuˈweʃa}{s/he makes a hole}), as well as -- less consistently and with a different outcome -- in Teniente Fraga (\wordnl{iˌtuˈweɕa}{s/he makes a hole}, but \wordnl{ˌtixeˈlis}{scissors}) and Misión El Carmen (\wordnl{iˌtuˈwexʲa}{s/he makes a hole}, but \wordnl{ˌtixeˈlis}{scissors}).

\subsubsection{\sound{PW}{*xʷ}} \label{wi-xw}

\sound{Proto-Wichí}{*xʷ} typically yields [xʷ] or [fʷ] in the Wichí dialects \citep[128]{EN71}. \citet[138]{MC09} states that “although the use of both variants is observed in all varieties \anjc{i.e., Teniente Fraga, Misión El Carmen, Colonia Muñiz, Bazán, and El Sauzalito} and in all age ranges, it is more common to hear the labiodental realization in the Eastern varieties than in the Western ones \anjc{translation ours}”: compare Misión El Carmen [ˈxʷala] and Colonia Muñiz [ˈfʷala] ‘day’ (<~\wordng{PW}{*ˣˈxʷála}), El Sauzalito [tʃexʷ] and Bazán [tʃefʷ] ‘sweat’ (<~\wordng{PW}{*kʲúxʷ}). Free variation between [xʷ] and [fʷ] is also documented in Misión Santa María \REF{msm-jfw}. A third common reflex of \sound{PW}{*xʷ} is [hʷ], as attested in the Misión Chaqueña subdialect of Vejoz \citep{VU74} and in the Misión La Paz subdialect of Vejoz \citep[44]{MA08}. All three allophones have been attested in the subdialect(s) of Guisnay described in \citet[154]{AFG-SS-09}, where the phoneme in question is realized as [xʷ] \recind [fʷ] word-initially \REF{wzn-duck}, as [xʷ] \recind [fʷ] \recind [hʷ] intervocalically, as in \REF{wzn-scratch}–\REF{wzn-chair},\footnote{The etymon of \REF{wzn-chair} did not in fact contain a \intxt{*xʷ} in Proto-Wichí. Both the consonant and the vowel appear to have evolved irregularly.} and as [xʷ] word-finally \REF{wzn-eats}.

\booltrue{listing}
\ea
Misión Santa María Wichí \citep{SS07} \label{msm-jfw}\\
    \begin{xlist}
        \ex \phonetic{fʷihˈnjoɬ} \recind \phonetic{xʷihˈnjoɬ}\gloss{charcoal} < \word{PW}{*xʷijhó\mbox{-}lʰ}{charcoal.\PL}
        \ex \phonetic{afʷenˈtʃe} \recind \phonetic{axʷenˈtʃe}\gloss{bird} < \wordng{PW}{*ʔaxʷén\mbox{-}kʲe}
    \end{xlist}
\z
\ea
Guisnay Wichí \citep[154]{AFG-SS-09}\\
    \begin{xlist}
        \ex \phonetic{xʷoʔˈjax} \recind \phonetic{fʷoʔˈjax}\gloss{Muscovy duck} < \wordng{PW}{*xʷóqˀjaχ} \label{wzn-duck}
        \ex \phonetic{oxʷiˈlax} \recind \phonetic{ofʷiˈlax}\gloss{I scratch myself} < \wordng{PW}{*n̩\mbox{-}xʷílåχ} \label{wzn-scratch}
        \ex \phonetic{oxʷeˈwet} \recind \phonetic{ohʷeˈwet}\gloss{my chair} < \word{PW}{*ń̩\mbox{-}ho\mbox{-}wet}{my seat} \label{wzn-chair}
        \ex \phonetic{ˈtuxʷ}\gloss{s/he eats} < \wordng{PW}{*tuxʷ} \label{wzn-eats}
    \end{xlist}
\z
\boolfalse{listing}

\cits{MC09} claim regarding the geographical distribution of the allophones [xʷ] and [fʷ] is confirmed by other sources on Wichí. In 'Weenhayek, \citet{KC94} documents only [xʷ]. \citet{AFG067} reports that [xʷ] is predominant in the Paraje La Paz subdialect of Vejoz, where [fʷ] has been attested in only one lexeme (and even then it is reported to be in free variation with [xʷ]: [qafʷaˈjax] \recind [qaxʷaˈjax] ‘magic’). Moving in the southeast direction, in Rivadavia only [xʷ] (alongside its metathesized variant [ʷx]) is attested \citep[45--46]{JT09-th}. By contrast, in the southeastern extreme of the Wichí-speaking zone [fʷ] is reported as the main allophone of the phoneme in question \citep[51]{VN14}, where [xʷ] is only occasionally found in free variation with [fʷ] (\wordnl{lafʷuɬ \recind la[xʷ]uɬ}{her/his musical instrument}).

In the varieties of Bazán (younger speakers) and Rivadavia, /xʷ/ may surface as prelabialized rather than postlabialized in the coda position:

\booltrue{listing}
\ea
Bazán Wichí\\
    \begin{xlist}
        \ex /laxʷtʃa/~\phonetic{laxəˈtʃa} (older) \recind \phonetic{lawxˈtʃa} (younger)\gloss{her/his father} \citep[140]{MC09} < \wordng{PW}{*ɬ̩\mbox{-}xʷkʲah}
        \ex /ɬexʷ/~\phonetic{ˈxlexɯ̥} (older) \recind \phonetic{ˈxlewx} (younger)\gloss{its wing} \citep[140]{MC09} < \wordng{PW}{*ɬ\mbox{-}exʷ}
        \ex /axʷtsinaχ/~\phonetic{awhtsiˈnah}\gloss{fork} \citep[6]{JB09}
        \ex /xʷexʷ/~\phonetic{ˈhwewh}\gloss{finger} \citep[6]{JB09} < \wordng{PW}{*\mbox{-}xʷuxʷ}
    \end{xlist}
\z

Yet in other varieties, /xʷ/ may optionally lose labialization in the coda position. The following examples are from Paraje La Paz, but the delabialization in the word for `father' \REF{plp-ojcha} is also seen in other dialects, such as ’Weenhayek \citep[60]{KC16}.

\ea
Paraje La Paz Wichí \citep{AFG067}\\
    \begin{xlist}
        \ex /ɬexʷ/~\phonetic{ˈɬexʷ} \recind \phonetic{ˈɬex}\gloss{its wing} < \wordng{PW}{*ɬ\mbox{-}exʷ}
        \ex /oxʷtʃa/~\phonetic{oxˈtʃa} \recind \phonetic{oxˈtɕa}\gloss{my father} < \wordng{PW}{*ń̩\mbox{-}xʷkʲah}\label{plp-ojcha}
    \end{xlist}
\z
\boolfalse{listing}

In the Rivadavia subdialect of Southeastern Wichí, /xʷ/ is delabialized to [x] before a high rounded vowel. For example, Rivadavia [nuxu] ‘all’ \citep[63]{JT09-cap} clearly goes back to \wordng{PW}{*noxʷ\mbox{-}o}, as suggested by its cognates in other dialects: \wordng{LB}{nufʷu}, \wordng{'Wk}{noxʷo} \citep{MC09,VN14,KC16}.

\subsubsection{\sound{PW}{*ɬ}}

\sound{PW}{*ɬ} is articulated as \phonetic{ɬ} in most Wichí varieties. \citet[137--138]{MC09} reports, however, that it is typically realized as [xl] in Lower Bermejeño Wichí as spoken in Colonia Muñiz \REF{wi:lh:cm}.

\booltrue{listing}
\ea\label{wi:lh:cm}
Colonia Muñiz Wichí \citep[137--8]{MC09}\\
    \begin{xlist}
        \ex \phonetic{n̩ˈxlam}\gloss{s/he} (probably mistranslated) <~\word{PW}{*n̩\mbox{-}ɬ\mbox{-}áˀm}{I} 
        \ex \phonetic{tʃ’n̩ˈxlos}\gloss{my son} < \wordng{PW}{*n̩\mbox{-}ɬ\mbox{-}ǻs}
        \ex \phonetic{xleˈtek}\gloss{her/his head} < \wordng{PW}{*ɬ\mbox{-}éteq}
        \ex \phonetic{n̩tʃemˈxli}\gloss{I work} < \wordng{PW}{*n̩\mbox{-}t\mbox{-}kʲúm\mbox{-}ɬih}
        \ex \phonetic{aˈxlu}\gloss{iguana} < \wordng{PW}{*ʔáɬu}
    \end{xlist}
\z
\boolfalse{listing}

\citet[162]{AFG-SS-09} document [x͡l̥] as a free variant of [ɬ] in the speech of a consultant from San Luis, a community located not far from Santa Victoria Este, as in \wordnl{ɬaˈmis \recind x͡l̥aˈmis}{necklace}. \citet[50–51]{MA08} explicitly claims that in the Misión La Paz subdialect of the Guisnay dialect of Wichí the sound in question is articulated as a voiceless approximant [l̥] and not as a fricative [ɬ] \REF{wi:lh:mlp}. In ’Weenhayek, \citet[31]{KC94} describes the sound in question as [ll̥] and analyzes it as an underlying cluster /lh/ (see \sectref{wi-ch} on other clusters of this type); in this book we represent it as \intxt{ɬ}.

\booltrue{listing}
\ea\label{wi:lh:mlp}
Misión La Paz Wichí \citep[50–51]{MA08} \\
    \begin{xlist}
        \ex \phonetic{ˈl̥up}\gloss{its nest} < \wordng{PW}{*ɬ\mbox{-}úp}
        \ex \phonetic{oniˈpil̥}\gloss{my stomach} < \wordng{PW}{*ń̩\mbox{-}nipiɬ}
        \ex \phonetic{qɑl̥qɑl̥tɑx}\gloss{turkey} < \wordng{PW}{*qáɬqaɬ\mbox{-}taχ}
    \end{xlist}
\z
\boolfalse{listing}

\subsubsection{Glottalized consonants}\label{wi-gl-cons}

In Proto-Wichí, the following glottalized consonants are reconstructed: \intxt{*p'}, \intxt{*t'}, \intxt{*ts'}, \intxt{*kʲ'}, \intxt{*q'}, \intxt{*kʷ'} (exceedingly rare), \intxt{*ˀw}, \intxt{*ˀl}, \intxt{*ˀj}, \intxt{*ˀm}, and \intxt{*ˀn}.

We start by discussing the realization of the glottalized stops and affricates in the dialects of Wichí. These are described as ejective consonants by authors such as \citet[128--131]{MC09} and \citet[49--51, 79--82]{VN14} for Lower Bermejeño Wichí, \citet{VU74} for the Misión Chaqueña subdialect of Vejoz, or \citet{MA08} for the Misión La Paz subdialect of Guisnay, and we reconstruct this state of affairs for Proto-Wichí. Some dialects, however, appear to have innovated in transforming ejectives into implosives, at least at some points of articulation.

This process is most advanced in the ’Weenhayek dialect, with its four implosive phonemes. \citet[29]{KC94} reports that the ’Weenhayek pronounce what he analyzes as ``/pʔ/, /tʔ/, /kʔ/, and /qʔ/ as glottalic ingressives (implosives), whereas the sounds with fricative release, /kyʔ/ and /tsʔ/ \anjc{our \intxt{kʲ’} and \intxt{ts’}}, are glottalic egressives (ejectives)''. In her study of the variety spoken in Paraje La Paz, \citet{AFG067} documents the glottalized labial stop as varying between [ɓ̥] and [ɓ], the glottalized alveolar stop as varying between [ɗ̥] and [ɗ], and the glottalized velar stop as varying between [ɠ̊] and [ɠ] \REF{plp-deg-bo}–\REF{plp-deg-guse}. The reflex of \sound{PW}{*kʲ’}, on the other hand, is apparently articulated as the plain affricate [tʃ], as in [otʃoˈte] `my ear'. Deglottalization may affect other consonants as well \REF{plp-deg-laqe}.

\booltrue{listing}
\ea
Paraje La Paz Wichí \citep{AFG067}\\
    \begin{xlist}
        \ex \phonetic{ˈɓ̥o}\gloss{to roast, to burn} < \wordng{PW}{*\mbox{-}p’o} \label{plp-deg-bo}
        \ex \phonetic{siˈɓ̥a}\gloss{soldier} < \word{PW}{*sip’å}{hat; fish sp.\species{Sorubim lima (?)}}
        \ex \phonetic{tsoˈɓ̥a} \recind \phonetic{tsoˈɓa}\gloss{heel bone} < (?) \wordng{PW}{*sóp’a}{wax}
        \ex \phonetic{ˈɗ̥ɯn}\gloss{hard} < \wordng{PW}{*t’ún}
        \ex \phonetic{oˈɗ̥ek}\gloss{I eat (intr.)} < \wordng{PW}{*n̩\mbox{-}t\mbox{-}’eq}
        \ex \phonetic{ɗiˈkʷa} \recind \phonetic{ɗ̥iˈkʷa}\gloss{swollen} < \wordng{PW}{*t’ukʷa}
        \ex \phonetic{ɗuˈɬu} \recind \phonetic{ɗ̥uˈɬu}\gloss{her/his urine} < \wordng{PW}{*t\mbox{-}’úɬu}
        \ex \phonetic{ɠ̊oˈnek}\gloss{sweet} < \wordng{PW}{*haq’óneq}
        \ex \phonetic{xʷaˈɠ̊an}\gloss{it is blue} < \wordng{PW}{*ʔixʷáq’an}
        \ex \phonetic{ɠ̊uˈse} \recind \phonetic{ɠuˈse}\gloss{jaw} < \word{PW}{*\mbox{-}q’úse}{beard, chin} \label{plp-deg-guse}
        \ex \phonetic{laˈqɛ} \recind \phonetic{lɑˈqɛ}\gloss{it shines} < \wordng{PW}{*laq’e}\label{plp-deg-laqe}
    \end{xlist}
\z
\boolfalse{listing}

In some dialects, the implosive realization is restricted to the reflexes of \intxt{*p'} and \intxt{*t'}, but not to those of the glottalized consonants articulated further back. For example, \citet[34--35]{JT09-th} explicitly claims that in the Rivadavia subdialect of Southeastern Wichí the labial and dental glottalized stops [ɓ], [ɗ] are articulated as implosives rather than ejectives, whereas \sound{PW}{*ts’}, \intxt{*kʲ’}, and \intxt{*q’} are deglottalized to \intxt{ts}, \intxt{kʲ}, \intxt{q} in that variety \REF{wi:degl:riv}. Similarly, \citet[128--131]{MC09} documents implosive reflexes of \intxt{*p'} and \intxt{*t'} in the varieties of El Sauzalito \REF{wi:degl:elsauz} and Teniente Fraga \REF{wi:degl:tf}; unlike in Rivadavia, these varieties do not show systematic deglottalization of the remaining glottalized stops (El Sauzalito \intxt{ts’}, \intxt{kʲ’}, \intxt{k’}; Teniente Fraga \intxt{ts’}, \intxt{kʲ \recind tʃ \recind ʔtɕ}, \intxt{k’}). In their description of the variety spoken by a consultant from Ingeniero Juárez, \citet{LCB-MBC09} systematically transcribe the reflex of \sound{PW}{*t’} as [ɗ] (no data on other points of articulation are available in the cited work).

\booltrue{listing}
\ea
Rivadavia Wichí \citep{JT09-th} \label{wi:degl:riv}\\
    \begin{xlist}
        \ex \wordnl{tatsi}{rufous hornero} < \wordng{PW}{*táts’i}
        \ex \wordnl{ha\mbox{-}kʲute}{your ear} < \wordng{PW}{*ha\mbox{-}kʲ’óte}
        \ex \wordnl{la\mbox{-}qax}{her/his mouth} < \wordng{PW}{*ɬ̩\mbox{-}q’áχ}
    \end{xlist}
\z
\ea
El Sauzalito Wichí \citep[128--131]{MC09} \label{wi:degl:elsauz}\\
    \begin{xlist}
        \ex \phonetic{muˈɓi}\gloss{white heron} < \wordng{PW}{*móp’i}
        \ex \phonetic{ɗiˈsan}\gloss{its flesh, meat} < \wordng{PW}{*t\mbox{-}’isaˀn}
    \end{xlist}
\z
\ea
Teniente Fraga Wichí \citep[128--131]{MC09} \label{wi:degl:tf}\\
    \begin{xlist}
        \ex \phonetic{muˈɓ̥i}\gloss{white heron} < \wordng{PW}{*móp’i}
        \ex \phonetic{ɗiˈsan}\gloss{its flesh, meat} < \wordng{PW}{*t\mbox{-}’isaˀn}
    \end{xlist}
\z

The variety spoken in Misión El Carmen is unusual in that it debuccalizes \sound{PW}{*t'}, \intxt{*kʲ'}, and \intxt{*q'} to [ʔ], [ʔʲ] \recind [ˀj], and [ʔ], respectively, as in \REF{mec-kyekye}–\REF{mec-toaj}. \sound{PW}{*ts'} is preserved as an ejective affricate [ts'] in Misión El Carmen, whereas the reflex \sound{PW}{*p'} is quite unexpectedly attested as [ɗ] (\intxt{sic}) in the only available example \REF{mec-mudi}.

\ea
Misión El Carmen Wichí \citep{MC09} \label{mec-deb}\\
    \begin{xlist}
        \ex \phonetic{kʲeˈʔʲe} (older) \recind \phonetic{tɕeˀˈeʔ} (younger)\gloss{parakeet sp.} < \wordng{PW}{*kʲékʲ’e} \label{mec-kyekye}
        \ex \phonetic{ʔʲuˈte}\gloss{ear} < \wordng{PW}{*\mbox{-}kʲ’óte}
        \ex \phonetic{ləˈʔax}\gloss{her/his mouth} < \wordng{PW}{*ɬ̩\mbox{-}q’áχ}
        \ex \phonetic{toˈʔɑx}\gloss{one’s mouth} < \word{PW}{*\mbox{-}q’áχ}{mouth} \label{mec-toaj}
        \ex \phonetic{muˈɗi}\gloss{white heron} < \wordng{PW}{*móp’i}\label{mec-mudi}
    \end{xlist}
\z
\boolfalse{listing}

The debuccalization has also been attested in \citet{AFG067} as an optional phenomenon in the variety of Paraje La Paz \REF{plp-deb}. \citet[167–168]{AFG-SS-09} report several examples of debuccalization in the variety of Lapacho Mocho, as in \wordnl{oʔahɬitʃu}{my tongue} < \wordng{PW}{*ń̩\mbox{-}q’aχ\mbox{-}ɬ\mbox{-}ɪkʲ’u}.

\booltrue{listing}
\ea
Paraje La Paz Wichí \citep{AFG067} \label{plp-deb}\\
    \begin{xlist}
        \ex \phonetic{ʔiˈkʷa} \recind \phonetic{ɗiˈkʷa} \recind \phonetic{ɗ̥iˈkʷa}\gloss{swollen} < \wordng{PW}{*t’ukʷa}
        \ex \phonetic{ʔuˈse} \recind \phonetic{ɠ̊uˈse} \recind \phonetic{ɠuˈse}\gloss{jaw} < \word{PW}{*\mbox{-}q’úse}{beard, chin}
    \end{xlist}
\z

Finally, \citet[125]{MC09} also reports that in some cases glottalized stops may be optionally articulated as aspirated \REF{wi-glotasp}, though in \REF{mec-qha} the aspirated variant is in fact more conservative.

\ea
Wichí \citep[125]{MC09} \label{wi-glotasp}\\
    \begin{xlist}
        \ex Colonia Muñiz [laˈp’i] \recind [laˈpʰi]\gloss{tayra} < \wordng{PW}{*ɬ̩p’í}
        \ex El Sauzalito [ˈt’i] \recind [ˈtʰi]\gloss{its liquid} < \wordng{PW}{*t\mbox{-}’í}
        \ex Misión El Carmen [k’ɑ] \recind [ˈkʰɑ] \recind [ˈqʰa]\gloss{no} < \wordng{PW}{*qhá} \label{mec-qha}
    \end{xlist}
\z
\boolfalse{listing}

The fate of the glottalized sonorants \intxt{*ˀw}, \intxt{*ˀl}, \intxt{*ˀj}, \intxt{*ˀm}, and \intxt{*ˀn} in the dialects of Wichí is less clear. These consonants are preserved in ’Weenhayek as described by \citet{KC94,KC16}, in the Misión Chaqueña subdialect of Vejoz as described by \citet{MG-MELO15}, in Lower Bermejeño Wichí as described by \citet{VN14}, and in the Misión La Paz subdialect of Guisnay as described by \citet{MA08}. Other sources that deal with the same varieties, such as \citet{VU74} and \citet{JB09}, may at times fail to document the glottalization contrast in sonorants, possibly due to mistranscription on part of non-Wichí researchers. The phonological descriptions of other Wichí dialects do not mention the existence of glottalized sonorants and usually transcribe the consonants in question as plain sonorants, as in \REF{wi-plp-degs}–\REF{wi-riv-degs}. In Misión Santa María, Proto-Wichí glottalized sonorants are usually reflected as plain sonorants, as in \REF{msm-tree}–\REF{msm-blood}, but occasionally clusters of the shape [ʔC] are attested \REF{msm-man}.

\booltrue{listing}
\ea
Paraje La Paz Wichí \citep{AFG067} \label{wi-plp-degs}\\
    \begin{xlist}
        \ex \phonetic{woˈna}\gloss{kind of bee\species{bala}} < \wordng{PW}{*wóˀnah}
        \ex \phonetic{ˈwet}\gloss{place} < \wordng{PW}{*\mbox{-}ˀwet}
        \ex \phonetic{ˈwen}\gloss{to find} < \word{PW}{*\mbox{-}ˀwén}{to see}
        \ex \phonetic{tʃaˈla}\gloss{lizard} < \wordng{PW}{*kʲáˀlah}
    \end{xlist}
\z
\ea
Rivadavia Wichí \citep[68, 146, 157, 220]{JT09-th} \label{wi-riv-degs}\\
    \begin{xlist}
        \ex \wordnl{\mbox{-}nojix}{road} < \wordng{PW}{*ˀnǻjiχ}
        \ex \wordnl{halo}{tree} < \wordng{PW}{*haˀlå}
        \ex \wordnl{wahat}{fish} < \wordng{PW}{*ˀwáhat}
        \ex \wordnl{ja\mbox{-}wen}{we see} < \wordng{PW}{*ˣjá{-}ˀwen}
    \end{xlist}
\z
\ea
Misión Santa María Wichí \citep{SS07}\\
    \begin{xlist}
        \ex \phonetic{haˈlaʔ}\gloss{tree} < \wordng{PW}{*haˀlå} \label{msm-tree}
        \ex \phonetic{woˈjiːs}\gloss{blood} < \wordng{PW}{*ˀwoj\mbox{-}ís}\label{msm-blood}
        \ex \phonetic{hiʔˈno}\gloss{man} < \wordng{PW}{*hiˀno}\label{msm-man}
    \end{xlist}
\z
\boolfalse{listing}

In the word-final position, glottalized sonorants merge with their plain counterparts in most varieties of Wichí, or at least most sources do not transcribe the distinction in a consistent way. \citet{KC16} is the most reliable source in this regard. Note that the plain sonorants of Proto-Wichí are devoiced before a pause in ’Weenhayek, whereas the glottalized sonorants are realized as sequences of the type [Cʔ] in that position: \word{PW}{*hósaˀn}{ax} >~\wordng{’Wk}{hósaˀn} (phonetically \phonetic{hõːˈsanʔ}), \word{PW}{*kʲ’utháˀn}{thistle} >~\wordng{’Wk}{kʲ’utʰáˀn} \phonetic{kʲ’uˈtʰãːnʔ}, but \word{PW}{*n̩\mbox{-}jáhin}{I watch} >~\wordng{’Wk}{ʔõ\mbox{-}jáhin̥} \phonetic{ʔõjaːˈhĩn̥} \citep{KC16}. Sequences of the type [Cʔ] before in pause have also been attested in the Misión La Paz subdialect of Guisnay, as in [hõˈsanʔ] `ax' and [kʲ’uˈtanʔ] `thistle', but since they also show up in words that originally ended in a plain sonorant ([ojaˈhĩnʔ] `I watch'), it is not clear to us that Guisnay retains the original opposition. Other dialects, such as Lower Bermejeño, have entirely lost the contrast in question in codas: \word{LB}{husan}{ax}, \wordnl{tʃ’itʰan}{thistle}, \wordnl{n̩\mbox{-}jahin}{I watch} \citep{VN14}.

\subsubsection{Consonants plus \sound{PW}{*h}}\label{wi-ch}

In Wichí, underlying sequences of plain supraglottal consonants (stops or nasals) and /h/ (in some analyses, /h̃/) in the onset position are typically articulated as single sounds (aspirated stops or devoiced nasals). Some authors, such as \citet{VN14}, map the resulting sounds to independent phonemes, whereas others, such as \citet{KC94}, analyze them as underlying consonant clusters. The following vowel is phonetically nasalized at least in some dialects thanks to rhinoglottophilia (\sectref{wi-nasalization}). No sequences involving a fricative followed by */h/ existed in Proto-Wichí thanks to a diachronic sound change whereby */h/ was deleted after fricatives (\sectref{wi-consonant-dorsal}).

In ’Weenhayek, \citet[29, 31]{KC94} analyzes the sounds in question as underlying clusters with /h/ as a second element. Of these, /ph/, /th/, /tsh/, /kh/, /kʲh/, and /qh/ are phonetically realized as aspirated consonants. The clusters involving a sonorant and /h/ are realized with a devoiced nasal phase: /mh/ [m̥m], /nh/ [n̥n], /wh/ [ŋ̥w], /jh/ [ɲ̥j], /lh/ [l̥l]. In this book, these sounds are represented as \intxt{pʰ}, \intxt{tʰ}, \intxt{tsʰ}, \intxt{kʰ}, \intxt{kʲʰ}, \intxt{qʰ}, \intxt{m̥}, \intxt{n̥}, \intxt{ʍ}, \intxt{ç}, and \intxt{ɬ}.

In Lower Bermejeño Wichí, \citet[49–53]{VN14} ascribes phonemic status to the following consonants: /pʰ/, /tʰ/, /tsʰ/, /qʰ/, /n̥/ (optionally articulated as breathy voiced [n̤]), /j̊/, /w̥/. The sounds [m̥] (also pronounced as breathy voiced [m̤]) and [tʃʰ] are claimed by \citet{VN14} to be allophonic realizations of /mh/ and /tʃh/, respectively, as in \wordnl{nom̥en}{they come} (underlying /nom+hen/) and \wordnl{totʃʰajaχ}{s/he worships a god} (no underlying representation given). In this book, /j̊/, /w̥/ are represented as \intxt{ç} and \intxt{ʍ}.

In the Rivadavia subdialect of Southeastern Wichí, \citet[27–30]{JT09-th} identifies the aspirated stops /pʰ/, /tʰ/, /qʰ/ as phonemes. \sound{PW}{*tsh} is reflected as \intxt{ts} in Rivadavia, as in \wordnl{tsot\mbox{-}oj}{animals}, \wordnl{watsan}{it is green}. The reflex of \sound{PW}{*kʲh} in Rivadavia is unknown. As for sequences of a nasal and /h/, \citet[38–41]{JT09-th} analyzes the instances of [m̥] and [n̥] as /mh/ and /nh/ (or /mh̃/ and /nh̃/ in \citnp{JT09-cap}) when there is morphological evidence that clearly shows that there is a morpheme ending in a nasal and another one starting with /h/ (/h̃/). In her discussion of the tautomorphemic occurrences of [m̥] and [n̥], as in \REF{wi-cx-riv-nemhe}–\REF{wi-cx-riv-atsinha}, \citet[41]{JT09-th} states that the low number of examples makes it implausible to posit /m̥/ and /n̥/ as phonemes and concludes that these segments are ``residues of a phonological opposition that no longer exists''. The Rivadavia reflex of \sound{PW}{*jh} is articulated as a voiceless nasalized approximant [j̃̊] \recind [h̃j̃], considered by \citet[48]{JT09-th} and \citet[79]{JT09-cap} to be a realization of /jh/, and is attested in roots such as \REF{wi-cx-riv-ember}–\REF{wi-cx-riv-clay}, among others.\footnote{In the closely related variety of Southeastern Wichí spoken by \cits{LCB-MBC09} consultant from Ingeniero Juárez, the reflex of \sound{PW}{*jh} is documented as a voiced nasalized approximant [j̃]: [lɔˈj̃ɛ̃n] `they are alive', [tɔkɐtˈj̃ɛ̃n] `we cook', [tɐtɔˈj̃ɪ̃ɗɛ] `they do not lose', [jɛˈj̃ɪ̃ɗɛ] `they are not sharpened', [tɐˈj̃ɪ̃] `forest', [fʷɪˈj̃ʊ̃] `charcoal', [nɪ̃ˈj̃ɔ̃j] `ropes', [nɔ̃ˈj̃ɔ̃j] `footprints' \citep[102–103]{LCB-MBC09}.} Finally, the reflex of \sound{PW}{*wh} in Rivadavia is documented as [ʍ̃] and analyzed as /hw/ or /wh/, as in \REF{wi-cx-riv-married}–\REF{wi-cx-riv-talk}.

\booltrue{listing}
\ea
Rivadavia Wichí \citep[38–41, 48]{JT09-cap,JT09-th} \label{wi-cx-riv}\\
    \begin{xlist}
        \ex \wordnl{ˈnem̥e}{not anymore} < \wordng{PW}{*nem\mbox{-}hV} \label{wi-cx-riv-nemhe}
        \ex \wordnl{n̥ete}{injure} < \wordng{PW}{*\mbox{-}nhéte}
        \ex \wordnl{pon̥on}{pepper} < \wordng{PW}{*pǻnhån}
        \ex \wordnl{atsin̥a}{woman} < \wordng{PW}{*ʔatsínha} \label{wi-cx-riv-atsinha}
        \ex \wordnl{hʷisj̃̊u}{ember} < \wordng{PW}{*xʷijho(ʔ)} \label{wi-cx-riv-ember}
        \ex \wordnl{ij̃̊ox}{some} < \wordng{PW}{*ʔi\mbox{-}jhåχ}
        \ex \wordnl{ta\mbox{-}qataj̃̊\mbox{-}ẽn}{they cook} < \wordng{PW}{*ta\mbox{-}qátaj\mbox{-}heˀn}
        \ex \wordnl{ij̃̊ot}{clay} < \wordng{PW}{*ʔijhåt} \label{wi-cx-riv-clay}
        \ex \wordnl{ta\mbox{-}ʍ̃ajej}{s/he gets married} < \wordng{PW}{*ta\mbox{-}whájej} \label{wi-cx-riv-married}
        \ex \wordnl{ta\mbox{-}ʍ̃ijej}{s/he talks} \label{wi-cx-riv-talk}
    \end{xlist}
\z
\boolfalse{listing}

In other dialects, the reflexes of the clusters of the shape \intxt{*Ch} are not so thoroughly documented. For example, the variety of Paraje La Paz is reported to lack aspirated stops \citep{AFG067}; concrete examples of deaspirated stops involve \sound{PW}{*tsh}~>~\intxt{ts}, as in \REF{cx-plp-green}–\REF{cx-plp-children}, and \sound{PW}{*ph}~>~\intxt{ɓ̥} \REF{cx-plp-isit}. As for Proto-Wichí clusters of the shape ``sonorant + \intxt{*h}'', all available examples involve \sound{PW}{*nh} (variably reflected as \intxt{n} or \intxt{hn}) or \sound{PW}{*jh} (reflected as \intxt{hnj}), as in \REF{cx-plp-tusca}–\REF{cx-plp-clay}. In the variety of Misión Santa María, at least \sound{PW}{*tsh} and \intxt{*qh} are deaspirated, as in \REF{cx-msm-girl}–\REF{cx-msm-wildcat}; \sound{PW}{*nh} and \sound{PW}{*jh} are reflected as preaspirated nasals in that variety, as in \REF{cx-msm-earth}–\REF{cx-msm-embers}. In the Misión La Paz subdialect of Guisnay, aspirated stops have not been attested in the onset position \citep{MA08}, suggesting that Proto-Wichí clusters of the shape ``stop~+~\intxt{*h}'' may have undergone deaspiration, as in \REF{cx-mlp-boys}. At least the clusters \intxt{*mh}, \intxt{*nh}, \intxt{*jh}, however, are retained as voiceless nasals \intxt{m̥}, \intxt{n̥}, \intxt{n̥ʲ} in Misión La Paz, as in \REF{cx-mlp-gorges}–\REF{cx-mlp-forest}; \citet[98]{MA08} also notes that voiceless nasals may be optionally realized as voiced. The reflex of \sound{PW}{*wh} in Misión La Paz is documented as \intxt{hʷ}, as in \REF{cx-mlp-pots}–\REF{cx-mlp-holes}. Note, however, that \sound{PW}{*wh}~>~\intxt{hʷ} does not completely merge with \sound{PW}{*xʷ}~>~\intxt{hʷ}: the following vowel is nasalized in the former situation but not in the latter.

\booltrue{listing}
\ea
Paraje La Paz Wichí \citep{AFG067}\\
    \begin{xlist}
        \ex \phonetic{waˈtsan}\gloss{it is green} < \wordng{PW}{*ˀwátshan} \label{cx-plp-green}
        \ex \phonetic{naˈtsas}\gloss{children} < \wordng{PW}{*nǻʔtsha\mbox{-}s} \label{cx-plp-children}
        \ex \phonetic{otajˈɓ̥ɑ}\gloss{I sit} < \wordng{PW}{*n̩\mbox{-}t\mbox{-}táj\mbox{-}phå} \label{cx-plp-isit}
        \ex \phonetic{naˈtek}\gloss{tusca bush} < \wordng{PW}{*ˣnháte\mbox{-}q} \label{cx-plp-tusca}
        \ex \phonetic{ɑnˈjax}\gloss{wild bean} < \wordng{PW}{*ʔǻnhjaχ}
        \ex \phonetic{tsohˈnat}\gloss{knife} < \wordng{PW}{*tsonhat}
        \ex \phonetic{hohˈnat}\gloss{earth} < \wordng{PW}{*honhat}
        \ex \phonetic{ɑhnalaˈtax}\gloss{capybara} < \wordng{PW}{*ʔǻnhålå\mbox{-}taχ}
        \ex \phonetic{ohˈnus}\gloss{my nose} < \wordng{PW}{*ń̩\mbox{-}nhus}
        \ex \phonetic{usehˈna}\gloss{\textit{anco} squash} < \wordng{PW}{*ʔúsenha}
        \ex \phonetic{ihˈnjɑt}\gloss{clay} < \wordng{PW}{*ʔijhåt} \label{cx-plp-clay}
    \end{xlist}
\z\ea
Misión Santa María Wichí \citep{SS07}\\
    \begin{xlist}
        \ex \phonetic{ɬuˈtsa}\gloss{girl} < \wordng{PW}{*ɬútsha} \label{cx-msm-girl}
        \ex \phonetic{waˈtsan}\gloss{it is green} < \wordng{PW}{*ˀwátshan}
        \ex \phonetic{tsaˈwet}\gloss{animal} < \wordng{PW}{*ˣtshǻwet}
        \ex \phonetic{silaˈka}\gloss{wild cat} < \wordng{PW}{*silǻqhåj} \label{cx-msm-wildcat}
        \ex \phonetic{hohˈnat}\gloss{earth} < \wordng{PW}{*honhat} \label{cx-msm-earth}
        \ex \phonetic{tsohˈnat} \recind \phonetic{tsoxˈnat} \recind \phonetic{tsoh̃ˈnat}\gloss{knife} < \wordng{PW}{*tsonhat}
        \ex \phonetic{ohˈnũs} \recind \phonetic{õh̃ˈnus}\gloss{my nose} < \wordng{PW}{*ń̩\mbox{-}nhus}
        \ex \phonetic{tah̃ˈnĩ}\gloss{mountain} < \word{PW}{*tájhi}{forest}
        \ex \phonetic{fʷihˈnjoɬ} \recind \phonetic{xʷihˈnjoɬ}\gloss{embers} < \word{PW}{*xʷijhó\mbox{-}lʰ}{charcoal.\PL} \label{cx-msm-embers}
    \end{xlist}
\z
\ea
Misión La Paz Wichí \citep{MA08}\\
    \begin{xlist}
        \ex \phonetic{ʔnɑˈtses}\gloss{boys} < \wordng{PW}{*nǻʔtsha\mbox{-}s} \label{cx-mlp-boys}
        \ex \phonetic{lawoˈm̥ãj}\gloss{gorges} < \wordng{PW}{*ɬ̩\mbox{-}wómh\mbox{-}ajʰ} \label{cx-mlp-gorges}
        \ex \phonetic{kʲuˈm̥as}\gloss{workers} < \wordng{PW}{*kʲum\mbox{-}há\mbox{-}s}
        \ex \phonetic{pãˈn̥an}\gloss{red pepper} < \wordng{PW}{*pǻnhån}
        \ex \phonetic{tsoˈn̥at}\gloss{knife} < \wordng{PW}{*tsonhat}
        \ex \phonetic{hõˈn̥at}\gloss{earth} < \wordng{PW}{*honhat}
        \ex \phonetic{oˈn̥ɑq}\gloss{sachasandía\species{Capparis salicifolia}} < \wordng{PW}{*ʔónha\mbox{-}q \recvar *ʔónha\mbox{-}kʷ}
        \ex \phonetic{hʷiˈn̥ʲol}\gloss{charcoal} < \word{PW}{*xʷijhó\mbox{-}lʰ}{charcoal.\PL}
        \ex \phonetic{taˈn̥ʲĩ}\gloss{forest} < \wordng{PW}{*tájhi} \label{cx-mlp-forest}
        \ex \phonetic{toˈhʷãj}\gloss{pots} < \wordng{PW}{*towhá\mbox{-}jʰ} \label{cx-mlp-pots}
        \ex \phonetic{kʲoˈhʷãj}\gloss{holes} < \wordng{PW}{*kʲowhá\mbox{-}jʰ} \label{cx-mlp-holes}
    \end{xlist}
\z
\boolfalse{listing}

Proto-Wichí also allowed clusters of the shape */Ch/ in the word-final position, the options being \sound{PW}{*jʰ} (underlying */jh/) and \intxt{*lʰ} (underlying /lh/). \sound{PW}{*jʰ} is consistently reflected as voiceless \intxt{ç} in ’Weenhayek, where it contrasts with \sound{PW}{*j}~>~\sound{’Wk}{jʔ} \REF{wi-whk-yhy}. In all other varieties of Wichí, \sound{PW}{*jʰ} and \intxt{*j} merge as \intxt{j}.

\booltrue{listing}
\ea
’Weenhayek \citep{KC16} \label{wi-whk-yhy}\\
    \begin{xlist}
        \ex \wordnl{ʔi\mbox{-}náç}{s/he bathes} < \wordng{PW}{*ʔi\mbox{-}nájʰ}
        \ex \wordnl{ta\mbox{-}pájʔ}{it is bitter} < \wordng{PW}{*ta\mbox{-}páj}
    \end{xlist}
\z
\boolfalse{listing}

As for \sound{PW}{*lʰ}, both \intxt{l} and \intxt{ɬ} are found throughout the Wichí-speaking zone. These reflexes are distributed as follows. In ’Weenhayek and in the variety of Misión Santa María, only \intxt{ɬ} is found, as shown in \REF{wi-whk-lhl}–\REF{wi-msm-lhl}. In the varieties of Paraje La Paz (Vejoz) and Misión La Paz (Guisnay), by contrast, only the voiced reflex is attested, as in \REF{wi-plp-lhl}–\REF{wi-mlp-lhl}. Some dialects show both \intxt{l} and \intxt{ɬ} as possible reflexes. A case in point is the Lower Bermejeño dialect, where \citet[52]{VN14} states that \intxt{ɬ} varies with \intxt{l}, especially in fast speech \REF{wi-lb-lhl}. This seems to also be the case in the closely-related Rivadavia subdialect of Southeastern Wichí as documented by \citet{JT09-th}: compare the voiced reflex in \REF{wi-riv-lhl-hepel} and the voiceless reflex in \REF{wi-riv-lhl-yil}–\REF{wi-riv-lhl-fwul}. Variation is also attested in the Misión Chaqueña subdialect of Vejoz, where \citet{VU74} mostly documents the voiced reflex \intxt{l}, as in \REF{wi-vu-lhl-shadow}–\REF{wi-vu-lhl-they}; the voiceless reflex is documented in \REF{wi-vu-lhl-locust}. In \cits{MG-MELO15} dictionary of the same variety, the voiced reflex is found in \REF{wi-mc-mgmelo-weexcl}–\REF{wi-mc-mgmelo-they}, whereas the voiceless reflex is documented in \REF{wi-mc-mgmelo-shadow}–\REF{wi-mc-mgmelo-stars}.

\booltrue{listing}
\ea
’Weenhayek \citep{KC16} \label{wi-whk-lhl}\\
    \begin{xlist}
        \ex \wordnl{qatéts\mbox{-}eɬ}{stars} < \wordng{PW}{*qatéts\mbox{-}elʰ}
        \ex \wordnl{xʷiçó\mbox{-}ɬ}{embers} < \word{PW}{*xʷijhó\mbox{-}lʰ}{charcoal.\PL}
        \ex \wordnl{ʔõ\mbox{-}ˀj\mbox{-}iɬ}{I die} < \wordng{PW}{*n̩\mbox{-}ˀj\mbox{-}ilʰ}
        \ex \intxt{∅\mbox{-}ʔám–eɬ} (rare)\gloss{you guys} < \wordng{PW}{*∅\mbox{-}ʔám\mbox{-}elʰ}
    \end{xlist}
\z
\ea
Misión Santa María Wichí \citep{SS07} \label{wi-msm-lhl}\\
    \begin{xlist}
        \ex \phonetic{kateˈtseɬ}\gloss{stars} < \wordng{PW}{*qatéts\mbox{-}elʰ}
        \ex \phonetic{fʷihˈnjoɬ} \recind \phonetic{xʷihˈnjoɬ}\gloss{embers} < \word{PW}{*xʷijhó\mbox{-}lʰ}{charcoal.\PL}
        \ex \phonetic{oˈjiɬ}\gloss{I die} < \wordng{PW}{*n̩\mbox{-}ˀj\mbox{-}ilʰ}
        \ex \phonetic{aˈmeɬ}\gloss{you guys} < \wordng{PW}{*∅\mbox{-}ʔám\mbox{-}elʰ}
    \end{xlist}
\z
\ea
Paraje La Paz Wichí \citep{AFG067} \label{wi-plp-lhl}\\
    \begin{xlist}
        \ex \phonetic{kateˈtsel}\gloss{stars} < \wordng{PW}{*qatéts\mbox{-}elʰ}
        \ex \phonetic{huˈpel}\gloss{shadow} < \wordng{PW}{*hpélʰ}
        \ex \phonetic{aˈmel}\gloss{you guys} < \wordng{PW}{*∅\mbox{-}ʔám\mbox{-}elʰ}
        \ex \phonetic{ɬaˈmel}\gloss{they} < \wordng{PW}{*ɬ\mbox{-}ám\mbox{-}elʰ}
        \ex \phonetic{ˈjil}\gloss{dead} < \wordng{PW}{*ˀj\mbox{-}ilʰ}
    \end{xlist}
\z
\ea
Misión La Paz Wichí \citep{MA08} \label{wi-mlp-lhl}\\
    \begin{xlist}
        \ex \phonetic{qɑteˈtsel}\gloss{stars} < \wordng{PW}{*qatéts\mbox{-}elʰ}
        \ex \phonetic{hãʔlateˈtsel}\gloss{tree trunks} < \wordng{PW}{*haˀlå téts\mbox{-}elʰ}
        \ex \phonetic{hʷiˈn̥ʲol}\gloss{charcoal} < \word{PW}{*xʷijhó\mbox{-}lʰ}{charcoal.\PL}
        \ex \phonetic{ʔnoɬaˈmel}\gloss{we (exclusive)} < \word{PW}{*ˀnó\mbox{-}ɬ\mbox{-}ám\mbox{-}elʰ}{one (indefinite pronoun)}
    \end{xlist}
\z
\ea
Lower Bermejeño Wichí \citep[52]{VN14} \label{wi-lb-lhl}\\
    \begin{xlist}
        \ex \phonetic{hĩˈˀnũɬ} \recind \phonetic{hĩˈˀnũl}\gloss{men} < \wordng{PW}{*hiˀnó\mbox{-}lʰ}
        \ex \phonetic{ʔaˈmĩɬ} \recind \phonetic{ʔaˈmĩl}\gloss{you guys} < \wordng{PW}{*∅\mbox{-}ʔám\mbox{-}elʰ}
    \end{xlist}
\z
\ea
Rivadavia Wichí \citep{JT09-th}\\
    \begin{xlist}
        \ex \wordnl{hepel / \mbox{-}qa\mbox{-}mpel}{shadow} < \wordng{PW}{*hpélʰ / *\mbox{-}qá\mbox{-}hpelʰ} \label{wi-riv-lhl-hepel}
        \ex \wordnl{jiɬ}{s/he dies} < \wordng{PW}{*ˀj\mbox{-}ilʰ} \label{wi-riv-lhl-yil}
        \ex \wordnl{hinu\mbox{-}ɬ}{men} < \wordng{PW}{*hiˀnó\mbox{-}lʰ}
        \ex \wordnl{\mbox{-}xʷuɬ}{flute} < \wordng{PW}{*\mbox{-}xʷólʰ}  \label{wi-riv-lhl-fwul}
    \end{xlist}
\z
\ea
Misión Chaqueña Wichí \citep{VU74}\\
    \begin{xlist}
        \ex \wordnl{hupel}{shadow} < \wordng{PW}{*hpélʰ} \label{wi-vu-lhl-shadow}
        \ex \wordnl{\mbox{-}hʷol}{flute} < \wordng{PW}{*\mbox{-}xʷólʰ}
        \ex \wordnl{\mbox{-}pil}{to return hither} < \wordng{PW}{*\mbox{-}pílʰ}
        \ex \wordnl{j\mbox{-}apil}{to return thither} < \wordng{PW}{*j\mbox{-}ǻpilʰ}
        \ex \wordnl{jijl}{s/he dies} < \wordng{PW}{*ˀj\mbox{-}ilʰ}
        \ex \wordnl{o\mbox{-}ɬ\mbox{-}am\mbox{-}el}{we (exclusive)} < \wordng{PW}{*n̩\mbox{-}ɬ\mbox{-}ám\mbox{-}elʰ}
        \ex \wordnl{n\mbox{-}am\mbox{-}el}{we (inclusive)} < \wordng{PW}{*ˣn\mbox{-}ám\mbox{-}elʰ}
        \ex \wordnl{∅\mbox{-}am\mbox{-}el}{you guys} < \wordng{PW}{*∅\mbox{-}ʔám\mbox{-}elʰ}
        \ex \wordnl{ɬ\mbox{-}am\mbox{-}el}{they} < \wordng{PW}{*ɬ\mbox{-}ám\mbox{-}elʰ} \label{wi-vu-lhl-they}
        \ex \wordnl{tʃoɬ}{locust} < \wordng{PW}{*kʲólʰ} \label{wi-vu-lhl-locust}
    \end{xlist}
\z
\ea
Misión Chaqueña Wichí \citep{MG-MELO15}\\
    \begin{xlist}
        \ex \wordnl{o\mbox{-}ɬ\mbox{-}am\mbox{-}el}{we (exclusive)} < \wordng{PW}{*n̩\mbox{-}ɬ\mbox{-}ám\mbox{-}elʰ} \label{wi-mc-mgmelo-weexcl}
        \ex \wordnl{ˀn\mbox{-}am\mbox{-}el}{we (inclusive)} < \wordng{PW}{*ˣn\mbox{-}ám\mbox{-}elʰ}
        \ex \wordnl{∅\mbox{-}am\mbox{-}el}{you guys} < \wordng{PW}{*∅\mbox{-}ʔám\mbox{-}elʰ}
        \ex \wordnl{ɬ\mbox{-}am\mbox{-}el}{they} < \wordng{PW}{*ɬ\mbox{-}ám\mbox{-}elʰ} \label{wi-mc-mgmelo-they}
        \ex \wordnl{hupeɬ}{shadow} < \wordng{PW}{*hpélʰ} \label{wi-mc-mgmelo-shadow}
        \ex \wordnl{\mbox{-}piɬ}{to return hither} < \wordng{PW}{*\mbox{-}pílʰ}
        \ex \wordnl{tʃoɬ}{locust} < \wordng{PW}{*kʲólʰ}
        \ex \wordnl{katets\mbox{-}eɬ}{stars} < \wordng{PW}{*qatéts\mbox{-}elʰ} \label{wi-mc-mgmelo-stars}
    \end{xlist}
\z
\boolfalse{listing}

\subsubsection{Word-initial consonant clusters}\label{wi-cc}

The word-initial clusters \intxt{*kʲt}, \intxt{*tkʲ}, and \intxt{*qs} have changed in all Wichí dialects: in Southeastern Wichí they are resolved by the epenthesis of \intxt{i}, \intxt{a}, and \intxt{a}, respectively, whereas in all other varieties the first element of these clusters is simply deleted. Four examples are currently known: \word{PW}{*kʲtáˀnih}{Chaco tortoise}, \wordnl{*kʲtéta}{white algarrobo fruit\species{Prosopis elata}}, \wordnl{*tkʲénaχ}{mountain}, and \wordnl{*\mbox{qséɬtaχ}}{chequered woodpecker}. Their reflexes are affected by vowel epenthesis in Lower Bermejeño \REF{wi-kt-t-lb}. Epenthetic vowels in these words are likewise attested in \word{Rivadavia}{takʲenax}{mountain} \citep[25]{JT09-th} and in the form \wordnl{tʃiteta}{white algarrobo fruit}, documented in an unspecified location in Salta by \citet{MS14}. In ’Weenhayek, Vejoz, and Guisnay the clusters in question are rather eliminated by means of consonant deletion. The examples in \REF{wi-kt-t-whk} are from ’Weenhayek, and those in \REF{wi-kt-t-mc} are from the Misión Chaqueña subdialect of Vejoz. Forms from other, understudied varieties that show the same kind of sound change include Misión La Paz [kʲeˈnɑx], Misión Santa María [tɕeˈnax] \recind [tʃeˈnax] `mountain' \citep[67]{SS07,MA08}.

\booltrue{listing}
\ea
Lower Bermejeño Wichí \citep{VN14,JB09} \label{wi-kt-t-lb}\\
    \begin{xlist}
        \ex \wordnl{tʃitaˀni}{Chaco tortoise} < \wordng{PW}{*kʲtáˀnih}
        \ex \wordnl{tatʃenaχ}{mountain} < \wordng{PW}{*tkʲénaχ}
        \ex \wordnl{qaseɬtaχ}{chequered woodpecker} < \wordng{PW}{*qséɬtaχ}
    \end{xlist}
\z
\ea
’Weenhayek \citep{KC16} \label{wi-kt-t-whk}\\
    \begin{xlist}
        \ex \wordnl{táˀnih}{Chaco tortoise} < \wordng{PW}{*kʲtáˀnih}
        \ex \wordnl{tétaʔ}{white algarrobo fruit} < \wordng{PW}{*kʲtéta}
        \ex \wordnl{kʲénax}{mountain} < \wordng{PW}{*tkʲénaχ}
        \ex \wordnl{séɬtax}{kind of small woodpecker with a white crest} < \wordng{PW}{*qséɬtaχ}
    \end{xlist}
\z
\ea
Misión Chaqueña Wichí \citep{VU74,MG-MELO15} \label{wi-kt-t-mc}\\
    \begin{xlist}
        \ex \wordnl{taˀni}{Chaco tortoise} < \wordng{PW}{*kʲtáˀnih}
        \ex \wordnl{tʃenah}{mountain} < \wordng{PW}{*tkʲénaχ}
    \end{xlist}
\z
\boolfalse{listing}

There are also a few roots where it is possible to reconstruct word-initial clusters of the shape \intxt{*FW}, where \intxt{F} stands for a fricative and \intxt{W} for a labial consonant. These are resolved by an epenthetic vowel, whose quality depends on the dialect. In Lower Bermejeño (but not in the closely related Rivadavia subdialect), the epenthetic vowel is \intxt{i} in such cases. In the Misión Chaqueña subdialect of Vejoz, the epenthetic vowel is \intxt{i} \recind \intxt{u} after \intxt{s} but \intxt{u} after \intxt{h}. In ’Weenhayek and in the Misión La Paz subdialect of Guisnay, the epenthetic vowel is \intxt{u} even after \intxt{s}. In the Paraje La Paz subdialect of Vejoz, the epenthetic vowel is \intxt{u} at least after \intxt{h} (no examples involving \intxt{s} are documented in that variety in our sources). Finally, Rivadavia shows \intxt{u} (<~\intxt{*o}) after \intxt{s} and \intxt{e} (<~\intxt{*u}) after \intxt{h}. The known examples are listed in \tabref{wi-ep-lab}.

\begin{table}
\caption{Vowel epenthesis between a fricative and a labial}
\label{wi-ep-lab}
 \begin{tabularx}{\textwidth}{llllQ}
  \lsptoprule
            & `ant' & `dove' & `shadow' & source\\\midrule
  Proto-Wichí & *swánaχ & *spúp & *hpélʰ & \\
  ’Weenhayek & suwán̥-is & supúp & hupéɬ & \citet{KC16}\\
  Misión La Paz & suwan̥a-s & — & — & \citet{MA08}\\
  Misión Chaqueña & suwanah & — & hupel & \citet{VU74}\\
  Misión Chaqueña & siwan̥a-s & sipup & hupeɬ & \citet{MG-MELO15}\\
  Paraje La Paz & — & — & hupel & \citet{AFG067}\\
  Rivadavia & suwana, suwan̥a-s & — & hepel & \citet{JT09-th}\\
  Lower Bermejeño & siwan̥a-s & sipep & hipeɬ & \citet{VN14,CS-FL-PR-VN13}\\
  \lspbottomrule
 \end{tabularx}
\end{table}

\subsubsection{Obstruent loss before glottalized sonorants}\label{wi-c'c}

Some dialects, notably Southeastern Wichí, have done away with the Proto-Wichí clusters such as \intxt{*pˀl}, \intxt{*qˀl}, \intxt{*qˀj} by deleting their first element, as in the examples from Lower Bermejeño in \REF{wi-lb-obgl}.

\booltrue{listing}
\ea
Lower Bermejeño Wichí \citep{VN14,JB09} \label{wi-lb-obgl}\\
    \begin{xlist}
        \ex \wordnl{\mbox{-}juˀle}{to hiccup} < \wordng{PW}{*ˀ[j]ópˀle}
        \ex \wordnl{\mbox{-}waˀla}{nephew} < \wordng{PW}{*\mbox{-}wáqˀlah}
        \ex \wordnl{\mbox{-}waˀlani}{niece} < \wordng{PW}{*\mbox{-}wáqˀlanih}
        \ex \wordnl{[t]oˀlej\mbox{-}\APPL}{to fight} < \wordng{PW}{*[t]ǻqˀlej}
        \ex \wordnl{fʷuˀjaχ}{Muscovy duck} < \wordng{PW}{*xʷóqˀjaχ}
    \end{xlist}
\z
\boolfalse{listing}

The same phenomenon is attested in some other varieties, as in Misión Santa María [owaˈlaʔ] `my nephew' \citep{SS07}, Lapacho Mocho \mbox{[taʔleˈhnjen]} alongside \mbox{[takleˈhnjen]} `they fight', [xʷoˈʔjah] alongside [xʷokˈjah] `duck' \citep[163–164, 167]{AFG-SS-09}. In the Rivadavia subdialect of Southeastern Wichí, the reflex of \word{PW}{*n̩\mbox{-}ˀjópˀle}{I hiccup} has been unexpectedly attested as \intxt{n̩\mbox{-}jutle} \citep[134]{MC09}. By contrast, varieties such as ’Weenhayek and Vejoz preserve the clusters in question, though Vejoz may lose the glottalization in the sonorant \REF{wi-vj-obgl}.

\booltrue{listing}
\ea
Vejoz \citep{RJH13a,VU74,MG-MELO15} \label{wi-vj-obgl}\\
    \begin{xlist}
        \ex \wordnl{[j]ople}{to hiccup} < \wordng{PW}{*ˀ[j]ópˀle}
        \ex \wordnl{\mbox{-}wakla}{nephew} < \wordng{PW}{*\mbox{-}wáqˀlah}
        \ex \wordnl{\mbox{-}waklani}{niece} < \wordng{PW}{*\mbox{-}wáqˀlanih}
        \ex \wordnl{hʷok(j)e-tah}{duck} < \wordng{PW}{*xʷóqˀja\mbox{-}taχ}
    \end{xlist}
\z
\boolfalse{listing}

Before non-glottalized sonorants, the change does not usually take place; for example, \word{PW}{*\mbox{-}t\mbox{-}’ótle}{heart} and \wordnl{*wáplu}{she is pregnant} consistently preserve the clusters \intxt{tl} and \intxt{pl} in the daughter lects, as in \wordng{LB}{\mbox{-}t\mbox{-}’utle}, \intxt{waple} \citep[97]{VN14}. The cluster \intxt{*tn} is typically preserved as \intxt{tn}, but it may also evolve to \intxt{kn}, as in Paraje La Paz [tokˈnah]\gloss{toad} < \wordng{PW}{*tǻtnaχ}.

\subsubsection{Insertion and deletion of \intxt{ʔ} before a pause}

In Proto-Wichí, \intxt{*ʔ} was contrastive in the word-final position, as evidenced by pairs such as \wordnl{*ɬ\mbox{-}óʔ}{its seed} vs. \wordnl{*ɬ\mbox{-}ó}{his penis}. This is preserved at least in the Lower Bermejeño variety of Wichí as documented by \citet{VN14}, as in \word{LB}{\mbox{-}ɬ\mbox{-}uʔ}{seed} vs. \wordnl{\mbox{-}ɬ\mbox{-}u}{penis} \citep[212–213]{VN14}.\footnote{The distinction is not consistently represented in \cits{JB09} vocabulary of the Bazanero subdialect of Southeastern Wichí.}

Other Wichí dialects are less conservative in this regard. For example, ’Weenhayek no longer allows vowels before a pause \citep[25–26]{KC94}: an epenthetic \intxt{ʔ} is systematically inserted after erstwhile utterance-final vowels (or after \intxt{*j}), and \word{’Wk}{ɬ\mbox{-}óʔ}{its seed} is now homophonous with \wordnl{ɬ\mbox{-}óʔ}{his penis} \citep[75]{KC16}. In the Rivadavia subdialect of Southeastern Wichí, \intxt{ʔ} is automatically inserted after utterance-final stressed vowels, even in borrowings, such as \wordnl{\mbox{kʲesuʔ} \phonetic{kʲeˈsuʔ}}{cheese} (from \wordng{Spanish}{queso}), \wordnl{klistinaʔ \phonetic{klistiˈnaʔ}}{Cristina}; unlike in ’Weenhayek, words with non-final stress, such as \wordnl{iˈxʷala}{morning}, do not show the \intxt{ʔ}-epenthesis \citep[48–51]{JT09-cap}. In the varieties of Misión Santa María and Misión La Paz, the epenthesis of \intxt{ʔ} is found in some words -- \REF{wi-msm-findeg-sleep}–\REF{wi-msm-findeg-fly}, \REF{wi-mlp-findeg-shoulder}–\REF{wi-mlp-findeg-rat} -- but not in others -- \REF{wi-msm-findeg-rat}–\REF{wi-msm-findeg-girl}, \REF{wi-mlp-findeg-stone}–\REF{wi-mlp-findeg-monk}.

\newpage
\booltrue{listing}
\ea
Misión Santa María Wichí \citep{SS07} \label{wi-msm-findeg}\\
    \begin{xlist}
        \ex \phonetic{iˈmaʔ}\gloss{s/he sleeps} < \wordng{PW}{*ʔi\mbox{-}må} \label{wi-msm-findeg-sleep}
        \ex \phonetic{weˈjaʔ}\gloss{s/he flies} < \wordng{PW}{*xʷeˀjå \recvar *weˀjå \recvar *xʷiˀjå \recvar *wiˀjå} \label{wi-msm-findeg-fly}
        \ex \phonetic{aˈma}\gloss{rat} < \wordng{PW}{*ʔáma} \label{wi-msm-findeg-rat}
        \ex \phonetic{ɬuˈtsa}\gloss{young woman} < \wordng{PW}{*ɬútsha}  \label{wi-msm-findeg-girl}
    \end{xlist}
\z
\ea
Misión La Paz Wichí \citep{MA08}\\
    \begin{xlist}
        \ex \phonetic{ohʷaˈpoʔ}\gloss{my shoulder} < \wordng{PW}{*ń̩\mbox{-}xʷapo} \label{wi-mlp-findeg-shoulder}
        \ex \phonetic{aˈl̥uʔ}\gloss{iguana} < \wordng{PW}{*ʔáɬu}
        \ex \phonetic{aːmaʔ}\gloss{rat} < \wordng{PW}{*ʔáma} \label{wi-mlp-findeg-rat}
        \ex \phonetic{tunˈte}\gloss{stone} < \wordng{PW}{*túnte} \label{wi-mlp-findeg-stone}
        \ex \phonetic{piˈnu}\gloss{sugarcane} < \wordng{PW}{*pínu}
        \ex \phonetic{kʲ’ekʲ’e}\gloss{monk parakeet} < \wordng{PW}{*kʲékʲ’e} \label{wi-mlp-findeg-monk}
    \end{xlist}
\z
\boolfalse{listing}

The \intxt{ʔ}-epenthesis is only rarely found in the variety of Paraje La Paz ([aˈmaʔ] `rat', [haˈlɑʔ] `tree' are the only examples documented in \citnp{AFG067}). More research is needed on the varieties such as those of Misión Santa María, Misión La Paz, and Paraje La Paz in order to verify the status of the word-final instances of [ʔ], with a special focus on pairs such as \wordnl{*ɬ\mbox{-}óʔ}{its seed} vs. \wordnl{*ɬ\mbox{-}ó}{his penis}. Note that in all these dialects the \intxt{ʔ}-epenthesis typically fails to occur in words that have diachronically lost \sound{PW}{*h} in the word-final position (\sectref{wi-jj-h}), as in Misión Santa María [oˈko] `my mother', [oˈʔo] `hen' (<~\wordng{PW}{*ń̩\mbox{-}qoh}, \intxt{*hóʔoh}), Misión La Paz [otkʲumˈɬi] (<~\wordng{PW}{*ń̩\mbox{-}t\mbox{-}kʲúm\mbox{-}ɬih}).

In the Misión Chaqueña subdialect of Vejoz, by contrast, \intxt{*ʔ} appears to have been eliminated in the word-final position even in words that originally ended in a glottal stop, as in \intxt{ɬa} <~\word{PW}{*ɬaʔ}{louse} \citep[64]{VU74}. As a result, \citet{VU74} and \citet{MG-MELO15} do not document \intxt{ʔ} in the word-final position in Vejoz at all.

\subsubsection{\sound{PW}{*ˣ\mbox{-}}}

In a limited number of words, \sound{’Weenhayek}{ʔi\mbox{-}} corresponds to zero in other Wichí varieties. While it is tempting to provide a morphological interpretation for this correspondence (e.g. by positing a fossilized semantically empty prefix \intxt{ʔi\mbox{-}} in Weenhayek), external comparanda in other Mataguayan languages suggest instead that one must seek a phonological explanation for it. In all likelihood, the correspondence between \sound{’Wk}{ʔi\mbox{-}} and zero in other Wichí dialects results from attrition of phonological material at the left margin of the word: compare \word{Nivaĉle}{xiβeˀk͡la}{moon}, \wordnl{jiˀjek͡le}{tapir}, \wordnl{ʃnåβåp}{spring} and \wordng{’Wk}{ʔiwéˀlah}, \intxt{ʔijéˀlah}, \intxt{ʔináwop} \recind \wordng{LB}{wela}, \intxt{jela}, \intxt{nawup}. It is unclear at present how the segment in question was articulated in Proto-Wichí (some possibilities that we have considered include \intxt{*ʔ\mbox{-}}, \intxt{*ʔᵊ\mbox{-}}, ultrashort \intxt{*ʔĭ\mbox{-}}); we symbolize it with an ad hoc character \intxt{*ˣ\mbox{-}} for the time being. It has been reconstructed, among other, in the following roots: \wordnl{*ˣwéˀlah}{moon}, \wordnl{*ˣjéˀlah}{tapir}, \wordnl{*ˣnáwop}{spring}, \wordnl{*ˣmáwoh}{fox}, \wordnl{*ˣnǻte}{rabbit}, \wordnl{*ˣˈxʷála}{sun, day}, \wordnl{*ˣmájeq}{thing, ghost}, \wordnl{*ˣsp(’)ólop}{thrush}, \wordnl{*ˣnɪ́kʲ’u}{black-legged seriema\species{Chunga burmeisteri}}, \wordnl{*ˣn̥átaχ}{tusca fruit} (whence \wordnl{*ˣn̥át\mbox{-}eq}{tusca bush}).

The Lower Bermejeño subdialect of Southeastern Wichí always loses \intxt{*ˣ\mbox{-}} \REF{wi-lb-xx}. This is corroborated by \citet[138]{MC09}, who documents forms such as Misión El Carmen [ˈxʷala], Colonia Muñiz [ˈfʷala] ‘day’. Total loss of \intxt{*ˣ\mbox{-}} is also found in the variety of Misión Santa María \REF{wi-msm-xx}.

\booltrue{listing}
\ea
Lower Bermejeño Wichí \citep{VN14,CS-FL-PR-VN13} \label{wi-lb-xx}\\
    \begin{xlist}
        \ex \wordnl{weˀla}{moon} < \wordng{PW}{*ˣwéˀlah}
        \ex \wordnl{jeˀla}{tapir} < \wordng{PW}{*ˣjéˀlah}
        \ex \wordnl{nawup}{spring} < \wordng{PW}{*ˣnáwop}
        \ex \wordnl{mawu}{fox} < \wordng{PW}{*ˣmáwoh}
        \ex \wordnl{note}{rabbit} < \wordng{PW}{*ˣnǻte}
        \ex \wordnl{fʷala}{sun, day} < \wordng{PW}{*ˣˈxʷála}
        \ex \wordnl{ma(je)q}{thing} < \wordng{PW}{*ˣmájeq}
        \ex \wordnl{sipulup}{thrush} < \wordng{PW}{*ˣsp(’)ólop}
        \ex \wordnl{netʃ’e}{black-legged seriema} < \wordng{PW}{*ˣnɪ́kʲ’u}
        \ex \wordnl{n̥ataχ}{tusca fruit} < \wordng{PW}{*ˣn̥átaχ}
    \end{xlist}
\z
\ea
Misión Santa María Wichí \citep{SS07} \label{wi-msm-xx}\\
    \begin{xlist}
        \ex \phonetic{naˈwop}\gloss{spring} < \wordng{PW}{*ˣnáwop}
        \ex \phonetic{maˈwo}\gloss{fox} < \wordng{PW}{*ˣmáwoh}
        \ex \phonetic{maˈjek}\gloss{ghost} < \word{PW}{*ˣmájeq}{thing}
    \end{xlist}
\z

In the Rivadavia subdialect of Southeastern Wichí, \intxt{*ˣ\mbox{-}} is usually lost, as in \REF{wi-riv-xx-tapir}–\REF{wi-riv-xx-thing}; in three cases, however, the vowel \intxt{i} is found as its reflex instead, as in \REF{wi-riv-xx-fox}–\REF{wi-riv-xx-seriema}. A similar tendency is found in the Paraje La Paz subdialect of Vejoz: compare \REF{wi-plp-xx-thing}–\REF{wi-plp-xx-tusca} with \REF{wi-plp-xx-tapir}; note that the latter root shows up without an \intxt{i} in the derivative in \REF{wi-plp-xx-horse}.

\ea
Rivadavia Wichí \citep{JT09-th}\\
    \begin{xlist} 
        \ex \wordnl{jela}{tapir} < \wordng{PW}{*ˣjéˀlah} \label{wi-riv-xx-tapir}
        \ex \wordnl{nawup}{spring} < \wordng{PW}{*ˣnáwop}
        \ex \wordnl{note}{rabbit} < \wordng{PW}{*ˣnǻte}
        \ex \wordnl{maq}{thing} < \wordng{PW}{*ˣmájeq} \label{wi-riv-xx-thing}
        \ex \wordnl{imawu}{fox} < \wordng{PW}{*ˣmáwoh} \label{wi-riv-xx-fox}
        \ex \wordnl{iˈxʷala \recind ˈxʷala}{sun, day} < \wordng{PW}{*ˣˈxʷála}
        \ex \wordnl{inekʲe}{black-legged seriema} < \wordng{PW}{*ˣnɪ́kʲ’u} \label{wi-riv-xx-seriema}
    \end{xlist}
\z
\ea
Paraje La Paz Wichí \citep{AFG067} \label{wi-plp-xx}\\
    \begin{xlist}
        \ex \phonetic{ˈmak}\gloss{something} < \word{PW}{*ˣmájeq}{thing} \label{wi-plp-xx-thing}
        \ex \phonetic{siɓ̥oˈlop}\gloss{thrush} < \wordng{PW}{*ˣsp(’)ólop}
        \ex \phonetic{naˈtek}\gloss{tusca bush} < \wordng{PW}{*ˣnháte\mbox{-}q} \label{wi-plp-xx-tusca}
        \ex \phonetic{ɪjeˈla}\gloss{tapir} < \wordng{PW}{*ˣjéˀlah} \label{wi-plp-xx-tapir}
        \ex \phonetic{jelaˈtax}\gloss{horse} < \wordng{PW}{*ˣjéˀla\mbox{-}taχ} \label{wi-plp-xx-horse}
    \end{xlist}
\z

In the Misión Chaqueña subdialect of Vejoz, \intxt{i} and \intxt{∅} are almost equally frequent as reflexes of \sound{PW}{*ˣ\mbox{-}}. The available examples in \cits{VU74} work are given in \REF{wi-vu-xx}; the noun in \REF{wi-vu-xx-rabbit} unexpectedly shows \intxt{hn} instead of \intxt{*n}. \cits{MG-MELO15} vocabulary of the same variety has \intxt{i} in a different set of words \REF{wi-mgmelo-xx}; in \REF{wi-mgmelo-xx-fox} and \REF{wi-mgmelo-xx-thing}, \intxt{*ˣm} reflected as \intxt{ˀm}.

\ea
Misión Chaqueña Wichí \citep{VU74} \label{wi-vu-xx}\\
    \begin{xlist}
        \ex \wordnl{iwela}{moon} < \wordng{PW}{*ˣwéˀlah} \label{wi-vu-xx-moon}
        \ex \wordnl{ijela}{tapir} < \wordng{PW}{*ˣjéˀlah}
        \ex \wordnl{nawop}{spring} < \wordng{PW}{*ˣnáwop}
        \ex \wordnl{maˀwo}{fox} < \wordng{PW}{*ˣmáwoh}
        \ex \wordnl{hnåte \recind hnote}{rabbit} < \wordng{PW}{*ˣnǻte} \label{wi-vu-xx-rabbit}
        \ex \wordnl{ihʷala \recind hʷala}{sun, day} < \wordng{PW}{*ˣˈxʷála}
        \ex \wordnl{mak \recind majek}{thing, something} < \wordng{PW}{*ˣmájeq}
        \ex \wordnl{sip’olop}{thrush} < \wordng{PW}{*ˣsp(’)ólop}
        \ex \wordnl{natek}{tusca bush} < \wordng{PW}{*ˣn̥át\mbox{-}eq} \label{wi-vu-xx-tusca}
    \end{xlist}
\z

\newpage
\ea
Misión Chaqueña Wichí \citep{MG-MELO15} \label{wi-mgmelo-xx}\\
    \begin{xlist}
        \ex \wordnl{wela \recind iwela}{moon} < \wordng{PW}{*ˣwéˀlah}
        \ex \wordnl{inawop}{spring} < \wordng{PW}{*ˣnáwop}
        \ex \wordnl{nåte \recind inåte}{rabbit} < \wordng{PW}{*ˣnǻte}
        \ex \wordnl{hʷala \recind ihʷala}{sun, day} < \wordng{PW}{*ˣˈxʷála}
        \ex \wordnl{n̥atek}{tusca bush} < \wordng{PW}{*ˣn̥át\mbox{-}eq} 
        \ex \wordnl{ˀmawo}{fox} < \wordng{PW}{*ˣmáwoh} \label{wi-mgmelo-xx-fox}
        \ex \wordnl{ˀmak}{thing, something} < \wordng{PW}{*ˣmájeq} \label{wi-mgmelo-xx-thing}
    \end{xlist}
\z
 
Finally, as noted above, ’Weenhayek consistently shows the reflex \intxt{ʔi} \REF{wi-whk-xx}.

\ea
’Weenhayek \citep{KC16} \label{wi-whk-xx}\\
    \begin{xlist}
        \ex \wordnl{ʔiwéˀlah}{moon} < \wordng{PW}{*ˣwéˀlah}
        \ex \wordnl{ʔijéˀlah}{tapir} < \wordng{PW}{*ˣjéˀlah}
        \ex \wordnl{ʔináwop}{spring} < \wordng{PW}{*ˣnáwop}
        \ex \wordnl{ʔimáwoh}{fox} < \wordng{PW}{*ˣmáwoh}
        \ex \wordnl{ʔinǻteʔ}{rabbit} < \wordng{PW}{*ˣnǻte}
        \ex \wordnl{ʔiˈxʷálaʔ}{sun, day} < \wordng{PW}{*ˣˈxʷála}
        \ex \wordnl{ʔimák}{thing}, \wordnl{ʔimájek}{thing, ghost} < \wordng{PW}{*ˣmájeq}
        \ex \wordnl{ʔispólop}{thrush} < \wordng{PW}{*ˣsp(’)ólop}
        \ex \wordnl{netʃ’e}{black-legged seriema} < \wordng{PW}{*ˣnɪ́kʲ’u}
        \ex \wordnl{ʔin̥átax}{tusca fruit} < \wordng{PW}{*ˣn̥átaχ}
        \ex \wordnl{ʔin̥át\mbox{-}ek}{tusca bush} < \wordng{PW}{*ˣn̥át\mbox{-}eq}
    \end{xlist}
\z
\boolfalse{listing}

\subsubsection{\sound{PW}{*n̩\mbox{-}}}

Syllabic \intxt{*n̩\mbox{-}} is reconstructed in the first-person singular prefix (\wordng{PW}{*n̩\mbox{-}} in verbs, \intxt{*ń̩\mbox{-}} in nouns). It is preserved as a syllabic nasal in some subdialects of Southeastern Wichí, including Lower Bermejeño \citep[52, 98, 163, 223]{VN14} and Rivadavia \citep[41–42]{JT09-th}. In fact, it is reported to preserve its syllabicity preceding vowels in the latter variety: \wordnl{n̩.ˈi.hi}{I am}, \wordnl{n̩.i.qa.ˈna}{I am here}. \citet[132, 134, 137–138]{MC09} documents this realization in Colonia Muñiz, Bazán, and El Sauzalito (all these communities are located within the Lower Bermejeño zone); in Misión El Carmen, \intxt{n̩\mbox{-}} has been attested alongside \intxt{ni\mbox{-}}: compare \REF{sn-mec-steal}–\REF{sn-mec-work} and \REF{sn-mec-thirsty}.

\booltrue{listing}
\ea
Misión El Carmen Wichí \citep[131–132, 138]{MC09}\\
    \begin{xlist}
        \ex \phonetic{n̩ˈskɑt}\gloss{I steal} < \wordng{PW}{*n̩\mbox{-}sqat} \label{sn-mec-steal}
        \ex \phonetic{n̩kɒˈʰɲi}\gloss{my pocket} < \wordng{PW}{*ń̩\mbox{-}qhå\mbox{-}j\mbox{-}hih}
        \ex \phonetic{n̩ˈɬos}\gloss{my son} < \wordng{PW}{*n̩\mbox{-}ɬ\mbox{-}ǻs}
        \ex \phonetic{n̩tʃemˈɬi}\gloss{I work} < \wordng{PW}{*n̩\mbox{-}t\mbox{-}kʲúm\mbox{-}ɬih} \label{sn-mec-work}
        \ex \phonetic{niˈkʲim}\gloss{I am thirsty} < \wordng{PW}{*n̩\mbox{-}kʲím} \label{sn-mec-thirsty}
    \end{xlist}
\z
\boolfalse{listing}

In some dialects, \wordng{PW}{*n̩\mbox{-}}, \intxt{*ń̩\mbox{-}} has become a nasal rounded vowel. Note that Wichí does not otherwise have phonemic nasal vowels (though vowels can be allophonically nasalized following a nasal consonant or /h/), meaning that the innovative reflex of Proto-Wichí syllabic \intxt{*n̩} becomes the first (and only) nasal vowel in the inventory of the dialects in question. In ’Weenhayek, the resulting prefix is \intxt{ʔõ\mbox{-}} in verbs, \intxt{ʔṍ\mbox{-}} in nouns (but \intxt{ʔõ\mbox{-}} in those affected by Watkins’ Law), as in \REF{sn-whk}. In the subdialect of Southeastern Wichí spoken by \cits{LCB-MBC09} consultant from Ingeniero Juárez, the prefix in question shows up as \intxt{ʊ̃\mbox{-}} \REF{sn-ij}.

\booltrue{listing}
\ea
’Weenhayek \citep[13]{KC94} \label{sn-whk}\\
    \begin{xlist}
        \ex \wordnl{ʔõ\mbox{-}ɬ\mbox{-}áˀm}{I} < \wordng{PW}{*n̩\mbox{-}ɬ\mbox{-}áˀm}
        \ex \wordnl{ʔõ\mbox{-}tuxʷ}{I eat} < \wordng{PW}{*n̩\mbox{-}tuxʷ}
        \ex \wordnl{ʔṍ\mbox{-}qoh}{my mother} < \wordng{PW}{*ń̩\mbox{-}qoh}
        \ex \wordnl{ʔṍ\mbox{-}puhxʷah}{my brother} < \wordng{PW}{*ń̩\mbox{-}puhxʷah}
    \end{xlist}
\z
\ea
Ingeniero Juárez Wichí \citep[98, 100–101]{LCB-MBC09} \label{sn-ij}\\
    \begin{xlist}
        \ex \phonetic{ʊ̃ˈɪh̃ɪ̃}\gloss{I am} < \wordng{PW}{*n̩\mbox{-}ˈʔí\mbox{-}hi}
        \ex \phonetic{ʊ̃ˈjɛn}\gloss{I fish} < \word{PW}{*n̩\mbox{-}j\mbox{-}én}{I put a snare}
        \ex \phonetic{ʊ̃ˈɗɛkʷɛ}\gloss{I search} < \wordng{PW}{*n̩\mbox{-}ˈt\mbox{-}’ú\mbox{-}kʷe}
        \ex \phonetic{ʊ̃sɛˈlɪt}\gloss{I feel sleepy} < \word{PW}{*n̩\mbox{-}t\mbox{-}’isélit}{I marvel, I shudder, I wake up}
    \end{xlist}
\z
\boolfalse{listing}

In quite a number of (sub)dialects, the first-person prefix is attested as \intxt{o\mbox{-}}, with no traces of nasality. This is the case in the variety of Misión Santa María \REF{sn-msm}. The same kind of reflex is documented in Vejoz, including the subdialects of Misión Chaqueña \citep[131]{MG-MELO15,VU74} and of Paraje La Paz \REF{sn-plp}. Some examples from the Misión La Paz subdialect of Guisnay are given in \REF{sn-mlp}. Numerous examples of the prefix \intxt{o\mbox{-}} are documented in \citet[163–164, 167–168]{AFG-SS-09} in the varieties of Lapacho Mocho, Misión San Luis, and El Cañaveral. Similarly, \citet{SS15} documents only \intxt{o\mbox{-}} as the first-person prefix in an article on causatives and applicatives, where all examples come from the varieties of Santa Victoria Este, Misión San Luis, El Cañaveral, and Misión Santa María.

\booltrue{listing}
\ea
Misión Santa María Wichí \citep{SS07} \label{sn-msm}\\
    \begin{xlist}
        \ex \phonetic{omakaˈtsi}\gloss{I lay down} < \wordng{PW}{*n̩\mbox{-}mǻ\mbox{-}qatsih}
        \ex \phonetic{otupeˈna}\gloss{I bend down} < \wordng{PW}{*n̩\mbox{-}t’úp…\mbox{-}\APPL}
        \ex \phonetic{oˈtsu}\gloss{I win} < \word{PW}{*n̩\mbox{-}ts’u(ʔ)}{I suck}
        \ex \phonetic{otʃunˈɬi}\gloss{I work} < \wordng{PW}{*n̩\mbox{-}t\mbox{-}kʲúm\mbox{-}ɬih}
        \ex \phonetic{oˈkoj}\gloss{I dance} < \word{PW}{*n̩\mbox{-}qój}{I play, I dance}
        \ex \phonetic{otʃ’oˈte}\gloss{I help} < \wordng{PW}{*n̩\mbox{-}t\mbox{-}ˈkʲ’ót\mbox{-}eh}
        \ex \phonetic{osunˈɬi}\gloss{I whistle} < \wordng{PW}{*n̩\mbox{-}sun\mbox{-}ɬih \recind *n̩\mbox{-}sún\mbox{-}ɬih}
        \ex \phonetic{oˈtuh}\gloss{I eat} < \wordng{PW}{*n̩\mbox{-}tuxʷ}
        \ex \phonetic{oˈjiɬ}\gloss{I die} < \wordng{PW}{*n̩\mbox{-}ˀj\mbox{-}ilʰ}
        \ex \phonetic{oˈɬam}\gloss{I} < \wordng{PW}{*n̩\mbox{-}ɬ\mbox{-}áˀm}
        \ex \phonetic{oniˈpiɬ}\gloss{my stomach} < \wordng{PW}{*ń̩\mbox{-}nipiɬ}
        \ex \phonetic{otsoˈte}\gloss{my tooth} < \wordng{PW}{*ń̩\mbox{-}tsote}
        \ex \phonetic{otʃoˈte}\gloss{my ear} < \wordng{PW}{*ń̩\mbox{-}kʲ’ote}
        \ex \phonetic{oˈtsak}\gloss{my navel} < \wordng{PW}{*ń̩\mbox{-}ts’aq}
        \ex \phonetic{okuˈse}\gloss{my chin} < \wordng{PW}{*ń̩\mbox{-}q’use}
        \ex \phonetic{otʃaˈji}\gloss{my waist} < \wordng{PW}{*ń̩\mbox{-}kʲåji}
        \ex \phonetic{okʷeˈtʃ’o}\gloss{my palm of hand} < \wordng{PW}{*ń̩\mbox{-}kʷe\mbox{-}kʲ’o}
        \ex \phonetic{oˈko}\gloss{my mother} < \wordng{PW}{*ń̩\mbox{-}qoh}
    \end{xlist}
\z
\ea
Paraje La Paz Wichí \citep{AFG067} \label{sn-plp}\\
    \begin{xlist}
        \ex \phonetic{ojiˈsit}\gloss{I cut} < \wordng{PW}{*n̩\mbox{-}j\mbox{-}íset \recvar *n̩\mbox{-}j\mbox{-}ísit}
        \ex \phonetic{opotˈpe}\gloss{I bury} < \wordng{PW}{*n̩\mbox{-}pót\mbox{-}pe}
        \ex \phonetic{oˈɗ̥ek}\gloss{I eat} < \wordng{PW}{*n̩\mbox{-}t\mbox{-}’eq}
        \ex \phonetic{oˈkoj}\gloss{I play} < \wordng{PW}{*n̩\mbox{-}qój}
        \ex \phonetic{okaˈsit}\gloss{I stand} < \wordng{PW}{*n̩\mbox{-}t\mbox{-}qásit}
        \ex \phonetic{oˈsek}\gloss{I sweep} < \wordng{PW}{*n̩\mbox{-}sék}
        \ex \phonetic{oˈqoj}\gloss{I put clothes on} < \word{PW}{*ń̩\mbox{-}qhå\mbox{-}jʰ}{my clothes}
        \ex \phonetic{oteˈnek}\gloss{I sing} < \word{PW}{*ń̩\mbox{-}ten\mbox{-}eq}{my song}
        \ex \phonetic{oˈtsut}\gloss{my walking stick} < \wordng{PW}{*ń̩\mbox{-}}
        \ex \phonetic{otsoˈte}\gloss{my tooth} < \wordng{PW}{*ń̩\mbox{-}tsut}
        \ex \phonetic{otʃoˈte}\gloss{my ear} < \wordng{PW}{*ń̩\mbox{-}kʲ’ote}
        \ex \phonetic{oˈkʷej}\gloss{my hand} < \wordng{PW}{*ń̩\mbox{-}kʷej}
        \ex \phonetic{oˈles}\gloss{my children} < \wordng{PW}{*ń̩\mbox{-}les}
    \end{xlist}
\z
\ea
Misión La Paz Wichí \citep{MA08} \label{sn-mlp}\\
    \begin{xlist}
        \ex \phonetic{otkʲuˈhʷiʔ}\gloss{I am dizzy} < \wordng{PW}{*n̩\mbox{-}t\mbox{-}kʲhúxʷi}
        \ex \phonetic{otkʲumˈɬi}\gloss{I work} < \wordng{PW}{*n̩\mbox{-}t\mbox{-}kʲúm\mbox{-}ɬih}
        \ex \phonetic{oˈkʲim}\gloss{I am thirsty} < \wordng{PW}{*n̩-kʲím}
        \ex \phonetic{oˈten}\gloss{I copy} < \wordng{PW}{*n̩\mbox{-}tén}
        \ex \phonetic{otkʲoiˈɬi}\gloss{I sing} < \wordng{PW}{*n̩\mbox{-}t\mbox{-}’ikʲój\mbox{-}ɬih}
        \ex \phonetic{otkʲuiˈɬi}\gloss{I vomit} < \wordng{PW}{*n̩\mbox{-}t\mbox{-}kʲ’új\mbox{-}ɬih}
        \ex \phonetic{oˈhũt}\gloss{I push} < \wordng{PW}{*n̩\mbox{-}hút}
        \ex \phonetic{oˈhʷut}\gloss{I sharpen} < \wordng{PW}{*n̩\mbox{-}xʷút}
        \ex \phonetic{ojaˈhĩnʔ}\gloss{I watch} < \wordng{PW}{*n̩\mbox{-}jáhin}
        \ex \phonetic{ohʷaˈpoʔ}\gloss{my shoulder} < \wordng{PW}{*ń̩\mbox{-}xʷapo}
        \ex \phonetic{oˈl̥ip}\gloss{my piece} < \wordng{PW}{*n̩\mbox{-}ɬ\mbox{-}íp}
        \ex \phonetic{owuˈkeʔ}\gloss{my house} < \wordng{PW}{*ń̩\mbox{-}wukʷ\mbox{-}e}
        \ex \phonetic{ol̥ejˈtek}\gloss{my head} < \wordng{PW}{*n̩\mbox{-}ɬ\mbox{-}éteq}
        \ex \phonetic{oˈkʷej}\gloss{my arm} < \wordng{PW}{*ń̩\mbox{-}kʷej}
        \ex \phonetic{oˈwex}\gloss{my buttocks} < \wordng{PW}{*ń̩\mbox{-}weχ}
        \ex \phonetic{oniˈpil̥}\gloss{my stomach} < \wordng{PW}{*ń̩\mbox{-}nipiɬ}
        \ex \phonetic{ʔowoˈleʔ}\gloss{my hair} < \wordng{PW}{*ń̩\mbox{-}ˀwole}
        \ex \phonetic{opaˈset}\gloss{my lip} < \wordng{PW}{*ń̩\mbox{-}påset}
    \end{xlist}
\z
\boolfalse{listing}

Finally, \citet[131]{VU74} describes the first-person prefix in the Tartagal subdialect of Guisnay as \intxt{no\mbox{-}}, as in \wordnl{no\mbox{-}ˈp’aɬi}{I punish}. Several apparent examples of this prefix are attested by \citet[167–168]{AFG-SS-09} in the variety of Misión Santa María \REF{sn-msm-no}; we believe, however, that these tokens contain an indefinite possessor prefix (\wordng{PW}{*ˀnó\mbox{-}}) and not a first-person prefix, since \citet{SS07} -- our primary source on the variety of Misión Santa María -- documents only \intxt{o\mbox{-}} as the first-person prefix.

\newpage
\booltrue{listing}
\ea
Misión Santa María Wichí \citep[167–168]{AFG-SS-09} \label{sn-msm-no}\\
    \begin{xlist}
        \ex \wordnl{no\mbox{-}qantʃete}{my knee} < \word{PW}{*ˀnó\mbox{-}qamkʲete}{one’s knee}
        \ex \wordnl{no\mbox{-}kaʔis}{my girlfriend} < \word{PW}{*ˀnó\mbox{-}qa\mbox{-}ʔis}{one’s loved one}
        \ex \wordnl{no\mbox{-}k’ahlitʃu}{my tongue} < \word{PW}{*ˀnó\mbox{-}q’aχ\mbox{-}ɬ\mbox{-}ɪkʲ’u}{one’s tongue}
    \end{xlist}
\z
\boolfalse{listing}

\subsubsection{\sound{PM}{*ɬ̩\mbox{-}}} \label{wi-syll-lh}

The third-person possessive and the second-person active prefixes are homonymous in Wichí. While before vowels both consistently take the allomorph /ɬ\mbox{-}/, before consonants their form varies from dialect to dialect.

The most common form is \intxt{la\mbox{-}}; it is found in ’Weenhayek \citep[215]{KC16}, Misión Santa María \citep{SS07}, Misión La Paz \citep[87, 93, 95]{MA08}, Vejoz as spoken in Paraje La Paz \citep{AFG067}, and in Southeastern Wichí, including Rivadavia \citep[67, 100]{JT09-th}, El Sauzalito, Colonia Muñiz, Teniente Fraga, El Sauzalito,  Bazán \citep[48–49]{JB09}, and Lower Bermejeño in general \citep[163, 223]{VN14}. \citet[53, 120]{VN14} documents [l̩] as an optional realization in Lower Bermejeño \REF{wi-lh-lal-lb}.

\booltrue{listing}
\ea
Lower Bermejeño Wichí \citep[53, 120]{VN14} \label{wi-lh-lal-lb}\\
    \begin{xlist}
        \ex \phonetic{laˈmuq} \recind \phonetic{l̩ˈmuq}\gloss{dust ($=$ its powder)} < \wordng{PW}{*ɬ̩\mbox{-}mókʷ}
        \ex \phonetic{laˈˀwu} \recind \phonetic{l̩ˈˀwu}\gloss{her/his neck} < \wordng{PW}{*ɬ̩\mbox{-}ˀwo}
        \ex \phonetic{lapaˈtʃ’u} \recind \phonetic{l̩paˈtʃ’u}\gloss{her/his foot} < \wordng{PW}{*ɬ̩\mbox{-}pákʲ’o}
        \ex \phonetic{laˈles} \recind \phonetic{l̩ˈles}\gloss{her/his children} < \wordng{PW}{*ɬ̩\mbox{-}lés}
    \end{xlist}
\z

In Misión El Carmen, the third-person possessive prefix is attested as [la\mbox{-}] or [lə\mbox{-}] \REF{wi-lh-lal-mec}.

\ea
Misión El Carmen Wichí \citep[127, 130]{MC09} \label{wi-lh-lal-mec}\\
    \begin{xlist}
        \ex \phonetic{laˈkɯ} \recind \phonetic{laˈkʲɯʰ}\gloss{her/his mother} < \wordng{PW}{*ɬ̩\mbox{-}qoh}
        \ex \phonetic{ləˈʔax}\gloss{her/his mouth} < \wordng{PW}{*ɬ̩\mbox{-}q’áχ}
    \end{xlist}
\z
\boolfalse{listing}

The form \intxt{le\mbox{-}} is attested in Vejoz as spoken in Misión Chaqueña by \citet[131]{VU74} and \citet[29]{MG-MELO15} as well as in Lapacho Mocho by \citep[164]{AFG-SS-09}. It is also documented by \citet[150–151]{AFG-SS-09} in the forms [leˈnix] ‘its smell’ and [leˈpes] ‘its end’, but unfortunately the dialectal provenance of these forms is not identified (in total, five varieties are discussed in the cited paper: Paraje La Paz, Misión Santa María, Lapacho Mocho, Santa Victoria Este, and Las Vertientes).

The most divergent form, \intxt{ha\mbox{-}}, is documented in Tartagal, as in [haˌwatsanˈtʃejah] `her/his life' \citep[131]{VU74}.

It seems unproblematic to reconstruct the preconsonantal allomorph of the PW third-person possessive and the second-person active prefix as \intxt{*ɬ̩\mbox{-}}. It is even possible that the sound in question occurred within roots, as in \word{PW}{*ɬ̩p’í}{tayra} > \wordng{LB}{lap’i} \citep[48]{VN14}, \wordng{Vejoz or Guisnay}{lep’i} \citep[54]{RL16}, \wordng{’Wk}{lap’íʔ} \citep[220]{KC16}.

\subsection{Vowels}

The Proto-Wichí vowel inventory */i~e~a~å~o~u/ is virtually identical to that of Proto-Mataguayan, except that \sound{PM}{*ä} merged with \sound{PM}{*e} as \sound{PW}{*e} (see \sectref{pm-wi-ae}).\footnote{\citet[129–130]{EN71} offers a reconstruction of Proto-Wichí (``Premataco'') vowels that differs considerably from ours; her proposed inventory of Proto-Wichí vowels includes ten phonemes: */i~ɪ~e~ɛ~ɨ~a~u~ʊ~o~ɔ/. Since the cited work does not present any linguistic data that would substantiate the analysis therein, we do not discuss \cits{EN71} proposal any further in this chapter. \citet{VN-NA-23} reconstruct a six-vowel inventory identical to ours, but their proposal diverges from ours in significant way, notably in their interpretation of the philological evidence. Regrettably, this book was already completed when we learned of \cits{VN-NA-23} study, and it will not be discussed further in this chapter.} These Proto-Wichí vowels are largely preserved in all dialects except Southeastern. In addition, there appears to have been a somewhat more marginal seventh vowel, which we symbolize as \sound{PW}{*ɪ}; it merged with \sound{PW}{*e} in the Southeastern dialect and with \sound{PW}{*i} in all other dialects (\sectref{wi-ɪ}).

In the Southeastern dialect, as discussed in \sectref{wi-vowel-shift}, the vowels of Proto-Wichí have undergone considerable change thanks to what we dub the Southeastern Wichí vowel shift (cf. \citnp{LCB15}). It likely originated as a pull chain, whereby \sound{PW}{*u} was fronted, lowered, and unrounded to \intxt{e} (merging with the reflexes of \sound{PM}{*i} and \intxt{*ɪ}), \sound{PW}{*o} was consequently raised to \intxt{u}, and \sound{PW}{*å} acquired rounding (the prototypical realization of the resulting vowel in the Southeastern dialect is [ɔ] in Rivadavia and Ingeniero Juárez and [o] in Lower Bermejeño).

Minor phenomena involving vowels are discussed in \sectref{wi-vh} (translaryngeal vowel copying), \sectref{wi-lowering} (vowel lowering before uvulars and glottals), and \sectref{wi-nasalization} (vowel nasalization).

\subsubsection{\sound{PW}{*ɪ}}\label{wi-ɪ}

The vowel \intxt{*ɪ} is not preserved in any known variety of Wichí as an independent phoneme (it is unrelated to the allophone [ɪ] of the phoneme /i/, which occurs in some Wichí dialects after the palatal approximant: /ji/ [jɪ]). It is reconstructed based on the correspondence between /e/ in Southeastern Wichí and /i/ in other dialects. Note that Southeastern /e/ may also reflect \sound{PW}{*e} (reflected as /e/ in all Wichí varieties) or \sound{PW}{*u} (reflected as /u/ in all varieties except Southeastern). That way, \sound{PW}{*ɪ} merges with \intxt{*e} and \intxt{*u} as /e/ in Southeastern Wichí, but with \sound{PW}{*i} in all other varieties.

Three clearest examples of \sound{PW}{*ɪ} are given in \tabref{wi-ie-ex}. In Southeastern Wichí, these are reflected with [e]; in other varieties, one finds [i].

\begin{table}
\caption{Development of \sound{PW}{*ɪ}}
\label{wi-ie-ex}
 \begin{tabularx}{\textwidth}{QllQQ}
  \lsptoprule
            & `egg' & `yica bag' & `black-legged seriema' & source\\\midrule
  Proto-Wichí & *\mbox{-}ɬ\mbox{-}ɪ́kʲ’u & *hɪ́lu & *ˣnɪ́kʲ’u & \\
  \tablevspace
  Rivadavia & \phonetic{ɬeˈkʲe} & \phonetic{h̃ẽˈle} & \phonetic{inẽˈkʲe} & \citet[89–90, 274]{JT09-th}\\
  \tablevspace
  \mbox{Ingeniero Juárez} (Barrio Viejo) & \phonetic{ɬɛˈtʃɛ} & — & — & \citet[360]{LCB15}\\
  \tablevspace
  Bazán & \phonetic{ɬeˈtʃ’e} & \phonetic{hẽˈleʔ} & — & \citet[41, 50]{JB09}\\
  \tablevspace
  ’Weenhayek & \phonetic{ɬiːˈkʲ’uʔ} & \phonetic{hĩːˈluʔ} & \phonetic{ʔinĩːˈkʲ’uʔ} & \citet[32, 75, 150, 263]{KC16}\\
  \tablevspace
  Tartagal & \phonetic{ɬiˈtɕu} & — & — & \citet[360]{LCB15}\\
  \tablevspace
  \mbox{Misión Chaqueña} (Vejoz) & \phonetic{ɬitʃ’uʔ} & \phonetic{hĩlu} & — & \citet[57, 106]{VU74}\\
  \lspbottomrule
 \end{tabularx}
\end{table}

Phonetically, \sound{PW}{*ɪ} must have occupied an intermediary position between \intxt{*e} and \intxt{*i} (maybe IPA~[ɪ], but also [e] is a possibility if the prototypical realization of \sound{PW}{*e} was closer to [ɛ]). Alternatively, it could have been a diphthong ([ei̯] or the like), as suggested by the Nivaĉle and Chorote cognates of \word{PW}{*\mbox{-}ɬ\mbox{-}ɪ́kʲ’u}{egg}: \wordng{Ni}{\mbox{-}ʃajk’u}, \wordng{PCh}{*\mbox{-}éjk’uʔ}.

\subsubsection{Southeastern Wichí vowel shift}\label{wi-vowel-shift}

One of the most notable features of the Southeastern dialect of Wichí is its vowel system. While the vowels \intxt{*i}, \intxt{*e}, and \intxt{*a} of Proto-Wichí are preserved intact,\footnote{There may be slight allophonic differences across dialects involving these vowels. For instance, in Lower Bermejeño /i/ surfaces as [ɪ] following /j/ and /χ/, as in [jɪk] `s/he goes away', [jɪˈwaɬ] `slow', [jukʷaˈχɪ] `s/he chews'; /e/ lowers to [ɛ] after a uvular consonant, as in [nãˈχɛt] `it is rotten', [t’aˈmãχɛx] `s/he looks after it'; /a/ surfaces as [ɑ] next to a tautosyllabic uvular, as in [qɑˈmɑ̃χ] `still', [taˌqʰɑ̃ˈχɑj] `s/he is strong', [ʔisˈtɑq] `white cactus' \citep[41]{VN14}. In the Ingeniero Juárez variety, \citet[362]{LCB15} describes the default realizations of /i/, /e/, and /a/ as [ɪ], [ɛ], and [ɐ], respectively, based on instrumental evidence.} all back vowels change in the following way.

\sound{PW}{*u} merges with \sound{PW}{*e} as \intxt{e} (narrow transcription: [e] or [ɛ]\footnote{The mid-low realization [ɛ] is reported for the Ingeniero Juárez variety \citep[362]{LCB15}. In Lower Bermejeño, [e] is the default allophone, whereas [ɛ] is found after a uvular consonant, as in [nãˈχɛt] `it is rotten', [t’aˈmãχɛx] `s/he looks after it'.}) in all subdialects of Southeastern Wichí. It is unknown whether this sound change involved any intermediate steps, such as \intxt{*ɨ} > \intxt{*ə}, \intxt{*y} > \intxt{*ø}, or \intxt{*ʊ} > \intxt{*ə}. At any rate, this non-trivial sound change is exceptionless, and examples abound: \word{PW}{*túnte}{stone}, \wordnl{*nap’u \recind *náp’u \recind *nap’úh}{s/he licks}, \wordnl{*púle}{sky, cloud} > \wordng{LB}{tente} [tenˈte], \intxt{nap’e} [nãˈp’e], \intxt{pele} [peˈle] \citep[161, 278, 459]{VN14}, \wordng{Rivadavia}{tente} [tenˈte], \intxt{nape} [nãˈpe] \citep[25, 37]{JT09-th}, \wordng{Ingeniero Juárez}{nap’e} [nãˈɓɛ], \intxt{pele} [pɛˈlɛ] \citep[367]{LCB15}.

\sound{PW}{*o} raises to \intxt{ʊ} in Ingeniero Juárez \citep[362]{LCB15} and to \intxt{u} in Rivadavia \citep[49]{JT09-th} and Lower Bermejeño \citep[41]{VN14}.\footnote{The allophone [ʊ] shows up in Lower Bermejeño only word-finally when stressed, as in [ʔaˌtsin̥ãjˈtsʊ] `these women' \citep[41]{VN14}.} Examples of this sound change include \word{PW}{*hiˀno}{man}, \wordnl{*hólo}{sand}, \wordnl{*wóq’oh}{owl} > \wordng{LB}{hiˀnu} [hĩˈˀnũ], \intxt{hulu} [hũˈlu], \intxt{wuq’u} [wuˈq’u] \citep[66, 161]{VN14}, \wordng{Rivadavia}{hinu} [hĩˈnũ], \intxt{hulu} [hũˈlu] `dust', \intxt{wuqu} [wuˈqu] \citep[25, 217–218]{JT09-th}, \wordng{Ingeniero Juárez}{hinu} [h̃ɪ̃ˈnʊ̃], \intxt{hulu} [h̃ʊ̃ˈlʊ], \intxt{wuk’u} [wʊˈk’ʊ] \citep[364, 367]{LCB15}.

In turn, \sound{PW}{*å} acquires rounding and raises to \intxt{ɔ} in Rivadavia \citep[77]{JT09-cap} and in Ingeniero Juárez \citep[362]{LCB15}, whereas in Lower Bermejeño its prototypical realization is \intxt{o} \citep[41]{VN14}.\footnote{The allophone [ɔ] occurs in Lower Bermejeño next to a tautosyllabic uvular, as in [ˈɬɔq] `food', [ʔaˈqɔχ] `it is tasty', [ˈtɔχ] `realis conjunction' \citep[41]{VN14}.} For simplicity’s sake, we represent the vowel in question as \intxt{o} even in the varieties of Rivadavia and Ingeniero Juárez (except in narrow transcriptions). Examples of this sound change include \word{PW}{*haˀlå}{tree}, \wordnl{*ˀnǻjiχ}{road} > \wordng{LB}{haˀlo} [hãˈˀlo], \intxt{ˀnojiχ} [ˀnõˈjɪx] \citep[66, 110]{VN14}, \wordng{Rivadavia}{halo} [hãˈlɔ], \intxt{\mbox{-}nojiχ} [\mbox{-}nɔ̃ˈjix] \citep[68, 83]{JT09-th}, \wordng{Ingeniero Juárez}{halo} [h̃ɐ̃ˈlɔ], \intxt{ˀnojiχ} [nɔ̃ˈjɪx] \citep[367, 372]{LCB15}. As a consequence, Southeastern Wichí no longer has a back low vowel that would contrast with /a/.

Finally, \sound{PW}{*ɪ}, as shown in \sectref{wi-ɪ}, also merges with \sound{PW}{*e} and \intxt{*u} as \intxt{e} in Southeastern Wichí.

In the variety spoken in Misión El Carmen, only \sound{PW}{*u} and \intxt{*o} change to \intxt{e}, \intxt{u}, respectively, as shown in \REF{mec-vs-paralytic}–\REF{mec-vs-ear}. The reflex of \intxt{*o} is also attested as [ɯ] following velar stops and nasals in Misión El Carmen (and in El Sauzalito), as in \REF{mec-vs-wound}–\REF{mec-vs-butterfly}. By contrast, \sound{PW}{*å} remains as a low vowel in that variety \citep[135–136]{MC09}, and its range of possible realizations includes [ɑ] and [ɤ], as in \REF{mec-vs-red}–\REF{mec-vs-pocket}. \sound{PW}{*a} is usually reflected as [a], though [ɑ] and [ɤ] have also been attested, as in \REF{mec-vs-tobacco}–\REF{mec-vs-steals}, suggesting that the contrast between /a/ and /ɑ/ is fading away at least in some environments in Misión El Carmen.

\booltrue{listing}
\ea
Misión El Carmen Wichí \citep[125, 127, 130, 132, 137]{MC09}
    \begin{xlist}
        \ex \phonetic{ˌts(ʰ)exʷəˈlax}\gloss{paralytic} < \wordng{PW}{*tsúxʷlaχ} \label{mec-vs-paralytic}
        \ex \phonetic{ʔʲuˈte}\gloss{ear} < \wordng{PW}{*\mbox{-}kʲ’óte} \label{mec-vs-ear}
        \ex \phonetic{aˈmɯ̃ʔ}\gloss{grain} < \word{PW}{*ʔamo}{wound} \label{mec-vs-wound}
        \ex \phonetic{laˈkɯ} \recind \phonetic{laˈkʲɯʰ}\gloss{her/his mother} < \wordng{PW}{*ɬ̩\mbox{-}qoh}
        \ex \phonetic{kʲɯˈkɯk}\gloss{butterfly} < \wordng{PW}{*kʲókʷokʷ} \label{mec-vs-butterfly}
        \ex \phonetic{iˈkʲɤt}\gloss{it is red} < \wordng{PW}{*ʔikʲǻt} \label{mec-vs-red}
        \ex \phonetic{n̩kɑˈʰɲi}\gloss{my pocket} < \wordng{PW}{*n̩\mbox{-}qhǻ\mbox{-}j\mbox{-}hih} \label{mec-vs-pocket}
        \ex \phonetic{juˈkɤs}\gloss{tobacco} < \wordng{PW}{*jókʷas} \label{mec-vs-tobacco}
        \ex \phonetic{ˈkʰɑ} \recind \phonetic{ˈqʰa} \recind \phonetic{ˈk’ɑ}\gloss{no} < \wordng{PW}{*qhá}
        \ex \phonetic{kɑˈnu}\gloss{needle} < \wordng{PW}{*\mbox{-}qáno}
        \ex \phonetic{\mbox{-}ˈʔax} \recind \phonetic{\mbox{-}ˈʔɑx}\gloss{mouth} < \wordng{PW}{*\mbox{-}q’áχ}
        \ex \phonetic{isˈkɑt}\gloss{s/he steals} < \wordng{PW}{*ʔi\mbox{-}sqat} \label{mec-vs-steals}
    \end{xlist}
\z
\boolfalse{listing}

The sound correspondences that arose as the result of the Southeastern Wichí vowel shift are discussed in \citet{CM-JB-90} and \citet{LCB15}, but no attempt at a comparative reconstruction is made in either of these works.

\subsubsection{Vowels outside Southeastern Wichí}

The Wichí dialects that did not undergo the Southeastern Wichí vowel shift typically have a vowel inventory composed of six phonemes: /i~e~a~å~o~u/ (the seventh vowel of Proto-Wichí, */ɪ/, merged with /i/ in these varieties, as discussed in \sectref{wi-ɪ}). Their typical realizations are, respectively, [i], [e], [a], [ɑ],\footnote{\citet{SS07} represents this vowel as [ʌ] in the variety of Misión Santa María but still describes it as a ``low back open unrounded vowel'', suggesting that IPA~[ɑ] is the correct symbol also in the Misión Santa María variety.} [o], [u]. In the variety of Tartagal, \citet[362]{LCB15} reports /i~e~a~u/ to stand for [ɪ~ɛ~ɐ~ʊ], based on acoustic evidence. In the variety of Misión Santa María, one minor allophone is [ɛ], which occurs as an optional realization of /e/ word-finally, as in [manˈse] \recind [manˈsɛ] `boy' \citep{SS07}. In the Paraje La Paz subdialect of Vejoz as described by \citet{AFG067}, /i~e/ can optionally surface as [ɪ~ɛ] in a number of environments, as in \REF{wi-vow-plp-iyela}–\REF{wi-vow-plp-laqe}; only [ɛ], but not [e], is reported to occur in the latter variety following uvulars \REF{wi-vow-plp-laqe}. The vowel /u/ has the unrounded allophone [ɯ], which occurs following glottalized stops, as in \REF{wi-vow-plp-lick}–\REF{wi-vow-plp-hard}.

\booltrue{listing}
\ea
Paraje La Paz Wichí \citep{AFG067}\\
    \begin{xlist}
        \ex \phonetic{ɪjeˈla}\gloss{tapir} < \wordng{PW}{*ˣjéˀlah} \label{wi-vow-plp-iyela}
        \ex \phonetic{okaˈsɪt}\gloss{I stand} < \wordng{PW}{*n̩\mbox{-}t\mbox{-}qásit}
        \ex \phonetic{puˈlɛʔ}\gloss{sky} < \wordng{PW}{*púle}
        \ex \phonetic{laˈqɛ} \recind \phonetic{lɑˈqɛ}\gloss{it shines} < \wordng{PW}{*laq’e} \label{wi-vow-plp-laqe}
        \ex \phonetic{onaˈɓ̥ɯ}\gloss{I lick} < \wordng{PW}{*n̩\mbox{-}náp’u} \label{wi-vow-plp-lick}
        \ex \phonetic{ˈɗ̥ɯn}\gloss{it is hard} < \wordng{PW}{*t’ún} \label{wi-vow-plp-hard}
    \end{xlist}
\z

In addition, the contrast between /å/ and /a/ has been reported to be fading away or altogether non-existent in quite a number of dialects. For example, \citet[359]{LCB15} explicitly claims that no back low vowel has been attested in the variety spoken in Tartagal, and documents forms such as those in \REF{wi-ao-a-tart}, suggesting that \sound{PW}{*a} and \intxt{*å} merged in Tartagal as \intxt{a} (phonetically [ɐ]).

\ea \label{wi-ao-a-tart}
Tartagal Wichí \citep[359]{LCB15}\\
    \begin{xlist}
        \ex \wordnl{o\mbox{-}t\mbox{-}’an}{I shout} < \wordng{PW}{*n̩\mbox{-}t\mbox{-}’ǻn}
        \ex \wordnl{hala}{tree} < \wordng{PW}{*haˀlå}
        \ex  \wordnl{sip’a}{police} < \word{PW}{*sip’å}{hat; fish sp.\species{Sorubim lima (?)}}
        \ex \wordnl{sop’a}{wax} < \wordng{PW}{*sóp’a}
        \ex \wordnl{ts’ak}{navel} < \wordng{PW}{*\mbox{-}ts’aq \recind *\mbox{-}ts’áq}
    \end{xlist}
\z

In the Misión La Paz subdialect of Guisnay, \citet{MA08} documents both [a] and [ɑ] but argues that [ɑ] is an allophone of /a/ in that variety, based on the absence of minimal pairs and on the fact that ``the consultants also inconsistently produced and identified the back low unrounded vowel [ɑ]'' \citep[71]{MA08}. In the available corpus of the Misión La Paz variety, there are examples both of [ɑ] going back to \sound{PW}{*a}, as in \REF{wi-ao-a-mlp-aloja}–\REF{wi-ao-a-mlp-horsefly}, and of [a] going back to \sound{PW}{*å}, as in \REF{wi-ao-a-mlp-word}–\REF{wi-ao-a-mlp-pepper}, though in most cases the lexical distribution of [a] and [ɑ] in \cits{MA08} description does match the state reconstructed for Proto-Wichí, as shown in \REF{wi-ao-a-mlp-ax}–\REF{wi-ao-a-mlp-rope}. Although \citet[71]{MA08} is unable to determine the conditioning environment for the occurrence of the back allophone, she notes that ``the majority of instances of [ɑ] occur before the following phonemes: /s/, /x/, /ʔ/, /q/, and /hʷ/'' and that ``[i]t also occurs after /q/ and /h/'', leaving the question for future research. We surmise that the Misión La Paz subdialect of Guisnay may actually preserve the contrast between \sound{PW}{*a} and \intxt{*å}, but in some cases \sound{PW}{*a} may have changed into \intxt{å} (especially next to uvulars) and vice versa.

\ea \label{wi-ao-a-mlp}
Misión La Paz Wichí \citep{MA08}\\
    \begin{xlist}
        \ex \phonetic{hɑ̃ˈt’es}\gloss{aloja, alcoholic beverage} < \wordng{PW}{*hat’es} \label{wi-ao-a-mlp-aloja}
        \ex \phonetic{qɑl̥qɑl̥ˈtɑx}\gloss{turkey} < \wordng{PW}{*qáɬqaɬ\mbox{-}taχ}
        \ex \phonetic{jaqɑʔˈtuʔ}\gloss{it is yellow} < \wordng{PW}{*qáʔtu}
        \ex \phonetic{laˈqɑs}\gloss{horsefly} < \wordng{PW}{*laqas} \label{wi-ao-a-mlp-horsefly}
        \ex \phonetic{ʔnoɬaˈmet}\gloss{one’s word} < \wordng{PW}{*ˀno\mbox{-}ɬ\mbox{-}ǻmet} \label{wi-ao-a-mlp-word}
        \ex \phonetic{pãˈn̥an}\gloss{red pepper} < \wordng{PW}{*pǻnhån} \label{wi-ao-a-mlp-pepper}
        \ex \phonetic{hõˈsanʔ}\gloss{ax} < \wordng{PW}{*hósaˀn} \label{wi-ao-a-mlp-ax}
        \ex \phonetic{toˈhʷãj}\gloss{pots} < \wordng{PW}{*towh\mbox{-}ájʰ}
        \ex \phonetic{hãˈʔjɑx}\gloss{jaguar} < \wordng{PW}{*haˀjåχ}
        \ex \phonetic{niˈjɑqʷ}\gloss{rope} < \wordng{PW}{*níjåkʷ} \label{wi-ao-a-mlp-rope}
    \end{xlist}
\z
\boolfalse{listing}

Although /å/ and /a/ are reported to contrast in the Paraje La Paz subdialect of Vejoz, as in the minimal pair \wordnl{\mbox{-}paq}{to paint} and \wordnl{påq}{here}, \citet{AFG067} notes that /a/ may surface as [ɑ] next to a uvular \REF{wi-ao-a-plp}.

\booltrue{listing}
\ea \label{wi-ao-a-plp}
Paraje La Paz Wichí \citep{AFG067}\\
    \begin{xlist}
        \ex \phonetic{qɑˈlɑq}\gloss{cocoi heron\species{Ardea cocoi}} < \wordng{PW}{*qaláq}
        \ex \phonetic{woˈtɑq}\gloss{necklace} < \wordng{PW}{*\mbox{-}ˀwó\mbox{-}t\mbox{-}’aq}
    \end{xlist}
\z

In the variety of Lapacho Mocho, instances of intraspeaker variation of the types [ʌ]~\recind~[a] \REF{wi-ao-a-lm-tail} and [ʌ]~\recind~[o] \REF{wi-ao-a-lm-clay} have been documented, corresponding to \sound{PW}{*å} \citep[164]{AFG-SS-09}.

\ea
Lapacho Mocho Wichí \citep[164]{AFG-SS-09}\\
    \begin{xlist}
        \ex \phonetic{leˈtʃas} \recind \phonetic{leˈtʃʌs}\gloss{its tail} < \wordng{PW}{*ɬ̩\mbox{-}kʲås} \label{wi-ao-a-lm-tail}
        \ex \phonetic{iˈhnjɑt} \recind \phonetic{iˈhnjot}\gloss{clay} < \wordng{PW}{*ʔijhåt} \label{wi-ao-a-lm-clay}
    \end{xlist}
\z
\boolfalse{listing}

In the variety of Misión Santa María as described by \citet{SS07} and in the Misión Chaqueña subdialect of Vejoz as described by \citet{VU74}, /å/ and /a/ are documented as contrastive units, but their distribution does not always match the state reconstructed for Proto-Wichí: compare Misión Santa María [aˈkɑs] `it is raw', [iˈtas] `matches', [iˈhnjat] `clay' and \wordng{PW}{*ʔaqas}, \wordnl{*ʔítå\mbox{-}s}{fire.\PL}, \intxt{*ʔijhåt} \citep[168]{SS07,AFG-SS-09}. At least in the case of the Misión Chaqueña subdialect of Vejoz, this may have to do with instances of mistranscription on \cits{VU74} part rather than with sound change, since a more recent work on the same variety by \citet{MG-MELO15} does attest /å/ (represented by means of the grapheme ‹ä›) and /a/ in accordance with our Proto-Wichí reconstruction. For example, the reflex of \word{PW}{*ɬ\mbox{-}ǻmte\mbox{-}s}{her/his words, language} is attested as \intxt{ɬ\mbox{-}amte\mbox{-s}} in \citet[65]{VU74}, but as \intxt{ɬ\mbox{-}åmte\mbox{-s}} in \citet[15, 79]{MG-MELO15}.

\subsubsection{Translaryngeal vowel copying}\label{wi-vh}

Translaryngeal vowel copying (sometimes referred to as vowel harmony in the literature) is a very limited phenomenon in Wichí. It has been documented in the Rivadavia and Ingeniero Juárez subdialects of Southeastern Wichí, where the vowel /i/ of the irrealis suffix /\mbox{-}h̃i/ and of the negative suffix /\mbox{-}h̃it’e/ progressively  assimilates to the vowel /u/ of the preceding applicative suffix /\mbox{-}hu/. Examples from the variety of Ingeniero Juárez include [jɐh̃ɛ̃ˈtʰʊh̃ʊ̃ɗɛ] (underlying /i\mbox{-}het\mbox{-}h̃u\mbox{-}h̃it’e/) `s/he does not gather', [ɪtsɔˈn̥ʊ̃h̃ʊ̃ɗɛ] (underlying /i\mbox{-}tson\mbox{-}h̃u\mbox{-}h̃it’e/) `s/he does not pin', [nɪ̃skɐˈtʰʊh̃ʊ̃ɗɛ] (underlying /ni\mbox{-}sqat\mbox{-}h̃u\mbox{-}h̃it’e/) `I do not hide' \citep[97]{LCB-MBC09}. An example from the Rivadavia variety is given below.

\ea
Rivadavia Southeastern Wichí \citep[50]{JT09-th}\\
    \begin{xlist}
        \ex\gll wahat-ɬe i-tson̥-u-hut’e n̩-qolo\\
                fish-fishbone 3{\textsc{i}}-pin-\APPL-{\NEG} 1\SG-foot\\
                \glt `The fishbone did not pin my foot.'
    \end{xlist}
\z

\subsubsection{Vowel lowering}\label{wi-lowering}

In some dialects of Wichí, the allomorph \intxt{*ji\mbox{-}} of the Proto-Wichí verbal I-class prefix, which shows up before uvular consonants and \intxt{*h} (see \sectref{wi-yi}), has changed into \intxt{ja\mbox{-}}. This development is regular in ’Weenhayek \REF{wi-lh-vl-whk}.

\booltrue{listing}
\ea
’Weenhayek \citep{KC16} \label{wi-lh-vl-whk}\\
    \begin{xlist}
        \ex \wordnl{[ja]qákʲu\mbox{-}\APPL}{s/he distrusts} < \wordng{PW}{*[ji]qákʲu\mbox{-}\APPL}
        \ex \wordnl{[ja]qǻnkʲiʔ}{s/he destroys} < \wordng{PW}{*[ji]qǻnkʲi}
        \ex \wordnl{[ja]qǻx}{s/he crushes} < \wordng{PW}{*[ji]qǻχ}
        \ex \wordnl{[ja]qójʔ}{s/he plays} < \wordng{PW}{*[ji]qój}
        \ex \wordnl{[ja]hǻn̥}{s/he follows} < \wordng{PW}{*[ji]hǻn}
        \ex \wordnl{[ja]hó\mbox{-}\APPL}{s/he goes} < \wordng{PW}{*[ji]hó\mbox{-}\APPL}
        \ex \wordnl{[ja]hút}{s/he pushes} < \wordng{PW}{*[ji]hút}
        \ex \wordnl{[ja]hán\mbox{-}ex}{s/he knows} < \wordng{PW}{*[ji]hán\mbox{-}eχ}
        \ex \wordnl{[ja]húmin̥}{s/he loves} < \wordng{PW}{*[ji]húmin}
    \end{xlist}
\z

By contrast, the change never occurs in the Lower Bermejeño subdialect of Southwestern Wichí \REF{wi-lh-vl-lb}. Note that the sequence /ji/ is pronounced [jɪ] in Lower Bermejeño, as in [jɪk] `s/he goes away', [jɪˈwaɬ] `slow' \citep[41]{VN14}.

\ea
Lower Bermejeño Wichí \citep{VN14,JB09} \label{wi-lh-vl-lb}\\
    \begin{xlist}
        \ex \wordnl{[ji]qontʃi}{s/he destroys} < \wordng{PW}{*[ji]qǻnkʲi}
        \ex \wordnl{[ji]qoχ\mbox{-}ɬi}{s/he crushes} < \wordng{PW}{*[ji]qǻχ\mbox{-}ɬih}
        \ex \wordnl{[ji]quj}{s/he plays} < \wordng{PW}{*[ji]qój}
        \ex \wordnl{[ji]hon}{s/he follows} < \wordng{PW}{*[ji]hǻn}
        \ex \wordnl{[ji]hu\mbox{-}\APPL}{s/he goes} < \wordng{PW}{*[ji]hó\mbox{-}\APPL}
        \ex \wordnl{[ji]het\mbox{-}tsi}{s/he pushes} < \wordng{PW}{*[ji]hút\mbox{-}tshi}
        \ex \wordnl{[ji]han\mbox{-}eχ}{s/he knows} < \wordng{PW}{*[ji]hán\mbox{-}eχ}
        \ex \wordnl{[ji]hemin}{s/he loves} < \wordng{PW}{*[ji]húmin}
    \end{xlist}
\z
\boolfalse{listing}

In the Rivadavia subdialect of Southeastern Wichí, according to \citet[134–135]{JT09-th}, verbs that took \intxt{*ji\mbox{-}} in Proto-Wichí may now take either \intxt{ja\mbox{-}} (if the agent acts with low intensity) or \intxt{ʔi{-}} (if the agent acts with high intensity), as in the following examples.

\newpage
\ea
Rivadavia Southeastern Wichí \citep[135]{JT09-th}\\
    \begin{xlist}
        \ex\gll sip’o ja-hon malewu\\
                police 3{\textsc{i}}-follow thief\\
                \glt `the police chases the thief' (without too much intention of actually catching up with the thief)
        \ex\gll sip’o ʔi-hon malewu\\
                police 3{\textsc{i}}\textsubscript{ACT}-follow thief\\
                \glt `the police chases the thief' (until actually catching up)
        \ex\gll atsin̥a ja-hanex to j-omɬi\\
                woman 3{\textsc{i}}-know {\textsc{sub}} 3{\textsc{i}}-speak\\
                \glt `the woman knows how to speak' (with some knowledge of the language)
        \ex\gll atsin̥a i-hanex to j-omɬi\\
                woman 3{\textsc{i}}\textsubscript{ACT}-know  {\textsc{sub}} 3{\textsc{i}}-speak\\
                \glt `the woman knows how to speak' (with a very good knowledge of the language)
        \ex\gll hinu ja-hemen atsin̥a\\
                man 3{\textsc{i}}-love woman\\
                \glt `the man loves the woman'
        \ex\gll hinu i-hemen atsin̥a\\
                man 3{\textsc{i}}\textsubscript{ACT}-love woman\\
                \glt `the man loves the woman' (and is deeply in love with her)
    \end{xlist}
\z

Little information is available to us on the reflexes of \wordng{PW}{*ji\mbox{-}} in other dialects, such as Vejoz and Guisnay. The reflex \intxt{ja\mbox{-}} is attested as far east as in the Ingeniero Juárez subdialect of Southeastern Wichí: [jɐˈh̃ɛ̃t] `s/he pushes' \citep[100]{LCB-MBC09}.

The same kind of allomorphy is seen in the ’Weenhayek vocative prefix found in some kinship terms (no cognates in other Wichí varieties are known): compare \word{’Wk}{ʔi\mbox{-}xkʲah}{father!} and \wordnl{ja\mbox{-}qoh}{mother!} \citep[445]{JAA-KC-14}. This prefix goes back to the erstwhile first-person singular prefix, \wordng{PM}{*ji\mbox{-}}, and is homophonous with the I-class prefix (itself a reflex of an erstwhile third-person prefix, \wordng{PM}{*ji\mbox{-}}, extended to other persons by means of Watkins’ Law).

The development in question is identical to a process that occurs optionally (or subdialectally) in Iyojwa’aja’ (\sectref{ch-lowering}).

\subsubsection{Nasalization}\label{wi-nasalization}

In many dialects of Wichí, vowels are allophonically nasalized following nasal onsets, but also following a /h/, represented by some authors as /h̃/ \citep{JT09-cap,LCB-MBC09}. This is described for ’Weenhayek by \citet[12–13]{KC94}, for the Ingeniero Juárez subdialect of Southeastern Wichí by \citet[100]{LCB-MBC09}, for the Rivadavia subdialect of Southeastern Wichí by \citet[78–79]{JT09-cap}, and for the Lower Bermejeño subdialect of Southeastern Wichí by \citet[41–42]{VN14}.

\booltrue{listing}
\ea
’Weenhayek \citep[12–13]{KC94}\\
    \begin{xlist}
        \ex /∅-nek/~\phonetic{ˈnẽk}\gloss{s/he comes}
        \ex /móp’i/~\phonetic{mõːˈp’iʔ}\gloss{white heron}
        \ex /nísåh-és/~\phonetic{nĩːsɑˈhẽs}\gloss{shoes}
        \ex /nú-lís/~\phonetic{nũːˈlis}\gloss{bones}
        \ex /hup/~\phonetic{ˈhũp}\gloss{hut}
        \ex /haˀjåx/~\phonetic{hãˈʔjɑx}\gloss{jaguar}
        \ex /ˀnó-nhus/~\phonetic{ʔnõːˈn̥nũs}\gloss{one's nose}
        \ex /tájhi/~\phonetic{tãːˈɲ̊jĩʔ}\gloss{woods}
        \ex /la-whǻj/~\phonetic{laˈŋ̥wɑ̃ːjʔ}\gloss{its time}
    \end{xlist}
\z

\ea
Southeastern (Ingeniero Juárez) \citep[100, 102–103]{LCB-MBC09}\\
    \begin{xlist}
        \ex /nojix/~\phonetic{nɔ̃ˈjɪx}\gloss{road, path}
        \ex /inot/~\phonetic{ɪˈnɔ̃t}\gloss{water}
        \ex /itox-muk/~\phonetic{ɪˌtɔxˈmʊ̃k}\gloss{ashes}
        \ex /mak/~\phonetic{ˈmɐ̃k}\gloss{thing}
        \ex /i=h̃i/~\phonetic{ˈɪh̃ɪ̃}\gloss{s/he is}
        \ex /jah̃et/~\phonetic{jɐˈhɛ̃̃t}\gloss{s/he pushes}
        \ex /loj-h̃en/~\phonetic{lɔˈj̃ɛ̃n}\gloss{they are alive}
        \ex /ta-toj-h̃it’e/~\phonetic{tɐtɔˈj̃ɪ̃ɗɛ}\gloss{they do not lose}
        \ex /j-op(i)l-h̃it’e/~\phonetic{jɔpˈn̥ɪ̃ɗɛ}\gloss{s/he does not come back}
        \ex /tajh̃i/~\phonetic{tɐˈj̃ɪ̃}\gloss{forest}
        \ex /fʷijhu/~\phonetic{fʷɪˈj̃ʊ̃}\gloss{charcoal}
    \end{xlist}
\z

\newpage
\ea
Southeastern (Rivadavia) \citep[78–79]{JT09-cap}\\
    \begin{xlist}
        \ex /inot/~\phonetic{iˈnɔ̃t}\gloss{water}
        \ex /la-h̃eseq/~\phonetic{lah̃ẽˈseq}\gloss{her/his spirit}
        \ex /h̃alo/~\phonetic{h̃ãˈlɔ}\gloss{tree}
        \ex /h̃inu/~\phonetic{hĩˈnũ}\gloss{man}
        \ex /ta-qataj-h̃en/~\phonetic{taqataˈh̃j̃ẽn}\gloss{they cook}
    \end{xlist}
\z

\ea
Southeastern (Lower Bermejeño) \citep[42]{VN14}\\
    \begin{xlist}
        \ex /ama/~\phonetic{ʔaˈmã}\gloss{rat}
        \ex /note/~\phonetic{nõˈte}\gloss{tapeti}
        \ex /hope/~\phonetic{hõˈpe}\gloss{copula}
        \ex /la-nhes/~\phonetic{laˈn̥ẽs}\gloss{her/his nose}
        \ex /la-whoj/~\phonetic{laˈʍõj}\gloss{its time}
        \ex /tajhi/~\phonetic{taˈj̊ĩ}\gloss{forest}
        \ex /∅-tijoχ-pho/~\phonetic{tiˌjɔχˈpʰõ}\gloss{s/he jumps over}
        \ex /j-uq-tʃhoχ/~\phonetic{juqˈtʃʰɔ̃χ}\gloss{s/he crushes}
    \end{xlist}
\z

In the Misión La Paz subdialect of Guisnay, only /h/ -- but not the nasals /m~n/ -- is reported to trigger nasalization in the following vowel, and sometimes in the preceding vowel as well \citep[69–71, 83–84]{MA08}.

\ea
Guisnay (Misión La Paz) \citep[46–47, 70–71, 92]{MA08}\\
    \begin{xlist}
        \ex /holoʔ/~\phonetic{hõˈloʔ}\gloss{dust}
        \ex /ˀno-kʲ’aheʔ/~\phonetic{ʔnokʲ’aˈhẽʔ}\gloss{arrow}
        \ex /o-jahiˀn/~\phonetic{ojaˈhĩnʔ}\gloss{I watch}
        \ex /ˀno-humin/~\phonetic{ʔnõhũˈmin}\gloss{lover}
        \ex /ˀwahat-woʔ/~\phonetic{ʔwãhãtˈwoʔ}\gloss{fisherman}
        \ex /la-womha-j/~\phonetic{lawoˈm̥ãj}\gloss{gorges}
        \ex /kʲowh-aj/~\phonetic{kʲoˈʍãj}\gloss{holes}
        \ex /amaʔ/~\phonetic{aːˈmaʔ}\gloss{rat}
        \ex /pinu/~\phonetic{piˈnu}\gloss{sugarcane}
        \ex /nahajox/~\phonetic{nahãˈjox}\gloss{heat}
    \end{xlist}
\z
\boolfalse{listing}

Some Wichí lects lack nasalization in the environments described above. In the Paraje La Paz subdialect of Vejoz, vowels are reported to be nasalized before nasal consonants \citep{AFG067}. In Misión Santa María, nasalization is reported to occur after the sequence [h̃n] (<~\sound{PW}{*jh}, \intxt{*nh}), as in [ohˈnũs] `my nose', and sometimes next to nasals, as in [asˈnãm] `blind'.

\subsection{Word-level prosody}
\fussy
We have seen in \sectref{wi-prosody} that two suprasegmental phenomena coexisted in Proto-Wichí: contrastive vowel length, which continues the left-aligned accent of Proto-Mataguayan, and right-aligned stress, which in all likelihood represents a Wichí innovation.

The only variety known to preserve the contrastive vowel length of Proto-Wichí is ’Weenhayek, whereas in all other lects no equivalent phenomenon has been documented so far. It is possible that it is preserved in some varieties spoken in Argentina, such as the variety of Misión Santa María, where forms such as [woˈjiːs] `blood' and [ˈaːm] `you' (<~\wordng{PW}{*ˀwojís} and \intxt{*ʔáˀm}) have been attested \citep{SS07}.\footnote{An anonymous reviewer remarks that the vowel length in these specific examples ``seems to be related to stress, but it does not necessarily mean that vowel length is contrastive'' in the variety of Misión Santa María. We agree that the evidence is inconclusive, especially given the fact that PW long vowels are often reflected as what \citet{SS07} documents as short vowels: [taˈtʃ’i] `rufous hornero', [ɬeˈtʃe] `her/his thigh' (<~\wordng{PW}{*táts’i}, \intxt{*ɬ\mbox{-}ékʲe}). Note, however, that the reflexes of PW short stressed vowels are uniformly attested as short by \citet{SS07}, with no exceptions: [oˈjiɬ] `I die', [iˈmaʔ] `s/he sleeps' (<~\wordng{PW}{*n̩\mbox{-}jilʰ}, \intxt{*ʔi\mbox{-}måʔ}). More data would be needed in order to arrive at robust conclusions regarding the status of vowel length in the variety of Misión Santa María.} \citet[63]{MA08} reports that in the Misión La Paz subdialect of Guisnay ``there is some slight vowel lengthening in certain environments, but at this time, these environments are not clear''. Future documentation is needed to ascertain the status of the long vowels in Misión Santa María, Misión La Paz, and possibly other varieties spoken in the vicinities of the Bolivian border. In the Lower Bermejeño subdialect of Southeastern Wichí, vowels that carry primary or secondary stress are automatically lengthened \citep[123]{VN14}, but this phenomenon clearly has nothing to do with the contrastive vowel length of Proto-Wichí.

As for the right-aligned stress, the general pattern is apparently preserved in all varieties of Wichí, though the underlying specifications of certain suffixes (i.~e., whether metrical or extrametrical) may differ across dialects, as we have seen in \sectref{wi-stress}. Secondary stress is relatively well described only for the Lower Bermejeño subdialect of Southeastern Wichí, where iambic feet are built from the left edge of the word and the heads of the non-final feet receive secondary stress, as in \wordnl{(n̩\mbox{-}ˌj\mbox{-}is)(t\mbox{-}ʰiˌla)\mbox{-}(ˈʔam)}{I will cut you}, \wordnl{(la\mbox{-}ˌqa)(tih\mbox{-}ˌjen)\mbox{-}(ˈˀnũ)}{you make me jump} \citep[122]{VN14}. Since no such information is available on other dialects of Wichí, it is currently not possible to reconstruct the secondary stress pattern of Proto-Wichí.
