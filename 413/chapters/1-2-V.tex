\chapter{Vowels} \label{pm-vowels}
This chapter deals with the reconstruction of the Proto-Mataguayan vowels. We reconstruct an inventory composed of seven vowels (\sound{PM}{*i}, \intxt{*e}, \intxt{*ä}, \intxt{*a}, \intxt{*å}, \intxt{*o}, \intxt{*u}), as discussed in \sectref{pm-i}–\sectref{pm-u}.

\section{\sound{PM}{*i}}\label{pm-i}
\sound{PM}{*i} is typically preserved as \intxt{i} in all daughter languages: Maká, Nivaĉle, Proto-Chorote, and Proto-Wichí. In Maká, it merges with \sound{PM}{*e}, which also yields \sound{Mk}{i} (see \sectref{pm-e}, \sectref{mk-vowel-shift}). Irregular reflexes include \sound{Mk}{u} in \REF{i-suckb}; \sound{PCh}{*a} in \REF{i-right}, probably due to a sporadic metathesis; and \sound{PW}{*u} in \REF{i-bromelia}, \intxt{*o} in \REF{i-blackalgarrobof}--\REF{i-blackalgarrobot}. In \REF{i-fillv}, \sound{PM}{*i} is unexpectedly lost in Nivaĉle, whereas the Maká form is restructured. The variation \intxt{i} \recind \intxt{e} in Nivaĉle in \REF{i-elderbro} is likewise irregular.

\begin{exe}
    \ex \mouth
    \ex \gofirst
    \ex \stinger
    \ex \rightn \label{i-right}
    \ex \sisinlaw
    \ex \notafraid
    \ex \coldweather
    \ex \hidev
    \ex \crab
    \ex \leech
    \ex \palmg
    \ex \goawayit
    \ex \cryit
    \ex \resin
    \ex \plits
    \ex \dew
    \ex \wax
    \ex \water
    \ex \sandisaj
    \ex \truev
    \ex \neighbor
    \ex \elderbro \label{i-elderbro}
    \ex \eldersis
    \ex \youngerbro
    \ex \youngersis
    \ex \snail
    \ex \whitelim
    \ex \languageword
    \ex \thread
    \ex \savannahhawk
    \ex \rope
    \ex \smelln
    \ex \smellv
    \ex \pathn
    \ex \longv
    \ex \fillv \label{i-fillv}
    \ex \whitequebracho
    \ex \spend
    \ex \suckb \label{i-suckb}
    \ex \shoot
    \ex \spinsew
    \ex \carrysh
    \ex \swallow
    \ex \invite
    \ex \plate
    \ex \dig
    \ex \movev
    \ex \rheum
    \ex \woodpecker
    \ex \hornero
    \ex \bromelia \label{i-bromelia}
    \ex \blackalgarrobof \label{i-blackalgarrobof}
    \ex \blackalgarrobot \label{i-blackalgarrobot}
    \ex \rib
    \ex \recipient
    \ex \waspaniti
    \ex \juice
    \ex \dryout
    \ex \good
    \ex \firei
\end{exe}

The very same correspondence is observed in etymologies with a limited distribution (Maká and Nivaĉle, Chorote and Wichí), whose PM~age is thus questionable.

\begin{exe}
    \ex \spin
    \ex \spillmn
    \ex \ameiva
    \ex \pocote
    \ex \dreamv
    \ex \dreamn
    \ex \tobacco
    \ex \foot
    \ex \coati
    \ex \hunger
    \ex \ocelot
    \ex \tortoise
    \ex \spitmn
    \ex \willow
    \ex \heartmn
    \ex \majan
    \ex \frog
    \ex \standv
    \ex \limpkin
    \ex \siyaj
    \ex \durmili
    \ex \weave
    \ex \metal
    \ex \whiteegret
    \ex \diecw
    \ex \mancw
    \ex \eel
\end{exe}

In Chorote and Wichí, \sound{PM}{*i} lowers to \intxt{*e} before \intxt{*ts}, provided that there is a low vowel in the preceding syllable. This regularly happens when the syllable has \intxt{*t} as the onset, but one example with \sound{PM}{*x} > \sound{PCh/PW}{*h} has also been identified.\footnote{A somewhat similar change has affected the nominal plural suffix \wordng{PM}{*\mbox{-}its} in some Nivaĉle varieties: in the Shichaam Lhavos dialect, \intxt{\mbox{-}is} varies with \intxt{\mbox{-}es} after coronals, whereas in the Chishamnee Lhavos dialect the allomorph \intxt{\mbox{-}es} may be found even after consonants such as \intxt{p} \citep[276–277]{AnG15}.} This proposed sound change admittedly lacks a clear phonetic motivation, but it still seems to be regular. As a consequence, the nominal plural suffix \intxt{\mbox{-}is} in the contemporary Chorote and Wichí varieties shows the allomorph \intxt{\mbox{-}es}, an alternation best described as an instance of progressive height harmony in these languages.

\begin{exe}
    \ex \drinknpl
    \ex \waterpl
    \ex \starn
    \ex \earthpl
    \ex \skinpl
\end{exe}

The examples below show that word-initial instances of \sound{PM}{*ji} > \sound{*ʔi} changed to \sound{PCh}{*ʔa} and \sound{PW}{*ha} preceding a glottalized consonant followed by a low vowel (\sectref{pm-ch-ji-a}, \sectref{pm-wi-ji-ha}).

\begin{exe}
    \ex \jaguar
    \ex \treen
    \ex \vulture
\end{exe}

\section{\sound{PM}{*e}}\label{pm-e}
\sound{PM}{*e} is typically preserved as \intxt{e} in Nivaĉle, Proto-Chorote, and Proto-Wichí. In Maká, it yields \intxt{i} and thus merges with \sound{PM}{*i}. In Chorote and Wichí, it merges with \sound{PM}{*ä} instead. Special reflexes of \sound{PM}{*e} are found before the uvular fricative \sound{PM}{*χ}, as discussed later in this section. Some representative examples follow. Note the irregular reflexes in Maká in \REF{e-bat}, in Nivaĉle in \REF{e-mosquito}, and in Chorote in \REF{e-mortar}.

\begin{exe}
    \ex \honeycomb
    \ex \daughter
    \ex \thorne
    \ex \namen
    \ex \distal
    \ex \mortar \label{e-mortar}
    \ex \rootn
    \ex \north
    \ex \welln
    \ex \monkparakeet
    \ex \sendv
    \ex \feminine
    \ex \earkfe
    \ex \arrowkaxe
    \ex \chaniarf
    \ex \chaniart
    \ex \offspring
    \ex \wash
    \ex \squash
    \ex \bow
    \ex \whitesnail
    \ex \firewoodlhet
    \ex \otter
    \ex \cavy
    \ex \hear
    \ex \rain
    \ex \fatpe
    \ex \beard
    \ex \costume
    \ex \spank
    \ex \whitequebracho
    \ex \eyelash
    \ex \eye
    \ex \plate
    \ex \eyebrow
    \ex \tooth
    \ex \tears
    \ex \rheum
    \ex \bromelia
    \ex \belly
    \ex \walk
    \ex \headn
    \ex \bat \label{e-bat}
    \ex \jelayuk
    \ex \shadow
    \ex \moon
    \ex \wildhoney
    \ex \femalebreast
    \ex \mosquito \label{e-mosquito}
    \ex \teach
    \ex \parrot
    \ex \othern
\end{exe}

The very same correspondence is observed in etymologies with a limited distribution (Maká and Nivaĉle, Chorote and Wichí), whose PM~age is thus questionable.

\begin{exe}
    \ex \jar
    \ex \inhabitant
    \ex \grandchild
    \ex \earcw
    \ex \pacu
    \ex \heartmn
    \ex \skycloud
    \ex \gutscw
    \ex \chachalaca
    \ex \saymn
    \ex \ashamedmn
    \ex \cloudmn
    \ex \onemn
    \ex \leafhaircw
    \ex \dirt
    \ex \orphanmn
    \ex \hiccup
    \ex \heartcw
\end{exe}

Before the uvular fricative \sound{PM}{*χ}, the vowel \intxt{*e} has a special lowered reflex in all languages except Nivaĉle: \sound{Mk}{a} (rather than \intxt{i}), \sound{PCh}{*a} (rather than \intxt{*e}), and \sound{PW}{*a} (rather than \intxt{*e}).

\begin{exe}
    \ex \fatv
    \ex \jabiru
    \ex \quick
    \ex \longv
    \ex \fullriver
    \ex \blackalgarrobof
    \ex \peccary
    \ex \hurt
    \ex \chaguara
    \ex \wildbean
    \ex \mistolf
    \ex \puma
\end{exe}

The very same correspondence is observed in etymologies with a limited distribution (Maká and Nivaĉle, Chorote and Wichí), whose PM~age is thus questionable.

\begin{exe}
    \ex \smoke
    \ex \mollef
\end{exe}

If a consonant intervenes between the target vowel and the uvular trigger, the lowering occurs only in Maká (but not in Chorote and Wichí), and in that case the outcome is \sound{Mk}{e} (rather than \intxt{i}, as in non-lowering environments, or \intxt{a}, as when a uvular consonant is adjacent to the vowel).

\begin{exe}
    \ex \redquebracho
\end{exe}

The lowering induced by the uvular fricative left behind a number of synchronically active alternations in Maká, Chorote, and Wichí. In forms that go back to PM~etyma with a \intxt{*χ}, the lowering applies, and one finds \sound{Mk}{a}, \sound{PCh}{*a}, \sound{PW}{*a}. By contrast, the reflexes of PM~forms derived from the vocalic stems of the same etyma (see \sectref{jj-suff}) show no lowering, because \sound{PM}{*χ} was absent in the respective protoforms. Consequently, one finds \sound{Mk}{i}, \sound{PCh}{*e} (raised to \intxt{i} in the unstressed position in the contemporary varieties), \sound{PW}{*e}. Some examples are given in \REF{ex:uvullow:maka}--\REF{ex:uvullow:whk}.

\booltrue{listing}
\ea\label{ex:uvullow:maka}
        Maká \citep[121, 130, 183]{AG99}
    \begin{xlist}
        \ex \intxt{anhejaχ}\gloss{wild bean} → \intxt{anheji-ʔp}\gloss{wild bean season}
        \ex \intxt{aʔtaχ}\gloss{it hurts} → \intxt{aʔti-ts}\gloss{they hurt}
        \ex \intxt{i-f’ilxetsaχ}\gloss{poor.\SG} → \intxt{i-f’ilxetsi-ts}\gloss{poor.\PL}
    \end{xlist}
\z
\ea
        Iyojwa’aja’ \citep[96, 143, 144]{ND09}
    \begin{xlist}
        \ex \intxt{pánsa}\gloss{fast, quick.\SG} → \intxt{pánsi-s}\gloss{fast, quick.\PL}
        \ex \intxt{p’élisʲe}\gloss{poor.\SG} → \intxt{p’ihlʲúxsi-s}\gloss{poor.\PL}
        \ex \intxt{ʔáʔtʲeh-eʔ}\gloss{it hurts} → \intxt{ʔáʔti-s-i}\gloss{they hurt}
    \end{xlist}
\z
\ea
        Iyo’awujwa’ \citep[120, 166]{AG83}
    \begin{xlist}
        \ex \intxt{álisa}\gloss{cháguar.\SG} → \intxt{álisi-s}\gloss{cháguar.\PL}
        \ex \intxt{tóxsa}\gloss{smoke.\SG} → \intxt{tóxsi-s}\gloss{smoke.\PL}
    \end{xlist}
\z
\ea
        Manjui \citep{JC18}
    \begin{xlist}
        \ex \intxt{p’ilisáh}\gloss{poor.\SG} → \intxt{p’ilisɛ́-s}\gloss{poor.\PL}
    \end{xlist}
\z
\ea\label{ex:uvullow:whk}
        ’Weenhayek \citep[8, 92, 293,  297, 426]{KC16}
    \begin{xlist}
        \ex \intxt{pitáx}\gloss{long.\SG} → \intxt{pité-s}\gloss{long.\PL}
        \ex \intxt{p’alítsax}\gloss{poor.\SG} → \intxt{p’alítse-s}\gloss{poor.\PL}
        \ex \intxt{(-)tútsax}\gloss{smoke} → \intxt{tútse-tax}\gloss{mist}
        \ex \intxt{ʔǻjtax}\gloss{it hurts} → \intxt{ʔǻjte-ts}\gloss{they hurt}
    \end{xlist}
\z
\boolfalse{listing}

In two examples, \sound{PM}{*e} appears to have acquired rounding in Chorote and Wichí before a cluster with a labial consonant, yielding Proto-Chorote and Proto-Wichí~\intxt{*o}.

\begin{exe}
    \ex \bite
    \ex \tooth
\end{exe}

Finally, some cognate sets show deviant correspondences, which seem to instantiate vowel assimilation processes in individual languages. In \REF{e-wildmanioc} and \REF{e-river}, Nivaĉle reflects \sound{PM}{*éwV} as \intxt{oβV}, which could represent a regular pattern of vowel assimilation. An apparently irregular pattern of progressive vowel assimilation is seen in Chorote in \REF{e-lip}.

\begin{exe}
    \ex \wildmanioc \label{e-wildmanioc}
    \ex \lip \label{e-lip}
    \ex \river \label{e-river}
\end{exe}

\section{\sound{PM}{*ä}}\label{pm-ae}

\sound{PM}{*ä} is reconstructed based on the correspondence between \sound{Mk}{e}, \sound{Ni}{a}, \sound{PCh}{*e}, and \sound{PW}{*e}. It therefore merges with \sound{PM}{*a} in Maká and Nivaĉle, but with \sound{PM}{*e} in Chorote and Wichí. Irregular reflexes are seen in Nivaĉle in \REF{ae-flyv}, possibly due to vowel assimilation, as well as in Chorote in \REF{ae-rootn}. The reflex \sound{PW}{*ɪ} in \REF{ae-egg} is apparently the regular continuation of \sound{PM}{*äj}. The reflex \sound{PCh}{*i} in \REF{ae-soninlaw} is due to harmonic rising triggered by the following \intxt{*u} – as opposed to the reflex \sound{PCh}{*e} in \REF{ae-sisinlaw}, a process that might be regular in the environment \intxt{*W\_Lu}, where \intxt{W} stands for a labial consonant and \intxt{L} for a coronal one (compare \word{PCh}{*\mbox{-}pél}{shadow}, but \word{Mj}{\mbox{-}péilik}{shadow} < \intxt{*\mbox{-}píl\mbox{-}uk}).

\begin{exe}
    \ex \burn
    \ex \wing
    \ex \yicaay
    \ex \goawayyou
    \ex \goawaycisl
    \ex \putv
    \ex \flyv \label{ae-flyv}
    \ex \tell
    \ex \sisinlaw \label{ae-sisinlaw}
    \ex \soninlaw \label{ae-soninlaw}
    \ex \fieldn
    \ex \rootn \label{ae-rootn}
    \ex \coldweather
    \ex \crab
    \ex \killbird
    \ex \spouse
    \ex \stretchout
    \ex \dividev
    \ex \chaniarf
    \ex \chaniart
    \ex \flu
    \ex \nightmonkey
    \ex \hither
    \ex \smellv
    \ex \tapeti
    \ex \mesh
    \ex \acquainted
    \ex \abdcavity
    \ex \basetrunk
    \ex \trunk
    \ex \allrcpr
    \ex \burrow
    \ex \walk
    \ex \seev
    \ex \placen
    \ex \egg \label{ae-egg}
    \ex \vrbpl
    \ex \headn
    \ex \eatvi
\end{exe}

The very same correspondence is observed in etymologies with a limited distribution (Maká and Nivaĉle, Chorote and Wichí), whose PM~age is thus questionable.

\begin{exe}
    \ex \dreamn
    \ex \hole
    \ex \bilecw
    \ex \queenpalmf
\end{exe}

The regular reflex in Chorote and Wichí seems to be \intxt{*i} rather than \intxt{*e} in syllables that precede the accented one, though the conditioning environment is not entirely clear at present.

\begin{exe}
    \ex \deep
    \ex \cat
    \ex \duraznillo
    \ex \meat
\end{exe}

\section{\sound{PM}{*a}}\label{pm-a}
\sound{PM}{*a} is typically preserved as \intxt{a} in Nivaĉle, Proto-Chorote, and Proto-Wichí. In Maká it is typically raised to \intxt{e} (whereas \sound{PM}{*e} is raised to \sound{Mk}{i}). Therefore, \sound{PM}{*a} usually merges with \sound{PM}{*ä} in Maká and Nivaĉle. However, \sound{PM}{*a} yields \sound{Mk}{a} before the uvular fricative \sound{PM}{*χ} – as in \REF{a-pseudo}, \REF{a-tsofatajf}, \REF{a-night}, \REF{a-tuscaf} – and assimilates to \sound{Mk}{o} if the following syllable contains an \intxt{*o} – as in \REF{a-armadillo}, \REF{a-wolf}, \REF{a-paralytic}, \REF{a-yawn}. The irregular reflexes in Maká include \intxt{a} in \REF{a-newadj} and \REF{a-knee}; \intxt{i} in \REF{a-hornero}. The irregular reflexes in Nivaĉle include \intxt{e} in \REF{a-thunder}; zero in \REF{a-tsofatajf}--\REF{a-tsofatajt}. The irregular reflexes in Chorote include an irregular metathesis in \REF{a-rightn}; \intxt{*e} in \REF{a-dew}; \intxt{*i} in \REF{a-armadillo}; assimilation to \intxt{*å} in \REF{a-snore} and to \intxt{*o} in \REF{a-face}, \REF{a-eyebrow}, \REF{a-heel}; \intxt{*ᵊ} in \REF{a-cardinal}. In Wichí, the irregular reflexes include \intxt{*i} in \REF{a-coal}; zero in \REF{a-redquebracho}, \REF{a-wildbean}, and \REF{a-hiccup}; \intxt{*a} \recvar \intxt{*e} in \REF{a-newadj}; assimilation to \intxt{*å} in \REF{a-leg} and \REF{a-wildcat}. The unaccented sequence \sound{PM}{*aju} may yield \sound{PW}{*e}, as in \REF{a-iscayante}, \REF{a-tuscat}.

\begin{exe}
    \ex \honeycomb
    \ex \plaj
    \ex \lick
    \ex \fallonitsown
    \ex \mouth
    \ex \fruit
    \ex \bite
    \ex \companion
    \ex \rightn \label{a-rightn}
    \ex \coal \label{a-coal}
    \ex \disease
    \ex \firef
    \ex \centipede
    \ex \cutdown
    \ex \algarrobof
    \ex \north
    \ex \elbow
    \ex \suncho
    \ex \breath
    \ex \dew \label{a-dew}
    \ex \tooln
    \ex \grove
    \ex \redquebracho \label{a-redquebracho}
    \ex \neighbor
    \ex \elderbro
    \ex \eldersis
    \ex \sunn
    \ex \armadillo \label{a-armadillo}
    \ex \spouse
    \ex \barnowl
    \ex \thorncutjan
    \ex \oldn
    \ex \iscayante \label{a-iscayante}
    \ex \louse
    \ex \daylhuma
    \ex \girl
    \ex \interr
    \ex \defect
    \ex \bathe
    \ex \nose
    \ex \newadj \label{a-newadj}
    \ex \snore \label{a-snore}
    \ex \mucus
    \ex \dayworld
    \ex \rain
    \ex \inorderto
    \ex \alienable
    \ex \distrust
    \ex \leg \label{a-leg}
    \ex \fishwithhook
    \ex \starn
    \ex \parakeet
    \ex \wildcat \label{a-wildcat}
    \ex \anteater
    \ex \face \label{a-face}
    \ex \eyebrow \label{a-eyebrow}
    \ex \thunder \label{a-thunder}
    \ex \pseudo \label{a-pseudo}
    \ex \tsofa
    \ex \tsofatajf \label{a-tsofatajf}
    \ex \tsofatajt \label{a-tsofatajt}
    \ex \hornero \label{a-hornero}
    \ex \termitehouse
    \ex \guayacan
    \ex \healthy
    \ex \price
    \ex \night \label{a-night}
    \ex \tuscaf \label{a-tuscaf}
    \ex \tuscat \label{a-tuscat}
    \ex \tuscag
    \ex \caracara
    \ex \earth
    \ex \moon
    \ex \woman
    \ex \ask
    \ex \iguana
    \ex \rat
    \ex \jararaca
    \ex \wildhoney
    \ex \knee \label{a-knee}
    \ex \snakeatuj
    \ex \peccary
    \ex \maguari
    \ex \femalebreast
    \ex \mistolf
    \ex \mistolt
    \ex \wildbean \label{a-wildbean}
    \ex \meat
    \ex \mosquito
    \ex \teach
    \ex \bro
    \ex \puma
    \ex \lessergrison
\end{exe}

The very same correspondence is observed in etymologies with a limited distribution (Maká and Nivaĉle, Chorote and Wichí), whose PM~age is thus questionable.

\begin{exe}
    \ex \brightness
    \ex \dreamv
    \ex \ocelot \label{a-ocelot}
    \ex \lizard
    \ex \heavyv
    \ex \tortoise
    \ex \whitealgarrobof
    \ex \cheek
    \ex \pacu
    \ex \smooth
    \ex \majan
    \ex \heel \label{a-heel}
    \ex \orphancw
    \ex \medicine
    \ex \standv
    \ex \belt
    \ex \yellowv
    \ex \tongue
    \ex \cardon
    \ex \vertical
    \ex \precipice
    \ex \throwcw
    \ex \redbrocket
    \ex \wolf \label{a-wolf}
    \ex \metal
    \ex \balawasp
    \ex \cardinal \label{a-cardinal}
    \ex \stagnant
    \ex \fox
    \ex \saber
    \ex \fatalha
    \ex \cord
    \ex \cebil
    \ex \aloja
    \ex \paralytic \label{a-paralytic}
    \ex \yawn \label{a-yawn}
    \ex \doradocw
    \ex \hiccup \label{a-hiccup}
\end{exe}

In a number of stems, all of which are provisionally reconstructed with the vowels \intxt{*a} and \intxt{*å} in two adjacent syllables, a correspondence is found between \sound{Mk}{a…a}, \sound{Ni}{å…å}, \sound{PCh}{*a…o}, and \sound{PW}{*a…o}. In each case there is a labial consonant either between the vowels or before the first of them. In \REF{a-shoulder}--\REF{a-shoulderblade}, Chorote shows \sound{PCh}{*o…o} instead, as in \REF{a-heel} above. In \REF{a-flower}, Nivaĉle has \intxt{a} instead of the expected \intxt{*å}, which is likely due to a sound change whereby \sound{PM}{*å} changed to \sound{Ni}{a} at least in some dialects (\sectref{ni-a-ao-labials}).

\begin{exe}
    \ex \flower \label{a-flower}
    \ex \shoulder \label{a-shoulder}
    \ex \shoulderblade \label{a-shoulderblade}
    \ex \paloflojof
    \ex \paloflojot
    \ex \spring
\end{exe}

\section{\sound{PM}{*å}}\label{pm-ao}
\sound{PM}{*å} is preserved as a low back unrounded vowel (distinct from the low non-back unrounded vowel /a/) in most dialects of Nivaĉle, in Proto-Chorote, and in Proto-Wichí. In Maká, it yields \intxt{a}, but does not merge with \sound{PM}{*a} in most environments because the default reflex of the latter vowel is \sound{Mk}{e}. In the contemporary Chorote varieties, it survives as an underlying segment in Iyojwa’aja’ (which consistently surfaces as \phonetic{a}, whereas underlying /a/ surfaces either as \phonetic{a} or as \phonetic{e}); in other Chorote dialects, it merged with \intxt{*a}. In Southeastern Wichí, it yields \intxt{ɔ} (in the Rivadavia subdialect) or even \intxt{o}, but no merger occurs because \sound{PW}{*o} yields \intxt{u} in the same varieties. In Nivaĉle, \intxt{*å} merges with \intxt{*a} in the Yita’ Lhavos dialect in all environments (\sectref{ni-a-ao-merger}); in other dialects the merger takes place before labial consonants (\sectref{ni-a-ao-labials}). It should be noted that in Wichí \sound{PM}{*å} exceptionally yields \sound{PW}{*a} preceding the coda \intxt{*ˀm}, as in \REF{ao-defecate} and \REF{ao-pronominal}. Irregular reflexes include \sound{Mk}{e} in \REF{ao-eyelash}, \intxt{o} in \REF{ao-carrysh}; \sound{Ni}{a} in \REF{ao-cactus} and \REF{ao-teach}; PCh zero in \REF{ao-rhea}, \intxt{*a} in \REF{ao-bat}, \intxt{*o} in \REF{ao-wildbean}; \sound{PW}{*o} in \REF{ao-powder}, \intxt{*a} in \REF{ao-bitter} and \REF{ao-rat}, zero in \REF{ao-rhea}. In addition, irregular reflexes are apparently found in Maká in \REF{ao-soul} and in Wichí in \REF{ao-argentineboa}, but it is unclear whether the words in question actually belong to the respective cognate sets.

\begin{exe}
    \ex \gofirst
    \ex \goawayi
    \ex \arrive
    \ex \shout
    \ex \stinger
    \ex \cryao
    \ex \returnth
    \ex \burn
    \ex \food
    \ex \son
    \ex \daughter
    \ex \bleedv
    \ex \drinkn
    \ex \fatv
    \ex \flyv
    \ex \suncho
    \ex \welln
    \ex \drinkv
    \ex \jaguar
    \ex \treen
    \ex \water
    \ex \vulture
    \ex \truev
    \ex \testicle
    \ex \tail
    \ex \fall
    \ex \cactus \label{ao-cactus}
    \ex \arrowkaxe
    \ex \youngersis
    \ex \withstand
    \ex \killv
    \ex \snail
    \ex \pet
    \ex \nightmonkey
    \ex \defecate \label{ao-defecate}
    \ex \lightfire
    \ex \sleep
    \ex \goimp
    \ex \powder \label{ao-powder}
    \ex \father
    \ex \rope
    \ex \pathn
    \ex \cavy
    \ex \snore
    \ex \mucus
    \ex \bitter \label{ao-bitter}
    \ex \lip
    \ex \shuck
    \ex \jabiru
    \ex \quick
    \ex \up
    \ex \leg
    \ex \fishwithhook
    \ex \soul \label{ao-soul}
    \ex \vein
    \ex \spank
    \ex \wildcat
    \ex \acquainted
    \ex \sprout
    \ex \dinlaw
    \ex \eyelash \label{ao-eyelash}
    \ex \shoot
    \ex \carrysh \label{ao-carrysh}
    \ex \woodpecker
    \ex \chaja
    \ex \tired
    \ex \badmood
    \ex \rhea \label{ao-rhea}
    \ex \butterfly
    \ex \bat \label{ao-bat}
    \ex \straw
    \ex \rat \label{ao-rat}
    \ex \wildhoney
    \ex \mistolf
    \ex \mistolt
    \ex \hurt
    \ex \chaguara
    \ex \argentineboa \label{ao-argentineboa}
    \ex \wildbean \label{ao-wildbean}
    \ex \waspaniti
    \ex \stepv
    \ex \wildpepper
    \ex \skin
    \ex \teach \label{ao-teach}
    \ex \firei
    \ex \doveula
\end{exe}

The very same correspondence is observed in etymologies with a limited distribution (Maká and Nivaĉle, Chorote and Wichí), whose PM~age is thus questionable.

\begin{exe}
    \ex \pronominal \label{ao-pronominal}
    \ex \wordamet
    \ex \spin
    \ex \jar
    \ex \spillmn
    \ex \pocote
    \ex \tobacco
    \ex \ocelot
    \ex \redv
    \ex \torn
    \ex \grandchild
    \ex \feel
    \ex \willow
    \ex \two
    \ex \frog
    \ex \gutscw
    \ex \cicada
    \ex \siyaj
    \ex \durmili
    \ex \chachalaca
    \ex \toad
    \ex \soundv
    \ex \spillcw
    \ex \piranhamn
    \ex \skymn
    \ex \cloudmn
    \ex \fatalha
    \ex \ashamedcw
    \ex \orphanmn
    \ex \diecw
    \ex \mollef
\end{exe}

In a number of stems, \sound{PM}{*å} yields \intxt{*o} in Chorote and Wichí, a development usually found in the vicinity of a labial consonant or \sound{PM}{*χ}. In the same words, \sound{PM}{*a} in the preceding syllable typically harmonizes to \sound{Mk}{a}, \sound{Ni}{å}.

\begin{exe}
    \ex \flower
    \ex \najendup
    \ex \shoulder
    \ex \shoulderblade
    \ex \abdcavity
    \ex \paloflojof
    \ex \spring
\end{exe}

\section{\sound{PM}{*o}}\label{pm-o}
\sound{PM}{*o} is typically preserved as \intxt{o} in all daughter languages: Maká, Nivaĉle, Proto-Chorote, and Proto-Wichí. Only a few cognate sets show deviant reflexes: \sound{Mk}{u} in \REF{o-throw}, \sound{Ni}{a} in \REF{o-blackalgarrobof}--\REF{o-blackalgarrobot}, \sound{PCh}{*ᵊ} in \REF{o-wildmanioc}.

\begin{exe}
    \ex \coal
    \ex \elbow
    \ex \hand \label{o-hand}
    \ex \roundkoy
    \ex \armadillo
    \ex \bottomn
    \ex \ashes
    \ex \winter
    \ex \iscayante
    \ex \many
    \ex \roast
    \ex \savannahhawk
    \ex \zorzal
    \ex \wildmanioc \label{o-wildmanioc}
    \ex \penis
    \ex \seed
    \ex \deep
    \ex \full
    \ex \fillv
    \ex \cover
    \ex \fence
    \ex \lid
    \ex \durmili
    \ex \face
    \ex \eyebrow
    \ex \snake
    \ex \far
    \ex \uncle
    \ex \aunt
    \ex \tsofa
    \ex \tsofatajf
    \ex \tsofatajt
    \ex \dov
    \ex \worm
    \ex \throwv \label{o-throw}
    \ex \whiteegret
    \ex \cardinal
    \ex \blackalgarrobof \label{o-blackalgarrobof}
    \ex \blackalgarrobot \label{o-blackalgarrobot}
    \ex \expert
    \ex \neck
    \ex \blood
    \ex \butterfly
    \ex \gov
    \ex \palosanto
    \ex \nightnw
    \ex \sandyplace
    \ex \pigeon
    \ex \extinguished
    \ex \bro
    \ex \hiccup
    \ex \chest
\end{exe}

The very same correspondence is observed in etymologies with a limited distribution (Maká and Nivaĉle, Chorote and Wichí), whose PM~age is thus questionable.

\begin{exe}
    \ex \throwpush
    \ex \arrowfok
    \ex \mancw
    \ex \heavyv
    \ex \locustcw
    \ex \hole
    \ex \cheek
    \ex \earcw
    \ex \heel
    \ex \noden
    \ex \utensil
    \ex \redbrocket
    \ex \wolf
    \ex \balawasp
    \ex \leafhaircw
    \ex \fox
    \ex \palocruzmn
    \ex \paralytic
    \ex \yawn
    \ex \ripe
    \ex \queenpalmf
    \ex \heartcw
\end{exe}

\section{\sound{PM}{*u}}\label{pm-u}
\sound{PM}{*u} is typically preserved as \intxt{u} in all daughter languages: Maká, Nivaĉle, Proto-Chorote, and Proto-Wichí. In the Chorote varieties, it may front to \intxt{i} after palatalized consonants, but this sound change must have occurred after the disintegration of Proto-Chorote into dialects (see \sectref{ch-u-i}). Note that the reflexes in \REF{u-beard} in Nivaĉle and Wichí are entirely irregular due to contamination with those of \word{PM}{*\mbox{-}pǻs(\mbox{-}eˀt)}{lip}; the regular reflexes are found in Maká and Chorote. The Wichí reflexes in \REF{u-kingvulture} and \REF{u-puma} are also irregular. In \REF{u-tuscaf}--\REF{u-tuscag}, \sound{PM}{*xu\mbox{-}} is reflected as \sound{PCh}{*ʔi\mbox{-}} and \sound{PW}{*ˣ\mbox{-}}, which could be a regular development in word-initial unstressed syllables. In \REF{u-dayworld}, Chorote has lost the original vowel before what looks like a fossilized vowel-initial suffix.

\begin{exe}
    \ex \lick
    \ex \centipede
    \ex \soninlaw
    \ex \palm
    \ex \fart
    \ex \finger
    \ex \wax
    \ex \sunn
    \ex \answer
    \ex \grabwork
    \ex \eatkun
    \ex \heat
    \ex \meet
    \ex \sweat
    \ex \hornclub
    \ex \coldn
    \ex \barnowl
    \ex \thorncutjan
    \ex \bow
    \ex \yicalhuk
    \ex \daylhuma
    \ex \girl
    \ex \feces
    \ex \bonenu
    \ex \dog
    \ex \dayworld \label{u-dayworld}
    \ex \tapeti
    \ex \beard \label{u-beard}
    \ex \distrust
    \ex \kingvulture \label{u-kingvulture}
    \ex \vagina
    \ex \anteater
    \ex \likelove
    \ex \cat
    \ex \blind
    \ex \ant
    \ex \yellowlegs
    \ex \eatvt
    \ex \hardv
    \ex \duraznillo
    \ex \leniosa
    \ex \tired
    \ex \nest
    \ex \termitehouse
    \ex \passv
    \ex \egg
    \ex \tuscaf \label{u-tuscaf}
    \ex \tuscat \label{u-tuscat}
    \ex \tuscag \label{u-tuscag}
    \ex \grass
    \ex \firewoodhuk
    \ex \pushv
    \ex \woman
    \ex \iguana
    \ex \knee
    \ex \snakeatuj
    \ex \doveula
    \ex \urinate
    \ex \urine
    \ex \puma \label{u-puma}
\end{exe}

The very same correspondence is observed in etymologies with a limited distribution (Maká and Nivaĉle, Chorote and Wichí), whose PM~age is thus questionable.

\begin{exe}
    \ex \spitcw
    \ex \hunger
    \ex \runv
    \ex \chaguark
    \ex \pacu
    \ex \skycloud
    \ex \drum
    \ex \yellowv
    \ex \dovesipup
    \ex \smoke
    \ex \burnvi
    \ex \suckcw
    \ex \burnvt
    \ex \wut
    \ex \clothes
    \ex \temperance
    \ex \caracara
    \ex \eel
\end{exe}

\section{Insufficient evidence for reconstruction of a specific vowel}\label{vowels-insufficient}
\largerpage
Some etymologies have a limited distribution (Maká and Nivaĉle, Chorote and Wichí), and their PM~age is thus questionable.
For cognate sets that involve the correspondence between \sound{Mk}{e} and \sound{Ni}{a} with no cognates in Chorote and Wichí, it may not be possible to distinguish between \sound{PM}{*a} and \intxt{*ä}.

\begin{exe}
    \ex \burnalh
    \ex \foodmn
    \ex \locustmn
    \ex \ameiva
    \ex \obey
    \ex \smooth
    \ex \agile
    \ex \hookmn
    \ex \dwarf
    \ex \leafmn
    \ex \utensil
    \ex \vertical
    \ex \tsaqaq
    \ex \tireddie
    \ex \ashamedmn
    \ex \mollef
    \ex \queenpalmf
\end{exe}

For cognate sets that involve the correspondence between \sound{PCh}{*e} and \sound{PW}{*e} with no cognates in Maká and Nivaĉle, it may not be possible to distinguish between \sound{PM}{*e} and \intxt{*ä}.

\begin{exe}
    \ex \inhabitant
    \ex \whitealgarrobof
    \ex \frighten
    \ex \cook
    \ex \orphancw
    \ex \bilecw
    \ex \precipice
    \ex \temperance
    \ex \cebil
    \ex \aloja
    \ex \ashamedcw
\end{exe}

For \intxt{*χ}\mbox{-}final stems that lack a known reflex in Nivaĉle and whose vocalic stem is not recoverable, it is impossible to distinguish between \sound{PM}{*a} and \intxt{*e} (and even \intxt{*å}, if no Wichí cognate is available), because all these vowels merge before a uvular fricative as \sound{Maká}{a}, \sound{Chorote}{a}, and \sound{Wichí}{a} (\sound{PM}{*å} remains distinct in Wichí, however).

\begin{exe}
    \ex \dividev
    \ex \chaguark
    \ex \sandisaj
\end{exe}

Finally, a divergent correspondence occurs in two examples, where \sound{Ni}{å} corresponds to \sound{PCh}{*u} and \sound{PW}{*u} following a \sound{PM}{*(ˀ)w} (only one of these cognate sets has a reflex in Maká, where \intxt{e} is found). It is unclear as of yet which vowel should be reconstructed to Proto-Mataguayan in these two cases.

\begin{exe}
    \ex \largefat
    \ex \climb
\end{exe}
