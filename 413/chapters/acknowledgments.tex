\addchap{\lsAcknowledgementTitle} 

The authors wish to thank the following individuals for making this study possible.

First of all, we are indebted to all Chorote and Manjui who patiently taught Javier Carol their language(s) in the communities of Misión Chorote I, Parcela 42, Lapacho I, Misión La Paz, La Merced Nueva, La Estrella, La Bolsa, Santa Rosa/Wonta, Abizai, and San Eugenio–San Agustín, and particularly to Víctor Díaz (“Pelayo”), Héctor Sarmiento, Julián Gómez, Nicasio Carrizo, Juan and Claudio González, Sebastián Frías, Roberto Valentín, the Palmas, Artín and Franco Bravo, Gustavo González, Pablo Segundo, Juan Paredes, Tomás Vera, César Pérez, Carlino Álvarez (Neiwi’), Florencio Vázquez, José López, Rogelio López, Maycol Saldívar, Silverio García, Aurelia Leguizamón, Víctor Fermín, Clarita Martínez, Lidia Martínez, and Fanny Díaz.

We are also grateful to Analía Gutiérrez for productive conversations on the phonology of Nivaĉle and for her generosity with linguistic data.

We thank Andrés Salanova for introducing the authors one to each other. Andrey Nikulin is further grateful to him for providing excellent working conditions during his postdoctoral fellowship at the University of Ottawa, with financial support from the Social Sciences and Humanities Research Council (SSHRC) by means of an Insight grant (\#435-2018-1173, Principal Investigator: Andrés Pablo Salanova).

We are further thankful to all individuals who shared relevant literature with us: Ana Fernández Garay, Analía Gutiérrez, Andrés Salanova, Cristina Messineo, Eric Hunt, Lyle Campbell, Micaela Gaggero Fiscella, Nicholas Drayson, pa’i Nilo Damián Zárate López, Paola Cúneo, J.~Pedro Viegas Barros, Silvia Spinelli, and Temis Tacconi.

The main results of our study were presented at a seminar held by the Institute of Linguistics of the Faculty of Philosophy and Letters at the University of Buenos Aires (\intxt{Instituto de Lingüística de la Facultad de Filosofía y Letras de la Universidad de Buenos Aires}) on May 15, 2023. We are grateful to the audience for the fruitful discussion.

We thank the editors of the Topics in Phonological Diversity series -- Natalia Kuznetsova, Cormac Anderson, Shelece Easterday -- and the technical staff of Language Science Press for making this publication possible, and are particularly indebted to the reviewers of this volume for their careful reading of different versions of the manuscript and for their insightful comments. The community proofreaders have helped us get rid of many typos and inconsistencies.

Needless to say, all remaining errors are our own.