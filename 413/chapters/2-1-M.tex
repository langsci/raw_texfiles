\chapter{Maká} \label{mk}

This chapter deals with the historical phonology of Maká [maca1260] (\sectref{intro-mk}), including the development of its consonants (\sectref{mk-cons}), vowels (\sectref{mk-vow}), and prosody (\sectref{mk-prosody}) from the PM stage to Maká.

In what follows, we rely on \cits{AG94} grammatical description (which incorporates most of her \citeyear{AG89} findings) and on \cits{AG99} dictionary. However, these sources do not faithfully represent the glottalized sonorants and the preglottalized codas; for these sounds, we rely on Wycliffe’s Bible translations, on \cits{JB81} work, and on recently published materials in Maká \citep{unuuneiki,maka-etnomat,PMA}.

The consonantal inventory we assume for Maká is given in \tabref{mk-inv-cons}. The status of the ejective fricatives is dubious; they have been alternatively analyzed as sequences of plain fricatives and a glottal stop \citep{AG94}. Note that we apply \cits{AnG15} analysis of the Nivaĉle preglottalized codas as complex codas to the Maká preglottalized codas, and do not posit a set of preglottalized stops and fricatives; therefore, \word{Maká}{feˀt}{fire} is analyzed as /feʔt/. The vocalic inventory we assume for Maká includes only five vowels, /i~e~a~o~u/. 

\begin{table}
\caption{Maká consonants}
\label{mk-inv-cons}
\fittable{
 \begin{tabular}{rcccccc}
  \lsptoprule
            & labial & dental & alveolar & velar & uvular & glottal\\\midrule
  plain stops & p & t & ts & k & q & ʔ\\
  ejective stops & p’ & t’ & ts’ & k’ & q’ & \\
  plain fricatives & f & ɬ & s & x & χ & h\\
  (ejective fricatives) & (f’) & (ɬ’) & (s’) & (x’) & & \\
  plain approximants & w & l & \multicolumn{2}{c}{j} & &\\
  glottalized approximants & ˀw & ˀl & \multicolumn{2}{c}{ˀj} & &\\
  plain nasals & m & n & & & &\\
  glottalized nasals & ˀm & ˀn & & & &\\
  \lspbottomrule
 \end{tabular}
 }
\end{table}

\section{Consonants}\label{mk-cons}
Maká is conservative in that it has retained most Proto-Mataguayan consonants intact.

\subsection{\sound{PM}{*ɸ}}\label{mk-f}

One minor (and unconditioned) sound change has transformed \sound{PM}{*ɸ}, reconstructed as a bilabial fricative, into \sound{Mk}{f}, explicitly stated to be articulated as labiodental by \citet[30]{AG89}. For examples, see \sectref{proto-f}.

In the variety of Maká attested by \citet[456]{AD60} under the name `Lengua', the sound in question is mostly represented as ‹fu›, as in ‹fuêté› `fire', ‹hiafué› `teeth', ‹hicfué› `ear' (modern Maká \intxt{feˀt}, ---, \intxt{ji\mbox{-}kfiʔ}), suggesting that it was articulated as [ɸ] or [fʷ] in that variety.

\subsection{Loss of the word-initial glottal stop}\label{mk-onsetless}

Another innovation is the loss of the word-initial glottal stop, which was not contrastive in that position in Proto-Mataguayan in any case (it is reconstructed as an epenthetic segment inserted before words that would otherwise begin with a vowel). \citet[26–27, 49]{AG89} is not explicit on whether word-initial \intxt{ʔ} actually contrasts with zero in Maká synchronically: although she documents forms such as \textit{ʔaftil} `you are orphan', in the vast majority of cases word-initial (non-phonemic) glottal stops of other Mataguayan languages correspond to zero in Maká.

\subsection{\wordng{PM}{*h}}\label{mk-h}

The glottal fricative \sound{PM}{*h} has been lost word-finally in Maká, and \intxt{h} no longer occurs in that position synchronically \citep[34]{AG89}. This includes \sound{PM}{*jʰ}, \intxt{*lʰ}.

\begin{exe}
    \ex \plaj
    \ex \distal
    \ex \companion \label{mk-h-companion}
    \ex \neighbor \label{mk-h-neighbor}
    \ex \goimp
    \ex \soul
    \ex \hornero
    \ex \recipient
    \ex \lessergrison
\end{exe}

\subsection{\wordng{PM}{*ji}}\label{mk-ji}

The sequence \sound{PM}{*ji} is reflected as \intxt{ji} or \intxt{i} in Maká, with no clear distribution. \citet[36–37]{AG89} states that the sequence /ji/ surfaces as [ʝi] in Maká.

\begin{exe}
    \ex \dew
    \ex \hunger
    \ex \truev
    \ex \ocelot
\end{exe}

The third-person active prefix (\wordng{PM}{*ji\mbox{-}}) is also variably reflected as \intxt{ji\mbox{-}} or \intxt{i\mbox{-}} in Maká: \wordnl{ji\mbox{-}lan}{kills}, \wordnl{ji\mbox{-}liˀx\mbox{-}xuʔ}{cleans}, \wordnl{ji\mbox{-}nxiˀwen}{smells}, \wordnl{ji\mbox{-}piˀjeʔ}{hears}, \wordnl{ji\mbox{-}suʔun}{loves}, \wordnl{ji\mbox{-}tiɬ}{sews}, \wordnl{ji\mbox{-}ˀwen}{sees}, \wordnl{ji\mbox{-}t’ix}{says}, \wordnl{ji\mbox{-}wef}{is tired}, but \wordnl{i\mbox{-}maʔ}{sleeps}, \wordnl{i\mbox{-}wuˀm}{pushes, throws}, \wordnl{i\mbox{-}k}{goes}, \wordnl{i\mbox{-}p}{cries}.

\subsection{Destiny of glottalized sonorants}\label{mk-deglottalization-sonorants}
Although our main sources on Maká (\citnp{AG89}, \citeyear{AG94}, \citeyear{AG99}) do not attest any traces of glottalization in sonorants, more recent publications suggest that Maká has actually preserved the preglottalized sonorant onsets of PM, at least word-internally. These are spelt as ‹'w›, ‹'l›, ‹'y›, ‹'m›, ‹'n› in Wycliffe's Bible translations, in \citet{unuuneiki}, and in \citet{maka-etnomat,PMA}. Some examples follow.

\begin{exe}
    \ex \languagewordpl
    \ex \defect
    \ex \smellv
    \ex \hear
    \ex \spousewh
    \ex \marry
    \ex \seev
    \ex \placen
    \ex \rib
\end{exe}

Word-initially, however, glottalized sonorants are not attested. We surmise that PM glottalized sonorants underwent deglottalization in that environment.

\begin{exe}
    \ex \dayworld
    \ex \rhea
    \ex \climb
\end{exe}

\subsection{Destiny of preglottalized codas}\label{mk-deglottalization-codas}
Although our main sources on Maká (\citnp{AG89}, \citeyear{AG94}, \citeyear{AG99}) do not attest any traces of glottalization in codas, more recent publications suggest that Maká has actually preserved most preglottalized codas of PM with no modifications. In Wycliffe's Bible translations, in \citet{unuuneiki}, and in \citet{maka-etnomat,PMA} codas spelt as ‹'C› (in the practical orthography) occur abundantly precisely in words whose PM~etyma are reconstructed with a glottalized coda; some examples are given below.

\begin{exe}
    \ex \burnalh
    \ex \stingerits
    \ex \sonits
    \ex \bite
    \ex \rightn
    \ex \firef
    \ex \sandisaj
    \ex \takeaway
    \ex \answer
    \ex \runv
    \ex \grabwork
    \ex \dividev
    \ex \winter
    \ex \defecate
    \ex \abdcavity
    \ex \suckb
    \ex \up
    \ex \sprout
    \ex \throwv
    \ex \climb
    \ex \night
    \ex \spring
    \ex \meatitssg
\end{exe}

There are also a few exceptions.

\begin{exe}
    \ex \rootn
    \ex \palm
    \ex \spinsew
    \ex \nest
    \ex \badmood
    \ex \paralytic
\end{exe}

The coda \intxt{*\mbox{-}ˀj} is reflected as \intxt{j} in Maká.

\begin{exe}
    \ex \weave
    \ex \soundv
    \ex \marry
\end{exe}

\subsection{Glottal insertion in monosyllables}\label{mk-glottal-insertion}

In some cases, word-final glottal stops in Maká appear not to reconstruct to Proto-Mataguayan, as evidenced by the Lower Bermejeño Wichí cognates (where no glottal stop is found). We suggest that Maká underwent \intxt{ʔ}\mbox{-}epenthesis in roots of the shape \intxt{(C)V} (shared with Nivaĉle, see \sectref{ni-glottal-insertion}).

\begin{exe}
    \ex \thorne
    \ex \hornclub
    \ex \sleep
    \ex \juice
\end{exe}

\subsection{Fricative + \textit{*χ}}\label{mk-fricative-dorsal}
In Maká, Proto-Mataguayan clusters of the shape ``fricative + \intxt{*χ}" have lost the uvular fricative.

\begin{exe}
    \ex \finger
    \ex \redquebracho
    \ex \thunder
    \ex \meat
\end{exe}

As a result, clusters such as \intxt{fχ}, \intxt{ɬχ}, \intxt{sχ}, \intxt{xχ}, \intxt{χχ} are synchronically illicit in Maká \citep[60–61]{AG89}.

\subsection{Other consonant clusters}\label{mk-clusters}

Word-initially, the following consonant clusters are synchronically licit in Maká: \intxt{ph}, \intxt{tsx}, \intxt{tsh}, \intxt{kh}, \intxt{qh}, \intxt{k’w}, \intxt{hw}, \intxt{ɬw} \citep[58]{AG89}. Other consonant clusters reconstructed for PM have been mostly resolved by means of an epenthetic \intxt{i}. We have identified examples involving \sound{PM}{*ɸts}, \intxt{*ˀnj}, \intxt{*nx}, \intxt{*st}, and \intxt{*xn}.

\begin{exe}
    \ex \palm
    \ex \sleepiness\footnote{Synchronically, \wordng{Mk}{-nixatiʔ} is a relational stem, meaning that the sequence \intxt{-nix-} is in fact found in word-medial position in this noun. The epenthesis of \intxt{i} must thus have occurred at a stage when \intxt{-nixatiʔ} was still an absolute stem, as are its cognates in other Mataguayan languages.}
    \ex \cavy
    \ex \whitequebracho
    \ex \spring
\end{exe}

Maká also employs \intxt{e}\mbox{-}epenthesis to resolve stem-initial clusters whose first member is a non-nasal sonorant.

\begin{exe}
    \ex \flu
    \ex \squash
    \ex \rib
\end{exe}

Word-internally, many more clusters are allowed \citep[59–63]{AG89}. Nevertheless, there are several gaps, and some of them likely result from sound changes specific to certain clusters, such as \sound{PM}{*lʔ} > \sound{Mk}{l}, \sound{PM}{*sˀw} > \sound{Mk}{sVʔ}, and \sound{PM}{*(ˀ)wt} > \sound{Mk}{t}. Most of these PM~clusters are reconstructed based on evidence from Nivaĉle.

\begin{exe}
    \ex \soninlaw
    \ex \likelove
    \ex \heartmn
\end{exe}

At least one of these changes -- \sound{PM}{*(ˀ)wt} > \sound{Mk}{t} -- has resulted in a synchronically active alternation in Maká, whereby the syncopated allomorph of the reflexive prefix \intxt{\mbox{-}wet\mbox{-}} is \intxt{\mbox{-}t\mbox{-}} rather than \intxt{*\mbox{-}wt\mbox{-}} \citep[114]{AG94}, as shown in \REF{ex:wet:1:mk}.

\ea\label{ex:wet:1:mk}
    \begin{xlist}
        \ex\gll ∅-wet-xili-nen-ɬe\\
                3-\REFL-dirty-\CAUS-\REFL\\
                \glt `s/he soils herself/himself'
        \ex\gll ɬe-wet-xili-nen-ɬe~\recind ɬe-t-xili-nen-ɬe\\
                2.{\textsc{act}}-\REFL-dirty-\CAUS-\REFL\\
                \glt `you soil yourself'
    \end{xlist}
\z

In some cognate sets, \intxt{*mt} and \intxt{*mq} appear to have yielded \intxt{nt} and \intxt{nq} in Maká. It is uncertain whether this sound change is regular, as the sequences \intxt{mt} and \intxt{mq} are synchronically licit in Maká, as in \wordnl{somtaχ}{kind of fruit\species{Harrisia bonplandii}}, \wordnl{jamqaχ}{buff-necked ibis\species{Theristicus caudatus}}. However, words that contain them tend to lack a known Mataguayan etymology.

\begin{exe}
    \ex \samto
    \ex \samtok
    \ex \wamqa
\end{exe}

\subsection{Syllabic consonants}\label{mk-syll-c}

In Maká, the syllabic consonants of Proto-Mataguayan evolve in the same way as the syllables of the structure \intxt{*Ca} or \intxt{*Cä}: they yield \intxt{Ce}, with the vowel harmonizing to \intxt{a} or \intxt{o} if the next syllable contains a low vowel. This includes the third-person possessive and the second-person active realis prefixes (\wordng{PM}{*ɬ̩\mbox{-}} before consonants), the third-person active irrealis prefix (\wordng{PM}{*n̩\mbox{-}} before consonants), and the third-person T-class realis prefix (\wordng{PM}{*t̩\mbox{-}} before consonants).

\ea
    Maká \citep[85, 148]{AG94}
    \begin{xlist}
        \ex\gll ɬe-k’inix\\
                3.{\textsc{poss}}-younger\_brother\\
                \glt `his/her younger brother'
        \ex\gll ɬe-fejejkiʔ\\
                2.{\textsc{act}}-rotate\\
                \glt `you rotate'
        \ex\gll ne-t-fejejkiʔ\\
                3.{\textsc{act.irr}}-3\textsubscript{T}-sleep\\
                \glt `(that) s/he rotate'
        \ex\gll te-fejejkiʔ\\
                3\textsubscript{T}-rotate\\
                \glt `s/he rotates'
    \end{xlist}
\z

\section{Vowels}\label{mk-vow}
\subsection{Maká vowel shift}\label{mk-vowel-shift}
A notable sound change involving vowels in Maká is the vowel shift, whereby \sound{PM}{*e} changed to \sound{Mk}{i} (thus merging with \sound{PM}{*i} > \sound{Mk}{i}), \sound{PM}{*a} and \intxt{*ä} changed to \sound{Mk}{e}, and \sound{PM}{*å} changed to \sound{Mk}{a} in most positions.

This shift must have occurred at a relatively late date, since earlier registers of Maká (co-)dialects often show ‹a› and ‹e› where contemporary Maká has ‹e› and ‹i›, respectively. In the following examples, forms marked as ``Towothli'' are from Barbrooke Grubb’s data collected in 1913 (cited {\textit{apud}} \citnp{RJH15}); those marked as ``Enimagá'', ``Guentusé'', and ``Lengua'' are from \citet{JFA93} (cited {\textit{apud}} \citnp{EP98}).

\newpage
\booltrue{listing}
\begin{exe}
    \ex Towothli ‹hual› > \word{modern Maká}{xuwel}{moon}
    \ex Towothli ‹sahat› > \word{modern Maká}{sehets}{fish}
    \ex Guentusé ‹sèehà›, Lengua ‹saha›, Towothli ‹saha› > \word{modern Maká}{seheʔ}{earth}
    \ex Towothli ‹hutan› > \word{modern Maká}{h-uten}{I hate}
    \ex Towothli ‹wotak› > \word{modern Maká}{wote-k}{achiote tree}
    \ex Enimagá ‹egualé›, Lengua ‹gualé›, Towothli ‹iwali› > \word{modern Maká}{iweliʔ}{water}
    \ex Towothli ‹witlapinak› > \word{modern Maká}{wit-lepin-ek}{salt}
    \ex Towothli ‹hekŏf› > \word{modern Maká}{xikaf}{fan}
    \ex Towothli ‹selel› > \word{modern Maká}{ts’ilil}{bee sp.}
    \ex Towothli ‹peno› > \word{modern Maká}{pinuʔ}{bee sp.}
    \ex Towothli ‹oita› > \word{modern Maká}{ute}{stone}
\end{exe}
\boolfalse{listing}

\subsubsection{\sound{PM}{*e}, \intxt{*i} > \sound{Mk}{i}} \label{mk-e-i}

The following examples show that \sound{PM}{*e} changed to \sound{Mk}{i}, except before the uvular fricative \intxt{*χ} (see \sectref{mk-uvul-retr} on the vowel development before \intxt{*χ}).

\begin{exe}
    \ex \daughter
    \ex \thorne
    \ex \namen
    \ex \distal
    \ex \bite
    \ex \mortar
    \ex \rootn
    \ex \welln
    \ex \sendv
    \ex \feminine
    \ex \earkfe
    \ex \arrowkaxe
    \ex \pacu
    \ex \chaniart
    \ex \offspring
    \ex \wash
    \ex \squash
    \ex \firewoodlhet
    \ex \heartmn
    \ex \otter
    \ex \cavy
    \ex \hear
    \ex \rain
    \ex \beard
    \ex \spank
    \ex \whitequebracho
    \ex \eyelash
    \ex \eyebrow
    \ex \tears
    \ex \saymn
    \ex \rheum
    \ex \ashamedmn
    \ex \cloudmn
    \ex \walk
    \ex \onemn
    \ex \dirt
    \ex \orphanmn
    \ex \teach
\end{exe}

The only instance of an irregular reflex is given below.

\begin{exe}
    \ex \bat \label{mk-e-bat}
\end{exe}

For examples of \sound{PM}{*i} being retained as \sound{Mk}{i}, see \sectref{pm-i}.

\subsubsection{\sound{PM}{*a}, \intxt{*ä} > \sound{Mk}{e}} \label{mk-a-ae}

Both \sound{PM}{*a} and \intxt{*ä} normally changed to \sound{Mk}{e} (except before the uvular fricative \intxt{*χ}, for which see \sectref{mk-uvul-retr}, and before syllables that contain \sound{Mk}{a} or \intxt{o}, on which see \sectref{mk-vh}). Note that these two phonemes also merged in Nivaĉle (\sectref{ni-vowels}). The following examples show the development of \sound{PM}{*a}.

\begin{exe}
    \ex \plaj
    \ex \mouth
    \ex \fruit
    \ex \bite
    \ex \companion
    \ex \rightn
    \ex \disease
    \ex \firef
    \ex \cutdown
    \ex \dew
    \ex \grove
    \ex \redquebracho
    \ex \neighbor
    \ex \pacu
    \ex \smooth
    \ex \louse
    \ex \interr
    \ex \defect
    \ex \nose
    \ex \dayworld
    \ex \rain
    \ex \inorderto
    \ex \fishwithhook
    \ex \vertical
    \ex \thunder
    \ex \tsofatajt
    \ex \guayacan
    \ex \tuscaf
    \ex \tuscat
    \ex \tuscag
    \ex \woman
    \ex \maguari
    \ex \wildbean
    \ex \meat
    \ex \mosquito
    \ex \teach
    \ex \lessergrison
\end{exe}

Only two examples instantiate what seems to be an irregular reflex of \sound{PM}{*a} in Maká: \intxt{a} in \REF{mk-a-newadj} and \intxt{i} in \REF{mk-a-hornero}.

\begin{exe}
    \ex \newadj \label{mk-a-newadj}
    \ex \hornero \label{mk-a-hornero}
\end{exe}

The following examples show the development of \sound{PM}{*ä}.

\begin{exe}
    \ex \wing
    \ex \goawaycisl
    \ex \putv
    \ex \tell
    \ex \sisinlaw
    \ex \soninlaw
    \ex \rootn
    \ex \stretchout
    \ex \dividev
    \ex \chaniart
    \ex \flu
    \ex \hither
    \ex \smellv
    \ex \abdcavity
    \ex \cat
    \ex \allrcpr
    \ex \walk
    \ex \seev
    \ex \placen
    \ex \eatvi
    \ex \queenpalmf
    \ex \meat
\end{exe}

Finally, in the following examples in absence of diagnostic cognates from Chorote and Wichí it is impossible to decide between the reconstruction of \sound{PM}{*a} or \intxt{*ä}.

\begin{exe}
    \ex \burnalh
    \ex \foodmn
    \ex \locustmn
    \ex \ameiva
    \ex \obey
    \ex \smooth
    \ex \agile
    \ex \hookmn
    \ex \dwarf
    \ex \leafmn
    \ex \vertical
    \ex \tsaqaq
    \ex \tireddie
    \ex \mollef
\end{exe}

\subsubsection{\sound{PM}{*å} > \sound{Mk}{a}} \label{mk-ao-a}

The following examples show that \sound{PM}{*å} changed to \sound{Mk}{a}, with very few exceptions.

\begin{exe}
    \ex \spin
    \ex \gofirst
    \ex \goawayi
    \ex \arrive
    \ex \shout
    \ex \stinger
    \ex \cryao
    \ex \returnth
    \ex \food
    \ex \son
    \ex \daughter
    \ex \bleedv
    \ex \spillmn
    \ex \tobacco
    \ex \welln
    \ex \drinkv
    \ex \truev
    \ex \ocelot
    \ex \cactus
    \ex \arrowkaxe
    \ex \youngersis
    \ex \killv
    \ex \willow
    \ex \defecate
    \ex \lightfire
    \ex \sleep
    \ex \goimp
    \ex \rope
    \ex \cavy
    \ex \lip
    \ex \up
    \ex \fishwithhook
    \ex \cicada
    \ex \vein
    \ex \spank
    \ex \siyaj
    \ex \sprout
    \ex \soundv
    \ex \shoot
    \ex \woodpecker
    \ex \chaja
    \ex \tired
    \ex \badmood
    \ex \piranhamn
    \ex \skymn
    \ex \cloudmn
    \ex \rhea
    \ex \bat
    \ex \orphanmn
    \ex \mollef
    \ex \hurt
    \ex \wildbean
    \ex \wildpepper
    \ex \skin
    \ex \teach
\end{exe}

Only three examples instantiate what seems to be an irregular reflex of \sound{PM}{*å} in Maká: \intxt{iˀn} in \REF{mk-ao-soul}, \intxt{e} in \REF{mk-ao-eyelash}, and \intxt{o} in \REF{mk-ao-carrysh}.

\begin{exe}
    \ex \soul \label{mk-ao-soul}
    \ex \eyelash \label{mk-ao-eyelash}
    \ex \carrysh \label{mk-ao-carrysh}
\end{exe}

\subsubsection{Pre-uvular lowering} \label{mk-uvul-retr}

Before the PM uvular fricative \intxt{PM}{*χ}, certain Proto-Mataguayan vowels -- at least \sound{PM}{*a} and \intxt{*e}, but possibly also \intxt{*ä} -- have distinct reflexes in Maká.

When \sound{PM}{*χ} is adjacent to the target vowel, \sound{PM}{*a} and \intxt{*e} merge as \intxt{a}. The development \sound{PM}{*aχ} > \sound{Mk}{aχ} is shown below.

\begin{exe}
    \ex \ocelot
    \ex \runv
    \ex \oldn
    \ex \pseudo
    \ex \night
    \ex \tsofatajf
    \ex \piranhamn
    \ex \tuscaf
\end{exe}

The following examples show that \sound{PM}{*eχ} also changes to \sound{Mk}{aχ}.

\begin{exe}
    \ex \blackalgarrobof
    \ex \hurt
    \ex \wildbean
    \ex \mollef
\end{exe}

In the following example, it is impossible to rule out the reconstruction of \sound{PM}{*aχ} or \sound{PM}{*eχ}.

\begin{exe}
    \ex \dividev
\end{exe}

If a consonant intervenes between the target vowel and \sound{PM}{*χ}, \intxt{*e} isʃreflected as \sound{Mk}{e} rather than \intxt{i} or \intxt{a}.

\begin{exe}
    \ex \redquebracho
\end{exe}

The lowering induced by the uvular fricative left behind a number of synchronically active alternations in Maká. In forms that go back to PM~etyma with \intxt{*eχ} or \intxt{*aχ}, the lowering applies, and one finds \sound{Mk}{aχ}. By contrast, the reflexes of PM~forms derived from the vocalic stems of the same etyma (see \sectref{jj-suff}) show no lowering, because \sound{PM}{*χ} was absent in the respective protoforms. Consequently, one finds \sound{Mk}{i} and \intxt{e}. Some examples are given in \REF{mk-uvullow}.

\booltrue{listing}
\ea\label{mk-uvullow}
        Maká \citep[121, 130, 183, 361]{AG99}
    \begin{xlist}
        \ex \intxt{anhejaχ}\gloss{wild bean} → \intxt{anheji-ˀp}\gloss{wild bean season}
        \ex \intxt{aʔtaχ}\gloss{it hurts} → \intxt{aʔti-ts}\gloss{they hurt}
        \ex \intxt{i-f’ilxetsaχ}\gloss{poor.\SG} → \intxt{i-f’ilxetsi-ts}\gloss{poor.\PL}
        \ex \intxt{wanaˀχ}\gloss{piranha} → \intxt{wanhe-ts}\gloss{piranhas}
        \ex \intxt{xaja-taχ}\gloss{western mastiff bat} → \intxt{xaja-te-ts}\gloss{western mastiff bats}
    \end{xlist}
\z
\boolfalse{listing}

Note that the lowering does not apply before the uvular stop \intxt{*q}, as the following example shows. 

\begin{exe}
    \ex \tsaqaq
\end{exe}

The sound change described in this subsection is thus unrelated to the process whereby \intxt{i} is lowered to \intxt{e} (or \intxt{a}, \intxt{o} as per vowel harmony) before the uvular stop \intxt{q} in Maká, as in the first-person singular possessive prefix \intxt{ji\mbox{-}} and in the homophonous third-person active realis prefix, seen in \wordnl{je\mbox{-}qekxiʔ}{my calf}, \wordnl{ja\mbox{-}q'astaliʔ}{my saliva}, \wordnl{jo\mbox{-}qofol}{my nail}, \wordnl{je\mbox{-}qekuʔ}{s/he doubts} \citep{AG94}.

\subsubsection{Vowel harmony}\label{mk-vh}

Above (\sectref{mk-a-ae}) we have seen that \sound{PM}{*a} and \intxt{*ä} have \sound{Mk}{e} as their default reflex. However, a special reflex is found when the following syllable contains one of \intxt{a} or \intxt{o}: in that case, \sound{PM}{*a} (and possibly \intxt{*ä}) harmonize to \sound{Mk}{a} or \intxt{o}, respectively, as the following examples show.

\begin{exe}
    \ex \armadillo
    \ex \alienable
    \ex \utensil
    \ex \wolf
    \ex \spring
    \ex \paralytic
    \ex \yawn
\end{exe}

This sound change gave rise to a synchronically active alternation in Maká whereby \intxt{e} alternates with \intxt{a} and \intxt{o} whenever a low vowel follows in the next syllable \citep[106–108]{AG-BG03}. This alternation affects prefixes that contain the vowel \intxt{e} < \sound{PM}{*a/*ä}, as is the case with the indirect possession prefix \intxt{qa\mbox{-}} \REF{ex-mk-qa-vh} and the second-person possessive prefix \intxt{a\mbox{-}} \REF{ex-mk-a-vh}. In addition, it affects prefixes that are reconstructed as syllabic consonants in Proto-Mataguayan. This includes the third-person possessive and the second-person active realis prefixes (\wordng{PM}{*ɬ̩\mbox{-}} before consonants), the third-person active irrealis prefix (\wordng{PM}{*n̩\mbox{-}} before consonants), and the third-person T-class realis prefix (\wordng{PM}{*t̩\mbox{-}} before consonants), whose Maká reflexes are \intxt{ɬe\mbox{-} / ɬa\mbox{-} / ɬo\mbox{-}} \REF{ex-mk-lha-vh}, \intxt{ne\mbox{-} / na\mbox{-} / no\mbox{-}} \REF{ex-mk-na-vh}, \intxt{te\mbox{-} / ta\mbox{-} / to\mbox{-}} \REF{ex-mk-ta-vh}.

\ea\label{ex-mk-qa-vh}
    Maká \citep[240]{AG-BG03,AG99}
    \begin{xlist}
        \ex\gll ɬe-qe-neneˀk\\
                3.{\textsc{poss-alz}}-spoon\\
                \glt `his/her spoon'
        \ex\gll in-qo-kojojoj\\
                1+2.{\textsc{poss-alz}}-car\\
                \glt `our car'
        \ex\gll ja-qa-lasxixu\\
                1{\textsc{sg.poss-alz}}-poncho\\
                \glt `my poncho'
    \end{xlist}
\z

\ea\label{ex-mk-a-vh}
    Maká \citep[107]{AG-BG03}
    \begin{xlist}
        \ex\gll e-kumkenet\\
                2.{\textsc{poss}}-thigh\\
                \glt `your thigh'
        \ex\gll a-qaweχ\\
                2.{\textsc{poss}}-throat\\
                \glt `your throat'
        \ex\gll o-nokiʔ\\
                2.{\textsc{poss}}-elbow\\
                \glt `your elbow'
    \end{xlist}
\z

\ea\label{ex-mk-lha-vh}
    Maká \citep[85, 88, 148]{AG94}
    \begin{xlist}
        \ex\gll ɬe-k’inix\\
                3.{\textsc{poss}}-younger\_brother\\
                \glt `his/her younger brother'
        \ex\gll ɬo-nokiʔ\\
                3.{\textsc{poss}}-elbow\\
                \glt `his/her elbow'
        \ex\gll ɬe-fejejkiʔ\\
                2.{\textsc{act}}-rotate\\
                \glt `you rotate'
        \ex\gll ɬa-maʔ\\
                2.{\textsc{act}}-sleep\\
                \glt `you sleep'
    \end{xlist}
\z

\ea\label{ex-mk-na-vh}
    Maká \citep[85, 88]{AG94}
    \begin{xlist}
        \ex\gll ne-n-ek\\
                3.{\textsc{act.irr-cisl}}-go\\
                \glt `s/he comes'
        \ex\gll no-t-otoj\\
                3.{\textsc{act.irr}}-3\textsubscript{T}-dance\\
                \glt `(that) s/he dance'
        \ex\gll na-maʔ\\
                3.{\textsc{act.irr}}-sleep\\
                \glt `(that) s/he sleep'
    \end{xlist}
\z

\ea\label{ex-mk-ta-vh}
    Maká \citep[106]{AG-BG03}
    \begin{xlist}
        \ex\gll te-fejejkiʔ\\
                3\textsubscript{T}-rotate\\
                \glt `s/he rotates'
        \ex\gll to-foχij-kij\\
                3\textsubscript{T}-play\_flute-{\textsc{antp}}\\
                \glt `s/he plays flute'
    \end{xlist}
\z

\subsection{\sound{Maká}{j} following high vowels}

The combination of \sound{Mk}{i} and \intxt{j} surfaces as \phonetic{iː}, either at morpheme boundaries or within morphemes. One example is \word{Mk}{witi-kfi-j}{one's ears}, pronounced \phonetic{witikfiː}. In this book, we represent the sequence in question as \intxt{ij}.

In a similar vein, \sound{PM}{*uj(ʰ)} is reflected as \sound{Mk}{wi} after obstruents, with the sonority reaching its peak during the final phase of the rhyme:  \word{Mk}{k'wi}{cold} (but \wordnl{k'uj\mbox{-}i\mbox{-}m}{s/he feels cold}, with the benefactive applicative), \wordnl{nimeɬkw\mbox{-}i}{tombs} (from \wordnl{nimeɬuk}{tomb} and \wordnl{\mbox{-j}}{plural}), \wordnl{k\mbox{-}’wi}{I enter} (but \wordnl{j\mbox{-}uj}{s/he enters}). In this case we follow our sources in representing the sequence in question as \intxt{wi}, because \intxt{uj} is also attested as a valid rhyme in the language: \wordnl{hejeftuj}{I fart}, \wordnl{wit'afthuj}{bile.\PL}, \wordnl{esupuj}{it is soft} \citep{AG99}.

\section{Word-level prosody}\label{mk-prosody}

According to \cits{AG89} description, Maká does not retain any traces of the prosodic distinctions that we reconstruct for Proto-Mataguayan. Instead, Maká has innovated an edge-demarcation pattern whereby the final syllable of a word receives primary stress \REF{mk-stress}.

\booltrue{listing}
\ea\label{mk-stress}
    Maká \citep[67–68]{AG89}\footnote{The preglottalization in the terms for `spider' and `alligator' is not represented in \cits{AG89} work; it is attested in \citet[16]{PMA} and \citet[71]{JB81}, respectively.}
    \begin{xlist}
        \ex \wordnl{saˈlal}{cicada}
        \ex \wordnl{salaˈlits}{cicadas}
        \ex \wordnl{foˈχits}{flutes}
        \ex \wordnl{siˀwaˈlaχ}{spider}
        \ex \wordnl{siˀwalaˈχits}{spiders}
        \ex \wordnl{najaɬeˈneˀχ}{alligator}
        \ex \wordnl{najaɬeneˈχits}{alligators}
        \ex \wordnl{honokok'enˈxuʔ}{I kneel}
    \end{xlist}
\z

In addition, words of four or more syllables receive secondary stress on their peninitial syllable if it is heavy (i.e., contains a coda), and on their initial syllable otherwise \citep[68]{AG89}. 

\ea\label{mk-sec-stress}
    Maká \citep[69]{AG94}
    \begin{xlist}
        \ex \wordnl{qoˌtextinˈheʔ}{bee sp. (mid-sized, dark brown, stings lightly)}
        \ex \wordnl{qoˌtextinheˈtaχ}{bee sp. (queen bee, large, dark brown, stings painfully)}
        \ex \wordnl{t'oˌkonkoteˈket}{acacia \species{Acacia bonariensis} grove}
        \ex \wordnl{ˌqets'ijoˈhol}{bee sp. (mid-sized, yellow, stings hard, produces small amounts of inedible honey)}
        \ex \wordnl{ˌqets'ijohoˈlits}{bees sp. (same species as above)}
        \ex \wordnl{ˌneqfejenˈhet}{wax}
        \ex \wordnl{ˌneqfejenheˈtits}{wax.\PL}
    \end{xlist}
\z
\boolfalse{listing}
