\chapter{Phonotactics and processes} \label{pm-processes}

This short chapter presents an overview of the Proto-Mataguayan phonotactics and of the most important phonological processes that occurred synchronically in the protolanguage. The processes discussed in this chapter quite likely result from sound changes that took place before the disintegration of Proto-Mataguayan. Internal reconstruction of pre-Proto-Mataguayan remains beyond the scope of this book (see \citnp{LC-VG-07} for an early attempt).

\section{Phonotactics}

This section surveys the restrictions on Proto-Mataguayan onsets (\sectref{onsets}), codas (\sectref{codas}), and nuclei (\sectref{nuclei}). It does not take into account syllables composed of a single syllabic coronal consonant, such as \intxt{*ɬ̩}, \intxt{*n̩}, \intxt{*t̩}; these are discussed in \sectref{syllabic-C}. 

\subsection{Onsets}\label{onsets}

Onsets are obligatory in most Mataguayan languages, including Nivaĉle \citep[5]{AnG16}, Iyojwa’aja’ \citep[90]{JC14b}, ’Weenhayek \citep[3]{KC94}, and Lower Bermejeño Wichí \citep[97]{VN14}. This was also the case in Proto-Mataguayan. As discussed in \sectref{proto-glottal}, some roots can start with a vowel in Proto-Mataguayan, but a glottal stop is inserted before that vowel unless the root takes a consonant-final prefix. For example, the root \word{PM}{*\mbox{-}éj}{name} starts with a vowel, as seen in its inflected forms such as \wordnl{*j\mbox{-}éj}{my name} or \wordnl{*ɬ\mbox{-}éj}{her/his name}, but when it combines with the zero allomorph of the second-person prefix, the outcome is \wordnl{*∅\mbox{-}ʔéj}{your name}, with an inserted \intxt{*ʔ}. At the stem--suffix boundary, the hiatus-avoiding strategies are more diverse. Some suffixes simply lose their initial vowel following a vowel-final stem (compare \word{PM}{*ji\mbox{-}koj\mbox{-}ájʰ}{my hands} and \wordnl{*ji\mbox{-}lǻ\mbox{-}jʰ}{my domestic animals}). For other suffixes, it is more difficult to ascertain their PM~allomorphy pattern, because the behavior of their reflexes differs across Mataguayan. For example, the form provisionally reconstructed as \word{PM}{*[t]póʔ\mbox{-}ex}{it is full}, where the applicative suffix is added to a vowel-final stem, is reflected as \wordng{Mk}{[to]póʔ\mbox{-}ox}, \wordng{Ni}{[ta]póˀ\mbox{-}x}, \wordng{Ijw}{[ti]pɔ́\mbox{-}ji}, \wordng{Mj}{[ta]pɔ́\mbox{-}we}, and \wordng{PW}{[t]ˈpó-jeχ}, with full translaryngeal assimilation in Maká, suffix vowel loss in Nivaĉle, \intxt{j}\mbox{-}epenthesis in Iyojwa'aja' and Wichí, and \intxt{w}\mbox{-}epenthesis in Manjui (and Iyo'awujwa'). The reconstruction of the allomorphy patterns of such suffixes awaits further research.

A number of complex onsets can be reconstructed for Proto-Mataguayan, with the onset patterns being quite permissive. Possible combinations include sequences of a fricative and a stop (\intxt{*ɸk}, \intxt{*ɸts}, \intxt{*sk}, \intxt{*st}, \intxt{*Xp}); a fricative and a sonorant (\intxt{*sl}, \intxt{*sˀw}, \intxt{*xn}, \intxt{*Xw}); a stop and a sonorant (\intxt{*tl}); a stop and a fricative (\intxt{*kɸ}, \intxt{*kh}, \intxt{*ph}, \intxt{*px}); a sonorant and a stop (\intxt{*lk}); two sonorants (\intxt{*ˀnj}); a fricative, a stop, and a sonorant (\intxt{*stw}). For Proto-Chorote–Wichí, sequences of two stops are also reconstructed (\intxt{*kp}, \intxt{*kt}, \intxt{*tk}). This list is probably not exhaustive.

Other consonant clusters occurred word-internally in Proto-Mataguayan, but it is difficult to determine whether they were tautosyllabic or heterosyllabic. It is often the case that Chorote and Wichí show a tautosyllabic reflex of a given cluster, as in \wordng{PM}{*kʼutX₂₃áˀn} > \word{PW}{*kʲ’u.tháˀn}{thorn}; \wordng{PM}{*\mbox{-}ʔaqhuˀts \recind *\mbox{-}ʔaqhúˀts} > \word{PCh}{*\mbox{-}ʔa.qús}{knee}. The Nivaĉle reflexes of such clusters are heterosyllabic, as in \word{Ni}{k’ut.xaˀn}{thorn} \citep[124]{AnG15}; this is also the case in Maká at least for the clusters of the shape \intxt{Ch}, as in \word{Mk}{wi.taq.huts}{one’s knee} \citep[21, fn. 3]{AG89}.\footnote{Based on the cognates in Nivaĉle and Chorote, we suspect that the Maká form given by \citet[21, fn. 3]{AG89} is a mistranscription for \intxt{wi.t’aq.huˀts}. See \chapref{etymdic} for details.} We are inclined to think that some or all of these clusters were originally tautosyllabic, as suggested by the fact that they commonly occur morpheme-initially and word-initially; the Nivaĉle and Maká syllabification would then be innovative. The issue requires further research.

\subsection{Codas}\label{codas}

Any plain (non-glottalized) consonant, with the possible exception of \intxt{*w},\footnote{\sound{PM}{*w} is reconstructed root-finally in \word{PM}{*[t]k’áw\mbox{-}\APPL}{to hold in one’s arms, to hug} and \wordnl{*\mbox{-}åˀw\mbox{-}\APPL}{to be}, but these roots are typically followed by applicative suffixes, meaning that their final consonants may have been often syllabified as parts of onsets of the subsequent syllable.} could occur as a simplex coda, though some codas are quite rare word-internally (\intxt{*ʔ} occurred in very few words, such as \wordnl{*\mbox{-}qáʔtu(ʔ)}{yellow}, and the coda \intxt{*h} was likely banned word-internally altogether). The coda \sound{PM}{*q} is reconstructed only following low vowels (\sound{PM}{*a} or \intxt{å}), whereas the coda \sound{PM}{*k} seems to have been ruled out following \sound{PM}{*a}.

Complex codas are not allowed in any Mataguayan language, including Maká \citep[58]{AG89}, Nivaĉle \citep[5]{AnG16}, Iyojwa’aja’ \citep[90]{JC14b}, ’Weenhayek \citep[3]{KC94}, and Lower Bermejeño Wichí \citep[98]{VN14}, with two exceptions involving glottal consonants. First of all, Nivaĉle has preglottalized codas, analyzed as sequences of the type */ʔC/ in \citet{AnG15,AnG16b}.\footnote{In fact, \citet{AnG15,AnG16b} provides evidence that /ʔ/ is parsed as belonging to the nucleus in the rhymes of the type /VʔC/ in Nivaĉle.} As discussed in \sectref{glott-codas}, some of these correspond to glottalized codas in Manjui and Wichí, where at least \citet{KC94} analyzes them as underlying sequences of a sonorant and a glottal stop. We reconstruct preglottalized codas to Proto-Mataguayan and follow \citet{AnG15,AnG16b} in analyzing them as underlying sequences of the type */ʔC/, though we chose to represent them as \intxt{*ˀC} for aesthetic reasons. Another type of complex coda, which occurs only before a pause, involves sequences of a non-nasal sonorant (\sound{PM}{*j} or \intxt{*w}) and a \intxt{*h}, represented as \intxt{*jʰ} and \intxt{*lʰ} in this book. These are best preserved in Chorote, where \citet[88]{JC14b} analyzes them as sequences of a sonorant and a so-called ``unstable /h/'' (at least in the Iyojwa'aja' variety, ``unstable /h/'' can also follow nasal sonorants, though such possibility is not reconstructed for~PM). Synchronically, the ``unstable /h/'' in \cits{JC14b} terminology is a kind of /h/ that is deleted word-medially, and in Chorote it may occur both as a part of a complex coda and as a simplex coda, as in \wordnl{máh / má\mbox{-}}{go!}. We also reconstruct \intxt{*jʰ} and \intxt{*lʰ} for Proto-Wichí (note that \sound{PW}{*lʰ} continues both \sound{PM}{*l} and \intxt{*lʰ} word-finally), which are reflected as voiceless consonants \intxt{ç},~\intxt{ɬ} in some Wichí varieties and as voiced \intxt{j},~\intxt{l} in others (\sectref{wi-ch}).

Glottalized stops cannot ever be followed by a consonant or pause at the surface in any Mataguayan language, including Maká \citep[58]{AG89} and Lower Bermejeño Wichí \citep[98]{VN14}. In Nivaĉle, however, a first stop in a consonant cluster may receive underlying specification as [constricted glottis], which surfaces as creaky voice in the preceding vowel \REF{ex:côts'ej:niv}.

\ea\label{ex:côts'ej:niv}
Nivaĉle \citep[6]{AnG16}\\
    \begin{xlist}
        \ex\gll -kåts’ex \phonetic{-qɑˈtsʼex}\\
                -diarrhea\\
                \glt `diarrhea'
        \ex\gll -kåʔtsxe-nax \phonetic{-ˌqɑ̰ʦxeˈnax}\\
                -diarrhea-\textsc{res}\\
                \glt `person that has diarrhea'
    \end{xlist}
\z

We assume that Proto-Mataguayan behaved just like Nivaĉle in this regard. If an underlying glottalized stop came to occur before a consonant, it apparently no longer surfaced as ejective but rather as preglottalized (see \sectref{metathesis-deglott} for more details). Underlying glottalized consonants are not reconstructed in the word-final position, where preglottalized codas can be found instead. We are not aware of any evidence that would suggest that glottalized onsets and preglottalized codas are related in any way.

\subsection{Nuclei}\label{nuclei}

The nucleus position in Proto-Mataguayan was filled by any of its seven vowels, though we have not found evidence for reconstructing the vowel \intxt{*ä} word-finally (or preceding a word-final \intxt{*h}). In addition, as described in \sectref{syllabic-C}, coronal consonants (at least \intxt{*ɬ}, \intxt{*n} and \intxt{*t}) could occur as nuclei, as most clearly seen in preconsonantal allomorphs of certain prefixes.

\section{Consonantal and vocalic stems}\label{c-v-stems}

A very important feature of the morphophonology of the Mataguayan languages is the fact that consonant-final stems may suffer changes if a morpheme is added to their right. These alternations most characteristically occur in plural formation and in compounding, but not with other types of affixes, such as applicatives. In what follows, we use the labels \conc{consonantal stem} for the allomorph that shows up if no suffix is present and \conc{vocalic stem} for the allomorph that shows up before certain suffixes. The alternation patterns are summarized in \tabref{pm-c-v-stems}. Note that only guttural (that is, velar, uvular, or glottal) stem-final consonants are subject to alternations other than metathesis, such as truncation or weakening, an observation we owe to an anonymous reviewer of this book.

\begin{table}
\caption{Consonantal and vocalic stems}
\label{pm-c-v-stems}
\fittable{
 \begin{tabular}{rccl}
  \lsptoprule
    subsection & consonantal stem & vocalic stem \\\midrule
    \sectref{glott-loss-suff} & \intxt{*-CVʔ} & \intxt{*-CV-} \\
    \sectref{glott-loss-suff} & \intxt{*-CVh} & \intxt{*-CV-} \\
    \sectref{jj-suff} & \intxt{*-CVχ} & \intxt{*-CV-} \\
    \sectref{jj-suff} & \intxt{*-CVχ} & \intxt{*-ChV-} \\
    \sectref{velar-weakening} & \intxt{*-CVk} & \intxt{*-ChV-} \\
    \sectref{fricative-h} & \intxt{*-FVk} & \intxt{*-FV-} \\
    \sectref{metathesis} & \intxt{*-C₁VC₂} & \intxt{*-C₁C₂V-} \\
    \sectref{metathesis-deglott} & \intxt{*-C₁’VC₂} & \intxt{*-ˀC₁C₂V-} \\
    \sectref{no-voc-stem} & \intxt{*-C₁VC₂} & no vocalic stem \\
  \lspbottomrule
  \multicolumn{3}{p{\textwidth}}{C~=~consonant, F~=~fricative, V~=~vowel}\\
 \end{tabular}
 }
\end{table}

Furthermore, multiple plural suffixes have two allomorphs, one that starts with a vowel and combines with consonantal stems, and another one that starts with a consonant and combines with vocalic stems.

\begin{exe}
    \ex \plaj
    \ex \pll
    \ex \plits
\end{exe}

\subsection{Glottal truncation in suffixation}\label{glott-loss-suff}

\sound{PM}{*ʔ\mbox{-}} and \intxt{*h\mbox{-}}final stems always form their vowel stems by deleting the glottal consonant altogether. Similar rules have been explicitly described for Maká by \citet[70–71]{AG89} and for Nivaĉle by \citet[271–272]{AnG15} and \citet[285]{AnG20}.

Some examples of \sound{PM}{*ʔ\mbox{-}}final stems follow. Note that when a plural suffix is enclosed in parentheses in our notation, it attaches directly to the stem if the stem ends in a vowel, but replaces the stem-final \intxt{ʔ} if the stem ends in it, that is, the notation ``\wordng{Mk}{\mbox{-}kiʔ\pl{j}}'' is to be read as ``\SG~\intxt{\mbox{-}kiʔ}, \PL~\intxt{\mbox{-}ki\mbox{-}j}''.

\begin{exe}
    \ex \fruit
    \ex \mouth
    \ex \coal
    \ex \locustmn
    \ex \soninlaw
    \ex \treen
    \ex \hunger
    \ex \vulture
    \ex \feminine
    \ex \elderbro
    \ex \eldersis
    \ex \youngersis
    \ex \pet
    \ex \louse
    \ex \seed
    \ex \fatpe
    \ex \beard
    \ex \leg
    \ex \dinlaw
    \ex \eyelash
    \ex \eye
    \ex \face
    \ex \eyebrow
    \ex \rheum
    \ex \woodpecker
    \ex \tooth
    \ex \expert
    \ex \rib
    \ex \bat
    \ex \mosquito
    \ex \pigeon
\end{exe}

Some examples of \sound{PM}{*h\mbox{-}}final stems are shown below. In this case, only Chorote and (rarely) Wichí show any trace of an original alternation, because word-final \sound{PM}{*h} was lost in Maká, Nivaĉle, and in some cases in Wichí (see \sectref{proto-h}).

\begin{exe}
    \ex \companion
    \ex \lizard
    \ex \neighbor
    \ex \dog
    \ex \hornero
    \ex \moon
    \ex \rat
    \ex \doveula
    \ex \lessergrison
\end{exe}

\subsection{Behavior of stem-final \intxt{*χ} in suffixation}\label{jj-suff}

\sound{PM}{*χ\mbox{-}}final stems typically form their vowel stems by deleting the uvular fricative altogether. They always select for the plural suffix \intxt{*\mbox{-}ts}.

\begin{exe}
    \ex \ocelot
    \ex \barnowl
    \ex \many
    \ex \longv
    \ex \anteater
    \ex \pseudo
    \ex \far
    \ex \largefat
    \ex \rhea
    \ex \argentineboa
    \ex \chaguara
    \ex \firei
\end{exe}

Yet other \sound{PM}{*χ\mbox{-}}final stems form their vowel stems by converting \intxt{*\mbox{-}Vχ} into \intxt{*\mbox{-}hV\mbox{-}}. They too select for the plural suffix \intxt{*\mbox{-}ts}.

\begin{exe}
    \ex \centipede
    \ex \oldn
    \ex \piranhamn
    \ex \caracara
\end{exe}

\subsection{Velar weakening}\label{velar-weakening}

\sound{PM}{*k\mbox{-}}final stems typically form their vowel stems by converting \intxt{*\mbox{-}Vk} into \intxt{*\mbox{-}hV\mbox{-}}.\footnote{If the application of the rule would result in an illicit consonant cluster, \intxt{*\mbox{-}Vk} can change to \intxt{*\mbox{-}VhV} instead. No clear instances of this avoidance strategy have been reconstructed so far, but its traces have been preserved in various languages: compare \word{Nivaĉle}{tak͡luk\plf{tak͡luhu\mbox{-}j}}{blind} \citep[248]{JS16}, \word{'Wk}{la\mbox{-}p'ok\plf{la\mbox{-}p'óho\mbox{-}ç}}{its fence} \citep[80]{KC94}, \word{Lower Bermejeño Wichí}{la\mbox{-}wekʷ\plf{la\mbox{-}wehe\mbox{-}j}}{its owner} \citep[192]{VN14}.} Similar rules have been described for Maká by \citet[72–73]{AG89} and for Nivaĉle by \citet[9–10]{LC-VG-07}. In Lower Bermejeño Wichí, \citet[192]{VN14} analyzes stem-final \intxt{\mbox{-}kʷ} and \intxt{\mbox{-}eq} as suffixes precisely because they alternate with \intxt{\mbox{-}hV\mbox{-}} in plurals (as in \word{LB}{nijokʷ\plf{niço\mbox{-}j}}{rope}); a similar stance is taken in \citet{JC14a} regarding Iyojwa'aja' pairs such as \wordnl{ʔimóhsik\plf{ʔimóhse\mbox{-}ˀl}}{devil, deity}, \wordnl{\mbox{-}ɛ́tik\plf{\mbox{-}ɛ́te\mbox{-}ˀl}}{head}. We believe that these alternations are best understood as phonological rather than morphological.

\begin{exe}
    \ex \tobacco
    \ex \powder
    \ex \rope
    \ex \bilecw
    \ex \plate
    \ex \headn
\end{exe}

It is quite possible that whenever the application of the velar weakening resulted in a cluster of a glottalized stop and \intxt{*h}, the former became a preglottalized coda, a phenomenon known from vocalic stems with metathesis and glottal reallocation (\sectref{metathesis-deglott}). However, we know of no relevant examples reconstructible to Proto-Mataguayan.\footnote{Synchronically, velar weakening combined with glottal reallocation has been marginally attested in Nivaĉle by \citet[182]{JS16}, who documents \word{Ni}{nap’uk\plf{naˀpxu\mbox{-}j}}{ashes used as salt; soda}. The existence of the plural form \intxt{naˀpxu\mbox{-}j} is, however, not confirmed by \perscommp{Analía Gutiérrez}{2023}. In addition, elsewhere \citet[177]{JS16} himself documents the vocalic stem of \intxt{nap’uk} as \intxt{naˀpku\mbox{-}}, without the velar weakening process. Unless this is a mistake on \cits{JS16} part, we may be dealing here with dialectal variation.}

Note that \sound{PM}{*k} does not simply fricativize to the homorganic \intxt{*x}: forms such as \word{Mk}{(\mbox{-})nijha\mbox{-}j}{ropes, cords} (with the glottal consonant \intxt{h}) as well as \word{Ni}{(\mbox{-})titxe\mbox{-}j}{plates}, \wordnl{\mbox{-}ʃatxe\mbox{-}s}{heads} (with the consonant \intxt{x} in a palatalizing environment) clearly show that \sound{PM}{*h} has to be reconstructed in these cases. Compare this to the following examples of \sound{PM}{*x\mbox{-}}final stems, where a velar fricative is unequivocally reconstructed in both the consonantal and in the vocalic stems (related by metathesis, see \sectref{metathesis}), as evidenced by the velar reflex \intxt{x} in Maká and by the palatalized reflex \intxt{ʃ} in Nivaĉle.

\begin{exe}
    \ex \bow
    \ex \nose
    \ex \pathn
    \ex \abdcavity
\end{exe}

Some \sound{PM}{*k\mbox{-}}final stems, all of which have a rounded vowel preceding the velar stop, are lexically specified for not undergoing the velar weakening process. Instead, they undergo metathesis (\sectref{metathesis}) or lack a vocalic stem altogether (\sectref{no-voc-stem}).

\begin{exe}
    \ex \leniosa
    \ex \zorzal
    \ex \cat
    \ex \duraznillosgpl
    \ex \palmsgpl
\end{exe}

One could suspect that at some stage, before the divergence of Proto-Mataguayan into the daughter languages, these stems ended in a uvular stop (\sound{PM}{*q}). Recall from \sectref{proto-q} that synchronically \sound{PM}{*q} in a coda position can only be preceded by a low vowel (\sound{PM}{*a} or \intxt{å}). Therefore, one can tentatively reconstruct a sound change whereby the Pre-Proto-Mataguayan rhymes \intxt{*\mbox{-}oq} and \intxt{*\mbox{-}uq} yielded Proto-Mataguayan \intxt{*\mbox{-}ok} and \intxt{*\mbox{-}uk}. Velar weakening would have arisen only in those stems that ended in a \intxt{*\mbox{-}k} -- but not in \intxt{*\mbox{-}q} -- in Pre-Proto-Mataguayan.

\subsection{Ban on \intxt{*h} after fricatives}\label{fricative-h}

Whenever velar weakening (\sectref{velar-weakening}) would result in a sequence of a Proto-Mataguayan fricative and \intxt{*h}, the glottal fricative does not surface altogether. If the velar weakening process operated ``normally'', one would expect the vocalic stem of nouns such as \wordnl{*\mbox{-}ɬuˀk}{yica bag, load} to have been \intxt{**\mbox{-}ɬhu\mbox{-}}, but the reflexes in the daughter languages rather point to \intxt{*\mbox{-}ɬu\mbox{-}}. Some examples follow.

\begin{exe}
    \ex \thread
    \ex \yicalhuk
    \ex \firewoodhuk
\end{exe}

We take this as evidence that synchronically sequences of a fricative and \intxt{*h} were banned in Proto-Mataguayan, possibly due to a Pre-Proto-Mataguayan sound change \intxt{*Fh > *F}, where \intxt{F} stands for any fricative. Note that the sequences \intxt{*ɸh}, \intxt{*ɬh}, \intxt{*sh}, \intxt{*xh}, \intxt{*χh}, or \intxt{*hh} are not reconstructed anywhere in the lexicon.

\subsection{Metathesis}\label{metathesis}

Stems that end in an obstruent may form their vocalic allomorph by means of metathesis of the final two segments of the stem. Similar rules have been described for Maká (plant names) by \citet[74]{AG89} and for Nivaĉle by \citet[272–274]{AnG15}. The latter author also claims that the metathesis in Nivaĉle is driven by two requirements, namely, the avoidance of complex codas and the satisfaction of the Syllable Contact Law \citep{RMTV83}, whereby ``sonority should not rise across a syllable boundary (from an obstruent to a sonorant)'' \citep[295]{AnG20}. Note that preglottalized codas undergo deglottalization upon metathesizing, as in \REF{met-fart}, \REF{met-nose}, \REF{met-abdcavity}; this is still synchronically the case in Maká and Nivaĉle.

\begin{exe}
    \ex \food
    \ex \wing
    \ex \fart \label{met-fart}
    \ex \leniosa
    \ex \grandchildmpl
    \ex \youngerbro
    \ex \bow
    \ex \nose \label{met-nose}
    \ex \abdcavity \label{met-abdcavity}
    \ex \trunk
\end{exe}

In some idiosyncratic cases, vocalic stems formed by means of metathesis select for the vowel-initial allomorph of the plural suffix, and the final vowel of the vocalic stem is therefore deleted. Synchronically, the resulting pattern has been described as vowel syncope.

\begin{exe}
    \ex \pathn
    \ex \burrow
\end{exe}

At least in Nivaĉle, the metathesis rule does not apply if it would result in an illicit consonant cluster: the vowel is copied instead, so that the stem-final consonant appears flanked by identical vowels in the vocalic stem, as in \word{Ni}{xot\plf{xoto\mbox{-}j}}{sandy place} \citep[277]{AnG15}. Even though similar alternations were found in other languages, as in \word{Ijw}{t\mbox{-}'ák\plf{t\mbox{-}'aká\mbox{-}ˀl}}{rope} \citep[92]{JC14b}, \word{Mj}{hi\mbox{-}hwɛ́tus\plf{hi\mbox{-}hwɛ́tusu\mbox{-}j}}{its root} \citep{JC18}, we have not been able to reconstruct any clear case of a PM~lexeme that would follow such a pattern.

\subsection{Metathesis and glottal reallocation}\label{metathesis-deglott}

The pattern described in this subsection must have been quite rare in Proto-Mataguayan. It arises when the application of metathesis (\sectref{metathesis}) would result in a consonant cluster whose first member is a glottalized stop. In this case, the stop surfaces as preglottalized rather than ejective, as in \word{Ni}{\mbox{-}kåts’ex}{diarrhea} and \wordnl{\mbox{-}kåˀtsxe\mbox{-}nax}{person that has diarrhea} \citep[227]{AnG15}.\footnote{Since in Nivaĉle only prosodically prominent syllables allow for a glottal or preglottalized coda, no preglottalization surfaces in forms such as \word{Ni}{ʔap’ax\plf{ʔapxa\mbox{-}}}{jararaca} \citep[273]{AnG15}.} This pattern has been preserved only in Nivaĉle, but it is evidently quite archaic.

Consider the following pair of nouns, both of each are securely reconstructible to Proto-Mataguayan. The derivational relation between them is not productive, but it is possible to speculate that the latter member of the pair contains a fossilized masculine suffix \intxt{*\mbox{-}ˀk}, added to the vocalic stem of the former (with metathesis and glottal reallocation).

\begin{exe}
    \ex \aunt
    \ex \uncle
\end{exe}

The [constricted glottis] feature in the initial consonant of the term for\gloss{uncle} can be seen in Nivaĉle forms where stress falls on the prefix, such as \wordnl{ji\mbox{-}ká\mbox{-}ˀtxok}{my brother-in-law} \citep[191]{AnG15}. In forms such as \word{Ni}{ji\mbox{-}txóˀk}{my uncle}, no preglottalization is found, because Nivaĉle systematically deglottalizes the codas in all prosodically weak syllables. In other languages, there are no traces of the [constricted glottis] feature in the term for\gloss{uncle} (in stark contrast with the term for\gloss{aunt}). Recall from \sectref{glott-codas} that Maká and Nivaĉle are the only languages that retain the contrast between preglottalized and plain obstruent codas. That way, the obvious solution is to reconstruct the vocalic stem of \wordng{PM}{*\mbox{-}t’ox \recind *\mbox{-}t’óx} as \wordng{PM}{*\mbox{-}ˀtxo\mbox{-} \recind *\mbox{-}ˀtxó\mbox{-}}, where metathesis is combined with the reallocation of the [constricted glottis] feature to the left. The Maká term for `uncle' is regrettably not attested in our sources that distinguish between plain and preglottalized codas. Other Nivaĉle stems that show the phenomenon in question, such as the pair \word{Ni}{nap’uk}{ashes used as salt; soda} and \wordnl{naˀpku\mbox{-}tax}{salt} \citep[177, 182]{JS16}, lack known cognates in other Mataguayan languages.

\subsection{Absence of a vocalic stem}\label{no-voc-stem}

Not all consonantal stems have a vocalic counterpart. Some of them remain unaltered before any suffixes, with the proviso that preglottalized codas deglottalize when they resyllabify as the onset of the next syllable before certain affixes (for example, the plural form of \wordnl{*kʼutX₂₃áˀn}{thorn} is reconstructed as \intxt{*kʼutX₂₃án\mbox{-}its}).\footnote{Only a subset of vowel-initial affixes behaves like this. Others can attach to stems that end in a preglottalized coda without triggering deglottalization, as in \wordnl{*ji\mbox{-}péˀj\mbox{-}aʔ}{s/he hears}.}

Some suffixes have dedicated allomorphs that co-occur with consonantal stems. For example, the plural suffixes surface as \intxt{*\mbox{-}áj}, \intxt{*\mbox{-}íts}, and \intxt{*\mbox{-}él} after consonants.\footnote{In fact, some authors, such as \citet[190]{VN14} for Lower Bermejeño Wichí and \citet[274-8]{AnG15} for Nivaĉle, have described the vowels appearing in such allomorphs as epenthetic. Note, however, that different suffixes show up with different vowels in Proto-Mataguayan, a fact that makes us think that the vowels in question are part of the underlying representation of the suffix. Of course, innovations in individual Mataguayan languages and dialects have altered the picture in some cases. For instance, in Nivaĉle the allomorphs \intxt{*\mbox{-}íts} and \intxt{*\mbox{-}él} are reflected as \intxt{\mbox{-}ik/\mbox{-}ek}, \intxt{\mbox{-}is/\mbox{-}es}, with the choice of the vowel depending on the dialect, on the preceding consonant, and even on the lexeme, with some inter\mbox{-} and intra-speaker variation \citep{AnG15}.} Other suffixes have only one allomorph. In Nivaĉle and Chorote, an epenthetic vowel may occur between a consonantal stem and a consonant-initial suffix: \word{Ni}{βosok͡l\mbox{-}[i]tax}{big butterfly}, \wordnl{ɬ\mbox{-}up\mbox{-}[i]tʃat}{group of nests}, \wordnl{p’ok\mbox{-}[i]βaʃ}{mark of an arrow} \citep[68--69]{AnG15};
\word{Ijw}{wiˀjít\mbox{-}[i]p}{winter}, \wordnl{hi\mbox{-}ˀwɛ́t\mbox{-}[i]hwa}{her/his neighbor} \citep{JC14a}. It is as of yet unclear whether the vowel epenthesis strategy was employed in Proto-Mataguayan, since some Mataguayan varieties lack it: compare \word{’Wk}{xʷitsúk\mbox{-}tax}{kind of palm}, \wordnl{ha\mbox{-}ˀwét\mbox{-}xʷah}{your neighbor} \citep[56, 172]{KC16}, with no vowel epenthesis.

The following nouns are reconstructed as lacking a vocalic stem, as seen in the respective plural forms.

\begin{exe}
    \ex \yicaay
    \ex \namen
    \ex \water
    \ex \tail
    \ex \hand
    \ex \thorncutjan
    \ex \winter
    \ex \languageword
    \ex \zorzal
    \ex \rain
    \ex \starn
    \ex \costume
    \ex \cicada
    \ex \wildcat
    \ex \kingvulture
    \ex \cat
    \ex \basetrunk
    \ex \snake
    \ex \chaja
    \ex \nest
    \ex \stagnant
    \ex \earth
    \ex \skin
    \ex \meat
\end{exe}

\section{Allomorphs of prefixes}

Many prefixes display an allomorphy pattern whereby a moraic allomorph is used before stems that start with a supraglottal consonant, and a non-moraic allomorph occurs with stems that begin with a vowel or a glottal stop (in which case the prefix coalesces with the glottal stop). Homophonous prefixes follow identical allomorphy patterns in Proto-Mataguayan.

\begin{table}
\caption{PM alternating prefixes}
\label{PM-alternating-prefixes}
 \begin{tabular}{ccccccc}
  \lsptoprule
          & before C & before V & before ʔ\\\midrule
  1.\textsc{poss}, 1.A/S\textsubscript{A}.{\textsc{irr}}, 3.A/S\textsubscript{I}.{\textsc{rls}} & *ji- & *j- & *ˀj-\\
  2.\textsc{poss}, 2.A/S\textsubscript{A}.{\textsc{irr}} & *ʔa- & *∅- & *∅-\\
  3.\textsc{poss}, 2.A/S\textsubscript{A}.{\textsc{rls}} & *ɬ̩- & *ɬ- & *ɬ’-\\
  2.S\textsubscript{P}/P.{\textsc{rls}}, 3.A/S.{\textsc{irr}} & *n̩- & *n- & *ˀn-\\
  3.S\textsubscript{T} & *t̩- & *t- & *t’-\\
  1.A/S\textsubscript{A}.{\textsc{rls}} & *ha- & *h- & *k’-\\
  \lspbottomrule
 \end{tabular}
\end{table}

For details, see \chapref{etymdic} and the discussion in \sectref{syllabic-C}.

\section{Irregular verbs} \label{irrv}

A very limited number of Proto-Mataguayan verbs are reconstructed as having an alternation between low vowels and \intxt{*i}, where the vowel \intxt{*i} appears after prefixes of the shape \intxt{*j\mbox{-}} (including 3.A/S\textsubscript{I}.{\textsc{rls}} and 1.A/S\textsubscript{A}.{\textsc{irr}}).

\begin{exe}
    \ex \goaway
    \ex \cry
    \ex \diecw
\end{exe}

In the latter two cases, Chorote has generalized the allomorph with a low vowel, and Wichí the one with a high vowel.
