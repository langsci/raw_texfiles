\chapter{Introduction}

Mataguayan is a small language family of Southern Chaco (South America). It includes at least four distinct languages, of which two show considerable internal diversity: \concl{Maká} (Glottocode [maca1260]), \concl{Nivaĉle} ([niva1238]), \concl{Chorote} (with its varieties \concl{Iyojwa’aja’} [iyoj1235], \concl{Iyo’awujwa’} [iyow1239], and \concl{Manjui}), and \concl{Wichí} (a dialect continuum which includes varieties such as \concl{’Weenhayek} [wich1262], \concl{Lower Pilcomayeño}, \concl{Vejoz}, and \concl{Southeastern}). In this book, we systematically apply the comparative method to the extant Mataguayan varieties in order to arrive at a reconstruction of Proto-Mataguayan (=~PM) phonology and lexicon.

\begin{sidewaysfigure}
\includegraphics[width=15cm]{figures/mataguayan_map.pdf}
\caption{Map of the Mataguayan-speaking area}
\label{fig-map}
\end{sidewaysfigure}

Basic facts on the individual Mataguayan languages are presented in \sectref{intro-mataguayan}. The theoretical tenets of this study are discussed in \sectref{intro-tenets}. \sectref{intro-previous} surveys all published studies which deal with the reconstruction of Proto-Mataguayan and the historical development of individual Mataguayan languages. \sectref{notation} makes explicit our notation conventions and \sectref{structure} details the structure of this book.

\section{Mataguayan languages} \label{intro-mataguayan}
\sloppy
This section presents some basic facts on each Mataguayan language: Maká (\sectref{intro-mk}), Nivaĉle (\sectref{intro-ni}), Chorote (\sectref{intro-ch}), and Wichí (\sectref{intro-wi}).
\fussy

\subsection{Maká} \label{intro-mk}
\concl{Maká} (Glottocode [maca1260]) is the native language of the Maká people of Paraguay. Most speakers currently live in Nueva Colonia Indígena Maká, a community located within the city of Mariano Roque Alonso, in the Gran Asunción metropolitan area (Central department). In addition, some Maká live in the communities of Qemkuket (Presidente Hayes department) and Ita Paso (Itapúa department), as well as in the proximities of Ciudad del Este (Alto Paraná department) \citep[128]{CM15}. The 2012 Paraguayan census \citep{ine-py-12} reports the following number of ethnic Maká by department: 1~228 in the Central department, 436 in Presidente Hayes, 32 in Itapúa, 167 in Alto Paraná, 20 in Boquerón (total population in Paraguay: 1~888). In the Argentine territory, the 2022 Argentine census reports 13 ethnic Maká, including 5 who speak or understand the language \citep{indec2024}. In earlier literature, the language and the people are sometimes called \concl{Enimagá}, \concl{Towothli}, \concl{Cochaboth}, or \concl{Lengua}.

Before the Chaco War (1932--1935), the Maká resided in the Paraguayan Chaco, between the headwaters of the Verde, Confuso, and Montelindo Rivers. Their centers were Cuatro Vientos, Nanawa, and Laguna-Guasú, and they are reported to have been divided into two groups, \concl{Fisket Ɫeiɫets} and \concl{Aseptiket Ɫeiɫets}, who possibly spoke slightly different dialects \citep[28]{JBe34,MCS72,AG94}. After the Chaco War, most of the Maká were transferred to Colonia Fray Bartolomé de las Casas, just across the Paraguay River from Puerto Botánico (Asunción), and in 1985 they relocated to their current location in the city of Mariano Roque Alonso. As of 1991, very few Maká were reported to still live in their homeland in the Chaco \citep[28--29]{AG94}.

\hspace*{-1.4pt}Until the 1990s, the Maká language had been known to Western scholars mostly through wordlists. One such wordlist, collected by Wilfrid Barbrooke Grubb and referred to as the \concl{Towothli} doculect in this book, is reproduced in \citet[238--256]{RJH15}, whereas several other wordlists \citep{VK31,JBe31,JBe34,MS37} are published in \intxt{Revista de la Sociedad Científica del
Paraguay} (partly reproduced in the Appendix in \citnp{TT15}). In addition, \citet[456]{AD60} documents a list of 16 words representing a language he calls \concl{Lengua}, which appears to be a divergent dialect of Maká.\footnote{Note that the ethnonym ``Lengua'' has also been historically used to refer to unrelated ethnic groups of the Chacoan region, including the Enlhet (also known as ``Lengua Septentrional'', ``Northern Lengua'', or ``Lengua Norte''), the Enxet (also known as ``Enxet Sur'', ``Lengua Meridional'', ``Southern Lengua'', or ``Lengua Sur''), and the Payaguá. The Enlhet and the Enxet are speakers of languages classified as members of the Enlhet–Enenlhet family. The extinct and scarcely attested Payaguá language is best classified as a linguistic isolate, though it may well turn out to be distantly related to Mataguayan \citep{PVB04}.} These sources do not faithfully reflect the phonological oppositions of Maká and are therefore of limited importance for our study, though they provide philological evidence for dating certain sound changes. Maká data in this book come mostly from \citet{AG89,AG94,AG99}, with \citet{CM15} and \citet{TT15} used as secondary sources. \citet{JB81}, \citet{maka-etnomat,PMA}, \citet{unuuneiki}, and Wycliffe’s Bible translations have also been consulted, especially with regard to the opposition between plain and glottalized codas and sonorant onsets, underdifferentiated in other sources.

\subsection{Nivaĉle} \label{intro-ni}
\concl{Nivaĉle} ([niva1238]) is spoken by the people of the same name in Paraguay and Argentina. The 2012 Paraguayan census \citep{ine-py-12} reports 14~768 ethnic Nivaĉle in the Paraguayan territory, including 11~705 in the department of Boquerón and 2~932 in the department of Presidente Hayes. In the Argentine territory, the Nivaĉle are known as \concl{Chulupí}, and their ethnic population is 878, 75.1\% of which speak or understand Nivaĉle (this corresponds to 659 speakers), according to the 2022 Argentine census \citep{indec2024}. Historically, the presence of the Nivaĉle in what is now Argentina was much more notable, and their area used to extend to the Bermejo River in the south; however, due to conflicts with the military in the early twentieth century they retreated north to the Pilcomayo River, and they abandoned their last village on the Bermejo River in 1913 \citep[258]{RJH15}. The migration patterns of the Nivaĉle in the first half of the twentieth century are particularly complex. Between 1900 and 1945, many Nivaĉle migrated seasonally from Paraguay to Argentina, seeking to work on sugar plantations in Salta and Tucumán. From 1930 on, a migration flow in the opposite direction -- towards the Mennonite colonies of New-Halbstadt and Filadelfia -- became increasingly intense \citep[7--10]{NS87}. In earlier literature, the language and the people are sometimes called \concl{Ashlushlay}.

\citet[7]{AnG15} reports at least three regional varieties of Nivaĉle as defined by linguistic criteria:
\begin{enumerate}
    \item \concl{Chishamnee Lhavos} (also known as the Arribeño, or Upriver dialect), spoken along the Pilcomayo River, from Fortín Magariños (to the west from Misión Esteros) in the southeast up to the Pedro P. Peña area (Paraguay) and Salta (Argentina) in the northwest \citep[21--22]{NS87};
    \item \concl{Shichaam Lhavos} (also known as the Abajeño, or Downriver dialect), spoken from Fortín Magariños up to the Missions of San José de Esteros and San Leonardo de Escalante/Fischat, both in Paraguay \citep[21--22]{NS87};
    \item and \concl{Yita’ Lhavos} (or the Bush dialect), whose zone lays to the north from the Chishamnee Lhavos area, entirely in Paraguay, reaching Mayor Infante Rivarola and approaching Mariscal Estigarribia, with speakers in the Mission of Santa Teresita.
\end{enumerate}

Little is known about the defining characteristics of the dialects spoken by other groups. The \concl{Jotoi Lhavos} live in the northern part of the Mennonite colonies area, around Campo Loa, to the southeast from Mariscal Estigarribia, Paraguay, whereas the \concl{Tavashai Lhavos} live northeast of the Mission of San Leonardo de Escalante/Fischat, between Fortín General Díaz and Tinfunké, along the northernmost extreme of Estero Patiño, also in Paraguay \citep[22--23]{NS87}.

Early work on the Nivaĉle language includes a short description and vocabulary in \citet[257--305]{RJH15} and some less accessible publications, surveyed in \citet[15–17]{LC20}. These early sources are not used in our study, because many phonological oppositions of Nivaĉle are not sufficiently well represented there. In this book, we rely on \citet{JS16} as our main source of the Nivaĉle lexicon, whereas \citet{AnG15}, \citet{AF16}, and \citet{LC20} have served as our main data sources on Nivaĉle phonology and grammar. Secondary sources include \citet{NS87} and the works by \citet{AnG15-evid,AnG16,AnG16c,AnG16b,AnG20,AnG21} and \citet{AnG-GE-23}.

\subsection{Chorote} \label{intro-ch}
\concl{Chorote} is a language, or maybe two closely related languages, spoken by the Iyojwa’aja’ and Iyo’awujwa’ peoples of Argentina and by the Manjui people of Paraguay. The varieties spoken by these peoples are referred to in this book, respectively, as \concl{Iyojwa’aja’} [iyoj1235], \concl{Iyo’awujwa’} [iyow1239], and \concl{Manjui} (no Glottocode assigned). Iyo’awujwa’ and Manjui are considerably closer to each other than any of them is to Iyojwa’aja’; they are sometimes collectively referred to as \concl{Forest Chorote} or, in Gerzenstein's works, as \concl{variety \#2} (\concl{V2}), and individually as \concl{Argentine V2} and \concl{Paraguayan V2}. By contrast, Iyojwa’aja’ is also known as \concl{Riverine Chorote} or as the \concl{variety \#1} (\concl{V1}). \citet{indec2024} reports 3~238 ethnic Chorote (Iyojwa’aja’ and Iyo’awujwa’) in the Argentine territory, 75.1\% of which speak or understand Chorote (this amounts to 2~431–2~433 speakers). Their main communities in the Chacoan region are Misión La Paz, La Bolsa, La Gracia, La Merced Vieja, and La Merced Nueva, although many have moved to the outskirts of Tartagal in the early twentieth century, more specifically, to the communities of Misión Chorote I, Misión Chorote II, Misión Chorote – Parcela 42, Lapacho I, Misión Kilómetro 4, Misión Kilómetro 6, and Misión El Cruce (the latter community is located in the municipality of General Mosconi rather than Tartagal). The 2012 Paraguayan census \citep{ine-py-12} reports 582 ethnic Manjui in the Paraguayan territory, almost all of them (579) in the department of Boquerón. Their main centers are Misión Santa Rosa (Wonta, more than 400 individuals), Abizai (close to Mariscal Estigarribia), and San Eugenio--San Agustín. The exonym \concl{Chorote} is also sometimes spelt \concl{Chorotí} in earlier literature. 

It should be noticed that in this book we reserve the term \intxt{Manjui} (originally a Nivaĉle exonym) for the dialect spoken in specific parts of Paraguay, and particularly in Santa Rosa (Wonta). It does not include the variety spoken in the community of San Eugenio, located in the surroundings of Pedro P. Peña near the Pilcomayo River (Paraguay), which is very close to Argentine Iyo'awujwa' spoken in Misión La Paz, Argentina (``almost identical'', according to a consultant that has lived in both places). Our usage of the term \intxt{Manjui} therefore differs from the everyday usage of the same term in Paraguay, where any Chorote person is referred to as ``Manjui'', irrespective of the dialect they speak (in Argentina, the term ``Chorote'' is employed in the same way).

The autonym of the Manjui is \wordnl{Inkijwas}{neighbors, those who live together}. Another glottonym found in the literature is \wordnl{Lumnanas}{Forest People}, spread in the 2000s, but not universally accepted at present (and rejected in Santa Rosa). In turn, \wordnl{Wikina Wos}{Northern People} is the name given by the Argentine Chorote to the ones that live in Paraguay.

The Manjui variety (excluding that of San Eugenio) has two subdialects, which according to \citet{GH94} are \wordnl{Jlimnájnas}{Forest People}, or \intxt{Dialect A}, and \wordnl{Jlawá'a Wos}{Outsiders}, or \intxt{Dialect B}. The first one corresponds to the original dwellers of the area of Santa Rosa, where a Mission of New Tribes was founded by the end of the 1960s, and the second one to neighboring groups, especially to the East, that arrived to Santa Rosa after the foundation of the Mission. The variety spoken in Mariscal Estigarribia is also Jlawá'a Wos. There are minor differences between them, which are mainly phonetic and, to a lesser extent, lexical. Unfortunately, we cannot reflect this variation in this book in a systematic way. Although we often report internal variation in Manjui, we are often not able to assign a specific dialectal form to either dialect.\footnote{In speakers born in the 1970s or later, with whom Carol's fieldwork was mainly conducted, both dialects seem to have mixed to some extent. Specific forms were often attributed to one or another dialect depending on the speaker, and different forms were sometimes recognized as representative of the same dialect. Most of Carol's consultants recognized themselves as Jlawá'a Wos.}

The varieties of Chorote are generally mutually intelligible to a great extent, except that Iyojwa'aja' and Iyo'awujwa' speakers from Argentina do not understand Manjui because of their increased speech rate (the reverse is, however, not true).

Early sources on Chorote include \cits{RJH15} description of Iyojwa’aja’ and \cits{RLN10} wordlists of Manjui (labeled as ``A'' and ``C'') and Iyojwa’aja’ (labeled as ``B''). However, the transcription in these works is quite unreliable, and we rely on them only when a certain lexeme is not attested in Carol's field materials. The Iyojwa’aja’ data in this book come from Carol's original fieldwork (published in \citnp{JC14b} and \citnp{JC14a}, among other works) and \cits{ND09} dictionary.\footnote{The pioneering study of Iyojwa’aja’ by \citet{AG78,AG79} was instrumental for Carol’s own work, but is not extensively cited in this book given our focus on phonetics and phonology. Subsequent research has revealed some inaccuracies in Gerzenstein’s transcriptions, especially regarding glottal and glottalized consonants.} For the Iyo’awujwa’ variety, we rely on \cits{AG83} grammatical description and vocabulary and on Carol's field notes. For Manjui, we also mostly rely on Carol's field data, published as \citet{JC18} and \citet{JC-amer}, and on \cits{GH94} vocabulary when a given datum is lacking from our corpus. \citet{GS10} is a useful source on Iyojwa’aja’ and Iyo’awujwa’ phytonymy.

\subsection{Wichí} \label{intro-wi}
\concl{Wichí} is a dialect continuum spoken by a people known as Wichí in Argentina and as ’Weenhayek in Bolivia. \citet{indec2024} reports 69~080 ethnic Wichí in the Argentine territory, 73.4\% of which speak or understand Wichí (this amounts to 50~671–50~739 speakers), distributed by province as follows: 45.9\% in Salta, 32.3\% in Formosa, 9.2\% in Chaco, 12.6\% elsewhere. \citet{indec2024} also reports 179 ethnic ’Weenhayek, 63.1\% of which speak or understand ’Weenhayek (this corresponds to 113 speakers). The 2012 Bolivian census \citep{ine-bo-12} reports that 4~551 individuals aged 4 or older learnt 'Weenhayek as their first language, and that 3~482 individuals aged 6 or older use it as their main language in daily life. In earlier literature, the language and the people are sometimes called \concl{Mataco}, an ethnonym now considered pejorative.

From a linguistic point of view, Wichí can be subdivided into at least four dialectal zones, as will be argued in \sectref{wi-dialects}.

\begin{enumerate}
    \item \concl{’Weenhayek} [wich1262], also called \concl{Noctén} or \concl{Noctenes} in earlier literature, is the variety spoken in Bolivia along the Pilcomayo River, between the city of Villamontes and the Argentine border;
    \item \concl{Lower Pilcomayeño} [wich1264] (or \concl{Guisnay}, from Wichí \intxt{W’enhayey} [w’en̥ã\-jej])  is a poorly described dialect (or perhaps a dialect cluster) spoken along the Pilcomayo River and around the city of Tartagal in the Argentine provinces of Salta and Formosa;
    \item \concl{Vejoz} [wich1263] is spoken in the Argentine province of Salta along the Bermejo River;
    \item \concl{Southeastern Wichí} (including subdialects such as \concl{Lower Bermejeño Wichí} and \concl{Rivadavia Wichí}) is spoken in the Argentine provinces of Salta, Formosa, and Chaco along the Bermejo River as well as between the Bermejo and Pilcomayo Rivers.
\end{enumerate}

The earliest known record of Wichí, representative of the Vejoz dialect, is Esteban Primo de Ayala's 1795 \intxt{Diccionario y arte de la lengua mataca}, published in \citet{IC-RM-20}. Other early sources include \citet{GP86,GP97,IM95,JR96,RLN10,RJH13a,RJH13b,RJH37,RJH40}. These works do not fully reflect the phonological oppositions of Wichí and are therefore not particularly useful for the purposes of our study. We rely on modern sources instead. For the ’Weenhayek variety, our preferred sources are \cits{KC16} dictionary and \cits{JAA-KC-14} grammatical description. For Vejoz, we have consulted the vocabularies by \citet{VU74} and \citet{MG-MELO15}. For the Lower Bermejeño subdialect of Southeastern Wichí, we mostly rely on \cits{VN14} grammar, whereas \cits{JB09} vocabulary serves as a secondary source; in addition, many flora and avifauna terms have been extracted from \citet{CS08} and \citet{CS-FL-PR-VN13}. \citet{MS14} is a useful source on plant names in the Southeastern variety as spoken in Salta. \citet{JT09-th} is a description of Southeastern Wichí as spoken in Rivadavia.

\subsection{Lexicostatistic classification} \label{intro-lexicostat}

We have conducted a lexicostatistic survey with the twofold purpose of obtaining a working model of a phylogenetic tree of Mataguayan and assessing the approximate chronological depth of Proto-Mataguayan. An analogous study with similar results had been carried out by \citet{AT64}, but it was based on imperfect data and did not take into account the dialectal diversity of Nivaĉle, Chorote, and Wichí (each of these languages is represented by only one lect in that study).

For our lexicostatistic calculations, we have used a list of 110 concepts (an extension of the 100-item version of the Swadesh list), which has been compiled for Maká, two Nivaĉle lects, three Chorote lects, and four Wichí lects \citep{GLD-mtg} in accordance with the standards adopted in the Global Lexicostatistic Database \citep{GLD}. Known loanwords have been excluded from the counts. We have also calculated approximate divergence dates for each purported intermediate protolanguage based on the formula proposed by \citet{MV-MS-17}.\footnote{The formula in question was chosen because it was designed and tested based on the same type of data sets as the one used here (110-item Swadesh lists for Romance languages compiled in accordance with the standards adopted in the Global Lexicostatistic Database). According to \cits{MV-MS-17} glottochronological model (the so-called \conc{flow model}), two languages whose most recent common ancestor was spoken \intxt{t} millennia ago are expected to share \(e^{-0.61t}(1~+~0.61t)\) cognates on the 110-item Swadesh list.} The resulting matrix is given in \tabref{matrix} (see the \hyperref[abbr]{list of abbreviations} for the glottonyms).

\begin{table}
\tabcolsep=.75\tabcolsep
\caption{Lexical distances between Mataguayan lects (all values in \%)}\label{matrix}
\begin{tabularx}{\textwidth}{r cc ccc cccc}
    \lsptoprule
    & Ni ShL & Ni ChL & Mj & I’w & Ijw & ’Wk & Vej & Riv & LB\\\cmidrule(lr){2-3}\cmidrule(lr){4-6}\cmidrule(lr){7-10}
    Mk & 38.10 & 36.80 & 28.57 & 31.00 & 32.65 & 22.64 & 24.51 & 20.00 & 19.81\\\addlinespace
    Ni~ShL & & 95.33 & 43.92 & 44.12 & 41.41 & 34.26 & 35.58 & 31.58 & 31.78\\
    Ni~ChL & & & 44.86 & 46.08 & 43.43 & 36.11 & 37.50 & 33.77 & 34.26\\\addlinespace
    Mj & & & & 94.12 & 81.82 & 54.63 & 54.37 & 52.63 & 49.53\\
    I’w & & & & & 84.38 & 54.37 & 54.00 & 54.17 & 49.02\\
    Ijw & & & & & & 59.00 & 56.25 & 57.14 & 54.54\\\addlinespace
    ’Wk & & & & & & & 93.27 & 89.47 & 92.59\\
    Vej & & & & & & & & 90.67 & 91.35\\
    Riv & & & & & & & & & 94.80\\
    \lspbottomrule
\end{tabularx}
\end{table}

The languages represented by multiple lects in our survey show unequal degrees of internal diversity. Nivaĉle and Wichí are quite internally close-knit: there are 95.33\% of matches between two Nivaĉle dialects (ca.~560 years of divergence), and 89.47\%--93.27\% of matches between the main dialects of Wichí (ca.~690--900 years); the Rivadavia and Lower Bermejeño subdialects of Southeastern Wichí show an even higher match percentage (94.80\%, or ca.~595 years of independent development). By contrast, Chorote is more internally diverse, with as little as 81.82\% of matches between Manjui and Iyojwa’aja’ (ca.~1270 years). Iyo’awujwa’ is obviously closer to Manjui (94.12\%, or ca.~640 years) than to Iyojwa’aja’, but it shares more cognates with the latter variety than Manjui due to the Iyojwa’aja’--Iyo’awujwa’ contact.

On a macro scale, the clearest node comprising multiple languages within Mataguayan is the so-called Chorote--Wichí branch, with the percentage of matches ranging between 49.02\% and 54.63\% for each pair of lects (we exclude Iyojwa’aja’, which shows up to 59.00\% matches with Wichí, because its speakers are known to have intensely contacted with the Wichí since at least 1900). \citet[371]{AT64} gives a similar figure, with 61\% of matches on a 100-item wordlist and 49\% on a 223-item wordlist. Proto-Chorote--Wichí must have split into Chorote and Wichí some 2,515--2,805 years before present. Note that Wichí could be viewed as the most divergent language within Mataguayan from a morphosyntactic point of view (two salient features are its lack of grammatical gender and its use of demonstrative suffixes rather than proclitics), whereas Chorote is much closer to Nivaĉle and Maká in this regard; we interpret this as an innovation specific to Wichí, whereby the latter language underwent considerable structural change in a relatively short period of time.

The position of Nivaĉle within the Mataguayan \intxt{Stammbaum} is less clear: the language shows comparable percentages of matches with Maká (36.80\%--38.10\%) and Wichí (31.58\%--37.50\%), whereas the Nivaĉle--Chorote matches total at an even higher rate (41.41\%--46.08\%) due to language contact between Nivaĉle and Chorote (note that Nivaĉle cannot form a clade with Chorote to the exclusion of Wichí, since Chorote is most closely related to Wichí). \citet[371]{AT64} finds that Nivaĉle shares the same number of cognates with Maká (44\% on a 100-item wordlist, 38\% on a 223-item wordlist) as with Chorote (44\% and 40\%, respectively), whereas the pair Nivaĉle--Wichí shows fewer matches (38\% and 33\%, respectively); \cits{AT64} opinion is that some Nivaĉle--Chorote matches are of a ``cultural'' nature. There are three possible interpretations, none of which can be discarded at present.

\begin{enumerate}
    \item Nivaĉle could be equidistant from Maká and Chorote--Wichí. In this case Proto-Mataguayan split into three branches (Maká, Nivaĉle, and Chorote--Wichí) somewhere around 4,460--4,930 years ago, as indicated by the low shares of cognates between Maká and the Wichí lects (19.81\%--24.51\%; \citnp{AT64} likewise identifies 15\% of cognates on his 100-item wordlist and 19\% on his 223-item wordlist). The higher shares involving pairs such as Nivaĉle--Chorote (especially Chishamnee Lhavos and Manjui/Iyo’awujwa’), Maká--Nivaĉle (especially the Shichaam Lhavos dialect), Nivaĉle--Wichí, and Maká--Chorote would be explained by undetected borrowings between sister languages.
    \item Nivaĉle could form a clade with Maká. This is proposed by \citet[296]{AF05,LC-VG-07,PVB13a}. Under this scenario, Proto-Mataguayan split in a binary way into Maká--Nivaĉle and Wichí--Chorote ca.~4,460--4,930 years ago. The higher shares involving pairs such as Nivaĉle--Chorote (especially Chishamnee Lhavos and Manjui/Iyo’awujwa’), Nivaĉle--Wichí, and Maká--Chorote would be explained by undetected borrowings between sister languages. Proto-Maká--Nivaĉle must have split into Maká and Nivaĉle ca.~3,520 years ago, based on the cognate share in the pair Maká--Chishamnee Lhavos (Shichaam Lhavos, which has some additional cognates with Maká, is spoken in an area adjacent to the Maká homeland, and the higher share of matches in the pair Maká--Shichaam Lhavos suggests that there has been some language contact between these lects).
    \item Nivaĉle could form a clade with Chorote--Wichí to the exclusion of Maká. In this case Proto-Mataguayan would have split into Maká and Nivaĉle--Chorote--Wichí ca.~4,460--4,930 years ago. The higher shares involving pairs such as Maká--Nivaĉle (especially the Shichaam Lhavos dialect) and Maká--Chorote would be explained by undetected borrowings between sister languages. Proto-Nivaĉle--Chorote--Wichí would have split into Nivaĉle and Chorote--Wichí ca. 3,470--3,880 years before present (based on 31.58\%--37.50\% of matches between Nivaĉle and Wichí). The higher share of cognates involving Nivaĉle (especially the Chishamnee Lhavos dialect) and Chorote (especially Manjui and Iyo’awujwa’) is due to language contact.
\end{enumerate}

In principle, it is conceivable that the low share of cognates between Maká and Wichí -- 19.81\% to 24.51\% -- is due to vocabulary loss in one of these languages (or maybe in both) due to lexical borrowing from unknown sources. If these figures are ignored, the disintegration of Proto-Mataguayan must be dated at 3,880--4,110 years before present, based on cognate shares such as 28.57\% (Maká--Manjui) or 31.58\% (Shichaam Lhavos Nivaĉle--Rivadavia Wichí).

\subsection{External relations} \label{intro-external}

The Mataguayan languages have prominently figured in a number of long-range proposals, most notably as a part of the so-called \concl{Mataco–Guaicuruan} or \concl{Macro-Guaicuruan} proposal (cf. \citnp{PVB13a} for the most recent evaluation and references), whereby Mataguayan is considered to be related to the Guaicuruan language family of Argentina, Paraguay, and Brazil (the extinct Guachí and Payaguá languages are also sometimes included into the proposal; \citnp{PVB04}). The hypothesis hinges on significant morphological similarities between Mataguayan and Guaicuruan, but there are also multiple lexical lookalikes involving reconstructed Proto-Mataguayan and Proto-Guaicuruan forms. We find the Mataco–Guaicuruan proposal plausible, though a detailed appraisal is beyond the scope of this book. Some lexical lookalikes involving Mataguayan and Guaicuruan are given below, and many more are pointed out in our etymological dictionary (\chapref{etymdic}), where we also indicate whether a given lookalike is mentioned in \cits{PVB13a} study. The Proto-Mataguayan reconstructions are ours, and the Proto-Guaicuruan ones come from \citet{PVB13b}.

\booltrue{listing}
\begin{exe}
    \ex \wordng{Proto-Mataguayan}{*-ǻɸe(ʔ)} {\sep} \word{Proto-Guaicuruan}{*-owe}{tooth}
    \ex \word{Proto-Mataguayan}{*[w]ǻpil}{to return hither}, \wordnl{*[t]píl}{to return thither} {\sep} \word{Proto-Guaicuruan}{*\mbox{-}op’il}{to return}
    \ex \word{Proto-Mataguayan}{*[n]åt \recind *[n]ǻt}{to bleed} {\sep} \word{Proto-Guaicuruan}{*\mbox{-}awot}{blood}
    \ex \wordng{Proto-Mataguayan}{*-äɸ} {\sep} \word{Proto-Guaicuruan}{*-aˀwá}{wing}
    \ex \wordng{Proto-Mataguayan}{*[j]ä́n} {\sep} \word{Proto-Guaicuruan}{*-a(ˀ)n}{to put}
    \ex \wordng{Proto-Mataguayan}{*[j]ékɸaˀx} {\sep} \word{Proto-Guaicuruan}{*-ewak}{to bite}
    \ex \wordng{Proto-Mataguayan}{*[ji]låt \recind *[ji]lǻt \recvar *[ji]let \recind *[ji]lét} {\sep} \word{Proto-Guaicuruan}{*\mbox{-}ʔi(ˀ)lote}{to flee}
    \ex \word{Proto-Mataguayan}{*(-)ɬé(ˀ)t}{firewood} {\sep} \word{Proto-Guaicuruan}{*-oˀlét}{fire}
    \ex \wordng{Proto-Mataguayan}{*(-)lo(ʔ) \recind *(-)ló(ʔ)} {\sep} \word{Proto-Guaicuruan}{*á(ˀ)lo}{ashes}
    \ex \word{Proto-Mataguayan}{*mǻh}{go!} {\sep} \word{Proto-Guaicuruan}{*mo}{you go; go!}
    \ex \wordng{Proto-Mataguayan}{*-mǻˀk} {\sep} \word{Proto-Guaicuruan}{*áˀmoqo}{powder}
    \ex \wordng{Proto-Mataguayan}{*-njiˀx} {\sep} \word{Proto-Guaicuruan}{*-(ˀ)nik}{smell}
    \ex \word{Proto-Mataguayan}{*ˀnáɬu(h)}{day, world} {\sep} \word{Proto-Guaicuruan}{*nalóʔ}{natural light, day, sun}
    \ex \wordng{Proto-Mataguayan}{*(-)ˀ<n>ǻjix} {\sep} \word{Proto-Guaicuruan}{*-aˀdíko}{path}
    \ex \wordng{Proto-Mataguayan}{*tsåhǻq} {\sep} \word{Proto-Guaicuruan}{*t’aqaqa}{chajá bird}
    \ex \word{Proto-Mataguayan}{*-wä́ˀx}{burrow, anus} {\sep} \word{Proto-Guaicuruan}{*-ˀwVˀg }{hole}
    \ex \wordng{Proto-Mataguayan}{*ˀwäleˀk} {\sep} \word{Proto-Guaicuruan}{*-awalek}{to walk}
    \ex \wordng{Proto-Mataguayan}{*[ji]ˀwä́n} {\sep} \word{Proto-Guaicuruan}{*-wen}{to see}
    \ex \wordng{Proto-Mataguayan}{*[t]’at’o} {\sep} \word{Proto-Guaicuruan}{*-at’ó}{to yawn}
    \ex \word{Proto-Mataguayan}{*-ʔåx}{skin, bark} {\sep} \word{Proto-Guaicuruan}{*-ʔáko}{skin, leather}
    \ex \wordng{Proto-Mataguayan}{*-(j)uˀk} {\sep} \word{Proto-Guaicuruan}{*-iko}{tree (suffix)}
    \ex \wordng{Proto-Mataguayan}{*-áwå(ʔ)} {\sep} \word{Proto-Guaicuruan}{*-awo<qó>}{flower}
\end{exe}
\boolfalse{listing}

Mataguayan also displays notable similarities with the Zamucoan language family of Paraguay and Bolivia, which is composed of three languages (Old Zamuco, Ayoreo, and Chamacoco). \citet{LC14} notes multiple morphological and lexical similarities between Zamucoan, Mataguayan, and Guaicuruan, and attributes them to language contact, but the nature of similarities involved (inflectional morphology, basic vocabulary, shared suppletion pattern in the verb `to go (away)') makes us think that Zamucoan could in fact share a distant common ancestor with Mataguayan (and Guaicuruan). An obstacle for pursuing this promising avenue of research is the fact that there have been no systematic attempts at reconstructing Proto-Zamucoan phonology and lexicon so far. Some lexical lookalikes involving Mataguayan and Zamucoan are given below; the Zamucoan forms are from \citet[778--791]{LC16}.

\booltrue{listing}
\begin{exe}
    \ex \wordng{Proto-Mataguayan}{*[t]’ä(ˀ)k} {\sep} \wordng{Old Zamuco}{[t]ak}; \wordng{Ayoreo}{[t]ak(e)}; \word{Chamacoco}{[t]aːk}{to eat (intransitive)}
    \ex \wordng{Proto-Mataguayan}{*tux} {\sep} \wordng{Old Zamuco/Ayoreo}{[t]agu}; \word{Chamacoco}{[t]ew}{to eat (transitive)}
    \ex \wordng{Proto-Mataguayan}{*[ji]må} {\sep} \wordng{Old Zamuco}{{\upshape{1\SG}}~a\mbox{-}imo}; \wordng{Ayoreo}{mo}; \word{Chamacoco}{umóʔ}{to sleep}
    \ex \word{Proto-Mataguayan}{*-éj}{name} {\sep} \word{Proto-Guaicuruan}{*-ej}{to name, to call} {\sep} \wordng{Ayoreo}{i}; \word{Chamacoco}{iː-tɕ}{name}
    \ex \wordng{Proto-Mataguayan}{*[j]ik / *-åk / *-äk} {\sep} \wordng{Proto-Guaicuruan}{*-eko \recind *-iko} {\sep} \wordng{Ayoreo}{dik}; \word{Chamacoco}{[d]ɨrk}{to go (away)}
\end{exe}
\boolfalse{listing}

It is possible that Mataguayan, Guaicuruan, and Zamucoan are all even more distantly related to a number of more northern language families. \citet{SLQ10} observes some similarities between the person indices of Guaicuruan and Chiquitano (a language now known to be classified as Macro-Jê; \citnp{WA08}). \citet{PVB05} notes some morphological and lexical similarities between Mataguayan, Guaicuruan, and Macro-Jê, a major language family of Brazil and Bolivia, with extinct members in Paraguay and Argentina. \citet[552--555]{AN-FC-18} tentatively suggest, based on limited evidence, that Mataguayan, Guaicuruan, and Zamucoan form a phylum which is distantly related to another phylum composed of Tupian, Macro-Jê, Bororoan, Cariban, and Karirian (cf. \citnp{ADR09} on this latter grouping); together, all these families are hypothesized to constitute the so-called \concl{Macro-Chacoan} macrofamily, to which \citet[79--80]{AN20} adds Yaathê and is currently inclined to think, based on unpublished evidence, that the Harakmbut--Katukina language family of Western Amazonia (established by \citnp{WA00}) also belongs there.

Some lexical lookalikes involving Mataguayan and other language families are given below. The sources are as follows: \citet{AN20} for Proto-Macro-Jê and for the Karirian varieties (Kipeá and Dzubukuá), \citet{LSC13} for Proto-Bororoan, \citet{SG-DP-07} for Proto-Cariban, \citet{MACS-ms} and \perscommp{Silva}{2022} for pre-Yaathê, \citet{ZdA11} for Katukina, \citet{RT95} for Harakmbut, and the first co-author's ongoing research for Proto-Tupian (partially published in \citnp{AN-FC-22}). The transcriptions have been adapted to the International Phonetic Alphabet, except for sounds whose reconstructed value has not been established with certainty (\sound{Proto-Macro-Jê}{*â}, \sound{Proto-Tupian}{*ḳ}).

\booltrue{listing}
\begin{exe}
    \ex \wordng{Proto-Mataguayan}{*\mbox{-}koj} {\sep} \word{pre-Yaathê}{*\mbox{-}kòj}{hand}
    \ex \wordng{Proto-Mataguayan}{*péɬaj} {\sep} \word{pre-Yaathê}{*pVlitɨ́\mbox{-}a \recind *pVlɨtɨ́\mbox{-}}{rain}
    \ex \word{Proto-Mataguayan}{*\mbox{-}xä́teˀk}{head} {\sep} \word{Proto-Guaicuruan}{*\mbox{-}(a)t'ek}{head, hair} {\sep} \word{pre-Yaathê}{*\mbox{-}d₂áká / *d₂áká\mbox{-}ka}{head}
    \ex \wordng{Proto-Mataguayan}{*\mbox{-}teʔ} {\sep} \wordng{Old Zamuco/Ayoreo}{edo}; \wordng{Chamacoco}{{\upshape{\PL~}}ɨl\mbox{-}e \recind ɨl\mbox{-}ɨ} \citep{LC22} {\sep} \wordng{Proto-Macro-Jê}{*\mbox{-}ndomᵊ} {\sep} \word{pre-Yaathê}{*\mbox{-}tò}{eye}
    \ex \wordng{Proto-Mataguayan}{*ʔítåχ} {\sep} \wordng{Proto-Tupian}{*atʲa / *\mbox{-}j\mbox{-}atʲa} {\sep} \wordng{Kipeá}{isu / \mbox{-}usu}; \word{Dzubukuá}{iðu / \mbox{-}uðu}{fire} {\sep} \wordng{Katukina}{ita}, \word{Harakmbut}{ʔɨtaʔ}{firewood}
    \ex \wordng{Proto-Mataguayan}{*[ji]kǻˀt-\APPL} {\sep} \wordng{Proto-Tupian}{*\mbox{-}ḳat} {\sep} \word{Harakmbut}{\mbox{-}kot}{to fall}
    \ex \wordng{Proto-Mataguayan}{*\mbox{-}ɸ’i(ʔ)} {\sep} \wordng{pre-Yaathê}{*\mbox{-}pè(j)}
    {\sep} (?)~\wordng{Proto-Tupian}{*\mbox{-}pɨ / *mbɨ} {\sep} \wordng{Proto-Macro-Jê}{*\mbox{-}pâɾᵊ} {\sep} \wordng{Kipeá}{bɨ(ri\mbox{-})}; \wordng{Dzubukuá}{bɨ} {\sep} \word{Proto-Bororoan}{*bɨɾe}{foot}
    \ex \word{Proto-Mataguayan}{*\mbox{-}k’u}{horn, club} {\sep} \word{pre-Yaathê}{*\mbox{-}kì}{horn} {\sep} \wordng{Proto-Tupian}{*(\mbox{-})ḳɯp} {\sep} \word{Proto-Macro-Jê}{*(\mbox{-})kɨ₁mᵊ}{tree, horn, club}
    \ex \word{Proto-Mataguayan}{*\mbox{-}k’o}{bottom, pit} {\sep} \wordng{Proto-Tupian}{*\mbox{-}kãʔãc} (preserved only in the Mundurukuan branch) {\sep} \word{Proto-Macro-Jê}{*\mbox{-}kuɲᵊ}{hole}
    \ex \wordng{Proto-Mataguayan}{*\mbox{-}óʔ} {\sep} \wordng{Proto-Macro-Jê}{*c(\mbox{-})ɜmᵊ} {\sep} \wordng{Proto-Bororoan}{*a} {\sep} \word{Proto-Cariban}{*a\mbox{-}ɾɨ \recind *a\mbox{-}tɨpə}{seed}
    \ex \wordng{Proto-Mataguayan}{*\mbox{-}áʔ} {\sep} \wordng{Proto-Guaicuruan}{*\mbox{-}a} {\sep} \wordng{Ayoreo}{a}; \word{Chamacoco}{eː\mbox{-}taʔ}{fruit} {\sep} \word{Proto-Tupian}{*-ʔa}{fruit; head}
\end{exe}
\boolfalse{listing}

\section{Theoretical tenets} \label{intro-tenets}
In this section we describe the theoretical tenets of our study, with an emphasis on how a bottom-up approach to the reconstruction of protolanguages can be meaningfully complemented with elements of a top-down approach. We also discuss the relevance of the different levels of phonological analysis to studies in historical linguistics, and make explicit our views on the best practices in the applications of the comparative method and etymological analysis.

The application of the comparative method in this book follows a \conc{bottom-up top-controlled approach}, composed of two important principles: the \conc{bottom-up reconstruction principle} (\ref{burp}) and the \conc{external control principle} (\ref{ecp}).

\begin{exe}
    \ex \conc{Bottom-up reconstruction principle}. If a given clade is subdivided into subclades, the reconstruction of each element of its protolanguage must be based on the reconstructions of the intermediate protolanguages (the ancestral languages of the aforementioned subclades).\label{burp}
    \ex \conc{External control principle}. If the languages of a given clade do not allow for an unambiguous reconstruction of a given element for its protolanguage (for example, when the evidence is conflicting or incomplete), it is permissible to take into account data from other related languages in order to decide which reconstruction is the most plausible one.\label{ecp}
\end{exe}

The principles in \REF{burp} and \REF{ecp} are applicable to phonological, lexical, morphological, and syntactic comparanda alike.

In order to comply with the bottom-up reconstruction principle, we make extensive use of Proto-Chorote and Proto-Wichí reconstructions in addition to the data of the contemporary Chorote and Wichí varieties. This is justified by the fact that in each Chorote and Wichí variety, at least some important distinction has been lost as compared to Proto-Chorote and Proto-Wichí, respectively. For example, Iyojwa’aja’ has merged the clusters of the shape \intxt{*hT} (where \intxt{T} stands for any stop; metathesized from earlier \intxt{*Th}) with plain stops, whereas Manjui and Iyo’awujwa’ have neutralized the opposition between \intxt{*a} and \intxt{*å}. Similarly, Southeastern Wichí has merged Proto-Wichí \intxt{*u} and \intxt{*e} and has apparently lost important prosodic distinctions of Proto-Wichí, as well as word-final instances of \intxt{*h}, whereas ’Weenhayek has suffered a partial merger of \intxt{*q} and \intxt{*kʷ}, among other likely innovations.

The external control principle allows us to choose between alternative reconstructions of Proto-Chorote and Proto-Wichí forms in a number of situations. For example, as noted above, Manjui and Iyo’awujwa’ have neutralized the opposition between \sound{PCh}{*a} and \intxt{*å}, preserved in Iyojwa’aja’ after palatal and palatalized consonants. This entails that whenever an Iyojwa’aja’ cognate is unavailable -- or if it is available but the vowel in question happens to occur after a non-palatal(ized) consonant -- it would be impossible to decide whether \sound{PCh}{*a} and \intxt{*å} should be reconstructed in a given protoform based on Manjui and Iyo’awujwa’ evidence alone. For instance, the Proto-Chorote etymon of \wordng{Manjui}{ʔahájuk} and \word{Iyo’awujwa’}{ahájik}{mistol tree} (without a cognate in Iyojwa’aja’) could be alternatively reconstructed as \wordng{PCh}{*ʔahájuk}, \intxt{*ʔåhájuk}, \intxt{*ʔahǻjuk}, or \intxt{*ʔåhǻjuk}. Cognates elsewhere in Mataguayan, such as \wordng{Nivaĉle}{ʔaxåjuk} and \wordng{’Weenhayek}{ʔahǻjuk}, clearly show that the correct Proto-Chorote reconstruction is \intxt{*ʔahǻjuk}.

Throughout this book, we adopt a relatively shallow representation of the data as opposed to sticking to an underlying phonological representation (\sectref{transcr}). This is done for a variety of reasons. First of all, using major allophones rather than phonemes helps circumvent the situation where multiple conflicting analyses have been proposed (for example, aspirated and ejective consonants in Wichí are analyzed as clusters by \citnp{KC94} and as phonemes by other authors), or where no deep analysis is available at all (this is the case of Iyo’awujwa’ and of the reconstructed protolanguages). Using a shallow representation also spares us the necessity of representing archiphonemes in neutralizing environments. Finally, representing the major allophones makes it easier for the reader to track instances of phonetic change in addition to those of phonological change.

The reconstruction of Proto-Mataguayan in this book is grounded in a solid etymological analysis of the extant comparative corpus. We take a strict approach to the etymologies, whereby only precise (or almost precise) formal and semantic matches between languages are considered to satisfy the criteria for cognation. In some cases, we argue that horizontal transmission (rather than cognation) accounts best for some of the observed similarities; this includes borrowings which have possibly been intermediated by non-Mataguayan languages.

\section{Previous research} \label{intro-previous}

The Mataguayan language family in its current limits has been recognized as a valid genetic unit at least since \citet{AM42}, who proposed the label \concl{Matako--Maká} for it. \citet[202--204]{JAM50}, who uses the spelling \concl{Mataco--Maká}, proposes that the family is split in a binary way into two branches (\concl{Mataco} and \concl{Maká}), and that the Mataco branch is further subdivided into \concl{Mataco--Mataguayo} (equivalent to the present-day Wichí) and \concl{Chorotí--Ashluslay}, which includes languages known today as Chorote and Nivaĉle. The label \concl{Matacoan} (or its variants), considered derogatory by the speakers, is sometimes used as a synonym of \concl{Mataguayan} even today, especially in English-language publications.

Although there have been attempts at a phonological reconstruction of PM \citep{EN84,PVB93,PVB02}, none of them can be considered conclusive. The first two predate the publication of two pioneering works on Maká \citep{AG94,AG99}, which appears to be a conservative language in many respects (for example, it preserves a contrast between uvulars and velars, mostly neutralized in other languages). \citet{PVB02} makes several improvements, especially regarding guttural (velar, uvular, and glottal) fricatives, but it still predates the publication of important descriptive work on Wichí, Chorote, and Nivaĉle, which appeared in the last two decades; therefore, many issues deserve revision in light of the new data. Indeed, recent documentation work has revealed important facts about the phonologies of Nivaĉle \citep{AF16,AnG15,AnG16,AnG16c,AnG16b,AnG19-L,AnG19,AnG21,LC20},\footnote{\cits{AF16} grammatical description has also been published as a book \citep{AF16-libro}, an edition we were unable to consult. Our mentions of Fabre’s grammar in this book rely on the 2014 version, in particular with regard to the page numbers.} Chorote \citep{JC14b,JC14a,JC18}, and various dialects of Wichí \citep{AFG067,SS07,MA08,AFG-MC-09,JT09-th,VN14}. \cits{AnG-VN-21} study on the glottal stop and glottalization in the Mataguayan family is the most recent contribution, whose main point is that */ʔ/ should be reconstructed as a phoneme in PM. In our book, all these recent works are taken into account, which at times prompt us to deviate in significant ways from decisions taken in earlier studies in Mataguayan historical linguistics.\footnote{This book was already completed when we learned of \cits{VN-NA-23} and \cits{LC-subm} relevant studies.} This is particularly relevant for Chorote (for which we rely on one of the authors’ field data); we show that previous accounts of its historical development have failed to recognize a significant number of phonological processes which are synchronically active in the Chorote varieties.

There are several published studies dedicated to the historical development or comparative studies centered on specific Mataguayan languages. Most of them are dedicated to the dialectal diversity of Wichí, with \citet{EN71,LCB15} focusing on phonology, \citet{VN19} on morphosyntax, \citet{VN-MA-21} on lexicon, whereas \citet{VN20} seeks to identify the defining traits of each major dialect of Wichí. In her description of Iyo’awujwa’ and Manjui, \citet{AG83} notes a number of differences between these varieties and Iyojwa’aja’ and makes an attempt at a reconstruction of Proto-Chorote forms. \citet{LC-VG-07} carry out an internal reconstruction of pre-Nivaĉle phonology based on the morphophonological alternations found in that language.

Finally, \citet{PVB13a} makes a pioneering attempt at a systematic comparison between reconstructed Proto-Mataguayan and Proto-Guaicuruan forms, which reveals a number of promising sound correspondences. The author concludes that a genetic link between those two families is likely (see \sectref{intro-external} for more details).

\section{Notation conventions}\label{notation}

This section presents the conventions used for the representation of linguistic data in this book.

\subsection{Transcription}\label{transcr}

Throughout our study, we resort to (and provide a justification for) using broad phonetic representation for the data of the contemporary languages in order to minimize the impact of one’s analytical choices on the validity of our statements. The transcription system used is the International Phonetic Alphabet (IPA), with the following exceptions.\footnote{These exceptions do not apply to narrow transcriptions, for which IPA is used.} The character \intxt{å} is employed for the back unrounded vowel /ɑ/ of Nivaĉle, ’Weenhayek, Vejoz, Proto-Chorote, Proto-Wichí, and Proto-Mataguayan in order to avoid confusion between the italic letterforms of \intxt{a} and \intxt{ɑ}. Similarly, \intxt{ä} is used for the near-low front unrounded vowel /æ/ of Proto-Mataguayan (and for the allophone [æ], occasionally found in Manjui) in order to avoid confusion between the italic letterforms of \intxt{æ} and \intxt{œ}. The character \intxt{β} stands for the labial approximant (IPA~/β̞/) of Nivaĉle in order to reduce the use of diacritics; note that there are no voiced fricatives in the Mataguayan languages. We also use the symbol \intxt{k͡l} for the dorsal–coronal laterally released stop of Nivaĉle (IPA /k͡ɫ/). The affricates are written without the tie diacritic for legibility purposes. Finally, the function of the acute accent depends on the language: it denotes stress in Chorote and Nivaĉle, long vowels in ’Weenhayek and Proto-Wichí, and in Proto-Mataguayan it indicates the abstract category of ``accent'', which mostly corresponds to stress in Chorote and Nivaĉle and to vowel length in ’Weenhayek and Proto-Wichí. The IPA characters \intxt{ˈ} and \intxt{ˌ} denote, respectively, primary and secondary accent in languages other than Chorote and Nivaĉle.

When citing data from individual Mataguayan languages, we opt for a relatively shallow level of representation, which in most cases corresponds quite straightforwardly to the orthographies used by their speakers. In some cases, this may result in representing a greater degree of phonetic detail than is actually contrastive in the respective languages (especially in ’Weenhayek and in the Chorote lects). A major advantage of this approach is that it spares us the need to use archiphonemes in forms where some distinctions are neutralized. It also ensures comparability of the data and allows us to eschew the need to choose between conflicting analyses of the same linguistic phenomena. Finally, this decision makes it easier for the reader to track sound changes that have applied in any specific form.

We employ capital letters as wildcard characters for natural classes of Proto-Mataguayan sounds. The complete list is as follows: \intxt{A}~=~low vowel, \intxt{C}~=~consonant, \intxt{F}~=~fricative, \intxt{L}~=~coronal, \intxt{M}~=~sonorant, \intxt{P}~=~stop, \intxt{V}~=~vowel, \intxt{W}~=~labial, \intxt{X}~=~guttural fricative. The term ``guttural'' in this book is used to refer to velar, uvular, and glottal segments, whereas the term ``dorsal'' refers to velar and uvular segments only (note that this usage differs from \cits{PVB02} terminology, who uses the term ``dorsal'' to refer to /h/ alongside velars and uvulars). We assume the feature [±grave] in order to capture the shared phonological behavior of labial and dorsal consonants as opposed to coronals.

A final remark is due on the representation of the glottal stop in what is usually analyzed as onsetless syllables. In most, if not all, Mataguayan lects, a phonetic glottal stop [ʔ] appears to be automatically inserted in any syllable which would otherwise lack an onset, as in Lower Bermejeño /inot/ [ʔiˈnot] ‘water’. Note, however, that in all Mataguayan languages there are morphemes whose underlying representations demonstrably start with a glottal stop (e.g. \word{PM}{*\mbox{-}ʔåx}{skin, bark} and its reflexes), which are opposed to morphemes whose underlying representations start with a vowel (e.g. \word{PM}{*\mbox{-}åq}{food} and its reflexes), as is evident from the interaction of these morphemes with the material attached to their left (\sectref{proto-glottal}, \sectref{glott-status}). Word-initially, \intxt{∅} (absence of an onset) and /ʔ/ are neutralized in favor of [ʔ] in the Mataguayan languages; we represent such instances of [ʔ] as \intxt{ʔ}. Even if some, most, or all instances thereof turn out to be ultimately epenthetic, representing them as actual segments is useful because they may be subject to sound change in some languages (notably in Wichí, where \intxt{*ʔ} dissimilated to \intxt{*h} in certain environments; see \sectref{wi-glottal-dissim}).

\subsection{Special characters}

Asterisked forms (such as \intxt{*\mbox{-}teʔ}) refer either to reconstructions or to hypothetical forms suggested by one’s expectations but contradicted by the actual data. Two asterisks are used for hypothetical reconstructions contradicted by the comparative data (as in “the reflexes in the daughter languages point to the reconstruction \intxt{*kajáh} rather than to the expected form \intxt{**hóhkajah}”). Slashes and brackets are used for phonological and phonetic representations, respectively, including reconstructed forms (for example, */k/ *[k̟]). Forms cited verbatim after premodern sources are given in chevrons (for example, Mk~‹hipès› ‘hand’). The symbol \recvar is used to separate alternative reconstructed forms where the evidence from the daughter lects is conflicting (some lects point to one reconstructed form, whereas other lects suggest a different reconstruction). By contrast, the symbol \recind is employed when the evidence from the daughter lects is insufficient to choose between two or more possible reconstructions. In addition, {\recind} is used when two or more forms are synchronically attested in a given lect as variants.

Much of the discussion in this paper is based on analyzing cognate sets. In some cases, a given form is not synchronically segmentable, but only a part of it is cognate with the material of other languages. The part which is deemed noncognate is then given in angle brackets, as in \intxt{*\mbox{-}lá<hwah>}.

The Mataguayan languages make a clear-cut distinction between \conc{absolute} (unpossessable without additional morphology) and \conc{relational} (obligatorily possessed) nominal stems \citep{APS-AN-nd}. Since relational stems always take a prefixal person index, they are given with a hyphen at the left margin of the stem. That way, the notation \word{PM}{*\mbox{-}éj}{name} signifies that the stem in question could not occur without a possessor in Proto-Mataguayan, and it needed to combine with a person index in order to constitute a valid wordform (as in \word{PM}{*j\mbox{-}éj}{my name}, \wordnl{*ʔ\mbox{-}éj}{your name}, \wordnl{*ɬ\mbox{-}éj}{her/his name}). Conversely, absolute nominal stems are given without a hyphen at the left margin, as in \word{PCh}{*kéɬ}{nasal mucus, cold}, implying that imaginary forms such as \word{PCh}{**ʔi\mbox{-}kéɬ}{my nasal mucus}, \wordnl{**ʔa\mbox{-}kéɬ}{your nasal mucus}, \wordnl{**hᵊ\mbox{-}kéɬ}{her/his nasal mucus} were not possible according to our reconstruction. For a handful of nominal stems, the expression of a possessor is optional; these are called \conc{relationally labile} stems. These are given with a hyphen enclosed in parentheses. For examples, \word{Mk}{(\mbox{-})fiɬik}{drum} signifies that the root \intxt{fiɬik} in Maká can occur both on its own and with prefixal person indices (as in \wordnl{ji\mbox{-}fiɬik}{my drum}). Such stems are a minority in the Mataguayan languages.

\subsection{Plurals}

In all Mataguayan languages, noun pluralization is attained by means of adding a plural suffix to the stem. There are multiple plural suffixes in each language, and the choice of a particular suffix is lexicalized to a great extent. Moreover, the accretion of a plural suffix often triggers alternations of different types in the stem, such as vowel syncope or metathesis, velar stop spirantization or deletion, and deglottalization, as shown in \REF{nivplurals}.

\booltrue{listing}
\ea \label{nivplurals}
    Nivaĉle
    \begin{xlist}
        \ex \wordnl{-k͡lutseʃ}{bow, gun} → \wordnl{-k͡lutsxe-s}{bows, guns}
        \ex \wordnl{jitsuˀx}{male} → \wordnl{jitsx-åj}{males}
        \ex \wordnl{maˀnuˀk}{Manjui.\SG} → \wordnl{manxu-j}{Manjui.\PL}
        \ex \wordnl{nijåk}{cord, rope} → \wordnl{nijxå-j}{cords, ropes}
        \ex \wordnl{jinkåˀp}{year} → \wordnl{jinkåp-es}{years}
    \end{xlist}
\z
\boolfalse{listing}
    
The application of the internal reconstruction method to such alternations in Nivaĉle by \citet[5--10]{LC-VG-07} unveiled a number of sound changes, which the authors attribute to the so called “pre-Nivaĉle” (“pre-Chulupí”) stage. It must be said, however, that analogous alternations are found not only in Nivaĉle, but also in all other Mataguayan languages. In this book, we assume that most of the sound changes reconstructed by \citet{LC-VG-07} based on the Nivaĉle data (i.e., the vowel syncope, the glottal stop deletion, and the velar stop spirantization) had already been complete by the Proto-Mataguayan stage. We thus reconstruct separately the singular and the plural Proto-Mataguayan forms for every noun for which it is possible.

In this book, the plural form is given after the singular form separated by a comma. For example, “\wordng{Ni}{nijåk\plf{nijxå\mbox{-}j}}” is to be read as “\wordng{Nivaĉle}{nijåk} (singular), \intxt{nijxå\mbox{-}j} (plural)”. If the accretion of a plural suffix causes no changes in the stem, only the form of the suffix is given after the stem, enclosed in parentheses. For example, “\wordng{'Wk}{\mbox{-}ɬ\mbox{-}úp\pl{is}}” stands for “\wordng{’Weenhayek}{\mbox{-}ɬ\mbox{-}úp} (singular), \intxt{\mbox{-}ɬ\mbox{-}úp\mbox{-}is} (plural)”. This notation is also used for the stems ending in \intxt{\mbox{-}ʔ}, which is always lost before a plural suffix (\sectref{glott-loss-suff}): “\wordng{Ni}{\mbox{-}ɬaʔ\pl{s}}” is to be read as “\wordng{Nivaĉle}{\mbox{-}ɬaʔ} (singular), \intxt{\mbox{-}ɬa\mbox{-}s} (plural)”.

\subsection{Allomorphy of the third-person index in verbs}

In verbs, it is sometimes useful to specify the allomorph of the third-person prefix they select for. In our notation, it is enclosed in square brackets immediately before the stem. For example, “\wordng{LB}{[ʔi]lon}” is to be read “\wordng{Lower Bermejeño}{\mbox{-}lon}, third person \intxt{ʔi\mbox{-}lon}”. In Chorote, the third-person prefix \intxt{ʔi\mbox{-}} often causes the palatalization of the initial consonant of the stem; in such cases, we give both the form inflected for the third person (with the prefix enclosed in square brackets) and the bare stem, with no palatalization effect, as in “\wordng{Mj}{[ʔi]lʲén / \mbox{-}lán}” (to be read as “\wordng{Manjui}{\mbox{-}lán}, third person \intxt{ʔi\mbox{-}lʲén}”). In a handful of irregular verbs, the third-person form (as well as any other irregular forms) is spelled out separately, as in \wordng{PM}{*\mbox{-}åp}, 3~\wordnl{*ˀ[j]ip}{to cry} (to be read as “\wordng{Proto-Mataguayan}{*\mbox{-}åp}, third person \intxt{*ˀj\mbox{-}ip}”).

In nouns, the choice of the allomorph of the third-person prefix is usually predictable (at least in Proto-Mataguayan and in some daughter languages), so we do not spell it out. It should be noted, however, that in some words -- especially those that denote parts of animals or plants -- the third-person prefix tends to fossilize to the stem in some languages; alternatively, it may remain analyzable, but the form inflected for the third person is the only one actually in use. Such cases will be commented on explicitly in \chapref{etymdic}.

\subsection{Glottonyms}

We have standardized the choice and the spelling of the glottonyms throughout this book in order to warrant consistency. That way, we always refer to the Nivaĉle language as \emph{Nivaĉle} (and not as \emph{Nivacle}, \emph{Niwaklé}, \emph{Chulupí}, \emph{Ashlushlay}, or \emph{Suhin}), even if the cited source uses an alternative name or spelling. In general, in-prose mentions of specific (proto-)languages and dialects in this book refer to each lect by its full name. At the same time, we employ \hyperref[abbr]{abbreviated glottonyms} when they are not syntactically integrated into the prose (for example, when presenting linguistic data).

\section{Structure of this book}\label{structure}

In Part I, we put forward a detailed proposal regarding the phonological reconstruction of Proto-Mataguayan. It contains four chapters, each dealing with a separate aspect of PM phonology: the reconstruction of consonants (\chapref{pm-consonants}), vowels (\chapref{pm-vowels}), word-level prosody (\chapref{prosody}), and morphophonological alternations (\chapref{pm-processes}). In each chapter, we provide a declarative account of the reconstructed inventory of segments and phonological processes that were synchronically active in Proto-Mataguayan. We then proceed to examine the sound correspondences on which our reconstruction is based. For non-trivial reconstructive decisions, a justification is provided.

In Part II, we outline the phonological evolution of each Mataguayan language all the way from Proto-Mataguayan to the contemporary lects. It contains four chapters, one on Maká (\chapref{mk}), one on Nivaĉle (\chapref{ni}), one on Chorote (\chapref{ch}), and one on Wichí (\chapref{wi}). For Nivaĉle, Chorote, and Wichí, we also provide a detailed description of the sound changes which have led to the diversification of Proto-Nivaĉle, Proto-Chorote, and Proto-Wichí.

Part III contains the Mataguayan etymological dictionary (\chapref{etymdic}), where we list the cognate sets on which our reconstruction is based. Each entry includes the reconstructed form (and some diagnostic inflected forms, when applicable); its gloss; its reflexes in each daughter variety (including Proto-Chorote and Proto-Wichí) with the respective sources; comments on irregular developments, non-trivial reconstructive decisions, and rejected cognates; comments on similar forms in the Guaicuruan languages; and references to earlier comparative studies when available.

We conclude the book by summarizing the main findings of the preceding chapters and the differences between our proposal and earlier ones (\chapref{concl}). We also discuss the distribution of the innovations identified in the chapters of Part II, and conclude that Chorote and Wichí likely form a valid clade of the family, whereas Nivaĉle shares some innovations with Chorote--Wichí and others with Maká, making its classification dubious. Finally, we briefly comment on the possible external relations of the Mataguayan family.
