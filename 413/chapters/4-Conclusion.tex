\chapter{Conclusion}\label{concl}

In this book, we put forward a phonological reconstruction of Proto-Mataguayan, and show the main developments from the protolanguage to the daughter languages, including the intermediate protolanguages such as Proto-Wichí and Proto-Chorote. In addition, we compiled a short etymological dictionary, which contains several hundred lexical and morphological entries with Proto-Mataguayan reconstructed etyma and their reflexes in the daughter languages and dialects.

Regarding the consonantal system of Proto-Mataguayan, our study by and large supports \cits{PVB02} findings, including the reconstruction of three ``dorsal'' fricatives (\intxt{*x}, \intxt{*χ}, \intxt{*h}). We depart from previous reconstructions in positing \intxt{*ɸ} instead of \intxt{*xʷ}, thus rendering the reconstructed inventory more symmetrical and accounting in an elegant way for the correspondence between \sound{Mk}{f’} and \sound{Ni/PCh/PW}{(*)p’}. We also find solid evidence for \intxt{*ʔ} as a Proto-Mataguayan phoneme, supporting \cits{AnG-VN-21} hypothesis. We reconstruct a glottalized counterpart for every plain supraglottal consonant except the dorsal fricatives. Although in many cases it is possible to derive them from underlying clusters of the shape */Cʔ/, there is evidence that \intxt{*ˀl} and \intxt{*ˀm} are phonologically different from \intxt{*lʔ} and \intxt{*mʔ} in Proto-Mataguayan as well as in the modern languages. Contrastive (pre)glottalization may also be reconstructed in the coda position, though in this case, too, it is possible to represent the preglottalized codas as sequences of the type */ʔC/, as proposed by \citet{AnG16b} for Nivaĉle. There is evidence for tautosyllabic consonant clusters of the structure */CX/ (where \intxt{X} stands for a velar, uvular, or glottal fricative), which have given rise to aspirated consonants in Wichí. Other types of tautosyllabic consonant clusters are reconstructed primarily based on evidence from Maká and Nivaĉle. In general, our proposal differs from the extant reconstructions of Proto-Mataguayan consonants in that our reconstructed inventory is quite symmetrical, and in that the development of each phoneme in the daughter languages can now be accounted for without major exceptions or irregularities.

As for the vowels, alongside the six ones of previous reconstructions (\intxt{*i}, \intxt{*e}, \intxt{*a}, \intxt{*å}, \intxt{*o}, \intxt{*u}) we posit a seventh vowel, \intxt{*ä}. This putative vowel accounts for the correspondence between \sound{Ni}{a} and \sound{Mk/PCh/PW}{*e}. We leave open the question whether it was a truly distinct phoneme in the protolanguage. At present, we cannot discard the possibility that the instances of \intxt{*ä} in our proposal should be reconstructed with \intxt{*a} instead, though we are currently unable to formulate the environment where \intxt{*a} would have yielded \intxt{*e} in Proto-Chorote and Proto-Wichí.

Another novelty of our proposal is the reconstruction of the prosodic system of Proto-Mataguayan (\chapref{prosody}), which has not been previously attempted. Our proposal is mainly based on evidence from Chorote, the ’Weenhayek dialect of Wichí, and Nivaĉle (the evidence from the latter language is rather limited, however). There is also limited evidence from the Lower Bermejeño variety of Wichí and Nivaĉle, which consists of a partial correlation between the position of the accent and deglottalization (loss of \intxt{*ʔ} or preglottalization in codas). The precise nature of the Proto-Mataguayan accent is still far from clear. Phonetically, its reflexes include stress (in Chorote and Nivaĉle) and vowel length (’Weenhayek).

We also describe the phonological innovations that characterize each Mataguayan language. Some of them are shared between two or three languages, providing grounds for establishing clades within Mataguayan, as detailed below.

There are multiple innovations shared by Wichí and Chorote, supporting the existence of a Chorote–Wichí clade within Mataguayan, as identified in our lexicostatistic survey (\sectref{intro-lexicostat}) and suggested in previous research \citep[296]{AF05,LC-VG-07,PVB13a}. Among the processes exclusively shared by Chorote and Wichí are sound changes such as the merger of the three dorsal fricatives as \intxt{*h} in simplex onsets and, with some provisos, in complex onsets (\sectref{ch-j-jj}, \sectref{wi-jj-j-h}); the glottal dissimilation (\sectref{ch-glot-dissim}, \sectref{wi-glot-dissim}); the merger of \sound{PM}{*ä} and \intxt{*e} as \intxt{*e} (\sectref{pm-ch-ae}, \sectref{pm-wi-ae}); the lowering of \intxt{*i} to \intxt{*e} in the environment \intxt{*At/x…ts} (\sectref{pm-ch-atits-ates}, \sectref{pm-wi-atits-atets}), the lowering of \intxt{*i} to \intxt{*a} in the environment \intxt{*j…C'Á} (\sectref{pm-ch-ji-a}, \sectref{pm-wi-ji-ha}), and the rounding of \intxt{*e} before clusters with a labial (\sectref{pm-ch-ekw-okw}, \sectref{pm-wi-ew-ow}).\footnote{The sound change \sound{PM}{*k(’)}~>~\sound{PW}{*kʲ(’)} in onsets (\sectref{wi-q-k}) is also closely paralleled by an analogous sound change in the Chorote varieties (\sectref{ch-k}, \sectref{ch-k'}), but in Chorote this sound change must have taken place quite late, after the disintegration of the Chorote varieties and the so-called first palatalization (\sectref{ch-pal1}). Since Proto-Wichí split into dialects at a much later date than Proto-Chorote (\sectref{intro-lexicostat}), it is likely that the sound change \intxt{*k(’)}~>\intxt{*kʲ(’)} in onsets was an areal one, and affected Proto-Wichí, pre-Iyojwa’aja’ and Proto-Manjui–Iyo’awujwa’ at some point between the 7\textsuperscript{th} and 13\textsuperscript{th} centuries. It is further conceivable that Enxet Sur (a language belonging to the geographically adjacent Enlhet–Enenlhet family), where one finds [c], [cʲ], or [kʲ] corresponding to [k] in the sister languages \citep[70–73]{JE21}, was also affected by the putative areal sound change. It is, however, also possible that \intxt{*k(’)} in onsets was simply articulated as a prevelar stop [k̟(’)] in the hypothetical Proto-Chorote–Wichí language, thus facilitating the independent development to \intxt{*kʲ(’)}.}. In previous studies, the similarities between Wichí and Chorote might have been somewhat exaggerated because Chorote was mostly represented by the better-known Iyojwa’aja’ variety, known to have been in close contact with Wichí since at least 1900 (see \chapref{etymdic} for a list of possible borrowings from Wichí into Iyojwa’aja’). However, the number of cognates shared by Wichí and Chorote only, including the Manjui and Iyo’awujwa’ variations, is still considerable, and the percentage of matches on the 110-item Swadesh list between Chorote (excluding Iyojwa’aja’) and Wichí ranges between 50.50\% and 55.77\% (\sectref{intro-lexicostat}).

The position of Nivaĉle is somewhat ambiguous. On the one hand, it shares some innovations with Maká but not with other languages, such as the merger of \sound{PM}{*ä} and \intxt{*a} as \sound{Mk}{e}, \sound{Ni}{a} (\sectref{pm-ae}, \sectref{mk-a-ae}) and the glottal insertion in monosyllables (\sectref{mk-glottal-insertion}, \sectref{ni-glottal-insertion}). On the other hand, it shares some innovations with Chorote and Wichí but not with Maká, such as the fortition of the Proto-Mataguayan glottalized fricatives (phonologically possibly analyzable as tautosyllabic sequences of a fricative and a glottal stop) to glottalized stops, whereby \sound{PM}{*ɸ’}, \intxt{*ɬ’} changed to \intxt{(*)p’}, \intxt{(*)t’} (\sectref{ni-glott-fric}, \sectref{ch-glott-fric}, \sectref{wi-glott-fric}), as well as the deaffrication of \sound{PM}{*ts} to \intxt{(*)s} in the coda position (\sectref{ni-ts-s}, \sectref{ch-ts}, \sectref{wi-ts-s}). As of now, it appears impossible to decide whether Nivaĉle is genetically closer to Maká, to Chorote--Wichí, or forms a clade on its own. Our lexicostatistic survey (\sectref{intro-lexicostat}) likewise allows for all three possibilities. Given the wide popularity of the hypothesis that Nivaĉle is most closely related to Maká \citep[296]{AF05,LC-VG-07,PVB13a}, we list Maká--Nivaĉle cognates in a separate section in our etymological dictionary (\chapref{etymdic}), but it should be kept in mind that this clade is less well-supported than Chorote–Wichí.

At least two processes -- the lowering of \intxt{*e} to \intxt{(*)a} before the coda \intxt{*χ} (\sectref{mk-uvul-retr}, \sectref{pm-ch-ejj-ah}, \sectref{pm-wi-ejj-ajj}) and the loss of \intxt{*χ} after fricatives (\sectref{mk-fricative-dorsal}, \sectref{ch-consonant-dorsal}, \sectref{wi-consonant-dorsal}) -- are shared by Maká, Chorote, and Wichí to the exclusion of Nivaĉle. These sound change must have occurred independently in Maká and Chorote--Wichí, since Maká is lexically distant from Chorote and especially Wichí.

As for the temporal depth of the family, a glottochronological assessment in \sectref{intro-lexicostat} suggests that Proto-Mataguayan was likely spoken some 4,630--5,060 years before present, or 3,785--3,945 years before present if one considers that the low share of cognates between Maká and Wichí results from contact-induced vocabulary loss in one of these languages (or maybe in both) due to lexical borrowing from unknown sources. This temporal depth is comparable to that of protolanguages such as Proto-Jê.

Future studies will need to consider evidence from other domains, such as morphology and syntax, in order to arrive at a reliable subgrouping of the Mataguayan family, in particular with regard to the status of Nivaĉle.

Finally, we hope that our reconstruction will prove helpful in establishing possible genetic links with other language families of South America through a comparison of reconstructed protolanguages between themselves. In particular, we consider that the possibility of a genetic relationship with Guaicuruan is very promising, in accordance with \citet{PVB93,PVB13a}. Other candidates for sister language families, even if very distantly related, include Zamucoan, Tupian, Macro-Jê, Bororoan, Cariban, Karirian, Yaathê, and Harakmbut--Katukina.