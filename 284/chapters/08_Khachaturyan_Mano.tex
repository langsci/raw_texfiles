\documentclass[output=paper]{langscibook}

\author{Maria Khachaturyan\affiliation{University of Helsinki}}
\title{Reflexive constructions in Mano} 
\abstract{This paper focuses on  reflexivity in Mano (Southern Mande). Mano has a dedicated reflexive pronoun \textit{ē} used with 3\textsc{sg} antecedents. It can be followed by the self-intensifier \textit{dìè} to form a complex reflexive. Among the highlights of the reflexivity system are (1) frequent non-subject orientation (direct objects, arguments of postpositions and subject’s possessors can serve as antecedents) challenges the current accounts of the syntax of Mande VPs; (2) the use of the intensifier cannot be explained by the semantic class of the verb alone (introverted vs extroverted), as \textit{dìè} assures a broader function of reference continuity; (3) there are marginal cases of reflexives in the subject position; and (4) against typological predictions, the intensifier \textit{dìè} can be used in middle constructions, reflexive constructions and for intensification, but not to express reciprocity.}

\IfFileExists{../localcommands.tex}{
  \usepackage{langsci-optional}
\usepackage{langsci-gb4e}
\usepackage{langsci-lgr}

\usepackage{listings}
\lstset{basicstyle=\ttfamily,tabsize=2,breaklines=true}

%added by author
% \usepackage{tipa}
\usepackage{multirow}
\graphicspath{{figures/}}
\usepackage{langsci-branding}

  
\newcommand{\sent}{\enumsentence}
\newcommand{\sents}{\eenumsentence}
\let\citeasnoun\citet

\renewcommand{\lsCoverTitleFont}[1]{\sffamily\addfontfeatures{Scale=MatchUppercase}\fontsize{44pt}{16mm}\selectfont #1}
   
  %% hyphenation points for line breaks
%% Normally, automatic hyphenation in LaTeX is very good
%% If a word is mis-hyphenated, add it to this file
%%
%% add information to TeX file before \begin{document} with:
%% %% hyphenation points for line breaks
%% Normally, automatic hyphenation in LaTeX is very good
%% If a word is mis-hyphenated, add it to this file
%%
%% add information to TeX file before \begin{document} with:
%% %% hyphenation points for line breaks
%% Normally, automatic hyphenation in LaTeX is very good
%% If a word is mis-hyphenated, add it to this file
%%
%% add information to TeX file before \begin{document} with:
%% \include{localhyphenation}
\hyphenation{
affri-ca-te
affri-ca-tes
an-no-tated
com-ple-ments
com-po-si-tio-na-li-ty
non-com-po-si-tio-na-li-ty
Gon-zá-lez
out-side
Ri-chárd
se-man-tics
STREU-SLE
Tie-de-mann
}
\hyphenation{
affri-ca-te
affri-ca-tes
an-no-tated
com-ple-ments
com-po-si-tio-na-li-ty
non-com-po-si-tio-na-li-ty
Gon-zá-lez
out-side
Ri-chárd
se-man-tics
STREU-SLE
Tie-de-mann
}
\hyphenation{
affri-ca-te
affri-ca-tes
an-no-tated
com-ple-ments
com-po-si-tio-na-li-ty
non-com-po-si-tio-na-li-ty
Gon-zá-lez
out-side
Ri-chárd
se-man-tics
STREU-SLE
Tie-de-mann
} 
  \togglepaper[1]%%chapternumber
}{}

\begin{document}
\maketitle

\section{Introduction}\label{sec:Kachaturyan:1}

Mano (\textit{máá}) is a Southern Mande language spoken by 305~000 people in Liberia and 85~000 in Guinea. It does not have an official status in the countries where it is spoken. In Guinea, Mano is a minority language, while in Liberia, it is the fifth most spoken language. Very little literature is produced in the language, with the high-quality translation of the New Testament published in Liberia as one of the exceptions (UBS 1978). 

\begin{figure}
\caption{A map of Mano}
\label{fig:Kachaturyan:1}
\end{figure}

Liberian Mano has three dialects, the Northern dialect Maalaa (\textit{máá lāā}), which is spoken around Sanniquellie; the Central dialect Maazein (\textit{máá zè\'{ŋ}}), spoken in Ganta; and the Southern dialect Maabei (\textit{máá bèí}), which is spoken in Saklepea. Guinean Mano also has three dialects, Zaan (\textit{zà̰à}), the easternmost dialect spoken around the town of Bossou, Maa (\textit{màá}), the central dialect spoken in the city of Nzérékoré and to the south of it, and Kpeinson (\textit{kpé\'{ŋ}sɔ}) spoken near Diecké. All dialects are mutually intelligible. This paper is based on the central Guinean dialect, Maa (\textit{màá}). On the dialectal situation, see \citet{Khachaturyan2018}. A grammatical description of Mano can be found in \citet{Khachaturyan2015}, for a typological portrait of the language, see \citet{KhachaturyanNodate2020}.

In Guinea, Mano is in intense contact with a South-Western Mande language, Kpelle, spoken by 460 000 people. This results in widespread and often unreciprocated bilingualism (Mano speaking Kpelle more often than the other way round) and unidirectional transfer of certain lexical \citep{KachaturyanNodate} and grammatical features \citep{Khachaturyan2019}. Contact arguably affects the reflexivity system, as well, in the speech of bilinguals and monolinguals alike. On contact between Mano and Kpelle, see \citet{KhachaturyanKonoshenkoToAppear}.

This paper is largely based on my first-hand fieldwork material from Mano, elicited (el.) or natural, coming from my oral corpus (MOC). A small number of examples are taken from the Bible translation (UBS 1978), all checked with my primary language consultant for natural sounding; the verses are marked correspondingly.

The discussion in this paper is organized as follows. In \sectref{sec:Kachaturyan:2}, I introduce the basics of Mano morphosyntax. In \sectref{sec:Kachaturyan:3}, I introduce the pronominal system, including the dedicated 3\textsc{sg} reflexive pronoun. In \sectref{sec:Kachaturyan:4}, I discuss the intensifiers used in reflexive and reciprocal constructions, in particular, \textit{dìè} which forms complex reflexive markers. \sectref{sec:Kachaturyan:5} is dedicated to the syntax of reflexivity: the coreference domain, subject-oriented and non subject-oriented uses, as well as reflexives in the subject position. In \sectref{sec:Kachaturyan:6}, I briefly discuss the valency changing function of reflexive markers. \sectref{sec:Kachaturyan:7} gives a preliminary assessment of the influence of Kpelle on Mano in the domain of reflexivity. I provide a concluding discussion of the findings in \sectref{sec:Kachaturyan:8}.


\section{Basics of Mano morphosyntax}\label{sec:Kachaturyan:2}

\subsection{Clause structure and word order}\label{sec:Kachaturyan:2.1}

Mano has a rigid word order typical of the Mande family: S Aux O V X, where Aux is an auxiliary expressing TAMP and functioning as a site of subject indexation, and X are postpositional phrases and adverbs. In \REF{ex:Kachaturyan:1}, the third person singular auxiliary, \textit{āà}, belongs to the perfect series. There are in total eleven auxiliary series occurring in different TAMP contexts. The full subject noun phrase is never obligatory \REF{ex:Kachaturyan:1b}, and reflexives can appear in clauses without overt subject noun phrase, as is typically the case of languages with pro-drop. In copular clauses, the word order is S Cop X, where the subject noun phrase is obligatory (see \REF{ex:Kachaturyan:5a} below).


 
\ea
\label{ex:Kachaturyan:1}
    \ea
    \label{ex:Kachaturyan:1a}
    \gll Pèé  āà      kɔnɔ  yà    Pólāá  sɔnɔ\\
     Pe  \textsc{3sg.prf}  food  put  Pola    near\\
    \glt ‘Pe has put the food near Pola.’ [el.]
    
    \ex
    \label{ex:Kachaturyan:1b}
    \gll āà  kɔnɔ  yà  Pólāá  sɔnɔ\\
     \textsc{3sg.prf}  food  put  \textsc{P}ola  near\\
    \glt ‘(S)he has put the food near Pola.’ [el.]
    \z
\z

Some series of auxiliaries are portmanteau forms, incorporating the 3\textsc{sg} pronominal direct object. In some cases, the portmanteau forms are distinct, as in the case of the past series (\ref{ex:Kachaturyan:2a}, \ref{ex:Kachaturyan:2b}), in some cases they coincide with non-portmanteau ones, as in the case of the perfect (\ref{ex:Kachaturyan:1b}, \ref{ex:Kachaturyan:2c}).


 
 \ea
\label{ex:Kachaturyan:2}
    \ea
    \label{ex:Kachaturyan:2a}
    \gll ē  ló\\
     3\textsc{sg.pst}  go\\
    \glt ‘(S)he went.’ [el.]

    \ex
    \label{ex:Kachaturyan:2b}
    \gll ā  yà\\
     \textsc{3sg.pst>3sg}  put\\
    \glt ‘(S)he put it.’ [el.]

    \ex
    \label{ex:Kachaturyan:2c}
    \gll āà  yà\\
     \textsc{3sg.prf>3sg}  put\\
    \glt ‘(S)he has put it.’ [el.]
    
\z
    \z

As argued in \citet{Nikitina2009}, all postpositional phrases are adjoined at the level of the clause, rather than belonging to the verb phrase (see also \citealt{Nikitina2018}). This issue presents a major challenge for the analysis of reflexivity in Mano in terms of c-command, a question that I return to  in \sectref{sec:Kachaturyan:7}.

 
\subsection{Noun phrase structure}\label{sec:Kachaturyan:2.2}

Mano has relatively limited nominal morphology, with only one productive derivational suffix (-\textit{là}, suffix of abstract nouns) and two tonal forms: high tone forms used in particular when the noun is followed by a demonstrative (\textit{gɔ̰} ‘man’, \textit{gɔ̰} \textit{wɛ } man:\textsc{h} \textsc{dem} ‘this man’) and low tone construct forms used to mark heads of noun phrases with specific preposed dependents (\textit{lēē} ‘woman’, \textit{gí} \textit{lèè} [stomach woman:\textsc{cstr}] ‘pregnant woman’). On construct forms in African languages, see \citet{CreisselsGood2018}. There is no morphological case in the language, and definiteness is not grammaticalized. Mano distinguishes between alienably and inalienably possessed nouns. Inalienable possession is expressed by juxtaposition of the possessor and possessee; the possessor can also be expressed by a basic pronoun \REF{ex:Kachaturyan:3}. Alienable possession is expressed by possessive pronouns or with full possessor NP + possessive pronoun + head noun, as seen in \REF{ex:Kachaturyan:4}. 


 
 \ea
\label{ex:Kachaturyan:3}
    \ea
    \label{ex:Kachaturyan:3a}
    \gll à  dàā\\
     \textsc{3sg}  father\\
    \glt ‘his father’

 \ex
    \label{ex:Kachaturyan:3b}
    \gll Pèé  dàā\\
     Pe  father\\
    \glt ‘Pe's father’\z
\z
    
  \ea
\label{ex:Kachaturyan:4}
    \ea
    \label{ex:Kachaturyan:4a} 
    \gll là  ká\\
     \textsc{3sg.poss}  house\\
    \glt ‘his house’

    \ex
    \label{ex:Kachaturyan:4b} 
    \gll Pèé  là  ká\\
     Pe  \textsc{3sg.poss}  house\\
    \glt ‘Pe’s house’\z
\z

Plurality is expressed by number words: one (\textit{vɔ}) for additive plural, as in \textit{gbá̰} \textit{vɔ} ‘dogs’, and one (\textit{nì})for non-additive, including associative and emphatic plural, as well as plural for kinship terms, as in \textit{dàā} \textit{nì} ‘fathers’ (father and his kins). The word order in noun phrases is typically genitival dependent – head noun – adjective – numeral – determinative. Determinatives include quantifiers, demonstratives, number words, as well as self-intensifiers, which will be discussed in detail in \sectref{sec:Kachaturyan:4}.


 \section{Pronouns}\label{sec:Kachaturyan:3}

 \subsection{Personal pronouns}\label{sec:Kachaturyan:3.1}

Mano has five series of pronominal forms used in different syntactic contexts: (1) basic pronouns, used in non-subject argument positions (direct object, argument of postposition, inalienable possessor, \REF{ex:Kachaturyan:5a}; (2) possessive pronouns used to express alienable possessors \REF{ex:Kachaturyan:5b}; (3) emphatic pronouns used for emphasis as well as for NP coordination \REF{ex:Kachaturyan:5c}; (4) high-tone pronouns used in the same contexts as high-tone nouns \REF{ex:Kachaturyan:5d}; and (5) inclusory pronouns used as heads in inclusory constructions \REF{ex:Kachaturyan:5e}. There are no subject pronouns, as auxiliaries are the sites of subject indexation. All pronouns distinguish between two numbers and three persons, with the exception of inclusory pronouns, which have only plural forms. Pronominal forms are given in \tabref{tab:Kachaturyan:1}.

\begin{table}
\begin{tabularx}{\textwidth}{p{3.1cm}p{0.8cm}p{0.8cm}p{2cm}p{1.2cm}p{0.8cm}p{0.8cm}} 
\lsptoprule
&\textsc{1sg} & \textsc{2sg} & \textsc{3sg} & \textsc{1pl} & \textsc{2pl} & \textsc{3pl}\\
\midrule
(1) basic pronouns & {\={ŋ}} & {ī} & {à} {/} {ā} {/} {á}\footnote{The tone of the \textsc{3sg} basic pronoun optionally assimilates to the tone of the preceding vowel.} & {kō} & {kā} & {ō}\\
(2) possessive pronouns & {\`{ŋ}} & {ɓà} & {là} & {kò} & {kà} & {wà}\\
(3) emphatic pronouns & {mā(ē)} & {ɓī(ē)} & à, (à)yē, (à)yé, yō & {kō(ē)} & {kā(ē)} & {ō(ē)}\\
(4) high-tone pronouns & {má} & {ɓí} & {(à)yé} & {kó} & {ká} & {ó}\\
(5) inclusory pronouns &  &  &  & {kò{\textasciitilde}kwà} & {kà} & {wà}\\
\lspbottomrule
\end{tabularx}
\caption{Personal pronouns in Mano}
\label{tab:Kachaturyan:1}
\end{table}


 \ea
\label{ex:Kachaturyan:5}
    \ea
    \label{ex:Kachaturyan:5a} 
    \gll ɲɛɛ  kɛ  mìà  wɔ  \textbf{ō}  ká\\
     fetish  do  person.\textsc{pl:cstr}  \textsc{cop.neg}  \textbf{\textsc{3pl}}  with\\
    \glt ‘They are not witches (lit.: fetish-doing-people they aren’t).’ [MOC]
 
    \ex
    \label{ex:Kachaturyan:5b} 
    \gll ō  \textbf{wà}    ká    dɔ\\
     3\textsc{pl.pst}  \textbf{\textsc{3pl.poss}}  house    build\\
    \glt ‘They\textsubscript{i} build their\textsubscript{i,j} house.’ [el.]
 
    \ex
    \label{ex:Kachaturyan:5c} 
    \gll \textbf{ōē}  ō  kɛɛ  lɛɛ  ɓɔ  nɛ  pèèlɛ  mɔ\\
     \textbf{3\textsc{pl.emph}  }\textsc{3pl}  year  3\textsc{sg.neg}  go.out  not.yet  two  on\\
    \glt ‘those (of them) who haven’t yet reached two years’ [Matthew 2:16; UBS 1978]

    \ex
    \label{ex:Kachaturyan:5d}
    \gll \textbf{ó}  ā,  ō  mɛ  ē  sí\\
     \textbf{3\textsc{pl.h}} \textsc{dem} \textsc{3pl} surface \textsc{3sg.pst} take\\
\glt ‘Those ones, they were cleansed.’ [el.]

    \ex
    \label{ex:Kachaturyan:5e}
    \gll gbóó-wè  \textbf{wà}    mīā  gbɛɛ-wè\\
     sobbing-speech:\textsc{cstr}    \textbf{\textsc{3pl.ip}}  person\textsc{.pl}  cry-speech:\textsc{cstr}\\
\glt ‘sobbing and people’s crying’ [Matthew 2:18; UBS 1978]
\z
\z

All transitive verbs are obligatorily used with a direct object, a noun phrase or a pronoun. In speech reports, a dummy pronoun is used: it is impossible to use a speech verb without a 3\textsc{sg} direct object pronoun. A typical introduction of a report would be \textit{láà} \textit{gèē} ‘(s)he is saying it’, followed by the reported discourse (see \ref{ex:Kachaturyan:19} and \ref{ex:Kachaturyan:29}). Thus, 3\textsc{sg} pronouns are not always referential.


 
 \subsection{Reflexive pronoun and basic pronouns in the reflexive function}\label{sec:Kachaturyan:3.2}

Mano has a dedicated \textsc{3sg} reflexive pronoun \textit{ē} which is used in the same positions as the basic pronouns: in the direct object position \REF{ex:Kachaturyan:6}, as well as the argument of postposition and as inalienable possessor. It is used with a third person sg. antecedent withint the same minimal finite clause (see \sectref{sec:Kachaturyan:5.1}), being in quasi-complementary distribution with the \textsc{3sg} basic pronoun \textit{à} (\ref{ex:Kachaturyan:6a}, \REF{ex:Kachaturyan:6b}, but see \sectref{sec:Kachaturyan:5.2.2} and \sectref{sec:Kachaturyan:7}) and is typically not used with antecedents other than \textsc{3sg} \REF{ex:Kachaturyan:6d}. In other persons and numbers, there aren’t any dedicated reflexives and instead basic pronouns are used in the reflexive function \REF{ex:Kachaturyan:6c}, in particular, the 3pl pronoun \textit{ō} which, unless it is accompanied by a self-intensifier (see \sectref{sec:Kachaturyan:4}), routinely has ambiguity between coreferential and disjoint readings \REF{ex:Kachaturyan:6e}. Thus, the paradigm of pronouns used in the contexts of coreferentiality between two arguments in the same clause constists of the basic pronouns plus the reflexive \textsc{3sg} pronoun \textit{ē} (on the \textsc{3sg} basic pronoun \textit{à} in that function, see \sectref{sec:Kachaturyan:5.2.2}).


\ea
\label{ex:Kachaturyan:6}
    \ea
    \label{ex:Kachaturyan:6a}  
\gll ē  \textbf{ē}  gḭ̀ḭ̄  ~\\
     \textsc{3sg.pst}  \textbf{\textsc{3sg.refl}}  wound\\
\glt ‘She wounded herself.’ [el.]

\ex
    \label{ex:Kachaturyan:6b}  
\gll ē  \textbf{à}  gḭ̀ḭ̄.  ~\\
     \textsc{3sg.pst}  \textbf{\textsc{3sg}}  wound\\
\glt ‘She wounded him.’ [el.]

\ex
    \label{ex:Kachaturyan:6c}  
\gll kō  \textbf{kō}  gḭ̀ḭ̄  ~\\
     \textsc{1pl.pst}  \textbf{\textsc{1pl}}  wound\\
\glt ‘We wounded ourselves.’ [el.]

\ex
    \label{ex:Kachaturyan:6d}  
\gll *kō  \textbf{ē}  gḭ̀ḭ̄\\
     \textsc{1pl.pst}  \textbf{\textsc{3sg.refl}}  wound\\
\glt (Intended reading: ‘We wounded ourselves.’) [el.]

\ex
    \label{ex:Kachaturyan:6e}  
\gll ō  \textbf{ō}  gḭ̀ḭ̄.~\\
     \textsc{3pl.pst}  \textbf{\textsc{3pl}}  wound\\
\glt ‘They wounded themselves/them.’ [el.]\z
\z

In some rare cases the reflexive pronoun can be used with other types of antecedents \REF{ex:Kachaturyan:7a}--\REF{ex:Kachaturyan:7b}. It can also sometimes be used without any antecedent, in a non-referential function, as in \REF{ex:Kachaturyan:8} where it occurs with the adjective \textit{yīè} ‘good’ in a comitative postpositional phrase whose overall meaning is adverbial, ‘well’. The exact contexts where there is a mismatch between the person and number value of the 3\textsc{sg} reflexive pronoun \textit{ē} and the antecedent require further investigation.


 
\ea
\label{ex:Kachaturyan:7}
    \ea
    \label{ex:Kachaturyan:7a} 
\gll kɔáà  wálà   pɛ̰  \textbf{ē}  \textbf{kíè}  bà \\
     \textsc{1pl.jnt}  God  pray\textsc{:jnt}  \textbf{\textsc{3sg.refl}}  \textbf{\textsc{recp}  }in \\
\glt ‘We pray together.’ [MOC] 
\ex
    \label{ex:Kachaturyan:7b} 
\gll kō  \textbf{kō}  \textbf{kíè}  bà\\
     1\textsc{pl.exi}  \textbf{\textsc{1pl}}  \textbf{\textsc{recp}}  in\\
\glt ‘We are together.’ [MOC]
\z
\z

\ea
    \label{ex:Kachaturyan:8} 
 \gll ō  ō  kɔ  yà  à  wì  \textbf{ē}  yīè  ká\\
     3\textsc{pl.pst}  \textsc{3pl}  had  put  \textsc{3sg}  under    \textbf{3\textsc{sg.refl}}  good  with\\
\glt ‘They welcomed him very well (lit.: with its goodness).’ [MOC]
\z


 \section{Reflexive and reciprocal determinatives}\label{sec:Kachaturyan:4}


 \subsection{Self-intensifier \textit{dìè} and complex reflexive markers}\label{sec:Kachaturyan:4.1}

Basic and reflexive pronouns can be accompanied by determinatives: \textit{dìè}, \textit{kíè} and \textit{zì}.

\textit{Dìè} is an intensifier, somewhat similar to English \textit{himself}, as in “The President himself came”. It derives from the adjective \textit{dìè} ‘true’.

\ea
    \label{ex:Kachaturyan:9} 
 \gll kɛ  kō  mììdàāmì  \textbf{dìè}  là  tíé  wɛ  é  kṵ́  kō  zò  píé\\
     so.that  \textsc{1pl}  Lord\textbf{  \textbf{int}}  \textsc{3sg.poss}  fire  \textsc{dem}  \textsc{3sg.conj}  catch  \textsc{1pl}  heart  at  \\
\glt ‘So that the fire of our Lord himself ignites in our hearts (lit.: the fire of our lord ignites in our hearts).’ [MOC]
\z

Crucially, \textit{dìè} can also be used with the reflexive \REF{ex:Kachaturyan:10a} and with basic personal pronouns (\ref{ex:Kachaturyan:10b}, c) to form complex (as opposed to simplex, bare pronouns) reflexive markers. While the basic 3pl pronoun is ambiguous between the coreferential and the disjoint readings \REF{ex:Kachaturyan:6e}, the complex marker \textit{ō} \textit{dìè} is unambiguously coreferential \REF{ex:Kachaturyan:10b}.


 
 \ea
    \label{ex:Kachaturyan:10} 
     \ea
    \label{ex:Kachaturyan:10a} 
\gll lɛ  bḭ́–pɛlɛ  \textbf{ē}  \textbf{dìè}  mɔ.\\
     \textsc{3sg.exi}  touch–\textsc{inf}  \textsc{3sg.refl}  \textsc{int}  on\\
\glt ‘He touches himself.’ [el.]

\ex 
\label{ex:Kachaturyan:10b}
\gll ō  bḭ́–pɛlɛ  \textbf{ō}  \textbf{dìè}  mɔ.\\
     \textsc{3pl.exi}  touch–\textsc{inf}  \textbf{\textsc{3pl}}  \textbf{\textsc{int}}  on\\
      \glt ‘They\textsubscript{i} touch themselves\textsubscript{i}/*them\textsubscript{j}’ [el.]

 \ex
\label{ex:Kachaturyan:10c}
\gll kō  bḭ́–pɛlɛ  \textbf{kō}  \textbf{dìè}  mɔ.\\
     \textsc{1pl.exi}  touch–\textsc{inf}  \textbf{\textsc{1pl}}  \textbf{\textsc{int}}  on\\
    \glt ‘We touch ourselves.’ [el.]\\
     \z
\z

 
 \subsection{Complex vs simplex reflexive markers}

While the complex reflexive marker – pronoun + \textit{dìè} – is always possible, there are some restrictions on the use of the simple reflexive and basic personal pronouns in reflexive contexts. In the direct object position, the simplex marker is acceptable with verbs such as \textit{zúlú} ‘wash’, \textit{gḭ̀ḭ̄} ‘hurt’, \textit{gélé} ‘burn’, \textit{bḭ̀ḭ̀} ‘hide’, \textit{kṵ́} ‘warm up’, \textit{mìīmíí} ‘move’. The simplex marker is marginally accepted with verbs such as \textit{lī} ‘make beautiful’, \textit{mɛ} ‘beat’, \textit{zɔ̰ɔ̰} ‘show’, \textit{dà} ‘drop’, \textit{gɔ̰} ‘fight against’, \textit{gɛ̰} ‘see’. The simplex marker is even less acceptable with verbs such as \textit{fòlō} ‘detach’, \textit{gɛ̰} ‘consider’, \textit{dɔkɛ} ‘give’, \textit{tɛnɛ} ‘appreciate’, \textit{kpàā} ‘annoy’. Corpus data partially confirms elicitation: simplex reflexive was amply attested with the verb \textit{zúlú} ‘wash’, while the complex one was attested with \textit{gélé} ‘burn’, \textit{zɔ̰ɔ̰} ‘show’, \textit{kɛ} ‘make, become’, \textit{tɛnɛ} ‘raise’, \textit{fɔɔ} ‘inflate‘ (in the reflexive context, means ‘to swagger‘), \textit{sí} ‘take’ (in the reflexive context, means ‘to boast’), \textit{sɔlɔ} \textit{ɓō} ‘obtain’ (in the reflexive context, means ‘to become fully formed, developed’). 

\ea
    \label{ex:Kachaturyan:11} 
 \gll lɔkɛmɔ  ɔ  yē  wɔ  mīī  í  \textbf{ī}  \textbf{dìè}  tɛnɛ,  í  \textbf{ī}  \textbf{dìè}  fɔɔ\\
     love  \textsc{dem}  \textsc{3sg.emph}  \textsc{cop.neg}  person  \textsc{2sg.conj}  \textbf{\textsc{2sg}  \textbf{int}}  raise  \textsc{2sg.conj}  \textbf{\textsc{2sg}  \textbf{int}}  swell\\
\glt ‘Love, it isn’t (, like,) man, you should raise yourself, you should swagger (lit.: inflate yourself).’ (MOC)
\z

The rules of distribution between the simplex and the complex markers in the direct object position require further investigation; so far, it seems that the verbs used with simplex and complex markers cannot be neatly divided into introverted and extroverted classes, respectively, as it is the case in some other languages (\citealt{KoenigVezzosi2004}). 

In oblique argument positions expressed with postpositional phrases, the complex marker is usually preferred. However, simplex marker is also marginally possible with the verbs \textit{nāā} ‘love’, \textit{yɛ} ‘stab’, \textit{tā̰ā̰} ‘annoy’, \textit{gbṵ̄} ‘help’. The simplex marker is inacceptable with the verbs \textit{túó} ‘frighten’, \textit{pá} ‘touch’, \textit{nū} ‘bring’, \textit{lèmā} ‘forget’. 

In the benefactive context, both complex and simplex markers are acceptable.

\ea
    \label{ex:Kachaturyan:12} 
 \gll Pèé  āà  ká  lɔ  \textbf{ē}  \textbf{(dìè)}  lɛɛ  \\
     Pe  \textsc{3sg.prf}  house    buy  \textbf{\textsc{3sg.refl}}  \textbf{\textsc{int}}  \textsc{pp}  \\
\glt ‘Pe bought a house for himself.’
\z

In non-argumental, locative PPs simplex markers seem to be preferred, at least according to the corpus where they occur more frequently than the complex ones. 

\ea
    \label{ex:Kachaturyan:13} 
 \gll é  ló  \textbf{ē}  mɛ\'{ŋ}\\
     3\textsc{sg.conj}    go  \textbf{\textsc{3sg.refl}}  behind\\
\glt ‘(So that) he returns.’ [MOC]
\z

If both a complex reflexive and a simplex one can be used, \textit{dìè} adds~intentionality and emphasis.


 \ea
    \label{ex:Kachaturyan:14} 
    \ea
    \label{ex:Kachaturyan:14a}
\gll ē  \textbf{ē}  gḭ̀ḭ̄  \\
     \textsc{3sg.pst}  \textbf{\textsc{3sg.refl}}  wound\\
\glt ‘He wounded himself’ [el.]


 \ex
    \label{ex:Kachaturyan:14b} 
\gll ē  \textbf{ē}  \textbf{dìè}  gḭ̀ḭ̄  \\
     \textsc{3sg.pst}  \textbf{\textsc{3sg.refl}  \textbf{int}}  wound\\
\glt ‘He wounded himself intentionally’ [el.]
\z
\z

\ea
    \label{ex:Kachaturyan:15} 
    \ea
    \label{ex:Kachaturyan:15a} 
     \textit{Pèé}  \textit{āà}  \textit{kɔnɔ}  \textit{yà}  \textbf{\textit{ē}}  \textit{sɔnɔ}\\
     Pe  \textsc{3sg.prf}  food  put  \textbf{\textsc{3sg.refl}}  near\\
\glt ‘Pe put food near himself.’ [el.]

    \ex
    \label{ex:Kachaturyan:15b} 
\gll Pèé  āà  kɔnɔ  yà  \textbf{ē}  \textbf{dìè}  sɔnɔ\\
     Pe  \textsc{3sg.prf}  food  put  \textbf{\textsc{3sg.refl}  \textbf{int}  }near\\
\glt ‘Pe put food near himself (contrastive: there are other people around).’ [el.] \z
\z

The two functions of the self-intensifier \textit{dìè}, reflexive and non-reflexive, should be considered functions of the same lexeme. In \REF{ex:Kachaturyan:16a}, \textit{dìè} follows the reflexive pronoun \textit{ē} forming a complex reflexive pronoun. In \REF{ex:Kachaturyan:16b}, an utterance that followed \REF{ex:Kachaturyan:16a} in the recording, it occurs in the subject noun phrase, has an intensifying reading and is used with a basic 3\textsc{sg} pronoun \textit{à} with the same reference as the reflexive pronoun in the preceding clause.


 \ea
    \label{ex:Kachaturyan:16} 
    \ea
    \label{ex:Kachaturyan:16a} 
\gll lɛfùnɔɔ  ēkílíɓɛ  ē  nū  \textbf{ē}  \textbf{dìè}  pàà.\\
     light  \textsc{dem}  \textsc{3sg.pst}  come  \textbf{\textsc{3sg.refl}  \textbf{int}}  at\\
\glt ‘The light came at his own (home).’ [MOC] 

    \ex
    \label{ex:Kachaturyan:16b} 
    \gll \textbf{à}  \textbf{dìè}  pàà  mìà  òó  gbāā  ō  kɔ  yà  à  wì.\\
     \textbf{3\textsc{sg}  \textbf{int}  }at  person.\textsc{pl}:\textsc{cstr}  \textsc{3pl.neg}  \textsc{neg}  \textsc{3pl}  arm  put  3\textsc{sg}  under\\
     ‘His own people (lit.: the people at his own) did not accept it.’ [MOC]\\
    \z
\z
 
 \subsection{Reciprocal marker \textit{kíè}}\label{sec:Kachaturyan:4.2}

Reciprocal constructions are formed with basic plural pronouns followed by the reciprocal determiner \textit{kíè.}

\ea
    \label{ex:Kachaturyan:17} 
 \gll kóò  \textbf{kō}  \textbf{kíè}  gɛ̰  tòò  ɲɛnɛ  dɔkézɛ\\
     \textsc{1pl.ipfv}  \textbf{\textsc{1pl}  \textbf{recp}}  see:\textsc{ipfv}  tomorrow  hour  same\\
\glt ‘We see each other tomorrow at the same hour.’ [el.]
\z


 \section{Syntax of reflexives}\label{sec:Kachaturyan:5}


 \subsection{Coreference domain}\label{sec:Kachaturyan:5.1}

The coreference domain of Mano reflexives is always the minimal finite clause. There cannot be antecedents for reflexive markers outside the minimal clause (with the rare exception of reflexives in the subject position, see \sectref{sec:Kachaturyan:5.4}). In \REF{ex:Kachaturyan:18a}, the subject is the antecedent of a reflexive marker situated in the argument position of a gerund. In \REF{ex:Kachaturyan:18b}, the reflexive marker is situated in the dependent finite clause. There is potential ambiguity: in case the two clauses’ subjects are coreferential, the subject of the main clause appears as the antecedent of the reflexive marker, but if the subject of the dependent clause is distinct from the subject of the main clause, then it is apparent that it is the subject of the minimal finite clause, and not the main clause, that is the antecedent.

\ea
    \label{ex:Kachaturyan:18} 
    \ea
    \label{ex:Kachaturyan:18a} 
\gll lɛɛ  nàà  \textbf{${\emptyset}$} bḭ́–à̰  ká  \textbf{ē}  \textbf{dìè}  mɔ.\\
     3\textsc{sg}.\textsc{ipfv}  want:\textsc{ipfv}  \textbf{\textsc{pro}}  touch–\textsc{ger}  with  \textbf{3\textsc{sg}}\textbf{.\textsc{refl}}  \textbf{\textsc{int}}  on\\
\glt ‘He\textsubscript{i} wants to touch himself\textsubscript{i}.’ [el.]
    
    \ex
    \label{ex:Kachaturyan:18b}
    \gll lɛɛ  nàà  \textbf{é}  bḭ́  \textbf{ē}  \textbf{dìè}  mɔ.  \\
     3\textsc{sg}.\textsc{ipfv}  want:\textsc{ipfv}  \textbf{3\textsc{sg}}\textbf{.\textsc{conj}}  touch  \textbf{3\textsc{sg}}\textbf{.\textsc{refl}}  \textbf{\textsc{int}}  on\\
\glt ‘She\textsubscript{i} wants to touch (lit.: that she\textsubscript{i} touches) herself\textsubscript{i}/ She\textsubscript{i} wants that he\textsubscript{j} touches himself\textsubscript{j}/*her\textsubscript{i}’ [el.]
\z
\z


To express coreference between the subject of the main clause and a pronoun in the finite dependent clause, the basic pronoun \textit{à} has to be employed. However, the intensifier \textit{dìè} is often added in such cases to mark that the antecedent is to be found in the immediate discourse context; it may be the subject of the main clause \REF{ex:Kachaturyan:19} or some other prominent referent \REF{ex:Kachaturyan:20}.

\ea
    \label{ex:Kachaturyan:19} 
 \gll Yèí  ā  gèē  Kɔɔ  lɛɛ  é  \textbf{à}  \textbf{dìè}  gɛ̰\\
     Yei  3\textsc{sg.pst>3sg}  say  Ko  \textsc{pp}  \textsc{3sg.conj}  \textbf{\textsc{3sg}  \textbf{int}}  see\\
\glt ‘Yei\textsubscript{i} said to Ko\textsubscript{j} (so that) she\textsubscript{j} looks at her\textsubscript{i}/him\textsubscript{k}/*herself\textsubscript{j.}’ [el.]
\z


\ea
    \label{ex:Kachaturyan:20} 
 \gll kɛ-ŋwɔ-yɔɔ  sé\'{ŋ}  lɛ  \textbf{mī}  à  kɛ  ɛ  à  tíé  lɛɛ  \textbf{à}  \textbf{dìè}  kṵ́\\
     do-problem:\textsc{cstr-}bad  every  \textsc{att}  \textbf{person}  3\textsc{sg.sbjv>3sg}  \textsc{bkgr}  \textsc{3sg}  fire  \textsc{3sg.neg}  \textbf{\textsc{3sg}  \textbf{int}}  catch\\
\glt “All sins that a person\textsubscript{i} commits, their fire does not catch him\textsubscript{i}.” (1 Corinthians 6:18; UBS 1978)\z

Unlike many African languages, including some very closely related, such as Dan \citep{Vydrin2017}, Mano does not have logophoric pronouns.


 
 \subsection{Subject orientation}\label{sec:Kachaturyan:5.2}

 \subsubsection{Possessive position}\label{sec:Kachaturyan:5.2.1}


The previous sections amply demonstrated the autopathic and oblique constructions with reflexive markers where the antecedent is the subject. Similarly, the reflexive pronoun can also be used in the inalienable possessor position and be coreferential with the subject. It can occur within the direct object NP \REF{ex:Kachaturyan:22} as well as within the NP occupying the role of the argument of a postposition \REF{ex:Kachaturyan:21}.

\ea
    \label{ex:Kachaturyan:21} 
 \gll máríá  lɛ  wéé-pɛlɛ  \textbf{ē}  yɔɔ  ŋwɛ\`{ŋ}\\
     Maria  \textsc{3sg.exi}  speak-\textsc{inf}  \textbf{\textsc{3sg.refl}}  in.law  about\\
\glt ‘Maria is speaking about her brother-in-law.’ [el.]\z

Typical grooming contexts (shaving, combing, brushing the teeth) are expressed with reflexive markers in the inalienable possessor position.

\ea
    \label{ex:Kachaturyan:22} 
 \gll lɛ  \textbf{ē}  sɔ̰ɔ̰  pɛlɛ-pɛlɛ\\
     3\textsc{sg.exi}  \textbf{\textsc{3sg.refl}}  teeth  wash-\textsc{inf}\\
\glt ‘She is brushing her teeth.’ [el.]\z

In the case the possessor coreferential with the subject is alienable, there are several strategies available. First, a possessive pronoun can be used. In the 3rd person, it is potentially ambiguouos between a coreferential and a disjoint readings.

\ea
    \label{ex:Kachaturyan:23} 
 \gll ē  \textbf{là}  pɔɔ  sí  \\
     3\textsc{sg.pst}  \textbf{\textsc{3sg.poss}}  thing.\textsc{pl}  take \\
\glt ‘(The spider) collected its belongings,’ potential additional reading: somebody else’s belongings [MOC]\z

Another option is to use a basic or, in 3sg, reflexive pronoun and the self-intensifier \textit{dìè}. In such a case, the verb optionally takes a low-tone construct form (compare with \REF{ex:Kachaturyan:19} where the lexical tone is used). The reading is unabmiguously coreferential.

\ea
    \label{ex:Kachaturyan:24} 
 \gll ō  \textbf{ō}  \textbf{dìè}  kà  gɛ̰-pɛlɛ\\
     3\textsc{pl.pst}  \textbf{\textsc{3pl}}  \textbf{\textsc{int}}  house:\textsc{cstr}  see\textsc{{}-inf}\\
\glt ‘They see their own house/*somebody else’s house.’ [el.]\z

The final option is to use the self-intensifier \textit{zì}. It is typically used in possessive contexts, even without the possessor \REF{ex:Kachaturyan:25}, but can also be used in reflexive possessive contexts \REF{ex:Kachaturyan:25}, \REF{ex:Kachaturyan:26}.

\ea
    \label{ex:Kachaturyan:25} 
 \gll \textbf{kā}  \textbf{zì}  ā  bɛ̰ɛ̰  káà  lɔɔ  dɔ\\
     \textbf{\textsc{2pl}  \textbf{poss.int}}  \textsc{dem}  too  \textsc{2pl.jnt>3sg}  trade\textsc{:cstr}  do\textsc{:jnt}\\
\glt ‘Your (share), you sell it.’ [MOC]\\
     \z

\ea
    \label{ex:Kachaturyan:26} 
 \gll yé  wè\={ŋ}  āà  \textbf{ē}  \textbf{zì}  kɛ  nɛ\'{ŋ}nɛ\`{ŋ}  kɔ  gínī  ā...\\
     when  salt  3\textsc{sg.prf}  \textbf{3\textsc{sg.refl}}  \textbf{\textsc{poss.int}}  do:\textsc{nmlz}  tasty  arm:\textsc{cstr}  lose  \textsc{bkgr}\\
\glt ‘But when the salt has lost its matter of being tasty... (lit.: its-being-tasty-manner) [how can it become tasty again?]’ (Matthew 5:13; UBS 1978)\z

\ea
    \label{ex:Kachaturyan:27} 
 \gll mīā  sé\'{ŋ}  wáà  \textbf{ō}  \textbf{zì}  ɓɛlɛ  kṵ̀,\\
     person.\textsc{pl}  every  \textsc{3pl.jnt}  \textbf{\textsc{3pl}}  \textbf{\textsc{poss.int}}  string  catch:\textsc{jnt}\\
\glt ‘Every person grasped his own rope.’ [MOC]\z

In \sectref{sec:Kachaturyan:5.3}, we will see multiple examples of non-subject orientation of reflexive markers, including in the inalienable possessor position. The possibility of non-subject orientation was not tested for reflexive possessives marked with \textit{dìè} and \textit{zì.}


 \subsubsection{Basic pronoun in the reflexive function}
\label{sec:Kachaturyan:5.2.2}

In the postpositional phrase, the basic pronoun \textit{à} coreferential with the subject can occasionally be used, as demonstrated by a handful of corpus examples. In \REF{ex:Kachaturyan:28}, the pronoun is an argument of postposition, in \REF{ex:Kachaturyan:29} it is used as an inalienable possessor within the argument of postposition and in \REF{ex:Kachaturyan:30} it is used within alienable possessor expressed with the self-intensifier \textit{dìè.} 
%changed example numbering in REFs - original said: 21, 22, 23; but it is clearly 28, 29, 30

\ea
    \label{ex:Kachaturyan:28} 
 \gll ē  nū  \textbf{à}  pà.  \\
     3\textsc{sg}.\textsc{pst}  come  \textbf{3\textsc{sg}}  at\\
\glt ‘He came back home (lit.: he came at him).’ [MOC]
\z

\ea
    \label{ex:Kachaturyan:29} 
 \gll à  gbē  áà  gèè  \textbf{à}  lòkó  lɛɛ...\\
     3\textsc{sg}  son  \textsc{3sg.jnt>3sg}  say:\textsc{jnt}\textbf{  \textbf{3sg}}  mother \textsc{pp}\\
\glt ‘Her son said to his mother.’ [MOC]
\z

\ea
    \label{ex:Kachaturyan:30} 
 \gll lɛ  tá̰  kɛ–pɛlɛ  \textbf{à}  \textbf{dìè}  ɓū  gā-à  yí.\\
     3\textsc{sg}.\textsc{exi}  dance    do–\textsc{inf}  \textbf{3\textsc{sg}}  \textbf{\textsc{int}}  rice  die–\textsc{ger}  in\\
\glt ‘She is dancing in her (field of) ripe (lit.: dead) rice.’ [MOC]
\z

Such examples are generally disapproved in elicitation and can be collected only through corpus methods.


 
 \subsection{Non-subject orientation}\label{sec:Kachaturyan:5.3}

 \subsubsection{Direct object}\label{sec:Kachaturyan:5.3.1}


In addition to subject antecedents, reflexives in Mano can have non-subject antecedents, as well: direct object, argument of postposition and subject’s possessor. In all examples attested, the reflexive marker was situated in the postpositional phrase. I begin with the DO position, illustrated by \REF{ex:Kachaturyan:31}.

\ea
    \label{ex:Kachaturyan:31} 
 \gll Ō nɛfú ā gɛ̰ \textbf{ē} lòóò Mēlé kɛlɛ. \\
     3\textsc{pl.pst} child \textsc{dem} \textsc{s}ee \textbf{\textsc{3sg.refl}} mother Mary hand\\
\glt ‘They saw the child in the hands of his mother Mary.’ (Matthew 2:11; UBS 1978)
\z

In \REF{ex:Kachaturyan:32}, the reflexive marker in the postpositional phrase has two readings: its antecedent is either the DO, or the subject. Without the self-intensifier \textit{dìè} the preferred interpretation is subject-oriented. 

\ea
    \label{ex:Kachaturyan:32} 
 \gll Pèé  lɛ  Máríá  zɔ̰ɔ̰-pɛlɛ  \textbf{ē}  \textbf{dìè}  lɛɛ  \\
     Pe  \textsc{3sg.exi}  Maria  show-\textsc{inf}  \textbf{\textsc{3sg.refl}}  \textbf{\textsc{int} }\textsc{pp}  \\
\glt ‘Pe is showing Maria to himself/to herself.’ [el.]
\z


 \subsubsection{Postpositional phrase}\label{sec:Kachaturyan:5.3.2}


The antecedent of a reflexive in a postpositional phrase can be found in another postpositional phrase, as in \REF{ex:Kachaturyan:33}. A full NP with the same referent, \textit{dɔwálàlélàmìà} \textit{nɔfé} \textit{dò} ‚any prophet‘, is in the topic position and cannot occupy the role of the syntactic antecedent.

\ea
    \label{ex:Kachaturyan:33} 
 \gll dɔwálàlélàmìà  nɔfé  dò  òó  ló  dō  ō  kɔ  yà  \textbf{à}  wì  ɓɛlɛyà  ká  \textbf{ē}  \textbf{dìè}  pàà\\
     prophet  each  \textsc{indef}  \textsc{3pl.neg}  go  once  \textsc{3pl}  hand  put  \textbf{\textsc{3sg}} under  respect  with  \textbf{\textsc{3sg.refl}}  \textbf{{int}}  at\\
\glt ‘Any prophet\textsubscript{i}, they (=people) have never welcomed him\textsubscript{i} (lit.: put their hands under him) in his own\textsubscript{i} country (lit.: at his own).’ [MOC]
\z

However, it seems that the basic pronoun \textit{à} is preferred to the reflexive pronoun if the antecedent is in a PP. It is also preferably, but not obligatorily, used with a self-intensifier \textit{dìè.} 

\ea
    \label{ex:Kachaturyan:34} 
 \gll Pèé  ē  wéé  Máríá  lɛɛ  \textbf{à}  \textbf{(dìè)}  ŋwɛ\`{ŋ}\\
     Pe  \textsc{3sg.pst}  speak  Maria  \textsc{pp}  \textbf{\textsc{3sg}}  \textbf{\textsc{int}}  about\\
\glt ‘Pe\textsubscript{i} spoke to Maria\textsubscript{j} about herself\textsubscript{j}/someone else\textsubscript{k}/*himself\textsubscript{i}.’ [el.]
\z


 \subsubsection{Subject’s possessor}\label{sec:Kachaturyan:5.3.3}


Some examples were attested where the antecedent of the reflexive was the subject’s possessor. Example \REF{ex:Kachaturyan:35} is a resultative copular construction where the syntactic position of the subject is occupied by a nominalized form of the verb whose thematic argument occupies the syntactic position of the inalienable possessor. There are examples where the subject is a noun whose inalienable \REF{ex:Kachaturyan:36} and alienable \REF{ex:Kachaturyan:37} possessors are antecedents of the reflexive. It is not yet clear what allows such uses, but in all examples attested, the antecedent was a human and a prominent discourse character.

\ea
    \label{ex:Kachaturyan:35} 
 \gll \textbf{à}  wàà  lɛ  \textbf{ē}  kèlè  yí\\
     \textbf{3\textsc{sg}} enter.\textsc{ger} \textsc{cop} \textbf{\textsc{3sg.refl}} shell in\\
\glt ‘She is stuck in her shell (said about a child who does not grow fast enough).’ [MOC]
\z

\ea
    \label{ex:Kachaturyan:36} 
 \gll \textbf{à}  ɓɛlɛyà  wɔ  à  ká  \textbf{ē}  \textbf{dìè}  pàà\\
     \textbf{\textsc{3sg}} respect  \textsc{cop.neg}  \textsc{3sg}  with  \textbf{\textsc{3sg.refl}}  \textbf{\textsc{int}}  at\\
\glt ‘He is not respected in his own country (lit.: his\textsubscript{i} respect isn’t in his own\textsubscript{i} country).’ [MOC]
\z

\ea
    \label{ex:Kachaturyan:37} 
 \gll \textbf{là}  ɓò  vɔ  ō  pɛɛ–pɛlɛ  \textbf{ē}  \textbf{dìè}  kɛlɛ.\\
     \textbf{3\textsc{sg}}\textbf{.\textsc{poss}  }  goat  \textsc{pl}  3\textsc{pl}.\textsc{exi}  multiply–\textsc{inf}  \textbf{3\textsc{sg}}\textbf{.\textsc{refl}}  \textbf{\textsc{int}}  hand\\
\glt ‘His\textsubscript{i} goats are breeding in his\textsubscript{i} possession.’ [MOC]
\z


 
 \subsection{Reflexives in the subject position}\label{sec:Kachaturyan:5.4}

In rare examples from my corpus, disapproved in elicitation, the subject NP can contain a reflexive marker in the possessor position. In \REF{ex:Kachaturyan:39}, the noun phrase ‘her skin’ was repeated twice, in the first case, with the reflexive pronoun, and in the second case, with the basic pronoun, which is the preferred variant.

\ea
    \label{ex:Kachaturyan:38} 
 \gll \textbf{ē}  dàā  ē  kɛ  dɔmì  ká\\
     \textbf{3\textsc{sg.refl}}  father  3\textsc{sg.pst}  do  chief  with\\
\glt “His (lit.: his own) father was a chief” [MOC]
\z

\ea
    \label{ex:Kachaturyan:39} 
 \gll \textbf{ē}\textsubscript{i}  kīī  ɓō–ò  \textbf{ē\textsubscript{i}}  mɔ  gbāā,  \textbf{à\textsubscript{i}}  kīī  āà  ɓō.\\
     \textbf{3\textsc{sg}}\textbf{.\textsc{refl}}  skin  take.off–\textsc{ger}  \textbf{3\textsc{sg}}\textbf{.\textsc{refl}}  on  now  \textbf{3\textsc{sg}}  skin  3\textsc{sg}.\textsc{prf}  take.off\\
\glt ‘Her\textsubscript{i} (lit.: herself’s) skin being peeled off from herself\textsubscript{i}, her\textsubscript{i} skin was peeled off.’ [MOC]
\z


 \section{Valency-changing function}\label{sec:Kachaturyan:6}

In Mano, as is typical of Mande languages, the majority of verbs are labile and can be employed in transitive and intransitive constructions with active/causative or passive/inchoative meaning, respectively, without overt marking (on passive lability in Mande, see \citealt{CobbinahLuepke2009}). However, to explicitly mark the inchoative nature of the action, a postpositional phrase \textit{ē} \textit{dìè} \textit{lɛɛ} ‘by itself’ can be added.


 \ea
    \label{ex:Kachaturyan:40} 
    \ea
    \label{ex:Kachaturyan:40a} 
\gll ē  ɓò  fóló\\
     3\textsc{sg.pst}  goat  detach\\
\glt ‘He detached the goat.’

\ex
    \label{ex:Kachaturyan:40b}
\gll ɓò  ē  fóló\\
     goat  3\textsc{sg.pst}  detach\\
\glt ‘The goat detached.’

\ex
    \label{ex:Kachaturyan:40c}
\gll ɓò  ē  fóló  ē  dìè  lɛɛ\\
     goat  3\textsc{sg.pst}  detach  3\textsc{sg.refl}  \textsc{int}  \textsc{pp}\\
\glt ‘The goat detached by itself.’
\z
\z

In some contexts, some speakers accept the complex reflexive marker in the direct object position, still in the valency-changing, rather than authopatic function. The context where such a construction sounded the most natural was a famous West-African cartoon about the child warrior Kirikou, who was born by itself.

\ea
    \label{ex:Kachaturyan:41}
 \gll Kíríkú    ē  ē  dìè  yē\\
     Kirikou  \textsc{3sg.pst}  \textsc{3sg.refl}  \textsc{int}  give.birth\\
\glt ‘Kirikou was born by itself.’ (in the French original: Kirikou s’est enfanté tout seul, lui-même)\footnote{\url{https://www.youtube.com/watch?v=yg8GcN0rBLA}}\z


 \section{Influence of Kpelle in the reflexive domain}
\label{sec:Kachaturyan:7}

As mentioned above, Mano is in intense contact with a Southwestern Mande language Kpelle. In contrast with Mano, Kpelle lacks a dedicated reflexive pronoun and employs either basic pronominal prefixes for the expression of reflexivity (in the \textsc{3sg}, the prefix is expressed by consonant alternation and tonal change), or a combination of a prefix with a self-intensifier. Compare the use of the reflexive \REF{ex:Kachaturyan:41} and basic \REF{ex:Kachaturyan:42} pronouns in Mano with the use of the basic prefix in Kpelle \REF{ex:Kachaturyan:43}.

\ea
    \label{ex:Kachaturyan:42}
 \gll ē  ē  zúlú\\
       \textsc{3sg.pst}  \textsc{3sg.refl}  wash\\
\glt ‘He washed himself\textsubscript{.}’ [el.] (Mano)
\z

\ea
    \label{ex:Kachaturyan:43}
 \gll ē  à  zúlú\\
       \textsc{3sg.pst}  \textsc{3sg}  wash \\
\glt ‘He\textsubscript{i} washed him\textsubscript{j}’ [el.] (Mano)
\z

\ea
    \label{ex:Kachaturyan:44}
 \gll àá  \textbf{ŋw}àa\\
     3\textsc{sg.res}  \textsc{3sg{\textbackslash}}wash\\
\glt ‘He\textsubscript{i} washed him\textsubscript{j}/himself\textsubscript{.i}.’ [el.] (Kpelle)\z

As a result of contact with Kpelle, some Mano – Kpelle bilinguals employ the Mano basic pronoun in their Mano speech even in the contexts where such use is normally disallowed, namely, in the direct object position. Such use is especially common in the speech of young bilingual children and of L2 speakers of Mano. The example \REF{ex:Kachaturyan:44} was obtained from a 19 year old speaker whose father is Mano and whose mother is Kpelle but who grew up in the Kpelle-speaking village of her maternal grandparents; in addition to a different pattern in the use of reflexives, her speech shows interference in the use of tones, which is why they are not marked.

\ea
    \label{ex:Kachaturyan:45}
 \gll nɛfu  lɛ  \textbf{a}  die  gɛ̰-pɛlɛ  gaazu  yi\\
       child  \textsc{3sg.exi}  \textbf{\textsc{3sg}} \textsc{int}  see\textsc{{}-inf}  mirror  in\\
\glt ‘The child is seeing her (meaning: herself) in the mirror.’\z

It was mentioned in \sectref{sec:Kachaturyan:5.2.2} that the basic pronoun is sometimes used in the reflexive function in the speech of (quasi-)monolinguals. The examples given above concerned the position within the postpositional phrase. Another context is the inclusory construction, which is the main means for expression of nominal coordination. In this construction, the inclusory pronoun expresses the entire set of coordinated participants, or the superset and is followed by a noun phrase expressing a subset of participants. In this construction, bilinguals and monolinguals alike employ both basic and reflexive pronouns with equal frequency. (Inclusory constructions in Mande languages in typological and diachronic perspectives were described in \citealt{Khachaturyan2019}.) Note also that it is a syntactically unusual position where the antecedent is not a subject and is not overtly expressed: the antecedent is included in the referent of the inclusory pronoun.

\ea
    \label{ex:Kachaturyan:46}
 \gll wà  ē/  à  lòkóò  \\
     \textsc{3pl.ip}  \textsc{3sg.refl}  \textsc{3sg}  mother \\
\glt ‘he\textsubscript{i} and his\textsubscript{i,} mother (lit: they (including) his mother)’ [el.] (Mano)
\z

The use of the non-reflexive pronoun in the inclusory construction may be a direct consequence of contact and the fact that that very construction (or, more specifically, the pronoun) was borrowed into Mano from Kpelle \citep{Khachaturyan2019}.

An interesting fact for the syntax of binding is that when the inclusory construction occurs in the non-subject position, the reflexive pronoun can only have a reading disjoint from the subject \REF{ex:Kachaturyan:47}. To express coreference with the subject, the basic pronoun must be chosen \REF{ex:Kachaturyan:48}. Thus, these contexts, which have been tested only in elicitation, provide an intriguing example of obligatory non-subject orientation of the reflexive pronoun and require a further explanation.

\ea
    \label{ex:Kachaturyan:47}
 \gll Pèé  ē  Máríá  wà    \textbf{ē}  yɔɔ  gɛ̰\\
     Pe  \textsc{3sg.pst}  Maria \textsc{3pl.ip} \textbf{\textsc{3sg.refl}} in.law see\\
\glt ‘Pe\textsubscript{i} saw Maria\textsubscript{j} and her\textsubscript{j}/*his\textsubscript{k}/*his\textsubscript{i} brother in law.’ [el.]\z

\ea
    \label{ex:Kachaturyan:48}
 \gll Pèé  ē  Máríá  wà    \textbf{à}  yɔɔ  gɛ̰\\
     Pe  \textsc{3sg.pst}  Maria \textsc{3pl.ip} \textbf{\textsc{3sg}} in.law see\\
\glt ‘Pe\textsubscript{i} saw Maria\textsubscript{j} and his\textsubscript{i}/his\textsubscript{k}/*her\textsubscript{j} brother in law.’ [el.]\z


 \section{Discussion}\label{sec:Kachaturyan:8}

Mano has one dedicated reflexive pronoun, \textit{ē}, typically used with \textsc{3sg} antecedents, and two self-intensifiers, \textit{dìè} and \textit{zì}, the latter being used only in possessive contexts. Alone, \textit{ē} forms a simplex reflexive marker, and accompanied by \textit{dìè} it forms a complex reflexive marker. Both simplex and complex markers are used in autopathic, oblique and possessive contexts and their use cannot be accounted for by the semantic class of the verb (introverted and extroverted). The self-intensifier \textit{dìè} is preferred in oblique argument position (\sectref{sec:Kachaturyan:4.2}), as well as in all cases where the coreference relation is less trivial: when the antecedent is not the subject (\sectref{sec:Kachaturyan:5.3}), when there is contrast involved \REF{ex:Kachaturyan:15}, or when it accompanies the basic pronoun \textit{à} in contexts where the coreference domain extends beyond the minimal final clause (\sectref{sec:Kachaturyan:5.1}). The function of \textit{dìè} is thus much more than to form a complex reflexive marker used in specific syntactic and semantic contexts: it is employed to ensure broader reference continuity (a somewhat similar account of logophoric marking can be found in \citealt{Dimmendaal2001}).

In the direct object position, the reflexive pronoun \textit{ē} is in complementary distribution with the basic pronoun \textit{à}. However, in the postpositional phrase, \textit{à} is also frequently allowed, especially for non-subject orientation. Such lack of complementarity of reflexive and non-reflexive markers in non-core domains has been attested cross-linguistically (\citealt{TesteletsToldova1998}). In addition, under the influence of Kpelle, which does not distinguish between reflexive and nonreflexive pronouns, in Mano the basic pronoun can replace the reflexive even in the direct object position in the speech of bilinguals and in the inclusory construction borrowed from Kpelle.

One distinctive feature of the Mano reflexivity system is the possibility of non-subject orientation, especially with direct object antecedents. \tabref{tab:Kachaturyan:2} summarizes the uses of reflexive and basic 3\textsc{sg} pronouns \textit{ē} and \textit{à} with different antecedents. The lines reflect the position of the antecedent and the columns reflect the position of the pronouns. 

\begin{table}
\begin{tabularx}{0.8\textwidth}{XXXp{5cm}} 
\lsptoprule
&\textsc{subj} & \textsc{do} & \textsc{pp}\\
\hline
\textsc{subj} & {}- & refl & refl (preferred in el., occurs in corpus); basic (corpus)\\
\textsc{do} & {}- & {}- & refl (preferred in el., occurs in corpus); no basic pronouns in the corpus\\
\textsc{pp} & {}- & {}- & basic (preferred in el., no corpus examples); refl (1 corpus example)\\
\lspbottomrule
\end{tabularx}
\caption{Subject and non-subject orientation in 3\textsc{sg}}
\label{tab:Kachaturyan:2}
\end{table}

According to the most recent analysis, Mande languages have a reduced verb phrase structure, with only the direct object belonging to the verb phrase, while all other verbal arguments being expressed by postpositional phrases and adjoined highly \citep{Nikitina2018}. Although there are arguments in support of this analysis for Mano, reflexivity presents a challenge for it, at least if analyzed within the framework of binding theory which imposes the restriction of c-commanding. The reason is that direct object NPs are widely accepted as antecedents to reflexive markers in the position of arguments of postpositions, which is a direct violation of c-commanding, assuming that postpositional phrases are base-generated in the IP-adjoined position, higher than the DO. Potentially, reflexivity represents a challenge to the view of SAuxOV as the underlying word order, and not a result of movement (\citealt{SandeEtAl2019}). Alternatively, if the choice of antecedent is regulated not by the principle of c-commanding, but by the scale of syntactic roles (\citealt{TesteletsToldova1998}), then the behavior of reflexive markers is much easier to explain: the antecedent is always found in the same position on the scale or higher. In addition, there is a potential case of obligatory non-subject orientation of reflexives as part of the inclusory construction which requires an additional explanation.

Final remark concerns the use of the self-intensifier \textit{dìè} in anticausative constructions. The prediction by \citet{KoenigMoyseFaurie2016} states that if a marker is used for middle voice (including anticausative), for coreference between the core arguments and in the self-intensifier function, which is the case for Mano, then it has to be used in the reciprocal function. Mano data clearly contradicts the prediction, since there is a dedicated reciprocal marker \textit{kíè.}


 \section*{Abbreviations}

\begin{tabularx}{.45\textwidth}{lQ}
\textsc{att} & attention drawer\\ 
\textsc{conj} & conjunctive\\ 
\textsc{cop} & copula\\ 
\textsc{cstr} & construct form\\ 
\textsc{dem} & demonstrative\\ 
\textsc{emph} & emphatic\\ 
\textsc{exi} & existential\\
\textsc{ger} & gerund\\ 
\textsc{h} & high tone\\ 
\textsc{indef} & indefinite\\
\textsc{inf} & infinitive\\ 
\textsc{int} & intensifier\\ 
\textsc{ip} & inclusory pronoun\\ 
\textsc{ipfv} & impervective\\ 
\textsc{jnt} & conjoint\\ 
\textsc{neg} & negative\\ 
\textsc{pl} & plural\\ 
\textsc{poss} & possessive\\ 
\textsc{pp} & postposition, postpositional phrase\\ 
\textsc{prf} & perfect\\ 
\textsc{pst} & past\\ 
\textsc{recp} & reciprocal\\ 
\textsc{refl} & reflexive\\ 
\textsc{res} & resultative\\ 
\textsc{sg} & singular.
\end{tabularx}


%\section*{Acknowledgements}
%\citet{Nordhoff2018} is useful for compiling bibliographies.

{\sloppy\printbibliography[heading=subbibliography,notkeyword=this]}
\end{document}
