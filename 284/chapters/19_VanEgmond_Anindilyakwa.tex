\documentclass[output=paper]{langscibook}
\author{Marie-Elaine van Egmond\affiliation{University of Greifswald}}
\title{The reflexive voice construction in Anindilyakwa}
\abstract{This chapter describes the reflexive voice in Anindilyakwa, a polysynthetic language of Northern Australia. In this language, up to two arguments of a verb are identified by means of pronominal prefixes on the verb. Reflexive voice in Anindilyakwa is marked by a verbal suffix that occurs on transitive verbs and reduces the valency of the verb by one. The suffix signals that the agent subject is co-referential with the referent that previously occurred as the transitive object pronominal prefix. This is mostly a patient referent, but it can also be a beneficiary introduced by the benefactive applicative, or the recipient referent of an inherently ditransitive verb. Although the language has free pronouns, there are no reflexive pronouns in Anindilyakwa; the sole reflexivizer is the verbal suffix.}
\IfFileExists{../localcommands.tex}{
 \addbibresource{localbibliography.bib}
 \usepackage{langsci-optional}
\usepackage{langsci-gb4e}
\usepackage{langsci-lgr}

\usepackage{listings}
\lstset{basicstyle=\ttfamily,tabsize=2,breaklines=true}

%added by author
% \usepackage{tipa}
\usepackage{multirow}
\graphicspath{{figures/}}
\usepackage{langsci-branding}

 
\newcommand{\sent}{\enumsentence}
\newcommand{\sents}{\eenumsentence}
\let\citeasnoun\citet

\renewcommand{\lsCoverTitleFont}[1]{\sffamily\addfontfeatures{Scale=MatchUppercase}\fontsize{44pt}{16mm}\selectfont #1}
  
 %% hyphenation points for line breaks
%% Normally, automatic hyphenation in LaTeX is very good
%% If a word is mis-hyphenated, add it to this file
%%
%% add information to TeX file before \begin{document} with:
%% %% hyphenation points for line breaks
%% Normally, automatic hyphenation in LaTeX is very good
%% If a word is mis-hyphenated, add it to this file
%%
%% add information to TeX file before \begin{document} with:
%% %% hyphenation points for line breaks
%% Normally, automatic hyphenation in LaTeX is very good
%% If a word is mis-hyphenated, add it to this file
%%
%% add information to TeX file before \begin{document} with:
%% \include{localhyphenation}
\hyphenation{
affri-ca-te
affri-ca-tes
an-no-tated
com-ple-ments
com-po-si-tio-na-li-ty
non-com-po-si-tio-na-li-ty
Gon-zá-lez
out-side
Ri-chárd
se-man-tics
STREU-SLE
Tie-de-mann
}
\hyphenation{
affri-ca-te
affri-ca-tes
an-no-tated
com-ple-ments
com-po-si-tio-na-li-ty
non-com-po-si-tio-na-li-ty
Gon-zá-lez
out-side
Ri-chárd
se-man-tics
STREU-SLE
Tie-de-mann
}
\hyphenation{
affri-ca-te
affri-ca-tes
an-no-tated
com-ple-ments
com-po-si-tio-na-li-ty
non-com-po-si-tio-na-li-ty
Gon-zá-lez
out-side
Ri-chárd
se-man-tics
STREU-SLE
Tie-de-mann
}
 \togglepaper[1]%%chapternumber
}{}

\begin{document}
\maketitle
%\shorttitlerunninghead{}%%use this for an abridged title in the page headers


\section{Introduction}
\label{sec:vanegmond:1}

Anindilyakwa (pronounced [ɛnin̪t̪iʎakʷa] in the language itself) is a non-Pama-Nyungan language spoken by over 1,400 people \citep{MarmionEtAl2014} living on Groote Eylandt and Bickerton Island in the Gulf of Carpentaria, Northern Territory, Australia (see \figref{fig:vanegmond:1}). It is one of the very few remaining Australian languages that is still acquired by children and is thus spoken by all generations. Nonetheless, despite the efforts of the community and linguists, the language is, as are all of Australia’s indigenous languages, endangered due to the pressure of English. Anindilyakwa was once thought to be “perhaps the most difficult of all Australian languages, with a very complex grammar” \citep[84]{Dixon1980}, and classified as a language isolate by \citet{OGradyEtAl1966Languages, OGradyEtAl1966Aboriginal}, and \citet[250]{Evans2005}. However, the language has recently been demonstrated to be closely related to Wubuy, its nearest geographical neighbour spoken on the mainland and is thus to be subsumed under the Gunwinyguan family (\citealt{VanEgmond2012}, \citealt{VanEgmondBaker2021}; see \figref{fig:vanegmond:1}). The previously presumed isolate status of Anindilyakwa may be due to:
(i) its unusual phonological inventory, which departs from both the typical Australian pattern (including e.g. the phoneme /ə/), and from the typical Gunwinyguan pattern (due to e.g. the lamino-dental /l̪/ and lamino-palatal /ʎ/ phonemes, written \textit{lh} and \textit{ly}, respectively),
(ii) its few recognizable verbal roots and inflections (\citealt{Baker2004}, fn 25), and
(iii) its idiosyncratic lexicon (\citealt{Capell1942}: 376; \citealt{Worsley1954}: 20; \citealt{Heath1981}; \citealt{Yallop1982}: 40). But despite its complexities, \citet{VanEgmond2012} shows that Anindilyakwa grammar is also fairly regular, and patterns much like the Gunwinyguan family of languages on the mainland to its west.



\begin{figure}
 \includegraphics[width=\textwidth]{figures/vanEgmondAnindilyakwa-img001.png}
\caption{Anindilyakwa and the Gunwinyguan family \citealt{Harvey2003}: 204; \citealt{BarryEtAl2003}; \citealt{Evans2017}; \citealt{VanEgmond2012}; \citealt{VanEgmondBaker2021})}
\label{fig:vanegmond:1}
\end{figure}


Like the other Gunwinyguan languages, Anindilyakwa is richly polysynthetic, exhibiting extensive cross-referencing of subject and object arguments on the verb by means of pronominal prefixes, noun incorporation, and a variety of valency-changing affixes, including the reflexive suffix that is the topic of this chapter. All nominals and verbs are obligatorily inflected for person, number and gender for humans, or one of five noun classes for non-humans.

The sole reflexivizer in the language is a verbal reflexive voice marker, which is a suffix that is added to the verb stem. There are no reflexive pronouns in Anindilyakwa. The reflexive suffix changes the argument structure of the verb: since the agent subject is now co-referential with the patient argument in object function, the verb becomes morphologically intransitive, and both agent and patient are represented by the same pronominal prefix on the verb. The reflexive suffix -\textit{jungwV}- is related to the reciprocal suffix -\textit{yi}-, which occurs in the same position and which also reduces the valency of the verb. Compare the transitive verb kill’ (literally make die’) in \REF{ex:vanegmond:1a} with the intransitive reflexive \REF{ex:vanegmond:1b} and reciprocal \REF{ex:vanegmond:1c} verbs:



\ea%1
 \label{ex:vanegmond:1}

 \ea
 \label{ex:vanegmond:1a}
\gll \textbf{\textit{nə-ma}}-\textit{jungwa-ju-wa}\\
\textbf{\textsc{3sg.m}}\textbf{-}\textbf{\textsc{veg}}-die-\textsc{caus}-\textsc{pst}\\
\glt he killed it (e.g. animal of \textsc{veg} noun class, such as \textit{mangma} \textsc{veg}.crab’)’


 \ex
 \label{ex:vanegmond:1b}
\gll \textbf{\textit{nə}}-\textit{jungwa-ja-}\textbf{\textit{jungu}}\textit{-na}\\
\textbf{\textsc{3sg.m}}-die-\textsc{caus}-\textbf{\textsc{refl}}-\textsc{pst}\\
\glt he killed himself’


 \ex
 \label{ex:vanegmond:1c}
\gll \textbf{\textit{na}}\textit{-jungwa-jee-}\textbf{\textit{yi}}-\textit{na}\\
\textbf{\textsc{3pl}}-die-\textsc{caus}-\textbf{\textsc{recp}}-\textsc{pst}\\
\glt they killed each other’
\z
\z



In \REF{ex:vanegmond:1a}, the verb has a subject prefix \textit{nə-} and an object prefix \textit{ma-} representing the agent (he’) and the patient (an animal of vegetable (\textbf{\textsc{veg}}) noun class, such as a crab), respectively. The reflexive verb in \REF{ex:vanegmond:1b}, on the other hand, is intransitive and the pronominal prefix \textit{nə-} represents both agent and patient, which are co-referential. Similarly, the verb in \REF{ex:vanegmond:1c} is also intransitive whilst specifying that the subject and object are co-referential, with the added reciprocal meaning of two or more agents each engaging in the same action (i.e. to verb each other’).



 After a brief overview of the principal typological features of the language (\sectref{sec:vanegmond:2.1}), nominals (\sectref{sec:vanegmond:2.2}), verbs (\sectref{sec:vanegmond:2.3}), the reflexive voice construction is described in detail in \sectref{sec:vanegmond:3}, followed by its potential historical source in \sectref{sec:vanegmond:4}, and a brief summary of the reflexive voice construction in \sectref{sec:vanegmond:5}



\section{Typological features}
\label{sec:vanegmond:2}



\subsection{Introduction}\label{sec:vanegmond:2.1}



The principal morphosyntactic typological features of Anindilyakwa are:


\begin{itemize}
\item As expected for a polysynthetic language, a single verb can express much of what is accomplished by the syntax in other languages - expression of arguments, causativization, reflexivization, reciprocity, and subordination

\item Arguments of the verb can be additionally expressed by optional free pronouns, demonstratives, or full nominals

\item Up to two arguments are prefixed to the verb (\sectref{sec:vanegmond:2.3}), and nominals are classified for one of five noun classes (non-humans) or one of three genders (humans) (\sectref{sec:vanegmond:2.2})

\item Four distinct series of pronominal prefixes on verbs encode an equal number of moods

\item Case-marking is primarily exploited as a strategy for roles such as locative, ablative, allative, instrumental, and to indicate relations between nominals. Anindilyakwa makes little use of nominal morphology to encode information about core syntactic functions; determination of subject (intransitive and transitive) and object functions is done by the pronominal prefixes on the verb

\item Most nominal case suffixes can also be used as complementizing cases on a verb in a subordinate clause to express temporal, causal and other relationships with the main clause (see examples in \REF{ex:vanegmond:7} below)

\end{itemize}
\begin{itemize}
\item A number of derivational affixes can alter the argument structure of the verb: the benefactive applicative prefix \textit{mən}- turns a beneficiary participant into an object that is prefixed to the verb, and the reciprocal and causative suffixes change the valency of the verb (\sectref{sec:vanegmond:2.3.3}), as does the reflexive suffix (\sectref{sec:vanegmond:3})

\item Body part and generic nominals can be incorporated into verbs and adjectives, leaving the valency of the verb unaffected; the incorporable syntactic functions are restricted to the absolutive pattern (e.g. example \REF{ex:vanegmond:2b} below)

\item Verb stems can be complex, historically consisting of an uninflecting plus an inflecting element, the latter determining the conjugational class of the stem

\item Since the arguments of the verb are identified by the pronominal prefixes on the verb, word order is syntactically free, and pragmatically determined

\item All words end in [a], and the vowel [u] is not contrastive but generated by adjacent [+round] consonants. The first [u] in the reflexive suffix -\textit{jungwV}-, for instance, is formed by assimilation of an underlying high vowel to the following labio-velar [ŋʷ]. The second vowel is realised as [a] when word-final (-\textit{jungwa}), but when followed by another suffix, this vowel absorbs the rounding of the preceding [ŋʷ] and is realised as [u] (e.g. -\textit{jungu-na}).

\end{itemize}

The following examples illustrate some of the above features: the pronominal prefixes on verbs and noun classes on nominals in (\ref{ex:vanegmond:2}a--c), noun incorporation in \REF{ex:vanegmond:2b} and derivational affixes in \REF{ex:vanegmond:2c}. All examples in this chapter come from \citet{VanEgmond2012} unless indicated otherwise.



\ea%2
 \label{ex:vanegmond:2}

 \ea
 \label{ex:vanegmond:2a}
\gll ngayuwa yiba-rrəngkə-na-ma nungkuwa adhalyəmə-manja arnungkwaya\\
\textsc{1sg.pro} \textsc{irr}.\textsc{1sg/2sg}-see-\textsc{npst}-1.\textsc{foc} \textsc{2sg}.\textsc{pro} \textsc{neut}.river-\textsc{loc} tomorrow\\
\glt I will see you at the river tomorrow’



\ex
\label{ex:vanegmond:2b}
\gll nanga-lyang-barra arəngkə-manja akinə-mərra dhukururrku-manja\\
\textsc{fem/fem}-head-hit.\textsc{pst} \textsc{neut}.head-\textsc{loc} \textsc{neut}.that-\textsc{instr} \textsc{fem}.Brolga-\textsc{loc}\\
\glt She [Emu(\textsc{fem})] hit Brolga on the head with that [stick(\textsc{neut}]’ \citep[310]{Leeding1989}


\ex
\label{ex:vanegmond:2c}
\gll kərrenə-mənə-muku+lharri-ju-wa merra\\
\textsc{3m/2pl}-\textsc{bene}-fluid+fall-\textsc{caus}-\textsc{pst} \textbf{\textsc{veg}}.blood\\
\glt he shed his blood for you’
\z
\z



As is common in Australian languages (e.g. \citealt{Dixon1980}), two major word classes can be identified in Anindilyakwa along the traditional lines of the affixational potential of the individual lexemes: nominals (\sectref{sec:vanegmond:2.2}) and verbs (\sectref{sec:vanegmond:2.3}). These two classes are differentiated by taking distinct sets of inflectional and derivational affixes.



\subsection{Nominals, noun classes and genders}\label{sec:vanegmond:2.2}

All nominals apart from loanwords are obligatorily inflected for person, number and gender (humans), or noun class (non-humans). Noun class systems are very common in the non-Pama-Nyungan languages of Australia. They are grammaticalized agreement systems, where class may be overtly marked on the noun, on articles and modifiers within the noun phrase, and on the predicate (e.g. \citealt{Dixon1986}; \citealt{Sands1995}; \citealt{Aikhenvald2000}). The most typical Australian system has four noun classes, which can be broadly labelled as masculine, feminine, vegetable, and neuter or residual (e.g. \citealt{Sands1995}: 258; \citealt{Evans2003}: 182). Anindilyakwa has five noun classes that classify non-humans and three genders that classify humans and domesticated animals, as outlined in \tabref{tab:vanegmond:1}--\tabref{tab:vanegmond:2}. The pronominal prefixes (1\textsuperscript{st} and 2\textsuperscript{nd} person) are identical on nominals and intransitive verbs, whereas the gender and noun class prefixes (3\textsuperscript{rd} person) differ. The table also lists the free pronouns for completeness.

%divided table into two because it was too complex otherwise

\begin{table}
\begin{tabularx}{\textwidth}{Xp{0.7cm}XXXX}
\lsptoprule
& gloss & nominals & intransitive verbs & free pronouns \\
\hline
pronominal prefixes & \textsc{1} & \textit{nəng-} & \textit{nəng-} & \textit{ngayuwa}\\
  & \textsc{1pl} & \textit{yirr-} & \textit{yirr-} & \textit{yirruwa}\\
 & \textsc{1fdu} & \textit{yirrəng-} & \textit{yirrəng-} & \textit{yirrənguwa}\\
  & \textsc{1mdu} & \textit{yin-} & \textit{yin-} & \textit{yinuwa}\\
  & \textsc{12} & \textit{y-} & \textit{y-} & \textit{yakuwa}\\
  & \textsc{12pl} & \textit{ngarr-} & \textit{ngarr-} & \textit{ngakurruwa}\\
  & \textsc{2} & \textit{nəngk-} & \textit{nəngk-} & \textit{nəngkuwa}\\
  & \textsc{2pl} & \textit{kərr-} & \textit{kərr-} & \textit{nəngkurruwa}\\
  & \textsc{2fdu} & \textit{kərrəng-} & \textit{kərrəng-} & \textit{nəngkərrənguwa}\\
  & \textsc{2mdu} & \textit{kən-} & \textit{kən-} & \textit{nəngkə(r)nuwa}\\
genders  & \textsc{3f} & \textit{dh-} & \textit{ying-} & \textit{ngalhuwa}\\
  & \textsc{3m} & \textit{n-} & \textit{n-} & \textit{enuwa}\\
  & \textsc{3pl} & \textit{wurr-} & \textit{na- {\textasciitilde} nuw-} & \textit{abərruwa}\\
  & \textsc{3fdu} & \textit{wurrəng-} & \textit{narrəng-} & \textit{abərrənguwa}\\
  & \textsc{3mdu} & \textit{wun-} & \textit{nen-} & \textit{abə(r)nuwa}\\
 \lspbottomrule
\end{tabularx}
\caption{Anindilyakwa free pronouns and prefixes on nominals and intransitive verbs -- humans and domesticated animals}
\label{tab:vanegmond:1}
\end{table}

\begin{table}
\begin{tabularx}{\textwidth}{XXXXXX}
\lsptoprule
& gloss & nominals & intransitive verbs & free pronouns \\
\hline
{noun classes} & {animate} & \textsc{masc} & \textit{y-} & \textit{n-} & \textit{(yi)ngalhuwa}\\
 & & \textsc{fem} & \textit{dh-} & \textit{ying-} & \textit{ngalhuwa}\\
& & \textsc{coll} & \textit{wurr-} & \textit{na- {\textasciitilde} nuw-} & \textit{abərruwa}\\
& {inanimate} & \textsc{veg} & \textit{m(a)-} & \textit{nəm}- & \textit{(mə)ngalhuwa}\\
 & & \textsc{neut} & \textit{a- {\textasciitilde} e-} & \textit{na- {\textasciitilde} nuw-} & \textit{(a)ngalhuwa}\\
 \lspbottomrule
\end{tabularx}
\caption{Anindilyakwa free pronouns and prefixes on nominals and intransitive verbs -- non-humans}
\label{tab:vanegmond:2}
\end{table}

%\begin{table}
%\begin{tabularx}{\textwidth}{XXXXXXX}
%\lsptoprule
%\hhline%%replace by cmidrule{~------}
%& \multicolumn{2}{c}{} & \textbf{gloss} & \textbf{nominals} & \textbf{intransitive verbs} & \textbf{free pronouns}\\
%\hhline%%replace by cmidrule{~------}
%\textbf{humans, domes-ticated animals} & \multicolumn{2}{c}{\textbf{pronominal prefixes}} & \textsc{1} & \textit{nəng-} & \textit{nəng-} & \textit{ngayuwa}\\
%& & & \textsc{1pl} & \textit{yirr-} & \textit{yirr-} & \textit{yirruwa}\\
%& & & \textsc{1fdu} & \textit{yirrəng-} & \textit{yirrəng-} & \textit{yirrənguwa}\\
%& & & \textsc{1mdu} & \textit{yin-} & \textit{yin-} & \textit{yinuwa}\\
%& & & \textsc{12} & \textit{y-} & \textit{y-} & \textit{yakuwa}\\
%& & & \textsc{12pl} & \textit{ngarr-} & \textit{ngarr-} & \textit{ngakurruwa}\\
%& & & \textsc{2} & \textit{nəngk-} & \textit{nəngk-} & \textit{nəngkuwa}\\
%& & & \textsc{2pl} & \textit{kərr-} & \textit{kərr-} & \textit{nəngkurruwa}\\
%& & & \textsc{2fdu} & \textit{kərrəng-} & \textit{kərrəng-} & \textit{nəngkərrənguwa}\\
%& & & \textsc{2mdu} & \textit{kən-} & \textit{kən-} & \textit{nəngkə(r)nuwa}\\
%& \multicolumn{2}{c}{\textbf{genders}} & \textsc{3f} & \textit{dh-} & \textit{ying-} & \textit{ngalhuwa}\\
%& & & \textsc{3m} & \textit{n-} & \textit{n-} & \textit{enuwa}\\
%& & & \textsc{3pl} & \textit{wurr-} & \textit{na- {\textasciitilde} nuw-} & \textit{abərruwa}\\
%& & & \textsc{3fdu} & \textit{wurrəng-} & \textit{narrəng-} & \textit{abərrənguwa}\\
%& & & \textsc{3mdu} & \textit{wun-} & \textit{nen-} & \textit{abə(r)nuwa}\\
%\textbf{non-humans} & \textbf{noun classes} & \textbf{animate} & \textsc{masc} & \textit{y-} & \textit{n-} & \textit{(yi)ngalhuwa}\\
%& & & \textsc{fem} & \textit{dh-} & \textit{ying-} & \textit{ngalhuwa}\\
%& & & \textsc{coll} & \textit{wurr-} & \textit{na- {\textasciitilde} nuw-} & \textit{abərruwa}\\
%& & \textbf{inanimate} & \textsc{veg} & \textit{m(a)-} & \textit{nəm}- & \textit{(mə)ngalhuwa}\\
%& & & \textsc{neut} & \textit{a- {\textasciitilde} e-} & \textit{na- {\textasciitilde} nuw-} & \textit{(a)ngalhuwa}\\
%\hhline%%replace by cmidrule{~~~----}
%\lspbottomrule
%\end{tabularx}
%\caption{Anindilyakwa free pronouns and prefixes on nominals and intransitive verbs}
%\label{tab:vanegmond:1}
%\end{table}


One way in which Anindilyakwa stands out from all other Gunwinyguan (and, indeed, non-Pama-Nyungan) languages is that the class prefixes on nouns are completely lexicalized and tightly bound to the noun root.\footnote{In other Gunwinyguan languages, noun class prefixes may be omitted (as indicated below by the -’ sign), but in Anindilyakwa they are tightly bound to the noun root (as indicated by the +’ sign): Anindilyakwa Wubuy Ngandi seagrass (\textsc{veg}) \textit{ma+wurrəra ama-wurruri ma-wurruri} ticks, fleas (\textsc{coll}) \textit{wurr+amərnda waa-murndik a-murndik} \textsc{neut}-louse’hawk (\textsc{masc}) \textit{ji+nəkarrka jii-nikarrka a-jikarrka} (\textsc{neut})} The class prefixes on adjectives, on the other hand, are variable, as illustrated in \REF{ex:vanegmond:3} for \textit{arəma} big’, as are the gender prefixes for humans \REF{ex:vanegmond:4}:



%\ea%3
 %\label{ex:vanegmond:3}
%  
% \gll y-arəma yaraja wurr-arəma wurrendhindha m-arəma memərrerra \textsc{veg}.flathead\\
% \textsc{masc}-big \textsc{masc}.goanna \textsc{coll}-big \textsc{coll}.rat \textsc{veg} -big \\
%\glt big goanna’ big rat’ big flathead’
%\z

\ea%3 in original in sequence raher than abc, see above
 \label{ex:vanegmond:3}
 \ea 
 \gll y-arəma yaraja\\
 \textsc{masc}-big \textsc{masc}.goanna \\
\glt big goanna’ 

\ex
 \gll wurr-arəma wurrendhindha\\ \textsc{coll}-big \textsc{coll}.rat \\
\glt big rat’

\ex
 \gll  m-arəma memərrerra\\
 \textsc{veg}-big \textsc{veg}.flathead\\

\glt big flathead’


\z
\z



\ea%4 same as 3
 \label{ex:vanegmond:4}
 \ea
 \label{ex:vanegmond:4a}
 \gll nə-balanda  \\
 \textsc{3m}-white.person\\ 
 \glt male non-Aborigine’  

 \ex
 \label{ex:vanegmond:4b}
 \gll dhə-balanda\\
 \textsc{3f}-white.person\\
 \glt female non-Aborigine’
 
\ex 
\label{ex:vanegmond:4c}
\gll wurrə-balanda\\
 \textsc{3pl}-white.person\\
\glt non-Aborigines’
\z
\z



Besides their ability to be used derivationally on nouns, as in \REF{ex:vanegmond:4}, where biological sex of the referent is determined by the prefix, gender prefixes differ from noun class prefixes in that they are used on loanwords, as in the Macassan loan \textit{balanda} above (which ultimately derives from \textit{Hollander}). Loanwords with non-human reference do not take noun class prefixes, such as the English loans \textit{jukwa} sugar’ and \textit{bajungkula} bicycle’, and the earlier Macassan loans \textit{jurra} paper, book’ (< \textit{surat}) and \textit{libaliba} canoe’ (< \textit{lepa-lepa}). Their noun class membership becomes apparent through agreement:



\ea%5
 \label{ex:vanegmond:5}

\ea
\label{ex:vanegmond:5a}
\gll \textbf{m}{-arəma} \textbf{{dəraka}}\\
\textbf{\textsc{veg}}-big \textbf{truck(\textsc{veg}})\\
\glt big truck’

 \ex
 \label{ex:vanegmond:5b}
\gll \textbf{{koton}} {nəngə-}\textbf{{nga}}{-rrəngka-ma} narrə-\textbf{{nga}}{-lhungkuwabi-ju-wa-ma}\\
\textbf{cotton(\textsc{fem}}) \textsc{1sg-}\textbf{\textsc{fem}}-see.\textsc{pst}-1.\textsc{foc} \textsc{3pl-}\textbf{\textsc{fem}}-grow-\textsc{caus}-\textsc{pst}-1.\textsc{foc}\\
\glt I saw cotton that they were growing’
\z
\z


\subsection{The verb}
\label{sec:vanegmond:2.3}

The verb is morphologically the most complex word class in Anindilyakwa. A single verb can express what may take a whole sentence in a language like English. Because of its internal complexity, much of what is accomplished by the syntax in other languages is carried out within the verb - expression of arguments, causativization, reflexivization, reciprocity and subordination. The complex templatic structure of the verbal word, where affix order is stipulated in the form of arbitrary position classes, is presented in \tabref{tab:vanegmond:3}.\footnote{A template is a flat structure where affixes are ordered with “no apparent connection to syntactic, semantic or even phonological representation” (\citealt{Inkelas1993}: 560, cited in \citealt{Nordlinger2010}). Templatic systems are not uncommon in the Australian context, especially for the head-marking polysynthetic languages of the north \citep{Nordlinger2010}} The verbal template has a finite number of slots with a fixed order, and no embedding possibilities.



\begin{table}
\begin{tabularx}{0.4\textwidth}{XX}
\lsptoprule
%slot & element
%\hline
(+5) & Case\\
(+4) & \textit{-ma {\textasciitilde} -mərra}\\
(+3) & Tense + Aspect\\
(+2) & Reflexive, reciprocal\\
(+1) & Causative\\
0 & Stem\\
-1 & Body part / generic\\
-2 & Benefactive\\
-3 & Quantifier\\
-4 & Object\\
-5 & Subject\\
-6 & Mood\\
\lspbottomrule
\end{tabularx}
\caption{Anindilyakwa verbal template (with optional elements in parentheses)}
\label{tab:vanegmond:3}
\end{table}


The only obligatory slots in this template are the pronominal prefixes in slots [-6] to [-4], the stem in [0] and the tense/aspect inflectional suffixes in [+3]. Note that the stem itself may be morphologically complex, and historically include compounded nominals (e.g. -\textit{muku+lharri}- [fluid+fall] to shed’ in \REF{ex:vanegmond:2c} above). Although they are given separate positions in the template, the valency-changing causative suffix in [(+1)] and reflexive and reciprocal suffixes in [(+2)] contribute to the formation of the verb stem.



\subsubsection{Main features of each slot}\label{sec:vanegmond:2.3.1}



This section briefly summarizes the main features of each slot of the verbal template, in order to understand the basic morphosyntax of the language, which will be necessary in our discussion of the reflexive construction of the language.



 The obligatory \textsc{pronominal} \textsc{prefixes} zone, in slots [-6] to [-4], contains up to two prefixes that represent the arguments of the verb, plus an indication of mood, as part of a complex paradigm. This zone includes the first and second person pronominal prefixes, and third person gender prefixes for humans, and noun class markers for non-humans. Transitive prefix complexes with human referents may be portmanteau forms, which is why the three slots are merged as a fusion zone in \tabref{tab:vanegmond:2}.



There are four distinct intransitive and four distinct transitive series of prefixes: (i) realis, (ii) irrealis, (iii) imperative, and (iv) hortative. As is characteristic of the non-Pama-Nyungan languages \citep{Verstraete2005}, the prefixes are combined with the tense/aspect suffixes (slot [+3]) to mark a variety of modal meanings. The Anindilyakwa system of eight series of (positive polarity) prefixes is unusually high: many non-Pama-Nyungan languages have a basic realis/irrealis distinction in the prefixes, but they do not differentiate between imperative or hortative mood, whereas some Gunwinyguan languages do not distinguish mood in the prefixes at all (e.g. Bininj Gun-wok, Ngalakgan, Ngandi), but employ suffixes instead.



The \textsc{quantifier} slot [(-3)] contains the quantifiers \textit{mərnda}- and \textit{wurra}- many’, which also occur on nominals.



The \textsc{benefactive} slot [(-2)] contains just one morpheme: the benefactive applicative \textit{mən}-, which introduces a beneficiary argument to the verb, which then knocks the theme argument out of object position. Compare the following examples, which are both transitive, but with a different argument structure: in \REF{ex:vanegmond:6a}, the theme argument in object function represented by the pronominal prefix on the verb is a neuter class item (i.e., \textit{akungwa} \textsc{neut}.water’), whereas in \REF{ex:vanegmond:6b} the object is the beneficiary introduced by the benefactive applicative:



\ea%6
 \label{ex:vanegmond:6}

 \ea
 \label{ex:vanegmond:6a}
\gll \textbf{n}{-akarrngə-na} \textbf{{akungwa}}\\
 \textbf{\textsc{3m>}}\textbf{\textsc{neut}}-get-\textsc{npst} \textbf{\textsc{neut}}\textbf{.water}\\
\glt he is getting water’



 \ex
 \label{ex:vanegmond:6b}
\gll \textbf{ngənə}-mən-akarrngə-na akungwa\\
 \textbf{\textsc{3m>1sg}}-\textsc{bene}-get-\textsc{npst} \textsc{neut}.water\\
\glt he is getting water for me’
\z

\z

In \REF{ex:vanegmond:6b}, the theme \textit{akungwa} water’ is no longer represented on the verb but only occurs outside of the verb.

The \textsc{body} \textsc{part} \textsc{/} \textsc{generic} slot [(-1)] is filled by a nominal root drawn from a set of about 80 forms, which are either body parts or generics that classify an external specific noun. An example was given in \REF{ex:vanegmond:2b} above.


As is typical of the Gunwinyguan languages (\citealt{BarryEtAl2003}), the \textsc{stem} slot [0] may be simple or complex. Simple stems consist of a verb root to which the inflection for tense and aspect may be added directly (e.g. -\textit{kwa}- give’, -\textit{lhəka}- go’). Complex verb stems, on the other hand, are synchronically frozen combinations of an uninflecting element followed by an element that takes the inflections (e.g. -\textit{yeng+bi}- speak’, consisting of the nominal root \textit{yeng}- voice’ and the inflecting element +\textit{bi}- ?’). Verb stems can furthermore be formed from nominals by the productive inchoative and factitive suffixes (see \sectref{sec:vanegmond:2.3.2} below).


The \textsc{causative} slot [(+1)] contains the causative suffix -\textit{ji}-, which derives transitive verbs from intransitive verbs. For example, -\textit{jungwa-ji}- to kill’ is derived from -\textit{jungwV}- to die’ in \REF{ex:vanegmond:7} below. See \sectref{sec:vanegmond:2.3.3}


The \textsc{reflexive} \textsc{/} \textsc{reciprocal} slot [(+2)] contains the reflexive suffix -\textit{jungwV}- and the reciprocal suffix -\textit{yi}-. These mutually exclusive suffixes derive intransitive verbs from transitive verbs, as was illustrated in \REF{ex:vanegmond:1} above and will be discussed in more detail in \sectref{sec:vanegmond:2.3.3} and \sectref{sec:vanegmond:3}, respectively.


The obligatory \textsc{tense+aspect} slot [+3] contains the tense and aspect inflections, which combine with the pronominal prefixes to express various modal meanings. There are six main conjugational classes, organised around the verb root or the inflecting element of the complex verb stem. The tense/aspect suffixes distinguish past (\textsc{pst}) and non-past (\textsc{npst}) tense, together with neutral aspect or a subtype of perfective aspect.


The very common -\textit{ma} {\textasciitilde} -\textit{mərra} suffix in slot [(+4)] occurs independently of tense and aspect, and is analysed by \citet[225-236]{VanEgmond2012} as a first person focalisation marker’ (1. \textsc{foc}), indicating that the speaker expresses his or her perception of an event or state of affairs.

The \textsc{case} slot [(+5)] contains case suffixes, which can be used on a verb in a subordinate clause to relate it to the main clause (as they can in many other, mainly Pama-Nyungan, Australian languages). Such cases are called complementizing cases in the literature (\citealt{DenchEvans1988}), and can be divided into two basic types: C-complementizer case, where members of the subordinate clause are case-marked in agreement with a coreferential NP in the main clause, as in \REF{ex:vanegmond:7a}, and T-complementizer case on members of the subordinate clause to express temporal, causal and other relationships with the main clause, as in \REF{ex:vanegmond:7b}. The subordinate clause appears in square brackets.



\ea%7
 \label{ex:vanegmond:7}

 \ea
 \label{ex:vanegmond:7a}
\gll Arakbawiya warnə-mamalya nuw-akbardha-ngə-ma y-akina-\textbf{{lhangwa}} [kənə-ngekbəraka-mə-\textbf{lhangwa} edhərra emindha-manja].\\
 long.time.ago \textsc{3pl.m}-people \textsc{3pl}-be.afraid-\textsc{pst}-1.\textsc{foc} \textsc{masc}-that-\textbf{\textsc{dat}} \textsc{irr}. \textsc{masc}> \textsc{neut}-make.\textsc{pst}-1. \textsc{foc}-\textbf{\textsc{dat}} \textsc{neut}.hole \textsc{neut}.nose-\textsc{loc}\\
\glt A long time ago people were afraid of them [\textit{yangungwa} \textsc{masc}.eel’] making a hole in their noses.’



\ex
\label{ex:vanegmond:7b}
\gll {[kenu-warde-na-}\textbf{{manja}}], nungkw-aja kənu-warde-na\\
 \textsc{irr}.\textsc{3m>2sg-}hit-\textsc{npst}-\textbf{\textsc{loc}} \textsc{2sg.pro}-\textsc{CofR} \textsc{irr}.2sg>\textsc{3m}-hit-\textsc{npst}\\
\glt if he hits you, you can hit him back’
\z
\z



In \REF{ex:vanegmond:7a}, the dative suffix on the verb in the relative clause agrees with the oblique object of the verb in the main clause (afraid of \textsc{x}-\textsc{dat}). The \textsc{loc} case on the verb in \REF{ex:vanegmond:7b} signals that the subordinate clause has a conditional meaning.



\subsubsection{Verbalizing suffixes}\label{sec:vanegmond:2.3.2}


New verbs can be created from nominals by the very productive inchoative -\textit{dhə}- and factitive -\textit{ka}- {\textasciitilde} -\textit{kwa} derivational suffixes.


%subsubsubsecction
\subsubsubsection{Inchoative -\textit{dhə} \textsc{inch}}\label{sec:vanegmond:2.3.2.1})


This suffix turns a noun or an adjective into an intransitive verb, which means to become [X]’. Some examples are listed in \REF{ex:vanegmond:8}, which also include the inchoative suffix added to recent loanwords.



\ea%8
 \label{ex:vanegmond:8}
\ea
\label{ex:vanegmond:8a}
 -\textit{arəma} big’ -\textit{arəmə-dhə}- to become big’

\ex
\label{ex:vanegmond:8b}
\textit{awinyamba} \textsc{neut}.anger’ -\textit{awinyamba-dhə}- to become angry’
 \ex
\textit{kərrəndəna} leprosy’ (< Eng \textit{quarantine}) -\textit{kərrəndəna-dhə}- to quarantine’
\ex
\textit{bungkawa} boss, ruler’ (< Mac \textit{puŋgawa}) -\textit{bungkawa-dhə}- to become ruler’
\z
\z



The following are some sentence examples.



\ea%9
 \label{ex:vanegmond:9}
 \ea
 \label{ex:vanegmond:9a}
\gll Wurr-adhədhiyara karrə-rrəngkə-na-manja akina karrə-m-abuwarrkə-na-ma abərra-lhangwa mingeemina mena kəm-arəmə-\textbf{dhə}-mə=baba.\\
\textsc{3pl-}young.girl \textsc{irr}.3pl> \textsc{neut}-see-\textsc{npst}-\textsc{loc} \textsc{neut}.that \textsc{irr}.\textsc{3pl-veg}-cover-\textsc{npst}-1.\textsc{foc} \textsc{3pl}.\textsc{pro}-\textsc{poss} \textsc{veg}.breast because \textsc{irr}. \textsc{veg}-big-\textbf{\textsc{inch}}\textbf{.\textsc{npst}}-1.\textsc{foc}=\textsc{reas}\\
\glt If young girls see them [\textit{engeemina} \textsc{neut}.legless lizard’], they cover their breasts because they will get bigger.’


\ex
\label{ex:vanegmond:9b}
\gll yirrə-ma-ngamba-ju-wa-ma nəmə-mərrkbalya-\textbf{{dhə}}-nə-ma ambaka\\
\textsc{1pl}-\textsc{veg}-bathe-\textsc{caus}-\textsc{pst}-1.\textsc{foc} \textsc{veg}-soft-\textbf{\textsc{inch}}-\textsc{pst}-1.\textsc{foc} later\\
\glt we soaked them [\textit{mənhənga} \textsc{veg}.burrawang’] in water, and later they became soft’
\z
\z


As these examples show, a denominal verb behaves like any other verb in Anindilyakwa in taking full person/number/gender/mood and tense/aspect affixation.



\subsubsubsection{Factitive -\textit{ka} {\textasciitilde} \textit{kwa} (factive)}\label{sec:vanegmond:2.3.2.2})



The factitive converts a noun or adjective into a transitive verb meaning to make something [X]’, as illustrated in the following dictionary entries.



\ea%10
 \label{ex:vanegmond:10}
\ea
\label{ex:vanegmond:10a}
 -\textit{dharrba} short’ -\textit{dharrbu-kwa}- shorten’

\ex
\label{ex:vanegmond:10b}
-\textit{abiyakarbiya} three’ -\textit{abiyakarbiya-ka}- divide into three’

\ex
\label{ex:vanegmond:10c}
\textit{awinyamba} \textsc{neut}.anger’ -\textit{awinyamba-ka}- to make angry’

\ex
\label{ex:vanegmond:10d}
\textit{alhəkəra} \textsc{neut}.house’ -\textit{lhəkəra-ka}- erect, raise, build’
\z
\z



The following are some textual examples of the factitive suffix.



\ea%11
 \label{ex:vanegmond:11}

 \ea
 \label{ex:vanegmond:11a}
\gll Nenə-ma-ngə-ma yərda biya nen-abiyarbuwa-\textbf{{ka}}{-ma} \textbf{y-akina}.\\
\textsc{3pl}>\textsc{masc}-take-\textsc{pst}-1.\textsc{foc} \textsc{masc}.supplejack and 3pl>\textsc{masc}-four-\textbf{\textsc{fact}}\textbf{.\textsc{pst}}-1.\textsc{foc} \textsc{masc}-that\\
\glt They took the supplejack cane and split it into four.’

 \ex
 \label{ex:vanegmond:11b}
\gll  a-mərndak-akina-ma amarda narr-ardadə-\textbf{ka}{-ma}\\
\textsc{neut}-many-that-\textsc{instr} \textsc{neut}.grass \textsc{3pl}>\textsc{neut}-hot-\textbf{\textsc{fact}}\textbf{.\textsc{pst}}-1.\textsc{foc}\\
\glt they heated them with leaves’
\z
\z


Factitive verbs can be reflexivized, as in example \REF{ex:vanegmond:20a} below.


\subsubsection{Argument-changing affixes}\label{sec:vanegmond:2.3.3}

As already mentioned, a number of derivational affixes alter the argument structure of the verb: the benefactive applicative prefix \textit{mən}- in slot [(-2)] of the verbal template introduces non-subcategorised arguments, while the causative, reflexive and reciprocal suffixes change the valency of the verb. They are discussed here in turn, with the reflexive suffix given its individual \sectref{sec:vanegmond:3}



\subsubsubsection{Benefactive applicative prefix (\textsc{bene})}\label{sec:vanegmond:2.3.3.1}

The prefix \textit{mən}- is an applicative that adds a beneficiary or maleficiary object argument to the verb, that is, a person positively or negatively affected by the action denoted by the verb. This new beneficiary/maleficiary argument knocks out the previous patient/theme object argument, which now appears as a free nominal (as we have already seen in \REF{ex:vanegmond:6} above). Compare the following examples taken from texts, where the object prefix indexes a patient referent in the (a) examples, while an introduced beneficiary referent occurs on the verb in the (b) examples.


\ea%12
 \label{ex:vanegmond:12}
 \ea
 \label{ex:vanegmond:12a}
\gll y-akina yikarba \textbf{{nəng-eni}}{-ngayindhu-ma}\\
\textsc{masc}-that \textsc{masc}.woomera \textbf{\textsc{1sg}}\textbf{-}\textsc{masc}-want.\textsc{pst}-1.\textsc{foc}\\
\glt I want that woomera’


 \ex
 \label{ex:vanegmond:12b}
\gll Akina awilyaba ngaya \textbf{{ngarra}}{-}\textbf{{mən}}{-ngayindhe-na-ma}.\\
\textsc{neut}.that \textsc{neut}.one \textsc{1sg.pro} \textbf{\textsc{1sg>2sg}}-\textbf{\textsc{bene}}-want-\textsc{npst}-1.\textsc{foc}\\
\glt That’s all I want for you.’
\z
\z



\ea%13
 \label{ex:vanegmond:13}

 \ea
 \label{ex:vanegmond:13a}
\gll {biya} \textbf{{na}}{-ma-nga}\\
and \textsc{neut}>\textsc{neut}-take-\textsc{pst}\\
\glt and it [mother cat] took another [kitten]’

 \ex
 \label{ex:vanegmond:13b}
\gll {Arakbawiya} \textbf{{narra}}{-}\textbf{{mənə}}-ma-ngə-ma wurrə-mərrə-mərrkbalya-lhangwa wurr-angarə-ngariya engengkuwa.\\
long.time.ago \textbf{\textsc{3pl>3pl}}-\textbf{\textsc{bene}}-take-\textsc{pst}-1.\textsc{foc} \textsc{3pl}-\textsc{rdp}-newborn.baby-\textsc{poss} \textsc{3pl}-\textsc{rdp}-young \textsc{neut}.spirit\\
\glt A long time ago they took the spirits of newborn babies.’
\z
\z



In \REF{ex:vanegmond:12b}, the argument introduced by the benefactive applicative is a beneficiary (you’), while in \REF{ex:vanegmond:13b} it is a maleficiary (they’, i.e. newborn babies’). A beneficiary verb is a regular transitive verb which can be reflexivized, as we will see in \sectref{sec:vanegmond:3} below.



\subsubsubsection{Causative -\textit{ji} (\textsc{caus})}\label{sec:vanegmond:2.3.3.2}


The most usual meaning of the causative suffix is causal, hence to make X [verb]’. The verb to which the suffix is added is normally intransitive and becomes transitive. The following are textual examples of causativized verbs.



\ea%14
 \label{ex:vanegmond:14}

 \ea
 \label{ex:vanegmond:14a}
\gll Adhənəbawiya nə-ma-beka-\textbf{{ju}}{-wa} \textbf{m-akina} \textbf{dəraka} {amalyirra-mərra}.\\
first \textsc{3m}-\textsc{veg}-drink-\textbf{\textsc{caus}}-\textsc{pst} \textsc{veg}-that truck(\textsc{veg}) \textsc{neut}.petrol-\textsc{instr}\\
\glt First he filled the truck with petrol.’ (Lit: he made the truck drink’)

 \ex
 \label{ex:vanegmond:14b}
\gll kureya ngə-ma-ngarre-na-ma m-ibina kə-ma-ngamba-\textbf{ji}{-ni=yadha} \\
have.a.try \textsc{hort.1sg-veg}-visit-\textsc{nps}-1.\textsc{foc} \textsc{veg}-that.same \textsc{irr}.1\textsc{sg}-\textsc{veg}-bathe-\textbf{\textsc{caus}}-\textsc{npst}=\textsc{purp}\\
\glt let me go and see if they [\textit{mənhənga} \textsc{veg}.burrawang’] are ready for me to soak them’
\z
\z

A causative verb is a regular transitive verb in that it can be reflexivized (\sectref{sec:vanegmond:3}).


\subsubsubsection{Reciprocal -\textit{yi} (\textsc{recp})}\label{sec:vanegmond:2.3.3.3}

The reciprocal suffix -\textit{yi}- occurs in slot [(+2)] together with the reflexive suffix discussed in the next section. The reciprocal decreases the verb’s valency by one, whilst specifying that the subject and object are co-referential, plus adding the reciprocal meaning of two or more agents each engaging in the same action (i.e. to verb each other’). The suffix is usually added to a transitive verb, which may also include causatives. A textual example is given in \REF{ex:vanegmond:15}.



\ea%15
 \label{ex:vanegmond:15}
 \gll kembirra arakba na-kwee-\textbf{yi}{-nə-ma} na-məng-barri-\textbf{yi}-nə-ma yimərnda na-kwee-\textbf{{yi}}{-nə-ma} \textbf{arəngka-manja} {nuw-arrka-milyi-jee-}\textbf{{yi}}{-nə-ma}\\
then compl.\textsc{act} 3pl-give-\textbf{\textsc{recp}}-\textsc{pst}-1.\textsc{foc} \textsc{3pl}-small.and.round-split-\textbf{\textsc{recp}}-\textsc{pst}-1.\textsc{foc} \textsc{masc}.louse \textsc{3pl-}give-\textbf{\textsc{recp}}-\textsc{pst}-1.\textsc{foc} \textsc{neut}.head-\textsc{loc} \textsc{3pl}-small.and.many-hold-\textsc{caus}-\textbf{\textsc{recp}}-\textsc{pst}-1.\textsc{foc}\\
\glt then they gave lice to each other and shared them and they held each other’s heads’
\z



The reciprocal suffix also has a collective reading, which is not uncommon cross-linguistically (see \citealt{Evans2003}: 495 and references therein), and which also happens in the related languages Bininj Gun-wok \citep{Evans2003} and Wubuy \citep{Heath1984}.



\ea%16
 \label{ex:vanegmond:16}

 \ea
 \label{ex:vanegmond:16a}
\gll nenə-rrəngka wurr-ambilyuma wurrajija nuw-angkarree-\textbf{yi}{-na-ma}\\
\textsc{3m>}\textsc{coll}-see.\textsc{pst} \textsc{coll}-two \textsc{coll}.bird \textsc{coll}-run-\textbf{\textsc{recp}}-\textsc{npst}-1.\textsc{foc}\\
\glt he saw the two birds flying away’ \citep[448]{Leeding1989}



 \ex
 \label{ex:vanegmond:16b}
\gll {yirrə-ngambee-}\textbf{{yi}}{-na}\\
\textsc{1pl}-bathe-\textbf{\textsc{recp}}-\textsc{pst}\\
\glt we all bathed’
\z
\z



The reciprocal suffix can co-occur with the transitivising benefactive applicative prefix, resulting in a morphologically intransitive verb:



\ea%17
 \label{ex:vanegmond:17}
\gll Kərr-ambarrngarna arakba karna na-\textbf{{mən}}{-angkarree-}\textbf{\textit{yi}}{-nə-ma?}\\
\textsc{2pl}-how.many? now \textsc{2pl}.this \textsc{3pl}-\textbf{\textsc{bene}}-run-\textbf{\textsc{recp}}-\textsc{pst}-1.\textsc{foc}\\
\glt How many of you [Aboriginal women] have they [whitefellas] run off with now?’
\z



Here, the \textsc{recp} suffix has scope over the \textsc{bene} prefix. The intransitive verb -\textit{angkarr}- run’ is made transitive by the \textsc{bene} (run off with’), which in turn is detransitivized by the \textsc{recp} (run off all together’): [\textsc{bene}-run]-\textsc{recp}.


\section{Reflexive -\textit{jungwV}}
\label{sec:vanegmond:3}

\subsection{Introduction}
\label{sec:vanegmond:3.1}

The reflexive voice marker in Anindilyakwa is the suffix -\textit{jungwV}-, which occurs in the same slot in the verbal template as the reciprocal suffix -\textit{yi}-. It reduces the morphological valency of the verb by one and indicates the coreference of two participants of the verb, as was illustrated in \REF{ex:vanegmond:1} above and again in \REF{ex:vanegmond:18} below. In \REF{ex:vanegmond:18a}, the intransitive verb -\textit{ngamba}- bathe’ is transitivized by the causative suffix -\textit{ja}- (bathe-\textsc{caus} = wash’), with the agent woman’ and the patient dress’ both represented on the verb by subject and object pronominal prefixes, respectively. In \REF{ex:vanegmond:18b}, by contrast, only the subject is cross-referenced on the verb, as agent and patient are now co-referential.


\ea%18
 \label{ex:vanegmond:18}

 \ea
 \label{ex:vanegmond:18a}
\gll dhə-dharrəngka \textbf{yingə-ma}-ngamba-ju-wa dhərija\\
\textsc{3f-}female \textbf{\textsc{3f}}\textbf{-}\textbf{\textsc{veg}}-bathe-\textsc{caus}-\textsc{pst} dress(\textsc{veg})\\
\glt the woman washed her dress’


\ex
\label{ex:vanegmond:18b}
\gll dhə-dharrəngka \textbf{yingə}{-ngamba-ja-}\textbf{jungu}-na\\
\textsc{3f}-female \textbf{\textsc{3f}}-bathe-\textsc{caus}-\textbf{refl}-\textsc{pst}\\
\glt the woman washed herself’
\z
\z


As this example shows, there are no reflexive pronouns in the language; reflexivity is only signalled by the suffix -\textit{jungwV}- on the verb. Identification of the arguments of the verb is done on the verb in Anindilyakwa; free pronouns are common but optional, as in \REF{ex:vanegmond:2a}, \REF{ex:vanegmond:7b}, \REF{ex:vanegmond:12b} above and other examples below.
In \REF{ex:vanegmond:18b}, the only possible reading is co-reference of agent and patient. The co-reference of the reflexive verb contrasts with the disjoint reference of the transitive verb in \REF{ex:vanegmond:19}:


\ea%19
 \label{ex:vanegmond:19}
\gll dhə-dharrəngka \textbf{nanga}-ngamba-ju-wa\\
\textsc{3f}-female \textbf{\textsc{3f>3f}}-bathe-\textsc{caus}-\textsc{pst}\\
\glt the woman\textsubscript{1} washed her\textsubscript{2}’
\z



Here, the verb does not have a reflexive marker, plus its pronominal prefix represents both an agent and a patient. Therefore, there is no other reading possible but disjoint reference.
The use of the reflexivizer is not subject to specific conditions relating to person or number: the same suffix is used for every person and number. Although the examples given so far all involve third person participants, the following textual examples involve first person plural \REF{ex:vanegmond:20a}, first person singular \REF{ex:vanegmond:20b} and second person singular \REF{ex:vanegmond:20c}.

\ea%20
 \label{ex:vanegmond:20}

 \ea
 \label{ex:vanegmond:20a}
\gll \textbf{Yirr}-akakərəma-ka-\textbf{jungu}-na-ma ngawa wurru-balanda-lhangwa a-mərndakijika adhuwaba ena-manja ayangkidharrba.\\
\textbf{\textsc{1pl}}-know.how.to-\textsc{fact}-\textbf{\textsc{refl}}-\textsc{npst}-1.\textsc{foc} cont.\textsc{act} \textsc{3pl}-non.Aborigine-\textsc{poss} \textsc{neut}-things today \textsc{neut}.this-\textsc{loc} \textsc{neut}.island\\
\glt We have learnt about white man’s things on this island.’

 \ex
 \label{ex:vanegmond:20b}
\gll ngalha-ja dh-akina narrang-anga-manja ena \textbf{nəngə}-dhaka-\textbf{jungu}-nu-ma\\
\textsc{fem}.\textsc{pro}-\textsc{emph} \textsc{fem}-that \textsc{fem}>\textsc{3pl}-bite.\textsc{pst}-\textsc{loc} \textsc{neut}.this \textbf{\textsc{1sg}}-burn-\textbf{\textsc{refl}}-\textsc{pst}-1.\textsc{foc}\\
\glt when she [spider] bit them [me or you] I just burnt myself [where I got bitten by the spider]’

\ex
 \label{ex:vanegmond:20c}
\gll Kemba kə-lhəka-ja-ma nəngk-ena m-ardədarra-manja \textbf{kə}{-karri-}\textbf{jungu}-na-ma m-ardədarra-manja.\\
then \textsc{irr}.\textsc{2sg}-go-\textsc{npst}-1.\textsc{foc} \textsc{2sg}-this \textsc{veg}-hot-\textsc{loc} \textbf{\textsc{irr.2sg}}-roast.in.ashes-\textbf{\textsc{refl}}-\textsc{npst}-1.\textsc{foc} \textsc{veg}-hot-\textsc{loc}\\
\glt Then you should go in the hot [sun(\textsc{veg})] and roast yourself in the hot [sand(\textsc{veg})].’
\z
\z


The suffix can equally well be used with non-human, even inanimate, participants:



\ea%21
 \label{ex:vanegmond:21}
 \ea
 \label{ex:vanegmond:21a}
\gll mema ma-mə-ki-yelhiya m-ibina nəmi-yelhiye-na-ma nəm-abuwarrka-\textbf{jungu}-na-ma\\
\textsc{veg}.this \textsc{veg}-\textsc{inalp}-\textsc{nzr}-be.shy \textsc{veg}-that \textsc{veg}-be.shy-\textsc{npst}-1.\textsc{foc} \textsc{veg}-hide-\textbf{\textsc{refl}}-\textsc{npst}-1.\textsc{foc} \\
\glt the name \textit{maməkiyelhiya} [shy crab’] means “that one that is shy” [because] it always hides itself’

\ex
\label{ex:vanegmond:21b}
\gll m-akinee=ka dəraka ngakurra-lhangwa, nəma-mənu-wardhi-\textbf{jungu}-na-ma\\
\textsc{veg}-that=\textsc{emph} truck(\textsc{veg}) \textsc{12pl}.\textsc{pro}-\textsc{poss} \textsc{veg}-\textsc{bene}-work-\textbf{\textsc{refl}}-\textsc{npst}-1.\textsc{foc}\\
\glt that truck of ours, it has to work for itself’
\z
\z



In \REF{ex:vanegmond:21b}, the reflexive construction involves coreference of the agent not with a patient argument but with a beneficiary, which is introduced by the benefactive applicative. Without the reflexive suffix, the verb would be transitive (e.g. the truck has to work for us’), with both the subject and the beneficiary represented on the verb by pronominal prefixes. The reflexive suffix detransitivizes the verb: the truck has to work for itself. Coreference of the subject agent with semantic roles other than patient is the topic of the next section.



 The reflexive suffix can also be used on nominalized verbs, which in Anindilyakwa can be used as non-finite verbs:



\ea%22
 \label{ex:vanegmond:22}
\gll Arakbawiya warnəmamalya nenə-ma-ngə-ma y-akaka-lhangwa yi-nə-m-akarrnga warni-ku-mərndi-\textbf{jungwi}=yadha.\\
 long.time.ago \textsc{3pl}.people \textsc{3pl}>\textsc{masc}-take-\textsc{pst}-1.\textsc{foc} \textsc{masc}-this-\textsc{poss}
 \textsc{masc}-\textsc{m}-\textsc{inalp}-teeth \textsc{3pl.m}-\textsc{nzr}-comb-\textbf{\textsc{refl}}=\textsc{purp}\\
\glt A long time ago people used to take the sawfish (\textit{yikurrərrəndhangwa}) teeth to comb their hair (Lit: to comb themselves’) (\citealt{GrooteEylandtDictionary1993}: 123)
\z


Since in Anindilyakwa, the pronominal prefix on the verb can either encode the possessor of the body part (the whole’), or the body part itself (the choice between the two is semantically motivated: see \citealt{VanEgmond2012}: Chapter 7), the subject agent argument being coreferential with the object theme argument in \REF{ex:vanegmond:22} is unproblematic: the combing of hair is perceived as not just affecting the hair but the whole person.



\subsection{Coreference of the subject with other semantic roles}\label{sec:vanegmond:3.2}



{While the reflexive construction frequently expresses coreference of the agent subject with the patient referent in object function, the subject can be co-referential with other semantic roles as well. This is only possible for participants registered on the non-reflexive verb by the object pronominal prefix, which are: (i) recipient argument of inherently ditransitive verbs, and (ii) beneficiary argument introduced by the benefactive applicative. Coreference of the subject with other semantic roles, such as (iii) possessor, and (iv) spatial referent, cannot be expressed by a reflexivized verb in Anindilyakwa. These four instances are discussed here in turn.}

\subsubsection{Reflexivized ditransitive verbs: coreference of subject with recipient} \label{sec:vanegmond:3.2.1}

For inherently ditransitive verbs, such as give’, tell’ and send’, the recipient is represented on the verb in object function, while the theme argument occurs outside of the verb, as shown in \REF{ex:vanegmond:23}. When such a ditransitive verb is reflexivized, it is business as usual: the verb becomes morphologically intransitive, with the subject agent being co-referential with the argument in object position, which now is the recipient, as in \REF{ex:vanegmond:24}.

\ea%23
 \label{ex:vanegmond:23}

 \ea
 \label{ex:vanegmond:23a}
\gll \textbf{nanga}-kwa jurra\\
\textbf{\textsc{3f>3f}}-give.\textsc{pst} book(\textsc{neut})\\
\glt she gave her a book’


 \ex
 \label{ex:vanegmond:23b}
\gll \textbf{yirrenə}-maka-mərra ena alhawudhawarra akina\\
\textbf{\textsc{3m>1pl}}-tell. \textsc{pst}-1.\textsc{foc} \textsc{neut}.this \textsc{neut}.story \textsc{neut}.that\\
 he told us this story’
\z
\z


\ea%24
 \label{ex:vanegmond:24}
 \ea
 \label{ex:vanegmond:24a}
\gll \textbf{yingu}{-kwa-}\textbf{jungu}-na jurra\\
\textbf{\textsc{3f}}\textsc{-}give-\textbf{\textsc{refl}}-\textsc{pst} book(\textsc{neut})\\
\glt she gave herself a book’

\ex
\label{ex:vanegmond:24b}
\gll \textbf{nə}-maka-\textbf{jungu}-na-mərra ena alhawudhawarra akina\\
\textbf{\textsc{3m}}-tell-\textbf{\textsc{refl}}-\textsc{pst}-1.\textsc{foc} \textsc{neut}.this \textsc{neut}.story \textsc{neut}.that\\
\glt he told himself this story’
\z
\z



The examples in \REF{ex:vanegmond:24} are regular reflexive constructions; the only difference is that the subject is now co-referential with the recipient, rather than the patient.



\subsubsection{Reflexivized benefactives: coreference of subject with beneficiary }\label{sec:vanegmond:3.2.2}



As already mentioned in \sectref{sec:vanegmond:2.3.3}, the benefactive applicative introduces a beneficiary argument to the verb, which knocks the theme/patient argument out of the object prefix position, as in \REF{ex:vanegmond:25a}, repeated from \REF{ex:vanegmond:13b} above. When reflexivized, the subject thus becomes co-referential with the beneficiary, as in \REF{ex:vanegmond:25b}.\footnote{This example is made up by me based on my knowledge of the language and has not been tested with speakers. However, \REF{ex:vanegmond:26} and \REF{ex:vanegmond:27} are real life examples taken from texts, which support the validity of \REF{ex:vanegmond:25b}.}



\ea%25
 \label{ex:vanegmond:25}

 \ea
 \label{ex:vanegmond:25a}
\gll Akina awilyaba ngaya \textbf{{ngarra}}{-}\textbf{{mən}}{-ngayindhe-na-ma}.\\
\textsc{neut}.that \textsc{neut}.one \textsc{1sg}.\textsc{pro} \textbf{\textsc{1sg>2sg}}-\textbf{\textsc{bene}}-want-\textsc{npst}-1.\textsc{foc}\\
\glt That’s all I want for you.’

 \ex
 \label{ex:vanegmond:25b}
\gll Akina awilyaba ngaya \textbf{nəngə}{-}\textbf{mən}{-ngayindhe-}\textbf{jungu}-na-ma.\\
\textsc{neut}.that \textsc{neut}.one \textsc{1sg}.\textsc{pro} \textbf{\textsc{1sg}}-\textbf{\textsc{bene}}-want-\textbf{\textsc{refl}}-\textsc{npst}-1.\textsc{foc}\\
\glt That’s all I want for myself.’
\z
\z



Here are some more examples of reflexivized benefactives:



\ea%26
 \label{ex:vanegmond:26}
\gll ngarrəbukurra-lhangwa engengkuwa \textbf{ngarrəbə}{-}\textbf{mənə}-rəngka-\textbf{jungwa} {ajungwa}\\
\textsc{12tri}.\textsc{pro}-\textsc{poss} \textsc{neut}.life \textbf{12tri}-\textbf{\textsc{bene}}-look.after-\textbf{\textsc{refl}}\textbf{.\textsc{npst}} \textsc{neut}.sickness\\
\glt we three must start looking after our own lives and sicknesses’
\z



\ea%27
 \label{ex:vanegmond:27}
\gll nungkuwa-lhangwa ngə-\textbf{məni}-yakuwerribika-\textbf{jungu}-ma nara, wurri-yukwayuwa yakuwa-lhangwa wurra-məni-yakuwerribiki-na\\
\textsc{2sg}.\textsc{pro}-\textsc{dat} \textsc{neg}-\textbf{\textsc{bene}}-think-\textbf{\textsc{refl}}\textbf{.\textsc{npst}}-1.\textsc{foc} \textsc{neg} \textsc{3pl}-small.PL \textsc{12pl}.\textsc{pro}-\textsc{dat} \textsc{imp.2pl>3pl-bene}-think\textsc{-npst}\\
\glt don’t think about yourself, think about our children’
\z



From these examples, it appears that the reflexive overrides the benefactive applicative. In \REF{ex:vanegmond:27}, for example, the intransitive verb -\textit{yakuwerribiki}- think’ is made transitive by the benefactive marker (think of our children’), which in turn is detransitivized by the reflexive marker (think of yourself’): [\textsc{bene}-think]-\textsc{refl}. Regarding the ordering of semantic composition, it has not been tested with speakers whether reflexive formation can precede the benefactive and I have not found any instances in the textual data. Hence the question of how examples such as he cut himself for them’ are realised, i.e. whether the object slot can be re-filled by the benefactive argument (i.e. \textsc{bene}-[cut-\textsc{refl}]), is an interesting topic for further research.



\subsubsection{Coreference of subject with possessor}\label{sec:vanegmond:3.2.3}



The possessor is expressed by a pronoun marked with possessive case, as in \REF{ex:vanegmond:28}. The head noun is represented on the verb. When the subject and the possessor referent are the same person, number and gender, this can result in ambiguity, as in \REF{ex:vanegmond:29}.



\ea%28
 \label{ex:vanegmond:28}
\gll nungkə-lhangwa yikarba \textbf{ \textbf{nəngen}}-ngayindha\\
\textsc{2sg}.\textsc{pro}-\textsc{poss} \textsc{masc}.woomera \textsc{1sg}>\textsc{masc}-want.\textsc{npst}\\
\glt I want your woomera’
\z



\ea%29
 \label{ex:vanegmond:29}
\gll \textbf{enuwə-lhangwa} {yikarba} \textbf{nenə}-ngayindha\\
3m.\textsc{pro}-\textsc{poss} \textsc{masc}.woomera \textsc{3m}>\textsc{masc}-want.\textsc{npst}\\
\glt he\textsubscript{1} wants his\textsubscript{1/2} woomera’
\z



Since the free pronoun \textit{enuwa} and the subject prefix on the verb both mean third person singular masculine’, they can be both coreferential and disjoint. The intended meaning must come from the context or by specifying the possessor. However, even though constructions such as \REF{ex:vanegmond:29} potentially express coreference between two clause participants (here, agent and possessor), there is no special form that signals the coreference. Therefore, I do not consider such examples to represent reflexive constructions (see Haspelmath, this volume).



\subsubsection{Coreference of subject with spatial referent}\label{sec:vanegmond:3.2.4}

A spatial referent is expressed by a nominal marked with e.g. locative case for a stative location. When the subject and the spatial referent have the same person, number, gender features, this again can result in ambiguity, as in \REF{ex:vanegmond:30}.



\ea%30
 \label{ex:vanegmond:30}
\gll \textbf{yingən}-rrəngka yingarna \textbf{dh-akina-manja}\\
 \textsc{3f}>\textsc{masc}-see.\textsc{pst} \textsc{masc}.snake \textsc{3f}-that-\textsc{loc}\\
\glt she\textsubscript{1} saw a snake next to her\textsubscript{1/2}’
\z



Since the demonstrative \textit{dhakina} and the subject prefix on the verb both mean third person singular feminine’, they can be coreferential and disjoint and the sentence is ambiguous. But again, since there is no special form that signals the coreference, such examples do not instantiate the reflexive construction.


\section{Related functions and diachronic development of -\textit{jungwV}}
\label{sec:vanegmond:4}

The reflexive suffix is homophonous to the verb -\textit{jungwV}- to die’, which belongs to the same verb class. The reflexive suffix and the die’ verb can co-occur, suggesting they are not the same morpheme:



\ea%31
 \label{ex:vanegmond:31}
\gll akina akwalya na-\textbf{jungwa}{-ja-}\textbf{jungu}-nə-ma\\
\textsc{neut}.that \textsc{neut}.fish \textsc{neut}-\textbf{die}-\textsc{caus}-\textbf{\textsc{refl}}-\textsc{pst}-1.\textsc{foc}\\
\glt the fish killed itself’
\z



{This could mean that the reflexive suffix is a grammaticalized form of the (intransitive) verb -\textit{jungwV}- die’ that has bleached of its semantics.}
{ However, there is another possible historical source for this suffix, which is the reflexive reconstructed for the ancestor of the Gunwinyguan languages, called proto-Gunwinyguan (\citealt{BarryEtAl2003}). Most Gunwinyguan languages have a suffix that derives reflexive and/or reciprocal verbs from transitive stems. \citet[342]{AlpherEtAl2003} note that in many Gunwinyguan languages, reflexive and reciprocal meanings are covered by the same suffix, except in Wubuy, Ngandi and Warray (see \figref{fig:vanegmond:1}) (as Anindilyakwa was still presumed an isolate then, they did not include this language in their discussion). Due to the great distance between Warray on the one hand, and Wubuy and Ngandi on the other, they argue, the distinctive reflexive and reciprocal forms cannot be an innovation (p. 342-343). The contrast between the two must therefore be archaic, and they reconstruct reflexive *-\textit{yi}- and reciprocal *-\textit{nji}- for proto-Gunwinyguan (p\textsc{gn}).}

The Anindilyakwa reciprocal -\textit{yi}- (which synchronically has a rare alternate form -\textit{nji}-) could then have derived from p\textsc{gn} *-\textit{nji}- in the following way:

%added example
\ea 
p\textsc{gn} reciprocal *-\textit{nji}- > *-\textit{ji}- (loss of nasal) > -\textit{yi}- (lenition)
\z

The reflexive suffix -\textit{jungwV}- is more difficult to derive from p\textsc{gn} *-\textit{yi}-. It is possible that it is segmentable into -\textit{ji.ngwV}-, where -\textit{ji}- represents a hardened *-\textit{yi}-. The high vowel obtains its rounding from the rounded dorsal segment -\textit{ngwV}- (recall that this is how [u] is formed in Anindilyakwa):

%added example, original was just a line
\ea
p\textsc{gn} reflexive *-\textit{yi}- > *-\textit{ji}- (hardening) > *-\textit{ji-ngwV}- (addition of \textit{ngwV} segment) > -\textit{jungwV}-
\z 

Perhaps it was the verb -\textit{jungwV}- die’ that triggered the formation of the reflexive suffix.

\section{Summary}
\label{sec:vanegmond:5}

As expected of a polysynthetic language, the arguments of a verb are identified on the verb, in the case of Anindilyakwa by means of pronominal prefixes. Free pronouns are common but optional. The language has a range of argument-changing affixes, one of which is the reflexive suffix. Anindilyakwa reflexive voice is marked by a verbal suffix that occurs on transitive verbs and reduces the valency of the verb by one. It is used for all persons, numbers and degrees of animacy of the participants involved. The suffix signals that the agent subject is co-referential with the referent that previously occurred as the transitive object pronominal prefix. This is mostly a patient referent, but it can also be a beneficiary introduced by the benefactive applicative, or the recipient referent of an inherently ditransitive verb.



\section*{Abbreviations}
\noindent
\begin{tabularx}{.45\textwidth}{lQ}
1.\textsc{foc} & first person focalization marker\\
\textsc{bene} & benefactive\\
\textsc{caus} & causative\\
\textsc{CofR} & change of referent\\
\textsc{coll} & collective noun class\\
\textsc{dat} & dative\\
\textsc{du} & dual\\
\textsc{emph} & emphatic\\
Eng & English\\
f & feminine\\
\textsc{fem} & feminine noun class\\
\textsc{hort} & hortative\\
\textsc{inalp} & inalienable possession\\
\textsc{inch} & inchoative\\
\textsc{incl} & inclusive\\
\textsc{instr} & instrumental\\
\textsc{irr} & irrealis\\
\textsc{loc} & locative\\
m & masculine\\
\end{tabularx}

\begin{tabularx}{.45\textwidth}{lQ}
Mac & Macassan language\\
\textsc{masc} & masculine noun class\\
\textsc{neg} & negative\\
\textsc{neut} & neuter noun class\\
\textsc{npst} & non-past\\
\textsc{nzr} & nominalizer\\
p\textsc{gn} & proto-Gunwinyguan\\
\textsc{pl} plural\\
\textsc{poss} & possessive\\
\textsc{pro} & pronoun\\
\textsc{pst} & past\\
\textsc{purp} & purposive\\
\textsc{recp} & reciprocal\\
\textsc{rdp} & reduplication\\
\textsc{reas} & reason\\
\textsc{refl} & reflexive\\
\textsc{sg} & singular\\
tri & trial\\
\textsc{veg} & vegetable noun class\\
\end{tabularx}


\sloppy\printbibliography[heading=subbibliography,notkeyword=this]
\end{document}
