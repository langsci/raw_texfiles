\documentclass[output=paper]{langscibook}

\author{Christian Döhler\affiliation{Leibniz-Zentrum Allgemeine Sprachwissenschaft, Berlin}}
\title{{The middle template and other ways of expressing coreference in Komnzo}} 
\abstract{Coreference in Komnzo is expressed by various levels of verbal and nominal morphology. Komnzo verb morphology provides a middle construction to express coreference between the agent and the patient argument. Coreference that involves oblique arguments makes use of nominal enclitics for contrastive focus and emphasis. Long distance coreference is always ambigous in Komnzo. Most notably, the grammatical markers found in coferences situations function at a much broader level, i.e. they are coexpressive for related meanings. In most cases, coreference has to be inferred from the context. This chapter argues that there is no dedicated reflexive construction and no dedicated reflexive marker in Komnzo. The argumentation is based on a corpus of natural speech.\\
}
\affiliation{Leibniz-Zentrum Allgemeine Sprachwissenschaft, Berlin}
\date{}
\IfFileExists{../localcommands.tex}{
 \addbibresource{localbibliography.bib}
 \usepackage{langsci-optional}
\usepackage{langsci-gb4e}
\usepackage{langsci-lgr}

\usepackage{listings}
\lstset{basicstyle=\ttfamily,tabsize=2,breaklines=true}

%added by author
% \usepackage{tipa}
\usepackage{multirow}
\graphicspath{{figures/}}
\usepackage{langsci-branding}

 
\newcommand{\sent}{\enumsentence}
\newcommand{\sents}{\eenumsentence}
\let\citeasnoun\citet

\renewcommand{\lsCoverTitleFont}[1]{\sffamily\addfontfeatures{Scale=MatchUppercase}\fontsize{44pt}{16mm}\selectfont #1}
   
 %% hyphenation points for line breaks
%% Normally, automatic hyphenation in LaTeX is very good
%% If a word is mis-hyphenated, add it to this file
%%
%% add information to TeX file before \begin{document} with:
%% %% hyphenation points for line breaks
%% Normally, automatic hyphenation in LaTeX is very good
%% If a word is mis-hyphenated, add it to this file
%%
%% add information to TeX file before \begin{document} with:
%% %% hyphenation points for line breaks
%% Normally, automatic hyphenation in LaTeX is very good
%% If a word is mis-hyphenated, add it to this file
%%
%% add information to TeX file before \begin{document} with:
%% \include{localhyphenation}
\hyphenation{
affri-ca-te
affri-ca-tes
an-no-tated
com-ple-ments
com-po-si-tio-na-li-ty
non-com-po-si-tio-na-li-ty
Gon-zá-lez
out-side
Ri-chárd
se-man-tics
STREU-SLE
Tie-de-mann
}
\hyphenation{
affri-ca-te
affri-ca-tes
an-no-tated
com-ple-ments
com-po-si-tio-na-li-ty
non-com-po-si-tio-na-li-ty
Gon-zá-lez
out-side
Ri-chárd
se-man-tics
STREU-SLE
Tie-de-mann
}
\hyphenation{
affri-ca-te
affri-ca-tes
an-no-tated
com-ple-ments
com-po-si-tio-na-li-ty
non-com-po-si-tio-na-li-ty
Gon-zá-lez
out-side
Ri-chárd
se-man-tics
STREU-SLE
Tie-de-mann
} 
 \togglepaper[1]%%chapternumber
}{}

\begin{document}

\maketitle

\textbf{Keywords:} coreference, middle, non-reflexive, Yam languages, Tonda languages, Komnzo


\section{Introduction}\label{intro}
This chapter describes the expression of various types of coreference in Komnzo, a language of Southern New Guinea. Komnzo has no dedicated reflexive construction and no set of reflexive pronouns to encode coreference. Instead verbs employ an inflectional pattern, which I call ``the middle template''. The middle template is used for situation types, which have been described as typical middle situation types \citep{Kemmer1993}, for example intransitive, reflexive, reciprocal, impersonal, and passive situation types. In addition to the middle template, a number of other factors are important for the expression of a reflexive situation, e.g. the case frame, the semantics of the verb lexeme, and the context. My argumentation is that the category ``reflexive'' is not a language internal category in Komnzo. Instead coreference is encoded by grammatical means that are much broader in their function.

This article is structured in the following way: \S\ref{yamlanguages} provides background on Komnzo and situates the language within the Yam language family. \S\ref{methods} provides details on the nature of the data, on which this article is based. \S\ref{grambackground} introduces the relevant grammatical structures: distributed exponence (\S\ref{distexp}), morphological templates (\S\ref{verbtemp}), pronominals (\S\ref{pronominals}) and enclitics (\S\ref{enclitics}). The main part of the article provides examples of reflexive situations (\S\ref{reflxexp}), that is coreference between agent and patient (\S\ref{reflxexp-mid}), coreference that involve other semantic roles (\S\ref{reflxexp-other}), and coreference across clauses (\S\ref{longstance}). \S\ref{conclusion} summarises the structures and offers a conclusion.
\section{Komnzo within the Yam languages}\label{yamlanguages}
Komnzo belongs to the Yam language family (formerly known as Morehead-Maro group) which is found in the south of the island of New Guinea. Yam languages are spoken on both sides of the border that divides Indonesia and Papua New Guinea. The language family comprises three subgroups: Nambu languages in the east, Tonda languages in the west and Yei in the north, which is a family-level isolate. Komnzo is the easternmost language of the Tonda subgroup. Together with Anta, Wára, Kánchá, Kémä and Wèré, it belongs to the Eastern Tonda dialect chain \citep[36]{Doehler2018}. Komnzo is spoken by around 250 speakers in the village of Rouku and Morehead Station. Figure \ref{fig:Yam-family-2} below provides a map of the Yam language family.

\begin{figure}
	\centering
		\includegraphics[width=\textwidth]{figures/komnzomap.jpg}
	\caption{The Yam language family}
	\label{fig:Yam-family-2}
\end{figure}

The Southern New Guinea region stretches from the mouth of the Fly River in the east to the Digul River in the west. Despite a growing interest in the region, the level of documentation of Yam languages is still low compared to other languages families in New Guinea, not to speak of other regions of the world. Over the last decade, a number of researchers have published on specific features of Yam languages, for example their unique senary number system \citep{Donohue2008, Evans2009, Hammarstrom2009, Plank2009}, their complex patterns of verb inflection with respect of TAM \citep{Siegel2015, Evans2015Oxford} and valency \citep{Evans2015Valency, Siegel2017}. There are two grammars available of individual Yam languages, namely Komnzo \citep{Doehler2018} and Ngkolmpu \citep{Carroll2016}. There is the Nen dictionary \citep{Evans2019} and text collections for Nen \citep{Evans2010} and Komnzo \citep{Doehler2010}. Finally, there are two overview articles of the Southern New Guinea region \citep{EvansKlamer2012, EvansEtAl2017}.
\section{Methods and Data}\label{methods}
The data discussed in this chapter has been collected during 16 months of fieldwork between 2010 and 2015 as part of the author's PhD project. The project resulted in a grammar of Komnzo \citep{Doehler2018}, a dictionary and a text corpus.

The examples in this article are either elicited or taken from the text corpus. The corpus comprises 12 hours of transcribed and translated speech of various text genres, including both natural and stimuli-based narratives, procedurals, conversations and public speech (See Table \ref{table1}). All corpus examples below are marked with a source code that has been formatted in the following structure: tciYYYYMMDD SSS \#\#. The first part identifies the transcription file: the three letter ISO 639-3 code for Komnzo (tci) and the date of the recording (YYYYMMDD). The second part identifies the annotation within the transcription file: the tiers are sorted by speaker (SSS) and the annotation number on the respective tier (\#\#). Note that only 8 out of 12 hours have been interlinearized at the current stage. Evidence about the frequency of individual verb lexemes in this article is based on the interlinearized subset of the corpus.

\begin{table}
	\centering
	\caption{Database (in hh:mm:ss)}
	\begin{tabular}{lll}
		\hline
		Text type&transcribed&interlinearized\\
		\hline
		Conversation&01:01:55&-\\
		Stimulus task&01:49:51&01:26:43\\
		Narrative&06:14:18&05:45:15\\
		Procedural&02:11:36&01:02:44\\
		Public speech&00:42:38&-\\
		\hline
		Total&12:00:18&08:14:41\\
		\hline
		%\lspbottomrule
	\end{tabular}
	\label{table1}
\end{table}

The corpus can be accessed in two ways. The complete collection has been archived with The Language Archive, Nijmegen \citep{Doehler2010}. This includes around 60 hours of audio-visual footage, text as well as observational recordings, transcribed as well as untranscribed. The corpus of transcribed texts has been archived at Zenodo \citep{Doehler2020}. The latter contains all transcription files in ELAN format (.eaf) in a single zip file. The associated audio and video files are accessible in separate session nodes at both locations. The title of a session node follows the formatting of the source code as described above.
\section{Grammatical background}\label{grambackground}
Komnzo is a double-marking language, in which the verb indexes core arguments and noun phrases are flagged for case. The case marking is organised in an ergative-absolutive system. In addition to four core cases (absolutive, ergative, dative, possessive), there are 13 semantic cases. The system of argument indexation in verbs is of the split-S type: The single argument of an intransitive verb is indexed in the same slot as the A argument of a transitive verb, if the event is dynamic. However, it is indexed in the same slot as the P argument, if the event is stative.

In the following sections I describe the principle of distributed exponence in \S\ref{distexp}, which is important for the understanding of verb morphology as well as for the glossing convention adopted in this article. In \S\ref{verbtemp} follows a description of verb templates. In \S\ref{pronominals}, I describe the pronominals in Komnzo: indexes (\S\ref{prefixseries}) and free pronouns (\S\ref{pron}). In \S\ref{enclitics}, I describe two nominal enclitics that play a role in the expression of reflexive situations.
\subsection{Distributed exponence}\label{distexp}
As other languages of the Yam family, Komnzo has complex verb morphology. Verbs express person, number and gender of up to two participants, 18 TAM categories, valency, directionality and deictic status. Complexity lies not only in the amount of grammatical categories that can be expressed morphologically, but also in the way how these categories are encoded. This is best described by the term ``distributed exponence'', which has surfaced in the recent literature on multiple exponence \citep{CaballeroHarris2012}. Carroll gives a precise definition of distributed exponence in his description of Ngkolmpu as ``the phenomenon in which morphosyntactic and morphosemantic properties are marked non-redundantly at multiple inflectional sites'' (\citeyear[268]{Carroll2016}).

In Komnzo verb morphology, this plays out as underspecification of individual morphs. Consider Table \ref{table2} below, in which the verb \emph{thoraksi} (\emph{thor}-|\emph{thorak}-) `appear' is inflected for different TAM categories.\textsc{f}ootnote{Komnzo verb lexemes have two stems, which are sensitive to aspect. The formal relationship between the two stems ranges from suffixation to consonant mutation to full suppletion. In this chapter, I will list the two stems in brackets after the infinitive in this way: \emph{thoraksi} (\emph{thor}-|\emph{thorak}-).}

\begin{table}
	\caption{\emph{thoraksi} `appear' in a \textsc{sg.masc} frame}
	\begin{tabular}{ll}
		\lsptoprule
		TAM category&inflected form\\
		\hline
		non-past&{y-thorak-wr}\\
		recent-past imperfective&{su-thorak-wr}\\
		recent-past durative&{y-thorak-wr-m}\\
		recent-past perfective&{sa-thor}\\
		past imperfective&{y-thorak-wr-a}\\
		past durative&{su-thorak-wr-m}\\
		past perfective&{sa-thor-a}\\
		iterative&{su-thor}\\
		\lspbottomrule
	\end{tabular}
	\label{table2}
\end{table}

It becomes clear from the table that the inflectional sites (the prefix, the verb stem, and the suffixes) contribute some information to TAM without encoding a particular TAM value. For example, the prefix \emph{y-} occurs in the non-past, the recent-past and the past tense, both in imperfective and durative aspect. Likewise, the verb stem {thor} is involved in expressing perfective aspect, but also the iterative. In other words, the morphs in each inflectional site are underspecified as to their grammatical meaning, in this case the TAM category. Underspecification of this type is also found in other grammatical categories, such as number and valency.

Distributed exponence prompts us to take the inflected verb, not the morpheme, as the level of analysis. As a practical consequence, I gloss verbs in a word-in-paradigm style \citep{Matthews1974}, as in (\ref{ex:doehler:1}) and (\ref{ex:doehler:2}) below. In the morpheme tier, slanted lines separate the verbstem from the inflectional material. In the gloss tier, the inflected verb form is placed in its paradigm by listing information in the following order: argument structure, TAM, directionality, and (following a forward slash) lexeme translation. Additionally, I put the entire verb gloss in square brackets followed by an abbreviation of the respective verb template in subscript. The copula in (\ref{ex:doehler:1}) occurs in the prefixing template (\textsc{pref}), while the verb in (\ref{ex:doehler:2}) occurs in the transitive template (\textsc{trans}). The role of verb templates will be addressed in the next section.

\ea
\label{ex:doehler:1}
	\gll {kabe} {y\textbackslash{thorak}/wr}\\
	man [\textsc{sg}.\textsc{masc}:\textsc{nonpast}:\textsc{ipfv}/appear]\textsubscript{\textsc{pref}}\\
	\glt `The man appears.'
	\z
	
\ea \label{ex:doehler:2}
	\gll {kabe=f} {nge} {wn\textbackslash{zä}/nzr}\\
	man=\textsc{erg}.\textsc{sg}{} child [\textsc{sg}>\textsc{sg}.\textsc{f}:\textsc{nonpast}:\textsc{ipfv}:\textsc{venit}/carry]\textsubscript{\textsc{trans}}\\
	\glt `The man carries the girl.'
\z

\subsection{Verb templates}\label{verbtemp}
Inflected verbs in Komnzo can be classified into prefixing, middle, and ambifixing depending on whether the prefix, the suffix or both are employed for indexing core arguments. I use the term ``verb template'' for this arrangement of morphological slots. Hence, we can say that a particular lexeme ``occurs in a prefixing template'' or that it ``occurs in an ambifixing template''. Templates are lexically determined for some verbs, which means that we can speak of ``a prefixing verb'' or of ``a middle verb''. However, for the majority of verbs, the system of templates is somewhat flexible, that is a verb stem can occur in different templates. Thus, we can describe a particular lexeme by stating that ``it occurs in the middle template and the ambifixing template, but not in the prefixing template''. Note that the distinction between prefixing, middle and ambifixing is based on a purely structural prespective for now. As we will see below, labels such as `intransitive' and `transitive' are a matter of token frequency of individual lexemes in Komnzo.

The morphological slots involved in the definition of templates are the following: (i) the undergoer prefix, (ii) the diathetic prefix, and (iii) the actor suffix. The undergoer prefix indexes a core argument, in which case it shows agreement in person, number and gender (See \S\ref{prefixseries}). The undergoer slot can also be filled with the middle prefix, which is invariant for these categories. The diathetic slot can be empty or be filled by the diathetic prefix \emph{a-}. The neutral label ``diathetic'' captures the fact that for some verbs its function is to increase valency, whereas for other verbs it decreases valency. Finally, the actor suffix indexes a core argument in the middle and ambifixing templates, in which case it shows agreement in person and number. In the prefixing template, the actor slot is absent. Table \ref{tab:doehler:3} provides a schematic overview of the possible templates. The column for the undergoer slot lists the morph \emph{y-} for \textsc{sg}.\textsc{masc}, with the exception of the middle template, where the morph is \emph{ŋ-}. The column for the actor slot lists the morph \emph{-th} for 2|3\textsc{nsg}.

\begin{table}
	\centering
	\caption{Verb templates}
	\begin{tabular}{lp{2cm}p{2cm}p{1cm}p{2cm}}
		\lsptoprule
		template&undergoer prefix&diathetic prefix&verb stem&actor suffix\\
		\hline
		prefixing&\langscicheckmark ({y-})&-&\langscicheckmark&-\\
		prefixing
		(indirect object)&\langscicheckmark ({y-})&\langscicheckmark ({a-})&\langscicheckmark&-\\
		middle&\langscicheckmark ({ŋ-})&\langscicheckmark ({a-})&\langscicheckmark&\langscicheckmark ({-th})\\
		ambifixing&\langscicheckmark ({y-})&-&\langscicheckmark&\langscicheckmark ({-th})\\
		ambifixing 
		(indirect object)&\langscicheckmark ({y-})&\langscicheckmark ({a-})&\langscicheckmark&\langscicheckmark ({-th})\\
		\lspbottomrule
	\end{tabular}
	\label{tab:doehler:3}
\end{table}

\tabref{tab:doehler:3} shows that there are more than the three templates mentioned above. This is caused by the absence versus presence of the diathetic prefix. Thus, the prefixing and the ambifixing template can be subdivided further. The prefixing template without the diathetic prefix indexes an S or P argument in the undergoer slot. It is labelled simply `prefixing template' (\textsc{pref} in the gloss). The prefixing template with the diathetic prefix indexes a beneficiary or possessor argument. It is labelled `indirect object prefixing template' (\textsc{io.pref} in the gloss). Likewise the ambifixing template can occur without or with the diathetic prefix. Without the diathetic prefix, the undergoer prefix indexes a P argument. With the diathetic prefix, it indexes a beneficiary or possessor. I label these two templates as `transitive ambifixing template' and `ditransitive ambifixing template' (\textsc{trans} and \textsc{ditrans} in the gloss). For reasons of better readability, I henceforth drop `ambifixing' from the labels and instead simply use `middle', `transitive' and `ditransitive template'. Note that these labels depart from a purely structural perspective and reflect the function of these templates. I provide more concrete examples in (\ref{ex:doehler:3a}--\ref{ex:doehler:3e}).

The system of verb templates is lexically determined for some verbs, while it is fluid for most verbs. This fluidity is one of the central aspects in understanding Komnzo verb morphology. That being said, there are only a handful of lexemes, which can enter into all five templates. Below, I present the verb \emph{migsi} (\emph{mig}-|\emph{mir}-) `hang' in all five templates to show how template choice impacts on argument structure and, more generally, on the meaning of the verb. The elicited examples in (\ref{ex:doehler:3}) are appear here in a reduced gloss, which ignores all TAM information and stem variations.\textsc{f}ootnote{The few lexemes, which can enter into all five template show a stem alternation that is only used in the prefixing template. The stem of \emph{migsi} for the prefixing template is \emph{mi}, while it is \emph{mig} or \emph{mith} for the middle and the ambifixing template depending on aspect. The \emph{thgr} element in (\ref{ex:doehler:3}a--b) is a stative non-dual suffix that has not been segmented here. Likewise (\ref{ex:doehler:3}c-e) also appear in a simplified gloss. The \emph{-wr} suffix is in fact marking aspect and non-dual number, while the \textsc{sg}{} is expressed as a zero. For further information on verb morphology, I refer the reader to the Komnzo grammar \citep{Doehler2018}.} Note that the examples (\ref{ex:doehler:3a}--\ref{ex:doehler:3d}) correspond to the five templates as they are listed in \tabref{tab:doehler:3} above.

\ea
	\label{ex:doehler:3}
	\ea
			\gll {y-mithgr}\\
			\textsc{sg}.\textsc{masc}-hang\\
			\glt `He hangs.'
			\label{ex:doehler:3a}
		\ex
			\gll {y-a-mithgr}\\
			\textsc{sg}.\textsc{masc}-\textsc{dia}-hang\\
			\glt `Something of his (or for him) hangs.'
			\label{ex:doehler:3b}
			\ex
			\gll {ŋ-a-mig-wr}\\
			\textsc{m}-\textsc{dia}-hang-\textsc{sg}\\
			\glt `It hangs itself up.'
			\label{ex:doehler:3c}
			\ex
			\gll {y-mig-wr}\\
			\textsc{sg}.\textsc{masc}-hang-\textsc{sg}\\
			\glt `S/He hangs him up.'
			\label{ex:doehler:3d}
			\ex
			\gll {y-a-mig-wr}\\
			\textsc{sg}.\textsc{masc}-\textsc{dia}-hang-\textsc{sg}\\
			\glt `S/He hangs up something of his (or for him).'
			\label{ex:doehler:3e}
	\z
	\z
	
The prefixing template (\ref{ex:doehler:3a}--\ref{ex:doehler:3b}) is used for intransitive event types that are stative, while the middle template (\ref{ex:doehler:3c}) is used for intransitive event types that are dynamic. Note that the diathetic prefix is part of the middle template. Thus, Komnzo has a split-S system that is based on event dynamicity. In terms of numbers of lexemes, the middle template is the preferred template for intransitive verbs. The coreference situation in (\ref{ex:doehler:3c}) is only caused by the semantics of the verb {migsi} `hang' and an alternative, though admittedly long-winded translation of (\ref{ex:doehler:3c}) would be `it assumes a hanging position'. As I argue in this chapter, the middle template is coexpressive for a range of functions, which would be termed intransitive, impersonal, reflexive, reciprocal and passive in languages that have dedicated constructions for these. However, there is no constructional distinction between these in Komnzo. Example (\ref{ex:doehler:3d}) shows the transitive template, which is the ``major biactant construction'' \citep{Lazard2002}. Finally, example (\ref{ex:doehler:3d}) shows the ditransitive template, which differs from the transitive template in that the diathetic prefix has been added to the verb. This is the way to express ditransitives in Komnzo, and one can argue that all ditransitives are derived in the language \citep[206]{Doehler2018}.

For the majority of verb lexemes in Komnzo, labels such as `transitive verb' or `intransitive verb' are a matter of frequency of template choice. I will give three examples to illustrate that claim by showing template frequencies in the text corpus. I start with \emph{msaksi} (\emph{msak}|\emph{ms}) `sit, dwell, stay', which occurs 331 times in the corpus. 296 tokens are in the prefixing template with the meaning `sit, dwell, stay', as in (\ref{ex:doehler:4}). 30 tokens occur in the middle template with the meaning `sit (self) down, assume a sitting position', as in (\ref{ex:doehler:5}). Finally, 5 tokens occur in the transitive template with the meaning `sit someone down', as in (\ref{ex:doehler:6}). Note that example (\ref{ex:doehler:6}) lacks noun phrases expressing the agent and patient. If these were expressed, they would appear in ergative and absolutive case, respectively. The skewing of the distribution allows us to characterise \emph{msaksi} as a stative, intransitive, prefixing verb (\ref{ex:doehler:4}). It follows that the occurence of this verb in the middle template in (\ref{ex:doehler:5}) should be analysed as an alternation that has to do with dynamicity. Likewise the occurence in the transitive template in (\ref{ex:doehler:6}) should be analysed as a causative alternation.


\ea
	\label{ex:doehler:4}
	\gll {nafa-ŋare} {komnzo} {wä\textbackslash{m}/nza} {masu=n}\\
	\textsc{3rd}.\textsc{poss}-wife(\textsc{abs}) just [\textsc{sg}.\textsc{f}:\textsc{pst}:\textsc{ipfv}/sit]\textsubscript{\textsc{pref}} Masu=\textsc{loc}{}\\
	\glt `His wife just stayed in Masu.'
	[tci20110810-2 MAB 8]
	%\Corpus{tci20110810-2}{MAB 8}
\z



\ea
    \label{ex:doehler:5}
	\gll {äusi} {fäth} {z-zrzü=me} {ŋa\textbackslash{msak}/wa}\\
	old.woman \textsc{dim}(\textsc{abs}) \textsc{redup}-knee=\textsc{ins}{} [\textsc{sg}:\textsc{pst}:\textsc{ipfv}/sit]\textsubscript{\textsc{mid}}\\
	\glt `The old woman sat down on her knees.'	[tci20120925-01 MKA 400]
	%\Corpus{tci20120925-01}{MKA 400}
\z


\ea
    \label{ex:doehler:6}
	\gll {wati} {y\textbackslash{msak}/wrth} {fof}\\
	then [2|3\textsc{pl}>3\textsc{sg}.\textsc{masc}:\textsc{nonpast}:\textsc{ipfv}/sit]\textsubscript{\textsc{trans}} \textsc{emph}\\
	\glt `Then they really sit him down.'
	[tci20120909-06 KAB 91]%\Corpus{tci20120909-06}{KAB 91}
\z

A second example is the verb \emph{brigsi} (\emph{brig-}|\emph{brim-}) `return', which occurs 181 times in the corpus. Note that the prefixing template is not available for this lexeme. 137 tokens occur in the middle template with the meaning `return, go back', as in (\ref{ex:doehler:7a}). Only 44 of the tokens occur in the transitive template with the meaning `bring something or someone back', as in (\ref{ex:doehler:7b}). In Example (\ref{ex:doehler:7}), the speaker describes the slash-and-burn agriculture, whereby gardens are shifted to a new location each year. Thus, \emph{brigsi} is a dynamic, intransitive, middle verb. The occurence in the transitive template in (\ref{ex:doehler:7b}) should be analysed as a causative alternation.

\ea
	\label{ex:doehler:7}
	\ea
		\gll {fthmäsü} {za\textbackslash{bth}/e} {bä} {we} {kwan\textbackslash{brig}/wre} {we} {z=n\textbackslash{rä}/} {zena}\\
			meanwhile [\textsc{f}pl>\textsc{sg}.\textsc{f}:\textsc{rpst}:\textsc{pfv}/finish]\textsubscript{\textsc{trans}} \textsc{med}{} also [\textsc{f}pl:\textsc{rpst}:\textsc{ipfv}:\textsc{vent}/return]\textsubscript{\textsc{mid}} also \textsc{prox}=[\textsc{f}pl:\textsc{nonpast}:\textsc{ipfv}/be]\textsubscript{\textsc{pref}} now\\
			\glt `Meanwhile we have finished (the soil) there and we returned now ...
			\label{ex:doehler:7a}
	
	\ex
			\gll {zane} {ysakwr=en} {zf} {za\textbackslash{thkäf}/e} {z=\textbackslash{rä}}/ {ŋarake} {thun\textbackslash{brig}/wre} {zena}\\
			\textsc{prox}{} season=\textsc{loc}{} \textsc{imm}{} [\textsc{f}pl>\textsc{sg}.\textsc{f}:\textsc{rpst}:\textsc{pfv}/start]\textsubscript{\textsc{trans}} \textsc{prox}=[\textsc{sg}.\textsc{f}:\textsc{nonpast}:\textsc{ipfv}/be]\textsubscript{\textsc{pref}} garden [\textsc{f}pl>2|3pl:\textsc{rpst}:\textsc{ipfv}:\textsc{vent}/return]\textsubscript{\textsc{trans}} now\\
			\glt ... this year we have started (making gardens) right here. We brought back the gardens now.' [tci20120922-08 DAK 80-81]
			%\Corpus{tci20120922-08}{DAK 80-81}
			\label{ex:doehler:7b}
		\z
	\z


The third example is the verb \emph{zrin} (\emph{zä-}|\emph{thor-}) `carry', which occurs 109 times in the corpus. Again, the prefixing template is not available for this lexeme. Only 3 tokens occurs in the middle template, as in (\ref{ex:doehler:8}), while the remaining 106 are in the transitive template, as in (\ref{ex:doehler:9}). Example (\ref{ex:doehler:8}) comes from a procedural text about yam cultivation, while example (\ref{ex:doehler:9}) is from a text about sorcery. It follows that we have to analyse \emph{zrin} as a transitive verb with the meaning `carry something'. Its occurence in the middle template in (\ref{ex:doehler:8}) is a passive alternation.


\ea
	\gll {ane} {thara=karä=sü} {kra\textbackslash{zä}/nzrth} {bobo}\\
	\textsc{dem}{} bundle=\textsc{prop}=\textsc{etc}{} [2|3pl:\textsc{irr}:\textsc{ipfv}/carry]\textsubscript{\textsc{mid}} \textsc{med}:\textsc{all}\\
	\glt `They (yams) will be carried there with the bundle and all.' [tci20121001 ABB 27]
	%\Corpus{tci20121001}{ABB 27}
	\label{ex:doehler:8}
\z


\ea
    	\label{ex:doehler:9}
	\gll {bäne} {zra\textbackslash{zä}/nzr} ... {fenz} {kzi=kaf}\\
	\textsc{recog}.\textsc{abs}{} [\textsc{sg}>\textsc{sg}.\textsc{f}:\textsc{irr}:\textsc{ipfv}/carry]\textsubscript{\textsc{trans}} (.) pus(\textsc{abs}) barktray=\textsc{prop}\\
	\glt `He will carry the watchamacallit ... the pus in the barktray.' [tci20130903-04 RNA 49-51]
	%\Corpus{tci20130903-04}{RNA 49-51}
\z


For some verb stems, the occurence in different templates may alter the meaning to such a degree, that these are best analysed as separate lexemes. One such example is \emph{rbänzsi} (\emph{rbänz-}|\emph{rbs-}), which has the meaning `untie' in the transitive template, but `explain' in the ditransitive template (lit. `to untie for someone'). A second example, is \emph{karksi} (\emph{kark-}|\emph{kar-}), which has the meaning `pull' or `smoke' in the middle template, but `take away from someone' in the transitive template.

As a summary to this section, I want to highlight two points. First, verb templates and the possibility for verb stems to occur in more than one verb template is central to the analysis of Komnzo verb morphology. Labels such as `intransitive verb' or `transitive verb' only make sense if one looks at the frequency of template choice in natural speech. Komnzo is thus a good example of what Lazard describes as ``scalar transitivity'' (\citeyear[166]{Lazard2002}).

Secondly, the middle template is a construction that is coexpressive for a number of semantic situation types. We can describe these types in the following way: single actor with coreference (reflexive), single actor without coreference (intransitive), dummy-actor or empty-actor (impersonal), mutual action (reciprocal), patient topicalization (passive), patient backgrounding (antipassive). Henceforth, I will use the labels in brackets to refer to the semantic situation types. In the main part of this chapter in \S\ref{reflxexp}, I give examples of these situation types to argue against a dedicated reflexive construction in Komnzo. 
\subsection{Pronominals}\label{pronominals}
\subsubsection{Indexes}\label{prefixseries}
For the purpose of describing argument indexation in verbs in more detail, it is useful to take a look at the prefixes. As we have seen in \S\ref{distexp}, distributed exponence means that the prefixes are underspecified with respect to TAM. For this reason, I have labelled the three series with Greek letters as α, β and γ in Table \ref{preftable}. Each series has distinct forms for person (1,2,3), number (singular vs. non-singular) and gender in third person singular (feminine vs. masculine). The prefixes are underspecified for number, hence the label \textsc{nsg}{} for non-singular. As in many Yam languages, Komnzo verbs have a dedicated verbal slot that marks duality (dual vs. non-dual). The three-value number system (singular, dual, plural) is constructed by combining two binary oppositions: singular vs. non-singular and dual vs. non-dual.

\begin{table}
	\centering
		\caption{Person prefixes}
		\begin{tabular}{llllll}
			\lsptoprule
			Gloss & α & β & γ\\
			\hline
			\textsc{f}\textsc{sg} &{wo-} & {kw-} & {zu-}\\
			\textsc{f}\textsc{nsg} &{n-} & {nz-} & {nzn-}\\
			\textsc{2sg} &{n-} & {nz-} & {nzn-}\\
			\textsc{sg}.\textsc{f} &{w-} & {z-} & {z-}\\
			\textsc{sg}.\textsc{masc} &{y-} & {s-} & {s-}\\
			2|3\textsc{nsg} &{e-} & {th-} & {th-}\\
			\textsc{m} &{ŋ-} & {k-} & {z-}\\
			\lspbottomrule
	\end{tabular}
	\label{preftable}
\end{table}

 As we can see in \tabref{preftable}, there are a number of syncretisms in the system. Most of them are disambiguated by other elements in the verb morphology. For example, the syncretism between the β and γ series in 3\textsc{sg}{} and 2|3\textsc{nsg} is disambiguated by the fact that these prefix series combine with different verb stems. Most Komnzo verbs have two verb stems that are sensitive to aspect. What is important for this chapter, is the fact that each prefix series has a morph that is invariant for number and person, which is shown in the last row of Table \ref{preftable}. This is the middle marker (\textsc{m}) used for the middle template as described in \S\ref{verbtemp}. We can see the middle marker as the first element in the verbs of the some of the above examples, even though the morphs have not been segmented these examples: \emph{ŋ-} (\ref{ex:doehler:5}), \emph{k-} (\ref{ex:doehler:7a}, \ref{ex:doehler:8}) and \emph{z-} (\ref{ex:doehler:9}).
\subsubsection{Free pronouns}\label{pron}
Komnzo has a rich set of free pronouns as we can see in \tabref{perspron}. Free pronouns encode the four core cases: absolutive, ergative, dative and possessive. Core cases flag those arguments that can be indexed in the verb. Furthermore, free pronouns express a number of obliques with a range of semantic cases, which cannot be indexed in the verb. Unlike other Yam languages, for example Nen \citep{Evans2015Valency} and Ngkolmpu \citep{Carroll2016}, there are no reflexive or reciprocal pronouns in Komnzo.

\begin{table}
        \caption{\label{perspron} Free pronouns}
	\fittable{
	\small
		\begin{tabular}{lllllll}
			\lsptoprule
			case&\textsc{fsg}&\textsc{fnsg}&\textsc{2sg}{}& 2\textsc{nsg}&\textsc{3sg}{}&\textsc{3sng}{}\\ \hline
			\textsc{abs}{}&{nzä}&\multirow{2}{*}{{ni}}&\multicolumn{2}{c}{{bä}}&\multicolumn{2}{c}{{fi}}\\
			\textsc{erg}{}&{nze}&&{be}&{bné}&{naf}&{nafa}\\
			\textsc{dat}{}&{nzun}&{nzenm}&{bun}&{benm}&{nafan}&{nafanm}\\ 
			\textsc{poss}{}&{nzone}&{nzenme}&{bone}&{benme}&{nafane}&{nafanme}\\ 
			\textsc{char}{}&{nzonema}&{nzenmema}&{bonema}&{benmema}&{nafanema}&{nafanmema}\\ 
			\textsc{assoc}{}&{ninrr}&{ninä}&{bnrr}&{bnä}&{nafrr}&{nafä}\\ 
			\textsc{loc}{}&{nzudben}&{nzedben}&{budben}&{bedben}&{nafadben}&{nafanmedben}\\ 
			\textsc{all}{}&{nzudbo}&{nzedbo}&{budbo}&{bedbo}&{nafadbo}&{nafanmedbo}\\ 
			\textsc{abl}{}&{nzudba}&{nzedba}&{budba}&{bedba}&{nafadba}&{nafanmedba}\\ 
			\textsc{purp}{}&{nzunar}&{nzenar}&{bunar}&{benar}&{nafanar}&{nafanar}\\
			\lspbottomrule
		\end{tabular}
		}

\end{table}
\subsection{Further nominal morphology}\label{enclitics}
There are two nominal enclitics in Komnzo, which play a role in the expression of reflexive situations. The two clitics are the exclusive clitic {=nzo} (\textsc{only}), which marks contrastive focus, and the emphatic clitic \emph{=wä} (\textsc{emph}), which marks emphasis. The former is related to the adverb \emph{komnzo} `only, still, just' on which the name of the language is based.\textsc{f}ootnote{There is no information as to the origin of the name Komnzo. However, it seems reasonable to assume that it originated in a misunderstanding on the part of a colonial officer. He must have mistaken \emph{komnzo} as a proper name in the phrase \emph{komnzo zokwasi}, which means `just language' or `only words', when he enquired about the language or tribe name. Note that a number of Yam language names of the Tonda branch are based on words that mean `only, still, just', e.g. Kánchá, Kémä, Wára, Wérè and Anta.} As can be seen in (\ref{ex:doehler:10}) and (\ref{ex:doehler:11}), neither of the two enclitics is a reflexive marker. There is no coreference in the two examples. In (\ref{ex:doehler:10}), \emph{=nzo} attaches to the S argument of the copula. In (\ref{ex:doehler:11}), \emph{=wä} attaches to the A argument. In \S\ref{reflxexp}, I will show examples in which these enclitics facilitate coreference. What is important here, it the fact that they do not encode coreference, i.e. they are non-reflexive markers.


\ea
    \label{ex:doehler:10}
	\gll {ni=nzo} {miyatha} {n\textbackslash{rä}/ra} {wämne} {dunzi=ma} {fof}\\
	\textsc{f}nsg=\textsc{only}{} knowledge [\textsc{f}pl:\textsc{pst}:\textsc{ipfv}/be]\textsubscript{\textsc{pref}} tree arrow=\textsc{char}{} \textsc{emph}{}\\
	\glt `Only we knew about the arrow in the tree.'
	[tci20120814 ABB 106]
	%\Corpus{tci20120814}{ABB 106}

\z


\ea
\label{ex:doehler:11}
	\gll {ni=wä} {komnzo} {ŋarake} {b=ä\textbackslash{fiyok}/wre}\\
	\textsc{f}nsg=\textsc{emph}{} still garden(\textsc{abs}) \textsc{med}=[\textsc{f}pl>2|3pl:\textsc{nonpast}:\textsc{ipfv}/make]\textsubscript{\textsc{trans}}\\
	\glt `We are still making gardens there.'
	[tci20120922-08 DAK 75]
	%\Corpus{tci20120922-08}{DAK 75}

\z


For other Yam languages, a dedicated set of reflexive/reciprocal pronouns (\textsc{r/r}) has been described. In Ngkolmpu, these are built from the ergative pronouns by adding a /\emph{to}/ element, for example: \emph{ngkai} 1\textsc{sg}.\textsc{erg} vs. \emph{ngkaito} 1\textsc{sg}.\textsc{r/r} or \emph{piengku} 3\textsc{sg}.\textsc{erg}{} vs. \emph{piengkuto} 3\textsc{sg}.\textsc{r/r} \citep[138]{Carroll2016}. The same is true in Nen, for which Evans describes a set of reflexive/reciprocal pronouns featuring a word-final /\emph{nzo}/ element, for example \emph{bm} \textsc{2sg}{} vs. \emph{benzo} \textsc{2sg}.\textsc{r/r} and \emph{bbenzos} 2\textsc{nsg}.\textsc{r/r} \citep[1091]{Evans2015Valency}.

The /\emph{to}/ element in Ngkolmpu and the /\emph{nzo}/ element in Nen are certainly cognate with the exclusive enclitic \emph{=nzo} in Komnzo. However, it has not grammaticalised into a set of reflexive/reciprocal pronouns. On the contrary, it may combine with any nominal, as we can see in (\ref{ex:doehler:12}), where it attaches to a proper name.


\ea
    \label{ex:doehler:12}
	\gll {bres=f=nzo} {kwrfar} {wämne} {zan=me} {di} {sa\textbackslash{frnz}/a}\\
	Bres=\textsc{erg}.\textsc{sg}=\textsc{only}{} big.wallaby(\textsc{abs}) tree beating=\textsc{ins}{} back.of.head(\textsc{abs}) [\textsc{sg}>3\textsc{sg}.\textsc{masc}:\textsc{pst}:\textsc{pfv}/belt]\textsubscript{\textsc{trans}}\\
	\glt `Only Bres struck down the big wallaby by beating it with a stick.' [tci20130927-06 MAB 6]
	%\Corpus{tci20130927-06}{MAB 6}
\z
	

\section{Reflexive situations}\label{reflxexp}
This section describes the expression of reflexive situations in Komnzo. I discuss coreference between agent and patient in \S\ref{reflxexp-mid}, which is followed by a description of coreference between agent or patient and other semantic roles in \S\ref{reflxexp-other}. Lastly, I discuss coreference across clauses in \S\ref{longstance}. In each section, I show that the relevant construction or relevant marker is coexpressive, i.e. it is not solely used for reflexive situations.
\subsection{Coreference between agent and patient}\label{reflxexp-mid}
As it has become clear from \S\ref{verbtemp}, the middle template is the strategy to express coreference between agent and patient. Recall that the middle template expresses reflexive situations in addition to a number of other situation types, such as intransitive, reciprocal, impersonal, passive, as well as antipassive. In this section, I discuss these situation types and show evidence from the Komnzo text corpus.

In example (\ref{ex:doehler:13}), we see the verb \emph{traksi} (\emph{trak}|\emph{tr}) `fall' used in the middle template. This verb can occur in the transitive template, with the somewhat unexpected meaning `catch fish'.\textsc{f}ootnote{By analogy to the lexemes discussed in \S\ref{verbtemp}, one would expect that a lexeme, which means `fall' in the middle template, would mean `drop' in the transitive template. Instead, the meaning of `drop' is expressed by a different lexeme.} Hence, the stem of \emph{traksi} has to be analysed as two different lexemes depending on template choice. There are 14 tokens of \emph{traksi} `fall' in the corpus and 13 of these occur in the middle template. There is one token in the ditransitive template, to which I will return in \S\ref{reflxexp-other}. The important point here is that the middle template is used to express an intransitive situation. Infact, this is the main function of this template. It is striking that the middle template is much more frequent than the prefixing template when comparing individual verb lexemes. The following list of intransitive verbs occur almost exclusively in the middle template: \emph{yak} (\emph{kwi}|\emph{math}) `run', \emph{mnzeraksi} (\emph{mnzerak}|\emph{mnzer}) `fall asleep', \emph{rninzsi} (\emph{rninz}|\emph{rnith}) `smile', \emph{borsi} (\emph{bor}|\emph{both}) `laugh', \emph{farksi} (\emph{fark}|\emph{far}) `set off', \emph{sogsi} (\emph{sog}|\emph{söbäth}) `ascend', \emph{rsörsi} (\emph{rsör}|\emph{rsöfäth}) `descend', \emph{bznsi} (\emph{bzn}) `work', \emph{rüsi} (\emph{rü}|\emph{rüth}) `rain', (\emph{rä}|\emph{r}) `think'.


\ea
	\gll {kwa} {ŋa\textbackslash{trak}/wr} {zane} {nge} {z=\textbackslash{rä}/}.\\
	\textsc{f}ut{} [\textsc{sg}:\textsc{nonpast}:\textsc{ipfv}/fall]\textsubscript{\textsc{mid}} \textsc{dem}:\textsc{prox}{} child(\textsc{abs}) [\textsc{prox}=\textsc{sg}.\textsc{f}:\textsc{nonpast}:\textsc{ipfv}/be]\textsubscript{\textsc{pref}}\\
	\glt `It will fall down, this baby girl here.'
	[tci20111004 TSA 110]
	%\Corpus{tci20111004}{TSA 110}
	\label{ex:doehler:13}
\z


In (\ref{ex:doehler:14}), we see an example of coreference between agent and patient with the verb \emph{ttüsi} (\emph{ttü}|\emph{ttüth}) `write, paint' used in the middle template. This lexeme occurs only 7 times in the corpus and (\ref{ex:doehler:14}) is the only example in the middle template. The remaining tokens are in the transitive template. Hence, the use of the middle template is a reflexive alternation of an otherwise transitive verb.


\ea
	\gll {zä} {kwa} {ŋa\textbackslash{ttü}/nzé}.\\
	\textsc{prox}{} \textsc{f}ut{} [\textsc{f}sg:\textsc{nonpast}:\textsc{ipfv}/paint]\textsubscript{\textsc{mid}}\\
	\glt `I will paint myself here.' [tci20130907-02 JAA 110-111] %\Corpus{tci20130907-02}{JAA 110-111}
	\label{ex:doehler:14}
\z


Example (\ref{ex:doehler:15}) shows another example of coreference between agent and patient. The verb \emph{marasi} (\emph{mar}) `see' is used in the middle template to express `look after yourself'. This example comes from a public speech, in which the speaker admonishes the audience about the excessive consumption of alcohol during an upcoming dance. For stylistic reasons, he uses the singular instead of the plural.


\ea
    \label{ex:doehler:15}
	\gll {ka\textbackslash{mar}anzé}! {bänema} {wri=f} {kwa} {n\textbackslash{zä}/nzr} {we} {bun} {we} {ane} {fäsi} {kwa} {\textbackslash{rä}/}\\
	[\textsc{2sg}:\textsc{imp}:\textsc{ipfv}/see]\textsubscript{\textsc{mid}} because drunkness=\textsc{erg}.\textsc{sg}{} \textsc{f}ut{} [\textsc{sg}>\textsc{2sg}:\textsc{nonpast}:\textsc{ipfv}/carry]\textsubscript{\textsc{trans}} also \textsc{2sg}.\textsc{dat}{} also \textsc{dem}{} shame \textsc{f}ut{} [\textsc{sg}.\textsc{f}:\textsc{nonpast}:\textsc{ipfv}/be]\textsubscript{\textsc{pref}}\\
	\glt `Look after yourself! Because when you get totally drunk, it will be embarassing for you.' [tci20121019-04 ABB 16-17] 
	%\Corpus{tci20121019-04}{ABB 16-17}
	
\z


The verb \emph{marasi} occurs 229 times in the corpus. 211 tokens are in the transitive template, 9 in the ditransitive template and 9 in the middle template. Of the 9 tokens in the middle template, only one example expresses a reflexive situation (\ref{ex:doehler:15}). 4 tokens express an antipassive situation, i.e. the patient argument is not indexed in the verb. I show an example of this in (\ref{ex:doehler:21}) below. The remaining 4 tokens express a reciprocal situation, as in (\ref{ex:doehler:16}).


\ea
	\gll {fi} {nm} {miyo-sé} {ŋa\textbackslash{mar}/nath}\\
	\textsc{3rd}.\textsc{abs}{} perhaps desire-\textsc{adjzr}{} [2|3du:\textsc{nonpast}:\textsc{ipfv}/see]\textsubscript{\textsc{mid}}\\
	\glt `Maybe they are in love?' (lit. `look at each other desiringly?') [tci20120925-01 MKA 39-40]
	%\Corpus{tci20120925-01}{MKA 39-40}
	\label{ex:doehler:16}
\z


Example (\ref{ex:doehler:17}) shows another example of a reciprocal situation. The verb \emph{zan} (\emph{fn}|\emph{kwr}) `hit, kill' occurs 172 times in the corpus: 165 in the transitive template versus 7 in the middle template, which are all reciprocal alternations. Note that the only constructional difference between reflexive and reciprocal situations is that the latter cannot be singular.

\ea
	\label{ex:doehler:17}%101
	\gll {zä} {zf} {ŋa\textbackslash{fn}/ath}.\\
	\textsc{prox}{} \textsc{imm}{} [2|3du:\textsc{pst}:\textsc{ipfv}/hit]\textsubscript{\textsc{mid}}\\
	\glt `They fought each other right here.' [tci20110802 ABB 23] %\Corpus{tci20110802}{ABB 23}
\z


The middle template in Komnzo is used in contexts that would employ reflexive constructions or reflexive pronouns in other languages, for example body-part or whole body actions. Example (\ref{ex:doehler:18}) shows an example with \emph{maiksi} (\emph{mayuk}|\emph{mayuf}) `wash'. This verb is basically transitive (`wash someone or something'), but it can appear in the middle template to express reflexive or reciprocal situations. Example (\ref{ex:doehler:18}) comes from a story about the brolga and the cassowary, who went washing together. As mentioned above, it is from context alone that we can infer that the two were washing themselves, rather than each other. Other lexemes in the same semantic domain are \emph{trisi} (\emph{tri}|\emph{trinz}) `scratch', \emph{royaksi} (\emph{royak}|\emph{royaf}) `dress, decorate' and \emph{rfrsi} (\emph{rfr}|\emph{rfrth}) `shave, trim'. 

Note that the first clause in (\ref{ex:doehler:18}) shows a raising construction. Therefore, the phasal verb \emph{thkäfaksi} (\emph{thkär}|\emph{thkäf}) `start' occurs in the middle template, while the lexical verb \emph{maikasi} has been nominalised. Only in the second clause, \emph{maikasi} is fully inflected.


\ea
	\label{ex:doehler:18}
	\gll {watik} {kra\textbackslash{thkäf}/th} {maik-si} {kwa\textbackslash{mayuk}/nmth} {kwras} {a} {yem}\\
	then [2|3du:\textsc{irr}:\textsc{pfv}/start]\textsubscript{\textsc{mid}} wash-\textsc{nmlz}{} [2|3du:\textsc{pst}:\textsc{dur}/wash]\textsubscript{\textsc{mid}} brolga(\textsc{abs}) and cassowary(\textsc{abs})\\
	\glt `Then they started to wash. The brolga and the cassowary were washing.' [tci20130923-01 ALB 9-12]
	%\Corpus{tci20130923-01}{ALB 9-12}
\z



Example (\ref{ex:doehler:19}) shows the middle template used for expressing an impersonal situation, i.e. the argument indexed in the verb is semantically empty. The closest translation of (\ref{ex:doehler:19}) is with an empty dummy-pronoun (`it'). Note that the verb in (\ref{ex:doehler:19}) is the light verb (\emph{ko}|\emph{kor}) `become', which lacks an infinitive. 


\ea
	\label{ex:doehler:19}
	\gll {aki} {zbo} {krä\textbackslash{kor}/}.\\
	moon(\textsc{abs}) \textsc{prox}:\textsc{all}{} [\textsc{sg}:\textsc{irr}:\textsc{pfv}/become]\textsubscript{\textsc{mid}}\\
	\glt `It became moon(light) here.' [tci20120904-02 MAB 47] %\Corpus{tci20120904-02}{MAB 47}
\z


Example (\ref{ex:doehler:20}) shows the use of the middle template to express a passive situation. In the example, the speaker explains the content and arrangement of his yam storage house. It is clear from the context that the argument indexed in the verb and expressed by the indefinite pronoun is the patient of the clause.


\ea
	\gll {fsan=ma} {nä} {kwa} {ŋan\textbackslash{zä}/nzrth} {zbo=wä} {zf}.\\
	Fsan=\textsc{char}{} \textsc{char}.\textsc{abs}{} \textsc{f}ut{} [2|3pl:\textsc{nonpast}:\textsc{ipfv}:\textsc{vent}/carry]\textsubscript{\textsc{mid}} \textsc{prox}:\textsc{all}=\textsc{emph}{} \textsc{imm}\\
	\glt `From Fsan, some more (yams) will be carried right here.' [tci20121001 ABB 45]
	%\Corpus{tci20121001}{ABB 45}
	\label{ex:doehler:20}
	\z


Example (\ref{ex:doehler:21}) and example (\ref{ex:doehler:22}) show an antipassive situation, in which the patient argument is not indexed in the verb. Instead, the verb occurs in the middle template. Here, we can speak of a dedicated antipassive construction because the case frame is different from all other situation types described above: the actor argument is flagged for ergative case. Note that the patient arguments are not indexed in the verb for semantic or pragmatic reasons, i.e. they often rank low in the animacy hierarchy or they are established in the preceding context. In (\ref{ex:doehler:21}) and (\ref{ex:doehler:22}), the respective patient arguments are given in brackets in the English translation. In both examples in the corpus, they are established in the preceding context.


\ea
	\gll {maureen=f} {zä} {zf} {ŋa\textbackslash{rg}/wrm} {efoth}.\\
	Maureen=\textsc{erg}.\textsc{sg}{} \textsc{prox}{} \textsc{imm}{} [\textsc{sg}:\textsc{rpst}:\textsc{dur}/wear]\textsubscript{\textsc{mid}} day\\
	\glt `Maureen was wearing them (the shoes) right here during the day.' 
	[tci20130901-04 MBK 15]
	%\Corpus{tci20130901-04}{MBK 15}
	\label{ex:doehler:21}
\z

\ea
	\gll {watik} {we} {masu} {kar=é} {kwe\textbackslash{karis}/th}\\
	then also Masu village=\textsc{erg}.\textsc{nsg}{} [2|3\textsc{pl}:\textsc{iter}/hear]\textsubscript{\textsc{mid}}\\
	\glt `Then, the villagers from Masu also heard it (the message).' [tci20131013-01 ABB 363]\\
	\label{ex:doehler:22}
\z


The set of examples in this section provides evidence that the middle prefix and the middle template are non-reflexive markers that happen to be coexpressive for reflexive, but also for a range of other situations. The only commonality between these situation lies in the fact that the event is about only one argument, which is one of the main criteria for ``middle situations'' according to Kemmer (\citeyear{Kemmer1993}). The role of the argument can be disambiguated only in the antipassive construction by the flagging of NPs with the ergative. For the other situation types, it is context alone that determines the correct state of affairs.
\subsection{Coreference involving other semantic roles}\label{reflxexp-other}
This section describes how coreference is expressed with other semantic roles, such as possessor, beneficiary, source, location and purpose. As will be become clearer, the markers and constructions that are employed are non-reflexives, i.e. they are coexpressive for other functions.

In Komnzo, possession is expressed by various contructions: (i) possessive pronouns and possessive case, (ii) possessive prefixes, (iii) the template of the verb. Example (\ref{ex:doehler:23}) shows the use of a possessive pronoun. Note that the empathic clitic \emph{=wä} attaches to the pronoun, which speakers often translate to English with `X's own'. Here, the speaker explains the different piles of yam tubers in his storage house and points out which yams are his. Note that in the last clause of (\ref{ex:doehler:20}) there is no emphatic clitic on the possessive pronoun. Thus, a more suitable translation is `my' or `mine' instead of `my own'.


\ea
	\gll {nzone=wä} {zane} {zf} {e\textbackslash{rä}/} ... {zane} {z=e\textbackslash{rä}/} ... {nzone} {zane} {zf} {e\textbackslash{rä}/}\\
	\textsc{f}sg.\textsc{poss}=\textsc{emph}{} \textsc{dem}.\textsc{prox}{} \textsc{imm}{} [2|3pl:\textsc{nonpast}:\textsc{ipfv}/be]\textsubscript{\textsc{pref}} (.) \textsc{dem}.\textsc{prox}{} \textsc{prox}=[2|3pl:\textsc{nonpast}:\textsc{ipfv}/be]\textsubscript{\textsc{pref}} (.) \textsc{f}\textsc{sg}.\textsc{poss}{} \textsc{dem}.\textsc{prox}{} \textsc{imm}{} [2|3pl:\textsc{nonpast}:\textsc{ipfv}/be]\textsubscript{\textsc{pref}}\\
	\glt `These (yams) here are my own. These ones are here ... these are mine here.' [tci20121001 ABB 129]
	%\Corpus{tci20121001}{ABB 129}
	\label{ex:doehler:23}
\z

Example (\ref{ex:doehler:24}) shows the use of possessive prefixes.\textsc{f}ootnote{The semantic difference between possessive case (pronouns, case enclitics) and the possessive prefixes is not based on alienability, but rather on a more general notion of closeness \citep[145]{Doehler2018}.} The example concludes an episode in a story with a quote by one of the protagonists. The possessive marker in this example is a prefix on the word \emph{zfth} `reason, cause' and not a possessive pronoun, as in (\ref{ex:doehler:23}). Note that the emphatic clitic \emph{=wä} attaches to \emph{zfth}. In this verbless clause we find coreference between `she' and `her' as the literal translation shows.


	\ea
	\gll {watik} ``{fi} {nafa-zfth=en=wä}''\\
	then \textsc{3rd}.\textsc{abs}{} \textsc{3rd}.\textsc{poss}-cause=\textsc{loc}=\textsc{emph}{}\\
	\glt `Well (he said) ``It was only her fault.''' (lit. `she in her own cause') [tci20120901-01 MAK 207]
	%\Corpus{tci20120901-01}{MAK 207}
	\label{ex:doehler:24}
\z

Another example of coreference is given in (\ref{ex:doehler:25}), which comes from the description of a picture card showing a man sitting in a prison cell.\textsc{f}ootnote{This is card \#16 of the Social Cognition Picture Task \citep{Carroll2009}.} The speaker uses direct speech to enact the character. The basic clause is expressed by the absolutive pronoun \emph{nzä} and the verb \emph{wothkgr} `I am inside'. The first singular argument is coreferential with \emph{nzonemäwä} `because of me', which is a possessive pronoun inflected with the characteristic case and the emphatic clitic.\textsc{f}ootnote{The characteristic case encodes the semantic roles of source `from', reason `because of' as well as purpose `for' \citep[157]{Doehler2018}. Note that the characteristic case always attaches to a nominal inflected for possessive case, if the referent is animate.}


\ea
	\gll ``{nzä} {nzone=ma=wä} {zfth=en} {zbo} {wo\textbackslash{thkgr}/}''\\
	\textsc{f}sg.\textsc{abs}{} \textsc{f}sg.\textsc{poss}=\textsc{char}=\textsc{emph}{} cause=\textsc{loc}{} \textsc{prox}.\textsc{all}{} [\textsc{f}sg.\textsc{nonpast}:2|3at/be.inside]\textsubscript{\textsc{pref}}\\
	\glt ``It's my own fault that I am in here.'' (lit. ``Because of me, (my) fault, I am in here.'') [tci20120925-01 KAB 23] %\Corpus{tci20120925-01}{KAB 23}
	\label{ex:doehler:25}
	\z

%sec 4.2
There are several layers of coreference in example (\ref{ex:doehler:26}) below. Recall that there are two verb templates in which the diathetic prefix increases valency (cf. \sectref{verbtemp}: (\ref{ex:doehler:3b}) and (\ref{ex:doehler:3d})). Both templates can be seen in (\ref{ex:doehler:21}). The example comes from the description of a picture card that shows a policeman, who hands over clothing to a man.\textsc{f}ootnote{This is card \#2 of the Social Cognition Picture Task \citep{CarrollEtAl2009}} In the first clause, we see that coreference is established between the beneficiary indexed by the verb prefix \emph{ya-} and the possessor expressed by the possessive pronoun \emph{nafane}. There is no free pronoun in the clause to express the beneficiary. Note that the demonstrative \emph{ane} refers to it, but \emph{ane} does not inflect for any of the core cases. The second clause contains the direct speech of the policeman. The coreferential elements are all in second singular: there is the topic expression (`As you are inside') followed by a speech formula that often accompanies transactions (`It is for you' or `It is yours'). In the last clause, the possessive pronoun (\textsc{2sg}) is indexed in the copula. However, as example (\ref{ex:doehler:27}) shows, the copula \emph{narä} can also index a beneficiary. This is caused by underspecification of the diathetic prefix, which results in an analytic problem, as we will see below.


	\ea
	\gll {frisman=f} {nafane} {slippers} {gwonyame} {ane} {bana} {fof} {ya\textbackslash{ri}/thr} ``{okay} {bä} {mane=me} {zä} {gu\textbackslash{thkgr}/} {bone} {b=na\textbackslash{rä}/}''\\
	policeman=\textsc{erg}.\textsc{sg}{} \textsc{sg}.\textsc{poss}{} slippers(\textsc{abs}) clothing(\textsc{abs}) \textsc{dem}{} pityful \textsc{emph}{} [\textsc{sg}>\textsc{sg}.\textsc{masc}:\textsc{ben}:\textsc{nonpast}:\textsc{ipfv}/give]\textsubscript{\textsc{ditrans}} okay \textsc{2sg}{} which=\textsc{ins}{} \textsc{prox}{} [\textsc{2sg}:\textsc{nonpast}:2|3at/be.inside]\textsubscript{\textsc{pref}} \textsc{2sg}.\textsc{poss}{} \textsc{med}=[\textsc{2sg}.\textsc{poss}:\textsc{nonpast}:\textsc{ipfv}/be]\textsubscript{\textsc{io.pref}}\\
	\glt `The policeman gives poor him his slippers and clothes (and says) ``Now that you are inside, those are your (things).''' [tci20111004 RMA 435-436]
	%\Corpus{tci20111004}{RMA 435-436}
	\label{ex:doehler:26}
	\z


	\ea
	\gll {wati} sa\textbackslash{kor}/a ``{bun} {bana} {ruga} {fof} {na\textbackslash{rä}}/''\\
	then [\textsc{sg}>\textsc{sg}.\textsc{masc}:\textsc{pst}:\textsc{pfv}/speak]\textsubscript{\textsc{trans}} \textsc{2sg}.\textsc{dat}{} pityful pig(\textsc{abs}) \textsc{emph}{} [\textsc{sg}.\textsc{masc}:\textsc{ben}:\textsc{nonpast}:\textsc{ipfv}/be]\textsubscript{\textsc{io.pref}}\\\
	\glt `Well, he said: ``The pig is for you poor guy.''' [tci20120805-01 ABB 814-815] %\Corpus{tci20120805-01}{ABB 814-815}
	\label{ex:doehler:27}
	\z


A similar strategy is used in (\ref{ex:doehler:28}), which comes from a conversational text. The speaker literally says `you will finish my wish'. Another possible translation of this clause is `you will finish my wish for me', if we assume that there is an additional argument, a beneficiary, which is not expressed in a separate noun phrase. In other words, there is an analytic problem with the diathetic prefix. When used to increase valency, as in (\ref{ex:doehler:26}) and (\ref{ex:doehler:28}), it is unclear whether the introduced argument is a beneficiary or a possessor. We can only tell from the flagging of the noun phrase, as in (\ref{ex:doehler:26}) and (\ref{ex:doehler:28}). Note that this overlap in the encoding of beneficiary and possessor roles is not uncommon in the Southern New Guinea region. In Ngkolmpu \citep{Carroll2016} and Bine (own fieldwork), both functions are expressed by the same case marker.

In (\ref{ex:doehler:26}), one can make an argument from frequency and say that the verb \emph{yarisi} (\emph{ri}|\emph{r}) `give' always has a beneficiary encoded in the prefix, and the corresponding noun phrase is flagged for dative. But examples like (\ref{ex:doehler:28}) are not as clear. The prefix could be indexing a possessor (as shown in the gloss), but also a beneficiary. Only in the latter case two arguments are coreferential and the translation would have to be `you will fulfil my wish for me'.


	\ea
	\gll {nzone} {miyo} {kwa} {wa\textbackslash{bthak}/wr}\\
	\textsc{f}sg.\textsc{poss}{} wish(\textsc{abs}) \textsc{f}ut{} [\textsc{sg}>\textsc{f}sg:\textsc{poss}:\textsc{nonpast}:\textsc{ipfv}/finish]\textsubscript{\textsc{ditrans}}\\
	\glt `You fulfil my wish.' (lit. `you will finish my wish') [tci20130823-06 CAM 23]
	%\Corpus{tci20130823-06}{CAM 23}
	\label{ex:doehler:28}
	\z


As I show in (\ref{ex:doehler:29}), autobenefactives cannot be expressed in this way. Coreference between the two arguments indexed in the verb renders the inflected form ungrammatical, as in (\ref{ex:doehler:29a}). Instead, the middle template has to be used, as in (\ref{ex:doehler:29b}).


	\ea
	\label{ex:doehler:29}
			\ea[*]{
			\gll {nzone} {miyo} {kwa} {wa\textbackslash{bthak}/é}\\
			\textsc{f}sg.\textsc{poss}{} wish(\textsc{abs}) \textsc{f}ut{} [\textsc{f}sg>\textsc{f}sg:\textsc{poss}:\textsc{nonpast}:\textsc{ipfv}/finish]\textsubscript{\textsc{ditrans}}\\
			}
			\label{ex:doehler:29a}
			
			\ex[]{
			\gll {nzone} {miyo} {kwa} {ŋa\textbackslash{bthak}/é}\\
			\textsc{f}sg.\textsc{poss}{} wish(\textsc{abs}) \textsc{f}ut{} [\textsc{f}sg:\textsc{nonpast}:\textsc{ipfv}/finish]\textsubscript{\textsc{mid}}\\
			\glt `I will fulfil my wish.'
			\label{ex:doehler:29b}
			}
		\z
		\z
	


Example (\ref{ex:doehler:30}) shows an autobenefactive expressed as an apposition. The speaker explains how they shared the meat after a pig hunt. The verb indexes a first plural actor (`we') and a second/third plural beneficiary (`for them'), which is also expressed by the dative pronoun. The first singular dative pronoun in the apposition is coreferential with the actor in the verb (`we cut for them (and) for us').


\ea
	\gll {sitau=aneme} {afa} {kwark} {b=ya\textbackslash{r}a} {nafanm} {ä\textbackslash{kwa}/ne} ... {nzenm=wä}\\
	Sitau=\textsc{poss}.\textsc{nsg}{} father deceased \textsc{med}=[\textsc{sg}:\textsc{pst}:\textsc{ipfv}/be]\textsubscript{\textsc{pref}} \textsc{3nsg}.\textsc{dat}{} [\textsc{f}pl>2|3pl:\textsc{ben}:\textsc{nonpast}:\textsc{ipfv}/cut.meat]\textsubscript{\textsc{distrans}} (.) \textsc{f}pl.\textsc{dat}=\textsc{emph}{}\\
	\glt `Sitau's father (and them) were there. We cut (meat) for them ... and for ourselves.' [tci20120821-02 LNA 96-97] %\Corpus{tci20120821-02}{LNA 96-97}
	\label{ex:doehler:30}
	\z


Source roles are expressed by the characteristic case (\textsc{char}), which - for animates only - attaches to a possessive inflection. In example (\ref{ex:doehler:31}), the agent ({Yasi}) is coreferential with the source (\emph{nafanemawä}) in the apposition. Note that the latter is marked with the emphatic clitic (\emph{=wä}).


\ea
	\gll  {yasi=f} {ane} {fof} {fam} {thn\textbackslash{r}/a} ... {nafane=ma=wä} {mrn} {fof}\\
	Yasi=\textsc{erg}.\textsc{sg}{} \textsc{dem}{} \textsc{emph}{} thought(\textsc{abs}) [\textsc{sg}>2|3\textsc{pl}:\textsc{pst}:\textsc{ipfv}:\textsc{vent}/do]\textsubscript{\textsc{trans}} (.) \textsc{sg}.\textsc{poss}=\textsc{char}=\textsc{emph}{} family \textsc{emph}{}\\
	\glt `Yasi thought of them, of his own family.'
	[tci20111107-01 MAK 176-177]
	%\Corpus{tci20111107-01}{MAK 176-177}
	\label{ex:doehler:31}
\z


In example (\ref{ex:doehler:32}), the speaker is giving advise to his interlocuter as to the right way of sharing ones' harvest. The agent indexed in the verb (\textsc{2sg}) is coreferential with the source (\emph{bonemawä}). Again the source is inflected with the emphatic enclitic (\emph{=wä}).


\ea
	\gll {keke} {kwa} {bone=ma=wä} {za\textbackslash{na}/thé} {we} {näbun=ane=ma} {be} {za\textbackslash{na}/thé}\\
	\textsc{neg}{} \textsc{f}ut{} \textsc{2sg}.\textsc{poss}=\textsc{char}=\textsc{emph}{} [\textsc{2sg}>\textsc{sg}.\textsc{f}:\textsc{imp}:\textsc{ipfv}/eat]\textsubscript{\textsc{trans}} also \textsc{indf}=\textsc{poss}.\textsc{sg}=\textsc{char}{} \textsc{2sg}.\textsc{erg}{} [\textsc{2sg}>\textsc{sg}.\textsc{f}:\textsc{imp}:\textsc{ipfv}/eat]\textsubscript{\textsc{trans}}\\
	\glt `Don't eat (the yam) from your own (harvest)! Eat (the yam) from another one's (harvest)!' %\Corpus{tci20120805-01}{ABB 760-761}
	[tci20120805-01 ABB 760-761]
	\label{ex:doehler:32}
\z


The role of location is expressed by one of the local cases: locative, allative and ablative. Coreference is achieved by a possessive construction. In example (\ref{ex:doehler:33}), a possessive prefix attaches to a place noun inflected for the locative case. The actor indexed in the verb is coreferential with the possessor of the locative marked role (`at your place'). In example (\ref{ex:doehler:34}), the agent of the verb which is also expressed in the noun phrase (`Babua's wife') is coreferential with the possessor of the allative marked role (`to her own village').\textsc{f}ootnote{Note that in (\ref{ex:doehler:33}) and (\ref{ex:doehler:34}), the gloss shows no person value, but only number (\textsc{sg}). This neutralization of the person value occurs in certain TAM inflections \citep[207]{Doehler2018}.}


\ea
	\gll ``{bu-kar=en} {ane} {fof} {bä} {safak} {e\textbackslash{mgthk}/wa}"\\
	\textsc{2sg}.\textsc{poss}-place=\textsc{loc}{} \textsc{dem}{} \textsc{emph}{} \textsc{med}{} saratoga(\textsc{abs}) [\textsc{sg}>2|3pl:\textsc{pst}:\textsc{ipfv}/feed]\textsubscript{\textsc{trans}}\\
	\glt ``You fed these saratoga fish there at your place.'' [tci20110802 ABB 121-122]
	%\Corpus{tci20110802}{ABB 121-122}
	\label{ex:doehler:33}
\z


\ea
	\gll {babu=ane} {ŋare} {zan\textbackslash{math}/a} {nima} {nafa-kar=fo=wä} {safs=fo}\\
	Babua=\textsc{poss}.\textsc{sg}{} wife(\textsc{abs}) [\textsc{sg}:\textsc{pst}:\textsc{pfv}:\textsc{vent}/run]\textsubscript{\textsc{mid}} like.this \textsc{3rd}.\textsc{poss}-village=\textsc{all}=\textsc{emph}{} Safs=\textsc{abl}\\
	\glt `Babua's wife ran to her own village, to Safs.' [tci20120814 ABB 211-213] %\Corpus{tci20120814}{ABB 211-213}
	\label{ex:doehler:34}
	\z

\subsection{Coreference across clauses}\label{longstance}
Coreference across clauses or long-stance coreference in Komnzo is always ambiguous and only the context resolves whether there is coreference or not. Hence, the elicited example in (\ref{ex:doehler:35}) can have two interpretations if it occurs out of context.


	\ea
	\gll {fi} {ŋa\textbackslash{ko}/nzrth} {nima} {fi} {kmam} {thrayak}\\
	\textsc{3rd}.\textsc{abs}{} [2|3pl:\textsc{nonpast}:\textsc{ipfv}/speak]\textsubscript{\textsc{mid}} like.this \textsc{3rd}.\textsc{abs}{} \textsc{appr}{} [2|3pl:\textsc{irr}:\textsc{ipfv}/walk]\textsubscript{\textsc{pref}}\\
	\glt `They\textsubscript{1} say that they\textsubscript{1} should not go.'
	\glt `They\textsubscript{1} say that they\textsubscript{2} should not go.'
	\label{ex:doehler:35}
\z

Example (\ref{ex:doehler:36}) is a corpus example from a conversational taks. It shows coreference between the oblique argument in the matrix clause (`located with you') and the actor argument indexed in the verb in the relative clause (\textsc{2sg}).


	\ea
	\gll {bun=dbo=nzo} {\textbackslash{rä}} {mane} {za\textbackslash{wok}/th}\\
	2\textsc{sg}.\textsc{dat}=\textsc{all}=\textsc{only}{} [\textsc{sg}.\textsc{f}:\textsc{nonpast}:\textsc{ipfv}/be]\textsubscript{\textsc{pref}} which(\textsc{abs}) [2\textsc{sg}>3\textsc{sg}.\textsc{f}:\textsc{imp}:\textsc{pfv}/choose]\textsubscript{\textsc{trans}}\\
	\glt `It is up to you, which one you choose!' [tci20111004 RMA 524] %\Corpus{tci20111004}{RMA 524}
	\label{ex:doehler:36}
\z

In (\ref{ex:doehler:37}), the agent of the first clause (\textsc{f}sg) is coreferential with the possessor in the second clause (`my eyes'). The possessive construction makes up a verbless clause (`but I saw it'), but it can also be translated as an apposition (`but in my eyes').


	\ea
	\gll {nzä} {keke} {skoro} {fthé} {kwof\textbackslash{rä}/rm} {fi} {nzu-sin=en=wä} {fof}\\
	\textsc{f}sg.\textsc{abs}{} \textsc{neg}{} school(\textsc{abs}) when [\textsc{f}sg:\textsc{pst}:\textsc{dur}/be]\textsubscript{\textsc{pref}} but \textsc{f}sg-eye=\textsc{loc}=\textsc{emph}{} \textsc{emph}{}\\
	\glt `I was not a school (child) at that time, but I witnessed this.' (lit. `but really in my eyes') [tci20150906-10 ABB 373-374] %\Corpus{tci20150906-10}{ABB 373-374}
	\label{ex:doehler:37}
	\z

\section{Conclusion}\label{conclusion}
As the preceding sections have shown, the grammatical markers and constructions that are used for the expression of reflexive situations are all coexpressive for a range of other functions. The middle template covers situation types that fall under label of ``middle'' as defined by \citep{Kemmer1993}. The exclusive clitic (\emph{=nzo}) and the emphatic clitic (\emph{=wä}) are used for creating contrastive focus and emphasis, respectively. The overlap of intensifiers and reflexives is known from the cross-linguistic literature \citep{KoenigSiemund2000}. Thus, it would be misleading to speak of a reflexive construction, reflexive pronouns or reflexive markers in Komnzo. Instead, express reflexive situations are inferred from constructions like the middle template, emphatic markers and contrastive focus markers that are much broader in their function.
\section*{Abbreviations}

\begin{tabularx}{.45\textwidth}{lQ}
        \textsc{1} & {first person}\\
		\textsc{2} & {second person}\\
		\textsc{3} & {third person}\\
		\textsc{abs} & {absolutive case}\\
		\textsc{abl} & {ablative case}\\
		\textsc{all} & {allative case}\\
		\textsc{alr} & {iamitive (`already')}\\
		\textsc{and} & {andative}\\
		\textsc{appr} & {apprehensive}\\
		\textsc{assoc} & {associative case}\\
		\textsc{ben} & {benefactive}\\
		\textsc{char} & {characteristic case}\\
		\textsc{dem} & {demonstrative}\\
		\textsc{di} & {diathetic prefix}\\
		\textsc{dist} & {distal (deictic)}\\
		\textsc{ditrans} & {ditransitive template)}\\
		\textsc{du} & {dual}\\
		\textsc{dur} & {durative}\\
		\textsc{emph} & {emphatic}\\
		\textsc{erg} & {ergative case}\\
		\textsc{etc} & {et cetera}\\
		\textsc{fem} & {feminine}\\
		\textsc{fut} & {future}\\
		\textsc{imm} & {immediate (`right here')}\\
		\textsc{imp} & {imperative}\\
		\textsc{indf} & {indefinite}\\
		\textsc{io.pref} & {indirect object prefixing template}\\
		\textsc{ins} & {instrumental case}\\
		\end{tabularx}
		\begin{tabularx}{.45\textwidth}{lQ}
		\textsc{ipfv} & {imperfective}\\
		\textsc{irr} & {irrealis}\\
		\textsc{loc} & {locative case}\\
		\textsc{m} & {middle}\\
		\textsc{masc} & {masculine}\\
		\textsc{med} & {medial (deictic)}\\
		\textsc{neg} & {negator}\\
		\textsc{npst} & {non-past}\\
		\textsc{nsg} & {non-singular}\\
		\textsc{obj} & {object}\\
		\textsc{only} & {exclusive marker (`only', `just')}\\
		\textsc{pfv} & {perfective}\\
		\textsc{pl} & {plural}\\
		\textsc{poss} & {possessive}\\
		\textsc{prox} & {proximal (deictic)}\\
		\textsc{pref} & {prefixing template}\\
		\textsc{pst} & {past}\\
		\textsc{purp} & {purposive case}\\
		\textsc{recog} & {recognitional}\\
		\textsc{redup} & {reduplication}\\
		\textsc{rpst} & {recent past}\\
		\textsc{sbj} & {subject}\\
		\textsc{sg} & {singular}\\
		\textsc{stat} & {stative}\\
		\textsc{trans} & {transitive template}\\
		\textsc{vent} & {venitive}\\
\end{tabularx}
%{\textsc{f}ootnotesize%
%\begin{multicols}{2}
%	\begin{tabbing}
%		{\textbackslash.../} = {verb stem (e.g. y\textbackslash fath/wr)}\\
%		{(.)} = {speech pause}\\\textsc{
%		{.} = {multi-item gloss (e.g. old.man)}\\
%		{\textbar} = {used in cases of syncretism (e.g. %2\textbar3)}\\
%		\textsc{1} = {first person}\\
%		\textsc{2} = {second person}\\
%		\textsc{3} = {third person}\\
%		\textsc{abs} = {absolutive case}\\
%		\textsc{abl} = {ablative case}\\
%		\textsc{all} = {allative case}\\
%		\textsc{alr} = {iamitive (`already')}\\
%		\textsc{and} = {andative}\\
%		\textsc{appr} = {apprehensive}\\
%		\textsc{assoc} = {associative case}\\
%		\textsc{ben} = {benefactive}\\
%		\textsc{char} = {characteristic case}\\
%		\textsc{dem} = {demonstrative}\\
%		\textsc{di} = {diathetic prefix}\\
%		\textsc{dist} = {distal (deictic)}\\
%		\textsc{ditrans} = {ditransitive template)}\\
%		\textsc{du} = {dual}\\
%		\textsc{dur} = {durative}\\
%		\textsc{emph} = {emphatic}\\
%		\textsc{erg} = {ergative case}\\
%		\textsc{etc} = {et cetera}\\
%		\textsc{fem} = {feminine}\\
%		\textsc{fut} = {future}\\
%		\textsc{imm} = {immediate (`right here')}\\
%		\textsc{imp} = {imperative}\\
%		\textsc{indf} = {indefinite}\\
%		\textsc{io.pref} = {indirect object prefixing template}\\
%		\textsc{ins} = {instrumental case}\\
%		\textsc{ipfv} = {imperfective}\\
%		\textsc{irr} = {irrealis}\\
%		\textsc{loc} = {locative case}\\
%		\textsc{m} = {middle}\\
%		\textsc{masc} = {masculine}\\
%		\textsc{med} = {medial (deictic)}\\
%		\textsc{neg} = {negator}\\
%		\textsc{npst} = {non-past}\\
%		\textsc{nsg} = {non-singular}\\
%%		\textsc{obj} = {object}\\
%		\textsc{only} = {ex:doehler:clusive marker %(`only', `just')}\\
%		\textsc{pfv} = {perfective}\\
%		\textsc{pl} = {plural}\\
%		\textsc{poss} = {possessive}\\
%		\textsc{prox} = {proximal (deictic)}\\
%		\textsc{pref} = {prefixing template}\\
%		\textsc{pst} = {past}\\
%		\textsc{purp} = {purposive case}\\
%		\textsc{recog} = {recognitional}\\
%		\textsc{redup} = {reduplication}\\
%		\textsc{rpst} = {recent past}\\
%		\textsc{sbj} = {subject}\\
%		\textsc{sg} = {singular}\\
		%		\textsc{stat} = {stative}\\
		\textsc{trans} = {transitive template}\\
%		\textsc{vent} = {venitive}\\
%	\end{tabbing}
%\end{multicols}}


\section*{Acknowledgements}
I express my deepest gratitude to Nakre Abia, Abia Bai, Daure Kaumb, and the people of Rouku village for welcoming me and their continued support of my work. I thank Martin Haspelmath and Katarzyna Janic and two anonymous reviewers for their helpful comments on earlier versions of this chapter. The fieldwork on which this chapter is based was funded by the DOBES program of the Volkswagen Foundation and by the Australian National University. I thank both of these institutions for their support of language documentation and description.

{\sloppy\printbibliography[heading=subbibliography,notkeyword=this]}
\end{document}
