\documentclass[output=paper]{langscibook}
\author{Françoise Rose\affiliation{Dynamique Du Langage, CNRS/Université Lyon2}}
\title{{Reflexive} {constructions} {and} {middle} {marking} {in} {Mojeño} {Trinitario}}
\abstract{Mojeño Trinitario (Arawak, Bolivia) shows a middle marker -\textit{wo} that encodes, among other functions, the coreference of subject and object within the same clause, within reflexive constructions. The middle marker -\textit{wo} is not only used for prototypical reflexive situations (the central interest of this volume), but also for situations types that are best considered middle (in line with \citealt{Kemmer1993}), including grooming, non-translational motion, other body actions, translational motion and positionals, reciprocals, mental events (cognition and emotion), and spontaneous events. The middle marker -\textit{wo} can also be used in situation types where it just adds various types of emphasis on the subject. Interestingly, the marker -\textit{wo} is only one of several middle-marking strategies in the language. Coreference other than between the subject and the object, within a clause or beyond the clause, are left unmarked, as the language has neither a set of reflexive pronouns nor of reflexive possessor indexes. Coreference beyond the reflexive construction is therefore left as a possible interpretation, depending on the semantico-syntactic and discourse context.
% \textbf{Keywords} reflexive, middle, Arawak, valency changes, S affectedness, activity
}
\IfFileExists{../localcommands.tex}{
 \addbibresource{localbibliography.bib}
 \usepackage{langsci-optional}
\usepackage{langsci-gb4e}
\usepackage{langsci-lgr}

\usepackage{listings}
\lstset{basicstyle=\ttfamily,tabsize=2,breaklines=true}

%added by author
% \usepackage{tipa}
\usepackage{multirow}
\graphicspath{{figures/}}
\usepackage{langsci-branding}

 
\newcommand{\sent}{\enumsentence}
\newcommand{\sents}{\eenumsentence}
\let\citeasnoun\citet

\renewcommand{\lsCoverTitleFont}[1]{\sffamily\addfontfeatures{Scale=MatchUppercase}\fontsize{44pt}{16mm}\selectfont #1}
   
 %% hyphenation points for line breaks
%% Normally, automatic hyphenation in LaTeX is very good
%% If a word is mis-hyphenated, add it to this file
%%
%% add information to TeX file before \begin{document} with:
%% %% hyphenation points for line breaks
%% Normally, automatic hyphenation in LaTeX is very good
%% If a word is mis-hyphenated, add it to this file
%%
%% add information to TeX file before \begin{document} with:
%% %% hyphenation points for line breaks
%% Normally, automatic hyphenation in LaTeX is very good
%% If a word is mis-hyphenated, add it to this file
%%
%% add information to TeX file before \begin{document} with:
%% \include{localhyphenation}
\hyphenation{
affri-ca-te
affri-ca-tes
an-no-tated
com-ple-ments
com-po-si-tio-na-li-ty
non-com-po-si-tio-na-li-ty
Gon-zá-lez
out-side
Ri-chárd
se-man-tics
STREU-SLE
Tie-de-mann
}
\hyphenation{
affri-ca-te
affri-ca-tes
an-no-tated
com-ple-ments
com-po-si-tio-na-li-ty
non-com-po-si-tio-na-li-ty
Gon-zá-lez
out-side
Ri-chárd
se-man-tics
STREU-SLE
Tie-de-mann
}
\hyphenation{
affri-ca-te
affri-ca-tes
an-no-tated
com-ple-ments
com-po-si-tio-na-li-ty
non-com-po-si-tio-na-li-ty
Gon-zá-lez
out-side
Ri-chárd
se-man-tics
STREU-SLE
Tie-de-mann
} 
 \togglepaper[1]%%chapternumber
}{}

\begin{document}
\maketitle 
%\shorttitlerunninghead{}%%use this for an abridged title in the page headers




\section{Introduction}
\label{sec:Rose:1}

Mojeño Trinitario is a language of the Arawak family spoken in Bolivia (\sectref{sec:Rose:2}). Reflexive constructions in Mojeño Trinitario make use of a middle marker -\textit{wo} as in \REF{ex:Rose:1}--\REF{ex:Rose:2} (\sectref{sec:Rose:3}). This encodes, among other functions, the coreference of what are subjects and objects in a corresponding non-reflexive clause \REF{ex:Rose:2}.


\ea
\label{ex:Rose:1}
\gll ñ-omuire=po t-etpiri-k\textbf{-wo}=po. \\
3\textsc{m}-also=\textsc{pfv} 3-arrange-\textsc{act\textbf{-mid}=pfv}\\
\glt ‘He got ready too (lit. he arranged himself too).’ [T38.186]
\z

\ea
\label{ex:Rose:2}
\gll p-etpiri-gi-a j-ma-ro-no. \\
2\textsc{sg}-arrange-\textsc{act-irr} \textsc{dem-nh.pl-med-pl}\\
\glt ‘Arrange these!’ [about pictures in the Family problem Solving Task]\footnote{This task is described in \citet{SanRoque2012}.} [T45.002]
\z


There are other types of coreference, within a clause or beyond the clause, that are left unmarked (\sectref{sec:Rose:4}), as the language does have neither a set of reflexive pronouns nor of reflexive possessor indexes. Coreference beyond the reflexive construction is therefore left as a possible interpretation, depending on the semantico-syntactic and discourse context. The middle marker -\textit{wo} is not only used for prototypical reflexive situations, the central interest of this volume (\sectref{sec:Rose:5}), but also for situation types that are best considered middle (in line with \citealt{Kemmer1993}), including grooming, non-translational motion, other body actions, translational motion and positionals, reciprocals, mental events (cognition and emotion), and spontaneous events. The middle marker -\textit{wo} can also be used in situations types where it does not show a middle function, but puts various types of emphasis on the subject. Interestingly, the marker -\textit{wo} is only one of several middle-marking strategies in the language, and it is the most agent-oriented one (\sectref{sec:Rose:6}).



The data on which this paper is based have been collected in the field by the author since 2005. It constitutes a database of 8 hours of (semi)-spontaneous texts, 2 hours of isolated sentences elicited with stimuli, and additionally 4900 elicited sentences \citep{Rose2018}.


\section{Introduction to Mojeño Trinitario}
\label{sec:Rose:2}
\subsection{The language}
\label{sec:Rose:2.1}

Mojeño (trin1274) is an endangered Arawak language (\citealt{Gill1957, Rose2015Mojeno}) spoken in Lowland Bolivia (Map 1).\footnote{This map is the English version of a map originally published in French in \citet{Rose2010}.} The Trinitario dialect is spoken by a few thousand speakers \citep{CrevelsMuysken2009}, most of which are bilingual in Spanish.


\begin{figure}
\caption{\label{fig:Rose:1}Geographical distribution of the Mojeño speakers}
\includegraphics[width=\textwidth]{figures/Rose-img001.jpg}
\end{figure}
 



Mojeño Trinitario is a highly agglutinating language, with a large number of suffix/enclitic slots and a few prefix slots. Lexical and grammatical morphemes display several surface forms, due to a rich system of morphophonemic rules and a pervasive process of vowel deletion \citep{Rose2019}. The next sections will present some aspects of the grammar of Mojeño Trinitario that are important for the issue of reflexivization: pronominal markers (\sectref{sec:Rose:2.2}), argument encoding (\sectref{sec:Rose:2.3}) and the active suffix (\sectref{sec:Rose:2.4}).


\subsection{Sets of pronominal markers}
\label{sec:Rose:2.2}

Mojeño Trinitario shows four sets of pronominal markers: free pronouns, demonstrative formatives,\footnote{These take a demonstrative prefix \textit{p}- and one of a set of distance/epistemic suffixes to form a demonstrative \citep{Rose2017}} person prefixes and person suffixes. \tabref{tab:Rose:1} shows that these sets share the same semantic categorization and cognate forms (demonstratives, of little relevance here, are left out).\footnote{For a full presentation and discussion of the pronominal paradigm, see \citet{Rose2015Mojeno}} For third person, note that number is neutralized for non-human, gender is neutralized for human plural, and there is a genderlect distinction for the third person human singular masculine depending on the gender of the speaker \citep{Rose2013, Rose2015Innovative}. Importantly, there is no set of reflexive pronominals. The same affix sets are used on both verbs and nouns: prefixes for subject on verbs and possessor on nouns, suffixes for object on verbs and subjects on non-verbal predicates. The only difference is that the semantically non-specific third person marker \textit{t}- is found on verbs only. The number of a third person subject marked with \textit{t}- can be specified with the plural suffix -\textit{ono}, also used to mark plurality on nouns. The use of the pronominal markers is discussed in the next section.


\begin{table}
\caption{\label{tab:Rose:1}Mojeño Trinitario pronominal markers}
\begin{tabularx}{\textwidth}{p{2.3cm}p{2.5cm}p{3cm}p{3.5cm}}
\lsptoprule
& \textsc{pronouns} & \textsc{prefixes}

(A, Sa, Sp, Poss) & \textsc{suffixes}

\textsc{(P,} \small argument of non-verbal predicate)\\
\hline
1\textsc{sg} & \textit{nuti} & \textit{n-} & \textit{-nu}\\
2\textsc{sg} & \textit{piti} & \textit{py-} & \textit{-vi}\\
1\textsc{pl} & \textit{viti} & \textit{vy-} & \textit{-(wok)ovi}\\
2\textsc{pl} & \textit{eti} & \textit{a-} & \textit{-'e}\\
3\textsc{m(sg.h)} speaker♂ & {\bfseries \textmd{\textit{ema}}} & {\bfseries \textmd{\textit{ma-}}\textmd{}\textmd{(}\textmd{\textit{{\textasciitilde} mu-, m-}}\textmd{)}} & \textit{-}\\
3\textsc{m(sg.h)} speaker♀ & {\bfseries \textmd{\textit{eñi}}} & {\bfseries \textmd{\textit{ñi-}}\textmd{}\textmd{(}\textmd{\textit{{\textasciitilde} ñ-}}\textmd{)}} & \textit{-}\\
3\textsc{f(sg.h)} & {\bfseries \textmd{\textit{esu}}} & {\bfseries \textmd{\textit{s-}}} & \textit{-}\\
3\textsc{nh(sg/pl)} & {\bfseries \textmd{\textit{eto}}} & {\bfseries \textmd{\textit{ta-}}\textmd{}\textmd{(}\textmd{\textit{{\textasciitilde} t-}}\textmd{)}} & \textit{-}\\
3\textsc{pl(h)} & {\bfseries \textmd{\textit{eno}}} & {\bfseries \textmd{\textit{na-}}\textmd{}\textmd{(}\textmd{\textit{{\textasciitilde} n-}}\textmd{)}} & \textit{-woko} (3\textsc{pl})\\
3 & & \textit{t-} (\textit{{\textasciitilde} ty-}) verbs only & \\
\lspbottomrule
\end{tabularx}
\end{table}

\subsection{Argument encoding}
\label{sec:Rose:2.3}

Argument encoding is essentially marked by the obligatory person indexation (the last two columns of \tabref{tab:Rose:1}). Noun phrases are indeed optional, and unflagged. Free pronouns are also used optionally, usually when the referent has been identified previously.\footnote{Free pronouns also show a use within noun phrases headed by a noun, where they either precede or replace the determiner (free pronoun + noun, or free pronoun + determiner + noun).} When noun phrases are overt, the basic order is SVO for transitive clauses and VS for intransitive clauses.



The obligatory person indexation works as follows. Subjects are indexed on verbs with prefixes \REF{ex:Rose:3}. First and second person objects are indexed on verbs with suffixes \REF{ex:Rose:4}, while third person objects are not overtly marked in the verb. Subject and object affixes on the same verb cannot be coreferential. A typological particularity of Mojeño Trinitario is that the presence and the person of an object triggers a differential indexation of third person subjects \citep{Rose2011}. On the one hand, the non-specific third person subject prefix \textit{t}- is found both on intransitive verbs as in \REF{ex:Rose:5} and on transitive verbs with a first or second person object as in \REF{ex:Rose:4}. On the other hand, a semantically specific third person subject prefix (\textit{ma}-, \textit{ñi}-, \textit{s}-, \textit{na}- and \textit{ta}-) is found when the object is a third person as in \REF{ex:Rose:3}. The selection of a third person subject prefix depends on transitivity, which does not solely depend on the number of the arguments and the person of the co-argument, but is also sensitive to various transitivity criteria, like aspect, mood, information structure, etc. \citep{Rose2011}.


\ea
\label{ex:Rose:3}
\gll ma `moperu-gra \textbf{mu-}em-'o=po to jani-ono\\
\textsc{art.m} child-\textsc{dim} \textsc{\textbf{3m}}\textbf{-}see-\textsc{act=pfv} \textsc{art.nh} bee-\textsc{pl}\\
\glt ‘The little boy saw the bees.’ [T11.019]
\z

\ea
\label{ex:Rose:4}
\gll \textbf{t}-im-it-ko\textbf{-wokovi}\\
\textbf{3-}\textsc{caus}-know-\textsc{act}\textbf{-1}\textbf{\textsc{pl}}\\
\glt ‘He teaches us.’ [T28.099]
\z

\ea
\label{ex:Rose:5}
\gll \textbf{t-}junopo=po \textbf{te} to smeno.\\
\textbf{3-}run=\textsc{pfv} \textbf{\textsc{prep.nh}} \textsc{art.nh} woods\\
\glt ‘It ran through the woods.’ [T11.018)
\z


Mojeño Trinitario shows A-preserving lability, also called agentive ambitransitivity: the same root can be used without any formal change either transitively with both A and P, or intransitively with a unique S participant (with S being semantically equivalent to A). Ambitransitivity is observable in the example \REF{ex:Rose:6}, where the root \textit{ew} ‘sow’ is used intransitively in the first clause and transitively in the second clause (observe the change in third person subject indexes).


\ea
\label{ex:Rose:6}
\gll ene \textbf{t-}ew-ko-m=po, \textbf{na-}ew-ko=po to arusu\\
and \textbf{3-}sow-\textsc{act-pl=pfv} \textbf{\textsc{3pl-}}sow-\textsc{act=pfv} \textsc{art.nh} rice\\
\glt ‘And they start to sow, they sow rice.’ [T21.038)
\z


Obliques (adjuncts or peripheral arguments) always occur with a preposition, and are also distinguished from objects by not being indexed on the verb. There is a single simple preposition \textit{te}, illustrated in \REF{ex:Rose:5},\footnote{The form \textit{te} is actually a reduced form of a prepositional root \textit{ye’e} with a third person non-human prefix \textit{ta}-. If the preposition introduces a first or second person, or a human third person, this is indexed as a prefix on \textit{ye’e}, as in \textit{p-ye’e} ‘with you, for you, etc.’} that shows multiple meanings such as ‘with’, ‘in’, ‘on’, ‘for’, ‘from’, etc.


\subsection{The active suffix}
\label{sec:Rose:2.4}

Mojeño Trinitario roots are either active (i.e. dynamic) or stative, and activity is overtly marked at the stem level with the active suffix (-\textit{ko} {\textasciitilde} -\textit{cho} {\textasciitilde} \nobreakdash-\textit{`o})\footnote{The allomorphs are selected depending on the preceding vowel (often not visible due to the rhythmic syncope process).}. This suffix comes almost at the end of the verb stem, made out of the root and its derivational morphology, and sketched in \figref{fig:Rose:2}.\footnote{\textrm{The interaction of the active suffix with the reciprocal will be discussed in \sectref{sec:Rose:6}. Also note that the middle marker present in the stem template is a prefix, distinct from the middle suffix -}\textrm{\textit{wo}}.}


\begin{figure}
\caption{\label{fig:Rose:2}Verbal stem template}

\textsc{caus/mid}-\textbf{root}-\textsc{red-clf/appl3-}N\textsc{-pluract-}\textbf{\textsc{act}}\textsc{/recp-appl1/2/pass}
\end{figure}

The active suffix is normally used with active roots (be they intransitive or transitive). The active suffix can be seen in examples \REF{ex:Rose:3}, \REF{ex:Rose:4} and \REF{ex:Rose:6} in the verb stems \textit{im-'o} ‘see/watch’, \textit{it-ko} ‘know’ and \textit{ew-ko} ‘sow’. However, it does not show in some active verb stems, as on \textit{junopo} ‘run’ in \REF{ex:Rose:5} and \textit{samo} ‘feel’ in \REF{ex:Rose:7} (see more below on this distribution). Stative roots such as \textit{itve} ‘be sweet’ do not normally take the active suffix, but when they do, an active (transitive) verb stem is derived, such as \textit{itve-cho} ‘sweeten’. Conversely, the active suffix is left out of constructions that are stativizing active roots, like the patient nominalization in \REF{ex:Rose:8} where the nominalizer replaces the active suffix.


\ea
\label{ex:Rose:7}
\gll je'e ty-uri p-samo?\\
so 3-good 2\textsc{sg}-feel\\
\glt ‘So, is it good how you feel?’ [T19.114]
\z

\ea
\label{ex:Rose:8}
\gll na-ni-ru\\
3pl-eat-\textsc{sp.p.nmlz}\\
\glt ‘their food’ [T19.102]
\z


There are two inflectional classes of active stems. The two rows of \tabref{tab:Rose:2} illustrate the behavior of the active suffix with respect to these two classes. Most active verbs always carry the active suffix. This is illustrated with \textit{j}\textit{año-ko} in the first row: the active suffix is present in the absence or presence of any other suffixes. A smaller number of active verbs (all with root-final /o/) obligatorily take the active suffix in some contexts only, basically when carrying stem-internal suffixes (the pluractional -\textit{ri,} a classifier, or the reduplicant). Otherwise, when carrying no suffix or stem-external suffixes (such as -\textit{nu}, first singular object), this class of active verbs does not show the active suffix. This is illustrated in the second row with \textit{j}\textit{ikpo} that does not show the active suffix in the first two columns, but does so in the third one. With this background in mind, we will see shortly that the middle-marker -\textit{wo} interacts unexpectedly with the active suffix \sectref{sec:Rose:5}.

\begin{table}
\caption{\label{tab:Rose:2}. The active suffix on the two classes of active stems}


\begin{tabularx}{\textwidth}{p{2cm}XX}

\lsptoprule
only active & with most stem-external suffixes & with all stem-internal suffixes\\
\hline
\textit{n-jaño}\textbf{\textit{-ko}}

1\textsc{sg}-watch-\textsc{act}

`I watch' & \textit{ty-jaño}\textbf{\textit{-k(o)-}}\textit{nu}

3-watch-\textsc{act-}1\textsc{sg}

`he/she/it watches me' & \textit{ty-jaño-ri}\textbf{\textit{-ko}}

3-watch-\textsc{pluract-act}

`he/she/it always watches'\\
\textit{n-jikpo}

1\textsc{sg}-answer

`I answer' & \textit{ty-jikpo-nu}

3-answer-\textsc{act-}1\textsc{sg}

`he/she/it answers me' & \textit{ty-jikpo-ri}\textbf{\textit{-ko}}

3-answer-\textsc{pluract-act}

`he/she/it always answers'\\
\lspbottomrule
\end{tabularx}
\end{table}

\section{Reflexive constructions in Trinitario}
\label{sec:Rose:3}

There is a single reflexive construction in Mojeño Trinitario. It involves the middle marker -\textit{wo} and marks the coreference of core participants. There is no other morphosyntactic strategy to encode reflexivity in the language (see \sectref{sec:Rose:4}.). This section first presents the morphological properties of the middle suffix -\textit{wo} \sectref{sec:Rose:3.1}, which are the same whatever its use, and then presents the semantics (\sectref{sec:Rose:3.2}) and the syntax (\sectref{sec:Rose:3.3}) of the reflexive construction only, in line with the focus of the volume. Other uses of the middle marker will be discussed in \sectref{sec:Rose:5}.


\subsection{Morphological properties of the middle suffix -\textit{wo}}
\label{sec:Rose:3.1}

The middle suffix -\textit{wo} attaches to the verb stem, in the same slot where object suffixes appear (although they never combine). This distributional fact could lead to an analysis where -\textit{wo} is a pronominal element, but this analysis does not hold because -\textit{wo} is invariant whatever the person of the subject, as shown in \REF{ex:Rose:9} and \REF{ex:Rose:10}. \figref{fig:Rose:2} outlines the verb template, where “V stem” stands for the template presented in \figref{fig:Rose:3}. Please note that the middle suffix -\textit{wo} occurs in a position outside of the stem (\figref{fig:Rose:3} shows a middle marker within the stem, which is the prefix \textit{ko}-, see \sectref{sec:Rose:6}.).

\begin{figure}

\caption{\label{fig:Rose:3}Verbal word template}

\textsc{s/a}-\textsc{irr-}\textbf{\textsc{Vstem}}\textsc{-irr}\textbf{\textsc{-mid}}\textbf{/\textsc{o}}\textsc{-comp-eval}-\textsc{pl=tame=degree=tame=dm}
\end{figure}

\ea
\label{ex:Rose:9}
\gll n-etpiri-k\textbf{-wo}=po nuti. \\
3-prepare-\textsc{act\textbf{-mid}=pfv} \textsc{1sg}\\
\glt ‘I got ready.’ [T38.182]
\z

\ea
\label{ex:Rose:10}
\gll v-echpu-ko vi-oso-ko\textbf{-wo} te yuku\\
1\textsc{pl}-get\_up-\textsc{act} \textsc{1pl}-heat-\textsc{act\textbf{-mid}} \textsc{prep.nh} fire\\
\glt ‘We would get up and warm up next to the fire.’ [T25.066]
\z


The middle marker -\textit{wo} has several allomorphs. The first three are predictable through general prosodic and phonological processes of the language, while the fourth results of a more restricted process.


\begin{itemize}
\item -\textit{mo} when it immediately follows an /m/, as in \textit{n-sam-mo} 1\textsc{sg}-listen-\textsc{mid} ‘I listen to myself’;

\item -\textit{v} (realized [β]) before front vowels (after hiatus resolution) or before \textit{y} as in the sequence -\textit{v=yore} ‘-\textsc{mid=fut}’ used in \REF{ex:Rose:21} (the sequence /w+j/ is often realized [ɥ]);

\item -\textit{w} when the \textit{o} is deleted through rhythmic syncope as in \REF{ex:Rose:13};

\item this /w/, stranded in coda position after the syncope of \textit{o}, is deleted and compensated by vowel lengthening when it precedes a labial consonant /p/ or /w/~- then the middle marker is not realized at all, but its presence is visible through lengthening of the preceding vowel, as in \REF{ex:Rose:41};

\item -\textit{po }when it follows the irrealis suffix -\textit{a,} as in \REF{ex:Rose:11}.

\end{itemize}

This last allomorph results from a very restricted rule: the labio-velar approximant /w/ (and its realization [β] before front vowels) plosivize in /p/ after the irrealis suffix -\textit{a} in the morphemes -\textit{wo} [wo] ‘\textsc{mid}’ and -\textit{wi} [βi] ‘2\textsc{sg}’ (see example \REF{ex:Rose:62}).\footnote{The syllable \textit{wo} is also realized \textit{po} with irrealis in roots that are likely the result of the lexicalization of the middle marker. The forms \textit{ʧowo}\textrm{}‘come back’, \textit{iʧmowo}\textrm{}\textit{\textup{‘}}find’, and \textit{itkowo}\textrm{}‘find, succeed’ are synchronically considered as roots, with /wo/ being part of the root. This analysis is due to the fact that the first two forms never occur without \textit{wo}, while the third one has quite a different meaning without \textit{wo}: \textit{it-ko} means ‘know’. Anyway, even though \textit{wo} is not segmentable as the middle marker in these forms, the irrealis is still added before \textit{wo \textup{rather than after the root}}, and \textit{wo}\textrm{}is realized as \textit{po}. As a consequence, roots \textit{ʧowo}\textrm{}‘come back’, \textit{iʧmowo}\textrm{}\textit{\textup{‘}}find’, and \textit{itkowo}\textrm{}‘find, succeed’ show suppletive irrealis forms \textit{ʧapo}, \textit{iʧmapo}\textrm{}and \textit{itkapo}.}


\ea
\label{ex:Rose:11}
\gll t-emtyo-k-a-\textbf{po}=pka\\
3-lose-\textsc{act-irr\textbf{-mid}=dub}\\
\glt ‘It may get lost.’ [T25.148]
\z



A surprising property of the middle suffix is that it makes the active suffix (presented in \sectref{sec:Rose:2.4}) appear on the class of active verbs that show the active suffix only when a stem-internal suffix is present. \tabref{tab:Rose:3} is similar to the last row of previous \tabref{tab:Rose:2} in showing that the active suffix is present on some verbs, here represented by the verb \textit{echo} ‘know’, \footnote{The example in the table is not illustrative of the reflexive meaning per se but of another middle use of the marker -\textit{wo} (see \sectref{sec:Rose:5}).} only if they take a stem-internal suffix. But the last column adds the information that the middle marker -\textit{wo \textup{is also a trigger of the presence of the active suffix}} on those active verbs that do not always show the active suffix. In a way, although the middle-marker -\textit{wo} occurs oustide of the verb stem, it behaves like a stem-internal suffix. This is coherent with the fact that stem-internal affixes are essentially derivational affixes and built up the semantic and syntactic argument structure of the stem. The position of the middle marker further away from the root is probably to be taken as a sign of a more recent grammaticalization.

\begin{table}
\caption{\label{tab:Rose:3}. Interaction of the active suffix with stem-internal suffix or middle marker}


\begin{tabularx}{\textwidth}{p{2.5cm}XXX}

\lsptoprule
only active & with most stem-external suffixes & with all stem-internal suffixes & with middle marker -\textit{wo}\\
\hline
\textit{n-echo}

1\textsc{sg}-know

‘I know’ & \textit{wo n-ech-a}

\textsc{neg} \textsc{1sg}-know-\textsc{irr}

‘I don’t know’ & \textit{n-ech-pi-ko}

1\textsc{sg}-know-\textsc{clf-act}

‘I know (a language, a song, a word)’ & \textit{wo n-echo-k-a}\textbf{\textit{-po}}

\textsc{neg} \textsc{1sg}-know-\textsc{act-irr\textbf{-mid}}

 ‘I did not know’\\
\lspbottomrule
\end{tabularx}
\end{table}

In addition, the middle suffix also applies on verbs that are not active, such as \textit{itna} ‘be used to’ in \REF{ex:Rose:42} (where it is realized \textit{etna} for phonotactic reasons).


\subsection{Semantics of the reflexive construction}
\label{sec:Rose:3.2}

This section reviews the situation types expressed by the middle marker that can be conceived as falling within the reflexive domain. “Situation types can be thought of as sets of situational or semantic pragmatic contexts that are systematically associated with a particular form of expression.” (\citealt[7]{Kemmer1993}; following \citealt{Talmy1972}). The Mojeño Trinitario middle marker -\textit{wo} is used on extroverted verbs like \REF{ex:Rose:9} to express true reflexive situation types in the sense of \citet[45]{Kemmer1993}: «~The direct reflexive situation type comprises semantic contexts which involve coreference in an event consisting of a single event frame~». Although this situation type is generally conceived as the prototypical reflexive function, it represents only a small part of the uses of the middle marker -\textit{wo}~in Mojeño Trinitario: in a random sample of 91 occurrences of -\textit{wo}, only 9 of them (i.e. less than 10\%) are actually expressing a direct reflexive situation type. The marker -\textit{wo} is also used on introverted verbs, in situation types often lumped with reflexive:\footnote{\citet[53-70]{Kemmer1993} considers these situation types to be distinct from the reflexive situation types because the participant roles are not as easily distinguishable as in reflexive situations.} these are body actions situation types, comprising grooming \REF{ex:Rose:12}, change in body posture \REF{ex:Rose:13}, other body actions \REF{ex:Rose:14}, translational motion \REF{ex:Rose:15} and positionals \REF{ex:Rose:16}. Other situation types that are clearly middle and do not belong to this intermediate body action types are described in \sectref{sec:Rose:5}..


\ea
\label{ex:Rose:12}
\gll t-vejamuiri-k\textbf{-wo} p-ñi `chane \\
3-undress-\textsc{act\textbf{-mid}} \textsc{dem-m} person\\
\glt ‘The man gets undressed.’ (PathC.031)
\z

\ea
\label{ex:Rose:13}
\gll powre-chicha ty-akyo-j-rii-ko\textbf{-w}=ri'i\\
poor-\textsc{emp} 3-fold-\textsc{clf}.amorph-\textsc{pluract-act\textbf{-mid}=ipfv}\\
\glt ‘Poor him, he is bent, crouched down.’ [T40.070]
\z

\ea
\label{ex:Rose:14}
\gll j-ma-ni ty-uuja-ja-me-k\textbf{-wo}-n=ri'i te n-chokio \\
\textsc{dem-nh.pl-prox} 3-scratch-\textsc{red-clf}.fabric-\textsc{act\textbf{-mid}-pl=ipfv} \textsc{prep.nh} \textsc{1sg}-be\_close\\
\glt ‘these (stinky dogs) are scratching themselves next to me.’ T29.046
\z

\ea
\label{ex:Rose:15}
\gll t-pojcha-j-ko\textbf{-wo} te j-ena `mu'ji\\
3-enter-\textsc{clf}:amorph-\textsc{act\textbf{-mid} prep.nh} \textsc{nh.sh-dist} husk\\
\glt ‘He got into that heap of corn husks (to hide).’ [T35.061]
\z

\ea
\label{ex:Rose:16}
\gll t-chum-ko\textbf{-wo} \\
3-hang-\textsc{act\textbf{-mid}}\\
\glt ‘It hangs.’ [Answer to the question: Where is the lamp?] (LocC.13)
\z

\subsection{The syntax of the reflexive construction}
\label{sec:Rose:3.3}

As mentionned above, the middle suffix -\textit{wo} can indicate coreference between two core participants that could be expressed as subject and object in a non-reflexive construction (compare \REF{ex:Rose:1} and \REF{ex:Rose:2}). These two participants can be agent and patient as in \REF{ex:Rose:17}, or other semantic roles like stimulus and experiencer as in \REF{ex:Rose:18}. Through combination with the benefactive applicative as in \REF{ex:Rose:19}, the subject of the reflexive construction can combine the roles of agent and benefactive (the applied object of the applicative construction).


\ea
\label{ex:Rose:17}
\gll s-yoyure\textbf{-wo}=richu s-echti-k=ri'i to s-ye'e=yo.\\
3\textsc{f}-rush\textbf{-}\textsc{\textbf{mid}=restr} 3\textsc{f}-cut\_soft-\textsc{act=ipfv} \textsc{art.nh} \textsc{3f-gpn=fut}\\
\glt ‘She rushed to cut her share.’ [T27.031]
\z

\ea
\label{ex:Rose:18}
\gll n-imooro-k\textbf{-wo} \\
1\textsc{sg}-watch-\textsc{act\textbf{-mid}}\\
\glt ‘‎I am looking at myself.’ [elicited]
\z

\ea
\label{ex:Rose:19}
\gll ma-wachri-s-no\textbf{-wo}\\
3\textsc{m}-buy-\textsc{act-appl\textbf{-mid}}\\
\glt ‘He bought it for himself.’ (adapted from \citealt{Gill1957}: 132)
\z


The middle marker is found with the reflexive meaning on transitive verb stems only, since this meaning involves a situation type with two distinguishable participant roles. I consider the Mojeño Trinitario reflexive constructions to be intransitive: no object noun phrase ever occurs (recall that there is no set of reflexive pronouns in the language (\sectref{sec:Rose:2.2})), and only the subject is indexed on the verb with a person prefix. However, since subject marking for first and second person subjects do not differ depending on transitivity and noun phrases are optional (\sectref{sec:Rose:0}) \todo{section 0?}, the transitivity analysis of individual sentences is often ambiguous at the surface level. Nevertheless, detransitivization is overtly marked when the subject is a third person, because it is then always indexed with \textit{t}-, as on intransitive verbs (and transitive verbs with a first or second person object).



This section has described the uses of the middle marker -\textit{wo} that can be considered to be reflexive, even though some of these are considered by other authors like \citet{Kemmer1993} not to carry a true reflexive meaning, but rather some senses of the middle. Other middle uses of -\textit{wo}, clearly distinct from the reflexive uses, are discussed in \sectref{sec:Rose:5}.


\section{The expression of coreference situations other than between core participants}
\label{sec:Rose:4}

The preceding section has shown that the middle marker -\textit{wo} is used to encode the coreference between two core participants. Coreference of two arguments other than the core participants are not usually marked with this marker in Mojeño Trinitario. This section inquires on how these situations can be encoded.



Non-core arguments are indexed by person prefixes. Person prefixes on nouns express their possessors, while person prefixes on prepositions express their object). These person prefixes can have either reflexive or non-reflexive interpretations. This indetermination is illustrated here for adnominal possession, and exemplified with the third person prefix for a feminine possessor \textit{s}- ‘her’. Obviously, the interpretation of coreference with the subject is excluded if the possessed noun is part of the subject noun phrase as in \REF{ex:Rose:20}, or if the subject is not a third person as in \REF{ex:Rose:21}. In examples where the possessed noun is not the subject, and the subject is a third person of the same gender/number, the referent of the possessor is interpreted as coreferential or not with the subject depending on the context. Most of the time, the context makes it transparent who is the referent of the possessor.\footnote{Searching for all nouns carrying a third person feminine possessive prefix in my corpus, there was no example the interpretation of which was in fact ambiguous.} In example \REF{ex:Rose:22} from a text, the feminine possessor of the object is interpreted as coreferential with the subject, but in some other context (for example, if we knew that the referent of the subject does not own a recorder), it could refer to another feminine third person. In \REF{ex:Rose:23}, it is also clearly coreferential with the subject.


\ea
\label{ex:Rose:20}
\gll ñi-ke=pripu=iji ñi \textbf{s-}ima\\
3\textsc{m}-be\_like=\textsc{conc.mot.ipfv=rpt} \textsc{3m} \textsc{\textbf{3f}}\textbf{-}husband\\
\glt ‘Her husband was coming.’ [T20.044]
\z

\ea
\label{ex:Rose:21}
\gll juiti v-naekcho-v=yore=po p-jo-ka \textbf{s-}emtone \\
today 1\textsc{pl}-start-\textsc{mid=fut=pfv} \textsc{dem-nh-prox} \textsc{\textbf{3f}}\textbf{-}work\\
\glt ‘Today we are going to start her work.’ [T04.001]
\z

\ea
\label{ex:Rose:22}
\gll kope s-era'i-ko to \textbf{s-}ye'e gravadora.\\
past\_day 3\textsc{f}-leave-\textsc{act} \textsc{art.nh} \textsc{\textbf{3f-}gpn} recorder\\
\glt ‘The other time, she\textsubscript{i} left her\textsubscript{i/j} recorder.’ [T26.037]
\z

\ea
\label{ex:Rose:23}
\gll p-su `seno t-ero=ri'i une s-ko-chane p-ñi \textbf{s}-ima=puka\\
\textsc{dem.f} woman 3-drink=\textsc{ipfv} water \textsc{3f-vz}-person \textsc{dem-m} \textsc{\textbf{3f}}-husband=\textsc{dub}\\
\glt ‘The woman\textsubscript{i} drinks water with a man who might be her\textsubscript{i/j} husband.’ (PathS.75) [the speaker is describing a video stimulus on the expression of path, and does not know the two actors nor their personal relationship]\footnote{This task is described in \citet{VuillermetKopecka2019}.}
\z


There is a subtype of the reflexive construction using the middle marker -\textit{wo} that encodes the coreference of the possessor of a noun with the subject: a noun expressing a body part is incorporated in a verb, which is reflexivized with the middle marker -\textit{wo}, as in \REF{ex:Rose:24}. There is another construction where the middle marker -\textit{wo} helps interpreting the coreference of the adnominal possessor and the subject, but where the middle marker expresses grooming situation types, or self-affectedness, rather than reflexivity (for example, in \REF{ex:Rose:25}, the woman is not literally plaiting herslef, her body). This can be used whether the object is a body part \REF{ex:Rose:25} or not \REF{ex:Rose:26}, and the clause is not detransitivized. In contrast, a lexical way to explicitly inform on the non-coreference with the subject is to use the adjective \textit{‘pona} ‘other’ \REF{ex:Rose:27}.


\ea
\label{ex:Rose:24}
\gll ñi t-yuk\textbf{-pan}-ne-ch\textbf{-wo}=o'i \\
\textsc{art.m} 3-touch\textbf{-jaw}-\textsc{clf.}back-\textsc{act\textbf{-mid}=ipfv}\\
\glt ‘the one who is pressing his cheeks’ [T45.066]
\z

\ea
\label{ex:Rose:25}
\gll su 'seno t-eja-ra-ko=o'i t-jigwaj-ji-ch\textbf{-wo} to \textbf{s-}chutmoko\\
3\textsc{f} woman 3-sit-\textsc{pluract-act=ipfv} 3-plait-\textsc{clf}:amorph-\textsc{act\textbf{-mid} art.nh} \textsc{\textbf{3f}}\textbf{-}hair\\
\glt ‘The woman is sitting and plaiting her hair.’ (PathM.12)
\z

\ea
\label{ex:Rose:26}
\gll t-vemju-ju-pew-cho\textbf{-wo} j-ma \textbf{s-}epkopewo \\
3-take\_off-\textsc{red}-\textsc{clf}.foot-\textsc{act\textbf{-mid}} \textsc{dem-nh.pl} \textsc{\textbf{3f-}}flipflop\\
\glt ‘She takes off her flipflop.’ (PathC.68)
\z

\ea
\label{ex:Rose:27}
\gll t-yusti-j-ko p-jo \textbf{s-}chutmoko su \textbf{`po-na} `seno\\
3-cut-\textsc{clf}:amorph-\textsc{act} \textsc{dem-nh.sg} \textsc{\textbf{3f-}}hair \textsc{art.f} \textbf{other-}\textsc{\textbf{clf}}\textbf{:h} woman\\
\glt ‘She cuts the hair of another woman.’ (Cut\& BreakF.33)
\z


As for obliques coreferential with the subject, the single inflectable preposition in the language takes a single person prefix paradigm, so that coreference cannot be marked in the obliques.\footnote{Most locative meanings are actually expressed either through verbs or relational nouns.} In elicitation as in \REF{ex:Rose:28}, a consultant made use of the unstressed restrictive clitic =\textit{(ri)chu} ‘only, just, exactly’ on a prepositional phrase to create a contrast between two possible interpretations of the person prefix on the preposition. The restrictive marker\footnote{The restrictive marker \textit{=(ri)chu} can be found on various parts of speech and is usually translated as ‘just, only, precisely’.} does not in itself express coreference, but refines the identifiability of the referent by excluding alternative referents. The only morphological resource to mark the coreference of a peripheral participant is the combination of an applicative \textit{-(')}\textit{u} and the middle marker -\textit{wo}, which marks the coreference of an object (the promoted oblique) and a subject. This is illustrated in \REF{ex:Rose:29} with the goal applicative -\textit{(')}\textit{u}, and had been illustrated in \REF{ex:Rose:19} with the benefactive applicative -\textit{(i)no}.


\ea
\label{ex:Rose:28}
\gll su `seno s-wachri-k=ri'i to charuji s-ye'e=yo / s-ye'e=yore\textbf{=richu}.\\
3\textsc{f} woman 3\textsc{f}-buy-\textsc{act=ipfv} \textsc{art.nh} food \textsc{3f-prep=fut} \textsc{/} \textsc{3f-prep=fut\textbf{=restr}}\\
\glt ‘‎The woman has bought food for her / herself (litt. for her precisely).’ [elicited]
\z

\ea
\label{ex:Rose:29}
\ea
\label{ex:Rose:29a}
\gll p-su `seno t-semo s-ye'e.\\
 \textsc{dem-f} woman 3-be\_angry \textsc{3f-prep}\\
\glt ‘The woman is angry with her.’ [elicited]

\ex
\label{ex:Rose:29b}
\gll p-su `seno t-sem-u-ch\textbf{-wo}=richu.\\
 \textsc{dem-f} woman 3-be\_angry\textsc{-appl-act\textbf{-mid}=restr}\\
\glt ‘The woman is angry with herself.’ [elicited]
\z
\z



There is no means of marking coreference between two non-core arguments. Again, the restrictive clitic =\textit{richu} can be used, at least in elicitation, to help the addressee interpret the potentially ambiguous reference of the person prefix.


\ea
\label{ex:Rose:30}
\gll n-ime-ri-ch=ri'i su Maria et-na s-kuna \\
1\textsc{sg}-show-\textsc{pluract-act=ipfv} \textsc{art.f} Maria one-\textsc{clf.}gen 3\textsc{f}-image\\

\gll s-ye'e / s-ye'e\textbf{=richu} \\
\textsc{3f-prep} \textsc{/} \textsc{3f-prep\textbf{=restr}}\\
\glt ‘I showed Maria a picture of her / herself only (litt. precisely her).’ [elicited]
\z


Middle-marking is not used for coreference across clauses. The examples \REF{ex:Rose:31} and \REF{ex:Rose:32} show that there is no marking for coreference between an element of a complement clause (here the subject) and the subject of the matrix clause. In discourse, a set of focus suffixes combinable with pronouns only can be useful for reference tracking across sentences, like -\textit{pooko} ‘the very same’ in \REF{ex:Rose:32}.\footnote{There is a set of focus suffixes used on pronouns only: -\textit{ji} illustrated in \REF{ex:Rose:45} and \REF{ex:Rose:46}, -\textit{koocho}, -\textit{pooko} in \REF{ex:Rose:33}, -\textit{yo} and -\textit{yumja}. They are used only on pronouns in core argument positions, but not in reflexive constructions.}


\ea
\label{ex:Rose:31}
\gll esu s-echo to ñ-epia-k=yore to peti.\\
\textsc{3f} \textsc{3f}-know \textsc{art.nh} \textsc{3m}-make-\textsc{act=fut} \textsc{art.nh} house\\
\glt ‘She knew that he was going to build a house.’ [elicited]
\z

\ea
\label{ex:Rose:32}
\gll esu s-echo=po to s-joch-ra=yre to tapajo to peti.\\
\textsc{3f} \textsc{3f}-know=\textsc{pfv} \textsc{art.nh} \textsc{3f-}close-\textsc{ev.nmlz=fut} \textsc{art.nh} door \textsc{art.nh} house\\
\glt ‘She remembered to close the house door.’ [elicited]
\z

\ea
\label{ex:Rose:33}
\gll tyompo esu t-k-ijare=e'i… esu\textbf{-pooko}=tse=ro,\\
and.also 3\textsc{f} 3-\textsc{vz}-name=\textsc{ipfv} \textsc{3f\textbf{-foc}=contrast=unq}\\
\gll esu tkijaree'i Dolorosa.\\
3\textsc{f} 3-\textsc{vz}-name=\textsc{ipfv} \textsc{\textup{Dolorosa}}\\

\glt [Preceding text: But there are only two: the Carmen Virgin and the mother of God, Holy Mary], and also the one called… the very same one, the one called Dolorosa. [the speaker realizes that the Holy woman he wanted to add to his list was the same person than the preceding one]. [T25.141]
\z

\section{Other functions of the middle marker -\textit{wo}}
\label{sec:Rose:5}

This section explores the functions of the middle marker -\textit{wo} other than its reflexive use. It first lists the situation types for which the middle marker -\textit{wo} is used. Then it lists the various semantico-syntactic changes produced in verbs stems by the use of -\textit{wo}. Finally, the use of -\textit{wo} on nominalizations is mentionned.



Middle situation types are events in which (a) the Initiator is also an Endpoint, or affected entity and (b) the event is characterized by a low degree of elaboration (\citealt{Kemmer1993}: 243), excluding reflexive and reciprocal proper. Below is a list of the middle situation types encoded in Mojeño Trinitario with the middle marker -\textit{wo}.


\begin{itemize}
\item the reflexive situation types (\sectref{sec:Rose:3});

\item some middle situation types : grooming, change in body posture, other body actions, translational motion and positionals (\sectref{sec:Rose:3});

\item prototypical reciprocal \REF{ex:Rose:34} and naturally reciprocal situation types \REF{ex:Rose:35};\footnote{\citet[17; 96--97]{Kemmer1993} defines these as follows: “The prototypical reciprocal context is a simple event frame expressing a two-participant event in which there are two relations; each participant serves in the role of Initiator in one of those relations and Endpoint in the other. ” and “Naturally reciprocal events are actions or states in which the relationship among two participants is usually or necessarily mutual or reciprocal. This class includes verbs of fighting, embracing, meeting, greeting, conversing, and so forth.”}

\item cognition \REF{ex:Rose:36};

\item emotion \REF{ex:Rose:37};

\item and spontaneous events \REF{ex:Rose:38},\footnote{A common example is the verb form \textit{t-ekti-k}\textbf{\textit{-wo}} 3-blow\_hard-\textsc{act\textbf{-mid}} ‘it blows hards’ used nominally with an article, \textit{to tektikwo} ‘a strong wind’.} including the expression of phases like ‘start’ in \REF{ex:Rose:39} or ‘end’.

\end{itemize}

\ea
\label{ex:Rose:34}
\gll juiti v-yon=ñore v-echji-ri-k\textbf{-wo}=yre na-e p-no-kro\\
today 1\textsc{pl}-go=\textsc{fut} 1\textsc{pl}-speak-\textsc{pluract-act\textbf{-mid}}=\textsc{fut} \textsc{3pl-prep} \textsc{dem-h.pl}-\textsc{pot.loc}\\
\glt ‘Today we are going to discuss with these’. [T24.087]
\z

\ea
\label{ex:Rose:35}
\gll esu t-itu-ch\textbf{-wo}=yre=ripu=ini=ji\\
3\textsc{f} 3-marry-\textsc{act\textbf{-mid}=fut=pfv=pst=rpt}\\
\glt ‘It is said that she was about to get married.’ [T19.177]
\z

\ea
\label{ex:Rose:36}
\gll t-ponre-ri-k\textbf{-wo}=ripo \\
3-think-\textsc{pluract-act\textbf{-mid}=pfv}\\
\glt ‘He is pensive/worried.’ [T40.154]
\z

\ea
\label{ex:Rose:37}
\gll n-yugiej-ko\textbf{-wo}\\
1\textsc{sg}-make\_uneasy-\textsc{act\textbf{-mid}}\\
\glt ‘I feel uneasy’ [T38.040]
\z

\ea
\label{ex:Rose:38}
\gll t-si-'o\textbf{-o}=po to une. \\
3-be.much-\textsc{act\textbf{-mid}=pfv} \textsc{art.nh} water \textsc{}\\
\glt ‘There had been a flood (lit. the water had been much).’ [T38.102]
\z

\ea
\label{ex:Rose:39}
\gll juiti v-naekcho\textbf{-v}=yore=po to v-ye`e gravasion. \\
today 1\textsc{pl}-start\textbf{-}\textsc{\textbf{mid}=fut=pfv} \textsc{art.nh} \textsc{1pl}-\textsc{gpn} recording\\
\glt ‘Today we are going to start our recording.’ [T30.001]
\z


Finally, there are some cases where the event does not seem to fall within a situation type described as middle, but are instead typically one- or two-participant events. In these cases, there is some emphasis on the subject. Three types of functions have been observed :


\begin{itemize}
\item the subject is particularly affected as in \REF{ex:Rose:40};\footnote{See \citet{Creissels2007} for a similar analysis of \textit{se}- verb forms in French involving no valency change.}
\item the subject is fully involved in the activity, with verbs strongly involving the agent, and not necessarily for their own benefice, as in ‘do fast’, ‘look for’, ‘carry’, or ‘pull’ illustrated in \REF{ex:Rose:41};
\item the subject is contrasted with other possible referents \REF{ex:Rose:42}.
\end{itemize}

\ea
\label{ex:Rose:40}
\gll ene takepo v-era'i-k\textbf{-wo}=po v-ke=ripo {una hora o dos horas}.\\
and then 1\textsc{pl}-leave-\textsc{act-mid=pfv} \textsc{1pl}-do.like=\textsc{pfv} one\_or\_two\_hours\\
\glt ‘And then we left it for one or two hourse (about a heavy load)’ [T25.004]
\z

\ea
\label{ex:Rose:41}
\gll t-chuu-ko\textbf{-o}=po to kareta to wiy-ono te to `chene\\
3-pull-\textsc{act\textbf{-mid}=pfv} \textsc{art.nh} cart \textsc{art.nh} ox-\textsc{pl} \textsc{prep.nh} \textsc{art.nh} path\\
\glt ‘ The oxen pull the cart on the path.’ [T28.057]
\z

\ea
\label{ex:Rose:42}
\gll n-itna, te p-jo-ka `wósare wo'=richu na-(a)-etna\textbf{-wo}.\label{bkm:Ref33103088}\\
1\textsc{sg}-be\_used \textsc{prep.nh} \textsc{dem-nh.sg-prox} village \textsc{neg=restr} \textsc{3pl}-\textsc{irr}-be\_used\textbf{-}\textsc{\textbf{mid}}\\
\glt ‘I am used to it, here in town they are not used to it.’ [T34.049]\footnote{The three vowels (/a/ of the prefix, /a/ of the irrealis prefix and the initial vowel of \textit{itna} ‘be used to’) merge into a diphthong \textit{ae}.}
\z


The middle uses have been up to now considered in terms of the situation types covered by this marker. The remainder of this section focuses on the various semantico-syntactic changes induced by the use of -\textit{wo} in the argument structure of the verb root. Detransitivization with subject and object being coreferential has been discussed in \sectref{sec:Rose:3}. (the reflexive construction). The middle marker -\textit{wo} involves four other types of detransitivization:


\begin{itemize}
\item decausative, as in \REF{ex:Rose:16} where the P participant is promoted as subject and the A is left unexpressed;
\item autocausative, as in \REF{ex:Rose:17}, where the subject has both A and P roles, but the action on oneself is not fully identical with the same action realized on some other participant;
\item antipassive with demotion of P as an oblique, as in \REF{ex:Rose:34} (the verb \textit{echijiriko} ‘speak to’ normally takes the addressee as the object, but in \REF{ex:Rose:34} the addressee is encoded in a prepositional phrase, in what is called a discontinuous reciprocal construction)\footnote{“Discontinuous constructions are those in which the second reciprocant is a non-subject.” \citep[396]{NedjalkovGeniusiene2007}};
\item antipassive with P deletion, as in \REF{ex:Rose:43} (the verb \textit{issiko} ‘whistle’ can normally take an object for the addressee).
\end{itemize}

\ea
\label{ex:Rose:43}
\gll t-issi-sio-k\textbf{-wo}=pri'i=ji. \\
3-whistle-\textsc{red}-\textsc{act}\textbf{\textsc{-mid}}\textsc{=conc.mot.ipfv=rpt}\\
\glt ‘He was coming whistling.’ [T6.093]
\z


Additionally, there are cases where no valency change is observed, on either transitive or intransitive verbs. First, a transitive verb affixed with -\textit{wo} can remain transitive, as in \REF{ex:Rose:39} and \REF{ex:Rose:41} for instance where an object noun phrase follows the verb. Second, the middle marker -\textit{wo} can be found on intransitive verbs, where it logically has no detransitivization effect either, as in \REF{ex:Rose:44}.


\ea
\label{ex:Rose:44}
\gll p-no po-mri-ono t-eja-ru-pue-k\textbf{-wo}-n=ri'i \\
\textsc{dem-pl} other-\textsc{clf}:group-\textsc{pl} 3-sit-\textsc{?-clf}:ground-\textsc{act\textbf{-mid}-pl=ipfv}\\
\glt ‘The others are sitting all over the ground.’ [T46.011]
\z


Finally, one observes the use of a sequence \textit{wo} on some other parts of speech that verbs. There are a few attestations of \textit{wo} on pronouns, after a focus marker -\textit{ji} as in \REF{ex:Rose:42}. This \textit{wo} could well be the middle marker, as it alternates in that position with the reciprocal marker –\textit{k(o)ko} shown in \REF{ex:Rose:43}.


\ea
\label{ex:Rose:45}
\gll nut-ji\textbf{-wo} m-ponre-ri-k-wo \\
\textsc{1sg-foc\textbf{-mid}} \textsc{1sg}-thing-\textsc{pluract-act-mid}\\
\glt ‘I have been thinking.’ [T43.029]
\z

\ea
\label{ex:Rose:46}
\gll eno-ji-kko t-imkata-koko-no \\
\textsc{3pl-foc-recp} 3-help-\textsc{recp-pl}\\
\glt ‘They both help each other.’ [elicited]
\z


Also, a sequence \textit{wo} is rather frequent after various nominalizers.\footnote{The location of the middle marker after the nominalizer may look surprising, but note that other verbal morphology like TAME occurs after nominalizers in Mojeño Trinitario, and that other Arawak languages also commonly show the sequence nominalizer + middle in that order, such as Yukuna (Lemus Serrano, in preparation).} Out of a small random list of 91 occurrences of \textit{wo} on an item comprising a verb root, 9 are nominalized. I consider this \textit{wo} to be the middle marker. In some examples, there is indeed a clear middle function, like the reciprocal one in \REF{ex:Rose:47}. In others, it can simply be interpreted as antipassive, since the patient of the ‘fool’ event is left unexpressed and is interpreted generically \REF{ex:Rose:48}. Since most nominalization processes are effectively reducing the valency of the affected clause, there is a logical link between nominalization and middle.


\ea
\label{ex:Rose:47}
\gll to v-itu-ch-ra\textbf{-wo}\\
\textsc{art.nh} \textsc{1pl}-marry-\textsc{act-ev.nmlz\textbf{-mid}}\\
\glt ‘our marriage’ [T42.008]
\z

\ea
\label{ex:Rose:48}
\gll to na-kitem-ra'\textbf{-wo} \\
\textsc{art.nh} \textsc{3pl}-fool-\textsc{hab.a.nmlz\textbf{-mid}}\\
\glt ‘their being tricksters’ [T6.021]
\z

\section{The middle marker -\textit{wo} among middle marking strategies}
\label{sec:Rose:6}

Mojeño Trinitario has many other strategies than the middle-marker -\textit{wo} that participate in the middle domain. They are briefly presented in \sectref{sec:Rose:6.1}. and then the overall coverage of the middle domain is discussed in \sectref{sec:Rose:6.2}.


\subsection{Mojeño Trintario middle-marking strategies}
\label{sec:Rose:6.1}

Lability has been mentioned in \sectref{sec:Rose:2.3}, but four other markers compete with the middle marker -\textit{wo} for the expression of either a low differentiation of A and P roles or demotion of one of these two roles.



First, there is a reciprocal marker, the verbal suffix -\textit{koko} (-\textit{kko} under syncope) used in the slot following that of the active suffix (see \figref{fig:Rose:1}).\footnote{When the reciprocal is supposed to follow the -\textit{ko} allomorph of the active suffix, only two \textit{ko} syllables are realized. For glossing purposes, I consider in those cases that the reciprocal -\textit{koko} then replaces the active suffix, as in \REF{ex:Rose:50}} It marks reciprocity between two core participants only, in prototypical reciprocal events. Unlike the middle marker -\textit{wo}, it is not used for naturally reciprocal events (see definitions in footnote 17). The use of the reciprocal marker is usually decreasing the valency of the verb root: in \REF{ex:Rose:50}, the verb is detransivitized, as is visible from the use of the semantically non-specific third person subject prefix \textit{t}-.


\ea
\label{ex:Rose:49}
\gll v-echem-cho\textbf{-kko}=po \\
1\textsc{pl}-understand-\textsc{act\textbf{-recp}=pfv}\\
\glt ‘Now we understand each other.’ [T24.131]
\z

\ea
\label{ex:Rose:50}
\gll ene t-emna\textbf{-kko}-no t-ko-chicha-m=po\\
and 3-love\textbf{-}\textsc{\textbf{recp}-pl} 3-vz-children-\textsc{pl=pfv}\\
\glt ‘And they love each other and have children.’ [T21.093]
\z


Second, there is another middle marker, a prefix \textit{ko}- immediately preceding the verb root (see \figref{fig:Rose:1}). It occurs only on transitive verb roots and detransitivizes them. When it is the only middle-marking device on a verb root, the verb does not carry the active suffix. Most of the time, it then shows some medio-passive meaning as in \REF{ex:Rose:51} and \REF{ex:Rose:52}. The agent is usually not expressed (either unknown, generic or not individually important) and there is no hint of agency (expression of will, or purpose). I hypothesize that in those cases the meaning is resultative, which the absence of active morphology seems to support. It is however sometimes found with a passive function as in \REF{ex:Rose:53}, but also with an autocausative meaning \REF{ex:Rose:54}, a reflexive meaning \REF{ex:Rose:55},\footnote{Out of context, this sentence could be interpreted as a medio-passive ‘we got covered with it’, but in the specific context of this biography, the subject plays both the A and P roles.} a reciprocal one \REF{ex:Rose:56}, on body actions like grooming \REF{ex:Rose:57}, and on positionals \REF{ex:Rose:58}.


\ea
\label{ex:Rose:51}
\gll to letra, t-\textbf{k-}aju\\
\textsc{art.nh} letter 3-\textsc{\textbf{mid}}\textbf{-}write\\
\glt ‘The letters, they are written [on a T-shirt].’ (LocL.68)
\z

\ea
\label{ex:Rose:52}
\gll to vaka t-\textbf{ko-}ywa \\
\textsc{art}.\textsc{nh} meat 3-\textbf{\textsc{mid}}\textbf{-}grind\\
\glt ‘The meat is ground.’ [T25.045]
\z

\ea
\label{ex:Rose:53}
\gll p-su-ka powre `chosi `seno s-imooro-o-ko=o'i to t-\textbf{k-}e'na=a'i.\\
\textsc{dem-f-prox} poor old woman 3\textsc{f}-watch-\textsc{pluract-act=ipfv} \textsc{art.nh} \textsc{3-\textbf{mid}}\textbf{-}hit=\textsc{ipfv}\\
\glt ‘This poor old woman, she watches them being hit\textit{.’} [40.168]
\z

\ea
\label{ex:Rose:54}
\gll t-\textbf{ko}-yumrugi t-piko-vi=i'i\\
3-\textsc{\textbf{mid}}\textbf{-}hide 3-be\_scared-\textsc{2sg=ipfv}\\
\glt ‘He hid himself, he was scared of you.’ [T35.092]
\z

\ea
\label{ex:Rose:55}
\gll eto v-\textbf{k-}epko-'u \\
\textsc{3nh} \textsc{1pl-\textbf{mid}}\textbf{-}cover-\textsc{appl}\\
\glt ‘We covered ourselves with it (lit. we put this over for ourselves).’ [about protecting oneselves from the cold with home-made blankets and hammocks] [T25.066]
\z

\ea
\label{ex:Rose:56}
\gll t-imo-ko-n=giereko=o'i t-\textbf{ko}-komji-wko \\
3-sleep-\textsc{act-pl=cnt=ipfv} 3-\textsc{\textbf{mid}}\textbf{-}embrace-\textsc{clf}.amorph\\
\glt ‘They are sleeping, they are embraced.’ [T30.073]
\z

\ea
\label{ex:Rose:57}
\gll t-\textbf{ko-}sp-ugi-ono ta-ye'e \\
3\textsc{-}\textbf{\textsc{mid-}}wash\textsc{-clf}.face\textsc{-pl} 3\textsc{nh-prep}\\
\glt ‘‎They wash their faces in it.' [T20.026]
\z

\ea
\label{ex:Rose:58}
\gll t-\textbf{ko-}kojaru-ji te p-jo aramre\\
3-\textsc{\textbf{mid}}\textbf{-}spread\_out-\textsc{clf}.amorph \textsc{prep.nh} \textsc{dem-nh.sg} wire\\
\glt ‘They are hanging on the barbed wire.’ [Answer to the question: Where are the clothes?] (LocC.037)
\z


Third, there is a less frequent suffix -\textit{si} that attaches to the verb in the slot after that of the active suffix (see \figref{fig:Rose:1}). It has no effect on the presence of the active suffix: it neither deletes it as does the middle-marker \textit{ko}-, nor forces its presence on those active verbs that do not always display it, as does the middle marker -\textit{wo} (see \sectref{sec:Rose:2.4}.). It is rare in discourse,\footnote{It occurs in the text corpus without the prefix \textit{ko}- in two examples only, \REF{ex:Rose:59} and \REF{ex:Rose:60}.} and attaches to transitive verbs as in \REF{ex:Rose:59} and \REF{ex:Rose:60}. In these examples, even though the person prefixes \textit{v}- and \textit{ñ}- on verbs marked with -\textit{si} refer to P, and the agent is expressed by a prepositional phrase introduced by the preposition \textit{mue'} {\textasciitilde} \textit{ñe}, the verb form does not seem to have been intransitivized: specific prefixes are used for third person subjects, such as \textit{ñ}- in \REF{ex:Rose:60}. However, the suffix -\textit{si} most often associates with the middle prefix \textit{ko}-, as in \REF{ex:Rose:61} and \REF{ex:Rose:62}. In these cases, the verb form looks detransitivized (see the non-specific third person prefix \textit{t-} in \REF{ex:Rose:61}. The function is always clearly passive, and most of the time an agent can be identified (even though it is actually not usually expressed).


\ea
\label{ex:Rose:59}
\gll v-icho-ri-k\textbf{-si}-po mue' ma viya\\
1\textsc{pl}-call-\textsc{pluract-act\textbf{-pass}=pfv} \textsc{prep.m} \textsc{art.m} Lord\\
\glt ‘We have been called by the Lord.’ [T24.061]
\z

\ea
\label{ex:Rose:60}
\gll eñi t-wonokore ñ-imit-ko-\textbf{si} ñe ñi ñi-chicha\\
\textsc{pro}.\textsc{m} 3-obey 3\textsc{m}-teach-\textsc{act}\textbf{-}\textbf{\textsc{pass}} \textsc{prep}.\textsc{m} \textsc{art}.\textsc{m} 3\textsc{m}-son\\
\glt ‘He was obeying, his son had taught him to be so. (lit. he had been taught by his son)’ [T19.164]
\z

\ea
\label{ex:Rose:61}
\gll eto=ri'i t-\textbf{k-}ijro-ri-k-\textbf{si} te to Trinra.\\
\textsc{pro}.\textsc{nh}=\textsc{pfv} 3-\textbf{\textsc{mid}}\textbf{-}vendre-\textsc{pluract}-\textsc{act}\textbf{-}\textbf{\textsc{pass}} \textsc{prep.nh} \textsc{art}.\textsc{nh} \textsc{T}rinidad\\
\glt ‘This was being sold in Trinidad’. [T25.033]
\z

\ea
\label{ex:Rose:62}
\gll p-a-\textbf{k-}kojcho\textbf{-si}, t-kojch-a-p=rine\\
\textsc{2sg-irr-\textbf{mid}}\textbf{-}scold\textbf{-}\textsc{\textbf{pass}} 3-scold-\textsc{irr-2sg=restr}\\
\glt ‘Be scold, let her just scold you!’ [T37.087]
\z


Finally, some middle situations are simply unmarked, like most changes in body posture like \REF{ex:Rose:63} and non-translational motions like \REF{ex:Rose:64}.


\ea
\label{ex:Rose:63}
\gll t-eja-k=po\\
3-sit-\textsc{act=pfv}\\
\glt ‘He sat.’ [T42.055]
\z

\ea
\label{ex:Rose:64}
\gll ene n-epñu-k=po te wowre\\
and 1\textsc{sg}-turn-\textsc{act=pfv} \textsc{prep.nh} left\\
\glt ‘And I turned to the left’. [elicited]
\z

\subsection{Mojeño Trintario middle domain}
\label{sec:Rose:6.2}

The middle marker -\textit{wo} that is used to mark reflexive constructions has a much wider extension, covering many of the situations types of the middle domain. The prefix \textit{ko}- can also be considered to be a middle marker, and also has a wide extension covering a rare reflexive use, but its most frequent use really is the middle passive. Finally the two other markers are highly specialized, one as the reciprocal, \textit{-koko}, and the other as a passive marker, -\textit{si}. In the end, the Mojeño Trinitario middle domain is unusual in showing two true middle markers, whereas \citet{Kemmer1993} was considering languages to have at best one middle marker. For a comparable situations in Bantu languages, \citet[146]{DomEtAl2016} suggest to add a fourth type to \citet{Kemmer1993}’s typology : multiple-form systems. “In such a system, multiple verbal morphemes cover different parts of the canonical middle, yet sometimes conveying meanings situated on the periphery of the canonical middle domain. In most Bantu languages, the semantic space of the middle voice seems to be organized along two domains, which can be qualified as agent-oriented vs. patient-oriented functions.” Such a complementary distribution does not obviously show for Mojeño Trinitario when looking at the distribution of the markers in \figref{fig:Rose::3}, but when the most frequent use of the two middle markers \textit{ko}- and -\textit{wo} are examined, then it is clear that \textit{ko}- is more patient-oriented (uses to the right of the figure) than -\textit{wo}. Middle \textit{ko}- blocks the expression of activity on the verb and always demotes or deletes A, while middle -\textit{wo} combines with stems marked for activity. A further remark is that the fact that non-transitional motion (called non-translational motion in the rest of Kemmer’s book and this paper) is always morphologically unmarked in Mojeño Trinitario is contradicting its supposed intermediary position in \citet[222]{Kemmer1993}’s typology.\footnote{A caveat that \citet[225]{Kemmer1993} gives herself is that verbs of non-translational motion are rare in her data, so that there is no positive evidence that they follow the predications of the semantic map in \figref{fig:Rose:3}.}

\begin{figure}
\caption{\label{bkm:Ref39672373}The middle domain in Mojeño Trinitario, based on \citet[202]{Kemmer1993}.}

\includegraphics[width=\textwidth]{figures/Rose-img002.png}
 \end{figure}

\section{Conclusion}
\label{sec:Rose:7}

This paper started off by exploring the encoding of reflexive constructions, which make use of a marker -\textit{wo}. Reflexive constructions are canonical: they are reduced to coreference between the two core arguments, and the valency of the verb root is decreased. The language shows neither coreferential person pronouns or indexes, nor any dedicated marker for the other types of coreference. Pronoun focus suffixes, the restrictive clitic and the middle marker -\textit{wo} can be helpful in tracking referents, but they are not dedicated markers either. As is frequent cross-linguistically (\citealt{Kemmer1993}), the marker -\textit{wo} used to encode reflexive situation types has a much wider use and can be considered a middle marker. Furthermore, the middle marker -\textit{wo} is one of the few markers that cover the middle domain in Mojeño Trinitario. Within that domain, reflexivity is neither central, salient nor really important. Not only is there no dedicated marker for reflexivity, but also the expression of reflexivity in discourse is not frequent: it is a minor use of middle -\textit{wo} and a rare use of middle \textit{ko}-, and is also expressed lexically by a few verb roots. The typologically most interesting aspects of the encoding of the middle-domain in this language are i) the semantic distribution of the various middle markers as illustrated in \figref{fig:Rose:3}, ii) the fact that two markers are best described as middle markers, which is not accounted for by the typology of middle systems (\citealt{Kemmer1993}), and iii) the complex relationship of middle-marking strategies with the encoding of activity/stativity.



\section*{Abbreviations}

\begin{tabularx}{.45\textwidth}{lQ}
\textsc{act} & active \\
\textsc{appl} & applicative \\
\textsc{art} & article\\
\textsc{caus} & causative\\ 
\textsc{clf} & classifier\\
\textsc{cnt} & continuative\\
\textsc{conc.mot} & concomitant motion\\
\textsc{contrast} & contrastive \\
\textsc{comp} & comparison\\
\textsc{dm} & discourse marker\\
\textsc{dem} & demonstrative\\
\textsc{deriv} & derivative\\
\textsc{dim} &  diminutive\\
\textsc{dist} & distal\\
\textsc{dub} & dubitative\\
\textsc{emp} & empathy\\
\textsc{eval} &  evaluative morphology\\
\textsc{ev.nmlz} & event nominalizer\\
\textsc{f} & feminine (singular)\\
\textsc{foc} & focus marker \\ 
\textsc{fut} & future\\  
\textsc{gpn} & generic possessive noun\\
\textsc{h} & human\\ 
\textsc{hab.a.nmlz} & habitual agent nominalizer\\
\textsc{indet} & indeterminate\\
\textsc{intens} & intensifier\\
\textsc{ipfv} & imperfective\\
\textsc{irr} & irrealis\\  
\textsc{m} & masculine (singular)\\ 
\textsc{mid} & middle\\   
\textsc{neg} &  negation\\ 
\textsc{nh} &  non-human\\
\end{tabularx}

\begin{tabularx}{.45\textwidth}[t]{lQ}
\textsc{pass} &  passive\\  
\textsc{pst} & past\\
\textsc{pfv} & perfective\\  
\textsc{pl} & plural\\
\textsc{pluract} & pluractional\\ 
\textsc{pot.loc} & potential location\\
\textsc{prep} & preposition\\  
\textsc{pro} &  pronoun \\ 
\textsc{prox} & proximal \\
\textsc{recp} & reciprocal \\ 
\textsc{red} & reduplication \\ 
\textsc{restr} & restrictive \\ 
\textsc{rpt} & reportative \\   
Sa & single argument of a canonical active intransitive verb \\    
Sp & single argument of a canonical stative intransitive verb \\
\textsc{sg} & singular\\
\textsc{sp.p.nmlz} & specific patient nominalizer\\
\textsc{unq} & unquestionable \\ 
\textsc{vz}  & verbalizer \\
\end{tabularx}
         

\sloppy\printbibliography[heading=subbibliography,notkeyword=this]

\end{document}