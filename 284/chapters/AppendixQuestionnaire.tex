\documentclass[output=paper]{langscibook}
\ChapterDOI{10.5281/zenodo.7874992}

\author{Katarzyna Janic\affiliation{Adam Mickiewicz University} and Martin Haspelmath\affiliation{Max Planck Institute for Evolutionary Anthropology \& Leipzig University}}

\title{Questionnaire on reflexive constructions in the world’s languages}

\abstract{}

\begin{document}
\maketitle

\noindent A short note on our terminology: the term \textsc{reflexivizer} refers to any specialized form that expresses coreference within a clause. By \textsc{specialized form} we understand a form which at least in certain conditions necessarily expresses the coreference meaning (even if non-coreference meanings are possible elsewhere). Reflexivizers can be dependent or non-dependent forms like reflexive (pro)nouns, reflexive argument markers, or reflexive voice markers. Languages that have not developed a specialized reflexivizer express coreference with the help of other linguistic forms, e.g. personal pronouns. In this context, we prefer to talk about a \textsc{non-reflexive form}.


\section*{Basic uses of reflexivizers}

\begin{enumerate}
\item Describe the \textit{personal pronouns} and \textit{reflexive pronouns} of the language. If the language also has \textit{verbal reflexivizers} (reflexive argument markers or reflexive voice markers), give a brief description of the relevant verbal marking patterns. 
\item  If the language uses a \textit{reflexive pronoun} to express coreference, does it have distinctions such as the following?\smallskip\\
\begin{tabularx}{\linewidth}{*5{Q}}
a. person  &  b. case &   c. number &  d. obviation  &  e. gender 
\end{tabularx}

\item  How is \textit{coreference of the agent subject with the patient referent in object function} expressed in the language? Give examples like (a--d) and indicate the form (if any) expressing coreference.
  \begin{enumerate}[label=\alph*.]
  \item I saw myself in the mirror.\footnote{Feel free to change the provided examples here and elsewhere in the questionnaire, if for some reasons they are problematic in your language.}  
  \item My friend hates himself. 
  \item She praised herself. 
  \item The man killed himself. 
  \end{enumerate}

\item If the language uses a specialized reflexivizer to express coreference, is this form \textit{obligatory or optional}? 

\item If the language has \textit{several different reflexivizers} used under different conditions, what determines their distribution? 
     \begin{enumerate}[label=\alph*.]
      \item If they are in a complementary distribution, define the conditions under which each reflexivizers is selected. 
      \item  If the forms can occur in the same environments, can you think of any context in which one form is preferred over another? 
     \end{enumerate}
\end{enumerate}


Some languages use a range of different forms to express agent-patient coreference. For instance, \ili{Dutch} employs reflexive pronouns only in the third person; coreference with the first and second person is expressed by ordinary personal pronouns.

\begin{enumerate}[resume]
\item Is the use of reflexivizers subject to specific conditions relating to person or number? If it is, define them and provide relevant examples. 

\end{enumerate}
\section*{Specialized reflexive form in other functions}
\begin{enumerate}[resume]
\item
Does the reflexive form have other uses? Specify and provide relevant examples. 

\end{enumerate}
\section*{Contrast between introverted and extroverted verbs}

Transitive verbs that allow a human object can be divided into introverted and extroverted classes 
(\citealt[803]{Haiman1980}; \citealt[61]{KoenigSiemund1999}). Extroverted actions express socially antagonistic events such as ‘kill’, ‘kick’, ‘attack’, ‘hate’ and ‘criticize’, whereas introverted actions include body care (or grooming) actions exemplified by ‘wash’, ‘shave’, ‘dress’, ‘bathe’, and a few others such as ‘defend oneself’.


\begin{enumerate}[resume]
\item
How are autopathic actions with extroverted verbs expressed in the language? Give examples like (a--c) and indicate the form (if any) responsible for the coreference interpretation. 

\begin{enumerate}[label=\alph*.]
\item The dog bit itself. 
\item The girl hates herself. 
\item The politician criticized himself. 
\end{enumerate}

\item
How are autopathic actions with introverted verbs expressed in your language? Translate the examples (a--c) and indicate the form (if any) responsible for the coreference interpretation. 
\begin{enumerate}[label=\alph*.]
\item The dog was washing himself. 
\item The girl washed. 
\item He shaved. 
\end{enumerate}
\end{enumerate}

\section*{Contrast between body-part and whole-body actions}

Some languages encode body-part actions (combing hair, brushing teeth, clipping nails) similarly to those involving whole-body actions (wash, bathe, get tented) i.e. with the help of the same reflexive form (e.g. \ili{French} \textit{se peigner} ‘to comb one’s hair’ vs. \textit{se laver} ‘to wash’). In other languages, body-part and whole-body actions are treated apart, the former being expressed through a transitive construction with the body part expressed as object (e.g. \ili{English}: \textit{I comb my hair.} vs. \textit{I washed.}). Moreover, some languages specify the body-part object in addition to the reflexive form (e.g. \ili{French}: \textit{Il se lave les mains} ‘he washes his hands’). If your language contrast body-part actions with whole-body actions in coding, proceed to point 20, otherwise skip it.


\begin{enumerate}[resume]
\item How are body-part actions expressed in your language: (i) through coreference, or (ii) through a transitive construction with the body part expressed as object? Translate (a--c). 

\begin{enumerate}[label=\alph*.]
\item  The men shaved their beard. 
\item  She scratched her back. 
\item  He brushed his teeth. 
\end{enumerate}

\item If your language employs a specialized reflexive form to express body-part actions, can the body part be expressed as well? 

\item How are whole-body actions expressed in your language? Translate the examples (a--c) and indicate the form (if any) responsible for the coreference interpretation. 

\begin{enumerate}[label=\alph*.]
\item The men got dressed.
\item She washed. 
\item I bathed. 
\end{enumerate}

\end{enumerate}

\section*{Reflexive pronoun in subject position}
\begin{enumerate}[resume]
\item
Except for a few cases (e.g. \ili{Georgian}), languages do not allow reflexive pronouns in subject function. Does your language support this crosslinguistic observation? If it does not, provide a relevant example. 

\end{enumerate}


\subsection*{Coreference of the subject with various semantic roles}
\begin{enumerate}[resume]
\item 
\begin{enumerate}[label=(\alph*)]
\item \textit{Possessor}. How is coreference of the subject with a possessor referent expressed in your language? Translate (a--c) and indicate the form (if any) triggering the coreference meaning. 
    \begin{enumerate}[label=\alph*.]
    \item She\textsubscript{1} took her\textsubscript{1} umbrella. 
    \item John\textsubscript{1} reads his\textsubscript{1} book. 
    \item The women\textsubscript{1} swept their\textsubscript{1} rooms. 
    \end{enumerate}
\item  \begin{sloppypar}Can you contrast examples from {14a} with those provided in {14b} in which the referent of possessor is not coreferential with a subject?\end{sloppypar}


   \begin{enumerate}[label=\alph*.]
    \item She\textsubscript{1} took her\textsubscript{2} umbrella. 
    \item John1 reads his\textsubscript{2} book. 
    \item The women1 swept their2 rooms. 
    \end{enumerate}
\end{enumerate}

\item \begin{enumerate}[label=(\alph*)]
\item \textit{Locative}. How is coreference of the subject with a spatial referent expressed in your language? Translate (a--c) and indicate the form (if any) triggering the coreference meaning. 
    \begin{enumerate}[label=\alph*.]
    \item She\textsubscript{1} saw a snake beside her\textsubscript{1}. 
    \item John\textsubscript{1} put a book next to him\textsubscript{1}. 
    \item She\textsubscript{1} left the traces behind her\textsubscript{1}. 
    \end{enumerate}

\item Contrast examples from {15a} with those in provided {15b} in which the spatial referent is not coreferential with a subject.

    \begin{enumerate}[label=\alph*.]
    \item She\textsubscript{1} saw a snake beside her\textsubscript{2}. 
    \item John\textsubscript{1} put a book next to him\textsubscript{2}. 
    \item She\textsubscript{1} left the traces behind her\textsubscript{2}. 
    \end{enumerate}

\end{enumerate}


\item \begin{enumerate}[label=(\alph*)] 
  \item \textit{Benefactive}. How is coreference of the subject with a beneficiary referent expressed in your language? Translate (a--c) and indicate the form (if any) triggering coreference. 
  \begin{enumerate}[label=\alph*.]
    \item She bought a book for herself. 
    \item The boy cooked a dinner for himself. 
    \item They built a house for themselves. 
   \end{enumerate}

  \item Contrast examples from {16a} with those provided in {16b} in which the referent of beneficiary is not coreferential with a subject.

  \begin{enumerate}[label=\alph*.]
  \item  She bought a book for her. 
  \item  He cooked a dinner for him. 
  \item  You built a house for them. 
  \end{enumerate}
  \end{enumerate}

\item \begin{enumerate}[label=(\alph*)] 
    \item \textit{Recipient}. How is coreference of the subject with a recipient referent expressed in your language? Translate (a--c) and indicate the form (if any) triggering the coreference meaning. 

    \begin{enumerate}[label=\alph*.]
    \item John talked to himself. 
    \item They sent a postcard to themselves. 
    \item The girl gave herself a present.
    \end{enumerate}

    \item Contrast examples from {17a} with those provided in {17b} in which the referent of recipient is not coreferential with a subject.

    \begin{enumerate}[label=\alph*.]
      \item John talked to him. 
      \item They sent a postcard to them. 
      \item The girl gave a present to her. 
    \end{enumerate}
  \end{enumerate}
\end{enumerate}


\subsection*{Coreference between non-subject arguments}
\begin{enumerate}[resume]
\item How is coreference between two non-subject arguments expressed in a single clause? Translate (a--c) and indicate the form (if any) responsible for the coreference interpretation. 

\begin{enumerate}[label=\alph*.]
\item She told us\textsubscript{1} about ourselves\textsubscript{1}. 
\item He spoke with John\textsubscript{1} about himself\textsubscript{1}. 
\item John showed Mary\textsubscript{1} a picture of herself\textsubscript{1}. 
\end{enumerate}
\end{enumerate}

\subsection*{Contrast between coreference and disjoint reference}
\begin{enumerate}[resume]
\item Contrast the subject-coreference pronoun in object position (examples’) with disjoint reference pronoun in object position (examples’’). Which pronouns does your language use to code these two types of situations? 

\begin{enumerate}[label=\alph*'.]
\item The man saw himself.     vs.   a’’ The man saw him. 
\item The woman criticized herself.   vs.   b’’ The woman criticized her. 
\item He admired himself.     vs.   c’’ He admired him. 
\end{enumerate}
\end{enumerate}

\subsection*{Contrast between object and nominal adpossessor}
\begin{enumerate}[resume]
\item Contrast the subject-coreferential pronoun in object position (examples') with the subject-coreferential pronouns in adnominal possessive position (examples’’). Which pronouns does your language use to code these two types of situations? 

\begin{enumerate}[label=\alph*'.]
\item She\textsubscript{1} killed herself\textsubscript{1}.    vs.   a’’ She\textsubscript{1} killed her\textsubscript{1/2} lover. 
\item He\textsubscript{1} admires himself\textsubscript{1}.  vs.   b’’ He\textsubscript{1} admires his\textsubscript{1/2} boss. 
\item She\textsubscript{1} saw herself\textsubscript{1}.    vs.   c’’ She\textsubscript{1} saw her\textsubscript{1/2} sister. 
\end{enumerate}
\end{enumerate}

\subsection*{Contrast between exact and inclusive coreference}
\begin{enumerate}[resume]
\item Contrast exact coreference (examples') with inclusive coreference (examples”). Which pronouns does your language use to code these two types of situations? 

\begin{enumerate}[label=\alph*'.]
\item She\textsubscript{1} admires herself\textsubscript{1}.   vs.   a’’ She\textsubscript{1} admires herself and the others\textsubscript{1+X}. 
\item He\textsubscript{1} criticized himself\textsubscript{1}.   vs.   b’’ He\textsubscript{1} criticized himself and the others\textsubscript{1+X}. 
\item He\textsubscript{1} defended himself\textsubscript{1}.   vs.   c’’ He\textsubscript{1} defended himself and the others\textsubscript{1+X}. 
\end{enumerate}
\end{enumerate}

\subsection*{Long-distance coreference}
\begin{enumerate}[resume]
\item How is coreference of the subject across clauses expressed in your language? Translate (a--c) and indicate the form (if any) responsible for the coreference interpretation. 
\begin{enumerate}[label=\alph*.]
\item She\textsubscript{1} thought that she\textsubscript{1} had enough money. 
\item The boy\textsubscript{1} said that he\textsubscript{1} must go home. 
\item We\textsubscript{1} said that we\textsubscript{1} worked the whole day. 
\end{enumerate}
\end{enumerate}

\printbibliography[heading=subbibliography,notkeyword=this]
\end{document}
