\documentclass[output=paper]{langscibook}

\author{Mahamane L. Abdoulaye\affiliation{Université Abdou Moumouni, Niamey}}
\title{Reflexive constructions in Hausa} 
\abstract{This contribution describes reflexive constructions in Hausa (Chadic, Niger, Nigeria). The reflexive pronouns are based on the word \textit{kâi} ‘head, self’, in a possessive construction with a person affix that is coreferential with the clause subject (or sometimes with a preceding direct object or applied object). Subject-coreferential direct objects or applied objects are almost always expressed as reflexive pronouns (with the partial exception of the direct objects of some mental/sensation verbs). Subject-coreferential possessive NPs can optionally be expressed as reflexive pronouns but with an emphasis on the possessive relation. Subject-coreferential locative, benefactive, and instrumental/associative NPs are normally expressed as non-reflexive pronouns but they can also be optionally expressed as reflexive pronouns. The chapter also describes three different constructions that are related to the typical reflexive construction and which may be relevant for an account of its development.}

\IfFileExists{../localcommands.tex}{%hack to check whether this is being compiled as part of a collection or standalone
  \usepackage{langsci-optional}
\usepackage{langsci-gb4e}
\usepackage{langsci-lgr}

\usepackage{listings}
\lstset{basicstyle=\ttfamily,tabsize=2,breaklines=true}

%added by author
% \usepackage{tipa}
\usepackage{multirow}
\graphicspath{{figures/}}
\usepackage{langsci-branding}

  
\newcommand{\sent}{\enumsentence}
\newcommand{\sents}{\eenumsentence}
\let\citeasnoun\citet

\renewcommand{\lsCoverTitleFont}[1]{\sffamily\addfontfeatures{Scale=MatchUppercase}\fontsize{44pt}{16mm}\selectfont #1}
  
  %% hyphenation points for line breaks
%% Normally, automatic hyphenation in LaTeX is very good
%% If a word is mis-hyphenated, add it to this file
%%
%% add information to TeX file before \begin{document} with:
%% %% hyphenation points for line breaks
%% Normally, automatic hyphenation in LaTeX is very good
%% If a word is mis-hyphenated, add it to this file
%%
%% add information to TeX file before \begin{document} with:
%% %% hyphenation points for line breaks
%% Normally, automatic hyphenation in LaTeX is very good
%% If a word is mis-hyphenated, add it to this file
%%
%% add information to TeX file before \begin{document} with:
%% \include{localhyphenation}
\hyphenation{
affri-ca-te
affri-ca-tes
an-no-tated
com-ple-ments
com-po-si-tio-na-li-ty
non-com-po-si-tio-na-li-ty
Gon-zá-lez
out-side
Ri-chárd
se-man-tics
STREU-SLE
Tie-de-mann
}
\hyphenation{
affri-ca-te
affri-ca-tes
an-no-tated
com-ple-ments
com-po-si-tio-na-li-ty
non-com-po-si-tio-na-li-ty
Gon-zá-lez
out-side
Ri-chárd
se-man-tics
STREU-SLE
Tie-de-mann
}
\hyphenation{
affri-ca-te
affri-ca-tes
an-no-tated
com-ple-ments
com-po-si-tio-na-li-ty
non-com-po-si-tio-na-li-ty
Gon-zá-lez
out-side
Ri-chárd
se-man-tics
STREU-SLE
Tie-de-mann
}
  \bibliography{localbibliography}
  %\togglepaper[5]
}{}

\begin{document}
\maketitle

\section{Introduction}\label{sec:Abdoulaye:1}

Hausa (Chadic, Niger, Nigeria) generally requires a distinctive marking for the coreference between a subject NP and another NP in the minimal clause, in particular when the second NP is a direct object, an applied object, and, optionally, an adnominal possessive pronoun, or the object of certain prepositions. This distinctive marking, the reflexive pronoun, is built on the noun \textit{kâi} ‘head, self’ combined in a possessive construction with a person suffix referring to the antecedent (e.g., \textit{kâ-n-shì} ‘himself’, lit. ‘self-of.\textsc{m-3sg.m}’). An example is given in \REF{ex:Abdoulaye:1}:


\ea%1
    \label{ex:Abdoulaye:1}
    \gll Yaa  bugè  kânshì.\\
        \textsc{3sg.m.cpl} hit \textsc{refl.3sg.m}\\
    \glt `He hit himself.’
  \z


In sentence (\ref{ex:Abdoulaye:1}), the person/tense/aspect marker \textit{yaa} (or ``subject pronoun” in Hausa linguistics) is coreferential with the person suffix \nobreakdash-\textit{shi}, which is embedded in a possessive construction with the noun \textit{kâi} `head, self’, forming the reflexive pronoun \textit{kânshì} `himself’. According to 
\citet[529]{Newman2000} reflexive pronouns based on a word (ultimately) meaning `head’ are widespread among Chadic languages.



This chapter describes the reflexive construction in Hausa, drawing heavily on \citet{Newman2000}, who gives the most detailed and exhaustive account of the construction in the language. The chapter also relies on the translation of the questionnaire sentences submitted to the judgment of informants (40 years-old and up, mostly Katsinanci dialect speakers) as well on data from published sources or collected otherwise, as indicated. The chapter also uses sentences constructed by the author, which are then checked with other native speakers.

The chapter is structured as follows. \sectref{sec:Abdoulaye:2} gives the overview of the pronominal system in Hausa. \sectref{sec:Abdoulaye:3}--\ref{sec:Abdoulaye:4} describe, respectively, the coreference patterns between the subject and the direct object and those between the subject and other syntactic functions. \sectref{sec:Abdoulaye:5} describes the coreference patterns between non-subject NPs. \sectref{sec:Abdoulaye:6} describes two types of self-intensifiers in Hausa. Finally, \sectref{sec:Abdoulaye:7} discusses the word \textit{kâi} in its usage as ‘self, oneself’ in compounds and fixed expressions.

\section{Overview of Hausa personal pronouns}\label{sec:Abdoulaye:2}

Hausa distinguishes various sets of pronouns depending on their syntactic function: the independent pronouns set (with a long final vowel or with two syllables), the object set with a reduced form (monosyllabic, and with a short final vowel), and the subject pronouns sets which combine (and are sometimes fused) with the tense/aspect markers. Some of the pronouns sets are illustrated in Table~\ref{tab:Abdoulaye:1} (see \citealt[72ff]{Caron1991}; \citealt[476ff]{Newman2000} for more details).


\begin{table}[ht]
    \centering
    \begin{tabular}{p{2cm}p{2cm}p{2cm}p{2cm}p{2cm}}
    \lsptoprule
    {Person} & {Independent} {pronouns} & {Direct} {object} {pronouns} & {Completive} {subject} {pronouns} &     {Future} {subject} {pronouns} \\
    \hline 
    1\textsc{sg} & nii & ni/nì & naa & zaa nì/zân\\
    2\textsc{sg.m} & kai & ka/kà & kaa & zaa kà\\
    2\textsc{sg.f} & kee & ki/kì & kin & zaa kì\\
    3\textsc{sg.m} & shii & shi/shì (ya/yà) & yaa & zaa shì/zâi\\
    3\textsc{sg.f} & ita & ta/tà & taa & zaa tà\\
    1\textsc{pl} & muu & mu/mù & mun & zaa mù\\
    2\textsc{pl} & kuu & ku/kù & kun & zaa kù\\
    3\textsc{pl} & suu & su/sù & sun & zaa sù\\
    Impersonal & -- & -- & an & zaa à\\
    \lspbottomrule
    \end{tabular}
    \caption{Some Hausa pronominal paradigms}\label{tab:Abdoulaye:1}
\end{table}

The independent pronouns appear in isolation, in topicalization, in nominal emphasis (e.g. \textit{ita} \textit{Maaɍìyaa} ‘as for Maria’), or as objects of some prepositions (e.g. \textit{dà} \textit{ita} ‘with her/it’). Direct object pronouns immediately following a verb assume a reduced form with a low or a high tone, as indicated in Table~\ref{tab:Abdoulaye:1} (the forms \textit{shi} vs. \textit{ya} for the 3\textsuperscript{rd} person masculine singular are free variants). Besides the regular 1\textsuperscript{st}, 2\textsuperscript{nd}, and 3\textsuperscript{rd} person, the subject pronouns also have an impersonal form, with usages similar to French \textit{on}, and for which there are no corresponding independent or direct object forms, as indicated. Since the subject pronouns are often morphologically fused with the tense/aspect markers, they are generally obligatory, whether or not a noun subject is specified in the clause.

However, possessive pronouns are the pronouns most relevant for the structure of the reflexive markers, in particular the adnominal ‘Noun-of-Pronoun’ possessive constructions, which can have both a possessive and a reflexive meaning with the noun \textit{kâi} ‘head, self’, as seen in le~\ref{tab:Abdoulaye:2} for the Katsinanci dialect.

\begin{table}[ht]
    \begin{tabularx}{0.95\textwidth}{p{3.5cm}p{4.5cm}p{3cm}}
    \lsptoprule
   Full ‘Noun-of-Noun’ & Full ‘Noun-of-Pronoun’ & Reduced ‘Noun-of-Pronoun’\\
   \hline 
    & kâi naa-shì/naa-sà  ‘his head’ (lit. ‘head that.of.\textsc{m-3sg.m}’) & kâ-n-shì/kâ-n-sà ‘his head, himself’\\ 
   \cmidrule{3-3}
  kâi na Abdù 
  
  ‘head that.of.\textsc{m} Abdu’ & & kâi-na-s ‘his head’\\
   \cmidrule{2-3}
   & kâi naa-yà ‘his head’ & kâ-n-yà ‘his head’\\
   \cmidrule{3-3}
   & & kâi-nâ-i ‘his head’\\
   \cmidrule{1-3}
   kâi na Maaɍìyaa ‘head that.of.M Maria’ & kâi naa-tà ‘her head’ (lit. ‘head that.of.M-3SG.F’) & kâ-n-tà ‘her head, herself’\\

	


%\multirow{4}{*}{\parbox{3.5cm}{kâi na Abdù \\‘head that.of.\textsc{m} Abdu’}} & \multirow{2}{*}{\parbox{4.5cm}{kâi naa-shì/naa-sà  ‘his head’ \\ (lit. ‘head that.of.\textsc{m-3sg.m}’)}} & {kâ-n-shì/kâ-n-sà ‘his head, himself’}\\
 %   &  & kâi-na-s ‘his head’\\
 %    \multirow{2}{*}{kâi naa-yà ‘his head’} & kâ-n-yà ‘his head’\\
 %  & &  kâi-nâ-i ‘his head’\\
 %%‘her head’ (lit. ‘head that.of.\textsc{m-3sg.f}’} & kâ-n-tà ‘her head, herself’\\
  % kâi na Maaɍìyaa ‘head that.of.M Maria’ & \tabincell{l}{kâi naa-tà ‘her head’\\(lit. ‘head that.of.\textsc{m-3sg.f}’)} & kâ-n-tà ‘her head, herself’\\
    \lspbottomrule
    \end{tabularx}
    \caption{Attributive possessive constructions in Hausa (3\textsuperscript{rd} person singular, Katsinanci dialect)}
    \label{tab:Abdoulaye:2}
\end{table}

To better show the structure of the possessive constructions in Hausa, the first column of Table~\ref{tab:Abdoulaye:2} gives the full ‘Noun-of-Noun’ constructions, where a masculine singular possessee noun (\textit{kâi} ‘head’) combines with a masculine and a feminine possessor noun (\textit{Abdù} and \textit{Maaɍìyaa}, respectively). In this column, the nouns are syntactically linked by a pronoun that refers and agrees in gender and number with the possessee noun \textit{kâi} (with a feminine possessee noun, the linking pronoun would be \textit{ta} ‘that.of.\textsc{f}’, as in \textit{mootàa} \textit{ta} \textit{Abdù} ‘the car of Abdu’, lit. ‘car that.of.F Abdu’; all plural possessee nouns use the pronoun \textit{na}; also, the ‘Noun-of-Noun’ constructions have reduced versions \textit{kâ-n} \textit{Abdù} ‘head of Abdu’/\textit{mootà-ɍ} \textit{Abdù} ‘car of Abdu’ which do not concern us here). In the second column, the noun \textit{Abdù} is replaced with a possessive pronoun, either \textit{shì}/\textit{sà} or \textit{yà} ‘\textsc{sg.m}’(cf. Table~\ref{tab:Abdoulaye:1}). In the full ‘Noun-of-Pronoun’ constructions of the second column, a possessive pronoun replaces the possessive noun (lit. ‘head of him/her’). These constructions are reduced in the third column in two ways: If the linking pronoun is reduced (\textit{na} > \textit{\nobreakdash-n}), then the derived form is ambiguous between a possessive and a reflexive form, as indicated. If, on the contrary, it is the possessive pronoun that is reduced (\textit{shì/sà} > \textit{\nobreakdash-s}) then only the possessive meaning is possible. When the variant \textit{yà} is used, as seen in the second row of the second column, again for many speakers, the resulting reduced forms do not have a reflexive use in Katsinanci dialect, no matter the reduction pattern followed (the western dialects, which only have the \textit{kâinâi} form, also use it as reflexive pronoun; see \citealt[74]{Caron1991}; see also the discussion in \sectref{sec:Abdoulaye:7}). With the 3\textsuperscript{rd} person feminine singular pronoun \textit{tà} (in the last row of Table~\ref{tab:Abdoulaye:2}), only the linking pronoun reduction is possible and the form is ambiguous between a possessive and a reflexive form. It may be noted that the reduced forms are more frequent than the full forms.

The reflexive forms in Table~\ref{tab:Abdoulaye:2} are clearly “Head” reflexives in \citeauthor{Faltz1985}'s (\citeyear{Faltz1985}:~32f, 44) typology, given their composite nature incorporating a head noun, a linking pronoun, and a possessive pronoun. Nonetheless, they will be referred to as “reflexive pronouns”, following a usage now established in Hausa literature (see also \citealt[74]{Caron1991}, \citealt[522]{Newman2000}; \citealt[413]{Jaggar2001}; but see \citealt[117]{Wolff1993} for a different label). Following a recent proposal (\citealt{Will2019}, see also \citealt[117]{Wolff1993}), I assume that the meaning of \textit{kâi} as ‘self’ (instead of ‘head’) is the meaning relevant to the reflexive pronouns (see the discussion in \sectref{sec:Abdoulaye:7}). Also, to simplify the data presentation, the reflexive pronouns will be glossed globally as ‘\textsc{refl}’ plus the person features (e.g., \textit{kânshì} ‘\textsc{refl.3sg.m}’, instead of \textit{kâ-n-shì} ‘self-of.\textsc{m-3sg.m}’). Finally, although Table~\ref{tab:Abdoulaye:2} focuses on the 3rd person, the pronouns for all persons in Table~\ref{tab:Abdoulaye:1} have corresponding reflexive pronouns, as we will see in the data throughout the chapter. The next section looks at the subject/object coreference. 

\section{Subject and direct object coreference} \label{sec:Abdoulaye:3}

In conformity with the general tendencies (see \citealt[16]{Haspelmath2020a} and references therein), sentences in Hausa with coreferring subject and direct object require -- with a few exceptions -- a distinctive reflexive marking. The following subsections present the basic uses of the reflexive pronouns, the contrast between exact and inclusive coreference, the contrast between extroverted and introverted verbs, and the contrast between body-part and whole-body actions.

\subsection{Basic uses in subject-object coreference}\label{sec:Abdoulaye:3.1}

Nearly all transitive verbs in Hausa require the reflexive form of the direct object when it is coreferential with the subject. This is illustrated in (\ref{ex:Abdoulaye:2}):


\ea%2
    \label{ex:Abdoulaye:2}
    \ea\label{ex:Abdoulaye:2a}
        \gll Taa  yàbi  kântà.\\
     \textsc{3sg.f.cpl}  praise  \textsc{refl.3sg.f}\\
        \glt `She praised herself.’
        
    \ex \label{ex:Abdoulaye:2b}
    \gll Ta-nàa  yàbo-n  kântà.\\
    \textsc{3sg.f-ipfv}   praise-of.\textsc{m}  \textsc{refl.3sg.f}\\
    \glt`She is praising herself.’
    
    \ex \label{ex:Abdoulaye:2c}
    \gll Mutàanê-n  sun  kashè  kânsù.\\
    people-\textsc{def}  \textsc{3pl.cpl}  kill  \textsc{refl.3pl}\\
     \glt `The men killed themselves.’
     
    \ex \label{ex:Abdoulaye:2d} 
    \gll Yaa  reenà  kânshì.\\
    \textsc{3sg.m.cpl}  belittle  \textsc{refl.3sg.m}\\
    \glt `He lost confidence in himself/renounced his ambitions.’
    
    \ex \label{ex:Abdoulaye:2e}
    \gll  Naa  ga  kâinaa  cikin  maduubii.\\
    \textsc{1sg.cpl}  see  \textsc{refl.1sg}  in  mirror\\
    \glt  `I saw myself in the mirror.’
    \z
\z 

The sentences in (\ref{ex:Abdoulaye:2}) illustrate basic direct object structures. Notably, most Hausa researchers consider that \textit{kântà} in the imperfective sentence (\ref{ex:Abdoulaye:2b}), where it appears formally as the “possessor” of the verbal noun \textit{yàboo} ‘praising’, is the sentence direct object (it can be focused or questioned like the object of the basic verb \textit{yàbi} ‘praise’ in \REF{ex:Abdoulaye:2a}, but unlike true adnominal possessive nouns like \textit{Abdù} in \textit{gidan} \textit{Abdù} ‘the house of Abdu’). Except for the verb \textit{ga/gan/ganii} ‘see’ in (\ref{ex:Abdoulaye:2e}), the reflexive pronouns in sentences (\ref{ex:Abdoulaye:2}) are obligatory. In sentence (\ref{ex:Abdoulaye:2c}), like in its English equivalent, the men could have killed themselves deliberately or by accident, separately or together (mutuality would require the reciprocal marking \textit{juunaa} ‘each other’). When a non-reflexive pronoun is used as direct object, then a disjoint reference interpretation is obligatory. This is illustrated in (\ref{ex:Abdoulaye:3}):


\ea%3
    \label{ex:Abdoulaye:3}
    \ea  \label{ex:Abdoulaye:3a}
    \gll Taa\textsubscript{1}  yàbee  tà\textsubscript{2}\\
    \textsc{3sg.f.cpl}  praise  \textsc{3sg.f}\\
   \glt  `She praised her.’
   
    \ex  \label{ex:Abdoulaye:3b}
   \gll Mutàanê-n\textsubscript{1}  sun  kashèe sù\textsubscript{2}\\
    people-\textsc{def}  \textsc{3pl.cpl}  kill  \textsc{3pl}\\
    \glt `The men killed them.’
    \z
\z 
         
Sentences (\ref{ex:Abdoulaye:3a}--\ref{ex:Abdoulaye:3b}) correspond to sentences (\ref{ex:Abdoulaye:2a}) and (\ref{ex:Abdoulaye:2c}), respectively. One may note that the reflexive pronoun, being morphosyntactically a noun, behaves like regular nouns in triggering the pre-nominal form of the verb (hence the contrast between \textit{yàbi} and \textit{yàbee} ‘praise’; see \citealt{Newman2000}:~627 for a complete description). Beside typical direct objects, the reflexive pronouns also occur in atypical direct object positions, such as in double object constructions, or as object of complex predicates, as seen in (\ref{ex:Abdoulaye:4}--\ref{ex:Abdoulaye:5}):

\ea%4
    \label{ex:Abdoulaye:4}
    \ea  \label{ex:Abdoulaye:4a}
    \gll  Taa  hanà  kântà  kwaanaa.\\
    \textsc{3sg.f.cpl}  deny  \textsc{refl.3sg.f}  sleep\\
    \glt  `She prevented herself from sleeping.’
    
    \ex \label{ex:Abdoulaye:4b}
    \gll  Yaa  biyaa  kânshì  Nairàa  goomà.\\
    \textsc{3sg.m.cpl}  pay  \textsc{refl.3sg.m}  Naira  ten\\
    \glt `Ali payed himself 10 Nairas.’
    \z
\z 


\ea%5
    \label{ex:Abdoulaye:5}
    \ea \label{ex:Abdoulaye:5a}
   \gll Abdù  yaa  mayaɍ\_dà  kânshì  waawaa.\\
    Abdu  \textsc{3sg.m.cpl}  return.\textsc{caus}  \textsc{refl.3sg.m}  idiot\\
   \glt `Abdu turned himself into an idiot.’
    \ex \label{ex:Abdoulaye:5b}
    \gll Abdù  yaa  maidà  kânshì  waawaa.\\
    Abdu  \textsc{3sg.m.cpl}  return.\textsc{caus}  \textsc{refl.3sg.m}  idiot\\
    \glt `Abdu turned himself into an idiot.’
    \z
\z 


In sentences (\ref{ex:Abdoulaye:4a}--\ref{ex:Abdoulaye:4b}), the reflexive pronouns are dative/deprivative arguments (\textit{hanà} basically means ‘deny’) and such arguments, when present, are the true direct objects of the verbs, not the theme arguments, which are placed away from the verb. Example \REF{ex:Abdoulaye:5a} illustrates a complex causative predicate, made up of the basic verb \textit{mayà} ‘replace, repeat’ and the particle \textit{dà} in a close-knit syntax. The two parts can in fact merge into one word, as shown in the equivalent sentence (\ref{ex:Abdoulaye:5b}).

As reported in \citet[524]{Newman2000}, a reflexive pronoun can alternate with a coreferential non-reflexive pronoun in direct object position with verbs he characterized as “mental/sensation” verbs. This is illustrated in (\ref{ex:Abdoulaye:6})--(\ref{ex:Abdoulaye:7}):


\ea%6
    \label{ex:Abdoulaye:6}
    \ea \label{ex:Abdoulaye:6a}
    \gll Naa  ganee  nì  cikin  maduubii.\\
     \textsc{1sg.cpl}  see  \textsc{1sg}  in  mirror\\
    \glt  `I saw myself in the mirror.’
    \ex  \label{ex:Abdoulaye:6b}
    \gll  Naa  ga  kâinaa  cikin  maduubii.\\
    \textsc{1sg.cpl}  see  \textsc{refl.1sg}  in  mirror\\
    \glt `I saw myself in the mirror.’
    \z
\z 



\ea%7
    \label{ex:Abdoulaye:7}
    \ea   \label{ex:Abdoulaye:7a}
    \gll Sai  Bàlki\textsubscript{1}  ta  gan  tà\textsubscript{1/2}  cikin  fîm.\\
    The  Balki  \textsc{3sg.f.rp}  see  \textsc{3sg.f}  in  film\\
     \glt `Then/suddenly, Balki saw herself in the movie.’ (cf. Sai Bàlki ta ga kântà cikin fîm.)
    \ex \label{ex:Abdoulaye:7b}
   \gll  Yâara\textsubscript{1}  sun  jii  sù\textsubscript{1/2}  cikin  ɍeediyòo.\\
    children  \textsc{3pl.cpl}  hear  \textsc{3pl}  in  radio\\
    \glt `The children heard themselves on the radio.’ (cf. Yâara sun ji kânsù cikin ɍeediyòo.)
    \z
\z 


In (\ref{ex:Abdoulaye:6a})--(\ref{ex:Abdoulaye:6b}), in the 1\textsuperscript{st} person, a non-reflexive pronoun can alternate with a reflexive pronoun with the same interpretation. For the 3\textsuperscript{rd} person in (\ref{ex:Abdoulaye:7a})--(\ref{ex:Abdoulaye:7b}), a non-reflexive pronoun can refer to the subject or to some other participant, giving rise to a disjoint reference interpretation. The alternative sentences given with reflexive pronouns are naturally unambiguous. There are, however, some strong restrictions on the alternation. For example, \citet[524]{Newman2000} lists 13 verbs allowing the alternation. Secondly, subject-coreference with a non-reflexive pronoun is more acceptable in the 1\textsuperscript{st} and 2\textsuperscript{nd} person than in the 3\textsuperscript{rd} person. For example, in Katsinanci dialect, the coreferential 3\textsuperscript{rd} person non-reflexive pronoun is restricted to about six verbs: \textit{ganii} ‘see’, \textit{jii} ‘hear, feel’, \textit{soo} ‘want’, \textit{sàamu} ‘find (oneself in a situation)’, \textit{gaanèe} ‘recognize’, and \textit{san} ‘be aware (of one’s own inclinations)’. Also, as hinted at in \citet[524]{Newman2000}, the subject-coreferential 3\textsuperscript{rd} person pronoun is also restricted to the Completive (with an anterior value) and the perfective aspect. This is illustrated in (\ref{ex:Abdoulaye:8}): 


\ea%8
    \label{ex:Abdoulaye:8}
    \ea  \label{ex:Abdoulaye:8a}
    \gll I-nàa  jîi-naa  ɗàazu  à  cikin  ɍeediyòo.\\
    \textsc{1sg-ipfv}  hear-of.\textsc{m.1sg}  moment  at  in  radio\\
    \glt `I was hearing myself a while ago on the radio.’
    \ex \label{ex:Abdoulaye:8b}
    \gll Su\textsubscript{1}{}-nàa  jî-n-sù\textsubscript{*1/2}  ɗàazu  à  cikin  ɍeediyòo.\\
    \textsc{3pl-ipfv}  hear-of.\textsc{m-3pl}  moment  at  in  radio\\
    \glt `They were hearing them a while ago on the radio.’
    \z
\z 

Examples (\ref{ex:Abdoulaye:8}), in the imperfective aspect, show a contrast between the 1\textsuperscript{st} person in (\ref{ex:Abdoulaye:8a}), where a subject-coreferring non-reflexive pronoun is possible, and the 3\textsuperscript{rd} person in (\ref{ex:Abdoulaye:8b}), where a disjoint reference interpretation of the pronoun is obligatory. These restrictions are in accordance with the general tendency whereby the 3\textsuperscript{rd} person requires the reflexive marking more than the 1\textsuperscript{st} and 2\textsuperscript{nd} person (for a discussion see \citealt[43]{Haspelmath2008} and references cited there).\footnote{The intransitive motion verbs \textit{jee} ‘go’ and \textit{zoo} ‘come’ can immediately be followed by a pronoun agreeing with the subject, a pronoun known as the Chadic “intransitive copy pronoun” (the pronoun is more common in other Chadic languages; e.g., \textit{sun} \textit{jee} \textit{sù} \textit{makaɍantaa}, lit. ‘they went they to school’, see \citealt[407]{Jaggar2001}; \citealt[479]{Newman2000} and references cited there). In another variant of the phenomenon, a possessive pronoun agreeing with the subject is adjoined to nominalized intransitive motion and stance verbs (e.g., \textit{yaa} \textit{koomàawa-ɍ{}-shì makaɍantaa}, lit. ‘he.CPL returning-of-him [i.e., he returned] to school’). Reflexive pronouns are not possible in both cases.}


\subsection{Contrast between exact and inclusive coreference} \label{sec:Abdoulaye:3.2}

As reported in \citet[524]{Newman2000}, Hausa marks the contrast between exact coreference, e.g., between a singular subject and an agreeing singular reflexive pronoun, and inclusive coreference between a singular subject and a plural reflexive pronoun. This is illustrated in (\ref{ex:Abdoulaye:9}):

\ea%9
    \label{ex:Abdoulaye:9}
    \ea   \label{ex:Abdoulaye:9a}
    \gll  Màccê-n\textsubscript{1}  taa  yàbi  kânsù\textsubscript{1+x} \\
    woman-\textsc{def}  \textsc{3sg.f.cpl}  praise  \textsc{refl.3pl}\\
    \glt `The woman praised herself and the others in her group.’
    \ex  \label{ex:Abdoulaye:9b}
    \gll Yaa\textsubscript{1}  kaarè  kânsù\textsubscript{1+x}  dàgà  muugù-n  zàrgii.\\
    \textsc{3sg.m.cpl}  protect  \textsc{refl.3pl}  from  serious-of.\textsc{m}  charge\\
    \glt `He defended himself and the others in his group against a serious charge.’
    \z
\z 
         
Beside the direct object position, \citet[524]{Newman2000} shows that the inclusive reflexive pronoun is also possible in the applied object position (see \sectref{sec:Abdoulaye:4.1} below).

\subsection{Contrast between extroverted and introverted verbs} \label{sec:Abdoulaye:3.3}

Reflexive marking in Hausa is apparently sensitive to the contrast between extroverted and introverted verbs (on this contrast see \citealt[44]{Haspelmath2008} and references cited there). With the extroverted verbs, defined as verbs expressing socially antagonistic actions, such as in Hausa \textit{cìiji} ‘bite’, \textit{hàlbi} ‘shoot’, etc., reflexive marking is obligatory in case of coreference. This is illustrated in (\ref{ex:Abdoulaye:10}):

\ea%10
    \label{ex:Abdoulaye:10}
    \ea \label{ex:Abdoulaye:10a}
    \gll Kàree  yaa  cìiji  kânshì.\\
    dog  \textsc{3sg.m.cpl}  bite  \textsc{refl.3sg.m}\\
    \glt `The dog bit itself.’
    \ex \label{ex:Abdoulaye:10b}
   \gll Yaarinyàa  taa  tsàni  kântà.\\
    girl  \textsc{3sg.f.cpl}  hate  \textsc{refl.3sg.f}\\
    \glt `The girl hates herself.’
    \ex \label{ex:Abdoulaye:10c}
    \gll Ɗan\_sìyaasàa  yaa  sòoki  kânshì.\\
    politician  \textsc{3sg.m.cpl}  criticize  \textsc{refl.3sg.m}\\
    \glt `The politician criticized himself.’
    \ex \label{ex:Abdoulaye:10d}
    \gll Soojà  yaa  hàlbi  kânshì.\\
    soldier  \textsc{3sg.m.cpl}  shoot  \textsc{refl.3sg.m}\\
    \glt `The soldier shot himself.’
    \z
\z
  

Beside the obligatory reflexive marking in all sentences (\ref{ex:Abdoulaye:10}), one can also note that extroverted sentences can have a simple ‘Subject + Verb + Object’ structure. By contrast, introverted verbs, defined as verbs expressing body-care actions and the like, may not appear in a simple ‘Subject + Verb + Object’ structure in their autopathic use. This is illustrated in (\ref{ex:Abdoulaye:11}): 


\ea%11
    \label{ex:Abdoulaye:11}
    \ea \label{ex:Abdoulaye:11a}
    \gll Yaaròo  ya-nàa  [yi-n]  wankaa.\\
    boy  \textsc{3sg.m-ipfv}  do-of.\textsc{m}  wash\\
    \glt `The boy was washing himself.’
    \ex \label{ex:Abdoulaye:11b}
     \gll Yaarinyàa  taa  yi  wankaa.\\
    girl  \textsc{3sg.f.cpl}  do  wash\\
    \glt `The girl washed.’
    \ex \label{ex:Abdoulaye:11c}
    \gll Yaa  yi  askìi.\\
    \textsc{3sg.m.cpl}  do  haircut\\
    \glt `He had a haircut (at the barber).’ Or: ‘He did a haircut (to himself).’
    \ex \label{ex:Abdoulaye:11d}
     \gll    Abdù  yaa  sâa  kaayaa.\\
    Abdu  \textsc{3sg.m.cpl}  put.on  clothes\\
    \glt `Abdu got dressed (dressed himself).’
    \ex \label{ex:Abdoulaye:11e}
     \gll Abdù  yaa  shiryàa.\\
    Abdu  \textsc{3sg.m.cpl}  prepare\\
    \glt `Abdu got ready.’
    \z
\z
    


Sentence (\ref{ex:Abdoulaye:11a}) is in the imperfective aspect, but the predicate \textit{wankaa} ‘wash, bathe, shower’ is more like an action noun that is the direct object of an understood generic verb \textit{yi} ‘do’ (see \citealt[281]{Newman2000}; \citealt[171]{Jaggar2001}). Indeed, the underlying \textit{yi} ‘do’ verb is obligatory when the sentence is in the Completive, as seen in (11b\nobreakdash-c) (in fact even in the imperfective, \textit{yi} is acceptable in the negative, e.g. \textit{bâi} \textit{yîn} \textit{wankaa} ‘he doesn’t wash’ or if \textit{wankaa} is modified, e.g., \textit{mun} \textit{iskè} \textit{yanàa} \textit{yî\nobreakdash-n} \textit{wani} \textit{irìn} \textit{wankaa} ‘we find him washing himself in a peculiar way’). In (\ref{ex:Abdoulaye:11d}) the sentence does have the structure ‘Subject + Verb + Object’ but the object is not coreferential with the subject. Finally in (\ref{ex:Abdoulaye:11e}) the sentence is intransitive. In all cases, a reflexive pronoun is not possible. It is possible however to express the introverted action with a reflexive pronoun in the applied object position, as seen in the following (for more on the applied object, see \sectref{sec:Abdoulaye:4.1}):

\ea%12
    \label{ex:Abdoulaye:12}
    \ea \label{ex:Abdoulaye:12a}
    \gll  Yaaròo  ya-nàa  mà  kânshì  wankaa. \\
    child  \textsc{3sg.m-ipfv}  \textsc{appl}  \textsc{refl.3sg.m}  wash\\
    \glt `The boy is washing by himself/on his own.’ (= Yaaròo yanàa wankaa dà kânshì)
    \ex \label{ex:Abdoulaye:12b}
    \gll Yaa  yi  mà  kânshì  askìi. \\
    \textsc{3sg.m.cpl}  do  \textsc{appl}  \textsc{refl.3sg.m}  haircut\\
    \glt `He did a haircut by himself.’ (=  Yaa yi askìi dà kânshì)
    \z
\z

  

Sentences (\ref{ex:Abdoulaye:12}) are used in contexts where it is assumed that the subject referent ordinarily cannot carry out the action but, as it happens, they did (for example a child may be too young to perform the action alone). These sentences, as indicated, are semantically equivalent to the ‘by himself’ emphatic sentences discussed later in \sectref{sec:Abdoulaye:6.1}, but formally they involve a bona fide reflexive pronoun in a verbal argument position, as we will see in \sectref{sec:Abdoulaye:4.1} To summarize, it can be said that overall Hausa clearly marks the contrast between extroverted and introverted verbs, and that only the former regularly require the reflexive pronoun in autopathic contexts. 

\subsection{Contrast between body-part and whole-body actions} \label{sec:Abdoulaye:3.4}

Actions on specified body-parts are expressed in Hausa in a simple ‘Subject + Verb + Object’  structure, as seen in (\ref{ex:Abdoulaye:13}). 

\ea%13
    \label{ex:Abdoulaye:13}
    \ea \label{ex:Abdoulaye:13a}
    \gll Yaa  askè  geemèe/  geemè-n-shì.\\
    \textsc{3sg.m.cpl}  shave  beard  beard-of.\textsc{m-3sg.m}\\
    \glt `He shaved (himself).’ Or: ‘He had his beard shaved (at the barber).’
    \ex \label{ex:Abdoulaye:13b}
    \gll Yaa  wankè  kâi/  kâ-n-shì.\\
    \textsc{3sg.m.cpl}  wash  head/  head-of.\textsc{m-3sg.m}\\
     \glt `He cleaned his head.’
    \ex \label{ex:Abdoulaye:13c}
    \gll Yaa  wankè  jìkii/  jìki-n-shì.\\
    \textsc{3sg.m.cpl}  wash  body  body-of.\textsc{m-3sg.m}\\
     \glt `He did a quick toilet.’ (Lit. ‘he cleaned his body’)
    \ex \label{ex:Abdoulaye:13d}
    \gll Yaa  shaacè  kâi/  kâ-n-shì.\\
    \textsc{3sg.m.cpl}  comb  head/  head-of.\textsc{m-3sg.m}\\
     \glt `He combed his head [hair].’
     \z
\z 


In sentences (\ref{ex:Abdoulaye:13}), simple verbs are followed by their direct objects expressing a body-part. There is hence a clear contrast with whole-body autopathic actions, which are expressed with the verb \textit{yi} ‘do’ plus a nominal (a verbal or an action noun) specifying the action, as seen in (\ref{ex:Abdoulaye:11})--(\ref{ex:Abdoulaye:12}) above (one may consider sentence (\ref{ex:Abdoulaye:11c}) to describe an action viewed holistically although it concerns the head only, in contrast to sentence (\ref{ex:Abdoulaye:13a}) with a specified body-part \textit{geemèe} ‘beard’). A possessive pronoun referring to the subject can be adjoined to the body-part noun in sentences (\ref{ex:Abdoulaye:13}), as indicated, although this is wholly unnecessary in normal contexts. One may note that even with the possessive \textit{kânshì} ‘his head’, sentences (\ref{ex:Abdoulaye:13b}) and (\ref{ex:Abdoulaye:13d}) are not really ambiguous, i.e., they do not have the reflexive meaning ‘he washed himself’ or ‘he combed himself’, respectively.\footnote{Sentence \ref{ex:Abdoulaye:13b}, with \textit{kânshì}, can take the reflexive meaning only in the context of a ceremonial cleansing. For example, in a marriage, a groom is ceremonially “washed” normally by female relatives (see \textit{sun} \textit{wankè} \textit{angòo} ‘they washed/cleansed the groom’). But a groom can also choose to retire aside and throw the ceremonial water on himself and, in that case, sentence (\ref{ex:Abdoulaye:13b}) with \textit{kânshì} ‘himself’ can be used to describe the situation. (\ref{ex:Abdoulaye:13b}), still with \textit{kânshì}, can also be used in the sense ‘he cleared himself (of some accusations).’}  Sentence (\ref{ex:Abdoulaye:13c}) illustrates an expression \textit{wankè} \textit{jìkii} ‘have a quick toilet’ which, despite using the noun \textit{jìkii} ‘body’, in fact refers to the cleaning of the limbs and face. Similarly, in sentence (\ref{ex:Abdoulaye:13d}) the hair is combed.

To conclude this section, one can say that in Hausa the use of a reflexive pronoun is obligatory for a direct object coreferential with the subject, except with a few mental/sensation verbs. Hausa also does not allow a reflexive pronoun in subject function.

\section{Coreference between the subject and various semantic roles} \label{sec:Abdoulaye:4}

Beside the direct object position, reflexive pronouns can also appear in positions not directly governed by the main verb. This section reviews the applied nominal position, the possessive NP, and the objects of various prepositions. The section also looks at long distance coreference cases.

\subsection{Recipients and other \textit{mà/wà}{}-marked applied nominals}\label{sec:Abdoulaye:4.1}

The applied nominal is the direct object of the applicative marker \textit{mà/wà}, a free particle that stands in a close-knit syntactic relation with the verb (see \citealt{Tuller1984}, \citealt{Abdoulaye1996}, \citealt{Newman2000}:~280). The applied object assumes a variety of semantic roles, chiefly the recipient role, but also the benefactive, malefactive, locative, and possessor roles, and other minor unspecified roles (most of these roles also have their proper, i.e., non-applied, morphosyntax, as discussed later in this section). Applied nominals that are coreferential with the subject are most naturally expressed as reflexive pronouns, as seen in (\ref{ex:Abdoulaye:14}):

\ea%14
    \label{ex:Abdoulaye:14}
    \ea \label{ex:Abdoulaye:14a}
    \gll John  yaa  bàa  (wà)  kânshì  shaawaɍàa. \\
    John  \textsc{3sg.m.cpl}  give  \textsc{appl}  \textsc{refl.3sg.m}  advice\\
    \glt `John advised himself/changed his mind.’
    \ex \label{ex:Abdoulaye:14b}
    \gll Sun  aikoo  mà  kânsù  wàsiiƙàa.\\
    \textsc{3pl.cpl}  send  \textsc{appl}  \textsc{refl.3pl}  letter\\
     \glt `They sent a letter to themselves.’
    \ex \label{ex:Abdoulaye:14c}
    \gll Yaarinyàa  taa  dafàa  mà  kântà  àbinci.\\
    girl  \textsc{3sg.f.cpl}  cook  \textsc{appl}  \textsc{refl.3sg.f}  food\\
     \glt `The girl cooked for herself.’
    \ex \label{ex:Abdoulaye:14d}
    \gll Yaa  zoo  yaa  ganaɍ  mà  kânshì  àlamàɍî-n.\\
    \textsc{3sg.m.cpl}  come  \textsc{3sg.m.cpl}  see  \textsc{appl}  \textsc{refl.3sg.m}  situation{}-\textsc{def}\\
     \glt `He came and saw the situation for himself.’
     \z
\z 
    



Sentences (\ref{ex:Abdoulaye:14a}--\ref{ex:Abdoulaye:14c}) illustrate recipient and benefactive nominals expressed as reflexive pronouns following the applied marker \textit{mà/wà} (the applied marker is normally omitted with the verb \textit{bâa} ‘give’, as seen in \ref{ex:Abdoulaye:14a}). Sentence (\ref{ex:Abdoulaye:14d}) shows that a mental/sensation verb, \textit{ganii} ‘see’, requires a reflexive applied object pronoun under subject coreference (by contrast, we have seen in the discussion of (\ref{ex:Abdoulaye:6}--\ref{ex:Abdoulaye:7}) that mental/sensation verbs can allow a non-reflexive subject-coreferential direct object pronoun). When the non-reflexive pronoun is used in the applied object position, then a disjoint reference reading is normally obligatory, as seen next in (\ref{ex:Abdoulaye:15}), unless there is a partial coreference between a singular subject and a plural applied object pronoun, as illustrated in (\ref{ex:Abdoulaye:16}):

\ea%15
    \label{ex:Abdoulaye:15}
    \ea \label{ex:Abdoulaye:15a}
    \gll John\textsubscript{1} yaa  baa  shì\textsubscript{*1/2} shaawaɍàa.\\
    John  \textsc{3sg.m.cpl}  give  \textsc{3sg.m}  advice\\
    \glt `John advised him.’
    \ex \label{ex:Abdoulaye:15b}
   \gll Sun\textsubscript{1} aikoo  mà-sù\textsubscript{*1/2} wàsiiƙàa.\\
    \textsc{3pl.cpl}  send  \textsc{appl-3pl}  letter\\
    \glt `They sent them a letter.’
    \ex \label{ex:Abdoulaye:15c}
   \gll *Naa  jaawoo  ma-nì  wàhalàa.\\
    \textsc{1sg.cpl}  draw  \textsc{appl-1sg}  troubles\\
    \glt `I invited troubles on myself.’
    \z
\z
 
       
\ea%16
    \label{ex:Abdoulaye:16}
   \ea  \label{ex:Abdoulaye:16a}
   \gll  Naa\textsubscript{1}  \textit{bâa}  \textit{kânmù}\textsubscript{1+x}\textit{/}  \textit{baa}  \textit{mù}\textsubscript{1+x}  wàhalàa.\\
    \textsc{1sg.cpl}  give  \textsc{refl.1pl}  give  \textsc{1pl}  troubles\\
    \glt `I (uselessly) tired us.’
    \ex  \label{ex:Abdoulaye:16b}
    \gll Kaa\textsubscript{1}  jaawoo  \textit{mà}  \textit{kânkù}\textsubscript{1+x}\textit{/}  \textit{ma-kù}\textsubscript{1+x}  wàhalàa.\\
    \textsc{2sg.m.cpl}  draw  \textsc{appl}  \textsc{refl.2pl}  \textsc{appl-2pl}  troubles\\
    \glt `You invited troubles on you and your associates.’
    \ex  \label{ex:Abdoulaye:16c}
   \gll Yaa\textsubscript{1} jaawoo  \textit{mà}  \textit{kânsù}\textsubscript{1+x}\textit{/}  \textit{ma-sù}\textsubscript{?1+x/2}\textit{\textsubscript{}  }wàhalàa.\\
    \textsc{3sg.m.cpl}  draw  \textsc{appl}  \textsc{refl.3pl}  \textsc{appl-3pl}  troubles\\
    \glt `He invited troubles on himself and his associates.’ OR: `He invited troubles on them.’
    \z
\z 
  
    
Sentences (\ref{ex:Abdoulaye:15a}--\ref{ex:Abdoulaye:15c}) show that a non-reflexive pronoun in the applied position, despite matching agreement features, cannot be coreferential with the subject. Sentence (\ref{ex:Abdoulaye:15c}) in particular shows that the non-reflexive pronoun is not possible even for the 1\textsuperscript{st} person (the same is true for the 2\textsuperscript{nd} person as well). But in plural pronoun constructions, as illustrated in (\ref{ex:Abdoulaye:16}a-b), the 1\textsuperscript{st} and 2\textsuperscript{nd} person may allow a non-reflexive subject-coreferential pronoun in the applied position, while for the 3\textsuperscript{rd} person the reflexive pronoun is strongly preferred by speakers, as seen in (\ref{ex:Abdoulaye:16c}).


\subsection{Possessive NPs} \label{sec:Abdoulaye:4.2}

When a possessive NP is coreferential with the subject, Hausa requires a simple possessive pronoun in basic, pragmatically neutral sentences, as illustrated in (\ref{ex:Abdoulaye:17}):

\ea%17
    \label{ex:Abdoulaye:17}
    \ea \label{ex:Abdoulaye:17a}
    \gll Taa\textsubscript{1} ɗàuki  laimà-ɍ{}-tà\textsubscript{1/2}.\\
    \textsc{3sg.f.cpl}  take  umbrella-of.\textsc{f-3sg.f}\\
    \glt `She took her umbrella.’
    \ex \label{ex:Abdoulaye:17b}
    \gll John\textsubscript{1} ya-nàa  kaɍàntà  littaafì-n-shì\textsubscript{1/2}.\\
    John  \textsc{3sg.m-ipfv}  read  book-of.\textsc{m-3sg.m}\\
    \glt `John is reading his book.’
    \ex \label{ex:Abdoulaye:17c}
    \gll  Maatâ-n\textsubscript{1} sun  shaarè  ɗaakì-n-sù\textsubscript{1/2}.\\
    women-\textsc{def}  \textsc{3pl.cpl}  sweep  room-of.\textsc{m-3pl}\\
    \glt `The women swept their rooms.’
    \z
\z
  
        
As shown in (\ref{ex:Abdoulaye:17}), the simple possessive pronoun can be coreferential with the subject or not. Nonetheless, and as \citet[525]{Newman2000} notes, the coreference between the subject and the possessive pronoun can also be expressed as a reflexive pronoun, but with a marked emphasis, as seen in (\ref{ex:Abdoulaye:18}):

\ea%18
    \label{ex:Abdoulaye:18}
    \ea \label{ex:Abdoulaye:18a}
    \gll  Sun  ginà  gida-n-sù.\\
    \textsc{3pl.cpl}  build  house-of.\textsc{m-3pl}\\
    \glt `They built their house.’
    \ex \label{ex:Abdoulaye:18b}
    \gll  Sun  ginà  \textit{gida-n}  \textit{kânsù}/  \textit{gidaa}  \textit{na}  \textit{kânsù/} gida-n-sù	na	kânsù.\\
    \textsc{3pl.cpl}  build  house-of.\textsc{m}  \textsc{refl.3pl}   \textsc{refl.3pl}  house-of.\textsc{m-3pl}  one.of.\textsc{m}  \textsc{refl.3pl}\\
    \glt `They built their own house.’
    \ex \label{ex:Abdoulaye:18c}
   \gll  {Ùbaa-naa}  {na}  {kâinaa!} {(cf. *ùba-n kâinaa/*ùbaa na kâinaa)}\\
    {father-of.\textsc{m.1sg}}  {one.of.\textsc{m}}  \textsc{refl.1sg} {}\\
    \glt `Hey you my dear [for me alone] “uncle”!’
    \z
\z

Sentence (\ref{ex:Abdoulaye:18a}), with a non-reflexive pronoun, has a pragmatically neutral interpretation, just like sentences (\ref{ex:Abdoulaye:17}). By contrast, sentence (\ref{ex:Abdoulaye:18b}) has a reflexive pronoun in a reduced, a full, or a double possessive construction. In all three options, sentence (\ref{ex:Abdoulaye:18b}) contrasts with sentence (\ref{ex:Abdoulaye:18a}) by being more emphatic and, naturally, the more profuse the formal means used, the greater the emphasis. Indeed in appropriate contexts, the emphasis can even imply an exclusive use by the possessor of the possessed object, beyond the state of possession itself. In particular, the double possessive appositional construction, i.e., the 3\textsuperscript{rd} option in (\ref{ex:Abdoulaye:18b}), is the one that mostly implies the exclusive use of the possessed object by the possessor. So, sentence (\ref{ex:Abdoulaye:18c}) expresses - jokingly – the exclusive use meaning and the shorter reflexive constructions cannot be used, as indicated (the expression is used to affectionately greet a familiar – but unrelated - senior person; the senior person greeted can in fact reply \textit{ɗìyaa-taa} \textit{ta} \textit{kâinaa} ‘my dear own “niece”, i.e., other kin relations can be used, but always between unrelated people). To summarize, Hausa likely does not have genuine reflexive adnominal possessives and sentence (\ref{ex:Abdoulaye:18b}) can be compared to English sentences with the emphatic possession marker \textit{own} (see \citealt{Haspelmath2008}:~51 for discussion).


\subsection{Locatives} \label{sec:Abdoulaye:4.3}

Hausa uses basic and derived prepositions to express static locative relations. The derived prepositions are generally homophonous with locational nouns that are formally heads of a possessive constructions taking as “possessor” the NP expressing the location ground (see \textit{baya\nobreakdash-n} \textit{iccèe} ‘behind the tree’, lit. ‘back-of.M tree’). Most of these possessive constructions have grammaticalized towards a prepositional phrase structure and no longer have the behavioral properties typical of true possessive constructions (see \citealt{Abdoulaye2018}:~48f). When the location ground NP is coreferential with the subject, a non-reflexive pronoun must be used. This is illustrated in (\ref{ex:Abdoulaye:19}):


\ea%19
    \label{ex:Abdoulaye:19}
    \ea \label{ex:Abdoulaye:19a}
    \gll Ta\textsubscript{1}  mayaɍ\_dà  yaaròo  baaya-n-tà\textsubscript{1}/  *baaya-n  kântà\textsubscript{1}.\\
    \textsc{3sg.f.rp}  return.\textsc{caus}  child  back-of.\textsc{m-3sg.f}  back-of.\textsc{m}  \textsc{refl.3sg.f}\\
    \glt `She moved the child behind her.’
    \ex \label{ex:Abdoulaye:19b}
    \gll Ka\textsubscript{1}{}-nàa\_dà  aikìi  gàba-n-kà\textsubscript{1} /  *gàba-n  kânkà\textsubscript{1}.\\
    \textsc{2sg.m}-have  work  front-of.\textsc{m-2sg.m} { } front-of.\textsc{m}  \textsc{refl.2sg.m}\\
    \glt `You have much work to do [in front of you].
    \z
\z


These sentences show that a locative ground NP coreferential with the subject cannot be a reflexive pronoun. There is hence a contrast between locative phrases based on the possessive construction and genuine possessive constructions which at least admit an emphatic reflexive pronoun optionally. The locative phrases based on the possessive constructions also contrast with locative phrases based on simple prepositions which, sometimes, allow a reflexive pronoun, as noted by \citet[522f]{Newman2000}. This is illustrated in (\ref{ex:Abdoulaye:20})--(\ref{ex:Abdoulaye:21}):


\ea%20
    \label{ex:Abdoulaye:20}
    \ea  \label{ex:Abdoulaye:20a}
    \gll  Ta\textsubscript{1} ga  wani  macìijii  kusa  \textit{gàree}  \textit{tà}\textsubscript{1/2}/  *\textit{gà}  \textit{kântà}\textsubscript{1}.\\
    \textsc{3sg.f.rp}  see  one  snake  near  on  \textsc{3sg.f}  on  \textsc{refl.3sg.f}\\
     \glt  `She saw a snake beside her/herself.’
    \ex  \label{ex:Abdoulaye:20b}
    \gll John\textsubscript{1} ya  ajè  littaafìi  neesà  dà  shì\textsubscript{1/2}/  *kânshì\textsubscript{1}.\\
    John  \textsc{3sg.m.rp}  put.down  book  away  to  \textsc{3sg.m}  \textsc{refl.3sg.m}\\
    \glt    `John put a book away from him.’
    \z
\z

        
\ea%21
    \label{ex:Abdoulaye:21}
    \ea \label{ex:Abdoulaye:21a}
    \gll Taa\textsubscript{1}  shaafà  fentìi  \textit{gàree}  \textit{tà\textsubscript{1/2}}/  \textit{gà}  \textit{kântà}\textsubscript{1}.\\
    \textsc{3sg.f.cpl}  rub  paint  on  \textsc{3sg.f}  on  \textsc{refl.3sg.f}\\
    \glt `She rubbed paint on her/herself.’
    \ex \label{ex:Abdoulaye:21b}
    \gll Sun\textsubscript{1}  jaawoo  bàɍgoo  bisà  suu\textsubscript{1/2}/  kânsù\textsubscript{2}.\\
    \textsc{3pl.cpl}  draw  blanket  on  \textsc{3pl}  \textsc{refl.3pl}\\
    \glt `They pulled the blanket over them/themselves.’
    \z
\z

        

In sentences (\ref{ex:Abdoulaye:20}--\ref{ex:Abdoulaye:21}), the particles \textit{gà} ‘on’ (\textit{gàree} before pronoun), \textit{dà} ‘with, and, to’ are basic prepositions (without an evident source). \textit{Bisà} ‘on, on top of’ is derived from the noun \textit{bisà} ‘top,  sky’ (see \textit{bisà-n-shì} ‘its top part’ or ‘on it’), but it can be used without possessive marking and behaves like basic prepositions. Sentences (\ref{ex:Abdoulaye:20}) require a non-reflexive pronoun even when subject-coreference is intended, as indicated by the ungrammaticality of a reflexive pronoun. This may be due to the fact that the sentences express a non-contact locative relation. Although this needs to be investigated more, one can see that in sentences (\ref{ex:Abdoulaye:21}), which express a contact location, a locative NP, which is coreferential with the subject, can be a reflexive or a non-reflexive pronoun. However, in sentences (\ref{ex:Abdoulaye:21}) a non-reflexive pronoun is still the most natural option.


\subsection{Benefactives with preposition \textit{don} ‘for’}\label{sec:Abdoulaye:4.4}

\sectref{sec:Abdoulaye:4.1} showed that benefactive NPs can be expressed as applied nominals. They can also be expressed as objects of the preposition \textit{don} ‘for, for the sake of’. Under subject-coreference, the benefactive argument is most naturally expressed as a reflexive pronoun, although the non-reflexive pronoun is also possible. This is illustrated in the following (see also \citealt[524f]{Newman2000}):


\ea%22
    \label{ex:Abdoulaye:22}
    \ea \label{ex:Abdoulaye:22a}
    \gll Taa\textsubscript{1} sàyi  littaafìi  don  kântà\textsubscript{1}/  ita\textsubscript{1/2}.\\
    \textsc{3sg.f.cpl}  buy  book  for  \textsc{refl.3sg.f}  \textsc{3sg.f}\\
    \glt `She bought a book for herself/for her.’
    \ex \label{ex:Abdoulaye:22b}
    \gll Yaaròo\textsubscript{1} yaa  dafà  àbinci  don  kânshì\textsubscript{1}/  shii\textsubscript{1/2}.\\
    boy  \textsc{3sg.m.cpl}  cook  food  for  \textsc{refl.3sg.m}  \textsc{3sg.m}\\
    \glt `The boy cooked food for himself/for him.’
    \ex \label{ex:Abdoulaye:22c}
    \gll Naa  ginà  gidaa  don  kâinaa/  nii.\\
    \textsc{1sg.cpl}  build  house  for  \textsc{refl.1sg}  \textsc{1sg}\\
    \glt `I built a house for myself/for me.’
    \ex \label{ex:Abdoulaye:22d}
    \gll (To)  don  kânkà!/  Don  kânshì!/  Don  kânsù!\\
    OK  for  \textsc{refl.2sg.m}  for  \textsc{refl.3sg.m}  for  \textsc{refl.3pl}\\
    \glt `OK, (that’s) your problem!/His problem!/Their problem!’
    \z
\z
  
    
In sentences (\ref{ex:Abdoulaye:22a}--\ref{ex:Abdoulaye:22c}) the reflexive pronoun is preferred, even for (\ref{ex:Abdoulaye:22c}) with a 1\textsuperscript{st} person pronoun. When a non-reflexive 3\textsuperscript{rd} person pronoun is used, it is naturally ambiguous between subject-coreference and disjoint reference, as indicated. Examples (\ref{ex:Abdoulaye:22d}) show that the benefactive phrase with the reflexive pronoun can be used as an idiomatic expression (which can be used by a speaker after hearing someone rejecting a sound advice). In this expression, the reflexive pronoun cannot be replaced with a non-reflexive pronoun (i.e., \textit{don} \textit{kuu} would mean ‘for you’, not ‘that’s your problem’).

\subsection{Instrumental, associative and other oblique NPs}\label{sec:Abdoulaye:4.5}


In \sectref{sec:Abdoulaye:3.1} (see discussion of sentence \ref{ex:Abdoulaye:4}) we saw that causative `Verb\nobreakdash-\textit{dà}’ constructions take true direct objects, which are expressed as reflexive pronouns in subject-coreference contexts. However, \textit{dà} is a multipurpose free particle which, in its basic functions, marks the comitative and the instrumental relations (it also marks ‘and’-conjunction, a function that does not concern us here). In these basic functions, \textit{dà}, like other oblique markers, can optionally take a reflexive complement. This is illustrated in (\ref{ex:Abdoulaye:23}):


\ea%23
    \label{ex:Abdoulaye:23}
    \ea \label{ex:Abdoulaye:23a}
    \gll   Naa  gamàa  da  nii/  kâina.\\
    \textsc{1sg.cpl}  include  with  \textsc{1sg}/  \textsc{refl.1sg}\\
    \glt `I included myself.’
    \ex \label{ex:Abdoulaye:23b}
    \gll  Balki\textsubscript{1}  taa  gamàa  dà  ita\textsubscript{1/2}/  kânta\textsubscript{1}.\\
    Balki  \textsc{3sg.f.cpl}  include  with  \textsc{3sg.f}/  \textsc{refl.3sg.f}\\
    \glt `Balki included her/herself.’
    \ex \label{ex:Abdoulaye:23c}
    \gll  Balki\textsubscript{1}  taa  yi  shaawaɍàa  gàme  dà  ita\textsubscript{1/2}/  kânta\textsubscript{1}.\\
    Balki  \textsc{3sg.f.cpl}  do  advice  about  with  \textsc{3sg.f}  \textsc{refl.3sg.f}\\
    \glt `Balki1 made a proposal concerning her/herself.’
    \z
\z 
    

    
It may be noted that in (\ref{ex:Abdoulaye:23a}--\ref{ex:Abdoulaye:23b}), the reflexive pronoun is the best option in case of subject-coreference. When a non-reflexive 3\textsuperscript{rd} person pronoun is used, as in (\ref{ex:Abdoulaye:23b}--\ref{ex:Abdoulaye:23c}), it can be coreferential with the subject or refer to another participant. It may also be noted that the reflexive pronouns in (\ref{ex:Abdoulaye:23}) are not emphatic pronouns and one must distinguish them from the adverbial self-intensifier constructions, which are also built with \textit{dà}{}-phrases (see \sectref{sec:Abdoulaye:6.1}).


\subsection{Long-distance coreference}\label{sec:Abdoulaye:4.6}

When a higher subject is coreferential with an NP in the lower clause, a non-reflexive pronoun is obligatorily used when the second NP is a subject, a direct object, an applied object, or a prepositional object. In fact, the only cases of long-distance reflexives concerns a position inside the adnominal possessive construction or a long-distance coreference mediated by an understood lower subject in a non-finite clause. This is illustrated in the following (sentence \ref{ex:Abdoulaye:25b} adapted from \citealt{Newman2000}:~523): 


\ea%24
    \label{ex:Abdoulaye:24}
    \ea \label{ex:Abdoulaye:24a}
    \gll Taa\textsubscript{1} azà  [(*kântà\textsubscript{1})  ta\textsubscript{1/2}{}-nàa\_dà  ìsàssun  kuɗii].\\
    \textsc{3sg.f.cpl}  think  \textsc{refl.3sg.f}  \textsc{3sg.f}-have  enough  money\\
    \glt `She thought that she had enough money.’
    \ex \label{ex:Abdoulaye:24b}
    \gll Yaa\textsubscript{1}  soo  Bintà\textsubscript{2}  tà  \textit{zàaɓee}  \textit{shì\textsubscript{1/3}}/  \textit{*zàaɓi}  \textit{kânshì}\textsubscript{1}/ zàaɓi	kântà\textsubscript{2}.\\
    \textsc{3sg.m.cpl}  want  B.  \textsc{3sg.f.sbj}  choose  \textsc{3sg.m}  choose  \textsc{refl.3sg.m}    choose  \textsc{refl.3sg.f}\\
    \glt `He wanted that Binta choose him/*himself/herself.’
    \z
\z



\ea%25
    \label{ex:Abdoulaye:25}
    \ea \label{ex:Abdoulaye:25a}
    \gll Yaa\textsubscript{1}  soo  Bintà\textsubscript{2}  tà  sàyi  hòoto-n  shì\textsubscript{1/3}/  kânshì\textsubscript{1}.\\
    \textsc{3sg.m.cpl}  want  B.  \textsc{3sg.f.sbjv}  buy  photo-of.\textsc{m}  \textsc{3sg.m}  \textsc{refl.3sg.m}\\
    \glt `Abdu wanted that Binta buy his picture/his own picture.’
    \ex \label{ex:Abdoulaye:25b}
    \gll Abdù\textsubscript{1}  yaa  tàmbàyi  Bintà\textsubscript{2}  [hanyà-ɍ  [kaarè  kânshì\textsubscript{1}/  kântà\textsubscript{2}]].\\
    Abdu  \textsc{3sg.m.cpl}  ask  B.  way-of.\textsc{f}  protect  \textsc{refl.3sg.m}  \textsc{refl.3sg.f}\\
    \glt `Abdu asked Binta how to protect himself/herself.’
    \ex \label{ex:Abdoulaye:25c}
    \gll Abdù\textsubscript{1}  yaa  tàmbàyi  Bintà\textsubscript{2}  [hanyà-ɍ  [kaarèe  shì\textsubscript{1/3}/  tà\textsubscript{2/3}]].\\
    Abdu  \textsc{3sg.m.cpl}  ask  B.  way-of.\textsc{f}  protect  \textsc{3sg.m}/  \textsc{3sg.f}\\
    \glt `Abdu asked Binta how to protect himself/herself/him/her.’
    \z
\z 
  
  

In sentences (\ref{ex:Abdoulaye:24a}-\ref{ex:Abdoulaye:24b}), the coreferential lower subject (pronoun \textit{ta}\nobreakdash- `3SG.F’) and direct object (pronoun \textit{shi} `3SG.M’), respectively, cannot be expressed as reflexive pronouns. By contrast, the coreferential adnominal possessive argument can be a reflexive pronoun but with an emphatic meaning, as seen in (\ref{ex:Abdoulaye:25a}). In sentence (\ref{ex:Abdoulaye:25b}), the main verb is followed by two object NPs. The second NP (in first brackets) contains a possessive construction with \textit{hanyàa} ‘way’ as head and an adnominal non-finite clause (inner brackets). The direct object of the non-finite clause, when coded as a reflexive pronoun, can refer to main subject (\textit{Abdù}) or the main direct object (\textit{Bintà}). In this case, the referent of the main subject or the main direct object would, respectively, be understood to be the agent of the verb \textit{kaarè} ‘protect’. When simple pronouns are used as direct objects of \textit{kaarè}, as seen in (\ref{ex:Abdoulaye:25c}), then these pronouns can refer to Abdu, Binta, or someone else. If the pronoun refers to Abdu, then Abdu cannot be the understood agent of verb \textit{kaarè}, and similarly with Binta. In other words, sentence (\ref{ex:Abdoulaye:25b}) may not illustrate a genuine long-distance coreference (see the discussion in \citealt[14, note 15]{Haspelmath2020a}).


\section{Coreference between non-subject arguments} \label{sec:Abdoulaye:5}

In Hausa, the coreference between non-subject arguments is most naturally expressed with non-reflexive pronouns or, alternatively, with a reflexive pronoun. The coreference relation can take place between a direct object, an applied object, or a prepositional object on the one hand, and an adnominal possessive pronoun or a prepositional object, on the other hand. This is illustrated in the following (see also \citealt[523]{Newman2000} for similar data):


\ea%26
    \label{ex:Abdoulaye:26}
    \ea \label{ex:Abdoulaye:26a}
    \gll Yaa\textsubscript{1} nuunàa  mà  Màaɍi\textsubscript{2} \textit{hòoto-n-tà}\textsubscript{2/3}/  \textit{hòoto-n}  \textit{kântà}\textsubscript{2}. \\
    \textsc{3sg.m.cpl}  show  \textsc{appl}  \textsc{m}.  photo-of.\textsc{m-3sg.f}  photo-of.\textsc{m}  \textsc{refl.3sg.f}\\
    \glt `He showed Mary her picture/a picture of herself (her own picture).’
    \ex \label{ex:Abdoulaye:26b}
    \gll  Muusaa\textsubscript{1}  yaa  yii  wà  Abdù\textsubscript{2}  zancee  gàme  dà  shii\textsubscript{1/2/3}/ kânshì\textsubscript{1/2}).\\
    Musa  \textsc{3sg.m.cpl}  do  \textsc{appl}  A.  talk  about  with  \textsc{3sg.m} \textsc{refl.3sg.m}\\
    \glt `Musa spoke with Abdu about himself.’
    \z
\z

        


Sentence (\ref{ex:Abdoulaye:26a}), with the reflexive pronoun \textit{kântà}, implies that the photo likely pictures Mary, whereas this reading is not obligatory with the non-reflexive pronoun \textit{tà}. In (\ref{ex:Abdoulaye:26b}), the (non-emphatic) reflexive pronoun \textit{kânshì} can only refer to either of the nouns, i.e. \textit{Muusaa} or \textit{Abdù.} The non-reflexive pronoun \textit{shii} can refer to either noun or a third understood participant. Sentence (\ref{ex:Abdoulaye:26b}) shows that Hausa reflexive pronouns are not exclusively subject-oriented.

\section{Self-intensifiers} \label{sec:Abdoulaye:6}

We have already seen in \sectref{sec:Abdoulaye:4.2} that adnominal possessive reflexive pronouns can put emphasis on the possessive relation (see \textit{mootàɍ} \textit{kânshì} ‘his own car’). \citet{Newman2000} discusses at length two other emphatic constructions in Hausa that are related to the reflexive constructions and which are referred to in typological studies as the adverbial and the adnominal self-intensifiers (see \citealt[43]{KoenigSiemund1999}). This section is largely based on Newman’s account, although I will use the general terminology. The section presents the two types of constructions, in turn.

\subsection{Adverbial self-intensifiers} \label{sec:Abdoulaye:6.1}

According to \citet[526]{Newman2000}, what he calls “pseudoemphatic” reflexives are prepositional phrases with the preposition \textit{dà} ‘with, and, to, etc.’ followed by an (apparent) reflexive pronoun which is coreferential with the sentence subject. Semantically, they emphasize the fact that the subject referent did an action or underwent a process on their own, by themselves. This is illustrated in (\ref{ex:Abdoulaye:27}--\ref{ex:Abdoulaye:28}):


\ea%27
    \label{ex:Abdoulaye:27}
    \ea \label{ex:Abdoulaye:27a}
    \gll Yâaraa  sun  koomàa  gidaa  dà  kâ-n-sù.\\
    children  \textsc{3pl.cpl}  return  home  with  self-of.\textsc{m-3pl}\\
    \glt `The children returned home by themselves.’
    \ex \label{ex:Abdoulaye:27b}
    \gll Wutaa  taa  mutù  dà  kâ-n-tà.\\
    fire  \textsc{3sg.f.cpl}  die  with  self-of.\textsc{m-3sg.f}\\
    \glt `The fired died out on its own.’
    \z
\z


\ea%28
    \label{ex:Abdoulaye:28}
    \ea \label{ex:Abdoulaye:28a}
   \gll  Yâaraa  dà  kâ-n-sù  su-kà  koomàa  gidaa.\\
    children  with  self-of.\textsc{M-3PL}  \textsc{3pl-rp}  return  home\\
    \glt `The children returned home all by themselves.’
    \ex \label{ex:Abdoulaye:28b}
   \gll  Yâaraa  sun  koomàa  gidaa  \textit{dà}  \textit{gudù}/  \textit{dà}  \textit{tàimako-n}  \textit{mutàanee}.\\
    children  \textsc{3pl.cpl}  return  home  with  running  with  help-of.\textsc{m}  people\\
    \glt `The children returned home running/with help from others.’
    \ex \label{ex:Abdoulaye:28c}
     \gll tàimako-n  kâi  (dà  kâi)\\
    help-of.\textsc{m}  self  with  self\\
    \glt `self-help (all by oneself)’
    \z
\z


 
\citet{Newman2000} calls the reflexive-like forms in (\ref{ex:Abdoulaye:27}) “pseudoemphatic” because he believes they are bona fide reflexive pronouns in an adjunct structural position and which are coreferential with the subject. He notes that they typically appear near or at the end of the sentence. He also notes that they can be focus-fronted, just like any other clause constituent, as seen in (\ref{ex:Abdoulaye:28a}). Furthermore, (\ref{ex:Abdoulaye:28b}) shows that they can alternate with manner phrases introduced with the same preposition \textit{dà} ‘with, and, to’. Nonetheless, it is clear that the reflexive pronouns in (\ref{ex:Abdoulaye:27}-\ref{ex:Abdoulaye:28}) signal emphasis and should be characterized accordingly. They are indeed used in contexts where a speaker believes the hearer does not expect the subject referent to be able to carry out the action on their own. Nonetheless, one may not consider them to be true reflexive pronouns. Indeed, example (\ref{ex:Abdoulaye:28c}) shows that \textit{kâi} meaning ‘self’ can appear without an adnominal possessive pronoun, i.e., a coreference with an antecedent noun is not required to mark the emphasis. These forms are very likely the Hausa instantiation of the adverbial self-intensifiers and can be glossed literally as ‘with self-of-pronoun’, marking more precisely the emphatic meaning ‘with (just) the self, all alone’ (see \citealt[44]{KoenigSiemund2000} who refer to this use of the intensifiers as the exclusive ‘alone’ use; for more on \textit{kâi} as ‘self’ see next section). Sentence (\ref{ex:Abdoulaye:28a}), without the intensifier, would have no implication on how the children returned home. \citet[529]{Newman2000} also notes that for an even greater emphasis, the intensifier can combine with true reflexive pronouns, as seen in (\ref{ex:Abdoulaye:29}):


\ea%29
    \label{ex:Abdoulaye:29}
    \ea \label{ex:Abdoulaye:29a}
    \gll Bintà  taa  zàrgi  kântà  dà  kâ-n-tà.\\
    Binta  \textsc{3sg.f.cpl}  accuse  \textsc{refl.3sg.f}  with  self-of.\textsc{m-3sg.f}\\
    \glt `Binta charged herself knowingly, deliberately.’
    \ex \label{ex:Abdoulaye:29b}
    \gll Sun  ƙaaràa  wà  kânsù  kuɗii  (suu)  dà  kâ-n-sù.\\
    \textsc{3pl.cpl}  augment  \textsc{appl}  \textsc{refl.3pl}  money  \textsc{3pl}  with  self-of.\textsc{m-3pl}\\
    \glt `They raised their pay all by themselves, deliberately.’
    \z
\z

       

Sentences (\ref{ex:Abdoulaye:29a}--\ref{ex:Abdoulaye:29b}) have, respectively, a direct object and an applied object reflexive pronoun combined with the emphatic \textit{dà\nobreakdash-}phrase, here underlining the deliberate aspect of the action. As \citet[527]{Newman2000} notes, an independent pronoun can optionally precede the \textit{dà\nobreakdash-}phrase, as seen in sentence (\ref{ex:Abdoulaye:29b}). In such cases, \citeauthor{Newman2000} proposes that the \textit{dà\nobreakdash-}phrase is not an independent sentence constituent but is simply adjoined to the pronoun. This construction then comes close to the second type of emphatic reflexive pronouns, which \citeauthor{Newman2000} also believes are adnominal adjunctions, and which are presented next.


\subsection{The adnominal self-intensifiers} \label{sec:Abdoulaye:6.2}

Indeed, according to \citet{Newman2000}, the genuine reflexive-like emphatic pronouns are not sentence-level constituents, that is, they do not fulfill a semantic or syntactic role in the clause. Instead, they always appear in apposition next to a noun or pronoun. Functionally, they seem to signal a scalar ‘even X’/‘X himself’ emphasis or contrast. This is illustrated in the following (see also \citealt[527]{Newman2000}):

\ea%30
    \label{ex:Abdoulaye:30}
    \ea \label{ex:Abdoulaye:30a}
    \gll Bellò  (shii)  kânshì  yaa  san  bâi\_dà  gaskiyaa.\\
    Bello  \textsc{3sg.m}  \textsc{emp.3sg.m}  \textsc{3sg.m.cpl}  know  \textsc{neg.3sg.m}.have  truth\\
    \glt `[Even] Bello himself knows he is wrong.’
    \ex  \label{ex:Abdoulaye:30b}
    \gll  Sun  ruusà  makaɍantâ-ɍ  (ita)  kântà.\\
    \textsc{3pl.cpl}  break.up  school-\textsc{def}  \textsc{3sg.f}  \textsc{emp.3sg.f}\\
    \glt `They destroyed the school itself.’
    \ex \label{ex:Abdoulaye:30c}
    \gll Ɗàalìbâ-n  duk  su-kà  gudù,  àmmaa  maalàmî-n  shii  kânshì  ya  tsayàa.\\
    students-\textsc{def}  all  \textsc{3pl-pf}  run  but  teacher-\textsc{def}  \textsc{3sg.m}  \textsc{emp.3sg.m} \textsc{3sg.m.rp}  stay\\
    \glt `The students all ran away, but the teacher himself stood.’
    \z
\z
 
        

In (\ref{ex:Abdoulaye:30a}--b), the self-intensifier follows the modified noun, with an optional (but preferred) pronoun between the two. The pronoun becomes obligatory if the modified noun is omitted or positioned after (or away from) the intensifier (e.g., \textit{shii} \textit{kânshì} ‘he himself’, \textit{shii} \textit{kânshì} \textit{Bellò} ‘Bello himself’). Consequently, one can easily formally distinguish the adverbial self-intensifier (see \sectref{sec:Abdoulaye:6.1}) from the adnominal self-intensifier, no matter their position in the sentence (see discussion of \ref{ex:Abdoulaye:31}--\ref{ex:Abdoulaye:32} below). Semantically, the adnominal self-intensifiers seem to primarily signal emphasis and, secondarily, contrast, but both in the background of a scalar context. For example, sentence (\ref{ex:Abdoulaye:30a}) expresses a clear scalar emphasis: i.e., adversaries and all other people, as expected, think Bello is wrong; however, and quite unexpectedly, Bello, too, knows he is wrong. As for sentence (\ref{ex:Abdoulaye:30b}), while it can be used in contexts where no other building was destroyed, it nonetheless supposes an understood scalar background, i.e., if a school can be destroyed, then other less important buildings might as well. This account is then similar to the one given in a number of studies, such as \citet{EdmondsonPlank1978, Primus1992, Kibrik1995}, as cited in \citet[47--48]{KoenigSiemund2000}, however, reject this type of account, citing as evidence English data on which sentence (\ref{ex:Abdoulaye:30c}) is modeled. They would argue that in (\ref{ex:Abdoulaye:30c}), it is well expected that the referent of the marked noun (\textit{maalàmîn} ‘the teacher’) is the one not afraid to face a danger. Nonetheless for Hausa, it can also be noted that sentence (\ref{ex:Abdoulaye:30c}), like sentences (\ref{ex:Abdoulaye:30a})--(\ref{ex:Abdoulaye:30b}), still has a scalar context: the marked noun refers to an entity situated at the higher end of a scale. The only difference is that sentence (\ref{ex:Abdoulaye:30c}) expresses a contrast (between the scaled entities ‘students’ and ‘teacher’; see also sentence (\ref{ex:Abdoulaye:32b}) below). That the adnominal self-intensifiers may express both emphasis and contrast should not be surprising, since in general focus studies, too, the same formal means can signal various pragmatic situations (such as when a cleft construction is claimed to signal new information focus, contrastive focus, and exhaustive listing focus). Nonetheless, this preliminary account may not extend to other languages like English, or even crosslinguistically, where the uses of the self-intensifiers are more diverse (see \citealt[224]{KoenigGast2006}) than it appears to be the case in Hausa (at least pending further data).


Adnominal self-intensifiers can be reinforced in a number of ways, for extra emphasis. They can also have idiomatic uses. This is illustrated in (\ref{ex:Abdoulaye:31})--(\ref{ex:Abdoulaye:32}):
\ea%31
    \label{ex:Abdoulaye:31}
    \ea \label{ex:Abdoulaye:31a}
    \gll  Bellò  shii  dà  kâ-n-shì  yaa  san  gaskiyaa.\\
    Bello  \textsc{3sg.m}  with  self-of.\textsc{m-3sg.m}  \textsc{3sg.m.cpl}  know  truth\\
    \glt `Bello, really he himself, knows the truth.’
    \ex \label{ex:Abdoulaye:31b}
    \gll Bello  shii  kân\_kânshì  yaa  san  gaskiyaa.\\
    Bello  \textsc{3sg.m}  \textsc{emp-emp.3sg.m}  \textsc{3sg.m.cpl}  know  truth\\
    \glt `Bello, really he himself, knows the truth.’
    \z
\z

   
\ea%32
    \label{ex:Abdoulaye:32}
    \ea \label{ex:Abdoulaye:32a}
    \gll Wâyyoo  mu(u)  kânmù!\\
    alas  \textsc{1pl}  \textsc{emp.1pl}\\
    \glt `Alas, poor us!’
    \ex \label{ex:Abdoulaye:32b}
    \gll  Kee  \textit{kânkì}/  \textit{dà}  \textit{kâ-n-kì}  zaa\_kì  kunnà  wutaa  à  nân!\\
     \textsc{2sg.f}  \textsc{emp.2sg.f}  with  self-of.\textsc{m-2sg.f}  \textsc{fut-2sg.f}  light  fire  at  here\\
    \glt `How come you [who should know better] would light a fire in this place!’
    \z
\z 



In (\ref{ex:Abdoulaye:31a}), the subject noun \textit{Bellò} is followed by a reinforced adnominal self-intensifier \textit{shii} \textit{dà} \textit{kânshì,} which clearly contains the adverbial intensifier \textit{dà} \textit{kânshì} (see \sectref{sec:Abdoulaye:6.1}). The pronoun \textit{shii} is obligatory hence, the noun \textit{Bellò} cannot be followed by just \textit{dà} \textit{kânshì.} Semantically, the modified noun in (\ref{ex:Abdoulaye:31a}) is emphasized, as indicated. Sentence (\ref{ex:Abdoulaye:31b}) shows that adnominal self-intensifiers can be partially repeated (or, more likely, reduplicated prefixally), for an even greater emphasis. The partial repetition/reduplication device seems not to be available to the adverbial self-intensifiers (in fact to no other reflexive or reflexive-like construction). I will follow \citet[527]{Newman2000} in separating out the two formal types of self-intensifiers and globally gloss the adnominal self-intensifiers as ‘EMP’, plus the person features (see also discussion of sentences \ref{ex:Abdoulaye:38} below). Nonetheless, as reported by other researchers (see \citealt[117]{Wolff1993}), it seems that speakers have come to make the two types of self-intensifiers overlap (see sentence \ref{ex:Abdoulaye:31a}, \ref{ex:Abdoulaye:32b}, but also sentence \ref{ex:Abdoulaye:38b} below with its double meaning). Sentences (\ref{ex:Abdoulaye:32}) show that adnominal self-intensifiers can partake in fixed or idiomatic expressions (sentences like \ref{ex:Abdoulaye:32b} are generally used for scolding, i.e., the referent of the pronoun \textit{kèe} ‘2SG.F’, in contrast to all other relevant people, should know that fire should not be lit at the place).


In conclusion, Hausa uses forms akin to reflexive pronouns as adverbial and adnominal intensifiers to mark, respectively, the ‘by himself’-action emphasis and the scalar ‘even X’/’X himself’ emphasis or contrast.

\section{The meanings of \textit{kâi} ‘head, self’}\label{sec:Abdoulaye:7}

In Hausa, as in many other languages in the area,\footnote{See for example \citet[39]{BernardWhite-Kaba1994} for Zarma.} the word for `head’ has many derived meanings, including: `intelligence’, `consciousness’, `mind’, `person’, and `self, oneself’ (see \citealt{Will2019} for a review). Indeed, in Hausa the noun \textit{kâi} ‘self, oneself’, independently from the reflexive pronouns in Table~3, can appear alone in many nominal compounds, semi-fixed verbal expressions, and even proverbs.\footnote{Some \textit{kâi}{}-based proverbs one can find in dictionaries and the internet are: \textit{iyà} \textit{ruwa} \textit{fit} \textit{dà} \textit{kâi} ‘saving oneself is the measure of one’s swimming skills’, lit. ‘swimming [is] saving self’ (a proverb used to mean one should first test oneself before claiming an expertise; a variant of which is: \textit{koowaa} \textit{ya} \textit{fid} \textit{dà} \textit{kâi} \textit{naa\nobreakdash-sà} \textit{shii} \textit{nèe} \textit{gwànii} ‘whoever saves himself is the expert’, using a full ‘self that.of.M\nobreakdash-3SG.M’ possessive construction.); \textit{yàbon} \textit{kâi} \textit{jaahilcìi} ‘bragging is shallowness’, lit. ‘praise of self [is] ignorance’; \textit{girman} \textit{kâi} \textit{rawànin} \textit{tsìyaa} ‘pride is destructive’, lit. ‘big-ness of self/head [is] turban of poverty’; \textit{anàa} \textit{ta} \textit{kâi} \textit{bâa} \textit{a} \textit{ta} \textit{kaayaa} ‘one should attend to the most urgent issue first’, lit. ‘while saving the self, one does not care about properties’. The proverbs usually shed the functional words, like copulas (see \citealt[164f]{Newman2000}), the light verb \textit{yi} ‘do’ (see \citealt[171]{Jaggar2001}, \citealt[281]{Newman2000}), or even reduce phonological material (cf. \textit{ruwa} above vs. the full form \textit{ruwaa} ‘water’).} Some of the \textit{kâi}{}-based compounds and idiomatic expressions are illustrated in (\ref{ex:Abdoulaye:33}).


\ea%33
    \label{ex:Abdoulaye:33}
    \ea \label{ex:Abdoulaye:33a}
    \gll àbu-n  kâi/  (àbù)  na  kâi\\
    thing-of.\textsc{m}  self  thing  one.of.\textsc{m}  self\\
    \glt `property, wealth, own item’
    \ex \label{ex:Abdoulaye:33b}
    \gll  kiishì-n  kâi\\
    jealousy-of.\textsc{m}  self\\
    \glt `self-protection’
    \ex \label{ex:Abdoulaye:33c}
    \gll sô-n  kâi\\
    loving-of.\textsc{m}  self\\
    \glt `selfishness’
    \ex \label{ex:Abdoulaye:33d}
    \gll yii  ta  kâi\\
    do  one.of.\textsc{f}  self\\
    \glt `save oneself’
    \z
\z
 
        


The expressions in (\ref{ex:Abdoulaye:33a})--(\ref{ex:Abdoulaye:33c}) are compound nouns which, like any noun, can be used independently from any previously mentioned referent (for example as subject in \textit{sôn} \textit{kâi} \textit{yaa} \textit{yi} \textit{yawàa} \textit{gidan} \textit{nàn} ‘there is too much selfishness in this house’, for the compound in (\ref{ex:Abdoulaye:33c}); for a crosslinguistic investigation of the reflexive compounds, see \citealt{Koenig2003}). Sentence (\ref{ex:Abdoulaye:33d}) presents an idiomatic expression. Compounds based on \textit{kâi} ‘self’, both with predictable or less predictable meanings, are numerous. Some frequent examples cited in the dictionaries are: \textit{ɓatàn} \textit{kâi} ‘confusion’, lit. ‘loss of self’; \textit{incìn} \textit{kâi} ‘independence, autonomy’; \textit{sanìn} \textit{ciiwòn} \textit{kâi} ‘self-care’, lit. ‘knowing of pain of self’ (cf. also \textit{ciiwòn} \textit{kâi} ‘headache’); \textit{girman} \textit{kâi} ‘pride, vanity’, lit. ‘big-ness of self’ (though this may also be  ‘big-ness of head’); \textit{jîn} \textit{kâi}, ‘pride, vanity’ lit. ‘feeling of self’; \textit{sâa} \textit{kâi} ‘volunteerism’, lit. ‘putting self’ (cf. \textit{aikìn} \textit{sâa} \textit{kâi} ‘voluntary work’); etc. These expressions and compounds can sometimes keep their idiomatic reading even when \textit{kâi} is adjoined a possessive pronoun (e.g., \textit{kâ-n-shì} ‘self-of-3SG.M’) referring to the sentence subject. This is illustrated in (\ref{ex:Abdoulaye:34})--(\ref{ex:Abdoulaye:35}):


\ea%34
    \label{ex:Abdoulaye:34}
    \ea \label{ex:Abdoulaye:34a}
    \gll Yaara  su-kà  yi  ta  kâ-n-sù.\\
    children  \textsc{3pl-rp}  do  one.of.\textsc{f}  self-of.\textsc{m-3pl}\\
    \glt `The children bolted away/escaped threat.’ OR `The children did their own [chair].’ (i.e., ‘they made one [chair] for themselves’)
    \ex \label{ex:Abdoulaye:34b}
    \gll Koo-waa  yà  yi  ta  kâ-n-shì!\\
    even-who  \textsc{3sg.m.sbjv}  do  one.of.\textsc{f}  self-of.\textsc{m-3sg.m}\\
    \glt `Every man for himself!’ (cf. Fr. \textit{sauve-qui-peut!}’); OR `May every one make his own [chair].’  ‘May every one follow his own way.’
    \z
\z

        

\ea%35
    \label{ex:Abdoulaye:35}
    \ea  \label{ex:Abdoulaye:35a}
    \gll  Abdù  yaa  nuunà  irì-n  [kiishì-n  kâ]-n-shì.\\
    Abdu  \textsc{3sg.m.cpl}  show  type-of.\textsc{m}  protection-of.\textsc{m}  self-of.\textsc{m-3sg.m}\\
    \glt `Abdu displayed his art of self-protection.’
    \ex \label{ex:Abdoulaye:35b}
    \gll  Abdù,  à  yi  kiishì-n  kâi/  *kâ-n-kà!\\
    Abdu  \textsc{imprs.sbjv}  do  protection-of.\textsc{m}  self/  self-of.\textsc{m-2sg.m}\\
    \glt `Abdu, you should protect yourself.’
    \ex \label{ex:Abdoulaye:35c}
    \gll   Abdù,  kà  yi  kiishì-n  kâ-n-kà!\\
    Abdu,  \textsc{2sg.m.sbjv}  do  protection-of.\textsc{m}  self-of.\textsc{m-2sg.m}\\
    \glt `Abdu, you should protect yourself.’
    \z
\z
 
   

Sentences (\ref{ex:Abdoulaye:34}) illustrate the expression \textit{yi} \textit{ta} \textit{kâi} ‘save self’ given in \REF{ex:Abdoulaye:33d}. In both sentences (\ref{ex:Abdoulaye:34a}--\ref{ex:Abdoulaye:34b}) the idiomatic meaning is still recoverable even though \textit{kâi} is adjoined a possessive pronoun referring to the subject. The sentences however are ambiguous, with possible true reflexive readings, as indicated.  Sentence (\ref{ex:Abdoulaye:35a}) shows that the compound \textit{kiishìn} \textit{kâi} ‘self-protection’, too, can take an adnominal possessive pronoun (see also \textit{irìn} \textit{[kiishìn} \textit{kâ]n} \textit{Abdù} ‘Abdu’s way in self-protection’, with an adnominal possessive noun). The compound structure is also clear in (\ref{ex:Abdoulaye:35b}) where an impersonal subject-pronoun occurs with a specified referent, yet the sentence cannot license an adnominal possessive pronoun. However, with a matching 2\textsuperscript{nd} person subject-pronoun, as in (\ref{ex:Abdoulaye:35c}), an adnominal possessive pronoun is possible and one gets a typical reflexive construction, no matter how one might analyze the sequence \textit{kiishì-n} \textit{kâ-n-kà} (as a compound ‘self-protection of you’, or as a reflexive pronoun ‘protection of yourself’). The typical reflexive reading is more easily available when the compound or fixed expression has a transparent meaning, as seen in the following case (examples adapted from \citealt[523]{Newman2000}):


\ea%36
    \label{ex:Abdoulaye:36}
    \ea \label{ex:Abdoulaye:36a}
    \gll Abdù  yaa  tàmbàyi  Bintà  hanyà-ɍ  kaarè  kâi.\\
    Abdu  \textsc{3sg.m.cpl}  ask  Binta  way-of.\textsc{f}  protect  self\\
    \glt `Abdu asked Binta about how to protect oneself [way of self-protection].’
    \ex \label{ex:Abdoulaye:36b}
    \gll  Abdù\textsubscript{} yaa  faɗàa  wà  Bintà  hanyà-ɍ  kaarè  kânshì/  kântà.\\
    Abdu  \textsc{3sg.m.cpl}  tell  \textsc{appl}  Binta  way-of.\textsc{f}  protect  \textsc{refl.3sg.m}  \textsc{refl.3sg.f}\\
    \glt `Abdu told Binta about how to protect himself/herself.’
    \z
\z
 
        

In (\ref{ex:Abdoulaye:36a}) with the bare expression \textit{kaarè} \textit{kai} ‘self-protection’, the person that needs to protect themselves can be Abdu, Balki, or some other person, while in (\ref{ex:Abdoulaye:36b}), with a reflexive pronoun, Abdu (with \textit{kânshì}) or Balki (with \textit{kântà}) are referred to by the reflexive pronoun, in a typical reflexive construction. Other semantically transparent \textit{kâi}{}-based compounds and expressions are: \textit{kaa\_dà} \textit{kâi} ‘falling all by oneself [self-defeat]’; \textit{kashè} \textit{kâi} ‘suicide’ (lit. ‘kill self’, cf. \textit{kisà\nobreakdash-n} \textit{kâi} ‘murder’, lit. ‘killing\nobreakdash-of head/person’); \textit{bìncìken} \textit{kâi} ‘self-exploration’; \textit{àmfàanin} \textit{kâi} ‘self-benefit’ (i.e., doing something for one’s own sake); \textit{tàimakon} \textit{kâi} ‘self-help’, etc. Some of these can be reinforced with the ‘by himself’ adverbial intensifiers seen in \sectref{sec:Abdoulaye:6.1}: \textit{bìncìken} \textit{kâi} \textit{dà} \textit{kâi} lit. ‘self-exploration by self’, \textit{tàimakon} \textit{kâi} \textit{dà} \textit{kâi} lit. ‘self-help by self’ (see also \citealt[523]{Newman2000}). As suggested already in \sectref{sec:Abdoulaye:6.1}, these reinforced compounds show that both \textit{dà} \textit{kâi} and \textit{dà} \textit{kânshì} can mark the ‘by himself’ emphasis. Finally, there is at least one case where \textit{kâi} ‘self’ appears embedded in typical reflexive constructions, i.e., when the plural form \textit{kaawunàa} ‘selves’ is used, as seen in the following (sentence \ref{ex:Abdoulaye:37a} from a radio broadcast and \ref{ex:Abdoulaye:37b} from \citealt[383]{Jaggar2001}; see also \citealt[45]{Abdoulaye2018}):


\ea%37
    \label{ex:Abdoulaye:37}
    \ea \label{ex:Abdoulaye:37a}
     \gll ...na  aamulàa  dà  tsaftàa  dà  kuma  kaarè  kaawunà-n-mù dàgà  cî-n  naamà-n  ɓeeràayee...\\
     one.of.\textsc{m}  practice  with  hygiene  and  also  protect  selves-of.\textsc{pl-1pl} from  eating-of.\textsc{m}  meat-of.\textsc{m}  rodents\\
     \glt `[appeals made to us] for practicing hygiene and protecting [restraining] ourselves from eating rodents...’
     \ex \label{ex:Abdoulaye:37b}
     \gll  Zaa\_mù  wankè  kaawunà-n-mù  dàgà  zàrgi-n  dà  a-kèe  ma-nà.\\
    \textsc{fut-1pl}  clear  selves-of.\textsc{pl-1pl}  from  charge{}-of.\textsc{m}  that  \textsc{imprs.ri}  \textsc{appl-1pl}\\
    \glt `We will clear ourselves of the accusation against us.’
    \ex \label{ex:Abdoulaye:37c}
     \gll Ɗaya  baayan  ɗaya,  su-kà  zwaagè  kaawunà-n-sù  dàgà  haɍakà-ɍ.\\
    one  after  one  \textsc{3pl-rp}  extract  selves-of.\textsc{pl-3pl}  from  matter-\textsc{def}\\
    \glt `One by one, they extracted themselves from the matter.’
    \z
\z
  
   

Sentences (\ref{ex:Abdoulaye:37}), with the plural form \textit{kaawunaà} ‘selves’, have a special semantics. Indeed, they tend to imply individualized actions by many people. This is clear in sentences (\ref{ex:Abdoulaye:37a}) and (\ref{ex:Abdoulaye:37c}), where it is understood that people performed the action separately and at various times. According to \citet[485]{Newman2000}, the building of the reflexive pronouns uses only the singular \textit{kâi} and this claim would be true if indeed it applies only to the reflexive pronouns that solely mark coreference between arguments, that is, without an added semantics or an emphasis. Indeed, if the regular reflexive pronoun \textit{kânmù} ‘ourselves’ (lit. ‘our-self’) is used in (\ref{ex:Abdoulaye:37a})--(\ref{ex:Abdoulaye:37b}), as is possible, then the sentences would not have the individualized actions reading.


Although most Hausa researchers assume that the reflexive pronouns are directly based on the meaning ‘head’ (see \citealt[74]{Caron1991},
;\citealt[529]{Newman2000}; \citealt[413]{Jaggar2001}; \citealt[147f]{Pawlak2014}; for a general proposal in this regard see \citealt[32f,109f]{Faltz1985}), a few sources have instead explicitly linked the reflexive pronouns with \textit{kâi} meaning ‘self’ (e.g., \citealt[117]{Wolff1993}; \citealt[161]{Will2019}). The data presented in this section show indeed that the meaning of ‘self’ may be relevant for an account of the development of the typical reflexive pronouns. Self-intensifier forms, too, are sometimes evoked as possible source of reflexive pronouns (see \citealt[44]{KoenigSiemund2000}; \citealt[105f]{Schladt2000}; and \citealt[22]{Haspelmath2020a} for discussions) and this proposal may be relevant for Hausa as well. We have seen in \sectref{sec:Abdoulaye:6} that Hausa has two types of self-intensifiers. There is some evidence in Katsinanci dialect that adnominal self-intensifiers are formally closer to typical reflexive pronouns than adverbial self-intensifiers. Indeed, adnominal self-intensifiers and reflexive pronouns tend to have less flexibility in their choice of the 3\textsuperscript{rd} person masculine singular pronoun variants, as given in Table~\ref{tab:Abdoulaye:2}, and so contrast with adverbial self-intensifiers and the \textit{kai} ‘self’ found in compounds and idiomatic expressions, as seen in (\ref{ex:Abdoulaye:38}):


\ea%38
    \label{ex:Abdoulaye:38}
    \ea  \label{ex:Abdoulaye:38a}
    \gll  Koo-waa  yà  yi  ta  kâ-n-shì/  kâ-n-yà/ kâi-nâ-i!\\
    even-who  \textsc{3sg.m.sbjv}  do  one.of.\textsc{f}  self-of.\textsc{m-3sg.m}  self-of.\textsc{m-3sg.m} self-of.\textsc{m-3sg.m}\\
    \glt `Every man for himself!’ (cf. sentence \ref{ex:Abdoulaye:34b} above)
    \ex  \label{ex:Abdoulaye:38b}
    \gll  Bello  yaa  jee  makaɍantâ-ɍ  dà  kâ-n-shì/  kâ-n-yà/ kâi-nâ-i.\\
    Bello  \textsc{3sg.m.cpl}  go  school-\textsc{def}  with  self-of.\textsc{m-3sg.m}  self-of.\textsc{m-3sg.m} self-of.\textsc{m-3sg.m}\\
    \glt`Bello went to the school by himself.’ (Also: ‘Bello himself went to the school.’)
    \ex  \label{ex:Abdoulaye:38c}
    \gll Bello  yaa  ga  kânshì/  ?kânyà/  ?kâinâi  cikin  maduubii.\\
    Bello  \textsc{3sg.m.cpl}  see  \textsc{refl.3sg.m}  \textsc{refl.3sg.m}  \textsc{refl.3sg.m}  in  mirror\\
    \glt `Bello saw himself in the mirror.’
    \ex  \label{ex:Abdoulaye:38d}
    \gll  Bello  shii  kânshì/  ?kânyà/  *kâinai  yaa  san  gaskiyaa.\\
    Bello  \textsc{3sg.m}  \textsc{emp.3sg.m}  \textsc{emp.3sg.m}  \textsc{emp.3sg.m}  \textsc{3sg.m.cpl}  know  truth\\
    \glt `Bello himself knows the truth.’
    \ex  \label{ex:Abdoulaye:38e}
     \gll Bello  shii  kân\_kânshì/  *kân\_kânyà/  *kân\_kâinai yaa  san  gaskiyaa.\\
    Bello  \textsc{3sg.m} \textsc{emp-emp.3sg.m}/  \textsc{emp-emp.3sg.m}  \textsc{emp-emp.3sg.m} \textsc{3sg.m.cpl}  know  truth\\
    \glt `Bello, really he himself, knows the truth.’
    \z
\z
      
  



As shown in Table~\ref{tab:Abdoulaye:2}, Katsinanci dialect has four reduced variants for the 3\textsuperscript{rd} person masculine singular possessive pronoun, three of which are relevant for our discussion here (the \textit{kâi-na-s} ‘his head’ variant is marginal even for typical possessive constructions). All speakers consulted agree without hesitation that the three variants are grammatical with \textit{kâi} ‘self’, as seen in (\ref{ex:Abdoulaye:38a}), and with the adverbial self-intensifiers, as seen in sentence (\ref{ex:Abdoulaye:38b}). This result, together with the fact that \textit{dà} \textit{kâi}, lit. ‘by self’, can alone mark emphasis (e.g., \textit{bìncìken} \textit{kâi} \textit{dà} \textit{kâi} lit. ‘self-exploration by self’), supports analyzing the ‘by himself’ emphatic constructions as having the literal comitative meaning ‘with (just) his self’, i.e. ‘alone’. By contrast, speakers are less firm in their judgments with the reflexive pronouns and the adnominal self-intensifiers. All speakers consulted immediately favor the form \textit{kânshì} for both constructions, as seen in (\ref{ex:Abdoulaye:38c})--(\ref{ex:Abdoulaye:38d}), respectively. Most consulted speakers tolerate \textit{kânyà} for both constructions. By contrast, \textit{kâinâi} is acceptable for the reflexive pronouns but is rejected by most speakers for the adnominal self-intensifiers. Finally, for all consulted speakers, in sentence (\ref{ex:Abdoulaye:38e}), the adnominal intensifier reinforced with partial repetition/reduplication (see sentence \ref{ex:Abdoulaye:31b} above) can only have the \textit{kânshì} form.


\section{Conclusion} \label{sec:Abdoulaye:8}

This contribution has shown that Hausa distinctively marks coreference between the subject and another NP in the same minimal clause using reflexive pronouns formally based on the possessive construction ‘\textit{kâi} + \nobreakdash-n + Pronoun’, lit. ‘self + of + Pronoun’, where the pronoun is coreferential with the clause subject (or sometimes with a preceding direct object or applied object). Subject-coreferential direct objects are almost always expressed as reflexive pronouns (with the exception of the direct objects of some mental and sensation verbs). Subject-coreferential applied objects are also always expressed as reflexive pronouns, except for the 1\textsuperscript{st} and 2\textsuperscript{nd} persons, where a non-reflexive pronoun is possible. Subject-coreferential locative NPs are always expressed as simple pronouns with prepositions derived from location nouns, but they can also be reflexive pronouns with simple, non-derived prepositions. Similarly, prepositional phrases with \textit{dà} ‘with, and’ basically accept simple pronouns,  but they also allow the reflexive pronouns, particularly in the 3\textsuperscript{rd} person. Subject-coreferential possessive NPs can optionally be expressed as reflexive pronouns but they then have a special ‘own’-emphasis on the possessive relation. The chapter also described three different constructions that are related to the typical reflexive constructions: compounds and semi-fixed expressions involving \textit{kâi} ‘self’, adverbial self-intensifiers marking the ‘by himself’ emphasis, and adnominal self-intensifiers marking the scalar ‘even X’/’X himself’ emphasis and contrast. These three constructions may be relevant for an account of the origin of the typical reflexive pronouns in Hausa.

\section*{Notes and abbreviations}

The data discussed in this paper are based on Katsinanci dialect. Katsinanci was the dialect of precolonial Katsina State, the territory of which today straddles the border between the Republic of Niger (towns of Maradi and Tessaoua) and the Federal Republic of Nigeria (town of Katsina). It is in a central position between the two main Hausa dialectal clusters, the western and the eastern dialects, but it shares more features with the western dialects (see \citealt[7]{Wolff1993}; \citealt[1]{Newman2000}). 

  The transcription in this chapter follows the Hausa orthography, with some changes. Long vowels are represented as double letters, low tone as grave accent and falling tone as circumflex accent. High tone is unmarked. The symbol `ɍ{}' represents an alveolar trill distinct from the flap `r'. Final ‘ɍ’ generally assimilates to the following consonant. Written `f' is pronounced [h] (or [hw] before [a]) in Katsinanci and other western dialects. 
  
  The abbreviations are:

\begin{tabularx}{.45\textwidth}[t]{lQ}
 1, 2, 3 & 1st, 2nd, 3rd person\\
 \textsc{appl} & applicative\\
 \textsc{cpl} & completive\\
 \textsc{def} & definiteness\\
 \textsc{emp} & emphasis\\
 \textsc{f} & feminine\\
 \textsc{fut} & future\\
 \textsc{imprs} & impersonal\\
 \textsc{ipfv} & imperfective\\
 \textsc{m} & masculine\\
 \textsc{neg} & negative\\
 \textsc{np} & noun phrase\\
 \textsc{pl} & plural\\
 \textsc{refl} & reflexive\\
 \textsc{ri} & relative imperfective\\
 \textsc{rp} & relative perfective\\
 \textsc{sg} & singular\\
 \textsc{sbjv} & subjunctive\\
\end{tabularx}
  

\begin{figure}[H]
     \centering
     \includegraphics[width=\textwidth]{figures/Hausa1.png}
     \caption{Hausa language primary area (from \citealt{Newman2000})}
     \label{fig:Abdoulaye:1}
\end{figure}

\begin{figure}[H]
     \centering
     \includegraphics[width=\textwidth]{figures/Hausa2.png}
     \caption{Hausa dialectal areas \textmd{(line = Niger/Nigeria border; from \citealt{Wolff1993})}}
     \label{fig:Abdoulaye:2}
\end{figure}
 
%%please move the includegraphics inside the {figure} environment
%%\includegraphics[width=\textwidth]{figures/Hausareflexivesabdoulaye20200526-img001.jpg}

 
%%please move the includegraphics inside the {figure} environment
%%\includegraphics[width=\textwidth]{figures/Hausareflexivesabdoulaye20200526-img002.jpg}


%\section*{Acknowledgements}


{\sloppy\printbibliography[heading=subbibliography,notkeyword=this]}
\end{document}
