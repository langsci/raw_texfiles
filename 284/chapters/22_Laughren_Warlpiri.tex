\documentclass[output=paper]{../langscibook}

\author{Mary Laughren\affiliation{University of Queensland}\orcid{}}

\title{Reflexive constructions in Warlpiri}

\abstract{Warlpiri is an Australian language which belongs to the Ngumpin-Yapa subgroup of Pama-Nyungan. Coreference between the subject and another argument of a finite clause – object or applicative – is marked by an anaphoric clitic in the auxiliary complex substituted for the person/number and case-marking clitic that would mark features of the corresponding non-subject argument disjoint in reference with the subject. Evidence that reflexive clauses with transitive verbs maintain their transitivity includes ergative case-marking of subject NP and the association of a part NP with the non-subject role. Formally similar pseudo-transitive reflexive clauses which express a change of state in a single argument are shown to be limited to situations in which the internal state of a being is altered by some external situation beyond that being's control. The role of the anaphor within complex NPs is compared with its role within the finite clause. Within a finite clause a strict coreference relation is limited to that between the subject and the non-subject role represented by the anaphor. Strict coreference between an argument of a matrix finite clause and an argument within a non-finite clause embedded within the finite clause is limited to the phonologically null subject of the non-finite clause. Given the lack of an anaphor in non-finite clauses, strict coreference between subject and object cannot be expressed. Where coreference is possible between an NP external to a non-finite clause and a pronoun internal to it, a disjoint reading is always available.}

\IfFileExists{../localcommands.tex}{
  \addbibresource{localbibliography.bib}
  \usepackage{langsci-optional}
\usepackage{langsci-gb4e}
\usepackage{langsci-lgr}

\usepackage{listings}
\lstset{basicstyle=\ttfamily,tabsize=2,breaklines=true}

%added by author
% \usepackage{tipa}
\usepackage{multirow}
\graphicspath{{figures/}}
\usepackage{langsci-branding}

  
\newcommand{\sent}{\enumsentence}
\newcommand{\sents}{\eenumsentence}
\let\citeasnoun\citet

\renewcommand{\lsCoverTitleFont}[1]{\sffamily\addfontfeatures{Scale=MatchUppercase}\fontsize{44pt}{16mm}\selectfont #1}
   
  %% hyphenation points for line breaks
%% Normally, automatic hyphenation in LaTeX is very good
%% If a word is mis-hyphenated, add it to this file
%%
%% add information to TeX file before \begin{document} with:
%% %% hyphenation points for line breaks
%% Normally, automatic hyphenation in LaTeX is very good
%% If a word is mis-hyphenated, add it to this file
%%
%% add information to TeX file before \begin{document} with:
%% %% hyphenation points for line breaks
%% Normally, automatic hyphenation in LaTeX is very good
%% If a word is mis-hyphenated, add it to this file
%%
%% add information to TeX file before \begin{document} with:
%% \include{localhyphenation}
\hyphenation{
affri-ca-te
affri-ca-tes
an-no-tated
com-ple-ments
com-po-si-tio-na-li-ty
non-com-po-si-tio-na-li-ty
Gon-zá-lez
out-side
Ri-chárd
se-man-tics
STREU-SLE
Tie-de-mann
}
\hyphenation{
affri-ca-te
affri-ca-tes
an-no-tated
com-ple-ments
com-po-si-tio-na-li-ty
non-com-po-si-tio-na-li-ty
Gon-zá-lez
out-side
Ri-chárd
se-man-tics
STREU-SLE
Tie-de-mann
}
\hyphenation{
affri-ca-te
affri-ca-tes
an-no-tated
com-ple-ments
com-po-si-tio-na-li-ty
non-com-po-si-tio-na-li-ty
Gon-zá-lez
out-side
Ri-chárd
se-man-tics
STREU-SLE
Tie-de-mann
} 
  \togglepaper[1]%%chapternumber
}{}

\begin{document}
\maketitle 
%\shorttitlerunninghead{}%%use this for an abridged title in the page headers

\section{Introduction}\label{sec:laughren:1}

\subsection{Classification, distribution and dialects of Warlpiri}\label{sec:laughren:1.1}

Warlpiri is one of the best documented Australian languages and is also one of the few indigenous Australian languages still being acquired by children.\footnote{See the bibliography of work on Warlpiri created and maintained by David Nash at \url{http://www.anu.edu.au/linguistics/nash/aust/wlp/wlp-lx-ref.html}.} According to the 2016 Australian Census, 2,276 people indicated that they spoke Warlpiri at home.

The traditional Warlpiri heartland is in the Tanami Desert in Australia's Northern Territory (see \figref{fig:laughren:1}). The language most closely related to Warlpiri is Warlmanpa, which borders Warlpiri on the northeast. These two languages, Warlpiri and Warlmanpa, form the Yapa branch of the larger Ngumpin-Yapa group of languages traditionally spoken on territory extending north and west from Warlpiri and Warlmanpa land (\citealt{McConvellLaughren2004, MeakinsEtAlForth}). Warlpiri is the southernmost member of the Ngumpin-Yapa group of languages which belong to the large Pama-Nyungan language family spoken over most of the continent. However, along their northern border, Ngumpin languages are in direct contact with non-Pama-Nyungan languages (see \citealt{Dixon2002, EvansEd2003}).

\begin{figure}
%\includegraphics[width=\textwidth]{figures/LaughrenWarlpiri-img001.jpg}
\caption{Warlpiri and Ngumpin-Yapa languages in relation to non-Pama-Nyungan Mirndi languages  \citealt{BrendaThornley2017}).}
\label{fig:laughren:1}
\end{figure}

Several dialects of Warlpiri can be identified reflecting to some extent the languages of neighbouring communities. These dialects vary mainly in vocabulary, with some minor phonological and grammatical differences which do not impact on the phenomena described herein.\footnote{The language described herein is traditional Warlpiri which is quite distinct from the variety dubbed "Light Warlpiri", which has developed among younger speakers at Lajamanu \citealt{OShannessy2005, OShannessy2006, OShannessy2013}.}

Some elementary facts about Warlpiri clause structure and morphology and the role and form of NPs are presented in \sectref{sec:laughren:1.2}. The remaining sections are organised as follows: \sectref{sec:laughren:2} sets out the pronominal system and the relationship between "free" pronouns and the markers of person and/or number in the auxiliary complex and the role of the anaphoric non-subject enclitic central to the reflexive construction in finite clauses; \sectref{sec:laughren:3} explores a range of relationships within finite reflexive clauses, while NP-internal reflexive relationships are discussed in \sectref{sec:laughren:4}; constraints on coreference within non-finite clauses are briefly discussed in \sectref{sec:laughren:5}; some uses of formal reflexive structure in clauses with monadic predicates are touched on in \sectref{sec:laughren:6}; the Warlpiri reflexive construction is placed in a wider Australian context in \sectref{sec:laughren:7}.

\subsection{General remarks on Warlpiri morphosyntax}\label{sec:laughren:1.2}

\subsubsection{Clause structure}\label{sec:laughren:1.2.1}

Warlpiri finite and non-finite clauses are quite distinct in structure. The core constituents of a finite clause are the auxiliary and the predicate; the latter may be verbal \REF{ex:laughren:1a} or nominal \REF{ex:laughren:1b}.\footnote{Unless otherwise indicated, the source of the Warlpiri sentences is the author's field notes and recordings.}

\ea%1
    \label{ex:laughren:1}
\ea
\label{ex:laughren:1a}
\gll \textbf{Nya-ngu}=lu=jana.\\
    see-\textsc{pst=pl.s=3pl}\\
\glt `They saw them.'\footnote{Suffix boundaries are marked by "-" and enclitic boundaries by "=". The subject clitic is glossed "\textsc{s}", but the grammatical function of the non-subject clitic is not glossed as it may mark the person and/or number and case features of several non-subject grammatical functions (discussed in some detail in \sectref{sec:laughren:2}).}

\ex
\label{ex:laughren:1b}
\gll \textbf{Jaja}=rna=ngku.\\
    grandmother=\textsc{1s=2}\\
\glt `I am grandmother to you.'
\z
\z

In verbal clauses, the auxiliary consists of one of two base morphemes: \textit{ka} `present indicative' only with the non-past verb form as in \REF{ex:laughren:3a} and \REF{ex:laughren:3b}, and \textit{lpa} `imperfective' with past and irrealis verbal inflections as in \REF{ex:laughren:11a} and \REF{ex:laughren:11b}. Each of these contrasts with its absence as in \REF{ex:laughren:1a}. A ``zero" base is compatible with all verbal inflections and is obligatory with a non-verbal predicate as in \REF{ex:laughren:1b}. The tense, aspect and mood (TAM) properties of a clause are marked by the auxiliary base in conjunction with verbal inflectional suffixes, and with a complementiser to which, if present, the auxiliary encliticises, as in \REF{ex:laughren:5a}.

Subject and non-subject enclitic pronouns attach to the auxiliary base \citep{Hale1973}. The auxiliary complex typically follows the clause-initial phrase, which may be of any category. Where the auxiliary base is phonologically null, as in \REF{ex:laughren:1a} and \REF{ex:laughren:1b}, the pronominal enclitics attach directly to the clause initial phrase.\footnote{Third person subjects are unmarked. The clitic =\textit{lu} in \REF{ex:laughren:1a} marks a plural subject and may combine with first or second person subject clitics. The dual clitic =\textit{pala} works the same way. While some clitics (such as =\textit{jana} in \REF{ex:laughren:1a}) mark both person and number features as well as case, others only mark features of either person or number – not both. Where only person features are marked, the absence of accompanying number marking typically defaults to a singular reading. \citet{Hale1973} provides a full account of Warlpiri person and number marking clitics.}

In a clause with an overt complementiser, the auxiliary complex must encliticise to the complementiser; this combination may occupy initial or second position in the clause. The choice of clause initial phrase is mainly determined by discourse factors (see \citealt{MushinSimpson2008}; \citealt{Swartz1991}), although the presence of the negative complementiser \textit{kula} excludes the inflected verb from the clause initial position. In finite clauses with a nominal predicate, there is no auxiliary base, or complementiser, so the pronominal clitics attach to the clause initial phrase as in \REF{ex:laughren:1b} in which the combination of subject clitic \textit{=rna} and dative clitic \textit{=ngku} encliticise to the nominal kin predicate \textit{jaja} `mother's mother'. Clauses with a nominal predicate lack markers of tense, aspect or mood (TAM) features and have a present or aorist interpretation. To overtly express TAM values, a copula-like 'stance' verb must be added which converts the clause from a nominal one to a verbal one.\footnote{\textrm{For more detail on basic clause structure in Warlpiri see \citet{Hale1982}.}}

Non-finite clauses, like finite clauses with a nominal predicate, lack TAM markers and have no auxiliary base. They also lack enclitic pronouns, which has implications for the expression of coreference. In this respect, Warlpiri differs from Western Romance languages in which accusative and/or dative person and number marking clitics including an anaphoric clitic occur in both finite and non-finite clauses. Unlike a nominal, a bare infinitive verb cannot function as the main predicate of a finite clause; it must host a complementiser suffix which signals the relationship between the non-finite clause and other constituents of the matrix finite clause in which it is embedded (see \citealt{Hale1982}; \citealt{Laughren2017}; \citealt{Nash1986}; \citealt{SimpsonBresnan1983}; \citealt{Simpson1991}; \textit{inter alia}).

\subsubsection{Noun phrases}\label{sec:laughren:1.2.2}

Warlpiri noun phrases are case-marked. Case is marked by a suffix (or its absence) which is obligatory on the final word of a phrase, although other words in a phrase may also be case-marked. In finite clauses, NPs whose number and person features are encoded by the subject pronominal enclitic are either marked by the ergative grammatical case suffix, e.g. \textit{karnta-ngku} as in (2b-d), or they are unmarked, e.g. \textit{karnta} in \REF{ex:laughren:3b}, depending on the verb.\footnote{The unmarked subject or object NP is traditionally said to be in the absolutive (ABS) case. In glossing Warlpiri examples, I omit this feature since it is redundant.} In finite nominal clauses, the subject NP is always unmarked. Similarly, NPs whose number and/or person features are marked by the non-subject pronominal enclitic are either unmarked, e.g. \textit{wati} 'man' in (2b-d) or marked with dative case, e.g. \textit{wati-ki} in \REF{ex:laughren:3b}.\footnote{\citet{Legate2002} argues that the dative-marked object of verbs like \textit{wangka-mi} `speak, talk' is a "low" applicative internal to the inner VP like the unmarked object of verbs with an ergative subject although in a different relationship to the verb. This "low" object-like applicative contrasts with the "upper" applicative generated above the inner VP but inside the higher vP. \citet{Simpson1991} also distinguishes these grammatical functions within an LFG framework.} Features of Warlpiri syntax that have been widely discussed in the linguistic literature are the grammatical optionality of NPs corresponding to the predicate's arguments, and the relative lack of constraints on word and phrase order, especially within finite clauses (e.g. \citealt{Hale1983, Jelinek1984, Laughren2002, Legate2002, MushinSimpson2008, Nash1986, Simpson1991}; \textit{inter alia}).\footnote{See also \citet{Pensalfini2004} for relevant discussion.} These features are illustrated by the contrast between the \REF{ex:laughren:2a} and (2b-d) and between \REF{ex:laughren:3a} and \REF{ex:laughren:3b}, and in other examples herein.

In the sentences in \REF{ex:laughren:2} and \REF{ex:laughren:3} the subject and object NPs refer to distinct entities.

\ea%2
    \label{ex:laughren:2}

\ea
\label{ex:laughren:2a}
\gll Nya-ngu=lu=jana.\\
    see-\textsc{pst=pl.s=3pl}\\
\glt `They saw them.’

\ex
\label{ex:laughren:2b}
\gll Karnta-ngku=lu=jana  wati  nya-ngu.\\
    woman-\textsc{erg=pl.s=3pl}  man  see-\textsc{pst}\\

\ex
\label{ex:laughren:2c}
\gll Nya-ngu=lu=jana  wati  karnta-ngku.\\
    see-\textsc{pst=pl.s=3pl}  man  woman-\textsc{erg}\\

\ex
\label{ex:laughren:2d}
\gll Wati=li=jana  karnta-ngku  nya-ngu.\\
 man=\textsc{pl.s=3pl}  woman-\textsc{erg}  see-\textsc{pst}\\
\glt `The women saw the men.'
\z
\z

With ditransitive verbs such as \textit{yinyi} `give', it is typically the animate recipient whose person and/or number features are marked by the non-subject enclitic pronoun while a co-referential NP is marked dative case as in \REF{ex:laughren:4a}. However, where the theme argument has an animate referent, its features are marked by the enclitic non-subject pronoun, and an NP referring to it is unmarked. The recipient NP is no longer marked by dative case, but is expressed in an optional phrase headed by a semantic case, the allative, as in \REF{ex:laughren:4b}.\footnote{Suffixes such as the allative 'to, towards' which behave rather like the heads of prepositional or postpositional phrases will be referred to herein as 'semantic cases'. \citet{Nash1986} classes them as 'cases' which contrast with the 'grammatical cases' in his ARG[ument] category. While a phrase marked by a semantic case may be further marked by a grammatical case (dative or ergative) suffix, the converse is not possible. See \citet{Simpson1991} and \citet{Legate2008} for detailed analyses of case in Warlpiri.} The person/number features of this allative phrase are not marked by an enclitic pronoun.

\ea%3
    \label{ex:laughren:3}

\ea
\label{ex:laughren:3a}
\gll Wangka-mi  ka=lu=jana.   \\
    speak-\textsc{npst}  \textsc{prs.ind=pl.s=3pl}\\
\glt `They are speaking to them.'

\ex
\label{ex:laughren:3b}
\gll Wati-ki  ka=lu=jana  wangka-mi   karnta.\\
    man-\textsc{dat}  \textsc{prs.ind=pl.s=3pl}  speak-\textsc{npst}  woman.\\
\glt `The women are speaking to the men.'
\z
\z

\ea%4
\label{ex:laughren:4}

\ea
\label{ex:laughren:4a}
\gll Kuyu  kapu=\textbf{ju}=lu yi-nyi \textbf{ngaju-ku}.\\
    meat \textsc{fut=}\textbf{\textsc{1}}\textsc{=pl.s}  give-\textsc{npst}  \textbf{me\textsc{{}-dat}}\\
\glt `They will give me meat.'

\ex
\label{ex:laughren:4b}
\gll {Kapu=}\textbf{{ju}}{=lu} \textbf{{ngaju}  }{yapakari-}\textbf{{kirra/}}{*yapakari-}\textbf{ki} {yi-nyi.} \\
    \textsc{fut}=\textbf{\textsc{1}}=\textsc{pl.s}  \textbf{me}  other-\textbf{\textsc{allat/*dat}  }give-\textsc{npst}\\
\glt `They will give me up to another.' [betray] [Warlpiri Bible, Matthew 17.22]
\z
\z

The non-subject enclitic pronoun also marks the person and number features of an applicative argument, such as the benefactive arguments in \REF{ex:laughren:5a} and \REF{ex:laughren:5b}. If present, an NP coreferential with the non-subject clitic is also marked by dative case as exemplified by \textit{jirrima-kari-ki} in \REF{ex:laughren:5a}.\footnote{\citet{Simpson1991} dubbed this class of applicative "external object" while \citet{Legate2002} dubbed it "higher applicative" in contrast with "lower applicatives", i.e., Simpson's "dative objects". Warlpiri has an array of adverbial preverbs such as benefactive expressed by dialect variants \textit{kaji/ngayi} which specify how the dative-marked applicative argument's role is interpreted (see also \citealt{Hale1982} and \citealt{Nash1986}).}

\ea%5
    \label{ex:laughren:5}

\ea
\label{ex:laughren:5a}
\gll {Yinga=}\textbf{{palangu}}\textbf{{\textsubscript{i}}} jinta-kari-rli   yangka   kuyu \textbf{jirrima-kari-ki}\textbf{{\textsubscript{i}}} \\
    \textsc{comp=du}  one-other-\textsc{erg}  that  meat  two-other-\textsc{dat}\\

\gll    ngayi  paji-rni.\\
    \textsc{ben}  cut-\textsc{npst}\\
\glt `So that the other person cuts up that meat \textbf{for} \textbf{the} \textbf{other} \textbf{two}.'

\ex
\label{ex:laughren:5b}
\gll {Kapu=rna=}\textbf{ngku} kaji  panti-rni.\\
    \textsc{fut=1s=2}  \textsc{ben}  spear-\textsc{npst}\\
\glt `I will spear (it/him/her) \textbf{for} \textbf{you}.'
\z
\z

NPs whose number features are not marked by pronominal enclitics, i.e., which are neither subject, object or applicative arguments, are typically marked by a case suffix with a complementising function such as the purposive \textit{ngapa-ku} in \REF{ex:laughren:6a}, marked by dative case, or a semantic case suffix such as the allative in \REF{ex:laughren:6b} or elative in \REF{ex:laughren:6c}.\footnote{The purposive phrase in \REF{ex:laughren:6a} marked by the dative case suffix differs from a dative object or applicative phrase in not being construed with a non-subject auxiliary pronominal enclitic.}

\ea%6
    \label{ex:laughren:6}
\ea
\label{ex:laughren:6a}
\gll \textbf{Ngapa-ku}  ka=rna  ya-ni.\\
    water-\textsc{dat}  \textsc{prs.ind=1s} go-\textsc{npst}\\
  \glt `I am going \textbf{for} \textbf{water}.' (i.e., to get water)

\ex
\label{ex:laughren:6b}
\gll \textbf{{Ngapa-kurra}} ka=rna  ya-ni.\\
    water-\textsc{allat}  \textsc{prs.ind=1s}  go-\textsc{npst}\\
\glt `I am going \textbf{to/towards} \textbf{the} \textbf{water}.'

\ex
\label{ex:laughren:6c}
\gll \textbf{{Ngapa-ngurlu}} ka=rna  ya-ni.\\
    water-\textsc{elat}  \textsc{prs.ind=1s}  go-\textsc{npst}\\

\glt `I am going \textbf{from} \textbf{the} \textbf{water}.'
\z
\z


Warlpiri lacks an article category but has an extensive set of determiners which may constitute an NP or combine with other nominal words in a complex NP. Determiners host the same set of case suffixes as other nominals.

\section{Pronouns and anaphors}\label{sec:laughren:2}

\subsection{Pronouns}\label{sec:laughren:2.1}

Warlpiri has two sets of pronouns: bound pronouns (auxiliary enclitics) and free pronouns which are set out in \tabref{tab:laughren:1} (see \citealt{Hale1973}). The former are obligatory in finite clauses, while the free pronouns behave like NPs in that their presence is not obligatory, but is determined by discourse factors. While the case-marking of the bound pronouns is Nominative vs Accusative/Dative, the free pronouns follow the same case-marking pattern as that of NPs. The non-subject pronominal enclitic has the same form irrespective of whether it marks the number and/or person features of an unmarked or dative-marked NP, except for the third person singular which has a marked dative form, \textit{{}-rla}, which contrasts with the phonologically null nominative and accusative, and a distinct "double dative" (DD) form. The DD form is used mainly when there is both a dative "object" and an "applicative" argument marked by the dative case suffix, or where there is one of these and an overt or implied purposive adjunct. The DD is formed by adding =\textit{jinta} to the third person dative enclitic =\textit{rla,} but by adding \textit{=rla} to all other non-subject enclitics. Unlike the other enclitic pronouns, the DD encodes no specific person or number features – it merely signals an additional clausal constituent marked by dative case.\footnote{In addition to the pronouns in \tabref{tab:laughren:1}, Warlpiri has a number of honorific addressee pronouns substituted for "standard" second person pronouns in particular circumstances; third person and plural forms may also be substituted for second person singular ones. These special register forms are not relevant to the subject matter herein.}

\begin{table}
\fittable{
\small
\begin{tabularx}{\textwidth}{p{0.7cm}p{1.3cm}p{1.5cm}p{1.3cm}p{1cm}p{1cm}X}
\lsptoprule
&  & Free pronouns & \multicolumn{4}{c}{ Enclitic pronouns}\\
\cmidrule{4-7}
&  &  & subject & \multicolumn{3}{c}{non-subject}\\
\cmidrule{5-7}
Person & Number &  & \textsc{nom} & \multicolumn{2}{c}{\textsc{acc/dat}} & {\textsc{double} 
\textsc{dat}}\\
\midrule
1 & \textsc{sing} & \textit{ngaju(lu)}  & \textit{=rna} & \multicolumn{2}{c}{\textit{=ju}\footnote{The distribution of 'i' and 'u' vowels in enclitic pronouns is determined by the preceding vowel: 'i' following 'i' and 'u' following 'u'. Following 'a' there is dialectal variation; in eastern Warlpiri 'a' is usually followed by 'i'; in southern and western Warlpiri 'a' is typically followed by 'u' although there is variation in the pronunciation of the 12 person subject pronouns.}, =ji}  & =rla\\
13 & \textsc{du} & {ngajarra} & =rlijarra, =rlujarra & \multicolumn{2}{c}{{=jarrangku}} &  \\
%\hhline%%replace by cmidrule{------~~}
& \textsc{plu} & {nganimpa} & {=rna=lu} & \multicolumn{2}{c}{{=nganpa}} &  \\
%\hhline%%replace by cmidrule{------~~}
12 & \textsc{du} & {ngali(jarra)} & =rli, =rlu & \multicolumn{2}{c}{{=ngalingki}} &  \\
%\hhline%%replace by cmidrule{------~~}
& \textsc{plu} & {ngalipa} & =rlipa, =rlupa & \multicolumn{2}{c}{{=ngalpa}} &   \\
%\hhline%%replace by cmidrule{------~~}
2 & \textsc{sing} & {nyuntu(lu)} & {=n(pa)} & \multicolumn{2}{c}{=ngku, =ngki} &   \\
%\hhline%%replace by cmidrule{------~~}
& \textsc{du} & nyumpala, nyuntu-jarra & =n=pala & \multicolumn{2}{c}{=ngku=pala, =ngki=pala} &  \\
%\hhline%%replace by cmidrule{~-----~~}
& \textsc{plu} & nyurrurla, nyuntu-patu & =nku=lu, =nki=li, =npa=lu & \multicolumn{2}{c}{{=nyarra}} &   \\
%\hhline%%replace by cmidrule{------~~}
3 & \textsc{du} & {nyanungu-jarra} & {=pala} & \multicolumn{2}{c}{{=palangu}} &  \\
%\hhline%%replace by cmidrule{------~~}
& \textsc{plu} & {nyanungu-rra} & =lu, =li & \multicolumn{2}{c}{{=jana}} &  \\
\cmidrule{5-6}
&  &  &  & \multicolumn{1}{c}{\textsc{acc}} & \multicolumn{1}{c}{\textsc{dat}} & \multicolumn{1}{c}{}\\
\cmidrule{5-6}
3 & \textsc{sing} & {nyanungu} & {ø} & {ø} & {=rla} & {=jinta}\\
\lspbottomrule
\end{tabularx}
}
\todo[inline]{Should phonemic or phonetic notation be used for the explanantion text? 'x' is used for glosses.}
\caption{\label{tab:laughren:1} Warlpiri pronouns}
\end{table}

As stated above, the case-marking on free pronouns is basically the same as on nouns, except for the possessor form: -\textit{kurlangu} on determiners, nouns (and infinitives), -\textit{nyangu} on pronouns. Exceptionally, as subject of a transitive clause, first and second person singular pronouns may be either marked ergative, or left in their unmarked form. The presence of free pronouns coreferential with corresponding bound pronouns generally marks contrastive focus, or emphasizes a topic function. Note the contrast between \REF{ex:laughren:7a} with no free subject pronoun coreferential with the enclitic subject pronoun \textit{=npa} and \REF{ex:laughren:7b} in which the presence of the ergative marked free pronoun \textit{nyuntulu-rlu} stresses the speaker's desire that the addressee execute the order. In \REF{ex:laughren:7c}, spoken in one sequence, the contrastive focus on the addressee relative to the speaker is marked by the free pronoun \textit{nyuntu} `you' coreferential with the 'object' enclitic \textit{=ngku} in the first sentence and with the subject enclitic \textit{=npa} in the second.

\ea%7
    \label{ex:laughren:7}
\ea
\label{ex:laughren:7a}
\gll  Kuntul-pi-nyi   ka=npa   yalyu-kurlu? \\
    cough-do-\textsc{npst}  \textsc{prs.ind=2s}  blood-with\\
\glt `Are you coughing up blood?' [HN66-7]\footnote{HN59 indicates Hale fieldnotes with transcriptions of oral recordings made in fieldwork season 1959-60; HN66-7 those from 1966-67.}

\ex
\label{ex:laughren:7b}
\gll Kuntul-pu-ngka   wakurturdu-rlu \textbf{{nyuntulu}}{{}-rlu!}\\
    cough-strike-\textsc{imp}  strong-\textsc{erg}  2\textsc{sg-erg}\\
\glt `Cough (it) up strongly \textbf{{you}}!' [HN66-7]

\ex
\label{ex:laughren:7c}
\gll \textbf{Nyuntu}{}-ku   ka=rna=\textbf{ngku} nyina  kurdiji-mardarnu.\\
you.\textsc{sg-dat}  \textsc{prs.ind=1s=2}  sit.\textsc{npst}  senior\\
\gll \textbf{Nyuntu} {ka=}\textbf{{npa}}=ju   nyina karli-parnta.\\
    you.\textsc{sg}  \textsc{prs.ind=2s=1} sit.\textsc{npst}  junior\\
\glt `I am senior to \textbf{\textit{you}}. \textbf{\textit{You}} are junior to me.'\footnote{\textit{Kurdiji-mardarnu} lit. 'shield-holder'; \textit{karli-parnta} lit. 'boomerang-having'.} [HN66-7]
\z
\z


The third person pronoun \textit{nyanungu}, in its singular, dual and plural forms, may constitute an NP and may refer to animate or non-animate entities. In \REF{ex:laughren:8a} \textit{nyanungu} is the unmarked subject NP, its presence marking contrastive focus. This pronoun may also have a specific determiner function, as in \REF{ex:laughren:8b}, in which it combines with \textit{wawirri} `kangaroo' to form a complex NP.\footnote{For an extensive discussion of reflexives and pronominal reference in Warlpiri, see Simpson (1991: §3.4), and Hale et al. (1996: §6).}

\ea%8
    \label{ex:laughren:8}

\ea
\label{ex:laughren:8a}
\gll {Kajika} \textbf{{nyanungu}} wangka   yangka  jinta-kari.\\
    might  3\textsc{sg}  say  aforementioned  one-other\\
\glt    \textbf{{}'He} might say -{}- that other one (that is): [...]' [HN66-7]

\ex
\label{ex:laughren:8b}
\gll Nyarrpara   ka=npa   nya-nyi  kuja \textbf{{nyanungu}}=ju  wawirri?\\
    Where  \textsc{prs.ind=2s}  see-\textsc{npst}  that  3\textsc{sg=top}  kangaroo\\
  \glt `Where is it that you can see \textbf{that/this/the} kangaroo (that you said you saw).' [HN66-7]
  \z
\z


\subsection{Anaphor and coreference}\label{sec:laughren:2.2}

Warlpiri also has an anaphoric non-subject enclitic pronoun \textit{=nyanu} used in both reflexive and, with dual or plural subjects, reciprocal constructions in finite clauses as shown in \REF{ex:laughren:9a} and \REF{ex:laughren:9c}.\footnote{Evans et al. (2007 \sectref{sec:laughren:3.1}) details properties of Warlpiri reciprocal clauses which are applicable to the reflexive clauses discussed herein.} Its referential value is always that of the subject.\footnote{\citet{Blake1988} reconstructs \textit{nyanu} as Eastern Pama-Nyungan feminine dative pronoun.} It is used with all subject enclitics with the exception of first person singular \REF{ex:laughren:9b} and the second person singular in imperative clauses \REF{ex:laughren:9d}, in which the non-anaphor second person non-subject enclitic is used. In non-imperative clauses with a second person singular subject =\textit{nyanu} must be used to signal coreference of an object or applicative with the subject as in \REF{ex:laughren:9c}. The NPs in \REF{ex:laughren:9a} and \REF{ex:laughren:9d} are in parentheses to indicate their grammatical optionality.

\ea%9
    \label{ex:laughren:9}

\ea
\label{ex:laughren:9a}
\gll Nya-ngu=rna=lu=nyanu. ({nganimpa-rlu})  (*{nganimpa})\\
    see-\textsc{pst=1s=pl.s=anaph}  (13\textsc{pl-erg)}   (1\textsc{3pl})\\
\glt `We saw each other/we saw ourselves.'

\ex
\label{ex:laughren:9b}
\gll Nya-ngu=rna=ju/*=nyanu (ngajulu-rlu) {/} ({ngaju(lu)})\\
    see-\textsc{pst=1s=1/*anaph}  (\textsc{1sg-erg}) {/} (\textsc{1sg)}\\
\glt `I saw myself.'

\ex
\label{ex:laughren:9c}
\gll Nya-ngu=npa=nyanu/*=ngku ({nyuntulu-rlu}) /  ({nyuntu(lu)})\\
    see-\textsc{pst=2s=anaph/*2}  (\textsc{2sg-erg)} { } (\textsc{2sg})\\
\glt `You saw yourself.'


\ex
\label{ex:laughren:9d}
\gll Nya-ngka=ngku/*=nyanu  ngapa-ngka ({nyuntulu-rlu}) / (nyuntu(lu))\\
    see-\textsc{imp=2/*anaph}  water-\textsc{loc}  (\textsc{2sg-erg})   { }   (\textsc{2sg)}\\
\glt `See/look at yourself in the water.'
\z
\z


Warlpiri has no subject reflexive pronoun, either free or bound, nor does it have a free reflexive pronoun akin to English pronouns with the suffix `self', or a form to mark long-distance anaphora (cf. \citealt{Giorgi2007}). The non-subject enclitic forms coreferential with the subject are set out in \tabref{tab:laughren:2}.

\begin{table}
\begin{tabularx}{0.7\textwidth}{p{1.3cm}XX} 
\lsptoprule
& \textsc{accusative/dative} & \\
\hline
1 & \textit{=ju, =ji} & Only with singular reference\\
2 & \textit{=ngku,=ngki} & Only with imperative verb and singular reference\\
\textsc{anaphor} & \textit{=nyanu} & Used with all other subject pronouns\\
\lspbottomrule
\end{tabularx}
\label{tab:laughren:2}
\caption{Reflexive/reciprocal enclitic pronouns and anaphor}
\end{table}

Irrespective of the case frame of the verb in the clause, the identical anaphor form is used, including with unmarked third person singular subjects as in \REF{ex:laughren:10b} and \REF{ex:laughren:11b}. The anaphor \textit{=nyanu} in \REF{ex:laughren:10b} contrasts with a zero marked disjoint accusative object in \REF{ex:laughren:10a}. A non-inflected object free pronoun (or other NP type) is grammatical in \REF{ex:laughren:10a} since it is referentially disjoint from the subject. In \REF{ex:laughren:10b} it is ungrammatical if coreferential with the subject as on the reading given.

\ea%10
    \label{ex:laughren:10}
\ea
\label{ex:laughren:10a}
\gll Paka-rnu  wati-ngki\textsubscript{i} ({nyanungu}\textsubscript{*i/j}).\\
    hit-\textsc{pst}  man-\textsc{erg}  3\\
\glt `The man\textsubscript{i} hit him\textsubscript{*i}/her/it.'

\ex
\label{ex:laughren:10b}
\gll {Paka-rnu=}\textbf{nyanu}{\textsubscript{i}} {wati-ngki\textsubscript{i} }*{nyanungu\textsubscript{i}}.\\
    hit-\textsc{pst=anaph}  man-\textsc{erg}  3\\
\glt `The man\textsubscript{i} hit \textbf{himself}\textsubscript{i/*j}.'
\z
\z


In \REF{ex:laughren:11a} the dative enclitic =\textit{rla} marks the third person singular features of the dative object which must have disjoint reference from that of the subject. In \REF{ex:laughren:11b}, the presence of the anaphor =\textit{nyanu} signals coreference of the dative object with the subject. In both sentences a dative-marked free pronoun coreferential with the non-subject enclitic pronoun is optional. In \REF{ex:laughren:10b} it is also coreferential with the subject.

\ea%11
    \label{ex:laughren:11}
\ea
\label{ex:laughren:11a}
\gll Wangka-ja=lpa=rla\textsubscript{*i/j} {wati\textsubscript{i}} ({nyanungu\textsubscript{*i/j}}{{}-ku}).\\
    say-\textsc{pst=}\textsc{impf=3dat}  man  3-\textsc{dat}\\
\glt `The man\textsubscript{i} spoke to him\textsubscript{*i/j}/her.'

\ex
\label{ex:laughren:11b}
\gll {Wangka-ja=lpa=}\textbf{{nyanu}} {wati\textsubscript{i}} ({nyanungu\textsubscript{i}}{{}-ku}).\\
    say-\textsc{pst=impf=anaph}  man  3-\textsc{dat}\\
\glt `The man\textsubscript{i} spoke to \textbf{himself}\textsubscript{i/*j}.'
\z
\z


As noted above, the addition of the third person free pronoun \textit{nyanungu} to \REF{ex:laughren:10b} is ungrammatical on the interpretation given. However on a disjoint reference reading between subject and object, and the anaphor \textit{{}-nyanu} coreferential with the subject being interpreted as a dative applicative argument, \REF{ex:laughren:10b} would be grammatical and interpretable as `The man\textsubscript{i} hit that one\textsubscript{*i/j} for himself\textsubscript{i/*j}{}'.

Unlike the verb's object which cannot be coreferential with an unmarked free pronoun as shown in \REF{ex:laughren:10b}, the dative object or applicative can be expressed by both the bound anaphor \textit{=nyanu} (signalling coreference with the subject) and an optional dative-marked free pronoun also coreferential with the subject. However this is only possible in a clause in which the subject NP is unmarked, as in \REF{ex:laughren:11b}. Where the subject NP is ergative-marked, coreference between subject and object – whether the latter is unmarked or dative – is ungrammatical. This contrast is illustrated in \REF{ex:laughren:12}.

In \REF{ex:laughren:12a} which is grammatical, the subject NP \textit{Jakamarra} is unmarked, and the dative marked pronoun \textit{nyanungu-ku} is coreferential with the anaphor \textit{=nyanu} which in turn is coreferential with the unmarked subject \textit{Jakamarra}. In \REF{ex:laughren:12b} in which the subject is marked with ergative case, the presence of the dative pronoun \textit{nyanungu-ku,} whether interpreted as coreferential or disjoint with the subject, renders the sentence ungrammatical.\footnote{See \citet[1440-1441]{HaleEtAl1995} and Simpson (1991: \sectref{sec:laughren:6.3}) for further examples and discussion of anaphora in Warlpiri.}

In \REF{ex:laughren:12b} the dative object argument of the verb \textit{yi-nyi} {} 'give' cannot be expressed by a dative-marked pronoun \textit{nyanungu-ku} interpreted as coreferential with the ergative subject \textit{Jakamarra-rlu} via the anaphoric enclitic \textit{=nyanu}. When the DD enclitic \textit{=rla} is added to the auxiliary in as in \REF{ex:laughren:12c}, the anaphor \textit{=nyanu} must be dative and coreferential with the subject, but it can be interpreted as either a dative object (recipient of giving) or as a dative applicative (e.g., ``higher" benefactive/possessive applicative). The DD enclitic =\textit{rla} is obligatorily disjoint in reference from the subject, and can be interpreted as linked to either an object or applicative role, but not the same role as the one associated with the anaphor.

\ea%12
    \label{ex:laughren:12}

\ea[]{
\label{ex:laughren:12a}
\gll Nyanungu-ku\textsubscript{i/*j} ka=nyanu\textsubscript{i/*j} Jakamarra\textsubscript{i} yulka-mi.\\
    3-\textsc{dat}  \textsc{prs.ind-anaph}  Jakamarra  love-\textsc{npst}\\
\glt `Jakamarra loves himself.' (Hale \textit{et al.} 1995: 1441 \#42a)
}

\ex[*]{
\label{ex:laughren:12b}
\gll Jakamarra-rlu\textsubscript{i} ka=nyanu\textsubscript{i}nyanungu-ku\textsubscript{*i/*j} kuyu  yi-nyi.\\
    Jakamarra-\textsc{erg} \textsc{prs.ind=anaph}  3-\textsc{dat} meat give-\textsc{npst}\\
 \glt 'Jakamarra is giving himself meat.' (\citealt{HaleEtAl1995}: 1440, (40c)) 
}

\ex[]{
\label{ex:laughren:12c}
\gll Jakamarra-rlu\textsubscript{i} ka=nyanu\textsubscript{i}=rla\textsubscript{j} nyanungu-ku\textsubscript{*i/j} kuyu  yi-nyi.\\
    Jakamarra-\textsc{erg} \textsc{prs.ind=anaph=dd}  3-\textsc{dat} meat give-\textsc{npst}\\
\glt `J\textsubscript{i} gives himself\textsubscript{i/*j} meat for him\textsubscript{*i/j}' \\
`J\textsubscript{i} gives him\textsubscript{*i/j} his\textsubscript{i/*j} meat.' \\
`J\textsubscript{i} gives him\textsubscript{*i/j} meat for himself\textsubscript{i/*j}.'
}
%\glt `J\textsubscript{i} gives him\textsubscript{*i/j} his\textsubscript{i/*j} meat.'

%\glt `J\textsubscript{i} gives him\textsubscript{*i/j} meat for himself\textsubscript{i/*j}.'
\z
\z


The DD structure in \REF{ex:laughren:12c} is similar to that in \REF{ex:laughren:13a} in which \textit{=nyanu} is coreferential with the dative-marked applicative argument \textit{nyanungu-ku}, and not the dative-marked object \textit{kuyu-ku} {}'meat' of the verb \textit{warri-rni} `look for'. As in \REF{ex:laughren:12c}, the presence of two dative-marked NPs, the dative object and the dative applicative, is marked by the invariant DD auxiliary enclitic =\textit{rla} added to the anaphoric enclitic \textit{=nyanu.} In \REF{ex:laughren:13b} which lacks a dative applicative argument the dative object is expressed by =\textit{rla} coreferential with \textit{kuyu-ku} `meat', but necessarily disjoint with the ergative subject \textit{Jakamarra-rlu}. The free dative-marked pronoun \textit{nyanungu-ku} in \REF{ex:laughren:13b} is coreferential with the dative object \textit{kuyu-ku}, thus functioning as a determiner within the same complex dative-marked NP as \textit{kuyu-ku}. In \REF{ex:laughren:13c} the presence of a dative object and a dative applicative is signalled by the DD enclitic sequence \textit{=rla=jinta}, in which each element has a different referent. The semantic ambiguity of \REF{ex:laughren:13c} derives from which grammatical function — goal of search (dative object)  or beneficiary of search (applicative) — is linked to the dative enclitic \textit{{}-rla} which is coreferential with the human referring dative-marked NP \textit{nyanungu-ku}, while \textit{=jinta} refers to the non-animate dative-marked NP \textit{kuyu-ku.} In \REF{ex:laughren:13d} in which both dative-marked NPs are coreferential with the dative enclitic \textit{=rla}, the DD \textit{=jinta} signals an implied purpose.

\ea%13
    \label{ex:laughren:13}
\ea
\label{ex:laughren:13a}
\gll Jakamarra\textsubscript{i}-rlu  ka=nyanu\textsubscript{i/*j}=rla\textsubscript{j}warri-rni  kuyu\textsubscript{j}-ku  nyanungu\textsubscript{i/*j}-ku.\\
    J-\textsc{erg}  \textsc{prs.ind}\textsc{=anaph=3dat}  seek-\textsc{npst} meat-\textsc{dat}  3-\textsc{dat}\\
\glt `Jakamarra\textsubscript{i} is looking for his\textsubscript{i} meat/is looking for meat for himself\textsubscript{i}.' (Hale \textit{et al}. 1995: 1440 \#41a)

\ex
\label{ex:laughren:13b}
\gll Jakamarra\textsubscript{i}-rlu  ka=rla\textsubscript{*i/j} {warri-rni} [{kuyu\textsubscript{j}}-ku  nyanungu{\textsubscript{j/*i/*k}}{{}-ku}].\\
    J-\textsc{erg}  \textsc{prs.ind}\textsc{=3dat}  seek-\textsc{npst}  meat-\textsc{dat}  3-\textsc{dat}\\
\glt `Jakamarra is looking for that meat.'

\ex
\label{ex:laughren:13c}
\gll Jakamarra\textsubscript{i}-rlu ka=rla\textsubscript{*i/j}=jinta\textsubscript{*i/*j/k} warri-rni  kuyu\textsubscript{*i/*j/k}-ku  nyanungu\textsubscript{*i/j}-ku.\\
J-\textsc{erg} \textsc{prs.ind=3dat=dd} seek-\textsc{npst} meat-\textsc{dat} 3-\textsc{dat}\\

\glt `Jakamarra\textsubscript{i} is looking for meat\textsubscript{*i/*j} for him\textsubscript{*i/j}.'

\glt `Jakamarra\textsubscript{i} is looking for him\textsubscript{*i/j} for meat\textsubscript{*i/*j}.'

\ex
\label{ex:laughren:13d}
\gll {Jakamarra\textsubscript{i}}{{}-rlu  ka=rla}{\textsubscript{*i/j}}{=jinta\textsubscript{*i/*j/k} }{warri-rni} [{kuyu\textsubscript{*i/j}}{{}-ku  nyanungu}{\textsubscript{*i/j}}{{}-ku}].\\
    J-\textsc{erg}  \textsc{prs.ind=3dat=dd}  seek-\textsc{npst}  meat-\textsc{dat}  3-\textsc{dat}\\
  \glt `Jakamarra is looking for that meat for some purpose (e.g. to cook/eat).'
\z
\z


\citet[167]{Simpson1991} points out that while the third singular dative enclitic  \textit{=rla} may be added to an anaphoric clitic as a DD marker as in \REF{ex:laughren:13a}, it is not possible to have a coreferential reading between these non-subject enclitics. Only the subject can determine the reference of an anaphor.

\ea[*]{
\label{ex:laughren:14}
\gll Wangka-ja=lpa=rna=nyanu=rla.\\
    speak-\textsc{pst=impf=1s=anaph=dd}\\
\glt    ${\neq}$ `I spoke to him about himself.' (\citealt{Simpson1991}: 167, 137)
}
\z


As will have been noted, the DD enclitic has the unique form \-\textit{=rla}, except when the preceding dative enclitic is also \textit{=rla,} as in \REF{ex:laughren:13c} and \REF{ex:laughren:13d}, in which case the DD is marked by \textit{{}-jinta}. The choice of which argument is represented by the first dative enclitic which encodes person and/or number features, and which by the DD enclitic is determined on the basis of grammatical function and a person feature hierarchy. This is partially exemplified by the auxiliary enclitics used with the verb \textit{kunka-mani} 'to get even with' in \REF{ex:laughren:15}. Here second person dative enclitic \textit{=ngku} refers to the person on whom the subject plans to take revenge, while the obligatory DD enclitic \textit{=rla} signals an understood applicative argument, i.e., because of what you did (to me/someone).

\ea%15
    \label{ex:laughren:15}
\gll {Kapu=rna=}\textbf{ngku}=rla  kunka-ma-ni  jalangu-rlu  ({nyuntu-ku}).\\
    \textsc{fut=1s=2=dd}  revenge-\textsc{caus-npst}  now-\textsc{erg}  (you-\textsc{dat)}\\
\glt `I'll get even with you now for it.'
\z


In \REF{ex:laughren:16} in which the goal of revenge is a third person, it is expressed by the DD, while the preceding dative clitic expresses the features of the person on whose behalf revenge is taken. In (16a-c) the dative enclitic — second person in \REF{ex:laughren:16a}, first person in \REF{ex:laughren:16b} and anaphoric in \REF{ex:laughren:16c} is coreferential with the subject thus encoding coreference between avenger and avenged.

\ea%16
    \label{ex:laughren:16}
\ea
\label{ex:laughren:16a}
\gll {Kunka-ma-nta=}\textbf{{ngku}}=rla  nyuntulu-rlu  wiyarrpa-rlu.\\
    revenge-\textsc{caus-imp=}\textbf{\textsc{2}}\textsc{=DD}  you-\textsc{erg}  poor\_thing-\textsc{erg}\\
  \glt `Take your revenge for it (on him/her/them), you poor thing.' [HN59]

\ex
\label{ex:laughren:16b}
\gll Kapu=rna=\textbf{{ju}}=rla  jukurra-rlu=jala  kunka-ma-ni.\\
    \textsc{fut-1s=}\textbf{\textsc{1}}\textsc{=dd}  tomorrow-\textsc{erg=cfoc}  revenge-\textsc{caus-npst}\\
\glt `I will get my revenge for it (on him/her/them) tomorrow (not now).'

\ex
\label{ex:laughren:16c}
\gll Ngilyi-parnta-rlu  ka=\textbf{{nyanu}}=rla  kunka-ma-ni.\\
    rotten\_one-\textsc{erg}  \textsc{prs.ind=}\textbf{\textsc{anaph}}\textsc{=dd}  revenge-\textsc{caus-npst}\\
\glt `That rotten one\textsubscript{i} is taking her\textsubscript{i} revenge for it (on him/her/them).'
\z
\z


\subsection{Coreference between subject and pronoun in a phrase introduced by a semantic case}\label{sec:laughren:2.3}

A semantic case-headed nominal expression, similar to an English prepositional phrase, acting as either a complement or an adjunct can consist of a free pronoun to which a semantic case, such as the perlative \textit{{}-wana} in \REF{ex:laughren:17a} and \REF{ex:laughren:17b} is added. It can be coreferential with the subject as in \REF{ex:laughren:17a} and \REF{ex:laughren:17b}.\footnote{A similar example with postposition -\textit{jangkardu} is cited in Simpson (1991: 169 \#140).}

\ea%17
\label{ex:laughren:17}
\ea
\label{ex:laughren:17a}
\gll Jakamarrai-rlu	yirra-rnu {/} nya-ngu	nyanungu\textsubscript{i/j}-wana.\\
J-\textsc{erg} put-\textsc{pst} {/} see-\textsc{pst}	3-\textsc{perl}\\
\glt `Jakamarrai put/saw it near himi/j.' \\

\ex
\label{ex:laughren:17b}
\gll Ngajulu-rlu=rna yirra-rnu {/} nya-ngu	ngaju-wana.\\
1-\textsc{erg}=1S put-\textsc{pst} {/} see-\textsc{pst} 1-\textsc{perl}\\
\glt `I put/saw it near me.'\\
\z
\z 
		 

\section{Other relationships within reflexive clauses}\label{sec:laughren:3}

\subsection{Reflexives and part-whole relations}\label{sec:laughren:3.1}

The syntax of part-whole, including body part, constructions has been described by \citet{Hale1981} and \citet{Laughren1992} \textit{inter alia}. In what \citet[338]{Hale1981} called the ``favorite mode of expression" of part-whole relations, the ``whole" is linked to a primary syntactic function such as subject and object while the ``part" is expressed by an NP assigned the same case as the whole, but not included in the NP referring to the whole. The ``part" NP acts like a secondary predicate which specifies the relevant ``part" of the ``whole".

In \REF{ex:laughren:18a} a third person singular subject acts on a third person singular object. Subject and object are referentially disjoint, hence the absence of auxiliary pronominal enclitic. The ergative-marked NP \textit{kurdu-ngku} `child' is associated with the subject, while the unmarked NP \textit{ngati} `mother' is associated with the object function. The ergative case on the NP \textit{rdaka-ngku} `hand/finger' identifies it as the relevant part of the child as the `poker' while the unmarked NP \textit{milpa} `eye' is the relevant part of the mother, the `poked'.

In \REF{ex:laughren:18b} the object is coreferential with the subject, indicated by the anaphoric enclitic \textit{{}-nyanu} (and the unacceptability of an object NP), so that the same `child' is both the `poker' and the `poked'. However the different parts of the child involved in the 'poking' event referred to by \REF{ex:laughren:18b} play different roles; as in \REF{ex:laughren:18a} they are aligned with the different thematic roles. Both \REF{ex:laughren:18a} and \REF{ex:laughren:18b} are transitive, but semantically vague in that they allow an interpretation in which the poking action is either intentional or not intentional.

\ea%18
\label{ex:laughren:18}
\ea
\label{ex:laughren:18a}
\gll Kurdu-ngku	ka	ngati	panti-rni	milpa	rdaka-ngku.\\
child-\textsc{erg}	\textsc{prs.ind} mother	poke-\textsc{npst}	eye	hand-\textsc{erg}\\
\glt `The child pokes mother in the eye with his finger.' \\

\ex
\label{ex:laughren:18b}
\gll Kurdu-ngku	ka=nyanu rdaka-ngku	panti-rni	milpa.\\
child-\textsc{erg}	\textsc{prs.ind=anaph}	hand-\textsc{erg}	poke-\textsc{npst}	eye	\\
\glt `The child pokes himself in the eye.'\\
\z
\z 



The sentences \REF{ex:laughren:18a} and \REF{ex:laughren:18b} in which intentionality on the part of the referent of the subject can be inferred contrast with those in \REF{ex:laughren:19}. In \REF{ex:laughren:19a} the pointed object which makes contact with the hand of the child is referred to by the ergative-marked subject NP \textit{jiri-ngki} 'prickle/thorn', while in \REF{ex:laughren:19b} it is a an illness whose symptoms include the production of quantities of nasal mucous (also called \textit{miirnta}) that is expressed as the subject which affects the child expressed as the object.\footnote{The verb \textit{pantirni} denotes contact between a pointed entity and the surface of some entity which may be pierced (cf. English \textit{jab, pierce, stab, stick into}) or not (cf. English \textit{poke}).}

\ea%19
    \label{ex:laughren:19}
\ea
\label{ex:laughren:19a}
\gll {Jiri-ngki} \textbf{{kurdu}} {pantu-rnu} \textbf{{rdaka}}.   \\
    prickle-\textsc{erg}  \textbf{child}  poked-\textsc{pst}  \textbf{hand}\\
  \glt `A prickle got stuck into the child's hand.'
    (Lit. A prickle stabbed/pierced the child hand)

\ex
\label{ex:laughren:19b}
\gll  Miirnta=rlu  kurdu  paka-rnu.\\
    flu=\textsc{erg}  child  hit-\textsc{pst}\\
  \glt `The child was struck by flu.'
    (Lit. Flu/nasal mucous struck the child.)
\z
\z

What is common to the sentences in \REF{ex:laughren:18} and \REF{ex:laughren:19} is that the 'patient', i.e., the entity/individual that is affected by the action, is expressed as the syntactic object, while the ergative-marked subject causes the occurrence of the event referred to, whether deliberately or not.

\subsubsection{Reflexive clauses with change of state verbs}\label{sec:laughren:3.2}

Verbs which express a change of state in a patient without denoting a cause or agent thematic role are typically formed in Warlpiri by complex verbs consisting of a preverbal predicate which combines with an intransitive 'change' verb. Sentences featuring the Warlpiri equivalent of the prototypical English 'change of state' verb \textit{break} are given in \REF{ex:laughren:20}. \textit{Rdilyki} 'broken' belongs to a set of 'stage' predicates which refer to the result of a change of state and which combine with an intransitive inflecting verb such as \textit{ya-} 'go' to create an inchoative verbal predicate.

The inflected verb \textit{ya-nu} 'go-\textsc{pst'} in \REF{ex:laughren:20a}, which in this context denotes a simple change of state undergone by the subject's referent, differs in form and meaning from the inflected transitive verb \textit{pu-ngu} {}'strike-\textsc{pst'} in \REF{ex:laughren:20b} which implies an action carried out by an agent which produces a change of state in a patient. In \REF{ex:laughren:20a} the patient role is borne by the unmarked subject NP \textit{kurdu}.\footnote{It is possible to add a dative-marked phrase to \REF{ex:laughren:20a} to refer to an entity which may be inferred to have 'caused' the situation referred to, but this is not relevant to the argument set out here (see \citet[386-8]{Simpson1991}).(i)   \textit{Waku=rla   marlaja   rdilyki-ya-nu   kurdu   watiya-ku/wati-ki.}  arm=\textsc{3dat}   because\_of   broken-go-\textsc{pst}   child   stick-\textsc{dat}/man-\textsc{dat}\glt `The child broke his arm because of the stick/man...'} The agent in \REF{ex:laughren:20b} is expressed by the ergative-marked subject NP \textit{wati-ngki}, while the patient object NP \textit{kurdu} is unmarked. The affected body part \textit{waku} 'arm' is also unmarked in both \REF{ex:laughren:20a} and \REF{ex:laughren:20b}. In the stative sentence in \REF{ex:laughren:20c} \textit{rdilyki} occurs as a nominal predicate external to the verb \textit{nguna} 'lie'. This contrasts with its use in \REF{ex:laughren:20a} and \REF{ex:laughren:20b} in which it is in a tighter preverbal relation with the inflecting verb.

\ea%20
    \label{ex:laughren:20}
\ea
\label{ex:laughren:20a}
\gll  Waku   rdilyki-ya-nu  kurdu.\\
    arm  broken-go-\textsc{pst}  child\\
  \glt `The child\textsubscript{i} broke his\textsubscript{i/*j} arm.' (Lit. the child broke arm(wise))

\ex
\label{ex:laughren:20b}
\gll  Waku  rdilyki-pu-ngu  kurdu  wati-ngki  punku-ngku.\\
    arm  broken-strike-\textsc{pst}  child  man-\textsc{erg}  bad-\textsc{erg}\\
  \glt `The nasty man broke the child's arm.' (Lit. the bad man broke the child arm(wise))

\ex
\label{ex:laughren:20c}
\gll  Kurlarda  yali  ka  nguna  rdilyki.\\
    spear  that  \textsc{prs.ind}  lie.\textsc{npst}  broken\\
\glt `That spear is lying broken.'
\z
\z


In contrast with \REF{ex:laughren:20a}, the reflexive sentence in \REF{ex:laughren:21} implies that the child's action of hitting himself (with a stick) caused his own arm to break.

\ea%21
    \label{ex:laughren:21}
\gll Waku=nyanu  kurdu-ngku  rdilyki-pu-ngu ({watiya-rlu}).\\
  arm=\textsc{anaph}  child-\textsc{erg}  broken-strike-\textsc{pst}  (stick-\textsc{erg)}\\
\glt `The child hit and broke his (own) arm (with a stick).'
\z


In this respect Warlpiri differs from Romance languages in which the reflexive sentence, as exemplified by the French sentence in \REF{ex:laughren:22a}, does not necessarily imply an agent, but is interpretable as an inchoative sentence featuring a patient subject and body part complement, equivalent in meaning – but not in form – to the Warlpiri sentence in \REF{ex:laughren:20a}.\footnote{\textsuperscript{} Change of state verbs such as \textit{casser} `break' are prototypical unaccusative verbs \citep{Perlmutter1973} in which the patient argument is first linked to the object function and then raised to the subject position (\citealt{Burzio1986}, \citealt{LevinHovav1995}). This construction, sometimes referred to as reflexive passive, differs from a reflexive construction in which distinct agent and patient roles are linked to a subject and object function associated with the same referent as in the Warlpiri sentence in \REF{ex:laughren:21}.}

\ea%22
    \label{ex:laughren:22}

\ea
\label{ex:laughren:22a}
\gll L' enfant  s'est  cassé  le  bras.\\
    the child  \textsc{refl}{}-is  broken  the  arm\\
\glt `The child broke his arm.'

\ex
\label{ex:laughren:22b}
\gll  Elle  lui  a  cassé  le  bras.\\
    she  3\textsc{sg}.\textsc{dat}  has  broken  the  arm\\
\glt `She broke his arm.'
\z
\z

Another difference between Romance languages and Warlpiri is that in the former it is the dative non-subject clitic pronoun, exemplified in \REF{ex:laughren:22b} by \textit{lui} (as opposed to accusative \textit{le} or \textit{la}) that refers to the whole while the affected part is referred to by the object NP as \textit{le bras} in \REF{ex:laughren:22a} and \REF{ex:laughren:22b}, whereas in Warlpiri, it is the affected whole which is the object in a transitive clause, signalled by the absence of an enclitic object pronoun for a third person singular in \REF{ex:laughren:20a} and \REF{ex:laughren:23a}. The dative enclitic third person singular pronoun \textit{=rla} in \REF{ex:laughren:23b} cannot be associated with the affected whole.

\ea%23
    \label{ex:laughren:23}
\ea[]{
\label{ex:laughren:23a}
\gll  Rdilyki-paka-rnu  waku.\\
    broken-hit-\textsc{pst}  arm\\
\glt `She hit and broke his arm.'
}

\ex[*]{
\label{ex:laughren:23b}
\gll Rdilyki-paka-rnu=\textbf{{rla}} waku.\\
    broken-hit-\textsc{pst=}\textbf{\textsc{3dat}}  arm.\\
    ${\neq}$ 'She hit and broke his arm.'
    }
    \z
\z


\subsection{Reflexive clauses with change of location verbs}\label{sec:laughren:3.3}

The location complement of 'change of location' verbs is expressed by a phrase headed by a semantic case such as the locative, allative, elative, or perlative. When the location is part of some whole as in \REF{ex:laughren:24}, there are two possible modes of expression. One is to place both the whole and the part in separate phrases headed by an identical semantic case as in \REF{ex:laughren:24a}, the other is to express the 'whole' as a dative object marked by a dative auxiliary enclitic while the 'part' is independently expressed in a semantic case headed phrase, as in \REF{ex:laughren:24b}.

\ea%24
    \label{ex:laughren:24}
\ea
\label{ex:laughren:24a}
\gll Nama  ka \textbf{{langa-kurra}} {yuka-mi} \textbf{{kurdu-kurra}}.\\
    ant  \textsc{prs.ind}  ear-\textsc{allat}  enter-\textsc{npst}  child-\textsc{allat}\\
\glt ‘The ant is entering the child’s ear.’ (Lit. ant into ear enters into child) (\citealt{Hale1981}: 341 \#24)

\ex
\label{ex:laughren:24b}
\gll Nama  ka \textbf{{rla}}{\textsubscript{i} }\textbf{{langa-kurra}} {yuka-mi} \textbf{{kurdu}}{\textsubscript{i}}\textbf{{{}-ku}}. \\
    ant  \textsc{prs.ind=3dat}  ear-\textsc{allat}  enter-\textsc{npst}  child-\textsc{dat}\\
\glt ‘The ant is entering the child’s ear.’ (Lit. ant to him into ear enters to child) (\citealt{Hale1981}: 341 \#24')
\z
\z


Where referential identity between the subject and the location is intended, only the dative object strategy of \REF{ex:laughren:24b} can force a reflexive interpretation, as shown in \REF{ex:laughren:25a}. The free third person pronoun \textit{nyanungu} in \REF{ex:laughren:25b} may be interpreted as coreferential with the subject or not.

\ea%25
    \label{ex:laughren:25}
\ea
\label{ex:laughren:25a}
\gll  Wati-ngki=nyanu  kuruwarri  kuju-rnu  rdukurduku-rla. \\
    man\textsc{{}-erg=anaph}  design  throw-\textsc{pst}  chest-\textsc{loc}\\
\glt `The man\textsubscript{i} painted a design on his\textsubscript{i/*j} chest.'

\ex
\label{ex:laughren:25b}
\gll  Wati-ngki  kuruwarri  kuju-rnu  nyanungu-rla  rdukurduku-rla.\\
    man-\textsc{erg}  design  throw-\textsc{pst}  \textsc{3-loc}  chest-\textsc{loc}\\
\glt `The man\textsubscript{i} painted a design on his\textsubscript{i/j} chest.'

\z
\z


\subsection{Reflexive clauses with 'bodily grooming' verbs}\label{sec:laughren:3.4}

Unlike English in which transitive verbs denoting acts of bodily grooming, especially with a human subject, may have a reflexive interpretation in the absence of an overt object NP, in Warlpiri the reflexive enclitic pronoun must be used, as with other transitive 'affect by contact' verbs. The self-grooming interpretation of the reflexive clause in \REF{ex:laughren:26a} contrasts with the other-grooming interpretation in the non-reflexive clause in \REF{ex:laughren:26b}.

\ea%26
    \label{ex:laughren:26}
\ea
\label{ex:laughren:26a}
\gll {Parlju-rnu=nyanu} ({nyanungu-rlu}).\\
    wash-\textsc{pst}{}-\textsc{anaph}  3-\textsc{erg}\\
\glt `She washed (herself).'

\ex
\label{ex:laughren:26b}
\gll {Parlju-rnu} ({nyanungu-rlu}).\\
    wash-\textsc{pst}  3-\textsc{erg}\\
  \glt `She\textsubscript{i} washed it/him/her\textsubscript{*i}.'
    ${\neq}$ She washed (herself).
    \z
\z


When an NP referring to the affected body part is added as in \REF{ex:laughren:27}, verbs like \textit{parljirni} 'wash' behave the same as the other transitive 'affect by contact' verbs seen in \sectref{sec:laughren:3.2}\footnote{\textsuperscript{} Simpson (1991: 170 \#142) cites a similar example with 'shave' taken from Hale's 1959 fieldnotes.  \textit{Jangarnka=npa=nyanu   jarntu-rnu?}  beard=\textsc{2s=anaph}  shave-\textsc{pst} \glt `Did you shave your beard off?'}

\ea%27
    \label{ex:laughren:27}
\ea
\label{ex:laughren:27a}
\gll {Parlju-rnu=nyanu} (\textit{nyanungu-rlu})  jurru.\\
    wash-\textsc{pst}{}-\textsc{anaph}  3-\textsc{erg}  head/hair\\
\glt `She\textsubscript{i} washed her\textsubscript{i/*j} hair.'

\ex
\label{ex:laughren:27b}
\gll {Parlju-rnu} ({nyanungu-rlu}) {jurru.}\\
    wash-\textsc{pst}  3-\textsc{erg}  head/hair\\
\glt `She\textsubscript{i} washed her\textsubscript{*i} /his/its hair.'
\z
\z

Similarly, verbs such as \textit{majarni} 'stretch, straighten' when used to express bodily self-manipulation must be used in a syntactically reflexive construction, as in \REF{ex:laughren:28a} and \REF{ex:laughren:28b}. The absence of the anaphoric non-subject enclitic as in \REF{ex:laughren:28c} and \REF{ex:laughren:28d} can only be interpreted with disjoint reference between subject and object. In \REF{ex:laughren:28d} the arm (\textit{waku}) that is straightened is part of the referent of the grammatical object not coreferential with the subject.

\ea%28
    \label{ex:laughren:28}
\ea
\label{ex:laughren:28a}
\gll  Maja-rnu=nyanu  (nyanungu-rlu). \\
    straighten\textsc{{}-pst-anaph  3-erg}\\
\glt `She straightened up/stretched (herself).'

\ex
\label{ex:laughren:28b}
\gll  Maja-rnu=nyanu  (nyanungu-rlu)  waku.\\
    straighten\textsc{{}-pst=anaph} 3-\textsc{erg}  arm\\
\glt `She\textsubscript{i} straightened her\textsubscript{i/*j} arm.'

\ex
\label{ex:laughren:28c}
\gll  Maja-rnu  (nyanungu-rlu).\\
    straighten-\textsc{pst}  3\textsc{{}-erg}\\
\glt `She\textsubscript{i} straightened him/her\textsubscript{*i}/it.'

\ex
\label{ex:laughren:28d}
\gll {Maja-rnu} ({nyanungu-rlu}) {waku.}\\
    straighten-\textsc{pst}  3-\textsc{erg}  arm\\
\glt `She\textsubscript{i} straightened her\textsubscript{*i/j} /his/its arm.'
\z
\z


Disjoint reference between subject and object is clear in \REF{ex:laughren:29a}. In \REF{ex:laughren:29b} the presence of the anaphor =\textit{nyanu} coreferential with the subject cannot be interpreted as the object of straightening, since that is the role of the NP \textit{kurlarda} 'spear' (which is not a 'part' of the subject's referent, unlike \textit{waku} in \REF{ex:laughren:28b}). The presence of =\textit{nyanu} in \REF{ex:laughren:29b} expresses a relationship of alienable possession between the subject and the object ('spear'). The presence of the DD enclitic =\textit{rla} in \REF{ex:laughren:29c} signals a purpose for which the spear is being straightened.

\ea%29
    \label{ex:laughren:29}

\ea
\label{ex:laughren:29a}
\gll Maja-rnu  kurlarda ({nyanungu-rlu}).\\
    straighten-\textsc{pst}  spear  3-\textsc{erg}\\
\glt `She straightened the spear.'\textsubscript{}


\ex
\label{ex:laughren:29b}
\gll Maja-rnu=nyanu  kurlarda ({nyanungu-ku}).\\
    straighten-\textsc{pst=anaph}  spear  3-\textsc{dat}\\
\glt `He\textsubscript{i} straightened his\textsubscript{i/*j} spear.'

\ex
\label{ex:laughren:29c}
\gll Maja-rnu=nyanu=rla  kurlarda ({nyanungu-ku}).\\
    straighten-\textsc{pst=anaph=DD}  spear  3-\textsc{dat}\\
\glt `He straightened the spear for himself (for some purpose).'
\glt `He\textsubscript{i} straightened his\textsubscript{i/*j} spear (for some purpose).'
\z
\z


Note that in \REF{ex:laughren:29b} and \REF{ex:laughren:29c} the anaphor =\textit{nyanu} may be coreferent with an overt dative marked pronoun (\textit{nyanungu-ku}), whereas in \REF{ex:laughren:28a} and \REF{ex:laughren:28b} \textit{=nyanu} is substituted for an unmarked object NP and cannot be coreferential with an unmarked pronoun.

\section{Reflexive relations within NP}\label{sec:laughren:4}

\subsection{Kin relation propositus anaphor -\textit{nyanu} }\label{sec:laughren:4.1}

Warlpiri employs three distinct syntactic constructions to express the binary relations expressed in English by the genitive construction: possessor in expressions of alienable possession \REF{ex:laughren:30a}, whole in expressions of a part-whole relation \REF{ex:laughren:30b}, and the propositus in kin relation expression \REF{ex:laughren:30c}. Kin terms denote binary relations, e.g., is mother of (x,y). A person may be referred to as a function of their relationship to another/others. The term 'propositus', taken from the anthropological linguistics literature, denotes the person(s) to whom the referent of an expression like 'John's mother' is related by the named kin relation. In this example, \textit{John} is the propositus.

\ea%30

    \label{ex:laughren:30}

\ea
Alienable possession\\
\label{ex:laughren:30a}
\gll {Jakamarra-}\textbf{kurlangu} {kurlarda.}  \\
    J-\textsc{poss}  spear\\
\glt `Jakamarra's spear.'

\ex Part whole\\
\label{ex:laughren:30b}
\gll {Jakamarra =}\textbf{nyanu} {yarnka-ja} \textbf{{jurru-ku}}. \\
    J=\textsc{anaph}  grab-\textsc{pst}  head-\textsc{dat}\\
\glt `Jakamarra\textsubscript{i} grabbed hold of his\textsubscript{i/*j} head.'

\ex Propositus\\
\label{ex:laughren:30c}
\gll [Jakamarra\textsubscript{i}-kungati-[\textbf{nyanu}\textbf{\textsubscript{i}}\textsubscript{/*j}]]-{rlu}  {purra-ja}.  \textsc{Kin}\\
    J-\textsc{dat}  mother-\textsc{anaph-erg}  cook-\textsc{pst}\\
\glt `Jakamarra's mother cooked it.'
\z
\z

Unlike the auxiliary anaphoric enclitic pronoun \textit{=nyanu} in \REF{ex:laughren:30b}, the nominal suffix \textit{{}-nyanu} in \REF{ex:laughren:30c} is hosted by a kin relation term \textit{ngati} 'mother' with which it forms a complex nominal which may host case suffixes, as exemplified by the ergative suffix. The syntactic scope of the ergative case extends to the entire NP which includes the dative-marked propositus \textit{Jakamarra-ku} which is coreferential with the anaphoric suffix \textit{{}-nyanu.} In the absence of a propositus phrase such as \textit{Jakamarra-ku} in \REF{ex:laughren:30c}, \textit{{}-nyanu} may be contextually bound and interpreted as 'his/her/its/their mother' or it may have an arbitrary interpretation as in 'the mother' implying 'the mother of someone'.

The anaphoric suffix -\textit{nyanu} contrasts with the special addressee propositus suffix \textit{\nobreakdash-puraji} in (31a-c).\footnote{There is also a speaker referring propositus suffix \nobreakdash-\textit{na} that is not as productive as the second person \nobreakdash-\textit{puraji;} it has been 'absorbed' into some kin term stems.} As shown in \REF{ex:laughren:31b} and \REF{ex:laughren:31c}, the second person kin propositus suffix \textit{{}-puraji} may be coreferential with the second person enclitic pronoun and with the free pronoun that is also coreferential with the enclitic pronoun.

\ea%31
    \label{ex:laughren:31}
\ea
\label{ex:laughren:31a}
\gll  Ngati-puraji. \\
    mother-\textsc{your.kin}\\
\glt `Your mother'

\ex
\label{ex:laughren:31b}
\gll  Ngati-puraji-rli=ngki  nya-ngu  (nyuntu).\\
    mother-\textsc{your.kin}{}-\textsc{erg=2}  see-\textsc{pst}  (you)\\
\glt `Your mother saw you.'

\ex
\label{ex:laughren:31c}
\gll Ngati-puraji=npa  nya-ngu {(nyuntulu-rlu).}\\
    mother-\textsc{your.kin} \textsc{=2s} see-\textsc{pst}  (you-\textsc{erg)}\\
\glt `You saw your mother.'
\z
\z


Unlike \textit{{}-nyanu} which may co-occur with a dative-marked propositus NP with which it is coreferential, the pronominal suffix -\textit{puraji} cannot. Rather a dative marked free pronoun propositus phrase is only compatible with the anaphoric propositus suffix \textit{{}-nyanu} as shown by the contrast between \REF{ex:laughren:32a} and \REF{ex:laughren:32b}.

\ea%32
    \label{ex:laughren:32}
\ea[]{
\label{ex:laughren:32a}
\gll Nyuntu-ku  ngati-\textbf{{nyanu}}.\\
    you-\textsc{dat}  mother-\textsc{anaph}\\
\glt `Your mother'
}

\ex[*]{
\label{ex:laughren:32b}
\gll Nyuntu-ku  ngati-\textbf{{puraji}}.\\
    you-\textsc{dat}  mother-\textsc{your.kin}\\
    }
\z
\z

The alienable possessor marked by the suffix \textit{-nyangu} on pronoun stems, \textit{{}-kurlangu} on other stems as in \REF{ex:laughren:30a}, can also mark a propositus phrase — especially in reference to descending generation kin — in which case coreference between possessive-marked free pronoun and pronominal propositus suffix is grammatical as shown in \REF{ex:laughren:33}.\footnote{For analysis of the syntactic contrast between the dative marked and possessive marked propositus phrase and its relationship to the kin term expression see \citet{Laughren2016}.}

\ea%33
\label{ex:laughren:33}
\gll Nyuntu-nyangu ngati-puraji.\\
you-\textsc{poss} mother-\textsc{your.kin}\\
\glt `Your mother'

\z

\subsection{Set reflexive use of anaphoric \textit{{}-nyanu}}\label{sec:laughren:4.2}

\citet[\S 3.4.3]{Simpson1991} describes another use of the anaphor \textit{nyanu} within a complex nominal expression of the form N-\textit{kari-yi-nyanu}. N-\textit{kari} means 'other N', while \textit{yi} (I gloss here as a ligative (\textsc{lig})) appears to be an old auxiliary base reserved for the expression of binary relations within a complex NP.\footnote{\textsuperscript{} \citet{McConvell1996} has documented auxiliary structures within complex NPs in Mudburra, another Ngumpin-Yapa language.} In \REF{ex:laughren:34a} the implication that the subject referent belongs to the class of \textit{Napaljarri} women is the only interpretation compatible with the dative object. Both 'giver' and 'recipient' belong to this same set. In \REF{ex:laughren:34b} the subject referent may or may not be a Napaljarri; what is presupposed here is that something has been previously given to another woman who is also a Napaljarri.
\ea%27
    \label{ex:laughren:34}
\ea
\label{ex:laughren:34a}
\gll Yi-nyi	ka=rla	Napaljarri-kari-yi-nyanu-ku.\\
    give-\textsc{npst} \textsc{prs.ind=3dat} Napaljarri-other-\textsc{lig-anaph-dat}\\
\glt 'She is giving (it) to another Napaljarri (woman) like herself.'

\ex
\label{ex:laughren:34b}
\gll Yi-nyi	ka=rla	Napaljarri-kari-ki.\\
    give-\textsc{npst} \textsc{prs.ind=3dat} Napaljarri-other-\textsc{dat}\\
\glt 'She is giving it to another Napaljarri.'
\z
\z




In Eastern Warlpiri -\textit{nyanu} in this set reflexive construction contrasts with the use of first and second person pronominal suffixes homophonous with the auxiliary enclitic forms: first person \textit{\nobreakdash-ji} and second person \textit{\nobreakdash-ngku}. In other dialects, -\textit{nyanu} is used irrespective of the subject's features. In \REF{ex:laughren:35} the implication is that both the addressee subject and the dative phrase belong to the set of big-headed creatures. With the second person pronoun \textit{-ngku}, vowel harmony applies so that the ligative is \textit{yu}.

\ea%35
    \label{ex:laughren:35}
\gll Wilypi-pardi-ya=rla jurru-lalykalalyka-kari-yu-\textbf{ngku}-ku.\\
    out-emerge-\textsc{imp=3dat}  head-big-other-\textsc{lig-}\textbf{\textsc{2}}\textsc{{}-dat}\\
\glt `Go out to that other big head like you/yourself!'
\z


The set reflexive relation may also hold between non-subject NPs as in \REF{ex:laughren:36a}. The anaphor may also be present in the subject NP as in \REF{ex:laughren:36b} where it forces the implication that Rocky is also dog.

\ea%36
    \label{ex:laughren:36}

\ea
\label{ex:laughren:36a}
\gll Kurlarda  ka=rna=lu=rla  limi-yirra-rni  kurlarda-kari-yi-\textbf{nyanu}-ku.  \\
spear  \textsc{prs.ind=1s=pl.s=3dat}  add-put-\textsc{npst}.  spear-other-\textsc{lig-}\textbf{\textsc{anaph}}\textsc{{}-dat}\\
\glt `We put spears with other spears like themselves.' (\citealt{Simpson1991}: 184 \#158)

\ex
\label{ex:laughren:36b}
\gll Maliki-kari-yi-\textbf{nyanu}-rlu  nya-ngu  Rocky.\\
    dog-other-\textsc{lig=}\textbf{\textsc{anaph}}\textsc{{}-erg} see-\textsc{pst}  R\\

\glt `Another dog like him\textsubscript{i} saw Rocky\textsubscript{i}.' (\citealt{Simpson1991}: 184 \#159a)
\z
\z

\section{Coreference relations in non-finite clauses}\label{sec:laughren:5}

As there is no auxiliary in non-finite clauses it is not possible to express coreference between subject and non-subject (object, applicative) by means of the auxiliary anaphor \textit{=nyanu}. In most non-finite clauses the understood subject is phonologically null and coreferential with the subject or object (or some other constituent) of the matrix finite clause. A pronoun in the non-finite clause has disjoint reference with the understood or 'controlled' subject of the non-finite clause containing it, as the following examples in \REF{ex:laughren:37} taken from \citet{Simpson1991} demonstrate.\footnote{\textrm{The non-finite clause is set out on the second line of sentences in \REF{ex:laughren:37} and \REF{ex:laughren:38}.}}

\ea%37
    \label{ex:laughren:37}
\ea
\label{ex:laughren:37a}
\gll Ngarrka-ngku ka  kurdu\textsubscript{j} {ngarri-rni} \\
    man-\textsc{erg}  \textsc{prs.ind}  child  tell-\textsc{npst}\\

\gll    nyanungu\textsubscript{*j}-ku  ngapa  yi-nja-ku\\
    3-\textsc{dat}  water  give-\textsc{inf-dat}\\
\glt ` man tells the child to give him (=man/other; ${\neq}$child) water.' (\citealt{Simpson1991}:178 \#150a)\\

\ex
\label{ex:laughren:37b}
\gll Marlu-ngku  ka  Jakamarra\textsubscript{j} nya-nyi\\
    kangaroo-\textsc{erg}  \textsc{prs.ind}  J  see-\textsc{npst}\\

    \gll  nyanungu\textsubscript{*j}-ku  wurru-ka-nja-kurra\\
   3-\textsc{dat}  creep-move-\textsc{inf-objcomp}\\
\glt `kangaroo sees Jakamarra sneaking up on it/him (${\neq}$Jakamarra).' (\citealt{Simpson1991}: 178 \#150b)
\z
\z


As expected, where the subject of the matrix finite clause such as \textit{wati-ngki} in \REF{ex:laughren:38} is coreferential with the understood subject of an embedded non-finite clause, the pronominal object within the non-finite clause cannot be interpreted as coreferential with the matrix subject.

\ea%38
    \label{ex:laughren:38}
\gll {Wati\textsubscript{i}}{{}-ngki}{\textsubscript{i}} ka=lu  yunpa-rni\\
  man-\textsc{erg}  \textsc{prs.ind=pl.s}  sing-\textsc{npst}\\

\gll  nyanungu-rra\textsubscript{*i}  paka-rninja-karra-rlu.\\
  \textsc{3-pl}  hit-\textsc{inf-subcomp-erg}\\

\glt `The men\textsubscript{i} are singing while striking them\textsubscript{*i}.'
\z

To express interclausal coreference relations as in \REF{ex:laughren:39}, two finite clauses are required so that the anaphor is locally bound within its clause by its subject, which can be coreferential (or disjoint) with an NP in the accompanying clause.

\ea%39
    \label{ex:laughren:39}
\gll [Wati-ngki=ka=lu  yunpa-rni]  [kujaka=lu=nyanu  panti-rni].\\
  man-\textsc{erg=prs.ind=pl.s=3pl}  sing-\textsc{npst}  \textsc{comp}\textsc{=pl.s=anaph}  pierce-\textsc{npst}\\
\glt `Men\textsubscript{i/j} are singing while they\textsubscript{i} are striking themselves\textsubscript{i}.'
\z

\section{Special uses of reflexive constructions}\label{sec:laughren:6}

\subsection{Inherent reflexive verbs}\label{sec:laughren:6.1}

Some Warlpiri verbs are only used in a reflexive construction and can be classed as 'inherently reflexive'. One of these is \textit{ngarrpangarrpa-ma-ni} 'to tell lies about' which is illustrated in \REF{ex:laughren:40}, in which the non-subject anaphor =\textit{nyanu} represents a dative applicative argument which must be coreferential with the subject. In \REF{ex:laughren:40b} the presence of an additional dative argument \textit{ngipiri-ki} is also registered by the DD enclitic \textit{=rla} in the auxiliary complex.

\ea%40
    \label{ex:laughren:40}

\ea
\label{ex:laughren:40a}
\gll Ngarrpangarrpa-ma-ni  ka=\textbf{nyanu} {kurdu-ngku}\\
    deceit-\textsc{caus-npst}  \textsc{prs.ind=anaph}  child-\textsc{erg}\\

\gll  {kuja} {kuyu} {nga-rnu.}\\
    \textsc{comp}  meat  eat-\textsc{pst}\\
\glt `The child is lying about (what he did) which was that he ate the meat.'

\ex
\label{ex:laughren:40b}
\gll {Ngarrpangarrpa-ma-nu=}\textbf{nyanu}=rla  ngipiri-ki  yapa-ngku,  palka=jala.\\
    deceit-\textsc{caus-pst=anaph=dd}  egg-\textsc{dat}  person-\textsc{erg}  present=\textsc{cfoc}\\

\glt `The child lied about the eggs — (they are) actually here.'
\z
\z

\subsection{Reflexive construction in inchoative monadic clause}\label{sec:laughren:6.2}

The reflexive constructions discussed in \sectref{sec:laughren:2.2} all involve two arguments with distinct thematic roles, one associated with the subject and the other with the object or applicative function, but with both linked to a single referent. Here I will briefly discuss monadic reflexive constructions in which a single thematic role is expressed by the subject in a clause that is formally reflexive. In Warlpiri these constructions are mainly confined to expressions of change in the internal state of a being (typically human) over which the undergoer has no control. Such a thematic role would be expected to be assigned to the object function. The obligatory non-subject enclitic coreferential with the subject would seem to represent this alignment of thematic role and grammatical function. These constructions are used with agent-patient verbs whose NP subject is marked ergative. In \REF{ex:laughren:41a} the enclitic anaphor \textit{=nyanu} signals coreference with the ergative marked NP subject \textit{yapa-ngku} whose plural number features are marked by the subject enclitic pronoun \textit{=lu}. The ergative-marked \textit{jarda-ngku} functions as an instrumental phrase, specifying the nature of the affect. An alternative construction expressing a similar meaning is shown in \REF{ex:laughren:41b} in which \textit{jarda-ngku} is the subject which brings about a change of state in the object \textit{yapa} whose number features are specified by the third person plural non-subject enclitic \textit{=jana} (cf. \REF{ex:laughren:19a} and \REF{ex:laughren:19b}). The intransitive \REF{ex:laughren:41c} differs from both \REF{ex:laughren:41a} and \REF{ex:laughren:41b} in being stative — not denoting a \textit{change} of state.\footnote{\textrm{The inchoative versus stative distinction exemplified by \REF{ex:laughren:41a} and \REF{ex:laughren:41c} is analogous to the distinction made in French in which the inchoative reflexive} \textrm{\textit{s'endormir}} \textrm{'to fall asleep' contrasts with stative} \textrm{\textit{dormir} }\textrm{{}'to sleep'.}}

\ea%41
    \label{ex:laughren:41}
\ea Inchoative\\
\label{ex:laughren:41a}
\gll Pirdi-pu-ngu=lu=nyanu  yapa-ngku  jarda-ngku. \\
    kill-strike-\textsc{pst=pl.s=anaph}  person-\textsc{erg}  sleep-\textsc{erg}\\
\glt `The people fell asleep.' (Lit. The people did themselves in with sleep.)

\ex Causative\\
\label{ex:laughren:41b}
\gll  Jarda-ngku=jana  yapa  pu-ngu. \\
    sleep-\textsc{erg=3pl}  person  strike-\textsc{pst}\\
\glt `The people were overcome by sleep. /The people became sleepy.'
  (Lit. sleep struck them.)

\ex Stative\\
\label{ex:laughren:41c}
\gll  Jarda  ka=lu  nguna. \\
    sleep  \textsc{prs.ind=pl.s}  lie.\textsc{npst}\\
\glt `They are sleeping/asleep.'
\z
\z


The use of monadic reflexive constructions to express externally caused changes of a person's internal state is also a feature of a special respect register used by initiated men, as shown in \REF{ex:laughren:42a} which contrasts with the 'standard' register sentence in \REF{ex:laughren:42b}.

\ea%42
    \label{ex:laughren:42}
\ea
\label{ex:laughren:42a}
\gll  Kati-ka=rra=ngku   lipakarra-rlu=lku!\\
  press\_on-\textsc{imp-away=2}  sleep\textsc{{}-erg=}now\\
\glt `Go off to sleep now.' (Lit. press down on yourself with sleep now) [HN59]

\ex
\label{ex:laughren:42b}
\gll  Jarda=lku   nguna-ka=rra!\\
 sleep=now  lie-\textsc{imp=away}\\
\glt `Go off to sleep now.'
\z
\z


It is especially emotional states that are expressed by a monadic reflexive construction in Warlpiri. These typically involve the figurative use of a body part in conjunction with a transitive agent-patient 'affect by contact' verb. In both \REF{ex:laughren:43a} and \REF{ex:laughren:43b} the relevant affected body part NP \textit{miyalu} 'belly/stomach' and the subject of which it is the relevant 'part' are marked by ergative case, in the case-matching structure discussed in \sectref{sec:laughren:3.1} The inchoative 'reflexive' sentences in \REF{ex:laughren:43a} and \REF{ex:laughren:43b} contrast with the stative sentence in \REF{ex:laughren:43c} in which the intransitive verb \textit{nyina} acts as a copula linking the predicate \textit{miyalu maju/warlu} with the first person subject, and allowing the specification of tense and mood features in the auxiliary complex.

\ea%43
    \label{ex:laughren:43}
\ea
\label{ex:laughren:43a}
\gll  Ngaju  ka=rna=ju  miyalu-rlu  yarlki-rni.\\
    I  \textsc{prs.ind=1s=1}  belly-\textsc{erg}  bite-\textsc{npst}\\
\glt `I’m getting really angry.’ (Lit. I am biting myself belly(-wise).)

\ex
\label{ex:laughren:43b}
\gll  Miyalu-rlu  ka=nyanu  pi-nyi  Jungarrayi-rli  miyi-ngirli.\\
    belly-\textsc{erg}  \textsc{prs.ind=anaph}  strike-\textsc{npst}  J-\textsc{erg}  food-\textsc{elat}\\
\glt `Jungarrayi is getting angry over the food.'
  (Lit. Jungarrayi is striking himself belly(wise) on account of the food.)

\ex
\label{ex:laughren:43c}
\gll  Ngaju  ka=rna  nyina  miyalu  maju/warlu.\\
    I  \textsc{prs.ind=1s}  sit  belly  bad/hot\\
 \glt `I am upset/angry.' (Lit. I am sitting stomach bad/hot)
 \z
\z


This aspectual contrast in the domain of emotion verbs, in which the formally reflexive construction signals an inchoative aspect, as opposed to the non-reflexive stative is also found in French: \textit{elle s'est fachée} {}'she got angry' versus \textit{elle est fachée} 'she is angry'. A similar contrast is between the reflexive inchoative \textit{Cécile s'énerve} 'Cécile is getting/gets irritated' and the causative \textit{Cécile énerve Karine} 'Cécile irritates Karine' (Maïa Ponsonnet, personal communication). Where Warlpiri differs from French (and many other languages including Australian ones) is in the restricted domain in which a formal reflexive construction (sometimes referred to as a pseudo-reflexive) signals an externally caused change of state. As noted in \sectref{sec:laughren:3.2}, the inchoative versus causative contrast involving change of state predicates such as 'break' is expressed in Warlpiri by the use of different inflecting verbs (intransitive vs transitive) rather than the contrast between a formal reflexive construction and a non-reflexive transitive one.\footnote{Typical of Australian languages, Warlpiri also has more generalised inchoative and causative inflecting verbs which combine with a predicative nominal, e.g., \textit{walyka-jarri} 'become cool', \textit{walyka-mani} 'make cool').}

\section{Wider perspective}\label{sec:laughren:7}

Warlpiri reflexive constructions within the domain of a tensed clause are marked by a non-subject enclitic pronoun having either identical person features with the subject enclitic or by an anaphor which has no person or number features and which may be an exponent of either accusative or dative case. This type of reflexive (and reciprocal) construction is characteristic of Ngumpin-Yapa languages. In fact \textit{=nyanu} is used in all Ngumpin-Yapa to express coreference and seems to be an innovation which distinguishes this group (\citealt{McConvellLaughren2004}). In some, such as Walmajarri, it replaces all person object enclitic pronouns including first person singular.

This type of reflexive construction is found more widely among Australian languages but it is not the only type of reflexive structure or even the most common. Many Pama-Nyungan languages express a reflexive relation by means of verbal morphology which has a detransitiving function. In fact the Arandic languages spoken to the immediate east of Warlpiri country are of this type. In many languages of eastern Australia the same morphology is also associated with an anti-passive construction. Some languages spoken along the southern part of the Gulf of Carpentaria such as Yanyula, Garrwa and Waanyi have distinct reflexive pronoun forms which replace both the nominative subject and coreferential accusative object. Like other pronouns they distinguish person and number.\footnote{\textrm{See the cross-linguistic account of Australian data, including Warlpiri, from the perspective of reciprocal clauses in Evans} \textrm{\textit{et al.}} \textrm{2007.} }

\section{Abbreviations:}

\begin{tabularx}{.45\textwidth}{>{\scshape}lQ}
1 &                  first person\\
12 &                  first and second person\\
13 &                  first and third person\\
2 &                  second person\\
3 &                  third person\\
\textsc{acc}  &  accusative\\
\textsc{allat} &                  allative\\
\textsc{anaph} &                  anaphor\\
\textsc{ben} &                  benefactive\\
\textsc{caus} &        causative\\
\textsc{cfoc} &      contrastive\\
\textsc{dat} &         dative\\
\textsc{dd} &                  double dative\\
\textsc{du} &                  dual\\
\textsc{elat} &                  elative\\
\textsc{erg} &                  ergative\\
\textsc{fut} &                  future\\
\textsc{imp} &                  imperative\\
\end{tabularx}
\begin{tabularx}{.45\textwidth}{>{\scshape}lQ}
\textsc{impf} &                  imperfective\\
\textsc{inc} &                  inceptive\\
\textsc{ind} &                  indicative\\
\textsc{irr} &                  irrealis\\
\textsc{lig} &                  ligative\\
\textsc{loc} &                  locative\\
\textsc{nom} &                  nominative\\
\textsc{npst} &                  non-past\\
\textsc{objcomp} &           object complementiser\\
\textsc{perl} &                  perlative\\
\textsc{pl} &                  plural\\
\textsc{poss} &                  possessive\\
\textsc{prs} &                  present\\
\textsc{pst} &                  past\\
\textsc{subcomp} &         subject complementiser\\ 
\end{tabularx}

\sloppy\printbibliography[heading=subbibliography,notkeyword=this]
\end{document}