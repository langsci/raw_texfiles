\documentclass[output=paper]{langscibook}

\author{Lea Brown\affiliation{University at Buffalo} \lastand Matthew Dryer\affiliation{University at Buffalo}}

\title{Reflexive constructions in Walman}

\abstract{Walman has two reflexive constructions, one involving a verbal prefix that occurs in the same position as first and second person object prefixes, the other a nominal construction that combines the genitive form of a pronoun with a word \emph{ein}, which otherwise means ‘base of tree’ or ‘reason’. The verbal prefix is also used as a reciprocal construction and the majority of instances of the verbal prefix in texts are either reciprocal or lexicalized.}

\IfFileExists{../localcommands.tex}{
  \usepackage{langsci-optional}
\usepackage{langsci-gb4e}
\usepackage{langsci-lgr}

\usepackage{listings}
\lstset{basicstyle=\ttfamily,tabsize=2,breaklines=true}

%added by author
% \usepackage{tipa}
\usepackage{multirow}
\graphicspath{{figures/}}
\usepackage{langsci-branding}

  
\newcommand{\sent}{\enumsentence}
\newcommand{\sents}{\eenumsentence}
\let\citeasnoun\citet

\renewcommand{\lsCoverTitleFont}[1]{\sffamily\addfontfeatures{Scale=MatchUppercase}\fontsize{44pt}{16mm}\selectfont #1}
   
  %% hyphenation points for line breaks
%% Normally, automatic hyphenation in LaTeX is very good
%% If a word is mis-hyphenated, add it to this file
%%
%% add information to TeX file before \begin{document} with:
%% %% hyphenation points for line breaks
%% Normally, automatic hyphenation in LaTeX is very good
%% If a word is mis-hyphenated, add it to this file
%%
%% add information to TeX file before \begin{document} with:
%% %% hyphenation points for line breaks
%% Normally, automatic hyphenation in LaTeX is very good
%% If a word is mis-hyphenated, add it to this file
%%
%% add information to TeX file before \begin{document} with:
%% \include{localhyphenation}
\hyphenation{
affri-ca-te
affri-ca-tes
an-no-tated
com-ple-ments
com-po-si-tio-na-li-ty
non-com-po-si-tio-na-li-ty
Gon-zá-lez
out-side
Ri-chárd
se-man-tics
STREU-SLE
Tie-de-mann
}
\hyphenation{
affri-ca-te
affri-ca-tes
an-no-tated
com-ple-ments
com-po-si-tio-na-li-ty
non-com-po-si-tio-na-li-ty
Gon-zá-lez
out-side
Ri-chárd
se-man-tics
STREU-SLE
Tie-de-mann
}
\hyphenation{
affri-ca-te
affri-ca-tes
an-no-tated
com-ple-ments
com-po-si-tio-na-li-ty
non-com-po-si-tio-na-li-ty
Gon-zá-lez
out-side
Ri-chárd
se-man-tics
STREU-SLE
Tie-de-mann
} 
  \togglepaper[1]%%chapternumber
}{}


\begin{document}
\maketitle
 
%Unknown Author
%July 24, 2020 6:56 PM
\section{Introduction}\label{sec:Brown:1}

In this paper, we discuss two reflexive constructions in Walman, a language in the Torricelli family spoken on the north coast of Papua New Guinea. One of these constructions is a verbal strategy; it involves a verbal prefix in the same position in the verb as first and second person object prefixes. The other construction is a nominal strategy and involves the genitive form of a personal pronoun followed by the word \emph{ein} ‘base (of tree), reason’. In \sectref{sec:Brown:2}, we give a brief overview of Walman morphology. In \sectref{sec:Brown:3}, we describe the verbal reflexive construction. In \sectref{sec:Brown:4}, we discuss lexicalized instances of the verbal reflexive construction. In \sectref{sec:Brown:5}, we describe the nominal reflexive construction. And in \sectref{sec:Brown:6}, we illustrate uses of the nominal reflexive construction as a marker of focus.


\section{Brief overview of Walman morphology}\label{sec:Brown:2}%orig 1, started counted at 0

  Walman verb morphology involves subject prefixes, object affixes, an applicative affix and a largely obsolete imperative construction.  In \REF{ex:Brown:1}, for example, all four verbs illustrate the \textsc{1sg} subject prefix \emph{m}{}-, while the verb \emph{maltawron} ‘I look for him’ also illustrates the \textsc{3sg.m} object suffix -\emph{n} and the verb \emph{mare} ‘I ask her’ (part of an idiom \emph{esi} \emph{are} ‘meet, encounter’) illustrates the null \textsc{3sg.f} object suffix.

\ea%1
    \label{ex:Brown:1}
    \gll Kum  pe  m-altawro-n  runon,  m-orou  m-esi  m-are-ø  chuto.\\
    \textsc{1sg}  still  \textsc{1sg}{}-look.for-\textsc{3sg.m}  \textsc{3sg.m}  \textsc{1sg}{}-go  \textsc{1sg}{}-arrive  \textsc{1sg}{}-ask-\textsc{3sg.f}  woman\\
    \glt ‘I was still looking for him when I met a woman.’
    \z 


For the majority of transitive verbs, the third person object affixes are suffixes, like \emph{{}-n} in \REF{ex:Brown:1}. However, for a minority of verbs, they are infixes, like the \textsc{3pl} object infix \emph{{}-y-} in \emph{kayko} ‘we eat them’ in \REF{ex:Brown:2}.

\ea%2
    \label{ex:Brown:2}
    \gll Kipin  mon  k-a<y>ko  wuem  alikiel.\\
   \textsc{1pl}  \textsc{neg} \textsc{1pl}{}-eat<\textsc{3pl}>  fish  gills\\
    \glt‘We don’t eat the gills of a fish.’\\
    \z


The first and second person object affixes are prefixes that follow the subject prefixes, like the first person object prefix \emph{p}{}- in \emph{npaltawro} ‘He looked for me/us’ in \REF{ex:Brown:3}.

\ea%3
    \label{ex:Brown:3}
    \gll Runon  n-arau  n-p-altawro  kum  m-ch-a.\\
        \textsc{3sg.m}  \textsc{3sg.m}{}-go.up  \textsc{3sg.m}{}-\textsc{1obj}{}-look.for  \textsc{1sg}  \textsc{1sg}{}-\textsc{2obj}{}-and\\
    \glt ‘He came up and looked for us.’
    \z

The first and second person object prefixes code person but not number. Example \REF{ex:Brown:3} also illustrates the second person object prefix \emph{ch}{}- in the form \emph{mcha} ‘me and you’, and also illustrates the use of a verb -\emph{a} for ‘and’ in Walman, where the first conjunct is the subject of the \emph{and}{}-verb and the second conjunct is the object (\citealt{BrownDryer2008}). \tabref{tab:Brown:1} lists the form of the subject and object affixes.

\begin{table}
    \centering
  \begin{tabularx}{0.5\textwidth}{lll}
  \lsptoprule
Subject & Prefixes  & Object Affixes\\
\hline
{\textsc{1sg}}  & m- &  \multirow{2}{*}{p-}\\
{\textsc{1pl}}  & k- &  \\
{\textsc{2sg}}  & n- & \multirow{2}{*}{ch-}\\
{\textsc{2pl}}  & ch- &  \\
{\textsc{3sg.f}} & w- & {}-ø\\
{\textsc{3sg.m}} & n- & {}-n\\
{\textsc{3sg.dim}}  & l- & {}-l\\
{\textsc{3pl}} &  y- & {}-y\\
\lspbottomrule
\end{tabularx}
\caption{Subject and object affixes}\label{tab:Brown:1}

\end{table}

 Walman has an applicative construction that usually has either benefactive or external possession meaning, the former illustrated in \REF{ex:Brown:4}, the latter in \REF{ex:Brown:5}. In \REF{ex:Brown:4}, for example, the verb \emph{nayawron} bears a \textsc{3sg.m} subject prefix \emph{n}{}-, an applicative suffix \emph{{}-ro}, and a \textsc{3sg.m} object suffix\emph{ -n} indexing the applied object.\footnote{ The regular form of the applicative suffix is \emph{-re {\textasciitilde} -ro}, the choice between these based on vowel harmony. Some applicative forms are irregular, like the stem -\emph{narin} in \REF{ex:Brown:6} below.}


\ea%4
    \label{ex:Brown:4}
    \gll Runon  \textbf{n-ayaw-ro-n}  nyi.\\
       \textsc{3sg.m}  \textbf{\textsc{3sg.m-}}\textbf{light.fire\textsc{{}-appli}} \textbf{\textsc{c-3sg.m}}  \\
    \glt ‘He lit a fire for him.’
    \z

          
 %
%Please follow the Leipzig Glossing Rules.” Applicative” should be APPL
%Unknown Author
%July 27, 2020 5:41 PM
  

\ea%5
    \label{ex:Brown:5}
    \gll Kum  \textbf{m-aram-re-n}  kayal  runon\\
        \textsc{1sg}  \textbf{\textsc{1sg}}\textbf{{}-step.on-}\textbf{\textsc{applic}}\textbf{{}-}\textbf{\textsc{3sg.m}}  foot  \textsc{3sg.m}\\
    \glt  ‘I stepped on his foot.’
    \z

The applicative construction is the only way to express a benefactive in Walman. Most applicative verbs in Walman are applicatives of transitive verbs. Applicatives of intransitive verbs do not have benefactive or external possession meanings, but simply add an argument. For example, the applicative of the intransitive verb for ‘speak’ adds a object denoting the addressee, as in \REF{ex:Brown:6}.

\ea%6
    \label{ex:Brown:6}
    \gll Ngan  \textbf{n-p-narin}  komunngan  kipin.\\
         father  \textbf{\textsc{3sg.m-1obj-}}\textbf{speak.\textsc{applic}}  story  \textsc{1pl}\\
    \glt  ‘Father told us a story.’
    \z
    
Applicatives of transitive verbs sometimes inflect for two objects, as in \REF{ex:Brown:7}, where the applied object is indexed by the first person prefix\emph{ p}{}- and the basic object by the third plural suffix \emph{{}-y}.\footnote{ With applicative verbs with two objects, we refer to the object that is not the applied object, the one that corresponds to the object of the corresponding nonapplicative verb, such as \emph{wiey} ‘two’ in \REF{ex:Brown:7}, as the basic object.} 


\ea%7
    \label{ex:Brown:7}
    \gll Chi  \textbf{n-p-olk-ro-y}  wiey  kum.\\
         \textsc{2sg}  \textbf{\textsc{2sg-1obj}}\textbf{{}-pick-}\textbf{\textsc{applic-3pl}}  two  \textsc{1sg}\\
    \glt ‘Pick two for me!’  
    \z

  The only case morphology in the language is genitive case forms of pronouns, illustrated by the forms \emph{wkum} ‘my’ and \emph{wchi} ‘your’ in \REF{ex:Brown:8}.

\ea%8
    \label{ex:Brown:8}
    \gll Chrieu  \textbf{w-kum}  y-ch-arien  nakol  \textbf{w-chi.}\\
        marks  \textbf{\textsc{gen}}\textbf{{}-}\textbf{\textsc{1sg}}  \textsc{3pl}{}-\textsc{2obj}{}-be.at:\textsc{applic}  house  \textbf{\textsc{gen}}\textbf{{}-}\textbf{\textsc{2sg}}\\
    \glt  ‘My books are in your house.’
    \z

These genitive forms are used in the nominal reflexive construction described in \sectref{sec:Brown:4} below, even when the reflexive is not functioning as a possessor. 



 The nongenitive and genitive forms of the personal pronouns are shown in \tabref{tab:Brown:2}. Except for the \textsc{3sg.m} form \emph{mnon}, the genitive forms are formed with a prefix \emph{w}{}-.


\begin{table}
\begin{tabularx}{\textwidth}{XXXp{0.5cm}XX}
 \lsptoprule
 & \multicolumn{2}{c}{Sg} & & \multicolumn{2}{c}{Pl}\\
 \cmidrule{2-3} \cmidrule{5-6}
 & {nongenitive} & {genitive} &  & {nongenitive} & {genitive}\\
 \hline
1  & kum  & wkum    & &  kipin &  wkipin\\
2  & chi & wchi   & &   chim & wchim\\
\textsc{3sg.f}  & ru  & wru   & \multirow{3}{*}{\textsc{3pl}} & \multirow{3}{*}{ri} & \multirow{3}{*}{wri}    \\
\textsc{3sg.m} & runon & mnon & & \\
\textsc{3sg.dim} & rul  & wrul    & &   \\
\lspbottomrule
\end{tabularx}
\caption{Personal pronouns}\label{tab:Brown:2}
\end{table}









\section{The reflexive-reciprocal prefix}\label{sec:Brown:3}%orig 2



 Walman has a reflexive-reciprocal prefix \emph{r-} that occurs in the same position as the first and second person object prefixes, immediately following the subject prefix, as in \REF{ex:Brown:9}, with the verb \emph{{}-eni {\textasciitilde} -enie} ‘to call someone something’.


\ea%9
    \label{ex:Brown:9}
    \gll  Runon  n-r-eni  Matthew.\\
        \textsc{3sg.m}  \textsc{3sg.m}{}-\textsc{refl/r}%
%“Reciprocal” should be RCP according to the Leipzig glossing rules 
%Unknown Author
%July 27, 2020 6:08 PM
\textsc{ecip}{}-call  Matthew\\
    \glt  ‘He calls himself Matthew.’
    \z

Compare \REF{ex:Brown:9} to \REF{ex:Brown:10}, where instead of a reflexive-reciprocal prefix, we have a first person object prefix \emph{p}{}-.

\ea%10
    \label{ex:Brown:10}
    \gll  Runon  n-p-eni  kum  Amos.\\
        \textsc{3sg.m}  \textsc{3sg.m}{}-\textsc{1obj}{}-call  \textsc{1sg}  Amos \\
    \glt   ‘He called me Amos.’
    \z
    
Example \REF{ex:Brown:11} illustrates the same verb with a \textsc{3sg.m} object suffix. 


\ea%11
    \label{ex:Brown:11}
    \gll Kum  m-enie-n  runon  Amos.\\
        \textsc{1sg}  \textsc{1sg}{}-call-\textsc{3sg.m}  \textsc{3sg.m}  Amos\\
    \glt ‘I called him Amos.’
    \z

The form of the stem in \REF{ex:Brown:11} is \emph{{}-enie}, in contrast to the form of stem in \REF{ex:Brown:9} and \REF{ex:Brown:10}, where it is just \emph{{}-eni}. Many Walman verbs use a different stem with object prefixes that is different from the stem used with object suffixes and infixes.



  The three examples in \REF{ex:Brown:12} to \REF{ex:Brown:14} are analogous to those in \REF{ex:Brown:9} to \REF{ex:Brown:11}, except that they involve an applicative verb, namely -\emph{ayakro} ‘to make something for someone (or of someone’s)’, the applicative of -\emph{ayako} ‘make, do, happen to’. Example \REF{ex:Brown:12} illustrates the reflexive/reciprocal prefix \emph{r}{}-, in this case coding the applied object. The verb also exhibits \textsc{3sg.f} agreement with the basic object \emph{nakol} ‘house’.



\ea%12
    \label{ex:Brown:12}
    \gll Runon  n-r-ayak-ro-ø  nakol.\\
 \textsc{3sg.m}  \textsc{3sg.m-refl/recip-}make\textsc{{}-applic-3sg.f}  house\\
    \glt  ‘He built a house for himself.’
    \z


In \REF{ex:Brown:13} is the same verb, but with a first person object prefix \emph{p}{}-. The verb also exhibits \textsc{3pl} agreement with the other object \emph{lei} ‘arrow(s)’.


\ea%13
    \label{ex:Brown:13}
    \gll Ngan  n-p-ayak-ro-y  lei  kum.\\
 father  \textsc{3sg.m-1obj-}make\textsc{{}-applic-3pl}  arrow  \textsc{1sg}\\
    \glt ‘Father made arrows for me. ‘
    \z

Example \REF{ex:Brown:14} illustrates the same verb with a \textsc{3sg.m} applied object. With applicative verbs that are applicatives of those verbs that take third person object suffixes (as opposed to infixes), the verb only inflects for the applied object, in \REF{ex:Brown:14} with the \textsc{3sg.m} suffix -\emph{n}.

\ea%14
    \label{ex:Brown:14}
    \gll Kum  m-ayak-ro-n  wako  runon.\\
 \textsc{1sg}  \textsc{1sg}{}-make-\textsc{applic}{}-\textsc{3sg.m}  boat  \textsc{3sg.m}\\
    \glt ‘I made a boat for him.’
    \z


 The reflexive-reciprocal prefix can be used for the recipient of the verb for ‘give’, as in \REF{ex:Brown:15}.


\ea%15
    \label{ex:Brown:15}
    \gll Kum  m-r-erie  oputo  nyukuel  chomchom.\\
 \textsc{1sg}  \textsc{1sg}{}-\textsc{refl/recip}{}-give  yam  food  much\\
    \glt  ‘I gave myself a lot of food.’
    \z

However, the form of the stem here is different from the normal stem of this verb and only occurs with the reflexive-reciprocal prefix. The usual stem for ‘give’ is -\emph{eyie} {\textasciitilde}\emph{ -e}, as in \REF{ex:Brown:16}.


\ea%16
    \label{ex:Brown:16}
    \gll Chi  n-eyie-n  runon  momol?\\
 \textsc{2sg}  \textsc{2sg}{}-give-\textsc{3sg.m}  \textsc{3sg.m}  what\\
    \glt ‘What did you give him?’
    \z

          
The reflexive of this verb is also used for dressing oneself, as in \REF{ex:Brown:17}.


\ea%17
    \label{ex:Brown:17}
    \gll Kamany  y-r-erie  chno  y-akie  porukul.\\
 person  \textsc{3pl}{}-\textsc{refl/recip}{}-give  traditional.dress  \textsc{3pl}{}-dance  dancing\\
    \glt ‘People put on traditional dress and dance.’
    \z

      
  As noted above and illustrated in \REF{ex:Brown:6}, expression of telling in Walman involves the applicative of the verb for ‘speak’ and the addressee can be reflexive, as in \REF{ex:Brown:18}.



\ea%18
    \label{ex:Brown:18}
    \gll Kum  m-r-narin.\\
 \textsc{1sg}  \textsc{1sg-refl/recip-}speak.\textsc{applic}\\
    \glt  ‘I talk to myself.’
    \z


  When the subject of a verb bearing the reflexive-reciprocal prefix is singular, only the reflexive reading is possible. When the subject is plural, sentences are ambiguous (or vague) out of context between a reflexive reading and a reciprocal reading, though in practice, the intended reading of such sentences is more often reciprocal, presumably because reciprocal readings are usually more natural than reflexive readings. In \REF{ex:Brown:19}, for example, the form \emph{yroko} is the reflexive-reciprocal form of the verb -\emph{oko} ‘take’, here meaning ‘marry’, and the intended reading is reciprocal, a reflexive reading not making sense here.

\ea%19
    \label{ex:Brown:19}
    \gll Nyakom  w-ri  ke  \textbf{y-r-oko,}  nyakom  y-awaro-y.  \\
 child.\textsc{pl}  \textsc{gen}{}-\textsc{3pl}  also  \textbf{\textsc{3pl}}\textbf{{}-}\textbf{\textsc{refl/recip}}\textbf{{}-take}  child.\textsc{pl}  \textsc{3pl}{}-be.parent.of-\textsc{3pl}  \\
    \glt ‘Their children also married each other and had children.’
    \z

  We will refer to the reflexive-reciprocal prefix as an object affix because it is in paradigmatic opposition to the first and second person object prefixes, as well as the fact that it codes the fact that the object is identical in reference to the subject. For present purposes, we treat something as an object grammatically if it is coded on the verb with an object affix. We are not aware of any useful criterion for objecthood in Walman other than the possibility of being coded by an object affix.



  Expressions of situations in which someone does something that affects a body part of their own frequently employ the reflexive-reciprocal prefix, as in \REF{ex:Brown:20} and \REF{ex:Brown:21}.


\ea%20
    \label{ex:Brown:20}
    \gll Kum  m-r-ulo  wi.\\
 \textsc{1sg}  \textsc{1sg}{}-\textsc{refl/recip}{}-cut  hand\\
    \glt ‘I cut my hand.’
    \z

\ea%21
    \label{ex:Brown:21}
    \gll Runon  n-r-ata  ngelie.\\
 \textsc{3sg.m}  \textsc{3sg.m-refl/recip-}bite  tongue\\
    \glt ‘He bit his tongue (accidentally).’
    \z

Sentences involving someone doing something that affects someone else’s body part are similar, with the verb exhibiting object inflection for the person, number and gender of the individual whose body part is affected, as in \REF{ex:Brown:22}.


\ea%22
    \label{ex:Brown:22}
    \gll  Ru  w-p-ulo  woruen.\\
 \textsc{3sg.f}  \textsc{3sg.f}{}-\textsc{1obj}{}-cut  hair\\
    \glt  ‘She cut my hair.’
    \z

In \REF{ex:Brown:22}, the noun \emph{woruen} ‘hair’ is not the object, but a type of nonobject complement, the object being expressed by the first person object prefix on the verb. Similar comments apply to \emph{wi} ‘hand’ in \REF{ex:Brown:20} and \emph{ngelie} ‘tongue’ in \REF{ex:Brown:21}.

  Expressions of washing are more complex. First, there is an intransitive verb \emph{okorue} \emph{{\textasciitilde} -korue} that denotes only washing oneself, without reflexive-reciprocal morphology, as in \REF{ex:Brown:23}.


\ea%23
\label{ex:Brown:23}
    \gll  { Kum}  { m-okorue}  { wul.}\\
 { \textsc{1sg}}  { \textsc{1sg}{}-bathe}  { water}\\
    \glt ‘I bathed.’
    \z

This verb normally combines with the noun \emph{wul} ‘water’, as in \REF{ex:Brown:23}. There is also a transitive verb -\emph{ko\_wue} for washing somebody else, as in \REF{ex:Brown:24}, where the subject and object involve distinct participants.\footnote{  The underscore in \emph{{}-ko\_wue} indicates that this is a verb that takes third person object infixes rather than object suffixes and the location of the underscore represents the location of the infix.}


\ea%24
    \label{ex:Brown:24}
    \gll Runon  n-p-kowue  wul  kum.\\
 \textsc{3sg.m}  \textsc{3sg.m}{}-\textsc{1obj}{}-wash  water  \textsc{1sg}\\
    \glt ‘He washed me.’
    \z

This verb can be used with a reflexive-reciprocal prefix, as in \REF{ex:Brown:25}, but expressions of washing oneself in our data usually involve the verb \emph{{}-okorue {\textasciitilde} -korue}, illustrated in \REF{ex:Brown:23} above. 


\ea%25
    \label{ex:Brown:25}
    \gll Kum  m-r-kowue.\\
 \textsc{1sg}  \textsc{1sg}{}-\textsc{refl/recip}{}-wash
        \\
    \glt ‘I washed myself.’
    \z

There is a separate transitive verb \emph{-olo} that is used for washing body parts, without reflexive-reciprocal morphology, illustrated in \REF{ex:Brown:26}, where the body part is object.


\ea%26
    \label{ex:Brown:26}
    \gll Ch-orou  ch-olo-y  motu-kol.\\
 \textsc{2pl}{}-go  \textsc{2pl}{}-wash-\textsc{3pl}  finger-\textsc{pl}\\
    \glt ‘Go and wash your hands.’
    \z

  
 This is one of several verbs used for washing things other than oneself.



  There are relatively few instances in our texts of uses of the reflexive-reciprocal prefix with specifically reflexive meaning. Two examples from texts are given in \REF{ex:Brown:27} and \REF{ex:Brown:28}. In \REF{ex:Brown:27}, \emph{yrsapur} ‘they untangle themselves’ is a form of the verb \emph{{}-sapur} ‘loosen, untangle’.

\ea%27
    \label{ex:Brown:27}
    \gll  Lasi  ru  w-aro-ø  \textbf{y-r-sapur} pra-pra  lasi  ru  w-aro-ø  y-otoplo-n  runon.\\
 immediately  \textsc{3sg.f}  \textsc{3sg.f}{}-and-\textsc{3sg.f}  \textbf{\textsc{3pl}}\textbf{{}-}\textbf{\textsc{refl/recip}}\textbf{{}-untangle}  loose-loose  immediately  \textsc{3sg.f}  \textsc{3sg.f}{}-and-\textsc{3sg.f}  \textsc{3pl}{}-tie-\textsc{3sg.m}  \textsc{3sg.m}\\
    \glt ‘They (literally ‘she and her’) suddenly wriggled free (literally ‘untangled themselves’) and quickly wrapped themselves around him (\emph{literally} ‘tied him’).’
    \z

There are two instances of the reflexive-reciprocal prefix in \REF{ex:Brown:28}, in \emph{nroko} and \emph{wrulo}. While the literal meaning of -\emph{oko} is ‘take’, it is combines in \REF{ex:Brown:28} with \emph{rele} ‘beard’ to mean ‘shave’, so with the reflexive-reciprocal prefix, the meaning is ‘he shaves himself’. 

\ea%28
    \label{ex:Brown:28}
    \gll Ngan  \textbf{n-r-oko}  rele,  nyue  \textbf{w-r-ulo}  woruen.\\
        father  \textbf{\textsc{3sg.m}}\textbf{{}-}\textbf{\textsc{refl/recip}}\textbf{{}-take}  beard  mother  \textbf{\textsc{3sg.-refl/recip}}\textbf{{}-cut}  hair\\
    \glt  ‘The father shaves, the mother trims her hair.’
    \z

 
The uses of the reflexive constructions in \REF{ex:Brown:28} involve body parts, analogous to \REF{ex:Brown:20} to \REF{ex:Brown:22} above.

  Some uses of the reflexive-reciprocal prefix are ones where the subject is semantically both agent and theme but where many languages would not use a reflexive form. For example, the basic meaning of the verb -\emph{a\_pulu} is ‘to spread something around, pour’, as in \REF{ex:Brown:29}.


\ea%29
    \label{ex:Brown:29}
    \gll ...  o  opucha  runon  n-oko-y  \textbf{n-a<y>pulu} alpa-y  alpa-y  y-anan  y-an  chapul.\\
   { } and  thing  \textsc{3sg.m}  \textsc{3sg.m}{}-take-\textsc{3pl}  \textbf{\textsc{3sg.m}}\textbf{{}-spread.around<}\textbf{\textsc{3pl}}> one-\textsc{pl}  one-\textsc{pl}  \textsc{3pl}{}-go.down  \textsc{3pl}{}-be.at  ground\\
    \glt 
  ‘... and he picked up things and spread them around on the ground.’
    \z

In \REF{ex:Brown:30}, this verb is used in its reflexive-reciprocal form, with the meaning ‘to spread oneself around’, but where many languages would simply say something like ‘spread around’, without a reflexive form, even though the subject is both agent and theme.


\ea%30
    \label{ex:Brown:30}
    \gll To  Walman  \textbf{y-r-apulu}  alpa-y  alpa-y  y-ara  y-ara  y-an  cha  w-kipin  eni  k-an  atuko.\\
         then  Walman  \textbf{\textsc{3pl}}\textbf{{}-}\textbf{\textsc{refl/recip}}\textbf{{}-spread.aroud}  one-\textsc{pl}  one-\textsc{pl}  \textsc{3pl}{}-come  \textsc{3pl}{}-come  \textsc{3pl}{}-be.at  place  \textsc{gen}{}-\textsc{1pl}  now  \textsc{1pl}{}-be.at  south\\
    \glt ‘The Walman people had spread out in separate groups all over the area, coming nearer and nearer (to the coast) and settling in the places where we now live.
    \z


Similarly, the verb -\emph{elie} \emph{{\textasciitilde} -eli} ‘throw’ when repeated means ‘to move something back and forth’, as in \REF{ex:Brown:31}


\ea%31
    \label{ex:Brown:31}
    \gll Runon  \textbf{n-elie-n}  \textbf{n-elie-n}  nyanam  n-roul  yie.\\
         \textsc{3sg.m}  \textbf{\textsc{3sg.m}}\textbf{{}-throw-}\textbf{\textsc{3sg.m}}  \textbf{\textsc{3sg.m}}\textbf{{}-throw-}\textbf{\textsc{3sg.m}}  child \textsc{3sg.m}{}-hang  string.bag\\
    \glt ‘He moved his baby son hanging in the string bag back and forth.’
    \z

In \REF{ex:Brown:32}, we find the same repeated verb with the reflexive prefix.

\ea%32
    \label{ex:Brown:32}
    \gll Runon  \textbf{n-r-eli}  \textbf{n-r-eli.}\\
 \textsc{3sg.masc}  \textbf{\textsc{3sg.masc-refl/recip}}\textbf{{}-throw}  \textbf{\textsc{3sg.masc-refl/recip}}\textbf{{}-throw}\\
    \glt  ‘He is swinging (on a swing).’
    \z


Again, the use of the reflexive form in \REF{ex:Brown:32} does involve identity of agent and theme, but many languages would simply express this meaning with something meaning ‘to move back and forth’, without any overt reflexive marking, as in English.



\section{Lexicalized reflexive-reciprocal forms }\label{sec:Brown:4}%orig 3



  There are many instances in which reflexive-reciprocal forms have apparently lexicalized with meanings that are not entirely predictable from the meaning of the verb that they are reflexive-reciprocal forms of (we say “apparently” since some of them might prove to simply be construals of verbs in particular contexts). Example \REF{ex:Brown:32} above illustrates the use of repeating -\emph{elie} \emph{{\textasciitilde} -eli} ‘throw’ with the reflexive-reciprocal prefix to mean ‘to move oneself back and forth’, where the subject is both agent and theme. Example \REF{ex:Brown:33} is similar, but because the subject is inanimate, it is not both an agent and a theme, but only a theme.








\ea%33
    \label{ex:Brown:33}
    \gll Yie  \textbf{w-r-eli}  \textbf{w-r-eli.}\\
 bilum  \textbf{\textsc{3sg.f-refl}}\textbf{{}-throw}  \textbf{\textsc{3sg.f-refl}}\textbf{{}-throw}\\
    \glt ‘The bilum is swinging (e.g., in the wind).’
    \z

   

This use involves removal of the agent role and could be classified as an anticausative use.

  The example in \REF{ex:Brown:34} also illustrates an instance where the semantic role normally associated with the subject of this verb is removed, but in this case the verb cannot be classified as anticausative because the role that is removed is that of a nonagentive experiencer of the verb -\emph{kay} ‘see’, rather than an agent, although there may still be an entailment of an unspecified experiencer, so that an English translation ‘it will be seen’ is natural. 

\ea%34
    \label{ex:Brown:34}
    \gll ...  cha  ru  \textbf{w-r-kay}  w-kipin  \textit{olsem}  ri welimi  wlapum.\\
   { } so.that  \textsc{3sg.f}  \textbf{\textsc{3sg.f-refl/recip}}\textbf{{}-see}  \textsc{gen}{}-\textsc{1pl}  like  \textsc{3pl} younger.sibling.\textsc{pl}  older.sibling:\textsc{pl}\\
    \glt ‘... so that it will be seen that we are just the same as our brothers and sisters.’
    \z


Normally, the subject of a reflexive form of this verb is both experiencer (the one seeing) and stimulus (the one seen), but in \REF{ex:Brown:34}, it is only stimulus.\footnote{ Grammatically, the subject in \REF{ex:Brown:34} is the \textsc{3sg.f} pronoun \emph{ru}, which can be analysed as an expletive subject like \emph{it} in the English translation. Semantically, the stimulus is the clause meaning ‘we are just the same as our brothers and sisters’, as it is in the English translation.}



  A different sort of lexicalization is reflected in \REF{ex:Brown:35}, where the reflexive-reciprocal form of the verb -\emph{e\_risi}, a transitive verb normally meaning ‘to cook by boiling’, means something like ‘to be very ripe, to be beginning to rot’. 



\ea%35
    \label{ex:Brown:35}
    \gll  Mikie  \textbf{w-r-erisi.}\\
 banana  \textbf{\textsc{3sg.f-refl}}\textbf{{}-cook.by.boiling}\\
    \glt ‘The bananas are rotting.’
    \z


The non-reflexive use of this verb is illustrated in \REF{ex:Brown:36}.

\ea%36
    \label{ex:Brown:36}
    \gll To  ngotu  y-ulue-ø  \textbf{y-e<ø>risi}    y-a<ø>ko.\\
 then  coconut  \textsc{3pl}{}-scratch-\textsc{3sg.f}  \textbf{\textsc{3pl}}\textbf{{}-cook.by.boiling<}\textbf{\textsc{3sg.f}}> \textsc{3pl}{}-eat<\textsc{3sg.f}>\\
    \glt  ‘Then they scraped coconut, boiled it, and ate it.’
    \z

In \REF{ex:Brown:36}, the subject is agent and the object is patient and with a ordinary reflexive verb, the subject would be both agent and patient. But like the verbs illustrated in \REF{ex:Brown:33} and \REF{ex:Brown:34}, the semantic role of agent that the subject would normally have with the verb is removed in \REF{ex:Brown:35}, so that the subject in \REF{ex:Brown:35} is just a patient. But in this case there is also an additional semantic change in that the banana is rotting, not undergoing the change of state associated with being cooked by boiling. 


  A similar example of lexicalization involves the reflexive-reciprocal form of the verb\emph{ -ikie} ‘put’, illustrated in \REF{ex:Brown:37}.

\ea%37
    \label{ex:Brown:37}
    \gll Runon  n-r-ikie  yal  ein  nganu  wiey  o kon  alpa-ø.\\
 \textsc{3sg.m}  \textsc{3sg.m}{}-\textsc{refl/recip}{}-put  breadfruit  tree  sun  two  and night  one-\textsc{f}\\\\
    \glt ‘He was stuck in the breadfruit tree for two days and a night.’
    \z

An example illustrating the non-reflexive use of this verb is given in \REF{ex:Brown:38}.


\ea%38
    \label{ex:Brown:38}
    \gll Chim  ch-p-ikie  kum  m-an  apar.\\
 \textsc{2pl}  \textsc{2pl}{}-\textsc{1obj}{}-put  \textsc{1sg}  \textsc{1sg}{}-be.at  bed\\
    \glt ‘Put me on the bed.’
    \z

          
A literal interpretation of \REF{ex:Brown:37} would be that the man put himself up in the tree, but in the text from which this example comes, the man was put up in the tree by a flock of birds. So, like the preceding examples, the use of the reflexive-reciprocal form in \REF{ex:Brown:37} involves the removal of the agent. However, if that were the only difference, \REF{ex:Brown:37} would simply imply that he was up in the breadfruit tree, but the lexicalized use of this verb more specifically means that he was actually stuck up in the breadfruit tree and had no way to get down. Hence the lexicalization of the reflexivization of this verb also involves an added element of meaning beyond simply the removal of the agent.



  A further example of a verb with lexicalized reflexive-reciprocal forms is the verb \emph{ayako} ‘make, do, cause, happen to’, whose stem with the reflexive-reciprocal prefix is \emph{any}. In fact, the reflexive-reciprocal form of this verb has a number of lexicalized meanings, though we restrict attention here to two of them. The first lexicalized meaning is ‘become’, as in \REF{ex:Brown:39}.

\ea%39
    \label{ex:Brown:39}
    \gll  W-an  w-an,  eni  \textbf{w-r-any}  siar.\\
 \textsc{3sg.fe}%
%Feminine is F in the LGR
%Unknown Author
%July 27, 2020 7:06 PM
\textsc{m}{}-be.at  \textsc{3sg.fem}{}-be.at  now  \textbf{\textsc{3sg.fem}}\textbf{{}-}\textbf{\textsc{refl/recip}}\textbf{{}-make}  reef\\
    \glt ‘And there it [the sago container] remained, until it became a reef.’ 
    \z


Again, this use involves removal of the semantic role that the subject of this verb would normally have (an agent, the maker). But if that were all that was involved, the meaning would be something like ‘the reef came into being’. In \REF{ex:Brown:39}, however, \emph{siar} ‘reef’ is not the subject, the subject (the sago container) being the thing that became a reef.\footnote{ That \emph{siar} ‘reef’ is not subject in \REF{ex:Brown:39} is clear from the fact that it follows the verb. Subjects in Walman invariably precede the verb.}



  A second lexicalized use of the reflexive-reciprocal form of -\emph{ayako} ‘do, make, happen to’ is ‘happen’, illustrated in \REF{ex:Brown:40}.\footnote{ Words in italics, like \emph{orait} in \REF{ex:Brown:40}, are Tok Pisin words from modern texts. Contemporary Walman is frequently a mixture of Walman and Tok Pisin.}


\ea%40
    \label{ex:Brown:40}
    \gll \textit{Orait}  ampa  ru  \textbf{w-r-any}   w-ama eni  nta.\\
 OK  \textsc{fut}  \textsc{3sg.fem}  \textbf{\textsc{3sg.fem}}\textbf{{}-}\textbf{\textsc{refl/recip}}\textbf{{}-make}  \textsc{3sg.fem}{}-like now  this\\
    \glt ‘Well, it should happen like this.’
    \z


This use is clearly related to the non-reflexive use of this verb with the meaning ‘happen to’, illustrated in \REF{ex:Brown:41}.


\ea%41
    \label{ex:Brown:41}
    \gll  Momol  \textbf{w-p-any}  kipin?\\
 what  \textbf{\textsc{3sg.fem}}\textbf{{}-}\textbf{\textsc{1obj}}\textbf{{}-make}  \textsc{1pl}\\
    \glt ‘What could have happened to us?’
    \z

  Although the use of this verb in \REF{ex:Brown:40} is semantically monovalent, it differs from the other lexicalized uses above in that in these other cases, it is the semantic role of the subject that is removed, while with this use of -\emph{rany} meaning ‘happen’, it is the semantic role of the object that is removed (the thing that something happens to) while the semantic role of the subject (the thing that happens) remains the same.



  The last case we will discuss of a lexicalized use of the reflexive-reciprocal prefix is with the verb \emph{awukul} ‘lift’, whose reflexive-reciprocal forms mean ‘jump’, as in \REF{ex:Brown:42}.

\ea%42
    \label{ex:Brown:42}
    \gll Lasi  n-ete-ø  may  w-ama  pino,  lasi \textbf{n-r-awukul}  n-aro-ø  tin  may  akou.\\
 immediately  \textsc{3sg.m}{}-see-\textsc{3sg.f}  rope  \textsc{3sg.f}{}-like  sling  immediately  \textbf{\textsc{3sg.m-refl/recip-}}\textbf{lift}  \textsc{3sg.m}{}-and-\textsc{3sg.f}  tightly  rope  finish\\
    \glt ‘He saw a vine like a sling, so he jumped and grabbed it tightly.’
    \z

          




It is not immediately obvious that this use is lexicalized since one might argue that jumping really is simply lifting oneself.  However, while this might apply to instances of jumping up, it is less obvious that jumping down, as in \REF{ex:Brown:43}, involves lifting oneself, although perhaps even jumping down often initially involves slightly jumping up.




\ea%43
    \label{ex:Brown:43}
    \gll Lasi  runon  \textbf{n-r-awukul}  n-anan  ...\\
 immediately  \textsc{3sg.m}  \textbf{\textsc{3sg.m-refl/recip}}\textbf{{}-lift}  \textsc{3sg.m}{}-go.down  \\
    \glt  ‘He immediately jumped down ...’
    \z

          



 



\section{The nominal reflexive construction}\label{sec:Brown:5} % orig 4



  In addition to the reflexive-reciprocal prefix on the verb, Walman also has a nominal reflexive construction, illustrated in \REF{ex:Brown:44} and \REF{ex:Brown:45}, that involves combining the genitive form of a personal pronoun with the word \emph{ein}, which has a range of meanings, the most basic of which is ‘base (of a tree)’ but which also can mean ‘cause, reason’. In both \REF{ex:Brown:44} and \REF{ex:Brown:45}, the nominal reflexive is functioning as object.


\ea%44
    \label{ex:Brown:44}
    \gll Runon  n-r-ulo  \textbf{mnon}  \textbf{ein.} \\
 \textsc{3sg.m}  \textsc{3sg.m}{}-\textsc{refl/recip}{}-cut  \textbf{\textsc{3sg.m.gen}}  \textbf{\textsc{refl}}\\
    \glt ‘He cut himself.’
    \z
 

\ea%45
    \label{ex:Brown:45}
    \gll  Runon  n-a  nyoko  seylieu  n-r-ao \textbf{mnon}  \textbf{ein.} \\
 \textsc{3sg.m}  \textsc{3sg.m}{}-use  bow  foreigner’  \textsc{3sg.m-refl/recip}{}-shoot \textbf{\textsc{3sg.m.gen}}  \textbf{\textsc{refl}}\\
    \glt ‘He shot himself with the gun.’
    \z


All instances of this construction in our data combine with the reflexive-reciprocal prefix construction when it is an object which is coreferential with the subject, as in \REF{ex:Brown:44} and \REF{ex:Brown:45}. We should also note that the only clear instances in texts of the nominal reflexive construction involve the focus use discussed in the next section. Two further examples illustrating the simultaneous use of the two constructions are given in \REF{ex:Brown:46} and \REF{ex:Brown:49}.

\ea%46
    \label{ex:Brown:46}
    \gll   Runon  n-r-arien  \textbf{mnon}  \textbf{ein}  “M-ayako-ø  momol?”\\
 \textsc{3sg.m}  \textsc{3sg.m}{}-\textsc{refl/recip}{}-ask  \textbf{\textsc{3sg.m.gen}}  \textbf{\textsc{refl}}  \textsc{1sg}{}-do-\textsc{3sg.f}  what\\
    \glt  ‘He asked himself “What should I do?”’
    \z


\ea%47
    \label{ex:Brown:47}
    \gll Runon  n-r-etiki  nyi  \textbf{mnon}  \textbf{ein.}\\
 \textsc{3sg.m}  \textsc{3sg.m-refl/recip}{}-cook  fire  \textbf{\textsc{3sg.m.gen}}  \textbf{\textsc{refl}}\\
    \glt ‘He burnt himself in the fire.’
    \z


Examples \REF{ex:Brown:48} and \REF{ex:Brown:49} are similar, except in these cases, the object is an applied object in an applicative clause. In \REF{ex:Brown:48}, the verb \emph{nroruen} ‘he cried for himself’ is the applicative of an intransitive verb -\emph{oruen} ‘cry’.


\ea%48
    \label{ex:Brown:48}
    \gll Nyue  w-elpete-n  runon  n-r-oruen \textbf{mnon}  \textbf{ein.}\\
 mother  \textsc{3sg.f}{}-quarrel.with-\textsc{3sg.m}  \textsc{3sg.m}  \textsc{3sg.m}{}-\textsc{refl/recip}{}-cry-\textsc{applic} \textbf{\textsc{3sg.m.gen}}  \textbf{\textsc{refl}}\\
    \glt ‘When his mother yelled at him, he cried for himself.’
    \z

In \REF{ex:Brown:49}, the verb \emph{nrapulun} ‘pour it for yourself’ is the applicative of a transitive verb \emph{a\_pulu} ‘pour, spread around’, so the clause contains two objects, the applied object \emph{wchi} \emph{ein} ‘yourself’, indexed on the verb by the reflexive reciprocal prefix \emph{r}{}-, and the basic object \emph{wul} ‘water’.
 

\ea%49
    \label{ex:Brown:49}
    \gll Chi  n-r-a<ø>pulun  wul  \textbf{w-chi}  \textbf{ein.}\\
 \textsc{2sg}  \textsc{2sg-refl/recip}{}-pour.\textsc{applic<3sg.f>}  water  \textbf{\textsc{gen}}\textbf{{}-}\textbf{\textsc{2sg}}  \textbf{\textsc{refl}} \\
    \glt  ‘Pour yourself some water.’
    \z

  The nominal reflexive construction in Walman normally consists of the genitive form of a pronoun followed by the word \emph{ein}. But an alternative to the use of a personal pronoun is a noun phrase consisting of an \emph{and}{}-verb where both conjuncts are pronominal. In \REF{ex:Brown:50}, for example, the nominal reflexive construction is \emph{wru} \emph{waro} \emph{ein}, where \emph{wru} \emph{waro}, literally ‘of her and her’ is functioning like a pronoun denoting the same two women as the subject \emph{ru} \emph{waro} ‘she and her’, where the first conjunct is represented by both the pronoun \emph{ru} and the \textsc{3sg.f} prefix on \emph{waro} and the second conjunct is represented by the null \textsc{3sg.f} object marking on \emph{waro}. Apart from the fact that \emph{wru} is in genitive form, \emph{wru} \emph{waro} is identical to \emph{ru} \emph{waro}. Since the nominal reflexive construction normally involves a personal pronoun followed by \emph{ein}, the use of \emph{wru} \emph{waro} in \emph{wru} \emph{waro} \emph{ein} means that \emph{wru} \emph{waro} is behaving here like a personal pronoun.

 

\ea%50
    \label{ex:Brown:50}
    \gll Ru  w-aro-ø  y-r-apulun  wul  \textbf{w-ru} \textbf{w-aro-ø}  \textbf{ein.}\\
 \textsc{3sg.f}  \textsc{3sg.f}{}-and-\textsc{3sg.f}  \textsc{3pl-refl/recip}{}-pour.\textsc{applic}  water  \textbf{\textsc{gen}}\textbf{{}-}\textbf{\textsc{3sg.f}}
        \textbf{\textsc{3sg.f}}\textbf{{}-and-}\textbf{\textsc{3sg.f}}  \textbf{\textsc{refl}} \\
    \glt ‘The two women poured water on themselves.’
    \z



  It is also possible to use the nominal reflexive construction with possessors, as in \REF{ex:Brown:51}, in which case we do not get the reflexive-reciprocal prefix on the verb.


\ea%51
    \label{ex:Brown:51}
    \gll Kum  m-a<ø>ko  ngu  \textbf{w-kum}  \textbf{ein}  m-apa-ø.\\
    \textsc{1sg}  \textsc{1sg}{}-eat<\textsc{3sg.f}>  excrement  \textbf{\textsc{gen}}\textbf{{}-}\textbf{\textsc{1sg}}  \textbf{\textsc{refl}}  \textsc{1sg}{}-excrete-\textsc{3sg.f}\\
    \glt ‘I was eating my own faeces, which I just excreted.’
    \z 

   
In fact, it is possible to have a reflexive-reciprocal verbal prefix in addition to the nominal reflexive construction on a possessor, if the verb is applicative, since one of the meanings associated with the applicative construction is that of external possession, as in \REF{ex:Brown:52} and \REF{ex:Brown:53}.\footnote{ Note that the verb \emph{nrlrey} in \REF{ex:Brown:52} bears both a reflexive-reciprocal prefix and a \textsc{3pl} object suffix. This object suffix is agreeing with \emph{nyi} ‘fire’, which is pluralia tantum in Walman and always triggers plural agreement.}








\ea%52
    \label{ex:Brown:52}
    \gll Runon  n-r-a<ø>pon  wuel  \textbf{mnon} \textbf{ein}  n-a<ø>ko.\\
 \textsc{3sg.m}  \textsc{3sg.m-refl/recip-}kill\textsc{.applic<3sg.f>}  pig  \textbf{\textsc{3sg.m.gen}} \textbf{\textsc{refl}}  \textsc{3sg.m}{}-eat<\textsc{3sg.f}>\\
    \glt ‘He killed his own pig to eat.’
    \z


 

\ea%53
    \label{ex:Brown:53}
    \gll Runon  n-r-lre-y  nchong  nyi  nakol \textbf{mnon}  \textbf{ein.}\\
 \textsc{3sg.m}  \textsc{3sg.m-refl/recip-}light.fire\textsc{.applic-3pl}  catch  fire  house \textbf{\textsc{3sg.m.gen}}  \textbf{\textsc{refl}}\\
    \glt ‘He set fire to his own house.’
    \z

         

  It is also possible to combine the reflexive-reciprocal prefix with the nominal construction marking a possessor if the thing possessed is a body part and the act denoted by the verb applies both to the referent of the subject and the body part, as in \REF{ex:Brown:54} and \REF{ex:Brown:55}.





\ea%54
    \label{ex:Brown:54}
    \gll Runon  n-r-kay  chkuel  nyamayki  \textbf{mnon}  \textbf{ein.}\\
 \textsc{3sg.m}  \textsc{3sg.m-refl/recip}{}-see  eye  nose  \textbf{\textsc{3sg.m.gen}}  \textbf{\textsc{refl}}\\
    \glt ‘He sees his own face.’
    \z

        
\ea%55
    \label{ex:Brown:55}
    \gll Runon  n-r-ako  motu  \textbf{mnon}  \textbf{ein}\\
 \textsc{3sg.m}  \textsc{3sg.m}{}-\textsc{refl/recip}{}-eat  finger  \textbf{\textsc{3sg.m.gen}}  \textbf{\textsc{refl}}\\
    \glt   ‘He bit his finger.’
    \z


  The possibility of using the nominal reflexive construction more generally on possessors is illustrated by \REF{ex:Brown:56} to \REF{ex:Brown:58}. In \REF{ex:Brown:56}, the possessor \emph{wkipin} \emph{ein} ‘of ourselves’ is modifying the noun \emph{wlroy} ‘desire’, which in turn is the complement of the word \emph{wama}, formally a form of the verb -\emph{ama} ‘be like’, but in an impersonal use since there is no apparent \textsc{3sg.f} trigger for the prefix \emph{w}{}- on \emph{wama} (suggesting that this has become grammaticalized as a preposition).



\ea%56
    \label{ex:Brown:56}
    \gll Kipin  k-oko-y  w-ama  wlroy  w-kipin  ein.\\
 \textsc{1pl}  \textsc{1pl}{}-take-\textsc{3pl}  \textsc{3sg.f}{}-like  desire  \textsc{gen}{}-\textsc{1pl}  \textsc{refl}\\
    \glt ‘We marry them of our own free will.’
    \z

    
  It is also possible for the nominal reflexive to function as a long distance reflexive, but only if it is a possessor in a subordinate clause, coreferential with the subject of the matrix clause. In \REF{ex:Brown:57}, for example, \emph{mnon} \emph{ein} is the possessor of the object in the subordinate clause but refers back to the subject of the matrix clause.



\ea%57
    \label{ex:Brown:57}
    \gll \textbf{Runon}  n-napi  kum  m-ao-n  ngan  \textbf{mnon}  ein.\\
 \textsc{3sg.m}  \textsc{3sg.m}{}-say  \textsc{1sg}  \textsc{1sg}{}-shoot-\textsc{3sg.m}  father  \textsc{3sg.m.gen}  \textsc{refl}\\
    \glt  ‘He said that I shot his father.’
    \z


Similarly, in \REF{ex:Brown:58}, \emph{mnon} \emph{ein} functions as possessor of the subject of the subordinate clause, but refers back to the subject of the matrix clause.


\ea%58
    \label{ex:Brown:58}
    \gll \textbf{Runon}  n-napi  ngan  \textbf{mnon}  ein  n-ao-n  runon\\
 \textsc{3sg.m}  \textsc{3sg.m}{}-say  father  \textsc{3sg.m.gen}  \textsc{refl}  \textsc{3sg.m}{}-shoot-\textsc{3sg.m}  \textsc{3sg.m}\\
    \glt  ‘He said that his very own father shot him.’
    \z


  We have one text example, given in \REF{ex:Brown:59}, in which the antecedent of a nominal reflexive is the subject of the first verb in a sequence of verbs with different subjects and where the clause in which the nominal reflexive occurs is not a subordinate clause. \emph{Mnon} \emph{ein} is the object of \emph{warien} ‘it hit him’ whose subject is the breadfruit which is also subject of the two verbs \emph{wan} ‘it was at’ and \emph{wanan} ‘it went down’ that precede warien ‘it hit him’. But the antecedent of \emph{mnon} \emph{ein} is \emph{runon} ‘\textsc{3sg.m}’ and intervening between \emph{runon} and the verbs whose subject is the breadfruit is another verb \emph{mlko} ‘I broke it off’ with a \textsc{1sg} subject \emph{kum}.




\ea%59
    \label{ex:Brown:59}
    \gll \textbf{Runon}  n-p-narin  kum,  to  kum  m-lko-ø yal  w-an  karwal,  w-anan,  w-arie-n  \textbf{mnon}  \textbf{ein}  woruen  amungko.\\
 \textbf{\textsc{3sg.m}}  \textsc{3sg.m}{}-\textsc{1obj}{}-speak.\textsc{applic}  \textsc{1sg}  then  \textsc{1sg}  \textsc{1sg}{}-break.off-\textsc{3sg.f}  breadfruit  \textsc{3sg.f}{}-be.at  tree.top  \textsc{3sg.f}{}-go.down  \textsc{3sg.f}{}-hit-\textsc{3sg.m} \textbf{\textsc{3sg.m.gen}}  \textbf{\textsc{refl}}  head  bone\\
    \glt ‘He spoke to me and then I picked a breadfruit that was at the top of the tree, and it came down and hit him on the head.’
    \z


Since this is the only example that we have like this, we are not sure what constraints there might be on how far a nominal reflexive can be separated from its antecedent. It is also possible that this is an instance of the focus use of the nominal reflexive discussed in the next section.



  We should note that the nominal reflexive construction is never obligatory for possessors. In \REF{ex:Brown:60}, for example, we get \emph{ngan} \emph{wkum} ‘my father’, not \emph{ngan} \emph{wkum} \emph{ein}, even though it refers back to the subject \emph{kum}.


\ea%60
    \label{ex:Brown:60}
    \gll Kum  m-tkre-n  ngan  w-kum.\\
 \textsc{1sg}  \textsc{1sg-}do.same\textsc{{}-3sg.m}  father  \textsc{gen}{}-\textsc{1sg}\\
    \glt  ‘I do things like my father.’
    \z

          
Similarly, in \REF{ex:Brown:61} we get \emph{cha} \emph{wri} ‘their village’, not \emph{cha} \emph{wri} \emph{ein}, even though it refers back to the subject \emph{ri} \emph{Chnapeli} ‘the Chinapelli’.


\ea%61
    \label{ex:Brown:61}
    \gll Ri  Chnapeli  y-orou  cha  w-ri.\\
 \textsc{3pl}  Chinapelli  \textsc{3pl}{}-go  place  \textsc{gen}{}-\textsc{3pl}\\
    \glt ‘The Chinapelli returned to their own village.’
    \z

          
{\textasciigrave}  The nominal construction can also be used for reciprocal situations, as in \REF{ex:Brown:62}, but again note that the verb contains the reflexive-reciprocal prefix \emph{r-}.



\ea%62
    \label{ex:Brown:62}
    \gll Ri  y-r-ao  \textbf{w-ri}  \textbf{ein.}\\
 \textsc{3pl}  \textsc{3pl-refl/recip}{}-shoot  \textbf{\textsc{gen}}\textbf{{}-}\textbf{\textsc{3pl}}  \textbf{\textsc{refl}}\\
    \glt ‘They shot each other.’
    \z

\section{Focus use of the nominal reflexive construction}\label{sec:Brown:6} %orig5
%According to König and Siemund, and to Haspelmath (this volume) these uses are defined as (self)intensifiers
%Unknown Author
%July 27, 2020 8:12 PM


  Similar to what we find in many other languages, the nominal reflexive construction in Walman is sometimes used as a marker of focus (\citealt{Koenigetal2013}), as in \REF{ex:Brown:63}.

\ea%63
    \label{ex:Brown:63}
    \gll  Walman  mlin  \textbf{w-ri}  \textbf{ein}  y-ayako-ø  woyue.\\
 Walman  true  \textbf{\textsc{gen}}\textbf{{}-}\textbf{\textsc{3pl}}  \textbf{\textsc{refl}}  \textsc{3pl}{}-make-\textsc{3sg.f}  bad\\
    \glt ‘The real Walman themselves made a mistake.’
    \z

When the item in focus is a personal pronoun not functioning as a possessor within a noun phrase, the pronoun occurs either in genitive form, as in \REF{ex:Brown:64}, or in nongenitive form, as in \REF{ex:Brown:65} and \REF{ex:Brown:66}. The focused element in \REF{ex:Brown:64} is the first conjunct of \emph{naro} ‘you (sg.) and her’, which functions, in turn, as the subject of \emph{charul} ‘you (pl.) flee’. 

\ea%64
    \label{ex:Brown:64}
    \gll   Korue,  \textbf{w-chi}  \textbf{ein}  n-aro-ø  ch-arul  ch-ara.\\
 no  \textbf{\textsc{gen}}\textbf{{}-}\textbf{\textsc{2sg}}  \textbf{\textsc{refl}}  \textsc{2sg}{}-and-\textsc{3sg.f}  \textsc{2pl}{}-flee  \textsc{2pl}{}-come\\
        \\
    \glt ‘Nothing, you [you yourself and her] have come here of your own accord (i.e. not through my magic).’
    \z

In \REF{ex:Brown:65}, \emph{kipin} \emph{ein} ‘we ourselves’ is subject.


\ea%65
    \label{ex:Brown:65}
    \gll \textbf{Kipin}  \textbf{ein}  monap  k-ayako-ø  koruen.\\
 \textbf{\textsc{1pl}}  \textbf{\textsc{refl}}  cannot  \textsc{1pl}{}-make-\textsc{3sg.f}  \textsc{neg}\\
    \glt ‘We ourselves are not able to make any.’
    \z

   
In \REF{ex:Brown:66}, \emph{kipin} \emph{ein} is the second conjunct of a conjoined noun phrase functioning as the goal of the verb \emph{wrukuel} ‘run’.


\ea%66
    \label{ex:Brown:66}
    \gll Ri  y-alma  yikiel  w-rukuel  kalway  ro  w-ri  y-an  Prou Wokau  o  \textbf{kipin}  \textbf{ein.}\\
 \textsc{3pl}  \textsc{3pl}{}-die  words  \textsc{3sg.f}{}-run  blood  part  \textsc{gen}{}-\textsc{3pl}  \textsc{3pl}{}-be.at  Prou Wokau  and  \textbf{\textsc{1pl}}  \textbf{\textsc{refl}}\\
    \glt ‘When they die, a message goes around to their blood relations living in Prou or Wokau, and even amongst ourselves,’
    \z

       

In \REF{ex:Brown:67}, \emph{wkipin} \emph{ein} is functioning as a possessor in \emph{ala} \emph{wkipin} \emph{ein} ‘our business’.



\ea%67
    \label{ex:Brown:67}
    \gll Kipin  \textit{save}  k-an  k-uruer  k-r-elpete  wkan a  pa  ala  \textbf{w-kipin}  \textbf{ein}  ...\\
 \textsc{1pl}  know  \textsc{1pl}{}-be.at  \textsc{1pl}{}-fight  \textsc{1pl}{}-\textsc{refl/recip}{}-quarrel.with  later \textsc{ptcl}  that  work  \textbf{\textsc{gen}}\textbf{{}-}\textbf{\textsc{1pl}}  \textbf{\textsc{refl}}\\
\glt  ‘We know that if we quarrel and fight later, that’s our business, [not yours].’
    \z

  As noted above, the only clear instances in texts of the nominal reflexive construction involve the focus use. Because of this and because of the fact that when it is used for the object of a verb, the verb invariably bears the reflexive-reciprocal prefix suggests that the focus use of this construction is basic and raises the possibility that all instances in elicited data are actually instances of the focus use.



\section{Conclusion}\label{sec:Brown:7} %orig 6

  In conclusion, Walman has two reflexive constructions, one involving a verbal prefix which is in paradigmatic opposition to first and second person object prefixes, the other a nominal reflexive construction that combines a personal pronoun with a word \emph{ein}, whose meanings outside this construction include ‘base of tree’ and ‘reason’. Both constructions are also used for reciprocals. The construction with the verbal prefix has also developed idiosyncratic meanings with some verbs. The nominal reflexive construction is also used as a focus construction and in fact it is possible that all instances of the nominal reflexive construction are really instances of focus.

\section*{Acknowledgments}

The field work which provided the data in this paper was funded initially by a Small Grant for Exploratory Research from the National Science Foundation. Later trips were funded by the Max Planck Institute for Evolutionary Anthropology in Leipzig, Germany, by an Endangered Languages Documentation Programme grant from the Hans Rausing Foundation, and by grants from the National Science Foundation (in the United States).

\sloppy\printbibliography[heading=subbibliography,notkeyword=this]

\end{document} 
 
 
%\section*{Abbreviations}
%\begin{tabularx}{.45\textwidth}{lQ}
%...  \\
%...  \\
%
%\begin{tabularx}{.45\textwidth}{lQ}
%...  \\
%...  \\
%


%\section*{Acknowledgements}
%\citet{Nordhoff2018} is useful for compiling bibliographies.

%{\sloppy\printbibliography[heading=subbibliography,notkeyword=this]}
%\end{document}
