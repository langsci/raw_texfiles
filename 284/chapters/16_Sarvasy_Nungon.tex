\documentclass[output=paper,colorlinks,citecolor=brown]{langscibook}
\author{Hannah S. Sarvasy\affiliation{Western Sydney University}}
\title{Reflexive constructions in Nungon}
\abstract{This chapter gives an overview of expression of reflexivity in the Papuan language Nungon of Morobe Province, Papua New Guinea. Nungon has two sets of free personal pronouns, a ‘basic’ set and an ‘emphatic’ set. The emphatic set includes more formal person/number distinctions than the basic set, and is used for various pragmatic effects relating to contrast and focus, as well as for the reflexive relationship between a transitive subject and object, when they are obligatory. Nungon has no set formal marking for reflexive relationships beyond this transitive subject/object coreference, however, with interpretation of reflexivity largely context-dependent for subject/oblique coreference and other coreferential combinations.}
\IfFileExists{../localcommands.tex}{
  \usepackage{langsci-optional}
\usepackage{langsci-gb4e}
\usepackage{langsci-lgr}

\usepackage{listings}
\lstset{basicstyle=\ttfamily,tabsize=2,breaklines=true}

%added by author
% \usepackage{tipa}
\usepackage{multirow}
\graphicspath{{figures/}}
\usepackage{langsci-branding}

  
\newcommand{\sent}{\enumsentence}
\newcommand{\sents}{\eenumsentence}
\let\citeasnoun\citet

\renewcommand{\lsCoverTitleFont}[1]{\sffamily\addfontfeatures{Scale=MatchUppercase}\fontsize{44pt}{16mm}\selectfont #1}
   
  %% hyphenation points for line breaks
%% Normally, automatic hyphenation in LaTeX is very good
%% If a word is mis-hyphenated, add it to this file
%%
%% add information to TeX file before \begin{document} with:
%% %% hyphenation points for line breaks
%% Normally, automatic hyphenation in LaTeX is very good
%% If a word is mis-hyphenated, add it to this file
%%
%% add information to TeX file before \begin{document} with:
%% %% hyphenation points for line breaks
%% Normally, automatic hyphenation in LaTeX is very good
%% If a word is mis-hyphenated, add it to this file
%%
%% add information to TeX file before \begin{document} with:
%% \include{localhyphenation}
\hyphenation{
affri-ca-te
affri-ca-tes
an-no-tated
com-ple-ments
com-po-si-tio-na-li-ty
non-com-po-si-tio-na-li-ty
Gon-zá-lez
out-side
Ri-chárd
se-man-tics
STREU-SLE
Tie-de-mann
}
\hyphenation{
affri-ca-te
affri-ca-tes
an-no-tated
com-ple-ments
com-po-si-tio-na-li-ty
non-com-po-si-tio-na-li-ty
Gon-zá-lez
out-side
Ri-chárd
se-man-tics
STREU-SLE
Tie-de-mann
}
\hyphenation{
affri-ca-te
affri-ca-tes
an-no-tated
com-ple-ments
com-po-si-tio-na-li-ty
non-com-po-si-tio-na-li-ty
Gon-zá-lez
out-side
Ri-chárd
se-man-tics
STREU-SLE
Tie-de-mann
} 
  \bibliography{../localbibliography}
  \togglepaper[16]%%chapternumber
}{}

\begin{document}

\maketitle 

\begin{figure}
    \caption{Linguistic context of the Uruwa River valley (shaded), Morobe Province, Papua New Guinea \citep[7]{Sarvasy2017Grammar}}
    %\includegraphics[width=\textwidth]{figures/sarvasy-fig1.jpg}
    \label{fig:sarvasy:1}
\end{figure}

\section{Introduction}\label{sec:sarvasy:1}

This chapter gives an overview of forms and structures related to the expression of reflexivity in the Papuan language Nungon of Morobe Province, Papua New Guinea. Nungon has no dedicated maker of reflexivity or reflexive pronoun. Instead, reflexivity is one function of the ‘emphatic’ set of personal pronouns.

%\todo[inline]{Single quotation marks are for linguistic meaning only.}

Nungon is a Papuan language of the Finisterre group within the Finisterre-Huon language family, spoken in northeastern Papua New Guinea. Nungon is an umbrella term applied to the southern four village-lects of an oval dialect continuum in the Uruwa River valley (see \figref{fig:sarvasy:1}), of which each village community historically had a distinct dialect. (The northernmost dialects are known collectively as Yau, source of the language code <yuw> that applies to the entire dialect continuum.) Nungon is spoken by approximately 1,000 people, but these are divided among the distinct dialects, such that there are no more than about 350 speakers of each dialect. All data and discussion in this chapter use the Towet village dialect. 

A full overview of Nungon grammar is in the Nungon reference grammar \citep{Sarvasy2017Grammar}.\footnote{Nungon quantification is discussed in full in \citet{Sarvasy2017Quant}; imperatives and commands are covered in \citet{Sarvasy2017Commands}; linguistic history and comparative structures is in \citet{Sarvasy2013, Sarvasy2014Finisterre}; more anthropological linguistic detail on covert expression of gender and secret language are in \citet{Sarvasy2016Gender, Sarvasy2019Taboo}.}  Some additional phonetic and phonological details are in \citet{SarvasyEtAl2019Vowel, SarvasyEtAl2019Analysis, SarvasyEtAl2020}. Nungon is an agglutinating language with some fusion. Constituent order is verb-final. Grammatical relations are indicated through indexation on the verb and through postpositions. There is no grammatical gender \citep{Sarvasy2016Gender}. Nungon number marking includes several ‘splits’ \citep{Sarvasy2018Multiple}, with different areas of the grammar involving different number systems. 

Like many Papuan languages, Nungon is a clause chaining language \citep{Sarvasy2015Imperative, Sarvasy2020Chains}, with several non-finite verb inflections that lack tense, mood, and, sometimes, subject person/number information, which typically serve as non-final members of clause chains or multi-verb predicates \citep{Sarvasy2020Predicates}. Finite verb inflections obligatorily mark subject person/number, distinguishing seven forms (second person dual always has an identical form to third person dual, and the same goes for second person plural and third person plural). A sub-class of 15 transitive verbs, most with prototypically human object arguments, also obligatorily mark object number or person/number through a verbal prefix.   

This chapter primarily draws on the author’s monolingual (Nungon-only; see \citealt{Sarvasy2016Fieldwork}) immersion fieldwork on Nungon grammatical structures over a total of nine months (2011-2013), during which a 140,700-word corpus of Nungon natural speech was created.\footnote{The Nungon corpus is archived in full with the Firebird Foundation. Individual components of the corpus may be obtained through written correspondence with the author. Open-access samples of Nungon natural speech are archived with CHILDES, at: \url{https://childes.talkbank.org/access/Other/Nungon/Sarvasy.html}.}  The corpus contains transcribed audio- and/or video-recorded texts (mostly narratives, but including some dialogues, procedural texts, and songs; from over 40 adult consultants), as well as the author’s transcriptions of non-recorded Nungon natural speech from observation and elicitation in the field. This chapter is also informed by the author’s continued involvement since 2015 with the Towet village community to document child acquisition of the Nungon language there (\citealt{Sarvasy2019Root, Sarvasy2020Predicates, Sarvasy2020Chains}); the two much-larger corpora of child-adult conversational interactions are not used in the present chapter. Where examples in this chapter come from texts or elicitation sessions from the Nungon adult corpus, the name of the text is given as an attribution after the example.  

\section{Nungon personal pronouns overview}\label{sec:sarvasy:2}

Like many other Finisterre-Huon languages \citep[21]{McElhanon1973}\todo{Missing from the bibliography}, Nungon has two personal pronoun paradigms, forming a ‘basic’ pronoun set and an ‘emphatic’ set (full discussion of personal pronouns is in \citealt[351--359]{Sarvasy2017Grammar}). The term ‘emphatic’ is used in deference to the tradition in Finisterre-Huon linguistics (e.g. \citealt{McElhanon1973}), though ‘self-intensifier’ cul be applicable (Haspelmath, Ch. 1, this volume)\todo{Link to be added}. Both sets combine with grammatical relation-marking postpositions to express agency, instrument, possession, location, and accompaniment. Third person pronouns in both sets can occur with reference to inanimate objects.

Formally, the Nungon basic set includes reduced person/number distinctions compared to the emphatic set, as seen in \tabref{tab:sarvasy:1}. While the emphatic paradigm includes the maximal nine distinct forms for the nine possible person/number categories, the basic paradigm includes only five distinct forms. Comparison with related Finisterre languages Nukna and Nek implies that Nungon first and second person basic pronouns could have originally included distinct forms for dual number, \textit{*not} (1du) and \textit{*hot} (2du), but that these have phased out over time, with the original plural (${\geq}$3) forms \textit{non} and \textit{hon} generalizing to encompass dual number as well (\citealt{Sarvasy2017Grammar, Sarvasy2018Multiple}). 

\begin{table}[]
    \centering
    \begin{tabularx}{\textwidth}{XXXXXXX} 
    \lsptoprule
    & \multicolumn{2}{c}{ singular} & \multicolumn{2}{c}{ dual} & \multicolumn{2}{c}{ plural}\\
    & basic & emphatic & basic & emphatic & basic & emphatic\\
    \hline
    1 & \textit{nok} & \textit{naga} & \textit{non} & \textit{nori} & \textit{non} & \textit{noni}\\
    2 & \textit{gok} & \textit{gaga} & \textit{hon} & \textit{hori} & \textit{hon} & \textit{honi}\\
    3 & \textit{yu} & \textit{ino} & \textit{yu} & \textit{yori} & \textit{yu} & \textit{yoni}\\
    \lspbottomrule
    \end{tabularx}
    \caption{Nungon personal pronouns}
    \label{tab:sarvasy:1}
\end{table}

Functionally, the basic personal pronouns are unmarked, compared with the emphatic personal pronouns. But use of even the basic personal pronouns is more functionally marked than the absence of any explicit personal pronoun, which is the norm in Nungon discourse. In example \REF{ex:sarvasy:1}, there is no personal pronoun or other noun phrase explicitly encoding the subject argument of the verb, which is indicated through the verbal inflection; here, no focus or contrast is entailed. But in the counterpart \REF{ex:sarvasy:2}, the presence of the basic personal pronoun with reference to the subject argument entails special focus with contrastive effect on the subject argument. Finally, example \REF{ex:sarvasy:3} shows a third option with maximal contrast, achieved through an explicit emphatic pronoun. Since Nungon has no grammatical gender (see \citealt{Sarvasy2016Gender} for more on covert gender marking), in this chapter the unwieldy “s/he” will be avoided by arbitrarily choosing male or female gender for each English free translation of third person singular pronouns and actors.

\ea%1
    \label{ex:sarvasy:1}
    \gll    Ongo-go-k.\\
            go-\textsc{rp-3sg}\\
    \glt    ‘She went.’
\z

\ea%2
    \label{ex:sarvasy:2}
    \gll    Yu ongo-go-k.\\
            \textsc{3.pro}  go-\textsc{rp-3sg}\\
    \glt    ‘She went.’ [contrastive; this particular actor, not one or more others, went]
\z

\ea%3
    \label{ex:sarvasy:3}
    \gll    Ino ongo-go-k.\\
            \textsc{3sg.pro.emph} go-\textsc{rp-3sg}\\
    \glt    ‘She herself went.’ [maximally contrastive; this actor, not one or more others, possibly with special reason or purpose, went]
\z

The emphatic pronouns always occur in focused, contrastive, or reflexive contexts. Because they are inherently focused, they rarely co-occur with the focusing postposition \textit{=ho}, but are attested with almost all other postpositions. The only exception is the genitive postposition \textit{=hon}, since the emphatic personal pronouns have special ‘emphatic possessive’ forms, used in contexts of focused, contrastive, or reflexive possession, marked with the suffix \textit{{}-in} (homophonous with one of the Nungon locative markers, \textit{{}-in}). Use of a genitive emphatic pronoun for contrast is exemplified in \REF{ex:sarvasy:4}: 

\ea%4
    \label{ex:sarvasy:4}
    \gll    Nan-na maa-no X, naga-in maa-na Y.\\
            father-\textsc{1sg.poss}  name-\textsc{3sg.poss}  X  \textsc{1sg.pro.emph-gen}  name-\textsc{1sg.poss} \textsc{Y}\\
    \glt    ‘My father’s name was X, my own name is Y.’ [Waasiöng inoin hatno]
\z

 Examples \REF{ex:sarvasy:5} and \REF{ex:sarvasy:6} show use of the emphatic personal pronouns to highlight the similarity in attributes of two sets of actors, as a special type of contrast. Example \REF{ex:sarvasy:5} includes two personal pronouns. The first actor mentioned is referred to by a basic personal pronoun; the second set of actors are referred to by an emphatic personal pronoun. Here, the basic pronoun occurs in a relatively neutral context, but the emphatic pronoun shows a relationship between the action by the first actor and that of a second set of actors (they all went in the same direction). 

\ea%5
    \label{ex:sarvasy:5}
    \gll    …  nok     e-ng    ngi-yo=gon,    yoni  ongo-gu-ng-an…\\
            {} \textsc{1sg.pro} come-\textsc{dep} \textsc{prox-dem=restr}  \textsc{3pl.pro.emph} go-\textsc{rp-2/3pl-loc}\\
    \glt    ‘…I coming along on this side, where they had gone…’ [Waasiöng inoin hatno]
\z

There is flexibility in the type of pronoun used to refer to the first actor presented in such contexts. For instance, in \REF{ex:sarvasy:6}, the first set of actors in a similar relational context is referred to with an emphatic pronoun, not a basic pronoun (as in \ref{ex:sarvasy:5}):

\ea%6
    \label{ex:sarvasy:6}
    \gll    Noni     ino     bom-mo.\\
            \textsc{1pl.pro.emph}  \textsc{3sg.pro.emph}  semblance-\textsc{3sg.poss}\\
    \glt    ‘We are like Him.’ [Church sermon, field notes]
\z

Note that \REF{ex:sarvasy:6} is a verbless clause; example \REF{ex:sarvasy:7} is another verbless clause. In \REF{ex:sarvasy:7}, the emphatic pronoun is used anaphorically, to refer back to a previously-mentioned tree species.

\ea%7
    \label{ex:sarvasy:7}
    \gll    Ino    wo-rok=gon.\\
            \textsc{3sg.pro.emph}  \textsc{dist-sembl=restr}\\
    \glt    ‘It is that same one.’ [Geisch nanno orugo yup]
\z

The emphatic pronouns (apparently, for all persons and numbers), but not the basic pronouns, can also occur as apparent nominal modifiers after a name or pronoun, similar to English \textit{he himself}. This is shown in \REF{ex:sarvasy:8}.

\ea%8
    \label{ex:sarvasy:8}
    \gll    Dono  oe-no=rot Yupna ongo-go-mok. Op-no, wo-ma-i, Dono ino, Lae ong-un-a.\\
            Dono  woman-\textsc{3sg.poss=comit}  Yupna  go-\textsc{rp-1du} husband-\textsc{3sg.poss} \textsc{dist-spec-top} Dono \textsc{3sg.pro.emph} Lae go-\textsc{ds.3sg-mv}\\
    \glt    ‘Dono’s wife and I went to Yupna. Her husband, that is, Dono himself, having gone to Lae.’ [Rosarin Yupna hain]
\z

Here, the emphatic pronoun in \textit{Dono ino} ‘Dono himself’ follows the name \textit{Dono} without any intervening pause, very similarly to English ‘Dono himself.’ 

\section{Expression of reflexivity in Nungon}\label{sec:sarvasy:3}

Demarcation of reflexivity is a specific sub-function of the Nungon emphatic pronouns. Emphatic pronouns are obligatory for reflexive reading when the transitive subject argument and object argument or oblique argument are coreferential. All person/number combinations are eligible for reflexive readings. Coreference between transitive subjects and objects is discussed in \ref{sec:sarvasy:3.1}, coreference between transitive subjects and other arguments is in \ref{sec:sarvasy:3.2}, and other coreference contexts are covered in \ref{sec:sarvasy:3.3}. Related expressions are in section \ref{sec:sarvasy:4}.

\subsection{Coreferential transitive subject and object}\label{sec:sarvasy:3.1}

As noted above, all Nungon finite verbs index subject argument person/number through verbal suffixes. A closed sub-set of 15 transitive verbs also obligatorily index object person/number through prefixes that are often fused with the verb root. No other verbs index object person/number. In Nungon transitive clauses, the object argument itself may be: a) omitted and understood from context; b) referred to with an explicit noun phrase; c) referred to with a demonstrative or personal pronoun. Example \REF{ex:sarvasy:9} shows a transitive clause with object argument omitted, example \REF{ex:sarvasy:10} shows a transitive clause with explicit object argument, and example \REF{ex:sarvasy:11} shows a transitive clause with basic personal pronoun referring to the object argument.

\ea%9
    \label{ex:sarvasy:9}
    \gll    honggit-ti!\\
            grab-\textsc{imp.2sg}\\
    \glt    ‘Grab it!’ [Field notes]
\z

\ea%10
    \label{ex:sarvasy:10}
    \gll    inowak na-go-mong.\\
            cassava  eat-\textsc{rp-1pl}\\
    \glt    ‘We ate cassava.’ [Rosarin Yupna hain]
\z

\ea%11
    \label{ex:sarvasy:11}
    \gll    nok na-no-ng n-u-ng=ir-a-ng.\\
            \textsc{1sg.pro} \textsc{1sg.o}{}-tell-\textsc{dep} \textsc{1sg.o}{}-roll.side.to.side-\textsc{dep}=be-\textsc{pres.nsg-2/3pl}\\
    \glt    ‘They lie to me.’ [Literally: ‘They address me and roll me from side to side’] [Field notes] 
\z

As seen in \REF{ex:sarvasy:11}, explicit free personal pronouns can co-occur with the object prefixes on verbs in which this is obligatory.

When a transitive subject and object are exactly coreferential, the object is referred to by an emphatic pronoun, as in examples \REF{ex:sarvasy:12} and \REF{ex:sarvasy:13}: 

\ea%12
    \label{ex:sarvasy:12}
    \gll    Ino    wet-do-k.\\
            \textsc{3sg.pro.emph}  \textsc{3sg.o.}kill-\textsc{rp-3sg}\\
    \glt    ‘He killed himself.’ (Field notes)
\z

\ea%13
    \label{ex:sarvasy:13}
    \gll    Amna   inggouk   dogu-hi-k=ko     ino aa-ng-a     it-ta-k.\\
            man  one    ghost-put\textsc{{}-nmz=foc}  \textsc{3sg.pro.emph} \textsc{3sg.o.}see-\textsc{dep-mv}  be-\textsc{pres.sg-3sg}\\
    \glt    ‘One man is looking at himself in a mirror.’ [Picture description task 4]
\z

The emphatic pronouns are necessary for reflexive reading in examples \REF{ex:sarvasy:12} and \REF{ex:sarvasy:13}. But the converse is not true: as noted in \citet[355]{Sarvasy2017Grammar}, representation of the object argument of a transitive verb with an emphatic pronoun does not necessarily entail coreference with the subject argument. The example given in \citet[355]{Sarvasy2017Grammar} is reproduced here as \REF{ex:sarvasy:14}:

\ea%14
    \label{ex:sarvasy:14}
    \gll    Yoiwet=ton  bök   obö-ng-a,  hara  ino we-k.\\
            Yoiwet=\textsc{gen}   house   break-\textsc{dep-mv}   almost   \textsc{3sg.pro.emph} \textsc{3sg.o.}kill-\textsc{np.3sg}\\
    \glt    ‘Yoiwet’s house breaking, it almost killed her.’ [Field notes]
\z

Here, the house (intransitive subject of the first clause, and transitive subject of the second clause) pertains to the person it nearly killed, who is mentioned in the first clause.

In contrast to antagonistic, ‘extroverted’ actions (\citealt[61]{KoenigSiemund1999}), as in \REF{ex:sarvasy:9}, typical ‘introverted’ actions that are expressed using reflexives in some languages take other forms in Nungon. For instance, Nungon \textit{guo-} ‘bathe’ is an intransitive verb, which requires a further transitivizing expansion to express bathing someone else \citep[513--516]{Sarvasy2017Grammar}. In Nungon, ‘introverted’ actions like ‘dress,’ ‘shave,’ and ‘apply make-up’ are expressed with the acted-upon element (a skirt or loincloth, or a possessed body part, see \ref{ex:sarvasy:20}--\ref{ex:sarvasy:22} below) as the transitive object, never exactly coreferential\footnote{Exact coreferentiality here means that two linguistic constituents refer to exactly the same referent. This is important in Nungon because such coreferentiality governs the distribution of switch-reference markers \citep{Sarvasy2015Imperative}. In Nungon switch-reference, body parts are not exactly coreferential with their possessors (the beings to which they belong).} with the transitive subject, as in \REF{ex:sarvasy:15}:

\ea%15
    \label{ex:sarvasy:15}
    \gll    Högök    oe  inggouk  yangam-o  uhok wo=hon  wo=hon  ta-a-k.\\
            white    woman  one    face-\textsc{3sg.poss}  color \textsc{dist=gen}  \textsc{dist=gen} do-\textsc{pres-3sg}\\
    \glt    ‘One Caucasian woman\textsubscript{i} applies make-up here and there to her\textsubscript{i/j} face.’ [Picture description task 6]
\z

In such expressions, the body part is usually marked as possessed in the usual way, without additional marking to show coreference between subject argument and the possessor of the body part. Removed from any particular discourse context, the most natural interpretation of \REF{ex:sarvasy:15} is one of coreferentiality. But if a non-coreferential context had already been introduced (one woman applies make-up to another person’s face), \REF{ex:sarvasy:15} would be acceptable in describing that situation as well. Introduction of the genitive emphatic pronoun to specify that only coreference is an acceptable interpretation would also introduce contrast, as seen in \REF{ex:sarvasy:16}.

\ea%16
    \label{ex:sarvasy:16}
    \gll    Högök    oe  inggouk  ino-in      yangam-o  uhok  wo=hon  wo=hon  ta-a-k.\\
            white    woman  one    \textsc{3sg.pro.emph-gen}  face-\textsc{3sg.poss}  color \textsc{dist=gen}  \textsc{dist=gen}  do-\textsc{pres-3sg}\\
    \glt    ‘One Caucasian woman\textsubscript{i} applies make-up here and there to her\textsubscript{i} own face.’ (Constructed)
\z

The addition of the genitive emphatic pronoun in \REF{ex:sarvasy:16} implies that there are other potential faces to which the woman could be applying make-up, but that the woman is applying it only to her own. In the absence of such a context, \REF{ex:sarvasy:16} is less natural than \REF{ex:sarvasy:15}.

It should further be noted that there are no clear examples in the Nungon adult corpus of ‘inclusive’ co-referentiality between transitive subject and object argument, where the referent of either subject or object is a larger group that includes the referent of the other argument. In Nungon, it is hypothetically possible, but not very natural, to explicitly break down complex groups into a pronoun conjoined with a noun phrase (?\textit{naga orin amna torop} ‘I myself and a group of men’). Thus cases of inclusive reference likely involve use of a single pronoun or a noun phrase (such as \textit{noni} ‘we’ or \textit{amna torop ambarak} ‘the whole group of men’) to describe the larger group. It seems likely that, if a pronoun is used, it would be the emphatic pronoun, but this remains to be tested. 

\subsection{Coreference between transitive subject argument and oblique argument}\label{sec:sarvasy:3.2}

As with coreference between transitive subject and object arguments, emphatic pronouns can be used to indicate coreference between a subject argument and oblique argument. However, unlike with coreference between transitive subject argument and object argument, it is unclear whether the emphatic pronouns are obligatory for obliques; it is likely that here basic pronouns can be substituted for emphatic pronouns with coreference still understood, in the right pragmatic and discourse-contextual circumstances. 

Where used, the emphatic pronouns can be marked with postpositions and preserve the reflexive reading. Example \REF{ex:sarvasy:17} shows coreference between a subject argument and oblique beneficiary, and \REF{ex:sarvasy:18} shows coreference between a subject argument and oblique accompanier:

\ea%17
    \label{ex:sarvasy:17}
    \gll    Hu-ng ino=ha=gon ho-ng na-ng to-ng  it-do-k.\\
            \textsc{nsg.o.}take.away-\textsc{dep} \textsc{3sg.pro.emph=ben=restr} cook-\textsc{dep} eat-\textsc{dep} do-\textsc{dep}  be-\textsc{rp-3sg}\\
    \glt    ‘Taking them away, he used to cook and eat them (just) for himself.’ [Fooyu ketket dogu]
\z

\ea%18
    \label{ex:sarvasy:18}
    \gll    Ni-ingat   h-e-ng-a     ino=rot     n-öö-go-k.\\
            \textsc{1nsg.o}{}-escort  \textsc{nsg.o}{}-come-\textsc{dep-mv}  \textsc{3sg.pro.emph=comit}  \textsc{1.o-}ascend-\textsc{rp-3sg}\\
    \glt    ‘Bringing us, he took us up along with him.’ [Nusek Finsch hat]
\z

Example \REF{ex:sarvasy:17} can be contrasted with \REF{ex:sarvasy:19}, where there is no coreference between the subject and beneficiary:

\ea%19
    \label{ex:sarvasy:19}
    \gll    Tanak   non=ta      h-i-ng.\\
            food  \textsc{1nsg.pro=ben}  cook-\textsc{np-2/3pl}\\
    \glt    ‘They cooked food for us.’ (Constructed)
\z

As the benefactive postposition can be used to mark recipients as well as beneficiaries, the same forms apply in such cases. 

\subsection{Other coreference contexts}\label{sec:sarvasy:3.3}

Reflexive interpretation is further possible in a range of other contexts, either with or without the emphatic pronouns. In these contexts, use of the emphatic pronouns usually entails a combination of reflexivity and contrast.

\subsubsection{Coreference between subject and possessor}\label{sec:sarvasy:3.3.1}

Coreference between the subject argument and possessor referent is not obligatorily indicated through use of the genitive emphatic pronouns, though this is a possibility. All possibilities are illustrated in (\ref{ex:sarvasy:20}-\ref{ex:sarvasy:22}). 

\ea%20
    \label{ex:sarvasy:20}
    \gll    Babiya-no  indar-a    it-ta-k.\\
            book-\textsc{3sg.poss}  read-\textsc{mv}  be-\textsc{pres.sg-3sg}\\
    \glt    ‘She\textsubscript{i} is reading her\textsubscript{i/j} book.’ (Constructed)
\z

\ea%21
    \label{ex:sarvasy:21}
    \gll    Yu=hon  babiya-no  indar-a    it-ta-k.\\
            \textsc{3.pro=gen}  book-\textsc{3sg.poss}  read-\textsc{mv}  be-\textsc{pres.sg-3sg}\\
    \glt    ‘She\textsubscript{i} is reading her\textsubscript{i/j} book.’ (Constructed)
\z

\ea%22
    \label{ex:sarvasy:22}
    \gll    Ino-in      babiya-no  indar-a    it-ta-k.\\
            \textsc{3.pro.emph-gen}  book\textsc{{}-3sg.poss} read\textsc{{}-mv} be\textsc{{}-pres.sg-3sg}\\
    \glt    ‘She\textsubscript{i} is reading her\textsubscript{i} book.’ (Constructed)
\z

All of the three options in (\ref{ex:sarvasy:20}-\ref{ex:sarvasy:22}) allow for reflexive interpretation; \REF{ex:sarvasy:20} is the most functionally unmarked and natural. In \REF{ex:sarvasy:20} and \REF{ex:sarvasy:2118}\todo{21?}, choice of a reflexive interpretation would depend on contextual knowledge. While the reflexive interpretation is the only possibility for \REF{ex:sarvasy:22}, use of the genitive emphatic pronoun there necessarily entails contrast along with reflexivity: either, a) that there are other potential books with different owners available to the reading person, or b) that the reader actually wrote the book herself.

The same options are available for possession of (non-co-referential) human object arguments, as in the Nungon translations of \textit{She killed her friend}, \textit{He saw his boss}, etc. With these, as with (\ref{ex:sarvasy:20}-\ref{ex:sarvasy:22}), the use of a genitive-marked pronoun introduces mild (with the basic pronoun) or strong (with the emphatic pronoun) contrast, as well as, if the emphatic pronoun is used, reflexivity.

\subsubsection{Coreference between subject and location}\label{sec:sarvasy:3.3.2}

In actual discourse, subjects are rarely coreferential with spatial referents. (English \textit{beside her} would be expressed with the comitative postposition \textit{=rot}; \textit{near her} would likely be expressed through the adjective \textit{ambek} ‘near’ alone, without \textit{her}; and \textit{behind him} would be expressed as \textit{mee\nobreakdash-no-n=dek} ‘at his back.’) In one example from the Nungon adult corpus, a speaker uses the locative postposition \textit{=dek} to describe location in discourse. Here, coreferentiality with the subject argument is expressed through use of the emphatic pronoun (marked with the locative postposition).

\ea%23
    \label{ex:sarvasy:23}
    \gll    Amna   maa-no    yo-wang-ka-t,     naga=dek    hi-ng-a,  oruk-na-i=dek.\\
            man  name-\textsc{3sg.poss}  say-\textsc{prob.sg-nf-1sg}   \textsc{1sg.pro.emph=loc}  put-\textsc{dep-mv} brother.of.male-\textsc{1sg.poss-pl=loc}\\
    \glt    ‘I will say the men’s names, starting from myself, on to my brothers.’ [Böas babiya bök]
\z

If the spatial referent were not coreferential with the subject argument and no contrast or focus were desired, either a basic pronoun or noun phrase would stand in for the emphatic pronoun \textit{naga} in \REF{ex:sarvasy:23}.

\subsubsection{Coreference of non-subject arguments}\label{sec:sarvasy:3.3.3}

Coreference of two non-subject arguments is rare-to-non-existent in the Nungon adult corpus. It may be assumed that this is dispreferred in discourse more generally. But if it were to occur, there would likely be three ways of expressing such co-reference, as with coreference of subject argument and possessor. Example \REF{ex:sarvasy:24} shows the absence of any pronoun referring to the recipient argument of ‘show’ (who is also the possessor of ‘her picture’), \REF{ex:sarvasy:25} shows the use of a basic pronoun for the possessor, and \REF{ex:sarvasy:26} shows the use of an emphatic pronoun.     

\ea%24
    \label{ex:sarvasy:24}
    \gll    Dogu-no   y-ande-ha-k.\\
            ghost-\textsc{3sg.poss}  \textsc{3.o-}show-\textsc{pres.sg-3sg}\\
    \glt    ‘She\textsubscript{i} shows her\textsubscript{j} her\textsubscript{i/j} picture.’ (constructed)
\z

\ea%25
    \label{ex:sarvasy:25}
    \gll    Yu=hon  dogu-no   y-ande-ha-k.\\
            \textsc{3.pro=gen}  ghost-\textsc{3sg.poss}  \textsc{3.o-}show-\textsc{pres.sg-3sg}\\
    \glt    ‘She\textsubscript{i} shows her\textsubscript{j} her\textsubscript{i/j} picture.’ (constructed)
\z

\ea%26
    \label{ex:sarvasy:26}
    \gll    Ino-in      dogu-no   y-ande-ha-k.\\
            \textsc{3.pro.emph-gen}  ghost\textsc{{}-3sg.poss  3.o-}show\textsc{{}-pres.sg-3sg}\\
    \glt    ‘She\textsubscript{i} shows her\textsubscript{j} her own\textsubscript{i/j} picture.’ (constructed)
\z

Here, even \REF{ex:sarvasy:26} is still ambiguous, in that the picture could pertain to the showing person or the viewing person. Such ambiguity would be reduced if one of the parties were first or second person, as in \REF{ex:sarvasy:27}:

\ea%27
    \label{ex:sarvasy:27}
    \gll    Ino-in      dogu-no   y-ande-ha-t.\\
            \textsc{3.pro.emph-gen}  ghost\textsc{-3sg.poss} \textsc{3.o-}show\textsc{{}-pres.sg-1sg}\\
    \glt    ‘I show her\textsubscript{i} her own\textsubscript{i/j} picture.’ (constructed)
\z

Here, the picture could still belong to a third party, distinct from the showing and viewing people, but it could not belong to the showing person, who is specified to be 1\textsc{sg}.

\subsubsection{Coreference across clauses}\label{sec:sarvasy:3.3.4}

Coreference across clauses—whether subordinate clauses within other clauses, coordinated independent clauses, or coordinated dependent clauses in chains—is most often not indicated through emphatic pronoun use. With clause chains, in fact, there is another, highly efficient, means to track subject reference across clauses: switch-reference marking. With Nungon switch-reference, any change in subject reference from clause A to clause B within a chain requires that the verb in clause A bears switch-reference marking, even if the referent of clause A’s subject is included within that of clause B, or vice versa. This means that a listener has a clear idea at any time of the co-referentiality of subjects across clauses; while there is no similar grammatical means for tracking object or other argument reference through a clause chain, it stands to reason that knowing the reference of the subject argument of each clause can help in whittling down options for object reference in cases of ambiguity. Nungon switch-reference marking is described in detail in Sarvasy \citet{Sarvasy2015Imperative, Sarvasy2017Grammar}. 

In clause chains, as elsewhere in the language, arguments normally lack expressions such as explicit pronouns or noun phrases if they are deemed recoverable from context. If reflexive and/or contrastive effects are desired, pronouns can be introduced: basic pronouns, for weak contrast, and emphatic pronouns for strong contrast, as in \REF{ex:sarvasy:28}, where a boy shoots at a ghost, but the arrows bounce back at him (referred to by the emphatic pronoun \textit{ino}), instead of hitting the target.

\ea%28
    \label{ex:sarvasy:28}
    \gll    Dogu   tem-un-a      wo-rok,   gun=to   hata-ng ino     hai-ng=gon     to-ng   it-do-k.\\
            ghost  \textsc{3sg.o.}shoot-\textsc{ds.3sg-mv}  \textsc{dist-sembl}  arrow=\textsc{foc}  jump-\textsc{dep} \textsc{3sg.pro.emph}  cut\textsc{{}-dep=restr} do\textsc{{}-dep} be\textsc{{}-rp-3sg}\\
    \glt    ‘He\textsubscript{i} having shot at the ghost, then, the arrow would just jump and strike him\textsubscript{i} (instead).’ [Fooyu ketket orin dogu]
\z

It is in these cross-clausal coreference contexts that Nungon emphatic pronouns indicating reflexivity frequently occur in grammatical subject function. Example \REF{ex:sarvasy:29} is reported speech from a woman observing that, while the person she sought to meet with was not at home, he had left his portable solar charger unattended on a mound beside his house, so he could not have gone very far.

\ea%29
    \label{ex:sarvasy:29}
    \gll    Maa-no     maa-no-no       imbange   orogo   hinom wo-ma-i   ngo-rok   it-ta-k,       ino     ma=ngo-k.\\
            name-\textsc{3sg.poss}  name-\textsc{3sg.poss-3sg.poss}  wonderful  good  \textsc{intens} \textsc{dist-spec-top}  \textsc{prox-sembl}  be-\textsc{pres.sg-3sg}  \textsc{3sg.pro.emph}  \textsc{neg}=go-\textsc{np.3sg}\\
    \glt    ‘His\textsubscript{i} wonderful, very nice stuff is here like this, (so) he himself\textsubscript{i} hasn’t gone.’ [Rosarin Yupna hain]
\z

In \REF{ex:sarvasy:29}, the initial reference to the absent man is as possessor reference, marked with the 3sg possessive suffix \textit{{}-no}, which does not have the possibility to be marked as reflexive or non-reflexive. The second reference to him is then through the emphatic pronoun \textit{ino}, which serves as intransitive subject argument. 

\section{Expanded types of reflexivity in Nungon}\label{sec:sarvasy:4}

The personal pronouns can further combine with three postpositions related to reflexivity. Two of these only co-occur with emphatic pronouns: \textit{=nang,} which relates to physical isolation (‘alone’), and the ‘autoreflexive’ \textit{=wut,} indicating ‘of one’s own power.’ The durative/restrictive \textit{=gon}, which means roughly ‘on one’s own’ when used with personal pronouns, can co-occur with either basic or emphatic pronouns, and indicates a more general type of aloneness than the physical isolation of \textit{=nang} or the use of solely one’s own force, as with \textit{=wut.} 

\section{Summary}\label{sec:sarvasy:5}

In sum, formal marking of reflexivity in Nungon is achieved through use of the emphatic personal pronouns: a second set of personal pronouns with more person/number distinctions than the ‘basic’ set. The emphatic pronouns also function more generally to indicate contrast and focus; reflexivity can be understood to be a restricted sub-category of contrast. 

That said, the Nungon emphatic personal pronouns are only obligatory for indication of reflexivity when the co-reference relation is between the transitive subject and object arguments. In all other contexts, the Nungon discourse style is highly permissive of formal ambiguity, apparently to be resolved by the listener based on discourse-contextual knowledge.

\section*{Abbreviations}

\begin{tabularx}{.45\textwidth}{lQ}
\textsc{1sg}, \textsc{2du}, etc. & person and number\\
\textsc{ben} & benefactive\\
\textsc{comit} & comitative\\
\textsc{dem} & demonstrative\\
\textsc{dep} & dependent\\
\textsc{dist} & distal\\
\textsc{ds} & different-subject\\
\textsc{du} & dual\\
\textsc{emph} & emphatic\\
\textsc{foc} & focus\\
\textsc{gen} & genitive\\
\textsc{imp} & immediate imperative\\
\textsc{intens} & intensifier\\
\textsc{loc} & locative\\
\textsc{mv} & medial verb\\
\textsc{neg} & negative\\
\end{tabularx}
\begin{tabularx}{.45\textwidth}{lQ}
\textsc{nf} & near future\\
\textsc{nmz} & nominalizer\\
\textsc{np} & near past\\
\textsc{nsg} & non-singular\\
\textsc{o} & object\\
\textsc{pl} & plural\\
\textsc{poss} & possessive\\
\textsc{pres} & present\\
\textsc{pro} & pronoun\\
\textsc{prob} & probable\\
\textsc{prox} & proximal\\
\textsc{restr} & restrictive\\
\textsc{rp} & remote past\\
\textsc{sembl} & semblance\\
\textsc{sg} & singular\\
\textsc{spec} & specifier\\
\textsc{top} & topicalizer\\

\end{tabularx}

\sloppy\printbibliography[heading=subbibliography,notkeyword=this]\end{document} 