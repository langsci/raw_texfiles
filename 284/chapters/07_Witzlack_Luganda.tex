\documentclass[output=paper,colorlinks,citecolor=brown,
% hidelinks,
% showindex
]{langscibook}
\author{Alena Witzlack-Makarevich\affiliation{Hebrew University of Jerusalem}\and Erika Just\affiliation{University of Kiel}\orcid{}\lastand Saudah Namyalo\affiliation{Makerere University}\orcid{}}

\title{Reflexive constructions in Luganda}

\abstract{This chapter describes the reflexive construction in Luganda, a Great Lakes Bantu language spoken in Uganda. The reflexive construction in Luganda is formed with the invariable reflexivizer \emph{ee-}, a verbal prefix immediately preceding the stem, which can be reconstructed to Proto-Bantu. There are no reflexive pronouns in Luganda. 
The prefix is obligatorily used to express coreference between the subject and the patient object in transitive verbs and there is no difference between introverted and extroverted verbs. 
The reflexivizer is also employed in case of coreference between an applied beneficiary and the subject. Apart from morphologically and semantically transparent reflexive constructions, Luganda also has a considerable number of fossilized reflexive verbs.}

\IfFileExists{../localcommands.tex}{
 \addbibresource{localbibliography.bib}
 \usepackage{langsci-optional}
\usepackage{langsci-gb4e}
\usepackage{langsci-lgr}

\usepackage{listings}
\lstset{basicstyle=\ttfamily,tabsize=2,breaklines=true}

%added by author
% \usepackage{tipa}
\usepackage{multirow}
\graphicspath{{figures/}}
\usepackage{langsci-branding}

 
\newcommand{\sent}{\enumsentence}
\newcommand{\sents}{\eenumsentence}
\let\citeasnoun\citet

\renewcommand{\lsCoverTitleFont}[1]{\sffamily\addfontfeatures{Scale=MatchUppercase}\fontsize{44pt}{16mm}\selectfont #1}
   
 %% hyphenation points for line breaks
%% Normally, automatic hyphenation in LaTeX is very good
%% If a word is mis-hyphenated, add it to this file
%%
%% add information to TeX file before \begin{document} with:
%% %% hyphenation points for line breaks
%% Normally, automatic hyphenation in LaTeX is very good
%% If a word is mis-hyphenated, add it to this file
%%
%% add information to TeX file before \begin{document} with:
%% %% hyphenation points for line breaks
%% Normally, automatic hyphenation in LaTeX is very good
%% If a word is mis-hyphenated, add it to this file
%%
%% add information to TeX file before \begin{document} with:
%% \include{localhyphenation}
\hyphenation{
affri-ca-te
affri-ca-tes
an-no-tated
com-ple-ments
com-po-si-tio-na-li-ty
non-com-po-si-tio-na-li-ty
Gon-zá-lez
out-side
Ri-chárd
se-man-tics
STREU-SLE
Tie-de-mann
}
\hyphenation{
affri-ca-te
affri-ca-tes
an-no-tated
com-ple-ments
com-po-si-tio-na-li-ty
non-com-po-si-tio-na-li-ty
Gon-zá-lez
out-side
Ri-chárd
se-man-tics
STREU-SLE
Tie-de-mann
}
\hyphenation{
affri-ca-te
affri-ca-tes
an-no-tated
com-ple-ments
com-po-si-tio-na-li-ty
non-com-po-si-tio-na-li-ty
Gon-zá-lez
out-side
Ri-chárd
se-man-tics
STREU-SLE
Tie-de-mann
} 
 \togglepaper[1]%%chapternumber
}{}
  
%move the following commands to the "local..." files of the master project when integrating this chapter


\begin{document}
\maketitle

\section{Introduction}\label{sec:Witzlack:1}

Luganda (or Ganda) is a Bantu language. It belongs to the West Nyanza branch of the Great Lakes Bantu languages of the East Bantu branch (on genealogical classification see \citealt{Schoenbrun1994, Schoenbrun1997}). It is spoken by the Baganda people in a wide area of Uganda including the capital, Kampala (see \figref{fig-map})\footnote{The map is credit to © OpenStreetMap contributors (\url{https://www.openstreetmap.org/copyright}) and \citet{glottolog}.}. 
As of 2014, 5.56 mil.\,of Ugandans identified themselves as being ethnically Baganda \citep{Uganda2016}. In addition to English, Luganda is also used as lingua franca across Uganda (\citealt{IsisngomaEtAl2016}, 
\citealt{Namyalo2016}).

\begin{figure}[!htb]
\begin{center}
  %\includegraphics[width=0.95\linewidth]{figures/Luganda-map.jpg}
  \caption{Luganda on a map of Uganda}\label{fig-map}
\end{center}
\end{figure}

The basic word order of Luganda is SVO, as is the case for the vast majority of Bantu languages, however, information structure considerations motivate various deviations from this basic word order (see e.g.\,\citealt{DowningMarten2019}). 
Nominal and verbal inflectional morphology is primarily prefixing. 
Nominal morphology is characterized by a system of noun class prefixes. 
Each noun in singular and plural belongs to one of the 23 noun classes. The noun classes are numbered from 1 to 23 corresponding to the reconstructed Proto-Bantu noun classes (see e.g.\,\citealt[237–239]{VanDerVeldeEtAl2019}). The nominal prefixes on the nouns are not segmented in the examples, the gloss indicates the inherent noun class in round brackets after the respective noun gloss. For instance, we do not segment the class 2 prefix \emph{ba-} in \emph{abakazi} ‘women’ in~(\ref{ex:Witzlack:1a}) but we indicated that this noun belongs to noun class 2 in the gloss ‘women(2)’. 
Luganda nouns regularly carry an augment, also known as pre-prefix or initial vowel (see e.g.\,\citealt[247–255]{VanDerVeldeEtAl2019}). 
The augment appears before the noun class prefix and has the forms \mbox{\emph{a-},} \emph{o-}, or \emph{e-}, e.g.\,\emph{a-ba-kazi} (\textsc{aug}-2-woman) ‘women’ in~(\ref{ex:Witzlack:1a}). The augment is neither segmented nor glossed in the examples in this paper. The noun class determines the shape of the agreement prefixes on dependents in a noun phrase, on the verb, as well as on a number of other constituents of the clause. We indicate the noun class agreement prefixes on dependents by segmenting them and providing the respective class number in Arabic numerals, as in the case of the subject prefix \emph{ba-} ‘2\textsc{sbj}’ on the verb \emph{ba-n-walan-a} (\textsc{2sbj}-\textsc{1sg.obj}-hate-\textsc{fv}) in~(\ref{ex:Witzlack:1a}). 
Most examples have class 1 or 2 subject agreement prefixes on the verb which index human singular and plural referents respectively. We also use Arabic numerals to indicate person indexing on the verb, as well as person information on pronouns. 
Note that in this case the Arabic numerals are always followed by the indication of number (\textsc{sg} or \textsc{pl}), for instance, \emph{n-} (\textsc{1sg.obj}) in~(\ref{ex:Witzlack:1a}). 
Verbs have multiple slots for inflectional morphology. Prefixes express such inflectional categories as negation, tense and aspect, as well as argument indexing (subject and optionally one or more objects). 
Suffixes express most voice categories, such as the causative and applicative, as well some other inflectional categories, such as aspect, and mood.

Luganda is a tone language and the tone of the reflexive prefix is reported to have different properties than the tone of object prefixes in many Bantu languages (e.g.\,\citealt{Marlo2015}), including closely related ones, such as Nkore \citep{Poletto1998}, 
but it goes beyond the scope of this paper to consider the tonal properties of the Luganda reflexive prefix. 
Tone is not marked in the standard orthography and we omit it from the examples.

The data used in the present study comes primarily from elicitations with two native speakers carried out in 2019–2020. 
They were supplemented with authentic examples from a corpus of naturalistic spoken language (over 50,000 words collected in 2019 in Kampala) and written language (over 200,000 words). 
Each example is indicated as coming from one of these sources with the labels `elicited', `spoken' and `written'.


\section{The reflexive prefix \emph{ee-} and its basic uses}\label{sec:Witzlack:2}

Luganda does not have reflexive pronouns. 
The Luganda reflexive prefix \emph{éé-} (\emph{ee-} in the rest of the paper) is used independently of the person or noun class of the subject. 
It derives from the common Bantu reflexive marker, reconstructed in Proto-Bantu as *\textit{-(j)i-} \citep[109–110]{Meeussen1967}. 
The reflexive marker is a prefix and immediately precedes the verb stem. 
Its position thus differs from all other Luganda affixes used to express the grammatical category of voice (often called \emph{extensions} in Bantu literature), such as applicative, causative, passive and reciprocal, which are suffixes (see e.g.\,\citealt[173]{SchadebergBostoen2019}).

The reflexive prefix \emph{ee-} is obligatorily used when the patient argument of a transitive verb is coreferential with its agent argument in the subject function. 
The examples in~(\ref{ex:Witzlack:1a}) and~(\ref{ex:Witzlack:1b}) have non-coreferential agents and patients. 
In~(\ref{ex:Witzlack:1a}) the pronominal patient is expressed by the pronominal index \emph{n-} (\textsc{1sg.sbj}) in the object slot, whereas in~(\ref{ex:Witzlack:1b}) the nominal patient is expressed by the noun \emph{abalokole} ‘born-again Christians’ following the verb. 
The examples in~(\ref{ex:Witzlack:2}) have coreferential agents and patients and employ the prefix \emph{ee-} in the object slot of the verb. 
As these examples illustrate, the same prefix is used with various person and number categories. Examples in~(\ref{ex:Witzlack:3}) support this point by providing an illustration with a different verb.

%%%%%%examples 1
\ea\label{ex:Witzlack:1}

\ea \label{ex:Witzlack:1a}
\glll Abakazi bampalana.\\
        abakazi		ba-n-walan-a\\
     women(2) \textsc{2sbj}-\textsc{1sg.obj}-hate-\textsc{fv}\\
\glt `Women hate me.' [written]

\ex \label{ex:Witzlack:1b}
\glll Muwalana abalokole.\\
    mu-walan-a	abalokole\\
\textsc{1sbj}-hate-\textsc{fv}	born\_again(2)\\
\glt ‘He hates born-again Christians.’ [spoken] %(ga\_MAK\_190715\_FS03:215)
\z 
\z 

\ea\label{ex:Witzlack:2}

\ea  \label{ex:Witzlack:2a}
\glll Neewalana.\\
    n-ee-walan-a\\
     \textsc{1sg.sbj}-\textsc{refl}-hate-\textsc{fv}\\
\glt ‘I hate myself.’ [elicited]

\ex \label{ex:Witzlack:2b}
\glll Weewalana.\\
    o-ee-walan-a\\
\textsc{2sg.sbj}-\textsc{refl}-hate-\textsc{fv}\\
\glt ‘You hate yourself.’ [elicited]

\ex \label{ex:Witzlack:2c}
\glll{Mukwano gwange yeewalana.}\\
 mukwano		gw-ange		a-ee-walan-a\\
friend(1)		\textsc{1sbj-1sg.poss} 	\textsc{1sbj}-\textsc{refl}-hate-\textsc{fv}\\
\glt ‘My friend hates himself/herself.’ [elicited]

\ex  \label{ex:Witzlack:2d}
\glll Tweewalana.\\
    tu-ee-walan-a\\
\textsc{1pl.}\textsc{sbj}-\textsc{refl}-hate-\textsc{fv}\\
\glt ‘We hate ourselves.’ [elicited]

\ex \label{ex:Witzlack:2e}
\glll Mweewalana.\\
    mu-ee-walan-a\\
\textsc{2pl.}\textsc{sbj}-\textsc{refl}-hate-\textsc{fv}\\
\glt ‘You hate yourselves.’ [elicited]

\ex \label{ex:Witzlack:2f}
\glll Beewalana.\\
 ba-ee-walan-a\\
\textsc{2sbj}-\textsc{refl}-hate-\textsc{fv}\\
\glt ‘They hate themselves.’ [elicited]
\z 
\z 

\ea\label{ex:Witzlack:3}
\ea  \label{ex:Witzlack:3a}
\glll	Neerabye mu ndabirwamu.\\
		n-ee-labye 			mu 		ndabirwamu\\
		\textsc{1sg.sbj}-\textsc{refl}-see.\textsc{pfv}	18.\textsc{loc} 	mirror(9)\\
		\glt	‘I saw myself in the mirror.’ [elicited]

\ex \label{ex:Witzlack:3b}
	\glll John yeerabye mu ndabirwamu.\\
		John 	a-a-ee-labye 				mu		ndabirwamu\\
		John(1)	\textsc{1sbj}-\textsc{pst}-\textsc{refl}-see.\textsc{pfv}		18.\textsc{loc} 	mirror(9)\\
		\glt ‘John saw himself in the mirror.’ [elicited]
\z 
\z 

Following \citet{Haiman1985, KoenigVezzosi2004} we distinguish between introverted verbs, which denote an action typically performed on oneself, such as grooming verbs, and extroverted verbs, which denote an action typically performed on others. 
The Luganda construction with the reflexive prefix \emph{ee-} is used to express autopathic situations with a wide range of extroverted verbs including ‘hate’ in (\ref{ex:Witzlack:2}) above, ‘see’ in (\ref{ex:Witzlack:3}), ‘kill’ in (\ref{ex:Witzlack:4}), ‘bite’ in~(\ref{ex:Witzlack:5}), ‘criticize’ in (\ref{ex:Witzlack:6}), and ‘praise’ in (\ref{ex:Witzlack:7}).

% EXAMPLE 4
\ea\label{ex:Witzlack:4}
	\glll	Omusajja yetta.\\
        	omusajja 	a-ee-tta-a\\
        	man(1)	\textsc{1sbj}-\textsc{refl}-kill-\textsc{fv}\\
	\glt    ‘The man killed himself.’ [elicited]
\z 
% EXAMPLE 5
\ea\label{ex:Witzlack:5}
	\glll   Embwa yeeruma.\\
        	embwa	e-a-ee-rum-a\\
        	dog(9)	\textsc{9sbj}-\textsc{pst}-\textsc{refl}-bite-\textsc{fv}\\
	\glt    ‘The dog bit itself.’ [elicited]
\z
% EXAMPLE 6
\ea\label{ex:Witzlack:6}
	\glll	Peter yeekolokota.\\
        	Peter	a-ee-kolokot-a\\
        	Peter(1) \textsc{1sbj}-\textsc{refl}-critisize-\textsc{fv}\\
	\glt    ‘Peter criticizes himself.’ [elicited]
\z
% EXAMPLE 7
\ea\label{ex:Witzlack:7}
	\glll   Ssaalongo atandika okwewaana nga bwali ssemaka.\\
        	ssaalongo	a-tandik-a 	oku-ee-waan-a nga		bu-a-li			ssemaka\\
        husband(1)	\textsc{1sbj}-start-\textsc{fv}	\textsc{inf}-\textsc{refl}-praise-\textsc{fv} how 14\textsc{sbj}-\textsc{pst}-\textsc{cop} head\_of\_household(1)\\
	\glt   ‘The husband starts to praise himself for being the head of the family.’ [written]%(LU-NEWS-160501-Baze:71)
\z

Introverted actions are expressed either by intransitive verbs or transitive verbs with a reflexive prefix. 
A few intransitive grooming verbs denote situations where the agent and the patient of an action have the same referent. 
These are \emph{naaba} ‘wash (oneself), clean up, bathe’, as in~(\ref{ex:Witzlack:8a}), and \emph{yambala} ‘dress, get dressed’, as in~(\ref{ex:Witzlack:8b}).

% EXAMPLE 8
\ea\label{ex:Witzlack:8}
\ea  \label{ex:Witzlack:8a}
	\glll	Yabadde afulumye okunaaba.\\
        	a-a-badde		a-fulumye		okunaaba\\
		\textsc{1sbj}-\textsc{pst}-\textsc{aux}		\textsc{1sbj}-go\_out.\textsc{pfv}	\textsc{inf}-bathe-\textsc{fv}\\
		\glt	‘She had gone outside to bathe.’ [written] %(LU-NEWS-190109-Abeebijambiya: 4)

\ex \label{ex:Witzlack:8b}
	\glll Omukyala anyirira ayambala bulungi.\\
		omukyala a-nyirir-a		a-yambal-a	bulungi\\
		wife(1)	\textsc{1sbj}-look\_good-\textsc{fv}	\textsc{1sbj}-dress-\textsc{fv}	nicely\\
		\glt ‘The wife looks good, she dresses nicely.’ [spoken] %(ga_MAK_190715_FS04:1184)

\z 
\z

To express other introverted actions, transitive verbs with the reflexive prefix are employed. These include the reflexive \emph{ee-yambula} ‘to undress (oneself)’ derived from the transitive \emph{yambula} ‘undress (somebody), take off (a piece of garment)’, the reflexive \emph{ee-mwa} ‘shave (oneself)’, as in~(\ref{ex:Witzlack:9a}), derived from the transitive \emph{mwa} ‘shave (somebody or something)’, the reflexive \emph{ee-sanirira} ‘comb (one’s hair)’, as in~(\ref{ex:Witzlack:10a}), derived from the transitive \emph{sanirira} ‘comb (e.g.\,hair)’, as well as \emph{ee-naaz-a} ‘wash (oneself)’ in~(\ref{ex:Witzlack:9b}), which is the reflexive of the transitive causative verb \emph{naaza} derived from the intransitive verb \emph{naaba} ‘wash (oneself)’, illustrated above in~(\ref{ex:Witzlack:8a}).

% EXAMPLE 9
\ea\label{ex:Witzlack:9}

\ea  \label{ex:Witzlack:9a}
	\glll	Yeemwa.\\
            a-a-ee-mwa-a\\
		\textsc{1sbj}-\textsc{pst}-\textsc{refl}-shave-\textsc{fv}\\
		\glt	‘He shaved (himself).’ [elicited]

\ex \label{ex:Witzlack:9b}
	\glll Embwa yali yeenaza.\\
	    embwa 	e-a-li		e-ee-naaz-a\\
		dog(9) 	\textsc{9sbj}-\textsc{pst}.be \textsc{9sbj}-\textsc{refl}-wash.\textsc{caus}-\textsc{fv}\\
		\glt ‘The dog was washing itself.’ [elicited]

\z 
\z

\section{Contrast between body-part and whole-body actions}

With most grooming verbs Luganda encodes whole-body actions (washing, bath\-ing, getting a shave, scratching) using the reflexive construction outlined in \sectref{sec:Witzlack:2}, as in~(\ref{ex:Witzlack:10a}),~(\ref{ex:Witzlack:11a}), and~(\ref{ex:Witzlack:12a}). 
Body-part actions (e.g.\,combing or shaving hair or scratching a body part) allow a range of constructions: 
a transitive construction with the respective body part expressed as the object, as in~(\ref{ex:Witzlack:10b}),~(\ref{ex:Witzlack:11b}), and~(\ref{ex:Witzlack:12b}), a reflexive construction with a body-part expressed as an oblique and marked by the locative preposition (nominal class 18) \emph{mu}, as in~(\ref{ex:Witzlack:11c}), and a reflexive construction with a body-part expressed as an object, as in~(\ref{ex:Witzlack:11d}) and~(\ref{ex:Witzlack:12c}). 
The respective body parts in~(\ref{ex:Witzlack:11d}) and~(\ref{ex:Witzlack:12c}) retain at least some of the properties of the morpho-syntactic object: apart from not being flagged, they can be indexed on the verb when fronted, as in~(\ref{ex:Witzlack:11e}). 

% EXAMPLE 10
\ea\label{ex:Witzlack:10}

\ea \label{ex:Witzlack:10a}
	\glll	John yeesaniridde.\\
           John	a-a-ee-saniridde\\
	John(1)	1\textsc{sbj}-\textsc{pst}-\textsc{refl}-comb.\textsc{pfv}\\
		\glt	‘John combed his hair (lit. combed himself).’ [elicited]

\ex \label{ex:Witzlack:10b}
	\glll John yasaniridde enviiri ze.\\
	    John	a-a-saniridde			enviiri		(ze)\\
	John(1)	\textsc{1sbj}-\textsc{pst}-comb.\textsc{pfv}	hair(10)		10.1\textsc{poss}\\
		\glt ‘John combed his hair.’ [elicited]

\z 
\z

% EXAMPLE 11
\ea\label{ex:Witzlack:11}
\ea  \label{ex:Witzlack:11a}
	\glll   Yeetakula.\\
            a-a-ee-takul-a \\
		\textsc{1sbj}-\textsc{pst}-\textsc{refl}-scratch-\textsc{fv}\\
		\glt	‘He scratched himself.’ [elicited]

\ex \label{ex:Witzlack:11b}
	\glll Yatakula omugongo (gwe).\\
	    a-a-takul-a 			omugongo 	gwe\\
		\textsc{1sbj}-\textsc{pst}-scratch-\textsc{fv}	back(3) 		3.1\textsc{poss}\\
		\glt ‘He scratched his back.’ [elicited]

\ex \label{ex:Witzlack:11c}
	\glll Yeetakula mu mugongo.\\
	    a-a-ee-takul-a 			mu 		mugongo\\
	\textsc{1sbj}-\textsc{pst}-\textsc{refl}-scratch-\textsc{fv} 	18.\textsc{loc}	back(3)\\
		\glt ‘He scratched himself on the back.’ [elicited]

\ex \label{ex:Witzlack:11d}
	\glll Yeetakula omugongo.\\
	    a-a-ee-takul-a 			omugongo\\
		\textsc{1sbj}-\textsc{pst}-\textsc{refl}-scratch-\textsc{fv}	back(3)\\
		\glt ‘He scratched his back.’ [elicited]

\ex \label{ex:Witzlack:11e}
	\glll Omugongo agwetakula buli kiro.\\
	    omugongo	a-gu-ee-takul-a			buli		kiro\\
		back(3)		\textsc{1sbj}-3\textsc{obj}-\textsc{refl}-scratch-\textsc{fv}	every	night(7)\\
		\glt ‘He scratches his back every night.’ [elicited]

\z 
\z

% EXAMPLE 12
\ea\label{ex:Witzlack:12}
\ea \label{ex:Witzlack:12a}
    \glll Yeemwa.\\
    a-a-ee-mwa-a\\
    \textsc{1sbj}-\textsc{pst}-\textsc{refl}-shave-\textsc{fv}\\
    \glt ‘He shaved (himself).’ [elicited]

\ex \label{ex:Witzlack:12b}
    \glll Abasajja baamwa ebirevu byabwe.\\
    a-basajja ba-a-mw-a ebirevu bi-abwe\\
    men(2) 	\textsc{2sbj}-\textsc{pst}-shave-\textsc{fv} beards(8) 	\textsc{8-2poss}\\
    \glt ‘The men shaved their beards.’ [elicited]

\ex \label{ex:Witzlack:12c}
    \glll Abasajja beemwa ebirevu.\\
    a-basajja ba-a-ee-mw-a 	ebirevu \\
    men(2) \textsc{2sbj}-\textsc{pst}-\textsc{refl}-shave-\textsc{fv} 	beards(8)\\
    \glt ‘The men shaved their beards.’ [elicited]

\z 
\z

In contrast to the patterns outlined above, the intransitive verb naaba ‘wash (oneself), clean up, bathe’ illustrated in~(\ref{ex:Witzlack:8a}) allows for only one way to express the relevant body part, viz. as an oblique phrase with the preposition mu, compare~(\ref{ex:Witzlack:13.1a}) and~(\ref{ex:Witzlack:13.1b}).
 %EXAMPLE 13A
\ea\label{ex:Witzlack:13.1}
\ea \label{ex:Witzlack:13.1a}
    \glll	Nanaaba.\\
    n-a-naab-a\\
    \textsc{1sg.sbj}-\textsc{pst}-bath-\textsc{fv}\\
    \glt ‘I bathed/took a bath/washed myself.’ [elicited]

\ex \label{ex:Witzlack:13.1b}
    \glll Nanaaba mu ngalo.\\
    n-a-naab-a	mu	ngalo\\
    \textsc{1sg.sbj}-\textsc{pst}-bath-\textsc{fv}	\textsc{18.loc}	10.hands\\
    \glt ‘I washed my hands.’ [elicited]

\z 
\z

\section{Coreference properties}\label{sec:Witzlack:4}
 
This section discusses coreference properties of the reflexive construction. In \sectref{sec:Witzlack:4.1} we discuss the coreference of the subject and various semantic roles. 
\sectref{sec:Witzlack:4.2} discuses coreference between non-subject arguments.

\subsection{Coreference of the subject with various semantic roles}\label{sec:Witzlack:4.1}


In this section we discuss the marking of the coreference of the subject and various semantic roles. 
We first consider the coreference between the subject and the possessor, as well as spatial referent, which are not overtly indicated in Luganda. 
We then discuss the coreference of the subject with the recipient with lexical ditransitive verbs and with the beneficiary of applicative verbs, which both use the regular reflexive prefix \emph{ee-}.

The coreference of the subject and of a possessor is not overtly indicated in Luganda: regular possessive pronouns are used and result in ambiguity between a coreferential reading and the reading with disjoint reference, as in~(\ref{ex:Witzlack:13.2}). 
For instance, the example from the corpus in~(\ref{ex:Witzlack:13.2c}) is open to multiple interpretations and only the context resolves the ambiguity: the house belongs to the official of the king.


\ea\label{ex:Witzlack:13.2}
\ea \label{ex:Witzlack:13.2a}
    \glll Yatwala manvuuli ye.\\
    a-a-twal-a 	manvuuli ye\\
    \textsc{1sbj}-\textsc{pst}-take-\textsc{fv}	umbrella(9) 	\textsc{9.1poss}\\
    \glt ‘He\textsubscript{i}/she\textsubscript{j} took his\textsubscript{i}/\textsubscript{k}/her\textsubscript{j}/\textsubscript{l} umbrella.’ [elicited]

\ex \label{ex:Witzlack:13.2b}
    \glll John asoma ekitabo kye.\\
   John	a-som-a ekitabo kye\\
    John(1)	\textsc{1sbj}-read-\textsc{fv}	book(7) \textsc{7.1poss}\\
    \glt ‘John\textsubscript{i} reads his\textsubscript{i}/\textsubscript{j}/her\textsubscript{j} book.’ [elicited]

\ex \label{ex:Witzlack:13.2c}
    \glll Omukungu wa Kabaka ali mu kattu oluvannyuma lw’ omukazi omukadde okufiira mu maka ge.\\
    omukungu wa	Kabaka	a-li mu	kattu oluvannyuma lw’ omukazi		omukadde	oku-fiir-a		mu		maka	ge\\
    official(1)	\textsc{1.gen}	king(1)	\textsc{1sbj}-be	\textsc{18.loc}	dilemma(12)	after \textsc{11.gen}	woman(1)	old(1)		\textsc{inf}-die-\textsc{fv}	\textsc{18.loc}	house(6)	\textsc{6.1poss}\\
    \glt ‘An official\textsubscript{i} of the King is in dilemma after the death of an old lady\textsubscript{k} in his\textsubscript{i}/\textsubscript{j}/her\textsubscript{k}/\textsubscript{l} house.’ [written]%(LU-NEWS-190110-Omukungu:1)

\z 
\z

The coreference of the subject and a spatial referent is not overtly coded either. 
Regular pronominal forms, such as the nominal class 1 pronoun \emph{we} ‘he/she’ in~(\ref{ex:Witzlack:14}), are used and the interpretation of their reference is determined by the context.

% EXAMPLE 14
\ea\label{ex:Witzlack:14}

\ea \label{ex:Witzlack:14a}
    \glll Yalaba omusota wabbali we.\\
    a-a-lab-a omusota	wabbali	we\\
    \textsc{1sbj}-\textsc{pst}-see-\textsc{fv}	snake(3) besides	1\\
    \glt ‘She\textsubscript{i} saw a snake beside her\textsubscript{i/j}/him.’ [elicited]

\ex \label{ex:Witzlack:14b}
    \glll Yaleka emikululo emabega we.\\
    a-a-lek-a emikululo	emabega	we\\
    \textsc{1sbj}-\textsc{pst}-leave-\textsc{fv}	traces(4)		behind	1\\
    \glt ‘She\textsubscript{i} left traces behind her\textsubscript{i/j}/him.’ [elicited]
\z 
\z

With ditransitive lexical verbs, both objects are not overtly flagged and can be indexed on the verb, as in~(\ref{ex:Witzlack:17}). 
The first token of the verb \emph{wa} ‘give’ indexes only the recipient, the theme is expressed by the noun \emph{olukusa} ‘permission(11)’, whereas the second token of \emph{wa} ‘give’ indexes both objects, in this case the theme prefix \emph{lu-} ‘11\textsc{obj}’ (indexing \emph{olukusa} ‘permission(11)’) precedes the recipient prefix of noun class 1 \emph{mu-} ‘1\textsc{obj}’. 
When the recipient is coreferential with the subject, the respective person index is replaced with the regular reflexive prefix \emph{ee-}, as in~(\ref{ex:Witzlack:18}). 
The theme can either be expressed by a noun phrase, e.g.\,\emph{ekirabo} ‘present(7)’ in~(\ref{ex:Witzlack:18a}), or by a theme index which precedes the reflexive prefix, as e.g.\,the class 7 prefix \emph{ki-} in~(\ref{ex:Witzlack:18b}).

% EXAMPLE 17
\ea
\label{ex:Witzlack:17}
 
    \glll […] ng’amuwadde olukusa oba talumuwadde.\\
    nga	a-mu-wadde	olukusa	oba	ti-a-lu-mu-wadde\\
    when \textsc{1sbj-1obj}-give:\textsc{pfv} permission(11)	or	\textsc{neg-1sbj-11obj-1obj}-give:\textsc{pfv}\\
    \glt ‘… whether he has given him a permission, or he has not given it to him.’ [written]%(LU-NEWS-160213-Engeri:32)

\z 

% EXAMPLE 18
\ea\label{ex:Witzlack:18}

\ea \label{ex:Witzlack:18a}
    \glll Omuwala yeewa ekirabo.\\
    omuwala	a-a-ee-w-a		ekirabo\\
    girl(1)	\textsc{1sbj}-\textsc{pst}-\textsc{refl}-give-\textsc{fv}		present(7)\\
    \glt ‘The girl gave herself a present.’ [elicited]

\ex \label{ex:Witzlack:18b}
    \glll Omuwala yakyeewa.\\
    omuwala	a-a-ki-ee-w-a\\
    girl(1)	\textsc{1sbj}-\textsc{pst-7obj}-\textsc{refl}-give-\textsc{fv}\\
    \glt ‘The girl gave it to herself.’ [elicited]


\z 
\z



Luganda has a productive applicative construction formed by the suffix \emph{-ir} and its variants. One of its functions is to introduce a beneficiary of an action expressed by the verb into the clause, as is illustrated twice in~(\ref{ex:Witzlack:15}). 
Pronominal beneficiaries are then expressed by the regular object prefixes on the verb, as e.g.\,class two object prefix \emph{ba-} on the last verb in~(\ref{ex:Witzlack:15}). 

% EXAMPLE 15
\ea
\label{ex:Witzlack:15}

\glll Nga mugogola enzizi, okuzimbira abakadde amayumba         n' okubalimirako.\\
    nga		mu-gogol-a	enzizi	oku-zimb-ir-a abakadde 	amayumba ne		oku-ba-lim-ir-a=ko\\
    when	\textsc{2pl.sbj}-clean-\textsc{fv}	well(10)	\textsc{inf}-build-\textsc{appl}-\textsc{fv} elderly(2)	houses(6) and	\textsc{inf-2obj}-dig-\textsc{appl}-\textsc{fv}=\textsc{part}\\
    \glt ‘You would clean the wells, constructing a house for the elderly and digging for them a bit.’ [written]	%(LU-NEWS-161008-Engeri:14)

\z

When the applied object is coreferential with the subject, the regular reflexive prefix replaces the object prefix to encode the beneficiary, as in the autobenefactive construction in~(\ref{ex:Witzlack:16}).

% EXAMPLE 16
\ea\label{ex:Witzlack:16}

\ea \label{ex:Witzlack:16a}
    \glll Yeegulira ekitabo.\\
    a-a-ee-gul-ir-a ekitabo\\
    \textsc{1sbj}-\textsc{pst}-\textsc{refl}-buy-\textsc{appl}-\textsc{fv}	book(7)\\
    \glt ‘She bought a book for herself.’ [elicited]

\ex \label{ex:Witzlack:16b}
    \glll Omulenzi yeefumbira ekyeggulo.\\
   omulenzi	a-a-ee-fumbir-a ekyeggulo.\\
    boy(1)	\textsc{1sbj}-\textsc{pst}-\textsc{refl}-cook-\textsc{appl}-\textsc{fv}	dinner(7)\\
    \glt ‘The boy cooked himself dinner.’ [elicited]

\ex \label{ex:Witzlack:16c}
    \glll Beezimbira ennyumba.\\
    ba-a-ee-zimb-ir-a	ennyumba.\\
    \textsc{2sbj}-\textsc{pst}-\textsc{refl}-build-\textsc{appl}-\textsc{fv}	houses(10)\\
    \glt ‘They built themselves houses.’ [elicited]

\ex \label{ex:Witzlack:16d}
    \glll Bampa ekirala kya kuzannya nga neekwanira omulenzi.\\
    Ba-m-p-a eki-lala	kya ku-zanny-a nga	n-ee-kwan-ir-a omulenzi\\
    \textsc{2sbj-1sg.obj}-give-\textsc{fv}	7-other 	\textsc{7.rel} \textsc{inf}-act-\textsc{fv} as		\textsc{1sg.sbj}-\textsc{refl}-seduce-\textsc{appl}-\textsc{fv}	boy(1)\\
    \glt ‘I was given another role of seducing a boy for myself.’ [written]%(LU-NEWS-150531-Omusajja:44)
    

\z 
\z

\subsection{Coreference between non-subject arguments}\label{sec:Witzlack:4.2}

No dedicated means exist in Luganda to express the coreference between two non-subject participants of the same clause. Regular possessive pronouns are used both in cases of the coreference of the possessor with one of the referents in the clause but also in case when the possessor is not mentioned in the clause at all, as the various readings in~(\ref{ex:Witzlack:19}) indicate.

% EXAMPLE 19
\ea\label{ex:Witzlack:19}
 
    \glll John yalaga Mary ekifaananyi kye.\\
    John	a-a-lag-a Mary ekifaananyi	ki-e\\
    John(1)	\textsc{1sbj}-\textsc{pst}-show-\textsc{fv}	Mary(1)	photo(7)	\textsc{7-1poss}\\
    \glt ‘John\textsubscript{i} showed Mary\textsubscript{j} a photo of himself\textsubscript{i}/herself\textsubscript{j}/him\textsubscript{k}/her\textsubscript{l}.’ [elicited]

\z

Attempts to obtain other cases of coreference between two non-subject participants following the questionnaire \citep{JanicHaspelmath2020} resulted in construction with a relative clause, as in~(\ref{ex:Witzlack:20}), and are ambiguous with third person referents, as the various readings of~(\ref{ex:Witzlack:20b}) suggest.

% EXAMPLE 20
\ea\label{ex:Witzlack:20}

\ea \label{ex:Witzlack:20a}
    \glll Yatubuulira ebitukwatako.\\
    a-a-tu-buulir-a	e-bi-tu-kwat-a=ko\\
    \textsc{1sbj}-\textsc{pst}-\textsc{1pl.obj}-tell-\textsc{fv}	\textsc{rel-8sbj}-\textsc{1pl.obj}-concern-\textsc{fv=17.loc}\\
    \glt ‘She told us about ourselves.’ [elicited]

\ex \label{ex:Witzlack:20b}
    \glll Yagogera 	ne John ebimukwatako.\\
    a-a-gog-era	ne	John e-bi-mu-kwat-a=ko\\
    \textsc{1sbj}-\textsc{pst}-speak-\textsc{appl}-\textsc{fv}	\textsc{COM} John \textsc{rel-8sbj-1obj}-concern-\textsc{fv=17.loc}\\
    \glt ‘He\textsubscript{i} spoke with John\textsubscript{j} about himself\textsubscript{i}/\textsubscript{j}/him\textsubscript{k}/her\textsubscript{l}.’ [elicited]


\z 
\z



\section{Contrast between exact and inclusive coreference}\label{sec:Witzlack:5}

In this section we briefly outline the structural difference between constructions used for exact coreference and constructions employed for inclusive coreference. 
The exact coreference between the agent and the patient arguments is expressed by the use of the regular reflexive prefix \emph{ee-}, as in many examples above, as well as in~(\ref{ex:Witzlack:21}). 
In case of inclusive coreference, the verb also carries the reflexive prefix \emph{ee-}. 
The patient argument coreferential with the agent can be optionally expressed overtly with a personal pronoun followed by the self-intensifier particle \emph{kennyini} (see below). 
The non-coreferential patient is expressed by a prepositional phrase with the preposition \emph{ne} ‘with’. 
Furthermore, the adverb \emph{wamu} ‘together’ can precede the prepositions phrase, compare~(\ref{ex:Witzlack:21a}) and~(\ref{ex:Witzlack:21b}).

% EXAMPLE 21
\ea\label{ex:Witzlack:21}

\ea \label{ex:Witzlack:21a}
    \glll Yeekolokota.\\
    a-a-kolokot-a\\
    \textsc{1sbj}-\textsc{pst}-critisize-\textsc{fv}\\
    \glt ‘He criticized himself.’ [elicited]

\ex \label{ex:Witzlack:21b}
    \glll Yeekolokota (ye kennyini) (wamu) n’ abalala.\\
    a-a-kolokot-a	ye	kennyini	wamu	ne	abalala\\
    \textsc{1sbj}-\textsc{pst}-critisize-\textsc{fv}	1	self		together	with		others(2)\\
    \glt ‘He criticized himself and the others.’ [elicited]

\z
\z

The self-intensifier particle \emph{kennyini} used in~(\ref{ex:Witzlack:21b}) or its agreeing forms (“emphatic pronoun” in \citet[178, 439]{Murphy1972}\footnote{What conditions the use of agreeing vs.\,non-agreeing forms is a topic for further investigations.} is otherwise used to emphasize the exclusive participation of the noun phrase it follows, as e.g.\,\emph{omulwanyi kennyini} ‘the fighter himself’ in~(\ref{ex:Witzlack:22a}) or \emph{ffe kennyini} ‘we ourselves’ in~(\ref{ex:Witzlack:22b}).

% EXAMPLE 22
\ea\label{ex:Witzlack:22}

\ea \label{ex:Witzlack:22a}
    \glll Naye omulwanyi kennyini ye yasabye nti tasobola musajja.\\
    naye omulwanyi	kennyini	ye	yasabye	nti	ti-a-sobol-a			musajja \\
    but	fighter(1)	self 1 \textsc{1sbj}-\textsc{pst}-ask:\textsc{pfv} \textsc{quot} \textsc{neg}-\textsc{1sbj}-cope\_with-\textsc{fv}	man(1)\\
    \glt ‘But it was the fighter himself who said that he can’t defeat the man.’ [written]%(LU-NEWS-120630-Golola:15)

\ex \label{ex:Witzlack:22b}
    \glll Eky’ ennaku mu ffe kennyini abaakukusanga emmwaanyi, mwabeerangamu bambega ba 	gavumenti.\\
    eky’ ennaku	mu	ffe	kennyini a-ba-a-ku-kus-a-nga emmwaanyi		mu-a-beer-a-nga=mu 	bambega	ba gavumenti\\
    \textsc{7.rel}	sadness(9)	\textsc{18.loc}	\textsc{1pl}	self(2)	\textsc{rel-2sbj}-\textsc{pst-prog}-smuggle-\textsc{fv-hab} coffee\_berries(10)	\textsc{18bj}-\textsc{pst}-be:\textsc{appl}-\textsc{fv-hab=18.loc}	spies(2) \textsc{2.gen}	government(9)\\
    \glt ‘What is sad is that among us ourselves, the ones who smuggled coffee, there also used to be government spies.’ [written] %(LU-NEWS-161029-Engeri:30)


\z
\z


\section{Long-distance coreference}\label{sec:Witzlack:6}

No dedicated means are used to express coreference across clauses, compare~(\ref{ex:Witzlack:23a}), where the agents of the two clauses have disjoint reference, with~(\ref{ex:Witzlack:23b}), where the agents of the two clauses are coreferential.

% EXAMPLE 23
\ea\label{ex:Witzlack:23}

\ea \label{ex:Witzlack:23a}
    \glll Agambye nti batandikira Ggulu mu Septembe.\\
    a-gambye	nti		ba-tandik-ir-a 	Ggulu	mu	September	 \\
    \textsc{1sbj}-say:\textsc{pfv}	\textsc{quot}	\textsc{2sbj}-start-\textsc{appl}-\textsc{fv}	Ggulu(9)	\textsc{18.loc}	September(9)	\\
    \glt ‘He said that they start from Gulu in September.’ [written] %(LU-NEWS-181221-Aba:9)

\ex \label{ex:Witzlack:23b}
    \glll Ababaka baagambye nti bateekateeka okusisinkana Pulezidenti Museveni.\\
    ababaka	ba-a-gambye nti ba-teekateek-a 	oku-sisinkan-a	Pulezidenti	Museveni\\
    representatives(2)	\textsc{2sbj}-\textsc{pst}-say:\textsc{pfv} \textsc{quot}	\textsc{2sbj}-arrange-\textsc{fv}		\textsc{inf}-meet-\textsc{fv}	president(1)	Museveni(1)\\
    \glt ‘The representatives said that they are organizing to meet President Museveni.’ [written]%(LU-NEWS-121124-KCCA:16)

\z
\z


\section{Specialized reflexive form in other functions}\label{sec:Witzlack:7}

This section focuses on two functions of the specialized reflexive prefix \emph{ee-}. 
We will first outline its use to express the reciprocal meaning (\sectref{sec:Witzlack:7.1}). 
We then briefly outline the impressive set of fossilized reflexives in Luganda (\sectref{sec:Witzlack:7.2}). 


\subsection{Reflexive-reciprocal polysemy}\label{sec:Witzlack:7.1}

Apart from the functions outline above, as in many other Bantu languages, the Luganda reflexive prefix is polysemous and can be used to express the reciprocal meaning (cf.\,the detailed study by \citealt{DomEtAl2016} of the polysemy of the Bantu reflexive marker, as well as other markers involved in the semantic domain of the middle, see also \citealt{Polak1983} and \citealt{Marlo2015-1}). %exceptional check Ref
Luganda has two dedicated reciprocal suffixes, 
viz. \emph{-an} (called “associative” in the Bantu inventory of extensions, 
see \citealt[173]{SchadebergBostoen2019}) and \emph{-agan},\footnote{This is a historically complex suffix made up of the repetitive *\textit{-ag/-ang} and associative *\textit{-an} (\citealt[173]{SchadebergBostoen2019}, see also \citealt{DomEtAl2016} on the origin of the reciprocal suffix \emph{-angan} in Cilubà). With monosyllabic roots and roots in /g/ the suffix is realized as \emph{-aŋŋan}, see \citet[356]{AshtonEtAl1954}.} both illustrated in~(\ref{ex:Witzlack:24}). Of the two markers, \emph{-agan} is more productive, though the exact conditions of the distribution of the two markers is a topic for future research (see also \citealt[ 44–45]{McPherson2008Descriptive}) . 
.

% EXAMPLE 24

\ea
\label{ex:Witzlack:24}
 
    \glll Ffe mu kkanisa bwe tuba tugatta abafumbo tubagamba baagalanenga, bakuumaganenga.\\
    ffe	mu	kkanisa	bwe	tu-ba tu-gatt-a	abafumbo tu-ba-gamb-a		ba-yagal-an-e-nga	ba-kuum-agan-e-nga\\
    we	\textsc{18.loc}	church(9)	when \textsc{1pl.sbj-aux}	\textsc{1pl.sbj}-join-\textsc{fv}	married\_couples(2) \textsc{1pl.sbj-2obj}-say-\textsc{fv} \textsc{2sbj}-love-\textsc{recp-sbjv-hab}	\textsc{2sbj}-protect-\textsc{recp-sbjv-hab} \\
    \glt ‘As for us, when in church we are joining married couples, we tell them to love each other, protect each other.’ [written]%(LU-NEWS-150213-Buli:12)

\z


In addition to the dedicated reciprocal markers, the reflexive prefix \emph{ee-} is occasionally used to render the reciprocal meaning, as in~(\ref{ex:Witzlack:25}). 

% EXAMPLE 25
\ea\label{ex:Witzlack:25}

\ea \label{ex:Witzlack:25a}
    \glll [B]atandise okwebba.\\
    ba-tandise	oku-ee-bb-a\\
    \textsc{2sbj}-start.\textsc{pfv}	\textsc{inf}-\textsc{refl}-steal-\textsc{fv}\\
    \glt ‘(Some Ugandans in South Africa have no job so) they started stealing from each other.’ [written]%(LU-NEWS-190312-Gavt:9)

\ex \label{ex:Witzlack:25b}
    \glll Twewalana.\\
    tu-ee-walan-a\\
    \textsc{1pl.sbj}-\textsc{refl}-hate-\textsc{fv}\\
    \glt ‘We hate each other/ourselves.’ [elicited]
\z
\z

In some cases, the reflexive is used in combination with the fossilized reciprocal stems, as in (\ref{ex:Witzlack:26}) (see also \citealt[122]{Murphy1972}).\footnote{\citet[46]{McPherson2008Descriptive} reports that one of her consultants used the reflexive prefix \emph{ee-} and the reciprocal suffix \emph{-agan} productively with the same verbs. 
Such examples are found unacceptable by the speakers we consulted and we did not find a single attestation of such a combination in our corpus.}
The functions and distribution of this construction remains a topic for further research. 

% EXAMPLE 26
\ea
\label{ex:Witzlack:26}
 
\glll Bejjukanya.\\
ba-ee-jjukany-a\\
\textsc{2sbj}-\textsc{refl}-remind.\textsc{recp.caus-fv}\\
\glt ‘They remind each other.’ %(ga_MAK_190715_FS02:42)

\z


\subsection{Lexicalized reflexive verbs}\label{sec:Witzlack:7.2}

The discussion in \sectref{sec:Witzlack:2}–\sectref{sec:Witzlack:6} focused on the reflexive construction proper, i.e.\,on a grammatical construction with a special form (the reflexivizer \emph{ee-}) employed when two participants of a clause are coreferential (as defined in Haspelmath, this volume), as well as on the use of \emph{ee-} to express the reciprocal meaning (\sectref{sec:Witzlack:7.1}). 
However, when one considers the distribution of the reflexive prefix \emph{ee-} in the corpus, these two constructions do not account for the most frequent types of constructions with the reflexive prefix \emph{ee-}. 
What are then these other uses of the reflexive prefix \emph{ee-}? 

\citet[31]{Geniusiene1987} makes a distinction between reversible reflexive verbs, which are usually in the focus of studies of reflexive vs.\,the less studied class of non-reversible  reflexive verbs.\footnote{This is originally \citeauthor{Nedjalkov1997}'s terms (\citeyear[10–15]{Nedjalkov1997}).} 
The following criteria of reversibility are suggested by \citet[145–148]{Geniusiene1987} to distinguish between the two: (1) morphological reversibility, i.e. a situation when a derived unit is formally related to a base word, morphological non-reversible are traditionally known as reflexiva tantum; (2) syntactic reversibility, viz. a change of reversible reflexive properties according to one of the regular patterns; (3) lexical reversibility, viz. the identity of lexical distribution relative to the corresponding syntactic positions in a non-reflexive construction and related reflexive construction; (4) semantic reversibility, viz. a regular, standard change of the meaning of a reflexive, semantic non-reversible reflexive verbs have the meaning which is related to that of the base non-reflexive way in some idiosyncratic way. We will first consider reflexiva tantum, and then we will proceed with what \citet{GotoSay2009} call “non-reversible reflexive verbs proper”, these are the verbs that are non-reversible according to one or often several of the criteria (2) to (4).

Reflexiva tantum and semantic non-reversible reflexive verbs proper are wide\-spread in the Bantu languages (see \citealt{Marlo2015-1} for examples from a range of Bantu languages). 
\citet{Polak1983} notes that this widespread pattern of reflexive lexicalization and fossilization may have already existed in Proto-Bantu. 
\citet[132–133]{AshtonEtAl1954} grammar of Luganda lists a small number of non-reversible reflexive verbs of various types, whereas a quick skim through \citet{Murphy1972} yields hundreds of candidates.\footnote{\citet{Murphy1972} also lists frequent non-lexicalized reflexives.}

Luganda reflexive-tantum verbs include e.g.\,the intransitive \emph{eedubika} ‘get stuck in the mud; be immersed’, and \emph{eegoota} ‘walk with a stiff, erect or proud gait’, as well as transitive \emph{eekeka} ‘suspect, beware of’, \emph{eebagala} ‘mount, ride (an animal)’, and \emph{eesigama} ‘lean on; rely on’. 

Non-reversible reflexives have idiosyncratic relations to the corresponding non-reflexive verbs. An example for a Luganda semantic non-reversible reflexive verb is given in~(\ref{ex:Witzlack:27}).
The reflexive-tantum verb \emph{eesiga} ‘trust, rely on’ has a formally non-reflexive counterpart \emph{siga} ‘to sow, plant’.

% EXAMPLE 27
\ea \label{ex:Witzlack:27}
 
    \glll Basobola okukwesiga okukuwola?\\
    ba-sobol-a	oku-ku-eesig-a	oku-ku-wol-a\\
    \textsc{2sbj}-can-\textsc{fv}	\textsc{inf-2sg.obj}-trust(\textsc{refl})-\textsc{fv}	\textsc{inf-2sg.obj}-lend-\textsc{fv}\\
    \glt ‘Can they trust you and lend you (money)?’ [written] %(LU-NEWS-171023-Kaamulali:37)

\z

Some non-reversible reflexives are semantically nearly identical with their non-reflexive counterparts and thus do not follow the standard change of the meaning of a reflexive, as e.g.\,\emph{gaana}~(\ref{ex:Witzlack:28a}) and \emph{eegana}~(\ref{ex:Witzlack:28b}): they both mean ‘reject, refuse, deny’ and in one of their senses entail an abstract patient (an idea, a proposal, a statement). 

% EXAMPLE 28
\ea\label{ex:Witzlack:28}

\ea \label{ex:Witzlack:28a}
    \glll Kino baakigaana.\\
    ki-no	ba-a-ki-gaan-a\\
    \textsc{7-prox} \textsc{2sbj}-\textsc{pst-7obj}-reject-\textsc{fv}\\
    \glt ‘They rejected it (the divorce proposal).’ [written] %(LU-NEWS-160501-Baze:57)

\ex \label{ex:Witzlack:28b}
    \glll kyokka China yo ebyegaana.\\
    kyokka	China 	yo 	e-bi-eegaan-a\\
    but		China(9) 9.\textsc{med}	\textsc{9sbj-8obj}.deny(\textsc{refl})-\textsc{fv}\\
    \glt ‘(…) but China denied them (the reports).’ [written] %(LU-NEWS-160207-Eyeebase:10)


\z
\z

Others are non-reversible with respect to several criteria at once. 
For example, the reflexive verb \emph{eetegereza} ‘comprehend, grasp; analyze; observe; recognize, make out’ derives from \emph{tegereza} ‘listen to, pay attention to’. 
Apart from semantic non-reversibility, this, as well as many other Luganda reflexive verbs, are syntactically non reversible, as both \emph{tegereza} and its morphologically reflexive counterpart \emph{eetegereza} are transitive, as the object prefix \emph{mu-} ‘1\textsc{obj}’ in~(\ref{ex:Witzlack:29b}) indicates.

% EXAMPLE 29
\ea\label{ex:Witzlack:29}

    \ea \label{ex:Witzlack:29a}
    \glll agambye nti agenda kusooka kwetegereza tteeka.\\
    a-gambye 	nti	a-gend-a	ku-sook-a	ku-eetegerez-a	tteeka\\
    \textsc{1sbj}-say:\textsc{pfv}	\textsc{quot}	\textsc{1sbj-aux-fv}	\textsc{inf}-do\_first-\textsc{fv}	\textsc{inf-}revise(\textsc{refl})-\textsc{fv}	bill(5)\\
    \glt ‘He has said that he is going to revise the bill first (before signing it).’ [written]%(LU-NEWS-131227-Museveni:3)
    
\ex \label{ex:Witzlack:29b}
    \glll Oluvannyuma lw’ okumwetegereza namutuukirira.\\
    oluvannyuma	lwa	oku-mu-eetegerez-a	n-a-mu-tuukirir-a\\
    after \textsc{11.gen} \textsc{inf-1obj-}observe(\textsc{refl})-\textsc{fv} \textsc{1sg.sbj-pst-1obj}-approach-\textsc{fv}\\
    \glt ‘After observing her, I approached her (and made a marriage proposal).’ [written]%(LU-NEWS-161218-Abooluganda:28)

\z
\z


Another example of non-reversibility with respect to several criteria is provided in (\ref{ex:Witzlack:30b}). 
The non-reflexive ditransitive verb \emph{buuza} ‘ask’ takes two arguments, viz.\,the person being asked and the question, as in~(\ref{ex:Witzlack:30a}). 
Its reflexive counterpart \emph{eebuuza} means ‘ask oneself, wonder’ but also ‘inquire, consult’. 
In this second usage in addition to mild semantic non-reversibility, we also observe a change of valency properties, as another participant – the one enquires from – can be added to the clause, though the argument role is in principle already occupied by the reflexive prefix.

%Example 30
\ea\label{ex:Witzlack:30}

\ea \label{ex:Witzlack:30a}
    \glll Baamubuuzizza lwaki tayagala kusooka kugattibwa.\\
    ba-a-mu-buuzizza lwaki	ti-a-yagal-a	ku-sook-a		ku-gattibw-a\\
    \textsc{2sbj}-\textsc{pst-1obj}-ask:\textsc{pfv}	why		\textsc{neg-1sbj}-want-\textsc{fv}	\textsc{inf}-do\_first-\textsc{fv}	\textsc{inf}-marry-\textsc{fv}\\
    \glt ‘They asked him why he does not want to do the wedding first.’ [written]%(LU-NEWS-120120-	Gwe:21)

\ex \label{ex:Witzlack:30b}
    \glll Mukyala wange takyampuliriza era buli kimu ky’ akola yeebuuza ku mikwano gye.\\
    mukyala wa-nge 	ti-a-kya-n-wuliriz-a era buli kimu kye a-kol-a a-eebuuz-a 	ku 	mikwano	gye\\
    wife(1)	1-\textsc{1sg.poss} \textsc{neg-1sbj-pers-1sg.obj}-listen.to-\textsc{fv} and every	thing(1) \textsc{7.rel}	\textsc{1sbj}-do-\textsc{fv} \textsc{1sbj}-consult(\textsc{refl})-\textsc{fv} \textsc{17.loc} friends(4)	\textsc{4.1poss}\\
    \glt ‘My wife no longer listens to me and she first consults her friends on whatever she does.’ [written] %(LU-NEWS-160419-Omusajja: 65))

\z
\z


\section{Conclusion}\label{sec:Witzlack:8}

This chapter addressed some questions regarding reflexive constructions in the Bantu language Luganda. It was shown that the prefix \emph{ee-} is used as a general reflexivizer, and that it does not show morphosyntactic agreement with person-number or noun class features of the subject. 
It is used productively to express coreference between the subject and the patient object in transitive verbs, and there is no difference between introverted or extroverted verbs.
Although Luganda has two dedicated reciprocal suffixes, \emph{ee-} can also be used to express reciprocal meaning, which is not uncommon for Bantu languages. 
The Luganda reflexivizer cannot be used to render coreferential relation between the subject and a possessor, nor for the subject and a spatial referent, and ambiguity has to be resolved by context. This is also true for the coreference between two non-subject arguments within the same clause for which there is no dedicated marker in Luganda. Despite its productivity, reflexive constructions proper do not account for the most frequent usage of the prefix \emph{ee-} in the corpus: it is especially noteworthy that the Luganda lexicon has quite a number of lexicalized reflexive verbs. In addition to reflexiva tantum, which are morphologically irreversible and cannot occur without the prefix, there are also non-reversible reflexives that have idiosyncratic relations (syntactic, lexical and/or semantic) to the corresponding non-reflexive verbs. The reflexivizer can also be used in combination with other verbal extensions such as fossilized reciprocals, which remains a topic for future research.

\section*{Abbreviations}
Abbreviations generally follow the Leipzig Glossing Rules. Additional abbreviations are the following:

\begin{tabularx}{.45\textwidth}{lQ}
    1\textsc{sg}, etc\, & person and number (only when followed by \textsc{sg} or \textsc{pl})\\
    1 to 23 & noun classes\\
    \textsc{fv} & final vowel\\
    \textsc{hab} & habitual\\
    \textsc{med} & medial demonstrative\\
    \textsc{part} & partitive\\
    \textsc{pers} & persistive\\
\end{tabularx}


{\sloppy\printbibliography[heading=subbibliography,notkeyword=this]}

\end{document}