\title{Semantic differences in translation}
\subtitle{Exploring the field of inchoativity} 
\BackBody{Although the notion of meaning has always been at the core of translation, the invariance of meaning has, partly due to practical constraints, rarely been challenged in Corpus-based Translation Studies. In answer to this, the aim of this book is to question the invariance of meaning in translated texts: if translation scholars agree on the fact that translated language is different from non-translated language with respect to a number of grammatical and lexical aspects, would it be possible to identify differences between translated and non-translated language on the semantic level too? More specifically, this books tries to formulate an answer to the following three questions: (i) how can semantic differences in translated vs non-translated language be investigated in a corpus-based study?, (ii) are there any differences on the semantic level between translated and non-translated language? and (iii) if there are differences on the semantic level, can we ascribe them to any of the (universal) tendencies of translation? In this book, I establish a way to visually explore semantic similarity on the basis of representations of translated and non-translated semantic fields. A technique for the comparison of semantic fields of translated and non-translated language called SMM++ (based on Helge Dyvik’s Semantic Mirrors method) is developed, yielding statistics-based visualizations of semantic fields. The SMM++ is presented via the case of inchoativity in Dutch (\textit{beginnen} `to begin'). By comparing the visualizations of the semantic fields on different levels (translated Dutch with French as a source language, with English as a source language and non-translated Dutch) I further explore whether the differences between translated and non-translated fields of inchoativity in Dutch can be linked to any of the well-known universals of translation. The main results of this study are explained on the basis of two cognitively inspired frameworks: Halverson’s Gravitational Pull Hypothesis and Paradis’ neurolinguistic theory of bilingualism.}
\dedication{%
Pour Alexis et Fabrice\vfill
On ne voit bien qu’avec le cœur\\
L’essentiel est invisible pour les yeux\\
(Antoine de Saint-Exupéry)
}
\renewcommand{\lsImpressumExtra}{The views expressed in this book are the author's and in no way reflect the views of the Council or European Council.}
\typesetter{Lukas Gienapp, Felix Kopecky}
\illustrator{Sebastian Nordhoff}
\proofreader{Amir Ghorbanpour,
Barend Beekhuizen,
Bev Erasmus,
Dan Bondarenko,
Jeroen van de Weijer,
Mario Bisiada,
Jean Nitzke,
Sebastian Nordhoff \&
Selçuk Eryatmaz}
\author{Lore Vandevoorde}
\BookDOI{10.5281/zenodo.2573677}%ask coordinator for DOI
\renewcommand{\lsISBNdigital}{978-3-96110-072-9}
\renewcommand{\lsISBNhardcover}{978-3-96110-073-6}
\renewcommand{\lsSeries}{tmnlp}
\renewcommand{\lsSeriesNumber}{13} %will be assigned when the book enters the proofreading stage 
\renewcommand{\lsID}{194}
\renewcommand{\lsCoverTitleFont}[1]{\sffamily\addfontfeatures{Scale=MatchUppercase}\fontsize{43.5pt}{15mm}\selectfont #1}
