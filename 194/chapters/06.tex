\chapter{Conclusion}\label{sec:6}
\addtocontents{toc}{\protect\enlargethispage{\baselineskip}}
\section{General conclusions}\label{sec:6.1}  
Impelled by the lack of empirical studies involved with meaning variation in translation, I decided to place the study of semantic differences in translated compared to non-translated texts at the center of my concerns. To date, much research in CBTS has focused on lexical and grammatical phenomena in an attempt to reveal presumed general tendencies of translation, but on the semantic level, these general tendencies have rarely been investigated. I therefore set out to answer three central questions.

The first question, how to investigate semantic differences in a translational setting, required a lengthy answer which was covered by the methodology proposed in \chapref{sec:3}. Given the attested lack of empirical studies of semantic phenomena in CBTS, no clear hypotheses could be drawn beforehand so that the proposed method necessarily had to be explorative in nature. In addition, CBTS offers very few methodological guidelines for semantic investigations. As a consequence, the first challenge was to develop a methodological technique able to measure semantic similarity of translated and non-translated language. I established a way to visually explore semantic similarity on the basis of representations of translated and non-translated semantic fields of a concept under study. More specifically, I developed the Extended Semantic Mirrors Method, a bottom-up, statistical visualization method of semantic fields in both translated and non-translated language. The method consists of (i) a translation-driven retrieval method for the selection of candidate-lexemes for a semantic field as well as (ii) a procedure to statistically visualize the retrieved data sets. In addition, different types of visualizations were proposed so as to investigate levelling out, shining through and normalization.

The application of the developed method to the case of inchoativity in Dutch allowed me to answer the second question: are there any differences on the semantic level between translated and non-translated texts? Since I did indeed observe differences on the semantic level between translated and non-translated Dutch, the third question required to be answered as well. Based on the additional visualization techniques proposed in \chapref{sec:3}, I made an attempt to link the observed differences to the universal tendencies of levelling out, shining through and normalization, which I considered to be the most suitable ones for semantic research.

I found evidence for the presence of semasiological levelling out in translated Dutch since in both TransDutch fields, some of the semasiological variation present in SourceDutch was ``absorbed'' by the \textsc{reference cluster}. As for semasiological shining through, I found that an influence of the source language possibly provoked the joint clustering of \textsc{action} and \textsc{state after onset} in\linebreak TransDutch\textsubscript{ENG}, the separate clustering of \textsc{action} and \textsc{state after onset} in Trans\-Dutch\textsubscript{FR} and the joint clustering of \textsc{action} and {\textsc{specific}} \textsc{action} in Trans\-Dutch\textsubscript{FR}. However, the specific clustering of \textsc{action} and \textsc{state after onset} in TransDutch\textsubscript{ENG} (into the \textsc{reference cluster}) and in TransDutch\textsubscript{FR} (into separate clusters) could also be explained as different degrees of target language influence and, hence, as semasiological normalization.

On the onomasiological level, I observed that the prototype-based organization of lexemes within the separate meaning distinctions differed in translated language, compared to non-translated language. Unfortunately, I could not connect my conclusions directly to the idea of onomasiological levelling out, since the number of lexemes in each visualization is kept stable. I did notice minimal changes in the prototype-based organization of the lexemes and found that lexemes which are near-synonyms in SourceDutch (such as \textit{starten} and \textit{beginnen}, \textit{start} and \textit{begin}, \textit{oprichten} and \textit{opzetten}) tend to become less near-synonymous in translated language. For onomasiological shining through, I found that the distinct clustering of \textit{opstarten} and \textit{begin} (as such semasiological phenomena) in TransDutch\textsubscript{ENG} could be explained as an influence of the source language, i.e. shining through on the onomasiological level. Furthermore, the prototype-based organization of \textit{oprichten} and \textit{opzetten} in TransDutch\textsubscript{ENG} showed signs of onomasiological normalization because of the similarity with the prototype-based organization of these lexemes in SourceDutch.

Unsatisfied with the limited explanatory power of the universals paradigm, I tried to explain the main results of this study on the basis of two cognitively inspired frameworks. The proposed cognitive frameworks – the Gravitational Pull Hypothesis and \citeauthor{paradis_language_1980}’ neurolinguistic theory of bilingualism – were applied to the results in an attempt to understand where levelling out, shining through and normalization on the semantic level might originate. Based on the idea of connectivity (a concept from the GPH) or direct transcoding (from Paradis’ model), I accounted for the separate clustering of \textit{begin} and \textit{opstarten} in TransDutch\textsubscript{ENG} (onomasiological shining through). In addition, by following the reasoning behind translating via the conceptual system (Paradis), I could tentatively explain how the observed instances of semasiological levelling out, semasiological shining through or normalization had come about.

\section{Retrospective insights}
\label{sec:6.2}  
The conclusions about tendencies of levelling out, shining through and normalization are arguably based on observations of minimal changes in the prototype-based organization of clusters and lexemes. It must be admitted that they are moreover post-hoc interpretations of the rendered visualizations and as such naturally open for discussion. Especially on the onomasiological level, it appeared hard to convincingly connect these minimal observations to larger tendencies of translational behavior. This might indeed merely come to show that the semantic changes are primarily taking place on the semasiological level, rather than on the onomasiological level, although it is also possible that the applied approach is better fitted to discern tendencies on the semasiological level than on the onomasiological level. I indeed concluded that (the few) striking observations on the onomasiological level are the ones that cause semasiological change (such as the separate clustering of \textit{opstarten} and \textit{begin}). Without a doubt, the limited number of lexemes within the visualizations (and the fact that the number of lexemes is furthermore kept stable throughout all visualizations) is one of the reasons why general tendencies seemed much more difficult to account for on the onomasiological level.

This brings me to an important point about the impact of methodological choices on my results. My interpretations of the observed phenomena in terms of general tendencies of translation are obviously heavily determined by the visualizations they rely on. These visualizations have come about as a result of a number of methodological choices which were taken primarily in the interest of the development of a viable visualization method of semantic fields in translated and non-translated language. Some of the choices undoubtedly impacted the overall appearance of the visualizations, and hence, influenced the further interpretation of the fields in terms of universal tendencies of translation.

Firstly, my decision to select the same lexemes for each visualization was taken to ensure the comparability of the visualizations but had the effect that onomasiological levelling out could not be investigated as such. Secondly, the observation of a frequency threshold of three observations impacted the number of selected lexemes. A frequency threshold of two observations would have resulted in the addition of the following lexemes: \textit{aangaan} `to start', \textit{aanvatten} `to commence', \textit{begin-} `initial', \textit{doen} `to do', \textit{lanceren} `to launch', \textit{maken} `to make', \textit{nemen} `to take', \textit{sinds} `since', \textit{start-} `starting'. Thirdly, my choice to base the developed method on the translational hypothesis rather than on the distributional hypothesis has obviously played a decisive role in the further visualization of the semantic fields. Finally, the determination of the meaning distinctions on the basis of cluster significance, and, more generally, the decision to carry out a HAC on the output of a CA, the chosen distance measure and clustering algorithm, have all been decisive in the ``shaping'' of the semantic field structures. As a result, it becomes clear that more research will be needed to verify the stability of the visualizations before more fine-tuned interpretations of the semantic fields can be given. A number of alternative methodological possibilities will need to be tested before a deeper level of analysis of the semantic fields can be pursued. For example, the possibilities and limitations of the SMM++ would certainly need to be further explored to see whether the annotation of verb patterns such as ‘to be+ing-form’ is realistic within SMM++ (taking into account the expansiveness of the technique). In addition, a comparison of the results based on the translational hypothesis with results for the same data based on a distributional hypothesis (which relies on context words) could serve as a useful assessment of the stability of this translational method and could be seen as a first step towards a more fixed visualization method for semantic research in translation. To this extent, a first comparison carried out by \citet{vandevoorde_distributional_2016} showed that the distributional and the translational method yield similar visualizations of the semantic field of inchoativity in Dutch.

Due to the lack of previous work on the subject, I was left in the dark about what semantic levelling out, shining through or normalization would look like. This explains the explorative character of this study and my primary concern with the operationalizability of these universal tendencies on the semantic level. In this regard, I did not choose the case of \textit{beginnen}\slash inchoativity in function of testing one or the other universal but rather out of pragmatic – corpus frequencies – considerations, as a ``good for all'' test case. As a consequence, most of the main observations of this study are not clearly illustrating the one or the other tendency of translational behavior. One of the striking differences between the translated fields and the non-translated field concerns the clustering of \textsc{action} and \textsc{state after onset}. I failed to ascribe this phenomenon to either normalization or shining through. Since verbs of \textsc{action} and verbs of \textsc{state after onset} exist (although to different extents) in French, English and Dutch, the visualizations did not allow me to determine which influence (source or target language) was causing the changes in the semantic structures. Most possibly, \textsc{action}\slash \textsc{state after onset} is a case where there is neither a strong source language influence nor a prevailing target language influence, and both normalization and shining through (or none) are at play. Although it would have been more gratifying to expound clear cases of shining through and normalization, the reality of translational behavior is most probably often very similar to this situation of \textsc{action} and \textsc{state after onset}, where various influences cause subtle changes which ultimately alter translated language (when compared to non-translated language) but stay extremely difficult to tease apart and to capture. 

Although the two cognitive frameworks which were subsequently applied did not miraculously enable me to differentiate between shining through and normalization, \citeauthor{paradis_language_1980}’ idea of translation via the conceptual system offered a possible explanation for the observed phenomena in the translated semantic fields, without creating the need to tease apart source and target language influence (since the observed translational outcome is accounted for by what happens in the non-observable, non-linguistic conceptual system).

With this work, I hope to have opened the way for more semantic research in TS. A number of methodological developments presented in this book might constitute a first small step towards more research into semantic differences in translation and more cognitive explanations for translational behavior. I believe to have shown that, despite the difficulties to empirically investigate semantic phenomena, and despite notorious TS-related obstacles such as equivalence, it is possible to empirically investigate translation universals on the semantic level. The method that was put forward in this study as well as the idea to rely on statistical visualization to investigate semantic differences in translation might be further used and developed to explore semantic differences in translation and gain more insights into the mechanisms of translation on a more abstract, semantic level. Further research will eventually lead to clear hypotheses about semantic changes in translation which can subsequently be submitted to the types of frameworks I now applied tentatively and post-hoc.
