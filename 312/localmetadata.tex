\title{Handbook of Lexical Functional Grammar}
\author{Mary Dalrymple}
\renewcommand{\lsSeries}{eotms}%use series acronym in lower case
\renewcommand{\lsSeriesNumber}{13}
\renewcommand{\lsID}{312}
\typesetter{Mary Dalrymple, Sebastian Nordhoff}
\proofreader{Alec Shaw,
Alena Witzlack,
Alexandr Rosen,
Amy Amoakuh,
Andreas Hölzl,
Andrew Spencer,
Bev Erasmus,
Bianca Prandi,
Bojana Bašaragin,
Brett Reynolds,
Cesar Perez Guarda,
Christian Döhler,
Craevschi Alexandru,
Daniel Siddiqi,
Elliott Pearl,
Emma Vanden Wyngaerd,
Fahad Almalki,
Georgios Vardakis,
Giorgia Troiani,
György Rákosi,
Hannah Booth,
James Gray,
Jamie Y. Findlay,
Jane Helen Simpson,
Janina Rado,
Jeroen van de Weijer,
Kalen Chang,
Katja Politt,
Keira Mullan,
Kiranmayi Nallanchakravarthi,
Daniela Kolbe,
Lachlan Mackenzie,
Liam McKnight,
Marten Stelling,
Matthew Windsor,
Miriam Butt,
Jean Nitzke
Patricia Cabredo,
Peter Sells,
Phil Duncan,
Piyapath T Spencer,
Prisca Jerono,
Sean Stalley,
Sebastian Nordhoff,
Steven Kaye,
Sylvain Kahane,
Troy E. Spier,
Valeria Quochi,
Wilson Lui,
Will Salmon}
\renewcommand{\lsISBNdigital}{978-3-96110-424-6}
\renewcommand{\lsISBNhardcover}{978-3-98554-082-2}
\BookDOI{10.5281/zenodo.10037797}
\BackBody{Lexical Functional Grammar (LFG) is a nontransformational theory of
linguistic structure, first developed in the 1970s by Joan Bresnan and
Ronald M. Kaplan, which assumes that language is best described and
modeled by parallel structures representing different facets of
linguistic organization and information, related by means of
functional correspondences. This volume has six parts. Part I,
\textit{Overview and introduction}, provides an introduction to core syntactic
concepts and representations.
Part II, \textit{Grammatical phenomena}, reviews
LFG work on a range of grammatical phenomena or constructions.
PartIII, \textit{Grammatical modules and interfaces}, provides an overview of LFG
work on semantics, argument structure, prosody, information structure,
and morphology.
Part IV, \textit{Linguistic disciplines}, reviews LFG work in
the disciplines of historical linguistics, learnability,
psycholinguistics, and second language learning.
Part V, \textit{Formal and computational issues and applications}, provides an overview of
computational and formal properties of the theory, implementations,
and computational work on parsing, translation, grammar induction, and
treebanks.
Part VI, \textit{Language families and regions}, reviews LFG work
on languages spoken in particular geographical areas or in particular
language families. The final section, Comparing LFG with other
linguistic theories, discusses LFG work in relation to other
theoretical approaches.}
