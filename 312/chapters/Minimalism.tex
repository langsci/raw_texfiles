\documentclass[output=paper,hidelinks]{langscibook}
\ChapterDOI{10.5281/zenodo.10186044}
\title{LFG and Minimalism}
\author{Peter Sells\affiliation{University of York}}
\abstract{I compare and contrast LFG and the Minimalist Program   (\citealp{chomsky1995the-minimalist}) with regard to different   overall aspects of the frameworks: fundamental design properties,   the representation of phrase structure, the representation of   clausal grammatical information, the nature and role of syntactic   features, and the analysis of agreement.}

\IfFileExists{../localcommands.tex}{
   \addbibresource{../localbibliography.bib}
   \addbibresource{thisvolume.bib}
   \usepackage{langsci-optional}
\usepackage{langsci-gb4e}
\usepackage{langsci-lgr}

\usepackage{listings}
\lstset{basicstyle=\ttfamily,tabsize=2,breaklines=true}

%added by author
% \usepackage{tipa}
\usepackage{multirow}
\graphicspath{{figures/}}
\usepackage{langsci-branding}

   
\newcommand{\sent}{\enumsentence}
\newcommand{\sents}{\eenumsentence}
\let\citeasnoun\citet

\renewcommand{\lsCoverTitleFont}[1]{\sffamily\addfontfeatures{Scale=MatchUppercase}\fontsize{44pt}{16mm}\selectfont #1}
  
   %% hyphenation points for line breaks
%% Normally, automatic hyphenation in LaTeX is very good
%% If a word is mis-hyphenated, add it to this file
%%
%% add information to TeX file before \begin{document} with:
%% %% hyphenation points for line breaks
%% Normally, automatic hyphenation in LaTeX is very good
%% If a word is mis-hyphenated, add it to this file
%%
%% add information to TeX file before \begin{document} with:
%% %% hyphenation points for line breaks
%% Normally, automatic hyphenation in LaTeX is very good
%% If a word is mis-hyphenated, add it to this file
%%
%% add information to TeX file before \begin{document} with:
%% \include{localhyphenation}
\hyphenation{
affri-ca-te
affri-ca-tes
an-no-tated
com-ple-ments
com-po-si-tio-na-li-ty
non-com-po-si-tio-na-li-ty
Gon-zá-lez
out-side
Ri-chárd
se-man-tics
STREU-SLE
Tie-de-mann
}
\hyphenation{
affri-ca-te
affri-ca-tes
an-no-tated
com-ple-ments
com-po-si-tio-na-li-ty
non-com-po-si-tio-na-li-ty
Gon-zá-lez
out-side
Ri-chárd
se-man-tics
STREU-SLE
Tie-de-mann
}
\hyphenation{
affri-ca-te
affri-ca-tes
an-no-tated
com-ple-ments
com-po-si-tio-na-li-ty
non-com-po-si-tio-na-li-ty
Gon-zá-lez
out-side
Ri-chárd
se-man-tics
STREU-SLE
Tie-de-mann
}
   \togglepaper[42]%%chapternumber
}{}

\begin{document}
\maketitle
\label{chap:Minimalism}

\section{A framework for comparison}
\label{sec:minimalism:framework}

LFG and the Minimalist Program (MP;
\citealp{chomsky1995the-minimalist,chomsky00}) are not
straightforwardly comparable, as they are articulated in quite
different ways. Going back to the 1980s, it could be said that LFG
and Government-Binding Theory \citep{chomsky1981lectures} had a
certain amount of commonality of approach, but as MP has developed
from the earlier Government-Binding theory (GB), more and more
emphasis has been placed in MP on derivation (see
e.g.~\citealp{hng06,hornstein18}), rather than on information and
representation, which are of course cornerstones of LFG.

As both LFG and GB were responses to theoretical concerns about
``classical'' transformational grammar, which was developed during the
1970s, it is useful to start with the legacy of the early
transformational period, which I summarize in (\ref{legacy}):

\ea\label{legacy}\ea
the overt part of syntax is represented in a phrase structure tree
\ex
all information in syntax is structured
\ex
different parts of a syntactic representation may share information
\z \z

As GB has developed into MP, it has been assumed that
(\ref{legacy}b-c) refer to the same structure as (\ref{legacy}a):
that only the structures of phrase structure represent syntactic
information, and that relationships are expressed in that structure,
being established by movement operations. For instance, topicalization
of an object creates a relationship between a topic position and an
object position, as a result of a derivational operation in the MP.

LFG is a framework which is also based on the principle that all
syntactic information is structured, but importantly that not all
syntactic information is structure in the sense of phrase structure,
and so it embodies (\ref{legacy}) by having (at least) three aspects
to the overall representation of a sentence:

\ea\label{3dims}\ea
overt phrase structure (c-structure)
\ex
a clause-level representation of the information it conveys
 (f-structure)
\ex
an argument-structure representation for predicate-argument structure
 (a-structure)
\z \z

All syntactic frameworks have a means to represent argument structure,
and for any given predicate, its argument structure is structured
according to the Thematic Hierarchy (e.g.~\citealp{jackendoff72}) or
something equivalent. This is a-structure in LFG (see
\citetv{chapters/Mapping}). There is a
mapping between this structure and the surface grammatical properties,
f-structure, which is the representation of (\ref{3dims}b). These
properties include the GFs such as \textsc{subj} and \textsc{obj}.
The representation of clausal grammatical information (\ref{3dims}b)
is not part of the phrase structure representation (\ref{3dims}a), but
rather is the information that the overt structure conveys. This
clausal representation is nevertheless structured in the sense that
the information it contains is organized and grouped, according to
principles of organization pertinent to this level.

This is different to the approach to clausal information in the MP,
where information may start out quite distributed throughout the
overall derivation, but can be aggregrated through successive
movements, but also modified (e.g.~a feature specification being used
to drive one operation, then being deleted subsequent to the
application of that operation; see
\sectref{sec:minimalism:features-mp}). The core arguments of a predicate
are merged first into a vP-VP structure (see
\citealp[315ff.]{chomsky1995the-minimalist}) and this is the
representation of argument structure; the internal argument(s) merge
into VP and the external argument is the specifier of vP. Then further
functional structure such as TP or CP is projected above vP.
Functional or relational properties such as `subject' and `object' are
characterized by the particular `Agree' relations between v and Obj
and between T and Subj.
  
Broadly speaking, the ``subjecthood'' properties identified by
\citet{keenan76} divide into those which properly refer to argument
structure, and those which refer to clausal grammatical information
(\citealp{Manning1996}). Different syntactic phenomena may relate to
either representation. For instance, anaphor binding is determined by
the argument structure hierarchy in some languages
(e.g.~\citealp{schachter76} on Tagalog; \citealp{wechslerarka} on
Balinese). In other languages, the hierarchy of grammatical functions
holding over f-structure is most relevant (see
e.g.~\citealp[212--213]{bresnan2001lexical};
\citealp[217]{BresnanEtAl2016} on `syntactic rank').

\hspace*{-.1pt}The representations of argument structure and of clausal information
are large\-ly langu\-age-invariant, though the mapping between them
shows more variation, as do the ways in which different syntactic
phenomena refer to them. The information that they represent is
carried by the overt phrase structures, (\ref{3dims}a), which are of
course subject to the most variation, and therefore the least
revealing about ``deep'' properties of language.

In this chapter I evaluate different aspects of the LFG and MP
approaches to grammatical theory. In \sectref{sec:minimalism:2} I
consider overall ``design features'' of the frameworks, and what
motivates them. In particular I outline how LFG took a different
direction from transformational grammar. In
\sectref{sec:minimalism:ps} I contrast the approaches of the
frameworks to phrase structure, and how the balance of analysis
between c-structure and f-structure falls in LFG. Finally in
\sectref{sec:minimalism:features} I compare the role(s) that features
play in LFG and in MP, and how featural specifications participate in
agreement.

\section{Design features of a grammatical framework}
\label{sec:minimalism:2}

\citet{kaplan19} gives a personal statement of how the passage below
from \citet{chomsky1965aspects} inspired his research which became part of the
foundation of LFG (see e.g.~\citealp[173--174]{kaplanbresnan82}):

\begin{quote}
No doubt, a reasonable model of language use will incorporate, as a
basic component, the generative grammar that expresses the
speaker-hearer's knowledge of the language; but this generative
grammar does not, in itself, prescribe the character or functioning of
a perceptual model or a model of speech production.
(\citealp[9]{chomsky1965aspects}) 
\end{quote}

Kaplan also pursued the idea that linguistic complexity will be best
modelled through (possibly complex) interactions of different
(relatively simple) components, different representational dimensions,
inspired by \citet{simon62}.

In this section I will consider how LFG addresses the core aims of a
generative grammar, and how it has done so according to certain key
foundational properties which set it apart from the procedural
approach which has characterized the GB/MP approach led by Chomsky.

\subsection{Levels of adequacy}

One way to approach how a given framework takes up the agenda for
Generative Grammar is to consider how the framework concerns itself
with Chomsky's successive levels of adequacy:

\begin{quote}
To summarize briefly, there are two respects
in which one can speak of ``justifying a generative grammar.'' On one
level (that of descriptive adequacy), the grammar is justified to the
extent that it correctly describes its object, namely the linguistic
intuition -- the tacit competence -- of the native speaker. In this
sense, the grammar is justified on external grounds, on grounds of
correspondence to linguistic fact. On a much deeper and hence much
more rarely attainable level (that of explanatory adequacy), a grammar
is justified to the extent that it is a principled descriptively
adequate system, in that the linguistic theory with which it is
associated selects this grammar over others, given primary linguistic
data with which all are compatible. In this sense, the grammar is
justified on internal grounds, on grounds of its relation to a
linguistic theory that constitutes an explanatory hypothesis about the
form of language as such. The problem of internal justification -- of
explanatory adequacy -- is essentially the problem of constructing a
theory of language acquisition, an account of the specific innate
abilities that make this achievement possible.
(\citealp[26--27]{chomsky1965aspects})
\end{quote}

\largerpage
Since the GB era, Chomsky has taken explanatory adequacy to be the
focus of syntactic theorizing (\citealp{rizzi16};
\citealp{dalessandro19}). Yet to do this presupposes that there is a
core of facts and generalizations so that there is a stable set of
grammars which satisfy descriptive adequacy. \citet[55]{hornstein18}
presents a list of structural properties that syntacticians might
agree are the ``mid-level generalizations'' of grammar; see also
\citet[8]{dalessandro19} for a summary. For instance, data as in
(\ref{appear1})--(\ref{appear2}), from
\citet[261]{chomsky1973conditions}, lead to well-established
generalizations to classify verbs as being raising or control
predicates, and binding conditions on anaphors and pronouns:

\ea\label{appear1}
\ea[]{
They appeared to John to like each other.
}
\ex
[*]{They appealed to John to like each other.}
\z \z

\ea\label{appear2}\ea
[*]{We appeared to John to like us.}
\ex[]{
We appealed to John to like us. 
}
\z \z

At the level of what facts and what kinds of facts are in the domain
of syntax -- such as those just given -- frameworks such as GB/MP,
LFG, and HPSG (\citealp{pollardsag87, pollard1994head-driven}) are
roughly commensurate, and so can be compared as to how they embody
descriptive adequacy. Of course the formal details of a syntactic
system which is intended to have a good ``correspondence to linguistic
fact'' vary between each framework, but these are the easiest points
of comparison. I take up this kind of comparison in
Sections~\ref{sec:minimalism:ps} and \ref{sec:minimalism:features}
below.

LFG is a framework which has been developed to address descriptive
adequacy, and which can be part of broader cognitive or computational
approaches to human language, following the first Chomsky quote above.
In this sense, it perhaps could be argued that LFG committed 40 years
ago to what has become known in the MP literature as Chomsky's ``third
factor'' (\citealp[6]{chomsky2005three}) for explaining the format of
grammatical knowledge:


\eabox{\label{3factors}
``\ldots\ we should, therefore, be seeking three factors that enter
into the growth of language in the individual:
\begin{enumerate}
\item Genetic endowment, apparently nearly uniform for the species,  which  interprets part of the environment as linguistic  experience, \ldots
\item Experience, which leads to variation, within a fairly narrow range, \ldots
\item Principles not specific to the faculty of language \ldots
  \begin{itemize}
    \item[--] principles of data analysis that might be used in language acquisition
    \item[--] principles of structural architecture
    \item[--] principles of efficient computation''
  \end{itemize}
\end{enumerate}
}

The GB perspective on the language faculty put a great burden on an
innate Universal Grammar which is essentially a parameterized
blueprint for any individual grammar. This corresponds to Chomsky's
first factor. Over the last 25 years, the trend in the development of
the MP has been to reduce reliance on this purely innate component of
grammar, in favor of the third factor. The reference in that factor
to principles of data analysis and of structural architecture is quite
salient as these are the principles at the basis of the considerations
I raise in the next subsection, though of course this is not to imply
that frameworks such as LFG deny that there are any `first factor'
properties or principles of our language capacity. However, as
\citet[498]{ogrady12} comments: ``\ldots\ the shift of focus to
third-factor effects in generative grammar marks a milestone of
sorts. Not because the idea is new, for it is not. Broadly speaking,
the rest of the field has been committed to the primacy of
third-factor explanations for decades.''

\subsection{Foundational properties of syntactic systems}

What kinds of property are fundamental to syntax, to be emergent from
a theory which ``constitutes an explanatory hypothesis about the form
of language'', as in Chomsky's notion of explanatory adequacy?  From
the Minimalist perspective, the key notion here is the binary merge of
abstract syntactic elements -- `External Merge' for initial
structure-building, and `Internal Merge' for movement from an existing
position to another one. The structure is built up incrementally, with
steps in the derivation driven by categorial requirements of
combination or by features (see \sectref{sec:minimalism:merge} and
\sectref{sec:minimalism:features-mp} below); the terminal nodes of the
structure are spelled out morphologically after the syntactic
operations have taken place.

LFG has taken a different starting point as to what the key properties
of syntax are; in the rest of this subsection I highlight the
consequences of a few examples which determine the `lexicalist' and
`functional' (that is, information-based) aspects of LFG.

\subsubsection{Lexicalist}

LFG is a lexicalist framework, built on the assumption that the
terminals in the phrase structure are word-level entities, the
X$^0$s of X$'$-theory. The roots of this approach are in the
Lexicalist Hypothesis of \citet{chomsky1970remarks}. Chomsky argued
that the shared properties of different words based on the same lexeme
could be accounted for without recourse to transformation (a
nominalization transformation for the specific examples considered in
that paper), and he introduced X$'$-theory to account for structural
similarities across categories. LFG, like other declarative
frameworks, expands on this perspective, using other syntactic
information not directly represented in the phrase structure
(cf.~(\ref{3dims})) to capture the appropriate similarities. An
X$^0$ may be internally complex, carrying the same kinds of
information as may be expressed by other elements or configurations in
the syntax, but formed according to its constraints on morphology, not
on syntax.

The following Swedish example from \citet[29]{mullerwechsler14}
illustrates several properties which motivate the lexicalist
analysis. It involves coordination of an active and a passive verb:

\ea Swedish\\
\gll
Golfklubben begärde och beviljade-s marklov för banbygget efter en hel
del förhandlingar och kompromisser med Länsstyrelsen och
Naturvårdsverket.\\ 
golf.club.\textsc{def} requested and granted-\textsc{pass}
ground.permit for track.build.\textsc{def} after a whole part
negotiations and compromises with county.board.\textsc{def} and
nature.protection.agency.\textsc{def}\\%{\pxl}
\glt
`The golf club requested and was granted a ground permit for track
construction after a lot of negotiations and compromises with the
County Board and the Environmental Protection Agency.'
\z

M\"uller and Wechsler argue that this example does not involve
Right-Node Raising, but rather coordination of two finite verbs at the
X$^0$-level (\textit{begärde och beviljades}). Each verb is a
syntactic word, marked for past tense (the \textit{de} part of each),
and the second one is marked for passive (the \textit{s}). Hence the
voice alternation active/passive is represented on single words, and
does not involve spans of structure involving separate heads such as
V, v, and Voice. Additionally, the second verb is a straightforward
counterexample to the `Mirror Principle' (\citealp{baker85}), which is
supposed to diagnose a close relationship between syntactic structure
and word-internal morpheme structure. Swedish passive {\it -s\/}
always appears external to other tense or aspectual suffixes on the
word, even though in an expanded MP-style clausal structure the Voice
head would be taken to be lower than and therefore closer to the
lexeme stem with respect to Aspect or Tense heads.

The French example in (\ref{heureuse-ex}) also motivates both the
lexicalist approach, as well as the design feature that agreement is
not directional.

\ea\label{heureuse-ex} French\\
\gll
Je suis heureuse.\\
I am happy.\textsc{f.sg}\\
\glt
`I am happy.' (spoken by a female)
\z

Neither the subject pronoun \textit{je} nor the inflected verb
\textit{suis} are categorized or marked for gender -- as in English and
many other languages -- yet the predicate adjective is marked as
feminine (and singular). The non-formal linguistic intuition that the
adjective agrees with its subject, or ``agrees with a noun'', has been
the basis of many formalized linguistic analyses: the predicate
adjective is a target and the subject should be its controller. Yet
there is no plausible source in the lexical content of
(\ref{heureuse-ex}) for a feminine gender specification except for the
adjective. It is certainly true of the \textit{sentence}
(\ref{heureuse-ex}) that it expresses a meaning involving a feminine
subject, but the morphosyntactic basis of that meaning could not be
\textit{je} or \textit{suis}, under any plausible analysis of those words.

This example is very powerful. From it, it follows then that
\textit{heureuse} is a lexical item marked for feminine gender
independently of the syntactic structure in which it appears, as there
is no source for feminine in the rest of the structure. This entails
the Lexicalist Hypothesis, as each X$^0$ in the syntactic
structure has properties that do not refer to any other X$^0$ in
the structure -- usually referred to as `Lexical Integrity'
(e.g.~\citealp[92]{bresnan2001lexical}; \citealp[92]{BresnanEtAl2016}).

\subsubsection{Information-based clausal representation}
\label{sec:minimalism:info-based}

Next, from the same example, it follows that ``agreement'' is the name
we give to a situation in which two or more syntactic elements put
constraints on a single informational unit, but that there is no
priority of one element over the other(s). In (\ref{heureuse-ex}), all
three such elements (the words in this case) put constraints on what
the subject is; and as the combination of those constraints is not
contradictory, the example is well-formed. The f-structure
contribution of each word is shown in
(\ref{heureuse}). (\ref{f-s-heureuse}) shows the f-structure for the
full example.

\eabox{\label{heureuse}\ea
%\evnup[2pt]{
 \avm[style=fstr]{%\begin{avm}
[pred & `pro'\\
  pers & 1\\
  num & sg ]
  }%\end{avm}
 % }
  \hbox to 14em{\hfill = je}
\vspace{0.5em}
\ex
%\evnup[2pt]{
\avm[style=fstr]{%\begin{avm}
[ subj&\rnode{a}{[pers & 1\\
                   num & sg]}\smallskip\\
         {tense} & pres\\
         pred & `{be}\arglist{(\UP \SUBJ)(\UP \XCOMP)}' \\ 
  xcomp & [ subj & !\rnode{b}{[\ \ ]}! ]  ]
}%\end{avm}
%}
\hbox to 6em{\hfill = suis}
\vspace{0.5em}
%\rtor[3.5]{a}{b}
%\nccurve[ncurv=3.5]{a}{b}
\CURVE[2.7]{-2pt}{0}{a}{-2pt}{0}{b}
\ex
%\evnup[2pt]{
\avm[style=fstr]{%\begin{avm}
[ subj&[ gend & fem ]\\
         pred & `happy\arglist{(\UP \SUBJ)}']
}
\hbox to 11em{\hfill = heureuse}
  \z }

  \eabox{\label{f-s-heureuse}%\evnup[2pt]{
  \avm[style=fstr]{%\begin{avm}
[ subj&\rnode{a}{[pred & `pro'\\
         pers & 1\\
         num & sg\\
         gend & fem]}\smallskip\\
         {tense}&pres\\
  pred&`be\arglist{(\UP \SUBJ)(\UP \XCOMP)}'\\
  xcomp & [ subj & !\rnode{b}{[\ \ ]}!\\
         pred&`happy\arglist{(\UP \SUBJ)}']  ]
}%\end{avm}
%}
%\rtor[3.5]{a}{b}
%\nccurve[ncurv=3]{a}{b}
\CURVE[2.5]{-2pt}{0}{a}{-2pt}{0}{b}  
}

Local syntactic relationships indicated by the structure-sharing seen
above are typically located in a predicate -- these will involve what
kind of arguments the predicate takes, possibly specifications of
case, whether it is a raising or control predicate, agreement
information, and so on. The apparent directionality of agreement seen
in canonical examples has nothing to do with ``agreement'' itself --
as a mechanism of agreement does not exist -- but rather comes from
the second property of \textit{suis}, that it is effectively a raising
predicate, and so whatever is true of its complement's subject is true
of its subject. In (\ref{f-s-heureuse}) the information shown as the
value of \textsc{subj} is the minimal amount of which the constraints
coming from each of the entries in (\ref{heureuse}) is true.

With regard to the implications for explanatory adequacy, this simple
example shows that the format of grammars is only consistent with
those that lack derivation and directionality -- in other words, if
the hypothesis space is restricted to declaratively stated grammars,
we expect that languages will quite generally show examples like
(\ref{heureuse-ex}). In \sectref{sec:minimalism:agreement}, I take up in
more detail the key information-based properties of what we informally
refer to as ``agreement''.

Other examples also show the importance of the information that an
item carries over its phrase structure properties. (\ref{hudson})
(originally from \citealp{hudson77}; see also \citealp[64]{gkps};
\citealp[19]{bresnan2001lexical}, \citealp[14]{BresnanEtAl2016})
illustrates one of the ``paradoxes of movement'':

\ea\label{hudson}\ea[*]
{I aren't happy.}
   \ex
Aren't I happy?
\z
\z

An initial positioning of an internally-negated auxiliary is taken as
evidence in movement analyses that the auxiliary has undergone several
movements, combining with Neg and then with T[ense] before moving to
C. However, from the notional analytic source \textit{I am not happy}
there is no pre-T-to-C version *\textit{I amn't happy}, as in (standard)
English there is no form \textit{amn't}, and (\ref{hudson}a) is also
ungrammatical. In this use, then, the form \textit{aren't} is a word
which can only appear in the C, or inverted-aux, position, but not in
any other position. In terms familiar from the early days of
transformational grammar, (\ref{hudson}b) would have to be analyzed as
a grammatical example derived from an ungrammatical source.

The contrast in (\ref{hudson}) shows that the syntactic properties
that an item has (being a tensed negated auxiliary in this case) are
not inexorably associated with structural derivations which aggregate
information. A movement-based account of (\ref{hudson}) would have to
assume that the syntactic features of \textit{aren't} can be assembled
on T, and from there moved to C, but that there is no lexical item
which can spell out those features on T, but only on C. In other
words, what actually matters is the surface position of the
realization of a set of syntactic properties, not where (or where
else) those properties came from. This is exactly what a declarative
framework such as LFG provides, with the same implication for
explanatory adequacy -- if an element in a higher position must
correspond to a derivationally related version of itself in a lower
position, pairs like (\ref{hudson}) should not exist. But they do
exist, and they show that the format of grammars should recognize that
words have bundles of features which are associated with (sets of)
syntactic positions. Within the broader Miminalist approach, the
realizational account of morphology in the Distributed Morphology
framework (\citealp{hallemarantz}; see \citealp{bobaljik17} for a
recent overview) has the same sensitivity to syntactic position: for
\textit{aren't}, a rule of vocabulary insertion could be made sensitive
to the collection of relevant syntactic features in the context of C,
but not of T.

\subsection{Rules and representations}

As syntactic frameworks have developed since the 1980s, they have
diverged as to whether the focus is on constraints stated on
representations, or on steps in a procedural
derivation. Government-Binding theory has a mix of properties:
conditions on rule application and conditions on representations. For
instance, the examples above in (\ref{appear1})--(\ref{appear2})
involve \textit{appear} as a raising predicate which requires an
operation of the rule Move-$\alpha$, while \textit{appeal} is a control
predicate which requires a representational check involving a
\textsc{pro} subject (see e.g.~\citealp{haegeman94} for a summary of
GB). More recently, the ``movement theory of Control''
(e.g.~\citealp{hornpoli10}) eliminates the representational condition
on the empty category subject in favor of a derivational analysis
similar in the relevant ways to the one for the raising predicate.

The GB Binding Theory Principles A and B were originally each stated
as a condition on a representation. For instance, Principle A looks
for a specific relationship of coindexing between antecedent and
anaphor within a certain domain; reinterpreted as a condition on rule
application, the principle must involve an operation of movement up to
(near) the antecedent, within a certain domain, following an idea
first proposed in \citet{lebeaux83}.

In the development of the MP, Chomsky has taken
the view that as some aspects of the grammar are procedural, and so
require conditions on rule application, parsimony would dictate that
all grammatical conditions are of that type, with no conditions on
representation. Hence the levels of GB over which representational
conditions were stated were eliminated. The MP is an
attempt to deconstruct GB along purely procedural aspects (see
e.g.~\citealp[54]{hornstein18}) -- in the limit, there is no
``stopping off'' at any point to evaluate a representation. In fact,
though, each step in a derivation must involve a local representation
-- but one within which or to which some further operation should take
place. The proposed operation of `Minimal Search' for the operation
{\it Agree\/} (\citealp[9]{chomsky07}) must inspect a structure to
find something within it -- here, an element W probing within a
structure Z -- and the outcome of that will constrain what
(procedurally) can happen next: ``Since W contains all information
relevant to further computation involving Z, W is also necessarily the
probe that selects a goal in any internal modification of Z. Minimal
search conditions limit the goal of the probe to its complement, the
smallest searchable domain.''

The output of syntax is fed to the `interfaces'. On the semantic
side, the end of the syntactic derivation corresponds to the GB level
of Logical Form (LF), which feeds to the `conceptual-intentional'
interface. On the phonetic side, the overt output of the derivation is
spelled-out to Phonetic Form (PF), which feeds to the interface
known as `articulatory-perceptual' or `sensorimotor' (see
e.g.~\citealp[2]{chomsky1995the-minimalist},
\citealp[5]{chomsky07}). One leading idea of the Minimalist Programme
is that LF and PF have no properties specific to them; rather, any
apparent well-formedness conditions are due entirely to properties of
the interfaces.

Within the core domain of syntax, there seem to be several phenomena
which bear on the issue of rules vs.~representations, and which appear
to favor the latter -- because their analysis seems irreducibly
representational. I will mention two different instances and then go
on to two others in more detail. First, as just noted, the MP
operation of Agree has to access a representation, in order to
establish a relation between Probe and Goal (see also
\sectref{sec:minimalism:features-mp}). Second, the approach to case marking
known as `Dependent Case' (e.g.~\citealp{baker15}) calculates the case
values of NPs by referring to larger structure -- the underlying
intuition being that in a clause containing two NPs, a subject
c-commanding an object, the marked case value of Accusative for the
object is the value assigned to an NP c-commanded by another, and in a
typologically different system, the marked case value of Ergative for
the subject is the value assigned to an NP which c-commands
another. Hence the computation of case values must refer to a
structural representation.

I now go in more depth into two instances which illustrate a different
kind of representational condition -- a negative condition. It is
difficult to imagine how such conditions could successfully be
captured procedurally. Returning to the binding conditions of GB, for
Principle A, there have been different proposals to reinterpret it
derivationally, for instance \citet{lidzidsardi98},
\citet{hornstein01} and \citet{boeckx07}, though others take a more
traditional view, such as \citet{safir08} and
\citet{charnavelsportiche16}. While Principle A requires two elements
to have a certain relationship, Principle B forbids two elements from
having a certain relationship -- it is a negative condition. A
procedural reinterpretation does not seem directly possible for
Principle B, as it requires disjointness (unless perhaps the system of
recording contra-indexing of \citealp{chomsky80} is revived), though
\citet{reuland11} presents a revised Binding Theory which refers to
properties of predicates and semantic constraints on the
interpretation of their arguments.

Principles A and B as they apply to English are familiar. In some
languages, with anaphoric systems more complex than that found in
English, conditions on the various elements of the system may involve
both positive and negative constraints -- such as in Norwegian
(\citealp[279--288]{dalrymple01},
\citealp[259--261]{BresnanEtAl2016}). Norwegian has four relevant
anaphor/pronoun forms, shown in (\ref{norwegian}) with their LFG
binding properties. The content of the binding properties is given in
(\ref{binding}):

\ea\label{norwegian}Featural analysis of Norwegian pronouns:
\ea
\makebox[5em][l]{seg}[$+$sbj, $-$ncl]
\ex
\makebox[5em][l]{ham}[$-$ncl]
\ex
\makebox[5em][l]{seg selv}[$+$sbj, $+$ncl]
\ex
\makebox[5em][l]{ham selv}[$-$sbj, $+$ncl]
\z\z

\ea\label{binding}\ea\relax
[$+$sbj, $-$ncl] – The antecedent must be a subject in the minimal
finite domain outside of the minimal nucleus containing the pronoun.
\ex\relax
[$-$ncl] – The antecedent must be outside of the minimal
nucleus containing the pronoun. 
\ex\relax
[$+$sbj, $+$ncl] – The antecedent must be a subject in the minimal
nucleus containing the pronoun. 
\ex\relax
[$-$sbj, $+$ncl] – The antecedent must be a nonsubject in the minimal
nucleus containing the pronoun.
\z\z

The negative conditions here seem to refer crucially to
representations -- to check that a relationship does not hold in a
certain local domain, or to check that a relationship does hold, but
not with a subject.

A different consideration about the role of representations comes from
the distribution of the depictive \textit{sisxoli} `alone' in Tsez, a
language of the Caucasus which has an ergative-absolutive case
marking system. The depictive may be associated with a preceding NP,
but may not itself precede its NP associate
(\citealp{polinsky00}). Hence the depictive has two possible
associates in (\ref{tsez-1}a), one in (\ref{tsez-1}b), and none in (\ref{tsez-1}c).

\ea\label{tsez-1} Tzez
\ea
\gll kid-b\={a}  ziya sisxoli bi\v{s}er-si\\
girl-{\ERG} cow.{\ABS} alone feed-\textsc{pst.evid}\\
\glt `The girl$_i$ alone$_i$ fed the cow.'\\
`The girl fed the cow$_i$ alone$_i$.'
\ex
\gll kid-b\={a} sisxoli ziya bi\v{s}er-si\\
girl-{\ERG} alone cow.{\ABS} feed-\textsc{pst.evid}\\
\glt `The girl$_i$ alone$_i$ fed the cow.'\\
{*}`The girl fed the cow$_i$ alone$_i$.'
\ex[*]
{\gll sisxoli kid-b\={a}  ziya bi\v{s}er-si\\
alone girl-{\ERG} cow.{\ABS} feed-\textsc{pst.evid}\\}
\z \z

The linear precedence condition is reinterpreted as one of c-command in
later discussions of these same examples in \citet{polipots06} and
\citet{fukuda08} -- the associate must c-command the depictive.

The distribution of the depictive becomes more interesting in the
context of raising and control predicates. In Tsez the predicate
\textit{yoq-} `begin' is ambiguous between control and raising, and in
fact is a backward control predicate in its control use or a forward
raising predicate in its raising use (\citealp{polipots06}). In LFG,
the higher and lower \textsc{subj} values of control or raising are
structure-shared in f-structure, and as discussed in
\citet{sellssubsump} that f-structure property is consistent with
c-structure expression of the relevant argument in the matrix clause
(`forward') or in the embedded clause (`backward'). (\ref{b-c}a) is an
interesting example regarding the syntax of the depictive, as it is
grammatical even though the depictive apparently precedes its
associate. \citet{polipots06} analyze this as a backward control
structure: a null (absolutive) subject of `begin', indicated by
$\emptyset$ in (\ref{b-c}b), controls the lower (ergative) subject of
`feed'. $\emptyset$ is used here as a notation to suggest the analysis
of the example, but it has no actual correspondent in the c-structure,
as is standard in the LFG analysis of control and raising. In this
example, it is the null matrix argument indicated by $\emptyset$ which
is the associate of the depictive, and both are constituents of the
main clause (see the f-structure in (\ref{fs-b-c}) below):

\ea\label{b-c} Tzez
\ea
\gll sisxoli kid-b\={a} ziya  bi\v{s}ra  yoq-si (backward control)\\
alone girl-{\ERG}  cow.{\ABS}  feed    begin-\textsc{pst.evid} {}\\
\glt `The girl$_i$ alone$_i$ began to feed the cow.'\\
{*}`The girl began to feed the cow$_i$ alone$_i$.'
\ex
\gll $\emptyset_i$ sisxoli [kid-b\={a}$_i$ ziya  bi\v{s}ra]  yoq-si\\
{} alone [girl-{\ERG}  cow.{\ABS}  feed]    begin-\textsc{pst.evid} \\
\z\z

The other use of \textit{yoq-} is as a regular forward raising
predicate. Its subject is in absolutive case as the predicate is not
formally transitive, and the subject in the lower clause is the empty
position, again indicated here by $\emptyset$. As seen in (\ref{f-r}a),
with the syntactic analysis in (\ref{f-r}b), the same order of elements as
in (\ref{b-c}a) is ungrammatical in this instance, as the depictive does
indeed precede its associate:

\ea\label{f-r} Tzez
\ea[*]
{\gll sisxoli kid [ziya  bi\v{s}ra]  yoq-si (forward raising)\\
  alone girl-{\ABS}  [cow.{\ABS}  feed]    begin-\textsc{pst.evid} \\
}
\ex[*]
{\gll sisxoli kid$_i$ [$\emptyset_i$ ziya  bi\v{s}ra]  yoq-si \\
alone girl-{\ABS}  [{} cow.{\ABS}  feed]    begin-\textsc{pst.evid} \\
}
\z\z

The LFG account of these data requires the concepts of f-command, which
is like c-command but stated on f-structure, and of f-precedence (see
Glossary for f-command and f-precedence). This latter concept makes
reference to the c-structure expression(s) -- if any -- of f-structure
elements. Crucially, an element such as a null argument which is
present only in f-structure, but not in c-structure, has no
(f-)precedence relations defined on it
(\citealp[195]{bresnan2001lexical};
\citealp[213]{BresnanEtAl2016}). The LFG analysis of the Tsez
depictive can be stated simply as in (\ref{lfg-depict}):

\ea\label{lfg-depict}\ea
The associate and the depictive f-command each other.
\ex
The depictive must not f-precede the associate.
\z\z

\newpage
(\ref{lfg-depict}a) is essentially a clause-mate condition, and (\ref{lfg-depict}b) is a
negative condition. It does not require that the associate f-precede
the depictive, but rather that the depictive does not f-precede the
associate.

The f-structure of (\ref{b-c}) is shown in (\ref{fs-b-c}), leaving out the
case values of the arguments, which would formally conflict under
straightforward structure-sharing (i.e.~formal equality in LFG
terms). The case values require a slightly nuanced analysis, whatever
the framework (see \citealp{polipots02}, \citealp{sellssubsump}). For
presentational purposes, I assume here that the formal relation
between depictive and associate is that they share an \textsc{index}
value. Their GFs which f-command each other are indicated by the
boldface GF names, in the matrix nucleus. The matrix \textsc{subj} is
structure-shared with the embedded \textsc{subj} as the predicate is
backward control. While there is a matrix \textsc{subj} in
f-structure, it has no c-structure expression (there is no
`$\emptyset$' in the c-structure); only the embedded subject is
present in c-structure. Consequently, limited to the matrix
f-structure in which the associate and depictive f-command each other,
no precedence relation is defined on the boldface \textsc{subj}, and
so the condition in (\ref{lfg-depict}b) is also satisfied.

\vbox{%
\ea\label{fs-b-c}
F-structure of (\ref{b-c}), ignoring the case values. \textbf{\textsc{adj}}
does not f-precede \textbf{\textsc{subj}}:\\
\evnup[2pt]{\avm[style=fstr]{%\begin{avm}
[  pred & `begin\arglist{(\UP \SUBJ)(\UP \XCOMP)}'\\  
   tense & past\\ %\\
   \textbf{subj} & \rnode{a}{[ \phantom{case}  & \phantom{abs} ]}\smallskip\\ %\\
   \textbf{adj} & %\osort{$3$}
          \rnode{c}%
                 {[ pred & `alone'\\
                      index & $i$ ]}\smallskip \\ %\\
   xcomp & [ subj & \rnode{b}{[ pred & `girl'\\
                                 pers & 3\\
                                 num & sg\\
                                 index & $i$\\
                                 %\phantom{case} & \phantom{erg}
                                 ]}\smallskip \\
              obj & [ pred & `cow' ] \\
              pred & `feed\arglist{(\UP \SUBJ)(\UP \OBJ)}'\\ 
             ] \\ 
]
}%\end{avm}
  }
\CURVE[2]{-2pt}{0}{b}{-2pt}{0}{a}
%\nccurve[ncurv=2]{b}{a}
\z
}
  
In contrast, for (\ref{f-r}), involving a forward raising use of the
predicate, the  constraint in (\ref{lfg-depict}b) is not satisfied,
because the \textsc{subj} is overt in the matrix clause, and so
f-precedes the depictive \textsc{adj}.

\vbox{%
\ea\label{fs-f-r}
F-structure of (\ref{f-r}), ignoring the case
  conflict. \textbf{\textsc{adj}} f-precedes \textbf{\textsc{subj}}:\\ 
\evnup[2pt]{\avm[style=fstr]{%\begin{avm}
[  pred & `begin\arglist{(\UP \XCOMP)}(\UP \SUBJ)'\\ 
   tense & past\\
   \textbf{subj} & \rnode{a}{[ pred & `girl'\\
                                 pers & 3\\
                                 num & sg\\
                                 index & $i$\\
                                 %\phantom{case} & \phantom{erg}
                                 ]}\smallskip\\
   \textbf{adj} & \rnode{c}{[ pred & `alone'\\
                      index & $i$ ]}\smallskip \\
   xcomp & [ subj & \rnode{b}{[ \phantom{case}  & \phantom{abs} ]}\smallskip \\ 
              obj & [ pred & `cow' ] \\
              pred & `feed\arglist{(\UP \SUBJ)(\UP \OBJ)}'\\ 
             ] \\ 
]
}%\end{avm}
}
\CURVE[2.5]{-2pt}{0}{b}{-2pt}{0}{a}
%\nccurve[ncurv=1.8]{b}{a}
\z
}

%\dashline
%\nccurve{c}{a}

The Minimalist account in \citet{polipots06} (see also
\citealp{fukuda08}) is stated in terms of positive conditions, of
which (\ref{mp-depict}b) is the important one.

\ea\label{mp-depict}\ea
The associate and the depictive are clause-mates.
\ex
The associate c-commands/binds ($\equiv$ precedes) the depictive.
\z\z

What is interesting about (\ref{mp-depict}b) is that it can only be successfully
interpreted representationally. Suppose that at one point in the
derivation, the associate (whether overt or covert) c-commands the
depictive, and the relevant syntactic relationship is established,
satisfying (\ref{mp-depict}b). However, what is to prevent some later operation
which scrambles the depictive higher, so that it c-commands its
associate, in violation of (\ref{mp-depict}b)? To prevent this possibility,
(\ref{mp-depict}b) must be interpreted as an output condition on the ``final''
representation, regardless of when during the derivation the relation
between associate and depictive has been established. Hence even
though (\ref{mp-depict}b) is a positive condition, not a negative one, it is
necessarily representational.

In the LFG analysis, (\ref{lfg-depict}b) is necessarily a negative
condition, as the null subject in backward control is only represented
in f-structure (\ref{b-c}), and so could never be evaluated against a
positive precedence condition like (\ref{mp-depict}b). Evidence from other
languages supports the position that null arguments are present in
f-structure but absent from c-structure. Null pronouns in Malayalam
are not sensitive to f-precedence conditions, unlike overt pronouns
(\citealp[664--665]{mohanan83}). \citet{kameyama85} presents a similar
argument for Japanese (summarized in \citealt[171ff.~and
  288ff.]{dalrymple01}). For Malayalam, Mohanan observes that an overt
pronoun may not precede its antecedent -- compare (\ref{malay-1}) and
(\ref{malay-2}a) with (\ref{malay-2}b) -- while a null pronoun (indicated
for presentational purposes by \textit{pro} in (\ref{malay-2}b)) may
`precede' its antecedent:

\ea\label{malay-1} Malayalam
\ea
\gll
[ku\d{t}\d{t}iyute ammaye] awan \b{n}u\d{l}\d{l}i\\
[child.\textsc{gen} mother.\textsc{acc}] he.\textsc{nom} pinched\\
\glt `He$_i$ pinched the child$_i$’s mother.'
\ex[*]
{\gll [awante ammaye] ku\d{t}\d{t}i \b{n}u\d{l}\d{l}i\\
[he.\textsc{gen} mother.\textsc{acc}] child.\textsc{nom} pinched\\
\glt  `The child$_i$ pinched his$_i$ mother.'}
\z\z

\ea\label{malay-2} Malayalam
\ea
\gll
{[}awan  aanaye \b{n}u\d{l}\d{l}iya\b{t}in{\textschwa} \={s}ee\d{s}am]
ku\d{t}\d{t}i{$_i$} ura{\texteng}{\texteng}i\\
{[}he.{\NOM} elephant.{\ACC} pinched.it after] child.{\NOM} slept\\
\glt `The child$_i$ slept [after he$_{*i/j}$ pinched the elephant].'
\ex
\gll {[}\textit{pro} aanaye \b{n}u\d{l}\d{l}iya\b{t}in{\textschwa}
  \={s}ee\d{s}am] ku\d{t}\d{t}i{$_i$} ura{\texteng}{\texteng}i\\
{[} elephant.{\ACC}  pinched.it after] child.{\NOM} slept\\
\glt `The child$_i$ slept [after he$_{i,j}$ pinched the elephant].'
\z\z

The overt pronoun `he' in (\ref{malay-2}a) may not take `child' as its
antecedent, as the former precedes the latter, but this restriction is
not there with the null pronoun in (\ref{malay-2}b). Why would overt and
null pronouns have different precedence conditions on them?
\citet[664]{mohanan83} proposes that the correct analysis is that a
pronoun cannot precede its antecedent, where precedence is defined on
c-structure elements, such as overt pronouns, but is not defined for
null pronouns, which are present only in f-structure.
% This condition has to be defined over a structure of arbitrary size.

Consider the c-structure relationships of the relevant
parts of the examples, shown in (\ref{malay-c-f}a), with the f-structure of
the example shown in (\ref{malay-c-f}b). The subscript numbers show the
c-to-f-structure correspondences:

\ea\label{malay-c-f}
\ea C-structure:\\
\rnode{a}{(pronoun$_1$)}\hspace{50pt}\rnode{b}{pinched}$_2$\hspace{50pt}
\rnode{c}{child}$_3$\\[-0.1em]
%\vbox
\ex
F-structure:\\[.5em]
{\avm[style=fstr]{%\begin{avm}
[ subj&\id{3}{[pred&`child']}\smallskip\\
   pred & `sleep\arglist{(\UP \SUBJ)}'\\ 
   tense & past\\
   adj& \id{2}{[ pred& `pinch'\\
       subj& \id{1}{[pred & `pro' ]}\smallskip\\
 \ldots & \ldots ]} ]
}%\end{avm}
}\z\z

In both examples in (\ref{malay-2}), the adjunct clause 2 f-precedes
3, `child', because the c-structure correspondent(s) of 2 precede the
correspondent(s) of 3. However with regard to 1 and 3, 1 f-precedes 3
only if 1 is present in c-structure, which is only the case in
(\ref{malay-2}a). Hence the apparently different binding properties of
pronouns reduce to their different properties in different parts of
the syntactic analysis.

The implications of this analysis are far-reaching: if certain
syntactic elements can have a range of grammatical properties without
being represented in phrase structure -- and the above is positive
evidence that they are not represented in phrase structure -- then
every aspect of grammatical analysis which can or must refer to those
properties must also be independent of any phrase structure
representation, including phenomena such as subjecthood, agreement,
binding, and so on.

Declarative frameworks have different dimensions of analysis -- such
as c-structure and f-structure as described below -- but not different
levels in the sense that GB had (e.g.~D-structure, S-structure). As
there are no rules or operations, there are no conditions on rules,
and so all conditions are stated over representations, as constraints.

\section{Phrase Structure}
\label{sec:minimalism:ps}
\largerpage[2]
\subsection{Heads and headed structures}
\label{sec:minimalism:heads}

LFG c-structures have some similarities with the S-structures of late
GB. A canonical clause (for an SVO language) is structured around
what I refer to as a `skeleton' with a `spine'
(\citealp[17]{sellssao}). (\ref{skeleton}) below shows the skeleton,
and the spine corresponds to all the non-argumental parts, V, I, C and
their projections. These are separate categories which participate in
the familiar clausal extended projection
(\citealp[116ff.]{grimshaw00ep}; \citealp[100]{bresnan2001lexical}),
often now referred to as the `Hierarchy of Projections'
(\citealp{adger03}).

The formal relation in the c-structure between V and I and C is
usually developed from the idea of `extended projection' of
\citet{grimshaw00ep}; see also \citet[100--101]{bresnan2001lexical},
\citet[103]{BresnanEtAl2016}. The clausal categories are all
projections of the category verb, which is specified by the
traditional labels [$+$V, $-$N] (\citealp{chomsky1970remarks}).
  
\ea\label{ext-proj}
Extended Projections
\ea
\makebox[2em][l]{V}\makebox[1em][l]{=}\makebox[7em][l]{[$+$V, $-$N, P0]} (the zeroth-level projection
of V) 
\ex
\makebox[2em][l]{I}\makebox[1em][l]{=}\makebox[7em][l]{[$+$V, $-$N, P1]} (the first-level
projection of V) 
\ex
\makebox[2em][l]{C}\makebox[1em][l]{=}\makebox[7em][l]{[$+$V, $-$N, P2]} (the second-level projection
of V) 
\z\z

The outline clause structure has specifiers of CP and IP, and
complement position(s) within VP, schematized here with the
placeholder label Complement.

\ea\label{skeleton}
\begin{tabular}[t]{lll}
\begin{forest}
  for descendants={base=bottom}
  [ \rnode{cp}{CP}
    [ XP ]
    [ \rnode{cb}{C$'$} [ \rnode{c}{C} ] 
      [ \rnode{ip}{IP}
        [ \rnode{s}{NP} ]
        [ \rnode{ib}{I$'$}
          [ \rnode{i}{I} ]
          [ \rnode{vp}{VP}\hspace*{-12mm}\rdash [ \rnode{v}{V} ]
            [ Comp\rlap{lement} ]
          ]  
        ]
      ]
    ]
  ]
\end{forest}
& \phantom{more space} &
\lower1em\hbox{\evnup[2pt]{\avm[style=fstr]{%\begin{avm}
\rnode{fs}{[ subj&\rnode{sbj}{[\ldots ]}\\
         {tense}&\ldots\\
  pred&\ldots\\
  \ldots & \ldots  ]}
}%\end{avm}
}}
\end{tabular}
\nccurve[linestyle=dashed,nodesepA=2pt,nodesepB=0pt,angleA={115},angleB={170},linewidth=.5pt]{->}{s}{sbj}
\nccurve[linestyle=dashed,nodesepA=2pt,nodesepB=0pt,angleA={0},angleB={175},linewidth=.5pt]{->}{cp}{fs}
\nccurve[linestyle=dashed,nodesepA=2pt,nodesepB=0pt,angleA={0},angleB={175},linewidth=.5pt]{->}{ip}{fs}
%\nccurve[linestyle=dashed,nodesepA=2pt,nodesepB=0pt,angleA={0},angleB={175},linewidth=.5pt]{->}{cb}{fs}
%\nccurve[linestyle=dashed,nodesepA=2pt,nodesepB=0pt,angleA={0},angleB={175},linewidth=.5pt]{->}{ib}{fs}
\nccurve[linestyle=dashed,nodesepA=2pt,nodesepB=0pt,angleA={0},angleB={175},linewidth=.5pt]{->}{vp}{fs}
\nccurve[linestyle=dashed,nodesepA=2pt,nodesepB=0pt,angleA={15},angleB={175},linewidth=.5pt]{->}{c}{fs}
\nccurve[linestyle=dashed,nodesepA=2pt,nodesepB=0pt,angleA={15},angleB={175},linewidth=.5pt]{->}{i}{fs}
\nccurve[linestyle=dashed,nodesepA=2pt,nodesepB=0pt,angleA={10},angleB={175},linewidth=.5pt]{->}{v}{fs}
\z

Each node in the c-structure is annotated as to how it contributes to
the f-structure. The formal annotations on nodes are not shown in
(\ref{skeleton}), for simplicity, but the dashed lines represent the way that
information flows from the c-structure to the f-structure.

Every node in the clausal spine contributes its information to the
main f-structure, as can be seen from the several lines converging on
the outer f-struc\-ture, which represents the grammatical information of
the clause (again for simplicity, I omit lines from the X$'$ nodes).
The other nodes, XP, NP, and Complement, have different annotations on
them, as they contribute to parts of the overall f-structure. For
instance, the annotation on the node NP would indicate that its
contribution is as the subject -- in other words, NP as specifier of
IP is the subject position. This is indicated by the dashed line going
from NP in the c-structure to the value of \textsc{subj} in the
f-structure.

As far as clausal information is concerned, the verb itself
contributes identically to the clause whether it is in V or in I or in
C, a property usually referred to as `head mobility' (see
e.g.~\citealp[126ff.]{bresnan2001lexical};
\citealp[129ff.]{BresnanEtAl2016}). For instance, unless extra
information is associated with the C node in (\ref{ex:Minimalism:25}b), both
c-structures in (\ref{ex:Minimalism:25}) would determine the same f-structure:

\ea\label{ex:Minimalism:25}\ea
%\begin{tabular}[t]{lll}
\begin{forest}
 [IP
   [ {(\UP\SUBJ)=\DOWN\\NP} [ Maria, roof ] ]
   [ {\UP=\DOWN\\I$'$}
    [ {\UP=\DOWN\\I} [ is ] ]
    [ {\UP=\DOWN\\VP} [{(\UP\XCOMP)=\DOWN\\AP} [ {\UP=\DOWN\\A} [ happy ] ] ] ] ] ]
\end{forest}
b. % fake \ex
\begin{forest}
  [ CP
    [ {\UP=\DOWN\\C$'$}
      [ {\UP=\DOWN\\C} [ is ] ]
      [ {\UP=\DOWN\\IP}   
        [ {(\UP\SUBJ)=\DOWN\\NP} [ Maria, roof ] ]
        [ {\UP=\DOWN\\I$'$}
          [ {\UP=\DOWN\\VP} [{(\UP\XCOMP)=\DOWN\\AP} [ {\UP=\DOWN\\A} [ happy ] ] ] ] ] ] ] ]  
\end{forest}
%\end{tabular}
\z\z

Head mobility can be illustrated with the c-structures above. On the
assumption that the only VP can be the c-structure complement of I,
then for the example {\it Maria is happy\/} in (\ref{ex:Minimalism:25}a) the VP lacks a
c-structure head V, for the verb \textit{is} is in I; and in (\ref{ex:Minimalism:25}b),
for the string {\it Is Maria happy\/}, both IP and VP lack their
X$^0$ heads. In these structures the finite form of {\it be\/}
acts as an auxiliary verb, and so does not head a surface VP, but
appears in a higher functional head position (in contrast {\it be\/}
as a non-finite form would head VP, as in {\it Maria could [be
    happy]\/}).

Formally, the theory requires that every XP either has a c-structure
head in the standard X$'$ sense, or that it maps to an f-structure
shared with at least one YP which is headed in c-structure. Such a
Y$^0$ is known as the `extended head' of XP (the notion is
originally due to \citealp[221]{zaenen-kaplan1995}, revised to the
formulation given here by \citealp[353]{bresnan00opt}). So in
(\ref{ex:Minimalism:25}a), I is the extended head of VP, and in (\ref{ex:Minimalism:25}b), C is the
extended head of IP and of VP, leading to the illusion that the head
is ``moving''. Different verbal categories may be restricted, though,
to particular c-structures positions: finite auxiliaries in English
may only appear in I or C, not in V; finite non-auxiliaries must
appear in V. Hence finite auxiliaries have the category [$+$V, $-$N,
  P\textgreater0] and finite verbs have the category [$+$V, $-$N, P0].


The discourse in the MP literature over the past 25 years as to
whether head movement exists or whether it is part of `narrow
syntax' (see e.g.~\citealp{roberts11}; \citealp{harigrib19} for
overviews) is quite puzzling from the perspective of declarative
frameworks such as LFG or HPSG, as heads are central to the syntactic
analysis. The issue arose in the development of the MP as
position-occupying head movement does not obey the Extension Condition
of \citet{chomsky1995the-minimalist}, requiring that every operation
of Merge extends the root node of the current tree. Head movement
violates this condition, as it formally involves adjunction to a node
lower than the root node (in contrast to XP adjunction, which does
adjoin at the root). Consequently Chomsky raised the issue of the
status of head movement (e.g.~\citealp[136--137]{chomsky00};
\citealp[38]{chomsky01}) within the MP approach.

As the mapping from c-structure to f-structure in LFG suggests, the
crucial fact about a clausal spine is that head positions share
information, each being a functional co-head (see
e.g.~(\ref{ex:Minimalism:25})). This is directly evidenced in various core cases of
multiple expresssion of the same grammatical information in a single
domain, as first described in LFG by \citet{Nino1997}. The same
properties of clausal information are expressed on more than one head
(see also \citealp{Sells2004}, \citealp{Lodrup2014}). The Finnish
examples in (\ref{135}) and (\ref{137}) \citep[135,~137]{Nino1997} show the
phenomenon:

\ea\label{135} Finnish\\
\gll \"Al-k\"a\"a puhu-ko.\\
\NEG.\IMP-\IMP.2.{\PL} speak-\IMP\\
\glt `Don't (you pl.)\ speak!'
\z
% from nino97 135

\ea\label{137} Finnish
\ea
\gll Ei ol-lut sano-ttu\\  %standard
\NEG.3.{\SG} \PRF-\PST.\PTCP.{\SG} say-\PASS.\PST.\PTCP.{\SG}\\
\glt `It has not been said.'
\ex
\gll Ei ol-ttu sano-ttu\\  %colloquial
{\NEG}.3.{\SG} \PRF-\PASS.\PST.\PTCP.{\SG} say-\PASS.\PST.\PTCP.{\SG}\\
\glt `It has not been said.'
\z\z
% from nino97 137

(\ref{135}) involves a special form of negation restricted to
imperatives, as well as imperative marking on both the auxiliary and
the main verb. In (\ref{137}), singular marking appears on all three
words: the negative, which is a kind of auxiliary; another auxiliary;
and the main verb. These examples also indicate that `passive' is a
feature in f-structure which can be accessed -- see also
\citet{Lodrup2014} for evidence in Norwegian for the same conclusion.
(\ref{137}b) is a colloquial variant of (\ref{137}a), in which the passive
marking on the main verb also appears on the medial auxiliary. The c-
and f-structure of (\ref{137}b) are shown in (\ref{ex:Minimalism:28}). It can easily be
seen that the constraints coming from each of the words in (\ref{137}b)
-- using the glosses as a guide -- are satisfied by this f-structure:

\ea\label{ex:Minimalism:28}
\begin{tabular}[t]{lll}
\begin{forest}
  for descendants={base=bottom}
  [  \rnode{ip}{IP} [ \rnode{i}{I} [ ei ] ] 
      [ \rnode{v1p}{VP}
        [ \rnode{v1}{V} [ ol-ttu ] ]
          [ \rnode{v2p}{VP} [ \rnode{v2}{V} [ sano-ttu ] ] ]
      ]
   ]    
\end{forest}
& \phantom{more space} &
\lower1em\hbox{\evnup[2pt]{\avm[style=fstr]{%\begin{avm}
      \rnode{fs}{[ subj&\rnode{sbj}{[ pers & 3\\
                                        num & sg ]}\\
  {tense} & pres \\
  neg & + \\
  asp & perf \\
  pass & + \\
  pred & \ldots  ]}
}%\end{avm}
}}
\end{tabular}
\nccurve[linestyle=dashed,nodesepA=2pt,nodesepB=0pt,angleA={0},angleB={175},linewidth=.5pt]{->}{ip}{fs}
\nccurve[linestyle=dashed,nodesepA=2pt,nodesepB=0pt,angleA={0},angleB={175},linewidth=.5pt]{->}{v1p}{fs}
\nccurve[linestyle=dashed,nodesepA=2pt,nodesepB=0pt,angleA={0},angleB={175},linewidth=.5pt]{->}{v2p}{fs}
\nccurve[linestyle=dashed,nodesepA=2pt,nodesepB=0pt,angleA={-10},angleB={175},linewidth=.5pt]{->}{i}{fs}
\nccurve[linestyle=dashed,nodesepA=2pt,nodesepB=0pt,angleA={15},angleB={175},linewidth=.5pt]{->}{v1}{fs}
\nccurve[linestyle=dashed,nodesepA=2pt,nodesepB=0pt,angleA={0},angleB={175},linewidth=.5pt]{->}{v2}{fs}
\z

The co-head approach of LFG allows for different sources of the same
constraint (e.g.~that the value of \NUM\ is \SG) which will be
true of just a single object (e.g.~the subject). Hence feature
exponence may be distributed or apparently multiplied. In the MP
approach, each feature necessarily originates in only one position in
the structure, and then must be copied or spread onto other positions,
for data such as that above. In MP analyses, `imperative' corresponds
to a high position in the clause, so the \textsc{imp} feature in
(\ref{137}) must spread downwards. However in (\ref{137}b), the
\textsc{pass} feature would originate on the lowest verb, the only one
marked in (\ref{137}a), and so would have to spread upwards.

The distribution of morphological exponence is probably not related to
direction of spreading, but rather concerns morphological constraints
on each type of word as to what features it must express, might
express, or cannot express. This can be seen clearly in the examples
in (\ref{liv-ex}) from Livonian \citep[131]{Nino1997}, which obey the
generalizations in (\ref{liv}):

\ea\label{liv}\ea
verbs are marked for number
\ex
participles are not marked for person
\z\z

\ea\label{liv-ex} Livonian
\ea
\gll \"a-b u-m and-\^en-{$\emptyset$}\\
\NEG-1 be-1.{\SG} give-\PST.\PTCP-{\SG}\\
\glt `I have not given.'
\ex
\gll \"a-b \`u-om and-\^en-d\\
\NEG-1 be-1.{\PL} give-\PST.\PTCP-{\PL}\\
\glt `We have not given.'
\ex
\gll \"a-d \`u-od and-\^en-{$\emptyset$}\\
\NEG-2 be-2.{\SG} give-\PST.\PTCP-{\SG}\\
\glt `You have not given.'
\ex
\gll \"a-d \`u-ot and-\^en-d\\
\NEG-2 be-2.{\PL} give-\PST.\PTCP-{\PL}\\
\glt `You have not given.'
\z\z

(\ref{ex:Minimalism:31}) shows the c- and f-structure of (\ref{liv-ex}d). Again following
the gloss, it can be seen that the \textsc{pers} value of the subject
is identically constrained by the first two words, while the
\textsc{num} value is constrained by the last two words:

\ea\label{ex:Minimalism:31}% F-structure of \LLast[d]:\\
\begin{tabular}[t]{lll}
\begin{forest}
  for descendants={base=bottom}
  [  \rnode{ip}{IP} [ \rnode{i}{I} [ \"a-d ] ] 
      [ \rnode{v1p}{VP}
        [ \rnode{v1}{V} [ \`u-ot ] ]
          [ \rnode{v2p}{VP} [ \rnode{v2}{V} [ and-\^en-d ] ] ]
      ]
   ]    
\end{forest}
& \phantom{more space} &
\lower1em\hbox{\evnup[2pt]{\avm[style=fstr]{%\begin{avm}
      \rnode{fs}{[ subj&\rnode{sbj}{[ pers & 2\\
                                         num & pl ]}\\
  {tense} & pres\\
  neg & + \\
  asp & perf \\
  pred&\ldots\\
  \ldots & \ldots  ]}
}%\end{avm}
}}
\end{tabular}
\nccurve[linestyle=dashed,nodesepA=2pt,nodesepB=0pt,angleA={0},angleB={175},linewidth=.5pt]{->}{ip}{fs}
\nccurve[linestyle=dashed,nodesepA=2pt,nodesepB=0pt,angleA={0},angleB={175},linewidth=.5pt]{->}{v1p}{fs}
\nccurve[linestyle=dashed,nodesepA=2pt,nodesepB=0pt,angleA={0},angleB={175},linewidth=.5pt]{->}{v2p}{fs}
\nccurve[linestyle=dashed,nodesepA=2pt,nodesepB=0pt,angleA={-10},angleB={175},linewidth=.5pt]{->}{i}{fs}
\nccurve[linestyle=dashed,nodesepA=2pt,nodesepB=0pt,angleA={15},angleB={175},linewidth=.5pt]{->}{v1}{fs}
\nccurve[linestyle=dashed,nodesepA=2pt,nodesepB=0pt,angleA={0},angleB={175},linewidth=.5pt]{->}{v2}{fs}
\z

These patterns of multiple expresssion extend beyond simple clauses,
into various kinds of complex predicate (see e.g.~\citealp{Sells2004},
\citealp{Lodrup2014}), which might require a more nuanced syntactic
analysis than simple embedding of f-structure nuclei -- as argued for
on the basis of entirely different data by
\citet{AndrewsManning1999}. The multiple expression data could
profitably be analyzed in a realizational framework (as suggested in
\citealp{Sells2004}) -- every informational element within a certain
domain must have at least one rule of realization applying to it (this
idea is formalized explicitly in \citealp{crysbona16}), but in certain
circumstances one piece of information can be referred to more than
once, as the generalizations in (\ref{liv}) suggest. Crucially, again,
it is not that one piece of morphological exponence on a c-structure
head is copied to another head, but rather that different (co-)heads
are acting as exponents of the same grammatical information.

\subsection{The MP approach to phrase structure: Merge}
\label{sec:minimalism:merge}

The legacy of the Government-Binding model of syntax into the MP is a
procedural approach to structure and structure-building. Binary
structures are built up by Merge of two elements, often known as
External Merge or `first merge'. The GB idea of movement is
reinterpreted in the MP as Internal Merge -- one element from within a
given structure is (re-)merged near the top of the structure. As noted
in \sectref{sec:minimalism:framework}, the argument structure of the
predicate is represented in a vP-VP structure, above which there are
further projections such as TP and CP. By the time the structure has
built up at least to TP, this structure effectively codes clausal
information.

Strictly speaking, the syntactic derivation is abstract, with
syntactic relationships referring to the structural notion of
c-command but not linear order, which comes in the mapping from syntax
to Phonetic Form (PF). The relevant terminal nodes of the structure
are spelled out as words via the principles of Distributed Morphology
(for an overview of this framework, see \citealp{harleynoyer03} or
\citealp{embicknoyer07}). Consider the derivation in (\ref{awarded-1}) of the
example {\it several prizes were awarded\/}, which will also feature
later on:

\ea\label{awarded-1}
\resizebox{.9\textwidth}{!}{
\begin{forest}
  for descendants={base=top}
 % [ \rnode{cp}{CP}
    % [ XP ]
  %  [ \rnode{cb}{C$'$} [ \rnode{c}{C} ] 
      [ \rnode{ip}{TP}
        [ \rnode{s}{DP}\\ {\cnom} [ {several prizes}, roof ] ]
        [ \rnode{ib}{T$'$}
          [ \rnode{i}{T} [ T\\ \fstackonefill ]
            [ \rnode{tpass} Pass [ be ] ] ]
          [ \rnode{vp}{PassP}
            [ \rnode{pass}{Pass} [ \tr{be} ] ]
            [ \rnode{vb}{vP}
              [ \rnode{v}{v} [ v\\ {\inflpass} ]
                              [ \rnode{vV}{V} [ award ] ] ]
              [ \rnode{VP}{VP}
                [ \rnode{V}{V} [ \tr{award} ]]
                [ \rnode{ob}{DP} [ \tr{several prizes}, roof ] ]
              ]
            ] 
          ]  
         ]
      ]
%      ]
% ]
  \end{forest}
\nccurve[ncurv=1.5,nodesepA=2pt,nodesepB=0pt,angleA={210},angleB={210},linewidth=.5pt]{->}{pass}{tpass}
\nccurve[ncurv=1.3,nodesepA=2pt,nodesepB=0pt,angleA={210},angleB={210},linewidth=.5pt]{->}{V}{vV}
\nccurve[ncurv=1.3,nodesepA=2pt,nodesepB=2pt,angleA={210},angleB={210},linewidth=.5pt]{->}{ob}{s}
}
\z

The internal argument of a transitive verb is merged with V (a kind of
root) within VP, at the lower right of the structure. The structure
builds from the bottom up via successive applications of Merge. The VP is
immediately the complement of a ``little v'' vP, which introduces the
external argument in a canonical transitive. The particular example
here is a passive, with the external argument suppressed. The two
components of the verb, V and v, are combined by head movement of V to
v, as the structure shows. The notation \tr{award} indicates the
original position of V before movement.

The passive is indicated here by PassP, following the analysis of
English auxiliaries in \citet{adger03}. The Pass head {\it be\/}
merges with vP as its complement. Next, above that, a T$'$ is created
with a formative for past tense in its T head. The auxiliary verb {\it
  be\/} also undergoes head movement, to combine with T. Finally,
following X$'$-theory, T$'$ has a specifier which hosts the surface
subject. In the case of a passive example, a DP is raised from a
VP-internal position to fill the subject position.

The arrows in (\ref{awarded-1}) indicate movement. Standardly in the
Minimalist approach, movement leaves behind a ``copy'' of the moved
constituent (\citealp{chomsky1995the-minimalist}), which the notation
such as \tr{award} etc.\ is intended to convey. Principles of
realization at PF determine which copy is overt (pronounced) --
usually the highest copy, as in the earlier versions of
transformational grammar where moved constituents leave behind a
trace, which is by definition unpronounced. The formalization of MP
syntactic derivations due to \citet{collstab16} captures the `copy'
idea by treating each operation of movement as creating a
multidominance structure from a single terminal element; that
formalization is extended to head movement by
\citet{bleaman21}. However, for presentational purposes, I show the
more familiar movement-with-copies structures here.

There are complex heads in T and v in (\ref{awarded-1}), both formed via head
movement. They also have a representation of the various features
which are present or which are valued during the course of the
derivation (see \sectref{sec:minimalism:features-mp}). These complex heads
will realize their lexical and featural information as the words {\it
  were\/} and {\it awarded\/}.

Hence, the phrase structure derivation in the Minimalist approach
represents all the clausal information, somewhat like LFG's
f-structure, which is then spelled out as the overt form, which
corresponds to some extent to LFG's c-structure.

\subsection{Phrase structures are not isomorphic to clausal
  information}

One difference between LFG and MP concerns how far the phrase
structure is a direct representation of the clausal information. As
just noted, the representation of clausal information in a Minimalist
approach is encoded within the phrase structure (in its configuration
and its derivation), while in LFG the relation between f-structure and
c-structure is fundamentally more flexible.  The Mandarin Verb Copy
Construction (\citealp{lithompson}, \citealp{huang82}) serves as a
good example of how clausal information at f-structure can exist
independently of any particular c-structure property. Postverbal
arguments and adjuncts appear to be in competition within a single VP:
in order to express an argument and an adjunct, the main verb must be
duplicated to form a second VP, as in (\ref{piano}b).

\ea\label{piano}Mandarin
\ea[*]
{\gll Zhangsan tan gangqin de hen hao\\
Zhangsan play piano \textsc{lnk} very well\\
\glt `Zhangsan plays piano very well'}
\ex
\gll Zhangsan tan gangqin tan de hen hao\\
Zhangsan play piano play \textsc{lnk} very well\\
\glt `Zhangsan plays piano very well'
\z\z

\citet{huang82} proposed a phrase structure filter which essentially
disallows arguments and adjuncts in the same VP. \citet{fangsells07}
note that both arguments of a ditransitive verb appear in the first VP
(underlined in (\ref{ex:Minimalism:34}a)), and that an object may be displaced from
within the first VP, otherwise preserving the phrase structure:

\ea\label{ex:Minimalism:34} Mandarin
\ea
\gll wo   song \tightuline{ta} \tightuline{zhe\ jian\ liwu} song  de hen
hao \\ 
I give him {this \textsc{cl}~~~gift} give \textsc{lnk} very well \\
\glt `I gave him this gift and it turned out to be a very good idea.'
\ex
\gll \tightuline{zhe\ jian\ liwu} wo song   ta   {\gapline} song de  hen hao\\
{this\ \textsc{cl}~~~gift}   I give  him {\gapline} give   \textsc{lnk} 
  very well\\
\glt `This gift, I gave (it) him and (it turned out to be) very good.'
\z\z

 
However, if an object is displaced from a monotransitive VP, verb
``copying'' is no longer an option (also see \citealp[53]{huang82}):

\ea\label{piano-top} Mandarin
\ea
[*]
{\gll \tightuline{{gangqin}}  Zhangsan tan {\gapline}   tan de  hen
  hao\\ 
piano Zhangsan play    {\gapline}    play    \textsc{lnk}  very    well\\}
\ex
\gll \tightuline{{gangqin}}  Zhangsan tan {\gapline} de  hen hao\\
piano Zhangsan play    {\gapline}      \textsc{lnk}  very   well\\
\glt `The piano, Zhangsan played (it) very well.'
\z\z

If we take (\ref{ex:Minimalism:36}) to be the basic f-structure of what should be
expressed in one structure of (\ref{piano}), given the constraint that
arguments and adjuncts cannot be in the same c-structure VP, it
follows that (\ref{piano}b) is the only possible expression.

\ea\label{ex:Minimalism:36}
%\evnup[2pt]{
{\avm[style=fstr]{%\begin{avm}
  [ pred & `play\arglist{(\UP \SUBJ)(\UP \OBJ)}'\\ 
   subj  &  [pred & `Zhangsan']\\
   obj & [ {pred} & `piano' ]\\
   adjunct & [pred & `very well']
]    
}%\end{avm}
}
\z

On the other hand, if `piano' appears as a structural \textsc{topic},
in clause-initial position, only a single VP is required to express
the in-situ material, which consists of the \textsc{pred} and its
\textsc{adjunct} in (\ref{ex:Minimalism:37}), as in example (\ref{piano-top}b). The
identification of \textsc{topic} and \textsc{obj} takes place only at
f-structure (\citealp{kaplzaen89}).

\ea\label{ex:Minimalism:37}
%\evnup[2pt]{
\avm[style=fstr]{%\begin{avm}
  [ {topic} & \rnode{a}{[pred & `piano' ]}\\
    pred & `play\arglist{(\UP \SUBJ)(\UP \OBJ)}'\\ 
   subj  &  [pred & `Zhangsan']\\
{obj} & \rnode{b}{[ \phantom{pred} & \phantom{`piano'} ]}\smallskip\\
  adjunct & [pred & `very well']
]  
}%\end{avm}
%}
\CURVE[1.6]{-2pt}{0}{a}{-2pt}{0}{b}
\z

Examples such as (\ref{piano-top}b) show that the competition between
arguments and adjuncts for the same VP is a phrase-structure
phenomenon, and is not relevant for the level of clausal grammatical
information: a verb in Mandarin can perfectly well have a full array
of arguments and any adjuncts, but only some of those can be expressed
within a single VP. Following a careful survey of the research on this
topic, \citet{bartos19} proposes an MP analysis which has to appeal to
haplology of V to derive (\ref{piano-top}b) from (\ref{piano-top}a)
(already suggested in \citealp[99]{huang82}), but this is merely
symptomatic of an underlying misanalysis, for the core relations
between a predicate and its arguments and adjuncts are not
isomorphically represented in phrase structure.

\section{Features and agreement}
\label{sec:minimalism:features}

\subsection{Feature theory and LFG}

LFG is built on the foundation that featural specifications in
morpho-syntax are of the form [attribute value], and that
well-formedness requires every attribute in a given representation to
have an appropriate value. The attribute-value format is used in LFG
to represent functional structure, which represents the relational and
featural content of a clause, but not constituent
structure. F-structure is deliberately designed to not look like a
phrase structure, to signify that it represents a different kind of
syntactic information, and also that the parts within it are
unordered. (The concept of f-precedence in LFG (see Glossary)
crucially makes reference to the c-structure realization of
f-structure elements.)

\citet{adger10} considers features and the format of features in
MP. He also concludes that features should be represented as
attribute-value pairs, but rejects the idea that feature names can
have structured values, because that re-creates the hierarchical
structure within the phrase structure (e.g.~a structured value for
\textsc{subj} corresponds to a DP in the phrase structure with
internal constituency). Of course, there is no claim in LFG that
every attribute in f-structure is the name of a feature -- `f' stands
for `functional', not `feature'. Hence the closest comparison will be
the atomically-valued attributes in f-structure, which will correspond
most closely to features in MP, and which also accord with the general
notion of morpho-syntactic features. More precisely, these will be the
`syntactic' features identified by \citet{sadlspen01} (see also
\citealp{spencer13}), which are the target of morpho-syntactic
exponence (as in the discussion of Finnish and Livonian above in
\sectref{sec:minimalism:heads}). In this subsection I compare the LFG and
MP approaches to such features. An extended discussion of features in
the MP in comparison to other frameworks can be found in
\citet[409--420]{asudtoiv06}.

Featural information associated with each word introduces constraints
on the well-formed\-ness of the whole structure, within a `monotonic'
system: information cannot be selectively ignored, nor can it be
changed. Hence declarative frameworks such as LFG necessarily have a
property which has come to feature in MP discourse -- the `No
Tampering Condition' (\citealp{chomsky07}), which does not allow
information on an item to be changed as it is merged in as part of the
derivation (see also \sectref{sec:minimalism:conclusion}).

For instance, (\ref{ex:Minimalism:38}) is ungrammatical as not all the constraints
coming from the lexical items can be satisfied simultaneously, and no
part of the information can be ignored:

\ea[*]
{You am happy.}\label{ex:Minimalism:38}
\z

In this example, \textit{you} will specify the value of
\textsc{person} of the subject as 2, but \textit{am} will specify that
same value as 1. There is no way to satisfy the requirements of these
first two words in a single structure.

LFG introduces featural information either via lexical items or by the
rules which license phrase structure. Every well-formed feature
specification in f-struc\-ture is of the form \mbox{\textsc[attribute value]}, by
definition (see e.g.~\citealp[181--182]{kaplanbresnan82}). If any
lexical item specifies a feature but without a value, that is an
`unvalued feature'; some other element in the structure must introduce
the value for that feature, or else the overall structure will be
ill-formed. Unvalued features play a significant role in MP analyses
(because their function goes beyond that of simply representing
information; see \sectref{sec:minimalism:features-mp}); they also find
their place in declarative analyses, as described below, although
valued features tend to be the norm.

The basic way for information to be specified is as a defining
equation -- for instance the information carried by the appropriate
lexical entries to give the f-structures in (\ref{heureuse}). There is
another kind of informational contribution, the constraining equation
of LFG. \citet[207--209]{kaplanbresnan82} motivate constraining
equations with familiar facts such as those in (\ref{handing}), with their
proposal for analysis in (\ref{pres-part}):

\ea\label{handing}
A girl is handing (*is hands, *is handed) the baby a toy.
\z
%
\ea\label{pres-part}
\textit{is}:\; (\UP\XCOMP\textsc{participle}) {=$_c$} \textsc{present}
\z

% \textit{handing} ({\UP} PARTICIPLE) = present

The VP complement of \textit{is} has the grammatical function
\textsc{xcomp} in the LFG analysis, and within that, the grammatical
form \textit{handing} would provide the value `\textsc{present}' for the
attribute \textsc{participle}. That fulfils the requirement in
(\ref{pres-part}). The important move to a constraining equation over a
defining one concerns the ungrammatical variants in
(\ref{handing}). For instance, as a finite form, \textit{hands} is not
specified at all for the attribute \textsc{participle}, and so does
not provide the information that (\ref{pres-part}) needs. However, if that
information in (\ref{pres-part}) were specified as defining information, it
would be unified in with the information from \textit{hands}, and -- at
least on that count -- the sequence \textit{*is hands} would not be
ungrammatical, as nothing would be inconsistent. Kaplan and Bresnan
note that in a unification-based system, constraining equations have
the important consequence that negative-value specifications for
otherwise unnecessary features can be avoided. (For more on features
see \citealp{kaplan18}.)

Accounts involving a constraining-type analysis are common. This is
the situation that is modelled in MP analyses with an
`uninterpretable' feature -- two elements between which there is
some dependency have the same feature specifications, but only one
such specification is the `real one'. An MP analysis of the English
auxiliary system by \citet{aelbharw15} involves the same idea as in
(\ref{pres-part}), proposing that uninterpretable but valued features
match between the governed verb and the higher auxiliary which governs
it.

The use of a constraining equation can be further illustrated in the
case of a strict Negative Polarity Item (NPI) -- an item that must appear
in the context of negation, but is not the expression of negation
itself. Such an NPI constrains its syntactic environment to have the
\textsc{neg} feature with the value + (see e.g.~\citealp{sellsneg});
this information must be present, but supplied by some other element,
namely overt negation. From the MP perspective, \citet{zeijlstra15}
discusses an analysis of NPIs in which they ``carry some
uninterpretable negative feature [u\textsc{neg}] that must be checked
against a higher, semantically negative element that carries an
interpretable formal negative feature [i\textsc{neg}].''  Again, in
the relevant sense, this is a valued feature which is contentful on
one element, and is on another for the purpose only of establishing an
abstract syntactic relation.

Returning to the case of an attribute introduced without a value, this
is an existential constraint on f-structure
(see \citealp[210ff.]{kaplanbresnan82} and
\citealp[112--114]{dalrymple01}). For instance, the complementizer {\it
  that\/} in English introduces a clause which is tensed, but it
places no restriction on the value of \textsc{tense}. Hence part of
the functional information associated with {\it that\/} will be the
existential constraint {(\UP\TENSE)}. This constrains the
f-structure of the clause to have the attribute \TENSE, and any
well-formed f-structure must have a value for that attribute. The
value is not supplied by {\it that\/}, so that information must come
from elsewhere in the clause introduced by {\it that\/}.

In summary, the notions of `unvalued' and `uninterpretable' features
which are important in MP analyses -- see immediately below -- have
formalized equivalents in LFG, and in LFG, neither can by
itself lead to a well-formed f-structure: an f-structure cannot
contain an attribute without a value, and the contribution of a
constraining equation must be matched by the contribution of a
(valued) defining equation. In keeping with the character of the
differences between the LFG and MP approaches, a clausal f-structure
in LFG is never partial nor ill-formed, unlike stages in an MP
derivation. Rather, each element in the c-structure in a given example
contributes to a set of constraints which the overall f-structure must
satisfy. If those constraints conflict, there is no f-structure which
satisfies them, and the example is thereby ungrammatical.

\subsection{Features in the MP}
\label{sec:minimalism:features-mp}
\largerpage[2]
Features are put to at least three uses in MP analyses (see
\citealp[200--212]{adger10}). The first is to represent information,
the second is to establish a relationship, and the third is to make
something happen. The representational aspect usually involves valued
features, and might involve unvalued ones. The second use involves the
notion of interpretable and uninterpretable features, which is the
mechanism for establishing a relationship known as `Agree' between a
Probe and a Goal (\citealp[101]{chomsky00}).

For instance, \citet[189]{adger10} gives the following illustrative
example of a feature that is unvalued, and also uninterpretable. The
idea that some features are uninterpretable was originally introduced
by \citet[277--278]{chomsky1995the-minimalist}. In (\ref{adger}), the
first group of features are features of the subject DP, and the second
group are features on the T head of TP.

\ea\label{adger}\relax
\{D, definite, plural\} \ldots\ \{T, past, \textit{u}plural\}
\z

The DP is definite and plural, and T is past and also marked as
plural, but the prefix notation \textit{u} indicates that the plural
feature, though present on T, is uninterpretable on it. Adger
describes (\ref{adger}) as follows:
\begin{quote}
The idea is that a feature like [plural] only has an interpretation
when specified on a category which can be potentially interpreted as
plural (e.g.~on a noun), otherwise an instance of this feature will be
uninterpretable: interpretability is detectable from a feature’s
syntactic/semantic context. The formal property of features (the
\textit{u} prefix) which enables them to enter into dependency relations
is thus linked to the interpretation of features\ \ldots~ \citep[189]{adger10}
\end{quote}
The [plural] feature is not interpretable on T -- the interpretation
of tense never makes reference to singular/plural -- but the matching
occurence of [plural] on the subject DP establishes the {Agree}
relation between these two groups of features. After it has been
checked by a matching interpretable feature, [plural] is then
eliminated on T.

From the perspective of LFG, the equivalent of
[\textit{u}plural] in (\ref{adger}) would be as in (\ref{ex:Minimalism:42}).

\ea\label{ex:Minimalism:42}
%\textit{is}:
(\UP\SUBJ\NUM) {=$_c$} \PL
\z


\largerpage[2]
Just as with (\ref{adger}), a structure described by (\ref{ex:Minimalism:42}) will
only be well-formed if some other element (e.g.~the subject) specifies
the \PL\ value for the feature, but it represents a different approach
to the role of features. In (\ref{adger}), the feature on T is
understood to convey ``I am plural'', which is uninterpretable; but
the specification in (\ref{ex:Minimalism:42}) conveys ``my subject's number is
plural'', which is actually straightforwardly interpretable.

The third use of features in the MP is to trigger an operation. Such
features do not seem to overlap with the features considered above,
and exist solely to make something happen. The canonical example is
the `EPP-feature' derived from GB, but used in different ways to
force either XP movement or X$^0$ movement (head movement) in many
MP analyses. This feature has been more recently cast as an `Edge
Feature' (e.g.~\citealp{chomsky2005three}). It is not clear formally
what kind of feature this is -- it must be satisfied, as an
instruction for some structure to be built, and once satisfied, there
are two options: either it becomes inactive, or it stays active,
allowing for multiple specifiers (\citealp[11]{chomsky07}).


\largerpage[2]
I now show in more detail how (un)valued and (un)interpretable
features participate in an MP derivation. What is shown here is based
on the presentation in \citet[284ff.]{radford09}, though using a
slightly different representational format which is more internally
consistent and which will also be more transparent in the context of
the LFG approach to features described above. There is in fact a close
relation between valued and interpretable features, as will be evident
in the structures below. However, the two notions are formally
distinct and can play different roles in an overall syntactic analysis
(see e.g.~\citealp{aelbharw15}). The structure in (\ref{awarded-4}) underlies
the fragment {\it were awarded several prizes\/}, which is our
illustrative example. The DP {\it several prizes\/} has interpretable
features of person and number (it is 3rd person plural), and in the
syntax it will have a value for case; but the case feature is
initially unvalued, as the particular value of case will depend on the
syntactic context of the DP. The v which ultimately hosts {\it
award\/} has an Infl feature (sometimes referred to as {\sc vform} in
LFG), which will also be valued according to the syntactic context of
the verb. Finally, the head T is specified as past tense, and it also
hosts agreement features for person and number, which are unvalued at
this initial stage. In the structure, the features shown in bold are
those which are interpretable, and they also are the ones which are
valued.

\ea\label{awarded-4}
{\begin{forest}
  for descendants={base=top}
 % [ \rnode{cp}{CP}
    % [ XP ]
  %  [ \rnode{cb}{C$'$} [ \rnode{c}{C} ] 
        [ \rnode{ib}{T$'$}
          [ \rnode{i}{T} [ T\\ \fstackoneint ]
            [ \rnode{tpass} Pass [ be\\{} ] ] ]
          [ \rnode{vp}{PassP}%\rdash
            [ \rnode{pass}{Pass} [ \tr{be} ] ]
            [ \rnode{vb}{vP}
              [ \rnode{v}{v}  [ v\\{\uinfl} ]
                              [ \rnode{vV}{V} [ award ] ] ]
              [ \rnode{VP}{VP}
                [ \rnode{V}{V} [ \tr{award} ]]
                [ \rnode{ob}{DP} [ {several prizes\\
                                    \fstacktwoint}, roof ] ]
              ]
            ] 
          ]  
         ]
%      ]
% ]
  \end{forest}
\nccurve[ncurv=1.5,nodesepA=2pt,nodesepB=0pt,angleA={210},angleB={210},linewidth=.5pt]{->}{pass}{tpass}
\nccurve[ncurv=1.3,nodesepA=2pt,nodesepB=0pt,angleA={210},angleB={210},linewidth=.5pt]{->}{V}{vV}
}
\z

Formally, Agree establishes a relation between two nodes, a Probe and
a Goal, one of which has an interpretable feature, with the other
having a (matching) uninterpretable feature. In the simple example
here, the uninterpretable features are the unvalued ones, and so they
will become valued once Agree takes place. For instance, the Pass head
{\it be\/} in (\ref{awarded-4}) enters into an Agree relation with v and
values the Infl feature as Pass(ive). The T head enters into an
Agreement relation with the DP {\it several prizes\/}, valuing the
Case on DP as Nom, and at the same time taking the values of person
and number from that DP. Hence, after these Agree relations are
established, all features are valued, as in (\ref{awarded-5}), and these
feature specifications will be relevant for morphological realization
(e.g.\ the surface form of {\it be\/} will be {\it
  were\/}). Nevertheless, the non-bold (formerly unvalued) features
are classed as uninterpretable, and must delete by the end of the
syntactic derivation, Logical Form. Finally, (\ref{awarded-5}) also shows one
more feature on T, [EPP], which is discussed immediately below.


\ea\label{awarded-5}
{\begin{forest}
  for descendants={base=top}
 % [ \rnode{cp}{CP}
    % [ XP ]
  %  [ \rnode{cb}{C$'$} [ \rnode{c}{C} ] 
        [ \rnode{ib}{T$'$}
          [ \rnode{i}{T} [ T\\ \fstackthrchk \\ \fepp]
            [ \rnode{tpass} Pass [ be\\{} ] ] ]
          [ \rnode{vp}{PassP}%\rdash
            [ \rnode{pass}{Pass} [ \tr{be} ] ]
            [ \rnode{vb}{vP}
              [ \rnode{v}{v}  [ v\\{[Infl:Pass]} ]
                              [ \rnode{vV}{V} [ award ] ] ]
              [ \rnode{VP}{VP}
                [ \rnode{V}{V} [ \tr{award} ]]
                [ \rnode{ob}{DP} [ {several prizes\\
                                    \fstackfouchk}, roof ] ]
              ]
            ] 
          ]  
         ]
%      ]
% ]
  \end{forest}
\nccurve[ncurv=1.5,nodesepA=2pt,nodesepB=0pt,angleA={210},angleB={210},linewidth=.5pt]{->}{pass}{tpass}
\nccurve[ncurv=1.3,nodesepA=2pt,nodesepB=0pt,angleA={210},angleB={210},linewidth=.5pt]{->}{V}{vV}
}
\z

T in (\ref{awarded-5}) has the [EPP] feature mentioned above, which has the
effect that at the next step of the derivation a specifier must be
created. This feature, then, does not represent a `featural' property
of the clause (unlike, say, `past tense'), but represents a structural
property.  One option for satisfying this feature is to merge in an
expletive placeholder, {\it there\/}:

\ea\label{awarded-6}
{\begin{forest}
  for descendants={base=top}
 % [ \rnode{cp}{CP}
    % [ XP ]
  %  [ \rnode{cb}{C$'$} [ \rnode{c}{C} ] 
      [ \rnode{ip}{TP}
        [ \rnode{s}{DP} [ there, roof ] ] % [ we, roof ] ]
        [ \rnode{ib}{T$'$}
          [ \rnode{i}{T} [ T\\ \fstackthrchk\\ \sepp ]
            [ \rnode{tpass} Pass [ be\\{} ] ] ]
          [ \rnode{vp}{PassP}%\rdash
            [ \rnode{pass}{Pass} [ \tr{be} ] ]
            [ \rnode{vb}{vP}
              [ \rnode{v}{v}  [ v\\{[Infl:Pass]} ]
                              [ \rnode{vV}{V} [ award ] ] ]
              [ \rnode{VP}{VP}
                [ \rnode{V}{V} [ \tr{award} ]]
                [ \rnode{ob}{DP} [ {several prizes\\
                                    \fstackfouchk}, roof ] ]
              ]
            ] 
          ]  
         ]
      ]
%      ]
% ]
  \end{forest}
\nccurve[ncurv=1.5,nodesepA=2pt,nodesepB=0pt,angleA={210},angleB={210},linewidth=.5pt]{->}{pass}{tpass}
\nccurve[ncurv=1.3,nodesepA=2pt,nodesepB=0pt,angleA={210},angleB={210},linewidth=.5pt]{->}{V}{vV}
}
\z

Once the specifier is merged, the EPP feature is thereby satisfied,
indicated in (\ref{awarded-6}) by the strikethrough. Another way of satisfying
this feature from the stage in (\ref{awarded-5}) is to raise the object DP to
the subject position, as a canonical passive:

\ea\label{awarded-7}
\hspace*{-5mm}\scalebox{.9}{\begin{forest}
  for descendants={base=top}
 % [ \rnode{cp}{CP}
    % [ XP ]
  %  [ \rnode{cb}{C$'$} [ \rnode{c}{C} ] 
      [ \rnode{ip}{TP}
        [ \rnode{s}{DP} [ {several prizes\\
                                    \fstackfouchk}, roof ] ]
        [ \rnode{ib}{T$'$}
          [ \rnode{i}{T} [ T\\ \fstackthrchk\\ \sepp ]
            [ \rnode{tpass} Pass [ be\\{} ] ] ]
          [ \rnode{vp}{PassP}%\rdash
            [ \rnode{pass}{Pass} [ \tr{be} ] ]
            [ \rnode{vb}{vP}
              [ \rnode{v}{v}  [ v\\{[Infl:Pass]} ]
                              [ \rnode{vV}{V} [ award ] ] ]
              [ \rnode{VP}{VP}
                [ \rnode{V}{V} [ \tr{award} ]]
                [ \rnode{ob}{DP} [ \tr{several prizes}, roof ] ]
              ]
            ] 
          ]  
         ]
      ]
%      ]
% ]
  \end{forest}
\nccurve[ncurv=1.5,nodesepA=2pt,nodesepB=0pt,angleA={210},angleB={210},linewidth=.5pt]{->}{pass}{tpass}
\nccurve[ncurv=1.3,nodesepA=2pt,nodesepB=0pt,angleA={210},angleB={210},linewidth=.5pt]{->}{V}{vV}
\nccurve[ncurv=1.3,nodesepA=2pt,nodesepB=2pt,angleA={210},angleB={210},linewidth=.5pt]{->}{ob}{s}
}
\z

\subsection{Agreement and the direction of Agree}
\label{sec:minimalism:agreement}

The second use of features in the MP noted above is that they
participate in the process of Agree (\citealp{chomsky00}), which is a
prerequisite to establishing a relation in the syntax. The exemplar
syntactic relation is that of agreement -- say between a subject and a
finite verb. As with canonical agreement (e.g.~\citealp{corbett06}),
there is assumed to be a controller of agreement and a target for
agreement, a directional or asymmetric relationship, formally
instantiated as a Probe and a Goal in the MP. There is considerable
debate in the MP literature as to the `directionality' of Agree -- is
it upwards, or downwards? -- as well as to whether feature valuation
passes from the higher element to the lower one, or vice versa. For
instance, \citet{poliprem19} make a linguistic argument about the
direction of agreement (specifically, $\phi$-feature agreement); they
argue that agreement must be directional, looking downwards, but
valuing features upwards. In contrast, \citet{bjorzeij19} argue for a
more complex system in which a checking relation is first established,
but only upwards, and then after that valuation can take place, in
either direction. Some examples which bear on these issues are given
below. These proposals are each `substantive' proposals, motivated by
empirical observations, as there is nothing formally about the MP
system which requires a given directionality for Agree.

As noted above in \sectref{sec:minimalism:info-based}, there is no sense in
LFG in which agreement can be directional, as ``agreement'' is the
informal notion we apply to a situation where more than one element
provides featural information about some (other) element. The Archi
examples below show that the distinction between Controller and
Target, or between Probe and Goal, cannot be sustained anyway.

\citet{poliprem19} present examples such as (\ref{tsez}) from Tsez, to
support their claim that a Probe looks downwards to find a Goal --
that the Probe c-commands the Goal -- and the relevant feature values
from the Goal are then valued upwards to the (previously incomplete)
Probe. The key property of this example is that it involves
long-distance agreement, in which the matrix verb does not agree with
any local argument but rather agrees with the absolutive argument
(object) of the embedded clause. Tsez has an ergative-absolutive
case-marking system, and the verb agrees with an absolutive
argument. The embedded absolutive in (\ref{tsez}) is `bread', class
\textsc{iii}, and both the local predicate `eat' and the higher
predicate `know' agree with it in class, shown in boldface in the
gloss:

\ea\label{tsez}
Tsez\\%\vspace{-0.8em}
\gll
eni-r [u\v{z}-\={a} magalu b-\={a}c’-ru-{{\textbeltl}}i] b-iy-xo\\
mother-\textsc{dat} [boy-\textsc{erg}
bread.{\bf\textsc{iii}}\textsc{(abs)}
{\bf\textsc{iii}}-eat-\textsc{pst.ptcp-nmlz}] 
{\bf\textsc{iii}}-know-\textsc{pres}\\%\vspace{-0.9em}
\glt`The mother knows that as for the bread, the boy ate it.'
\z

The particular argument that Polinsky and Preminger make is based on
the observation that the opposite configuration appears to be
unattested -- we never find a structure in which a verb in a lower
clause agrees with an argument in a higher clause. To rule out this
logical possibility, they argue that syntactic theory should only
allow downwards Agree/upwards valuation. The detail of their argument
is not crucial here -- what is relevant are the relative structural
relations between the two elements in the agreement relationship. In
(\ref{tsez}), as the absolutive controls agreement on the higher
predicate `know', I will categorize this example as one in which the
target must c-command the controller (hence, valuation is upwards).

The rest of the examples in this section are taken from Archi, another
language with an ergative-absolutive system. Archi has a wide range of
potential targets for agreement, but the controller is always the
absolutive. (\ref{behind}) from \citet[67]{boch16} shows various agreement
targets (boldfaced in the gloss, each corresponding to the exponent
\textit{d-}):

\vbox{
\ea\label{behind}Archi\\%\vspace{-0.8em}
\gll
[d-ez {\textchi}ir] d-e{$\scriptstyle\langle$}r{$\scriptstyle\rangle$}q\textsuperscript{\textrevglotstop}a-r-\v{s}i d-i\\
[{\bf\textsc{ii.sg}}-\textsc{1sg.dat} behind]
{\bf\textsc{ii.sg}}-\textsc{{$\scriptstyle\langle$}ipfv{$\scriptstyle\rangle$}}go-\textsc{ipfv-cvb} 
{\bf\textsc{ii.sg}}-be.\textsc{pres}\\%\vspace{-0.9em}
\glt`She follows me.' (male speaking)
\z
}


\largerpage[2]
Both the main verb and the auxiliary `be' show agreement with the
absolutive for gender and number. The gender system in Archi consists
of four noun classes, and in this example, the gender agreement is for
class \textsc{ii}. The controller of agreement is not overt -- it is
the implicit subject of the intransitive predicate, formally in
absolutive case. In addition, the first singular pronoun \textit{d-ez}
which is the object within the directional PP headed by `behind'
agrees with the absolutive of its clause, even though the pronoun is
not a direct co-argument of the absolutive in this example. The
pronoun is itself first person singular, but it also has an `external
agreement' slot for the clausal absolutive. Now that pronoun, inside
the PP, cannot c-command anything outside that PP, and yet the
intuition here is that it is the target of agreement: so for this
example it must be the case that the controller (a null subject
absolutive) c-commands the target. Valuation, if directional, should
be downwards -- exactly reversed from the Tsez example (\ref{tsez}).

The LFG analysis of Archi agreement in \citet{sadler16} codes each
agreeing element for the relevant features of the notional agreement
controller -- the argument in absolutive case. As the GF of that
argument could be \textsc{subj} or \textsc{obj} depending on the
transitivity of the predicate, Sadler uses the designator
\textsc{piv}, proposed by \citet{falk06}. \citet[161]{sadler16} also
uses the template approach (\citealp{dalrymple2004linguistic}) to
schematize over different agreement combinations. For the form in
(\ref{dez}), @\textsc{ii.sg} associates the gender and number agreement
values with the word, as in the second commentary enclosed in [ ]:

\ea\label{dez}\begin{tabular}[t]{ll}
\textit{d-ez}, Pronoun & \\
(\UP\textsc{pers}) = 1 & \\
(\UP\textsc{num}) = sg & [it is a first singular pronoun] \\
@\textsc{ii.sg}((\GF\UP)\,\textsc{piv}) &
[its external agreement features are class \textsc{ii} singular]\\
  \ldots
  \end{tabular}
\z

For (\ref{behind}), (\GF\UP)\,\textsc{piv} instantiates as
(\OBL\OBJ\UP)\,\SUBJ. The f-structure of the example is
shown in (\ref{ex:Minimalism:50}), where the external agreement path for the first
person pronoun -- the inner \mbox{[\textsc{pred~`pro'}]} -- follows this
instantiation and specifies values for gender and number, shown in
boldface:

\vbox{
  \ea\label{ex:Minimalism:50}
  F-structure of (\ref{behind}); agreement of the pronoun with the
  absolutive\\
((\OBL\OBJ\UP)\,\SUBJ) 
must be: \textsc{class\ ii, num} sg\\[0.5em]
\avm[style=fstr]{%\begin{avm}
    [ \textbf{\textsc{subj}} & [ pred & `pro' \\
    pers & 3\\
    \textbf{\textsc{num}} & \textbf{sg} \\
   \textbf{\textsc{class}} & \textbf{\textsc{ii}}\\
    case & abs ]\\
  {tense} & pres\\
   pred & `go\arglist{(\UP \SUBJ)(\UP obl)}'\\
   \textbf{\textsc{obl}}  & [ pred & `behind\arglist{(\UP \OBJ)}'\\
   \textbf{\textsc{obj}} & [ {pred} & {`pro'} \\
             pers & 1\\
            num & sg \\
            case & dat ]
            ] ]
}%\end{avm}
\z
}


\largerpage[2]
Note that the first person pronoun itself does not have any
``agreement slot'' within its own feature structure: it has no
agreement feature specification which is supposed to match or be
copied somewhere else in the (f-)structure.

The informal notions of controller and target have no embodied
representation, which ultimately proves to be an important fact about
the LFG analysis -- because there are examples in which `controller'
and `target' are the same single syntactic element. There are
different types of example in Archi where an absolutive argument
``agrees with itself'' -- a given syntactic element has an external
agreement slot, to agree with the absolutive of its clause, but that
element happens to be the absolutive itself. (See also
\citealp[68--69]{corbett06}, \citealp[137]{borsley16}.) In these
examples, the distinction between controller and target -- as two
distinct elements in an asymmetric relationship -- is invalid, but on
a co-description account of the kind illustrated by (\ref{dez}) the
examples work out straightforwardly.

(\ref{refl}) is one such example. A reflexive pronoun in Archi has two
slots for agreement -- one for the features of its antecedent, as is
familiar, and another one for the features of the absolutive of the
clause. In (\ref{refl}) (from \citealp[70]{boch16}) the subject is the
pronoun `I', in dative case, and the object is the reflexive, in
absolutive case, and it is class \textsc{ii}, signifying a female
referent. The subject pronoun, main verb and auxiliary verb each agree
in class with the absolutive, as does one of the slots in the
reflexive -- the whole form is \textsc{1sg}, agreeing with the subject
antecedent, and there is also an infixed class \textsc{ii} agreement,
again agreeing with the absolutive, which is the reflexive itself.
%(Sadler 166)

\ea\label{refl}Archi\\
\gll d-ez zona{$\scriptstyle\langle$}r{$\scriptstyle\rangle$}u d-ak:u-r-\v{s}i d-i
da{\textchi}on-n-a-\v{s}\\
{\bf\textsc{ii.sg}}\textsc{-1sg.dat}
\textsc{1sg.refl.abs}{$\scriptstyle\langle$}{\bf\textsc{ii.sg}}{$\scriptstyle\rangle$}
{\bf\textsc{ii.sg}}-see-\textsc{ipfv-cvb}
{\bf\textsc{ii.sg}}-be.\textsc{prs}
mirror\textsc{(iv)-sg.obl-in-el}\\
\glt `I am seeing myself in the mirror.'
\z


\largerpage[2]
So here, (\GF\UP)\,\textsc{piv} instantiates as
(\OBJ\UP)\,\OBJ, one of the logical possibilities. The
f-structure is shown in (\ref{ex:Minimalism:52}):

\vbox{
\ea\label{ex:Minimalism:52}
F-structure of (\ref{refl}); 
agreement of the reflexive with the absolutive\\
((\OBJ\UP)\,\OBJ) must be:
      \textsc{class\ ii, num} sg\\[0.5em]
\avm[style=fstr]{%\begin{avm}
    [ subj & [ pred & `pro'\\
            pers & 1\\
          num & sg \\
        class & \textsc{ii}\\
    case & dat ]\\
  {tense} & pres\\
   pred & `see\arglist{(\UP \SUBJ)(\UP \OBJ)}'\\
   \textbf{\textsc{obj}}  & [ pred & `pro'\\
                  refl & + \\
               pers & 1\\
             \textbf{\textsc{num}} & \textbf{sg}\\
             \textbf{\textsc{class}} & \textbf{\textsc{ii}} \\
            case & abs ]\\
  obl & [ & ``{mirror}''\phantom{e} ] ] %phantom to center `mirror'
}%\end{avm}
\z
}
  
Like other aspects of the grammar, the correct account of
``agreement'' does not involve moving something -- in this case,
featural information -- from one place to another, but rather is a
partial specification of featural information in a larger structure.

\section{Conclusion}
\label{sec:minimalism:conclusion}

LFG takes up the challenges of accounting for human language precisely
as Chomsky first articulated them, yet continuing with a view quite
different from his as to what the core non-negotiable properties of
the syntactic system should be. It was developed as a systematic and
coherent framework for the representation of grammatical information,
based on certain key design features. While these design features give
these frameworks a very different character from a procedural
framework such as the MP, many of the components of analysis which MP
has developed are already present in declarative frameworks (see
\sectref{sec:minimalism:features}), as is the convergence of interest in
exploring `third factor' considerations (see (\ref{3factors})).

As Chomsky has noted, the choice between grammatical frameworks can be
understood in terms of the ``extra burdens'' that an over-exuberant
approach will entail (\citealp[10--11]{chomsky07}). From a declarative
perspective, any procedural approach creates such burdens, as the
necessary mechanisms are either too powerful or are not well-founded.
If those mechanisms can change or even delete syntactic information or
syntactic substance, it is necessary to constrain those destructive
operations with the `No Tampering Condition' -- indeed, a very natural
property of a grammatical system, but one that should be intrinsic to
it.

The `copy theory' of movement (\citealp{chomsky1995the-minimalist}) is
a way of expressing the intuition that some information is
shared. However, copies involve duplication of substance, which
amounts to more than the sharing of information.  The discussion of
structure in \sectref{sec:minimalism:ps} provides a perspective on two
kinds of further burden that necessarily arise in a copy-based
approach. First, with regard to head mobility, the LFG view is that
the issue is one of alternative positions, rather than successive
positions which exist to provide hosts for position-occupying
movement. The head is indeed only ``in'' one head position, but it
makes the same contribution that it would have made from any of its
alternative possible positions, and so might appear as if it were
contributing from each position. The evidence from multiple expression
of clausal information supports this view. A recent MP account of the
syntax of heads by \citet{arrepiet20} associates only the
informational part of a given head with several head positions,
effectively recapitulating the LFG analysis of head mobility through
various operations to create the right representations. Second, the
facts of Mandarin verb copying show that certain parts of the
syntactic analysis indeed call for a duplication of substance (when
there are both in-situ arguments and adjuncts in Mandarin), while
other parts involve more abstract syntactic information (the notion of
a verb \textit{having} arguments and having adjuncts). If that abstract
information is conflated with the phrase structure substance, the
system generates too much, and then extra operations have to be
invoked, pruning or conflating substance.

The formalization of the MP by \citet{collstab16} is designed in part
to address the No Tampering Condition, and the apparent duplication of
substance. Instead of creating copies, in this formalization,
successive movements of a given element create new multidominance
relations from that single element, which therefore does not change
during the derivation. Their formalization is extended to head
movement by \citet{bleaman21}. This particular formalization might
make MP derivations slightly closer in nature to f-structures, in that
each object in the derivation is a single informational unit which may
have multiple grammatical relations and phrase-structural relations
(e.g.\ a topicalized object is both an object and a topic, but with
just one overt realization, in topic position).

A different kind of burden of potential complexity falls on the
feature system of the MP, as features are used to do more than
represent information (see \sectref{sec:minimalism:features}). It becomes
necessary to posit ``bad'' feature specifications, such as
uninterpretable features, which by design are not interpretable on
their hosts, and which must be eliminated during the derivation. LFG
has constraints of a different character for checking that certain
grammatical relationships exist, and which do not involve recourse to
local pockets of uninterpretability.

\largerpage
\section*{Acknowledgments}

The content of this chapter has benefitted considerably
  from the comments of three anonymous reviewers, whose contributions
  I gratefully acknowledge; and thanks to Mary Dalrymple for her
  continued guidance.

\sloppy
\printbibliography[heading=subbibliography,notkeyword=this]
\end{document}
