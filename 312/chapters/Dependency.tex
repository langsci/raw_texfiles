\documentclass[output=paper,hidelinks]{langscibook}
\ChapterDOI{10.5281/zenodo.10186040}
\title{LFG and Dependency Grammar}
\author{Dag Haug\affiliation{University of Oslo}}
\abstract{This chapter discusses Dependency Grammar from the perspective of LFG. I first introduce the key ideas behind Dependency Grammar and how they relate to LFG concepts. I then show how both LFGs and Dependency Grammars can be translated into Multiple Context-Free Grammars to study formal differences between the frameworks. Next I discuss two recent efforts to translate from LFG analyses to the version of Dependency Grammar adopted in Universal Dependencies. Finally I show how Glue semantics can be applied to dependency structures.}

\IfFileExists{../localcommands.tex}{
   \addbibresource{../localbibliography.bib}
   \addbibresource{thisvolume.bib}
   \usepackage{langsci-optional}
\usepackage{langsci-gb4e}
\usepackage{langsci-lgr}

\usepackage{listings}
\lstset{basicstyle=\ttfamily,tabsize=2,breaklines=true}

%added by author
% \usepackage{tipa}
\usepackage{multirow}
\graphicspath{{figures/}}
\usepackage{langsci-branding}

   
\newcommand{\sent}{\enumsentence}
\newcommand{\sents}{\eenumsentence}
\let\citeasnoun\citet

\renewcommand{\lsCoverTitleFont}[1]{\sffamily\addfontfeatures{Scale=MatchUppercase}\fontsize{44pt}{16mm}\selectfont #1}
  
   %% hyphenation points for line breaks
%% Normally, automatic hyphenation in LaTeX is very good
%% If a word is mis-hyphenated, add it to this file
%%
%% add information to TeX file before \begin{document} with:
%% %% hyphenation points for line breaks
%% Normally, automatic hyphenation in LaTeX is very good
%% If a word is mis-hyphenated, add it to this file
%%
%% add information to TeX file before \begin{document} with:
%% %% hyphenation points for line breaks
%% Normally, automatic hyphenation in LaTeX is very good
%% If a word is mis-hyphenated, add it to this file
%%
%% add information to TeX file before \begin{document} with:
%% \include{localhyphenation}
\hyphenation{
affri-ca-te
affri-ca-tes
an-no-tated
com-ple-ments
com-po-si-tio-na-li-ty
non-com-po-si-tio-na-li-ty
Gon-zá-lez
out-side
Ri-chárd
se-man-tics
STREU-SLE
Tie-de-mann
}
\hyphenation{
affri-ca-te
affri-ca-tes
an-no-tated
com-ple-ments
com-po-si-tio-na-li-ty
non-com-po-si-tio-na-li-ty
Gon-zá-lez
out-side
Ri-chárd
se-man-tics
STREU-SLE
Tie-de-mann
}
\hyphenation{
affri-ca-te
affri-ca-tes
an-no-tated
com-ple-ments
com-po-si-tio-na-li-ty
non-com-po-si-tio-na-li-ty
Gon-zá-lez
out-side
Ri-chárd
se-man-tics
STREU-SLE
Tie-de-mann
}
   \togglepaper[40]%%chapternumber
}{}

\begin{document}
\maketitle
\label{chap:Dependency}

\section{Introduction}\label{sec:Dependency:intro}
Dependency Grammar (DG) is a tradition for syntactic analysis based on
binary, asymmetric relations (called dependency relations or just
\emph{dependencies}) between words. These relations are typically
labelled, giving rise to a set of labels that can be thought of as
grammatical functions, which are of course also important in
LFG. In fact, the correspondence between dependencies in DG and
grammatical functions in LFG and their central role in both theories
is the main similarity, formally and conceptually, between the two
frameworks.

The primacy of dependencies is what holds together work in the DG
tradition. As we will see, it is characteristic of almost all
DG theories that they acknowledge a level of syntax that we will call
the \emph{core dependencies}. This is a set of dependencies restricted
so as to form a tree over the words of a sentence, i.e. a structure
where each word has exactly one head, except the root word, which has
none (or equivalently, is attached to a synthetic root node). (\ref{ex:simpletree})
shows a very simple example of this.

\begin{exe}
  \ex\label{ex:simpletree}
  \begin{dependency}[baseline=-0.6ex,theme=simple]
    \begin{deptext}
      Tracy \& loves \& Chris\\
    \end{deptext}
    \deproot[edge unit distance=1.5ex]{2}{\textsc{root}}
    \depedge{2}{1}{subj}
    \depedge{2}{3}{obj}
  \end{dependency}
\end{exe}
Most theoretical work and concrete analyses have seen the need to
introduce additional mechanisms or levels of structure beyond core
dependencies to give the theory more analytical bite; this goes all
the way back to \citet{Tesniere1959}, the founding work of modern DG.
However, there is typically little agreement about the additional
mechanisms or levels of structure between individual scholars working
in the DG tradition. So, while the core dependency representation is
often acknowledged as theoretically inadequate, it has enjoyed
considerable popularity as a simplified representation with practical
applications in computational linguistics and natural language
processing.

But even restricting attention to core dependencies, there are a
number of choice points where different dependency frameworks make
different decisions. For example, when the core dependencies model structures with a lexical word and one or more function words (for example, articles and
nouns, auxiliaries and full verbs, or prepositions and their
complements), we must take a stance on
whether the lexical or the functional word is the head: the co-head
option often used in LFG is not available. (\ref{ex:functionwords}) shows what the
(unlabelled) dependency structure of a simple sentence would look
like if we take function words as heads (left) or lexical
words as heads (right).
\begin{exe}
  \ex\label{ex:functionwords}
  \begin{dependency}[baseline=-0.6ex,theme=simple]
    \begin{deptext}
      Fido \& has \& slept \& on \& the \& mat \\
    \end{deptext}
    \deproot[edge unit distance=2ex]{2}{\textsc{root}}
    \depedge{2}{1}{}
    \depedge{2}{3}{}
    \depedge{3}{4}{}
    \depedge{4}{5}{}
    \depedge{5}{6}{}
  \end{dependency}
  \begin{dependency}[baseline=-0.6ex,theme=simple]
    \begin{deptext}
      Fido \& has \& slept \& on \& the \& mat \\
    \end{deptext}
    \deproot[edge unit distance=2ex]{3}{\textsc{root}}
    \depedge{3}{1}{}
    \depedge{3}{2}{}
    \depedge{3}{6}{}
    \depedge{6}{5}{}
    \depedge{6}{4}{}
  \end{dependency}
\end{exe}
It is obviously not necessary to treat all function words the same,
and so there are intermediate variants between these two extremes,
taking for example prepositions and articles as heads, but not auxiliary verbs.

Another point at which dependency grammarians diverge is the treatment
of coordination. Because coordination is normally thought of as
symmetric, it is not easy to represent with directed
dependencies. Here the most common competing analyses, shown in
(\ref{ex:conjunctions}), involve taking the first conjunct as the head (left),
which entails giving up on symmetry; or to make the conjunction the
head (right) and maintain symmetry, but at the cost of dissociating the
conjuncts from their normal head (e.g. the verb), which is the basis
for most morphosyntactic and semantic constraints.



\begin{exe}
  \ex\label{ex:conjunctions}
  \begin{dependency}[baseline=-0.6ex,theme=simple]
    \begin{deptext}
      Mary \& likes \& fruit \& and \& vegetables\\
    \end{deptext}
    \deproot[edge unit distance=2ex]{2}{\textsc{root}}
    \depedge{2}{1}{}
    \depedge{2}{3}{}
    \depedge{3}{4}{}
    \depedge{3}{5}{}
  \end{dependency}
  \begin{dependency}[baseline=-0.6ex,theme=simple]
    \begin{deptext}
      Mary \& likes \& fruit \& and \& vegetables\\
    \end{deptext}
    \deproot[edge unit distance=2ex]{2}{\textsc{root}}
    \depedge{2}{1}{}
    \depedge{2}{4}{}
    \depedge{4}{3}{}
    \depedge{4}{5}{}
  \end{dependency}
\end{exe}

Faced with the choices illustrated in (\ref{ex:functionwords}) and
(\ref{ex:conjunctions}) many linguists in the DG tradition have felt
that neither analysis is satisfactory, and they have therefore reacted
by enriching the dependency formalism in various ways that result in
data structures that have more in common with LFG. I discuss some key
examples of this in \sectref{sec:Dependency:keyideas}.  Even if much
theoretical work in DG assumes such enriched data structures, most
practical applications of DG rely on core dependencies, thereby
forcing choices that, at least from an LFG perspective, are somewhat
arbitrary.

One key difference between DG and LFG is that dependency
grammarians typically do not formalize their work and in many cases do
not provide (even informal) rules that generate the constructions they
are interested in but content themselves with providing analyses of
the whole structure. This goes back to the earliest dependency
grammarians such as Tesnière, but has become even more prominent with
the increasing use of dependency structures in data-driven parsing,
where the goal is not to define a grammar that recognizes (or
generates strings from) a formal language, but to parse strings into a
single plausible structural representation. Nevertheless, it \emph{is}
possible to conceive of DGs as formal grammars. In
\sectref{sec:Dependency:wordorder} I discuss how this can be done using the
framework of Multiple Context-Free Grammars. While this is not an
approach that most dependency grammarians follow, it yields a useful 
framework for comparing DG and LFG. Another useful perspective on DG
and LFG is offered by recent efforts to translate LFG resources into
DG resources, which I discuss in \sectref{sec:Dependency:computational}.
\sectref{sec:Dependency:semantics} explores the potential for combining
dependency grammars with Glue semantics, the standard semantic
framework in LFG.


\section{The dependency grammar tradition and LFG}\label{sec:Dependency:keyideas}
The idea of using binary, labelled, asymmetric relations to analyze
syntax is found in the work of P\=aṇini, Ancient Greek and Roman grammarians and
the speculative grammarians of the Middle Ages
\citep{Covington1984}. On its own, this idea is too vague to define a
theoretical framework and both P\=aṇini and the speculative
grammarians have also been seen as forerunners of generative grammar
\citep{Kiparsky1994,Chomsky1966}. What defines the modern dependency
grammar tradition, which started with \citet{Tesniere1959}, is the attempt to base syntax primarily, or even
exclusively, on the concept of core dependencies, as opposed to the
concept of constituency developed in American structuralism and the
generative tradition. Although there have been a number of attempts to
develop dependency grammar into a full-fledged grammatical theory (the
most well-known ones being Functional Generative
Description \citep{SgallEtAl1986}; Meaning--Text Theory (MTT) \citep{Melcuk1988}; and
Word Grammar \citep{Hudson84,Hudson2010}), none of these are very
widespread beyond the environments where they originated and hence
there is no single, coherent version of DG as a formal framework.  The
focus of this section is therefore not to identify assumptions made in
specific frameworks, but rather to compare ideas that are common in
the dependency grammar tradition with LFG.

\subsection{Dependency graphs and f-structures}
There is an obvious similarity between dependencies, as found in DG,
and the binary, labelled, asymmetric relations between the nodes of an
LFG f-structure.\footnote{To emphasize the parallelism between
f-structures and dependency graphs, we rely here on the
graph-theoretic interpretation of attribute-value matrices, where
feature structures and atomic values are nodes, and attributes are
labelled edges between these nodes, and not the ``official''
interpretation of f-structures as functions
\citep{kaplan1995formal,kaplanbresnan82}. The graph-theoretic interpretation is standard in most other
unification-based frameworks from Functional Unification Grammar \citep{kay1979functional} onwards, and was, to my knowledge, first formalized by \citet{MoshierRounds}. It is used in HPSG \citep{rich:20}; see
\citetv[section 4]{chapters/HPSG} for discussion of the
differences between the two views.} In both cases, the
relations form a directed labelled graph over nodes corresponding to
linguistic material. The similarity even extends to the set of labels
used, which in both cases contain traditional grammatical functions
such as subject and object. Formally, however, there are two important
differences: First, the nodes of the f-structure are not words, but
correspond to zero, one or several words/c-structure terminals. This
is how LFG escapes the indeterminacy of direction of headedness in
constructions which combine lexical and functional words that we saw
in (\ref{ex:functionwords}).
Second, labelled dependencies are
not necessarily functional, i.e. there may be two or more daughters
bearing the same relation to the same head, in violation of LFG's
uniqueness condition.\footnote{LFG can deal with several dependents bearing the same relation by using set-valued attributes e.g. for \textsc{adjunct}; this introduces the concept of sets, which also has no counterpart in DG.}

In addition to these two formal differences, there are in practice
many more differences, because DG analyses rarely use the full power
of a directed graph and instead typically emphasize the core
dependencies, which form a tree spanning the words of the sentence. To the extent
that e.g. multiple heads are used, one of the heads is typically considered ``primary''. Even so, the
formal similarities between dependencies and f-structures mean that
similar theoretical questions can arise in both DG and LFG and even
that one can think of LFG's f-structures as dependency graphs that
take a particular view on certain foundational questions in DG.\footnote{Furthermore, on the implementation side, \citet{broker-1998} shows how DGs can be encoded as LFGs and implemented in the XLE platform.}

An overarching question in the DG tradition (see e.g. 
\citealt[199f.]{deMarneffeNivre2019}) is whether dependency relations
are sufficient for analyzing syntax. In one sense, the answer is
obviously no. Like f-structures, dependency structures say nothing
about word order. This is dealt with in the c-structure in LFG, and
scholars within dependency grammar have also seen the need to enrich
the theory with a mechanism for constraining word order. I return to
this in \sectref{sec:Dependency:wordorder}. But more fundamentally, one
might ask whether core dependencies, tree structures over words, are
sufficient to capture functional aspects of syntax like f-structures
do in LFG.

In fact, it is not too hard to see that core dependencies cannot fully
represent the functional relations of a sentence. Consider for example, the
subject in a raising construction.

\begin{exe}
  \ex
    It seems to rain.
\end{exe}
%
The expletive \textit{it} bears a functional relation to the raising
verb \textit{seems} as witnessed by agreement; but the form of the expletive is
licensed by the lexical verb \textit{rain} (and would be different in
e.g. \textit{There seems to be a problem}), giving evidence for a
second functional relation. If one insists on core dependencies, one
of the two relations must be privileged.

The alternative is to increase the expressivity of the theory, and
this is in fact what Tesnière did when he introduced two other kinds of
relations beside dependencies that can hold between words, namely
junction (\emph{jonction}) and transfer (\emph{translation}). Junction
is the relation that holds between coordinated items that are either
dependents of the same head or heads of the same
dependent. Translation is the relation that holds between lexical
words and functional words that license their appearing in various
dependencies. For example, complementizers ``translate'' verbs so as
to license their appearing in object position according to the
analysis in \citet[24]{Tesniere1959}; similar analyses are given for
determiners and adpositions.

Crucially, words that are linked by
junction or transfer form a complex node (\textit{nucleus dissocié})
in the dependency graph and jointly contract dependency relations. In
this way, their dependents end up having more than one head; and
they can collectively bear a single dependency relation to their
head. In this respect, Tesnière's analyses are in fact quite close to
standard LFG f-structures, where coordination is analyzed in terms of
a set-valued attribute (\ref{ex:dance}) and function words such as
e.g. auxiliaries form a single f-struture node with their lexical verb
(\ref{ex:maryhasarrived}).

\begin{exe}
  \ex\label{ex:dance}
  \begin{xlist}
  \ex
  All boys and girls dance.
  \ex\evnup{
    \avm[style=fstr]{
      [pred & `dance\arglist{\textsc{subj}}' \\
        subj & \{ [pred & `boy' \\
                   spec & \1[ pred & `all' ] ]\\\\
        [pred & `girl' \\
        spec & \1 ] \} ]
    }}
    \ex\label{ex:junction}
    \begin{dependency}[baseline=-0.6ex,theme=simple]
      \begin{deptext}
        All \& boys \& --- \& and \& --- \& girls \& dance \\
      \end{deptext}
      \wordgroup{1}{2}{6}{subject}
      \depedge{2}{1}{det}
      \depedge{7}{2}{subj}
      \depedge{6}{1}{det}
      \depedge{7}{6}{subj}
      \deproot{7}{\textsc{root}}
    \end{dependency}
  \end{xlist}
\end{exe}


\begin{exe}
  \ex\label{ex:maryhasarrived}
  \begin{xlist}
    \ex
    Mary has arrived.
    \ex\evnup{
    \avm[style=fstr]{
      [pred & `arrive\arglist{\textsc{subj}}' \\
        tense & perfect \\
        subj & [ pred & `Mary' ] ]
    }}
    \ex\label{ex:translation}
    \begin{dependency}[baseline=-0.6ex,theme=simple]
      \begin{deptext}
        Mary \& has \& arrived \\
      \end{deptext}
      \wordgroup{1}{2}{3}{verb}
      \wordgroup[color=white]{1}{1}{1}{subject}
      \groupedge{verb}{subject}{subj}{2}
      \deproot{3}{\textsc{root}}
    \end{dependency}
    \end{xlist}
\end{exe}
In this respect, both Tesnière's theory and LFG's f-structure reject
the idea that syntactic dependencies can be adequately captured in a
tree structure over the words of a sentence. Nevertheless, LFG's
approach is much more general than Tesnière's. Tesnière allows
many-to-one relations between words and dependency nodes based on
relations that are not dependencies, but he maintains the tree
structure over dependency nodes. Therefore, the only way a word can
have two heads is if those heads form a single node by junction or
transfer, as in (\ref{ex:junction}) and (\ref{ex:translation}); but
LFG also allows for a word to have two heads that do not form a group,
as in the analysis of functional control verbs (\ref{ex:functionalcontrol}).

\begin{exe}
  \ex\label{ex:functionalcontrol}
  \begin{xlist}
    \ex Chris persuaded Mary to come
    \ex\evnup{
    \avm[style=fstr]{
      [pred & `persuade\arglist{\textsc{subj}, \textsc{obj}, \textsc{xcomp}}' \\
        subj & [pred & `Chris' ] \\
        obj &  \1[pred & `Mary'  ] \\
        xcomp & [pred & `come\arglist{\textsc{subj}}' \\
          subj & \1  ]
    ] }}
  \end{xlist}
\end{exe}
%
Such dependencies cannot be expressed in Tesnière's formalism, because
\textit{persuade} and \textit{come} share the dependent \textit{Mary},
despite not forming a group. Moreover, \textit{Mary} bears a different
syntactic relation to each of them, which again is not possible in
Tesnière's formalism. More recent versions of dependency grammar have
typically accounted for control and raising verbs by positing more
levels of representation, see \sectref{subsec:morelevels}.

Finally, an important difference between Tesnière's dependency graphs and
f-structures is that f-structures may contain nodes that correpond to
no overt word. A typical case is pro-drop, as in (\ref{ex:prodrop}) from Italian.

\ea\label{ex:prodrop}
  \ea
    \gll vengono\\
    come-\PRS.3\PL\\

    \ex\evnup{
    \avm[style=fstr]{
      [pred & `come\arglist{\textsc{subj}}' \\
        subj & [ pred & `pro' ] ]
      }}

    \ex
    \begin{dependency}[baseline=-0.6ex,theme=simple]
      \begin{deptext}
        vengono \\
      \end{deptext}
      \deproot{1}{\textsc{root}}
    \end{dependency}
  \z
\z
%
Again, Tesnière's formalism cannot capture this: dependency nodes may
correspond to one word, or more words if they form a group by junction
or transfer, but not to zero. The strategy in later versions of DG has
been the same as that used to address phenomena where LFG uses
structure sharing, namely to introduce more levels of representation.

\hspace*{-4.3pt}In sum, one can say from an LFG perspective that Tesnière's dependency
graphs, while certainly more expressive than core dependencies, are
insufficiently general to deal with the complex functional relations
that exist in natural language sentences.

\subsection{Other levels of syntactic representation}\label{subsec:morelevels}
Tesnière's strategy was to enrich dependency graphs so as to be able
to represent more functional relations than core dependencies can do.
More recent versions of DG have instead opted to keep the core
dependencies simple and instead go beyond a single level of
grammatical description to accommodate more information. One prominent
example is the so-called tectogrammatical layer found in Functional
Generative Description \citep{SgallEtAl1986} and the associated Prague Treebanks \citep{HajicEtAl2020}. This layer
is annotated with an enriched dependency tree that will contain nodes
that do not correspond to words (e.g.\ pro-dropped subjects) and
secondary edges capturing multiple head-phenomena such as
control.\footnote{The status of the tectogrammatical layer is not
  entirely clear: the Prague Dependency Treebank annotation guidelines
  (\url{https://ufal.mff.cuni.cz/pdt2.0/doc/manuals/en/t-layer/html/ch02.html})
  say that it ``represents the semantic structure of the sentence'',
  but \citet{HajicEtAl2020} describe it as ``deep
  syntax''. The difference may be merely terminological.}

Mel\v{c}uk's Meaning--Text Theory explicitly distinguishes a
deep syntactic level between the semantic level and surface
syntax. However, as pointed out by \citet{Kahane2003}, the deep
syntactic level is the least defined level of MTT and it is not clear
how much information it is supposed to contain. What is clear,
however, is that grammatically imposed coreference relations are
resolved in deep syntax, opening up a way to deal with, e.g., control.

In Word Grammar \citep{Hudson84,Hudson2010}, too, control is treated by loosening the tree
constraint on dependency structures. Example~(\ref{ex:WGstructureshare}), from \citet[521]{Hudson2003}, illustrates how structure
sharing is used to analyze raising (\textit{you} shared by
\textit{have} and \textit{been})\footnote{Instead of Hudson's \textit{sharer}, I have used the LFG relation \textsc{xcomp} which Hudson explicitly mentions as an alternative name for the same concept. The diagram in \citet[521]{Hudson2003} does not have a
  subject relation between \textit{you} and \textit{looking}, although
  \textit{looking} is an \textsc{xcomp} of \textit{been}. It is
  unclear whether this is just an error.   } and extraction
(\textit{what} shared by \textit{have}, \textit{been},
\textit{looking} and \textit{at}).

\begin{exe}
  \ex \label{ex:WGstructureshare}
  \begin{dependency}[baseline=-0.6ex,theme=simple]
    \begin{deptext}
      What \& have \& you \& been \& looking \& at?\\
    \end{deptext}
    \deproot{2}{\textsc{root}}
    \depedge{2}{1}{extractee}
    \depedge{2}{3}{subj}
    \depedge{2}{4}{xcomp}
    \depedge{4}{5}{xcomp}
    \depedge{5}{6}{prep}
    \depedge[edge below]{4}{3}{subj}
    \depedge[edge below]{4}{1}{extractee}
    \depedge[edge below]{5}{1}{extractee}
    \depedge[edge below, label style={below}]{6}{1}{extractee,complement}
  \end{dependency}
\end{exe}

The dependency graph in (\ref{ex:WGstructureshare})  is
essentially identical to the standard LFG analysis (except that in
extraction, LFG usually has structure sharing only between the gap and
the filler position, without involving the inter\-mediate f-struc\-tures).
However, in Word Grammar, the edges above and below the words have
different status:
\begin{quote}This diagram also illustrates the notion `surface
structure' [\dots]. Each dependency is licensed by the grammar
network, but when the result is structure-sharing just one of these
dependencies is drawn above the words; the totality of dependencies
drawn in this way constitutes the sentence's surface structure. In
principle any of the competing dependencies could be chosen, but in
general only one choice is compatible with the `geometry' of a
well-formed surface structure, which must be free of `tangling'
(crossing dependencies -- i.e.\ discontinuous phrases) and `dangling'
(unintegrated words). There are no such constraints on the non-surface
dependencies.'' \citep[521]{Hudson2003}
\end{quote}

This illustrates the point that I made in the introduction: different
varieties of dependency grammar may have different notions of ``deep
syntax'', but they all share the idea that there is an interesting
representation of syntactic dependencies that is a rooted tree over
nodes that stand in a one-to-one correspondence with the words of the
sentence. This is very different from LFG: all edges of an f-structure
graph are equal. The subject edge that connects the subject of a
control construction to the control verb has exactly the same status
as the subject edge that connects the subject to the non-finite
verb. Thus, there is no ``privileged subgraph'' of the f-structure
that forms a rooted tree over the words. By contrast, Hudson's
distinction between the surface structure and the non-surface
dependencies gives rise to such a privileged subgraph, although it
must be said that the distinction between surface and non-surface
dependencies is not further developed in Word Grammar.

Dependency grammars also differ in their treatment of ``null words'',
i.e. cases where LFG would have an f-structure node that does not
correspond to any surface word, as in e.g.\ pro-drop. Most dependency
analyses would simply leave out such subjects, as we saw in (\ref{ex:prodrop}).
But here too, many dependency grammars introduce the missing subjects
in ``deeper'' projections, for example in the tectogrammatical layer
of Functional Generative Description. In fact, Word Grammar is one of
the few dependency grammar frameworks that acknowledge empty elements
in the core syntactic graph. \citet{CreiderHudson2006} present an argument for this
that runs along standard lines of LFG thinking. In Ancient Greek,
predicate nouns and adjectives agree in case (and adjectives also in number and
gender) with their subjects; and subjects of infinitives are in the
accusative.

\ea Ancient Greek (Xenophon, Anabasis 1.3.6)\\
\gll {nomízo:} {gàr} {humâ:s}   emoì      eînai  kaì  patrída             kaì  phílous\\
{think-1.\PRS} {for}  {you-\ACC} {me-\DAT} {be-\textsc{inf}} and {fatherland-\ACC} and {friends-\ACC}\\
\glt `For I think you are to me both fatherland and friends'
\z
But crucially, the predicative is accusative also when the accusative subject is absent (\ref{ex:noaccusative}), even in cases where there is a coreferential element in the higher clause (\ref{ex:longdistance}).

\ea\label{ex:noaccusative} Ancient Greek (Isocrates 2.15)\\
\gll philánthro:pon eînai deî \\
{humane-\ACC} {be-\textsc{inf}} must\\
\glt `one must be humane' 
\z

\ea\label{ex:longdistance} Ancient Greek (Plato, Alcibiades 2, 141a7)\\
\gll exarkései        soi         túrannon  genésthai \\
{suffice-\FUT} {you-\DAT} {king-\ACC} {become-\textsc{inf}} \\
\glt `it will be enough for you to become king' 
\z
%
In (\ref{ex:longdistance}), we observe that the predicate noun \textit{turannon} does not agree directly with its logical subject \textit{soi}, but rather with the unexpressed subject of the infinitive.
Since case agreement is generally agreed to be syntactic (whereas
agreement in number and gender could potentially be semantic), \citet{CreiderHudson2006}
conclude that the unexpressed subject of the infinitive must
nevertheless be present in the syntax. This is unsurprising from an
LFG point of view, but does not seem to be generally accepted in
DG. It is unclear, for example, how Functional
Generative Description would deal with this kind of data, since null words are
inserted only at the tectogrammatical layer, where there is no case
feature.

\section{Word order and generative power in DG and LFG}\label{sec:Dependency:wordorder}
In most versions of dependency grammar, it is assumed that the nodes
of a dependency structure are not linearly ordered in themselves: a dependency relation implies no particular linear order between a head and its dependents, but
can be related to different surface linearizations. This view goes
back to \citet[chapter 7]{Tesniere1959}, who distinguishes sharply
between structural order (dependencies) and linear order. The main
exception to this is Functional Generative Description, which assumes
a linear order on the nodes even in the tectogrammatical layer, to
capture information structure.

But even if the nodes of the dependency structures are not linearly
ordered, it is possible (and in fact necessary for most languages) to
constrain the relation between dependency structure and
linearization. One much-discussed constraint is \emph{projectivity}.\footnote{It seems that this term originated with a technical report by P. Ihm and Y. Lecerf ``Eléments pour une grammaire générale des langues projectives'', Bruxelles 1960, but I have been unable to find this paper.}

\ea\label{def:projectivity} A dependency graph is projective iff for every edge $n_h
\rightarrow n_d$ it contains, $n_h$ dominates all nodes that occur
between $n_h$ and $n_d$ (where domination is the transitive closure of
the edge/dependency relation)
\z
%
An early result due to \citet{Gaifman1965} is that projective
dependency grammars are weakly equivalent to context-free
grammars.\footnote{See also \citet{Hays1964}.} This result may in fact
have led to a lack of interest in dependency grammar because it was
widely believed in the sixties and seventies (and eventually proved in
the eighties) that natural languages are \emph{not} context-free. On
the other hand, the recognition problem for a dependency grammar with
no linearization constraints at all (thus allowing arbitrary
discontinuities) is NP complete \citep{NeuhausBroeker97}.\footnote{As
we will see in \sectref{sec:Dependency:computational}, this is not an issue
in data-driven parsing, which sidesteps the recognition problem and
aims directly at providing a contextually plausible parse.}

With the increasing popularity of dependency grammars in the 2000s,
this led to the search for \emph{intermediate} linearization
constraints between strict projectivity and arbitrary
non-projectivity. One important class of constraints is based on the
notion of \emph{block degree} \citep{HolanEtAl1998}. Intuitively,
projectivity as defined in (\ref{def:projectivity}) ensures that the
subgraph of $n_h$ (i.e. $n_h$ and the set of nodes it dominates) forms
a single block of adjacent nodes. We can instead allow the subgraph to
form \emph{two} blocks of adjacent nodes, interrupted by a continuous
set of words. We say that $n_h$ has block degree 2; and the block
degree of a dependency tree is the highest block degree of any of its
nodes. Equivalently, we can speak of gap degree, which is block degree
minus 1 (i.e., the number of allowed gaps). (\ref{ex:latinex}) illustrates this
with an example from Latin.


\begin{exe}
  \ex\label{ex:latinex}
  \begin{xlist}
    \ex Latin\\
    \gll Mihi nullus est terror\\
    me.\textsc{dat} none.\textsc{nom} is fear.\textsc{nom}\\
    \glt `I have no fear.'
    \ex \label{ex:latintree}
      \begin{dependency}[baseline=-0.6ex,theme=simple]
        \begin{deptext}[column sep=1cm]
          mihi \& nullus \& est \& terror \\
        \end{deptext}
        \deproot{3}{\textsc{root}}
        \depedge{3}{4}{{subj}}
        \depedge{4}{2}{{adj}}
        \depedge{3}{1}{{obl}}
      \end{dependency}
  \end{xlist}
\end{exe}

The gap degree of \textit{est} is 0, since its subgraph is continuous;
but the gap degree of \textit{terror} is 1, since there is one gap in
its subgraph -- \textit{est} intervenes between \textit{terror} and
\textit{nullus}, but is not dominated by \textit{terror}. As a result,
the gap degree of the whole tree is 1.

\hspace*{-2.5pt}To study the computational complexity of the dependency grammars that
could generate structures like (\ref{ex:latinex}), and their relationship to LFG
grammars, it is convenient to use phrase structure-based systems
that allow discontinuities, so-called Linear Context-Free Rewriting
Systems (LCFRS, \citealt{vijay-shanker-etal}) or the notational variant Multiple Context-Free
Grammars (MCFG, \citealt{SekiEtAl1991}).  The MCFG formalism is a generalization of CFG which
retains ordinary CFG productions for the expression of categorial
structure, but uses explicit \emph{yield functions} to compute the
yield of the mother node from the yields of the daughters. In an
ordinary CFG, yield computation is conflated with category formation:
a rule such as DP $\rightarrow$ D NP says both that the category DP is
formed of a D and an NP, and that the yield of the resulting DP is
formed by concatenating the yields of D and NP. In effect, then, a CFG
can be seen as an MCFG with concatenation as the only yield
function.\footnote{See \textcite{clark2014} for an accessible
introduction for linguists and \citet[chapter 6]{Kallmeyer2010} for a
more formal introduction.}

To allow for greater expressivity, MCFG allows yields to be
\emph{tuples} of strings. For example, we may want to say
that the yield of DP is a pair (2-tuple) consisting of the yields of D
and NP. This pair will then be the input to further yield functions
that apply to productions with DP on the right-hand side. More
generally, we may allow yields to be $n$-tuples of strings. The interesting point is that there is a close
correspondence between yield components in an MCFG and blocks in a
corresponding dependency structure. We can extract MCFG rules from
dependency trees, as shown in \textcite{Kuhlmann2013}, where a formal
exposition is given. Here I just provide an intuitive understanding of
how the tree in (\ref{ex:latintree}) gives rise to the rules
in \tabref{tbl:mcfgrules}.

\begin{table}
  \begin{tabular}{lll}\lsptoprule
    rule & yield function & compact notation \\\midrule
  $\textsc{adj} \rightarrow g()$ & $g=\langle$\textit{nullus}$\rangle$   & $\textsc{adj} \rightarrow \langle$\textit{nullus}$\rangle$ \\
  $\textsc{obl} \rightarrow h()$ & $h=\langle$\textit{mihi}$\rangle$ & $\textsc{obl} \rightarrow \langle$\textit{mihi}$\rangle$\\
  $\textsc{subj} \rightarrow i(\textsc{adj})$ & $i=\langle x_1,$\textit{terror}$\rangle$ & $\textsc{subj} \rightarrow \langle x_1,$\textit{terror}$\rangle$ (\textsc{adj})\\
  $\textsc{root} \rightarrow j(\textsc{obl subj})$ & $j=\langle x_1 y_1$ \textit{est} $y_2\rangle$ & $\textsc{root} \rightarrow \langle x_1 y_1$ \textit{est} $y_2\rangle$(\textsc{obl subj})\\\lspbottomrule
\end{tabular}
\caption{Rules extracted from the tree in (\ref{ex:latintree})}\label{tbl:mcfgrules}
\end{table}

Looking at \textit{nullus} in (\ref{ex:latintree}), we see that it has no dependents, hence the right-hand side of the first rule is a constant function which fixes the yield to the string \textit{nullus}, and similarly for \textit{mihi}. For \textit{terror}, things are more interesting. It takes one dependent, an \textsc{adj}, and hence its yield function $i$ depends on the value of that argument. Concretely, the yield of the node \textit{terror} is a tuple, consisting of the yield of the \textsc{adj} dependent which is represented as $x_1$,\footnote{The convention is that we use $x$ for the yield of the first dependent and $y$ for the yield of the second dependent, and subscript those variables with an index referring to blocks of the yield.} and the string \textit{terror}. Finally, the verb takes two arguments, $\textsc{subj}$ and $\textsc{obl}$. The yield is constructed by concatenating the yield of the $\textsc{obl}$ (i.e.\ $x_1)$, the first component of the $\textsc{subj}$ (i.e.\ $y_1)$, the string \textit{est}, and the second component of \textsc{subj} ($y_2$).

With the rules in \tabref{tbl:mcfgrules}, we can construct the MCFG derivation tree in (\ref{ex:mcfg}).

\begin{exe}
\ex\label{ex:mcfg}
\begin{forest}
  [{$\langle x_1 y_1$ \textit{est} $y_2\rangle$}
    [{$\langle$\textit{mihi}$\rangle$} ]
    [{$\langle x_1,$\textit{terror}$\rangle$}
      [{$\langle$\textit{nullus}$\rangle$} ]
    ]
  ]
\end{forest}
\end{exe}

But notice that because the MCFG grammar is lexical, i.e. each rule introduces exactly one lexical item, the tree in (\ref{ex:mcfg}) is isomorphic to the dependency tree in (\ref{ex:latintree}). In other words, a lexicalized MCFG can simply be interpreted as a dependency grammar which simultaneously restricts word order.

This allows us to compare the generative capacity and the parsing
complexity of dependency grammars with other formalisms. Under a
reasonable constraint on discontinuities,\footnote{Namely
  wellnestedness; a tree is wellnested if there are no disjoint subtrees that overlap linearly.} the expressivity of an MCFG depends only
on the maximal block degree of the grammar, giving rise to a hierarchy
of $k$-MCFGs, where $k$ is the block degree of the most complex yield
function in the grammar. It turns out that 2-MCFGs (and hence
dependency grammars that allow maximally one gap) are weakly
equivalent to Tree Adjoining Grammars and `classical' Combinatory
Categorial Grammar, as was proven by \citet{BodirskyEtAl2005}.\footnote{See also \citet{Kuhlmann2007,Kuhlmann2010}.}

Even more interesting from an LFG perspective, there is also a result
that a subclass of LFG grammars, so-called \emph{finite copying LFGs},
can be translated into weakly equivalent MCFGs/LCFRSs
\citep{SekiEtAl1993}. Finite copying LFGs are quite restricted in what
functional annotations they allow, in particular they do not allow
head annotations (\ua=\da) or reentrancies, and also impose the
crucial constraint that the grammar puts an upper bound on the number
of c-structure nodes corresponding to a single
f-structure. \citet{wed:kap:20} show that we can impose this upper
bound while still allowing head annotations and reentrancies, as long
as they are nonconstructive. This allows most functional equations
that are used in linguistic work, including functional control
equations of the type (\ua\,\textsc{f
  g})=(\ua\,\textsc{h}). \citet{wed:kap:20} call these grammars
$k$-bounded LFGs and prove that for any $k$-bounded LFG, a weakly
equivalent $k$-MCFG can be constructed. Moreover, the MCFG rules can
be annotated with functional descriptions that allow us to construct
the f-structure that the corresponding $k$-bounded LFG assigns to the
sentence, yielding a strongly equivalent MCFG.

These results allow us to compare dependency grammars and LFGs in a
precise way. First of all, dependency grammars and $k$-bounded LFGs
are weakly equivalent. Nevertheless, although strongly equivalent
MCFGs can be constructed from both dependency grammars and $k$-LFGs,
it is not the case that we can construct a strongly equivalent
dependency grammar from an LFG. The interpretation of an MCFG as a
dependency grammar relies on unique lexicalization: each rule contains
a single lexical item interpreted as the head. The MCFGs that
\citet{wed:kap:20} construct from LFGs are not lexicalized in this
way. They do contain functional descriptions that allow us to identify
the head but, since LFG allows co-heads, the head is not guaranteed to
be unique. Moreover, the functional descriptions in the MCFG
constructed from an LFG may contain reentrancies, i.e. words having
more than one head, which have no interpretation on the dependency
grammar side, thus losing information. A final, minor point is that
Kuhlmann's interpretation of MCFGs as dependency grammars say nothing
about edge labels; it would be natural and straightforward to
interpret LFG's ordinary function assignments as such labels.

\hspace*{-.1pt}In sum, then, the formal analysis tells us that the difference between
$k$-bounded LFGs and dependency grammars resides exactly in the
availability of co-heads and reentrancies, which provide important
information from a linguistic point of view. Finally, it should be
noted that the restriction to $k$-bounded LFGs, while preserving
coverage of many, perhaps most, linguistic phenomena, is nevertheless
not trivial. \citet{Rambow2014} argued that unbounded scrambling as
found in German and other free word order languages falls outside the
generative capacity of MCFGs (and mildly context sensitive grammar
formalisms in general) and hence $k$-bounded LFGs.

The comparison of dependency grammars and LFGs through MCFGs is also
interesting from other points of view. As \citet{wed:kap:20} point
out, the effect of converting an LFG to an MCFG is to precompute the
interaction between f- and c-structure and construct a grammar that
recognizes all and only the c-structures whose f-descriptions are
satisfiable. From a practical point of view, this may be an advantage
in parsing. But from the perspective of theoretical LFG, it can be
argued that MCFGs and the dependency grammars they give rise to
conflate c- and f-structure, making it harder to state linguistic
generalizations. The advantage of LFG's projection architecture is
precisely ``to account for signifiant linguistic generalizations in
a factored and modular way by means of related but appropriately
dissimilar representations'' \citep[309]{kapl:89}. Seen from the
dependency grammar side, the formal results offer a choice:
Kuhlmann's translation to MCFGs makes it possible to enrich dependency
grammars with an account of word order in a single component; but
\citegen{wed:kap:20} results show that MCFGs can be ``modularized''
into a word order component and a functional component (which is not
surprising given that MCFGs generalize CFGs precisely by dissociating
dominance and linearization) to give something very close to LFG.
Either way, the formal analysis exposes similarities and differences
between the frameworks. In principle, this paves the way for
cross-fertilization on the theoretical side, but in practice such
gains are limited by the fact that, as I pointed out in
\sectref{sec:Dependency:intro}, dependency grammarians typically do not think
in terms of (formal or informal) rules that generate the constructions
they are interested in but content themselves with providing analyses
of the whole structure.

%% \subsection{Long distance dependencies}
%% Let us look more closely at one particular kind of discontinuity, long
%% distance depencies. The Word Grammar analysis was shown in Figure
%% \ref{fig:WGstructureshare}. It involves secondary dependencies beyond
%% the dependency tree proper. For the surface tree, the Word Grammar
%% analysis chooses to represent only the \emph{extractee} relation
%% between \textit{what} and \textit{have}, but this choice is governed
%% by the requirement that the surface structure be projective. From a
%% semantic point of view, it is clear that we can never represent the
%% correct meaning of this sentence if we do not have the
%% \emph{complement} relation between \textit{at} and \textit{what}. So
%% we could choose to have represent only this relation and this is in fact the choice of most dependency treebank.

%% Figure showing projective analysis (e.g. of Heringer/Erom/some treebank, see Müller)

%% Because such an analysis allows nonprojective dependencies, it needs
%% to be constrained in order to not overgenerate. 

%% \begin{exe}
%%   \ex What have you been looking at?
%%   \ex At what have you been looking?
%%   \ex *What have at you been looking?
%% \end{exe}

%% One way to constrain the construction is to use MCFG rules to take
%% care of the word order. This minimizes functional information: we only
%% know that \textit{what} depends on \textit{at} and the linearization
%% rules take care of the rest. The other possibility is to allow
%% multiple dependencies as in Word Grammar. On this view, we use more
%% functional information, but the linearization is simpler:
%% \textit{what} simply occurs next to (one of) its heads. This could be
%% translated into an MCFG that is actually a context free grammar (with
%% only continuous yield functions), but would need functional
%% annotations to retrieve the secondary dependencies/reentrancies needed
%% to recover the meaning. Unfortunately, the existing dependency grammar
%% analyses that exists, surveyed in Müller XXX, are not sufficiently
%% explicit to allow a comparison of the pros and cons of the different approaches.



\section{DG and LFG in computational linguistics}\label{sec:Dependency:computational}
\subsection{Data-driven dependency parsing}
On the computational side, there is a similar difference between DG on
the one hand, and LFG and most other formal linguistic traditions on
the other hand, in
that there has generally been little interest in developing formal
grammars that can generate or parse languages. There are some
exceptions to this: in the framework of Constraint Dependency Grammar
\citep{Maruyama1990}, there is for example a broad-coverage parser of
German \citep{FothEtAl2005}; and Constraint Grammar
\citep{karlsson1995} is a widely used system in which implemented
grammars have been created for a wide variety of languages. Many of
these grammars content themselves with assigning syntactic function
labels to words, without building a full syntax tree, but even so,
many have proven useful in practical tasks.

Nevertheless, the dominant use of DG in computational linguistics is
closely associated with machine learning approaches where computers
find patterns in human annotated data. For such approaches, it is
sufficient that annotators provide case-by-case analyses of the corpus
without actually abstracting the rules that would create these
analyses. Consistency remains a goal, since it makes the patterns
easier to learn, but it is not enforced in the way it would be in
grammar-based annotation such as typical scenarios for creating LFG
parsebanks, where annotators choose between alternative analyses
provided by the underlying grammar.

As we have seen several times so far, the constraints on core
dependency syntax, namely the unique mother and the one-to-one
correspondence between nodes and tokens, mean that many theoretically
relevant distinctions cannot be encoded. On the flip side, this makes
the annotation task easier as the annotator does not have to be
trained in drawing the distinctions. The result is also often more
accessible to end users: while grammar-based treebanks contain much
more information than dependency trees, this information is typically
encoded in a specific theoretical framework and not always easily accessible to
users without training in that framework. In short, core dependency
trees offer a tradeoff between practical considerations and
theoretical depth, which may be attractive for many applications where
the deeper linguistic distinctions do not matter much.

On top of that, the simple target structure makes it possible to train
very efficient statistical dependency parsers. This approach is
fundamentally different from the formal grammar approach to DG
developed by \citet{Kuhlmann2013}, which we saw in
\sectref{sec:Dependency:wordorder}. Data-driven parsers learn from human
annotation and try to provide the most plausible parse in context,
without judging acceptability or enumerating possible parses. In this
context, non-projective dependencies are not an issue and can be
captured efficiently \citep{mcdonald-etal-2005}. \citet{nivre2008}
introduced algorithms that could produce projective dependency parses
in time linear of the input and algorithms that allow non-projective
parses and run in quadratic time. Such results led to a huge increase
of interest in dependency parsing, which quickly became dominant in
statistical approaches to computational linguistics.

Data-driven parsing requires annotated data and the last decade has
seen a large increase in the number of dependency treebanks that are
available, especially driven by the Universal Dependencies (UD)
initiative.\footnote{See \url{https://universaldependencies.org/} and
\citet{deMarneffeEtAl2021}.} UD developed out of the Stanford
dependencies for English \citep{DeMarneffeManning2008} (which means
that there is a certain amount of LFG heritage) as an effort to create
an annotation scheme that can be used across languages. Though it has
been driven mainly by practical considerations in NLP research,
it has in recent years also been used for linguistic research (e.g. \citealt{Hahn2347,berdicevskis-piperski-2020-corpus}).

As of release 2.9 (November 2021), UD contains 217 treebanks from 122
languages. A comparison with LFG's ParGram approach reveals the
strengths and weaknesses of the approach.\footnote{For more on ParGram, see \citetv{chapters/ImplementationsApplications}.} Drawing on the long
tradition of using DG to provide case-by-case analyses rather than
abstracting grammars has made it possible to achieve an unprecedented
breadth of coverage. On the other hand, the analyses are more shallow
than those provided by LFG grammars and the lack of underlying
grammars makes the UD project much more prone to inconsistencies both
within and across treebanks. 

\subsection{Converting LFG parsebanks to dependency treebanks}
The existence of annotated resources in both LFG and DG formats makes
it possible to study differences between the two from a different 
perspective than the formal language approach we adopted in
\sectref{sec:Dependency:wordorder}. In this section, we look at work on
converting LFG-based resources to dependency structures to see how the
two formats compare and to what extent information can be preserved
when converting to the less expressive DG format.

For completeness, we mention that there has also been some work on
enriching them to yield LFG-structures, e.g. by
\citet{forst-2003-treebank} and \citet{haug2012dependency}. However,
both Forst and Haug started from relatively rich dependency
annotations (with secondary edges), so that the conversion to
f-structures was not difficult and other issues were more important
(e.g. the creation of c-structures from the dependency representations
by Haug).

Several conversion algorithms have been developed to convert LFG
structures to dependency structures. Here I discuss two recent
approaches, by \citet{Meurer17LFG} and
\citet{prz:pat:19:lre},\footnote{\citet{dione2020b} presents an
approach that combines \citet{Meurer17LFG} and
\citet{prz:pat:19:lre}. For older work, see \citet{Ovrelid2009} and
\citet{Cetinoglu2010}.} which contrast in interesting ways, since
Meurer starts from the c-structure and Prze\-pi{\'o}r\-kowski and
Pate\-juk from the f-structure. Both are natural starting points: the
f-structure represents grammatical functions, just like the target
dependency structure; but the c-structure has the advantage that its
terminal nodes are in one-to-one correspondence with the words of the
sentence, just like in the dependency structure. Both algorithms
target the particular style of dependency annotation adopted in UD,
but proceed in two steps, namely first the creation of a dependency
structure, and second, the modification of that structure to comply
with the exact representation chosen in UD. Here we focus on the first
step. To illustrate how the two algorithms work, we consider the LFG
structure in (\ref{ex:depconvertC})--(\ref{ex:depconvertF}).

\begin{exe}
  \ex \label{ex:depconvertC}
  \begin{forest}
  [IP
    [{(\ua\ \TOPIC)=\da\\NP$_1$} [{\ua=\da\\N} [this ] ] ]
    [{\ua=\da\\IP}
      [{(\ua\ \SUBJ)=\da\\NP$_2$} [{\ua=\da\\N} [John ] ] ]
      [{\ua=\da\\V$'$} [{\ua=\da\\V} [wants ] ]
        [{(\ua\ \XCOMP)=\da\\VP} [{\ua=\da\\{PART}} [to ] ]
          [{\ua=\da\\V} [see ]]]]]]
  \end{forest}
  \ex \label{ex:depconvertF}\evnup{
  \avm[style=fstr]{
    [pred & `want\arglist{\textsc{subj, xcomp}}' \\
      topic & \1[ pred & `pro' ] \\
    subj  & \2[ pred & `John' ] \\
    xcomp & [ pred & `see\arglist{\textsc{subj, obj}}' \\
      subj & \1 \\
      obj & \2 ]
    ]
  }}

\end{exe}
%
In Meurer's approach, the first step is to ``lexicalize'' the
c-structure tree by recursively replacing each non-terminal node with
its functional head node, as determined by the annotation
$\ua=\da$. This is straightforward for IP, NP$_1$ and NP$_2$ in
(\ref{ex:depconvertC}): \textit{wants}, \textit{this} and \textit{John}
are uniquely linked to these nodes via an unbroken chain of
$\ua=\da$. But more generally, the challenge here is the same as in
lexicalizing an MCFG that results from the Wedekind-Kaplan
construction: co-heads and absence of heads mean there might be no
unique daughter to lift. To find a unique head in such cases the
algorithm proceeds as follows: 1) if no daughter of $x$ is a
functional head, attach all daughters to the mother of $x$ and proceed
as before; 2) if more than one daughter of $x$ is a functional head,
choose the one with the shortest embedding path; 3) if there is a tie,
choose the leftmost node. For the VP in our example, case 3 applies and we choose
\textit{to} as the head; it is therefore lifted to the VP node, while
\textit{see} is only lifted to the V node. These lifting operations
yield the tree in (\ref{ex:liftedtree}).

\begin{exe}
  \ex \label{ex:liftedtree}
  \begin{forest}
    [wants
      [this ]
      [John ]
      [to [see ]]]
  \end{forest}
\end{exe}
%
\largerpage
We then need to label the edges. Meurer's algorithm does that by
labelling the edge between nodes $x$ and their daughter $y$ in the
resulting tree with the f-structure path from $\phi(x)$ to
$\phi(y)$. So, the edge from \textit{John} to \textit{wants} is
labelled \textsc{subj} since that is the path from the f-structure of
\textit{wants} to the f-structure of \textit{John}. But because of
reentrancies in the f-strucure, the path between two f-structures is
not always unique: for example, there is a path from the f-structure
of \textit{wants} to the f-structure of \textit{this} that is labelled
\textsc{topic}, but there is another path that is labelled
\textsc{xcomp obj}. In such cases, Meurer chooses the shortest path
that contains only grammatical functions (i.e. no discourse
functions); in our case that yields the complex label \textsc{xcomp.obj} where the two elements of the f-structure path have been concatenated with a dot. Co-heads present another problem for the labelling approach:
\textit{to} and \textit{see} share the same f-structure, so there is
no path. In such cases a dummy relation $=$ is used.


This algorithm produces a projective dependency graph with complex
labels, as shown for our example in the lefthand side of (\ref{ex:meurerconvert}). In the
next step, the complex labels are resolved and nodes attached
accordingly, potentially introducing non-projectivity. For our
example, when the complex relation \textsc{xcomp.obj} is
resolved, we obtain the non-projective tree on the right-hand side of (\ref{ex:meurerconvert}).

\begin{exe}
  \ex\label{ex:meurerconvert}
  \begin{minipage}{0.4\textwidth}
  \begin{dependency}[baseline=-0.6ex,theme=simple]
    \begin{deptext}
      This \& John \& wants \& to \& see\\
    \end{deptext}
    \deproot{3}{\textsc{root}}
    \depedge{3}{1}{\textsc{xcomp.obj}}
    \depedge{3}{2}{\textsc{subj}}
    \depedge{3}{4}{{\textsc{xcomp}}}
    \depedge{4}{5}{=}
  \end{dependency}
  \end{minipage}
  \begin{minipage}{0.1\textwidth}
    $\Rightarrow$
  \end{minipage}
  \begin{minipage}{0.4\textwidth}
  \begin{dependency}[baseline=-0.6ex,theme=simple]
    \begin{deptext}
      This \& John \& wants \& to \& see\\
    \end{deptext}
    \deproot{3}{\textsc{root}}
    \depedge{4}{1}{\textsc{obj}}
    \depedge{3}{2}{\textsc{subj}}
    \depedge{3}{4}{\textsc{xcomp}}
    \depedge{4}{5}{=}
  \end{dependency}
  \end{minipage}
\end{exe}
%
This then is the input to the final step, where the dependency tree is
normalized according to the UD annotation standard.

\largerpage
\citet{prz:pat:19:lre}, by contrast, start from the f-structure, which
already represents the syntactic dependencies. This means the
challenge is different, namely to match the nodes of the f-structure
to the words of the sentence, which are the nodes of the target
dependency graph. F-structure nodes may correspond to zero, one or
several words; they are given by the $\phi^{-1}$, which is part of the
source annotation. F-structures that correspond to no words (e.g. in
pro-drop) may simply be ignored in the dependency structure; but for
f-structures that correspond to more than one word, the ``true'' head
that will take the f-structure's place in the corresponding dependency
structure must be identified, and the other words in $\phi^{-1}$ must
be attached with appropriate relations. The basic algorithm is simple:
if there is a verbal token in $\phi^{-1}$, choose that as the true
head; otherwise, choose a nominal or adjectival token; otherwise an
explicit lexical conjunction. The other nodes are then attached to the
true head with a relation labelled by their own preterminal
category. This produces the structure in (\ref{ex:ppconvert}) from the f-structure in
(\ref{ex:depconvertF}).

\begin{exe}
  \ex\label{ex:ppconvert}
  \begin{dependency}[baseline=-0.6ex,theme=simple]
    \begin{deptext}
      This \& John \& wants \& to \& see\\
    \end{deptext}
    \deproot{3}{\textsc{root}}
    \depedge{3}{1}{\textsc{topic}}
    \depedge{3}{2}{\textsc{subj}}
    \depedge{3}{5}{\textsc{xcomp}}
    \depedge{5}{4}{\textsc{part}}
    \depedge[edge below]{5}{1}{\textsc{obj}}
  \end{dependency}
\end{exe}

As we can see, the output from the algorithm of \citet{prz:pat:19:lre}
is not a tree, but a graph, where all f-structure relations are preserved, including two incoming edges to \textit{this}. This is exploited to produce enhanced UD, which allows for this kind of graph structure; but the
output is also trimmed to produce a basic UD structure.

(\ref{ex:meurerconvert})--(\ref{ex:ppconvert}) illustrate the output
of the first steps in the conversions, where the target is to produce
the desired data structure, namely a dependency tree or graph over
words. As mentioned, the next step is to normalize this structure to
the concrete requirements of the UD annotation standard. This is less
interesting from our point of view, but it is worth looking at a few
topics that display divergences between standard LFG solutions and
choices that are made in the dependency grammar community as
exemplified by UD.

First, UD subscribes to the primacy of content
words. This means that content words are typically heads of function
words, for example in structures consisting of auxiliary and verb,
adposition and noun, and determiner and noun, as illustrated in the lower graph of (\ref{ex:functionwords}) in \sectref{sec:Dependency:intro}. Also, there are
no nested structures of function words, so e.g. in structures with multiple auxiliaries (\textit{may have been understood}), all the auxiliaries attach directly to the lexical verb. While UD may be extreme among
dependency grammar approaches in adopting this principle across the
board, similar analyses are found for some of these structures in
other frameworks. By contrast, such analyses are non-existing in the
LFG literature, except for noun-determiner structures (where the
determiner is often analyzed as a \textsc{spec} dependent of the
noun): function words are typically either co-heads, or lone heads,
taking a lexical word as their dependent. For example, there are
analyses of auxiliaries as co-heads specifying features of the
f-structure where the lexical verb contributes the \textsc{pred}, and
alternative analyses where auxiliary verbs take \textsc{xcomp}
dependents, potentially in a cascading sequence ending in the lexical
verb.

In fact, the difference between the co-head analysis and the UD
dependent analysis of function words is rather slight, as revealed by the
conversion procedure of \citet{prz:pat:19:lre}. In
f-structures that have functional co-heads, the lexical head will be
chosen as the head during conversion, and hence the function words
will end up as dependents. And in fact, given that UD uses a flat
structure for multiple function words means that the two
representations are more or less equivalent, a point made in the UD
documentation
too,\footnote{\url{https://universaldependencies.org/u/overview/syntax.html}}
where it is said that function word relations are different from
dependency relations between content words and in fact form
Tesnière-style nuclei.

This in turn opens the door to theoretical cross-fertilization. What
are good criteria for choosing between the two analyses? The UD
argument is that primacy of lexical words maximizes parallelism across
languages, and the exact same argument has been raised in the LFG
literature \citep{buttetal96}. On the other hand, \citet{dyvik99} has
countered that this leads to a stipulative, rather than empirical,
notion of language universality and also that it can lead to analyses
that are language-internally unmotivated. Recently,
\citet{OsborneGerdes2019} criticized the UD approach and argued that
functional words should always be heads. They were apparently unaware
of the LFG literature on the topic, perhaps because it is cast in terms of
co-heads vs. \textsc{xcomp}s. But as we have seen, the
difference between a co-head analysis and a UD-style annotation is
very slight, and so the arguments made in the LFG context are certainly relevant also for the DG community.

The other main divergence between initial dependencies, as resulting
from the conversion algorithms, and the target UD structures concern
coordination. Here LFG makes use of an additional data structure,
sets, which have no equivalents in standard dependency grammar or in
UD. (Although as we saw above, Tesnière's junction comes close.) There
are many competing analyses of coordination in the dependency
literature,\footnote{And also in (pre-UD) dependency treebanks, see \citet{pop:etal:13}.} maybe suggesting that the basic data structure of
dependency trees is ill-suited to model coordination, as
Tesnière argued. The UD choice is to take the first conjunct
as the head and attach the other conjuncts to it with a special
dependency relation \textsc{conj}, whereas conjunctions and
punctuation marks are attached to their following conjunct with
\textsc{cc} and \textsc{punct}. It is known that this annotation style
cannot capture all important structural differences, such as the
difference between a dependent of the first conjunct and a shared
dependent of multiple conjuncts, or different style of nested
coordinations. The conversion procedure exposes this lack of
expressivity, but also makes it possible to quantify its effect. As
observed by \citet{prz:pat:19:lre}, only twelve out of 21,732 utterances in the Polish LFG structure bank are effected.\footnote{See \citet{przepiorkowski-patejuk2019} for a proposal as to how nested coordination could be captured in UD.}

More generally, \citet{prz:pat:19:lre} conclude that the information
loss in converting from LFG to (enhanced) UD is in fact negligible,
except in the case of pro-drop structures. As the UD effort continues
to expand, there is therefore considerable potential for theoretical
cross-fertilization.


\section{Semantics}\label{sec:Dependency:semantics}
Tesnière in general pays much less attention to meaning than to
structure, but at various points he does talk about semantic
dependencies. Several versions of dependency grammars (Functional
Generative Description, Meaning--Text Theory) have taken this up and
operate with a separate level of semantic structure. There are also
various graph-based semantic representation languages such as Abstract
Meaning Representation (AMR, \citealt{banarescu-etal-2013}), which arguably are semantic
dependency representations without an accompanying syntactic
representation. All such semantic dependency graphs, whether they are coupled
to syntax or not, differ considerably from standard logic-based
formalizations of meaning as used in LFG and most other formal
frameworks. They will not be further discussed here.

Robaldo's Dependency Tree Semantics \citep{Robaldo2006} is much closer to standard conceptions of formal semantics, as it aims to transform dependency trees into structures that can be interpreted model-theoretically. But for the purposes of comparison with LFG, it is more interesting to observe that \citet[308]{Broeker2003}, in his discussion of the formal
foundations of dependency grammar, briefly suggested that the
similarity between dependency trees and LFG's functional structure
could make the application of Glue semantics
(\citealt{dalrympleetal93,Dalrymple:Glue}; see also
\citetv{chapters/Glue}) to dependency grammar a promising research
area. \citet{gotham-haug:2018} flesh out this idea and show how to
combine Universal Dependencies with Partial Compositional Discourse Representation Theory \citep{Haug2014}.

On the formal side, there are few if any obstacles to such an
application.  The fundamental idea behind glue semantics is to have
linear logic terms guide the composition of corresponding lambda
terms.  In the first order glue setting, the terms of the linear logic
are the f-structures, atomic formulae are formed by applying
predicates to the f-structures (in type-theoretic terms, these
predicates act like unary type constructors) and complex formulae are
formed with $\multimap$, which acts as a binary function type
constructor. Consider (\ref{ex:semantics}), which gives the meanings, dependency structure and f-structure for \textit{Everybody loves somebody}. We write $e_1$ for $e(1)$, i.e. the application of the type constructor/predicate $e$ to the syntactic object/term with index $1$.

\begin{exe}
  \ex\label{ex:semantics}
  \begin{xlist}
    \ex \label{ex:gluesem} \begin{tabular}[t]{lll}
      \textit{everybody} & $\lambda P.\forall x.\IT{person}(x) \rightarrow P(x)$ & $(e_1 \multimap t_2) \multimap t_2$ \\ 
      \textit{somebody}  & $\lambda P.\forall x.\IT{person}(x) \land P(x)$ & $(e_3 \multimap t_2) \multimap t_2$ \\
      \textit{loves}     & $\lambda x.\lambda y.\IT{love}(x,y)$ & $e_1 \multimap e_3 \multimap t_2 $ \\
      \end{tabular}
    \ex \label{ex:d}
      \begin{dependency}[baseline=-0.6ex,theme=simple]
        \begin{deptext}[column sep=1cm]
          everybody \& loves \& somebody \\
        \end{deptext}
        \deproot{2}{\textsc{root}}
        \depedge{2}{1}{{subj}}
        \depedge{2}{3}{{obj}}
      \end{dependency}

      \ex\label{ex:f}\evnup{
      \avm[style=fstr]{
        [pred & `love\arglist{ \textsc{subj} }' \\
          subj & [ pred & `everybody ] \\
          obj  & [ pred & `somebody' ] ]
      }}

  \end{xlist}
\end{exe}
%
Clearly, it makes no difference whether we interpret the glue types in
(\ref{ex:gluesem}) over the dependency tree in (\ref{ex:d}) or the
f-structure in (\ref{ex:f}): in both cases we just need the same mapping between the indices 1, 2, 3 and the corresponding f-structures or dependency nodes.

However, while the formal properties of the two theories are similar
enough that Glue semantics can be used for both LFG and DG, a practical consideration is that dependency trees typically do not contain all the semantically relevant information that we find in the 
corresponding f-structure. Control structures are a case in point (\ref{ex:abrams}).
\begin{exe}
  \ex\label{ex:abrams}
\begin{dependency}[baseline=-0.6ex,theme=simple]
  \begin{deptext}
    Abrams \& persuaded \& the \& dog \& to \& bark \\
  \end{deptext}
  \deproot{2}{\textsc{root}}
  \depedge{2}{1}{{subj}}
  \depedge{2}{4}{{obj}}
  \depedge{4}{3}{{det}}
  \depedge{2}{6}{{xcomp}}
  \depedge{6}{5}{{mark}}
\end{dependency}
\end{exe}
%
The dependency tree in (\ref{ex:abrams}) lacks the information that
\textit{the dog} is the subject of \textit{to bark}. However, the
label \textsc{xcomp} does tell us that the missing subject of
\textit{to bark} is one of the dependents of \textit{persuaded}. As a
result, the best we can do is to introduce a discourse referent $x_2$
that is the subject of the infinitive clause and must be linked to one
of the participants in the matrix event, though we do not know which
one, unless we have access to the lexical information that
\textit{persuade} is an object control verb.
%
%  \begin{exe}
%    \ex\label{ex:gluelambda}
%    $\lambda P.\lambda y.\lambda x.\lambda F.$\SDRS{e_1~x_1~x_2}{\IT{persuade}(e_1)\\\IT{controldep}(e_1,x_2)\\\IT{xcomp}(e_1,q)\\\IT{obj}(e_1,y)\\\IT{subj}(e_1,x)\\q:P({x_2})(\lambda \_.[\ |\ ]) } ; $ F(e_1)$
%  \end{exe}
%
We see that it is possible to compensate for some of the information
loss in dependency trees, although the result only becomes useful if
we have other, lexical information sources available: dependency trees
on their own do not typically come with the rich semantic
lexical entries that glue (and other formal semantic theories)
require.
We refer to
\citet{gotham-haug:2018} for more details.


\section{Summary}
We have seen that the basic relations for analyzing functional syntax,
dependencies in DG and grammatical functions in LFG, are very similar,
both formally and conceptually. Nevertheless, the focus on core
dependencies that is often seen in DG work leaves other levels of
analysis less well-developed than in LFG. Word order, in particular,
has not received much attention, but we have seen that it can be
interestingly restricted through the use of lexicalized MCFGs,
offering a point of comparison to LFGs, which can also be translated
to MCFGs. Another point of comparison is offered by work on converting
LFG parsebanks to dependency treebanks. Finally, we saw that the
similarity between DG and LFG also means that they can use the same
syntax-semantics interface in the form of Glue semantics.

All in all, the considerable similarities between the two theories
suggest there is ample room for mutually benefiting discussion,
especially if the increasing use of DG in computational linguistics
triggers a corresponding interest in theoretical DG.

\section*{Acknowledgements}
I thank the two anonymous reviewers, one non-anonymous reviewer (Adam Przepiórkowski) and the editor Mary Dalrymple for very helpful comments on a previous version of this chapter.
 
\sloppy
\printbibliography[heading=subbibliography,notkeyword=this]
\end{document}
