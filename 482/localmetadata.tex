\title{Teacher education for working in linguistically diverse classrooms}
\subtitle{Nordic perspectives}
\BackBody{This volume presents studies on aspects of teacher education that prepare teachers for working in linguistically diverse classrooms and schools in five Nordic countries; Denmark, Finland, Iceland, Norway and Sweden. This twin focus (teacher education in linguistically diverse contexts; and Nordic perspectives) makes the volume unique in its field, and contributes to international discussions on how teacher education can prepare pre-service and in-service teachers for working with linguistically diverse student groups. 
 
The volume includes contributions on teacher education policies, teacher educators’ perspectives on teacher education, and pre-service teacher perspectives on teacher education.
    
The ways in which teacher education prepares educators for working with newcomers and multilingual students has attracted considerable attention in recent years. This reflects the increasingly linguistically diverse nature of classrooms that teachers around the world meet, that is in turn, a direct result of intensified globalisation and transnational migration. Clearly, teacher education is crucial for successful implementation of educational provisions for multilingual students. Teacher knowledge, gained partly through teacher education, plays a central role in creating educational environments where multilingual students can thrive. 

This volume focuses specifically on teacher education in a Nordic context, a region traditionally associated with progressive approaches in education based on principles of inclusivity, social justice and equal opportunity (Blossing et al. 2014, Frønes et al. 2020). In the twenty-first century, most Nordic countries have experienced increasing levels of migration. While multilingualism and transnational migration are not new phenomena in the region, geographical and social factors, as well as the ways humans communicate have helped make multilingualism more visible in the twenty-first century (Aronin \& Singelton 2008). Schools in the Nordic countries have had to act quickly and think flexibly to meet the needs of an increasingly linguistically and culturally heterogenous group of students. The ability of the Nordic countries to provide these students with “inclusive, equal education and a fair chance to start a new life” constitutes in some ways the ultimate test of the “Nordic model” of education (Lundahl 2016: 10). Investigating how this challenge is addressed in different forms of teacher education is the topic to which this volume turns its attention.}
\author{Anne Reath Warren and Jonas Yassin Iversen and Boglárka Straszer} 
 

\renewcommand{\lsISBNdigital}{978-3-96110-507-6}
\renewcommand{\lsISBNhardcover}{978-3-98554-136-2}
\BookDOI{10.5281/zenodo.15147416}
\typesetter{Ivo Boers, Sebastian Nordhoff}
\proofreader{Amir Ghorbanpour,
Brett Reynolds,
David Carrasco Coquillat,
Eliane Lorenz,
Elliott Pearl,
Ikmi Nur Oktavianti,
Jeroen van de Weijer,
Leonie Twente,
Mary Ann Walter,
Nicoletta Romeo,
Rainer Schulze,
Thera Crane,
Viola Wiegand
}
\lsCoverTitleSizes{45pt}{15mm}% Font setting for the title page


\renewcommand{\lsSeries}{cib}
\renewcommand{\lsSeriesNumber}{5}
\renewcommand{\lsID}{482}
