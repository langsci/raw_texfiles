\documentclass[output=paper]{langscibook}
\ChapterDOI{10.5281/zenodo.15280893}
\author{Anne Reath Warren\orcid{}\affiliation{Uppsala University} and Jonas Yassin Iversen\orcid{}\affiliation{Inland Norway University of Applied Sciences} and Boglárka Straszer\orcid{}\affiliation{University of Dalarna}}
\title[Setting the scene]{Setting the scene: Teacher education for linguistically diverse classrooms in the Nordic region}
\abstract{This introductory chapter provides a contextual background to the topic addressed in the volume’s chapters and discusses common themes across the contributions. After introducing the volume and its Nordic context, we give a brief description of the Nordic educational and migration context from a multilingual perspective. The terminology used in this volume to describe categories of students, teachers and forms of education is then introduced. An overview of the literature on teacher education and linguistic diversity is presented, followed by a summary of the ways that teacher education is organised in the five Nordic countries represented in the volume. Each of the chapters is then introduced. Issues common across contexts include tensions regarding how multilingual perspectives can be included in teacher education programmes, which pre-service teachers benefit from knowledge about multilingualism and how to incorporate both theoretical and practical knowledge into teacher education. The volume points to an ongoing need for researchers to engage with teacher educators in questions relating to linguistic diversity.}
\IfFileExists{../localcommands.tex}{
  \addbibresource{../localbibliography.bib}
  \usepackage{langsci-optional}
\usepackage{langsci-gb4e}
\usepackage{langsci-lgr}

\usepackage{listings}
\lstset{basicstyle=\ttfamily,tabsize=2,breaklines=true}

%added by author
% \usepackage{tipa}
\usepackage{multirow}
\graphicspath{{figures/}}
\usepackage{langsci-branding}

  
\newcommand{\sent}{\enumsentence}
\newcommand{\sents}{\eenumsentence}
\let\citeasnoun\citet

\renewcommand{\lsCoverTitleFont}[1]{\sffamily\addfontfeatures{Scale=MatchUppercase}\fontsize{44pt}{16mm}\selectfont #1}
   
  %% hyphenation points for line breaks
%% Normally, automatic hyphenation in LaTeX is very good
%% If a word is mis-hyphenated, add it to this file
%%
%% add information to TeX file before \begin{document} with:
%% %% hyphenation points for line breaks
%% Normally, automatic hyphenation in LaTeX is very good
%% If a word is mis-hyphenated, add it to this file
%%
%% add information to TeX file before \begin{document} with:
%% %% hyphenation points for line breaks
%% Normally, automatic hyphenation in LaTeX is very good
%% If a word is mis-hyphenated, add it to this file
%%
%% add information to TeX file before \begin{document} with:
%% \include{localhyphenation}
\hyphenation{
affri-ca-te
affri-ca-tes
an-no-tated
com-ple-ments
com-po-si-tio-na-li-ty
non-com-po-si-tio-na-li-ty
Gon-zá-lez
out-side
Ri-chárd
se-man-tics
STREU-SLE
Tie-de-mann
}
\hyphenation{
affri-ca-te
affri-ca-tes
an-no-tated
com-ple-ments
com-po-si-tio-na-li-ty
non-com-po-si-tio-na-li-ty
Gon-zá-lez
out-side
Ri-chárd
se-man-tics
STREU-SLE
Tie-de-mann
}
\hyphenation{
affri-ca-te
affri-ca-tes
an-no-tated
com-ple-ments
com-po-si-tio-na-li-ty
non-com-po-si-tio-na-li-ty
Gon-zá-lez
out-side
Ri-chárd
se-man-tics
STREU-SLE
Tie-de-mann
} 
  \togglepaper[1]%%chapternumber
}{}

\begin{document}
\maketitle 
%\shorttitlerunninghead{}%%use this for an abridged title in the page headers

\section{{Introduction}}\label{sec:reath:1}

In the twenty-first century, countries in the Nordic region, along with many other countries, have experienced increasing levels of migration. While neither multilingualism nor transnational migration are new phenomena in the region, geographical and social factors, as well as the ways humans communicate, have helped make multilingualism more visible (\citealt{AroninSingleton2008}). One implication of increased migration is an increase in the number of school students who speak languages in addition to or other than the languages in which education is conducted. Some of these students may have recently migrated, and are therefore often referred to as \textit{newcomers}, while others were born in the region and raised in families where languages in addition to the official majority languages are used. A complex range of factors (individual, family language policy, educational policy, sociolinguistic and language ideological) {i}nfluence multilingual development, and the learning experiences that multilingual students meet at school (\citealt{HornbergerSkilton-Sylvester2000,The_douglas_fir_group2016}). In contexts where factors that are conducive for language learning and development compound, there is potential for these students to become multilingual, maintaining and developing all the languages in their repertoire. However, if factors that work against language learning and development are dominant, there is a risk that multilingual students lose the language(s) they speak in contexts outside of school and/or face challenges at schools where instruction is in a less familiar language. As language and learning are intrinsically connected, the language development outcomes for multilingual students impact the extent to which they achieve their full potential at school. In this volume, we turn our attention to a significant factor in language development and educational outcomes, namely, teacher education and the ways it prepares teachers for working in linguistically diverse classrooms in five Nordic countries: Denmark, Finland, Iceland, Norway and Sweden.

The Nordic countries are often lauded as societies based on values of equal opportunity, social justice and modernity (\citealt{BlossingEtAl2014}). In education, these values have in part been translated into policies, which are intended to be operationalised through the development of different forms of support and subjects for newcomers and multilingual students, including the right to second language education, bilingual tutoring and mother tongue education. These policies have existed in different forms and to different extents in the Nordic region since the 1960s. Schools and classrooms in the Nordic region have become increasingly multilingual in the twenty-first century and are thus well-served by such policies and rights. However, the sociopolitical setting has also changed in the Nordic region, and debate about the raison d'être for the above-mentioned policies and rights has become louder, in some regions resulting in more restrictive policies regarding mother tongue education for example (\citealt{AlisaariEtAl2023}). Neither is the region immune from “the monolingual mindset” (e.g., \citealt{HermanssonEtAl2022}) or other forms of inequality (e.g., \citealt{Peterson2022}). Indeed, \citet[100]{Habel2012} argues that in the Swedish context, there is a need for critical reflection on the country’s “lingering attachment to collectively held conceptions of political, historic and cultural innocence vis-à-vis racial issues”. This complex mixture of historical, ideological and political factors makes teacher education for linguistically diverse classrooms in the Nordic region an interesting case, in a context ripe for critical analysis.

This volume comprises empirical studies from five Nordic countries which describe and analyse how different aspects of teacher education prepare and support teachers for working with multilingual students in different classroom contexts. The volume includes contributions exploring aspects of teacher education policies, teacher education pedagogy as well as pre- and in-service teacher perspectives on teacher education. Moreover, the volume presents research on a relatively unique educational provision for multilingual students that has been developed in the Nordic countries (study guidance in the mother tongue) to an international audience.

The editors of this volume all work in teacher education programmes, often teaching courses with a focus on the preparation of pre- and in-service teachers for work in linguistically diverse contexts. In our work and in our research, we have seen a need for a volume that presents research on the opportunities and challenges pre- and in-service teachers and teacher educators in the Nordic region face as they prepare for work in linguistically diverse contexts. We also have seen a need for analysis and critical discussion of practices in teacher education and in linguistically diverse classrooms. Bringing together these perspectives and practical examples from different Nordic contexts, this volume contributes to international discussions on these topics. 

In this introductory chapter, we present the Nordic context.  In \sectref{sec:reath:2}, a basic description of the demographics of each of the countries represented in the volume is presented. As the terms used for categories of students, teachers and forms of education are different in each national context, we also introduce the terminology used in this volume, in \sectref{sec:reath:3}. This is followed in \sectref{sec:reath:4} by a brief review of the literature on teacher education internationally and in the Nordic context. In \sectref{sec:reath:5} the ways that teacher education is organised in Denmark, Finland, Iceland, Norway and Sweden are presented, followed in \sectref{sec:reath:6}, by a brief introduction to each of the chapters in the volume. In the final \sectref{sec:reath:7}, common themes emerging from the chapters are discussed.  

\section{The Nordic context: Demographics of migration and education}\label{sec:reath:2}

The Nordic region has traditionally been associated with progressive, centralised approaches to education, based on principles of inclusion, social justice and equal opportunity, to compensate for the disadvantages that students from diverse social and cultural backgrounds, abilities, genders or geographical locations face at school (\citealt{BlossingEtAl2014,Elstad2020,Frones2020}). Students who speak languages other than or in addition to the language of schooling at home are in some cases included in this group, and are offered forms of support (for example, second language instruction, multilingual tutoring and/or mother tongue education). However, teachers do not always recognise or understand how to support students who have a mother tongue other than the majority language, which increases the risk that the education system will not fulfil its duty to provide compensatory approaches.

While the advantages of multilingualism and well-designed long-term bilingual education (e.g., \citealt{ThomasCollier2002}) are frequently quoted in research on the Nordic educational contexts, education is still overwhelmingly conducted in Nordic languages only. Some multilingual students, newcomers in particular, face challenges in schools where achievement is most commonly measured monolingually (e.g. \citealt{Cummins2000}). Some years ago, in an international comparison of social justice measures in education, Denmark, Finland, Iceland, Norway and Sweden were ranked highly on the Justice Index, particularly in terms of access to education, social cohesion and intergenerational justice (\citealt{Schraad-Tischler2011}). This ranking has changed in Sweden, where the impact of socio-economic background on students’ performance in science at age 15 was more recently ranked as “around average” (\citealt{OECD2017}). While Denmark, Finland, Iceland and Norway are still ranked highly in terms of socio-economic equity, a need to better support learners of migrant backgrounds has been identified in all these countries (\citealt{OECD2016, OECD2020, OECD2020-1, OECD2020-2}). These reports indicate that teacher education in the Nordic countries would benefit from critical analysis, to identify ways in which it can better prepare teachers for meeting the needs of their increasingly linguistically and culturally heterogeneous student bodies. According to \citet[10]{Lundahl2016}, the ability of the Nordic countries to provide all students, including newcomers and other multilinguals, with “inclusive, equal education and a fair chance to start a new life” constitutes a significant test for the “Nordic model” of education. Critical examination of teacher education plays a crucial role in rising to this challenge. This volume contributes to this critical examination.

In the next section, Nordic demographics are described, with a specific focus on the growing proportion of the population who have multilingual backgrounds. The countries are presented in alphabetical order and sometimes different kinds of data are provided, because the Nordic countries do not always collect statistics on the linguistic backgrounds of their citizens or students; and when these are collected, they are described in different ways in each of the five countries. For example, in schools in Denmark and Norway, the country where a student was born in is recorded, but not the language(s) that the student speaks. Different terms are used in the different Nordic countries to describe and categorise students and the forms of education that they participate in. In the following presentation of each respective country’s demographics, these country-specific terms are used. The terminology used in this volume (presented in \sectref{sec:reath:3}) has been standardised, to facilitate understanding and improve readability. 

Denmark has a population of 5,873,420 inhabitants (\citealt{Statistics_denmark2023}). No official information is available on the number of languages other than Danish spoken by inhabitants in Denmark since official national statistics focus on national origin rather than language. In national statistics, the categories “immigrants” and “descendants of immigrants” are used. Immigrants and descendants currently constitute 14\% of the total Danish population, and they originate from more than 200 countries across the world. Among immigrants, those from Poland currently constitute the largest subgroup regarding country of origin, followed by those from Syria, Turkey and Germany. Among descendants of immigrants, the largest group is those whose national origin lies in Turkey, followed by Lebanon, Iraq, Pakistan and Somalia. In 2019/20, immigrants and descendants defined as “non-Western” constituted 10\% of all students in primary and lower secondary school. Mother tongue education was introduced in Denmark in the National Curriculum of 1975, but since 2002 municipalities have only been mandated to offer mother tongue education in languages with official recognition within the EU or the European Economic Area, as well as in Faroese and Greenlandic (\citealt{AlisaariEtAl2023}).

Finland has a population of 5,550,066 inhabitants (\citealt{Statistics_finland2023}). Of these, 442,399 speak languages other than Finnish, Sámi and Swedish. Finland is constitutionally a bilingual country (Swedish and Finnish), and Sámi languages have official status in northern Finland (\citealt[55]{AlisaariEtAl2023}). 21,215 students participated in mother tongue education in a total of 57 languages other than the official languages in Finland in 2019 (\citealt{Finnish_national_agency_for_education2022}). These numbers do not represent all the students with migrant backgrounds in Finland, but they give an indication of the growing linguistic diversity of Finnish schools.

Iceland’s population was registered at 368,792 in January 2021 (\citealt{Statistics_iceland2022}). Icelandic is the only official language recognised in Iceland. Over the past 20 years, Iceland has experienced increased levels of immigration transforming what was once a largely monolingual and monocultural school population, to one characterised by ethnic, linguistic and religious diversity. In 1997, 0.8\% of primary school students were registered as speaking a mother tongue other than Icelandic, and by the latter half of 2017, this percentage had increased to 9.9\% or 4,470 primary school students (\citealt{Statistics_iceland2018}). In the autumn of 2020, the percentage increased again, to 12\%, or 5,611 students (\citealt{Statistics_iceland2020-children}). The increase in the number of children speaking languages other than Icelandic as a mother tongue in Icelandic schools has resulted in heightened awareness of the importance of learning Icelandic and gaining skills and confidence in using Icelandic in these children’s education. In Iceland, mother tongue education is only offered in Norwegian, Polish and Swedish, although community-based mother tongue education is widespread (\citealt{PeskovaEtAl2023}).

Norway’s population in 2023 was 5,488,984 (\citealt{Statistics_norway2023}). Norwegian is the official language of Norway, with the indigenous Sámi languages as co-official within certain jurisdictions. People who were born in countries outside of Norway to two foreign-born parents are defined as immigrants in Norway. This group constitutes 14.8\% of the Norwegian population, while children born in Norway to immigrant parents (as per previous definition) represent 3.7\% of the total population. As in Denmark, statistics on languages spoken by individuals are not collected in Norway. Instead, only the countries of origin are recorded. The five countries from which the majority of immigrants originate are Poland, Lithuania, Sweden, Syria and Somalia. For children born in Norway to immigrant parents, the five largest countries of origin are Poland, Lithuania, Somalia, Pakistan and Sweden. There are no official statistics regarding the number of multilingual children in schools. However, it is estimated that one in five students speaks a language other than Norwegian at home (\citealt{Kulbrandstad2020}). Although mother tongue education was officially introduced in Norway in 1987, it has become a transitional provision for students with limited knowledge of Norwegian, and very few students are currently enrolled in mother tongue education in public schools in Norway.

In 2023, Sweden had 10,521,556 inhabitants (\citealt{Statistics_sweden2023}). The official national language is Swedish and the five official national minority languages are Finnish, Meänkieli (Tornedal Finnish), Yiddish, Romany Chib (all varieties) and Sámi (all varieties). In the academic year 2023/24, 28.9\% of students in the compulsory schools used languages other than Swedish at home on a regular basis with at least one caregiver (\citealt{Swedish_national_agency_of_education2024}). These students are thus eligible to study that language through the elective subject of mother tongue education. 187 languages were taught through mother tongue education in 2023\slash 24 (\citealt{Swedish_national_agency_of_education2024}). Although these figures apply to the compulsory school only (grades 1–9), they reflect to some extent the linguistic diversity in broader social contexts in Sweden.

This brief demographic overview reveals that in all the Nordic countries investigated in this volume, there are significant numbers of school-age children who, in addition to the national majority languages, speak other languages. Sweden has the highest proportion, while Iceland has the lowest. In Norway and Finland, indigenous languages have official status in particular regions, while in Sweden they have official status in the whole country. Finland is the only officially bilingual country, with Swedish and Finnish both accorded the status of national official languages.

\section{Terminology in this volume}\label{sec:reath:3}

As \citet[3]{Skutnabb-KangasMcCarthy2008} have pointed out, the concepts researchers use are never neutral:

\begin{quote}
In contested arenas such as bilingual education, words and concepts frame and construct the phenomena under discussion, making some persons and groups invisible; some the unmarked norm, others marked and negative.
\end{quote}

Across the Nordic countries, in different fields of studies and theoretical traditions, the terminology used to describe linguistic diversity in schools varies significantly. To improve coherence in this volume, the editors and contributing authors have discussed and agreed to use specific terms to describe similar educational phenomena that are investigated in this volume. In some cases, it was not obvious which terms were the most suitable and negotiations were needed. We therefore provide the reasoning behind our choices below.   

We have chosen to use the phrase \textit{linguistically diverse classrooms} to describe how classrooms, including teachers and students, are made up of individuals with different linguistic backgrounds. In linguistically diverse classrooms, students speaking national, national minority, indigenous and other minority languages can study alongside newcomers and other students with histories of migration. All of these individuals have different relationships to their languages. The term \textit{multilingualism} is used to describe any individual’s regular use of more than one language. From this perspective, multilingualism also encompasses other concepts, such as bilingualism and plurilingualism. In line with this definition, we describe students as \textit{multilingual} when they speak more than one language in their daily life. Hence, the term \textit{multilingual student} encompasses terminology found in Nordic policy documents, when they, for instance, refer to “other multilingual students” (Finnish: “muut monikieliset oppilaat”) and “linguistic minority students” (Norwegian: “minoritetsspråklege elevar”). The concept \textit{multilingual student} also includes students who have recently moved to and begun school in a Nordic country, who are, when referred to, described more specifically as \textit{newcomers}. The term \textit{newcomer} is found in policy documents in different Nordic countries (Swedish: “nyanlända”, Norwegian: “nykomne”). This is a term that also covers asylum seekers who are enrolled in schools. 

In this volume, a \textit{teacher education programme} refers to the particular programme in focus in the study, for example the Finnish class-teacher education programme for grades 1 through 6 or the Norwegian teacher education programme for grades 5 through 10. A \textit{teacher education subject} refers to the different subjects that pre-service teachers study during their teacher education programme, for example Danish, Pedagogy, and Science. Subjects are often divided into and taught through several \textit{courses}. Throughout this volume, the authors will use \textit{programme}, \textit{subject}, and \textit{course} to describe the specific context under study. These terms correspond respectively to the Nordic equivalents “program”/“fag”/“ämne”, and “emne”/“kurs”/“modul”. Prospective teachers enrolled in a teacher education programme in order to become teachers are described as \textit{pre-service teachers} in this volume, while practising teachers participating in \textit{in-service education} or courses after graduation are called \textit{in-service teachers.} When referring to the period of practice teaching that pre-service teachers do in schools, the term \textit{practicum} is used. 

The word “didaktik” or “didaktiikka” (in Finnish) refers to methods and approaches used in learning and teaching contexts. It is often translated directly as \textit{didactics} in English-medium texts, including the chapters in this volume. As this term is not used widely outside the Northern European context, we have included this brief explanation here to improve comprehension of the chapters and also to add some contextual information on Nordic education.

While the authors have mostly used this terminology consistently throughout the volume, other terms are also sometimes used, to reflect a particular theoretical framework. For example, both \textcite{chapters/6_iversen} and \textcite{chapters/8_heikkola} draw on \citet{García2009} when they use the term \textit{emergent multilingual}.

\section{{Teacher education and linguistic diversity: Earlier studies in the field}}\label{sec:reath:4}

Nordic classrooms are populated by students who use a variety of languages and have varying levels of competency in them. These linguistically diverse classrooms reflect the regions’ multilingual population as well as the fact that newcomers are increasingly often directly placed in mainstream classrooms, rather than being educated in separate introductory programmes. When newcomers are placed directly in mainstream classrooms, they spend most of their school time with other students with different backgrounds and mainstream teachers, which, some argue, can support integration (\citealt{KorpEtAl2019}). However, it has been pointed out in Anglophone contexts, that while EAL (English as an additional language) students are being mainstreamed, EAL pedagogy and practice is not. In other words, few accommodations have been made in teacher education for the presence of linguistic, cultural and other forms of diversity in classrooms (\citealt[3]{Liddicoat2022}). Teacher education in the Nordic region has long been in a similar situation: being largely unprepared for the relatively sudden expansion of linguistic diversity in mainstream classrooms (\citealt{Iversen2020, BrorssonLainio2015}), but this seems to be slowly changing.

The ways in which teacher education prepares educators for working with multilingual students has attracted considerable attention in recent years in countries outside of the Nordic region, see for example \citet{Benholz2017}, \citet{Cochran-smith2015}, \citet{FreemanFreeman2014}, \citet{KaramKibler2024}, \citet{Lucas2011}, \citet{WernickeEtAl2021} and \citet{FoleyEtAl2022}. This reflects the linguistically diverse nature of classrooms that teachers around the world meet in the twenty-first century. Research on teacher education for working in linguistically diverse classrooms and schools builds on the research on learning for multilingual students. In this section, a range of the principal findings in this field is presented and connected to the field of teacher education. This is followed by a brief overview of recent research on teacher education in contexts of linguistic diversity.  

Learning among multilinguals and newcomers is supported when all teachers are able to integrate content and language teaching (\citealt{Nusche2009}), incorporate interaction into classroom activities and scaffold the use of languages that multilingual students already know as resources for learning (\citealt{AxelssonMagnusson2012}). It thus follows that pre-service teachers need knowledge about these aspects, and classroom strategies for implementation. Equally important, however, is the extent to which pre-service and in-service teachers are willing to engage with, reflect on, and challenge their own conceptualisations of key concepts in the field of diversity (\citealt{Scarino2022}). This could include critical discussions on concepts such as \textit{additional languages}, \textit{language and culture in learning and knowing}, \textit{competence}, \textit{diversity} and \textit{difference} (\citealt{Scarino2022}). Education policies play a key role in the provision of knowledge about and opportunities for implementing such critical discussions (\citealt{BurtonEtAl2024,OjhaEtAl2024}).

Pre- and in-service teachers also benefit from being made aware of the importance of collaborations between different kinds of teachers (\citealt{Creese2005,GanuzaHedman2015}) and teachers and administrative instances (\citealt{WedinWessman2017}) to provide the best support for multilinguals in schools. Communication between and mutual understanding of the roles of all language teachers (second language, majority language, foreign/modern language, mother tongue), {s}pecial education teachers, principals and municipal leaders responsible for education and caregivers is crucial for successful implementation of policies and organisation of educational provisions that promote learning among multilingual students (\citealt{Dewilde2013,Warren2017}). Teacher knowledge, gained partly through teacher education, plays a central role in establishing and facilitating these collaborations and in creating educational environments where all multilingual students, including newcomers, can thrive.

Teachers, their students and their schools are embedded in complex sociopolitical contexts (\citealt[289]{Cochran-SmithLytle1999}). Understanding the dynamic and dialogical interplay between this context and the teaching situation, as well as having the ability to act agentively, is increasingly important for teachers. This can be addressed in teacher education by placing emphasis on teaching as an ongoing process of enquiry rather than attaining mastery (\citealt{TooheySmythe2022}). Teachers need to be educated as professionals who not only understand teaching techniques and how to apply them, but who can also take action informed by theory and understanding of the sociocultural context, in their classrooms and schools (\citealt{Leung2022,VillegasEtAl2018}). In other words, if teacher education aims to prepare teachers for work in linguistically and culturally diverse classrooms, it needs to emphasise and critically examine the situatedness of learning (\citealt{Liddicoat2022,OjhaEtAl2024}).

There is increasing awareness among teacher educators across contexts that teacher education needs to prepare pre-service teachers to engage with the linguistic diversity present in the classroom in ways that go beyond “best practices” for teaching the language of instruction (\citealt{OjhaEtAl2024}). One aspect of this is nurturing interest in and a positive attitude towards linguistic diversity among pre-service teachers. Some recent studies have found that short courses spanning only a semester or two can contribute to changing negative beliefs about multilingualism (\citealt{Aleksic2023,Mahalingappa2024,SchroedlerEtAl2023}). However, other researchers stress that bringing sustained change, including changes in beliefs and teaching practices, requires time and effort (\citealt{Duarte2022,KirschEtAl2020}). Furthermore, when teacher education prepares teachers to work in linguistically diverse classrooms, the content of the coursework matters. \citet{DollGuldenschuh2023} first found that a teacher education course on multilingualism in education did not significantly alter the participants’ negative perspectives on multilingualism. However, after the course content had been revised to focus more on pedagogies of migration and different models of multilingual education, rather than second language acquisition and educational disadvantages, the course was successful in changing the participants’ negative perspectives. As well as the importance of content, this also indicates that critical evaluation and revision of teacher education courses that include students’ perspectives can make a difference to the impact the courses have.

\citet{WernickeEtAl2021} present findings from an international research project, MultiTEd, describing and comparing different approaches to preparing pre\hyp service teachers for linguistically diverse contexts. The volume includes contributions from several countries in North America and Europe describing current approaches regarding linguistic diversity in teacher education, as well as promising initiatives to better prepare pre-service teachers to work in linguistically diverse classrooms. The contributions also highlight how sociopolitical circumstances, language policies and institutional and programme preferences in a range of contexts play an important role and make the development of teacher education complex. There are two contributions from the Nordic context in that volume. \citet{PaulsrudLundberg2021} analysed a selection of primary school teacher education programmes in Sweden and found that there is no standardised or national requirement to provide courses on multilingualism in teacher education at this level. They argue that there are only “vague spaces for multilingualism” \citep[55]{PaulsrudLundberg2021} in the programmes, the responsibility being implicitly passed on to pre- and in-service teachers. \citet{Szabo2021} examined teacher education at two universities in Finland and found that teacher education was slow to respond to societal changes, including the growing proportion of multilingual students in schools.

Another study examining attitudes and beliefs about multilingualism in Swedish educational contexts shows that teacher educators and in-service teachers alike regard Swedish as the only legitimate language for learning in teacher education programmes and at school, indicating a monolingual mindset (\citealt{PaulsrudEtAl2023}). Both pre- and in-service teachers in this study expressed concern that they were unprepared for work in linguistically diverse classrooms, leading the authors to conclude that teacher education in Sweden has not caught up with the multilingual reality of the Swedish school today.

The extent to which pre-service teachers are given the opportunity to reflect on and develop a teacher identity that is consonant with the linguistically diverse classrooms and schools in which they will be working is another salient factor for teacher education. A recent survey and interview study in England investigated pre-service teachers’ perceptions of their teacher education programme, their own experiences of teaching students from a range of linguistic backgrounds, and the challenges faced by all students in developing language and literacy skills (\citealt{FoleyEtAl2022}). The pre-service teachers’ responses indicated both commitment and a will to take responsibility for the learning needs of students with English as an additional language. In spite of this, pre-service teachers nearing the end of their teacher education programme also reported that they had acquired few strategies for working with such students. Moreover, approximately 20\% had “little confidence” in their ability to support students with diverse linguistic backgrounds (\citealt[117]{FoleyEtAl2022}). The authors recommend both “core sessions” (\citealt[120]{FoleyEtAl2022}) on linguistic diversity in teacher education programmes, as well as {i}nput on linguistic diversity infused in all other subject areas. The latter is a measure to place emphasis on supporting pre-service teachers “to make all lessons more accessible to, and inclusive of, the multilingual and multicultural pupils whom they will encounter” (\citealt[120]{FoleyEtAl2022}).

\begin{sloppypar}
Teachers are often expected and accustomed to calibrating their approaches to teaching, to meet the needs of academically diverse learners (\citealt{Cochran-SmithEtAl2015}). However, adapting their teaching repertoire in order to nurture the development of subject and language knowledge among newcomers and other multilingual students who study in the same classroom as students who have spent their entire school career in the same country, accumulating everyday language and scientific language and knowledge in the majority language of that country, presents different and specific challenges (\citealt{WernickeEtAl2021}). For newcomers in secondary schools especially, the pressure of learning the meaning of complex subject-specific words and concepts and how to embed them in longer texts to produce knowledge that meets the requirements of the national curricula, can be a stressful and exhausting challenge (\citealt{Sharif2017}).
\end{sloppypar}

This brief overview of the international and Nordic research reveals that a wide range of factors are important to consider when preparing pre-service teachers for working in linguistically diverse classrooms. These include incorporating knowledge about multilingual development and how to integrate content and language learning, particularly for newcomers, in all teacher education programmes. In addition, critical reflection on the terms that are used to describe and discuss linguistic diversity should be encouraged among pre-service teachers, and the importance of collaborating within school ecologies emphasised. Nurturing awareness of the sociocultural context in which classrooms are situated and a positive attitude towards linguistic diversity are important. Finally, supporting pre-service teachers as they gain a deeper understanding of the nature of teacher identity, and learn that teaching knowledge evolves over time, through experience rather than through the use of a specific set of tools “delivered” by teacher educators, can prepare them for their future work in linguistically diverse contexts. The research indicates that such knowledge is present in different forms of teacher education, but is organised differently. Core courses in addition to strands addressing multilingualism across teacher education as well as evaluation and revision of the content addressing linguistic diversity are all beneficial for meeting the needs that pre-service teachers themselves identify as important. 

In this volume, we focus specifically on research on teacher education in the Nordic region, which has traditionally been associated with progressive, centralised approaches to education, based on principles of inclusion, social justice and equal opportunity (\citealt{BlossingEtAl2014,Elstad2020,Frones2020}). We see a need to examine if these progressive approaches include preparing teachers for working with multilingual students, given that Nordic school populations have become increasingly multilingual. To our knowledge, research on teacher education in a Nordic context has not yet been collated in a single volume, although there are a number of researchers working in this field. This volume therefore is the first to present research on teacher education for linguistic diversity in a Nordic context to an international audience. Gathering critical studies of teacher education programmes from one region in a single volume also facilitates broader patterns to be observed. This can provide researchers and teacher educators in the Nordic region with knowledge that helps them to analyse and improve teacher education in local contexts. Teacher educators in countries outside the Nordic region may also benefit from reading about broader tendencies in a particular geographical region, to identify and address research gaps and work with development in their own region.

\section{Organisation of teacher education in the Nordic countries}\label{sec:reath:5}

While there is a long-standing tradition of collaboration and mutual inspiration between teacher educators and researchers in teacher education in the Nordic countries (\citealt{Elstad2020,HadzialicEtAl2017}), there is considerable variation in how teacher education is organised (\citealt{Elstad2020}). Historically, teacher education was not conducted within universities. Rather, pre-service teachers for primary and lower secondary schools were educated in separate teacher colleges. The exception is Finland, where teacher colleges were integrated into the universities in 1979. Since then, teacher education has been organised as an integrated five-year master’s degree (\citealt{Sahlberg2015}). For the past two decades, the now internationally acclaimed Finnish teacher education system (e.g., \citealt{Darling-HammondEtAl2017,Tatto2015}) has served as an inspiration for teacher education in the other Nordic countries. Finland’s results on international standardised tests, such as PISA, TIMS, and PIRLS (\citealt{Ahonen2021}), have motivated politicians across the Nordic countries to model their respective teacher education systems on the Finnish model (\citealt{Elstad2020}). Iceland integrated teacher education into their university system in 2008 (\citealt{Sigursson2020}), and Norway and Sweden started similar processes in 2005 and 2011, respectively (\citealt{Astrand2020,SkagenElstad2020}). Consequently, Denmark is the only Nordic country where the traditional teacher colleges still dominate and where pre-service teachers are not expected to complete a master’s degree (\citealt{MadsenJensen2020}). Rather, Danish teacher education comprises a four-year programme at a teacher’s college. In addition to the initial pre-service teacher education programmes, all Nordic countries provide comprehensive in-service education for in-service teachers.

Teacher education is organised differently across the Nordic region. Some countries have programmes aimed at educating teachers for specific levels of the education system (primary or secondary), while other countries have opted for more general teacher education programmes (\citealt{Elstad2020}). Danish pre-service teachers, for example, specialise in particular school subjects, but have differentiated courses depending on the level at which the subject (e.g., Danish, Mathematics, Physical Education and English) will be taught (\citealt{MadsenJensen2020}). A mandatory course called “Teaching bilingual students” in such a general teacher education programme in Denmark is in focus in \chapref{chap:5}. In Finland, on the other hand, so-called class teachers are educated to teach all subjects from grade 1 through grade 6, while subject teachers, who teach grade 7 through grade 12, specialise in particular school subjects (\citealt{Hansen2020}). Teacher education for subject teachers in Finland comprises a one-year programme, which is studied after subject-specific studies at Master's level have been completed. The first kind of programme (for pre-service teachers, grades 1 through 6) is reported on in \chapref{chap:7} and the second kind of programme (for pre-service subject teachers, grades 7 through 12) in Finland is in focus in \chapref{chap:8}. As in Denmark, Icelandic teacher education is not specialised according to the age group the pre-service teachers are expected to teach, but they have the opportunity to specialise in particular school subjects (\citealt{Sigursson2020}). Studies investigating two general teacher education programmes in Iceland are presented in Chapters~\ref{chap:2} and~\ref{chap:4}. In Norwegian teacher education for compulsory school, pre-service teachers follow either a programme preparing them for grades 1 through 7 or for grades 5 through 10. These two different programmes have nonetheless much in common, and in \chapref{chap:7}, researchers from the Norwegian educational context present findings derived from data collected from both. Swedish teacher education institutions are largely free to select and organise the content of their respective teacher education programmes independently. Hence, Swedish teacher education is characterised by a diversity of approaches (\citealt{Astrand2020}). In-service education for teachers is widespread throughout the Nordic countries and often delegated to teacher educators, through centrally funded initiatives or centres for in-service education. Policies and practices in study guidance in the mother tongue in the Swedish context are analysed in \chapref{chap:3}. There is no compulsory educational programme for the tutors who conduct study guidance, rather they usually learn about their profession through in-service education or on the job.

In some of the Nordic countries, education policies have resulted in subjects and forms of support available in schools for multilinguals that are relatively unique internationally. These include the elective subject of mother tongue education, for students who speak languages in addition to the majority language at home, and the right to temporary support in the form of tutoring in their mother tongue or strongest school language, for students whose knowledge of the majority language is deemed insufficient for them to pass core subjects. These subjects and forms of support are organised differently in each of the Nordic countries. Not all pre-service and in-service teachers have relevant knowledge about the subjects and forms of multilingual support (\citealt{BrorssonLainio2015}). Without a deep understanding of not only the needs but also the strengths of their multilingual students, teachers are less prepared to help them reach their full learning potential. Teacher education is thus crucial for successful implementation of educational provisions for multilingual students.

\section{{The contributions}}\label{sec:reath:6}

This volume significantly contributes to our understanding of teacher education in different Nordic contexts, presenting a wide range of cases, all of which in some way address how teacher education prepares pre- and in-service teachers for working in linguistically diverse schools and classrooms.  Following this introductory chapter, the subsequent two chapters of this volume focus on policies regulating teacher education, the syllabi defining what pre-service teachers are expected to learn, and in-service teachers’ perspectives on teacher education’s effort to prepare teachers for working in linguistically diverse schools. 

In \chapref{chap:2}, Edda Óskarsdóttir and Hermína Gunnþórsdóttir present their analysis of course descriptions from the two main teacher education institutions in Iceland, based on focus group interviews with in-service teachers who have recently graduated from each respective teacher education programme. Their objective is to investigate how Icelandic teacher education prepares pre-service teachers for working in linguistically diverse classrooms. The study highlights the in-service teachers’ concern with the lack of practical pedagogy to support multilingual students in Icelandic teacher education, reflecting perspectives expressed by pre-service teachers in other chapters of this volume (e.g., \cite{chapters/6_iversen, chapters/5_ostergaard}).

In \chapref{chap:3}, Jenny Rosén and Åsa Wedin  apply nexus analysis to explore how national policies on tutoring in the mother tongue (in Swedish, called “study guidance in the mother tongue”, hereafter SGMT) are translated into course syllabi at a Swedish university and how these are interpreted by one in-service SGMT assistant. Rosén and Wedin also recorded classroom interactions to investigate how national policies are enacted in particular classroom settings. They find that the SGMT assistant does, in fact, fulfil the various roles expected of him, according to key policy documents. In spite of this, his extensive competence is not recognised by the school.

\begin{sloppypar}
The only contribution that presents the experiences, beliefs and knowledge of teacher educators is from Iceland again. \chapref{chap:4} acts as a bridge between the previous two chapters (principally addressing policy, syllabi and in\hyp service teacher perspectives) and the final three (principally addressing pre\hyp service teachers’ beliefs and knowledge). In \chapref{chap:4}, Hafdís Guðjónsdóttir, Jónína Vala Kristinsdóttir, Gunnhildur Óskarsdóttir and Samúel Lefever analyse the experiences of Icelandic teacher educators as they prepare pre-service teachers for working in linguistically diverse classrooms. They present findings from focus group interviews with teacher educators at the largest teacher education institution in Iceland. \citeauthor{chapters/4_gudjonsdottir} find that although teacher educators do not always emphasise multicultural education in their teaching, they nonetheless attempt to model effective teaching practices that are learner\hyp centred and inclusive.  
\end{sloppypar}

Pre-service teachers’ beliefs and knowledge about, as well as skills for, teaching in linguistically diverse classrooms are investigated in Chapters~\ref{chap:5}--\ref{chap:8}. In \chapref{chap:5}, Winnie Østergaard, Anna-Vera Meidell Sigsgaard, Christine Worm, Lone Wulff, Thomas Roed Heiden and Anne-Louise Markussen analyse Danish pre-service teachers’ examination papers from a mandatory course on teaching bilingual students. \citeauthor{chapters/5_ostergaard} explore how pre-service teachers conceptualise multilingualism after completing the mandatory course. Although there are obvious advantages associated with a mandatory course on teaching bilingual students, \citeauthor{chapters/5_ostergaard} conclude that as long as this is the only course in teacher education where multilingualism is positioned as a resource, it will be ineffective in altering pre-service teachers’ preconceptions of multilingual students. In addition, the study identifies limited ability among the pre-service teachers to connect theoretical knowledge to pedagogical practice.   

In \chapref{chap:6}, Jonas Yassin Iversen, Wenche Elisabeth Thomassen and Sandra Fylkesnes analyse interviews with 106 Norwegian pre-service teachers from three different studies to explore their orientations, knowledge and skills for teaching multilingual and emergent multilingual students. \citeauthor{chapters/6_iversen} introduce Lucas and Villegas’ framework for linguistically responsive teaching, and use this as a theoretical framework for their own analyses. They find that although the pre-service teachers articulate adequate orientations towards multilingualism, they are nevertheless unable to demonstrate the necessary knowledge and skills to enact linguistically responsive teaching. Based on their findings, the authors identify key areas where Norwegian teacher education needs to improve in order to prepare pre-service teachers to enact linguistically responsive teaching.

Chapters~\ref{chap:7} and~\ref{chap:8} shift the focus to Finnish teacher education in primary schools and upper secondary schools respectively. In \chapref{chap:7}, Jenni Alisaari, Leena Maria Heikkola and Raisa Harju-Autti investigate Finnish third-year, pre-service primary school teachers’ preparedness to teach linguistically diverse students, at the beginning of a course on multilingual pedagogies. The Finnish teacher education programme for primary school teachers prepares pre-service teachers to teach in grades 1 through 6. \citeauthor{chapters/7_alisaari} find that the pre-service teachers demonstrate an adequate understanding of language learning, and reports skills to support academic language development. However, they have limited awareness of how to provide linguistic support for multilingual students. The authors argue that all teachers need to be able to identify language demands involved in different learning tasks. Moreover, they advocate for a focus both on pre-service teachers’ ability to support the acquisition of the language of instruction, as well as subject-specific content.

In \chapref{chap:8}, Leena Maria Heikkola, Elisa Repo and Niina Kekki analyse Finnish pre-service subject teachers’ preparedness to support multilingual students’ language learning in Finnish schools. The Finnish teacher education for subject teachers prepares pre-service teachers for teaching grade 7 to grade 12. In line with \citeauthor{chapters/7_alisaari}, this chapter also explores pre-service teachers’ knowledge about academic language development. \citeauthor{chapters/8_heikkola} follow the pre-service teachers’ development through a one-year teacher education course. They find that the pre-service teachers were able to identify the language demands of academic tasks, although their awareness of language-related practices was limited. Based on their analyses, \citeauthor{chapters/8_heikkola} discuss the opportunities and limitations of this one-year course, and highlight that the pre-service teachers’ previous educational background influences how they develop their preparedness to support multilingual learners. 

In the Epilogue, Ingrid Piller offers critical and thought-provoking observations on the many contradictions found in Nordic education. She notes that the Nordic countries are often seen as global beacons of modernity, social inclusion, and equal opportunities. At the same time, students with a migrant background in the Nordic countries underachieve in schools – even after controlling for socio-economic and language status. Piller agrees with the authors in this volume that the gap between inclusive policies and their operationalisation in teacher education documented in this volume may be part of the explanation for why Nordic countries continue to struggle to support students with a migrant background. What readers beyond the Nordic countries can learn from the chapters in this volume, Piller argues, is that changes at the policy level are insufficient to challenge the monolingual habitus of multilingual schools. Consequently, it is necessary to disentangle migration and linguistic diversity. Linguistic diversity is, indeed, a fact of life and should be treated accordingly in schools. Finally, Pillers provides three challenges for future language education. 

\section{Common themes and future directions}\label{sec:reath:7}

While compiling the volume it became clear that although the aspects of teacher education programmes addressing linguistic heterogeneity in the five Nordic countries represented in this volume differ, there are several common threads connecting the contributions. The editors have grouped together these threads under three themes. 

The first of these themes regards the integration of multilingual perspectives in teacher education, specifically, whether to incorporate one core subject focusing on theories and approaches to teaching in linguistically diverse classrooms, such as the Danish model (see \citealt{chapters/5_ostergaard}), or include a strand that addresses these issues across all subjects. While a core course ensures that all pre-service teachers have the opportunity to learn about linguistic diversity and education, the results reported by \textcite{chapters/5_ostergaard} imply that it cannot be taken for granted that this results in the acquisition of in-depth theoretical or pedagogical knowledge. None of the chapters in this volume present an alternative model. However, the potential advantages of a “multilingual strand” running through all courses include that pre-service teachers could learn subject-specific multilingual approaches, and also that they would be regularly reminded about the importance of applying multilingual perspectives in teaching. Another alternative is to have both a core course on linguistic diversity and input on linguistic diversity being infused in all other subject areas, as recommended by \citet{Foley2022}. As the results of the chapters in this volume indicate that many pre-service teachers \parencite{chapters/7_alisaari, chapters/8_heikkola, chapters/6_iversen} and some in-service teachers (\citealt{chapters/2_Gunnthorsdottira}) feel unprepared to work practically with multilingual students, an approach combining theoretical knowledge in a core subject with practical application in all subjects may also have potential in a Nordic context. Regardless of model, it is necessary to bridge the existing disciplinary divides within teacher education to prepare teachers for linguistically diverse classrooms (\citealt{Bale2024,BurtonEtAl2024}).

\begin{sloppypar}
The second theme concerns which groups of pre- and in-service teachers should be educated about working with linguistically heterogeneous groups. There are tensions in schools concerning which teachers should be responsible for multilingual students’ language development and thus their learning and success in different school subjects, and a clear tendency to delegate this responsibility to language teachers (\citealt{HermanssonEtAl2022}). Moreover, the findings presented by \citet{chapters/8_heikkola} suggest that language teachers are those whose knowledge about multilingualism appeared to increase the most after teacher education on that topic. Previous studies, however, show that language is an intrinsic aspect of learning and producing knowledge in \textit{every} subject (\citealt{Cummins2000,Gibbons2014}). This indicates the importance of \textit{all} pre-service teachers being educated about linguistic diversity and approaches for teaching that promote the development of both subject knowledge and the language used to express that knowledge. Implementation of this kind of approach in teacher education could be facilitated by communication and collaboration between teacher educators in all subjects and disciplines. In schools, collaboration between {d}ifferent categories of educational workers could support ongoing language and knowledge development for multilingual students, as has been pointed out in earlier research (\citealt{Creese2005,Dewilde2013,Warren2017,WedinWessman2017}).
\end{sloppypar}

The third theme concerns the divide between theory and practice, which is salient on different scales. In the presented cases, see \citet{chapters/8_heikkola, chapters/6_iversen, chapters/2_Gunnthorsdottira, chapters/3_rosen}, many pre-service teachers report having acquired theoretical knowledge about multilingual development but lack preparation for applying this to classroom practice. It is positive that many pre-service teachers report having theoretical knowledge about multilingualism, and moreover, often frame multilingualism as something beneficial, as theoretical knowledge is necessary for developing classroom practice. However, the results in these collected studies also indicate that the capacity of teachers to take action informed by theory and understanding of the sociocultural context, in their classrooms and schools (\citealt{Leung2022}), is still limited in the Nordic context. With the exception of Denmark, teacher education in all the Nordic countries has moved from colleges to universities, to ensure that pre-service teacher knowledge is theoretically as well as practically grounded. As teacher education evolves along with the societies in which it operates, it is important that theoretical knowledge about relevant developments in social and educational theory is not only presented and discussed, but also translated by teacher educators, into practical educational approaches and activities. Presentation and discussion of, as well as practice with approaches and activities that aim to explicitly scaffold the development of language and content knowledge among multilingual students can provide pre-service and in-service teachers with the tools they need to operationalise theoretical knowledge into their teaching practice in linguistically diverse classrooms (\citealt{Gibbons2014,Iversen2020}). On another scale, \citet{chapters/3_rosen} illustrate that there is a gap, if not a gulf, between how forms of educational support for multilingual students are described in steering documents and how they are enacted in school contexts. The findings in this chapter echo previous research in the Swedish context (\citealt{Warren2017}) and illustrate that knowledge about the different forms of support that are available for multilingual students are unlikely to support learning until teachers have knowledge about them, and actively collaborate. \citegen{chapters/3_rosen} findings provide a compelling argument for the inclusion of knowledge about all forms of education and support for multilingual students in teacher education programmes.

By learning from the experiences from colleagues and researchers in a range of settings in the Nordic region, local policies and pedagogies that meet the needs of both teacher educators and pre-service teachers in other contexts can be discussed and developed. In future research, we see a need for more fine-grained analyses of the content of teacher education programmes, with a focus on the aspects that help teachers promote learning and inclusion in linguistically diverse classrooms. There is also a need to investigate how these aspects are best incorporated into teacher education programmes. The research presented in this volume indicates that teacher educators and researchers need to engage in questions about how to prepare pre-service teachers to \textit{enact} a multilingual pedagogy once they transition into teaching. 

Finally, the contributions indicate an ongoing and pressing need for teacher educators to be vigilant and engaged in their local and national social and educational contexts in order to be able to respond flexibly to the changing nature of schools in their region. Teacher educators in the Nordic region cannot rely on past political and educational achievements, nor delegate responsibility for multilingual learners’ needs to language teachers alone. Critical self-reflection on teacher education that also encourages dynamic, contextually-based approaches to policy enactment is crucial for ensuring that Nordic teachers are prepared to work with linguistically heterogeneous groups of students and help them thrive at school and in the region. This contribution is significant for other national contexts as well; politicians and policy-makers come and go, adjusting approaches to education, and social attitudes shift and sway, so teacher educators must remain alert, responsive and prepared to meet the needs that teachers face in changing social contexts.

\section*{Acknowledgments}

The contributions in this volume have been peer-reviewed (double-blind). We are very grateful to all the authors, the participants in each of the studies, and the reviewers in both Nordic and international contexts who raised questions and suggested clarifications that have made each chapter and the volume as a whole more coherent and accessible to a wider audience. We hope that the contributions in this volume contribute to a broader and deeper understanding of teacher education in this context, and to preparing pre-service teachers for their future work in linguistically diverse classrooms.

The editors would like to dedicate this volume to the memory of our Icelandic colleague, Gunnhildur Óskarsdóttir,\footnote{\url{https://www.mbl.is/frettir/innlent/2023/03/18/andlat_gunnhildur_oskarsdottir/}} who sadly passed away before this volume was published.

\printbibliography[heading=subbibliography,notkeyword=this]
\end{document}
