\documentclass[output=paper]{langscibook}
\ChapterDOI{10.5281/zenodo.15280907}
\author{Leena Maria Heikkola\orcid{}\affiliation{UiT The Arctic University of Norway} and Elisa Repo\orcid{}\affiliation{University of Jyväskylä} and Niina Kekki\orcid{}\affiliation{University of Turku}}
\title[Mapping teachers’ preparedness for linguistically diverse classrooms]{Mapping pre-service subject teachers’ preparedness for linguistically diverse classrooms in Finland}
\abstract{This study investigates pre-service teachers’ knowledge of academic language demands, and their awareness of language-related practices. Data were collected via an online survey from three pre-service subject-teacher groups (studying to teach students aged 13–18) at the beginning and end of a year-long teacher education programme. Based on content analysis of the responses, all three groups were able to identify the language demands of academic tasks. However, awareness of language-related practices was vague in all three groups. Although the curricula for basic and upper secondary education in Finland emphasise the role of language in all learning, our study found that current teacher education programme may not be sufficient to prepare pre-service subject teachers for the multilingual realities of today’s schools. The results deepen our understanding of the gradual process of developing teachers’ preparedness for linguistically diverse classrooms, and indicate that linguistically responsive teaching practices should be modelled to support learning and multilingualism in the classroom, in Finland and globally.}
\IfFileExists{../localcommands.tex}{
  \addbibresource{../localbibliography.bib}
  \usepackage{langsci-optional}
\usepackage{langsci-gb4e}
\usepackage{langsci-lgr}

\usepackage{listings}
\lstset{basicstyle=\ttfamily,tabsize=2,breaklines=true}

%added by author
% \usepackage{tipa}
\usepackage{multirow}
\graphicspath{{figures/}}
\usepackage{langsci-branding}

  
\newcommand{\sent}{\enumsentence}
\newcommand{\sents}{\eenumsentence}
\let\citeasnoun\citet

\renewcommand{\lsCoverTitleFont}[1]{\sffamily\addfontfeatures{Scale=MatchUppercase}\fontsize{44pt}{16mm}\selectfont #1}
   
  %% hyphenation points for line breaks
%% Normally, automatic hyphenation in LaTeX is very good
%% If a word is mis-hyphenated, add it to this file
%%
%% add information to TeX file before \begin{document} with:
%% %% hyphenation points for line breaks
%% Normally, automatic hyphenation in LaTeX is very good
%% If a word is mis-hyphenated, add it to this file
%%
%% add information to TeX file before \begin{document} with:
%% %% hyphenation points for line breaks
%% Normally, automatic hyphenation in LaTeX is very good
%% If a word is mis-hyphenated, add it to this file
%%
%% add information to TeX file before \begin{document} with:
%% \include{localhyphenation}
\hyphenation{
affri-ca-te
affri-ca-tes
an-no-tated
com-ple-ments
com-po-si-tio-na-li-ty
non-com-po-si-tio-na-li-ty
Gon-zá-lez
out-side
Ri-chárd
se-man-tics
STREU-SLE
Tie-de-mann
}
\hyphenation{
affri-ca-te
affri-ca-tes
an-no-tated
com-ple-ments
com-po-si-tio-na-li-ty
non-com-po-si-tio-na-li-ty
Gon-zá-lez
out-side
Ri-chárd
se-man-tics
STREU-SLE
Tie-de-mann
}
\hyphenation{
affri-ca-te
affri-ca-tes
an-no-tated
com-ple-ments
com-po-si-tio-na-li-ty
non-com-po-si-tio-na-li-ty
Gon-zá-lez
out-side
Ri-chárd
se-man-tics
STREU-SLE
Tie-de-mann
}
  \togglepaper[1]%%chapternumber
}{}

\begin{document}
\maketitle 
\label{chap:8}
%\shorttitlerunninghead{}%%use this for an abridged title in the page headers




\section{Introduction}\label{sec:heikkola:1}

This study focuses on pre-service teachers’ preparedness for multilingual learners. Specifically, it examines their developing orientations, knowledge and skills in instructing culturally and linguistically diverse students, which is referred to as linguistically responsive teaching (henceforth LRT, \citealt{LucasVillegas2013}). LRT supports learning content and language with a special focus on students from diverse linguistic and cultural backgrounds (\citealt{LucasVillegas2013}, see also \sectref{sec:heikkola:3}). The study is part of a longitudinal project examining the effects of teacher education on teachers’ beliefs and practices regarding working with emerging multilingual Finnish language learners (henceforth EMLLs). In this study, EMLLs refer to migrant background students studying in a Finnish-medium school in Finland. The ultimate goal for the present study is to propose recommendations for how teacher education in Finland, as well as globally could be developed to better prepare future subject teachers to respond to increasing linguistic (and cultural) diversity. In particular, we focus on the language demands of academic tasks and language-related practices that operationalise scaffolding in the learner’s zone of proximal development (henceforth ZPD, \citealt{Vygotsky1978}), that is, how students can reach a higher level of knowledge and skills with the support of a more skilful instructor or in interaction with their peers (see further in \sectref{sec:heikkola:3}).

Teacher professionalisation should include \textit{pedagogical language knowledge}: knowledge of language that is “directly related to disciplinary teaching and learning and situated in the particular (and multiple) contexts in which teaching and learning take place” (\citealt{Bunch2013}: 307). In this study, teachers’ knowledge is understood as “cognitive dispositions that are functionally responsive to situations and demands in certain contexts” (\citealt{Kaiser2019-1}: 32; see also \citealt{KliemeEtAl2008}). We analyse how pre-service subject teachers’ knowledge of academic language demands, and awareness of language-related practices develop during a year-long teacher education programme. Thus, the study aims to contribute to the field of teachers’ professionalisation with regard to linguistic diversity.

The following research questions guided our investigation: 

\begin{enumerate}
\sloppy
\item How does pre-service subject teachers’ knowledge of academic language demands develop during a year-long teacher education programme? 
\item How does pre-service subject teachers’ awareness of language-related scaffolding practices develop during a year-long teacher education programme? 
\end{enumerate}

This study was motivated by the latest PISA assessments, which suggest a significant gap between the learning outcomes of native Finnish speakers and migrant-background students (\citealt{Harju-LuukkainenEtAl2015,LeinoEtAl2019,VettenrantaEtAl2016}). Similar trends can be seen for other OECD countries as well (\citealt{Schleicher2019}). In addition, the study is positioned in the context of curricular reforms in Finland that require schools to promote cultural diversity, language awareness and multilingual approaches as key values (\citealt{EDUFI2014, EDUFI2019}). In today’s Finnish schools, \textit{all} teachers are language teachers within their subjects, and language development and the attainment of the literacy needed for successful academic participation is central to all instruction. Thus, every teacher in Finland is required to be linguistically responsive (\citealt{LucasVillegas2013}), that is, they need to understand the processes of language learning and ways to scaffold students’ learning of language and content in their ZPD. This knowledge is important as it affects how teachers teach multilingual students (\citealt{LucasVillegas2013}).

However, studies have shown that not all teachers in Finland have the pedagogical orientations, knowledge and skills needed to respond to increasing linguistic diversity (\citealt{AlisaariEtAl2019,Repo2020,SuuriniemiSatokangas2023}). A similar situation has been reported in other Nordic countries (see \citealt{Iversen2021,Lundberg2019,Rosen2019-1}). Although language awareness could revolutionise schools and classrooms, questions have been raised about how the concept has been translated into linguistically responsive practices (\citealt{AhlholmEtAl2021,ZilliacusEtAl2017}). In Finland, for instance, teachers’ classroom practices sometimes reflect a persistent “Finnish only policy” (\citealt{AlisaariEtAl2019}). As the current curricula are relatively new, it is to be expected that teachers’ practices are still developing. Indeed, changes in school culture and discourses are slow (\citealt{Repo2020,TarnanenPalviainen2018}), and often change does not happen when ordered from above (\citealt{HornbergerJohnson2007}).

As there is evidence that LRT is crucial for newcomers to a school system whose first language is not the same as the language of instruction (\citealt{Gibbons2014,LucasVillegas2013,Schleppegrell2002}), it is important to study pre-service subject teachers’ preparedness to support EMLLs and how their preparedness is supported in teacher education. There is evidence that training focused on cultural diversity enable teachers to affirm students’ identities (\citealt{KimanenEtAl2019}), and professional development increases early childhood practitioners’ awareness of language learning (\citealt{Kirsch2018-1}). The present study adopted the view that pre-service subject teachers’ preparedness for linguistically diverse classrooms consists of their abilities to identify potentially challenging features of academic language and their awareness of language-related and interactional scaffolding practices. However, we are aware that being linguistically responsive also requires other types of knowledge and skills.

The chapter is structured as follows. In \sectref{sec:heikkola:2}, we present the context of the study. The theoretical background of the study is described in \sectref{sec:heikkola:3}, with subsections on academic language demands (\sectref{sec:heikkola:3.1}) and language\hyp related scaffolding (\sectref{sec:heikkola:3.2}). In \sectref{sec:heikkola:4}, the methodology of the study is explained, with separate subsections focusing on participants and data collection (\sectref{sec:heikkola:4.1}) and data analysis (\sectref{sec:heikkola:4.2}). Results are presented in \sectref{sec:heikkola:5}, with subsections focusing on academic language demands (\sectref{sec:heikkola:5.1}) and language\hyp related practices (\sectref{sec:heikkola:5.2}). Finally, in \sectref{sec:heikkola:6}, the results of the study are discussed, and conclusions drawn.

\section{The context of the study: One-year subject teacher education programme}\label{sec:heikkola:2}

In Finland, subject teachers teach grades 7–9 (students aged 13–15) in basic education and grades 10–12 (students aged 15–18) in upper secondary school. In contrast to primary school teachers, who usually teach all subjects to students aged 7–12 (see \citealt{chapters/7_alisaari}), subject teachers typically teach two different subjects. All subject teachers are required to obtain a three-year bachelor’s degree and a two-year master’s degree (five years total). This includes 120 ECTS (= European Credit Transfer and Accumulation System) in their major, the primary teaching subject, and 60 ECTS in their minor, the other teaching subject(s). One ECTS equals approximately 27h of work. In addition, subject teachers are required to complete pedagogical studies (60 ECTS) lasting one academic year either as a part of their bachelor’s or master’s degree or as additional studies.

The one-year teacher education programme consists of theoretical and subject-specific didactic studies, as well as practicums. Currently, there are eight universities that educate teachers in Finland. Finnish teacher education is research-based and highly valued. Even though teacher education is guided by the national core curricula, teacher education departments at universities create their curricula autonomously; thus, the curricula diverge quite significantly between the universities (\citealt{Szabo2021}). Furthermore, subject teacher education in Finland is organised according to the subjects, and there can be substantial differences in how broadly various topics, such as LRT, are discussed across diverse subjects.

At the teacher education department from which the participants in this study were recruited, the curriculum for subject teachers’ pedagogical studies includes one course (5 ECTS) on subject\hyp specific didactical skills that covers topics such as language-aware teaching, awareness of subject\hyp specific language features, challenges of teaching subject-specific terminology, and significance of language and culture in learning. However, the course covers many other topics as well; thus, the time allocated to language awareness is brief.

In the curriculum of the investigated teacher education department, one of the four practicums is dedicated to linguistically aware teaching taking multilingual students into consideration in teaching a subject. According to the curriculum, all pre-service subject teachers shall become aware of the role of language in learning and teaching during their year of pedagogical studies. This study investigates pre-service teachers’ awareness of the demands of academic language and their language-related practices for scaffolding instruction; both topics are addressed in the teacher education programme at the university in question.     

\section{Theoretical background}\label{sec:heikkola:3}

This study is based on \posscitet{LucasVillegas2011,Lucas2013} LRT framework, which draws on a sociocultural approach to learning and teaching language(s) (\citealt{Donato2000,Gibbons2014,LantolfThorne2006,Lier2000,Vygotsky1978}). Viewed through a sociocultural lens, language learning is a social and cognitive process (\citealt{LantolfThorne2006,Vygotsky1978}). Language skills develop through repetitive interaction, collaborating (\citealt{Dufva2020,Vygotsky1978}) and sharing and recycling language or linguistic resources (\citealt{Dufva2013}). The LRT framework consists of \textit{orientations} (beliefs and values regarding language and linguistic diversity) and \textit{pedagogical knowledge and skills} (\citealt{chapters/7_alisaari}; see \citealt{chapters/6_iversen}). In this study, these are assumed to be part of teachers’ professionalism and cognitive competencies (\citealt{Kaiser2019-1}) to teach EMLLs, in line with \citegen{Bunch2013} call for pedagogical language knowledge. Below, we present selected studies on identifying academic language demands (\sectref{sec:heikkola:3.1}) and language-related practices (\sectref{sec:heikkola:3.2}), the two topics of investigation in the present study.

\subsection{Identifying the language demands of academic tasks and discourse}\label{sec:heikkola:3.1}

Different contexts require different types of language use (\citealt{Cummins2000, Cummins2021-1}). In the context of schooling, language plays a central role in managing classroom activities and presenting content knowledge (\citealt{Cummins2021,Schleppegrell2002}). Thus, linguistically responsive teachers should be aware of the relationship between language and content. Furthermore, to support students who are simultaneously learning language and content, teachers must understand the demands of academic language and discourse (\citealt{Cummins2000,Gibbons2014,Schleppegrell2002}). The term \textit{academic language} refers to the language and literacy skills needed to function in the school context (\citealt{Cummins2000, Cummins2021-1,Wong_fillmore2000-2}) and is used to identify the linguistic features of school subjects and activities likely to pose challenges for EMLLs (\citealt{LucasVillegas2013}). Academic language was included in the study’s theoretical framework because developing higher levels of language proficiency is crucial for fully participating in today’s text-oriented society (\citealt{Cummins2021,Haneda2014}). Traditionally, discipline-specific content learning has been based on textual artefacts and literacy-focused tasks (\citealt{Barton2007,LuukkaEtAl2008}), and according to \citet[162]{Cummins2021-1}:

\begin{quote}
success in school for all students depends on the extent to which they develop competence in reading increasingly complex written texts and learning how to write coherently for a variety of audiences across the curriculum. 
\end{quote}

Academic language is demanding, as reading and writing in the language of schooling requires a specific set of linguistic resources (\citealt{Cummins2021,Gibbons2014}). Indeed, the syntactic and semantic features of academic language and different registers set higher cognitive demands than informal oral interactions, where the meaning is co-constructed (\citealt{Schleppegrell2002}). This study draws on the assumption that academic language differs fundamentally from conversational language (\citealt{Cummins2000,Gibbons2014,Schleppegrell2002}).

While oral academic situations do occur, research on students' language proficiency development often focuses on the challenges associated with written academic language (\citealt{Biber1986,GumperzEtAl1984,MichaelsCollins1984,Wong_fillmore2000-2}). To function in academic contexts, learners need linguistic resources to seek, analyse, and interpret information; understand and explain abstract concepts; and produce and edit written knowledge presentations (\citealt{Cummins2000,Gibbons2014}). These skills form the  that is needed to become an active member of today’s societies (e.g. \citealt{CopeKalantzis2000}). Multiliteracy, which is required by the Finnish national core curricula for basic education and general upper secondary education (\citealt{EDUFI2014, Edufi2019}), refers to the competent use of different texts and genres in appropriate contexts; thus, just knowing subject-specific vocabulary is not enough for success in an academic language situation.

\subsection{Practices for scaffolding instruction for multilingual learners}\label{sec:heikkola:3.2}

\citet{LucasVillegas2013} use the term \textit{scaffolding instruction} to refer to supporting multilingual learners’ learning of the content and language of instruction by drawing on their linguistic resources (see also \citealt{Vygotsky1978}). For teachers, \textit{scaffolding} means the instructional adaptations used to make academic content understandable for learners (see \citealt{Gibbons2014}) while keeping the cognitive demands of instruction high (\citealt{VillegasEtAl2018}). Leaning on sociocultural understanding, teaching is facilitating, and learning happens through interaction (\citealt{Teemant2014}).

Through scaffolding, learners can collectively complete academic tasks they could not do alone. Indeed, scaffolding is required for learners to participate in the teaching within their ZPD (\citealt{Vygotsky1978}). Scaffolding and ZPD are two sides of the same coin, as it is only when scaffolding is needed – and provided – that learning takes place (\citealt{Gibbons2014}). In other words, learning occurs when a more knowledgeable other, such as a teacher or a parent, collaborates with a student to help them function beyond their current capabilities (\citealt{Gibbons2014, Walqui2010-1}). Scaffolding can also happen between peers, suggesting that learners can support each other’s learning, as learners can recycle resources in interaction depending on their world knowledge (\citealt{Walqui2010-1}). In language learning, scaffolding aims to enable students to notice and adopt affordances that can be used for interaction (\citealt{Lier2000}).

\citet{Lucas2008-3} list seven different practices for scaffolding instruction for multilingual students: 1) using extra-linguistic aids, such as visual tools (pictures, videos) and graphic organisers (graphs, timelines); 2) supplementing or modifying written texts, for example, by developing study guides with questions, giving definitions and highlighting key terms, in order to enhance students’ reading; 3) supplementing and modifying oral texts, for example, by reducing speech rate, avoiding idiomatic expressions and pausing more frequently to allow more time for processing and responding; 4) giving clear and explicit instructions; 5) facilitating and encouraging the use of students’ first language(s); 6) engaging language learners in meaningful activities in which they can interact with others and negotiate meaning; and 7) minimising the potential for anxiety when using language in the classroom. Indeed, employing students’ entire linguistic repertoires through multilingual (\citealt{Cummins2021}) or translanguaging pedagogy (\citealt{Garcia2009-2}) can scaffold higher levels of academic performance. In addition, providing outlines for lectures, repeating key ideas (\citealt{Gibbons2014}) and establishing classroom routines so learners know what is expected of them (\citealt{WillettEtAl2007}) support students’ learning. The current study investigates whether pre-service subject teachers are aware of these kinds of linguistically responsive practices.

\section{Methodology}\label{sec:heikkola:4}

In this section, we introduce the participants and data (\sectref{sec:heikkola:4.1}), and theory-driven content analysis (\citealt{Krippendorf2012, Tuomi2018-1}), which was the method of analysis used in the study (\sectref{sec:heikkola:4.2}).

\subsection{Participants and data collection}\label{sec:heikkola:4.1}

The data for the study were collected via an online survey in Finnish.\textbf{ }The data of the study consists of pre-service subject teachers’ responses to one open-ended question that sought to measure participants’ preparedness  linguistically diverse learners. Analysis of the participants’ responses to other questions in the survey has been reported elsewhere (see \citealt{HeikkolaEtAl2022}).

The data were collected at the beginning and end of a year-long subject teacher education programme at a Finnish university; each participant responded to the same question twice. The data were collected during two consecutive academic years from two separate teacher education programmes. The investigated pre-service subject teachers did their practicums at a school where 64\% of the students spoke first languages other than Finnish (40 different languages).

74 participants responded to the survey both at the beginning and end of the teacher education programme and gave their written consent to participate in the study. Of the participating pre-service subject teachers, 39 were linguistics/literature students (henceforth linguistics students, majoring in Finnish, or foreign languages, such as English, Swedish or German, or literature),\footnote{In Finland, one can become a teacher of L1 Finnish or Swedish by majoring in literature, as the subject in school is called Finnish/Swedish and literature.} 21 studied natural sciences (majoring in mathematics, physics or chemistry), and 14 studied social sciences (majoring in religious studies or history). The different group sizes reflect the number of students in the different subjects undergoing teacher education. 

Participation in the study was voluntary. Participants’ privacy rights were respected by pseudonymisation of the data, that is, by using identification numbers for the participants and removing all personal information from the data for analysis, and by following the guidelines for ethical research. 

The open-ended question included in the study introduced a fictional scenario in a linguistically diverse school. Translated into English by the authors it read:

\begin{quote}
Imagine yourself in a situation in which you, as a teacher, are having a conversation at recess with an immigrant-background student. You notice that the student speaks Finnish fluently. However, once the lesson begins, you notice that the student has difficulties participating in the lesson. What do you think is the reason for this, and what would you, as a teacher, do in this situation?
\end{quote}

The question was two-fold: asking “What do you think is the reason?” sought participants’ understanding of academic language demands, while asking “What would you, as a teacher, do?” was intended to elicit ideas for suitable pedagogical practices for EMLLs. We expected the participants to reflect on why the student was having trouble in class and how they could make the academic content understandable to the student. The question intended to provide information on pre-service teachers’ knowledge of academic language demands and language-related pedagogical practices that operationalise scaffolding in a learner’s ZPD. It was assumed that these reflections stem from LRT, and it was designed to be unequivocal and accessible to all participants. The open-ended question was originally based on a question in the survey used by \citet{Alisaari2020_Apples}. 

\subsection{Data analysis}\label{sec:heikkola:4.2}

The data were analysed using theory-driven content analysis (\citealt{Krippendorf2012, Tuomi2018-1}), namely, reading and analysing the participants’ responses based on current theories regarding LRT (see \sectref{sec:heikkola:3}). This method allowed us to compare, contrast, categorise and test the pre-service subject teachers’ theoretical knowledge and skills related to LRT, and thus, enhance the consistency of the analysis (\citealt{Tuomi2018-1}). Theory-driven content analysis was chosen as the method of analysis because it enabled us to investigate the pre-service subject teachers’ preparedness for teaching linguistically diverse classrooms while comparing it to current theoretical understandings. We understand content analysis as a method to capture both qualitative and quantitative data (\citealt{Krippendorf2012,Tuomi2018-1}).

\begin{sloppypar}
To ensure unbiased analysis, the data were randomised, so the authors were not aware of the respondents’ subject groups. The analysis was done in four stages: 1) theory-driven coding of the data (\citealt{Krippendorf2012,Saldana2008-1,Tuomi2018-1}); 2) organising the responses according to the three subject teacher groups; 3) making comparisons between the responses from three teacher groups; and 4) looking at possible changes in responses before and after the year–long teacher education programme. Stage 1 was done in two steps, aligning with the two-fold question. Drawing on identifying the demands of academic language (e.g. \citealt{Cummins2000,Schleppegrell2002}), the potential reasons for difficulties in following instruction were coded according to the following scale (D~= demands):
\end{sloppypar}

\begin{enumerate}
  \item[D0:]   Language demands are not mentioned as a reason 
  \item[D1:]   Narrow knowledge of academic language demands as a reason 
  \item[D2:]   Partly theory-based knowledge of academic language demands as a reason 
  \item[D3:]  Theory-based knowledge of academic language demands as a reason
\end{enumerate}
  
Responses coded as D3 showed theory-based knowledge regarding the demands of academic language by explicitly describing the differences between conversational and academic language and at least hinting at contextual language use, such as the language demands of subject-specific tasks, genres and discourses. Responses coded as D2 were expected to contain similar knowledge of language demands but could contain some conceptual inaccuracies, or a mention of only one explicit context for language use. Those coded as D1 referred to language implicitly and did not explain how language demands could vary in different situations, or included responses only mentioning vocabulary. If language demands were not mentioned at all, the responses were coded as D0. Notably, some responses were coded into the lowest category even though they showed some knowledge of how situational factors affect language use (e.g. the student is more nervous in the classroom than during recess). However, as these did not mention academic language demands explicitly, they did not fully align with the theoretical framework of the study, and were thus coded as D0. 

Next, looking at language-related and interactional pedagogical practices that operationalise scaffolding in a learner’s ZDP (e.g. \citealt{Gibbons2014,LucasVillegas2013}), the participants’ responses regarding their reported practices were coded using the following scale (P~= practices):

\begin{enumerate}
  \item[P0:]   Language is not mentioned in the practices 
  \item[P1:]   Language is mentioned in the practices 
  \item[P2:]   Language is mentioned, and a concrete pedagogical practice is presented 
  \item[P3:]   Language is mentioned, and multiple concrete pedagogical practices are presented
\end{enumerate}

Responses coded as P3 mentioned multiple, concrete, language-related and interactional practices that operationalise scaffolding in a learner’s ZPD. Responses coded as P2 included one concrete, language-related scaffolding practice, while those in P1 mentioned a language-related practice without a concrete proposition or an explanation of how such a practice would scaffold instruction for EMLLs. Responses naming practices not related to language were coded as P0. Similar to the D-categories, some of these responses were coded as P0 even though they were useful and suitable for classroom instruction (e.g. helping students to overcome anxiety in performing in the classroom). 

\begin{sloppypar}
After the initial coding, the responses were organised according to the three student groups, and the groups’ responses regarding academic language demands (D) and language-related practices (P) were compared.  The final analysis focused on the changes that took place when contrasting the participants’ responses  the beginning and end of the year-long teacher education programme. The analysis was conducted through cycles of identifying patterns, reading the responses at different levels of abstraction, and reflecting against the theoretical framework. Throughout the analysis, we kept analytic memos and systematically took different perspectives on the data. In addition, short descriptive codes were added to describe the main content of the coded parts of the responses for both categories.
\end{sloppypar}

\section{Findings}\label{sec:heikkola:5}

In this section, we describe the participants’ knowledge of academic language demands (\sectref{sec:heikkola:5.1}) and discuss their awareness of language-related scaffolding practices (\sectref{sec:heikkola:5.2}). 

\subsection{Pre-service subject teachers’ knowledge regarding academic language demands}\label{sec:heikkola:5.1}

To address research question 1, the response categories regarding the participants’ knowledge of academic language demands are presented in \tabref{tab:heikkola:1} for the whole group, as well as for the three subject groups separately.     


\begin{table}
\small
\tabcolsep=.66\tabcolsep
%raw numbers are copied to the end of this chapter
% % % \includegraphics[width=\textwidth]{figures/Chapter8Heikkolaetalfinalversionauthornames240814-img001.png}
\begin{tabular}{l *3{S[table-format=2.0] S[table-format=2.0]} 
                  *2{S[table-format=2.1] S[table-format=2.0]}
                  *3{S[table-format=2.0] S[table-format=2.0]} 
               }
\lsptoprule
 & \multicolumn{4}{c}{All ($N=74$)} & \multicolumn{4}{c}{L ($n=39$)} & \multicolumn{4}{c}{NS ($n=21$)} & \multicolumn{4}{c}{SS ($n=14$)}\\
   \cmidrule(lr){2-5}\cmidrule(lr){6-9}\cmidrule(lr){10-13}\cmidrule(lr){14-17}
 & \multicolumn{2}{c}{before} & \multicolumn{2}{c}{after} & \multicolumn{2}{c}{before} & \multicolumn{2}{c}{after} & \multicolumn{2}{c}{before} & \multicolumn{2}{c}{after} & \multicolumn{2}{c}{before} & \multicolumn{2}{c}{after}\\
    & {\%} & {$n$}  & {\%} & {$n$}  & {\%} & {$n$}  & {\%} & {$n$} & {\%}  & {$n$} & {\%} & {$n$} & {\%} & {$n$} & {\%} & {$n$}\\\midrule
 D3 & 35 & 26 & 53 & 39 & 39 & 15 & 64  & 25 & 19  & 4 & 24 & 5 & 50 & 7 & 57 & 8\\
 D2 & 31 & 23 & 23 & 17 & 41 & 16 & 23  & 9  & 24  & 5 & 24 & 5 & 14 & 2 & 36 & 5\\
 D1 & 26 & 19 & 18 & 13 & 15 & 6  & 8   & 3  & 43  & 9 & 38 & 8 & 29 & 4 & 7  & 1\\
 D0 & 7  & 5  & 1  &  1 & 5  & 2  & 2.5 & 1  & 9.5 & 2 & 0  & 0 & 7  & 1 & 0  & 0\\
 mr & 1  & 1  & 5  &  4 & 0  & 0  & 2.5 & 1  & 4.5 & 1 & 14 & 3 & 0  & 0 & 0  & 0\\
\lspbottomrule
\end{tabular}
\caption{Development of knowledge regarding academic language demands from before to after the teacher education programme. Percentages were calculated without the missing values. L = linguistics students, NS = natural science students, SS = social science students, mr = missing response}
\label{tab:heikkola:1}
\end{table}

Looking at the whole group, the pre-service teachers’ theoretical understanding of language demands had increased during their year-long teacher education programme. Thus, it seems as though teacher education has benefited the pre-service subject teachers in supporting and developing their understanding of academic language demands. When comparing the three subject groups, namely linguistics, natural sciences and social sciences, the linguistics group seemed to benefit the most from the teacher education programme when looking at the most theory-based responses. They increased their theory-based knowledge of academic language demands from 39\% to 64\%, compared to a moderate increase from 19\% to 24\% in the natural sciences group and from 50\% to 57\% in the social sciences group. 

In the linguistics group, the percentage of responses categorised as reflecting a partly theory-based understanding went down from 41\% to 23\%, whereas the numbers stayed the same in the natural science group (24\%), and more than doubled (from 14\% to 36\%) in the social sciences group. Responses categorised as reflecting a narrow understanding of academic language demands decreased in all the three groups: from 15\% to 8\% in the linguistics group, from 29\% to 7\% in the social sciences group, and from 43\% to 38\% in the natural sciences group. A substantial number of responses coded as partly theory-based knowledge decreased at the end of the teacher education programme only in the linguistics group, with many responses being coded to the highest category, namely, theory-based knowledge of academic language demands. This trend was not seen in the other two groups.

Based on these findings, it seems that the linguistics group was best able to benefit from the teacher education programme in developing their understanding of the demands of academic language, followed by the social sciences group. The findings suggest that only a few pre-service subject teachers in the natural sciences group were prepared to use the opportunities offered in the teacher education programme to fully develop their understanding. These findings are in line with previous research, stating that (future) language teachers often have a broader understanding of the role of language, and thus have a higher level of understanding regarding language demands compared to future natural sciences teachers (\citealt{AlisaariEtAl2019,HeikkolaEtAl2021}).

Next, we compared the responses given before and after the year-long teacher education to see how individual participants’ knowledge had developed (see \tabref{tab:heikkola:2}). Three different response types emerged: 1) increased knowledge regarding academic language demands; 2) no change in knowledge of academic language demands; and 3) decreased knowledge regarding academic language demands. 


\begin{table}
\small
\tabcolsep=.66\tabcolsep
\begin{tabular}{l S[table-format=2.1]@{~}S[table-format=2.0] *3{S[table-format=2.0]@{~}S[table-format=2.0]} }
\lsptoprule
 & \multicolumn{2}{c}{All} & \multicolumn{2}{c}{L} & \multicolumn{2}{c}{NS} & \multicolumn{2}{c}{SS}\\
 & \multicolumn{2}{c}{($N=74$)} & \multicolumn{2}{c}{($n=39$)} & \multicolumn{2}{c}{($n=21$)} & \multicolumn{2}{c}{($n=14$)}\\
   \cmidrule(lr){2-3}\cmidrule(lr){4-5}\cmidrule(lr){6-7}\cmidrule(lr){8-9}
                        & {\%} & {$n$} & {\%} & {$n$} & {\%} & {$n$} & {\%} & {$n$}\\\midrule
Increased knowledge     & 40.5 & 30 & 41 & 16 & 48 & 10  & 29 &  4\\
No change in knowledge  & 40.5 & 30 & 46 & 18 & 14 & 3   & 64 &  9\\
Decreased knowledge     & 14   & 10 & 10 & 4  & 24 & 5   & 7  &  1\\
Missing responses       & 5    & 4  & 3  & 1  & 14 & 3   & 0  &  0\\
\lspbottomrule
\end{tabular}
% % % \includegraphics[width=\textwidth]{figures/Chapter8Heikkolaetalfinalversionauthornames240814-img002.png}
\caption{Development in knowledge regarding academic language demands from before to after the teacher education programme. L = linguistics students, NS = natural science students, SS = social science students.}
\label{tab:heikkola:2}
\end{table}

The understanding of academic language demands increased in the responses of 40.5\% of the pre-service teachers from before to after the teacher education programme. However, another 40.5\% of the pre-service teachers’ responses reflected no change in their understanding during the teacher education. In 14\% of the pre-service teachers’ responses, there was a decrease in the understanding of the demands of academic language. Based on the findings, it seems that the teacher training programme has supported understanding for approximately 40\% of the pre-service teachers. However, it should be noted that 35\% of the pre-service teachers were already at the highest level of understanding at the beginning of the year-long teacher education programme. As it has been suggested that theoretical knowledge is the base for pedagogical practice (see \citealt{Kirsch2018-1}), more focus is required on the demands of academic language, and, generally, on the role of language in all learning and teaching in teacher training.

Response examples illuminating the changes in the participants’ knowledge are presented in \tabref{tab:heikkola:3}. Participants’ subject group and identification number are presented together with the example number.

\begin{table}
\caption{Examples of responses regarding knowledge of the demands of academic language. Each example includes responses from the same pre-service teacher before and after the teacher education programme. L = linguistics student, NS = natural science student, SS = social science student}
\label{tab:heikkola:3}
\footnotesize
\begin{subtable}{\textwidth}\centering
\caption{Increased knowledge}
\begin{tabularx}{\textwidth}{>{\raggedright}p{\widthof{Example 6}} QQ}
\lsptoprule
                & \multicolumn{2}{c}{\itshape What do you think is the reason?}\\
                & Before & After\\\midrule
Example 1 (SS3) & The student has not learned vocabulary related to the subject. [D1]                         & The student has learned conversational Finnish but has not learned the school and subject-specific language. [D3]\\\addlinespace
Example 2 (NS12)& The concepts and phrasal structures can be very different compared to spoken language. [D2] & The language used during lessons is different from spoken language. [D3]\\
\lspbottomrule
\end{tabularx}
\end{subtable}\smallskip

\begin{subtable}{\textwidth}\centering
\caption{No change in knowledge}
\begin{tabularx}{\textwidth}{>{\raggedright}p{\widthof{Example 6}} QQ}
\lsptoprule
                & \multicolumn{2}{c}{\itshape What do you think is the reason?}\\
                & Before & After\\\midrule
Example 3 (SS2) & The language used during lessons might be foreign and academic for the student. [D2] & It might be that the student has difficulty mastering the language and vocabulary of the subject that is being taught. On the other hand, there might be peer pressure, and the student does not have the courage to ask/say things out loud in front of others. [D2]\\\addlinespace
Example 4 (L25) & Subject-specific language is often very different compared to language proficiency needed for everyday language use. Also, vocabulary is much more challenging in written theoretical language than in everyday speech. [D3] & For many, spoken language proficiency might be almost at a native level because the same structures and words are often repeated in spoken language. Modelling is essential for learning spoken language. However, the subject-specific languages are completely different from spoken language. The subject-specific languages are also challenging for native speakers, so special attention is needed with Finnish-as-a-second-language learners. [D3]\\
\lspbottomrule
\end{tabularx}
\end{subtable}
\end{table}

\begin{table}
\footnotesize
\ContinuedFloat
\caption{Examples of responses regarding knowledge of the demands of academic language. Each example includes responses from the same pre-service teacher before and after the teacher education programme. L = linguistics student, NS = natural science student, SS = social science student (cont.)}
\begin{subtable}{\textwidth}
\caption{Decreased knowledge}
\begin{tabularx}{\textwidth}{>{\raggedright}p{\widthof{Example 6}} QQ}
\lsptoprule
                & \multicolumn{2}{c}{\itshape What do you think is the reason?}\\
                & Before & After\\\midrule
Example 5 (NS13) & The reason might be that the student is fluent in everyday language. [D2] & [The student] doesn’t know the vocabulary of the subject in Finnish. They are not willing to communicate with the teacher or do not want the teacher to understand the topic of the conversation. [D1]\\\addlinespace
Example 6 (L22)  & The student can speak spoken language, but possibly the written language used in class creates a problem. [D2] & The student is insecure when using the language and may be afraid of being ridiculed if they make mistakes. It is also possible that the student has problems with reading and writing. [D0]\\
\lspbottomrule
\end{tabularx}
\end{subtable}
% % % \includegraphics[width=\textwidth]{figures/Chapter8Heikkolaetalfinalversionauthornames240814-img003.png}
\end{table}


\begin{sloppypar}
At the beginning of the teacher education programme, many of the participants understood language and academic language demands as comprising only vocabulary or terminology rather than entire texts and genres of discipline\hyp specific contexts (Examples 1 and 2): such responses were coded as D1. However, other participants indicated that the student may be more fluent in some language domains, such as spoken language, than in other domains (Example 6). These participants often highlighted the student’s conversational proficiency without explicitly distinguishing this from the language and literacy skills needed to function in academic contexts (Example 5). Often, situational language use was hinted at, but responses only explicitly named one situation. These responses were seen as reflecting partly theory-based knowledge and coded as D2. At the end of the teacher education programme, many of the participants were aware that academic language demands are fundamentally different from conversational language demands, and they were able to articulate their knowledge with concepts that aligned with the theoretical framework (Examples 1, 2 and 4), explicitly stating that language is used differently in different contexts. These responses were coded as D3. 
\end{sloppypar}

The participants whose knowledge of academic language demands increased during the teacher education programme had often broadened their understanding of language from it being limited to vocabulary or a specific language domain (e.g. speaking or writing) to it being different in different contexts, namely conversational language being different from academic language (Examples 1 and 2). When there were no changes in the participants’ knowledge regarding academic language, it was often because they already possessed more knowledge at the beginning of the programme (Example 4). This was especially true for the linguistics students, as they had already been studying language and literature for years. In the other two subject groups, the responses hinted at a moderate level of understanding at the beginning of the teacher education programme (Example 3). However, these participants were often not as well equipped to articulate their understanding as the linguistics students. As this may be a factor affecting the coding, responses have been interpreted based on their intended meaning, not the wording of the responses.

Reasons for decreased understanding of academic language demands varied. Often, the participants’ understanding was verbalised somewhat vaguely at the beginning of the teacher education programme (Examples 5 and 6). At the end of the programme, some responses only mentioned vocabulary (Example 5), which may reflect what the pre-service teachers had learned during their practicum. It is understandable that the participants’ responses may have focused on something concrete, such as specific words, instead of holistically focusing on language as a situational phenomenon. The participants’ experiences from their practicum are reflected in many responses given in the survey at the end of the programme. For example, some focused on affective factors that may be hindering students’ learning, such as peer pressure (Example 3), attitude (Example 5) or anxiety (Example 6). While knowledge of these factors is essential for teachers, the students’ responses may not have been coded highly on our scale on understanding academic language demands, as affective factors were not within the scope of the current study.

\subsection{Pre-service subject teachers’ awareness of language-related practices}\label{sec:heikkola:5.2}

In this section, we address research question 2. The response categories regarding pre-service subject teachers’ awareness of linguistically responsive practices are presented in \tabref{tab:heikkola:4}. In the analyses, we take all responses that mention language in some way to be promising signs of the participants’ awareness of language-related scaffolding practices.

     
\begin{table}
\small
\tabcolsep=.66\tabcolsep
% % % \includegraphics[width=\textwidth]{figures/Chapter8Heikkolaetalfinalversionauthornames240814-img004.png}
\begin{tabular}{l *3{S[table-format=2.0] S[table-format=2.0]} 
                  *2{S[table-format=2.1] S[table-format=2.0]}
                  *3{S[table-format=2.0] S[table-format=2.0]} 
               }
\lsptoprule
 & \multicolumn{4}{c}{All ($N=74$)} & \multicolumn{4}{c}{L ($n=39$)} & \multicolumn{4}{c}{NS ($n=21$)} & \multicolumn{4}{c}{SS ($n=14$)}\\
   \cmidrule(lr){2-5}\cmidrule(lr){6-9}\cmidrule(lr){10-13}\cmidrule(lr){14-17}
 & \multicolumn{2}{c}{before} & \multicolumn{2}{c}{after} & \multicolumn{2}{c}{before} & \multicolumn{2}{c}{after} & \multicolumn{2}{c}{before} & \multicolumn{2}{c}{after} & \multicolumn{2}{c}{before} & \multicolumn{2}{c}{after}\\
    & {\%} & {$n$}  & {\%} & {$n$}  & {\%} & {$n$}  & {\%} & {$n$} & {\%}  & {$n$} & {\%} & {$n$} & {\%} & {$n$} & {\%} & {$n$}\\\midrule
 P3 & 8  & 6  & 31 & 23 & 7.7  & 3  &  38.5 & 15 & 5  & 1   & 9.5  & 5  & 14.3 & 2  & 43 & 6\\
 P2 & 35 & 26 & 19 & 14 & 43.6 & 17 &  13   & 5  & 29 & 6   & 19   & 4  & 21.4 & 3  & 36 & 5\\
 P1 & 31 & 23 & 24 & 18 & 13   & 5  &  15   & 6  & 52 & 11  & 42.9 & 9  & 50   & 7  & 21 & 3\\
 P0 & 22 & 16 & 8  & 6  & 28   & 11 &  13   & 5  & 14 & 3   & 4.8  & 1  & 14.3 & 2  & 0  & 0\\
 mr & 4  & 3  & 18 & 13 & 7.7  & 3  &  20.5 & 8  & 0  & 0   & 23.8 & 5  & 0    & 0  & 0  & 0\\
\lspbottomrule
\end{tabular}
\caption{Development of awareness of linguistically responsive practices from before to after the teacher education programme. Percentages were calculated without the missing values. L = linguistics students, NS = natural science students, SS = social science students}
\label{tab:heikkola:4}
\end{table}

When looking at multiple language-related practices in the pre-service subject teachers’ responses, the awareness of linguistically responsive practices increased from 8\% to 31\% during the teacher education. The awareness increased greatly both in the linguistics (8\% → 38.5\%) and social sciences (14.3\% → 43\%) groups, whereas in the natural sciences group the increase was more moderate (5\% → 9.5\%). 

When analysing the whole group, the number of responses coded as including only one language-related scaffolding practice went down from 35\% to 19\%. Similar trends could be seen in the linguistics group (43.6--13\%) and natural sciences group (29--19\%). However, in the social sciences group these responses went up from 21.4\% to 36\%. On the other hand, in the social sciences group, there were fewer responses coded into the lower categories: language mentioned, or language not mentioned, at the end of the teacher education. 

Based on these findings, it seems that the linguistics and social sciences groups benefited the most from their teacher education programme year when looking at their awareness of linguistically responsive practices. As with their knowledge of academic language demands, the natural science group is clearly different from the linguistics and social sciences groups also with regard to awareness of language\hyp related scaffolding practices: a large part of this group’s responses reflected no change in awareness, and, in addition, this group also had a high number of missing responses at the end of their teacher education programme.

It was expected that the linguistics group would be (the most) aware of linguistically responsive practices, as they are interested in language(s) and have studied language and language learning in their major studies. Based on the findings, it does seem that the linguistics group were more able to benefit from their teacher education in a way that supported them to become more aware of linguistically responsive practices. Moreover, the social sciences group was able to benefit from the teacher education programme in such a way that their awareness regarding linguistically responsive practices increased during the teacher training programme. Similar findings pointing to the linguistics and social sciences groups’ increased ability to benefit from teacher education and to increase their understanding of the overall importance of the role of language in learning and teaching have been shown (\citealt{HeikkolaEtAl2021}). \citeauthor{HeikkolaEtAl2021} also showed that natural science students did not benefit as much as the other two groups, when it comes to understanding the role of language in learning and teaching.

When comparing the responses given before and after the teacher education programme, three different outcomes emerged: 1) increased awareness of linguistically responsive practices, 2) no change in awareness and 3) decreased awareness of linguistically responsive practices. The findings are presented in \tabref{tab:heikkola:5}. Looking at the whole group, 38\% increased their awareness. In the linguistics (46\%) and social sciences (57\%) groups approximately half of the participants had gained a higher awareness during their teacher education, whereas only 19\% of the natural science pre-service teachers had done the same. 

   
%%please move the includegraphics inside the {figure} environment
%%\includegraphics[width=\textwidth]{figures/Chapter8Heikkolaetalfinalversionauthornames240814-img005.png}
\begin{table}
\begin{tabular}{l *4{S[table-format=2.0] @{~} S[table-format=2.0]}}
\lsptoprule
                     & \multicolumn{2}{c}{All} & \multicolumn{2}{c}{L} & \multicolumn{2}{c}{NS} & \multicolumn{2}{c}{SS}\\
                       \cmidrule(lr){2-3}\cmidrule(lr){4-5}\cmidrule(lr){6-7}\cmidrule(lr){8-9}
Change in awareness  & \multicolumn{2}{c}{($N=74$)} & \multicolumn{2}{c}{($n=39$)} & \multicolumn{2}{c}{($n=21$)} & \multicolumn{2}{c}{($n=14$)}\\
of LR practices      & {\%} & {$n$} & {\%} & {$n$} & {\%} & {$n$} & {\%} & {$n$}\\\midrule
Increase             & 38 & 28 & 46 & 18 & 19 & 4 & 57 & 8\\
No change            & 31 & 23 & 21 & 8  & 43 & 9 & 36 & 5\\
Decrease             & 12 & 9  & 10 & 4  & 14 & 3 & 7  & 1 \\
Missing responses    & 19 & 14 & 23 & 9  & 24 & 5 & 0  & 0 \\
\lspbottomrule
\end{tabular}
\caption{\label{tab:heikkola:5}Development of awareness of linguistically responsive practices from before to after the teacher education programme. LR = linguistically responsive, L = linguistics students, NS = natural science students, SS = social science students}
\end{table}

31\% of the whole group remained at the same level of awareness regarding language-related practices during their teacher education. The natural science (43\%) and social sciences (36\%) groups had higher numbers of participants who remained at the same level compared to the linguistics group (21\%). Looking at the whole group, 12\% had a decreased awareness of language-related practices at the end of their teacher training. Here, there were no major differences between the groups: linguistics (10\%), natural sciences (14\%) and social sciences (7\%). Thus, based on our findings, it seems that the majority of the pre-service subject teachers had benefitted from their teacher training programme when it comes to pedagogical practices supporting multilingual students. Again, the natural sciences group differs somewhat from the two other groups, as not all pre-service teachers in natural sciences had gained as high an awareness regarding language-related scaffolding practices as the pre-service teachers in the other groups. 

In \tabref{tab:heikkola:6}, examples of changes in the participants’ concrete, language-related scaffolding practices are presented. Similar to the responses regarding academic language demands, at the beginning of the teacher education programme, many participants focused only on word-level scaffolding, for example, supporting the development of the student’s vocabulary, terminology and concepts (Examples 8 and 10). Responses focusing on vocabulary alone were coded as P1. At the beginning of their teacher education, many participants were able to name one language-related scaffolding practice, and emphasised modifying teacher speech or materials (Example 7), using plain Finnish (Examples 9 and 12) and being familiar with students’ language proficiency levels (Example 11). However, few participants were able to give more than one language-related practice at the beginning of the teacher education programme.

\begin{table}
% % % % \includegraphics[height=\textheight]{figures/Chapter8Heikkolaetalfinalversionauthornames240814-img006.png}
\footnotesize
\caption{Examples of responses regarding awareness of linguistically responsive practices. Each example includes responses from the same pre-service teacher before and after the teacher education programme. L = linguistics student, NS = natural science student, SS = social science student}
\label{tab:heikkola:6}
\begin{subtable}{\textwidth}
\caption{Increased practices}
\begin{tabularx}{\textwidth}{p{\widthof{Example 8}} QQ}
\lsptoprule
\multicolumn{3}{c}{What would you, as a teacher, do?}\\
 & Before & After\\
\midrule
Example 7 (L1) & As a teacher, I would pay attention to my own word choices, to how assignments are formulated and to how topics are taught. [P2]  & During lessons, the ways language is being used should be examined critically. The student could be given wordlists or things could be made easier for them. [P3]\\\addlinespace
Example 8 (L16) & In my teaching, I would try to pay attention to this and explain difficult terms carefully. [P1] & I would ask the student about the matter. I would also ask how I could help the student to better follow the teaching. I would also try to pay attention to Finnish-as-a-second-language  students in my teaching, e.g. by creating keyword lists to accompany assignment handouts and by supporting my speech, e.g. with the help of written notes on the smartboard. [P3]\\
\lspbottomrule
\end{tabularx}
\end{subtable}\medskip\\
\begin{subtable}{\textwidth}
\caption{No change in practices}
\begin{tabularx}{\textwidth}{p{\widthof{Example 10}} QQ}
\lsptoprule
\multicolumn{3}{c}{What would you, as a teacher, do?}\\
 & Before & After\\
\midrule
Example 9 (SS6) & As a teacher, I would aim for clear instruction and blackboard notes. I would also support the student when they are doing assignments. [P2] & As a teacher, I would try to explain concepts, so that they are easier to understand, and I would make sure that everyone understands the content no matter how difficult the language is. [P2]\\\addlinespace
Example 10 (NS8) & Terms and concepts should be explained as clearly as possible. [P1] &I would pay attention to explaining terms. [P1]\\
\lspbottomrule
\end{tabularx}
\end{subtable}
\end{table}

\begin{table}
\footnotesize
\ContinuedFloat
\caption{Examples of responses regarding awareness of linguistically responsive practices. Each example includes responses from the same pre-service teacher before and after the teacher education programme. L = linguistics student, NS = natural science student, SS = social science student (cont.)}
\begin{subtable}{\textwidth}
\caption{Decreased practices}
\begin{tabularx}{\textwidth}{p{\widthof{Example 10}} QQ}
\lsptoprule
\multicolumn{3}{c}{What would you, as a teacher, do?}\\
 & Before & After\\
\midrule
Example 11 (NS22) & It is important that the teacher understands how well the students can follow the teaching. The teaching should also happen without hurry and in peace. [P2] & I would contact special education, and we would think of a solution together. [P0]\\\addlinespace
Example 12 (L2) & It would be good for the teacher to coinsider this in their teaching and help the language learner by making the language used in teaching clearer. [P2] & As a teacher, I would speak with them about the matter during recess, and I would encourage them to use Finnish also during lessons. [P1]\\
\lspbottomrule
\end{tabularx}
\end{subtable}
\end{table}

At the end of the teacher education programme, most of the linguistics students were able to name two or more concrete language-related scaffolding practices (Examples 7 and 8). Some of these responses still focused on vocabulary, but holistic language-related scaffolding practices, such as examining language during lessons or using both spoken and written language when giving assignments, were also given. The linguistics students were more eloquent in verbalising language-related scaffolding practices at the end of their training than the other two groups. In the coding process, however, we aimed to interpret the content of the responses instead of the language, and did not penalise students for more straightforward responses.  

Some participants were not able to verbalise more concrete language-related scaffolding practices at the end of the teacher education programme, and many remained at the vocabulary level (Example 10). Interestingly, some participants learned to “outsource” issues related to language to other professionals, such as special educators (Example 11) or Finnish-as-a-second-language teachers. Although Example 11 is about multi-professional collaboration, which may be beneficial for the individual EMLL, the response lacks active language support for the student, which conflicts with the principles of language-aware schools stated in the Finnish core curricula (\citealt{EDUFI2014, Edufi2019}). In addition, the participants often had very high expectations of their students’ metacognitive skills regarding language learning; thus, the issue presented in the survey question was often seen as something the teacher could solve just by speaking with the student (Example 12). Compared to the linguistics and social sciences pre-service teachers, the natural sciences group had the most responses reflecting no change in or even decreased awareness of language-related practices. Often missing in the responses of all groups (even in the highest category level) were practices drawing on joint productive activities, collaboration and co-construction of knowledge (cf. \citealt{LucasVillegas2013}), such as group work, mini-experiments or talking about the content with a partner. However, this may also reflect the survey question, which focused on the teacher and their actions.

\section{Discussion and conclusions}\label{sec:heikkola:6}

Language is an essential mediator of teaching and learning (\citealt{Bunch2013}). To be linguistically responsive and support multilingual students in learning both language and content, teachers need pedagogical language knowledge and pedagogical content knowledge and skills (\citealt{Bunch2013,LucasVillegas2013}).

Teacher education plays a role in developing pre-service teachers’ knowledge about academic language demands and language-related practices. Many of the participants in this study increased their knowledge of academic language demands during the teacher education programme; their responses showed an understanding of the language (and even discourse) demands of academic tasks (\citealt{Cummins2000,Gibbons2014,Schleppegrell2002}). Previous research from Finland and the Nordic countries has shown that often teachers’ understanding of LRT is at a solid level (\citealt{AlisaariEtAl2019,Alisaari2020_Apples,Iversen2021,Lundberg2019}). In addition, \citet{chapters/7_alisaari} showed similar results, although for a different pre-service teacher population, namely pre-service primary school teachers, although they raise concerns regarding the pre-service primary teachers’ grammar knowledge. In the present study, the pre-service subject teachers’ responses often focused on language as a word-level phenomenon; thus, to gain theory-based knowledge regarding the demands of academic language, pre-service teachers should be better prepared to analyse syntactic and semantic characteristics of academic texts (\citealt{Cummins2000}).

The participants were able to name concrete practices that support multilingual students’ learning at the end of the teacher education programme more readily than at the beginning. The practices focused on supplementing teaching and modifying both written and oral texts. Notably, many of the practices regarding modifying written texts were primarily concerned with vocabulary at both the beginning and the end of the teacher education programme. This is in line with previous studies (\citealt{Aalto2019,HeikkolaEtAl2021,Heikkola2022-1}); (pre-service) teachers’ pedagogical understanding of language often remains on the vocabulary level, and when asked for possible teaching practices, the focus is on terminology. However, the vocabulary-based practices were somewhat refined towards the end of the programme; at the beginning, the responses included explaining difficult words to students, while at the end, many participants suggested giving students lists of keywords to scaffold assignments. Furthermore, at the beginning of the programme, modifying oral texts usually meant that the teacher were being mindful of their word choices, speech tempo and the provision of explicit instructions (cf. \citealt{HiteEvans2006}). However, towards the end of the training, practices of oral text modification became more holistic and included a critical examination of language during lessons. The characteristic responses of the participants lacked practices listed by \citeauthor{LucasEtAl2008} (\citeyear{LucasEtAl2008}; see also \sectref{sec:heikkola:3.2}). There were few mentions of visual aids, although previous research has shown that, of all the practices used, Finnish in-service teachers use extra-linguistic cues, such as graphic organisers and visual tools, the most (\citealt{HeikkolaEtAl2022}). The participants never reported using students’ first language(s) as a resource in instruction, which resonates with previous findings from Finland: not many teachers are equipped to include students’ first language(s) in multilingual pedagogies (\citealt{AlisaariEtAl2019}). However, multilingualism is generally viewed positively (\citealt{AlisaariEtAl2019}); teachers support immigrant-background children speaking their first language(s) at home (\citealt{AlisaariEtAl2021}), and pre-service primary school teachers seem to understand the importance of L1 in learning (see \citealt{chapters/7_alisaari}). As strong first-language skills support all learning (\citealt{Gibbons2014,LucasVillegas2010}), it is important for future teachers to understand the value of harnessing students’ entire linguistic repertoires to support (content) learning, and to provide scaffolding for language learners by drawing on their linguistic resources (\citealt{Cummins2021,Garcia2009-2}).

The participants rarely reported meaningful collaborative activities where students co-construct knowledge or negotiate meaning. The lack of such practices contradicts sociocultural understanding of (language) learning through social interaction (cf. \citealt{Donato2000,Dufva2020,LantolfThorne2006,Vygotsky1978}), which is the premise for the Finnish national core curricula, and widely seen as the basis for all learning in school across the globe.

It was hypothesised that the linguistics students would have the highest levels of awareness of academic-language demands and linguistically responsive practices. Based on our findings, this seems to be the case. In addition, this group increased its awareness of LRT the most during the teacher education programme compared to the social sciences and natural sciences groups. This may be linked to the content of the teacher education programme: language-teaching-related studies potentially focus more on the implementation of language awareness. In addition, due to their area of expertise, linguistics students may have had the concepts and vocabulary to verbalise their thoughts on the role of language in teaching and learning more accurately than students of other majors (cf. \citealt{HeikkolaEtAl2021}). However, such a finding raises questions about the challenges of teaching LRT principles to those pre-service teachers who possibly need this knowledge the most. Developing preparedness for linguistically diverse learners requires both time and interest; those who deem a topic relevant are those who learn the most (\citealt{Repo2020}).

Given the qualitative nature of the study and the relatively small sample size, it is impossible to make strong generalisations from the findings. Nevertheless, the analysis provides a window into pre-service subject teachers’ reasoning and the ways their thinking regarding linguistic diversity issues shifted during the year-long teacher education programme. A special strength of the study is its longitudinal nature. In contrast to many other Finnish and Nordic studies, this study has a longitudinal design, and has focused on the development of pre-service subject teachers’ preparedness for linguistically diverse classrooms during a year-long teacher education programme. Through this design, we were able to investigate how pre-service subject-teachers’ understanding of academic language demands and their awareness of language-related scaffolding practises developed within a year. These results can thus be applied to other teacher education programmes to support courses that promote LRT. 

\begin{sloppypar}
The data collection method (an online survey) may have influenced participants’ responses: short responses were expected. Asking respondents to write a short essay focusing on the different theories and practices taught in the teacher education programme could have captured more of the participants’ knowledge and awareness. However, some development in the participants’ knowledge about the demands of academic language and language-related practices was found, and participants reflections could be seen in their responses. In addition, a survey enabled reaching a larger group of participants than an essay may have. As the survey only took a short time to respond to, it was reasonable to ask participants to do so twice, which may not have been possible with an essay. Furthermore, the open question format allowed us to examine the pre-service teachers’ development in their responses. When interpreting the findings, however, consideration should be given to whether the decrease in understanding of academic language demands and language-related scaffolding practices was due to lessened knowledge or low motivation to respond to the survey at the end of the academic year. 
\end{sloppypar}

The findings of this study focus on language-related issues. We did not consider affective factors or classroom dynamics in the categorisation of the responses if language was not included, although these are important practices that support learning in general. Further research is needed to investigate the development of teachers’ expertise related to non-linguistic pedagogical practices, including taking into consideration the affective factors related to learning.

Based on the findings of this study, a one-year teacher education programme may not be long enough to fully prepare pre-service subject teachers for the societal and curricular changes inherent to increasing linguistic diversity. Teacher’s professionalisation develops slowly, and time is needed for teachers to reflect on the impact of their knowledge and practices. In the future, teacher education should have a strong(er) emphasis on LRT, especially in programmes for subject teacher groups other than future language teachers. Thus, more theoretical and practical education is warranted, especially for other subject groups. Furthermore, it is important to ensure that the topic of LRT is covered in professional development for in-service teachers as well. The extensive role of language in schooling should play a role in the subject teacher education programme, and LRT practices should be modelled to support (language) learning and multilingualism in the classroom. In this way, more pre-service subject teachers will have agency in supporting their EMLLs.

\printbibliography[heading=subbibliography,notkeyword=this]
\end{document}



% \begin{tabularx}{\textwidth}{XXXXXXXXX}
%  & { \textbf{All}}
% \lsptoprule

%  \textbf{(\textit{N}} \textbf{=} \textbf{74)} &  & { \textbf{L}}

%  \textbf{(\textit{n} }\textbf{=} \textbf{39)} &  & { \textbf{NS}}

%  \textbf{(\textit{n} }\textbf{=} \textbf{21)} &  & { \textbf{SS}}

%  \textbf{(\textit{n}} \textbf{=} \textbf{14)} & \\
% & { \textbf{before}}

%  \textbf{\%} \textbf{(n)} & { \textbf{after}}

%  \textbf{\%} \textbf{(n)} & { \textbf{before}}

%  \textbf{\%} \textbf{(n)} & { \textbf{after}}

%  \textbf{\%} \textbf{(n)} & { \textbf{before}}

%  \textbf{\%} \textbf{(n)} & { \textbf{after}}

%  \textbf{\%} \textbf{(n)} & { \textbf{before}}

%  \textbf{\%} \textbf{(n)} & { \textbf{after}}

%  \textbf{\%} \textbf{(n)}\\
%  \textbf{Theory-based} \textbf{knowledge} \textbf{(D3)} & 35\% \REF{ex:heikkola:26} & 53\% \REF{ex:heikkola:39} & 39\% \REF{ex:heikkola:15} & 64\% \REF{ex:heikkola:25} & 19\% \REF{ex:heikkola:4} & 24\% \REF{ex:heikkola:5} & 50\% \REF{ex:heikkola:7} & 57\% \REF{ex:heikkola:8}\\
%  \textbf{Partly} \textbf{theory-based} \textbf{knowledge} \textbf{(D2)} & 31\% \REF{ex:heikkola:23} & 23\% \REF{ex:heikkola:17} & 41\% \REF{ex:heikkola:16} & 23\% \REF{ex:heikkola:9} & 24\% \REF{ex:heikkola:5} & 24\% \REF{ex:heikkola:5} & 14\% \REF{ex:heikkola:2} & 36\% \REF{ex:heikkola:5}\\
%  \textbf{Narrow} \textbf{knowledge} \textbf{(D1)} & 26\% \REF{ex:heikkola:19} & 18\% \REF{ex:heikkola:13} & 15\% \REF{ex:heikkola:6} & 8\% \REF{ex:heikkola:3} & 43\% \REF{ex:heikkola:9} & 38\% \REF{ex:heikkola:8} & 29\% \REF{ex:heikkola:4} & 7\% \REF{ex:heikkola:1}\\
%  \textbf{Language} \textbf{not} \textbf{mentioned} \textbf{(D0)} & 7\% \REF{ex:heikkola:5} & 1\% \REF{ex:heikkola:1} & 5\% \REF{ex:heikkola:2} & 2.5\% \REF{ex:heikkola:1} & 9.5\% \REF{ex:heikkola:2} & 0 (0\%) & 7\% \REF{ex:heikkola:1} & 0 \% \REF{ex:heikkola:0}\\
%  \textbf{Missing} \textbf{responses} & 1 \% \REF{ex:heikkola:1} & 5\% \REF{ex:heikkola:4} & 0 \% \REF{ex:heikkola:0} & 2.5\% \REF{ex:heikkola:1} & 4.5\% \REF{ex:heikkola:1} & 14\% \REF{ex:heikkola:3} & 0\% \REF{ex:heikkola:0} & 0\% \REF{ex:heikkola:0}\\
% \lspbottomrule
% \end{tabularx}
% \tabref{tab:heikkola:2}


% \begin{tabularx}{\textwidth}{XXXXX}
%  & { \textbf{All}}
% \lsptoprule

%  \textbf{(\textit{N}} \textbf{=} \textbf{74)} & { \textbf{L}}

%  \textbf{(\textit{n} }\textbf{=} \textbf{39)} & { \textbf{NS}}

%  \textbf{(\textit{n} }\textbf{=} \textbf{21)} & { \textbf{SS}}

%  \textbf{(\textit{n}} \textbf{=} \textbf{14)}\\
% & \textbf{\%} \textbf{(n)} & \textbf{\%} \textbf{(n)} & \textbf{\%} \textbf{(n)} & \textbf{\%} \textbf{(n)}\\
%  \textbf{Increased} \textbf{knowledge} & 40.5\% \REF{ex:heikkola:30} & 41\% \REF{ex:heikkola:16} & 48\% \REF{ex:heikkola:10} & 29\% \REF{ex:heikkola:4}\\
%  \textbf{No} \textbf{change} \textbf{in} \textbf{knowledge} & 40.5\% \REF{ex:heikkola:30} & 46\% \REF{ex:heikkola:18} & 14\% \REF{ex:heikkola:3} & 64\% \REF{ex:heikkola:9}\\
%  \textbf{Decreased} \textbf{knowledge} & 14\% \REF{ex:heikkola:10} & 10\% \REF{ex:heikkola:4} & 24\% \REF{ex:heikkola:5} & 7\% \REF{ex:heikkola:1}\\
%  \textbf{Missing} \textbf{responses} & 5 \% \REF{ex:heikkola:4} & 3 \% \REF{ex:heikkola:1} & 14\% \REF{ex:heikkola:3} & 0\% \REF{ex:heikkola:0}\\
% \lspbottomrule
% \end{tabularx}
% \tabref{tab:heikkola:3}


% \begin{tabularx}{\textwidth}{XXX}

% \lsptoprule
% \multicolumn{3}{c}{ \textit{What do you think is the reason?}}\\
% & Before & After\\
% \multicolumn{2}{c}{\textbf{Increased} \textbf{knowledge}

% } & \\
% \textbf{Example} \textbf{1~}

% \textbf{(SS3)} & The student has not learned vocabulary related to the subject.

% [D1] & The student has learned conversational Finnish but has not learned the school and subject-specific language.

% [D3]\\
% \textbf{Example} \textbf{2~}

% \textbf{(NS12)} & The concepts and~phrasal structures can be very different compared to spoken language.~

% [D2] & The language used during lessons is different from spoken language.

% [D3]~\\
% \multicolumn{2}{c}{\textbf{No} \textbf{change} \textbf{in} \textbf{knowledge}} & \\
% \textbf{Example} \textbf{3~}

% \textbf{(SS2)} & The language used during lessons might be foreign and academic for the student.

% [D2] & It might be that the student has difficulty mastering the language and vocabulary of the subject that is being taught. On the other hand, there might be peer pressure, and the student does not have the courage to ask/say things out loud in front of others.

% [D2]\\
% \textbf{Example} \textbf{4~}

% \textbf{(L25)} & Subject-specific language is often very different compared to language proficiency needed for everyday language use. Also, vocabulary is much more challenging in written theoretical language than in everyday speech.

% [D3] & For many, spoken language proficiency might be almost at a native level because the same structures and words are often repeated in spoken language. Modelling is essential for learning spoken language. However, the subject-specific languages are completely different from spoken language. The subject-specific languages are also challenging for native speakers, so special attention is needed with Finnish-as-a-second-language learners.

% [D3]~\\
% \multicolumn{2}{c}{\textbf{Decreased} \textbf{knowledge}

% } & \\
% \textbf{Example} \textbf{5~}

% \textbf{(NS13)} & The reason might be that the student is fluent in everyday language.

% [D2]~ & [The student] doesn’t know the vocabulary of the subject in Finnish. They are not willing to communicate with the teacher or do not want the teacher to understand the topic of the conversation.

% [D1]~\\
% \textbf{Example} \textbf{6~}

% \textbf{(L22)} & The student can speak spoken language, but possibly the written language used in class creates a problem.

% [D2]~ & The student is insecure when using the language and may be afraid of being ridiculed if they make mistakes. It is also possible that the student has problems with reading and writing.

% [D0]~\\
% \lspbottomrule
% \end{tabularx}
% \tabref{tab:heikkola:4}


% \begin{tabularx}{\textwidth}{XXXXXXXXX}
%  & { \textbf{All}}
% \lsptoprule

% { \textbf{(\textit{N}} \textbf{=} \textbf{74)}}

% { \textbf{before}}

%  \textbf{\%} \textbf{(n)} & { \textbf{after}}

%  \textbf{\%} \textbf{(n)} & { \textbf{L}}

% { \textbf{(\textit{n} }\textbf{=} \textbf{39)}}

% { \textbf{before}}

%  \textbf{\%} \textbf{(n)} & { \textbf{after}}

%  \textbf{\%} \textbf{(n)} & { \textbf{NS}}

% { \textbf{(\textit{n} }\textbf{=} \textbf{21)}}

% { \textbf{before}}

%  \textbf{\%} \textbf{(n)} & { \textbf{after}}

%  \textbf{\%} \textbf{(n)} & { \textbf{SS}}

% { \textbf{(\textit{n}} \textbf{=} \textbf{14)}}

% { \textbf{before}}

%  \textbf{\%} \textbf{(n)} & { \textbf{after}}

%  \textbf{\%} \textbf{(n)}\\
%  \textbf{Multiple} \textbf{language-related} \textbf{practices} \textbf{(P3)} & { 8\%}

%  \REF{ex:heikkola:6} & 31\% \REF{ex:heikkola:23} & 7.7\% \REF{ex:heikkola:3} & 38.5\% \REF{ex:heikkola:15} & 5\% \REF{ex:heikkola:1} & 9.5\% \REF{ex:heikkola:5} & 14.3\% \REF{ex:heikkola:2} & 43\% \REF{ex:heikkola:6}\\
%  \textbf{One} \textbf{language-related} \textbf{practice} \textbf{(P2)} & 35\% \REF{ex:heikkola:26} & 19\% \REF{ex:heikkola:14} & 43.6\% \REF{ex:heikkola:17} & 13\% \REF{ex:heikkola:5} & 29\% \REF{ex:heikkola:6} & 19\% \REF{ex:heikkola:4} & 21.4\% \REF{ex:heikkola:3} & 36\% \REF{ex:heikkola:5}\\
%  \textbf{Language} \textbf{mentioned} \textbf{(P1)} & 31\% \REF{ex:heikkola:23} & 24\% \REF{ex:heikkola:18} & 13\% \REF{ex:heikkola:5} & { 15\%}

%  \REF{ex:heikkola:6} & { 52\%}

%  \REF{ex:heikkola:11} & 42.9\% \REF{ex:heikkola:9} & 50\% \REF{ex:heikkola:7} & 21\% \REF{ex:heikkola:3}\\
%  \textbf{Language} \textbf{not} \textbf{mentioned} \textbf{(P0)} & 22\% \REF{ex:heikkola:16} & { 8\%}

%  \REF{ex:heikkola:6} & 28\% \REF{ex:heikkola:11} & { 13\%}

%  \REF{ex:heikkola:5} & { 14\%}

%  \REF{ex:heikkola:3} & 4.8\% \REF{ex:heikkola:1} & 14.3\% \REF{ex:heikkola:2} & 0\% \REF{ex:heikkola:0}\\
%  \textbf{Missing} \textbf{responses} & 4\% \REF{ex:heikkola:3} & 18\% \REF{ex:heikkola:13} & 7.7\% \REF{ex:heikkola:3} & 20.5\% \REF{ex:heikkola:8} & 0\% \REF{ex:heikkola:0} & 23.8\% \REF{ex:heikkola:5} & 0\% \REF{ex:heikkola:0} & 0\% \REF{ex:heikkola:0}\\
% \lspbottomrule
% \end{tabularx}
% \tabref{tab:heikkola:5}


% \begin{tabularx}{\textwidth}{XXXXX}
%  & { \textbf{All}}
% \lsptoprule

% { \textbf{(\textit{N}} \textbf{=} \textbf{74)}}

%  \textbf{\%} \textbf{(n)} & { \textbf{L}}

% { \textbf{(\textit{n} }\textbf{=} \textbf{39)}}

%  \textbf{\%} \textbf{(n)} & { \textbf{NS}}

% { \textbf{(\textit{n} }\textbf{=} \textbf{21)}}

%  \textbf{\%} \textbf{(n)} & { \textbf{SS}}

% { \textbf{(\textit{n}} \textbf{=} \textbf{14)}}

%  \textbf{\%} \textbf{(n)}\\
%  \textbf{Increase} \textbf{in} \textbf{the} \textbf{awareness} \textbf{of} \textbf{LR} \textbf{practices} & 38\% \REF{ex:heikkola:28} & 46\% \REF{ex:heikkola:18} & 19\% \REF{ex:heikkola:4} & 57\% \REF{ex:heikkola:8}\\
%  \textbf{No} \textbf{change} \textbf{in} \textbf{the} \textbf{awareness} \textbf{of} \textbf{LR} \textbf{practices} & 31\% \REF{ex:heikkola:23} & 21\% \REF{ex:heikkola:8} & 43\% \REF{ex:heikkola:9} & 36\% \REF{ex:heikkola:5}\\
%  \textbf{Decrease} \textbf{in} \textbf{the} \textbf{awareness} \textbf{of} \textbf{LR} \textbf{practices} & 12\% \REF{ex:heikkola:9} & 10\% \REF{ex:heikkola:4} & 14\% \REF{ex:heikkola:3} & 7\% \REF{ex:heikkola:1}\\
%  \textbf{Missing} \textbf{responses} & 19\% \REF{ex:heikkola:14} & 23\% \REF{ex:heikkola:9} & 24\% \REF{ex:heikkola:5} & 0\% \REF{ex:heikkola:0}\\
% \lspbottomrule
% \end{tabularx}
% \tabref{tab:heikkola:6} 


% \begin{tabularx}{\textwidth}{XXX}

% \lsptoprule
% \multicolumn{3}{c}{ \textit{What would you, as a teacher, do?}}\\
% & Before & After\\
% \multicolumn{2}{c}{\textbf{Increased} \textbf{practices}} & \\
% \textbf{Example} \textbf{7}

% \textbf{(L1)} & As a teacher, I would pay attention to my own word choices, to how assignments are formulated and to how topics are taught.

% [P2] & During lessons, the ways language is being used should be examined critically. The student could be given wordlists or things could be made easier for them. 

% [P3] \\
% \textbf{Example} \textbf{8}

% \textbf{(L16)} & In my teaching, I would try to pay attention to this and explain difficult terms carefully. 

% [P1] & I would ask the student about the matter. I would also ask how I could help the student to better follow the teaching. I would also try to pay attention to Finnish{}-as{}-a{}-second{}-language  students in my teaching, e.g. by creating keyword lists to accompany assignment handouts and by supporting my speech, e.g. with the help of written notes on the smartboard.

% [P3] \\
% \multicolumn{2}{c}{\textbf{No} \textbf{change} \textbf{in} \textbf{practices}} & \\
% \textbf{Example} \textbf{9}

% \textbf{(SS6)} & As a teacher, I would aim for clear instruction and blackboard notes. I would also support the student when they are doing assignments.

% [P2] & As a teacher, I would try to explain concepts, so that they are easier to understand, and I would make sure that everyone understands the content no matter how difficult the language is.

% [P2]\\
% \textbf{Example} \textbf{10}

% \textbf{(NS8)} & Terms and concepts should be explained as clearly as possible. 

% [P1] & I would pay attention to explaining terms.

% [P1] \\
% \multicolumn{2}{c}{\textbf{Decreased} \textbf{practices}} & \\
% \textbf{Example} \textbf{11}

% \textbf{(NS22)} & It is important that the teacher understands how well the students can follow the teaching. The teaching should also happen without hurry and in peace. 

% [P2] & I would contact special education, and we would think of a solution together.

% [P0] \\
% \textbf{Example} \textbf{12}

% \textbf{(L2)} & It would be good for the teacher to coinsider this in their teaching and help the language learner by making the language used in teaching clearer.

% [P2] & As a teacher, I would speak with them about the matter during recess, and I would encourage them to use Finnish also during lessons. 

% [P1] \\
% \lspbottomrule
% \end{tabularx}
 
