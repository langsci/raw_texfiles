\documentclass[output=paper]{langscibook}
\ChapterDOI{10.5281/zenodo.15280899}
\author{Hafdís Guðjónsdóttir\orcid{}\affiliation{University of Iceland} and Jónína Vala Kristinsdóttir\orcid{}\affiliation{University of Iceland} and Gunnhildur Óskarsdóttir\orcid{}\affiliation{University of Iceland} and Samúel Lefever\orcid{}\affiliation{University of Iceland}}
\title[Preparing teachers in Iceland to work in multilingual classrooms]{Preparing pre- and in-service teachers in Iceland to work in multilingual classrooms}
\abstract{Increased immigration in Iceland has led to growing ethnic, linguistic and religious diversity among students. This diversity presents challenges for teachers and teacher educators. The aim of this study was to deepen understandings of teacher educators’ perceptions of preparing pre- and in-service teachers for working in linguistically diverse classrooms. The study was carried out by four teacher educators at the University of Iceland. Twenty-eight participants (teacher educators and colleagues of the researchers) were interviewed in 13 focus groups.  Teacher educators say that while they do not always place emphasis on theories of multicultural or multilingual education, they attempt to model effective teaching practices and provide pre- and in-service teachers with opportunities to broaden their experience and understanding of working with diverse student groups. These findings can help teacher educators understand how diversity in the student population impacts on teacher education, and shed light on needs in teacher education regarding work in linguistically diverse contexts.}
\IfFileExists{../localcommands.tex}{
  \addbibresource{../localbibliography.bib}
  % add all extra packages you need to load to this file

\usepackage{tabularx,multicol}
\usepackage{url}
\urlstyle{same}

\usepackage{listings}
\lstset{basicstyle=\ttfamily,tabsize=2,breaklines=true}

\usepackage{langsci-basic}
\usepackage{langsci-optional}
\usepackage{langsci-lgr}
\usepackage{langsci-osl}
% \usepackage{./langsci/styles/langsci-lgr}
% \usepackage{./langsci/styles/langsci-osl}
% \usepackage{langsci-gb4e}

\usepackage{tikz}
\usetikzlibrary{patterns,calc}
\pgfdeclarepatternformonly{south east lines}{\pgfqpoint{-0pt}{-0pt}}{\pgfqpoint{3pt}{3pt}}{\pgfqpoint{3pt}{3pt}}{
    \pgfsetlinewidth{0.6pt}
    \pgfpathmoveto{\pgfqpoint{0pt}{3pt}}
    \pgfpathlineto{\pgfqpoint{3pt}{0pt}}
    \pgfpathmoveto{\pgfqpoint{.2pt}{-.2pt}}
    \pgfpathlineto{\pgfqpoint{-.2pt}{.2pt}}
    \pgfpathmoveto{\pgfqpoint{3.2pt}{2.8pt}}
    \pgfpathlineto{\pgfqpoint{2.8pt}{3.2pt}}
    \pgfusepath{stroke}}
    
\usepackage{stmaryrd}
\usepackage{wasysym}
\usepackage{multirow}
\usepackage{caption}
\usepackage{subcaption}
\usepackage{mathrsfs}
\usepackage{qtree}

\usepackage{linguex}


  %pminos do not split footnotes
% \interfootnotelinepenalty=10000 %Footnote in Laporte chapters has to be split SN


%\DeclareIndexNameFormat{default}{%
%\nameparts{#1}%
%\usebibmacro{index:name}%
%{\index[names]}%
%{\namepartfamily}%
%{\namepartgiveni}%
% {}% L1
% {}% L2
%{\namepartprefix}% generates spurious space L3
%{\namepartsuffix}% generates spurious space L4
%}

%  {\DeclareIndexNameFormat{default}{%
%     \usebibmacro{index:name}{\index[names]}{#1}{#3}{#5}{#7}}}

%\DeclareIndexNameFormat{default}{%
%  \usebibmacro{index:name}{\sindex[nom]}{#1}{#3}{#5}{#7}}

%\DeclareIndexNameFormat{default}{%
%  \usebibmacro{index:name}{\sindex[person]}{#1}{#3}{#5}{#7}}
%\DeclareIndexNameFormat{default}{%
%\nameparts{#1} \usebibmacro{index:name}{\sindex[person]]}{\namepartfamily}{‌​\namepartgiven}{\nam‌​epartprefix}{\namepa‌​rtsuffix}}

%\newcommand{\smiley}{:)}

%\renewbibmacro*{index:name}[5]{%
%\usebibmacro{index:entry}{#1}%
%{\iffieldundef{usera}{}{\thefield{usera}\actualoperator}\mkbibindexname{#2}{#3}{#4}{#5}}}

% \newcommand{\noop}[1]{}

%remove for final
%\overfullrule=1mm

\newcommand{\tobi}[2]}}
\renewcommand{\S}[1]{\tobi{#1}{\textsc{*}}}

% this volume references
% puts: [this volume]
% already defined: \citetv
%\newcommand{\citepv}[1]{(\citeauthor{#1} \citeyear*{#1} [this volume])}
\newcommand{\citealtv}[1]{\citeauthor{#1} \citeyear*{#1} [this volume]}

%parentheses around example number
\newcommand{\pref}[1]{(\ref{#1})}

% in-text examples

\newcommand{\lnex}[1]{\textit{#1}} %target lang word
\newcommand{\lnlit}[1]{(lit.: `#1')} %literal reading
\newcommand{\lnlat}[1]{(#1)} % latinization
\newcommand{\lntrans}[1]{`#1'} %translation
\newcommand{\lnexl}[2]%
{\lnex{#1}{} \lnlat{#2}} % ex with latinization
\newcommand{\lnexlat}[3]{\lnex{#1}{} \lnlat{#2}{} \lntrans{#3}} % ex with latinization and tranl.

%ch01
\newcommand{\co}[1]{\mbox{\textbf{#1}}}

%ch09

\newcommand{\cyrbulg}[1]{\begin{otherlanguage*}{bulgarian}#1\end{otherlanguage*}}


%ch10
\newcommand{\nlp}{{\small NLP}}
\newcommand{\mwe}{{\small MWE}}
\newcommand{\rae}{{\small RAE}}
\newcommand{\lvc}{{\small LVC}}
\newcommand{\pos}{{\small P}o{\small S}}
%\newcommand{\todo}[1]{ \textcolor{red}{#1} }

%\renewcommand{\labelenumi}{\theenumi}
%\ainamefmt{{vv}{ll}{, ff}{, jj}} % fullname

\newcommand{\biberror}[1]{{\color{red}#1}}

\newcommand{\osenovaitem}{--~} 
  %% hyphenation points for line breaks
%% Normally, automatic hyphenation in LaTeX is very good
%% If a word is mis-hyphenated, add it to this file
%%
%% add information to TeX file before \begin{document} with:
%% %% hyphenation points for line breaks
%% Normally, automatic hyphenation in LaTeX is very good
%% If a word is mis-hyphenated, add it to this file
%%
%% add information to TeX file before \begin{document} with:
%% %% hyphenation points for line breaks
%% Normally, automatic hyphenation in LaTeX is very good
%% If a word is mis-hyphenated, add it to this file
%%
%% add information to TeX file before \begin{document} with:
%% \include{localhyphenation}
\hyphenation{
    Beck-man
    Ngu-yen
    back-chan-nel
    back-chan-nels
    mo-not-o-nous
    ste-reo-typ-i-cal
}

\hyphenation{
    Beck-man
    Ngu-yen
    back-chan-nel
    back-chan-nels
    mo-not-o-nous
    ste-reo-typ-i-cal
}

\hyphenation{
    Beck-man
    Ngu-yen
    back-chan-nel
    back-chan-nels
    mo-not-o-nous
    ste-reo-typ-i-cal
}
 
  \togglepaper[1]%%chapternumber
}{}

\begin{document}
\lehead{Hafdís Guðjónsdóttir et al.}
\maketitle 
\label{chap:4}
%\shorttitlerunninghead{}%%use this for an abridged title in the page headers


\section{Introduction}

One of the major challenges for teachers in modern times is the continuous search for pedagogy and teaching approaches to meet the growing diversity within schools. Increasing rates of immigration to Iceland, especially in the last 20 years, is leading to a growing population of students with different ethnic, linguistic, and religious {backgrounds}{.} In 1997, 0.8\% of compulsory school students had a language other than Icelandic as their mother tongue (\citealt{Statistics_iceland2018}). In the autumn of 2017, the proportion was 9.9\% of the student population, which represents an increase of over 300 students from the previous year. In the autumn of 2022, 6,570 students or 14\% of the students in compulsory schools had a mother tongue other than Icelandic, an increase of 670 students from the previous year (\citealt{Statistics_iceland2023}).

These changes in recent years call for a reconceptualisation of how teachers are prepared for teaching in diverse settings.{ To face these challenges, it is necessary for school leaders and teachers to consider how they can create effective learning environments for diverse groups of students} (\citealt{Ainscow2008, Day2010-1, Florian2017-1}). In the Icelandic national curriculum, there is a strong focus on ensuring that all children have access to meaningful learning in schools. It emphasises that diversity and varied abilities of students are respected and any form of discrimination and exclusion in schools is unacceptable (\citealt{Ministry_of_education_science_and_culture2011}). An audit of the Icelandic system of inclusive education reported a solid foundation in legislation and policy, which is in line with relevant international conventions (\citealt{European_agency_for_special_needs_and_inclusive_education2017}). According to the audit, education legislation in Iceland builds on a broad conceptualisation of inclusion with a focus on justice and equity. Schools are advised to create spaces for all students to participate, enjoy learning and make progress. However, the auditors concluded that the legislation has not been well implemented. Many teachers and administrators also question whether teacher education and professional development opportunities in Iceland have enabled schools to implement inclusive education for all learners (\citealt{European_agency_for_special_needs_and_inclusive_education2017}).
This is corroborated by other studies from Iceland and the UK, where pre-service teachers report a lack of relevant courses in the study programme and too little focus on practical pedagogy aimed at teaching multilingual students (\citealt{FlocktonCunningham2021, chapters/2_Gunnthorsdottira}).

To meet the demands of increasing diversity in schools and classrooms, teacher education needs to place emphasis on inclusive and multicultural education, together with the teaching of additional languages (\citealt{Banks2016,Ladson-billings1995-multicultural}). Teacher education must attend to this focus in two different ways, in the content area (what to teach) and through pedagogy (how to teach) (\citealt{European_agency_for_special_needs_and_inclusive_education2011}). With these two aspects in mind, teacher educators must consider how they can best prepare pre- and in-service teachers to work in inclusive schools and classrooms with diverse groups of students.

In this chapter, we will present and discuss the findings of a research study conducted at the University of Iceland’s School of Education whose purpose was to explore teacher educators’ perceptions of preparing pre- and in-service teachers for working with students with multicultural and multilingual backgrounds.

\section{Conceptual framework}
\begin{sloppypar}
The conceptual framework for this chapter centres around teacher education pedagogy, as well as insights from multicultural and multilingual education. These perspectives afford us with the necessary theoretical foundation to explore teacher educators’ perceptions of preparing pre- and in-service teachers for working with students with a multicultural and multilingual background. In the following, we first outline some key characteristics of teacher education, before we briefly introduce the benefits associated with a multicultural and multilingual approach to education. 
\end{sloppypar}

\subsection{Teacher education pedagogy}

Developing a pedagogy of teacher education involves more than simply delivering information about teaching and subject content (\citealt{Loughran2005}). The pedagogy of teacher education is about understanding the complex interplay between human, material, and non-tangible elements (\citealt{Hordvik2020-1}){. It is about understanding the relationship between teacher educator and pre-service teacher, between pre-service teachers and subject matter, and finally between context and content} (\citealt{Loughran2013}){.} {It encompasses understanding how learning influences teaching and the ways teacher educators teach. Each teacher educator brings multiple aspects to their teaching, such as personal practical knowledge, perspectives, perceptions, and expectations} (\citealt{Russell2007}){. Each university classroom is therefore influenced by the knowledge, experience, and beliefs of both the teacher educator and the pre- and in-service teachers. The learning material, discourse, traditions, and university rules also affect the teaching} (\citealt{Loughran2014}){.}

{All teachers, including teacher educators, deal with complex realities in their everyday practice. Therefore, the ability to critically reflect on their practice and to develop an understanding by questioning and systematically exploring theories of teaching and learning is important} (\citealt{Loughran2002,Watts2009-1}){. This ability to analyse and make meaning from personal experience is crucial for the development of professional knowledge. Reflective learning involving various epistemological challenges, including reasoning and sense-making, is widely used in teacher education programmes} (\citealt{Russell2017-1}){. Reflection, however, can be complex, and pre- and in-service teachers need to be guided through their cognitive thinking, have a space where they can reflect on their learning and develop their work through continuous feedback} (\citealt{Korthagen2010-1}){. Along with their emotions and personal needs, pre- and in-service teachers also need to explore how their mission influences their professional identities, their behaviour, and the competences they develop for teaching in different environments} (\citealt{Korthagen2010-1}){.}

{Teaching diverse groups of students integrates professional knowledge about teaching and learning and involves an ethical and social commitment to students. The pedagogy of responsive teaching builds on pedagogical knowledge, understanding and skills of the teacher. It centres on students’ responses, fostering flexibility and setting clear goals} (\citealt{Dozier2005-1}){. The pedagogy of responsive teaching facilitates differentiation among students, contexts, methods, materials, and resources. These pedagogical qualities are witnessed in teachers who understand individual differences, are committed to the education of all students, and have a knowledge base which enables them to differentiate between students. Responsive teaching is student-centred learning where diversity is the norm} (\citealt{Gujonsdottir2003-1,Gujonsdottir2016-1}){. In the current study, these aspects of  teacher education pedagogy help us gain a better understanding of the complex interplay between all actors and elements of the university classroom.}

\subsection{Multicultural education}

{In response to the increase in the student population with diverse racial, ethnic, linguistic and cultural backgrounds comes the need to incorporate multicultural education and its principles in teacher education. The aim of multicultural education is to create equal educational opportunities for all students and to remove barriers to educational opportunities. It emphasises that teachers value multilingual students’ experiences and knowledge, build on their strengths and support them in their learning} (\citealt{Gay2018,Nieto2010}){. Multicultural education aims at empowering the school culture through equity pedagogy, content integration, knowledge construction and prejudice reduction} (\citealt{Banks2016}){. Dialogue and positive interaction between students and teachers are particularly important in the language learning environment, where students must learn to communicate in an additional language. In multicultural education, it is believed that the way students learn is deeply influenced by their cultural identity} (\citealt{Banks2016}){, and therefore it is important when teaching diverse groups of students that the educational approaches value and recognise students’ cultural and linguistic backgrounds} (\citealt{Gay2018}){.}

{Culturally responsive teaching is such an educational approach. It refers to teaching methods that build on cultural knowledge, experiences, and learning styles of diverse groups of students} (\citealt{Ladson-billings1995-but}){. The method empowers students by drawing on their linguistic and cultural resources and enables them to become more successful learners by increasing their self-confidence and willingness to participate} (\citealt{Gay2018,Ljunggren2016,Prediger2023-1}){. Student-oriented and flexible methods of teaching and assessment help to improve the educational experience of all students} (\citealt{Al-azawei2017}){. This approach to teaching and learning strives to give all students equal opportunities to learn and succeed and therefore it is consistent with multicultural education. In the study at hand, we are interested in what teacher educators say about the incorporation of elements of multicultural education into their own teaching practice and whether they mention teaching pre-service teachers about multicultural education.}

\subsection{Multilingualism and additional language teaching}
\begin{sloppypar}
The benefits of multilingualism for individuals and societies have been researched and discussed by many scholars (\citealt{Chumak-Horbatsch2012,Cummins2000,Garcia2010-1,Garcia2014-4}). There is consensus among researchers that multilingualism is a valuable resource that should be recognised and nurtured as it paves the way for broad perspectives, {tolerance, and active participation in a democratic society} (\citealt{Garcia2014-3,Ljunggren2016}){.}
\end{sloppypar}

\begin{sloppypar}
\citet{Cummins2000} argues that schools and educational communities should invest in social justice and acknowledge how policy making, attitudes, beliefs and expectations exclude some children while welcoming others. Schools should strive to create learning environments that respond to the needs of multilingual students and their families and develop ways to implement inclusive and socially just practices which acknowledge multilingual students’ backgrounds, previous experience and knowledge. Research in Nordic countries has shown how a focus on inclusion, social justice and equity in schools promotes active participation of all students and their families (\citealt{LefeverEtAl2018,RagnarsdottirKulbrandstad2018,Tran2016-1,Ragnarsdottir2014-1}).
\end{sloppypar}

A common factor shared by many students with a migrant background is that of learning an additional language, typically the majority or school language. These students have often grown up in a multilingual environment and bring that knowledge with them when they enter school. \citet[18]{Prediger2023-1} found that “language-responsive instructional approaches can enhance the learning of both monolingual and multilingual students (newly immigrated and resident students), and students with all degrees of language proficiency”.

All schools have language policies, but they are often not clearly articulated. As pointed out by \citet[7]{CumminsEarly2015}, school language policies should address two aspects: that multilingual students extend their knowledge of the school language while learning academic content, and secondly, that “linguistic and cultural diversity are positioned within the school not as problems to be resolved but as instructional assets”.

Generally speaking, language{ teaching in Iceland has focused on foreign language learning, where exposure to the languages being taught (such as Danish) is limited and language learning takes place mainly in the classroom. On the other hand, Icelandic is taught as a mother tongue or the majority language, which differs greatly from teaching Icelandic as an additional (or second) language.  Being able to communicate in the majority language is crucial for the social participation and educational attainment of multilingual students. Research has shown that limited language skills, lack of support and inadequate language teaching for multilingual students can impact negatively on their learning outcomes and sense of belonging in the learning environment} (\citealt{SinacoreLerner2013}){.}

\citet{CumminsEarly2015} explain that students who are learning the majority (or school) language as an additional language are dealing with two types of language ability, conversational skills and academic language skills. Most students will acquire proficiency in conversational language within one or two years of exposure to the language. However, it typically takes from 5 to 10 years for students to acquire the academic language proficiency necessary for successful school study (\citealt{Wong_fillmore2000-2}). \citet{Lucas2008-3} have identified important principles for teaching multilingual students. One principle is to provide multilingual students with ample opportunities for comprehensible input and output in the additional language. Active use of the language (both oral and written) deepens students’ awareness of language forms and functions and is essential for language acquisition. Another principle is the recognition of multilingual students’ first language as an important resource for additional language learning. It is a valuable tool for thinking and communication, and cross-lingual transfer between the first language and additional languages enriches both languages. Thirdly, \citeauthor{Lucas2008-3} point out how multilingual students are often prone to school-related anxiety. They face difficulties in expressing themselves due to their limited language proficiency in the school language and are not able to demonstrate their knowledge, interests, or personalities easily. This may distract students from focusing on language learning and inhibit their social interaction, which is important for language development. Therefore, providing a welcoming and supportive classroom is an important principle for ensuring multilingual students’ well-being and sense of belonging.  In summary, a multitude of factors influence multilingual students’ additional language learning. While additional language teaching should aim at developing both communicative skills and academic language, it must also take into account the individual needs of multilingual students, that can vary greatly due to linguistic background, age, personality and home environment. It is immensely important that teachers (and teacher educators) understand the impact that multilingualism and language learning have on students’ academic and social development (\citealt{FlocktonCunningham2021,PaulsrudEtAl2023}).

\section{Research methodology}

The aim of this qualitative study was to develop an understanding of how teacher educators at the School of Education, University of Iceland, prepare pre- and in-service teachers for teaching in linguistically diverse classrooms. The educational and professional development of both pre-service and in-service teachers in Iceland are primarily in the hands of teacher educators from two tertiary-level institutions; the University of Iceland and the University of Akureyri. Participants in this study, teacher educators at the University of Iceland, were chosen because of their expertise in general pedagogy, pedagogy of subject teaching, and inclusive education. It was important to us to collect data from the perspectives of the teacher educators, to learn how they address issues of diversity and multilingualism in their teaching. Thus, the main research question guiding our study was: What are teacher educators’ perceptions of preparing pre- and in-service teachers for working with students with multicultural and multilingual backgrounds?

The School of Education is divided into four faculties responsible for the education of teachers: the Faculty of Education and Pedagogy, the Faculty of Health Promotion, Sport and Leisure Studies, the Faculty of Subject Teacher Education and the Faculty of Education and Diversity. The teacher education programmes are mostly situated in the first three faculties mentioned, and they all offer five-year study programmes that end with a master’s degree in education which is required to receive teacher certification. Since initial teacher education was extended from a three-year bachelor’s degree to a master’s degree in 2008, many in-service teachers have come back to obtain a master’s degree in education. In addition, individuals who have received a bachelor’s degree in fields other than education can complete a two-year teacher certification programme at the master’s level. Thus, the teacher education programmes at the School of Education include bachelor and master level programmes, and the programmes at the master’s level are attended by both pre-service and in-service teachers. 

The research study was carried out by four teacher educators, the authors of this chapter. Twenty-eight teacher educators participated in the study; all faculty members in the School of Education. Ten of the participants were specialists in teacher pedagogy at preschool and compulsory school levels, and four were specialists in special education or inclusive education. Ten participants were members of the Faculty of Subject Teacher Education and were experts in the teaching of mathematics, natural sciences, social sciences, foreign languages, Icelandic, arts and crafts or ICT. The remaining participants were experts in the fields of upper-secondary education, physical education, literacy or assessment methods. All but two of the participants had many years of experience as teacher educators.

Data were collected through 13 focus group discussions with up to four participants in each group. The focus groups were organised according to teaching area and school level. For example, specialists in language teaching at the compulsory level were grouped together, specialists in upper-secondary teaching formed another group, and teacher educators with a focus on general pedagogy at the compulsory school level formed a third group. In some cases, there was overlap between school levels such as in a group which consisted of general pedagogy teachers at both pre-school and compulsory school levels. The focus group discussions took place from spring 2019 to spring 2021. The discussions were 30–45 minutes in length and were recorded and transcribed verbatim.

The focus group approach was selected for this study to create a space for dialogue about the participants’ teaching. The interaction within the focus group was of importance because it contributed to the development of ideas and responses within the group. The questions used with the participants were open-ended and non-directive since our intention was to give the participants an opportunity to share their personal experience of how they prepare pre- and in-service teachers to work with multilingual students (\citealt{Bryman2004,EatoughSmith2017}).

The teacher educators were given the focus questions beforehand and were asked to discuss how they prepare pre- and in-service teachers for teaching diverse groups of students. The four researchers, individually, had the role of moderating 2-4 focus group discussions each. As moderators we posed questions to the group, invited participants to speak, managed time and the recording device. In addition, each researcher participated as a member of one focus group discussion. As researchers, we were aware of potential advantages and disadvantages of our dual role as both moderators and participants. Our aim was to explore our own perspectives and practices as well as those of our colleagues. As participants, we could take an active part in the focus group discussion with our colleagues in a safe environment. The discussions gave us an opportunity to reflect, share and develop ideas jointly with other specialists. Furthermore, all the participants expressed their gratitude for receiving the opportunity to reflect on and discuss their professional beliefs and practices with other specialists. Nevertheless, there could be a possibility that participants did not wish to discuss aspects openly due to work relationships. Although there was no indication of this in the data, this should be considered one of the possible limitations of the current study.

In analysing the data, we applied an interpretive phenomenological analysis (IPA) approach, beginning by identifying essential elements and relationships related to the aim of the research. The coding was selective, and the focus was on teacher educators’ descriptions of the pedagogical approaches and practices they used to prepare their students for teaching in linguistically diverse classrooms (\citealt{EatoughSmith2017}). As the analysis developed, we catalogued codes, identified patterns, and grouped them according to the conceptual framework of teacher education pedagogy, multicultural education, and multilingualism and additional language teaching.

The next step was to go through each theme and summarise it. We then structured and supported the themes by bringing in evidence from the data in the form of participant quotes. Next, we wrote a description with verbatim examples and a structural description where we reflected on what we learned from analysis of the focus group discussions (\citealt{BraunClarke2013,Wolcott2001}). Finally, we wrote a composite description of our findings where the voices of participants were in the forefront (\citealt{Creswell2008}). The focus group meetings were all conducted in Icelandic and therefore the transcripts were also in Icelandic. We, the researchers, translated the quotes that are presented in the findings.

All ethical standards for research prescribed by the University of Iceland were followed in the study, including the Act on personal data protection and processing of personal data\footnote{\url{https://www.personuvernd.is/media/uncategorized/Act_No_90_2018_on_Data_Protection_and_the_Processing_of_Personal_Data.pdf}} and the \citegen{UniversityofIceland2019} Code of Ethics. Pseudonyms are used to protect the identity of the participants.

\section{Findings}

The following themes were constructed during the analysis process: teacher education pedagogy, multicultural education, and multilingualism and additional language teaching. The participant quotes presented here are representative of the teacher educators’ views and perceptions of how they prepare pre- and in-service teachers for teaching in multilingual classrooms.

\subsection{Teacher education pedagogy}

Teacher education pedagogy was discussed at length; it was important for the teacher educators to be a model for teaching diverse students in inclusive schools, in other words to “practice what they preach” or “walk the talk”:

\begin{quote}
We do not lecture about teaching methods. We teach about them by using them ourselves. Our students get open tasks and need to explore, discuss, collaborate and use all kinds of tools, such as digital tools. They get the chance to reflect on their own understanding of teaching and adapt these methods to their own ideas. We finish the course by summing up the methods we have used to help our students focus on important features of teaching. Our goal is that they will be able to use rich and creative methods in their own teaching. (Rúna)
\end{quote}

The teacher educators recognise the importance of providing pre- and in\hyp service teachers with opportunities to experience first-hand what they are being taught – those pedagogical practices are the core of teacher education:

\begin{quote}
We are faced with the challenges of having diverse groups of students and need to find ways to respond to their needs. By using project-based learning, our students can relate to their own interests and use their findings in  their daily life and work. By sharing their findings with each other, they learn about the quality of having students with diverse backgrounds within the group. (Margrét)
\end{quote}

Some teacher educators gave examples of activities or projects that they implemented in their teaching as a means of enriching students’ own learning experiences:

\begin{quote}
For example, a classroom activity that we do in a course for pre- and in-service teachers is to have them create their own “identity text” and share it with the others. It’s great to see how they tackle the task and see the outcomes. They show a lot of creativity and depth of understanding. (Árni)
\end{quote}

Many of the discussions reflected the teacher educators’ vision that all learners should flourish in school. One example comes from a teacher of a course about inclusive practices:

\begin{quote}
…if one looks at this as if we are preparing people for working in inclusive education, education for all, and dealing with diverse groups of students, then what is perhaps reflected in our teaching is precisely the flexibility in projects. …I think we are working in a way that we would like them [students] to adopt …And the mindset really is that we [students] are all different, and we need different things to be successful. (Sjöfn)
\end{quote}

In courses about information technology, student-oriented learning is strongly reflected:

\begin{quote}
…I emphasise student-oriented learning, students have a choice of projects. We always have a diverse group of students in the information technology course. A strong emphasis is on [appealing to] students’ interests and that the project they are working on is something they find interesting, …something they can take advantage of, and sometimes it's something related to daily life; this interests the student. (Dísa)
\end{quote}

Although the teacher educators come from different disciplines and differ in their pedagogical focus, their theoretical stance was similar. The teacher educators expressed a common belief in inclusive and equitable education. Reflection on their own learning is an integral part of many of the courses, for example in this course for pre-school teachers:

\begin{quote}
In this course about early childhood education, we try to build a platform where our students discuss and reflect together on inclusion and how we can support children with diverse needs, such as children with multicultural backgrounds, to participate in all pre-school activities. (Auður)
\end{quote}

This emphasis on reflection opens the way for discussions in class that help the pre- and in-service teachers to better understand how to respond to children with diverse backgrounds. A teacher educator who teaches a course about inclusive practices also focused on reflection:

\begin{quote}
In one of the course’s assignments our students are urged to analyse something that is at their heart, a challenge they have found difficult to overcome, something that they really want to dig into and become a specialist in. They then introduce their findings to other participants in the course and in that way we develop a learning community together. In these communities our students’ collaborative reflection results in rich funds of knowledge that everyone gains from. (Nanna) 
\end{quote}

The reflection becomes a part of how learning is interpreted in different settings and a source for the pre- and in-service teachers to develop their own beliefs and understanding of teaching and learning. The aim is that, as professionals, they can identify the situations they are working in, and the issues or challenges they face. 

\subsection{Multicultural education} %4.2 /

Although specific emphasis on multicultural education theory was not frequent in the teacher educators’ discourse, they were aware of the necessity of preparing pre- and in-service teachers to teach students with diverse backgrounds and needs. Their strength in using inclusive teaching approaches at university level helped them understand the importance of meeting the needs of multilingual students{.} The discussion in the focus group meetings reflected the teacher educators’ vision to prepare pre- and in-service teachers to work in multilingual classrooms. 

In the programme for physical training instructors, the teacher educators emphasise that their students learn to embrace diversity within schools: 

\begin{quote}
We prepare them to meet the learning needs of all their students. We are still developing our ways of preparing them to teach in multicultural and multilingual settings since this is relatively new in our country. Children who don’t understand each others’ language participate in games and instinctively play together. Through sport activities they learn to communicate through body language and gradually with spoken language. When our students do their practicum they experience working in multicultural settings in some of our schools. (Ösp)
\end{quote}

An arts educator also mentioned the practicum as an opportunity for students to experience working in multicultural settings:

\begin{quote}
It depends on which school they go to. In some schools there are many children of foreign background. What visual arts and sports have in common is that everyone can participate. I introduce my students to the work of artists who emphasise multiculturalism and urge them to use that approach in their own work. (Hulda)
\end{quote}

The teacher of a mathematics course emphasised an inquiry approach and open tasks for pre- and in-service teachers. {The intention in this course was to make learning appropriate for diverse learners, by building on their resources and recognising both their cultural and linguistic backgrounds:}

\begin{quote}
…these are people who are learning to become teachers. They will meet all kinds of children and they need to be able to respond to and work with a diverse group of learners. I organise my teaching so that all my students feel at home, and everyone can cope with the challenges I’m presenting. (Kristín)
\end{quote}

A newly hired teacher educator talked about her own lack of knowledge with regard to working with linguistically diverse classrooms:

\begin{quote}
\sloppy
The area of students with foreign backgrounds is my weakness. Although it’s something that is there subconsciously, I feel like I need more information. So, I’ve been looking for information about ways to prepare pre-service teachers for working with students with diverse cultural backgrounds. (Jóna)
\end{quote}

The quotes from the teacher educators show an emerging awareness of responding to the growing diversity of languages spoken in Icelandic schools and finding ways to effectively prepare pre- and in-service teachers to work in multicultural classrooms. The teacher educators {critically reflect on their own practice and work towards developing their professional knowledge and teaching practices.}

\subsection{Multilingualism and additional language teaching}

Some of the teacher educators talked about the need for more knowledge and a greater understanding of multilingualism and additional language teaching. They acknowledged that students need to re-examine their views towards language learning and language use in the classroom and gain an understanding of how learning Icelandic as an additional language differs from learning Icelandic as a mother tongue. A teacher educator who teaches Icelandic grammar expressed: 

\begin{quote}
[We need to] help pre- and in-service teachers take a wider look at language, both to increase their tolerance of Icelandic spoken with an accent, and to realise that there is not just one type of Icelandic, that there are many types. I put a lot of effort into these things in order to prepare them for teaching students with diverse backgrounds. (Þórunn)
\end{quote}

In a course about reading instruction, the learning of children with a mother tongue other than Icelandic is discussed: 

\begin{quote}
We are aware of the relation between language development and learning to read. We need to attend to the children’s social communication. The learning of language is strongly connected to communication. The multilingual children need to talk to the Icelandic children because that is how they learn the language. (Anna)
\end{quote}

This teacher educator emphasises that children who are learning an additional language need to practice the new language but does not acknowledge that multilingual students’ first language is an important resource for additional language learning. This seems to embody a monolingual viewpoint which puts emphasis on the majority language and fails to recognise knowledge of other languages as an asset.  

In the pre-school teacher education programme, multilingualism is addressed in a course about language development. The teacher educator expressed the necessity of a compulsory course about multilingualism for all pre- and in-service teachers:

\begin{quote}
In my opinion, all pre-school teachers should be required to take a course on multilingualism and Icelandic as an additional language. It is so important because there are so many children with foreign backgrounds in our pre-schools. (Vala)
\end{quote}

A teacher educator who teaches courses about language teaching methods and language learning expressed similar concerns:

\begin{quote}
I would like to see more emphasis put on learning about language teaching methods and language learning in our teacher education programme, including Icelandic as an additional language, so that everyone has some insight into what it means to learn a new language and how language is so important for communication at school and with parents and others. (Árni)
\end{quote}

\begin{quote}
Additional language teaching focuses on a variety of ways to teach vocabulary and communicative skills. A lot of emphasis is placed on communication, that students learn how to express themselves in the new language. Less emphasis is put on learning grammar, especially with younger learners, instead the focus is on how we use the language to communicate with others and understand our surroundings. (Árni)
\end{quote}

In summary, the findings of the study show that the teacher educators do not always place specific emphasis on multicultural education or additional language learning, but attempt to prepare pre- and in-service teachers for teaching in multilingual classrooms by organising their courses in ways which consider students’ diverse backgrounds and needs. They utilise a variety of approaches and practices in their teaching which are learner-centred and inclusive, thus creating learning spaces built on students' resources (experiences, strengths, interests). They model effective teaching practices and provide students with opportunities to broaden their experience and understanding of working with diverse groups of students.

\section{Discussion and conclusions}

The purpose of the research was to develop an understanding of how teacher educators at the School of Education at the University of Iceland perceive the preparation of pre- and in-service teachers for working with students with multicultural and multilingual backgrounds. {Similar to findings from other contexts, the teacher educators in the current study are conscious of the diverse student population in Icelandic schools at all levels and admit that they need to consider that in their teaching at the School of Education} (\citealt{FlocktonCunningham2021,PaulsrudEtAl2023}){. They are aware that the increase in the number of} multilingual students calls for changes in classroom practices and teacher education (\citealt{Day2010-1}). The findings suggest that a specific focus on multicultural education in the teacher education programme is lacking and depends mostly on the interest and knowledge of individual teacher educators (e.g. \citealt{PaulsrudEtAl2023}). On the other hand, the teacher educators are aware of the necessity of preparing pre- and in-service teachers to teach students with diverse linguistic backgrounds, and several of the teacher educators feel that more emphasis is needed on teaching Icelandic as an additional language.

\begin{sloppypar}
Developing a pedagogy of teacher education is more than simply delivering information about teaching and subject content (\citealt{Loughran2005}); it also involves an understanding of the complex interplay between human, material, and non-tangible elements (\citealt{HordvikEtAl2020}). The pedagogy of teacher education stresses that it is important for teacher educators to serve as models for teaching diverse students, in other words to “practice what they preach” or “walk the talk”.
\end{sloppypar}

{Another important factor of teacher education is the ability to critically reflect on one’s own practice} (\citealt{Loughran2002,WattsLawson2009}){. Data from our study provided examples of teacher educators’ critical reflection and their efforts to improve their professional development. Reflection on learning is an integral part of many courses taught by the teacher educators, and they emphasised providing pre- and} in-service teachers{ with opportunities to reflect on their own learning. Such reflections can enable pre- and} in-service teachers{ to develop their own beliefs and understanding of teaching in linguistically diverse classrooms. Being able to analyse and make meaning from one's own experience is crucial for the development of professional knowledge} (\citealt{RussellMartin2017}){.}

Teaching approaches and practices which are learner-centred and inclusive are discussed widely in the focus groups. The teacher educators put emphasis on creating learning spaces which build on students’ resources and give them opportunities to learn how to work with diverse groups of students. However, there is little indication of teacher educators working specifically with theories of multilingualism or additional language learning. {There is no discussion of the importance of} taking advantage of the whole linguistic repertoire of students and the positive impact it can have on students’ self-esteem and well-being. In fact, in some cases there seems to be tension between monolingual and multilingual approaches.

Inadequate language teaching and lack of support for multilingual students can have a negative impact on their academic progress and sense of belonging (\citealt{SinacoreLerner2013}). \citet{CumminsEarly2015} and \citet{LucasEtAl2008} stress the importance of creating learning environments that recognise multilingual students’ language knowledge as an asset, not as a deficit. Our findings indicate that knowledge and understanding of multilingualism and additional language teaching is not widespread among the teacher educators and was only dealt with in a few courses. To overcome this deficit, it is essential that teacher education provides pre- and in-service teachers with knowledge about additional language teaching that will help multilingual students to develop language skills necessary for positive learning outcomes and social participation in school and society (\citealt{LucasEtAl2008}). At the same time, it has to be acknowledged that multilingual students’ needs vary greatly according to both background and environment.

The findings from this study shed light on both strengths and weaknesses within the group of teacher educators regarding teaching in multilingual classrooms. The teacher educators are specialists in teacher education pedagogy but are lacking in knowledge about multicultural and multilingual education{. The need for strengthening the multicultural and multilingual focus of the teacher education programmes is clear.}

{The outcomes of the study can help teacher educators understand how diversity in student population impacts on initial teacher education. They shed light on the importance of integrating both theory and practice in teaching and in the development of educational programmes whose aims are to prepare pre- and in-service teachers for working in multilingual classrooms. Culturally responsive teacher education which recognises and responds to the abilities and needs of multilingual students is essential for ensuring social justice and equity for all in today’s schools.}

\sloppy\printbibliography[heading=subbibliography,notkeyword=this]
\end{document}
