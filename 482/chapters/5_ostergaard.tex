\documentclass[output=paper]{langscibook}
\ChapterDOI{10.5281/zenodo.15280901}
\author{Winnie Østergaard\orcid{}\affiliation{VIA University College} and Anna-Vera Meidell Sigsgaard\orcid{}\affiliation{University College of Copenhagen} and Thomas Roed Heiden\orcid{}\affiliation{UCL University College ; University of Aarhus} and Lone Wulff\orcid{}\affiliation{University College of Copenhagen} and Christine Worm\orcid{}\affiliation{University College of Northern Denmark}}
\title[Multilingualism in Danish pre-service teachers’ writing assignments]{Multilingualism in Danish pre-service teachers’ writing assignments: Between theory and practice}
\abstract{The study is conducted in the context of Danish teacher education, and explores pre-service teachers’ reported knowledge about multilingualism and multilingual pedagogy. Through analysis of successful pre-service teachers’ examination papers, we explore how well pre-service teachers demonstrate their knowledge about multilingualism. Using semantic gravity analysis, we also explore how they describe implementing this knowledge in the context of supporting multilingual students’ learning in mainstream classrooms. This analysis provides a useful tool for reflecting on pedagogic and assessment practices for teacher educators and other professional degree educators. Our analyses show that pre-service teachers value multilingualism but have difficulty demonstrating in writing how to implement multilingual pedagogies. Investigating the degree to which pre-service teachers can demonstrate their understandings of theoretical concepts has broader implications  for both teacher educators and educators of other professions, in terms of how we assess tertiary students’ understandings of theory, and how they describe implementing theory in their future practice.}
\IfFileExists{../localcommands.tex}{
  \addbibresource{../localbibliography.bib}
  % add all extra packages you need to load to this file

\usepackage{tabularx,multicol}
\usepackage{url}
\urlstyle{same}

\usepackage{listings}
\lstset{basicstyle=\ttfamily,tabsize=2,breaklines=true}

\usepackage{langsci-basic}
\usepackage{langsci-optional}
\usepackage{langsci-lgr}
\usepackage{langsci-osl}
% \usepackage{./langsci/styles/langsci-lgr}
% \usepackage{./langsci/styles/langsci-osl}
% \usepackage{langsci-gb4e}

\usepackage{tikz}
\usetikzlibrary{patterns,calc}
\pgfdeclarepatternformonly{south east lines}{\pgfqpoint{-0pt}{-0pt}}{\pgfqpoint{3pt}{3pt}}{\pgfqpoint{3pt}{3pt}}{
    \pgfsetlinewidth{0.6pt}
    \pgfpathmoveto{\pgfqpoint{0pt}{3pt}}
    \pgfpathlineto{\pgfqpoint{3pt}{0pt}}
    \pgfpathmoveto{\pgfqpoint{.2pt}{-.2pt}}
    \pgfpathlineto{\pgfqpoint{-.2pt}{.2pt}}
    \pgfpathmoveto{\pgfqpoint{3.2pt}{2.8pt}}
    \pgfpathlineto{\pgfqpoint{2.8pt}{3.2pt}}
    \pgfusepath{stroke}}
    
\usepackage{stmaryrd}
\usepackage{wasysym}
\usepackage{multirow}
\usepackage{caption}
\usepackage{subcaption}
\usepackage{mathrsfs}
\usepackage{qtree}

\usepackage{linguex}


  %pminos do not split footnotes
% \interfootnotelinepenalty=10000 %Footnote in Laporte chapters has to be split SN


%\DeclareIndexNameFormat{default}{%
%\nameparts{#1}%
%\usebibmacro{index:name}%
%{\index[names]}%
%{\namepartfamily}%
%{\namepartgiveni}%
% {}% L1
% {}% L2
%{\namepartprefix}% generates spurious space L3
%{\namepartsuffix}% generates spurious space L4
%}

%  {\DeclareIndexNameFormat{default}{%
%     \usebibmacro{index:name}{\index[names]}{#1}{#3}{#5}{#7}}}

%\DeclareIndexNameFormat{default}{%
%  \usebibmacro{index:name}{\sindex[nom]}{#1}{#3}{#5}{#7}}

%\DeclareIndexNameFormat{default}{%
%  \usebibmacro{index:name}{\sindex[person]}{#1}{#3}{#5}{#7}}
%\DeclareIndexNameFormat{default}{%
%\nameparts{#1} \usebibmacro{index:name}{\sindex[person]]}{\namepartfamily}{‌​\namepartgiven}{\nam‌​epartprefix}{\namepa‌​rtsuffix}}

%\newcommand{\smiley}{:)}

%\renewbibmacro*{index:name}[5]{%
%\usebibmacro{index:entry}{#1}%
%{\iffieldundef{usera}{}{\thefield{usera}\actualoperator}\mkbibindexname{#2}{#3}{#4}{#5}}}

% \newcommand{\noop}[1]{}

%remove for final
%\overfullrule=1mm

\newcommand{\tobi}[2]}}
\renewcommand{\S}[1]{\tobi{#1}{\textsc{*}}}

% this volume references
% puts: [this volume]
% already defined: \citetv
%\newcommand{\citepv}[1]{(\citeauthor{#1} \citeyear*{#1} [this volume])}
\newcommand{\citealtv}[1]{\citeauthor{#1} \citeyear*{#1} [this volume]}

%parentheses around example number
\newcommand{\pref}[1]{(\ref{#1})}

% in-text examples

\newcommand{\lnex}[1]{\textit{#1}} %target lang word
\newcommand{\lnlit}[1]{(lit.: `#1')} %literal reading
\newcommand{\lnlat}[1]{(#1)} % latinization
\newcommand{\lntrans}[1]{`#1'} %translation
\newcommand{\lnexl}[2]%
{\lnex{#1}{} \lnlat{#2}} % ex with latinization
\newcommand{\lnexlat}[3]{\lnex{#1}{} \lnlat{#2}{} \lntrans{#3}} % ex with latinization and tranl.

%ch01
\newcommand{\co}[1]{\mbox{\textbf{#1}}}

%ch09

\newcommand{\cyrbulg}[1]{\begin{otherlanguage*}{bulgarian}#1\end{otherlanguage*}}


%ch10
\newcommand{\nlp}{{\small NLP}}
\newcommand{\mwe}{{\small MWE}}
\newcommand{\rae}{{\small RAE}}
\newcommand{\lvc}{{\small LVC}}
\newcommand{\pos}{{\small P}o{\small S}}
%\newcommand{\todo}[1]{ \textcolor{red}{#1} }

%\renewcommand{\labelenumi}{\theenumi}
%\ainamefmt{{vv}{ll}{, ff}{, jj}} % fullname

\newcommand{\biberror}[1]{{\color{red}#1}}

\newcommand{\osenovaitem}{--~} 
  %% hyphenation points for line breaks
%% Normally, automatic hyphenation in LaTeX is very good
%% If a word is mis-hyphenated, add it to this file
%%
%% add information to TeX file before \begin{document} with:
%% %% hyphenation points for line breaks
%% Normally, automatic hyphenation in LaTeX is very good
%% If a word is mis-hyphenated, add it to this file
%%
%% add information to TeX file before \begin{document} with:
%% %% hyphenation points for line breaks
%% Normally, automatic hyphenation in LaTeX is very good
%% If a word is mis-hyphenated, add it to this file
%%
%% add information to TeX file before \begin{document} with:
%% \include{localhyphenation}
\hyphenation{
    Beck-man
    Ngu-yen
    back-chan-nel
    back-chan-nels
    mo-not-o-nous
    ste-reo-typ-i-cal
}

\hyphenation{
    Beck-man
    Ngu-yen
    back-chan-nel
    back-chan-nels
    mo-not-o-nous
    ste-reo-typ-i-cal
}

\hyphenation{
    Beck-man
    Ngu-yen
    back-chan-nel
    back-chan-nels
    mo-not-o-nous
    ste-reo-typ-i-cal
}
 
  \togglepaper[1]%%chapternumber
}{}

\begin{document}
\maketitle 
\lehead{Winnie Østergaard et al.}
\label{chap:5}
%\shorttitlerunninghead{}%%use this for an abridged title in the page headers
\newpage

\section{Introduction}

Teaching in any setting is a multifaceted and complex endeavour. When preparing pre-service teachers for their profession, their degree programme requires them to adopt changes in perceptions of phenomena in teaching and learning, as well as to gain knowledge about situations on which they often have intuitive views. Social discourses, political debates, and personal experiences all inform the pre-existing knowledge that pre-service teachers bring to their educational settings. The primary aim of this chapter is to explore how pre-service teachers demonstrate understandings of one such topic in their examination papers. The particularly politicised and debated topic explored here is supporting multilingual students in mainstream education.

Everyday opinions on educational topics are different from professional and research-based knowledge in the field. The everyday understandings of pre-service teachers at the start of their degree programmes are unsurprisingly influenced by personal school experiences and public discourse. One role of teacher education is therefore to support pre-service teachers’ transition from intuitive to theoretically founded understandings. This could support future teachers in making informed and theoretically founded decisions in their practice (\citealt{Rusznyak2021,TilakaratnaSzenes2021}). In Denmark, strengthening pre-service teachers’ connection to pedagogical practice \citep{Rasmussen2022} as a way of meeting the well-known challenge pre-service teachers have in combining theory and practice  has received attention in recent years (\citealt{Cochran-SmithZeichner2005}). This study, therefore, can shed light on some of the difficulties pre-service teachers experience in what is often referred to as the theory-practice divide \citep{Gravett2012}.

In this chapter, we explore pre-service teachers’ development of knowledge about multilingualism and multilingual students, as demonstrated in examination papers written for a course (\textit{Teaching bilingual students,} see \sectref{sec:ostergaard:2.2}) in Danish teacher education programmes. Specifically, we examine how pre-service teachers move from intuitive to more theoretical understandings in the field of additional language education and how they connect this theoretical knowledge to classroom practice, by analysing pre-service teachers’ examination papers. We are interested in exploring if (and if so, how) pre-service teachers view students’ multilingualism as a resource for learning (\citealt{Garcia2017-1}), as opposed to the more commonly held, uncritical viewpoint, in which multilingualism is seen as an impediment to learning. After completing a course on teaching multilingual students in mainstream classrooms, all pre-service teachers in Denmark must write an examination paper in which they analyse pedagogic practice from an additional language perspective and provide suggestions for pedagogic practice that will support multilingual students’ learning in mainstream classrooms. In this examination paper, pre-service teachers must therefore apply theories on multilingualism to the context of their teaching subjects, to demonstrate their ability to support multilingual students’ learning. Doing so requires them to make connections between theory and practice. In this chapter, therefore, we explore how pre-service teachers demonstrate their knowledge about multilingualism in their written examination papers by analysing how they describe implementation of theories on multilingualism into pedagogic practice.

In Denmark, as in other countries, making use of students’ multilingual repertoires in mainstream teaching is commonly seen as a challenge, despite research pointing to the advantages of multilingual approaches (\citealt{Dewilde2020,Laursen2019,PaulsrudEtAl2017}).  For example, although an established perspective in additional and foreign language education internationally, the concept of \textit{translanguaging} (\citealt{Garcia2017-1})  has only recently become a topic of interest among researchers in Denmark. The publication of the first (and as of yet, only) Danish-language book on translanguaging in educational practice (\citealt{HolmenThise2021}) has however facilitated the inclusion of multilingual approaches in both in\hyp service and pre-service teaching degree programmes. This is also the case in the obligatory course for all pre-service teachers, called \textit{Teaching bilingual students}, which we explore in this chapter.

The learning outcomes and assessment criteria of the \textit{Teaching bilingual students} course are informed by a positive and \textit{additive} perspective on multilingualism and multilingual students, in contrast to the more \textit{subtractive} perspective \citep{Holmen2019} apparent in the Danish Public School Act. How pre-service teachers resolve this seeming paradox, and how well they demonstrate their knowledge about multilingualism, can potentially influence pre-service teachers’ pedagogic practices significantly, motivating our investigation. Learning more about how students in a professional degree programme take up new and developing perspectives in associated research fields has implications both locally, i.e. in structuring and restructuring related degree programmes and curriculum content, but also more generally in terms of how knowledge from research fields is recontextualised to a professional degree programme.

An underlying assumption in professional degree programmes, such as the one in focus here, is that engagement with theory will inform and improve future practice (\citealt{ShayClarence-Fincham2016}). However, pre-service teachers often have difficulty writing about pedagogic practice in a theoretically informed way (\citealt{NielsenPaulsen2006}). This chapter builds on earlier work by \citet{Meidell_sigsgaard2021} that also explored pre-service teachers’ examination papers for the \textit{Teaching bilingual students} course, comparing how successful\footnote{\textrm{Successful papers in \citet{Meidell_sigsgaard2021} were papers achieving a passing grade of average or above. We apply the same criteria for a successful paper in the present study.}} and less successful examination papers respectively connected theory with practice. That study did not reflect on which of the knowledge and competency areas pre-service teachers chose to write about, considering instead students’ comprehension of relevant theories in general. For the purpose of this study, we are specifically interested in exploring how pre-service teachers demonstrate knowledge about theories of multilingualism. This can provide insights into tendencies in the field of multilingual and additional language education as they develop in the context of teaching multilingual and additional language learners in mainstream classes in Denmark. The specific research questions for this chapter are therefore: 

\begin{itemize}
\item  How do pre-service teachers demonstrate understandings of multilingualism in their written examination papers for the obligatory teacher education course, \textit{Teaching bilingual students}, in Danish teacher education?
\item  To what degree are they able to link theoretical understandings with examples of pedagogic practice?
\end{itemize}

Analysing pre-service teachers’ reported ability to recontextualise theoretical knowledge into specific contexts of pedagogic practice can also provide more general insights from an assessment perspective. In tertiary education fields such as teacher education, as well as many other fields, it is relevant to investigate how well students demonstrate connections between their observations of practice and theoretical concepts and how well they are able to recontextualise theoretical concepts to new practice situations (\citealt{Berry2008,RusznyakEtAl2021}). It is in doing both that they convincingly can demonstrate their professionality. How pre-service teachers do so with presumably new understandings of multilingualism is the focus of this chapter, and it provides broader insights for reflecting on, working with, and assessing students’ knowledge about both theory and practice.

\section{Background}

To set the context for our study we first map connections between politics, policy and practice \citep{Biesta2011}, focusing on the terms \textit{bilingualism} and \textit{multilingualism} in Danish school-policy documents. Investigating these uncovers current theories on multilingualism in existing research in both Denmark and in the Nordic countries and underpins the specific context of this study: assessment for the course \textit{Teaching bilingual students}.

\subsection{Multilingualism in Danish school-policy documents}
\begin{sloppypar}
The relevant policy documents regulating Danish public schools, the Public School Act, and The Decree of Public School Teaching in Danish as a Second Language do not include the terms \textit{multilingualism}, \textit{linguistically diverse classrooms} or \textit{multilingual practices}\slash\textit{resources}. Instead, the Public School Act uses the term \textit{tosproget} [bilingual],\footnote{All Danish language citations included in this chapter have been translated into English by this chapter’s authors. This includes both quotes from curriculum documents/mandates and from the analysed pre-service examination papers.} defining the relevant students as “children who have a different mother tongue than Danish, and who learn Danish only through contact with the surrounding society, perhaps for the first time through formal education”  (\citealt{MinistryofChildrenandEducation2022}).\footnote{This official definition means students growing up with Danish and another language at home are not technically categorised as bilingual.}
\end{sloppypar}

The use of the term \textit{bilingualism} in Danish policy documents is problematised by \citet{Kristjánsdóttir2011}, who points out that these documents enact and represent social practices of discrimination with significant pedagogical consequences. For a student to command more than one language is seen as a challenge for social integration in Danish schools (\citealt{Kristjánsdóttir2010,Kristjánsdóttir2011}). The monolingual language ideology makes it difficult to leverage students’ multilingual backgrounds as a resource within the framework of the public school \citep{Kristjánsdóttir2018}.  Together with political and media discourses on, e.g., immigration, this can result in pre-service teachers expressing scepticism about including “immigrant languages” in a public school setting. \citet{Iversen2021}, \citet{Kulbrandstad2004} and \citet{DaugaardDewilde2017}  challenge these monolingual tendencies, maintaining that multilingual practices support the Nordic educational system’s values of child\hyp centred pedagogies, respectful of diversity.

\subsection{The \textit{Teaching bilingual students} course in Danish teacher education programmes}\label{sec:ostergaard:2.2}

The Danish teaching degree programme is a standardised, nationally regulated, four-year bachelor’s degree programme, offered by the Danish University Colleges. It prepares primary and lower-secondary school teachers, and contains obligatory courses in Teacher-Professional Competency, practicum placement and three teaching subjects (e.g. Arts, Biology, English) chosen by the pre-service teacher. Content and assessments for all the degree programme’s subjects and courses are defined by the Ministry of Higher Education and Science. 

This chapter focuses on the assessment of the obligatory teacher education course, \textit{Teaching bilingual students} (10 ECTS). The aim of the course is to prepare pre-service teachers to meet the needs of multilingual learners in their future classrooms. Upon completing the course, the pre-service teachers submit an examination paper individually or in groups, on a self-chosen topic based on the following national prompt: 

\begin{quote}
In the examination paper, students must analyse a situation from a lesson, a sequence of lessons, and/or a textbook or other teaching materials from one of the students' present or future teaching subjects based on a relevant issue concerning second language teaching. The analysis must lead to suggestions for revised didactic measures as a means of scaffolding bilingual students in the subject or themes in question. (\citealt{MinistryHigherEducationScience2020})
\end{quote}

This question requires pre-service teachers to develop and demonstrate theoretically informed knowledge and how to apply this in practice. Pre-service teachers must demonstrate their understanding of several associated knowledge and competency criteria, e.g., “knowledge of multilingual students’ linguistic practices and resources and being able to use this knowledge in application to pedagogic practice” (\citealt{MinistryHigherEducationScience2020}), within one of their teaching subjects. In requiring pre-service teachers to focus on one of their teaching subjects, this course compels them to focus on the teaching of that subject, incorporating their knowledge about multilingualism in order to meet the learning needs of multilingual learners in mainstream classrooms. How pre-service teachers demonstrate this in their written examination papers is the focus of this chapter and compels a description of the theories on multilingualism (see \sectref{sec:ostergaard:3.1}) underpinning the \textit{Teaching bilingual students} course.

\section{Understanding multilingualism: Bridging theory and practice}

Because the teaching degree programme aims to prepare teachers for the teaching profession, pre-service teachers must demonstrate that they know both relevant content from their course studies and how to apply it in teaching. To explore \textit{how} pre-service teachers demonstrate knowledge about multilingualism in their written exams, we include two different theoretical perspectives. \sectref{sec:ostergaard:3.1} concentrates on current theories on multilingualism from relevant literature in Denmark and internationally. This also serves as the theoretical perspective used in the initial analysis of our data, addressing our first research question. \sectref{sec:ostergaard:3.2} introduces the theoretical concept \textit{semantic gravity} from Legitimation Code Theory (\citealt{Maton2014}) for identifying varying degrees of context-dependence of knowledge, between theory and practice. This analytical perspective is used to illustrate how pre-service teachers operationalise in writing their knowledge about multilingualism in educational practice, and addresses our second research question.

\subsection{Theories on multilingualism}\label{sec:ostergaard:3.1}

Multilingualism can be understood as people’s ability to navigate a complex diversity of languages in social practices (\citealt{Hodge1988,Laursen2019}). Operationally, multilingualism can be understood as knowing and using more than two languages, while at the same time also experiencing a sense of belonging to specific social groups (\citealt{GarciaWei2014}). In educational settings, multilingual approaches can build an important foundation for multilingual learners’ opportunities to make meaning in school and in the world.

\begin{sloppypar}
\citet{Garcia2008} connects the term multilingualism to the term \textit{multilingual awareness}, while at the same time problematising the lack of focus in (American) teacher education on this area. She proposes \textit{translanguaging pedagogy} as a way of providing all learners with equal educational opportunity and building a more just society, arguing that all teacher education programmes must prepare teachers for multilingual education (\citealt{Garcia2017-1}). Translanguaging pedagogy can be subdivided into \textit{translanguaging to learn,} which is focused on students' opportunities to draw on available language resources, and \textit{translanguaging to teach,} which focuses on teachers' organisation of teaching practices concerning multilingual students (\citealt{GarciaWei2014}). These themes are expanded into three strands of translanguaging pedagogy: \textit{translanguaging} \textit{stance, translanguaging design}, and \textit{translanguaging shifts} (\citealt{Garcia2017-1}). \textit{Translanguaging stance} refers to teachers’ beliefs that multilingual students have a holistic language repertoire, and their leveraging of students’ multilingualism for learning. \textit{Translanguaging design} refers to planning instruction to support students’ capacity to engage meaningfully with content, promoting language development and making space for multilingualism. Finally, \textit{translanguaging shifts} refers to the flexibility and responsibility in “moment-to-moment moves” concerning multilingual students’ needs (\citealt[61]{Garcia2017-1}).
\end{sloppypar}

\citet{DaugaardLaursen2021} discuss translanguaging pedagogy and its possibilities in the Nordic context. They point out, however, that Nordic public school teachers do not have the same opportunities for making use of translanguaging pedagogy, especially translanguaging to teach, as referred to by \citet{GarciaWei2014}. As the teachers in García and Wei’s study work in Spanish and English bilingual contexts, they understand their students’ language repertories better than teachers in many Nordic classrooms, where many different language backgrounds are present. In a Nordic context, therefore, it may be preferable to explicitly develop pre-service teachers’ opportunities to enact translanguaging pedagogy which supports students’ potential for translanguaging to learn (\citealt{DaugaardLaursen2021}).

\subsection{Semantic gravity: Navigating between theory and practice}\label{sec:ostergaard:3.2}

In the context of our study, pre-service teachers must demonstrate how they can apply knowledge about multilingualism to pedagogic practice. To trace this, we employ \textit{semantic gravity} from Legitimation Code Theory (LCT) (\citealt{Maton2014}).

LCT provides a sociological framework for researching and informing practice (\citealt{Maton2014}), providing powerful analytical tools for revealing what is seen as valued in various contexts. LCT’s dimension of \textit{semantics} provides analytical and conceptual tools for making organising principles of \textit{meaning-making} visible \citep{Maton2014}. The semantics dimension includes two concepts, semantic density and semantic gravity, of which semantic gravity is the relevant concept for investigating connections between different levels of context-dependent meaning.

Semantic gravity (SG) refers to “the degree to which meaning relates to its context, and is seen as stronger (+) or weaker (−) along a continuum of strengths” \citep[63]{Maton2014}. Here, the meaning explored is pre-service teachers’ knowledge about multilingualism, as expressed in the analysed examination papers. To demonstrate how varying strengths of semantic gravity are enacted in the examination papers analysed, we introduce a so-called \textit{translation device} (\citealt{MatonChen2016,MatonDoran2017}), see \tabref{tab:ostergaard:1}.

\begin{table}[hp]
\small
\begin{tabularx}{\textwidth}{>{\raggedright}p{\widthof{semantic gravity}}XX}
\lsptoprule
Strength of semantic gravity & Descriptor: How each strength of SG manifests in the examination papers  & Examples from examination paper 2 \\\midrule
Weaker (SG−) & Abstractions, theoretical concepts or ideas concerning multilingualism and names of theorists\newline
No direct connections to the data & Recognition of the mother tongue as a resource can contribute to establishing a metalinguistic affordance, which can benefit all students’ understanding of languages and linguistic development. (p. 21) \\\addlinespace

Mid-strength (SG0) & Generalisations about multilingualism\slash interpretations about multilingualism, multilingual pedagogy and multilingual students & When the Syrian students are given the opportunity to use their mother tongue, they will, [...] take part in filling out the metalinguistic affordance. This [...] could contribute to a more explorative conversation about language (\citealt[113--114]{Daugaard2013}). (p. 13)\newline
Since the explanation has taken place in the mother tongue, the teacher in the case has no way of knowing if the explanation is correct. (p. 12) \\\addlinespace

Stronger (SG+)  &Description of an observation or activity exhibiting multilingualism\slash excerpt from transcriptions including multilingualism   & The teacher’s reaction to this is to ask: “Do you know the word in Syrian?” to which student B answers: “I know what you mean, but we don’t have a word for it.” [...] the teacher turns to student A and says: ”oh, so you just explained what it means?” (p.\,12) \\
\lspbottomrule
\end{tabularx}
\caption{Translation device showing the operationalisation of semantic gravity} 
\label{tab:ostergaard:1}
\end{table}

The left column of the translation device shows semantic gravity as a continuum of relative strengths.\footnote{LCT conventions prescribe placing the strongest semantic gravity at the bottom and weakest semantic gravity at the top of the continuum (see e.g. \citealt[111]{Maton2014}).} Following \citet{Kirk2017}, we operate with three relative strengths of semantic gravity: strongest semantic gravity (SG+) at the bottom, mid-strength semantic gravity (SG0) in the middle, and weakest semantic gravity (SG−) at the top of the continuum (see the second column of the translation device, \tabref{tab:ostergaard:1}). The third column provides a descriptor of what characterises each of these strengths of semantic gravity in the context of the examination papers analysed. The fourth column gives examples from the data of each of the three strengths of semantic gravity. For clarity, all the data excerpts in the translation device are from examination paper 2, in which the pre-service teachers analysed a case exploring how micro-scaffolding and including multiple languages in instruction could support multilingual students’ language development in upper-primary Danish lessons.

In the context of the examination papers for \textit{Teaching bilingual students}, \citet[182]{Meidell_sigsgaard2021} defined stronger semantic gravity (SG+) as being “associated with descriptions of teaching materials, cases, teaching plans, activity descriptions and/or excerpts of dialogue from the classroom and can include quotes from transcriptions…”. Working from this point of departure and with an eye for meanings relating to multilingualism, stronger semantic gravity (SG+) refers to pre-service teachers demonstrating knowledge of multilingualism closely related to the context of  their analysed data. This happens, for example, when pre-service teachers include a transcript of a teacher talking to a student referring to aspects of multilingual\slash linguistic awareness or incorporating multiple languages. An example of this is seen in the excerpt from the analysis section of paper 2, where the pre-service teachers summarise what they observed in a lesson: “The teacher’s reaction to this is to ask: `Do you know the word in Syrian?'” (see the bottom row of the translation device); including a description of an incident from the observed lesson, and a quote in which the observed teacher refers to a students’ native language is an example of stronger semantic gravity (SG+) in the examination paper.

Conversely, weaker semantic gravity (SG$-$) is seen when the pre-service teachers “directly name theories and models which they have learned about throughout the course, such as when [...] students name concepts and terms from theories of (second) language and literacy development” (\citealt{Meidell_sigsgaard2021}: 183). In the context of this chapter, weaker semantic gravity is when pre-service teachers demonstrate understandings of multilingualism that are less context-dependent. This occurs when pre-service teachers refer to theoretical concepts of multilingual language development or pedagogy such as translanguaging, often with reference to relevant literature in the form of a citation. An example of this can be seen in the top row of the translation device, when the authors refer to theoretically informed perspectives on multilingualism, referring to “…the mother tongue as a resource” and connecting it to another theoretical term, in this case, “metalinguistic affordance”. Here, the pre-service teachers are making a claim about multilingualism which is not related to the context of their data, thus exhibiting weaker semantic gravity (SG−).

The mid-strength section of semantic gravity scale (SG0) refers to “…meanings which generalise over specific episodes or illustrations, but which are not entirely abstracted from a contextual base” \citep[4]{Kirk2017}. In this chapter, mid-strength semantic gravity can be seen when pre-service teachers make generalisations or interpret their data. An example of this occurs when the authors of paper 2 suggest that giving Syrian students the opportunity to use their mother tongue can help foster other students’ curiosity and contribute to more explorative conversations about language (see the middle row in the translation device). Here, the pre-service students are making a generalisation based on their specific pedagogical context (referring to the Syrian and other students in the observed class), as well as interpreting this generalisation by connecting it to theoretical concepts (the terms \textit{scaffolding} and \textit{metalinguistic affordance}). The mid-strength section of the semantic gravity scale also includes pre-service teachers interpreting their data, as seen in the other example in the SG0 row above, where the authors write: “Since the explanation has taken place in the mother tongue, the teacher in the case has no way of knowing if the explanation is correct”. Mid-strength semantic gravity (SG0) is seen in the examination papers when the authors interpret their data or make generalisations.

It is important to note that the classifiers stronger (+) and weaker (−) used with semantic gravity do not carry positive or negative connotations: the “+” simply refers to a relatively stronger semantic gravity (SG+) (i.e., a greater degree of context-dependence), and vice versa, with “−” denoting less context-dependent meanings or a weaker semantic gravity (SG−). Semantic gravity analysis reveals variations in context-dependence in a text. In the context of this chapter, we use semantic gravity to show how pre-service teachers demonstrate their understanding of multilingualism, and how they are able to link between theory and practice in their written exams. Such analysis allows us to see how pre-service teachers are able to interpret specific pedagogic practice by referring to theory (i.e. can they link specific examples of pedagogic practice with theoretical perspectives, moving from stronger semantic gravity to weaker semantic gravity), and conversely, how convincingly they can implement these theoretical perspectives to inform pedagogic practice (moving from weaker semantic gravity to stronger semantic gravity). As in other studies of English for Academic Purposes and reflective writing, successful analyses and discussions include so-called \textit{semantic waves} in which different strengths of semantic gravity are woven together (\citealt{Kirk2017, Meidell_sigsgaard2021, TilakaratnaSzenes2021}).

\section{Method and steps in the analytical process}

To explore pre-service teachers’ understandings of multilingualism we contacted all the teacher educators teaching the obligatory course \textit{Teaching bilingual students} and asked them to forward all successful examination papers that had a focus on multilingualism from 2020 and 2021. Our aim was to collect examination papers where pre-service teachers explored, reflected on and implemented multilingual students’ diverse linguistic repertoires as a resource for learning. Based on our own experience as teacher educators of this course, however, we anticipated finding only few such examples. Our teacher educator colleagues confirmed that hunch in their responses, commenting that multilingualism and\slash or multilingual pedagogies were not common topics chosen by the pre\hyp service teachers for their examination papers. Of the examination papers collected (see Table~\ref{appendix:ostergaard} for an overview of the examination papers collected and their topics), only those which mentioned multilingualism in the problem statement of the paper or made explicit use of multilingual approaches, e.g., in their ``Suggestions for Pedagogic Practice" section, were included in our study. This resulted in a dataset of only five examination papers.\footnote{All authors of the collected examination papers were informed about the research purpose, and all have provided written permission for the use of their examination papers in this study.}

\begin{table}[hp]
\caption{An overview of the collected examination papers}
\label{appendix:ostergaard}
\small
\begin{tabularx}{\textwidth}{lllQ}
\lsptoprule
Paper       & Subject      & \#A\footnote{Number of authors (pre-service teachers).} &  Problem statement, concerning additional language didactics\\\midrule
 1     & Danish       & 3          &  How can a teacher scaffold both academically and socially the transition from the reception class, i.e. sheltered Danish instruction, to mainstream instruction classes?\\\addlinespace
 2     & Danish       & 2          &  How can a teacher, through micro-scaffolding and the inclusion of multiple languages in their teaching, support multilingual students ’development from everyday language to subject-specific language in the subject Danish in upper primary school classes?\\\addlinespace
 3     & Arts         & 1          &  How can arts education, specifically initial interpretation and analysis, support language development in both mono- and multilingual students?\\\addlinespace
 4     & Danish       & 2          &  Which challenges can be identified in the Clio-unit\footnote{Clio is a publisher of digital teaching materials, part of Alinea Publishing, offering topics in many school subjects online for subscription members. Many Danish schools subscribe to Clio and other digital teaching materials platforms.} “Managing the instructive text” with regard to multilingual students’ academic and linguistic development, and how may the unit be reframed so these challenges are minimised while simultaneously strengthening a resource perspective on multilingualism in the classroom?\\\addlinespace
 5     & Mathematics  & 2          &  How does the unit on “Angles”\footnote{The unit on “Angles” refers to a chapter from a Mathematics textbook, which the authors of paper 4 analysed.} fit into the Pedagogic Register model? What kind of challenges may this have for multilingual students’ learning, and how can we restructure the unit in order to accommodate these challenges?\\\lspbottomrule
\end{tabularx}
\end{table}

The analysis was completed in two phases. The first phase addresses the first research question: How do pre-service teachers demonstrate understandings of multilingualism in their written examination papers for the obligatory teacher education course, \textit{Teaching bilingual students}, in Danish teacher education? In this phase we identified and categorised pre-service teachers’ understandings of multilingualism (described in \sectref{sec:ostergaard:5.1}). The second phase uses semantic gravity to explore the degree to which pre-service teachers demonstrate their ideas for implementing multilingualism approaches in practice, thus addressing research question 2 (\sectref{sec:ostergaard:5.2}).

Inspired by the theoretically informed perspectives on multilingualism and in particular by translanguaging pedagogy as described in \sectref{sec:ostergaard:3.1}, four categories of knowledge about multilingualism were identified in the examination papers, in the first phase of analysis:

\begin{enumerate}
\item referencing multilingualism and literature on and related to multilingualism,
\item theoretically informed arguments including multilingualism and multilingual pedagogies,
\item analyses of cases and/or teaching materials from a multilingual perspective, and
\item examples of teaching/learning activities including multilingualism and/or multilingual approaches.
\end{enumerate}

All four categories reveal different aspects of pre-service teachers’ knowledge about multilingualism. The first two categories reflect pre-service teachers’ translanguaging stance while the last two reflect translanguaging pedagogy from the perspective of translanguaging design and translanguaging shifts. The last category also connects to translanguaging to learn and translanguaging to teach.

Because we are also interested in determining how pre-service teachers express their ideas for implementing their theoretical knowledge about multilingualism in pedagogic practice, we completed a second analysis phase using semantic gravity, thus addressing the second research question: To what degree are they able to link theoretical understandings with examples of pedagogic practice? This phase of analysis focused on the two sections of the examination papers where students must transform their theoretical knowledge into pedagogic practice, i.e., the analysis section and the suggestions for pedagogic practice section.

Applying semantic gravity analysis to all examination papers’ \textit{analysis} section gives us insight into how well pre-service teachers are able to interpret pedagogic practices using theories on multilingualism. Semantic gravity analysis of each examination papers’ \textit{suggestions for pedagogic practice} section demonstrates pre-service teachers' ability to recontextualise in writing their knowledge of theories on multilingualism in terms of future pedagogic practice. Although such analysis does not demonstrate how well these pre-service teachers actually teach, the semantic gravity analysis does allow us to draw some conclusions about how well they understand theories of multilingualism and multilingual pedagogy, and their ability to describe how they could apply these in pedagogic practice. 

\section{Analyses and results}\label{sec:ostergaard:5}

This section presents our two analyses and related findings. First, we present analysis and findings from examining the examination papers for understandings of multilingualism, and in doing so comment on their connection to translanguaging pedagogy. Second, we examine the pre-service teachers’ reported ability to implement multilingualism and multilingual approaches from the perspective of semantic gravity.

\subsection{Exploring the examination papers for understandings of multilingualism}\label{sec:ostergaard:5.1}

All five examination papers from our dataset include examples of the pre-service teachers demonstrating knowledge concerning multilingualism (category 1). This is seen, for example, when pre-service teachers define their use of terms connected to multilingualism, as seen here:

\begin{quote}
In this examination paper we have chosen to use the term multilingual students rather than bilingual [students], given the fact that students with mother tongues other than Danish have very different language profiles (\citealt{KnudsenWulff2017}: 3) […], therefore, [we] speak of multilingual students in order to capture all the differences in the language use of the individual student […]. (Paper 5, p. 3) 
\end{quote}

All five papers also demonstrate theoretically informed arguments including multilingualism and multilingual pedagogies (category 2). This is seen, for example, when pre-service teachers explain the reason for choosing a theoretical term or position, as in the following:

\begin{quote}
We are using García and Wei's notion of translanguaging which must be understood as a pedagogical praxis where one acknowledges the linguistic repertoire of the students and makes use of them in teaching in order to strengthen their participation and learning (\citealt[14]{HolmenThise2019}). (Paper 5, p. 3)
\end{quote}

Four of the five collected papers included analyses of cases and/or teaching materials from a multilingual perspective (category 3). That this category was missing in paper 5 can, perhaps, be explained by the fact that the pre-service teachers of that paper analysed Danish teaching materials which did not incorporate or make room for other languages. The other four papers all include analyses of teacher-student interactions and/or teaching materials in which other languages were used or talked about. An example of this category can be seen in the following citation from an analysis of teacher-student interaction:

\begin{quote}
Muhammed will probably be left with the feeling that multilingualism in this class is not viewed as a resource. The class discussions are, as previously mentioned, conducted solely in Danish – a monolingual classroom. (Paper 1, p. 15)
\end{quote}

Concerning category 4, examples of teaching/learning activities including multilingualism, the overall picture varied. All five examination papers included some suggestions for allowing students to make use of their full linguistic repertoire in classroom activities. These varied, however, on a wide spectrum from optional (as when allowing students to make use of their mother tongue in the classroom on their own initiative) to necessary (as when using other languages is essential for completing an activity). When using other languages is necessary for completing an activity, we see this as an example of translanguaging to teach. 

In the following example from the \textit{suggestions for pedagogic practice} section of examination paper 4, the pre-service teachers suggest that the students make use of their “strongest” language. This suggestion falls somewhere in the middle of the spectrum from optional to necessary, and is suggestive of more of a translanguaging to learn activity:

\begin{quote}
After the short individual thinking-time, the students must be given e.g. 2 minutes to talk to their partner about the answer. Here it is once again possible to make use of grouping students with the same strongest language who can work together. After this the whole class can answer (Podcast n.d.). The method gives more and better answers, because the students have been given the time to think and to formulate their thoughts […], and the important thing about the pyramid model is that thinking-time and room for the students’ own languages is given (Podcast n.d.). In this way, the method supports the resource-perspective on multilingualism in the class.  (Paper 4, p. 15)
\end{quote}

In contrast to the above example, two of the examination papers demonstrated examples of multilingual approaches (also category 4) leaning more towards the necessary end of the spectrum. In both of these cases, the pre-service teachers included activities involving students’ parents (and the languages spoken at home) as a necessary resource for learning. An example of this is seen in paper 4, where the pre-service teachers include the activity known as \textit{conversation homework} as a valuable learning activity for multilingual children:

\begin{quote}
In order to further support translanguaging and a resource-perspective in general, one could consider using conversation homework as an element. The students could, for example, instruct their parents (or other significant adults) [in their home language] in the same or similar assignment as they themselves carried out during the “action” phase activities […]. (Paper 4, p. 13)
\end{quote}

Analysing the dataset from the perspective of the four categories of multilingualism shown here provides an initial overview of pre-service teachers’ understandings of multilingualism in the context of teaching in the mainstream classroom. In all five papers, pre-service teachers demonstrate an emerging translanguaging stance in that they all have examples of category 1 when referencing multilingualism and literature related to multilingualism. An emerging translanguaging stance is also detectible in that all five papers also include theoretically informed arguments including multilingualism and multilingual pedagogies (category 2). Finding these in the examination papers suggests that the authors of all five papers demonstrate a theoretically underpinned understanding of multilingualism, where multilingualism is seen as a resource for student learning.

Translanguaging design and translanguaging shifts, however, were demonstrated less clearly. Analyses of cases and/or teaching materials from a multilingual perspective (category 3) only appeared in those examination papers which analysed teacher-student interactions in which multilingual students were involved. Category 4 was the least convincing category, in that the examples of teaching/learning activities including multilingualism and/or multilingual approaches included were often optional and up to students’ own initiative rather than an integral part of the teaching activities suggested. This implies pre-service teachers are more able to implement translanguaging to learn and less translanguaging to teach.

Analysis based on the four categories of multilingualism indicates that the pre-service teachers do demonstrate understandings of multilingualism as a resource for learning from a theoretical or logical point of view. The four categories did not, however, provide adequate nuance for determining \textit{to what degree} pre-service teachers can describe how they would apply their theoretical knowledge in pedagogic practice. For this we turn to the semantic gravity analysis.

\subsection{Weaving together theory with pedagogic practice: Semantic gravity analysis and results}\label{sec:ostergaard:5.2}

In the first stage of analysis (above), we identified where in their examination papers the pre-service teachers referred to multilingualism and multilingual approaches. Conclusions based on that first stage of analysis were unclear in terms of how the pre-service teachers implemented multilingual approaches in their own reported and/or predicted practice. In this section of the chapter, therefore, we look more closely at the examples of multilingualism identified in categories 3 (analyses of cases and/or teaching materials from a multilingual perspective) and 4 (examples of teaching/learning activities including multilingualism and/or multilingual approaches) from the perspective of semantic gravity. In the examination papers these occur in the \textit{analysis} and \textit{suggestions for pedagogic practice} sections, respectively.

Analysing pre-service teachers’ understandings of multilingualism from the perspective of semantic gravity reveals both their ability to interpret examples of practice from a theoretical point of view, and their potential ability to implement theoretically informed perspectives on multilingualism in their pedagogic practice. The identified examples of category 3 (in the analysis section) were analysed and categorised into the three strengths of semantic gravity. An example can be seen here from paper 2, where the pre-service teachers analyse an observed interaction between multilingual students and a teacher:

\begin{quote}
When the teacher turns to student A and says: ``well, so you've just explained what that means?" \textbf{[SG+]}, she actually engages in a completely different and new conversation \textbf{[SG0]}, which moves on a more metalinguistic level (\citealt[28]{KnudsenWulff2021}) \textbf{[SG−]}. (Paper 2, p. 12)\footnote{To demonstrate the semantic gravity coding in this and the following examples, we have added our coding directly in the quote from the examination papers using bolded square brackets. For example, the first \textbf{[SG+]} in this example refers to the semantic gravity strength of the text up to the square bracket, “When the teacher turns to student A and says: ‘well, so you've just explained what that means?’”.}
\end{quote}

In the example above, the beginning of the sentence up to and including the quote is coded as stronger semantic gravity (SG+), as it refers directly to an observation of pedagogic practice. The second part of the sentence makes an interpretation of what happened, commenting on how what the teacher said can be understood in terms of how conversations flow and what is being talked about. This is therefore coded with mid-strength semantic gravity (SG0). The last part of the sentence makes specific reference to literature in which multilingualism is described and includes both a theoretical term \textit{metalinguistic understanding} and a literature reference (\citealt{KnudsenWulff2021}). Here, the pre-service teachers refer to theoretical understandings of multilingualism, demonstrating weaker semantic gravity (SG−).

Seeing all three strengths of semantic gravity as in the example above makes for a more convincing demonstration of the pre-service teachers’ ability to interpret pedagogic practice from theoretical perspectives on multilingualism. To show how often these manifest in the collected dataset, the following table lists the number of instances of each strength of semantic gravity for each examination paper’s analysis section.

\begin{table}
\begin{tabular}{l *5{r}}
\lsptoprule
SG strength &  Paper 1  &  Paper 2  & Paper 3  & Paper 4 & Paper 5\\\midrule
SG− & 7 & 5 & 1 & 1 & 0\\
SG0 & 10 & 22 & 3 & 7 & 0\\
SG+ & 2 & 10 & 2 & 0 & 0\\
\lspbottomrule
\end{tabular}
\caption{Semantic gravity (SG) in understandings of multilingualism in the examination (analysis)}
\label{tab:ostergaard:2}
\end{table}

\tabref{tab:ostergaard:2} shows that two of the examination papers (4 and 5) do not include examples of stronger semantic gravity (SG+) in their analysis section. While paper 5 does mention translanguaging in its theory section, this theoretical perspective is not used to analyse their data. As a result, paper 5 shows no instances on the semantic gravity scale above. In contrast, paper 4 does implement multilingualism in the analysis section. As the table shows, however, the bulk of their understandings of multilingualism are in the mid-strength level of semantic gravity (SG0). When concerning multilingualism, their analysis section includes mostly generalisations such as the following:

\begin{quote}
It is not at any point considered in the activities that multilingual students who have limited linguistic competence in the new language can benefit from using their strongest language \textbf{[SG0]}, which would give them a better opportunity for learning, as expressed in the following quote. “[…] By including the students' strongest language, the students get other opportunities […] to enter into a dialogue with the academic material. This provides better opportunities for learning […]” (\citealt[95]{HolmenThise2020}) \textbf{[SG−]}. (Paper 4, p. 10)\footnote{Part of the citation included by the authors of paper 4 has been omitted.}
\end{quote}

Although the authors of paper 4, in the example above, do connect to translanguaging theory by including a quote from relevant literature, they do not make an explicit connection to their data. This makes for a less convincing demonstration of the pre-service teachers’ ability to interpret pedagogic practice and leaves it up to the examiner to trust both that this summary is true and that the interpretation of the data holds.

Similarly, the suggestions for pedagogic practice section of the examination paper also provides opportunities for the pre-service students to demonstrate their understandings of multilingualism (category 4) by writing about how they would implement theory in the practice of teaching a specific school subject. \tabref{tab:ostergaard:3} below shows the strengths of semantic gravity of each occurrence of understandings of multilingualism identified in the papers’ suggestions for pedagogic practice sections.


\begin{table}
\begin{tabular}{l *5{r}}
\lsptoprule
SG strength & Paper 1 & Paper 2  & Paper 3 & Paper 4 & Paper 5\\\midrule
SG− & 8 & 3 & 2 & 6 & 3\\
SG0 & 9 & 5 & 3 & 6 & 3\\
SG+ & 9 & 4 & 0 & 3 & 2\\
\lspbottomrule
\end{tabular}
\caption{Semantic gravity (SG) in understandings of multilingualism in the examination (pedagogic practice)} 
\label{tab:ostergaard:3}
\end{table}

An example of this can be seen in paper 5, where the pre-service teachers describe an activity.  They want the students to perform, generalise and interpret the activity, and finish with a reference to translanguaging theory:

\begin{quote}
Last but not least, we will give the students the task of going home and showing their parents, siblings or others the video from our action phase activity – without sound \textbf{[SG+]}. In this way, the teacher can once again utilise translanguaging by having the multilingual students explain what is happening in their own language \textbf{[SG0]}, and the teaching thus integrates multilingualism \textbf{[SG0]} – it is not about knowing the most technical words in Danish, but about communicating about something they have learned in a language in which they feel safe at home (\citealt{HolmenThise2019}) \textbf{[SG−]}. (Paper 5, p. 17)
\end{quote}

This example starts with stronger semantic gravity (SG+) in that it gives a fairly concrete description of a so-called “conversation homework” activity, in which students must talk with their parents about a video seen in class. This is followed by two sentences at mid-strength semantic gravity: first a sentence generalising how the activity is an example of a multilingual activity, followed by a sentence offering an interpretation of what such an activity can mean in terms of activating students’ multilingual repertoire. The last sentence, starting with “– it is not about knowing”, connects their generalisation and interpretation to an aspect of translanguaging pedagogy (feeling safe) with a literature reference to \citet{HolmenThise2019}.

In comparison, the pre-service teacher in the following example does not make use of stronger semantic gravity when making a suggestion for pedagogic practice:

\begin{quote}
In the keyword dialogue activity, the inclusion of English is suggested \textbf{[SG0].} This could also open up to other languages, thereby increasing the opportunity for multilinguals to contribute, which could increase the diversity of found images \textbf{[SG0].} This speaks to Holmen \& Thise's thesis that multilingualism can be a resource for students and for teaching \textbf{[SG−]}. (Paper 3, p. 13)
\end{quote}

Unlike the previous example from paper 5, this suggestion from examination paper 3 does not include a concrete description of the activity suggested. The authors only refer to the \textit{keyword dialogue activity} from their analysis section without expanding on it. The pre-service teacher could have presented more details about how including English and other languages could be implemented, for example, by providing a list of relevant English keywords for the students to use in their search, or by asking students to brainstorm keywords in any other known (to them) languages and translate these prior to searching. Such examples would have strengthened the semantic gravity and made a more convincing demonstration in writing of the pre-service teachers’ potential ability to implement a multilingual approach in their teaching.

Our semantic gravity analysis of the collected examination papers’ \textit{analysis} section and the \textit{suggestions for pedagogic practice} section shows that all five papers include references to literature about multilingualism. In the examination papers, pre-service teachers thus demonstrate that they are able to make less context-dependent meanings about multilingualism (SG−). Furthermore, they demonstrate an ability to generalise about multilingual theories and interpret their data from multilingual perspectives, displaying mid-strength semantic gravity (SG0). However, describing how they would implement this theoretical knowledge in practice, describing concrete examples of multilingual practices in classrooms (SG+) is less prevalent in the examination papers analysed.

As seen above, both the \textit{analysis} sections and the \textit{suggestions for pedagogic practice} sections of the papers contain relatively few examples of stronger semantic gravity (SG+), compared to the other two strengths of semantic gravity. This suggests that the pre-service teachers may face challenges in operationalising their understandings of multilingualism and incorporating elements of e.g. translanguaging to learn and translanguaging to teach (\citealt{Garcia2017-1}) in their future practice. While referencing theory and making generalisations about its usefulness in an educational context is necessary for demonstrating comprehension of a concept such as translanguaging, not including concrete examples of what it could look like in an educational setting results in less convincing examination papers.

\section{Discussion and conclusion}

For most pre-service teachers, the \textit{Teaching bilingual students} course is the only one in their degree programme which requires them to learn about multilingual students’ learning prerequisites. It is arguably also the only course which explicitly positions students’ multilingualism as a resource. The course content draws on fields including but not limited to linguistics, pedagogy, literacy development, cultural studies and language education. The teacher educators teaching this course are experts in additional language teaching and learning rather than in teaching subjects (Mathematics, Science, History etc.) or the core pedagogy subjects of the teaching degree programme. It is up to the pre-service teachers themselves, therefore, to do the work of integrating theories of, for example, multilingualism, from the \textit{Teaching bilingual students} course with the teaching of their chosen subjects. That, combined with the fact that the course is obligatory for all pre-service teachers, makes it unique in a Nordic context. Further research in the Nordic context of teacher education would be valuable, for example in comparing how multilingualism and linguistic diversities are included in pre-service teachers’ degree programmes, with the purpose of examining the relevance of promoting equal and democratic learning opportunities for all students in both general instruction as well as within each subject.

When teaching the obligatory teacher education course \textit{Teaching bilingual students}, we occasionally experienced opposition from the pre-service teachers. A common challenge for pre-service teachers is accepting that students’ diverse language repertoires can be activated in subject teaching as a way of supporting development of both students’ subject content knowledge as well their linguistic competencies. While such an understanding of multilingualism is well supported by research in the fields of additional language education (\citealt{DaugaardDewilde2017, Garcia2017-1,Schleppegrell2002_Challenges}), changing the mindsets of pre-service teachers who believe that multilingual students are in some way deficient (\citealt{Kristjánsdóttir2006,Kristjánsdóttir2011,Laursen2019}) proves challenging \citep{Jaspers2022}.

While our study cannot answer the question of whether the \textit{Teaching bilingual students} course has changed any pre-service teachers’ perceptions on multilingualism towards a more additive perspective, some interesting insights may nonetheless be gained. One of the aims of this chapter was to investigate the degree to which established concepts from the field of additional language education (in this case, multilingualism) are taken up, understood and operationalised by pre-service teachers in their examination papers. The fact that only very few pre-service teachers choose to focus on this aspect in their examination papers, combined with the findings presented above, gives us as teacher educators cause for reflection and suggests that this perspective is either not very well understood by the pre-service teachers and/or otherwise under-prioritised as a useful perspective. This suggests that knowledge about translanguaging and seeing multilingualism as a resource for learning is not common among pre-service teachers and could therefore be strengthened in their degree programme. This gives teacher educators food for thought in that this knowledge is indeed something we wish our pre-service teachers to develop.

As our first phase of analysis shows, those pre-service teachers who chose to include a multilingual perspective in their examination papers do recognise the value of multilingualism as a tool for supporting students’ subject learning. Our second analysis phase shows, however, that their operationalisation of this perspective into pedagogical practice is less convincing (see also \citealt{chapters/7_alisaari, chapters/8_heikkola, chapters/6_iversen}).

While in this chapter we have looked at one specific theoretical perspective (multilingualism) from a relatively small educational field (additional language education), the findings from our semantic gravity analysis are more far\hyp reaching. This kind of analysis of pre-service teachers’ writing is more broadly relevant for educators in any professional field requiring practitioners to reflect on practice in a theoretically informed way. Examining how convincingly students demonstrate understandings of relevant theoretical concepts in their writing and how they apply these understandings to real scenarios is not limited to the case of teacher education. Nevertheless, studies in Denmark (\citealt{NielsenPaulsen2006}) as well as internationally (\citealt{Calderhead1987,LeCornuEwing2008,Rusznyak2021}) show that it is often difficult for students to do this convincingly.

Our study confirms results of previous research, suggesting that students’ abilities to “weave together” various levels of context-dependent understandings can provide convincing demonstrations of reflective thinking (\citealt{Hood2016,Kirk2017,SzenesEtAl2015}). Our findings provide teacher educators with a tool for reflecting on pre-service teachers’ reported understandings of a central theoretical perspective (multilingualism as an additive perspective) in our field, and can be useful in terms of deciding how to (re-)focus on this perspective in our teaching. This perspective will be useful in the coming years with the implementation of the latest teacher education reform in Denmark, in which a greater emphasis is placed on pre-service teachers’ practicum, requiring pre-service teachers to spend more time in practice while continuing to require reflecting on practice through theoretical perspectives.

\printbibliography[heading=subbibliography,notkeyword=this]
\cleardoublepage
\end{document}
