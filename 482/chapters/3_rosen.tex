\documentclass[output=paper]{langscibook}
\ChapterDOI{10.5281/zenodo.15280897}
\author{Jenny Rosén\orcid{Stockholm University}\affiliation{} and Åsa Wedin\orcid{}\affiliation{Dalarna University}}
\title[Study guidance in the mother tongue]{Study guidance in the mother tongue: Legitimate knowledge and the emergence of a profession in Swedish schools}
\abstract{In this chapter we problematise the role and the training for assistants in Study Guidance in the Mother Tongue (SGMT) in Swedish schools, through the case of one SGMT assistant. The chapter addresses questions of competence and legitimacy on different scales using the framework of nexus analysis, basing its analysis on national policy documents, syllabi from university courses for SGMT assistants, interviews with the assistant, teachers and principals and observations of classroom and school practices. The findings show that the specific competences mentioned in official documents become visible in the different roles the assistant performs, both in classroom practices and in negotiation with other school staff. However, lack of recognition of his competence leaves him frustrated in the school practice, and creates a position of in-betweenness for the assistant. The contrast between the knowledge the SGMT assistants gained through academic courses, and the attitudes of teachers and principals reveals the need to include knowledge about SGMT in teacher education, to create equitable education and understanding of SGMT across school contexts.}
\IfFileExists{../localcommands.tex}{
  \addbibresource{../localbibliography.bib}
  % add all extra packages you need to load to this file

\usepackage{tabularx,multicol}
\usepackage{url}
\urlstyle{same}

\usepackage{listings}
\lstset{basicstyle=\ttfamily,tabsize=2,breaklines=true}

\usepackage{langsci-basic}
\usepackage{langsci-optional}
\usepackage{langsci-lgr}
\usepackage{langsci-osl}
% \usepackage{./langsci/styles/langsci-lgr}
% \usepackage{./langsci/styles/langsci-osl}
% \usepackage{langsci-gb4e}

\usepackage{tikz}
\usetikzlibrary{patterns,calc}
\pgfdeclarepatternformonly{south east lines}{\pgfqpoint{-0pt}{-0pt}}{\pgfqpoint{3pt}{3pt}}{\pgfqpoint{3pt}{3pt}}{
    \pgfsetlinewidth{0.6pt}
    \pgfpathmoveto{\pgfqpoint{0pt}{3pt}}
    \pgfpathlineto{\pgfqpoint{3pt}{0pt}}
    \pgfpathmoveto{\pgfqpoint{.2pt}{-.2pt}}
    \pgfpathlineto{\pgfqpoint{-.2pt}{.2pt}}
    \pgfpathmoveto{\pgfqpoint{3.2pt}{2.8pt}}
    \pgfpathlineto{\pgfqpoint{2.8pt}{3.2pt}}
    \pgfusepath{stroke}}
    
\usepackage{stmaryrd}
\usepackage{wasysym}
\usepackage{multirow}
\usepackage{caption}
\usepackage{subcaption}
\usepackage{mathrsfs}
\usepackage{qtree}

\usepackage{linguex}


  %pminos do not split footnotes
% \interfootnotelinepenalty=10000 %Footnote in Laporte chapters has to be split SN


%\DeclareIndexNameFormat{default}{%
%\nameparts{#1}%
%\usebibmacro{index:name}%
%{\index[names]}%
%{\namepartfamily}%
%{\namepartgiveni}%
% {}% L1
% {}% L2
%{\namepartprefix}% generates spurious space L3
%{\namepartsuffix}% generates spurious space L4
%}

%  {\DeclareIndexNameFormat{default}{%
%     \usebibmacro{index:name}{\index[names]}{#1}{#3}{#5}{#7}}}

%\DeclareIndexNameFormat{default}{%
%  \usebibmacro{index:name}{\sindex[nom]}{#1}{#3}{#5}{#7}}

%\DeclareIndexNameFormat{default}{%
%  \usebibmacro{index:name}{\sindex[person]}{#1}{#3}{#5}{#7}}
%\DeclareIndexNameFormat{default}{%
%\nameparts{#1} \usebibmacro{index:name}{\sindex[person]]}{\namepartfamily}{‌​\namepartgiven}{\nam‌​epartprefix}{\namepa‌​rtsuffix}}

%\newcommand{\smiley}{:)}

%\renewbibmacro*{index:name}[5]{%
%\usebibmacro{index:entry}{#1}%
%{\iffieldundef{usera}{}{\thefield{usera}\actualoperator}\mkbibindexname{#2}{#3}{#4}{#5}}}

% \newcommand{\noop}[1]{}

%remove for final
%\overfullrule=1mm

\newcommand{\tobi}[2]}}
\renewcommand{\S}[1]{\tobi{#1}{\textsc{*}}}

% this volume references
% puts: [this volume]
% already defined: \citetv
%\newcommand{\citepv}[1]{(\citeauthor{#1} \citeyear*{#1} [this volume])}
\newcommand{\citealtv}[1]{\citeauthor{#1} \citeyear*{#1} [this volume]}

%parentheses around example number
\newcommand{\pref}[1]{(\ref{#1})}

% in-text examples

\newcommand{\lnex}[1]{\textit{#1}} %target lang word
\newcommand{\lnlit}[1]{(lit.: `#1')} %literal reading
\newcommand{\lnlat}[1]{(#1)} % latinization
\newcommand{\lntrans}[1]{`#1'} %translation
\newcommand{\lnexl}[2]%
{\lnex{#1}{} \lnlat{#2}} % ex with latinization
\newcommand{\lnexlat}[3]{\lnex{#1}{} \lnlat{#2}{} \lntrans{#3}} % ex with latinization and tranl.

%ch01
\newcommand{\co}[1]{\mbox{\textbf{#1}}}

%ch09

\newcommand{\cyrbulg}[1]{\begin{otherlanguage*}{bulgarian}#1\end{otherlanguage*}}


%ch10
\newcommand{\nlp}{{\small NLP}}
\newcommand{\mwe}{{\small MWE}}
\newcommand{\rae}{{\small RAE}}
\newcommand{\lvc}{{\small LVC}}
\newcommand{\pos}{{\small P}o{\small S}}
%\newcommand{\todo}[1]{ \textcolor{red}{#1} }

%\renewcommand{\labelenumi}{\theenumi}
%\ainamefmt{{vv}{ll}{, ff}{, jj}} % fullname

\newcommand{\biberror}[1]{{\color{red}#1}}

\newcommand{\osenovaitem}{--~} 
  %% hyphenation points for line breaks
%% Normally, automatic hyphenation in LaTeX is very good
%% If a word is mis-hyphenated, add it to this file
%%
%% add information to TeX file before \begin{document} with:
%% %% hyphenation points for line breaks
%% Normally, automatic hyphenation in LaTeX is very good
%% If a word is mis-hyphenated, add it to this file
%%
%% add information to TeX file before \begin{document} with:
%% %% hyphenation points for line breaks
%% Normally, automatic hyphenation in LaTeX is very good
%% If a word is mis-hyphenated, add it to this file
%%
%% add information to TeX file before \begin{document} with:
%% \include{localhyphenation}
\hyphenation{
    Beck-man
    Ngu-yen
    back-chan-nel
    back-chan-nels
    mo-not-o-nous
    ste-reo-typ-i-cal
}

\hyphenation{
    Beck-man
    Ngu-yen
    back-chan-nel
    back-chan-nels
    mo-not-o-nous
    ste-reo-typ-i-cal
}

\hyphenation{
    Beck-man
    Ngu-yen
    back-chan-nel
    back-chan-nels
    mo-not-o-nous
    ste-reo-typ-i-cal
}
 
  \togglepaper[1]%%chapternumber
}{}

\begin{document}
\maketitle 
\label{chap:3}
%\shorttitlerunninghead{}%%use this for an abridged title in the page headers



\section{Introduction}

The focus of this chapter is the role of and training for assistants in Study Guidance in the Mother Tongue (SGMT, \textit{studiehandledning på modersmålet}) in Swedish schools, and the education these assistants are offered in relation to teacher education in general.  Sweden has provided support for multilingual students since the 1970s, mainly through mother tongue education (\citealt{GanuzaHyltenstam2020}). Besides mother tongue education, students that need support in a language other than Swedish to achieve the goals of various school subjects should be offered such support by an assistant, through study guidance in their so-called “mother tongue” or  their strongest language (\citealt{Swedish_ministry_of_education2010} chapter 3, 12i§). Newcomers  in lower-secondary school should be provided with SGMT unless it is clearly unnecessary (\citealt{Swedish_ministry_of_education2010} chapter 3 12i§, \citealt{Swedish_ministry_of_education2011} chapter 5 4§). According to the Swedish National Agency of Education, SGMT includes not only subject teaching through the mother tongue but also support regarding language awareness, contrastive aspects between Swedish and the mother tongue as well as intercultural perspectives on school subjects (SNAE, \citealt{Swedish_national_agency_of_education2015}).

The importance of SGMT for the success of newcomers in school has been emphasised by the School Inspectorate (\citealt{Swedish_school_inspectorate2017}) and a governmental inquiry (\citealt{Swedish_ministry_of_education2019}). Previous studies (\citealt{Rosen2019-1, Rosen2020, St_john2021}) have illustrated the complex role of SGMT assistants in relation to teachers, as well as the multilingual and multimodal character of the tuition. In a study of SGMT in compulsory schools, \citet{Reath_warren2017} characterises SGMT as a multilingual practice, functioning to raise awareness of lexical, conceptual, metalinguistic, task-oriented and sociocultural issues. In an action research project, \citet{St_john2021} studied the collaboration between teachers and SGMT assistants in municipal adult education in Swedish Tuition for Immigrants (SFI). The study showed that the SGMT assistants supported communication between the students and the teachers, making it possible for the teachers to see if their students had understood the lesson content. This in turn allowed teachers to adjust their feedback and teaching to the needs of the students. Moreover, the SGMT assistants could present the content and relate it to contexts that were familiar to students, making it more relevant and comprehensible. Thus, the assistants could be described as intermediaries that move between teachers and students and provide “strategic multilingual, adaptive, inclusive and pedagogical assistance” (\citealt[230]{St_john2021}).

The role of SGMT assistants is unique to the Swedish school system, even though there are similarities to the role of “bilingual assistants” in Norway.  In a study in the Norwegian context, \citet{Eek2021} has explored the use of bilingual assistants in second language tuition for adult migrants. Through observations and interviews, \citet{Eek2021} investigated the significance of these assistants for adult students’ investment in second language learning. Her study shows how the bilingual assistants enable the students to participate more actively in the classroom, and how important they are for students’ confidence, understanding and recognition. The degree to which SGMT assistants themselves actively participate during lessons varies. In a study of the Language Introduction Programme in Upper Secondary School, for multilingual students aged 16--19 who have not yet reached a sufficient level of Swedish for mainstream programmes, \citet{Wedin2022} showed that SGMT assistants played a less passive role in classrooms than assistants examined in other studies (e.g. \citealt{Rosen2019-1, Rosen2020}). In Wedin’s study, SGMT assistants interacted actively with students in subjects such as Natural Sciences, Social Sciences and Mathematics.

Even though SGMT has been offered in Swedish schools for more than forty years, education for these assistants has not been incorporated into teacher education. Although there are SGMT assistants that have teaching qualifications, often from countries other than Sweden, there is no formal requirement for teacher education to be an SGMT assistant. Since the qualifications to work as an SGMT assistant are not officially outlined, there is considerable variation in terms of subject knowledge and language skills among them, and SGMT assistants need to collaborate with subject teachers in order to support the students. The vagueness of the description and understanding of the skills and professional role of SGMT assistants underpins the aim of this chapter, which is to examine and problematise SGMT in Swedish schools with a focus on how legitimate knowledge for assistants is constructed in the school context, in policy documents and in courses offered at universities. Through a study of SGMT at one upper secondary school, we address questions of competence and legitimacy on different scales using the framework of nexus analysis (\citealt{ScollonScollon2004, ScollonScollon2007}). The following research questions have guided the study:

\begin{itemize}
\item How is the role and competence of one SGMT assistant negotiated in relation to teachers and students in the classrooms?
\item What knowledge is highlighted as central for the profession of SGMT assistant in policy documents, including course syllabi?
\end{itemize}

Thus, from the nexus analysis perspective, the first research question examines the historical body of the SGMT assistant and the interaction order in the classroom, while the second research question focuses on the discourses in place. The framework of nexus analysis is presented in \sectref{sec:rosen:2}.

\section{Nexus analysis} \label{sec:rosen:2}

Nexus analysis has developed from ethnography and discourse analysis, and includes both historical and ethnographic dimensions (\citealt{Hult2010, Pietikainen2011}). The main concern in the analysis is the complex relationship between discourse and action, including both macro and micro perspectives (\citealt{Lane2014}), which is relevant for this study. In their pioneering work, \citet{ScollonScollon2004} emphasise how social action, as the unit of analysis, occurs in the intersection between the discourses in place, the participants’ historical bodies, and the interaction order produced (see below for a description of the terms). Nexus analysis

\begin{quote}
  entails not only a close, empirical examination of the moment under analysis but also a historical analysis of […] trajectories or discourse cycles that intersect [at a given] moment as well as an analysis of the anticipations that are opened up by the social actions taken in that moment. (\citealt[8]{ScollonScollon2004})
\end{quote}

\begin{sloppypar}
Thus, the analysis includes the intersection of wider socio-historical discourses and a specific social action. A social action that is repeated is understood as a social practice. Social actions occur at the intersection between what in nexus analysis is framed as i) the discourses in place which enable that action or are used by the participants as meditational means in their action, ii) the historical bodies of the participants in action and the institutional settings focused on, and iii) the interaction order which they mutually produce (\citealt{ScollonScollon2004}).
\end{sloppypar}

The concept of historical bodies comes from the work of \citet{Nishida1998} and is understood as the collectively constituted experiences of individuals as well as institutions across time that become a “natural” part of them. The concept has similarities to Bourdieu’s concept of habitus (\citealt{Bourdieu1984}). Interaction order can be described as ways in which individuals behave and use language as they form relationships with others in social interaction. Thus, nexus analysis does not focus on language per se but on how languages and other semiotic and material resources are used to mediate action (\citealt{Lane2014}).

\section{Material and method}

To include both micro and macro perspectives in the analysis, the empirical material used consists of i) national policy documents about SGMT produced by the Swedish National Agency of Education, ii) syllabi from university courses for SGMT assistants from three universities (University A, B and C), iii) one interview with an SGMT assistant and extracts from interviews with 13 teachers and four principals at the same school, and iv) video-recorded observations from 23 lessons in various school subjects during which the SGMT assistant participated. We particularly draw on transcripts from a video recording and field notes taken during one of these lessons, in an upper secondary school. 

The university courses for the SGMT assistants were offered as independent courses, not included in regular teacher education. Courses for SGMT assistants have been provided by a number of universities as part of an initiative by the Swedish government to increase competence in schools regarding the education of newcomers (\citealt{Swedish_ministry_of_education2019}). Courses for SGMT assistants are also provided by other educational institutions. However, since this chapter focuses on SGMT in relation to teacher education, which in Sweden is conducted in universities, those courses have not been included. Since there are no general guidelines for who can be employed as an SGMT assistant, they are, on a group level, extremely heterogeneous. The specific case examined in this chapter cannot thus be understood as representative.

As nexus analysis brings into focus relations between discourse and action as well as macro and micro perspectives, the analysis in this study started from ethnographic field work in one school and the work of one SGMT assistant, given the pseudonym Barzan. He was selected by the second author as a participant in a two-year ethnographic study, as he was the only SGMT assistant who was employed full-time at the Language Introduction Programme (LIP) at the upper secondary school under investigation. Students who arrive in Sweden at age 16--19 are placed in the LIP to learn Swedish and to complement earlier studies in various subjects. One semi-structured interview (35 minutes long) was conducted with Barzan at the end of the study, and several informal occasions of small talk between him and the second author took place earlier, during the observations at the school. Barzan mastered languages that many of the students understood and thus he worked with multiple students in various subjects during the two-year ethnographic study. Although the interview with Barzan is the main interview used in the analysis in this study, extracts from other interviews with 17 teachers and four principals (conducted during the ethnographic study) have also informed the analysis. The lesson observed was in social sciences. It was one hour long and students were preparing for an upcoming test by reading the textbook and answering questions in an exercise book. Video recordings were made of individuals and pairs of students working on this task and fieldnotes were written. The transcript of the classroom video recording was conducted in collaboration with a colleague who is a researcher in Arabic. The translations (from Arabic and Swedish, the only languages used in this lesson, to English) were done by the authors. 

In line with nexus analysis, the first step in the analysis was inductive and focused on the research questions. We identified central practices in SGMT that showed how competence and legitimacy were negotiated in the classroom (RQ 1) by listening to the recordings of the lesson observations and reading and categorising the fieldnotes and the transcripts of the interviews. After the central practices were identified, the analysis then focused on the interaction order, bringing into focus relations between Barzan, students and teachers. The analysis of the interaction order was based on an earlier analysis of the observed lessons where Barzan had been a participant (see \citealt{WedinAho2022}).

Secondly, we focused the analysis on how Barzan positioned himself as an SGMT assistant in the school, including his historical body. In the analysis we looked at how he talked about his experiences of language and education. We particularly focused on his talk about his role as an SGMT assistant and then we compared it to teachers’ and principals’ talk. 

In the third step of the analysis, we looked into the discourses made relevant in those actions. Thus, national policy documents and courses for SGMT assistants were included in the analysis. The analysis of the policy documents focused on the content, and was thus guided by content analysis (\citealt{Swedish_ministry_of_education1992,Swedish_ministry_of_education1993}), identifying which competences and what knowledge SGMT assistants were expected to develop through the courses, or were expected to perform, as expressed in policy documents.   

Ethical issues were considered throughout the study and data are presented in ways that avoid recognition of the participants and the specific school. Persons involved were informed about the aims of the project and gave their consent to participate. All data have been stored in accordance with the project’s data management plan.

\section{Findings}

Through the framework of nexus analysis, we examined the professional role and work of one SGMT assistant, Barzan, in order to critically discuss the role of SGMT in Swedish schools. We analysed the action by focusing on the interaction order, historical body and discourses in place.  This section presents our results, and begins by zooming in on the interaction order in the classrooms where Barzan was working as an SGMT assistant. Following this, we address questions of position and the historical body of the SGMT assistant. We then address discourses in place by focusing on national policy for SGMT and university courses targeting SGMT assistants. Finally, we problematise the skills and professional role of SMGT assistants in relation to the Swedish school and teacher education. 

\subsection{Interaction order}\label{sec:rosen:4.1} %4.1 /

Starting from the social actions of SGMT assistants, students and teachers, we focus on the interaction order by examining how the role of the SGMT assistant was negotiated in the classrooms at the school and how Barzan was positioned and positioned himself as he formed relationships in social interactions in classrooms.


Two interactional patterns dominated in the observed classrooms where Barzan was present (see \citealt{WedinAho2022}); teachers led the whole-class instruction, what is commonly called “chalk and talk”, and students worked with exercises. Barzan was only included in the second type of lessons and spent his time in class walking around helping students who asked for help, mainly those who were newcomers to Swedish schools, and with whom he shared languages other than Swedish. In lessons in Natural Sciences and Mathematics, students were sometimes divided up, so that Barzan could teach those who spoke Arabic, Kurdish and Dari in a separate classroom, while the others stayed with the teacher. During other lessons, students worked on their own or in groups, and Barzan interacted with more students, not only those he shared a language with. Barzan’s role was to support students who asked for help, and he was never observed interrupting the teacher or taking initiative for the teaching of the class.

Several examples of how Barzan supported students were during the observed lesson in Social Sciences when he spent most of the time with Maryam, who had recently migrated to Sweden and knew Arabic. Their interaction, during which Maryam worked with exercises on a text in the textbook, was video-recorded. She first read the text aloud to Barzan, then the questions one at a time, looked for the answer in the text, discussed it with him and then wrote the answers. 

In the interaction, Barzan and Maryam negotiated roles and positioned themselves. In the following, it will be shown how Barzan positioned himself and was positioned as 1. the one who knows Swedish, who translates and explains, taking on a role that may be denoted as an extended interpreter role (Tables~\ref{tab:rosen:excerpt1} and~\ref{tab:rosen:excerpt2}); 2. the “Master” who supports learning by encouraging, instructing, giving advice, and who shows and is shown respect (Tables~\ref{tab:rosen:excerpt3} and~\ref{tab:rosen:excerpt4}); and 3. the one who is included in “we second language learners of Swedish” who sometimes make mistakes and who may be nervous before a test (\tabref{tab:rosen:excerpt5}).

The first position as an extended interpreter, who supports the students by translating and explaining, is illustrated in the following excerpt. (All transcripts are adapted to Standard Arabic. The original transcript is to the left and the translation to the right. In the translation, \textbf{bold} is used for words said in Swedish and \textit{italics} for what is said in English.)

\begin{table}
\caption{Excerpt 1}
\label{tab:rosen:excerpt1}
\begin{tabularx}{\textwidth}{XX}
\lsptoprule
M: (läser) hyreshus? & M: (reads) \textbf{apartment} \textbf{block?}\\
B: {\textarab{بَيت ايجار}} hyreshus & B: Apartment block\\
& \textbf{apartment} \textbf{block} \\
(…) & (…) \\
M (läser) hittills? & M: (reads) t\textbf{o} \textbf{date?} \\
B:  {\textarab{يعني لحد الآن}} & B: It means to date\\
M: Jaha, ok, {\textarab{كلمة جديدة}} & M: \textbf{Oh,} \textbf{I} \textbf{see,} a new word.\\
\lspbottomrule
\end{tabularx}
\end{table}

The excerpt in \tabref{tab:rosen:excerpt1} shows how Maryam turns to Barzan, asking for his explanation of the words \textit{hyreshus} ‘apartment block' and \textit{hittills} ‘to date’ and Barzan translates. In the first case he also repeats the Swedish word to support the student. Support with pronunciation becomes even more explicit in the excerpt in \tabref{tab:rosen:excerpt2}.


\begin{table}
\caption{Excerpt 2}
\label{tab:rosen:excerpt2}
\begin{tabularx}{\textwidth}{XX}
\lsptoprule
M: (läser) hjälpa (skrattar lite)

B:  {\textarab{أي صح}} & M: (reads) \textbf{help} (laughs a little)

B: Correct \\
\lspbottomrule
\end{tabularx}
\end{table}

In \tabref{tab:rosen:excerpt2}, Maryam indicates that she feels insecure about the pronunciation of \textit{hjälpa} ‘help’, (pronounced initially with [j]) by laughing a little and Barzan confirms that her pronunciation is correct. 

The second role, as the Master,  {\textarab{أستاذ}} (\textit{Ustaaz}) or teacher, who supports in various ways, appears in the excerpts in Tables~\ref{tab:rosen:excerpt3} and~\ref{tab:rosen:excerpt4}. \textit{Ustaaz} is a title used in apprenticeship contexts, when learners address their instructor (Master), and is also used by Maryam when addressing Barzan. 


\begin{table}
\caption{Excerpt 3}
\label{tab:rosen:excerpt3}
\begin{tabularx}{\textwidth}{XX}

\lsptoprule
M: {\textarab{لي زمان ما كتبت}}

B: {\textarab{أول شي، بأي شي تبدينه}}

M: {\textarab{أي بلحرف الكبير}} 

{\textarab{بس أنا خربتت أستاذ}}

B: {\textarab{هونة هم، حرف كبير، تصير لك تعويض بلامتحان}}

M: {\textarab{لاـ بلامتحان أنا بدقدق}}

B: {\textarab{تمام}} & M: I haven’t written for a long time 

B: First of all, when you begin 

M: Yes, with capital letter, but I got confused, \textit{Ustaaz} 

B: Also here, capital letter, so that you won’t have to repeat the test 

M: No, in the test then I’m 

meticulous

B: Good\\
\lspbottomrule
\end{tabularx}
\end{table}

Maryam excuses herself for making a mistake in her writing and Barzan takes the opportunity to point out the importance of starting with a capital letter, something that is not used in Arabic script. In this classroom the teacher was never observed being addressed by a title such as “Teacher” or “Sir”. Thus, the way Maryam addresses Barzan by {\textarab{أستاذ}}, \textit{Ustaaz}, is notable, positioning him as the Master who knows and who has a specific role of supporting learning. Barzan, as the  Master, supports Maryam by encouraging her and giving her advice before a test. Maryam continues to excuse herself for the mistakes, addressing him by {\textarab{أستا}},  \textit{Ustaaz}, and when he points out another instance of this in the text, she stresses that she will be more careful in the test situation, to which Barzan responds: 

\begin{quote}
{\textarab{تمام}}, Good.
\end{quote}


\begin{table}[b]
\caption{Excerpt 4}
\label{tab:rosen:excerpt4}
\begin{tabularx}{\textwidth}{XX}

\lsptoprule
M: Det är jag

B: {\textarab{ودي قلك، انتي هوية شايفة نفسك لا شي بالعكس انتي كلش}} 

M: {\textarab{أنا؟}}

B: {\textarab{كلش}}

M: Aha, jag fattar

B: \textarab{أكتر من اللازم ذكية ، انا تعجبت لما قالت لي أنك ماخذة} betyg C  \textarab{ما كنت بتوقع}

M: {\textarab{قصدك...}}

B: Betyg C wow & M: \textbf{That’s} \textbf{me}

B: I’ll tell you, you see yourselves as worthless but on the contrary so

M: Me?

B: Contrary …

M: \textbf{Oh,} \textbf{I} \textbf{got} \textbf{it}

B: so you are really clever, I was surprised when I heard that you got \textbf{grade} \textbf{C} I didn’t expect that.

M: You mean … 

B: \textbf{Grade} \textbf{C} \textit{wow}\\
\lspbottomrule
\end{tabularx}
\end{table}

In apprenticeship relations, the role of the Master and the learner is characterised by mutual respectful behavior, something that is apparent in Extract 4, where Maryam is reading about a student who does not do very well in school, and comments.

Barzan responds to Maryam’s devaluing of herself by encouraging her and saying that her performance is better than what he had expected. He ends the interaction with the comment “wow”. In Barzan’s role of \textit{Ustaaz}, the Master, in \tabref{tab:rosen:excerpt4} he shows respect for his apprentice, Maryam. 

In the third role, Maryam and Barzan construct a “we”, learners of Swedish, who sometimes make mistakes and who may be nervous before a test. In Tables~\ref{tab:rosen:excerpt5} and~\ref{tab:rosen:excerpt6}, Barzan positions himself as a learner of Swedish, taking on the role of a student, who struggles and sometimes makes mistakes.

\begin{table}
\caption{Excerpt 5}
\label{tab:rosen:excerpt5}
\begin{tabularx}{\textwidth}{XX}

\lsptoprule
M: {\textarab{أستاذ أنا بتكون عم بقرأ هيك لحالي بقرأ كثير منيح بس بلامتحان بلسويدي بلامتحان كان}}

(skakar händerna)

B: {\textarab{أي أنا كان عندي هَيّ المشكلة من أقرأ من أقرأ ما بقدر أركز بس إذا  أقرأ بقلبي هون لا أفتهم}}

M: {\textarab{لا أنا بقدر أركز}} men om jag har prov, katastrof & M: Ustaaz, when I read like this on my own I read very well but in the test, in the Swedish test, it was (shakes both her hands)

B: Yes, I had the same problem, when I read, when I read out loud I can’t concentrate but when I read silently to myself I understand

M: No, I can concentrate \textbf{but} \textbf{when} \textbf{I} \textbf{have} \textbf{a} \textbf{test,} \textbf{catastrophe.}\\
\lspbottomrule
\end{tabularx}
\end{table}

When Maryam refers to problems in a test situation, Barzan responds by saying that he had similar problems. The use of the past tense (“had” “read”) implies that this was earlier, positioning him as a former student. Here, however, Maryam addresses the problem of being nervous, pointing out that she can concentrate but is nervous during tests, while Barzan described a lack of concentration when reading out loud. Thus, this is an example of how Barzan positions himself as one among Swedish learners. He sits down next to Maryam and describes that he also experienced the hardship of learning in a second language.  In \tabref{tab:rosen:excerpt6}, they both admit having made the same mistake when reading in the textbook. Barzan points to the word \textit{fritid} ‘spare time’ and and says that he read it as the similar word \textit{framtid} ‘future’. This action by Barzan can be seen as face-saving for Maryam, by showing that we all make mistakes, and that it is okay to do so when learning.

\begin{table}
\caption{Excerpt 6}
\label{tab:rosen:excerpt6}
\begin{tabularx}{\textwidth}{XX}

\lsptoprule
B: Fritiden {\textarab{أنا قريتليك فرامتيدن}}

M: {\textarab{وأنا كمان،}} & B: \textbf{Spare} \textbf{time} I read ‘future’ to you  

M: Oh, me too \\
\lspbottomrule
\end{tabularx}
\end{table}

In common with \tabref{tab:rosen:excerpt5}, \tabref{tab:rosen:excerpt6} shows Barzan sympathising with Maryam in the hardship of learning a second language and admitting that he too sometimes makes mistakes.  

In the excerpts in Tables~\ref{tab:rosen:excerpt1}--\ref{tab:rosen:excerpt5}, analysis of the interaction order in the classroom shows how Barzan positions himself and is positioned as SGMT assistant by his student, thus contributing to our understanding of the roles that the SGMT assistant assumes in this context.

\subsection{Historical bodies of the participants and the institutional setting}

In this section, an analysis of the historical bodies will be presented. The analysis is based on the collectively constituted experiences of individuals as well as institutions across time, expressed in interviews and talks with Barzan, teachers and principals, as well as observations of their practices.  

In interviews, Barzan described his educational and linguistic background. His parents were Kurds from Iraq who migrated to Iran where Barzan and his siblings were born. The dominant language in the family was the Kurdish variety Sorani, while the language he used with his friends in the neighborhood was Arabic. Farsi was the medium of instruction in the school. At the age of 20, Barzan returned to Iraq with his parents, and he studied general Physics at the university, where Arabic was the dominant language. Thus, he had his first twelve years of schooling in Farsi but took his academic degree in Arabic. 

As an immigrant in Sweden, he studied Swedish as a second language and he has children in pre-school and primary school. Thus, Barzan shared the experience of being a second language student, in his case in Iran, Iraq and Sweden, with the students in the LIP. He had also studied a university course to prepare for being a SGMT assistant (7.5 ECTS), similar to those presented in \sectref{sec:rosen:4.3}, “Discourses in place”. Thus, his main competences in relation to those competences outlined in national policy documents as important for SGMT assistants (\citealt{Swedish_national_agency_of_education2015}) are linguistic competence in Swedish, Farsi, Arabic and Sorani and academic competence from his education in Iraq and Iran including general subject knowledge; specifically in Natural Sciences and Mathematics. He also has experience of the Swedish education system as a parent, a student in Swedish as a second language, as a university student, studying a course for SGMT assistants, and from his work as a SGMT assistant in secondary school. From observations, it is also clear that he has developed knowledge about his role, students' needs, teachers’ varied understanding of his role and about his place in the organisation, gained through his experience as an SGMT assistant,

Teachers who worked in the LIP at this particular school had varying qualifications and competences. All teachers who were observed (14) had a teacher’s degree, and eight of them were qualified teachers in Swedish as a second language with experience of teaching in multilingual classrooms.  During observations and interviews, both principals and many of the teachers expressed uncertainty concerning the role of SGMT assistants in the school. One principal referred to the assistants as “mother tongue teachers” and another said that she was happy to have hired “a teacher in Tigrinya”, referring to another SGMT assistant. When a third principal talked about SGMT assistants she referred to them as “reception teachers I may say Mother Tongue Tuition teachers. This is what they’re called on paper but here they also work with a reception group, those who are most new with us”. This statement refers to the fact that Barzan had come to an agreement with the principal that he took the role as a teacher in Mathematics for a group of recently arrived students, and that this was also the case with some of the other SGMT assistants, despite the fact that they lacked teacher education in the subject. Thus, it seems that from the perspective of the principal, competence in a language shared with the students was perceived as more important than Swedish language skills, pedagogical knowledge or subject knowledge. Both principals and teachers, however, also expressed ambiguous views of students’ linguistic competence. On one hand they were aware of the importance of acknowledging students’ varied linguistic repertoires in the classrooms, but on the other hand, some of them expressed doubts about the benefit of multilingual practices in classrooms. For example, while one of the three principals stressed that it was important for students to be able to use their other languages and to explain things to each other using languages they shared, the two other expressed skepticism towards multilingualism. One claimed that students should be encouraged to use only Swedish and the other argued: “They have to start talking Swedish with each other” [otherwise] “they easily only resort to Somali”. While all the teachers of Swedish as a second language expressed positive views towards the use of various linguistic resources among students, some of the other teachers explicitly stated that they preferred that students use only Swedish in the classroom, such as in \tabref{tab:rosen:excerpt7}, from an interview with one teacher (T) and the interviewer (I) (emphasis added).


\begin{table}
\caption{Excerpt 7}
\label{tab:rosen:excerpt7}
\begin{tabularx}{\textwidth}{XX}

\lsptoprule
I: Så du menar att det är mer ett problem i och med att dom inte pratar svenska?

L: Kaos. Det är som, ja det är som ett kaotisk ja. Man kan, alltså man kan använda ordet mångfald bara för att, va heter det skönmåla det. [skratt] Nej det är sådär mångfald man använder positiv.

I: Så du menar att det är ett språkligt kaos?

L: Alltså jo, jo vilken, jo det är språkligt kaos eftersom alltså vi som skola vi vill att dom ska lära sej svenska och borde ge och ha nåt, borde ge verktyg för att komma på ett annat sätt att dom kan tala svenska, i den sociala miljön. & I: So you mean that it is more of a problem when they don’t talk Swedish

T: Chaos. It’s like, yes it’s like  chaotic yes. You can, that is you can use the word diversity to, you know, give it a positive twist [laughs] No it’s like diversity, you use that to describe something in positive terms.

I: So, do you mean that it is linguistically chaotic?

T: I mean, yes, yes what, yes it’s linguistically chaotic because, that is, we as a school we want them to learn Swedish and ought to give and have something, ought to give them tools to find another way for them to talk Swedish, in the social environment. \\
\lspbottomrule
\end{tabularx}
\end{table}



This teacher in Social Sciences expressed a negative view of students’ use of other languages, which he compares to chaos and also expresses a wish for a school policy stating that Swedish should be used.

Explicit collaboration between Barzan and the teachers was only observed during classes in Swedish as a second language and not in other subjects. During the interview, Barzan said that in the course for SGMT assistants that he had attended, he had received a guide planning collaboration with teachers, which he then had presented for the teachers at a team meeting. However, only one teacher had accepted the guide and only in relation to one student. As a result of the course for SGMT assistants, Barzan also took the initiative to create bilingual wordlists for students in various subjects. He collaborated with other SGMT assistants and mother tongue tuition teachers. Moreover, to help as many of the students as possible, he found other people in and outside school that spoke the same languages he did and also Somali and Tigrinya. However, he said that only some of the teachers had shown interest. Barzan’s conclusion from this experience was that the teachers saw him (and other SGMT assistants) as interfering with their work rather than as supporting colleagues. He claimed that when the School Inspectorate had investigated the school, one thing that they had questioned was the role of SGMT assistants. Thus, he had suggested to one principal that someone should be appointed to take responsibility for SGMT in the LIP programme. 

Barzan expressed his understanding of power relations at the school claiming that teachers are the ones who decide, and not SGMT assistants or principals. He said that during meetings with the principal, he had suggested changes to support the students’ subject learning, but that this advice had been rejected by teachers, who did not want to collaborate with SGMT assistants. However, this was not the case for all teachers; Barzan had also experienced more successful collaborations. Regardless, his overall impression was that he was treated as inferior to the teachers, who decided if, when and how collaboration could take place.  

Through the historical body of Barzan, represented in this section through his talk and classroom practices, a picture emerges of how his competence and role are negotiated. Barzan’s linguistic competence, university studies in Physics and his own experiences of being a multilingual student are not valued in the school. Rather, he is positioned as inferior by some teachers. His perceptions, firstly of not being given agency in relation to teachers and secondly, that it is teachers and not him who are the ones who have the power to decide, shows that he does not feel heard, not even when he presents material related to the university course that he took part in. Most teachers did not acknowledge his experiences or knowledge. Thus, his inferior position in relation to the teachers stands in stark contrast to the mutual respect expressed between him and Maryam in the previous \sectref{sec:rosen:4.1}. The professional role negotiated in the interaction order with the students collides with how Barzan’s historical body is positioned and negotiated in relation to teachers, placing the SGMT assistant in an ambivalent position. 

\subsection{Discourses in place} \label{sec:rosen:4.3}%4.3. /

In this section we present our analysis concerning research question 2 that asks what knowledge is expressed as central for the profession of the SGMT assistant in policy documents and course syllabi. As mentioned earlier, Barzan had participated in a university course for SGMT assistants which had provided him with ideas about how to collaborate with the teachers at the school he worked at. 

University courses as well as national policy documents produced discourses in place that enabled certain practices, presented in previous sections. While teacher education is regulated by the Swedish Higher Education Act (\citealt{Swedish_ministry_of_education1992}) and the Higher Education Ordinance (\citealt{Swedish_ministry_of_education1993}), there are no formal educational requirements for assistants in SGMT. It is the principal at each school who decides who can be employed as an SGMT assistant. Hence, in order to analyse the discourses in place in the practice of SGMT, we include the policy material produced by the Swedish National Agency of Education and syllabi from three universities that offer courses aimed at SGMT assistants. The Swedish National Agency of Education has highlighted that the following aspects have a positive impact on SGMT (\citealt[32–33]{Swedish_national_agency_of_education2015}):

\begin{itemize}
\item The SGMT assistant has good competence in the mother tongue and the school system in the country where the student previously attended school.
\item The SGMT assistant has well-developed linguistic awareness, including subject language both in Swedish and the mother tongue and can use that in order to plan how to use the languages to support the student.
\item The SGMT assistant is knowledgeable in pedagogy, including pedagogical awareness and the ability to use different methods. Moreover, the assistant has knowledge about subject teaching in the Swedish school and can evaluate how to support the student.
\item The SGMT assistant is familiar with the Swedish school system and the legal regulations for the type of school that she/he is working in. The assistant needs to be knowledgeable about the national curriculum, syllabi and individual support activities.
\item The SGMT assistant is focused on and familiar with the goals and requirements that the students need to achieve. The assistant can plan the guidance with the student and the subject teacher in relation to the goals and requirements and be familiar with the prior knowledge needed to comprehend the subject.
\item The SGMT assistant is familiar with the school subjects that she/he is guiding the students in.
\end{itemize}

The Swedish National Agency of Education  argues that these six aspects in relation to the competence of SGMT assistant are important in order for the support to be successful, but also that many schools face challenges in finding assistants that comply with the recommended competences  (SNAE, \citealt{Swedish_national_agency_of_education2015}). The competence of SGMT assistants can be described in terms of a) linguistic competence in the two languages (including linguistic awareness) and competence in b) the Swedish school system and policy, c) the school system in other countries from where the students have migrated, d) subject knowledge, and e) pedagogy.

During the last decade, various courses have been developed to educate SGMT assistants, courses that commonly cover about 7.5 ECTS. For this study we have included courses from three universities. Two courses are oriented towards SGMT in general and are both worth 7.5 ECTS (University B and C), whereas one is oriented towards SGMT and mother tongue teaching and is worth 15 ECTS (University A). The analysis of the syllabi shows that students in these courses are expected to achieve knowledge in several fields, including linguistic knowledge and awareness, pedagogy, intercultural aspects, Swedish school system and policy and collaboration.

Linguistic knowledge and awareness are mentioned in all syllabi but mainly emphasised in syllabi oriented towards subject language. None of the syllabi mention general competence in either Swedish or the other languages in which the assistant is expected to work. In one syllabus, students are expected to identify and use different types of written expressions, and during the course they are given the opportunity to practice different types of writing (Uni A). In all syllabi, the language development of students is mentioned, stating learning outcomes such as knowledge about theories and research in multilingualism and their importance for language and learning among multilingual students and newcomers (Uni B), parallel development of language and knowledge and relations between everyday language, academic language and subject language (Uni C). Moreover, the syllabus at Uni A includes learning outcomes related to working with texts across subjects, differences between oral and written texts and text genres in different subjects.  

Knowledge about pedagogy is included in the learning outcomes of all syllabi. Students are expected to plan and discuss successful learning situations and environments for their students’ language development (Uni A), analyse pedagogical choices in their own practices in relation to the theoretical framework of the course and their students’ learning abilities (Uni B), and describe how digital tools can be integrated into the pedagogical practices (Uni B and C). Moreover, students should discuss the pedagogical approaches used in SGMT and their role as an assistant and describe pedagogical models and approaches (Uni C). 

Intercultural competence or an intercultural approach is mentioned in two of the syllabi, one stating that:

\begin{quote}
students should be able to account for central aspects of the school’s values and views of knowledge and to discuss these from the point of intercultural competence. (Uni B)
\end{quote}

Similarly, the learning outcomes in the syllabus for Uni C are about describing and discussing an inclusive and intercultural approach in school in general and in SGMT specifically. 

In regard to the Swedish school system and policy, two syllabi include learning outcomes where students are expected to account for basic policy and regulations in the Swedish school system and the fundamental values and tasks of the school (Uni B and C). Finally, two syllabi include learning outcomes where students should discuss successful forms of cooperation between SGMT assistants and teachers (Uni B and C).

Comparing the six aspects formulated by the Swedish National Agency of Education and the learning outcomes and content in the three syllabi, there are similarities and also differences in what is constructed as legitimate knowledge for SGMT assistants. The linguistic competence of the assistants in Swedish or the language/languages that they are expected to give support in are not mentioned in the syllabi. Language competence, including the ability to mediate and move between languages, is thus either expected to be something that assistants already have, or it is not considered important. The “in-betweenness” in regard to interculturality is mentioned in two syllabi but then mainly in terms of approaching the values and goals in the Swedish curriculum rather than as an aspect of the instruction in the different subjects. To sum up the analysis of discourses in place: even though there is a lack of a general teacher education for SGMT assistants, the national policy documents as well as the syllabi of the university courses analysed show the diverse competences needed to perform the work successfully. Thus, the course material as well as national policy documents became relevant for and were used by Barzan to claim his position in the school in relation to the principal, as apparent through interaction order and historical bodies.

\section{Discussion and conclusions}
\begin{sloppypar}
The aim of this chapter was to examine and problematise the social action of SGMT assistants by focusing on the construction of legitimate knowledge. The chosen example is from an upper secondary school and some aspects may thus not be valid for primary school, lower secondary school, or adult education which have other types of organisation. Through nexus analysis, here with Barzan in focus, the complex and ambiguous role of SGMT assistants, highlighted also by \citet{Rosen2019-1, Rosen2020}, \citet{Reath_warren2017} and \citet{St_john2021}, becomes apparent. The analysis of discourses in place shows that although the SGMT assistants in most cases lack a formal teacher degree, eligibility criteria for the job include advanced language skills (Swedish and the language of SGMT), pedagogy, subject knowledge and teaching, interculturality as well as knowledge about the Swedish school system and policy. These diverse and advanced competences also become visible through the different roles Barzan takes on in the classroom in relation to the students and in his reported conversations with teachers and principals. However, in relation to the teachers in the school, Barzan is mainly expected to translate with minor impact on the content and the instruction. He expresses feelings of inferiority and lack of agency in relation to teachers.  Despite his several attempts to develop collaboration, using tools that he had developed based on the university course on SGMT that he took, few teachers showed interest or engagement. Barzan perceives that most teachers do not acknowledge his competence except that as an interpreter in the classroom, something that left him feeling frustrated, and that he was not able to support students in an effective way. At the same time, however, Barzan was also expected to take on the responsibility as a teacher for the students during certain classes.
\end{sloppypar}

The in-betweenness of the role of SGMT assistant (\citealt{St_john2021})  was expressed in some of the course syllabi in relation to interculturality. This in-betweenness also became visible through the historical bodies and interaction order in the classroom. The difference between Barzan’s perception of how he was positioned by teachers and how he was positioned and positioned himself in relation to Maryam shows how the role and legitimate knowledge of the SGMT assistants are intertwined and contextual in the social practices. Although Barzan is positioned as valuable and knowledgeable both in policy and by the principals, this had only a minor impact on his role in the classroom and in his relationships with the teachers. Thus, Barzan had few opportunities to actually use the knowledge and competences he developed in the university course on SGMT. Competence and expected competence are related to this variation in roles, where Maryam positions him as knowledgeable while many teachers position him as mainly a translator. Also, the principal’s decision to position him as teacher based on his linguistic competence to some extent makes his lack of teacher education invisible, putting linguistic competence in focus.

The role of principals becomes clear here, both in terms of giving the SGMT assistants at the school clear instructions regarding the expected outcomes of their work and also creating a framework for collaboration between teachers and SGMT assistants. Although Barzan talks about attempts made by the principal to improve the support through SGMT at the school, these attempts have not resulted in any real change. The lack of knowledge among principals and teachers about the function and role of SGMT at this specific school and also in Swedish schools in general (\citealt{WedinRosen2022}) means that there is a two-fold risk. On the one hand SGMT assistants may be given responsibility for teaching, without having the necessary competence, while on the other hand they may be perceived as lacking competence and thus become marginalised at the school. The lack of a clear policy in regard to SGMT assistants’ competence and professional role in the Swedish school creates a situation of uncertainty and ambivalence for both the assistants and teachers as well as students. Thus, the conditions for SGMT may differ between schools, and students are not provided with equal possibilities; a problem also raised by the \citet{Swedish_school_inspectorate2017}.  Furthermore, the uncertainty may result in the work of SGMT assistants becoming invisible in the school or that SGMT assistants are left with the responsibility of newcomers without the support from qualified subject teachers. Newcomers, such as those in the LIP, are often under a lot of pressure to complete their studies in order to be able to continue in the education system. SGMT assistants can play a key role in creating a successful learning environment for these students, but this requires structured collaboration with teachers, building on a shared responsibility for the students. In order to enable a successful collaboration, the asymmetric relations between teachers and SGMT assistants needs to be addressed.

This study highlights the importance of teacher education for preparing pre-service teachers to work with newcomers as well as with second language learners more generally. Since newcomers have the right to SGMT, teachers need to be prepared to collaborate with SGMT assistants and to understand their different roles and responsibilities. As has been shown by \citet{Hermansson2022} and \citet{WedinRosen2022}, education in Sweden for teachers and principals does not prepare them to support newcomers or second language learners. The discontinuity between documents and policies on an official level and attitudes and practices on classroom levels that these earlier studies showed are reflected in the results of this study as well. Here, the lack of alignment between the course given to SGMT assistants and the attitudes of teachers and principals reveals a need for consistency through all teacher education to create conditions for equitable education for students who are newcomers and more generally for multilingual students.

\section*{Acknowledgements}

We would like to express our thanks to Lovisa Berg who transcribed the Arabic dialogue, and helped us with the interpretation of the interaction between Barzan and Maryam. The study of Barzan was part of the research project \textit{Recently arrived students in Swedish upper secondary school:  A multidisciplinary   study on language development, disciplinary literacy and social inclusion} (2018--2021), which was financed by the Swedish Research Council, grant number 2017-03566.

\printbibliography[heading=subbibliography,notkeyword=this]
\end{document}
