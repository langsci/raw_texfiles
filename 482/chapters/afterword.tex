\documentclass[output=paper]{langscibook}
\ChapterDOI{10.5281/zenodo.15280909}
\author{Ingrid Piller\orcid{}\affiliation{Macquarie University, Australia}}
\title{Changing teachers’ monolingual habitus}
\abstract{\noabstract}
\IfFileExists{../localcommands.tex}{
  \addbibresource{../localbibliography.bib}
  % add all extra packages you need to load to this file

\usepackage{tabularx,multicol}
\usepackage{url}
\urlstyle{same}

\usepackage{listings}
\lstset{basicstyle=\ttfamily,tabsize=2,breaklines=true}

\usepackage{langsci-basic}
\usepackage{langsci-optional}
\usepackage{langsci-lgr}
\usepackage{langsci-osl}
% \usepackage{./langsci/styles/langsci-lgr}
% \usepackage{./langsci/styles/langsci-osl}
% \usepackage{langsci-gb4e}

\usepackage{tikz}
\usetikzlibrary{patterns,calc}
\pgfdeclarepatternformonly{south east lines}{\pgfqpoint{-0pt}{-0pt}}{\pgfqpoint{3pt}{3pt}}{\pgfqpoint{3pt}{3pt}}{
    \pgfsetlinewidth{0.6pt}
    \pgfpathmoveto{\pgfqpoint{0pt}{3pt}}
    \pgfpathlineto{\pgfqpoint{3pt}{0pt}}
    \pgfpathmoveto{\pgfqpoint{.2pt}{-.2pt}}
    \pgfpathlineto{\pgfqpoint{-.2pt}{.2pt}}
    \pgfpathmoveto{\pgfqpoint{3.2pt}{2.8pt}}
    \pgfpathlineto{\pgfqpoint{2.8pt}{3.2pt}}
    \pgfusepath{stroke}}
    
\usepackage{stmaryrd}
\usepackage{wasysym}
\usepackage{multirow}
\usepackage{caption}
\usepackage{subcaption}
\usepackage{mathrsfs}
\usepackage{qtree}

\usepackage{linguex}


  %pminos do not split footnotes
% \interfootnotelinepenalty=10000 %Footnote in Laporte chapters has to be split SN


%\DeclareIndexNameFormat{default}{%
%\nameparts{#1}%
%\usebibmacro{index:name}%
%{\index[names]}%
%{\namepartfamily}%
%{\namepartgiveni}%
% {}% L1
% {}% L2
%{\namepartprefix}% generates spurious space L3
%{\namepartsuffix}% generates spurious space L4
%}

%  {\DeclareIndexNameFormat{default}{%
%     \usebibmacro{index:name}{\index[names]}{#1}{#3}{#5}{#7}}}

%\DeclareIndexNameFormat{default}{%
%  \usebibmacro{index:name}{\sindex[nom]}{#1}{#3}{#5}{#7}}

%\DeclareIndexNameFormat{default}{%
%  \usebibmacro{index:name}{\sindex[person]}{#1}{#3}{#5}{#7}}
%\DeclareIndexNameFormat{default}{%
%\nameparts{#1} \usebibmacro{index:name}{\sindex[person]]}{\namepartfamily}{‌​\namepartgiven}{\nam‌​epartprefix}{\namepa‌​rtsuffix}}

%\newcommand{\smiley}{:)}

%\renewbibmacro*{index:name}[5]{%
%\usebibmacro{index:entry}{#1}%
%{\iffieldundef{usera}{}{\thefield{usera}\actualoperator}\mkbibindexname{#2}{#3}{#4}{#5}}}

% \newcommand{\noop}[1]{}

%remove for final
%\overfullrule=1mm

\newcommand{\tobi}[2]}}
\renewcommand{\S}[1]{\tobi{#1}{\textsc{*}}}

% this volume references
% puts: [this volume]
% already defined: \citetv
%\newcommand{\citepv}[1]{(\citeauthor{#1} \citeyear*{#1} [this volume])}
\newcommand{\citealtv}[1]{\citeauthor{#1} \citeyear*{#1} [this volume]}

%parentheses around example number
\newcommand{\pref}[1]{(\ref{#1})}

% in-text examples

\newcommand{\lnex}[1]{\textit{#1}} %target lang word
\newcommand{\lnlit}[1]{(lit.: `#1')} %literal reading
\newcommand{\lnlat}[1]{(#1)} % latinization
\newcommand{\lntrans}[1]{`#1'} %translation
\newcommand{\lnexl}[2]%
{\lnex{#1}{} \lnlat{#2}} % ex with latinization
\newcommand{\lnexlat}[3]{\lnex{#1}{} \lnlat{#2}{} \lntrans{#3}} % ex with latinization and tranl.

%ch01
\newcommand{\co}[1]{\mbox{\textbf{#1}}}

%ch09

\newcommand{\cyrbulg}[1]{\begin{otherlanguage*}{bulgarian}#1\end{otherlanguage*}}


%ch10
\newcommand{\nlp}{{\small NLP}}
\newcommand{\mwe}{{\small MWE}}
\newcommand{\rae}{{\small RAE}}
\newcommand{\lvc}{{\small LVC}}
\newcommand{\pos}{{\small P}o{\small S}}
%\newcommand{\todo}[1]{ \textcolor{red}{#1} }

%\renewcommand{\labelenumi}{\theenumi}
%\ainamefmt{{vv}{ll}{, ff}{, jj}} % fullname

\newcommand{\biberror}[1]{{\color{red}#1}}

\newcommand{\osenovaitem}{--~} 
  %% hyphenation points for line breaks
%% Normally, automatic hyphenation in LaTeX is very good
%% If a word is mis-hyphenated, add it to this file
%%
%% add information to TeX file before \begin{document} with:
%% %% hyphenation points for line breaks
%% Normally, automatic hyphenation in LaTeX is very good
%% If a word is mis-hyphenated, add it to this file
%%
%% add information to TeX file before \begin{document} with:
%% %% hyphenation points for line breaks
%% Normally, automatic hyphenation in LaTeX is very good
%% If a word is mis-hyphenated, add it to this file
%%
%% add information to TeX file before \begin{document} with:
%% \include{localhyphenation}
\hyphenation{
    Beck-man
    Ngu-yen
    back-chan-nel
    back-chan-nels
    mo-not-o-nous
    ste-reo-typ-i-cal
}

\hyphenation{
    Beck-man
    Ngu-yen
    back-chan-nel
    back-chan-nels
    mo-not-o-nous
    ste-reo-typ-i-cal
}

\hyphenation{
    Beck-man
    Ngu-yen
    back-chan-nel
    back-chan-nels
    mo-not-o-nous
    ste-reo-typ-i-cal
}
 
  \togglepaper[1]%%chapternumber
}{}

\begin{document}
\maketitle 
%\shorttitlerunninghead{}%%use this for an abridged title in the page headers

\noindent 
Education systems around the world struggle with the reality of linguistic diversity and frame it in contradictory ways. There is the positive goal of language learning and admiration for elite bilinguals. This coexists with language panic about the decline of the national language and deficit views of the language practices of migrant students.

\textit{Teacher education for working in linguistically diverse classrooms: Nordic perspectives} exposes how these contradictions and tensions play out in the Nordic countries in the terrain of teacher education.

The Nordic countries are often seen as global beacons of modernity, social inclusion, and equal opportunities \parencite{chapters/1_reath}. Yet, even so, contradictions abound. There are the contradictions between policies that champion linguistic inclusion and the reality of their absence in the monolingual curricula for preservice teachers (\citealt{chapters/2_Gunnthorsdottira}). There are also the contradictions between hiring bilingual teachers to ensure support for newcomer students and the realities that there are no clear roles for these teachers within a school’s organisation (\citealt{chapters/3_rosen}). Then, there are the contradictions between championing multilingual learning in theory but reducing it to a language-free learner-centered approach in practice \parencite{chapters/4_gudjonsdottir}. And finally, there are the contradictions between teachers being equipped with theoretical foundations in multilingual learning but not the practical tools to teach multilingual students \parencite{chapters/7_alisaari,chapters/8_heikkola,chapters/6_iversen,chapters/5_ostergaard}. 

\begin{sloppypar}
All these contradictions are embedded within the wider contradiction of the Nordic countries being internationally touted as educational success stories, while the reality of widespread migrant student failure is swept under the carpet. As attested by the exceptional performance of Nordic countries on the OECD’s Programme for International Student Assessment (PISA) (\citealt{OECD2023}), the level of educational attainment is high across the region. Due to these successes, some Nordic countries, most notably Finland, have been held up as international role models for strong education systems throughout the 21st century (\citealt{Ahonen2021}). 
\end{sloppypar}

Yet the same PISA reports that place Nordic countries in the international vanguard when it comes to overall student performance in Mathematics, Reading, and Sciences also place them last when it comes to immigrant student performance. Non-immigrant students significantly outperform immigrant students in the Nordic countries. The same is true in many parts of the world. Elsewhere, however, the score difference between immigrant and non-immigrant students disappears after controlling for socio-economic status and language status. This is not the case in the Nordic countries, with Finland, Sweden, and Denmark having the greatest score differences between immigrant and non-immigrant students internationally in Mathematics, \textit{after} controlling for socio-economic and language status (\citealt[217]{OECD2023}). The same is true for disparate Reading scores, where all Nordic countries featured in this book (Finland, Sweden, Iceland, Denmark, Norway, in this order) are assembled at the bottom of the international comparative dataset \citep[218]{OECD2023}.

How can the Nordic countries’ outstanding commitment to educational equity and the high quality of education there co-exist with such dismal results when it comes to the education of immigrant children? The gap between inclusive policies and their operationalisation in teacher education documented in this volume may be part of the explanation. 

But the studies assembled here raise a broader conundrum: how can we get linguistically diverse education right globally if the Nordic countries with all their advantages cannot get it right?

\begin{sloppypar}
The contradictions and tensions related to the education of linguistically diverse students have been termed “the monolingual habitus of multilingual schools” by German education researcher Ingrid \citet{Gogolin1994,Gogolin1997}. Schools erase the de facto linguistic diversity that is present in any community by instituting a narrow set of monolingual policies and practices. 
\end{sloppypar}

The monolingual habitus goes back to the 19\textsuperscript{th} century when both the European nation state and formal education were simultaneously institutionalised. Although migration and globalisation have diminished the power of the nation state, the monolingual habitus of schooling remains in force (\citealt{PillerEtAl2024}). 

Old habits die hard.

What the contributions assembled in this volume show is, in fact, substantial change on the policy level: there are policies to support the language learning of newcomers, there are policies to strengthen heritage languages, there are policies to include new migrants, and there are policies to educate prospective teachers about multilingualism.

Yet, changing habitus, “a system of durable and transposable dispositions” (\citealt[261]{Bourdieu1993}), requires more than policies, because it is sedimented both in the individual and the collective.

The dispositions teachers bring to linguistically diverse classrooms are not only informed by ideas and experiences with language but also by ideas about the nation. Discussions about linguistic diversity are too often discussions about migrants and migration, as is evident from the definitional efforts expanded on classifying and counting immigrant students, who may or may not be immigrants themselves \parencite{chapters/1_reath}.

To make progress towards changing the monolingual habitus of schooling we need to disentangle migration and linguistic diversity.

Linguistic diversity is a fact of life, “the normal human experience” (\citealt{Goodenough1976}), and all students – not just students classified as migrants – need support to develop high levels of multilingual proficiencies.

The language education challenge is threefold: first, all students must develop their proficiency in the school language. The academic proficiencies in the school language required to progress through education are no one’s native language. Some students however can build on the linguistic resources acquired in the home to extend their academic literacies in the school language while others start from scratch, with many different linguistic and cultural constellations in between (\citealt{Heath1982}). Teachers need to be socialised into a habitus that supports the academic language development of all students along with their content learning as a key task of schooling.

Second, most students need to learn one or more instructed languages. Except for students in the core Anglophone countries, English language learning is today inevitable for students internationally. How to teach English (and other instructed languages) effectively so that students can communicate in the global lingua franca constitutes a major teaching challenge that today falls no longer only on English language teachers but, with the expansion of English as medium of instruction programmes, is a question ever more teachers need to grapple with.

Third, students whose home language is substantially different from the school language should have the opportunity to also develop high levels of proficiency, including academic literacies, in their home language. To achieve this, the expansion of heritage language programmes is sorely needed. In fact, what is desirable is the expansion of dual language programmes that raise the status of languages often considered of little value. The usual objection “we can’t teach every language” does not hold because an increasing body of research now shows that dual immersion programmes are beneficial for all minoritised children, even if their personal heritage language is not involved. For instance, \citet{Purkarthofer2021} shows that smaller class sizes, better resources, and teachers who are more supportive of multilingual students benefit all students in a dual immersion programme regardless of their actual heritage language.

\textit{Teacher education for working in linguistically diverse classrooms: Nordic perspectives} marks an important milestone in the ongoing quest to change the monolingual habitus of the education system. The findings are relevant internationally: we need to break the ideological connection between migration and linguistic diversity to help build rich multilingual repertoires for all students.

\sloppy\printbibliography[heading=subbibliography,notkeyword=this]
\end{document}
