\documentclass[output=paper]{langscibook}
\ChapterDOI{10.5281/zenodo.15280903}
\author{Jonas Yassin Iversen\orcid{}\affiliation{Inland Norway University of Applied Sciences} and Wenche Elisabeth Thomassen\orcid{}\affiliation{University of Stavanger} and Sandra Fylkesnes\orcid{}\affiliation{Østfold University College}}
\title[Teachers’ orientations, knowledge and skills regarding multilingualism]{Norwegian pre-service teachers’ orientations, knowledge and skills regarding multilingualism}
\abstract{In this chapter, we investigate 106 Norwegian pre-service teachers’ views and experience with multilingualism from different stages of their teacher education and from four different teacher education institutions. The data were collected as part of three recent doctoral studies (\citealt{Fylkesnes2019,Iversen2020,Thomassen2021})  and consist primarily of focus group and individual interviews. Drawing on \citeauthor{LucasVillegas2011}' (\citeyear{LucasVillegas2011,Lucas2013}) framework for preparing linguistically responsive teachers, we focus our analytical attention on  what orientations, knowledge and skills the pre-service teachers demonstrate. Our analysis indicates that pre-service teachers generally have the fundamental orientations necessary for linguistically responsive teaching, yet they seem to lack the necessary pedagogical knowledge and skills needed to enact these orientations in their teaching. Based on our findings, we argue that Norwegian teacher education needs to put greater emphasis on developing pre-service teachers’ pedagogical knowledge and skills if they are to enact linguistically responsive teaching.}
\IfFileExists{../localcommands.tex}{
  \addbibresource{../localbibliography.bib}
  % add all extra packages you need to load to this file

\usepackage{tabularx,multicol}
\usepackage{url}
\urlstyle{same}

\usepackage{listings}
\lstset{basicstyle=\ttfamily,tabsize=2,breaklines=true}

\usepackage{langsci-basic}
\usepackage{langsci-optional}
\usepackage{langsci-lgr}
\usepackage{langsci-osl}
% \usepackage{./langsci/styles/langsci-lgr}
% \usepackage{./langsci/styles/langsci-osl}
% \usepackage{langsci-gb4e}

\usepackage{tikz}
\usetikzlibrary{patterns,calc}
\pgfdeclarepatternformonly{south east lines}{\pgfqpoint{-0pt}{-0pt}}{\pgfqpoint{3pt}{3pt}}{\pgfqpoint{3pt}{3pt}}{
    \pgfsetlinewidth{0.6pt}
    \pgfpathmoveto{\pgfqpoint{0pt}{3pt}}
    \pgfpathlineto{\pgfqpoint{3pt}{0pt}}
    \pgfpathmoveto{\pgfqpoint{.2pt}{-.2pt}}
    \pgfpathlineto{\pgfqpoint{-.2pt}{.2pt}}
    \pgfpathmoveto{\pgfqpoint{3.2pt}{2.8pt}}
    \pgfpathlineto{\pgfqpoint{2.8pt}{3.2pt}}
    \pgfusepath{stroke}}
    
\usepackage{stmaryrd}
\usepackage{wasysym}
\usepackage{multirow}
\usepackage{caption}
\usepackage{subcaption}
\usepackage{mathrsfs}
\usepackage{qtree}

\usepackage{linguex}


  %pminos do not split footnotes
% \interfootnotelinepenalty=10000 %Footnote in Laporte chapters has to be split SN


%\DeclareIndexNameFormat{default}{%
%\nameparts{#1}%
%\usebibmacro{index:name}%
%{\index[names]}%
%{\namepartfamily}%
%{\namepartgiveni}%
% {}% L1
% {}% L2
%{\namepartprefix}% generates spurious space L3
%{\namepartsuffix}% generates spurious space L4
%}

%  {\DeclareIndexNameFormat{default}{%
%     \usebibmacro{index:name}{\index[names]}{#1}{#3}{#5}{#7}}}

%\DeclareIndexNameFormat{default}{%
%  \usebibmacro{index:name}{\sindex[nom]}{#1}{#3}{#5}{#7}}

%\DeclareIndexNameFormat{default}{%
%  \usebibmacro{index:name}{\sindex[person]}{#1}{#3}{#5}{#7}}
%\DeclareIndexNameFormat{default}{%
%\nameparts{#1} \usebibmacro{index:name}{\sindex[person]]}{\namepartfamily}{‌​\namepartgiven}{\nam‌​epartprefix}{\namepa‌​rtsuffix}}

%\newcommand{\smiley}{:)}

%\renewbibmacro*{index:name}[5]{%
%\usebibmacro{index:entry}{#1}%
%{\iffieldundef{usera}{}{\thefield{usera}\actualoperator}\mkbibindexname{#2}{#3}{#4}{#5}}}

% \newcommand{\noop}[1]{}

%remove for final
%\overfullrule=1mm

\newcommand{\tobi}[2]}}
\renewcommand{\S}[1]{\tobi{#1}{\textsc{*}}}

% this volume references
% puts: [this volume]
% already defined: \citetv
%\newcommand{\citepv}[1]{(\citeauthor{#1} \citeyear*{#1} [this volume])}
\newcommand{\citealtv}[1]{\citeauthor{#1} \citeyear*{#1} [this volume]}

%parentheses around example number
\newcommand{\pref}[1]{(\ref{#1})}

% in-text examples

\newcommand{\lnex}[1]{\textit{#1}} %target lang word
\newcommand{\lnlit}[1]{(lit.: `#1')} %literal reading
\newcommand{\lnlat}[1]{(#1)} % latinization
\newcommand{\lntrans}[1]{`#1'} %translation
\newcommand{\lnexl}[2]%
{\lnex{#1}{} \lnlat{#2}} % ex with latinization
\newcommand{\lnexlat}[3]{\lnex{#1}{} \lnlat{#2}{} \lntrans{#3}} % ex with latinization and tranl.

%ch01
\newcommand{\co}[1]{\mbox{\textbf{#1}}}

%ch09

\newcommand{\cyrbulg}[1]{\begin{otherlanguage*}{bulgarian}#1\end{otherlanguage*}}


%ch10
\newcommand{\nlp}{{\small NLP}}
\newcommand{\mwe}{{\small MWE}}
\newcommand{\rae}{{\small RAE}}
\newcommand{\lvc}{{\small LVC}}
\newcommand{\pos}{{\small P}o{\small S}}
%\newcommand{\todo}[1]{ \textcolor{red}{#1} }

%\renewcommand{\labelenumi}{\theenumi}
%\ainamefmt{{vv}{ll}{, ff}{, jj}} % fullname

\newcommand{\biberror}[1]{{\color{red}#1}}

\newcommand{\osenovaitem}{--~} 
  %% hyphenation points for line breaks
%% Normally, automatic hyphenation in LaTeX is very good
%% If a word is mis-hyphenated, add it to this file
%%
%% add information to TeX file before \begin{document} with:
%% %% hyphenation points for line breaks
%% Normally, automatic hyphenation in LaTeX is very good
%% If a word is mis-hyphenated, add it to this file
%%
%% add information to TeX file before \begin{document} with:
%% %% hyphenation points for line breaks
%% Normally, automatic hyphenation in LaTeX is very good
%% If a word is mis-hyphenated, add it to this file
%%
%% add information to TeX file before \begin{document} with:
%% \include{localhyphenation}
\hyphenation{
    Beck-man
    Ngu-yen
    back-chan-nel
    back-chan-nels
    mo-not-o-nous
    ste-reo-typ-i-cal
}

\hyphenation{
    Beck-man
    Ngu-yen
    back-chan-nel
    back-chan-nels
    mo-not-o-nous
    ste-reo-typ-i-cal
}

\hyphenation{
    Beck-man
    Ngu-yen
    back-chan-nel
    back-chan-nels
    mo-not-o-nous
    ste-reo-typ-i-cal
}
 
  \togglepaper[1]%%chapternumber
}{}

\begin{document}
\maketitle
\label{chap:6}
%\shorttitlerunninghead{}%%use this for an abridged title in the page headers

\section{Introduction}

Since the teacher education reform of 2010 (\citealt{Ministry_of_education_and_research2010, MER2016}), Norwegian teacher education has increasingly emphasised educating teachers for working with multilingual and emergent multilingual learners. Between 2013 and 2017, this attention towards multilingualism in teacher education was supported through a state-funded initiative referred to as “Competence for diversity” (\citealt{UHR2016}). This initiative was meant to support teacher educators in their efforts to realise the ambitions of preparing pre-service teachers for the multicultural and multilingual reality of today’s classrooms, described in national guidelines (\citealt{UHR2016}) and regulations (\citealt{MER2010, MER2016}).

Nonetheless, researchers have continued to identify a concerning lack of attention towards multilingualism and pre-service teachers have continued to report limited knowledge about multilingualism and lacking skills to teach multilingual and emergent multilingual learners (\citealt{DyrnesEtAl2015,RandenEtAl2015,Skrefsrud2015-1,The_evaluation_group2015}). Over a decade since the teacher education reform of 2010 was implemented, the question still remains whether the emphasis on educating teachers for working with multilingual and emergent multilingual learners has led to the education of linguistically responsive teachers. This is investigated though the following research question: What orientations, knowledge and skills about multilingualism do Norwegian pre-service teachers demonstrate?

In order to answer this research question, we analyse empirical data from three recent doctoral studies (\citealt{Fylkesnes2019,Iversen2020,Thomassen2021}) in light of \citeauthor{LucasVillegas2011}' (\citeyear{LucasVillegas2011,Lucas2013}) framework for preparing linguistically responsive teachers. In conclusion, we discuss what potential and limitations in Norwegian teacher education’s preparation of pre-service teachers for multilingual classroom can be identified from our findings. Our concern is not to assess the individual pre-service teachers’ orientations, knowledge, nor skills. Rather, our concern is to investigate to what degree Norwegian teacher education is able to educate linguistically responsive teachers.

\section{Background and previous research}
\begin{sloppypar}
In Norway, primary and lower secondary school teacher education programmes are integrated five-year master’s programmes regulated by the government through national guidelines (\citealt{UHR2016}) and regulations (\citealt{MER2016}). Together these guidelines and regulations contribute to a rigorous standardisation of teacher education programmes across institutions. Up until the 2010 teacher education reform, a four-year general teacher education prepared teachers to work at all levels throughout primary and lower secondary education (grades 1–10), and to teach across all subjects. However, the 2010 teacher education reform divided the general teacher education into two separate programmes, thus qualifying teachers either for teaching in grades 1–7, or 5–10. In 2017 an additional reform was implemented. This reform extended both teacher education programmes from four-year programmes to five-year master’s programmes (\citealt{MER2016}).
\end{sloppypar}

The year before the implementation of the latest reform, the Norwegian government\hyp appointed expert group on the teacher role argued that it was necessary to continue to improve the presence of multicultural perspectives and Norwegian as a second language across all subjects within teacher education (\citealt{DahlEtAl2016}). The expert group’s call for improving the multicultural focus within teacher education was based on research on how teacher education programmes and teacher educators had met the burgeoning diversity in the student population. This research had not been reassuring in the sense that teacher educators had reported that they were struggling to adapt to the multilingual reality (\citealt{RandenEtAl2015}), and that there was a lack of awareness about issues relating to multiculturalism and multilingualism within teacher education (\citealt{DyrnesEtAl2015}). Furthermore, research suggested that diversity has been given limited attention within the different subjects of teacher education (\citealt{Skrefsrud2015-1}), and that many pre-service teachers felt unqualified to work with multilingual and multicultural students (\citealt{The_evaluation_group2015}).

Despite these findings, the policy documents presented above suggest an emergent tendency to acknowledge the multilingualism found in Norwegian society at large, and in schools. This is most prominent in the national guidelines and regulations for teacher education (\citealt{MER2016,UHR2016}). As part of the 2017 reform, a revised version of the national guidelines (\citealt{UHR2016}) and regulations for the teacher education programmes were adopted (\citealt{MER2016}). These regulations state that pre-service teachers should acquire “comprehensive knowledge about children’s development, education and learning in different social, linguistic and cultural contexts”\footnote{All quotes from Norwegian policy documents have been translated from Norwegian into English by the authors.}  (\citealt{MER2016}). In the national guidelines for teacher education (both 1--7 and 5--10), there are also some relevant learning outcomes for different subjects, but they are still rather general and superficial. For examples, the guidelines for mathematics (1--7) state that “teaching should be adapted to the different needs of students, where the different cultural, linguistic and social backgrounds should both be taken into account but also considered a resource” (\citealt{MER2016}: 23). The most concrete guidelines are described in the subject of Norwegian, where it is stated that students should have “knowledge about language and identity, Norwegian as a second language and multilingual practice” and also have knowledge about how to assess language competence (\citealt{MER2016}: 28). With this context in mind, it is hence necessary to explore whether the increased emphasis on educating pre-service teachers for working with multilingual and emergent multilingual learners has led to the education of linguistically responsive teachers in Norway.

Internationally, there is an extensive body of research that focuses on pre-service teachers’ knowledge – or lack thereof – about teaching multilingual students or second language learners (\citealt{AcquahEtAl2020,AcquahSzelei2020,AndersonStillman2013,BravoEtAl2014,GroulxSilva2010,TandonEtAl2017,VillegasEtAl2018}). Recent studies have produced promising findings regarding in- and pre-service teachers’ attitudes and beliefs concerning multilingualism in education (\citealt{Duarte2022,PaulsrudEtAl2023,SchroedlerEtAl2023,Thoma2022}). Studies also find that pre-service teachers’ positive beliefs do tend to lead to more multilingual practices in the classroom (\citealt{KirschEtAl2020,PalviainenEtAl2016,SchroedlerFischer2020}). Despite a comprehensive review by \citet{VillegasEtAl2018} confirming that there are many studies on pre-service teachers’ \textit{beliefs} about multilingualism, it also points out that these studies frequently lack a concern for how to prepare pre-service teachers to analyse the language demands embedded in academic text and learning tasks. The tendency to focus on in- and pre-service teachers' beliefs related to multilingualism, rather than practices, has persisted in recent research in the field (\citealt{Aleksic2023,Doll2023,Duarte2022,PaulsrudEtAl2023,Thoma2022}). Consequently, pre-service teachers might become insecure about how to translate positive beliefs about multilingualism into their own teaching practice (\citealt{SchroedlerEtAl2023}). In conclusion, \citet{VillegasEtAl2018} therefore argue that there is a need for pre-service teachers to be provided with knowledge about a) second language learning and b) the comprehensible input students receive, as well as c) the differences between everyday language and academic language. Furthermore, quite a few studies that focus on how teacher education can better prepare pre-service teachers for working with multilingual students highlight the importance of providing pre-service teachers with sound opportunities to meet, teach and practice with emergent multilingual learners in multilingual and multicultural classroom settings (\citealt{BravoEtAl2014,GroulxSilva2010}) and to model culturally responsive teaching (\citealt{AcquahEtAl2020}).

\section{Theoretical framework}

As pointed out in the previous section, much research has explored pre-service teachers’ preparedness for teaching multilingual and emergent multilingual students. Several researchers have attempted to summarise key insights from this research and identify what knowledge and skills pre-service teachers need to have in order to provide quality education for multilingual students. For instance, \citet{HowardAleman2008} concluded that pre-service teachers need knowledge of “subject matter and pedagogical content knowledge, knowledge of effective practice about teaching in diverse settings and development of a critical consciousness”. A widely recognised framework summarising the expertise of linguistically responsive teachers was developed by \citet{LucasVillegas2013}. This framework has been applied in teacher education research in diverse settings over the course of the past decade (cf. \citealt{chapters/7_alisaari, chapters/8_heikkola, TandonEtAl2017}).

In their framework, \citet{LucasVillegas2013} have included three orientations and four types of pedagogical knowledge and skills they see as fundamental for linguistically responsive teaching. The framework was originally developed to assess teacher preparedness to teach “English language learners” in the US context. The framework was not intended to be used as a formula for teacher educators, but rather developed as an attempt to combine central insights from the field of second language learning. In this chapter, when applying Lucas and Villegas’ framework for preparing linguistically responsive teachers to a Norwegian context, we have replaced the concept \textit{English language learners} with \citegen[60]{Garcia2009} concept \textit{multilingual} and \textit{emergent multilingual learners}. Following \citegen{Garcia2009} conceptualisation of multilingual and emergent multilingual learners, we use these concepts in order to accentuate that students who are in the process of acquiring the language of instruction (i.e. emergent multilinguals) bring a wide repertoire of linguistic resources with them to school and that students who already have acquired high proficiency in the language of instruction (e.g. multilinguals) also benefit from linguistically responsive teaching (e.g., \citealt{Garcia2009}).  

\citet[101]{Lucas2013-3} first describe what they refer to as three fundamental orientations necessary for linguistically responsive teaching. They define orientations as “inclinations or tendencies toward particular ideas and actions, influenced by attitudes and beliefs” (\citealt{LucasVillegas2011}: 56, 101). According to \citet[101]{LucasVillegas2013}, the three fundamental orientations necessary for linguistically responsive teaching are:

\begin{itemize}
\item \textit{Sociolinguistic consciousness:} An understanding that language, culture, and identity are deeply interconnected, and an awareness of the sociopolitical dimensions of language use and language education. 
\item \textit{Value for linguistic diversity:} Knowledge of key psycholinguistic, sociolinguistic, and sociocultural processes involved in learning a second language, and of ways to use that knowledge to inform instruction.
\item \textit{Inclination to advocate for multilingual and emergent multilingual learners:} Understanding of the need to take action to improve multilingual and emergent multilingual learners’ access to social and political capital and educational opportunities, and willingness to do so. 
\end{itemize}

Furthermore, \citet[101--102]{LucasVillegas2013} describe four elements of pedagogical knowledge and skills they consider fundamental for linguistically responsive teaching. They define knowledge and skills as “the complex and interconnected disciplinary knowledge, pedagogical content knowledge, knowledge of learners, and pedagogical skills needed by successful teachers” (\citealt{LucasVillegas2011}: 56). These knowledge and skills are:

\begin{itemize}
\item \textit{A repertoire of strategies for learning about the linguistic and academic backgrounds of multilingual and emergent multilingual learners in the language of instruction and their native languages:} Understanding of the importance of knowing about the backgrounds and experiences of multilinguals and emergent multilingual learners, and knowledge of strategies for learning about them.
\item \textit{An understanding of and ability to apply key principles of second language learning:} Knowledge of key psycholinguistic, sociolinguistic, and sociocultural processes involved in learning a second language, and of ways to use that knowledge to inform instruction. 
\item \textit{Ability to identify the language demands of classroom tasks}: Skills for determining the linguistic features of academic subjects and activities likely to pose challenges for multilinguals and emergent multilingual learners, including identifying key vocabulary, understanding syntactic and semantic features of academic language, and the linguistic expectations for successful completion of tasks.
\item \textit{A repertoire of strategies for scaffolding instruction for multilingual and emergent multilingual learners:} Ability to apply temporary supports to provide multilingual and emergent multilingual learners with access to learning the language of instruction and content taught in this language, including using extralinguistic supports such as visuals and hands-on activities; supplementing written and oral text with study guides, translation, and redundancy in instruction; and providing clear and explicit instructions.               
\end{itemize}

As stated earlier, Lucas and Villegas’ framework was primarily developed to meet the needs of \textit{emergent} multilingual students. When we expand the use of the framework to include multilingual learners in general, this can potentially raise some issues that need to be addressed. In line with other researchers focusing on multilingual learners, we argue that sociolinguistic consciousness, appreciation of linguistic diversity, and an inclination to advocate for multilingual and emergent multilingual learners is also fundamental for students who are not in a process of learning a language of instruction (\citealt{Garcia2016-1,Garcia2014-4}). All multilingual learners also benefit from meeting teachers with an awareness of the sociopolitical dimensions of language use and language education and who let their appreciation of multilingualism inform their instruction, and advocate for their students (\citealt{Garcia2017-1}). Moreover, all multilingual learners benefit from meeting teachers who are able to consider their students’ complete linguistic repertoire, who are able to identify language demands, and provide the necessary scaffolding as part of their instruction (\citealt{Garcia2016-1,Garcia2014-4}). This is obviously most critical for emergent multilingual learners, but also fundamental for all multilingual learners – regardless of their proficiency in the language of instruction.

\section{Method}

In the following, we first present the data and participants involved in the current study. We also include some ethical considerations involved in re-using qualitative data for secondary analyses, before we describe how we have analysed our data. 

\subsection{Data collection, participants, and ethical considerations} 
\begin{sloppypar}
In our analysis, we revisit our three doctoral studies’ data on Norwegian pre-service teachers’ orientations, knowledge, and skills regarding multilingualism, at different stages in teacher education (\citealt{Fylkesnes2019,Iversen2020,Thomassen2021}). The three studies are based on qualitative data acquired from a total of 106 pre-service teachers, primarily through focus group and individual interviews, with additional data collected through classroom observation, and the collection of pre-service teachers’ linguistic autobiographies. To analyse data from different research projects with distinct research questions and research designs is complex (e.g., \citealt{IrwinWinterton2011}). Consequently, we decided to focus our common analysis on the data we shared across projects, namely the focus group and individual interviews. In the following, we refer only to our interview data.
\end{sloppypar}

\begin{sloppypar}
The data were gathered from four different teacher education institutions across Norway and at different stages in the participants’ teacher education (see \tabref{tab:iversen:1}). Because of the rigorous standardisation of teacher education programmes in Norway, there is limited variation in the orientations, knowledge, and skills pre\hyp service teachers are expected to develop over the course of their teacher education.
\end{sloppypar}

\begin{table}
\caption{Four groups of participants. “PST”: pre-service teachers} 
\label{tab:iversen:1}
\begin{tabularx}{\textwidth}{lQQQQ}
\lsptoprule
 & \multicolumn{4}{c}{Research project}\\\cmidrule(lr){2-5}
 & \citet{Iversen2020} & \multicolumn{2}{c}{\citet{Thomassen2021}} & \citet{Fylkesnes2019}\\\midrule
Participants: & Year 1 PST ($n=24$) & Year 2 PST ($n=56$) & Year 4 PST ($n=11$) & Year 4 PST ($n=15$)\\
Universities: & A and B &  C &  A, B, and C & A and D\\
\lspbottomrule
\end{tabularx}
\end{table}

\citet{Iversen2020} collected data from first-year pre-service teachers, \citet{Thomassen2021} from both second- and fourth-year pre-service teachers, and \citet{Fylkesnes2019} from fourth-year pre-service teachers. Whilst Iversen’s doctoral study explored pre-service teachers’ first encounter with multilingualism during their practicum from a translanguaging perspective, Thomassen’s doctoral study investigated pre-service teachers’ competence to teach in multicultural and multilingual classrooms, from a Norwegian as a second language perspective. Fylkesnes’ doctoral study investigated how actors in teacher education use and make meaning of the floating signifier \textit{cultural diversity}. Although Fylkesnes’ doctoral study was primarily interested in the conceptualisation of cultural diversity, the data collected from the focus groups are nonetheless relevant for this chapter’s analysis, due to how pre-service teachers’ answers extensively focused on issues related to multilingualism and emergent multilingual learners. 

The re-use of qualitative data for secondary analyses has become increasingly common over the past decades (\citealt{BishopKuula-Luumi2017}). Notwithstanding this trend, there are specific challenges associated with secondary analyses of qualitative data, and with the combination of previous separate data sets for a secondary analysis. \citet{IrwinWinterton2011} have pointed out that qualitative data might not be suitable for re-use due to the researcher’s distance from the collection of primary data, and related knowledge of the particular contexts of data collection. In the case of the study at hand, the decision to re-use and combine our own previous data for new analyses enabled us to draw on our intimate knowledge of the data and context for data collection. Nonetheless, due to the qualitative nature of the three research projects and the methodological differences in data collection, we cannot describe a clear progression in the participants’ orientations, knowledge and skills from early to later stages in their teacher education. Hence, in the analysis, we describe the most prominent patterns in our data material without connecting the results to the participants' stage in their teacher education. With this in mind, we still argue that the patterns we describe in part can be ascribed to their teacher education, and that they provide important insights into the degree to which Norwegian teacher education is perceived to be able to educate linguistically responsive teachers.

\subsection{Analysis}

Despite the three studies’ different foci, they all provided rich data on pre-service teachers’ views and experiences with multilingualism. Due to our common interest in pre-service teachers’ orientations, knowledge and skills regarding multilingual classrooms, we found the framework for preparing linguistically responsive teachers, developed by \citet{LucasVillegas2011,Lucas2013}, to be useful for our analysis. 

We conducted a theory-driven thematic analysis (\citealt{BrinkmannKvale2015}), where each researcher deductively categorised their transcribed focus group interviews according to the three orientations and four elements of fundamental knowledge, and skills, which \citet{LucasVillegas2011} describe in their framework for preparing linguistically responsive teachers. Each statement could be simultaneously categorised according to multiple categories. To illustrate this process, we have included a few brief examples in \tabref{tab:iversen:2} (page~\pageref{tab:iversen:2}).

\begin{table}
\caption{Example of categorisation. All excerpts from the empirical data have been translated from Norwegian into English by the authors. All names are pseudonyms} 
\label{tab:iversen:2}
\small
\begin{tabularx}{\textwidth}{XQQQ}
\lsptoprule
Category & Iversen & Thomassen & Fylkesnes\\\midrule
Sociolinguistic consciousness & \textit{Nora:} They preserve their mother tongue. That’s positive, I would say. 

\textit{Elise:} It’s important for the students’ identity, and, well, their personal development. & \textit{Christina:} They shouldn’t be robbed of their own language and their own identity. You do not leave your culture or your language at the door. & \textit{Tor:} I think some teachers think that minority students should (…) forget about their culture and language (…) I think that’s wrong. \\
Value for linguistic diversity & \textit{Josefine:} It’s a resource, to know several languages. They’re, well, lucky to be able use them. & \textit{Emil:} I think it’s very interesting with students who have other languages (…) It’s a strength. & \textit{Liv:} Recognition of cultural and linguistic background is very important in schools today.\\
\lspbottomrule
\end{tabularx}
\end{table}

Following the initial categorisation, we compared and discussed our findings within each of the three categories (e.g., \citealt[594--596]{JohnsonChristensen2019}). Through this comparison, we noticed several converging patterns between the different data sets: The pre-service teachers expressed similar orientations, similar levels of knowledge about multilingualism, and similar levels of skills regarding multilingualism in education, but their particular knowledge and skills varied. Moreover, the analysis revealed some salient limitations, particularly relating to the pre-service teachers’ knowledge and skills.

\begin{sloppypar}
In line with our research question and qualitative approach to thematic analysis, we were not concerned with quantifying how many pre-service teachers demonstrated “adequate” or “inadequate” orientations, knowledge, and skills (e.g., \citealt{BrinkmannKvale2015}). Rather, we focused our analysis on the most salient tendencies in the pre-service teachers’ orientations, knowledge, and skills, and which potential limitations in Norwegian teacher education’s preparation of pre-service teachers for multilingual classrooms this variation might indicate.
\end{sloppypar}

\section{Findings}

In the following, we will describe and exemplify the most prominent patterns found in the data. We will start the presentation of findings by describing the pre-service teachers’ orientations, before we turn to the pre-service teachers’ knowledge and skills. 

\subsection{Norwegian pre-service teachers’ orientations}

According to \citet{LucasVillegas2013}, sociolinguistic consciousness, value for linguistic diversity, and inclination to advocate for multilingual and emergent multilingual learners are necessary orientations for linguistically responsive teaching. 

Our analysis revealed that the pre\hyp service teachers exhibited an emergent, though limited, sociolinguistic consciousness. Regardless of how far the pre\hyp service teachers had come in their teacher education, the participants generally articulated an acceptance of the dominant policy discourses about multilingual and emergent multilingual learners, emphasising the importance of developing high proficiency in Norwegian and a swift mainstreaming into the monolingual norms of Norwegian education. Nonetheless, they also displayed what could be described as an emergent sociolinguistic consciousness (e.g., \citealt{LucasVillegas2013}), in terms of how they seemed to connect language, culture and identity. This is evident from the extract below:

\begin{quote}
\emph{Kim:} (…) I think some teachers have this thought that the students are supposed to … the linguistic minority students are supposed to become similar, that they are supposed to, in a way forget about their own culture and their language and only adopt the Norwegian, in a way. So, I believe that that’s a bit wrong, a bit wrong way of thinking because they, all students have, also the minority students have the right to keep their language as they learn the Norwegian language, and I believe that it is important that the teacher recognises the students’ first language as well. 
\end{quote}

In line with the general patterns in our data, the excerpt above exemplifies how Kim challenged the ongoing assimilationist attitudes within Norwegian education and opened up for a more multilingual approach to education. Kim displays an understanding of the connections between language, culture, and identity, and comments on the limitations of a one-sided emphasis on developing students’ Norwegian language skills. As such he reflects how most of the pre-service teachers in our study displayed an awareness of the sociopolitical dimensions of language use and language education (e.g., \citealt{LucasVillegas2013}).

There was a salient pattern in our data where the pre-service teachers express appreciation for linguistic diversity by describing multilingualism as something “positive” (cf. \citealt{chapters/5_ostergaard}). The pre-service teachers mentioned the advantages of being multilingual for further language learning, the benefits of being able to maintain relationships with relatives in other countries, and the importance of preserving students’ languages for their identity development. However, some pre-service teachers struggled to articulate what was “positive” about multilingualism and how this appreciation could inform their own instruction: 

\begin{quote}
\emph{Greta:} Opportunities… I see several… I think it’s very important that the world, and Norway as well, becomes wider, because I think that we can develop further if we have knowledge about the multicultural and the multilingual… I think that the interest in languages can increase… I think that students with Norwegian as a second language have a lot to offer us (…)
\end{quote}

Despite Greta’s positive remarks about multilingualism, she positions “the multicultural” and “multilingual” as an Other who can benefit an implicit “us”. Furthermore, Greta’s comments regarding multilingualism illustrate how the pre-service teachers’ positive statements often were somewhat superficial. Nevertheless, the pre-service teachers were able to identify psycholinguistic, sociolinguistic and sociocultural benefits of multilingualism, particularly in acquiring the language of instruction (e.g., \citealt{LucasVillegas2013}).

The pre-service teachers’ inclination to advocate for multilingual or emergent multilingual learners was not consistent – neither at an individual level, nor at group level (cf. \citealt{chapters/5_ostergaard}). Nonetheless, the pre-service teachers were concerned with central values in Norwegian education, such as equality and inclusion, and they were usually aware of the rights of emergent multilingual learners to differentiated Norwegian instruction, mother tongue education, and bilingual subject instruction. Across different student groups, there was a concern for securing students’ rights:

\begin{quote}
\emph{Greta:} I would’ve checked, well, if the student came from somewhere else and didn’t know a single word in Norwegian, I would of course take her or him to adapted language education.

\emph{Researcher:} Yes. 

\emph{Greta:} Because that’s exactly the law. Yes, I know that much, but not exactly word by word.  
\end{quote}

“Adapted language education” includes both differentiated Norwegian instruction, and sometimes even mother tongue education and bilingual subject instruction. The participants saw access to advanced competence in Norwegian as the primary social and political capital, and as the principal requirement to access educational opportunities. Hence, the pre-service teachers’ potential advocacy for students’ access to “adapted language education” was, in effect, geared towards a rapid inclusion of students into a more or less monolingual mainstream. Despite their concern for students’ academic success, certain pre-service teachers considered challenges associated with multilingualism in education to be the responsibility of other actors in the education system. 

In conclusion, the pre-service teachers demonstrated an emergent sociolinguistic consciousness and appreciation of linguistic diversity. However, their inclination to advocate for multilingual learners was inconsistent. Furthermore, their positive orientations were often superficial, and the monolingual bias found in Norwegian education also influenced the pre-service teachers’ emphasis on rapid acquisition of and high proficiency in Norwegian.  

\subsection{Norwegian pre-service teachers’ pedagogical knowledge and skills}

\citet{LucasVillegas2013} describe four elements of pedagogical knowledge and skills they consider fundamental for linguistically responsive teaching. These four elements are: strategies for learning about multilingual and emergent multilingual learners’ background, understanding of and ability to apply key principles of second language learning, ability to identify the language demands of classroom tasks, and a repertoire of strategies for scaffolding instruction for multilingual and emergent multilingual learners.

The pre-service teachers’ interest in their students’ linguistic and academic backgrounds varied and they consequently demonstrated a limited repertoire of strategies for learning about multilingual and emergent multilingual learners’ background (e.g., \citealt{LucasVillegas2013}). Rather, they were eager to consider all students with apparently high competence in Norwegian as monolingual, even when they suspected that many students in fact were multilingual. This is exemplified in the excerpt:

\begin{quote}
\emph{Nora:} Had I entered the classroom without knowing anything about anybody’s linguistic background, I wouldn’t have thought about it at all. That this student wasn’t Norwegian. Because I haven’t, maybe you have, but I haven’t. 
\end{quote}

\begin{sloppypar}
In this example, it seems that the pre-service teacher was unaware of the importance of knowing about the backgrounds and experiences of multilingual and emergent multilingual learners, regardless of their proficiency in the language of instruction. From the analysis of our data, it seemed that the pre-service teachers only found the multilingual students’ background relevant when they believed the students faced linguistic challenges. Nonetheless, when the researchers explicitly asked for strategies to learn about the linguistic and academic backgrounds, the pre-service teachers were able to describe such strategies: 
\end{sloppypar}

\begin{quote}
\emph{Betty:} (…) If there was something I wanted to ask about the student, I think I would have included the mother tongue teacher and asked, yes, I think so. And also asked for suggestions, for example about [the students’ proficiency in] Arabic, which I have very limited knowledge about, and find out at which level it was… and then… and the parents as well, to find out what they were talking at home, how they do it at home (…)
\end{quote}

The pre-service teacher in the example above suggested that the teacher can contact the mother tongue teacher of the student and parents. As part of learning more about and getting to know the students’ background, Betty’s aim seemed to be to figure out “which level the student is at”. In addition to this example, one pre-service teacher proposed that students could present their own cultural and linguistic background to the whole class, as a way for both the teacher and the rest of the class to learn more about each other. 

\begin{sloppypar}
The pre-service teachers articulated mostly a limited understanding of and ability to apply key principles of second language learning. One pre-service teacher mentioned Cummins’ dual iceberg model, however, without being able to adequately describe its content:
\end{sloppypar}

\begin{quote}
\emph{Alex:} I agree that teachers need more competence related to this area. It is a theory, it is, or it is a model. Iceberg-model. The Cummins model. I do not know if you have heard about it. And it is concerned with, that foundation that it, well, both languages, and was it not about how this foundation is both languages and that they build on each other? Or, that the mother tongue is the foundation perhaps. I do not exactly remember how it was, but at least I think that it should be included in teacher education.
\end{quote}

Another pre-service teacher said that they “had learned a lot about this as part of the Norwegian subject”, also apparently without being able to articulate how principles from second language learning could inform the pre-service teacher’s teaching practice. Pre-service teachers who had studied Norwegian as part of their teacher education mentioned that they had learned about contrastive grammar, while pre-service teachers who had studied English mentioned key principles for second language learning when discussing multilingual and emergent multilingual students’ needs. Yet, another pre-service teacher described how the instruction about multilingual and emergent multilingual learners’ needs in teacher education had been rather superficial:  

\begin{quote}
\emph{Anna:} The thing I most remember is that, when we were learning about this topic, I remember it was mostly to categorise the different concepts, what is mother tongue, what is bilingual, and then a little about what is the most typical [language] mistakes [made by emergent multilingual learners]. 
\end{quote}

The pre-service teachers in our data seemed to find it challenging to recall and articulate concrete theoretical or methodological principles for working with multilingual or emergent multilingual learners. In individual interviews, five pre-service students mentioned “knowledge about the students’ language system” as important. Nonetheless, the pre-service teachers’ knowledge of key psycholinguistic, sociolinguistic, and sociocultural processes involved in learning a second language, and of ways to use that knowledge to inform instruction, was quite limited (e.g., \citealt{LucasVillegas2013}), regardless of where the pre-service teachers were in their teacher education.

\begin{sloppypar}
Furthermore, the pre-service teachers’ ability to identify the language demands of classroom tasks was also limited (cf. \citealt{chapters/8_heikkola}). This pedagogical skill involves the ability to determine the linguistic features of academic subjects and activities likely to pose challenges for multilingual and emergent multilingual learners, and to consider this in the planning and conducting of teaching. When explicitly asked in the interviews, the participants provided examples of language demands of classroom tasks:
\end{sloppypar}

\begin{quote}
\emph{Greta:} I think that many of the concepts we have in Norway are rather similar, or that we have many concepts for similar things, like in the movie we watched about multilingualism, no, about second language learning in science class, where they were talking about the difference between a branch and a twig and a tree. And that second language learners may know the word and how to pronounce it, but they don’t know that it refers to exactly that thing, and that’s where they misunderstand (…)
\end{quote}

As with the example above, the pre-service teachers understood that subject-specific terminology can be challenging for some multilingual and emergent multilinguals, just as it can pose a challenge for all students. Notwithstanding such examples, their comments on language demands of classroom tasks were generic and did not provide any concrete examples of how they analysed language demands as part of their lesson planning. 

Finally, the pre-service teachers described their repertoire of strategies for scaffolding instruction for multilingual and emergent multilingual learners (cf. \citealt{chapters/7_alisaari, chapters/8_heikkola}). This skill involves the ability to apply temporary supports to provide multilingual and emergent multilingual learners with access to learning Norwegian and content taught in Norwegian (e.g., \citealt{LucasVillegas2013}). The pre-service teachers provided several examples of how they had spontaneously scaffolded instruction for emergent multilingual students as part of their practicum. These examples included using extralinguistic supports such as visuals and hands-on activities; translation through peer-support and digital tools, as well as providing clear and explicit instructions. Below is an example of how one participant had learned how to provide clear instructions in a class made up of mostly multilingual and emergent multilingual learners:

\begin{quote}
\emph{Marthe:} We were instructed [by the supervising teacher] that we had to give very clear instructions and assignments because the students kind of listen to you, but it’s not certain that they understand you the first time. So, what we were told from the beginning was that we had to be super explicit, really. Things that were really not natural to explain several times, we had to do it. And that might be because of the students’ language background. 
\end{quote}

This example aligns with the recommendations provided by \citet{LucasVillegas2013}. Although it is necessary to provide explicit instructions, the pre-service teachers usually described the different scaffolding strategies as spontaneous and quite unstructured solutions to problems that suddenly occurred in the learning situation. For example, pre-service teachers mentioned the use of Google Translate as a strategy when communication was breaking down in the classroom. 

In conclusion, the pre-service teachers had a limited repertoire of strategies for learning about multilingual and emergent multilingual learners’ background, an inadequate understanding of and ability to apply key principles of second language learning, restricted ability to identify the language demands of classroom tasks, and an insufficient repertoire of strategies for scaffolding instruction for multilingual and emergent multilingual learners.

\section{Discussion}

The aim of this study has been to explore whether the increased emphasis on educating pre-service teachers for working with multilingual and emergent multilingual learners within Norwegian teacher education over the past decade has led to the education of linguistically responsive teachers. This has been investigated though the following research question: What orientations, knowledge and skills about multilingualism do Norwegian pre-service teachers demonstrate? In the following, we will first discuss the main findings from our analysis, before we discuss the potential and limitations in Norwegian teacher education’s preparation of pre-service teachers for multilingual classrooms. In conclusion, we point out a few implications for language policy and education. 

Through the application of \citeauthor{LucasVillegas2013}' (\citeyear{LucasVillegas2013}) framework for preparing linguistically responsive teachers, we identified a consistent discrepancy in the pre-service teachers’ orientations, on the one hand, and their knowledge and skills, on the other hand. On the one hand, the pre-service teachers demonstrated orientations that suggested that they were willing and able to improve multilingual and emergent multilingual learners’ access to social and political capital and educational opportunities. They demonstrated an understanding for the ways that language and identity are deeply connected, and they appreciated linguistic diversity. Despite these encouraging findings, there were still certain limitations to their critical awareness related to current language policies in education and a hesitance to advocate for multilingual and emergent multilingual students. These findings reflect findings from previous studies, which have also pointed out certain shortcomings in pre-service teachers’ orientations (\citealt{AndersonStillman2013,PaulsrudEtAl2023,VillegasEtAl2018}). Nevertheless, research suggests that pre-service teachers’ orientations can potentially be influenced and changed through teacher education (\citealt{Aleksic2023,Doll2023,AndersonStillman2013,SchroedlerEtAl2023,VillegasEtAl2018}).

The pre-service teachers seemed more limited in their disciplinary knowledge, pedagogical content knowledge, knowledge of learners, and pedagogical skills needed to enact linguistically responsive teaching. Even pre-service teachers who had studied second language acquisition as part of their Norwegian and/or English subject education in teacher education were unable to clearly articulate how they could support multilingual and emergent multilingual students in their future classrooms. It is discouraging that our data do not indicate a clear progression in the pre-service teachers’ knowledge and skills from the first-year pre-service teachers to the fourth-year pre-service teachers. Our findings align with previous Norwegian research (\citealt{DyrnesEtAl2015,RandenEtAl2015,Skrefsrud2015-1,The_evaluation_group2015}), as well as international research (\citealt{AcquahEtAl2020,AcquahSzelei2020,AndersonStillman2013,BravoEtAl2014,GroulxSilva2010,SchroedlerEtAl2023,TandonEtAl2017,VillegasEtAl2018}), which also report that pre-service teachers are not sufficiently prepared to work with multilingual students. Specifically, analyses of teacher education programmes in Iceland and Sweden suggest that these programmes do not provide prospective teachers with the pedagogical skills necessary to enact linguistically responsive teaching (\citealt{chapters/2_Gunnthorsdottira}). These findings are also confirmed in studies of pre-service teachers’ knowledge and skills reported elsewhere in this volume \parencite{chapters/7_alisaari, chapters/8_heikkola, chapters/5_ostergaard}.

It might be unsurprising that the pre-service teachers in our study articulated the necessary orientations, while at the same time were frequently unable to demonstrate the same level of knowledge and skills related to linguistically responsive teaching. \citet{VillegasEtAl2018} found in their review of research on pre-service teacher education for emergent multilinguals that studies on pre-service teachers’ \textit{beliefs} completely dominated the field of research. Similar tendencies were reported in \citegen{AndersonStillman2013} review and can be observed in recent research (e.g. \citealt{Aleksic2023,Doll2023,Duarte2022,SchroedlerEtAl2023,Thoma2022}). If researchers’ over-emphasis on beliefs is reflected in teacher educators’ instruction, it would be unsurprising if pre-service teachers develop the necessary orientations, without the pedagogical knowledge and skills necessary to enact linguistically responsive teaching. Researchers have also previously pointed out that pre-service teachers’ positive positioning towards multilingual learners does not necessarily transfer into teaching practice (\citealt{SchroedlerEtAl2023}).

Our findings suggest that the emphasis on educating pre-service teachers for working with multilingual and emergent multilingual learners within Norwegian teacher education since 2010 (\citealt{MER2016,UHR2016}) seems to have contributed to the development of the fundamental orientations identified by \citet{Villegas2018} as crucial for preparing pre-service teachers for working in multilingual classrooms. Simultaneously, our findings also indicate that the most critical aspect for Norwegian teacher education in the years to come will be to provide all pre-service teachers with the knowledge and skills needed to enact linguistically responsive teaching. Furthermore, pre-service teachers – regardless of subject background – need to be able to visualise what linguistically responsive teaching looks like and be introduced to appropriate strategies and methods for enacting such teaching.

Moreover, our findings indicate that teachers would benefit from the  current regulations and guidelines for teacher education in Norway (\citealt{MER2016, UHR2016}) being more explicit about the specific knowledge and pedagogical skills needed for teaching multilingual and emergent multilingual students. Considering the linguistic diversity currently characterising schools across Norway, we argue that future national guidelines and regulations for teacher education should include explicit requirements regarding teacher educators’ knowledge and skills. Findings from other Nordic countries and beyond indicate that teacher education needs a stronger emphasis on practical knowledge and pedagogical skills related to teaching linguistically diverse students (\citealt{chapters/2_Gunnthorsdottira}, \citealt{SchroedlerEtAl2023}). Then it would be possible for teacher educators to capitalise on pre-service teachers’ positive orientations and prepare them for enacting linguistically responsive teaching when they begin teaching.

\printbibliography[heading=subbibliography,notkeyword=this]
\end{document}
