\documentclass[output=paper]{langscibook}
\ChapterDOI{10.5281/zenodo.15280895}
\author{Hermína Gunnþórsdóttir\orcid{}\affiliation{University of Akureyri} and Edda Óskarsdóttir\orcid{}\affiliation{University of Iceland}}
\title[Working with multilingual students in Iceland]{Working with multilingual students in Iceland: Exploring educational experiences of newly graduated teachers}
\abstract{The number of multilingual students has increased rapidly in Iceland over the past two decades. Recent research in Iceland indicates that the school system has been challenged in meeting the needs of this group of students. The aim of this chapter is to shed light on how the teacher education programmes at the Universities of Iceland (UI) and Akureyri (UNAK) prepare and support pre-service teachers in their initial education to work with learners for whom Icelandic is an additional language. Analysis of documents (current teacher education course syllabi) and three focus group interviews with newly graduated students at both universities was conducted. Findings revealed a lack of relevant courses in the study programme and all teachers discussed a lack of practical pedagogical strategies to support multilingual students. In sum, the findings indicate that teacher education in Iceland does not sufficiently prepare pre-service teachers for teaching multilingual students. }
\IfFileExists{../localcommands.tex}{
  \addbibresource{../localbibliography.bib}
  \usepackage{langsci-optional}
\usepackage{langsci-gb4e}
\usepackage{langsci-lgr}

\usepackage{listings}
\lstset{basicstyle=\ttfamily,tabsize=2,breaklines=true}

%added by author
% \usepackage{tipa}
\usepackage{multirow}
\graphicspath{{figures/}}
\usepackage{langsci-branding}

  
\newcommand{\sent}{\enumsentence}
\newcommand{\sents}{\eenumsentence}
\let\citeasnoun\citet

\renewcommand{\lsCoverTitleFont}[1]{\sffamily\addfontfeatures{Scale=MatchUppercase}\fontsize{44pt}{16mm}\selectfont #1}
   
  %% hyphenation points for line breaks
%% Normally, automatic hyphenation in LaTeX is very good
%% If a word is mis-hyphenated, add it to this file
%%
%% add information to TeX file before \begin{document} with:
%% %% hyphenation points for line breaks
%% Normally, automatic hyphenation in LaTeX is very good
%% If a word is mis-hyphenated, add it to this file
%%
%% add information to TeX file before \begin{document} with:
%% %% hyphenation points for line breaks
%% Normally, automatic hyphenation in LaTeX is very good
%% If a word is mis-hyphenated, add it to this file
%%
%% add information to TeX file before \begin{document} with:
%% \include{localhyphenation}
\hyphenation{
affri-ca-te
affri-ca-tes
an-no-tated
com-ple-ments
com-po-si-tio-na-li-ty
non-com-po-si-tio-na-li-ty
Gon-zá-lez
out-side
Ri-chárd
se-man-tics
STREU-SLE
Tie-de-mann
}
\hyphenation{
affri-ca-te
affri-ca-tes
an-no-tated
com-ple-ments
com-po-si-tio-na-li-ty
non-com-po-si-tio-na-li-ty
Gon-zá-lez
out-side
Ri-chárd
se-man-tics
STREU-SLE
Tie-de-mann
}
\hyphenation{
affri-ca-te
affri-ca-tes
an-no-tated
com-ple-ments
com-po-si-tio-na-li-ty
non-com-po-si-tio-na-li-ty
Gon-zá-lez
out-side
Ri-chárd
se-man-tics
STREU-SLE
Tie-de-mann
} 
  \togglepaper[2]%%chapternumber
}{}

\begin{document}
\maketitle 
\label{chap:2}
%\shorttitlerunninghead{}%%use this for an abridged title in the page headers



\section{Introduction} %1. /

Over the past two decades, the number of students with a migrant background has been growing in the Icelandic education system. This increase is in line with changes in the society regarding the percentage of the country’s inhabitants defined as immigrants. In 2000, 2.6\% of the population were defined as immigrants, but twenty years later the percentage was 15\% (\citealt{Statistics_iceland2020-children}). These changing demographics are reflected in student populations at all school levels, as in 2019, 14.5\% of all preschool children (\citealt{Statistics_iceland2020-foreign}) and 11.5\% of all compulsory school students had a mother tongue other than Icelandic (\citealt{Statistics_iceland2020-children}). This situation has called for the school system to respond to the unique situation these students are in; learning Icelandic and gaining skills and confidence in using the language in their education.

Recent research in Iceland indicates that the school system faces challenges in meeting the educational and social needs of this group of students. In sum, findings have shown that multilingual students are often socially isolated, many are at a greater risk of developing mental health conditions than their peers, and they seem less inclined to participate in leisure and sports activities (\citealt{Gumundsdottir2013, Ministry_of_education_science_and_culture2020-3, Runarsdottir2015-1, Orisdottir2018-1}). In addition, their educational attainment is below that of their Icelandic peers, according to PISA findings (\citealt{Oecd2019-1}), and they are less likely to graduate from upper secondary school (\citealt{RagnarsdottirLefever2018}). The situation in the other Nordic countries is somewhat similar. When looking at educational outcomes in mathematics, reading and science for immigrant students in Denmark, Finland, Norway and Sweden, their outcomes are worse than those of students with non-immigrant background (\citealt{Torslev2017}).

Studies of teachers of Icelandic (\citealt{Gunnorsdottir2020-1, Oskarsdottir2017-1}) report that teachers seem to lack the professional knowledge and skills to meet the educational needs of diverse and multilingual students. Furthermore, school principals and school authorities have  called for measures to support, encourage and stimulate teachers’ interest in and ambition for including multilingual students in their classrooms (\citealt{Gunnorsdottir2017-1}). However, other studies have found examples of successful inclusive practices in schools, most often at the initiative of individual teachers rather than a whole school sustainable practice (\citealt{Gujonsdottir2016}). Similar challenges to those mentioned above have been identified in other Nordic countries as well, such as difficulties for students in introductory classes in relating to and making friendships with other children in the regular class (Norway); good practices being more commonly the work of individual teachers rather than a common school culture (Finland); and teachers arguing that they do not have time to give the newly arrived students the extra support they needed which resulted in stress and frustration among the teachers (Sweden) (\citealt{RagnarsdottirKulbrandstad2018}). In the UK context, \citet{FlocktonCunningham2021} argue that there has been a continuous and distinct lack of external guidance and support for teachers teaching learners who speak English as an additional language. Furthermore, no real consensus has yet been achieved as to what constitutes an appropriate pedagogical framework for this group. In Germany, the DaZ module (German as a second language) has been incorporated into teacher education at a national level since 2009. Research shows that the module has potential to support teachers to work with multilingual students; however, it does not utilise this potential fully as its realisation in teacher training mainly depends on the perspectives that educators individually bring from their own experiences (\citealt{GoltsevEtAl2022}). This turns the focus on teacher education and how pre-service teachers are prepared to work in multilingual classrooms. Although there have been some positive changes in recent years, teacher education programmes in the Nordic countries have not emphasised multilingual pedagogy (\citealt{Calafato2020,Iversen2021,KieranAnderson2019,RaudOrehhova2020}). {The above-mentioned lack of support and guidance for teachers and of consensus on the pedagogy employed is reflected in the research field. Compounding these challenges, there is a lack of research~– actually no research at all~– that addresses this issue from the perspective of teacher education in Iceland. This study is therefore an important and valuable contribution to this research field.}

This chapter will concentrate on the situation in Iceland based on interviews with recently graduated teachers and by analysing the existing teacher education course syllabi. The aim of this chapter is to cast a light on how the teacher education programmes at the Universities of Iceland (UI) and Akureyri (UNAK), prepare and support pre-service teachers in their initial education to work with learners for whom Icelandic is an additional language. The research questions addressed are:

\begin{itemize}
\item How do the teacher education programmes at UI and UNAK prepare pre-service teachers to work with multilingual students? 
\item How do recently graduated teachers feel they are prepared for teaching multilingual students? 
\end{itemize}

\section{Background}

In this section we first introduce the main policy documents that have a direct bearing on the situation, that is the \textit{Current Act on Education in Iceland,} \textit{The National Curriculum Guide for Compulsory Schools, The Action Plan for immigrants 2016–2019} and \textit{Draft policy: Education of children and youth with diverse language and cultural background from 2020.} Secondly, we describe the organisation of teacher education in Iceland.

\subsection{Policy background} %2.1 /

\textit{The Current Act on Education}\footnote{\url{https://www.government.is/media/menntamalaraduneyti-media/media/frettatengt2016/91_2008-Compulsory-School-Act-ENGLISH-Uppfaert-Jan-2017.pdf}} in Iceland strongly emphasises equality at the preschool, compulsory, and upper secondary school levels. Schools are expected to adapt their operations as closely as possible to the situation and needs of the students; thus, in a broad sense, supporting every student’s development, welfare and education (\citealt{Ministry_of_education_science_and_culture2011}). The education policy is based on ideas of inclusive education, focusing on meeting students’ diverse needs, whether those be academic or social; thus, schools should ensure that students are given equitable educational opportunities. According to the laws governing different educational levels, all students are entitled to an equitable education at the preschool, compulsory, and upper secondary school levels (\citealt{Ministry_of_education_science_and_culture2011}).

Thus, teachers are expected to respond to the educational needs of all students, including those with different language backgrounds. This calls for an inclusive pedagogy where teachers organise teaching, apply teaching methods and evaluate students’ learning according to their needs. However, research on the Icelandic school system has shown that the implementation of inclusive approaches  has lacked guidance, support and a structured approach (\citealt{European_agency_for_special_needs_and_inclusive_education2011}). The cause might be a lack of focus on both initial teacher education and teachers’ professional development.

\textit{The National Curriculum Guide for Compulsory Schools} (\citealt{Ministry_of_education_science_and_culture2011}) emphasises improving the Icelandic language skills of students with a foreign language background. Proficiency in Icelandic is considered a key prerequisite for becoming active participants in society, based on democratic principles of equal opportunity.  However, schools should also encourage parents to support their children’s Icelandic learning and at the same time cultivate and develop their own languages, to promote active bilingualism. The only guidance provided states that when the teaching of Icelandic as a second language in compulsory education is organised, the age, maturity and needs of students should be considered. Furthermore, students’ experiences, cultural backgrounds and academic status should be taken into account The criteria or learning outcomes in Icelandic as a second language are set out in four categories: spoken language and listening, reading, literature and writing (\citealt{Ministry_of_education_science_and_culture2013}). The chapter on Icelandic as a second language in the National Curriculum Guide for Compulsory Schools is currently being revised.

The \textit{Parliamentary resolution on an action plan on immigrants for the years 2016–2019} recommends action within five main categories: society, family, education, the labour market and refugee issues (\citealt{Alingi2016}). A total of 30 measures were introduced and they aimed at ensuring equal opportunities for everyone living in Iceland, regardless of individual factors and circumstances. Six measures focus on educational support, and all of those are aimed at harnessing the education and human resources of immigrants, both for their own benefit and for society. Measure C.1. focuses on equal opportunities for learning, measure C.2. on active bilingualism/multilingualism to enhance the importance of mother tongue teaching in pre-schools, compulsory and secondary schools. Finally, measure C.3. aims at developing steps to respond to student school dropout rates and to increase the number of students with immigrant backgrounds who graduate from upper secondary schools.  

A draft policy issued by the Ministry of Education, Science and Culture (\citealt{Ministry_of_education_science_and_culture2020-3}) on the education of children and youths with diverse language and cultural backgrounds summarises the situation in detail, based on Icelandic and international research, Icelandic laws and regulations, curricula and school levels. The document highlights the urgency of responding to the current situation at various levels and presents proposals for action in seven sections. Section six refers to the education of teachers and after-school staff, emphasising that the teaching of children and young people with immigrant backgrounds should be an indispensable part of the basic education of all teachers and after-school staff. At the same time, they should receive on-going professional development on multiculturalism and plurilingualism. Furthermore, the draft policy emphasises that multicultural education, which celebrates diversity and is based on the resources and strengths of children and young people, should be the hallmark of the school system in future education policy (\citealt{Ministry_of_education_science_and_culture2020-3}). The draft policy both responds to and confirms recent research findings which state that the Icelandic school system seems to have difficulty educating children and youths with a foreign linguistic and cultural background (\citealt{Ministry_of_education_science_and_culture2020-3}).

These policy documents recommend that teachers of groups of linguistically diverse students   need to meet the needs of all their students, through a focus on equity and by employing diverse teaching and evaluation methods to accommodate students. Teachers have, however, found it challenging to translate these policies into pedagogical praxis to fulfil curricular demands (\citealt{Oskarsdottir2017-1}). The cause might be a lack of focus on such approaches in both initial teacher education and continuing professional development.  

\subsection{Teacher education in Iceland} %2.2 /

Currently, comprehensive teacher education in Iceland is provided at two universities: the University of Iceland (UI) and the University of Akureyri (UNAK). Since 2008, a 180 ECTS bachelor’s degree and a 120 ECTS master’s degree is required by law to gain a license to teach in pre-schools, compulsory schools and upper-secondary schools. The teacher education programme has no centrally defined compulsory core subjects and the teacher education institutions set their own curriculum protocols for study programmes in initial teacher education and determine content areas, competences and learning outcomes.  

The University of Akureyri, UNAK, is a small university, located in Akureyri, a town in the north of Iceland. UNAK has 2540 students (2020) of whom 395 were enrolled in educational programmes at both undergraduate and graduate level (\citealt{Unak2020-1}). Approximately 100 students enrol in educational studies every year (MEd or MT) leading to teacher certification, and 30 students register for educational science (MA) programmes; that is, theoretical studies which do not lead to certification, although a few courses are compulsory for MEd and MT students.

The School of Education at the University of Iceland, based in Reykjavík, had 2466 enrolled undergraduate and graduate students in 2020. There are two faculties that teachers graduate from with a certification: the Faculty of Education and Pedagogy (E\&P) and the Faculty of Subject Teacher Education (STE). The E\&P faculty graduates are general pre- and compulsory school teachers and those who focus on educational leadership. The STE faculty graduates have a focus on teaching a specific subject (such as Icelandic, foreign languages, mathematics, natural sciences, art and design) and aim to teach at the compulsory or upper secondary school level. The faculties have several programmes of study and pre-service teachers can select elective courses from across different faculties.

A recent legislative act on the education, competences and employment of teachers and school administrators includes a specification of the general and specialised knowledge, skills and competences that teachers and school administrators must possess (Lög Um Menntun, Hæfni Og Ráðningu Kennara Og Skólastjóra í Leik-, Grunn- Og Framhaldsskólum [Act on the education, competency {and} recruitment of teachers]\footnote{\url{https://www.althingi.is/lagas/nuna/2019095.html}}). A framework of competences for teachers that builds on the act was developed by the Ministry in collaboration with the teacher education institutions, the teachers’ union, and other stakeholders in 2022 (Reglugerð Um Hæfniramma Með Viðmiðum Fyrir Almenna Og Sérhæfða Hæfni Kennara Og Skólastjórnenda Við Leik-, Grunn- Og Framhaldsskóla [Regulation on a competence framework with criteria for the general {and} specialised qualifications of teachers {and} school administrators at preschool, compulsory {and} secondary schools]\footnote{\url{https://island.is/reglugerdir/nr/1355-2022}}). This means that the teacher education universities are currently in the process of reviewing their programmes in accordance with the requirements laid out in the act.

\section{Conceptual framework}  %3. /

The conceptual framework on which this chapter draws comprises aspects of inclusive education, culturally responsive pedagogy and teacher knowledge. They form the ideological foundations on which the research is built. The conceptual framework is thus the compass to navigate the topic on multilingualism in teacher education as presented in the research questions.

Inclusive education implies that all students take part in and receive a quality education. It places a particular focus on reaching students who are at risk of being excluded or marginalised in schools (\citealt{Black-Hawkins2017,Florian2009}). As a social justice issue, inclusive education focuses on the intersection of diversities such as ability, gender, and sexuality, as well as culture, language, and socio-economic background. It builds on the premise that, without exception, quality education is a democratic right for all (\citealt{Pantic2015-1,Reay2012}). Organising instruction in a way that truly includes all learners calls for educators to consider how student differences affect learning, and to choose pedagogical strategies that effectively address those differences (\citealt{KieranAnderson2019}).

Through a focus on culture, language and experiences, culturally responsive pedagogy is viewed as an approach to achieve inclusion and student engagement. Other approaches grounded in empowering pedagogies and focusing on teachers’ roles in the plurilingual classrooms are outlined in, for example, \textit{Linguistically Appropriate Practice} (\citealt{Chumak-Horbatsch2012}) and \textit{Linguistically Responsive Teaching} (\citealt{LucasEtAl2008,LucasVillegas2013}), which both challenge the monocultural knowledge that has dominated teaching in schools for decades. In these publications, plurilingualism is instead framed as the norm, directing attention towards an understanding that language, culture, and identity are deeply interconnected, and teachers need to be aware of the sociolinguistic dimension of language education (\citealt{LucasVillegas2013}). The strategies that build on this approach include teachers using students’ cultural and linguistic resources as enabling resources rather than as a barrier to learning (\citealt{LefeverEtAl2018}). Individual experiences and interests are incorporated to facilitate learning and build on students’ cultural, linguistic, and ethnic experiences as a basis for interactive and collaborative teaching methods (\citealt{Chumak-Horbatsch2012}).

An oft-cited literature review by \citet{RychlyGraves2012} outlines four characteristics of culturally responsive teachers. These characteristics can be employed to frame what is needed in preparing pre-service teachers for teaching diverse students. The first characteristic emphasises that teachers are caring and empathetic, and persistent in their efforts to ensure the success of culturally diverse students. The second states that teachers reflect on their attitudes and beliefs about other cultures and languages. Thus, teachers acknowledge their biases and beliefs about students from diverse backgrounds. Third, teachers are also reflective about their own cultural frames of reference. This characteristic refers to teachers’ worldviews and their influence on instructional practices. The fourth and final characteristic is teachers’ progressive knowledge about other cultures and languages, that they actively seek out information about the cultures and languages represented in their classrooms, extending it beyond foods, flags, and holiday celebrations, and using it for adapting classroom curriculum and instruction to be more inclusive.

When the above characteristics are viewed in the framework of teacher knowledge (\citealt{Shulman1986}), they can, in combination, provide a three-dimensional picture of teacher education. In Shulman’s framework, specific teacher knowledge can be viewed as residing in propositional knowledge, case knowledge and strategic knowledge. These knowledge forms apply to the domains of content, pedagogy, and curriculum as well as those of management, organisation, and individual differences among students. Propositional knowledge is, according to \citet{Shulman1986}, normative and reflects on the norms, values, ideological or philosophical commitments of justice, fairness, equity, and such concepts that construct teacher knowledge. Case knowledge has to do with how teachers learn about principles of practice through narratives and parables and the experience of others, whereas strategic knowledge is the learning that takes place when case knowledge collides with experience in practice, thereby forming new knowledge (\citealt{Shulman1986}).

Later, Shulman expanded his theory and referred to “three apprenticeships” (\citealt{Shulman2007} in \citealt[192]{FlorianRouse2010}). The first apprenticeship is of the head, which refers to the knowledge and theoretical foundations of the profession. The second is the apprenticeship of the hand, which represents the technical and practical skills teachers employ and the third apprenticeship is of the heart, which, like propositional knowledge, refers to the attitudes, values and moral dimensions essential for the profession (\citealt{Shulman2007} in \citealt[192]{FlorianRouse2010}).

Recent emphasis in multilingual pedagogy has focused on the importance of supporting students in maintaining their mother tongue as they learn the new language (\citealt{BaileyMarsden2017}) and building on their strengths and prior knowledge to help compensate for what they are lacking regarding the new language and culture (\citealt{Cummins2014}). According to \citet{Cummins2017}, educators who work in linguistically diverse contexts must teach through a multilingual lens in order to teach the whole child. Such an instructional approach has been shown to have a positive effect on students’ cognitive and metalinguistic development as well as on the cross-lingual relationship between students’ first and second languages.

\section{Methods and data} %4. /

The aim of this chapter is to shed light on how the teacher education programmes at UNAK and UI prepare and support pre-service teachers to work with multilingual learners, through analysis of documents (current teacher education course syllabi) and three focus group interviews with newly graduated teachers who had studied at the Universities of Akureyri (UNAK) and Iceland (UI). 

\subsection{Data collection}

Data collection involved two data sources. Firstly, the current teacher education course syllabi (from online course catalogues) in the two  teacher education institutions in Iceland (UNAK and UI). The focus was on the Master of Education (MEd) and Master of Teaching (MT) degree programmes, as they lead to licenced teaching certification. We read through the syllabi and selected courses relevant for our topic.

The criteria for choosing which course syllabi to analyse were that their description had to include the words ‘multicultural’ (Icelandic: \textit{fjölmenning}), ‘language’ (I: \textit{tungumál}), and/or ‘diversity’/’inclusion’ (I: \textit{fjölbreytileiki\slash menntun fyrir alla\slash menntun án aðgreiningar\slash skóli án aðgreiningar}). Secondly, three focus group interviews with newly graduated teachers were conducted, with two to three participants in each focus group (see Table~\ref{tab:gunnthorsdottira:1}). The aim of the focus group interviews was to explore whether, how and in what way the newly graduated teachers feel prepared to teach multilingual students.

The method of finding participants was by sending out emails to newly graduated teachers and following up on these emails multiple times. It turned out to be a challenge to recruit participants for the interviews due to Covid-19 and the difficult work situation teachers consequently found themselves in. Thus, the number of participants was lower than we had planned for.

The interviews were conducted in autumn 2021, recorded via TEAMS or Zoom and lasted {approximately} one hour. The focus groups were conducted in Icelandic, recorded and transcribed verbatim. Translation of data extracts was done by the authors. The two participants from UI were interviewed together and the ones from UNAK were interviewed in two groups.


\begin{table}
\begin{tabularx}{\textwidth}{llcQ}
\lsptoprule
       &  & Graduation & \\
{Name} &  & {year} & {Specialisation}\\
\midrule
Eva   & UNAK & 2021 & Biology lower and upper secondary school \\
María & UNAK & 2021 & Preschool\\
Harpa & UNAK & 2021 & Math and Natural science lower secondary \\
Karen & UNAK & 2020 & Primary education\\
Erna  & UNAK & 2019 & Primary education \\
Alma  & UI & 2020 & Subject teaching\\
Ester & UI & 2020 & Preschool\\
\lspbottomrule
\end{tabularx}
\caption{Focus group participants}
\label{tab:gunnthorsdottira:1}
\end{table}

\begin{sloppypar}
The guiding questions were related to how the teacher education programme had prepared them as teachers to work with multilingual students, if they recalled courses with this emphasis, their experience from the practicum related to multilingual students and suggestions for improvement. Collecting data from both the course syllabi in the teacher education programmes and the focus groups provided information from different perspectives, as they have different origins and roles and also serve as a type of triangulation method, drawing on multiple sources (\citealt{Stake2000}). Document analysis is a systematic procedure employed to review diverse forms of printed or electronic documents. This research method can provide a way of tracing developments and change (\citealt{Bowen2009}) and it is suited to deducing meaning, developing a deeper understanding, and uncovering new perspectives on a research problem or question (\citealt{Merriam2009}). Focus group interviews are useful for capturing a variety of viewpoints on a topic where the aim is to reveal multiple perspectives, trends or patterns regarding an issue (\citealt{BrinkmannKvale2015}).
\end{sloppypar}

\subsection{Data analysis} %4.2 /

The data were analysed through two methods. Document analysis was used on the course syllabi in the teacher education programmes at UNAK and UI. The purpose of the analysis was to highlight the kind of courses offered for pre-service teachers regarding multilingual students. Thematic analysis was employed on the focus group data (\citealt{BraunClarke2006}).

The analytical process involved three steps. First, we, both separately and together, analysed the course syllabi documents. This involved identifying, selecting and making sense of the data from each document. The data was then categorised based on content.

Secondly, we, also both individually and together, used thematic analysis on the interview data. The third step was to create data tables in Excel and search for common themes and contradictions across the documents and the interviews to look for answers to the research questions. We read through the data multiple times and used different colours to find themes and in collaboration we systematically compared the themes from the interviews to findings from the document analysis. The dominant categories were then identified, summarised and are presented in \sectref{sec:gunnþórsdóttir:5} “Findings”.

\subsection{Ethical considerations} %4.3 /

Informed consent was obtained from the participants as they accepted the invitation to the focus group interviews. The names of participants are all pseudonyms, and the study was conducted in accordance with the Act on personal data protection and processing of personal data (Lög Um Persónuvernd Og Vinnslu Persónuupplýsinga\footnote{\url{https://www.personuvernd.is/media/uncategorized/Act_No_90_2018_on_Data_Protection_and_the_Processing_of_Personal_Data.pdf}}). 

\section{Findings}\label{sec:gunnþórsdóttir:5} %5. /

The findings are presented in two main sections: first course syllabi in the teacher education programme (\sectref{sec:gunnþórsdóttir:5.1}) and then findings from the focus groups with newly graduated teachers (\sectref{sec:gunnþórsdóttir:5.2}). In \sectref{sec:gunnþórsdóttir:6}, Discussions and implications, we bring the findings together and summarise the main implications.

\subsection{Course syllabi in the teacher education programmes}\label{sec:gunnþórsdóttir:5.1} %5.1 /

We analysed the course syllabi at both universities that provide teacher education in the country. Seven course syllabi from UNAK and nine from UI were analysed. The findings are presented in Sections~\ref{sec:gunnþórsdóttir:5.1.1} and~\ref{sec:gunnþórsdóttir:5.1.2}, respectively. 

\subsubsection{Courses offered at UNAK}\label{sec:gunnþórsdóttir:5.1.1} %5.1.1 /

When looking at the courses for pre-service teachers offered by UNAK that met the criteria of including a focus on multicultural or multilingual students, only one is mandatory for pre-service teachers at all levels (pre-, primary and secondary level), and one is mandatory for the preschool level only. The course \textit{The Student in a Diverse and Inclusive School} (10 ECTS) is aimed at all school levels. It focuses on changes in Icelandic society, globalisation and migration and their effects on educational work and students’ learning. Theoretical concepts of multiculturalism, democracy and inclusive education are discussed and related to policy and everyday work in school. Particular emphasis is placed on how teachers can meet the needs of a diverse student group with flexible organisation and a variety of teaching methods.

\begin{sloppypar}
Those who are registered in preschool teacher education study a mandatory course: \textit{Language Stimulation and Literacy} (10 ECTS). Here, the focus is on language stimulation for younger children, including second language learners through books, poems and rhymes. Pre-service teachers learn how to organise an environment which contributes to language stimulation and lays the foundation for literacy and writing.
\end{sloppypar}

Two of the five elective courses have some focus on language but are based in the MA programme which is more theoretically oriented than the MEd and MT programmes. The first course is \textit{Language and Literacy – the first steps} (10 ECTS) and focuses on language, language acquisition and literacy of students in preschool and the first years of primary school. The social and cultural aspect of language and literacy in society is discussed as well. During the course, pre-service teachers write a reflective paper, where multilingualism is one of the topics. They prepare a lesson plan for a diverse group in which they need to include multilingual students. The second course is \textit{Language and Reading Difficulties} (10 ECTS) which covers language and reading problems, their causes and manifestations, assessment, and follow-up. Pre-service teachers learn how to support students with purposeful learning and teaching methods, create an encouraging learning environment and provide positive reinforcement. They can choose to focus on multilingualism in assignments regarding language and reading difficulties.

The three remaining courses are all part of the MA programme but are electives for MEd and MT students: \textit{Democracy, Human Rights and Multiculturalism} (10 ECTS), \textit{Ideology and Policy in Inclusive Education} (10 ECTS) and \textit{School Counselling and Interviewing} (10 ECTS).  The content of these courses is on global issues such as changes in Icelandic society regarding globalisation and migration and the impact of such changes on education and schools. There is a discussion of concepts and ideas on human rights, social justice, democracy, inclusive school and minority groups, refugees and families. Issues such as multilingualism, translanguaging and mother tongue represent only a small part of this broad content. One of these courses focuses on the different learning needs of individuals and diverse teaching methods, emphasising how the teacher meets the needs of all students.

Taken together, the analysis shows that only one course is mandatory at all levels and to some extent focuses on multilingual students as part of meeting students’ diverse needs, multiculturalism, democracy and inclusive school. The two courses that are about language and reading are offered as an elective option, but the focus is mainly on general aspects of language and reading and less on how this relates to multilingual students.

\subsubsection{Courses offered at UI}\label{sec:gunnþórsdóttir:5.1.2} %5.1.2 /

Ten courses for pre-service teachers at the UI met the criteria of including a focus on multicultural or multilingual students. The courses are mandatory for pre-service teachers, depending on the programme or specialisation, but none of the courses are mandatory for all. There is a difference of approach apparent within the courses. 

Three courses have a specific focus on working with multilingual students: \textit{Bilingualism and Literacy, Icelandic as a Second Language} and \textit{Teaching Language in the Multicultural Classroom}. These courses are mandatory for the pre-service teachers in the faculty of Subject Teacher Education (STE) who are focusing on teaching Icelandic as a subject or on working with multilingual students. For other pre-service teachers these courses are elective.

Three courses have a specific focus on multiculturalism: \textit{Multicultural Society and Schools – Ideology and Research, Leadership in Inclusive Schools in a Multicultural Society,} and \textit{Religion in a Multicultural Society.} In these, the focus is on culture rather than language and on building knowledge and understanding of the multicultural society from various perspectives, such as that they should “have knowledge, overview and understanding of the ideology and research in the field of multicultural studies” (from the syllabus \textit{Multicultural Society and Schools – Ideology and Research}). These courses are mostly taught as electives for pre-service teachers; a few study programmes have them as mandatory in the STE faculty. 

The remaining four courses have a focus on inclusion, pedagogy and/or special needs. In the course \textit{Learning and Teaching: Supporting Children with Special Needs}, an emphasis is placed on addressing the “most common students’ special needs” and this includes the “students learning Icelandic as a second language”. This course is mandatory for pre-service teachers in the STE faculty and elective for others. The course \textit{Pedagogy, Diversity and Inclusion} has a focus on participants becoming familiar with main concepts and ideas regarding inclusion and diversity. Another course, \textit{Working in Inclusive Practices}, has a similar focus; however, the aim is to enable participants to strengthen their pedagogical competences in working with diverse groups of students. These two courses are mandatory in a few study programmes but elective in others. The last course is a pedagogy course with a focus on mathematics: \textit{Mathematics for All}. Here the focus is on how teachers can design and adapt a curriculum and teach mathematics to diverse groups of students. Furthermore, the study of teaching and learning in multicultural settings is discussed. This course is elective for pre-service teachers in both the STE and E\&P faculties.

The above analysis shows that pre-service teachers can choose from a range of courses. However, very few of them are focused on language learning or how to teach multilingual students. The courses that concentrate on pedagogy and diversity, in some instances, have a special needs focus that links multilingualism\slash multiculturalism to being a problem rather than a resource to build on. 

\subsection{Pre-service teachers’ experiences}
\label{sec:gunnþórsdóttir:5.2}%5.2 /

Findings from the focus group interviews with newly graduated teachers revealed their experiences of teacher education and their ideas for improvement. The following sections indicate the themes generated from the data.  

\subsubsection{Lack of focus on teaching multilingual students}  %5.2.1 /

In general, the teachers reported that the courses they took opened their eyes towards diverse cultures and supported them in gaining an understanding of the complexities relating to “all kinds of diversity and cultures”. However, most stated that they had no recollection of taking a course that emphasised teaching multilingual learners or offered a focus on pedagogy or teaching methods for working with these students. As Karen (UNAK) said “I can’t remember any course, reading material or assignment where this was the focus”. Ester, a pre-school teacher (UI), remembered that in one of the courses she took, “one lecture mentioned this, I looked it up in my course notes”.

The teachers stated that they knew about courses on offer that had a focus on multiculturalism, but all mentioned that they had little flexibility of choice. Ester felt that “there is a lack of motivation for students to focus on multiculturalism”. Her view was that pre-service teachers need to be encouraged to focus on how to teach this group of students as this is the reality they will face in schools. 

\subsubsection{The value of practicum in schools} %5.2.2 /

The teachers mentioned that while there was not much emphasis on practical teaching in courses, more knowledge was gained through their practicum in schools. María (UNAK) said: “I took my practicum in a very multicultural school, so really I learned much there; it was a really valuable experience”. The same experience was echoed by the other participants. Alma (UI) stated that her knowledge about how to teach multilingual students came from her practicum. She also said that “it was because I was interested in the matter”, not that this was required in her studies. In one of the interviews Ester (UI) mentioned the importance of having a choice of diverse types of schools for practicum placement, including schools with high numbers of multicultural and multilingual students. In her case (pre-primary level), she had to obtain special permission to take her practicum in a school located in a neighbourhood with a high immigrant population. In general, the schools in this area were not listed as available choices. 

\subsubsection{Assignments and optional course choice}  %5.2.3 /

In~general, the participants reported that their assignments had little emphasis on multilingual students and their needs. UNAK graduates, however, mentioned that when writing lesson plans in some courses, they had to account for students who needed additional support or did not have Icelandic as their mother tongue and explain how they would include support for them in their planning. Alma, a participant from UI, mentioned she thought it important to include awareness of this group of students in all assignments and also during the practicum. However, reflecting on her studies, she did not recall being required to pay special attention to multicultural or multilingual students in her assignments. Ester agreed and said: 

\begin{quote}
All these years in my pre-primary study there were no assignments on multiculturalism, and that surprised me a lot.
\end{quote}

According to the participants, course assignments do not place emphasis on multilingual students or multiculturalism.

\subsubsection{Ideas for improvement} %5.2.4 /

When asked what might have been addressed better or was lacking in their studies, Harpa (UNAK), a teacher at the lower secondary level, mentioned that her focus was on natural science and maths. She would have liked to have had more information incorporated in the courses on how she could teach math and natural science to students who are learning Icelandic, in conjunction with the language content. The focus in her study was mostly on learning the subjects per se.

The UNAK students believed that slightly more courses including a focus on multicultural or multilingual students were offered for those preparing to work in pre-school than for those at the compulsory level, as the courses at that level focused on subjects such as math, natural science, etc. Alma, a UI student, mentioned that she would like to have had more room for elective credits in her studies. Teaching multilingual students is of great interest to her, but she had very little flexibility to choose courses relevant to the topic. She and Eva (UNAK) enrolled in teacher education with a BS degree (Bachelor of Science) from another department and thus studied for two years to get a teacher certification (MEd). Both of them emphasised how important it is for pre-service teachers with that background to have access to courses that focus on pedagogy and practical teaching. All of the participants, from both universities, stated that they would have liked to take practical courses focusing on pedagogy and teaching strategies that benefit multilingual students and to receive more information on the kind of materials available to teachers.

\section{Discussions and implications}\label{sec:gunnþórsdóttir:6} %6. /

The findings were structured around two focus points:  teacher education course syllabi and teachers’ experiences. We will now discuss the findings and summarise the main implications.

All the policy documents clearly state that schools in Iceland should emphasise equality and adapt their operation as closely as possible to the situation and needs of the students (\citealt{Ministry_of_education_science_and_culture2011}). The recent draft policy by the \citet{Ministry_of_education_science_and_culture2020-3}, on the education of children and youths with diverse language and cultural backgrounds takes a very clear stance and emphasises the urgency of responding to the current situation at all levels of the education system, including teacher education.

Both universities offer courses that contribute to pre-service teachers’ general knowledge on student diversity in schools and stress the importance of looking at students’ education in a globalised, multicultural and plurilingual world (\citealt{Gay2002}). The courses also offer a solid theoretical base on relevant topics. There is, however, some divergence between the universities as to the number of courses offered; for example, fewer courses are offered at UNAK than UI, which is not unexpected, given the difference in size between the two universities. Another contrast is that the courses at UNAK mostly offer students a theoretical background on concepts such as multiculturalism, multilingualism, democracy, and inclusive schools and how these relate to policies, laws and curricula. There are no separate courses focusing on pedagogy and how to manage everyday work in schools with students with a multilingual background, although there seems to be a certain awareness in some courses.

The general and broad emphasis in the policy documents is thus reflected in the courses offered to students at both universities. In the interviews with the teachers, the same priority is expressed; that is, the courses have opened their eyes towards diverse cultures and helped them gain an understanding of a complex situation relating to multicultural/multilingual education. However, little attention was given to pedagogy or practical teaching methods, most of which they obtained through their practicum in schools. In an interview study with pre-and in-service teachers in Iceland presented in this volume \parencite{chapters/4_gudjonsdottir}, similar findings were identified, that is that a focus on multicultural education in the teacher education programme is lacking, although teacher educators are aware of the necessity of preparing pre- and in-service teachers to teach students with diverse linguistic backgrounds. The changes that need to take place pertaining to strengthening the focus on multicultural and multilingual education in teacher education, depend on the teacher educators’ perspectives and the spaces they create in their teaching for multilingualism (\citealt{GoltsevEtAl2022}).

Several courses at the University of Iceland emphasise the pedagogy of how to teach diverse learners; it is interesting to note, however, that the pre-service teachers who are studying subject teaching (other than teaching Icelandic) have fewer courses to choose from. The mandatory course for this group of students is \textit{Learning and Teaching: Supporting Children with Special Needs.} The name of this course indicates that diversity is linked to disability. This seems to go against the ideas of culturally responsive pedagogy requiring educators to design instruction from the perspective that student diversity is a strength rather than a deficit (\citealt{KieranAnderson2019}).

Interpreting the above findings from the perspective of \citet{Shulman1986}: propositional knowledge, case knowledge and strategic knowledge, it is apparent that students are mostly offered propositional knowledge. This knowledge is normative and reflects on the norms, values, ideological or philosophical commitments of justice, fairness, equity, and such. There is less focus on Shulman’s second and third type of knowledge, that is, case knowledge – how teachers learn about principles of practice through cases and parables – and strategic knowledge, which is the learning that takes place when case knowledge collides with experience in practice (\citealt{Shulman2007} in \citealt[192]{FlorianRouse2010}). \citet{FlocktonCunningham2021} have pointed out the need to establish and identify what constitutes good practice when teaching multilingual students and also to establish how pre-service (and in-service) teacher training can be developed and improved in relation to developing teachers’ teaching skills relating to teaching this group of students.

As summarised above, recent research in Iceland indicates that the school system has not quite managed to meet the educational or social needs of students with diverse language backgrounds (\citealt{Gumundsdottir2013, Runarsdottir2015-1, Orisdottir2018-1}), and teachers lack the professional knowledge and skills to meet the educational needs of this group of students (\citealt{Gunnorsdottir2020-1}). The analysis of the teacher education course syllabi and the interviews with teachers show that there is an opportunity to increase emphasis on culturally and linguistically responsive pedagogy in teacher education, as this approach includes using students’ cultural and linguistic resources in their schooling (e.g. \citealt{BaileyMarsden2017}).

Our findings indicate that in part we see the same pattern in teacher education as other researchers in the field: teachers are aware of the situation (the challenges) of their students, everyone is “willing to do their best” but teachers lack the practical pedagogy to support multilingual students (\citealt{FlocktonCunningham2021, Gunnorsdottir2020-1, chapters/6_iversen, Oskarsdottir2017-1}). This is not solely an Icelandic phenomenon, as research findings from the UK indicate that new teachers there do not feel confident or prepared to work effectively with multilingual students either, and that it is naïve to assume that teachers will be able to “learn on the job” how to successfully support this group of students (\citealt{FlocktonCunningham2021}). The lack of relevant courses is more visible in the study programme at the University of Akureyri, as was confirmed in interviews with the teachers. It is thus vital that teacher education institutions design their study programmes and courses with a focus on these needs, and offer courses on how to adapt the curriculum to teaching diverse groups of students in general, not least in individual subjects such as mathematics, natural and social sciences, that will lay the foundation for their future study.

Although our sample size was limited, our findings indicate that teacher education in Iceland does not sufficiently prepare future teachers to teach multilingual students. This is in line with recent research findings in Iceland (see \citet{chapters/4_gudjonsdottir, Ministry_of_education_science_and_culture2020-3}). There is room to improve the course catalogue and include mandatory courses on multilingual pedagogy for all pre-service teachers as well as enable pre-service students to choose more courses and thus specialise in multilingual teaching. Adding a focus on this issue in the practicum in the pre-service teacher education would also be valuable. In past decades, there has been a call for emphasis on this specialisation by teachers in Iceland and elsewhere. Due to increased re-location of people in the world, teachers find themselves teaching classrooms where an increasingly larger number of students do not speak the majority language. If teachers worldwide are not supported to work with multilingual students, they and their communities will be deprived of the experience, knowledge and skills of immigrant populations.

\sloppy\printbibliography[heading=subbibliography,notkeyword=this]
\end{document}
