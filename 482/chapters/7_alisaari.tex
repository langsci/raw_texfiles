\documentclass[output=paper]{langscibook}
\ChapterDOI{10.5281/zenodo.15280905}
\author{Jenni Alisaari\orcid{}\affiliation{Stockholm University} and Leena Maria Heikkola\orcid{}\affiliation{Åbo Akademi University} and Raisa Harju-Autti\orcid{}\affiliation{University of Jyväskylä}}
\title[“You have to choose your words wisely”]{“You have to choose your words wisely”: Finnish pre-service teachers’ understandings of, and support for, multilingual students’ academic language development}
\abstract{The aim of this study was to examine Finnish pre-service primary school teachers’ ($n = 92$) understandings of academic language development and self-reported preparedness to support language learning when teaching multilingual students (MLSs) in Finnish primary schools. Data were gathered with a survey and analysed qualitatively and quantitatively. Results indicated that the pre-service teachers had a reasonable understanding of language learning and reported skills to support academic language development, but there were gaps in awareness of how to provide concrete linguistic support for MLSs. Moreover, there were considerable gaps in the pre-service teachers’ grammar knowledge and one-third of them did not see a connection between language and mathematical problem-solving skills. The results indicate that the pre-service teachers’ awareness of linguistically responsive teaching is still developing. Providing pre-service teachers with research-based and practice-oriented instruction for supporting academic language development in all subjects is needed.}
\IfFileExists{../localcommands.tex}{
  \addbibresource{../localbibliography.bib}
  \usepackage{langsci-optional}
\usepackage{langsci-gb4e}
\usepackage{langsci-lgr}

\usepackage{listings}
\lstset{basicstyle=\ttfamily,tabsize=2,breaklines=true}

%added by author
% \usepackage{tipa}
\usepackage{multirow}
\graphicspath{{figures/}}
\usepackage{langsci-branding}

  
\newcommand{\sent}{\enumsentence}
\newcommand{\sents}{\eenumsentence}
\let\citeasnoun\citet

\renewcommand{\lsCoverTitleFont}[1]{\sffamily\addfontfeatures{Scale=MatchUppercase}\fontsize{44pt}{16mm}\selectfont #1}
   
  %% hyphenation points for line breaks
%% Normally, automatic hyphenation in LaTeX is very good
%% If a word is mis-hyphenated, add it to this file
%%
%% add information to TeX file before \begin{document} with:
%% %% hyphenation points for line breaks
%% Normally, automatic hyphenation in LaTeX is very good
%% If a word is mis-hyphenated, add it to this file
%%
%% add information to TeX file before \begin{document} with:
%% %% hyphenation points for line breaks
%% Normally, automatic hyphenation in LaTeX is very good
%% If a word is mis-hyphenated, add it to this file
%%
%% add information to TeX file before \begin{document} with:
%% \include{localhyphenation}
\hyphenation{
affri-ca-te
affri-ca-tes
an-no-tated
com-ple-ments
com-po-si-tio-na-li-ty
non-com-po-si-tio-na-li-ty
Gon-zá-lez
out-side
Ri-chárd
se-man-tics
STREU-SLE
Tie-de-mann
}
\hyphenation{
affri-ca-te
affri-ca-tes
an-no-tated
com-ple-ments
com-po-si-tio-na-li-ty
non-com-po-si-tio-na-li-ty
Gon-zá-lez
out-side
Ri-chárd
se-man-tics
STREU-SLE
Tie-de-mann
}
\hyphenation{
affri-ca-te
affri-ca-tes
an-no-tated
com-ple-ments
com-po-si-tio-na-li-ty
non-com-po-si-tio-na-li-ty
Gon-zá-lez
out-side
Ri-chárd
se-man-tics
STREU-SLE
Tie-de-mann
} 
  \togglepaper[1]%%chapternumber
}{}

\begin{document}
\maketitle 
\label{chap:7}
%\shorttitlerunninghead{}%%use this for an abridged title in the page headers



\section{The aim and background}\label{sec:alisaari:1}

The aim of this survey study is to examine Finnish pre-service primary school teachers’ ($n = 92$) understandings of academic language development and self-reported preparedness to support language learning when teaching 7 to 13-year-old (grades 1 to 6) multilingual students in Finnish primary schools. In Finland, most primary school teachers teach all subjects in grades 1 to 6. The term \textit{multilingual student} (MLS) refers to students who have a first language (L1) other than Finnish or Swedish, the languages of instruction in Finnish schools. However, we acknowledge that multilingualism is a much broader concept that can also include students whose L1 is Finnish or Swedish. We want to avoid deficit terms when referring to students, and thus, in this article, the term MLS covers both students with a migrant background and other students whose L1 is different than the language of instruction, including national minority language students.  Since many of the MLSs in Finland have a migrant background, we pay special attention to this group in the theoretical part of this chapter. 

As the number of MLSs continues to increase worldwide (\citealt{Migration_data_portal2020}), the role of languages in learning is taking center stage in education. Studies have shown a significant gap in learning outcomes between students with a migrant background and majority-language speakers in many member countries of the Organisation for Economic Co-operation and Development (OECD), including Finland (\citealt{Ahonen2021}). While mastering the more formal language used in school can be challenging for all students, regardless of their linguistic background, students with a migrant background often face a range of educational obstacles, including learning gaps, challenges in transitions, and lower educational attainment than their majority peers (\citealt{Borgna2017}). Teachers play a significant role in making instruction comprehensible for their students. Teaching language and content simultaneously is necessary in order to help students understand and produce language in the ways it is used in different subjects (\citealt{CumminsEarly2015}). However, several separate studies from different perspectives have shown a gap in Finnish in-service teachers’ pedagogical skills and their knowledge about linguistic diversity (\citealt{AlisaariEtAl2019,Alisaari2020_Apples,HeikkolaEtAl2022}). Surprisingly, few studies have focused on how teachers can “analyze the language demands embedded in academic text and learning tasks, an indispensable skill to scaffold instruction adequately” (\citealt[152]{VillegasEtAl2018}) for learners of the language of instruction. This study aims to contribute to this topic.

Learning a language takes time. Attaining academic language proficiency may take five to seven years (\citealt{Cummins2021}). To promote inclusive education for all learners, regardless of their linguistic backgrounds, developing meaningful approaches and methods for teaching both subject content and the language of schooling is essential (\citealt{Harju-AuttiEtAl2022}). Thus, language-related pedagogical matters should be incorporated into teacher education. Little, however, is known about pre-service teachers’ competencies in regard to this issue in Finland. See, however, \citet{chapters/8_heikkola}, regarding pre-service subject teachers’ preparedness for linguistically diverse classrooms.

The Finnish national core curriculum for basic education (\citealt{Finnish_national_agency_of_education2014}) states that all teachers should integrate language and content in teaching. However, without specific and thorough teacher education concerning linguistic development and the role of languages in learning, the necessary language awareness is not attained. It is important to investigate how teacher education prepares pre-service teachers in this regard. This study examines pre-service teachers’ understandings of academic language development and preparedness to scaffold their learners’ learning of the language of instruction and subject-specific content (see \sectref{sec:alisaari:3.2} of this chapter, also \citealt{CarlsonEtAl2018}). The study is guided by the following research questions:

\begin{enumerate}
\item  How do pre-service primary school teachers understand and report supporting academic language development?
\item  How prepared are pre-service teachers to support students in learning both the language of instruction and subject-specific content?
\end{enumerate}

In this chapter, we first introduce notions of linguistically responsive teaching and teacher competencies and present some studies in these fields (\sectref{sec:alisaari:2}). After introducing the methodology (\sectref{sec:alisaari:3}), we present the study findings and discussion (\sectref{sec:alisaari:4}), then finally our conclusions and practical implications based on this study (\sectref{sec:alisaari:5}).

\section{Language in teaching}\label{sec:alisaari:2}
\subsection{Linguistically responsive teaching}\label{sec:alisaari:2.1}

In today’s linguistically diverse schools, we must look beyond traditional language teaching to gain a deeper understanding of the role languages play in all learning. Language is essential for becoming socialised into the linguistic and cultural behaviours of different communities (\citealt{PhinneyOng2007}). Thus, the theoretical foundation for this study lies in sociocultural language learning theories (\citealt{LantolfThorne2006, Lier2000,Vygotsky1986}), which view languages as being learned in social interaction and mediated by other language users in specific contexts. Importantly, language learners need comprehensible input as affordance to develop their language skills (\citealt{Lier2000}).  Linguistically responsive teaching means that teachers understand the significance of language in all learning and possess pedagogical skills that support learning in various situations (\citealt{Alisaari2020_Apples,LucasVillegas2013}). Knowing students' backgrounds and recognising the value of their linguistic resources is essential in linguistically responsive teaching (henceforth LRT) (\citealt{LucasVillegas2013}). Several studies have indicated the importance of students’ L1s in learning other languages and school subjects (e.g. \citealt{AgirdagVanlaar2018,Cummins2021,GanuzaHedman2018}).

Linguistically responsive teachers also need to recognise the challenges that the language of instruction may pose to learners (\citealt{Cummins2021,Gibbons2014,LucasVillegas2013,SchleppegrellEtAl2004}). Academic language differs noticeably from everyday social language (e.g. \citealt{Beacco2017,BeaccoEtAl2016,Cummins2021,SchleppegrellEtAl2004}). While basic everyday language covers the vocabulary and grammar used in informal, spoken social interaction, academic language is more abstract in terms of both vocabulary and grammar. Moreover, subject-specific language includes linguistic features typical of a subject. As all learning is connected to language, it is essential to understand that all subjects have terms, concepts, and vocabulary that can be considered a new form of language, even for students who are fluent in the language of instruction (\citealt{LahtiEtAl2020}). However, academic language is particularly challenging for MLSs, and the optimal way to develop it is in discussions between the students and the teacher (\citealt{TharpEtAl2018}). Therefore, academic and subject-specific language should be taught together with content knowledge in all subject areas (\citealt{CarlsonEtAl2018}). Moreover, teachers should scaffold instruction so learners can accomplish academic assignments at otherwise unattainable cognitive and language levels (\citealt{Gibbons2014,TharpEtAl2018,VillegasEtAl2018,Vygotsky1986}).

Some teachers have the misconception that the acquisition of language in general, and academic language in particular, occurs automatically in classrooms where academic language is used, regardless of the students’ linguistic backgrounds (\citealt{CarlsonEtAl2018}). Furthermore, it seems that teachers do not automatically develop an understanding of language dimensions (e.g. \citealt{Alisaari2020_Apples}), that is, the ways language varies between everyday language, academic language and subject-specific language (e.g. \citealt{BeaccoEtAl2016}). Mastering subject-specific language requires literacy skills that can only be acquired when literacy instruction is embedded in content classes (\citealt{ShanahanShanahan2008}).

Teachers’ knowledge concerning the role of languages in learning should be an essential part of their pedagogical and didactic skills (\citealt{CarlsonEtAl2018,ShanahanShanahan2008}). The ways teachers adapt their pedagogical practices to learners’ language skills affects how learners benefit from the teaching. Furthermore, it is optimal to facilitate the development of students’ academic language skills in a structured and systematic manner in all subject teaching (\citealt{CarlsonEtAl2018,ShanahanShanahan2008}), not exclusively during language lessons.  This can be effectively done through dialogue, where teachers facilitate learning processes (\citealt{TharpEtAl2018}). These kinds of dialogues can be described as a form of scaffolding, part of the multifaceted support teachers provide to help learners learn (\citealt{Gibbons2014}). Students who immigrated and began school in the new country at an older age are often highly dependent on teachers’ support (\citealt{Sharif2017}). However, effective pedagogy develops  {all} learners’ language and literacy skills across the curriculum (\citealt{CumminsEarly2015,TharpEtAl2018}).

\subsection{Teacher competencies}\label{sec:alisaari:2.2}

According to a review of studies on the effects of pedagogical competences, there seems to be a relationship between teachers’ general pedagogical knowledge, teaching quality, and students’ learning outcomes (\citealt{UlfertsEtAl2019}). In Finland, the current national core curriculum for basic education requires teachers to possess specific pedagogical knowledge in relation to language awareness, including using students’ multilingual repertoires as learning resources, and considering students’ diverse backgrounds when planning instruction (\citealt{Finnish_national_agency_of_education2014}). However, policy requirements alone do not necessarily lead to adequate changes on a practical level. Teachers need support in developing their competencies in fostering the learning of the language of instruction (\citealt{KiefferLesaux2012}), with special attention paid to using students’ entire linguistic repertoires in learning (\citealt{Cummins2021}).

Recent studies from many countries and different perspectives indicate that pre- and in-service teachers’ competencies in supporting multilingual learners are still developing (e.g. \citealt{AgirdagEtAl2014,AlisaariEtAl2019,Alisaari2020_Apples,FanecaEtAl2016,HeikkolaEtAl2021,Iversen2020,Lundberg2019,Rodriguez-izquierdo2020}). The deeper the teachers’ understanding of language learning and the importance of linguistic support, the better they are able to facilitate students’ learning and pedagogically justify their practices (\citealt{Alisaari2020_Apples,HeikkolaEtAl2022}). Thus, it is important to investigate how teacher education prepares future teachers to support all learners, regardless of their linguistic backgrounds.

\section{Methods}\label{sec:alisaari:3} %3. /
\largerpage[-2]
The aim of this study is to investigate Finnish pre-service teachers’ understandings of academic language development and their reported preparedness to support students in simultaneously learning the language of instruction and subject-specific content. In this section, we first present the research instrument and the participants of the study, and then we explain the data analysis. 

\subsection{Research instrument}\label{sec:alisaari:3.1} %3.1 /

The data were collected by means of a survey developed in Germany (DaZKom) and modified for the Finnish context. The original instrument was “intended to give empirically supported insights into how learning opportunities in academic teacher education must be designed for enabling the acquisition of a substantiated and standardized [German as L2] competency” (\citealt{CarlsonEtAl2018}). In this study, as in the original instrument (\citealt{CarlsonEtAl2018}), mathematics serves as an example for the realisation of subject-content integrated language scaffolding.

The original research instrument included a total of 68 items: 32 multiple choice, 14 closed-constructed, and 22 open-ended items. The Finnish version was modified due to differences in the educational contexts and the languages in question (German and Finnish). For example, in the Finnish study, only pre-service primary school teachers were studied, while in the original research, pre-service secondary school teachers were studied (\citealt{CarlsonEtAl2018}). Some mathematical items that did not correspond to the Finnish primary school mathematics curriculum were changed; the new examples were modified based on Finnish primary school textbook examples (see Appendix~\ref{appendix:alisaari}, Item A). Further, some language issues in German were not relevant in Finnish, so language tasks were adapted to reflect meaningful grammatical issues that can be challenging in Finnish. This required changing the whole task (see Appendix~\ref{appendix:alisaari}, Item B). The instrument adaptation was conducted by the first author, an expert in language education, in collaboration with a university mathematics teacher.

The original research instrument was relatively long, and thus was shortened for this pilot study. The Finnish version consisted of 22 items: seven multiple-choice questions with three to six choices each (29 possible responses), 12 open-ended questions (e.g. regarding the use of students’ linguistic resources for learning), and three background questions about the studies pre-service teachers had completed.\footnote{The Finnish version of the survey can be obtained from the first author (see also Appendices~\ref{appendix:alisaari} and~\ref{appendix:alisaari2}). There are also English versions developed, both in the US (\citealt{HammerViesca2023}) and in Canada (\citealt{Bale2023}).} In this study, the background questions and one multiple-choice question were omitted from the analysis because of a lack of space in this chapter and the questions’ ambiguity.

The original research instrument was divided into three dimensions: 1) the role of language as a medium for interaction and classroom actions; 2) language learning processes, multilingualism, and learners’ linguistic repertoires; and 3) teaching strategies associated with linguistic support. Recent research conducted in Finland suggests that the importance of L1 in learning is often overlooked in teacher education (e.g. \citealt{AlisaariEtAl2019,HeikkolaEtAl2022,Repo2020}). Therefore, a fourth dimension was added to the Finnish research instrument to reflect this research. Additionally, due to the modifications of the research instrument, we also adapted the descriptions of the dimensions of the instrument. Thus, the four dimensions in the Finnish instrument were: 1) developing skills in the language of instruction for learning; 2) knowledge of language, grammar, and semiotic symbols; 3) providing linguistic support; and 4) L1 as a tool for learning. To answer research question 1, we investigated pre-service teachers’ responses across the four dimensions.

In order to investigate pre-service teachers’ competencies to support students in learning (see also research question 2), we further analysed the pre-service teachers’ competency at group level within the four dimensions. This analysis was based on \posscitet{Dreyfus1986} model of adult skill acquisition that classifies learners as novice, advanced beginner, and competent. The competency levels were calculated based on the means of the participants’ responses:\footnote{See \sectref{sec:alisaari:3.3} for more information on the scoring.} novice (0–1.49), advanced beginner (1.5–2.99), and competent (3–4) following the original model by \citet{CarlsonEtAl2018}. According to \citet{DreyfusDreyfus1986}, novices determine their actions based on facts that can be identified without specific expertise, whereas advanced beginners already have experience in similar situations and base their actions thereon. Finally, competent teachers are able to make decisions based on experience and the ability to recognise a situation’s most important features, and they also feel secure about and committed to their decisions (\citealt{DreyfusDreyfus1986}). This was the first time this research instrument was used in the Finnish context; therefore, this study serves as a pilot version for further research projects in Finland.

\subsection{Participants and the context}\label{sec:alisaari:3.2} %3.2 /

The data were collected from third-year pre-service primary school teachers ($n = 92$) at the beginning of a pilot course on multilingual pedagogies, to gain knowledge on the need for such a course at this stage of their pedagogical studies. Since the course was in its piloting phase, it was beneficial to investigate pre-service teachers’ prior knowledge of academic language development and their preparedness to support (school) students in learning the language of instruction and subject-specific content simultaneously. 

The main theme of the course was LRT, and it comprised issues related to multilingual pedagogy, teaching different subjects’ content according to the principles of LRT, language development, assessment of language skills, supporting the development of MLSs’ language skills, the role of L1 in learning, and teaching the subject “Finnish as a Second Language and Literature”. Finnish grammar was not taught, but grammatical terms were used when talking about language; the pre-service teachers had already been taught grammar-related content during their first year, in a course related to the didactics of the Finnish language and literature. The course included both lectures (8 hours in total) and seminars in smaller groups (12 hours in total per group), and the pre-service teachers completed independent assignments to examine the themes covered in more detail.

Before starting the course and participating in the survey, the pre-service teachers were tasked with reading a book on learning Finnish as an additional language (\citealt{VaaralaEtAl2016}), as framed by the Finnish national core curriculum for basic education. The book covers multilingualism as a resource only very briefly, the main topic being teaching Finnish as a second language. Only 76\% of the pre-service teachers reported reading the book, and only 30\% stated that the book had helped them in answering the survey. Of the pre-service teachers ($n = 96$) who responded to the survey, 92 gave written permission to use their answers in this study.

In interpreting the results, the fact that one of the researchers in this study was also a teacher on the course has to be taken into account, as that may have caused some power imbalance. At the beginning of the study, the pre-service teachers were informed that the results could not be connected with individual participants, since we anonymised the responses before the analysis. In addition, the positionality of the teacher/author was thoroughly discussed among the researchers during the analysis process, and all the interpretations were done by the three authors.

\subsection{Analysis}\label{sec:alisaari:3.3} %3.3 /
\begin{sloppypar}
The data were analysed first qualitatively and then quantitatively. A rubric, namely the guidelines for the analysis,\footnote{The guidelines included possible responses for each survey item and their scoring based on how correct the answers would be. See an example of the Finnish version of the rubric in Appendix~\ref{appendix:alisaari2}. The original guidelines have not been published.} which was created for analysing the data gathered from the original survey, was adapted for the Finnish instrument by all the authors. Next, all three authors analysed the open-ended items independently using the rubric and discussed the analysis of each item until consensus was reached in the negotiations. The first author analysed the multiple-choice questions, which had correct and incorrect answers. 
\end{sloppypar}

After the answers were analysed qualitatively based on the guidelines created for the original survey, standardised total scores (max. 4 points) were calculated for the entire questionnaire, as well as separately for each of the four dimensions of the survey: 1) developing language skills for learning; 2) knowledge of language, grammar, and semiotic symbols; 3) providing linguistic support; and 4) using L1 as a tool for learning. The normalised total scores were calculated so that all total scores became 4. The results of the normalised total scores are presented in \tabref{tab:Alisaari:1} (see \sectref{sec:alisaari:4.1}). After presenting the normalised total scores, we discuss issues emerging from the qualitative analysis and give examples of participants’ responses. These analyses were done to answer research question 1.

Next, competency levels were investigated to answer research question 2. As we collected the data for this study using \citegen{Carlson2018} survey, we also followed their categorisation of competency, forming three teacher profiles based on the standard total scores: novice (0–1.49), advanced beginner (1.5–2.99), and competent (3–4) (\citealt{DreyfusDreyfus1986}). The frequencies of the competency levels are presented in \tabref{tab:Alisaari:2} (see \sectref{sec:alisaari:4.2}).

\section{Results and discussion}\label{sec:alisaari:4}

In this section, in order to answer research question 1, we present the pre-service primary school teachers’ understanding of language learning in the classroom based on the total scores of their responses (see \sectref{sec:alisaari:4.1}). We also describe interesting themes that emerged from the qualitative data. Finally, in order to answer research question 2 concerning the pre-service teachers’ reported preparedness to support their future students’ language learning, we investigate the participants’ competency levels based on their total scores (see \sectref{sec:alisaari:4.2}).

\subsection{Pre-service primary school teachers’ understanding of language learning}\label{sec:alisaari:4.1}  %4.1 /

In \sectref{sec:alisaari:4.1}, we present the results of our analysis of participants’ responses. We calculated total scores for all questions included in the survey and for the four dimensions (see \tabref{tab:Alisaari:1}).

\begin{table}
\begin{tabular}{l rrrrr}
\lsptoprule
Dimension & {$n$} & {M} & {SD} & {min} & {max}\\\midrule
Developing language skills for learning & 67 & 2.8 & 0.5 & 1.1 & 3.6\\
Knowledge of language, grammar,  & 70 & 2.8 & 0.5 & 1.5 & 3.7\\
   \quad and semiotic symbols\\
Providing linguistic support & 75 & 2.9 & 0.8 & 0.0 & 4.0\\
L1 as a tool for learning & 81 & 3.2 & 0.7 & 0.9 & 4.0\\
Total score (max. 4 points) & 51 & 2.9 & 0.4 & 1.6 & 3.7\\
\lspbottomrule
\end{tabular}
\caption{Total scores for the entire questionnaire and the four dimensions}
\label{tab:Alisaari:1}
\end{table}

On average, the pre-service teachers’ knowledge of the role of languages in teaching was relatively high. As not all participants answered all the questions, the total score for the entire survey could only be calculated for 51 participants. When looking at the four dimensions, the participants’ knowledge related to the use of L1 to support learning was at a higher level than for other dimensions. This result differs from previous findings from Finland that have indicated relatively low awareness of the importance of L1 for learning (\citealt{AlisaariEtAl2019,HeikkolaEtAl2022,Repo2020}). However, these previous studies were conducted mainly among in-service teachers. The reason behind the more asset-oriented attitudes toward L1 might be related to the paradigm shift that has recently taken place in different societies and is leading to higher levels of awareness of the importance of L1 for learning (\citealt{AroninSingleton2018}). This is also reflected in the curriculum for basic education in Finland (\citealt{Finnish_national_agency_of_education2014}) which has potentially had an influence on teacher education, and thus on the participants of this study. However, a closer look at the pre-service teachers’ attitudes toward L1 use reveals some contradictory attitudes (see below).

Qualitative analysis of the responses to individual items revealed great variation. For example, to investigate understandings of developing language skills for learning, the participants were asked to list what they believed to be the reasons for linguistic difficulties in a mathematical task. Only 27\% of the participants clearly analysed the difficulties, naming abstract concepts, grammatical knowledge, and vocabulary knowledge as possible reasons (Example 1). However, 60\% of the participants were unable to identify specific reasons for the vocabulary or grammar being difficult (Example 2), and 14\% gave insufficient responses (Example 3).\footnote{Examples from the responses were translated from Finnish into English by the authors.}

\ea
The first sentence is very long, and it is difficult to find the main points. Furthermore, the words are not used in their basic forms ("kepiltä," from the pole; "kepille," to the pole): [in Finnish,] you can only tell the difference by the locative suffixes.

\ex Long sentences in which it is difficult to find the main point. 

\ex In my opinion, the sentence order is quite clear, and the sentences are short. In the assignment, there are abbreviations (m, cm), which may not be familiar concepts to all students. 
\z

In addition, only 28\% of the participants recognised the connection between mathematical problem solving and language skills, while 41\% argued that mathematical problems are usually presented in a linguistic format and thus require appropriate interpretation of the language used. However, 31\% did not see a connection between language and problem-solving skills. Therefore, it seems that one-third of the pre-service teachers investigated in this study need to develop their awareness of LRT, as understanding the intertwined nature of language, thinking, and all learning is crucial for teachers (\citealt{Alisaari2020_Apples,CumminsEarly2015,LucasVillegas2013,Vygotsky1986}).

\begin{sloppypar}
The participants’ degree of knowledge regarding language, grammar, and semiotic symbols was concerning. The qualitative analysis revealed gaps in basic grammatical knowledge. When the pre-service teachers were asked whether there were comparative or superlative forms of adjectives (e.g., \textit{faster, fastest}) in the text, one fifth of the participants found such forms even though they were not present in the assignment. Furthermore, as many as 61\% incorrectly identified possessive suffixes in the assignment, mainly identifying genitive case suffixes as possessive suffixes (e.g., \textit{hänen} `hers' instead of \textit{ikänsä} `her age'). When the participants were asked to give examples of different grammatical forms, the possessive suffix was incorrectly produced by 62\% of the participants. Past tense was correctly produced by only 37\% and present tense by 44\% of the participants.
\end{sloppypar}

This finding is in line with previous Finnish studies indicating that pre-service teachers’ language knowledge is relatively weak (\citealt{TainioMarjokorpi2014,TainioRoutarinne2012}). At the time of data collection, the participants had completed all the Finnish language and grammar courses of their teacher studies, as they will teach Finnish after graduating. Thus, this result raises concerns about future teachers’ abilities to integrate language and content teaching.

When examining what was reported regarding how to provide linguistic support, 67\% of the participants mentioned more than one way to provide linguistic support (Example 4), 25\% mentioned only one means of linguistic support (Example 5), and 8\% were unable to provide any examples of linguistic support (Example 6). 

\ea Visual support when solving the assignment. I would go through the terms, for example, what does three times total age mean? 
\ex I would draw a picture of the situation on the board somehow. I would illustrate. 
\ex  It should be clear to the students where to begin solving the assignment, using arithmetic sequence. 
\z

The following types of support were mentioned: visual support (65\%), elaborating on vocabulary (46\%) and structures (26\%), breaking the sentence into fragments (21\%), additional oral instructions (17\%), modeling the mathematical procedures (7\%), using plain Finnish (7\%), and peer support (1\%). This is in line with previous research showing that Finnish teachers seem to be competent in supporting learning by using visual aids, while using more linguistically oriented support is lacking (\citealt{HeikkolaEtAl2022}). More multifaceted support for learners would enable them to better understand the content (\citealt{Gibbons2014}), which all students should have equal opportunities to access (\citealt{ComminsMiramontes2006}).

When asked to reflect on the students’ needs for developing academic competence specifically for producing mathematical texts in the future, 44\% named either supporting the development from everyday language to academic language or elaborating on the structure of a word problem and guiding the student. For example, one pre-service teacher wrote: “You have to choose your words wisely” when describing how to scaffold the learners’ understanding. In addition, 39\% of the participants recognised the importance of subject-specific language development but gave fewer specific responses. However, 17\% were not able to provide adequate responses. These results indicate that a notable number of the pre-service teachers in this study need more focused training in linguistically responsive pedagogy, especially in how language is essential to subject learning, thinking, and expressing ideas (e.g. \citealt{CumminsEarly2015,ShanahanShanahan2008}). For example, in teaching mathematics, it is crucial that a teacher know how to actively support the development of subject-specific language and create opportunities for students to participate in cognitively challenging discussions (e.g., mathematical problem solving) as well as to understand instructions (\citealt{AhlholmPortaankorva-Koivisto2018,CarlsonEtAl2018,JoutsenlahtiTossavainen2018}).

When looking at the participants’ understanding of the use of L1 as a learning tool, somewhat contradictory results were found. Half of the pre-service primary school teachers reported that they would limit the use of L1s during lessons to ensure that MLSs develop their Finnish. Furthermore, almost one-third of the participants reported allowing the use of L1s only if someone else, mainly the teacher, knew the language. Although half of the participants reported that they would restrict the use of L1s in the classroom to support Finnish language learning, 97\% would nevertheless encourage the use of L1s to promote content learning. These somewhat contradictory findings resonate with earlier research from Finland, in which teachers’ positive stances toward multilingualism and the use of L1 in general often did not actualise in their reported classroom practices (\citealt{AlisaariEtAl2019}). This may stem from the practical difficulty of dealing with a range of languages in class that teachers do not speak or understand.

Even though the participants gave contradictory responses regarding L1 use in the classroom, they were able to name reasons for the importance of L1s being present in the classroom: 85\% of the participants provided more than one reason for advocating the use of L1 (for example, appreciating linguistic diversity, strengthening student identities, supporting learning outcomes, or highlighting language awareness). 15\% responded more vaguely that L1 use might help students understand a lesson’s content. Importantly, only 2\% of the participants would allow the use of L1 only at recess. Thus, although the participants were unwilling to encourage L1 use in their classrooms, they acknowledged its importance for students’ learning. This indicates positive attitudes that could be used as grounds for pedagogical practices that consider MLSs’ L1s as learning resources. However, more work needs to be done to actualise L1 use in learning according to the requirements of the Finnish core curriculum for basic education (\citealt{Finnish_national_agency_of_education2014}) and for the multilingual stage to actualise in Finnish classrooms (see also \citealt{AroninSingleton2018}).

\subsection{Levels of competency}\label{sec:alisaari:4.2}

In this section, we present the frequencies of the participants’ responses as coded into the three different categories of competency (see \tabref{tab:Alisaari:2}).


\begin{table}
\begin{tabularx}{\textwidth}{X cccc}
\lsptoprule
          &     & \multicolumn{3}{c}{\%}\\\cmidrule(lr){3-5}
          &     &          & Advanced   & \\
Dimension & $n$ & {Novice} & {beginner} & {Competent}\\\midrule
Developing language skills for learning     & 67 & 1.5 & 64.2 & 34.3\\
Knowledge of language, grammar, and symbols & 70 & 0.0 & 51.4 & 48.6\\
Providing linguistic support                & 75 & 4.0 & 24.0 & 72.0\\
L1 as a tool for learning                   & 81 & 2.5 & 27.2 & 70.4\\
Total score                                 & 51 & 0.0 & 56.9 & 43.1\\
\lspbottomrule
\end{tabularx}
\caption{Percentages of participants in the three competency groups: novice (0–1.49), advanced beginner (1.5–2.99), and competent (3–4)}
\label{tab:Alisaari:2}
\end{table}

\hspace*{-3.9pt}Although the detailed investigation of each individual dimension revealed some shortcomings (\sectref{sec:alisaari:4.1}), when looking at the dimensions as whole, there are noticeably few novices. This indicates that a broader investigation provides a different and more positive perspective on the competence of the pre-service teachers.  Thus, at the university where the study took place, teacher education seems to provide pre-service teachers with at least an elementary understanding of academic language development and ways to support it. When looking at the total score for the entire survey, 56.9\% of the participants were advanced beginners and 43.1\% were categorised as competent and therefore could be expected to be prepared to support language learning in the classroom. In the dimensions of providing linguistic support and L1 as a tool for learning, over 70\% of the participants were categorised as competent. However, it has to be kept in mind that this result is based only on their self-reports. Although single items seem to indicate knowledge gaps (see section 4.1), since there were only few novice-level competencies, as a whole, teacher education at this particular university appears to promote understanding of the importance of L1 use and providing linguistic support in general. Our results indicate that in addition to encouraging pre-service teachers to better understand the importance of language in learning, thinking, and communicating, it is also of crucial importance to deepen their knowledge of language, including grammar. Half of the participants had significant gaps in their grammatical knowledge of the language that they will use for instruction in the future.

\section{Conclusions}\label{sec:alisaari:5} %5. /

The aim of this study was to examine Finnish pre-service primary school teachers’ understandings of academic language development and their reported preparedness to support language learning in the classroom when teaching MLSs in Finnish basic education. Since the data were collected from all the pre-service teachers in their third year at one teacher education institution only, the results cannot be generalised to other contexts (Finnish or international).

Based on this study, the Finnish pre-service teachers’ overall understandings of academic language development and their reported preparedness to support language learning in the classroom vary. However, the results reflect the overall tendency seen in recent studies conducted in Finland (\citealt{AlisaariEtAl2019,Alisaari2020_Apples,Harju-AuttiEtAl2022,HeikkolaEtAl2022,chapters/8_heikkola}). Although there is a relatively good understanding of and support for academic language development, there are some gaps in awareness of how to provide linguistic support for MLSs (cf. \citealt{Harju-AuttiSinkkonen2020}) and how to use L1s as learning resources (\citealt{AlisaariEtAl2019}). However, the pre-service teachers in this study seemed to have a somewhat better understanding of the importance of L1 use and providing linguistic support than of the importance of developing language skills for learning or knowledge of language and grammar (cf. \citealt{chapters/2_Gunnthorsdottira,chapters/5_ostergaard}).

One-third of the pre-service teachers did not see a connection between language and mathematical problem-solving skills, which leads us to conclude that their LRT awareness is still developing. The same conclusion could be drawn about the linguistic support, mainly visual or vocabulary- and structure-related, that the pre-service teachers said they would offer during a mathematics lesson. This seems to be common in Finnish pedagogical practices (e.g. \citealt{HeikkolaEtAl2022}). Moreover, even though pre-service teachers in this study had high competence in awareness of the importance of students’ L1s in learning, many of them reported that they would nevertheless restrict students’ L1 use, indicating that monolingual ideologies exist among pre-service as well as in-service teachers (cf. \citealt{AlisaariEtAl2019}). Additionally, based on our results, pre-service teachers need support in how to scaffold their students in developing subject-specific language. This would provide more equal opportunities for all learners, especially MLSs, who may need extra practice with vocabulary, sentence structures, and meaning (\citealt{ComminsMiramontes2006}).

While not a specific focus of the study, the fact that there were considerable gaps in the pre-service teachers’ grammar knowledge after they had taken all the Finnish grammatical courses of their teacher education is worrying. Grammatical knowledge is needed to enable future teachers to perceive language as a whole and support students’ learning (\citealt{Myhill2000}). In addition, classroom teachers have a responsibility to teach students the basics of grammar, especially in the upper grades of primary school (\citealt{Tainio2020}). Therefore, they must have a strong knowledge of the language, including the basics of grammar. Knowledge of language can be seen as an important part of pedagogical competence; teachers’ weak grammatical skills are likely to affect their ability to apply language knowledge in practice (\citealt{Rattya2013}).

The findings also suggest that the LRT requirements stipulated in the Finnish core curriculum for basic education are still difficult to achieve, especially considering that the third-year pre-service teachers in this study seemed to lack essential Finnish language skills. This would suggest that the current curriculum for teacher education at the university where this study took place does not provide sufficient knowledge regarding the Finnish language and grammar or language in general. Thus, offering pre-service teachers comprehensive information on language as well as research-based and practice-oriented instruction to deepen their knowledge of how to support academic language development in all subjects is recommended.

The instrument used in this study gave interesting insights into pre-service teachers’ understandings of academic language development and preparedness to teach MLSs. However, the instrument would benefit from refinement, as this pilot study illustrates. For example, following the original guidelines for the analysis (\citealt{CarlsonEtAl2018}) sometimes resulted in giving full scores even though the responses included negative attitudes or clear misunderstandings of issues related to language development. Even though these kinds of responses were rare, they highlighted these issues in the analysis process. This study was intended to be a pilot for further development and use of the research instrument; thus, both the questions and the guidelines for analysis would benefit from being rewritten and adjusted to better suit the context in which they are used.

Since immigration is rising globally, there is an increased need for awareness of the role of language in learning and teaching. Thus, the results of this study can be of interest to teacher educators and researchers in contexts beyond Finland, stimulating discussion and examination of similar issues in those contexts. In addition, the study makes a methodological contribution; the instrument used in this study was adapted from use in a German context to use in a Finnish context. The results indicate that this adaptation was fruitful, and thus, the research instrument could be used in different multilingual educational contexts in other parts of the world, although it must be carefully adapted to country-specific educational policies.

\appendixsection{Items of original assignments and their modifications}\label{appendix:alisaari}

Items were translated into English by the authors of the original research instrument and the modified research instrument.

\appendixsubsection{Original assignment}

\begin{enumerate}[label=\Alph*.]
\item  Auction\\
\emph{Stimulus}: The following test item comes up on a standardised test in your 7th grade math class.

\begin{itemize}
\item A dealer paid \$10,000 for a boat at an auction. At the dealership, a salesperson sold the boat for 30\% more than the auction price. The salesperson received a commission of 25\% of the difference between the auction price and the dealership price. 

\item What was the salesperson’s commission? 

\begin{enumerate}
\item \$750
\item \$1,750
\item \$3,250
\item \$5,500
\end{enumerate}

\item Why does the contextual vocabulary (e.g. \emph{dealer}, \emph{auction}) cause more difficulties for multilingual learners than the subject-specific vocabulary (e.g. \emph{difference}, \emph{more than})? 

\item Name at least one other difficult term and justify your answer.

\item To what extent does this test item pose additional challenges for multilingual learners at the sentence and text level? 

\item Name at least one feature and support it with evidence from the text.
\end{itemize}

\item You want to use the following task for the next lesson: 

\begin{itemize}
\item Leonardo da Vinci (1452--1519) describes the ideal measures of the human body as follows:

\begin{quote}
The open hand from the wrist to the tip of the middle finger is a tenth of   the entire height. The length of the  foot is one-sixth of the height of the body. The head from the chin to the crown is one-eighth. The distance  from the chin to the nostrils and     from the lowest roots of the hair to    the eyebrows is always the same and equals, just as the ear, a third of the face.
\end{quote}


\item Measure your height. What would the measurements of your hands, feet,  face, and ears be according to da Vinci?

\item Which linguistic elements can you identify in the task? Mark the right answer with a cross in each line and give an example if necessary.

\begin{enumerate}
\item Demonstrative pronouns
\item Prepositional verbs
\item Prepositions
\item Past participles
\item Imperatives
\item Compounds
\end{enumerate}

\end{itemize}
\end{enumerate}

\appendixsubsection{Modified assignment}

\begin{enumerate}[label=\Alph*.]
\item In the fifth-grade mathematics textbook (\emph{Tuhattaituri} 5a, \emph{Otava}) is the following assignment:

\begin{itemize}
\item In a dog agility competition, the distance between two poles on the straight weave\hyp pole path is always 60cm. The distance from the first pole to the last pole is 6m 60cm. How many poles are there on the straight weave-pole path in total?

\item Why does the contextual vocabulary (e.g., \emph{agility competition}) pose more difficulties for multilingual learners than mathematics subject-specific vocabulary (e.g., \emph{total})? Name at least one other difficult word and justify your answer.
\end{itemize}

\item You want to use the following assignment in your lesson:

\begin{itemize}
\item In three years, Sonja’s grandfather will be three times as old as Sonja was last year. When adding Sonja’s current age to her grandfather’s current age, the total is 68. How old are each of them now?

\item What kind of grammatical factors can you identify in the task? Select either “yes” or “no” for each listed item. If you select “yes,” give an example.

\begin{enumerate}
\item Past tense
\item Comparative/superlative
\item Imperative
\item Present tense
\item Participial phrase 
\item Possessive suffix
\end{enumerate}
\end{itemize}
\end{enumerate}

\appendixsection{Examples of the scoring guidelines concerning the responses for some of the survey items and their scoring based on how correct the answers would be}\label{appendix:alisaari2}

The questions are previously published in \citealt{AlisaariEtAl2023}.

% \begin{table}
% \footnotesize
% \begin{tabularx}{\textwidth}{XQX}
% \lsptoprule
% Survey questions & Scoring & Responses for specific scores\\
% \midrule
% \textbf{Question 1: Contextual and mathematical vocabulary} & & \\
% \midrule
% The following assignment is in a fifth-grade mathematics textbook (\textit{Tuhattaituri 5a}): “In a dog agility competition, the distance between two poles on the straight weave-pole* trajectory is always 60 cm. The distance from the first pole to the last pole is 6 m 60 cm. How many poles are on the straight weave-pole trajectory in total?” \newline A) Why does the contextual vocabulary (e.g., \textit{agility}, \textit{competition}) pose more difficulties for multilingual learners than mathematics subject-specific vocabulary (e.g., \textit{total})? Name at least one other difficult word and justify your answer. (0–2 points) & 2 points \newline 1 point \newline 0 points & Mention of at least one difficult word AND explanation of why contextual vocabulary might be more challenging than mathematical vocabulary (e.g., the notion that life experience influences the development of contextual vocabulary) \newline Mention of one difficult word OR an example of why contextual vocabulary is more challenging than mathematical vocabulary \newline No mention of a difficult word OR suitable line of reasoning regarding vocabulary\\
% \midrule
% B) What other difficulties might the assignment cause for a multilingual learner at the sentence and text levels? (0–2 points) & 2 points \newline 1 point \newline 0 points & Mention of concrete, challenging words/grammatical features OR discussing difficulties on an abstract level \newline No mention of concrete, challenging words/structures OR of the task being too long\newline No mention of challenging words/structures OR an inappropriate response\\
% \midrule
% \end{tabularx}
% \end{table}
% 
% \begin{table}
% \footnotesize
% \begin{tabularx}{\textwidth}{XQX}
% \midrule
% \textbf{Question 2: Linguistic support} & &\\
% \midrule
% A) What kind of linguistic support would you offer multilingual learners either before or during the assignment to make the assignment easier to understand? Name at least two. (0–2 points) & 2 points \newline 1 point \newline 0 points & Mention of at least two of the following reasons: \begin{itemize} \item Ensure students understand challenging words/phrases/concepts 
% \item Break the sentence into fragments
% \item Use sentence/summary frames
% \item Provide visuals that correspond with the word problem
% \item Rewrite the question to be less linguistically complex/give students the formula/simplify the vocabulary/use imperatives
% \item Strategy (e.g., use translators or Google) \end{itemize} Mention of at least one of the reasons listed above \newline No mention of the reasons listed above \\
% \midrule
% B) Different types of linguistic support listed by the pre-service teachers based on their responses to question 2A. & & The first three listed by the participants were included in the analysis.\\
% \midrule
% \textbf{Question 3: Link between mathematical and linguistic skills} & & \\
% \midrule
% What is the link between mathematical skills and linguistic skills? Explain. (0–2 points) & 2 points \newline 1 point \newline 0 points & Understanding and explaining the link between mathematical and linguistic skills\newline Naming the link between mathematical and linguistic skills \newline Vague response not naming the link between mathematical and linguistic skills\\
% \midrule
% \end{tabularx}
% \end{table}
% 
% \begin{table}
% \footnotesize
% \begin{tabularx}{\textwidth}{XQX}
% \midrule
% \textbf{Question 4: A) Justify your response to the following statements:} & & \\
% \midrule
% \begin{enumerate}[label=\Alph*),nosep,leftmargin=*]
% \item I would restrict conversations in first languages during lessons to ensure that my multilingual learners develop their basic and academic language skills in Finnish. (0–2 points)
% \item First languages should be used in classrooms, but only when the teacher also knows the language(s). (0–2 points)
% \item I would encourage multilingual learners to use their first languages to understand content in different subjects more easily. (0–2 points)
% \item Name reasons why it may be good for multiple languages to be present in different subject classes. (0–2 points) \end{enumerate} & 2 points \newline 1 point \newline 0 points & Mention of at least two of the following reasons: \begin{itemize} \item L1s are appreciated
% \item Learners’ identities as multilingual individuals are strengthened
% \item Some learners have more courage to contribute during a lesson (e.g., group work) if they are can use their L1
% \item Learners can understand the lessons’ content (e.g., a task was understood)
% \item Learners’ L1s should be used as a resource 
% \item Language contrasts can highlight linguistic structures
% \item Highlights awareness of languages 
% \item Helps build community in the classroom 
% \item Learning is encoded in L1
% \item Encourage maintenance of L1 
% \item Building communities through L1 \end{itemize} Mention at least one of the reasons listed above \newline No mention of the reasons listed above \\
% \midrule
% \textbf{B) Importance of L1} & & \\
% \midrule
% \begin{enumerate}[label=\Alph*),nosep,leftmargin=*] \item I would restrict conversations in first languages during lessons to ensure that multilingual learners develop their basic and academic language skills in Finnish. (0–1 points) 
% \item First languages should be used in classrooms, but only when the teacher also knows the language(s). (0–1 points)
% \item I would encourage multilingual learners to use their first languages to understand content in different subjects more easily. (0–1 points) \end{enumerate} & 1 point \newline 0 points & Understanding the importance of multilingual learners’ L1s \newline Not understanding the importance of multilingual learners’ L1s\\
% \lspbottomrule
% \end{tabularx}
% \end{table}
% \clearpage

\appendixsubsection*{Question 1: Contextual and mathematical vocabulary}

The following assignment is in a fifth-grade mathematics textbook (\textit{Tuhattaituri 5a}):

\begin{itemize}
\item[] In a dog agility competition, the distance between two poles on the straight weave-pole\footnote{A weave-pole is an agility obstacle consisting of a series of poles, and it is traditionally used in dog agility training and competitions.} trajectory is always 60 cm. The distance from the first pole to the last pole is 6 m 60 cm. How many poles are on the straight weave-pole trajectory in total?
\end{itemize}
\begin{itemize}
	\item[A)] Why does the contextual vocabulary (e.g., \textit{agility}, \textit{competition}) pose more difficulties for multilingual learners than mathematics subject-specific vocabulary (e.g., \textit{total})? Name at least one other difficult word and justify your answer. (0–2 points)
		\begin{itemize}
			\item Mention of at least one difficult word AND explanation of why contextual vocabulary might be more challenging than mathematical vocabulary (e.g., the notion that life experience influences the development of contextual vocabulary) (2 points)
			\item Mention of one difficult word OR an example of why contextual vocabulary is more challenging than mathematical vocabulary (1 point)
			\item No mention of a difficult word OR suitable line of reasoning regarding vocabulary (0 points)
		\end{itemize} 
	\item[B)] What other difficulties might the assignment cause for a multilingual learner at the sentence and text levels? (0–2 points)
		\begin{itemize}
			\item Mention of concrete, challenging words/grammatical features OR discussing difficulties on an abstract level (2 points)
			\item No mention of concrete, challenging words/structures OR of the task being too long (1 point)
			\item No mention of challenging words/structures OR an inappropriate response (0 points)
		\end{itemize}
\end{itemize}
	
\appendixsubsection*{Question 2: Linguistic support}

\begin{itemize}
	\item[A)] What kind of linguistic support would you offer multilingual learners either before or during the assignment to make the assignment easier to understand? Name at least two. (0–2 points)
		\begin{itemize} 
			\item Mention of at least two of the following reasons (2 points): 
				\begin{itemize}
					\item Ensure students understand challenging words/phrases/con\hyp cepts
					\item Break the sentence into fragments
					\item Use sentence/summary frames
					\item Provide visuals that correspond with the word problem
					\item Rewrite the question to be less linguistically complex/give students the formula/simplify the vocabulary/use imperatives
					\item Strategy (e.g., use translators or Google)
				\end{itemize}
			\item Mention of at least one of the reasons listed above (1 point)
			\item No mention of the reasons listed above (0 points)
		\end{itemize}
	\item[B)] Different types of linguistic support listed by the pre-service teachers based on their responses to question 2A.
		\begin{itemize}
			\item The first three listed by the participants were included in the analysis.
		\end{itemize}
\end{itemize}

\appendixsubsection*{Question 3: Link between mathematical and linguistic skills}

What is the link between mathematical skills and linguistic skills? Explain. (0–2 points) 
\begin{itemize}
\item Understanding and explaining the link between mathematical and linguistic skills (2 points)
\item Naming the link between mathematical and linguistic skills  (1 point)
\item Vague response not naming the link between mathematical and linguistic skills (0 points)
\end{itemize}


\appendixsubsection*{Question 4}
\appendixsubsubsection*{A) Justify your response to the following statements:}

\begin{itemize}
	\item[A)] I would restrict conversations in first languages during lessons to ensure that my multilingual learners develop their basic and academic language skills in Finnish. (0–2 points)
	\item[B)] First languages should be used in classrooms, but only when the teacher also knows the language(s). (0–2 points)
	\item[C)] I would encourage multilingual learners to use their first languages to understand content in different subjects more easily. (0–2 points)
	\item[D)] Name reasons why it may be good for multiple languages to be present in different subject classes. (0–2 points)
\end{itemize}

\begin{itemize}
  \item Mention of at least two of the following reasons (2 points):            
    \begin{itemize}
		\item L1s are appreciated
		\item Learners’ identities as multilingual individuals are strengthened
		\item Some learners have more courage to contribute during a lesson (e.g., group work) if they are can use their L1
		\item Learners can understand the lessons’ content (e.g., a task was understood)
		\item Learners’ L1s should be used as a resource 
		\item Language contrasts can highlight linguistic structures			
		\item Highlights awareness of languages 
		\item Helps build community in the classroom 
		\item Learning is encoded in L1
		\item Encourage maintenance of L1 
		\item Building communities through L1
	\end{itemize}
  \item Mention at least one of the reasons listed above (1 point) 
  \item No mention of the reasons listed above (0 points)
\end{itemize}

\appendixsubsubsection*{B) Importance of L1}
		
\begin{itemize}
	\item[A)] I would restrict conversations in first languages during lessons to ensure that multilingual learners develop their basic and academic language skills in Finnish. (0–1 points)
	\item[B)] First languages should be used in classrooms, but only when the teacher also knows the language(s). (0–1 points)
	\item[C)] I would encourage multilingual learners to use their first languages to understand content in different subjects more easily. (0–1 points)
\end{itemize}

\begin{itemize}
\item Understanding the importance of multilingual learners’ L1s (1 point)
\item Not understanding the importance of multilingual learners’ L1s (0 points)
\end{itemize}
\printbibliography[heading=subbibliography,notkeyword=this]
\end{document}
