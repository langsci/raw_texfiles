\title{Crossroads between Contrastive Linguistics, Translation Studies and Machine Translation}
\subtitle{TC3 II}
\author{Oliver Czulo  \lastand Silvia Hansen-Schirra} 
\renewcommand{\lsSeriesNumber}{4}  
\renewcommand{\lsCoverTitleFont}[1]{\sffamily\addfontfeatures{Scale=MatchUppercase}\fontsize{40pt}{12.75mm}\selectfont #1}

\renewcommand{\lsISBNdigital}{978-3-946234-26-5}                     
\renewcommand{\lsISBNhardcover}{978-3-946234-98-2}                   
\renewcommand{\lsID}{102}                     
\renewcommand{\lsSeries}{tmnlp} % use lowercase acronym, e.g. sidl, eotms, tgdi
\renewcommand{\lsURL}{http://langsci-press.org/catalog/book/102} 
\typesetter{Sebastian Nordhoff, Iana Stefanova, Florian Stuhlmann}
\BackBody{Contrastive Linguistics (CL), Translation Studies (TS) and Machine Translation (MT) have common grounds: They all work at the crossroad where two or more languages meet. Despite their inherent relatedness, methodological exchange between the three disciplines is rare. This special issue touches upon areas where the three fields converge. It results directly from a workshop at the 2011 German Association for Language Technology and Computational Linguistics (GSCL) conference in Hamburg where researchers from the three fields presented and discussed their interdisciplinary work.
While the studies contained in this volume draw from a wide variety of objectives and methods, and various areas of overlaps between CL, TS and MT are addressed, the volume is by no means exhaustive with regard to this topic. Further cross-fertilisation is not only desirable, but almost mandatory in order to tackle future tasks and endeavours.}
\proofreader{ 
Felix Kopecky,
Jean Nitzke,
Valeria Quochi
}
\BookDOI{10.5281/zenodo.1019701}
