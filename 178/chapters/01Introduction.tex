% !TEX root = ../main.tex
\chapter{Introduction}\label{Introduction}

%%% 最後に数字合わせをやること。
%%% それぞれ微妙に違うデータを元に集計している
%%% Principle -> もっと良い名前

%%% グロス整理

%%% Abstract, Acknowledgements

%%% Justification
%%% (1) 問題の設定 (問い)
%%% (2) 問題設定の背景(なぜ問うのか)
%%%     問題の議論の歴史的な流れ
%%% 
%%% (3) 方法論
%%%     理論(談話+認知)
%%% (4) アウトライン



%%----------------------------------------------------
\section{Aims of the study}\label{IntroAims}
\largerpage
The goal of this study is twofold.
First,
I will investigate the relationships between \isi{information structure} and linguistic forms in spoken Japanese.
Second,
I will propose a method to investigate
this kind of relations in any language by using corpora. %in any language.%the relations between \isi{information structure} and linguistic forms in any language using corpora.

Speakers of Japanese, like speakers of many other languages,
infer other people's knowledge and
express their assumptions about it using various linguistic and non-linguistic tools.
Consider a conversation between three people, A, B, and C,
from \ci{the Chiba three-party conversation corpus} \cite{Den_2007_SAC}.
In \Next[-A1],
one of the participants, A, starts talking about \ci{ano koohii-meekaa} `that coffee machine'.
In B2 to B4, B, explains why A started to talk about it;
it is related to the previous \isi{topic} (too many people gathered in a small room).
C just adds a weak backchannel response in C5.
In A6--A7,
A asks C whether she knows about the new coffee machine that arrived in building E.
In C8--C11,
C answers to A that
she knows about it but has never tried it.%
	\footnote{
	Some of the utterances were omitted for the sake of simplicity.
	}
%
\ex.
 \ag.[A1:] ano \EM{koohii-meekaa} sugoi-yo-ne \\
      that coffee-maker great-\ab{fp}-\ab{fp} \\
      `That coffee machine is excellent, isn't it?'
 \bg.[B2:] \EM{koohii-meekaa}-o mi-tai \\
          coffee-maker-\ab{acc} see-want \\
          `(I) wanna see the coffee machine.'
 \bg.[B3:] tukat-teru-no-o mi-tai-tte iu-no-to \\
           use-\ab{pfv}-\ab{nmlz}-\ab{acc} see-want-\ab{quot} say-\ab{nmlz}-and \\
           `(They) want to see (us) use (the coffee machine), and'
 \bg.[B4:] koohii nom-e-nai san-nin-gumi-mo ita-kara otya non-de-ta \\
           coffee drink-\ab{cap}-\ab{neg} three-\ab{cl}.person-group-also exist-because tea drink-\ab{prog}-\ab{past} \\
           `since there were also three people who cannot drink coffee, they drank tea.'
 \bg.[C5:] un \\
           uh-huh \\
           `Uh-huh.'
 \bg.[A6:] [to C] ii-too-no sit-teru: \\
           {} {} E-building-\ab{gen} know-\ab{prog} \\
           %`(Do you) know (that) in Building E?'
           `(Do you) know about (that) in Building E?'
 \bg.[A7:] ano \EM{koohii-meekaa} \\
           that coffee-maker \\
           `That coffee machine.'
 \bg.[C8:] un un un un un \\
           yeah yeah yeah yeah yeah \\
           `Yeah yeah!'
 \bg.[A9:] tukat-ta koto aru \\
        use-\ab{past} thing exist \\
        `Have (you) used (it)?'
 \bg.[C10:] atasi-sa: \EM{are}-ga-ne: ki-te-kara-ne: \\
         \ab{1}\ab{sg}-\ab{fp} that-\ab{nom}-\ab{fp} come-and-since-\ab{fp} \\
         `Since it arrived, I'
 \bg.[C11:] moo hotondo sagyoo sun-no-ga nooha-beya-ni nat-tyat-ta-kara-ne: \\
           already almost work do-\ab{nmlz}-\ab{nom} brainwave-room-\ab{dat} become-\ab{pfv}-\ab{past}-because-\ab{fp} \\
           `almost always work in the brainwave room, so...'
          \src{chiba0932: 172.39-191.46}
%172.3866 174.0174 A: あのコーヒーメーカーすごいよね
%173.9564 175.1409 B: コーヒーメーカーを見たい
%174.0174 174.5300 A: メーカー
%174.6300 175.0125 A: <笑>
%176.5177 180.4778 B: 使ってるのを見たいっていうのと(0.446)コーヒー飲めない組三人もいたからお茶飲んでた
%178.0445 178.4060 C: (I_うん)
%178.0693 178.5800 A: そうそう
%180.4794 180.9095 A: そうそう
%181.2900 181.4900 B: <笑>
%181.4871 182.3664 A: E棟の知ってる:
%181.5620 182.1821 C: (D_パエー)
%182.3664 183.3625 A: あのコーヒーメーカー
%183.6570 184.9120 C: (I_うん)(I_うん)(I_うん)(I_うん)(I_うん)
%184.1662 185.1013 B: 純喫茶(R_吉田谷)
%184.8765 186.0390 A: (T_ツコ-|使っ)(.)使ったことある
%186.2325 191.4593 C: あたしさ:あれがね:(.)来てからね:もうほとんど作業すんのが脳波部屋になっちゃったからね:

From this short conversation,
observers (namely, we) can infer that
A in A1 assumed that the other participants already knew about the great coffee machine that was introduced in their lab.
One can also infer that B in B2--B4 already knew about the coffee machine.
In A6--A7,
A appears to think that C might not know about the coffee machine.
However, C in C8 explicitly denies A's concern. 

Why is it possible for us to infer the speakers' assumptions about the knowledge of other participants?
In this case, linguistic expressions such as
\ci{ano (koohii meekaa)} `that (coffee machine)' in A1 and
\ci{sit-teru:} `(do you) know...?' in A6
indicate A's assumptions about the other participants' knowledge.

This study investigates more subtle linguistic expressions than these determiners in spoken Japanese, namely
particles, \isi{word order}, and intonation.
As an example, let us discuss
the distinction between the particles \ci{ga} and \ci{wa},
that has been discussed for a long time in the literature on Japanese linguistics.
Examples \Next[a], containing the particle \ci{ga}, and \Next[b], containing the particle \ci{wa},
express the same proposition (`A/the dog is running'),
where \isi{definiteness} is not explicit in the original Japanese sentences.
The expression \ci{inu} `dog' followed by \ci{ga} in \Next[a]
can be interpreted as either definite or \isi{indefinite},
while the same expression followed by \ci{wa} in \Next[b]
can only be interpreted as definite:
from \Next[b] we can infer that the speaker assumes that the \isi{hearer} already knows about the dog.
%
\ex.
 \ag. inu-\EM{ga} hasit-teiru \\
      dog-\ab{nom} run-\ab{prog} \\
      `A/the dog is running.'
 \bg. inu-\EM{wa} hasit-teiru \\
      dog-\ab{top} run-\ab{prog} \\
      `The dog is running.'
 \hfill{(Constructed)}

As will be discussed in Chapter \ref{Particles},
however,
it is not the case that
the NP coded by \ci{wa} is always definite,
nor is it the case that
the NP coded by \ci{ga} is always \isi{indefinite}.
What determines the usage of the particles?
Moreover,
particle choice interacts with other factors such as
\isi{word order} and intonation.
This study investigates how \isi{information structure} affects
particle choice, \isi{word order}, and intonation
employing a corpus of spoken Japanese.

%%----------------------------------------------------
\section{Background}\label{IntroBackground}

Information structure in this study comprises
``the utterance-internal structural and semantic properties reflecting the relation of an \isi{utterance} to the \isi{discourse} context,
in terms of the \isi{discourse} status of its content,
the actual and attributed attentional status of the \isi{discourse} participants,
and the participants' prior and changing attitudes
(knowledge, beliefs, intentions, expectations, etc.)''
\cite[250]{kruijff-korbayovasteedman03}.
I assume that
\isi{information structure} is a subordinate part of \isi{discourse} structure,
which is a clause-level unit and does not allow recursivity.
Also, I assume that \isi{information structure} should be analyzed at the surface level rather than at the level of underlying semantics (or logical form).

Studies on \isi{information structure} can be brought back to two sources
(see \citeA{kruijff-korbayovasteedman03} for a useful survey). %There are two kinds of roots of studies on \isi{information structure}
%(see \citeA{kruijff-korbayovasteedman03} for a useful survey).
One originates in the studies on definite and \isi{indefinite} descriptions by \citeA{russell05} and \citeA{strawson50,strawson64}. These studies triggered the discussion on \isi{presupposition} and assertion which are still a matter of debate now. 
%One started from studies on definite and \isi{indefinite} descriptions by \citeA{russell05} and \citeA{strawson50,strawson64}. These studies triggered the discussion on \isi{presupposition} and assertion, which has been at issue until the present time.
In particular, this line of research has influenced contemporary scholars of logic, formal semantics, and generative grammar \cite{chomsky65,jackendoff72,selkirk84,rooth85,rizzi97,erteschik-shir97,erteschik-shir07,buring07,ishihara11,krifkamusan12,endo14}.
The other source originates from the Prague School \cite{mathesius28,mathesius29,sgall67,firbas75}, whose studies have particularly inspired functional linguistics
%The other source originates from the Prague School \cite{mathesius28,mathesius29,sgall67,firbas75}, whose studies have particularly inspired functional linguistics
\cite{bolinger65,halliday67,kuno73,gundel74,chafe76,chafe94,prince81,givon83,tomlin86,lambrecht94,birnerward98,birnerward09}.
Some scholars were influenced by both of these traditions
\cite{vallduvi90,steedman91,vallduvivilkuna98}. 

Almost independently from this European and American tradition of linguistics, Japanese linguistics focused its attention on the so-called \isi{topic} particle \ci{wa} in Japanese, often as opposed to the \isi{case particle} \ci{ga} \cite{matsushita28,yamada36,tokieda50,mikami53,mikami60,onoe81,kinsui95,kikuchi95,noda96,masuoka00,masuoka12}.
%the so-called \isi{topic} particle \ci{wa} in Japanese has gathered the attention of Japanese linguists for a long time,
%often as opposed to the \isi{case particle} \ci{ga} \cite{matsushita28,yamada36,tokieda50,mikami53,mikami60,onoe81,kinsui95,kikuchi95,noda96,masuoka00,masuoka12}.
In addition to its use, the discussion on \ci{wa} also elicited the question on the nature of the subject
because, on the surface, \ci{wa} frequently alternates with \ci{ga},
the so-called subject particle.
See Chapter \ref{Background} for details.
%%%ISSUE ' is not displayed properly in vallduvi90. 

Recently, more studies have investigated the actual production and understanding of language rather than just the acceptability judgements of constructed examples.
Corpus-oriented studies  
\cite[e.g.,][]{calhounetal05,gotzeetal07,chiarcosetal11}
inherit from the two \isi{information structure} traditions:
the logical tradition and the functional one.
Other corpus-oriented studies such as \citeA{hajicovaetal00},
annotating \ili{Czech}, are based on the work of the Prague School.
There are also questionnaires for eliciting expressions related to \isi{information structure} cross-linguistically \cite{skopeteasetal06}. Further, 
\citeA{cowles03} and \citeA{cowlesferreira12} investigate \isi{information structure} mainly by employing psycholinguistic experiments.

I am mostly influenced by the traditions of functional linguistics and corpus linguistics.
Although I tried to include the work of other traditions as much as possible, sometimes readers from other schools might have difficulties understanding my assumptions.
%Although I tried to include knowledge from studies of other traditions as much as possible,
%sometimes my assumptions might be difficult to understand for readers of other traditions.
I assume that usage shapes a language \cite{givon76,comrie83,comrie89,bybeehopper01} and
am interested in how linguistic usage affects its shape. %the shape of a language.
In this study,
I focus on the question of how language usage related to \isi{information structure} affects linguistic form in Japanese.


%%----------------------------------------------------
\section{Methodology}\label{IntroMethodology}

I investigate linguistic forms associated with \isi{information structure} in spoken Japanese mainly by examining spoken corpora.
It is well known that \isi{information structure} phenomena are so subtle
that slight changes in the context can affect the judgement of the sentence in question, meaning that
acceptability judgements from a single person (i.e., the author) are not reliable.
This is the reason why I employ spoken corpora, in which
the speakers produce utterances naturally without concentrating on \isi{information structure} too much like linguists.
Moreover, contexts are available in spoken corpora,
which are crucial for observers to determine the \isi{information structure} of a sentence.
%Corpus investigation is a scientific method to investigate language
%because everybody can test the result using the same methodology.
It is also well known, however,
that \isi{information structure} annotation is very hard.
There are studies on the annotation of \isi{information structure} for various types of corpus and for different languages
%There are studies on annotating \isi{information structure} in various corpora in different languages
\cite{hajicovaetal00,calhounetal05,gotzeetal07,ritzetal08,chiarcosetal11}.
Some use syntactic information to decide the \isi{information structure} of a sentence \cite{hajicovaetal00};
some use intonation \cite{calhounetal05};
others use linguistic tests \cite{gotzeetal07,chiarcosetal11}; but many studies decide on the basis of several features.
For example, in annotating ``\isi{aboutness} \isi{topic}'',
\citeA{gotzeetal07} employ not only tests such as whether the NP in question can be the answer to the question ``let me tell you something about X'',
but also morphological information of the NP
such as referentiality, \isi{definiteness}, genericity, etc.
In the present work, I annotate multiple features of topichood and focushood,
rather than annotating homogeneous ``\isi{topic}'' and ``focus'' categories.
I consider a \isi{topic} to be a cluster of features, comprising ``presupposed, ``evoked, ``definite'', ``specific'', ``\isi{animate}'', etc. I also see \isi{focus} as a cluster of features, comprising ``asserted'', ``brand-new'', ``\isi{indefinite}'', ``non-specific'', ``\isi{inanimate}'', etc.
I assume that \isi{topic} and \isi{focus} typically (frequently) have these features, but that these are not always all necessarily present. There could be infrequent (i.e., atypical) topics that are \isi{indefinite} or \isi{inanimate}, or there could be foci that are definite or \isi{animate}.
%or there could be foci which are definite or \isi{animate}.
%I assume that they typically (frequently) have these features.
%Not all the features are necessarily present in topics or foci;
%there could be infrequent (i.e., atypical) topics which are \isi{indefinite} or \isi{inanimate},
%or there could be foci which are definite or \isi{animate}.
See discussion in Chapter \ref{Framework} for details.

I sometimes employ acceptability judgements and production experiments
to support my argument.
I believe that, in the future,
it will be necessary to test all the hypotheses using multiple methods
for a scientific investigation of language.


%%----------------------------------------------------
\section{Overview}\label{IntroOverview}

I will now outline the chapters of this book.
In Chapter \ref{Background},
I provide an overview of the previous studies on \isi{information structure} across languages.
I also describe the basic features of Japanese and review studies on Japanese related to this study.
In Chapter \ref{Framework},
I outline the framework employed in the study;
the notions of \isi{topic}, focus, and features related to them.
Moreover, I introduce the nature of the corpora,
the annotation procedure, and the methods employed to analyze the results.
The following three chapters analyze linguistic forms found in spoken Japanese.
Chapter \ref{Particles} investigates particles,
Chapter \ref{WordOrder} analyzes \isi{word order}, and
Chapter \ref{Intonation} inquires into intonation.
In Chapter \ref{Discussion},
I summarize the study and discuss its theoretical implications.

















