\documentclass[output=paper,colorlinks,citecolor=brown]{langscibook}
\ChapterDOI{10.5281/zenodo.14282818}
\author{Sally J. Delgado\orcid{}\affiliation{University of Puerto Rico at Cayey}}
%\ORCIDs{}

\title[Language contact in Puerto Rico]
      {Language contact in Puerto Rico: Documenting an emerging variety of English}

\abstract{Spanish and English have a unique contact history in Puerto Rico, where English operates in diglossia with vernacular Spanish for most Puerto Rican bilinguals who live on the island. Yet, few speakers recognize a local variety of English, and scholarship on the speech of Puerto Ricans living on the island is limited. This paper tests the hypothesis that speaker involvement in generating a database of linguistic resources on Puerto Rican English can increase awareness of how the Spanish-dominated context of the island has given rise to a distinct variety of English. A language documentation project aimed to test this hypothesis using eight experimental strategies: six data-collection methodologies that documented spoken and written forms of English, and two surveys that measured language attitudes. This paper presents exploratory analysis on three of these data-collection activities relating to: 1) lexical variation in speech, 2) semantic and structural variation in writing, and 3) attitudes on accent and confidence. Findings indicate that Puerto Rican English still tends to be understood and expressed through a Spanish matrix framework, but speakers increasingly assert ownership over their unique variety of English and use it to express Puerto Rican culture and identity. Speaker participation and the involvement of early-career scholars across all eight data-collection activities indicate that these stakeholders are motivated to document and describe their local variety of English. Furthermore, their participation not only resulted in increased recognition of Puerto Rican English but also positive re-evaluation of how local variation in English is manifested through language mixing in the contact situation of the island. The central recommendation of this paper encourages speakers, educators, community leaders and scholars of Puerto Rican studies to document the local variety of English and advocate for its local and international recognition.}


\IfFileExists{../localcommands.tex}{
   \addbibresource{../localbibliography.bib}
   \usepackage{langsci-optional}
\usepackage{langsci-gb4e}
\usepackage{langsci-lgr}

\usepackage{listings}
\lstset{basicstyle=\ttfamily,tabsize=2,breaklines=true}

%added by author
% \usepackage{tipa}
\usepackage{multirow}
\graphicspath{{figures/}}
\usepackage{langsci-branding}

   
\newcommand{\sent}{\enumsentence}
\newcommand{\sents}{\eenumsentence}
\let\citeasnoun\citet

\renewcommand{\lsCoverTitleFont}[1]{\sffamily\addfontfeatures{Scale=MatchUppercase}\fontsize{44pt}{16mm}\selectfont #1}
  
   %% hyphenation points for line breaks
%% Normally, automatic hyphenation in LaTeX is very good
%% If a word is mis-hyphenated, add it to this file
%%
%% add information to TeX file before \begin{document} with:
%% %% hyphenation points for line breaks
%% Normally, automatic hyphenation in LaTeX is very good
%% If a word is mis-hyphenated, add it to this file
%%
%% add information to TeX file before \begin{document} with:
%% %% hyphenation points for line breaks
%% Normally, automatic hyphenation in LaTeX is very good
%% If a word is mis-hyphenated, add it to this file
%%
%% add information to TeX file before \begin{document} with:
%% \include{localhyphenation}
\hyphenation{
affri-ca-te
affri-ca-tes
an-no-tated
com-ple-ments
com-po-si-tio-na-li-ty
non-com-po-si-tio-na-li-ty
Gon-zá-lez
out-side
Ri-chárd
se-man-tics
STREU-SLE
Tie-de-mann
}
\hyphenation{
affri-ca-te
affri-ca-tes
an-no-tated
com-ple-ments
com-po-si-tio-na-li-ty
non-com-po-si-tio-na-li-ty
Gon-zá-lez
out-side
Ri-chárd
se-man-tics
STREU-SLE
Tie-de-mann
}
\hyphenation{
affri-ca-te
affri-ca-tes
an-no-tated
com-ple-ments
com-po-si-tio-na-li-ty
non-com-po-si-tio-na-li-ty
Gon-zá-lez
out-side
Ri-chárd
se-man-tics
STREU-SLE
Tie-de-mann
}
   \boolfalse{bookcompile}
   \togglepaper[7]%%chapternumber
}{}


\begin{document}
\maketitle


\section{Introduction}

Spanglish is a poorly defined but well-recognized language contact phenomenon among Latino communities \citep{Ardila_2005,Otheguy_Stern_2010,MontesAlcala_2018}. The name itself, derived from splicing the names of the two languages from which it emerged, embodies a problematic history of colonial conflict and linguistic domination in the Americas that prompted situations of both cultural accommodation and re-appropriation among indigenous populations and second language speakers. The term Spanglish -- as the sequence of morphemes that compose its name suggest -- has been historically recognized as “[a] type of Spanish contaminated by English words and forms of expression, spoken in Latin America.” (\cite[v. XVI: 105]{OxfordEnglishDictionary_1989}). Many authors have accordingly recognized the mixture of languages as a problematic and potentially dangerous phenomenon. For example, Salvador Tio describes the state of Spanish in Latin America as follows: “se pudre en la frontera nuevo-mejicana donde [hablan]… la burundanga lingüística” (“[Spanish] is rotting on the New Mexican border where [they speak]… linguistic nonsense”) (Tio, 1992, as cited in \cite[1]{Lipski_2004}). Likewise, Roberto González-Echeverria describes “spanglish, la lengua compuesta de espanol e ingles que salió de la calle, […] plantea un grave peligro a la cultura hispanica y al progreso de los hispanos” (“Spanglish, the language composed of Spanish and English, […] represents a grave danger for Latino culture and the progress of Latinos.”) (González-Echeverria, 1997, as cited in \cite[1--2]{Lipski_2004}). In his paper on the linguistic typology of Spanglish, Lipski addressed the controversy and confusion of definitions and recognizes that “[i]n most cases the word Spanglish and the related connotations of linguistic hybridity \emph{qua} illegitimate birth are used to denigrate the linguistic abilities of Hispanic speakers born or raised in the United States” \citep[1]{Lipski_2004}. Others have considered the mixed language as a temporary means of communication used by those who do not yet have multilingual competence (Betances Palacios, 1997, as cited in \cite[1]{Lipski_2004}), or a communicative necessity comparable to a pidgin \citep[78]{Ardila_2005}. Although many authors agree that Spanglish is essentially a dialect of Spanish, more neutrally defined as “Spanish characterized by numerous borrowings from English” (\cite{AmericanHeritage_dict_EnglishLanguage_2020}), others recognize it as a separate form of speech defined as a “Fusión de español y el inglés” (“fusion of Spanish and English”) (\cite{RealAcademiaEspañola_2020}) created by groups of Hispanics, specifically in the U.S., who mix lexical and grammatical elements of both languages.

Attitudes towards Spanglish as “an ongoing controversy” have been traditionally polemic and often stigmatize the mixed language practice in a U.S. context \citep[136]{Dumitrescu_2010}. However, the use of English and Spanish mixed language expression in the spoken and visual arts, particularly among young people, have been receiving increasing recognition -- and, in some respects, appreciation -- for the mixed language practices of Hispanic bilinguals in recent decades \citep{Hernandez_1997,Stavans_2000,Pousada_2017}.\footnote{Defining types of bilingualism is beyond the scope of this study, but I acknowledge the continuum of language competencies that are encompassed by the word “bilingual” \citep[see][4--7]{Pousada_2017} and use the term in an inclusive manner for the purposes of this paper.} Concurrently, as the literature of education explores the value of translanguaging strategies for increasing numbers of migrant children and mixed language communities, interest in understanding how young people mix Spanish and English has gained momentum \citep{Sayer2008}. Interestingly, and regardless of whether the mixture of languages is perceived in linguistic terms as a dangerous threat to Spanish purity, an emerging Spanish dialect, or an intermediary form among bilingual populations, the phenomenon has predominantly been considered as a Hispanic issue. In contrast, the ideological framework of this paper encourages readers to consider the phenomenon of language mixing\footnote{This paper defines language mixing as “the merging of characteristics of two or more languages in any verbal communication” \citep[6]{Odlin_1989}. In this sense, language mixing practices incorporate a wide range of language contact phenomena including transfer, borrowing, and code-switching in both language acquisition and fluent bilingual contexts \citep{Muysken_2000}. Each of these phenomena have their own complexities and sub-classifications and are often discussed as part of an overlapping system of contact features \citep{Romaine_Kachru_1992,Yumoto_1995,Muysken_2000}.} from a specifically Puerto Rican and an Anglophone perspective. In full recognition of the ongoing controversy regarding the definition of the mixed language practices commonly known as Spanglish and the perceptions of its many culturally and ethnically distinct bilingual speakers, the scope of this paper does not seek to re-define Spanglish but instead considers how the discourse surrounding it may have affected the recognition of a Puerto Rican dialect of English. As such, this paper engages with the Puerto Rican language contact situation and specifically explores the effects of mixed language practices on how English is spoken and perceived on the island. 

This paper does not propose to negate nor diminish the existence of Puerto Rican Spanglish as a separate form of speech. Indeed, international scholarship in contexts of sustained language contact has shown that mixed languages can thrive in multilingual spaces. This is specifically among young speakers who creatively navigate the fluid spaces in between languages, for example, among the predominantly children, adolescents and young adult speakers of Sheng, a mixed language of Nairobi City \citep[176]{Githiora2002} and among the students who have innovated a new form of Estonian-Russian \citep{Zabrodskaja_2013}. For many such contact languages popularized among young people, students and their academic institutions often pioneer efforts in documentation and advocacy for recognized status of the new language. For example, the Jamaican Language Unit of the University of the West Indies in Mona, Jamaica, continues to serve a central role in advocacy for Jamaican Creole (despite a failed petition to make the language official alongside English in 2019 and public debate over appropriate use of the language -- exemplified in the polemic responses to a Southwest Airline employee using his native Jamaican Creole in a security announcement, cf. \cite{Dawkins_2020}). In addition to their lack of consensus about what uses are appropriate for languages in contact, scholars, activists and speakers commonly disagree on the definitions and names of what they are talking about, including whether the mode of communication in question is a new language or a dialectal variety. Although it is not within the scope of this paper to engage with debates of nomenclature regarding contact languages, I hereby acknowledge the complex continuum of codes that exist along a 3-dimentional scale of communication and recognize that for some Puerto Rican bilinguals, such a continuum could be envisioned as having Standard Castilian Spanish at one end and Standard American English at the other, see \figref{fig:fig1_07}. The central argument of this paper is that the polemic complexity of what lies in between incorporates a Puerto Rican variety of English, which is defined as intelligible to other speakers of English and characterized by its contact with Puerto Rican Spanish, Standard American English, and potentially also Puerto Rican Spanglish, a creole or contact language that may lie at the center of the contact situation.

% Figure 1:
\begin{figure}
    \includegraphics[width=\linewidth]{figures/fig1_07.png}
    \caption{Visual representation of the bilingual language continuum in Puerto Rico}
    \label{fig:fig1_07}
\end{figure}

Few studies, with the notable exceptions of \citet{Nash_1971} and \citet{Walsh_1994}, have attempted to document how the English used in Puerto Rico has been affected by contact with Spanish. More recently, \citet{Nickels_2005} has advocated for the recognition of Puerto Rican English as a regional dialect that has emerged in a contact situation. The paucity of academic interest in the variety has not made an impact on public opinion. Most speakers fail to recognize that what they speak is a variety of English; on the other hand, many bilinguals living in Puerto Rico\footnote{This paper refers to its Puerto Rican study population as bilingual given the educational context of the island; English is taught as a compulsory second language in K-12 grades and is a requirement for two years at the undergraduate level.} downplay the functions and importance of the English they use, which they regularly stigmatize as “inaccurate”, or “deficient.” To compound such speaker perceptions, there are no repositories of the spoken or written English used in Puerto Rico, which speakers might refer to as examples of local usage and academics might use as linguistic data upon which predictions can be formed and tested. In response, this paper tests the hypothesis that speaker involvement in generating a database of linguistic resources on Puerto Rican English can increase awareness of how the Spanish-dominated context of the island has given rise to a distinct variety of English.  Specifically, it presents the experimental findings of ongoing work with documentation conducted through a series of pilot studies designed to build capacity at the University of Puerto Rico (UPR) to enable research on mixed language practices and features of Puerto Rican English on the island. The objectives of the work were, firstly, to adapt, devise and pilot-test methodologies that would generate linguistic data for exploratory analysis and, secondly, to establish practices that meaningfully involved participants and early-career scholars in order to help them recognize and value their own Puerto Rican variety of English. 


\section{The context of the study and related literature}

Language mixing practices in Puerto Rico, often referred to with the problematic term “Spanglish”, have a unique contact history in the associated U.S. territory where English and Spanish have been in contact for over a century. Spanish colonialization in the 15\textsuperscript{th} century pre-dates the imposition of English at the turn of the 20\textsuperscript{th} century after the island was ceded to the U.S. after the Spanish-American war. Since a polemic 1902 law designated both English and Spanish as official languages, the status of both languages continues to incite debate, motivate policy change and entrench political divisions \citep{MunizArguelles_1989}. Spanish remains the vernacular in Puerto Rico, but explicit English instruction has long been an objective of the public education system \citep{TorresGonzales_2002}. Despite what \citet[17]{Pousada_2006} recognizes as a collective resistance and developmental delay in the acquisition of English in the island, she recognizes that the use of English and increasing exposure to English media on the island has resulted in “competent bilinguals scattered around the island and concentrated in the coastal and San Juan metro areas” \citep[1]{Pousada_2017}. Aligning with this observation, the \citet[February 17]{US_Census_Bureau_2020} reports that 23\% of the Puerto Rican population speak English “only” or “very well” with the rest of the population reporting proficiency, but less than “very well”.\footnote{Based on 2018 data indicating a total population of 3,069,357 and 715,370 of those (23.3\%) fluent in English.} Thus, for the majority of Puerto Ricans, language contact is perceived as an issue affecting the vernacular Spanish language of the island and little attention has been paid to how it has also affected the English that is spoken on the island.

Aligned with perceptions of Spanglish as a Hispanic issue, much academic interest has focused on how Spanish has been affected by contact with English and on how mixed language practices function among young speakers in Puerto Rico. Moreover, given that the vernacular Spanish operates in diglossia with English for Puerto Rican bilinguals who live on the island, studies conducted in Puerto Rico with bilinguals have invariably focused on Anglicized forms of Spanish or lexical borrowing from English within a Spanish matrix framework \citep[e.g.][]{LopezMorales_1999,Morales_2000,Holmquist_2013}. Thus, academics and speakers alike recognize the existence and characteristics of a Puerto Rican Spanish that is influenced by English at the phonological, morphological, syntactic and discourse levels \citep[see][]{Poplack_1984,Torres_2002,Armstrong_2010,Brown_Rivas_2011}. However, there is much less research and much less consensus on the existence and characteristics of a Puerto Rican English that is influenced by Spanish, with most academics passing over the concept completely in favor of considering English use as something intrinsically associated with the mainland U.S., as exemplified in Domínguez-Rosado’s chapter on language issues in the island, entitled “Puerto Rican Spanish and American English” \citep[vii]{Dominguez-Rosado_2015}. The different historical and social contexts of the two languages, reinforced by their anticipated cultural functions and the prescriptive framework of the Real Academia Española, that all students are taught in school, maintains the division of English and Spanish not only in the educational system and print publication but also in the mental models of speakers. As a result, Puerto Ricans living on the island default to considering the two languages as isolated codes, and so the common practice of mixing the two languages is often stigmatized as inappropriate or incorrect -- at best -- or considered as indicative of a speaker’s linguistic deficiency -- at worst. In addition, the Spanish linguistic heritage of the island that indexes Latino identity is often held in opposition to the English language practices of U.S. citizenship, military service, and academic and economic advancement. Hence, language choice and language mixing can indicate (or be perceived to indicate) political orientation and economic status. Yet, among young people, expression of Puerto Rican identity through mediums that permit and even encourage mixed language content, coupled with the scope of language contact in the tourism industry that drives the island’s economy and offers many entry-level employment opportunities, has begun to shift perspectives on the acceptability and value of code mixing in professional and social contexts. 

As academic interest in language-contact phenomena over recent decades has increased, a growing number of studies on language use in Puerto Rico have explored code-mixing, defined by \citet{Odlin_1989} and \citet{Muysken_2000} as incorporating a wide range of language contact phenomena including transfer, borrowing, and codeswitching in both language acquisition and fluent bilingual contexts. In an early study on what he called “code-shifting” in Puerto Rico, \citeauthor{Lawton_1979} suggests that “the constant interplay and interlocking of Puerto Rican Spanish with English suggests incipient creolization” (\citeyear[257]{Lawton_1979}). However, in the four decades since this study was published, no other scholars have argued that the language contact situation has led to an emergent creole. Furthermore, theoretical frameworks for creolization, such as the theory of abnormal transition or nongenetic development \citep[211]{Thomason_Kaufman_1988} do not align with the language contact situation in Puerto Rico where speakers appear to be undergoing a process of contact-induced dialectal divergence in both English and Spanish rather than prompting the creation a new language. Consequently, most recent studies on Puerto Rican bilingualism investigate the influence of context on language choice, for example, the role of social media platforms in computer-medicated communication \citep{Carroll_2008,Carroll_Mari_2017} and the relevance of individual preferences and attitudes that impact codeswitching behaviors among young people \citep{Tamargo_VelezAviles_2017,PerezCasas_2016}. Such studies on speaker behaviors indicate that Puerto Rican bilinguals increasingly recognize their own language switching practices and are more likely to evaluate such practices favorably \citep{GuzzardoTamargo_etal_2019}. In sum, much of the scholarship on language practices in Puerto Rico focus on either: the local variety of Spanish that is influenced by English, commonly known as Spanglish, or the sociolinguistic context of codemixing practices in informal speech among young people. 

English use among Puerto Ricans living on the island was, and still is, a controversial topic \citep{Velez_Schweers_1993,Dumitrescu_2010,Shenk_2012,GonzalesRivera_OrtizLopes_2018}. Since the establishment of the free public-school system in Puerto Rico following the U.S. military occupation of the island in 1898, English as a language of instruction has been a recurrent and problematic issue. Hence, early research on English in Puerto Rico often focused on education \citep{Marvin_1904,Falkner_1908,Desing_1947,Manuel_1952}. Towards the second half of the twentieth century, although publications about English use in Puerto Rico continued to have a predominantly pedagogical focus, scholarship also began to address issues of identity and question the imposition of English as a political strategy \citep{Pattee_1945}. Publications with a sociolinguistic focus addressed shifting perceptions of English among the Puerto Rican population \citep{Epstein_1968,McCroskey1985,Clachar1997} and began to address how growing numbers of English-speaking Puerto Ricans were changing the role of English on the island \citep{Fayer_2000}. A handful of scholars have attempted to define the functions and forms of English in Puerto Rico \citep{Nash_1971,Walsh_1994,Nash_1996,Fayer_Castro_Diaz_Plata_1998}, yet there is still little recognition that a local variety of English exists. \citeauthor{Nickels_2005}’ paper on “English in Puerto Rico” (\citeyear{Nickels_2005}) provides an overview of the political, educational, sociolinguistic and literary contexts of the language on the island and explains, “[t]he variety of English spoken in Puerto Rico is only beginning to be identified as a variety in its own right” (234). She additionally recommends that we should start explicitly calling the variety “Puerto Rican English” \citep[235]{Nickels_2005}. However, since this study was published, no subsequent scholarship has aimed to explicitly locate Puerto Rican English in a wider discourse on New Englishes. Instead, a handful of recent publications and unpublished dissertations show that there is continued interest in researching the sociolinguistic context of Puerto Rican English as it relates to political and social identity \citep{Falcon_2004,PerezCasas_2008}, the role of English in daily life \citep{RuizCorrea_2019}, and the use of English in Puerto Rican literature \citep{MendezAlberich_2006}. However, the educational importance of English acquisition is still very much at the forefront of scholarship \citep{Eisenstein_Ebsworth_Cai_2018}, and that has taken on a new importance after the economic migration of Puerto Ricans to the mainland U.S. in the aftermath of Hurricane Maria. Yet, very little scholarship, published or otherwise, has offered a detailed description of Puerto Rican English. And only one scholar, Rose Nash, has attempted a comprehensive description of how Puerto Rican English exists on a continuum of language contact in the island; Nash published a series of three papers documenting language contact on the island in terms of English-influenced Spanish \citep{Nash_1970}, Spanish-influenced English \citep{Nash_1971} and a type of contact code occupying the middle space \citep{Nash_1982}.\footnote{Nash includes an endnote in the last of these three publications to explain her intention to write another paper on the second-language Spanish variety of native English speakers on the island. However, if such a paper exists, it remains unpublished.} However, since this work was published nearly four decades ago, little progress has been made in documenting Puerto Rican English. 


\section{Documentation methodologies}

In order to test the hypothesis that speaker involvement in generating a database of linguistic resources on Puerto Rican English can increase awareness of how the Spanish-dominated context of the island has given rise to a distinct variety of English, this experimental study adapted, devised and pilot-tested eight distinct data-collection methodologies relating to spoken language, written language, and surveys on language issues, see \figref{fig:fig2_07}. All the data-collection activities were conducted in Puerto Rico at the UPR Cayey, a four-year public institution with an enrollment of around 3,000 Hispanic students located in a mountain municipality of the central-eastern region of the island. In all these data-collection activities, undergraduate students of the UPR Cayey were involved as participants and as project-documentation assistants.

% Figure 2:
\begin{figure}
    \includegraphics[width=0.75\linewidth]{figures/fig2_07.png}
    \caption{Summary of the eight data collection activities in the pilot project.}
    \label{fig:fig2_07}
\end{figure}


\subsection{Spoken language}

\begin{sloppypar}
Three distinct data-collection methodologies to capture spoken language were adapted from models devised by the British Library and catalogued in their Sounds archive: 1) a lexical repository called the WordBank, 2) a collection of readings called the VoiceBank, and 3) a series of recorded spontaneous dialogues called Voices. The catalogued holdings of the British Library’s ongoing efforts to capture variation in the accents and dialects of English are available online at \url{https://sounds.bl.uk/Accents-and-dialects/} (\citeauthor{BritishLibraryBoard}, n.d.). Using this website, and specifically the pages linking to the WordBank, VoiceBank and Voices collections, audio files were accessed to serve as a reference point to facilitate the design of the adapted methodology. In addition, descriptions of the methodologies and their associated documentation (including template consent forms) was shared by the lead curator of the British Library’s Spoken English archive, Jonathan Robinson, initially in a conference paper \citep{Robinson_2018} and subsequently in meetings and email-communications. As this paper presents the exploratory analysis of only one of these methodologies, the Puerto Rican WordBank, this is the only methodology explained here in detail.\footnote{Interested readers may contact the author at \texttt{sallyj.delgado@upr.edu} for detailed descriptions of any of the methodologies not described in full in this paper.} 
\end{sloppypar}

The PR WordBank methodology aimed to record lexical items and their associated meaning, etymology, and usage from among Puerto Rican participants. In the first pilot test of this methodology, conducted between October and December of 2018, instructions given to participants were to think of and then explain a word or phrase in English that they often used, for example with family or friends. Participants were asked to state and spell the word, explain what it means and where it comes from. They were additionally asked to give some context or an example in terms of how they used the word or phrase provided. Recordings were made and catalogued with non-identifying codes that were then associated with the participant’s socio-demographic data that they provided on the consent forms. In the second pilot test of this methodology, conducted between August and December of 2019, fieldworkers who were making the recordings similarly instructed participants to say and explain a word or phrase in English that they often used. However, if participants requested more instruction or if they struggled to think of words they might contribute (as happened in the 2018 testing stage), fieldworkers also offered a series of common suggested words from which the participant might choose. This list of possible words and phrases was generated by project-documentation assistants with input from previous work on lexical variation \citep{Nash_1971,Nash_1996} supplemented by suggestions from the Project Director and the faculty of the English department of the UPR Cayey based on their experience with lexical variation in local uses of English. The PR WordBank methodology has, to date, generated 142 audio recordings of lexical variation with transcripts and associated metadata. Ongoing exploratory analysis on this data uses techniques in trend analysis to determine patterns in the data which are then analyzed in terms of linguistic parsing and associations with socio-demographic markers.

\subsection{Written language}

Three distinct data-collection methodologies to document written language were designed to capture variation in both informal and formal contexts for expressive, analytical and academic purposes: 1) public writing on message boards, 2) creative writing for an anthology of narrative work, and 3) developed responses to academic prompts. All written material was collected on location at the UPR Cayey and most participants were undergraduate students of the campus.\footnote{The public writing component captured anonymous contributions, so the research team cannot guarantee that all participants of this data-collection activity were undergraduate students. It is possible that visitors to the campus, faculty, and/or service and administration personnel also contributed to the message boards.} As this paper presents the exploratory analysis of only one of these methodologies, viz. public writing on message boards, this is the only methodology explained here in detail.

\begin{sloppypar}
After the academic calendar resumed on October 23, 2017 following an interruption after Hurricane Maria that made landfall in Puerto Rico on September 20, 2017, a series of public message-boards were posted around campus to spread positive messages in response to written prompts. The prompts, location and dates that message-boards were posted are summarized in \tabref{tab:tab1_07}. An example of the photographic images that generated the data is shown in \figref{fig:fig3_07}. Project-documentation assistants collected the handwritten comments posted on these public message boards with photographic images, re-wrote the comments in electronic documents and determined a system of classification to conduct exploratory research. This methodology to capture public writing on message boards has generated a total of 443 language items, defined as units of meaning as they occur in single words, phrases, or sentences of either one or multiple clauses. Exploratory data-analysis was conducted at three levels. First, a language choice analysis determined participant preferences for either Spanish, English, an alternative language, or mixed code.\footnote{Images and comments that were not decipherable or were composed of unknown acronyms were discounted. Words such as “pizza” that were potentially English, Spanish (or another language) were classified separately so as not to artificially inflate language category results.} Second, a readability analysis was conducted using the Flesch Kincaid Reading Ease test, the Flesch Kincaid Grade Level indicator, the Gunning Fog index, the Coleman Liau index, the Automated Readability Index (ARI), and the SMOG index using the software at \url{https://www.online-utility.org/english/readability\_test\_and\_improve.jsp} and at \url{http://www.readabilityformulas.com/freetests/six-readability-formulas.php}. Lastly, a content analysis was conducted by classifying the focus of responses into four broad categories determined by trends in the data. The four categories related to: Leisure, Community, Physical Fulfillment, and Spiritual Fulfillment, and within these four broad categories, sub-classifications were also determined, e.g., in the “Leisure” category, sub-classifications included: TV, Music, Social Media and Video Games.
\end{sloppypar}

% Table 1:
\begin{table}
\begin{tabularx}{\textwidth}{lQll}
    \lsptoprule
    {Prompt} & {Location} & {Date posted} & {Date collected} \\ \midrule
    \textit{Razones para   vivir} & Outside cafeteria in Student Centre & Sept. 15, 2017 & Nov. 27, 2017 \\
    positive comments & In corridor of English Department classrooms & Oct. 23, 2017 & Nov. 27, 2017 \\
    \textit{Soy agradecido por…} & Outside cafeteria in Student Centre & Nov. 11, 2017 & Nov. 27, 2017 \\ 
    \lspbottomrule
\end{tabularx}
\caption{Summary of the prompts, location and dates that message-boards were posted at the UPR Cayey}
\label{tab:tab1_07}
\end{table}

% Figure 3:
\begin{figure}
    \includegraphics[width=\linewidth]{figures/fig3_07.png}
    \caption{Aligned photographic images taken from the message board “Razones para vivir” that generated data on informal written language.}
    \label{fig:fig3_07}
\end{figure}

\subsection{Language attitudes}

Two distinct data-collection methodologies to document participant’s attitudes towards language issues were pilot tested at UPR Cayey: 1) a survey on accent and confidence, and 2) a survey on discipline-specific competency using English as a Second Language. Survey responses to questions and prompt statements were given on a Likert scale to generate qualitative data with additional short-response questions to generate qualitative data. In both surveys, quantitative and qualitative data generated from the data collection activities were concurrently analyzed, descriptively and analytically, following a model suggested by \citet[23--28]{Auerbach_Silverstein_2003}. Quantitative data were processed using the average, maximum, minimum, standard deviation, and the range of the variable created to evaluate the perception of the participants, and qualitative data were analyzed by preparing the transcription of responses and then using exploratory and inferential analysis methods to identify statistically significant trends. Both data-collection methodologies were adapted from previous surveys that had been subject to experimental analysis and evaluated as suitable methods for data-collection. As this paper presents the exploratory analysis of the methodology related to accent and confidence, this is the only methodology explained here in detail.

The UPR Cayey survey on accent and confidence was adapted from a methodology described in a paper entitled “Student’s self-perceived English accent and its impact on their communicative competence and speaking confidence” \citep{Norman_2017}. This survey proposes to measure how self-perceptions of accent impact language attitudes and self-confidence in relation to English as a second language in the academic environment, originally among students taking English 6, a course the students must pass to be eligible for further academic studies, in one Upper-Secondary School in Sweden.\footnote{Note that this methodology was selected for its experimental design and does not intend to suggest that the language context of Puerto Rico is the same as the Swedish context of the original study.} The survey adapted for use at UPR Cayey used similar questions as the model questionnaire, that \citet{Norman_2017} provided in an index to the published study, with adapted wording that reflected the Puerto Rican context of the new study, for example, the question: “How often do you speak English with someone that is not a native speaker of Swedish?” \citep[39]{Norman_2017} was re-worded to: “How often do you speak English with someone that is not a native speaker of Spanish?” The UPR Cayey study also included demographic information and asked students to answer some questions in a short response, for example: “Is it important to sound like a native speaker of English?” and “Do you think people judge you because of your accent?” The survey was made available from August 16, 2019 and initially promoted among students enrolled in INGL 3265 “English Across Cultures”, before being subsequently shared among a wider community of undergraduates at the UPR Cayey using social media and direct email communication with potential participants. The survey on accent and confidence gathered data from 215 anonymous participants between August 16 and December 29, 2019, of whom the majority self-reported an age of between 16 and 25 and are assumed to be students of the university. The survey was administered using Google Forms, a web-based survey platform that provides rudimentary graphics indicating percentage responses to each question. These graphics and the associated summative data were used to enable exploratory analysis. 


\section{Results: Exploratory analysis}

This section outlines some of the main findings of ongoing exploratory analysis conducted by this author and collaborative groups of early-career scholars at UPR Cayey as they relate to the methodologies of the three data-collection areas of spoken language, written language, and language attitudes.

\subsection{Spoken language: PR WordBank}

Exploratory research on spoken language using data generated in the PR WordBank indicate three significant categories of local usage that participants recognized as lexical items of Puerto Rican English in addition to usage that might be considered widespread across the U.S. and in African American Vernacular English. The first, and by far the most sizable category representing more than half of the overall contributions at 55\%, is composed of English lexicon with comparable meaning to Standard English but demonstrates Spanish bound morphology that typically determines grammatical function. Furthermore, the majority of these lexical items are verbal and have been assigned the default (most common and regular) \emph{-ar} verbal suffix in Spanish.\footnote{The orthography of all lexical items indicated in quotation marks derive from participant contributions and/or the determination of Puerto Rican documentation assistants and thus reflect local spelling conventions.} For example, \emph{textear} (‘to text’), \emph{ chillear} (‘to chill [out]’), and \emph{postear} (‘to post[upload] social media’). Although the majority of the data from this category shows words expressed in the infinitive form, a significant groups of words were also expressed in adjectival participle forms with the affixation of the \emph{-ado} or \emph{-iado} verbal suffix in Spanish that is typically realized as a falling triphthong which is fronted at the end [iao] in Puerto Rican Spanish, for example, \emph{hookiao} (‘hooked’) which is also orthographically represented as \emph{juqueao} and \emph{jukiao}. Unsurprisingly, because these words have undergone a lexicalization process and are realized in a Spanish matrix framework, their orthography aligns with Spanish norms, as shown in other words such as \emph{chonquear} (‘to chunk[vomit]’) and \emph{janguear} (‘to hang [out]’). Furthermore, phonemes are added if the lexeme defies underlying Spanish phonological rules, e.g., the affixed vocal /e/ that forces the /s/ into the coda of a newly created syllable at the start of the words \emph{esnackear} (‘to snack’) and \emph{estartear} (‘to start’). This category of data reflects a type of interference that \citep[212]{Thomason_Kaufman_1988} describe as resulting from borrowing in the language contact situation and leading to lexical diffusion.\footnote{One caveat is that, although speakers have contributed these terms as examples of their local use of English, they may 
be considered as features of interference in Puerto Rican Spanish. However, given that these findings are exploratory, and that the conceptual space in which Puerto Rican English manifests necessarily intersects with Puerto Rican Spanish, they are documented here as their speakers intended, as examples of PR English.} 

The second most significant category of the data generated in the PR WordBank, representing about a quarter of the overall contributions at 24\%, is composed of English-derived words that are adapted for Puerto Rican Spanish phonology and subsequently assigned an extended or alternative semantic scope. Some of these words are derived from English brand names and have undergone extension such as \emph{Conflei} (‘Cornflakes’) used as a common noun meaning all breakfast cereal, \emph{chicle} (‘Chiclets’) used as a common noun meaning gum, and \emph{pampel} (‘Pampers’) used as a common noun meaning diapers. Other words have undergone semantic shift such as “ready”, which one participant explains:

\begin{quote}
    So, between my friends and I, and I think most uh part of the island, we use the word “ready” which actually means uh that something is pre-is like prepared. Like if you’re ready to go out that means you’re prepared to go out, you feel dressed and all that. Um, but we use the word “ready” as like as is a synonym of cool. So, we say, “oh eso está ready”, (that’s-that’s ready, that’s cool). It really doesn’t have any correlation with it, that’s how we use it. (Transcript BL.WB.2018.10.27.KV.IC.1.)
\end{quote}

Another participant explains how the word “pitcher” is used locally to denote “a person that ignores you or ignores other people” (Transcript BL.WB.2018.11.1.F R.FR.2.). The same participant explains, “It comes from, like, baseball, from pitcher, but I don’t really know what relevance it has with ignoring people but that's what it means to people here” (Ibid.).

The third, and perhaps the most innovative category of the data generated in the PR WordBank, composing 15\% of the overall contributions, is composed of newly coined words or phrases that have a very precise local meaning, for example the term “island dog” that one participant explains is a term for mixed breed stray dogs and is used across the Anglophone Caribbean. Many of these terms that index relationships or salutations are restricted to family or close social networks in usage, and thus may indicate incipient coinage that may yet become widespread, for example, the term \emph{broki} that is described as a synonym for friend and may derive from the English word ‘brother’ or ‘bro’, the word \emph{obb} which one participant describes as a salutation, and the word \emph{peje} that means a close friend or child, and may derive from Spanish \emph{pexe} meaning ‘fish’ or \emph{peje} meaning ‘a cunning person’ or ‘mermaid’ \citep[329]{Roberts_2014}. One particularly interesting coinage that was contributed is the term \emph{cangriman} that was explained by one participant as a Puerto Rican adaptation of standard English word \emph{congressman} which his grandfather used to refer to any person with high social status. However, this word also has roots in the Spanish word \emph{cangrejo} (‘crab’) historically used as a derogatory term for foreigners in Latin America and may have been used as a reference to users of cannabis in the late 1970s and 80s because of the way people held their fingers like a crab’s claws to smoke cannabis joints \citep{UrbanDictionary_2007}. Hence, the lexicalized expression may have more localized and covert meanings that the contributing morphemes suggest. 

\subsection{Written language: Public writing on message boards}

Exploratory research on public writing using data sourced from public message boards indicate that the language used by participants is strongly influenced by the language used to contextualize the writing activity. Of the three message boards that were used to generate data, the two that used a Spanish prompt: \emph{Razones para vivir} (‘reasons to live’) and \emph{Soy agradecido por...} (‘I am grateful for...’) generated most responses in Spanish (65.5\% of responses and 78.8\% of responses, respectively). The one message board that used an English prompt that read “The English Department wants to share your positive comments” (subsequently shortened to “Positive comments”) generated most of its responses in English (76.2\%), see \tabref{tab:tab2_07}. The aggregate data indicate an average of 73.5\% of responses were provided in the same language as the prompt. A greater percentage of responses used code mixing in response to the English prompt rather than the two Spanish prompts, potentially indicating preference for an increased use of the Spanish vernacular in a second language context. In addition, the Spanish prompts generated responses in other languages (e.g., French, Japanese) which was not evident in response to the English prompt, potentially suggesting an increased level of comfort integrating other languages into a Spanish matrix context rather than an English matrix context. It is likely that in addition to the explicit language of the prompt, the location of the message board also implicitly determined the language of responses: the two Spanish-language message boards were located in the student center, a social area where students are more likely to use their Spanish vernacular. The message board with the English prompt was located in the corridor adjacent to English Department classrooms, where students are more likely to use English for academic purposes. Interestingly, this finding appears to corroborate the data in a study of Facebook posts that indicates Puerto Ricans only use English to reply to previous messages written in English \citep{Carroll_Mari_2017}.

% Table 2:
\begin{table}
\tabcolsep=.9\tabcolsep
\small\begin{tabular}{llcccccc}
    \lsptoprule
    &  & \multicolumn{5}{c}{{language of responses (\%)}} &  \\ \cmidrule(lr){2-8}
    {prompt} & {language} & {Spanish} & {English} & mixed & other & misc. & {total} \\\midrule
    \textit{Razones para vivir} & Spanish & \cellcolor[HTML]{D9D9D9} 65.1* & 18.1 & 2.9 & 1.7 & 12.2 & 100 \\
    \textit{Soy agradecido por…} & Spanish & \cellcolor[HTML]{D9D9D9} 78.8* & 11.8 & 1.2 & 1.2 & 7.0 & 100 \\
    positive comments & English & 19.0 & \cellcolor[HTML]{D9D9D9} 76.2* & 4.8 & 0.0 & 0.0 & 100 \\ 
    \lspbottomrule
\end{tabular}
\caption{Summary of data on message-boards classified by language of response. *: Same language as the prompt.}
\label{tab:tab2_07}
\end{table}

The composition of responses, determined from the application of readability indicators, show that there were a significantly higher percentage of clauses in the message board with the English prompt compared to the two with the Spanish prompts, see \tabref{tab:tab3_07}. These results may corroborate the previous finding that responses follow the context of the prompt because neither of the two prompts written in Spanish were clauses, but the English prompt was expressed as a clause. Therefore, just as participants overwhelmingly matched the language of their response to the language of the prompt, participants also matched the grammatical composition of their response to the composition of the prompt. In addition, it is possible that the grid-style organization of the message board with the English prompt may have promoted more clauses as participants may have wanted to fill a pre-determined space. Comparatively, the organization of the two message boards with the Spanish prompts were presented on a blank background with no suggestion of pre-determined response space, see \figref{fig:fig3_07}. In addition to phrase and clause analysis, it was determined that the message board with the English prompt had slightly fewer words per clause than the two message boards that used a Spanish prompt (based only on responses that were composed in clauses, as indicated in \tabref{tab:tab3_07}, column 4). The message board with the highest number of words per clause was the one with the prompt \emph{Soy agradecido por…} which may have been motivated by the increased formality attached to some of the religious content in responses. There was no significant difference in the average words per syllable among the three message boards.

% Table 3:
\begin{table}[!ht]
\centering
\setlength{\tabcolsep}{4pt}
\resizebox{\linewidth}{!}{
\begin{tabular}{lccccc}
    \toprule
    \textbf{prompt} & \textbf{prompt} & \textbf{percent} & \textbf{percent} & \textbf{average syllables} & \textbf{average words} \\
    & \textbf{composition} & \textbf{phrases} & \textbf{clauses} & \textbf{per word} & \textbf{per clause} \\ \midrule
    Razones para vivir & phrase & \cellcolor[HTML]{D9D9D9} 91.6* & 8.4 & 1.5 & 7.8 \\
    Soy agradecido por… & phrase & \cellcolor[HTML]{D9D9D9} 85.9* & 14.1 & 1.6 & 8.2 \\
    positive comments & clause & 8.6 & \cellcolor[HTML]{D9D9D9} 91.4* & 1.6 & 6.5 \\ \bottomrule
    \multicolumn{6}{l}{\footnotesize *Same composition as the prompt}
\end{tabular}
}
\caption{Summary of data on message-boards classified by composition of responses}
\label{tab:tab3_07}
\end{table}

The subject-matter content of responses, determined by the classification of responses into broad categories, showed that participants referred to spiritual fulfillment and community subjects more than leisure or physical fulfillment, see \tabref{tab:tab4_07}. This may, in turn, suggest that spiritual fulfillment and a sense of community are more valued than leisure activities or physical fulfillment. Inherent values reflect the nature of the prompts that were designed to focus on subjects of high value to participants that they considered were worth living for, were thankful for, or were positive elements in their lives. This was an important finding for early-career scholars who had initially been shocked by some of the comments related to physical fulfillment, specifically in the sub-categories of sexual desire, (fast)food and alcohol consumption, that had promoted them to initially consider participants’ responses as fairly shallow and superficial. After analysis, these same scholars recognized that the number of responses focused on spiritual fulfillment, specifically in the sub-categories of positive self-affirmation and religion, were more than double those of physical needs and suggested that most participants did not respond with superficial comments. Instead, they determined that participants responded with an honesty that indexed both profound psychological needs and more immediate needs for physical fulfillment and distraction.\footnote{The timeframe of this study is notable in that message boards were active both before and after the prolonged academic recess caused by Hurricane Maria, a time in which many Puerto Ricans were experiencing significant trauma, insecurity and anxiety.}

% Table 4:
\begin{table}[!ht]
\centering
\setlength{\tabcolsep}{3pt}
\resizebox{\linewidth}{!}{
\begin{tabular}{lF{1cm}F{2cm}F{2.5cm}F{2.5cm}F{0.75cm}F{1cm}}
    \toprule
    & \multicolumn{4}{c}{\textbf{content focus of responses (\%)}} &  \\ \cmidrule{2-5}
    \textbf{prompt} & \multicolumn{1}{l}{\textbf{leisure}} & \multicolumn{1}{l}{\textbf{physical fulfillment}} & \multicolumn{1}{l}{\textbf{spiritual fulfillment}} & \multicolumn{1}{l}{\textbf{community}} & \textbf{total} \\
    &  &  &  &  & \textbf{\%} \\ \midrule
    Razones para vivir & 22.4 & 18.7 & 31.1 & 27.8 & 100 \\
    Soy agradecido por… & 18.3 & 14.0 & 41.9 & 25.8 & 100 \\
    positive comments & 4.5 & 8.9 & 65.7 & 20.9 & 100 \\ \midrule
    \multicolumn{1}{r}{\cellcolor[HTML]{F2F2F2}\textbf{average percent}} & \cellcolor[HTML]{F2F2F2} 15.1 & \cellcolor[HTML]{F2F2F2} 13.9 & \cellcolor[HTML]{F2F2F2} 28.9 & \cellcolor[HTML]{F2F2F2} 24.8 & \\ \bottomrule
\end{tabular}
}
\caption{Summary of data on message-boards classified by content of responses}
\label{tab:tab4_07}
\end{table}


\subsection{Language attitudes: Accent and confidence}

Exploratory research on both quantitative and qualitative data collected from the accent and confidence survey shows that Puerto Ricans, who predominantly learn English as a second language, have positive self-perceptions of communicative competence and speaking confidence. Furthermore, the Puerto Rican results are similar to the Swedish participants who use English as a second language in \citeauthor{Norman_2017}’s survey (\citeyear{Norman_2017}) which the methodology of the present study replicated.\footnote{Although the history of English in Sweden is not the same as Puerto Rico, a comparable situation of diglossia exists in which 1) advanced competencies in English are required for many programs in institutions of tertiary education, and 2) English competency is a characteristic in elite professions among the middle and upper classes \citep{Berg2001}.} Findings from the Puerto Rican data align with Norman’s conclusion from the original study of 80 Swedish participants:

\begin{quote}
    students seem to think that having a native-like accent is overvalued and that communication is to favour over their perceived English accent... most of the students value communication over perceived accent, and many of them say that they do not care how they sound as long as what they say is conveyed. \citep[ii]{Norman_2017} 
\end{quote}

Among the 215 Puerto Ricans who participated in the Puerto Rican survey, a significant majority, representing nearly four out of five participants, said that their English is either “very good” or “quite good” and most others evaluated their competency as “average”, see \tabref{tab:tab5_07}. Only 8 participants evaluated their English as “not good”, and it is interesting to note that this confidence is notably in excess of the Swedish participants, more than a fifth of whom considered their English “not good” \citep[13]{Norman_2017}. Similar to the Swedish participants, when asked if they aim to speak with a specific accent, three out of every five of the Puerto Rican participants said that they do not aim to sound a certain way. Of those students who do aim to speak with an accent, the most common response was a preference for an American accent, followed closely by a preference for a Puerto Rican or Hispanic accent, see \tabref{tab:tab6_07}. For those who consciously chose an accent, when asked to indicate the factors that may have influenced their choice, about half of all participants indicated that movies, TV series or online games and social media sites such as Facebook, X (formerly Twitter), Instagram and YouTube have influenced them.\footnote{Although these social media sites began as predominantly image and text-based platforms, the increasing availability of audio-visual media mean that they are becoming relevant phonological sources for ESL learners.} However, one in four participants also chose the option “I think it sounds like me.” When explaining their conscious use of a native-like accent (or not) in written responses, many participants recognized intelligibility in addition to educational and social expectation as reasons to sound like a native speaker, for example comments included: “a lot of Americans can’t understand you perfectly if you have a heavy accent”, and “some native english speakers tend to make fun of hispanic accents when speaking English”.\footnote{All syntax, word choice and orthography represent what the participants wrote in their comments.} Others recognized but did not internalize external pressure, for example: “I believe it is expected, not necessarily important”, and “I dobt[don’t] think it is important, but socially there is a lot of pressure.” However, many responses also celebrated diversity and stressed intelligibility as the main objective of communication, for example, one participant said: “I do not think it is important to sound like a native speaker. I just think it is important that you speak a way the other person is able to understand.” Another commented: “diversity in accents exists so there is no reason to try to sound in another way that isn't natural to yourself.” Some participants even recognized diversity as a critical component of language change, for example: “Accents are like the spices that allow the food that is the language to expand and diversify. Without accents, language would have a harder time to expand and grow.” Overall, Puerto Rican participants had overwhelmingly positive perceptions of local phonological variation despite their recognition of external pressure to adopt native-like American accents.

% Table 5:
\begin{table}[!ht]
\centering
\begin{tabular}{lcccc}
    \toprule
    \textbf{My English is...} & ...very good & ...quite good & ...average & ...not good \\ \cmidrule{2-5}
    \textbf{responses (\%)} & 41.9 & 36.3 & 18.1 & 3.7 \\
    \bottomrule
\end{tabular}
\caption{Self-evaluation of competency in English expressed as a percentage of participant responses ($n=215$)}
\label{tab:tab5_07}
\end{table}

% Table 6:
\begin{table}[!ht]
\centering
\setlength{\tabcolsep}{4pt}
\resizebox{\linewidth}{!}{
\begin{tabular}{lF{4cm}ccc}
    \toprule
    \textbf{accent choice} & I do not aim to use an accent & \multicolumn{3}{F{4cm}}{I consciously choose an accent} \\ \cmidrule{2-5}
    \textbf{accent type} & \cellcolor[HTML]{F2F2F2} & American & Puerto Rican & Hispanic or “other” \\
    \textbf{responses (\%)} & 63.7 & 18.6 & 13.0 & 4.7 \\ \bottomrule
\end{tabular}
}
\caption{Participant responses to the question “when you speak English, do you aim to speak with an accent?” expressed as a percentage ($n=215$)}
\label{tab:tab6_07}
\end{table}

When participants were asked to rank whether it was more important to sound like a native speaker or to express themselves easily, four out of every five participants asserted that ease of expression was more important; fewer participants reported that they had confidence with their accents. In response to a question to determine if participants feel awkward or embarrassed about their accents when they speak English, four out of every five participants said that they feel little-to-no negative feeling, and of the remaining participants, most reported feeling only a little awkward or embarrassed, see \tabref{tab:tab7_07}. Participants demonstrated insightful awareness of language variety and high self-confidence in written responses to the question: do you think that a person with English as their first language will understand you better if you sound like a native speaker? Some speakers shifted the burden of comprehension to the listener with comments such as: “Don’t care if they do. While they only speak one language, I speak two. They should make the effort to understand THEIR language regardless the accent” (emphasis by the participant) and “They would need to pay more attention, but if I speak correctly they could still understand me.” Others recognized the scope of variety and varying degrees of intelligibility as natural phenomenon among speakers of any language, for example comments such as “sometimes native speakers don’t understand others very well due to their accents”, “a lot of people aren't used to hearing other accents” and “It depends on the English accent that they are accustomed[to].” One participant spoke for many with the comment “one should always aim to be understood. But i don’t feel it’s important to hide my accent or pretend my first language is english.” In agreement with such sentiments, it is not surprising that nine in ten participants said they would not be concerned if they were identified as second language speakers of English, and the two highest factors that participants identified as important were being understood (ranked first) and sounding confident (ranked second), both considered more important in speaking than grammatical accuracy (ranked third).

% Table 7:
\begin{table}[!ht]
\centering
\begin{tabular}{lcccc}
    \toprule
    \multicolumn{1}{l}{Awkward or embarrassed} & \multicolumn{2}{c}{no}    & \multicolumn{2}{c}{yes} \\ \cmidrule{2-5}
    \textbf{extent of feeling} & not at all & not really & a little & often / a lot \\
    \textbf{responses (\%)} & 48.4 & 16.7 & 27.9 & 7.0 \\ \bottomrule
\end{tabular}
\caption{Participant responses to the question “do you feel awkward or embarrassed about your accent when you speak English?” expressed as a percentage ($n=215$)}
\label{tab:tab7_07}
\end{table}


\section{Discussion: Increasing recognition and value of Puerto Rican speech}

This section demonstrates how language documentation and exploratory research has increased awareness of the characteristics of Puerto Rican speech, and specifically the existence of Puerto Rican English, among speakers and early-career scholars. It furthermore validates the claim that the methodologies used in this pilot study might be effectively replicated on a larger scale to promote public recognition of a local variety of English and combat some of the social stigma associated with ubiquitous language mixing. 

\begin{sloppypar}
Before working on language documentation activities, responses from undergraduate students generally indicate that many are unaware of dialect varieties, including their own. When asked to define the term “Puerto Rican English”, responses included comments such as: “well, that actually sounds racist”, and “it just means USA English with another accent. There’s nothing deep to it.” Other comments demonstrated misconceptions of language homogeneity: “English is just a language a simple one, there isn’t different types of it.” Unsurprisingly, given the political and economic status of the island, participants generally agreed on the advantages of knowing and being able to use English, as indicated in one comment that “someone bilingual has more open doors.” Yet, students intuitively know that English and Spanish are polarized with each code operating in isolation according the ideologies of exonormative standardization. Students recognized that Spanish is governed by the prescriptive norms of the Real Academia Española of Madrid, Spain, and English is governed by descriptive U.S. standards that are promoted through the curricula of Puerto Rico’s Department of Education and codified in publications such as the Merriam-Webster dictionary. Language mixing is therefore perceived as a dangerous unregulated middle zone that might positively index Latinx identity, particularly in the context of U.S. migration, but can also create uncomfortable points of contrast for Puerto Ricans who live on the island and do not identify with a wider Hispanic identity or (im)migrant status that is often stigmatized in public discourses of the U.S.. Underlying judgements about identity construction through language choices reinforce what Pérez Casas identifies as a characteristic among Puerto Ricans to think of English and Spanish individually despite the common mixing of both languages in daily communication, which she observes “does not necessarily mean that they embrace a ‘bicultural’ identity” \citep[56--57]{PerezCasas_2016}. Unsurprisingly, given the dominance of Spanish vernacular on the island, the unsettling effects of language contact are typically seen in terms of a Spanish matrix framework, as indicated in one speaker’s observation: “the more someone knows English, the more it’s integrated into their Spanish.” In such ways, most students, before working with the documentation project, demonstrated perceptions that English in Puerto Rico is an encroachment of modern U.S. influence on the Hispanic identity of Puerto Ricans. Consequently, no participants recognized a local variety of English or engaged with ideas about emerging localized norms even when they recognized extensive bilingualism and the profusion of local forms that derive from contact with Spanish. 
\end{sloppypar}

When asked to explain how Puerto Ricans use language, most students recognize that language mixing is integral to expressions of Puerto Rican identity. Student perspectives on codeswitching generally recognized the phenomena as “normal” yet many also described or implied that it is stigmatized as deficient. In such a context, many participants referred to negative emotions attached to codeswitching practices, for example, one student acknowledged, and appeared to apologize for, how she mixes languages: “I use English and Spanish all the time. \emph{I try not to do it on purpose}” (emphasis added). In other cases, students seemed to justify the stigma of codeswitching by explaining how it is triggered by low competency and lapses of memory, for example, one student explained: “often, speaking more than one language can get a bit confusing sometimes and can make you forget.” Many students commented on what they received as their own deficient recall: “I forget how to say a word in Spanish” and “I can’t seem to think of the right word… this is specially frustrating.” Yet, even when not linked to frustration, participants often made comments that suggest they associate codeswitching with carelessness or comfort: “It’s just convenient…I just can’t remember certain words in Spanish”, and “basically, I use the first language that comes to mind.”  Negative perceptions of mixed codes are not unique to the Puerto Rican context. Indeed, scholarship attests to the commonality of derogatory discourse in contexts where mixed languages emerge, exemplified by movements such as “Speak Good English” in Singapore, which was launched in 2000 in Singapore to diminish the influence of Singlish and promote what was considered a more comprehensible standard variety \citep[262]{Lim2015}. Negative mental models associated with codeswitching have been historically linked with the stigmatized Puerto Rican migrant experience; Tato Laviera, a foundational figure of the Nuyorican Poetry Movement, expresses:

\begin{tabular}{ll}
    hablo lo inglés matao & [I speak broken (lit. “killed”) English] \\
    hablo lo español matao & [I speak broken (lit. “killed”) Spanish] \\
    no sé leer ninguno bien & [I don’t know how to read either well] \\
    & \citep[7]{Laviera_2014}
\end{tabular}

Negative mental models are also seemingly legitimized by some of the scholarship, for example Zentella describes a type of codeswitching in Puerto Rico as “crutching” which “occurs when speakers are at a temporary loss for a word or construction in the language which they were speaking before the switch” \citep[49]{Zentella1982}. Two sub-classifications of such crutching, according to Zentella, are “Not Known” when speakers \emph{do not know} the target construction and “Lapse” when speakers \emph{do not recall} the target construction (50).\footnote{I acknowledge that the intention of this author, and many others who codify types of codeswitching, is not to 
stigmatize speakers but to identify linguistic triggers. However, the framework for discussion often implements a 
monolingual ideology by assuming that the speaker’s intent is to remain in one language throughout the speech act, when there is often little evidence to support such an assumption.} Given the negative associations and wording of scholarship, reinforced by an education system that promotes the separation of the two languages, it is perhaps not surprising to see how young people have adopted similar negative terms to explain their own language mixing choices, which they recognize as part of a more widespread “problem.” This is illustrated by comments in our data such as: “We just forget the word in english and say it in Spanish or vice versa”, and “we don’t remember... in the end we just correct what we are trying to say.” Although this paper does not suggest that the linguistic context for English among bilinguals in Puerto Rico is the same as regions where creoles are spoken, negative attitudes towards Puerto Rican English are comparable to attitudes among speakers of non-standard or non-official languages such as indigenous creoles. For example, \citet[50]{GarciaLeon_2013} found that, in comparison to the positive attitudes towards both Standard English and Standard Spanish, speakers of Trinidadian Creole considered that their Caribbean language was not apt for education or economic advancement although it was found to be a strong indicator of national and ethnic identity. Similarly, Puerto Ricans recognize that their local speech practices represent their identity but do not index intelligence or wealth. Consequently, positive orientation toward codeswitching as a marker of local identity appears to force a stigma of deficiency on speakers that, for some, becomes an associated marker of Puerto Rican identity. 

However, despite perceptions of individual or collective anxiety about language mixing, after participating in research, many students were able to associate more positive emotions with widespread Puerto Rican codeswitching practices. The same participant who had previously stated “I use English and Spanish all the time. I try not to do it on purpose” later explained how using both languages could be beneficial: “I like to take advantage of both languages... my brain is thinking and processing information in two languages.” And this comment reflects the increasing awareness among education scholars, especially in multilingual contexts around Europe, that translanguaging “promotes a deeper and fuller understanding of the subject matter” \citep[281]{Baker_2001}. The word choices of other comments indicated an increased positive association of language mixing, for example, describing it as “our original ‘mezcla’” and explaining local speech practices as “\emph{natural} code switching between English and Spanish” (emphasis added). Students made comments that show how language mixing was not a coping strategy or “crutch” but an integrated cognitive process, as illustrated by one student (the same participant who previously stated “I forget how to say a word in Spanish”) who commented: “My mind \emph{unconsciously combines} both languages to create a message \emph{without separating} both Spanish and English” (emphasis added). Furthermore, the social effects of such recognition had already become apparent to one participant who described a situation when a friend mixed codes creatively and committed what would be perceived as errors in Standard Spanish: “its funny because at first we, his friends, corrected him… but now we noticed that this is his way of speaking.” This recognition aligns with the observations of Pérez Casas that “speakers define and co-construct different identities through a CS[codeswitching] style…[and] Their bilingual speech style displays identities that reflect the reality of the two linguistic worlds that coexist in their habitus” \citep[58]{PerezCasas_2016}. Overall, participants recognized and attached positive value to what they recognized as “a new branch into variations of the Spanish\slash English language” and “a mix of social and culture interactions.”

After working on the documentation activities of the project, when asked what the phrase “Puerto Rican English” means, all participants gave responses had positive value associated with the term and many explicitly recognized the variety as a local dialect. One participant explained, “Puerto Rican English is a dialect variation of the English language spoken in Puerto Rico. It includes verbal particularities such as phonological or lexical features, and also non-verbal features, like body language or facial expressions.” Others recognized -- and rejected -- the stigma attached to the local variety of English, for example: “I believe that Puerto Rican English has and is becoming an [sic] important for the identity for many Puerto Ricans, and although it may [be] seen as an incorrect way is [of] speaking English it's not.” After documenting the effect of language contact on the local variety of English, many participants positively associated Puerto Rican identity with the effects of language contact; one participant explained, “Puerto Rican English to me is a variation of English that is influenced by Puerto Rican Spanish” and another explained, “this phrase [Puerto Rican English] means the combination of spanish and English.” Many participants recognized language mixing as an integral part of Puerto Rican English and one participant described codeswitching as “something that gives us pride.” Most students, after participating in documentation activities, were able to discuss Puerto Rican English in terms that positively indexed local identity rather than U.S. political or economic dependence, for example, one student explained how “the unique puerto rican people mold the English language to there [their] native culture and life style.” Many students were able to recognize non-standard features in positive ways, for example, one student explained, “it means giving another language the puerto rican flare and identity, which makes us stand out from the crowd” and another recognized, “Puerto Ricans have develop \emph{their own} features” (emphasis added). Furthermore, students were able to appreciate these non-standard features in terms of linguistic anthropology, “It's a dialect with distinctive features that evolved through historical points and represents our culture.” Not only were participants able to positively re-evaluate their own language practices, they demonstrated attitudes indicative of socio-linguistic activism, for example, one participant commented: “The English dialect used by the Puertorican people, one that is growing everyday and \emph{must be accepted} as a dialect of English like any other” (emphasis added). In sum, trend analysis demonstrates overwhelmingly that early-career scholars who participated in the documentation project demonstrated not only increased awareness of the characteristics of Puerto Rican speech but also increased recognition of Puerto Rican English as a local dialect that positively indexes the bilingual identity of its speakers. Furthermore, this positive effect emerges across the spectrum of bilingual participants, whether they described their competency in terms that would render them as incipient, receptive, or functional bilinguals, or whether they reported their abilities as equivalent to balanced, simultaneous bilinguals with native speaker or native-like competency.

The relevance of these exploratory findings, much like the language contact situation in Puerto Rico itself, does not easily compare with other situations in which new dialects of English have emerged. Yet, this documentation project and the positive effects on speaker perceptions it has engendered may benefit from contrast with studies on attitudes towards varieties of English that have emerged in similar socio-historical situations. For example, educated young Puerto Ricans, much like the Saudi Arabian students of the study by \citet[1047]{AlDosari2011}, value listener comprehension over perceptions of native fluency or prestige when evaluating the accents of second language speakers. The speakers of Puerto Rican English involved in the data-collection activities of this study appear similar to speakers of Indian English who display positive attitudes towards their own variety of English despite residual preference for the prestige variety associated with colonial history \citep{Bernaisch_Koch_2015}. In terms of usage, the complexities of language convergence and divergence between a local variety of English alongside Standard American English and Puerto Rican Spanish that Puerto Ricans experience could be compared to linguistic accommodation in the multilingual context of Singapore \citep{Ng_Cavallaro_Koh_2014}. \citet{Ruanni_Tupas_2016} explains how recognizing both the local variety of English in Singapore, in addition to an international standard, can promote attitudinal change that facilitates second language acquisition without compromising cultural identity. Perhaps, as studies and participant involvement in documentation activities continue to promote awareness of the Puerto Rican variety of English, the island might undergo steps towards additive bi-dialectalism in language classrooms that address perpetual issues of “non-standard” usage in such contexts, comparable to concerns in Singapore \citep{Tan_2008}. However, Puerto Rico has a unique language contact situation, unlike the multilingual environments of Singapore and India. \citet[17]{Pousada_2006} explains:

\begin{quote}
    Puerto Rico is distinct from other countries such as Singapore, Hong Kong, the Philippines, and India where English has been successfully implanted. Those countries are linguistically very heterogeneous and have acquired a local variety of English for diplomatic, commercial, and technological communication among diverse populations. For them, English is an ethnically neutral language that does not threaten their nationality and is utilized primarily as a lingua franca for pragmatic purposes. In contrast, in Puerto Rico, because of its historical domination by the United States, planning for improving English learning is often viewed with suspicion as an attempt to unseat Spanish which is the native language of almost all residents on the island.
\end{quote}

Thus, although comparative approaches with other contexts of new Englishes may be informative, they are unlikely to provide satisfactory models for Puerto Rican English, a variety trapped between two colonial giants on a binary continuum. Perhaps, instead of looking towards studies on World Englishes, comparative studies on Hispanic varieties of English in the U.S. might provide better context for the language contact situation in Puerto Rico. In their study of Hispanic English in the mid-Atlantic south, \citet{Wolfram_Carter_Moriello_2004} highlight the importance of examining the dynamic early stages of English dialect emergence in contact with a local variety of Spanish. Perhaps studies on Puerto Rican English, with its geographic and cultural distance from mainstream U.S. influence, might provide comparative insights that inform what we know about how Hispanic communities play a role in forming the symbolic role of regional varieties of English among both second language speakers and multilingual communities.


\section{Conclusions and recommendations}

The data presented in this paper serves to support the hypothesis that speaker involvement in generating a database of linguistic resources on Puerto Rican English can increase awareness of how the Spanish-dominated context of the island has given rise to a distinct variety of English. The eight methodologies that have been tested as part of this pilot project in addition to the experimental analysis that has been conducted on their data demonstrate that they are feasible strategies to begin documenting the characteristics of Puerto Rican English. The scope and meaningful involvement of participants and early-career scholars has demonstrated that there is interest in local variation. Furthermore, the effects of their involvement in documentation activities indicate that participants not only demonstrate increased recognition and positive evaluation of Puerto Rican English, but also recognize language mixing as an integral expression of Puerto Rican identity rather than a marker of linguistic deficiency. 

Exploratory research suggests that, without the opportunity to study their own language variation, young Puerto Ricans do not recognize that they speak a unique dialect of English. Furthermore, they continue to attach stigma to their own mixed language practices and often consider mixing languages in terms of how English transfer marks linguistic deficiency in Spanish. They are more likely to perceive code-switching as something only appropriate to informal contexts, thus restricting the potential to use translanguaging practices to maximum advantage in an academic environment. However, young Puerto Ricans also recognize that their mixed language practices are a key component of their bilingual identity and they typically mix languages in emotive contexts and in ways that bolster social unity and emotional health. In addition, perception surveys indicate that young Puerto Ricans are increasingly confident about their language abilities and reject attitudes that might stigmatize their accents and language mixing practices. Based on the findings of these pilot studies in documenting Puerto Rican English, a central recommendation of this paper encourages speakers, educators, community leaders and other stakeholders in Puerto Rican English to document the variety and advocate for its local and international recognition. Perhaps as we continue to document and discuss Puerto Rican English, more speakers might share the feelings of one participant who attended and gave feedback on a student presentation of exploratory findings:

\begin{quote}
After listening to this research presentation, I surely felt less insecure, more confident about my “accent.” We spend so much time trying to sound perfect when being understood is enough. I wasn’t really aware that Puerto Rican English was actually a thing, but I’m glad it is and that I know of it.
\end{quote}

\section*{Acknowledgements}
  This work is dedicated to the many participants and the documentation assistants of the University of Puerto Rico in Cayey who helped provide authentic language samples and local interpretations about their usage. Without them, none of this would have been possible.

%\section*{Contributions}
%John Doe contributed to conceptualization, methodology, and validation.
%Jane Doe contributed to writing of the original draft, review, and editing.

{\sloppy\printbibliography[heading=subbibliography,notkeyword=this]}

\end{document}
