\documentclass[output=paper,colorlinks,citecolor=brown]{langscibook}
\ChapterDOI{10.5281/zenodo.14282812}
\author{Angela Bartens\orcid{}\affiliation{University of Turku;University of Helsinki} and Kwaku Owusu Afriyie Osei-Tutu\affiliation{University of Ghana} and Tamirand Nnena De Lisser\affiliation{University of Guyana}}
%\ORCIDs{}

\title[Postulating Atlantic English Pidgin\slash Creole as a pluriareal language]
      {Postulating Atlantic English Pidgin\slash Creole as a pluriareal language: A perception study}

\abstract{This paper aims at verifying the observation that many speakers of closely related Atlantic English(-lexifier) Pidgin(s) and Creole(s), or EP/C(s), can understand each other. In order to test mutual intelligibility of different EP/Cs, an electronic survey was completed by 56 participants from Ghana, Guyana, and Nigeria. In addition, 20 interviews were conducted with informants representing eight different nationalities and residing in Guyana at the time. The data were also supplemented by a reading comprehension and translation task of short passages in eight Atlantic EP/Cs conducted in Ghana and Guyana. The results of the study indicate that intelligibility exists between most (but not all) language pairs. We therefore suggest that we are dealing with a case of pluriareality rather than pluricentricity, as the habitual criteria for the latter are not (yet) fulfilled. We believe that this can be accounted for by a postcolonial understanding of language variation, allowing for the postulating of a language system with fluid borders for use in wider contexts, including writing. Nevertheless, the postcolonial aspect shall be addressed in more detail in a posterior study.}


\IfFileExists{../localcommands.tex}{
   \addbibresource{../localbibliography.bib}
   \usepackage{langsci-optional}
\usepackage{langsci-gb4e}
\usepackage{langsci-lgr}

\usepackage{listings}
\lstset{basicstyle=\ttfamily,tabsize=2,breaklines=true}

%added by author
% \usepackage{tipa}
\usepackage{multirow}
\graphicspath{{figures/}}
\usepackage{langsci-branding}

   
\newcommand{\sent}{\enumsentence}
\newcommand{\sents}{\eenumsentence}
\let\citeasnoun\citet

\renewcommand{\lsCoverTitleFont}[1]{\sffamily\addfontfeatures{Scale=MatchUppercase}\fontsize{44pt}{16mm}\selectfont #1}
  
   %% hyphenation points for line breaks
%% Normally, automatic hyphenation in LaTeX is very good
%% If a word is mis-hyphenated, add it to this file
%%
%% add information to TeX file before \begin{document} with:
%% %% hyphenation points for line breaks
%% Normally, automatic hyphenation in LaTeX is very good
%% If a word is mis-hyphenated, add it to this file
%%
%% add information to TeX file before \begin{document} with:
%% %% hyphenation points for line breaks
%% Normally, automatic hyphenation in LaTeX is very good
%% If a word is mis-hyphenated, add it to this file
%%
%% add information to TeX file before \begin{document} with:
%% \include{localhyphenation}
\hyphenation{
affri-ca-te
affri-ca-tes
an-no-tated
com-ple-ments
com-po-si-tio-na-li-ty
non-com-po-si-tio-na-li-ty
Gon-zá-lez
out-side
Ri-chárd
se-man-tics
STREU-SLE
Tie-de-mann
}
\hyphenation{
affri-ca-te
affri-ca-tes
an-no-tated
com-ple-ments
com-po-si-tio-na-li-ty
non-com-po-si-tio-na-li-ty
Gon-zá-lez
out-side
Ri-chárd
se-man-tics
STREU-SLE
Tie-de-mann
}
\hyphenation{
affri-ca-te
affri-ca-tes
an-no-tated
com-ple-ments
com-po-si-tio-na-li-ty
non-com-po-si-tio-na-li-ty
Gon-zá-lez
out-side
Ri-chárd
se-man-tics
STREU-SLE
Tie-de-mann
}
   \boolfalse{bookcompile}
   \togglepaper[4]%%chapternumber
}{}


\begin{document}
\maketitle

%-------------- the text goes here

\section{Introduction}\label{sec:04:1}

It is almost taken for granted that speakers of many, albeit not all, Atlantic English-lexifier Pidgins and Creoles, can understand each other and are aware of the linguistic similarities.\footnote{The Suriname creoles are notoriously different from the majority of the others.} We aim to go beyond this impressionistic observation by using, for the time being, three distinct verification methods: an online survey, interviews, and a reading comprehension and translation task for checking the plausibility of this observation.

As far as terminology is concerned, an abundance of definitions for the terms PIDGIN and CREOLE exists, highlighting the differences between what are often conceived of as distinct types of languages as far as complexity and range of use are concerned \citep[cf.][6]{Bakker_Matras_2013}. However, linguistic reality is more multifaceted and, without proposing to return to the discussion regarding the “life cycle” of pidgins \citep{Hall_1962} or the complexity of Creoles (see, for example, \cite{McWhorter_2011}), it is evident that some pidgins become the dominant or native languages of their speakers\footnote{Instead of “native”, we prefer “dominant” language as this language or language may change over the lifetime of a person \citep[431]{Chernobilsky_2008}.},  resulting in what were initially called “expanded pidgins” and later on “pidgincreoles”, a term attributed to Philip Baker \citep[113]{Bakker_2008}. Nigerian Pidgin or Naija, one of the languages and speech communities studied here, is a case in point: widely spoken as a second language, it has a growing number of native speakers, especially in some areas such as the Warri-Sapele area in Delta State (cf. \cite{Deuber_2005}: 1, 3). For our purposes, attributing language varieties to such categories is irrelevant. Rather, we used as inclusive a denomination as possible so that informants\footnote{We use the now-controversial term “informants” instead of, for example, “consultants” employed by, e.g., \citet[125]{Bowern_2008}, for the sake of convenience; and, more importantly, for covering distinct functions in the data gathering process.} could identify with a large range of such languages and therefore chose “English(-lexifier) Pidgin(s) and Creole(s)”, henceforth EP/C(s).\footnote{For the sake of clarity, we employ “EP/C” for one variety or language, “EP/Cs” when more than one is referred to, and “EP/C(s)” when applicable to one or more varieties/languages.} 

We aim at testing mutual intelligibility. Intelligibility can be defined as “the degree to which a spoken message can be understood” (Lingualinks cited by \cite{Bouwer_2007}: 257), and is determined by such criteria as “rate of speech, intonation, location of pauses, topic familiarity, discourse structure, grammatical complexity, and hearer’s attitude toward the text or speaker” (ibid.). As we shall see, some of these are highly relevant for our informants. It can be argued that intelligibility is essentially about social relationships, rather than linguistic relationships between varieties \citep[14]{Romaine_1994}. This means that self-reporting of intelligibility can be highly subjective, as it is partly dependent on collective language attitudes. As a result, self-report may result in a distorted appreciation of intelligibility and language practices (see also \sectref{sec:04:4}). Exposure to the other speech varieties leads to what can be termed `acquired intelligibility', whereas `inherent intelligibility' stems from a genetic relationship between distinct varieties, although this dichotomy has at times been considered problematic \citep[257--258]{Bouwer_2007}.

We now turn to PLURICENTRIC languages, a concept which we need to elucidate in order to explain why we have opted for referring to Atlantic EP/Cs as PLURIAREAL languages \citep{Scheuringer_1996} in the (re)making. \citet[e.g.,][66]{Kloss_1978} introduced the term “pluricentric” for standard languages with several varieties with equal distribution as, for example, official and administrative languages in distinct independent countries, e.g., Portugal and Brazil, as well as other Portuguese ex-colonies.\footnote{When compared to, e.g., English, the formation of standard varieties is still very much in progress, cf. \citet{Buque_etal_2008}, \citet{Sobrinho_2009}, \citet{Goncalves_2013}.} Several criteria have to be fulfilled for a language to be considered pluricentric. These are, essentially, the following: 

\begin{itemize}
\item The language occurs in usually several but at least two nations that function as “interacting centers”.
\item Official status at state (sole official or co-official language) or regional level, allowing for the setting of norms through these interacting centers.
\item Linguistic distance (abstand as defined by \cite{Kloss_1967}) that distinguishes it from other languages and may serve as a symbol of linguistic and cultural identity.
\item Acceptance of the pluricentric nature but also the perceived relevance of the 	national/regional variety for specific communities.
\item Codification completed or in progress coupled with dissemination by institutions and model speakers, e.g., through formal education. (cf. \cite{Muhr_2012}: 20; \cite{Clyne_1992}: 1).
\end{itemize}

The concept of pluricentricity is, to some extent, problematic. Despite endorsing the existence of varieties, for example, within the Francophonie; for mostly historical reasons only one is seen as (genuinely) legitimate; distinctions may be political and not linguistic; variants cross national borders; and there is no “pure” variety \citep[cf.][383]{Vigouroux2013}. The criteria most problematic for Atlantic EP/C to be perceived of as a pluricentric language are official status and codification, necessarily having an impact upon the acceptance of its pluricentricity as well as the creation of abstand through ausbau in the terminology established by \citet{Kloss_1967}, which is frequent in standardization and codification. Indeed, “[p]luricentricity refers to the development of multiple standards, often national standards of a given language, while ‘pluri-areality’ downplays if not negates, any national level” \citep[7]{Dollinger_2019}. 

It could be argued that “[p]luriareality merely says there is regional variation” \citep[7]{Dollinger_2019}, thence making it a basically useless concept for some scholars. Nevertheless, it is employed by others who wish to focus on linguistic differences in these language forms independent of national and political borders \citep{Niehaus_2015,ElspaB_Durscheid_Ziegler_2017}. As in the case of the pidgin-creole divide, we have opted for “pluriareality” for what we consider relative neutrality especially \emph{vis-à-vis} official status and codification.  

Our study is based on the existence of a great number of shared structures of Atlantic EP/C(s) (\cite{Alleyne_1980,Holm_1988_1989,Holm_Patrick_2007,Michaelis_Maurer_Haspelmath_Huber_2013,Muhleisen_2018}, etc.).\footnote{Frequently cited examples of shared Atlantic EP/C features would be the equative copula \emph{da/na/a}, the locative copula \emph{de}, modal \emph{fi/fo/fu/fa}, the 2\textsc{pl} pronoun \emph{unu}, the \textsc{ant} marker \emph{bin}, and adverbial \emph{se(l)f} `even’ \citep{McWhorter_1995}.} As a matter of fact, it is possible to go beyond the Atlantic region and actually speak of worldwide features of EP/C(s). \citet{Faraclas_Corum_Arrindell_etal_2019}, building on previous work, compile a list of 153 features. At least in the Atlantic region, these shared features bolster the observation that mutual intelligibility between EP/Cs exists. They are attributed to possible common origin as well as a physical, cultural, and an increasingly virtual diaspora \citep[cf.][]{Hancock_1986,Mair_Muhleisen_Pierker_2015,McWhorter_1996}.\footnote{Critique of the common origin scenario, called the “Domestic Hypothesis” by \citet{Hancock_1986} has been formulated by, for example, \citet{Baker_1999} and \citet{Huber_1999a} based on historical records.} 

Finally, we are interested in whether this could eventually lead to the “making” or (re)construction of a language system with fluid borders for use in wider contexts, understood here in terms of \citet{Makoni_2005} on the disinvention and subsequent reconstitution of “language”. This may be a conscious or unconscious process where language awareness plays a significant role in the creation of imagined units with clear-cut boundaries, labels, names, and neatly defined norms established by distinct agents through at times overlapping mechanisms (both top-down and bottom-up; cf. \cite{Huning_Kramer_2018}). The question arises as to how language is deployed in advancing social agendas and in constructing identities in specific communities of practice and praxis. After all, in the case of Atlantic EP/Cs, we are potentially dealing with a community of 130,000,000 speakers \citep{Faraclas_2020}. 

In the following section, we shall present the details of our methodology followed by the results of our online survey, interviews, and reading comprehension and translation task in \sectref{sec:04:3}.

\section{Methodology}\label{sec:04:2}

This paper relies on a multi-method approach, integrating online surveys and face-to-face interviews, which employs self-reported data, combined with a more objective, experimental reading comprehension and translation task. While self-reporting is a great tool to collect individual assumed beliefs and attitudes relatively quickly, it has its limitations. Responses may be biased based on one’s perceptions, and subject to societal expectations. To mitigate the aforementioned limitations, the experimental comprehension and translation task was employed to test actual linguistic competences. Nonetheless, the paper is largely based on descriptions of language attitudes, perceptions, and intelligibility, rather than on grammatical analyses accounting for the intelligibility or lack thereof. Details of the three methods are provided in Sections~\ref{sec:4:2.1}--\ref{sec:4.2.3}.

\subsection{The online survey}\label{sec:4:2.1}

Our online survey was conducted from June 15, 2019 until October 9, 2019, via the platform purdue.ca1.qualtrics.com.\footnote{The survey was available at: \href{https://purdue.ca1.qualtrics.com/jfe/form/SV_5cnFhaMpzwc2zKB}{https://purdue.ca1.qualtrics.com/form/SV\_5cnFhaMpzwc2zKB}, accessed November 12, 2019.} A total of 56 informants were included in the analysis, with the breakdown by nationality and gender as outlined in \figref{fig:fig1_04}. Three nationalities (Ghanaian, Guyanese, and Nigerian) were represented, 24 of the informants being male and 32 female,\footnote{The category “other” now commonly used in sociolinguistic research was excluded from this survey for the same pragmatic reasons as, for example, using the label “EP/C”.}  with their ages ranging from 18 to above 50. An informant who identified as Ghanaian-British was grouped with the Ghanaians for sake of clarity and for the reason that her answers did not stand out in any way from the answers of the Ghanaians. When processing the results, informants’ answers on their own variety were eliminated.

% Figure 1 here (graphic):
\begin{figure}
    \includegraphics[height=0.3\textheight]{figures/fig1_04.png}
    \caption{Breakdown of participants in the electronic survey}
    \label{fig:fig1_04}
\end{figure}

The informants were presented with audio samples of Ghanaian, Nigerian, Jamaican, and Sranan EP/C. The Ghanaian sample is from a TV show (\emph{Things We Do for Love}) that ran in the early 2000s and the selected scene features a humorous conversation between two men, one of whom had come to borrow money and clothes in order to take a girl out on a date. The Nigerian clip is about two men fighting over a telephone recharge card after finding out the last person to recharge it had won a substantial amount of money. The Jamaican one is an interview taken from a news interview of a female Kingston resident voicing her opinion and pleading for justice after her house had been flooded the night before. Finally, the Sranan clip retells the biblical story of Noah’s ark. All four samples were taken from YouTube.\footnote{Ghanaian sample: \url{https://www.youtube.com/watch?v=bC9I79DOmC8}, last accessed 7-28-2020; Nigerian sample:  \url{https://www.youtube.com/watch?v=D69cixGHRIw}, last accessed 7-28-2020; Jamaican sample: \url{https://youtu.be/J6uwQ-HmgD8}, last accessed 7-28-2020; Sranan sample: \url{https://www.youtube.com/watch?v=BmVFzVUAr88}, no longer available 7-28-2020.} 

In addition to the results discussed in Sections 3.1–3.3, informants were also asked about their attitudes towards their own variety of EP/C and those of other varieties they were familiar with. However, these results are dealt with in detail in the discussion section (i.e., \sectref{sec:04:4}).

\subsection{The interviews}\label{sec:4.2.2}

In order to supplement the data gathered by means of the survey, 20 interviews were conducted in Guyana in October 2019, with informants representing eight different nationalities and residing in the country at the time. The breakdown by nationality and gender is outlined in \figref{fig:fig2_04}. The sample was balanced as far as gender is concerned: There were ten female and ten male informants.

% Figure 2 here (graphic):
\begin{figure}
    \includegraphics[height=0.3\textheight]{figures/fig2_04.png}
    \caption{Breakdown of participants in the interviews}
    \label{fig:fig2_04}
\end{figure}

The age range of the informants was from 31 years to over 61 years. The informants had lived in Guyana from less than one year up to over 10 years. Informants were selected based on convenience sampling, and were asked about their abilities to understand Guyanese Creole, specific areas of ease and difficulty, examples of differences between varieties, etc. The interviews also collected perception data on whether different varieties of EP/Cs can share a common writing system.


\subsection{The reading comprehension and translation task}\label{sec:4.2.3}

The reading comprehension and translation task was conducted in December 2020. Short passages were presented in eight EP/Cs: Ghanaian Pidgin, Guyanese Creole, Jamaican, Krio, Nicaraguan Creole, Nigerian Pidgin, San Andr\'es Creole, and Sranan. A total of 34 informants were included in the analysis, with the breakdown by nationality and gender as outlined in \figref{fig:fig3_04}. A total of 24 of the informants were Guyanese and 10 Ghanaians. All informants were university students, age ranging from 18 to 50 years.

% Figure 3 here (graphic):
\begin{figure}
    \includegraphics[height=0.3\textheight]{figures/fig3_04.png}
    \caption{Breakdown of participants in the reading comprehension and translation task}
    \label{fig:fig3_04}
\end{figure}

The informants were required to identify the languages, say what the short excerpts were about and rewrite the first two lines of each excerpt in English. Additionally, informants were tasked with indicating words which they would have written differently in their language, words which were found to be difficult to understand, and rating the relative ease or difficulty in understanding the paragraph. Finally, a judgement call was made on whether they liked the writing system or not. 

We now turn to the results of the distinct samples.

\section{Results}\label{sec:04:3}
In this section, we detail the results from the online survey (\sectref{sec:04:3.1}), the interviews (\sectref{sec:04:3.2}), and the reading comprehension and translation task (\sectref{sec:04:3.3}).

\subsection{The online survey}\label{sec:04:3.1}

Here we provide an analysis of the data collected from the Ghanaian, Guyanese, and Nigerian participants on the Ghanaian, Nigerian, Jamaican, and Sranan samples.

\subsubsection{Ghanaian sample}

The breakdown of the results we obtained when we asked the informants to identify Ghanaian Pidgin can be seen in \figref{fig:fig4_04}.

% Figure 4 here (graphic):
\begin{figure}
    \includegraphics[height=0.3\textheight]{figures/fig4_04.png}
    \caption{Variety identified by informants for the Ghanaian sample}
    \label{fig:fig4_04}
\end{figure}

In their answers, the majority of the Guyanese (11/18) picked Africa, while six selected Nigeria and one indicated that the speakers were from either Nigeria or Ghana. The majority of Nigerians (10/17) understandably picked Ghana as they are in closer contact with the variety. Of the remaining 7 Nigerian informants, 2 cited both Ghana and Nigeria; 1 could not decide between the two countries; 1 was certain that the speakers were from Nigeria; and 3 could not specify which country the Ghanaian speakers were from (with 2 selecting West Africa, while 1 simply picked Africa).

The informants were also asked how difficult or easy it was for them to understand the speakers of the sample.

% Figure 5 here (graphic):
\begin{figure}
    \includegraphics[height=0.3\textheight]{figures/fig5_04.png}
    \caption{Difficulty of understanding Ghanaian speakers}
    \label{fig:fig5_04}
\end{figure}

As can be seen from \figref{fig:fig5_04}, as expected, the Guyanese informants had more difficulty in understanding the Ghanaian speakers than the Nigerians did. More precisely, the majority of the former (14/18) found the sample to be difficult (extremely or moderately so). There were also a number of Nigerian informants (6/17), including the one who said the sample was either Ghanaian or Nigerian, who found the speech moderately difficult to understand. Of the four Nigerians who found the sample neither easy nor difficult to understand, two had earlier identified it as coming from both Ghana and Nigeria, one had said Nigeria, while one had identified it as Ghanaian.

Particular difficulties cited were constituted by what the informants identified as “indigenous words”. One such word which was frequently mentioned was \emph{charley} (spelt variably as \emph{chale/charlie}), which is typically one of the most noticeable vocabulary items to those who hear Ghanaian EP/C for the first time, and whose function, according to \citet[276]{Huber_1999b}, is “to keep the channel of communication open”. Other vocabulary items which posed a challenge to the informants were (understandably) slang items (such as \emph{dog-gone} ‘stupid’) and personal names (such as \emph{Marsha}) that were thought to be content words. In addition to stating what they found difficult to understand, informants were also asked to provide alternative ways of saying (in their own EP/C) some of the things they had understood from the audio. Here also, most of the informants noted vocabulary differences between their own variety and Ghanaian EP/C; for example, one of the Nigerians said he would use \emph{chop} ‘eat’, instead of \emph{munch}, which was used by one of the Ghanaian speakers. One of the Guyanese informants said, instead of \emph{mek you no mind am} ‘don’t mind him’, she would say \emph{na bada wit he}. When discussing the Ghanaian audio, a Guyanese informant mentioned \emph{jiggaboo} and \emph{kangalang}, both essentially meaning ‘a stupid person’, the latter also meaning ‘hooligan, ruffian, an undisciplined person’. Nevertheless, these lexical items are not widely known to Ghanaian speakers; they would be more likely to use \emph{dog-gone} or another slang item (for example, \emph{john}) which means the same thing. 


\subsubsection{The Naija sample}


As far as the Nigerian Pidgin or Naija sample is concerned, the same questions were asked.

% Figure 6 here (graphic):
\begin{figure}
    \includegraphics[height=0.3\textheight]{figures/fig6_04.png}
    \caption{Variety identified by informants for the Nigerian sample}
    \label{fig:fig6_04}
\end{figure}

The majority of Ghanaians (20/21) identified the variety as from Nigeria, with only one indicating that he thought the speakers were from Ghana. Again, this is expected given the geographical and cultural proximity. The majority of Guyanese (12/18) identified the variety as from somewhere in Africa but five of them also specifically cited Nigeria, something that was not the case with the Ghanaian sample. Indeed, even the single undecided Guyanese informant indicated that the speakers could be Nigerian (or Ghanaian). It can be hypothesized that this is a result of Guyanese speakers being more exposed to Nigerian than Ghanaian popular culture, especially Nollywood film productions. This is also reflected in the results presented as \figref{fig:fig7_04}. “Moderately easy to understand” was the most common answer for all groups: 14/21 of the Ghanaians and 9/18 of the Guyanese.

% Figure 7 here (graphic):
\begin{figure}
    \includegraphics[height=0.3\textheight]{figures/fig7_04.png}
    \caption{Difficulty of understanding Nigerian speakers}
    \label{fig:fig7_04}
\end{figure}

Among the special difficulties, informants cited pronunciation and rapidity of speech as well as indigenous words, e.g., \emph{oya} ‘all right, okay’ (cf. Yoruba \emph{oya jeka lo} ‘all right, let’s go’), which is very noticeable because of its role in discourse and frequency of occurrence. This is distinct from the Ghanaian sample where the main difficulty for informants was constituted by the lexicon. With regard to what informants would say differently in their own varieties, responses revealed (as expected) regional differences, often based on adstrate languages. For example, one of the Ghanaian informants pointed out that instead of the Nigerian Pidgin \emph{na im} ‘he’s the one’, he would say \emph{ibi him}. This is consistent with \citegen[92]{Huber_1999b} observation that the equative copular \emph{na} is noticeably absent from Ghanaian Pidgin, whereas it is found in the other West African EP/Cs. For many Ghanaians it is indeed an emblematic marker of Naija, the variety they are most familiar with (cf. \sectref{sec:04:4}), alongside with the completive marker \emph{don(e)}. Another example, this one from a Guyanese informant, is saying \emph{yuh battry dead already} instead of \emph{battery done die}, which was used by one of the Nigerian Pidgin speakers in the audio file. However, Guyanese actually also employ \emph{don} in this context.

\subsubsection{The Jamaican sample}\label{sec:04:3.1.3}

As can be seen from \figref{fig:fig8_04}, the sample was surprisingly well identified by most informants (18/18 Guyanese, 15/21 Ghanaians and 9/18 Nigerians); with the majority of those who could not place the variety as Jamaican, still identifying it as from the Caribbean. This must be due to the fact that Jamaican speech and culture are known and imitated worldwide, albeit at times in a stylized manner, especially as a result of the spread of reggae (cf. \cite{Tomei_Hollington_2018}; \cite{Moll_2017}: 72), and may therefore stand for Caribbean varieties for speakers from outside of the region, e.g., West Africa (see \sectref{sec:04:4}).

% Figure 8 here (graphic):
\begin{figure}
    \includegraphics[height=0.3\textheight]{figures/fig8_04.png}
    \caption{Variety identified by informants for the Jamaican sample}
    \label{fig:fig8_04}
\end{figure}

When considering the ease of understanding Jamaican, Guyanese consultants understandably scored the highest as can be gleaned from \figref{fig:fig9_04}. The majority of them (14/18) found Jamaican extremely easy to understand, while the rest found it moderately easy (3/18) or neither easy nor difficult (1/18). Globally speaking, speakers of Naija appeared to have less difficulty than Ghanaians, which may point into the direction that the influence of Jamaican (pop) culture is more widespread in Nigeria than Ghana. More precisely, the sample was judged moderately difficult (11/21) or extremely difficult (6/21) to understand by the majority of Ghanaians; whereas a whole range of answers was given by Nigerian informants, “moderately easy” being the most frequent (6/17), followed by “moderately difficult” (5/17).

% Figure 9 here (graphic):
\begin{figure}
    \includegraphics[height=0.3\textheight]{figures/fig9_04.png}
    \caption{Difficulty of understanding Jamaican speakers}
    \label{fig:fig9_04}
\end{figure}

Among the difficulties mentioned in this case, we find again vocabulary as well as pronunciation, e.g., \emph{wombo} < \emph{waahn bak} ‘want back’, and phrasal expressions, e.g., \emph{kom ina di yaad} ‘enter/visit the yard’. Indeed, the majority of those who found the sample difficult to understand stated that they were unable to write down specific words or phrases (possibly because they could not isolate them from the stream of words; this also applies to both examples given above). Regarding what informants would say differently in their variety, the main difference appeared to be vocabulary items such as one of the Guyanese informants pointing out that “they say \emph{ina} and we say \emph{in}” and a Nigerian saying he would use \emph{for area} instead of \emph{ina di yaad}. 

\subsubsection{The Sranan sample}

As suggested in \sectref{sec:04:1}, Sranan (also called Sranantongo) was likely to be the EP/C most difficult to identify and understand. This hypothesis is borne out by the results we present in Figures \ref{fig:fig10_04} and \ref{fig:fig11_04}.

% Figure 10 here (graphic):
\begin{figure}[t]
    \includegraphics[height=0.3\textheight]{figures/fig10_04.png}
    \caption{Variety identified by informants for the Sranan sample}
    \label{fig:fig10_04}
\end{figure}

Clearly, the variety was very difficult to identify – as witnessed by answers such as “India” (three Guyanese, one Ghanaian, and one Nigerian informant), “Some Island”, “Haiti”, etc. The category “others” includes the following responses: “Cape Verde''; ``China''; ``Guyana''; ``Liberia''; ``Papua New Guinea''; ``Some part of Mexico''; ``South Africa”. Suriname was mentioned by two Guyanese who must be familiar with Sranan due to geographic, cultural, or historical proximity. As far as the difficulty of understanding Sranan is concerned, we find the following scenario where only one Guyanese, who was among the two previously cited informants, found the sample moderately easy to understand.

% Figure 11 here (graphic):
\begin{figure}[t]
    \includegraphics[height=0.3\textheight]{figures/fig11_04.png}
    \caption{Difficulty of understanding Sranan speakers}
    \label{fig:fig11_04}
\end{figure}

When asked about specific difficulties, one (Ghanaian) informant mentioned the lexical item \emph{boato} ‘boat’, but the rest of the informants typically indicated that everything that was said was difficult. 

\subsection{The interviews}\label{sec:04:3.2}

Results from the interviews revealed that a total of 11 informants said they understood Guyanese Creole (also called Creolese) at arrival and 18 at the time of conducting the interviews. The time range mentioned by those who have acquired passive knowledge (or receptive competence, in the terminology of \cite{Romaine_1994}: 23) of Guyanese Creole was cited as from two months to three years. Nevertheless, the question on the length of acquisition was open and answered subjectively (see also \sectref{sec:04:1}). For example, a Nigerian informant stated he had acquired passive knowledge in two months after arriving in Guyana, while an informant from Suriname, despite being in Guyana for over 10 years, stated that she still did not understand Guyanese Creole. This appreciation of one’s receptive competence (“very difficult to understand”; see \sectref{sec:04:3.2}.1) was shared by another female informant from Jamaica who had also resided in the country for over ten years. Could this be linked to the finding that in western-oriented societies, female speakers tend to be more conscious of linguistic norms and differences in register (cf. \cite{Labov_1972}: 243; \citeyear{Labov_2001}: 266; \cite{Lakoff_1973}: 48)?\footnote{See \citet{Keenan_1989} on Malagasy speakers for a counterexample of the Western Gender Paradox \citep[292--293]{Labov_2001}.}

\subsubsection{The ease/difficulty of understanding Guyanese}

\figref{fig:fig12_04} depicts the relative difficulty experienced by the interviewees in understanding Guyanese Creole on a scale ranging from 1 (very easy) to 5 (very difficult). One informant chose not to comment on this matter.

% Figure 12 here (graphic):
\begin{figure}
    \includegraphics[width=\textwidth]{figures/fig12_04.png}
    \caption{Relative ease/difficulty of understanding Guyanese Creole}
    \label{fig:fig12_04}
\end{figure}

It can be seen that 50\% of the informants (10/20) reported that Guyanese Creole was easy or very easy to understand, while five informants were neutral, finding it neither easy nor difficult to understand Guyanese Creole. It may be of interest to note that both informants who found Guyanese Creole very difficult to understand are females who have been living in Guyana for over 10 years. The difficulties mentioned included specific vocabulary items and pronunciation. Among the former, for example the following were cited:  \emph{ai paas} ‘disrespect’, \emph{gyaaf}  ‘chat’,  \emph{jos nou} ‘soon’, \emph{bok maan} ‘Amerindian’, \emph{balaanjii} ‘eggplant’, \emph{oplif} ‘collect’, \emph{push yo bodii} ‘persevere’, \emph{yuuzin} (at restaurant) ‘having/eating in’,  \emph{tek out} (a photo) ‘take’, and a range of names for various fruits and vegetables. For pronunciation, informants mentioned lengthened vowels as in \emph{maan} ‘man’, insertions as in \emph{giyorl} ‘girl’, elision as in \emph{wam} ‘what’s happening’, and r-fullness as in \emph{korna} ‘corner’.\footnote{Among the Caribbean EP/Cs, rhoticity is typical of Bajan but reported by \citet[570]{Wells_1982} for Guyanese as well.}  As in the case of the Naija sample mentioned in 3.1.2 above, rapidity of speech was also mentioned as a factor making it relatively difficult to understand Guyanese.

On the other hand, there were also linguistic features that were identified as helpful in understanding Guyanese, particularly words and structures similar to other EP/Cs, such as \emph{gyal} to mean ‘girl’, the use of \emph{dem} as a plural marker, invariability of the 3\textsuperscript{rd} person object and subject pronouns, and general English-based lexical items. Additionally, the informants tend to rely on the context of utterance to decipher seemingly unfamiliar terms or structures. It was also reported by most informants (13/20) that speakers of Guyanese Creole would understand them when they use their variety of EP/C.

\subsubsection{Language attitudes manifest in the interviews}

Language attitudes can be divided into three components: 1. cognitive or knowledge; 2. affective or evaluative; and 3. conative or action \citep[139]{Agheyisi_Fishman_1970}. Our online survey and the issues raised in the interviews discussed in the previous section essentially deal with the first component. The third is not for us as authors but eventually for the speakers to decide; we nonetheless hope to make some recommendations. In the following part of the analysis of the interviews conducted in Guyana, we deal with the affective or evaluative aspect expressed by the interviewees. To a certain extent, this was also included in the survey and the reading comprehension and translation task (see \sectref{sec:04:4}).

A total of 15 out of the 20 informants reported that they like Guyanese Creole, whereas 5 reported a neutral attitude towards it. Among the answers, we find the following observations:

\begin{enumerate}
    \item It is nice to understand other languages and cultures. 
    \item It is not very different from my language.
    \item It is funny, sounds nice, interesting, unique, different.
    \item It is not bad.
    \item It is not a matter of like or dislike, I accept it.
\end{enumerate}

When asked whether they liked their own variety of Atlantic EP/C, 18 answered affirmatively whereas two did not respond. Arguments in favor of one’s own variety were:

\begin{enumerate}
    \item It creates identity, unity.
    \item It is great for communication.
    \item It is what I grew up with, and it preserves my culture.
    \item It is expressive, versatile, colorful, sentimental.
    \item It makes me unique.
\end{enumerate}

We were also particularly interested in the feasibility of a common writing system for Atlantic EP/C as a manifestation of its pluriareal character and part of (future) conative language attitudes. A total of 13 informants were favorable to the idea, 7 were not. Arguments cited in favor of a common writing system were that it would be practical for sharing literature and music, the linguistic background is similar, it could contribute to community cohesion (probably in the sense of the `imagined communities' proposed by \cite{Anderson_1983}), and the fact that writing is already similar (the specific example of tweets was cited, something we will be looking at in the future). The informants not favorable to a common writing system stated it would be a challenging task to complete because of differences between distinct varieties of Atlantic EP/C and expressed their fear of losing cultural diversity and identity.\footnote{Note that this fear exists even in communities deemed to speak one creole language such as Nicaraguan EP/C \citep[cf.][]{Freeland_2004}.}  It can be concluded that identity-related factors were cited both in favor and against the proposal of a common writing system.

\subsection{Reading comprehension and translation task}\label{sec:04:3.3}

In this section we present the results of the Jamaican extract as a sample of the findings from the eight EP/Cs as gathered from our Guyanese and Ghanaian informants. A more detailed analysis is presented in another study \citep{Bartens_inpreparation}. Jamaican was chosen to exemplify the results of the reading comprehension task because of its worldwide diffusion (see Sections 3.1.3, 4, \& 5). In addition, Jamaican is of particular interest for our purposes as the writing system is fairly phonemic (see, for instance, the examples below) and quite well established \citep[see][]{Cassidy_LePage_1967}.

As detailed in \figref{fig:fig13_04}, 90\% of the Guyanese informants were able to correctly identify the Jamaican text. This proved to be a bit more difficult for the Ghanaians as only 50\% were able to correctly identify the language.

% Figure 13 here (graphic):
\begin{figure}
    \includegraphics[height=0.3\textheight]{figures/fig13_04.png}
    \caption{Identification of the Jamaican excerpt}
    \label{fig:fig13_04}
\end{figure}

While the Guyanese demonstrated that their understanding and translations (67\% and 50\%, respectively) were highly acceptable or acceptable, for the Ghanaians, the results revealed the contrary. As shown in \figref{fig:fig14_04}, 60\% of the Ghanaians did not respond to these two questions, and of the other 40\%, the response from 30\% of the informants was not acceptable. This shows that the Ghanaians did not understand the Jamaican excerpt.

% Figure 14 here (graphic):
\begin{figure}
    \includegraphics[height=0.25\textheight]{figures/fig14_04.png}
    \caption{Understanding and translating the Jamaican excerpt}
    \label{fig:fig14_04}
\end{figure}

Half of the Ghanaians, as detailed in \figref{fig:fig15_04}, reported that the Jamaican data was very difficult to comprehend, while 63\% of the Guyanese were neutral or indicated that it was easy to understand.

% Figure 15 here (graphic):
\begin{figure}
    \includegraphics[height=0.25\textheight]{figures/fig15_04.png}
    \caption{Difficulty of understanding the Jamaican excerpt}
    \label{fig:fig15_04}
\end{figure}

% Figure 16 here (graphic):
\begin{figure}
    \includegraphics[height=0.25\textheight]{figures/fig16_04.png}
    \caption{Liking the writing system used in the Jamaican excerpt}
    \label{fig:fig16_04}
\end{figure}

As a result, it is not surprising to note (\figref{fig:fig16_04}) that 70\% of the Ghanaians, as compared to only 33\% of the Guyanese, reported not liking the writing system.


Examples of Jamaican words which are different or are written differently across languages were the third person singular pronoun \emph{im} in Jamaican which is reflected as \emph{ii} in Guyanese Creole; the lexical verb ‘eat’ which is \emph{nyam} in Jamaican, but Guyanese informants suggested \emph{iit}, and also ‘little’ which is \emph{likl} in Jamaican but \emph{lil} and \emph{little} in Guyanese and Ghanaian, respectively. The Ghanaians also reported the English spelling for ‘tree’ and ‘try’ rather than the Jamaican \emph{chrii} and \emph{chrai}. The words \emph{eleva} ‘hell of a’, \emph{guda} ‘probably’, and \emph{kies} ‘case’ are examples of Jamaican words that were difficult for the Guyanese informants, while \emph{nyam} ‘eat’, \emph{Alis} ‘Alice (name of someone)’ and \emph{chrech} ‘stretch’ were identified as difficult for the Ghanaians.


\section{Discussion}\label{sec:04:4}

Based on the literature on Atlantic EP/Cs, it is no surprise that Sranan was found to be the most difficult variety to identify and understand. Familiarity with a variety, either as a result of geographic proximity or cultural influence, played a major role in the evaluations made by our informants. Here the online survey results stand out as it is more difficult to estimate the real contact informants have with a given EP/C as opposed to the interviews: self-report, especially in the form of written/online surveys, can notoriously distort the big picture (cf. \cite{Codo_2008}: 171; see also \sectref{sec:04:1}).\footnote{Survey answers also quite typically feature inconsistencies, which would be detected in, for instance, ethnographically oriented fieldwork. For example, one Ghanaian informant initially did not list “Broken English” (i.e., Ghanaian EP/C) as one of the languages he speaks but stated in subsequent affirmations that he actually speaks and likes it.} The findings from these were therefore checked against the more objective reading comprehension and translation task. The accuracy of the results gained from the self-reports is validated, as far as possible, by the actual competency demonstrated in the comprehension and translation task. The fact that Sranan was difficult to identify and understand apparently independent of the data gathering method consolidates our hypothesis of its linguistic distance from the other EP/Cs studied here.

Geographic and cultural proximity explains mutual identification and perceived intelligibility between Ghanaian and Nigerian Pidgin as well as the ease with which the Guyanese consultants identified and understood Jamaican. In these cases, we are talking about acquired intelligibility as discussed in \sectref{sec:04:1}, whereas the postulate of inherent intelligibility requires the existence of a genetic relationship as suggested, for example, by \citet{Hancock_1986} and \citet{McWhorter_1996}. In fact, in the Atlantic EP/Cs, structural similarity crosses with cultural proximity and sociolinguistic accessibility. At the same time, structural similarity facilitates cognitive recognition. Therefore, the measuring of the cognitive component of language attitudes as postulated by \citet[139]{Agheyisi_Fishman_1970} is relevant for our study.

As stated above, the cultural presence of Naija, e.g., through Nollywood productions, in the Caribbean probably explains why the Guyanese informants were more familiar with it than with Ghanaian Pidgin. Based on this sample, Jamaican also appears to be culturally more present in Nigeria than in Ghana, and may be additionally identified by African informants as the Caribbean EP/C by default (cf. \sectref{sec:04:3.1.3}).

While the previous discussion is essentially based on the online survey, the interviews conducted in Guyana confirmed that time of exposure is a key factor: Most informants found Guyanese Creole easy to understand within less than three years. What constitutes passive knowledge is again subjective and based on the standards and goals an individual sets for themselves in an intercultural context \citep[cf.][]{Dornyei_Otto_1998,Dornyei_Csizer_2005}. At the same time, most interviewees not only displayed positive attitudes towards both Guyanese Creole and their own variety but also supported the idea of a common writing system. This is consistent with the views of the online survey informants as the majority of them (44 out of the 47 who responded to that question) reported positive attitudes towards their own varieties. Additionally, out of the 51 online informants who answered the question on the idea of a common writing system, the overwhelming majority (i.e., 48) were in favor of such a development.

Arguments in favor of and against a common writing system appear to concentrate on identity-related issues. The major difficulties perceived were not only in the field of vocabulary, typical of any pluriareal or pluricentric language,\footnote{In fact, \citet[382]{Berschin_FernandezSevilla_Felizberger_2012} define Spanish, a case in point, as a pluricentric language based on the lexicon, not “grammar”, i.e, language structure.} but also in pronunciation. Pronunciation is easily mistaken for – but also closely related to – orthographies and writing systems, especially if these are devised for languages with a short tradition of writing. \citet[192--193]{Schieffelin_Doucet_1994} state that:

\begin{quote}
    When a language is codified and an orthography is officially adopted, this is usually interpreted to mean that there is one correct way to spell and write the language, and that all others are simply wrong. [...] And when a variety through its officialization is given the status of a standard, the users of the other varieties sometimes react with surprising virulence because they feel that their language variety and its speakers are denied representation.
\end{quote}

Opting for “normalization” in the sense of allowing for regional variation in the initial stages of writing as proposed by \citet{Koskinen_2010} for Nicaraguan Creole constitutes a sound option – when we are dealing with a geographically limited area.\footnote{\citet[313]{DeGraff_2014} cautions that “uniform conventions for written representations [...] [should] not distract from other valuable priorities that function to promote the language”.}  Our interest lies in the feasibility of writing Atlantic EP/C on a much larger scale which, as some informants’ responses reveal, may turn out a much more complicated endeavor. In this context, it is relevant to bear in mind that certain graphemes, for example \{k\}, \{w\}, and \{y\}, have led to  heated debates \citep{Bartens_inpreparation} and that, on the other hand, not writing does not constitute an option in today’s world. Writing is a fundamental condition for languagehood according to present language ideologies \citep[e.g.,][]{Lupke_2018}. In the long run, writing is not only a key element for constituting “language” in the sense of H vs. L languages/varieties \citep{Ferraz_1979,Fishman_1967}, but also for passing from a pluriareal to a pluricentric language through distinct processes of codification as outlined in \sectref{sec:04:1}. Writing may, however, be achieved not only through flexibility (see Koskinen’s interpretation of the term “normalization” above) but also through grassroots literacy \citep{Blommaert_2008}. These are possible paths of action to counter the fears of those who maintain that the inherent heterogeneity and creativity of creoles has to be sacrificed when, for instance, creating an orthography \citep[cf.][124]{Freeland_2004}.

Assessing the informants’ competencies in applying different orthographies in the reading comprehension and translation task is therefore fitting to the current study. Again, geographic and cultural proximity accounts for the actual intelligibility as demonstrated within the Caribbean and West African region. However, delving deeper into the intricacies of the writing systems employed for each language, we noted that informants who indicated that they liked a specific writing system liked it because they could read/understand it; the sole exception here was Krio. While 75\% of the Guyanese and 40\% of the Ghanaians demonstrated that they understand Krio, only 25\% and 10\% Guyanese and Ghanaians respectively indicated that they liked the writing system. This is because while Krio uses the Latin script which the informants are accustomed to, it also employs three phonetic letters – \{ɛ\}, \{ɔ\}, and \{ŋ\} – which were reported to be difficult.

The orthography used to write Sranan was not liked by any of the informants, though it was understood by 21\% of the Guyanese informants. The reason for not liking the writing system of Sranan was simply because the language was seen as difficult and the informants did not understand it. This is so because most of the words in Sranan, though English-based, do not look like their English etymons and structural and grammatical contrasts may obstruct intelligibility. Sranan has been in contact with Dutch rather than English and therefore, unlike other EP/Cs, it has no post-creole continuum and did not get to decreolize. (See \citealt{Sebba_2000} for a full description of Sranan’s orthography.)

The Guyanese informants were more likely to report that they liked the Guyanese, Nicaraguan, and Jamaican writing systems (83\%, 75\%, and 54\%, respectively) when compared to the other languages. Conversely, Ghanaian informants were more likely to report that they liked the Nigerian and Ghanaian writing systems (70\% and 50\%, respectively). The structural and lexical similarity of the languages is assumed to account for the cognitive recognition, comprehension, and translation of the languages in the reading comprehension and translation task. This is also very much conditioned by the familiarity with distinct writing systems. (See \citealt{Winer_1990} for four different models for writing an EP/C.)

A postcolonial approach is called for precisely when we consider the necessity of writing. What is meant by postcolonial in this context? Linguists, especially, tend to associate “postcolonial” with literary studies. However, a field of postcolonial linguistics is starting to emerge and the research questions we are interested in – Atlantic EP/C as a pluriareal language, mutual intelligibility, and the possibility of writing – would fall into the domain of postcolonial sociolinguistics \citep[cf.][]{Makoni_2011,Levisen_Sippola_2019}. For our purposes, it seems more useful to define “postcolonial” as a time-defining concept implying change (cf. \cite{Calabrese_2015}: 1; \cite{Anchimbe_2018}: xiii) than within the framework of power which postcolonial studies in general operate with (a position advocated, e.g., by \cite{Warnke_2017}). Both stances, the descriptive-causal and the critical-reflective one, share their opposition to Eurocentrism \citep[2]{Levisen_Sippola_2019}. 

Though space will not permit us to fully explore the postcolonial approach in this present paper, it is our aim to return to this in subsequent works. Nonetheless, in our understanding of postcolonial, i.e., as an essentially time-defining concept, it is important to consider language ecologies, both past (cf. The Founder Principle) and present, which draw on the feature pool available through population groups at a given time and the possibility of translanguaging in the ongoing formation of Atlantic EP/C(s) (\cite{Haugen_1971,Mufwene_1996,Mufwene_2013,Schneider_2007}: 22--23, \cite{Garcia_Li_2014}). For language ecologies to emerge, the concept of “community” is highly relevant. Community is a highly elusive concept \citep[xii]{Muhleisen_2017} and represents “a relation constantly under negotiation” \citep[2]{Brydon_DonaldColeman_2008}. We envision that speakers of various Atlantic EP/Cs may actually at some point conceive of themselves as speaking a single Atlantic EP/C (i.e., a pluriareal language, albeit with varieties; see \sectref{sec:04:1}). But a (speech) community does not exist without the individuals who make it up. Indeed, \citet[323]{Mufwene_2013} stresses the role of the individual in the ecology of language and affirms that:

\begin{quote}
    [...] factoring in the speaker as the most direct external ecological factor to language, as he/she contributes variation to the emergent, ever-evolving language and participates in: 1) the spread or elimination of variants through the selections he/she makes from among the competing variants (be they languages or linguistic features), 2) the emergence of new norms, and 3) sometimes the emergence of new varieties \citep[311]{Mufwene_2013}.
\end{quote}

We are arguing that competition and selection of variants is not necessarily one of the first steps, at least not on a larger scale, in the (re)constitution of an Atlantic EP/C which will undoubtedly retain variation as a pluriareal language. On the other hand, the emergence of new linguistic norms, including norms of usage, and their consolidation as an acknowledged diasystem, albeit with fluid borders, is a more realistic outcome. This diasystem could theoretically at some point be on its way to a pluricentric language with speakers increasingly accommodating their linguistic production in the sense of convergence towards the other speakers’ varieties (cf., for example, \cite{Giles_Ogay_2007}: 295 for the Communication Accomodation Theory) due to the fulfillment of the criteria stipulated, e.g., by \citet[20]{Muhr_2012} and \citeauthor{Clyne_1992} (\citeyear{Clyne_1992}: 1; see \sectref{sec:04:1}), but for the time being this may seem to be a relatively farfetched scenario. Indeed, if we prefer to consider languages less as “linguistically defined objects” and rather as speech and resources, “the real bits and chunks of language that make up a repertoire, and [...] real ways of using this repertoire in communication” \citep[5, 173]{Blommaert_2010}, continuing to conceive of Atlantic EP/Cs as pluriareal languages may constitute a more feasible solution even in the long run. However, and as stated above (\sectref{sec:04:1}), in the “making of languages” – which may actually consist of the disinvention and subsequent reconstitution of linguistic varieties \citep{Makoni_2005} – clear-cut boundaries, labels, names, and norms are required especially in cases of long-standing stigmatization and minoritization to which Atlantic EP/Cs also belong \citep[cf.][]{Huning_Kramer_2018}. This is especially true in sociolinguistic terms, whereas structural fluidity is necessary for use in wider contexts, including writing, when dealing with as complex a diasystem as Atlantic EP/Cs.\footnote{Note, however, that fluidity (and hybridity) is problematic in that, as advocated in postcolonial studies, this concept overlaps with neoliberalism \citep{Kubota2016}. In the case of Atlantic EP/C, structural fluidity is a necessity which does not clash with the time-defining, descriptive causal perspective adopted here.}

The impact and the peculiarities of virtual communities are no longer to be underestimated. Whereas cultural diffusion (cf. the above-mentioned example of Jamaican reggae) and diaspora formation lead to the spread, acceptance, and consolidation of certain varieties and their norms of usage (see \cite{Muhleisen_Schroder_2017} on the shift in role and prestige of Caribbean Creoles and Cameroonian EP/C in new diasporic metropolitan environments), new technology has opened totally new opportunities for language diffusion and development, including the democratization of the latter process \citep[cf.][]{Eisenlohr_2004,Bartens_2019}. For example, cyberspace has developed a sociolinguistic order of its own. \citet{Mair_2019} on Jamaican Standard English vs. Jamaican Creole and Nigerian Standard English vs. Naija found that what is stigmatized on the ground often actually becomes prestigious in cyberspace. At the same time, it becomes increasingly difficult to apply the concept of “speech community” to, for example, (diasporic) web-forum interaction \citep[91]{Moll_2017}. In the case of Atlantic EP/C(s), this may actually constitute an opportunity: If instead of the classical speech community (\cite{Gumperz_1968}; \cite{Labov_1972}: 120--121) we focus on communities of practice in the sense of “an agreement of people who come together around mutual engagement in an endeavour” \citep[464]{Eckert_McConnellGinet_1992}, less competent or so-called peripheral speakers \citep{Labov_1972} may be integrated by means of computer mediated communication.

We maintain that the affirmation “the language of the imagined global community is clearly English and the written script is digital” \citep[xvii]{Muhleisen_2017} is not true, especially not in the case of very widely spoken pluriareal Atlantic EP/C (see \sectref{sec:04:1}). Its different varieties are already widely used in digital communication and we believe that in the future this use will further increase and may lead to a consciousness of belonging to an imagined community, which is already in place among speakers of Caribbean varieties. The linguistic practices of this imagined community are able to counter elite closure by enabling alternative routes in the flow of information \citep[cf.][284, 311]{DeGraff_2014}.

\section{Conclusions}\label{sec:04:5}

The point of departure for this study was the observation that many speakers of closely related varieties of Atlantic EP/Cs can understand each other. First, we defined and justified the use of certain key concepts such as Atlantic EP/C(s). Our argument is that distinct varieties, i.e., EP/Cs, could be conceived of – and in part already are viewed as – the Atlantic EP/C diasystem. As we insist, this language system has fluid borders, allowing for use in distinct contexts and domains. Throughout the paper, we were concerned with intelligibility. We also discussed why we consider Atlantic EP/C a pluriareal language in the making and not a pluricentric one despite the fact that the former term is contested by some scholars \citep[e.g.,][7]{Dollinger_2016}. This (re)constitution of a language system with as many as 130,000,000 speakers is facilitated by a great number of shared linguistic structures.

Three lines of inquiry were pursued in order to verify – or falsify – the hypothesis of mutual intelligibility: 1. an online survey available from June 15, 2019, until October 9, 2019 which included audio clips of Ghanaian, Nigerian, Jamaican, and Sranan EP/C; 2. interviews of other Atlantic EP/C speakers residing in Guyana at the time of the study (October 2019); and 3. a reading comprehension and translation task conducted with university students in Guyana and Ghana (December 2020). The third line of inquiry was adopted in order to mitigate the caveats of self-report of intelligibility arising from the dependence on language attitudes (see Sections 1 and 2).

The data reveal that geographic and cultural proximity is taken to explain acquired intelligibility between Ghanaian Pidgin and Naija as well as the ease with which the Guyanese informants identified and understood Jamaican. Familiarity with Nollywood and Jamaican culture, especially reggae, were suggested to account for those varieties being better known and probably taken as representative of West Africa and the Caribbean, respectively.

Structural similarity was assumed to facilitate cognitive recognition, as was demonstrated in the reading comprehension and translation task. In the case of reading comprehension, cognitive recognition is also very much conditioned by the familiarity with distinct writing systems (see \cite{Winer_1990} for four different models for writing an EP/C). Writing was argued to constitute one of the major criteria for languagehood and therefore it is relevant to state that attitudes not only towards one’s own but also other varieties as well as writing them were fairly positive. For example, 13 out of the 20 interviewees were favorable towards a common orthography for writing EP/C languages. 

The question of writing can be framed in postcolonial thought. For our purposes, defining “postcolonial” as a time-defining concept seemed the most adequate approach. Despite this descriptive-causal (as opposed to critical-reflective) interpretation, we believe that it is precisely new technologies, offering new possibilities for the revitalization and development of minorized languages, which will enable Atlantic EP/Cs to consolidate into a community of practice in its own right. For the time being, we leave exploring the topic from a postcolonial angle – as well as a more in-depth discussion of the reading comprehension and translation task – for future studies.


\section*{Acknowledgements}
We would like to gratefully acknowledge the invaluable contribution of Dr. Uchenna Oyali, University of Abuja, during the initial phase of the study (online survey). We would also like to thank the two anonymous reviewers for their comments and suggestions.

\printbibliography[heading=subbibliography,notkeyword=this]

\end{document}
