\documentclass[output=paper,colorlinks,citecolor=brown]{langscibook}
\ChapterDOI{10.5281/zenodo.14282808}

\author{Kristoffer Friis Bøegh\orcid{}\affiliation{Utrecht University;Aarhus University}}
\title[African ethnolinguistic diversity in the colonial Caribbean]
      {African ethnolinguistic diversity in the colonial Caribbean: The case of the eighteenth-century Danish West Indies} 
\abstract{This study reconstructs a profile of the ethnolinguistic groups originating in Africa represented in the eighteenth-century Danish West Indies (i.e., St. Thomas, St. John, and St. Croix, today’s US Virgin Islands), based on triangulation of four bodies of direct and indirect evidence of African languages in the colony. I analyze i) data on the recorded slave trade to the colony; ii) data on 26 African languages collected in peer-group interviews conducted in the late 1760s by the German Moravian missionary and historian C.G.A. Oldendorp, with an estimated 70 enslaved individuals on St. Thomas and St. Croix; iii) data on the “nations” (i.e., approximate ethnic backgrounds) represented in a sample of 418 Christianized Africans and African Caribbeans of the Moravian congregation on St. Thomas, recorded in 1753 by the missionary Nathanael Seidel; and iv) baptismal records from 1744–1832 collected by the Moravians on St. Croix, featuring systematic information on more than 6,000 baptismal candidates of African descent. The results show that a wide range of languages were spoken within the enslaved population. In addition, a number of differences are identified in the distribution of groups and their proportions between St. Thomas and St. Croix. The findings, I argue, have implications by extension for contact linguistics more generally, shedding new light on the sociohistorical settings characterizing high-contact sites in the circum-Caribbean region.}

\IfFileExists{../localcommands.tex}{
   \addbibresource{../localbibliography.bib}
   \usepackage{langsci-optional}
\usepackage{langsci-gb4e}
\usepackage{langsci-lgr}

\usepackage{listings}
\lstset{basicstyle=\ttfamily,tabsize=2,breaklines=true}

%added by author
% \usepackage{tipa}
\usepackage{multirow}
\graphicspath{{figures/}}
\usepackage{langsci-branding}

   
\newcommand{\sent}{\enumsentence}
\newcommand{\sents}{\eenumsentence}
\let\citeasnoun\citet

\renewcommand{\lsCoverTitleFont}[1]{\sffamily\addfontfeatures{Scale=MatchUppercase}\fontsize{44pt}{16mm}\selectfont #1}
  
   %% hyphenation points for line breaks
%% Normally, automatic hyphenation in LaTeX is very good
%% If a word is mis-hyphenated, add it to this file
%%
%% add information to TeX file before \begin{document} with:
%% %% hyphenation points for line breaks
%% Normally, automatic hyphenation in LaTeX is very good
%% If a word is mis-hyphenated, add it to this file
%%
%% add information to TeX file before \begin{document} with:
%% %% hyphenation points for line breaks
%% Normally, automatic hyphenation in LaTeX is very good
%% If a word is mis-hyphenated, add it to this file
%%
%% add information to TeX file before \begin{document} with:
%% \include{localhyphenation}
\hyphenation{
affri-ca-te
affri-ca-tes
an-no-tated
com-ple-ments
com-po-si-tio-na-li-ty
non-com-po-si-tio-na-li-ty
Gon-zá-lez
out-side
Ri-chárd
se-man-tics
STREU-SLE
Tie-de-mann
}
\hyphenation{
affri-ca-te
affri-ca-tes
an-no-tated
com-ple-ments
com-po-si-tio-na-li-ty
non-com-po-si-tio-na-li-ty
Gon-zá-lez
out-side
Ri-chárd
se-man-tics
STREU-SLE
Tie-de-mann
}
\hyphenation{
affri-ca-te
affri-ca-tes
an-no-tated
com-ple-ments
com-po-si-tio-na-li-ty
non-com-po-si-tio-na-li-ty
Gon-zá-lez
out-side
Ri-chárd
se-man-tics
STREU-SLE
Tie-de-mann
}
   \boolfalse{bookcompile}
   \togglepaper[2]%%chapternumber
}{}

\begin{document}
\maketitle

\section{Introduction}

During the seventeenth and eighteenth centuries, European colonial powers divided the Virgin Islands in the Caribbean into two primary geopolitical units: the British Virgin Islands and the Danish West Indies.\footnote{In addition, the islands of Vieques and Culebra, located just east of Puerto Rico and both designated as municipalities of that territory, are also geographically part of the Virgin Islands.} The latter group, which became part of the United States as the US Virgin Islands in 1917, comprises St. Thomas (permanently settled in 1672), St. John (1718), and St. Croix (1734). As in the case of many other colonies in the circum-Caribbean region, the developmental history of the Danish West Indies was shaped by trade, plantation slavery, and contacts between diverse population groups of predominantly African and European origins (for an overview, see, e.g., \cite{JensenSimonsen_2016}). Although the islands were Danish by name for centuries, the vast majority of their population did not originate from Denmark\footnote{From 1672 to 1814, the Danish West Indies belonged to the double monarchy of Denmark-Norway, after which time Norway gained independence from Denmark. In the present study, I refer to the double monarchy simply as “Denmark.”} but had African or African Caribbean backgrounds. The enslaved Africans transported to the islands carried with them a large measure of their cultural systems, including their languages; and because they originated from numerous localities, an ethnolinguistically diverse society formed early on, and continued throughout the period of slavery. The colonizing population was linguistically heterogeneous as well. However, whereas information on the different backgrounds represented in the latter group is readily accessible (including in the form of copious census data), some of the central primary sources on the provenance and linguistic backgrounds of the enslaved population have yet to be brought together in an investigation of the ethnolinguistic diversity of people of African descent represented in the slave society.

This study aims to reconstruct a profile of the ethnolinguistic groups originating in Africa represented in the eighteenth-century Danish West Indies. In addition, it seeks to uncover differences in the distribution of groups and their proportions between the colony’s two main islands, St. Thomas and St. Croix, which were settled around half a century apart, and whose demographic profiles, consequently, reflect both differences in local developments and shifting waves in the slave trade. Moreover, because of the general scarcity of documentary evidence of African languages in the context of the colonial-period Caribbean, the study also has a wider relevance, extending beyond the scope of Virgin Islands language history. There is a seemingly unparalleled volume and breadth of documentation from the Danish West Indies relevant to a case study nuancing current notions about the extensiveness of diversity in the ethnolinguistic composition of “typical”\footnote{In a classic study of the Danish West Indies under Company rule, \citet[121]{Westergaard_1917} characterized the colony as a “fairly typical plantation society.”} Caribbean plantation slave societies. From a contact linguistics perspective, tapping into this documentation has the potential to cast new light on the sociohistorical settings and hence contributions of agents involved in the development of creole languages in the region – provided the situation characterizing the Danish West Indies is comparable to those characterizing other territories. This is most likely the case; although slavery and settlement varied considerably across the Caribbean, there are also many parallels from territory to territory \citep[see, e.g.,][]{Higman_2011}. Against this backdrop, the present study can be viewed as a response to calls from within the field for studying language contact phenomena, above all else, from a historically realistic perspective (along the lines of, e.g., \cite{Arends_2008}; cf. also \cite{Kouwenberg_Singler_2018}).

The Danish West Indies occupy a special position in the context of documentation of African languages in colonial settings. In addition to data on the recorded slave trade (the existence of which is commonplace throughout the Caribbean), there is an abundance of direct and indirect evidence of African languages, among others, in the form of data recorded \emph{in situ} by the Moravian Brethren, the pietistic missionary movement active in the Danish West Indies since 1732 (cf. §2.3 for details). Focusing on the eighteenth century (and for some aspects extending into the first part of the nineteenth century), I base my analysis primarily on triangulation of the following four bodies of evidence:

\begin{itemize}[label=-]
    \item data on the recorded transatlantic and intra-Caribbean slave trade to the Danish West Indies (including records from the online repository available at \url{https://www.slavevoyages.org/}, henceforth the Voyages Database);

    \item direct and indirect evidence of 26 African languages collected in peer group interviews with 70 or so enslaved individuals on St. Thomas and St. Croix, conducted by the Moravian missionary C.G.A. Oldendorp in the late 1760s for an ethnographic survey (\cite{Oldendorp_1777}; English edition: \cite{Highfield_1987}; critically-edited unabridged version: \cite{Oldendorp_2000, Oldendorp_2002});

    \item self-reported data on the ``nations'' (i.e., approximately, ethnic backgrounds) from a sample of 418 Christianized Africans and African Caribbeans of the Moravian congregation on St. Thomas, recorded in 1753 by the missionary Nathanael Seidel (\cite{Seidel_1753}; list published as an appendix in \cite{Sebro_2010});

    \item baptismal records for the period 1744–1832 from Friedensthal, the oldest of the Moravian mission stations in St. Croix, featuring systematic information on the ethnic backgrounds of over 6,000 baptismal candidates with an African or African Caribbean background (tabulated in \cite{Pope_1970}).
\end{itemize}

Additionally, where relevant, I draw on the historical literature on the Danish West Indies, including Danish-language research based on sources from the Danish National Archives (cf. \cite{JensenSimonsen_2016} for an overview).

The study is organized as follows. In §2, I offer background information on the Danish West Indies and discuss the colony’s suitability as a case study on African ethnolinguistic diversity in the colonial Caribbean setting. Moreover, I contextualize the sources of evidence I use to trace members of the African population to specific areas and ethnolinguistic groups. In §3, I examine the recorded slave trade to the Danish West Indies, offering an assessment based on estimates of imports from both the transatlantic and intra-Caribbean slave trade. Based on this exercise, I argue for a need to consider additional sources of evidence so as to be able to qualify the big-picture result which obtains. In §4, accordingly, I shift perspective to locally-recorded evidence. I compare linguistic data (as well as other cultural evidence) collected by Oldendorp with modern, carefully transcribed African language data, aiming to identify present-day correlates to the languages documented in the historical material. Extrapolating from the results of the analysis of Oldendorp’s data, I then examine Seidel’s list of Christianized Africans and African Caribbeans from St. Thomas in §5 and the baptismal records from St. Croix in §6, looking at similarities and differences in the distribution of the identified ethnolinguistic groups and their proportions between the two islands. In §7, finally, I provide a summary of the ground covered in the study and present some concluding remarks.

\section{The Danish West Indies as a case study on African origins and ethnolinguistic diversity: Background and sources of evidence}

Setting the scene, this section presents background information on the Danish West Indies and its enslaved population, focusing on the setting in the eighteenth century. I discuss the problem of ethnolinguistic identifications in the colonial Caribbean, and I contextualize the various sources of evidence drawn upon in the study.

\subsection{The slave society in the eighteenth century: Demography and ethnolinguistic diversity}

In 1672, the Danish West India and Guinea Company founded a permanent colony on the island of St. Thomas. The aim was to follow the example set by other expansionist European nations that had established exploitation colonies overseas in order to gain a part of the wealth associated with the transatlantic trade (cf. \cite{JensenSimonsen_2016}; see also \cite{Hall_1992}; \cite{Gobel_2016}; \cite{Olsen_2017}; \cite{Robertsetal_2024}, among other works). St. Thomas is hilly and was not particularly well suited for large-scale plantation agriculture, but it is endowed with an excellent natural harbor and, therefore, gained importance as a transshipment port over the course of the eighteenth century. In 1718, the Danish colony expanded with the annexation of the unsettled neighboring island of St. John. In 1733–34, more significantly, St. Croix was acquired from the French, and its colonization was initiated. The topography of St. Croix was well suited to plantation agriculture, and as a result, following its incorporation into the colony, sugar cultivation increasingly became an economic mainstay of the Danish West Indies. This relied on highly labor-intensive cultivation techniques and involved an ever-growing demand for an imported workforce: the enslaved Africans and African Caribbeans. The following section will consider the historical demography of the enslaved population across the three islands.

\subsubsection{Slave demography in St. Thomas, St. John, and St. Croix}

In its first century, the Danish colony was characterized by a rapidly growing enslaved population. As early as 1686, fourteen years after the permanent settlement of St. Thomas, people of African descent exceeded the European population, and that development only continued to accelerate throughout the eighteenth century. Thus, in 1755, when the Danish Crown took over the colony from the Company, there were 1,760 inhabitants of European origin across the three islands combined, whereas the enslaved numbered 14,409 people, or nearly 90\% of the total population \citep[58]{Gobel_Sebro_2017}. However, the demographic development was not the same on the three islands.

The main population growth on St. Thomas occurred in the first two decades of the eighteenth century. After that, the number of enslaved people remained stable for the remainder of the century. Based on an analysis of the colony’s extant tax-rolls, \citet[157]{GreenPedersen_1971} reported the following population figures. In 1686, the colonizing population numbered 300 and the colonized 333; by 1720, the numbers were 524 against 4,504; by 1754, they were 228 against 3,481 (with this decline in absolute numbers being offset by an increase on St. Croix). The European population segment comprised 7\% of the island’s total number of people in 1755, just under 8\% in 1775, and some 15\% in 1803.

St. John was settled from St. Thomas, and it developed largely as an adjunct of that island. Its colonization proceeded rapidly until 1733, at which time it became the scene of a large-scale slave insurrection. The rebellion was put down only after six months, and the island remained sparsely inhabited thereafter \citep[91--96]{Hornby_1980}. There were 2,031 enslaved people on St. John in 1755, corresponding to approximately 90\% of the island’s population. This number fluctuated a bit in the latter eighteenth century, rising to 2,598 in 1803, after which it remained stable (\cite{GreenPedersen_1971}: 157–58; \cite{Gobel_Sebro_2017}: 59).

St. Thomas' economy relied much less on plantation agriculture than those of St. John and, in particular, St. Croix. This is reflected in the fact that, peaking in 1797, as many as 34\% of the enslaved on St. Thomas worked in the town of Charlotte Amalie \citep[234]{GreenPedersen_1981}. \citet[128]{Sebro_2016} notes that society in St. Thomas ``was based on the principally racial divide where black meant slave and white meant free.'' However, as in slave societies elsewhere, this principal divide was not absolute. From the very beginning, a small group of free African Caribbeans was present as well. They managed in different ways to find space for themselves in the slave society, especially in the urban setting, by negotiating ideas of race and social position. However, in the eighteenth century, St. Thomas was still in the early stages of that process. Therefore, the free African Caribbean population, Sebro (\emph{ibid.}) indicates, ``was structurally of no great importance.'' The same was the case on early Danish St. Croix. Urban slavery also became firmly established there early on (see \cite{Tyson_2011}), but by far the largest proportion of the enslaved on St. Croix spent their days on large-scale rural sugar plantations. In 1792, for example, 86\% of enslaved people on St. Croix were engaged in field labor \citep[75]{Hall_1992}.

St. Croix was surveyed and brought under European cultivation in the course of the 1730s and 1740s, and the island’s plantation sector was consolidated in the 1750s (\cite{Hopkins_1987}). The following years were characterized by a rapid growth in the enslaved population, and by high mortality rates. Approximately 1\% more died than were born \citep[150]{Simonsen_Olsen_2017}, meaning that a constant inflow of new arrivals was needed to maintain existing numbers, not to mention meeting the requirement for absolute growth (cf. \cite{Robertsetal_2024}). In terms of demography, St. Croix thus followed the developmental pattern typical of a Caribbean plantation colony \citep[cf.][130]{Higman_2011}. According to \citet[150--151]{Simonsen_Olsen_2017}, in 1736, two years after the island’s settlement, there were 137 enslaved people on St. Croix; in 1742, this number had risen to around 1,700; and in 1765, it had increased to almost 19,000. This growth continued into the 1770s, after which the number stabilized at around 23,000 or 24,000. Between 1793 and 1803, the enslaved population grew once again, peaking at approximately 27,000 people in 1804. After that, the number decreased steadily. The enslaved claimed their freedom in 1848. In 1847, the last year a census was taken of the enslaved population, there remained 16,673 people \citep[33]{Holsoe_1994}, or about 62\% of the all-time maximum recorded 44 years earlier.

\subsubsection{Ethnolinguistic diversity in the Danish West Indies: A tentative orientation}

From the outset, Danish colonization in the Caribbean was characterized by a lack of much of an actual Danish colonizing population. This resulted in what has been described as a \emph{de facto} policy of “colonization by invitation” \citep[6]{Hall_1992}, which gave rise to substantial numbers of foreign Europeans being introduced into the population. These people were particularly of Dutch heritage on St. Thomas and St. John \citep[51]{Arends_Muysken_1992}, and of British stock on St. Croix \citep[13]{Hall_1992}. As for the languages used among the Europeans, \citet[357]{Oldendorp_2000} comments, writing in the 1760s, that at least English, Dutch, Dutch Creole (more on this below), German, Danish, French, and Spanish were in use. Although considerable, this diversity did not match that represented by the African languages that were in use in the colony. By way of illustration of this point, Johann Gottlieb Miecke, the principal of the Moravian Mission in St. Croix in the late eighteenth century, wrote in his 1796 letter of resignation (quoted in \cite{Lawaetz_1902}: 124) about the linguistic diversity characterizing the island. Miecke stated that he was losing faith in the mission work since it would be insufficient, he felt, to know even as many as 20 different languages if one was to effectively reach the enslaved population. Thus, the eighteenth-century context of the Danish West Indies was clearly characterized by extensive multilingualism, both among the enslaved and colonizing subsets of the population.

Meanwhile, this sociohistorical setting had led to the establishment of creole languages with lexical bases in Dutch and English, respectively, in the colony.\footnote{In addition, \citet[131–32]{Highfield_1993} conjectured that an undocumented French-lexifier creole was in use in St. Croix prior to the island’s abandonment by the French in 1696. Whether or not such a variety did exist, it is not to be confused with the French dialect of Frenchtown, St. Thomas (described in \cite{Highfield_1979}).} The first of these, Virgin Islands Dutch Creole, most likely formed locally on St. Thomas around 1700 \citep[cf.][52--54]{Sabino_2012}. It subsequently gained currency as a lingua franca in the colony, especially on St. Thomas and St. John. This language was the first creole to be described in a printed grammar \citep[see][]{Magens_1770}, and it is also notable for having been documented extensively for more than 250 years (for an anthology of Virgin Islands Dutch Creole texts, see \cite{vanRosseman_van_der_Voort_1996}; see also \cite{vanRosseman_2017}; \cite{vanSluijs_2017}; \cite{Boegh_2022}, along with other recent works). By the second half of the 1700s, a variety of Caribbean English Creole had become sufficiently well-established on St. Croix (where English had become established to the virtual exclusion of Dutch, cf. \cite{Oldendorp_2000}: 358, 682) for contemporary observers to make reference to its existence (e.g., \cite{Auerbach_1774} quoted in \cite{vanRosseman_2017}: 90). English Creole gained ground on St. Thomas and St. John from the early 1800s onward, leading to the eventual decline of Virgin Islands Dutch Creole in the nineteenth century (as observed by, e.g., \cite{Pontoppidan_1881}; cf. \cite{Stein_2014}; for further discussion and documentary evidence, see, e.g., \cite{BoeghBakker_2021}).

The creoles underwent influence from the diverse African languages represented in the colony. As for the identities of the African languages, earlier studies supported the claim that most of the Africans brought to St. Thomas during the last part of the seventeenth century spoke Kwa languages such as G\~a and Akan. Speakers of Akan, in particular, were estimated, by both historians and linguists, to have predominated throughout the peopling of St. Thomas (e.g., \cite{Feldbaek_Justesen_1980}; \cite{Stolz_Stein_1986}: 118; \cite{Parkvall_2000}: 135). In addition, \citet{Sabino_1988} suggested that Ewe (Gbe subgroup of Kwa) was a main early substrate in the colony (cf. also \cite{Sabino_2012}). More recent discussions of Danish-African trade relations and the transatlantic slave trade to the Danish West Indies (e.g., \cite{Sabino_2012}: 64–67; \cite{vanSluijs_2017}: 27–32) suggest that this picture is incomplete, in that a much larger set of languages were spoken in the regions from which enslaved people were transported to the Danish colony, and from a much wider area of Africa. \citet[65]{Sabino_2012} opts to focus on the Danish slaving operations as those that were most likely to have impacted the early demographic development of St. Thomas. Yet, the significance of other sources of trade to the growth of the colony’s plantation sector should not be underestimated. For instance, in 1685, the North German Brandenburg Africa Company gained the right to use St. Thomas as a transshipment port, after which it supplied thousands of enslaved people to the colony, some of whom were not re-exported, but incorporated into the local slave society \citep[254--257]{Weindl_2008}, \emph{pace} \citet[65]{Sabino_2012}. 

The most important previous analysis of slave imports with a focus on St. Croix is an unpublished anthropology dissertation (\cite{Pope_1970}). That study's main finding was that the ethnic diversity on St. Croix was greater and somewhat differently composed than that of St. Thomas. The archival data undergirding the analysis in \citet{Pope_1970} are available for scrutiny in its appendices. In §6, I analyze these unique data anew, hence the short discussion here. Re-evaluation of the data is warranted seeing as how a number of \citeauthor{Pope_1970}’s (\citeyear{Pope_1970}) conclusions have been subject to revision in later research (as detailed in §6).

In 1803, the official termination of the Danish slave trade effectively ended the flood of new contingents of native speakers of African languages. \citet[48]{Asmussen_1983} calculated that 46.9\% of the plantation laborers present on St. Croix in 1804 had been born in Africa, whereas only 7.3\% had been in 1846. Thus, although the prospects for their maintenance and long-term survival were poor, we can assume that African languages were in (recessive) use well into the nineteenth century. Despite this, there has been only limited attention devoted in the literature to identifying the African languages. Besides the studies cited above, there are some previous tentative analyses of Oldendorp’s material (\cite{Fodor_1975}; notes from Peter Stein in \cite{Oldendorp_2000}: 365–465; \cite{Jones_2010}). A more precise overview of the ethnolinguistic diversity represented in the colony can be obtained through the use of new methods and materials.

\subsection{The problem of ethnolinguistic identifications in the context of the colonial Caribbean}

There is little direct linguistic evidence in the form of quotes or texts in African languages from the Caribbean in the colonial period. Outside of the Virgin Islands, there is some documentation from 1741 of a Gbe language in Brazil \citep{Peixoto_1944}, and from the late 1700s of Kikongo in Haiti (Peter Bakker, p.c.), but not much else has survived. The lack of direct evidence is a fact commonly acknowledged in creole studies, and one which routinely obstructs what \citet[2]{Kouwenberg_2008} has termed “[t]he first task of the substratist creole researcher” – namely, identifying as precisely as possible the input languages involved in a given contact setting so as to be able to apply historical-comparative methods to investigate the origins of the various creole languages as contact phenomena. On the other hand, there is abundant indirect evidence for the use of African languages in the Caribbean, including in the form of lexical Africanisms and other instances of substrate influence in today’s creoles (see, e.g., \cite{Parkvall_2000, Parkvall_2016}; \cite{Bartens_Baker_2012}; \cite{Muysken_Smith_Broges_2015}). Beyond this, however, evidence for the African input in sociohistorical linguistic research – such as in reconstructions of the settlement histories of individual creole-speaking societies – tends to be limited to metalinguistic observations coupled with documentation of slave voyages (see, e.g., \cite{Kouwenberg_2008}, on Jamaica; \cite{Prescod_Fraser_2015}, on St. Vincent; \cite{GalarzaBallester2016}, on Antigua, to mention but a few studies based primarily on such sources of evidence).

As discussed by \citet[109]{Velupillai_2015}, archival information on the location of the trade forts and the points of embarkation for the slave ships has often been taken as an indication of the ethnolinguistic origins of substrate populations (cf. also §3). This has become possible because detailed information on the ports of embarkation, the number of captives shipped out, and their destinations has been collected and made available for the transatlantic slave trade (see \cite{Eltis_Behrendt_Richardson_Klein_1999} and, especially, the updated online database based on this publication, available at \url{https://www.slavevoyages.org/}, i.e., the Voyages Database). On the downside, the location of the forts and the points of embarkation for the ships cannot always be taken to coincide with the origins of the enslaved individuals that those forts traded with. These may have been taken from different hinterland sources, or bought from traders from other regions. According to \citet[109]{Velupillai_2015}, we should aim to establish the likely areas that would supply certain forts and ships, and which languages were likely to have been spoken in those areas at the time; yet, critically, as she notes (\emph{ibid.}), “even if we were able to establish those facts, we would have extremely limited access, if any, to the kinds of varieties spoken in the given areas at the time.” In a good number of instances, however, this problem can be bypassed in the case of the Danish West Indies, as I will proceed to show in the present study.

Meanwhile, metalinguistic comments from contemporary observers – while also useful – likewise present us with a number of challenges. In the context of the Danish West Indies, this can be illustrated by the following quote from a police journal sourced from the Danish National Archives:

\begin{itemize}[label={}]
    \item Carolina... Mandungo... She speaks no understandable language, and none of the black people in the [police] yard were able to understand her, so no one knew how she had come to the country. The police chief sent out the officers to summon various black people from the [West African] coast who were able to speak English, too [in addition to African languages, KFB]. After numerous attempts, a black boy, Peter, was summoned, who was able to understand the black woman. (Danish National Archives: Sheriff of Christiansted: Police journals, 3 September 1805, my translation)
\end{itemize}

Although this extract does present us with information relevant to the identification of the mentioned individual’s ethnic background (``Mandungo''), evidence of this kind is too limited to allow researchers to engage with questions pertaining to ethnolinguistic origins more generally. There is a lack of representativeness (in being anecdotal) and contextual information (on what basis did the record keeper reach the conclusion?), and it appears unclear whether there exists any direct evidence of the language used among the ``Mandungo,'' which, if present, could be compared with corresponding data from modern African languages to help establish its identity.\footnote{Oldendorp did in fact record ``Mandungo'' (or ``Mandongo'') language data, which I track to either \citeauthor{Guthrie_1948}'s (\citeyear{Guthrie_1948}) Bantu Zone B or H, i.e., approximately between southern Gabon and northwestern Angola on the West-Central African coast (cf. §4.1.18).} 

In light of the potential pitfalls of working with sources of evidence such as those discussed above, \citet[6]{Prescod_Fraser_2015}, in their study on the settlement history of St. Vincent, sum up the main problem as follows: ``The ethnolinguistic tapestry of the African continent is intricate and ascertaining ethnolinguistic origins is an arduous task.'' Indeed, the African continent is characterized by vast linguistic diversity. The number of African languages is estimated to be over 2,000 in \citet{Eberhard_Simons_Fenning_2019}, and the lexical diversity is enormous. In addition, typological unity cannot be attributed to the various families and subgroupings \citep[e.g.,][]{Boegh2016}. Importantly, the availability of information on modern African languages for comparative research purposes has increased considerably over the course of the past few decades. African linguistic relationships are not yet fully delineated, though, and the internal structures and/or external affiliations of a number of groupings continue to be subject to regular proposals for revision \citep[see, e.g.,][]{Guldemann_2018}. As a point of departure, therefore, the present study adopts the broad classification of the \emph{World Atlas of Language Structures} \citep{Dryer_Haspelmath_2013} into linguistic genera, i.e., groups of languages ``whose relatedness is fairly obvious without systematic comparative analysis'' \citep[584]{Dryer_2005}. Where pertinent, more specialized divisions of groups commonly subsumed under the Niger-Congo phylum (or superfamily) follow the classification proposed by \citet[18]{Williamson_Blench_200}.

\subsection{The Moravian witnesses to slavery in the Danish West Indies}

In 1732, the Moravian Brethren – missionaries operating from Herrnhut, Germany – established their first overseas mission on St. Thomas, expanding to St. Croix in 1734 (see, e.g., \cite{Meier_Stein_Palmie_Ulbricht_2010}). They recorded the most central of the linguistic sources considered in the present study. This section details their activities in the Danish West Indies. 

The Moravian presence in the colony was important in several respects. The missionaries were present from early in the colony, arriving 60 years after the settlement of St. Thomas, 14 years after the settlement of St. John, and at the very beginning of the Danish settlement of St. Croix. They were on a mission to Christianize the enslaved African labor force, and they were from the start looking to expand to other locations. The movement subsequently spread to Suriname (1735), Jamaica (1754), Antigua (1756), Barbados (1765), and elsewhere in the Caribbean and beyond. Thus, and seeing as how they were gathering experience for future expansion, the missionaries had embarked on a project that would periodically require evaluation. Accordingly, as noted by \citet[141]{Highfield_1994}, they kept detailed records in the form of “[c]hurch books, journals, diaries, registers of all kinds – but especially those relative to baptisms – correspondence, and reports,” all of which were carefully copied and preserved. The records were regularly inspected by visitants, whose task it was to supervise the progress made and to report back to Herrnhut. Their accounts, of which Oldendorp's (\citeyear{Oldendorp_1777}/\citeyear{Oldendorp_2000, Oldendorp_2002}) is the best-known example, add important supplementary information to that collected on an ongoing basis in the colony, such as the baptismal records from St. Croix. All in all, the Moravians left an enormous amount of detailed information about their activities, and about the enslaved population in the Danish West Indies.

Initially, the Moravian effort was concentrated on St. Thomas. Later, that effort was largely redirected to St. Croix. After a difficult beginning in the 1730s and 1740s, the Moravians’ work in the islands succeeded remarkably in the second half of the eighteenth century.\footnote{During this period, a second, Danish-operated Lutheran mission was also established in the Danish West Indies \citep[see, e.g.,][]{Larsen_1950}. The Lutherans likewise left important documentation relevant to the study of Virgin Islands language history.} By 1768, they had baptized some 4,500 Africans and African Caribbeans; by 1794, that number had climbed to 9,300, and by 1815 it reached approximately 15,000 \citep[146–47]{Highfield_1994}. In 1804, by comparison, the total (and all-time maximum) number of enslaved people on St. Croix came to approximately 27,000 \citep[151]{Simonsen_Olsen_2017}. In each decade from 1760 to 1800, \citet[190]{Simonsen_2017} estimates, the Moravians baptized between 8\% and 16\% of the enslaved African population in St. Croix. By \citeauthor{Highfield_1994}’s (\citeyear{Highfield_1994}: 147) estimate, the Brethren had, in effect, enrolled more than half of all the enslaved people in the Danish West Indies within approximately 70 years of the onset of their enterprise.

The success of the Moravians may be attributed largely to the fact that the enslaved had lost their kinship ties after having been captured, sold, and forcibly transported to the Caribbean. \citet[92--95]{Sensbach_2005} describes how the Moravians offered an alternative kinship system that appealed to the enslaved, and which afforded the members of the congregation certain opportunities and a chance for social elevation that was otherwise unavailable to those at the bottom of the slave society.

In the following sections, I will first consider quantitative evidence of the slave trade to the Danish West Indies, after which I turn to analyzing the various sources of documentation recorded by the Moravian Brethren. In this connection, it is important to note that I assume, though not uncritically, that Seidel’s list from St. Thomas, the baptismal records from St. Croix, and Oldendorp’s account based on slave interviews conducted on both islands are in general alignment with respect to their application of ethnic categories such as, e.g., “Kongo” and “Amina,” and that the connection between such ethnic categories and their correlated languages is extrapolatable from the linguistic data presented by Oldendorp in his ethnographic survey. This assumption, corroborated by \citet[190--191]{Simonsen_2017}, rests on the fact that the information considered was obtained directly from the people concerned. Oldendorp stated that his information was “based on their own narrations” \citep[159]{Highfield_1987}, often in the context of peer group interviews, on both St. Thomas and St. Croix.


\section{Evidence of the transatlantic and intra-Caribbean slave trade to the Danish West Indies}

\subsection{The transatlantic slave trade and Danish-African trade relations}

Following an unsuccessful experience with indentured labor \citep{Hvid_2016}, importation of enslaved people directly from Africa to the Danish West Indies began in the 1670s, most likely in 1673 \citep[40]{Westergaard_1917}. In total, the number of enslaved Africans exported by the Danes and through Danish establishments probably totaled no fewer than around 100,000 people \citep[15]{Gobel_2016}. As the Danish slave trading forts were located on the so-called “Gold Coast” (present-day Ghana), and as the Danish West India and Guinea Company had been granted a national monopoly on shipping and trade to Africa and the West Indies \citep[12]{Gobel_2016}, it would at first sight appear reasonable to assume that a preponderance of the enslaved were drawn from the ethnic groups of that region, where especially Akan lects are, and were, widespread \citep{Hair_1967,Hair_1968}. From 1661, the Danish headquarters in Africa was Fort Christiansborg at Accra \citep{Norregaard_1966}. \citet[66]{Sabino_2012} describes the Danes’ trade relations with local groups there. The Danes’ first trading partners there were the G\~a, who were at war with the expanding Akan-speaking Akwamu. When slave import to St. Thomas commenced, the G\~a were thus in a position to sell Akan-speaking prisoners of war from the interior. After defeating the G\~a in 1681, the Akwamu began supplying the Danes with G\~a prisoners, as well as with captives from western Gbe-speaking areas that they already dominated. The Danes continued to trade with the Akwamu until the 1730s. According to \citet[135]{Parkvall_2000}, the situation sketched out here ``eventually led to an exceptionally heavy Kwa bias in the workforce of the Danish West Indies.''

The reality was not as straightforward as that, however, in that even more enslaved people were drawn from other areas along the West African littoral, from Senegambia in the north to Angola in the south. Also, in the early period, many of the captives landed in the colony were not necessarily from the Gold Coast region. Danish ships routinely took onboard captives from ports of call along the way to the forts, and these embarkation points were often not registered anywhere. Moreover, for a variety of reasons, the Company’s purported monopoly on shipping and trade was not actually complied with, meaning that parties other than the Danes became involved in the trade (e.g., the earlier-mentioned Brandenburg Company). Thus, a Danish planter (quoted in \cite{Nielsen_1981}: 83, my translation) wrote circa 1740 about the enslaved on St. Croix, that there were ``as many nations [\ldots] among them as there are landscapes, towns, and places in both America and especially Africa, where most of them hail from.''\footnote{This text is traditionally attributed to the plantation owner J. L. Carstens (1705–1747), but there is ``plenty of evidence that Carstens [...] did not write it,'' as \citet[219]{Sebro_2016} cautions.} 

The quote points to a mixed African-origin population, an observation that can be corroborated by considering data from the Voyages Database. A search for records of all ships with the Danish West Indies as the principal place of landing produced 381 records of voyages, representing a total of 89,829 disembarked individuals (105,253 embarked). \citet[15]{Gobel_2016} notes that, despite allowing itself some extrapolation, the Voyages Database's total number of people embarked is in overall alignment with more specialized studies on the volume and composition of the slave trade to the Danish West Indies. \tabref{tab:tab1_02} (page~\pageref{tab:tab1_02}) shows an overview of extractable data on these voyages’ embarkation regions. The figures are based on arrivals. The regions of origin are those of \citet{Eltis_Behrendt_Richardson_Klein_1999}. Based on these data, a diversity of embarkation regions for the slave trade in the Danish West Indies can be observed. The figures are visualized in Figure \ref{fig:fig1_02}.

% Figure 1:
\begin{figure}[!ht]
    \centering
    \includegraphics[width=\linewidth]{figures/fig1_02.png}
    \caption{Proportions of enslaved people from the different embarkation regions}
    \label{fig:fig1_02}
\end{figure}

Who were transported, from where they were transported, and by whom, was contingent on numerous factors, and it is evident from the slave trade data considered here that these factors varied over time. There would appear, indeed, to have been an early dominance of groups from the Gold Coast and the Bight of Benin – Akan, Gã, and Gbe speakers – and of people with unknown embarkation points, who, consequently, are subsumed under “Other Africa.”\footnote{ ``Other Africa'' is for the most part used for captives whose point of embarkation is unknown, i.e., 31,052 (35\%) of the people accounted for in Table 1 (the remaining 1,150 people of the 32,220 subsumed under this category were tracked to more specific locations). The category is in practice often used for voyages that acquired captives along the Windward Coast. This required several stops, as only small numbers of individuals could be negotiated at the different ports of call \citep[cf.][]{Vos_2010}.} West-Central Africa, where Bantu languages predominate, was a major embarkation region in the first quarter of the eighteenth century, after which the numbers dropped to zero, bouncing back only in the 1790s. The captives who landed in the later period came from a more diverse set of ethnic backgrounds, reflecting the fact that the slave trade spread to a wider range of regions in the latter part of the eighteenth century. The Gold Coast still supplied the most enslaved people to the colony, whereas the Bight of Benin evidently was of much less relevance in this later phase. Moreover, between the mid-1770s and 1800, the Bight of Biafra, which is home to a different group of languages (especially different Benue-Congo subgroups), contributed about 12\% of the enslaved people landed in the Danish West Indies. In addition, Senegambia, Sierra Leone, and the Windward Coast, which likewise are representative of a different range of languages (Mande, North-Atlantic, Kru, etc.), continuously supplied smaller quantities of human cargo, which, as noted, can be presumed to have been larger in reality than the recorded trade indicates.

In evaluation, it would appear plausible, based on these data, that “substantial ‘ethnic diversification’ did not occur until the latter part of the eighteenth century,” as \citet[66]{Sabino_2012} has argued. However, the transatlantic trade was not the only source of slave imports to the islands. Indeed, when factoring in other sources of evidence, it becomes difficult to uphold what might be termed the ``heavy Kwa bias'' hypothesis, at least in its strong form. The remainder of the study corroborates this view.

% Table 1:
\begin{table}[p]
\centering
\resizebox{\linewidth}{!}{
\begin{tabular}{lllllllllll}
    \lsptoprule
    \multicolumn{1}{r}{\rotatebox{90}{\text{Year range}}} & \multicolumn{1}{r}{\rotatebox{90}{\text{Senegambia and offshore Atlantic}}} & \multicolumn{1}{r}{\rotatebox{90}{\text{Sierra Leone}}} & \multicolumn{1}{r}{\rotatebox{90}{\text{Windward Coast}}} & \multicolumn{1}{r}{\rotatebox{90}{\text{Gold Coast}}} & \multicolumn{1}{r}{\rotatebox{90}{\text{Bight of Benin}}} & \multicolumn{1}{r}{\rotatebox{90}{\text{Bight of Biafra and Gulf of Guinea islands}}} & \multicolumn{1}{r}{\rotatebox{90}{\text{West-Central Africa and St. Helena}}} & \multicolumn{1}{r}{\rotatebox{90}{\text{Southeast Africa and Indian Ocean islands}}} & \multicolumn{1}{r}{\rotatebox{90}{\text{Other Africa}}} & \multicolumn{1}{r}{\rotatebox{90}{\text{Totals}}} \\ \cmidrule(lr){1-11}
    1676–1700 & 299 & 0 & 0 & 1,658 & 3,508 & 0 & 938 & 0 & 9,937 & 16,340 \\
    \textit{Percentage} & \textit{1.8} & \textit{0.0} & \textit{0.0} & \textit{10.1} & \textit{21.5} & \textit{0.0} & \textit{5.7} & \textit{0.0} & \textit{60.8} & \textit{100.0} \\
    \tablevspace
    1701–1725 & 97 & 0 & 0 & 2,124 & 295 & 536 & 6,433 & 0 & 3,918 & 13,403 \\
    \textit{Percentage} & \textit{0.7} & \textit{0.0} & \textit{0.0} & \textit{15.8} & \textit{2.2} & \textit{4.0} & \textit{48.0} & \textit{0.0} & \textit{29.2} & \textit{100.0} \\
    \tablevspace
    1726–1750 & 0 & 0 & 0 & 3,368 & 242 & 0 & 0 & 21 & 742 & 4,373 \\
    \textit{Percentage} & \textit{0.0} & \textit{0.0} & \textit{0.0} & \textit{77.0} & \textit{5.5} & \textit{0.0} & \textit{0.0} & \textit{0.5} & \textit{17.0} & \textit{100.0} \\
    \tablevspace
    1751–1775 & 883 & 639 & 636 & 7,284 & 62 & 0 & 0 & 0 & 3,172 & 12,676 \\
    \textit{Percentage} & \textit{7.0} & \textit{5.0} & \textit{5.0} & \textit{57.5} & \textit{0.5} & \textit{0.0} & \textit{0.0} & \textit{0.0} & \textit{25.0} & \textit{100.0} \\
    \tablevspace
    1776–1800 & 99 & 543 & 0 & 10,251 & 0 & 3,602 & 5,346 & 0 & 9,312 & 29,153 \\
    \textit{Percentage} & \textit{0.3} & \textit{1.9} & \textit{0.0} & \textit{35.2} & \textit{0.0} & \textit{12.4} & \textit{18.3} & \textit{0.0} & \textit{31.9} & \textit{100.0} \\
    \tablevspace
    1801–1825 & 58 & 220 & 53 & 780 & 61 & 3,485 & 2,724 & 0 & 4,695 & 12,076 \\
    \textit{Percentage} & \textit{0.5} & \textit{1.8} & \textit{0.4} & \textit{6.5} & \textit{0.5} & \textit{28.9} & \textit{22.6} & \textit{0.0} & \textit{38.9} & \textit{100.0} \\
    \tablevspace
    1826–1850 & 0 & 277 & 0 & 0 & 0 & 1,087 & 0 & 0 & 444 & 1,808 \\
    \textit{Percentage} & \textit{0.0} & \textit{15.3} & \textit{0.0} & \textit{0.0} & \textit{0.0} & \textit{60.1} & \textit{0.0} & \textit{0.0} & \textit{24.6} & \textit{100.0} \\
    \midrule
    Totals & 1,436 & 1,679 & 689 & 25,465 & 4,168 & 8,710 & 15,441 & 21 & 32,220 & 89,829 \\
    \textit{Percentage} & \textit{1.6} & \textit{1.9} & \textit{0.8} & \textit{28.3} & \textit{4.6} & \textit{9.7} & \textit{17.2} & \textit{0.0} & \textit{35.9} & \textit{100.0} \\
    \lspbottomrule
\end{tabular}
}
\caption{Estimates of slave imports to the Danish West Indies via the transatlantic trade in 25-year periods, arranged by embarkation regions (based on the Voyages Database, accessed 30 September 2019). Absolute numbers in plain font, percentages for each period in italics.}
\label{tab:tab1_02}
\end{table}




\subsection{The intra-Caribbean slave trade}

In the eighteenth century, \citet[150]{Simonsen_Olsen_2017} note, “[i]t was the import of enslaved people from Africa and, secondarily, the import of enslaved people from the surrounding Caribbean islands, inter alia St. Christopher [i.e., St. Kitts, KFB], Anguilla, and Tortola, which created the population growth in the Danish West Indies” (my translation). I will not attempt to account for the local trade in its totality, part of which is unrecorded and hence unquantifiable, but I will offer an example that can illustrate what it implies in terms of modifying and nuancing the picture based solely on the transatlantic slave trade data.

Between 1766 and 1802, based on the years for which reliable figures are available \citep[cf.][208--209]{GreenPedersen_1975}, approximately 30,150 enslaved people were imported to St. Croix, both directly from Africa and via the local intra-Caribbean trade. Of these people, 8,444 were landed via the local trade, i.e., at least some 28\% of the total number. As these figures do not take into account the unrecorded and illicit trade (cf. \cite{Gobel_2016}: 174 and the references therein), the actual share was larger. Moreover, when looking into individual years, this percentage was often higher yet. This was especially so prior to the ten-year grace period following 1792, the year that the proclamation abolishing the Danish slave trade ten years hence went into effect, which prompted a significant increase in the transatlantic trade to the Danish West Indies (cf. \cite{Gobel_2016}: 138 ff.).

Based on inspection of tax rolls for the period 1767–1776 in the Danish National Archives, \citet[229]{Olsen_1988} calculated the percentages of enslaved people landed on St. Croix from Africa and other (i.e., primarily, Caribbean) locations during this period. \tabref{tab:tab2_02} summarizes these results. For a number of the years accounted for by Olsen, the majority of the enslaved landed via the recorded trade did not actually arrive at the islands on vessels directly from Africa. Over the entire period, roughly half of the enslaved came from other parts of the Atlantic world, many of them from other Caribbean islands. It should be noted that some of the enslaved landed via this trade did, in fact, come from Africa but had been re-exported via other islands \citep[cf.][233--235]{Olsen_1988}. Needless to say, this either complicates the picture or renders impossible the task of tracing them to specific embarkation points.

% Table 2:
\begin{table}
\begin{tabular}{rrrr}
    \lsptoprule
    {} & {} & \multicolumn{2}{c}{Directly from Africa} \\ \cmidrule(lr){3-4}
    Year & Slave imports & $N$ & \%\\\midrule
    1767 & 1,962 & 468 & 23.8 \\
    1768 & 1,686 & 1,163 & 70.1 \\
    1769 & − & (68) & − \\
    1770 & 1,862 & 1,131 & 60.7 \\
    1771 & 1,157 & 835 & 72.1 \\
    1772 & 850 & 588 & 69.1 \\
    1773 & 241 & 0 & 0.0 \\
    1774 & 1,017 & 321 & 31.5 \\
    1775 & 1,670 & 743 & 44.4 \\
    1776 & 1,080 & 706 & 65.3 \\\midrule
    1767–1776 & 11,525 & 6,023/5,955 & 52.0 \\ \lspbottomrule
\end{tabular}
\caption{Percentages of enslaved people landed on St. Croix from Africa and elsewhere, 1767–1776 (based on \cite{Olsen_1988}: 229)}
\label{tab:tab2_02}
\end{table}

Due to the extensive local trade, it follows that it is necessary, in addition, to consider other sources of evidence for the origins of the enslaved population in the Danish West Indies. Accordingly, the remainder of the study will focus on the language and ethnicity documentation provided by the Moravian Brethren, recorded locally in the colony.


\section{Oldendorp's interview data}

``Ideally,'' \citet[2]{Kouwenberg_2008} states, the substratist researcher ``hope[s] to find clear and incontrovertible linguistic evidence'' for the presence of particular African languages, ``supported by the historical record.'' Following this lead, this section aims to establish credible links between the African languages documented in the historical material from the Danish West Indies with modern-day correlate languages and language groups. To this end, I analyze linguistic data and (other) cultural information recorded by C.G.A. Oldendorp (1721–1787) of the Moravian Brethren for an ethnographic survey.

Oldendorp spent about two years in the Danish West Indies, from 1767 to 1768. His stay resulted in a highly influential account of the islands and the Moravian Mission’s history up to that point in time, including a detailed description of the life and cultural affinities of the enslaved in the colony (for background information on Oldendorp's account, see \cite{Meier_Stein_Palmie_Ulbricht_2010}).\footnote{In addition, Oldendorp also compiled his \emph{Criolisches Wörterbuch} (\citeyear{Oldendorp_1996}) during his stay.} Oldendorp completed his massive report in 1776, and it was published – albeit in significantly abridged form – the following year \citep{Oldendorp_1777}.\footnote{For a discussion of Oldendorp’s reaction to the abridgment of his report, see, e.g., \citet{Ahlback_2016}.} In 1987, an English translation of the 1777 edition appeared \citep{Highfield_1987}, which was then followed by a critically edited multi-volume version of the original manuscript in its entirety \citep{Oldendorp_2000, Oldendorp_2002}. Note that in the following, I quote from both the critically edited German version and the English translation of the account. 

An essential part of Oldendorp’s account is an ethnographic survey, based on interviews with approximately 70 informants in St. Thomas and St. Croix. Oldendorp’s study was to provide insights into the conditions of the enslaved by mapping the various “nations” represented in the colony, and by providing an overview of the various religious ideas represented among them. Oldendorp was well aware that there were cultural differences among the enslaved, remarking that “[t]he Negroes are divided into many nations, among which greater or lesser differences can be found in regard to language, as well as morals, customs, and religion” \citep[160]{Highfield_1987}. Oldendorp describes how he gathered his informants into peer groups according to their diverse origins, and how he then asked them about their (ancestral) religious customs and practices, the geographical location of their homelands, neighboring groups (both allies and enemies), trade relations, etc. There is reason to highlight the importance of the fact that Oldendorp based his description on information from groups that agreed that they belonged to the same ethnic category, and on that basis shared information. When several people agreed to accept each other as peers, there is good reason to believe that the information that came out of it also reflected real notions of a community, or at least cultural commonalities (cf. also \cite{Sebro_2010}: 69).

In addition, Oldendorp collected linguistic data from his interviewees. He recorded label-meaning correlations of the numerals ‘one’ to ‘ten’ (and selected higher numbers) across 22 languages \citep[cf.][458--460]{Oldendorp_2000}, as well as a set of thirteen ``basic'' meanings, such as, e.g., `head' and `sun,' across 26 languages (pp. 460--463), with some gaps in the distribution. There is, in addition, a single sentence translated into the various languages: \emph{Christus hat uns geliebet und gewaschen von den Sünden mit seinem Blut} ‘Christ has loved us and has washed away our sins with his blood’ (pp. 464--465). According to \citeauthor{Jones_2010} (\citeyear{Jones_2010}: 181), the full corpus consists of 667 words in the different African languages.

\citet[194]{Ahlback_2016} notes that the publication of Oldendorp’s original manuscript in 2000–2002 “has been welcomed as providing an outstanding source to eighteenth-century African Caribbean cultures, languages and identities.” As of yet, however, no actual language-by-language examination of Oldendorp’s linguistic data, as found in this version of the account, has appeared. The two previous analyses, namely Peter Stein’s editorial notes in \citet{Oldendorp_2000} and \citeauthor{Jones_2010}’ (\citeyear{Jones_2010}) useful overview, constitute valuable contributions to such an end, as does the examination by \citet{Fodor_1975} of the 1777 edition. In the following, I consider, in a concise analysis, the information collected by Oldendorp, identifying the languages (or, alternatively, the closest matches found) with present-day correlates organized by affiliation, on the basis of the linguistic data. I compare Oldendorp's collected numerals from `one' to `ten' with precisely transcribed modern lists of numerals in African languages. Unless otherwise noted, the modern numerals have been extracted from the database \emph{Numeral Systems of the World’s Languages} – an online repository of basic data on more than 4,000 of the world’s languages, formerly hosted at the Department of Linguistics at the MPI for Evolutionary Anthropology in Leipzig.\footnote{The database is currently available at \url{https://lingweb.eva.mpg.de/channumerals/} (last accessed 29 November 2024). The database forms the basis for the ``Numeralbank'' project carried out under the Glottobank research consortium: \url{https://glottobank.org/} (last accessed 29 November 2024).} Where available and where relevant (i.e., in practice, where the numerals data are either lacking or inconclusive), I refer also to some of the other linguistic data recorded by Oldendorp. In addition, I refer to cultural and geographic clues offered in the account, as well as to previous work within the Africanist literature and the aforementioned previous analyses of Oldendorp’s material.

\subsection{Identification of languages in Oldendorp's account}

The sets of numerals referred to in the following analysis are presented collectively in \tabref{tab:tab3_02}. The presentation of languages is based on the order in which they are discussed below. All modern languages are identified by their three-letter ISO 639-3 codes, marked by square brackets. The designations used by Oldendorp (which may represent ethnonyms or glossonyms, or both) are presented within double quotation marks. All languages which Oldendorp recorded direct evidence on are represented in the discussion.

\renewcommand{\arraystretch}{0.8} % Adjusts row height; default is 1
\setlength{\tabcolsep}{4pt} % Reduces column padding; default is usually 6pt

\begin{sidewaystable}
\scriptsize % Reduce font size
\begin{tabular}{p{3cm}llllllllll}
    \lsptoprule
    ``Nation''/language & 1 & 2 & 3 & 4 & 5 & 6 & 7 & 8 & 9 & 10 \\ \midrule
    \multicolumn{11}{c}{\textbf{North Atlantic (region: Senegambia)}} \\
    \midrule
    ``Fula'' & \textit{go} & \textit{didi} & \textit{taddi} & \textit{nei} & \textit{djoi} & \textit{djego} & \textit{tjedidi} & \textit{jenai} & \textit{jädet} & \textit{sappoi} \\
    Pulaar {[}fuc{]} & \textit{ɡoo} & \textit{ɗiɗi} & \textit{tati} & \textit{naj} & \textit{d͡ʒoj} & \textit{d͡ʒeeɡom} & \textit{d͡ʒeeɗiɗi} & \textit{d͡ʒeetati} & \textit{d͡ʒeenaj} & \textit{sappo} \\
    \tablevspace
    \multicolumn{11}{c}{\textbf{Mande (region: interior Windward Coast)}} \\
    \midrule
    ``Jalunkan'' & \textit{keling} & \textit{filla} & \textit{saba} & \textit{nani} & \textit{lolu} & \textit{worro} & \textit{orwila} & \textit{sagi} & \textit{kononto} & \textit{tan} \\
    Yalunka {[}yal{]} & \textit{kèdé} & \textit{fìríŋ} & \textit{sàkáŋ} & \textit{nànì} & \textit{sùlù} & \textit{sènì} & \textit{fòlófɛ̀rɛ́} & \textit{fòlòmàsàkáŋ} & \textit{fòlòmànànì} & \textit{fù} \\
    Dyula {[}dyu{]} & \textit{kelen} & \textit{filà} & \textit{sàbà} & \textit{nàànìn} & \textit{dùùrù} & \textit{wɔ̀ɔ̀rɔ̀} & \textit{wolon fìlà} & \textit{sieɡi} & \textit{kɔ̀nɔ̀ndon} & \textit{tan} \\
    Jalkunan {[}bxl{]} & \textit{dúlì} & \textit{fìlɑ̀} & \textit{siɡ͡bù} & \textit{nɑ̄ːnī} & \textit{sōːlō} & \textit{mìːlù} & \textit{mɑ̀ɑ́lɑ̀} & \textit{mɑ̀sīɡ͡bū} & \textit{mɑ́nɑ̄nì} & \textit{tɑ̄} \\
    ``Sokko'' & \textit{külle} & \textit{felaa} & \textit{sauaa} & \textit{nani} & \textit{duli} & \textit{woro} & \textit{ornala} & \textit{setti} & \textit{konundo} & \textit{tang} \\
    \tablevspace
    \multicolumn{11}{c}{\textbf{Gur (region: North Ghana, South Burkina Faso)}} \\
    \midrule
    ``Tembu'' & \textit{kuddum} & \textit{noalee} & \textit{nodosoo} & \textit{nonasaa} & \textit{nonoaa} & \textit{lodo} & \textit{lubbe} & \textit{lütoso} & \textit{kandilee} & \textit{figuh} \\
    Tem {[}kdh{]} & \textit{káɔ́ɖe} & \textit{sííɛ̀} & \textit{tóózo} & \textit{nááza} & \textit{nʊ́ʊ́wa} & \textit{loɖo} & \textit{lʊbɛ} & \textit{lutoozo} & \textit{kéénííré} & \textit{fuú} \\
    Lukpa {[}dop{]} & \textit{kʊ̀lʊ̀m} & \textit{naalɛ̀} & \textit{tòòsó} & \textit{naasá} & \textit{kàk͡pásɪ̀} & \textit{náátòsò} & \textit{náátòsòm̀pɔ̀ɣɔ̀laɣá} & \textit{pə́lé fɛ́jɪ́} & \textit{pɔ̀ɣɔ̀láɣáfɛ́jɪ́́} & \textit{náánʊ́á} \\
    ``Tjamba'' & \textit{obaa} & \textit{illee} & \textit{ittaa} & \textit{inna} & \textit{immu} & \textit{iloop} & \textit{illelee} & \textit{imenn} & \textit{üwäh} & \textit{piek} \\
    Kasem {[}xsm{]} & \textit{kàlʊ̀} & \textit{ǹlè} & \textit{ǹtɔ̀} & \textit{ǹnā} & \textit{ǹnū} & \textit{ǹdʊ̀n} & \textit{ǹpɛ̀} & \textit{nānā} & \textit{nʊ̀ɡʊ̄} & \textit{fúɡə́} \\
    Konkomba {[}xon{]} & \textit{-bàa} & \textit{-lèe} & \textit{-tàa} & \textit{-nāa} & \textit{-nmúu} & \textit{-lúub} & \textit{-lílé} & \textit{-niín} & \textit{-wɛ́ɛ} & \textit{píìk} \\
    \tablevspace
    \multicolumn{11}{c}{\textbf{Kru (region: Liberia, Ivory Coast)}} \\
    \midrule
    “Gien” & \textit{do} & \textit{sung} & \textit{ta} & \textit{nje} & \textit{mu} & \textit{medu} & \textit{mesong} & \textit{medda} & \textit{menje} & \textit{wo} \\
    Krahn {[}krw{]} & \textit{tòò} & \textit{sɔɔ̌n} & \textit{ta̓a̓n} & \textit{nyìɛ̓} & \textit{m̀m̌} & \textit{mɛ̀o̓} & \textit{mɛ̀sɔɔ̌n} & \textit{mɛta̓a̓ǹ} & \textit{mɛ̀nyìɛ̓} & \textit{pùèè} \\
    ``Kanga'' & \textit{aniandu} & \textit{aniasson} & \textit{anietan} & \textit{ananje} & \textit{aneamu} & − & − & − & − & \textit{aniepun} \\
    Kyanga {[}tye{]} & \textit{dúú} & \textit{fʸáā} & \textit{ˀāàː} & \textit{ʃíí} & \textit{sɔ́ɔ́rū} & \textit{pɛ́ɛ́níí} & \textit{jɪ́í} & \textit{ɔ́ɖʊ́ɔ́} & \textit{dʊ́kʊ́} & \textit{ɪ́ɪ́} \\
    \bottomrule
\end{tabular}
\caption{Numerals from `one' to `ten' across 23 languages recorded by Oldendorp compared with data on 26 modern African (Niger-Congo) languages and their subgroups}
\label{tab:tab3_02}
\end{sidewaystable}

\renewcommand{\arraystretch}{0.8} % Adjusts row height; default is 1
\setlength{\tabcolsep}{4pt} % Reduces column padding; default is usually 6pt

\begin{sidewaystable}
\ContinuedFloat
\centering
\scriptsize % Reduce font size
\fittable{
\begin{tabular}{p{3cm}llllllllll}
    \toprule
    ``Nation''/language & 1 & 2 & 3 & 4 & 5 & 6 & 7 & 8 & 9 & 10 \\ \midrule
    \multicolumn{11}{c}{\textbf{Gbe, Kwa; Yoruboid, Benue-Congo (region: Togo, Benin)}} \\
    \midrule
    ``Papaa'' & \textit{depoo} & \textit{auwi} & \textit{ottong} & \textit{enne} & \textit{attong} & \textit{attugo} & \textit{atjuwe} & \textit{attiatong} & \textit{atjenne} & \textit{awo} \\
    ``Wavu'' & \textit{depoo} & \textit{awee} & \textit{etong} & \textit{enne} & \textit{attong} & \textit{a-isee} & \textit{djaui} & \textit{tiatong} & \textit{tienee} & \textit{wo} \\
    \textit{Anders} (or: “Wavu II”) & \textit{baba} & \textit{bauli} & \textit{janna} & \textit{tofla} & \textit{guena} & \textit{brong} & \textit{jegra} & \textit{khiboa} & \textit{boafri} & \textit{magro} \\
    ``Watje'' & \textit{de} & \textit{ewee} & \textit{etong} & \textit{enne} & \textit{attong} & \textit{andee} & \textit{anderee} & \textit{enni} & \textit{enjidee} & \textit{owoo} \\
    Fon {[}fon{]} & \textit{ɖě} & \textit{we} & \textit{atɔn} & \textit{ɛnɛ} & \textit{atɔ́ɔ́n} & \textit{ayizɛ́n} & \textit{tɛ́nwe} & \textit{tántɔn} & \textit{tɛ́nnɛ} & \textit{wǒ} \\
    Aja {[}ajg{]} & \textit{eɖé, ɖeka} & \textit{èvè, amɛ̃ve} & \textit{etɔ̃̂, amɛ̃tɔ̃} & \textit{enɛ̀, amɛ̃nɛ̃} & \textit{atɔ̃, amãtɔ̃} & \textit{adɛ̃, amãdɛ̃} & \textit{adɾɛ, amãdɾɛ} & \textit{eɲĩ, amɛ̃ɲĩ} & \textit{ɲíɖe, aʃiɖekɛ} & \textit{ewó} \\
    Ewe {[}ewe{]} & \textit{èɖé} & \textit{èvè} & \textit{ètɔ̃} & \textit{ènè} & \textit{àtɔ̃} & \textit{àdẽ́} & \textit{adrẽ́} & \textit{èɲí} & \textit{aʃíeké} & \textit{èwó} \\
    \tablevspace
    \multicolumn{11}{c}{\textbf{Gã-Dangme, Kwa (region: coastal Ghana)}} \\
    \midrule
    ``Tambi'' & \textit{kaki} & \textit{ennu} & \textit{ette} & \textit{ewe} & \textit{enu} & \textit{ekba} & \textit{pagu} & \textit{panjo} & \textit{ne} & \textit{njomma} \\
    ``Akkran'' & \textit{eaku} & \textit{eenjo} & \textit{ette} & \textit{eedjee} & \textit{ennumo} & \textit{epa} & \textit{paggu} & \textit{paniu} & \textit{nehung} & \textit{jungma} \\
    G\~a {[}gaa{]} & \textit{ékòmé} & \textit{éɲɔ̀} & \textit{étɛ̃} & \textit{éɟwɛ̀} & \textit{énùmɔ̃} & \textit{ék͡pàa} & \textit{k͡pàwo} & \textit{k͡pàaɲɔ̃} & \textit{nɛ̀ɛhṹ} & \textit{ɲɔ̀ŋmá} \\
    Dangme {[}ada{]} & \textit{kákē} & \textit{éɲɔ̃̀} & \textit{étɛ̃̄} & \textit{éywɛ̀, éwìɛ̀} & \textit{énũ̄ɔ̃̄} & \textit{ék͡pà} & \textit{k͡pààɡō} & \textit{k͡pàaɲɔ̃̄} & \textit{nɛ̃̀ɛ̃́} & \textit{ɲɔ̃̀ŋ͡mã́} \\
    \tablevspace
    \multicolumn{11}{c}{\textbf{Cross-River, Benue-Congo; Igboid, Benue-Congo (region: Nigeria)}} \\
    \midrule
    ``Karabari'' & \textit{otuh} & \textit{abolam} & \textit{attoo} & \textit{abanna} & \textit{abisee} & \textit{abisih} & \textit{abassa} & \textit{abassatto} & \textit{abitollu} & \textit{abilli} \\
    ``Ibo'' & \textit{otuh} & \textit{aboa} & \textit{attoo} & \textit{anoo} & \textit{issee} & \textit{issih, tschi} & \textit{assaa} & \textit{assatto} & \textit{itlelite} & \textit{ili} \\
    Ijo Kalabari {[}ijn{]} & \textit{ŋ̀ɡèi} & \textit{màɪ̃} & \textit{tɛrɛ} & \textit{ineĩ} & \textit{sɔnɔ} & \textit{sonio} & \textit{sɔnɔmɛ̀} & \textit{ninè} & \textit{esenie} & \textit{oji, \`at\`ei} \\
    Igbo {[}ibo{]} & \textit{ótù} & \textit{àbʊ̄ɔ́} & \textit{àtɔ́} & \textit{ànɔ́} & \textit{ìsé} & \textit{ìsiì} & \textit{àsáà} & \textit{àsátɔ́} & \textit{ìtólú} & \textit{ìri} \\
    ``Mokko'' & \textit{kiä} & \textit{iba} & \textit{itta} & \textit{inan} & \textit{üttin} & \textit{itjüekee} & \textit{ittiaba} & \textit{itteiata} & \textit{huschukiet} & \textit{büb} \\
    Efik {[}efi{]} & \textit{kíét} & \textit{íbá} & \textit{ítá} & \textit{ínáŋ} & \textit{ítíón} & \textit{ítíókíét} & \textit{ítíábá} & \textit{ítíáitá} & \textit{úsúk-kíét} & \textit{dúóp} \\
    \tablevspace
    \multicolumn{11}{c}{\textbf{Bantu, Benue-Congo (region: coastal and interior West-Central Africa)}} \\
    \midrule
    ``Loango'' & \textit{bosse} & \textit{quari} & \textit{tattu} & \textit{ena} & \textit{tanu} & \textit{sambaan} & \textit{sambueri} & \textit{nane} & \textit{iwoa} & \textit{kumi} \\
    ``Congo'' & \textit{moschi} & \textit{sole} & \textit{sitattu} & \textit{sija} & \textit{sittan} & \textit{issamban} & \textit{samboari} & \textit{sinaan} & \textit{siwoa} & \textit{sikumi} \\
    ``Camba'' & \textit{moschi} & \textit{soli} & \textit{tattu} & \textit{ja} & \textit{tanu} & \textit{saman} & \textit{sambari} & \textit{nane} & \textit{wa} & \textit{komi} \\
    Kikongo {[}kng{]} & \textit{mɔ́sì} & \textit{zòólè} & \textit{tátù} & \textit{yá} & \textit{tánù} & \textit{sámbánù} & \textit{nsàmbwádì} & \textit{nànà} & \textit{vwà} & \textit{kùmì} \\
    ``Mandongo'' & \textit{omma} & \textit{meere} & \textit{metutu} & \textit{mina} & \textit{metaan} & \textit{schiauno} & \textit{entschewine} & \textit{ennane} & \textit{woa} & \textit{kumi} \\
    Kimbundu {[}kmb{]} & \textit{môxì} & \textit{yâdí} & \textit{tâtù} & \textit{wânà} & \textit{tânù} & \textit{sámánù} & \textit{sámbwádì} & \textit{(dí)nâkè} & \textit{(dí)vwà} & \textit{(dí)kwìnyì} \\
    Ndumu {[}nmd{]} & \textit{-mɔ} & \textit{-ɛɛlɛ, -ɔɔlɔ} & \textit{-tati} & \textit{-na} & \textit{-taani} & \textit{-sameni} & \textit{tsaami} & \textit{pwɔmɔ} & \textit{wua} & \textit{kumu} \\
    Mbere {[}mdt{]} & \textit{-mɔ} & \textit{-ele} & \textit{-tare} & \textit{-na} & \textit{-taani} & \textit{-syaami} & \textit{ntsaami} & \textit{mpfuɔmɔ} & \textit{wa} & \textit{kuomi} \\ \lspbottomrule
\end{tabular}
}
\caption{Numerals (continued)}
\label{}
\end{sidewaystable}



\subsubsection{``Fula'': Fula, North Atlantic}

Oldendorp’s ``Fula'' numerals match those of varieties in the Fula cluster [inclusive ISO code: ful], except for the forms for ‘eight’ and `nine.' This discrepancy is most likely due to Oldendorp, or his informant, having switched the forms \emph{jenai} and \emph{j\"adet} around. In addition, the ``Fula'' informant reported having traveled for two months to reach the ocean from his homeland, located along a great river \citep[373]{Oldendorp_2000}, which was probably the Senegal or Gambia River. Pulaar [fuc] is included in \tabref{tab:tab3_02} as the probably closest-matching candidate lect.

\subsubsection{``Mandinga'': Mandinka, Mande}

Oldendorp described the ``Mandinga'' (or ``Mandingo'') as a neighboring tribe of the ``Fula'' and ``Jalunkan'' (see below), a people related to them but differing in language \citep[376]{Oldendorp_2000}. Oldendorp recorded no ``Mandinga'' numerals, but he did include word forms for some of the other basic meanings. \tabref{tab:tab4_02} shows that correspondences with the Mande language Mandinka [mnk] can be found for four of six of these (Mandinka data adapted from \cite{Peace_Corps_The_Gambia_1995}). Thus, Oldendorp’s ``Mandinga'' most likely corresponds to a Mandinka lect, but the linguistic evidence is too scanty to allow any precise identification.

% Tabale 4:
\begin{table}[!ht]
\centering
\setlength{\tabcolsep}{4pt}
\resizebox{\linewidth}{!}{
\begin{tabular}{lllllll}
    \lsptoprule
    & {`God'} & {`Sun'} & {`Mouth'} & {`Hand'} & {`Father'} & {`Mother'} \\ \cmidrule(lr){2-7}
    “Mandinga” & \textit{Allah, Kanniba} & \textit{tille} & \textit{pandintee} & \textit{bulla} & \textit{ba} & \textit{jem} \\
    Mandinka {[}mnk{]} & \textit{Ala} & \textit{tìli} & \textit{dáa} & \textit{búlu} & \textit{baaba} & \textit{baa, naa} \\ 
    \lspbottomrule
\end{tabular}
}
\caption{Oldendorp’s “Mandinga” compared with Mandinka [mnk]}
\label{tab:tab4_02}
\end{table}

\subsubsection{``Jalunkan'' and ``Sokko'': Dyula-Bambara, Mande}

It is suggested by Peter Stein (\cite{Oldendorp_2000}: 376, fn. 54) that Oldendorp’s ``Jalunkan'' refers to the Yalunka language [yal] of Guinea and Sierra Leone. Conversely, \citet[104]{Fodor_1975} suggested that the language could be Dyula [dyu], which is mutually intelligible with the West African lingua franca (and, since the 1960s, a national language of Mali) Bambara [bam]. In addition, ``Jalunkan'' could also refer to Jalkunan [bxl] of Burkina Faso. Note that the modern Yalunka numerals `seven' through `nine' are compounds based on a unit `five.' The collected numerals display closest alignment with those of modern Dyula.

Oldendorp noted about the ``Sokko'' (or ``Asokko'') that they were Islamic, and that it took them from six to seven weeks to reach the coast from their home region, neighboring the ``Amina'' \citep[407--408]{Oldendorp_2000}, i.e., peoples of the Gold Coast \citep[cf.][247--248]{Law2005}. ``Sokko'' may well refer to the Dyula town Begho-Nsoko in interior Ghana, close to the Ivory Coast border (\cite{Jones_2010}: 186; \cite{Oldendorp_2000}: 407, fn. 84; \cite{Stahl_2001}: 124). \citet{Migeod_1913} linked the area’s people, whom \citet{Koelle_1854} called ``Jalonke,'' with the Dyula, and he wrote that they self-identified as the ``Sako'' \citep[346]{Migeod_1913}. Indeed, the ``Sokko'' numerals recorded by Oldendorp show close similarity to those of Dyula-Bambara. Moreover, nine out of thirteen of the ``Jalunkan'' and ``Sokko'' lexical items in \citet[460--461]{Oldendorp_2000} show close similarity, corroborating further that also the ``Sokko'' used a Dyula-Bambara lect.

\subsubsection{``Tembu'': Tem, Gur}

The ``Tembu'' were noted to live further inland than the ``Amina,'' four days' travel from the land of the coastal ``Akkran'' \citep[399--400] {Oldendorp_2000}. The ``Tembu'' numerals match those of the interior Gur language Tem [kdh] closely, except for the form for `two.' The related language Lukpa [dop] has a form corresponding to that for `two' given by Oldendorp, but this language does not otherwise match the “Tembu” numerals closely. Overall, the ``Tembu'' data recorded by Oldendorp are closest to Tem.

\subsubsection{``Tjamba'': Konkomba, Gurma, Gur}

Oldendorp reported that his ``Tjamba'' informants had traveled for as long as six months to reach the Gold Coast from their homeland, and that their king had his residence in the city of ``Gambaak'' \citep[403--404]{Oldendorp_2000}, i.e., Gambaga in the North East Region of Ghana. \citet[186]{Jones_2010} has suggested that Oldendorp’s ``Tjamba'' (also called ``Kassenti'' – supposedly meaning `I do not understand you') refers to the Kasem, or Chamba, language [xsm] of southern Burkina Faso, a language of the Gur group, but the linguistic data do not lend support to this view. Instead, the ``Tjamba'' numerals match those of Konkomba [xon] of the Gurma cluster, also a Gur language, spoken in Northern Ghana.

\subsubsection{``Gien'' and ``Kanga'': Kru lects}

\begin{sloppypar}
According to \citet[50--51]{Westermann_Bryan_1952}, the name “Gien” was used in reference to the Southern Wee language Krahn [krw] of the Ivory Coast and Liberia. Indeed, the collected numerals match closely between Oldendorp's ``Gien'' and Krahn. Another candidate language is the closely related lect Grebo [grj] (Grebo numerals adapted from \cite{Innes_1967}). As for the “Kanga,” although there is a Mande language called Kyanga [tye], the numerals (of which Oldendorp did not record `six' to `nine') indicate it was a Kru lect, too. Note that none of the modern Kru lects considered retains the (presumed) prefix \emph{ania}-, or \emph{anie}-, included by Oldendorp. Some additional clues given by Oldendorp indicate that these languages were varieties of Kru, and not, e.g., Mande. Oldendorp indicated that the ocean formed the western border of the ``Kanga'' territory, and that they often traded with Europeans. With their land bounded to the west by the sea, they otherwise shared borders with the ``Mandinga'' and ``Fula'' – but they did not understand their languages \citep[378]{Oldendorp_2000}. Mandinga and Fula are Mande and North Atlantic, respectively, indeed very different from Kru languages.
\end{sloppypar}

\subsubsection{``Mangree'': Another Kru language?}

\citet[382]{Oldendorp_2000} noted that a great river, which must be either the Sassandra or the Bandama \citep[cf.][162]{Fodor_1975}, constituted the border between the ``Gien'' and the ``Mangree,'' and that their languages ``[did] not differ much from one another'' \citep[162]{Highfield_1987}. Oldendorp recorded no ``Mangree'' numerals (for which reason the language does not figure in \tabref{tab:tab3_02}), but he noted that the ``Mangree'' people understood the ``Kanga'' (\citeyear{Oldendorp_2000}: 378), and that they lived deep in the interior. In addition to the ``Gien,'' the ``Mangree'' lived near the ``Mandinga'' and ``Amina'' (\citeyear{Oldendorp_2000}: 381–82). Thus, the ``Mangree'' could correspond to the Ngere (or Wee) of the Ivory Coast (cf. \cite{Jones_2010}: 186; \cite{Oldendorp_2000}: 381, fn. 58) who spoke a Wee lect (close to that of the Krahn-speaking ``Gien''). Oldendorp did record some basic ``Mangree'' lexicon. There are few correspondences, however, between the ``Mangree'' and, respectively, ``Gien'' and ``Kanga'' items \citep[cf.][460--461]{Oldendorp_2000}.

\subsubsection{``Amina'' and ``Akkim'': Akan, Potou-Tano, Kwa}

The ``Amina'' were described by Oldendorp as the most powerful ``nation'' east of the Gold Coast. Their land extended from close to the coast and well into the interior, and they waged war on their various neighbors \citep[383]{Oldendorp_2000}. The ``Amina'' numerals recorded by Oldendorp correspond closely to those of the Akan [aka] lects Twi and Fante (Fante numerals adapted from \cite{Bureau_of_Ghana_Languages_1986}), except for the form for `two' – and in the case of Fante, ‘one’. Moreover, the language of the ``Amina'' was described as being the same as that of the ``Akkim'' tribe (corresponding to modern-day Akyem), which is corroborated by the linguistic data. \citet[392]{Oldendorp_2000} noted about the ``Akkim'' that they lived close to the sea, a day’s journey from Danish Fort Christiansborg. For a recent in-depth study on Oldendorp's ``Amina,'' see also \citet[]{KelleyLovejoy_2023}.

\subsubsection{``Okwa''}

In addition, \citet[396]{Oldendorp_2000} noted having spoken with one individual from the same region as the two aforementioned groups who self-identified as an ``Okwa'' (or ``Okwoi''). However, as only the word \emph{Tschabee} `God' was recorded in this language, it has not been possible to establish its identity.

\subsubsection{``Akripon'': Larteh, South Guang, Potou-Tano, Kwa}

\citet[395]{Oldendorp_2000} noted that the ``Akripon'' shared a common border with the ``Amina,'' but they constituted a separate kingdom, and they also used the same language. ``Akripon'' is most likely a toponym corresponding to the modern town of Akropon in southern Ghana. The South Guang language Larteh [lar] is spoken there \citep{Bello_2013}. Its numerals match Oldendorp's ``Akripon'' numerals closely. Some North Guang lects match for the most part, but forms resembling Oldendorp's \emph{ebnoo} `nine' are only found in the South Guang cluster.

\subsubsection{``Papaa'' and ``Wavu'': Fon, Gbe, Kwa}

Both the ``Papaa'' and ``Wavu'' numerals match those of modern Fon [fon] (Fon numerals extracted from \cite{Lefebvre_Brousseau_2002}). Some other differences can, however, be noted between the two seemingly distinct groups. \citet[411--412]{Oldendorp_2000} stated that the ``Affong'' were the rulers of the ``Papaa'' kingdom, and that his ``Papaa'' informants were familiar with both the Danes and other Europeans, as well as with the ``Akkran'' and ``Amina'' (from whose raids they suffered). Their land extended to the sea on one side. The ``Wavu'' likewise lived in part on the coast, but also in part deep within the interior where they comprised a populous nation ``throughout which the same language is not spoken uniformly'' \citep[166]{Highfield_1987}.

\subsubsection{``Wavu II''}

\citet[132--137]{Fodor_1975} presented an in-depth discussion of Oldendorp's ``Wavu,'' which he argued actually referred to two distinct languages. The first of these is the Gbe variety identified above; the second – labeled \emph{anders} `different' (i.e., different from ``Wavu'') by Oldendorp and ``Wavu II'' by Fodor – remains obscure.\footnote{Cf. the entry for ``Wavu II'' in \emph{Glottolog}: \url{https://glottolog.org/resource/languoid/id/wavu1234} (last accessed 29 November 2024).} I have not been able to link ``Wavu II'' to any modern language.

\subsubsection{``Watje'' and ``Atje'': Aja/Ewe, Gbe, Kwa}

\citet[413, 419]{Oldendorp_2000} noted that the “Watje” were the rulers of a separate kingdom, their territory extending far inland from the sea. Their neighbors included the “Amina” (whom they were at war with), the “Tjamba” (or “Kassenti”), and the “Sokko.” The “Watje” and “Atje” were described as closely related groups, with languages that were “almost identical” (Oldendorp in \cite{Highfield_1987}: 166). Oldendorp recorded no “Atje” numerals, but the modern Gbe lects Aja [ajg] and Ewe [ewe] match the “Watje” numerals closely.

\subsubsection{``Tambi'' and ``Akkran'': G\~a-Dangme, Kwa}

The ``Tambi'' and ``Akkran'' groups were described as living next to the Danish fort. They were reported to understand the language of the neighboring war-like ``Amina,'' although they did have a language of their own \citep[396, 399, 411]{Oldendorp_2000}. The “Akkran” and ``Tambi'' numerals show alignment with the two closely related Kwa languages G\~a [gaa] and Dangme [ada], spoken in and near Accra, Ghana.

\subsubsection{``Karabari'' and ``Ibo'': Igbo, Benue-Congo}

\citet[426]{Oldendorp_2000} described the ``Karabari'' as living on the Calabar River, far from the sea. The ``Ibo,'' a numerous people who lived in a vast land in the interior, were their neighbors, and were noted to use the same language. Peter Stein \citep[426, fn. 108]{Oldendorp_2000} has suggested that the ``Karabari'' were speakers of Ijo Kalabari [ijn] (Ijoid), but the collected numerals resemble those of modern Igbo [ibo], not any of the Ijoid languages. Indeed, ``Karabari'' was a generic term used about peoples living along the Calabar River \citep[cf.][138]{Fodor_1975}. Also the ``Ibo'' numerals match those of modern Igbo.

\subsubsection{``Mokko'': Efik, Cross-River, Benue-Congo}

The ``Mokko'' shared a border with the ``Karabari'' \citep[434]{Oldendorp_2000}. The ``Mokko'' designation is not unproblematic, in that it was used for several peoples \citep[434, fn. 115]{Oldendorp_2000}, but in this context it most likely refers to the Efik-speaking Ibibio of Nigeria \citep[263]{Hair_1967}. Indeed, the ``Mokko'' numerals recorded by Oldendorp correspond to those of modern-day Efik [efi].

\subsubsection{``Loango,'' ``Congo,'' and ``Camba'': Kikongo, Bantu, Benue-Congo}

\citet[436--445]{Oldendorp_2000} noted that the ``Loango'' lived about a month's journey from the West-Central African coast, which can presumably be identified with present-day Gabon. The “Congo” lived close to the coast, and one of them knew the Portuguese-founded city Luanda, the capital of present-day Angola. The ``Camba'' lived near the ``Loango,'' not far from the ``Sundi'' (or ``Nsundi''). According to \citet[85]{Dalby_1964}, the group described as ``Sundi'' by \citet{Koelle_1854} lived in \citeauthor{Guthrie_1948}'s (\citeyear{Guthrie_1948}) Bantu Zone H. The numerals of Oldendorp's ``Camba'' and ``Congo'' match those of Kikongo [kng], except that Oldendorp's ``Congo'' takes what may be a concord prefix, \emph{si}-, for `three' to `six' and `eight' to `ten.' The ``Loango'' numerals are similar to those of Kikongo, with the exception of the ``Loango'' form \emph{quari} `two.' Moreover, eleven out of thirteen of the ``Loango'' and ``Congo'' lexical items show close similarity, and the same is the case for ten out of thirteen ``Camba'' and ``Loango'' items. Thus, the three groups all spoke varieties of Kikongo.

\subsubsection{``Mandongo'': A language of Bantu Zone B or H}

Finally, Oldendorp described the ``Mandongo'' as a widely dispersed people consisting of three scattered groups, ``bound to one another by a common language'' \citep[168]{Highfield_1987}. Some of Oldendorp's ``Mandongo'' interviewees indicated that it took an entire year for the journey from their homeland to the land of the ``Loango,'' and from there it took approximately one month to reach the sea \citep[442]{Oldendorp_2000}. It is suggested by Peter Stein (\cite{Oldendorp_2000}: 441, fn. 123, citing \cite{Birmingham_1966}) that the ``Mandongo'' designation corresponds to Ndongo, a Mbundu chiefdom near Luanda, in \citeauthor{Guthrie_1948}'s (\citeyear{Guthrie_1948}) Bantu Zone H (roughly, northwestern Angola to western Congo). However, there is only partial alignment between Oldendorp's ``Mandongo'' and the then expected correlate language Kimbundu [kmb]. \citet[152]{Fodor_1975} suggested that the ``Mandongo'' language belonged in the Mbete (or Mbere) cluster of Bantu Zone B (roughly, southern Gabon over western Congo), the same as, e.g., Ndumu [nmd] or Mbere [mdt]. Its precise identity remains uncertain, however. The collected numerals show closest similarity with various Zone B and H Bantu languages, but no one modern Bantu language emerges as a surefire match.


\subsection{Summary and short assessment}

Whereas early African language materials (where available) often suffer from imprecise transcription \citep[95--97]{Jones_1991}, Oldendorp's reporting has long been recognized, and commended, for its accuracy (e.g., \cite{Herskovits_1958}: 44; see also, e.g., \cite{Ahlback_2016}). Corroborating this view, the above analysis has shown that, for the most part, the linguistic data in Oldendorp's account are readily comparable with data on modern African languages. Differences could be due to dialectal varieties, or language changes in the past two and a half centuries. Oldendorp's ``Mangree'' and ``Okwa'' could not be identified with any present-day language, neither could the language \citet{Fodor_1975} labeled ``Wavu II.'' These cannot be ruled out to have been spurious. Further, the Bantu language ``Mandongo'' was only tracked to an approximate location on the West-Central African coast. Besides these, however, it was possible to link the remainder of Oldendorp's languages confidently with modern-day correlates (as summarized in \tabref{tab:tab5_02}).

% Table 5:
\begin{table}[t]
\begin{tabularx}{\textwidth}{lQl}
    \lsptoprule
    Language in Oldendorp’s account & Modern-day African language(s) & Affiliation \\ \midrule
    ``Fula'' & A language of the Fula cluster {[}ful{]}, perhaps Pulaar {[}fuc{]} & North Atlantic \\
    ``Mandinga'' & Presumably Mandinka {[}mnk{]} & Mande \\
    ``Jalunkan'' and “Sokko” & Dyula-Bambara {[}dyu{]} & Mande \\
    ``Tembu'' & Tem {[}kdh{]} & Gur \\
    ``Tjamba'' & Konkomba {[}xon{]} & Gur \\
    ``Gien'' and ``Kanga'' & Kru lects, presumably Krahn {[}krw{]} and/or Grebo {[}grj{]} & Kru \\
    ``Mangree'' & − & Kru, presumably \\
    ``Amina'' and ``Akkim'' & Akan {[}aka{]} lects & Kwa \\
    ``Okwa'' & − & − \\
    ``Akripon'' & Larteh {[}lar{]} & Kwa \\
    ``Papaa'' and ``Wavu'' & Fon {[}fon{]}, Gbe & Kwa \\
    ``Watje'' and ``Atje'' & Aja {[}ajg{]} and/or Ewe {[}ewe{]}, Gbe & Kwa \\
    ``Tambi'' and ``Akkran'' & G\~a {[}gaa{]} and/or Dangme {[}ada{]} & Kwa \\
    ``Karabari'' and ``Ibo'' & Igbo {[}ibo{]}, Igboid & Benue-Congo \\
    ``Mokko'' & Efik {[}efi{]}, Cross-River & Benue-Congo \\
    ``Loango,'' ``Congo,'' and ``Camba'' & Kikongo {[}kng{]}, Bantu & Benue-Congo \\
    ``Mandongo'' & A Bantu language of Zone B or H & Benue-Congo \\ \lspbottomrule
\end{tabularx}
\caption{Modern-day correlates to the African languages documented by Oldendorp}
\label{tab:tab5_02}
\end{table}

The map in Figure \ref{fig:map1_02} shows the approximate geographic distribution of the African languages found to be represented in the Danish West Indies based on the analysis of Oldendorp's material. In line with the analysis in §3, the languages can be seen to come from a stretch that extends from present-day Senegal to Angola – over more than 4,000 km. Most languages are spoken off the coastline from Ghana to Benin, but also Nigerian and West-Central African languages have a clear representation, as do languages from the continent’s westernmost coast. Strikingly, several languages are from far into the interior, months of travel from the coast.

\begin{figure}
    \centering
    \includegraphics[width=0.8\linewidth]{figures/map1_02a.png}
    \caption{Approximate locations of the identifiable languages listed in Table 5, represented by ISO 639-3 codes (excepting the imprecisely identified ``Mandongo'' Bantu language)}
    \label{fig:map1_02}
\end{figure}

Building on the above analysis, the following two sections will examine Seidel's list of Christianized Africans and African Caribbeans from St. Thomas and the baptismal records from St. Croix, looking at similarities and differences in the distribution of the identified ethnolinguistic groups and their proportions between the two islands.


\section{St. Thomas: Seidel’s list of Christianized Africans and African Caribbeans}

In 1753, the Moravian missionary Nathanael Seidel (1718–1782) visited the Danish West Indies for a duration of two months. During his time there, he recorded observations for a report on the status of the mission work (\cite{Seidel_1753}). As part of his report, Seidel compiled a list of the different African ``nations'' represented among the communicants of the Moravian congregation on St. Thomas, i.e., those of the enslaved who were entitled to receive communion. The report was unearthed from the Moravian Archives in Herrnhut, Germany, by the Danish historian Louise Sebro, in connection with her archival research for \citet{Sebro_2010}, in which the information pertaining to the interviewees’ various origins was tabulated as an appendix (pp. 215–16). Seidel interviewed 418 communicants in St. Thomas, representing some 12\% of the island’s total enslaved population at the time (estimated on the basis of data from the Danish National Archives in \citealp[157]{GreenPedersen_1971}). All but 36 (8.6\%) of Seidel's informants stated ethnic groups/places of origin, which are potentially identifiable and indicative of the range of ethnolinguistic groups represented in St. Thomas. Expanding on Sebro's work on Seidel's list, the present analysis is the first to make its contents available to an international audience (although see \cite{Bakker_2016b}: 226 for a brief characterization).

Along with Oldendorp's interview data (§4) and the baptismal records from St. Croix (§6), Seidel's work is unique as a source on African origins. According to \citet[67--68]{Sebro_2010}, Seidel kept a journal in which he detailed how he was in sustained contact with the enslaved, and he described how he interviewed each member of the congregation separately, presumably with the aid of an interpreter. Nothing indicates that Seidel knew Virgin Islands Dutch Creole, not to mention any African languages, so he would have required the assistance from someone who did. At any rate, Seidel's approach yielded good results. While it would have been possible for the informants to supply only vague information on their ethnic identities, such as ``Guinea,'' the large majority of them chose to self-identify using more salient, lower-level descriptive categories, such as ``Allada'' or ``Watje,'' which can be linked to modern-day African groups and, by extension, languages.

The information in Seidel's list as tabulated in \citet[125--126]{Sebro_2010} is summarized in \tabref{tab:tab6_02}. All entries are presented as originally recorded by Seidel. Sebro also includes a number of variant spellings in parentheses, which I have not included. Note that the 36 people who did not state an identifiable group are omitted from the table. Of those whose ethnic self-designations are obscure or lacking altogether, two indicated to have been born during the crossing of the Middle Passage; one simply self-identified as being from ``Guinea''; and 33 apparently did not provide an answer to the question, or the answer was not recorded by Seidel. In Sebro's tabulation, a number of individuals can be seen to have used ethnonyms relating to their parents rather than to themselves (e.g., as \emph{geb. St. Thomas Eltern Aja} `born on St. Thomas to Aja parents'), indicating that they themselves were locally born but continued to identify with their African heritage (\emph{vis-\`a-vis} the 125 people who self-identified as being ``Criol,'' i.e., locally-born Creoles). Drawing on secondary sources (e.g., \cite{Thornton_1998}; \cite{Hall_2005}; \cite{Law2005}), as well as considering information from \citet{Oldendorp_2000, Oldendorp_2002} and other primary sources, Sebro tentatively identified some of the languages/language groups presumed to correlate with the attested ethnonyms. Others are added by me in \tabref{tab:tab6_02}, informed primarily by the analysis of Oldendorp's data (cf. §4). Thirteen designations in the list have not been traced to any specific group.

% Table 6:
\begin{longtable}{lrrp{7cm}l}
\caption{Seidel’s list (1753) summarized, arranged alphabetically}\label{tab:tab6_02}\\
    \lsptoprule
    {``Nation''} & {$N$} & {\%} & {Correlate language/language group} \\ \midrule\endfirsthead
    \midrule
    {``Nation''} & {$N$} & {\%} & {Correlate language/language group} \\ \midrule\endhead 
    \endfoot\lspbottomrule\endlastfoot
    Aja & 13 & 3.1 & Gbe, Kwa \\
    Ajonga & 1 & \textless{}1 & − \\
    Akrum & 1 & \textless{}1 & − \\
    Allada & 4 & \textless{}1 & Gbe, Kwa \\
    Amina & 34 & 8.1 & Akan, Kwa \\
    Amombamba & 1 & \textless{}1 & Mbamba dialect of Kimbundu, Bantu \\
    Angkrang & 1 & \textless{}1 & G\~a-Dangme, Kwa \\
    Bundu & 1 & \textless{}1 & − (Mandinka, Mande? Kimbundu, Bantu?) \\
    Chamba & 13 & 3.1 & Gurma, Gur \\
    Comba & 1 & \textless{}1 & − (Kikongo, Bantu, Benue-Congo? Gur? Kru?) \\
    Criol & 125 & 29.9 & Virgin Islands Dutch Creole (and presumably other languages) \\
    Esina & 1 & \textless{}1 & − \\
    Fanti & 2 & \textless{}1 & Akan, Kwa \\ 
    Fon & 7 & 1.7 & Gbe, Kwa \\ 
    Gango & 1 & \textless{}1 & − (Kikongo, Bantu, Benue-Congo? Kru?) \\
    Ibo & 8 & 1.9 & Igbo, Benue-Congo \\
    Kamba & 1 & \textless{}1 & Kikongo, Bantu, Benue-Congo \\
    Karabari & 5 & 1.2 & Igbo, Benue-Congo \\
    Kongo & 19 & 4.5 & Kikongo, Bantu, Benue-Congo \\
    Kpesi & 2 & \textless{}1 & Gbe, Kwa \\
    Lique & 1 & \textless{}1 & Likpe {[}lip{]}, Kwa \\
    Loango & 28 & 6.7 & Kikongo, Bantu, Benue-Congo \\
    Lunda & 1 & \textless{}1 & Lunda, Bantu, Benue-Congo \\
    Mandinga & 12 & 2.9 & Mandinka, Mande \\
    Mandongo & 5 & 1.2 & Kimbundu(?), Bantu, Benue-Congo \\
    Nago & 5 & 1.2 & Yoruba, Benue-Congo \\
    Ongokalla & 1 & \textless{}1 & − \\
    Ouidah & 12 & 2.9 & Gbe, Kwa \\
    Popo & 15 & 3.6 & Gbe, Kwa \\
    Poshee & 1 & \textless{}1 & − \\
    Rentha & 1 & \textless{}1 & − \\
    Rüba & 1 & \textless{}1 & − \\
    \midrule
    Soko & 3 & \textless{}1 & Bambara, Mande \\
    Sundi & 1 & \textless{}1 & Kikongo, Bantu, Benue-Congo \\
    Tem & 6 & 1.4 & Tem, Gur \\
    Tori & 4 & 1 & Gbe, Kwa \\
    Vuningah & 1 & \textless{}1 & − \\
    Watje & 41 & 9.8 & Gbe, Kwa \\
    Wenwig & 1 & \textless{}1 & − \\
    Wungsoko & 1 & \textless{}1 & − \\
\end{longtable}

Figures \ref{fig:fig2_02} and \ref{fig:fig3_02} summarize the distributions in terms of the ethnolinguistic correlate groups and the regions of provenance of these, respectively. The two charts highlight the fact that the locally-born Creoles ($N=125$ out of 418, 29.9\%) comprise the largest group in Seidel's list, which can be seen to reflect a difference between St. Thomas and St. Croix in terms of the proportion of Creole to African enslaved people at this point in time. This is a consequence of the fact that St. Thomas had been colonized more than half a century before St. Croix, and that the population on St. Thomas had better conditions for reproduction. Moreover, the actual share of local-borns was in fact higher than reflected in these figures, as some opted to identify with their African heritage, as discussed above.

% Figure 2 (original order):
\begin{figure}[!ht]
    \centering
    \includegraphics[width=0.9\linewidth]{figures/fig2_02.png}
    \caption{The distribution of language groups summarized}
    \label{fig:fig2_02}
\end{figure}

% Figure 3 (original order):
\begin{figure}[!ht]
    \centering
    \includegraphics[width=0.9\linewidth]{figures/fig3_02.png}
    \caption{The geographic distribution of the groups summarized}
    \label{fig:fig3_02}
\end{figure}

Turning to the African groups, we can note that all of the unidentified ``groups'' consist of just one individual. This means that they are marginal, only thirteen individuals out of 418 people (i.e., 3.1\%).

104 individuals (24.9\%) were seemingly related to the Bight of Benin, and 98 of these people can be presumed to have been (heritage) speakers of Gbe lects. Thus, based on its representation in Seidel’s list, it would appear that \citeauthor{Sabino_1988} (e.g., \citeyear{Sabino_1988}, \citeyear{Sabino_2012}) is correct in considering Gbe a likely main substrate language of Virgin Islands Dutch Creole.

The second-largest African group was that of West-Central Africa ($N=56$, 13.4\%), which undoubtedly would have principally included speakers of Bantu languages. A roughly comparable proportion of at least eight (more than 10\%) of Oldendorp’s informants were Bantu speakers. This point is noteworthy in light of the documented absence of enslaved people originating from Bantu-speaking regions via the transatlantic slave trade between 1725 and the 1790s. However, it has to be kept in mind that the enslaved people who came to the Danish West Indies from within the Caribbean, e.g., from Cura\c{c}ao or other transshipment ports, could well have been Bantu speakers. A supposition that Bantu languages played a prominent role in the early language history of Danish St. Thomas can be backed up by means of linguistic evidence. \citet{Parkvall_2016} investigated the etymology of 44 African-derived words in Virgin Islands Dutch Creole as recorded by \citet{de_Josselin_de_Jong_1926}. He found that 35\% of the lexical Africanisms under consideration could be traced to a Bantu source, which must be viewed as a high percentage. By comparison, 29\% had a Kwa (including Gbe) source. These figures are particularly instructive if one supposes that early substrate influences would have been those to manifest themselves most clearly in the creole \citep[e.g.,][]{Mufwene_1996}. When based on figures from the transatlantic trade, the percentage of Bantu-derived lexical items appears disproportional to the number of early enslaved people on St. Thomas taken from Bantu-speaking areas. Based on those data, as noted by \citet[153]{Parkvall_2000}, “we would only expect Bantu languages to have had a decisive impact [in] the two decades between 1700 and 1720.” Here, Seidel's list would appear to offer a more realistic reflection of a continued presence of Bantu speakers on St. Thomas.

Coming in third, if we disregard the more numerous ``Unknown'' category, is the Gold Coast ($N=40$, 9.6\%), which supplied mostly non-Gbe Kwa speakers. Thus, Seidel's list does not support the idea of (at least a continued) Akan predominance on St. Thomas. The idea of a strong early Akan presence can, however, be backed by linguistic evidence, in the form of substrate features in Virgin Islands Dutch Creole \citep[153]{Parkvall_2000}. This apparent discrepancy between demographic and linguistic data can be viewed as weakening the ``heavy Kwa bias'' hypothesis (cf. §3.1) – especially the form of it assuming a bias in favor of non-Gbe lects. It could be read as pointing to the Akan influence having occurred elsewhere than on St. Thomas (see, e.g., \cite{Goodman_1985}, who speculated that the earliest enslaved people on St. Thomas brought a Dutch pidgin with them). Although such a scenario would go against the mainstream view on the genesis of Virgin Islands Dutch Creole \citep[see][199]{Sabino_2012}, it cannot at present be ruled out with certainty.

Finally, Mande-, Igboid-, Gur-, and Yoruboid-speaking people from Senegambia, the Bight of Biafra, and the West African interior all appear to have made up comparatively small groups on St. Thomas in 1753. Yet, when these smaller groups are considered in addition to the more prominent ones discussed above, the overall picture that obtains, is one of a varied population.


In the end, it would appear that there was a (small) preponderance of people originating in Gbe-speaking areas in West Africa on St. Thomas in 1753, provided we disregard the locally-born subset of the population. However, there was no single dominant group or African language. The Bantu presence was stronger than hitherto assumed in the literature (cf. also \cite{Bakker_2016a} for a discussion of underestimated numbers of Bantus in the Caribbean). It is especially striking that there were more Bantu than non-Gbe Kwa speakers – which was the group traditionally thought to have predominated in the colony. Two caveats have to be mentioned in connection with this assessment. First, it is unknown to what degree Seidel's sample was representative of the island’s enslaved population as a whole; for instance, they were all converted Christians. Second, it cannot be taken for granted that the 1753 snapshot of the substrate population is indicative of the half century or so that came before it. In the next section, I discuss ethnic information over a 90-year period.

\section{St. Croix: Baptismal records from the Moravian mission station Friedensthal}

After having been acquired from the French in 1733 (as detailed, e.g., by G. H. Høst [1791] in \cite{Highfield_2018}), St. Croix was settled from 1734 onward, with settlers arriving from St. Thomas, St. John, and elsewhere in the Leeward Islands. St. Croix soon took on the role of the Danish West Indies’ most important sugar-producing island. As a result, new arrivals were increasingly sent there, making St. Croix a demographically volatile society characterized by high immigration in the eighteenth century. The enslaved population on St. Croix not only grew much larger than those on St. Thomas and St. John but also became more diversified in terms of its members' ethnolinguistic origins. This conclusion partly reflects the fact that the transatlantic slave trade expanded to encompass a wider range of West African regions in the later eighteenth century, with the primary area of trade shifting eastward and southward on the continent (cf. §3). Even more tellingly, however, it can be corroborated by data on the origins of the enslaved population collected and stored by the Moravian missionaries in St. Croix.

The Moravians established two mission stations on St. Croix in the eighteenth century. In 1734, they built Friedensthal `The Valley of Peace' at the western end of Christiansted town. In 1771, they opened a second mission station, Friedensberg `Hill of Peace,' overlooking the newly established town of Frederiksted. Between 1744 and 1832, the Moravians at Friedensthal, the first of these mission stations, baptized a total of 6,783 Africans and African Caribbeans. In the 1960s, the anthropologist Pauline H. Pope carried out archival research on St. Croix for her dissertation (\cite{Pope_1970}), in which she tabulated data copied directly from the baptismal records stored at the island’s Moravian mission stations.\footnote{In addition to the baptismal records from Friedensthal, we also have data recorded at Friedensberg. In the present study, I focus on the Friedensthal data, as this material extends further back in time than the material kept at Friedensberg.} The records have since been moved to Bethlehem, Pennsylvania. Parallel to Seidel in St. Thomas, the missionaries in St. Croix included as part of the record for each candidate for baptism the name of their ``nation.'' However, whereas Seidel's approach was detailed in his journal, the exact method for obtaining this information on St. Croix is unknown. It may be assumed that most missionaries took down information directly from the candidates in question, but, inevitably, many people were involved in this work, so the information may not have as clear an origin in the people it concerned as in the cases of Seidel's list and Oldendorp's survey. On the other hand, despite these uncertainties, the scope and volume of the records are both exceptional.

\citeauthor{Pope_1970}'s (\citeyear{Pope_1970}) tabulations show that 757 individuals gave no identifiable ``nation'' or other meaningful designation for the records, or that this information was not taken down. In contrast, 6,026 (or nearly 90\%) of the total number of baptismal candidates did provide such information. About 50 different groups appear in the records. Pope attempted to identify these with geographical locations and modern-day African peoples, basing her analysis largely on secondary sources. According to \citet[35]{Holsoe_1994}, she was for the most part successful, albeit with some exceptions. In subsequent research, \citet{Sebro_2010} and \citet{Simonsen_2017} have accordingly revised a number of Pope's conclusions. Summarizing their results, it has turned out that the records, in a number of instances, actually reflect a greater diversity than Pope had recognized. Accordingly, since it is possible to work directly with the source material as tabulated in \citet{Pope_1970}, I present my own analysis of the baptismal records below, incorporating the mentioned revisions in the process, and adding some of my own.

The data in \tabref{tab:tab7_02} show a distribution by decade\footnote{Note the columns for 1744 through 1759 have been conflated in the table since there was only a single individual recorded in the 1744–1749 column. Likewise, the 1830–1832 column only recorded two individuals, and it was therefore merged with the column for 1820–1829. The data in \tabref{tab:tab9_02} (page~\pageref{tab:tab9_02}) are arranged the same way.} for a subsample of the 15 African ``nations'' which had the largest representation in the Friedensthal records. Figure \ref{fig:fig4_02} visualizes changes in the distribution over time.

% Table 7:
\begin{table}
% \setlength{\tabcolsep}{4pt} % Adjusted column spacing
% \renewcommand{\arraystretch}{1.2} % Adjusted row height
\resizebox{\linewidth}{!}{ % Resizes the table to fit the width of the page
\begin{tabular}{@{}lrrrrrrrrr} % Using '@{}' to remove extra space on the sides
    \lsptoprule
    {``Nation''} & {1744–59} & {1760s} & {1770s} & {1780s} & {1790s} & {1800s} & {1810s} & {1820–32} & {Totals} \\ \midrule
    Mandingo & 6 & 56 & 182 & 165 & 179 & 73 & 19 & 11 & 691 \\
    Ibo & 10 & 78 & 156 & 112 & 53 & 53 & 23 & 4 & 489 \\
    Watyi & 16 & 36 & 79 & 91 & 122 & 97 & 35 & 10 & 486 \\
    Amina & 66 & 84 & 88 & 83 & 67 & 40 & 17 & 1 & 446 \\
    Kongo & 17 & 37 & 72 & 61 & 33 & 97 & 51 & 21 & 389 \\
    Ashanti & 21 & 42 & 58 & 96 & 70 & 53 & 18 & 0 & 358 \\
    Kalabari & 29 & 56 & 62 & 36 & 15 & 12 & 4 & 0 & 214 \\
    Mokko & 3 & 14 & 43 & 51 & 24 & 32 & 17 & 8 & 192 \\
    Bambara & 0 & 13 & 38 & 32 & 50 & 18 & 9 & 0 & 160 \\
    Akanda & 0 & 31 & 40 & 48 & 9 & 16 & 3 & 0 & 147 \\
    Loango & 12 & 34 & 28 & 4 & 3 & 1 & 1 & 1 & 84 \\
    Sokoto & 4 & 18 & 13 & 12 & 9 & 2 & 0 & 0 & 58 \\
    Kwahu & 4 & 7 & 18 & 19 & 8 & 2 & 2 & 0 & 60 \\
    Popo & 16 & 3 & 1 & 3 & 16 & 2 & 2 & 0 & 43 \\
    Nupe & 3 & 9 & 3 & 8 & 0 & 4 & 0 & 0 & 27 \\ \lspbottomrule
\end{tabular}%
}
\caption{Distribution by decade of 15 major African ``nations'' represented in the Friedensthal mission records, 1744–1832 (source: \cite{Pope_1970}: 70–79)}
\label{tab:tab7_02}
\end{table}


% Figure 4 (original order):
\begin{figure}
    \includegraphics[width=\linewidth]{figures/fig4_02.png}
    \caption{Decade-by-decade and total distribution of 15 major African “nations” present on St. Croix, 1744–1832}
    \label{fig:fig4_02}
\end{figure}

Notwithstanding the issue of representativeness, the data point in the direction of a predominance of people identified as belonging to the ``Mandingo'' (i.e., presumably, speakers of the Mande language Mandinka or a closely related lect), ``Ibo'' (language group: Igboid, Benue-Congo, of Nigeria), ``Watyi'' (Gbe, Kwa), ``Amina'' (Akan, Kwa), ``Kongo'' (Kikongo, Bantu, Benue-Congo), and ``Ashanti'' (Akan, Kwa) groups. There were likewise concentrations of people identified as “Kalabari” (Igboid, Benue-Congo), ``Mokko'' (Efik, Cross River, Benue-Congo), ``Bambara'' (Bambara, Mande, of Mali), and ``Akanda,'' as well as more modestly sized groups of ``Loango'' (Kikongo, Bantu, Benue-Congo), ``Sokoto'' (or ``Sokko'': Bambara, Mande), ``Kwahu,'' ``Popo'' (Gbe, Kwa), and ``Nupe'' (presumably Nupoid, Benue-Congo, of Nigeria) people.

The identity of thirteen of the fifteen groups is straightforward. Only the identification of the aforementioned ``Akanda'' and ``Kwahu'' groups is problematic. \citet[25]{Pope_1970} preferred to link the ``Akanda'' with Oldendorp’s Kru-speaking ``Kanga,'' but she also considered the possibility that the ``Akanda'' were a Yoruboid-speaking group of coastal Nigeria. Alternatively, the group could be traced to an eponymous toponym in Gabon, just as the word may refer to, e.g., Akan or Nupoid groups. In the absence of direct linguistic evidence, I opt not to specify a group. It is likewise not possible to ascertain whether the ``Kwahu'' group should be counted among the Gbe (i.e., corresponding to Oldendorp's ``Wavu'') or not (cf. Fodor's \citeyear{Fodor_1975} ``Wavu II''). In addition, the term might refer to the Twi-speaking Kwahu group of Ghana. Thus, both terms are ambiguous.

A preponderance of ``Amina'' people ($N=66$, 32\%) can be identified early on (1744–1759), but a shift can be seen from the 1760s onward, in the direction of a more heterogeneous population. Incidentally, the point about early “Amina” (Akan) dominance aligns with another source on the demographic composition of early Danish St. Croix – specifically, a roster of Company-owned enslaved individuals working in Christiansted from 1740 to 1755, sourced from the Danish National Archives (tabulated in \cite{Tyson_2011}: 45–48). Conversely, the ``Amina'' are absent from later tallies of a similar kind \citep[e.g.,][36]{Holsoe_1994}.

\tabref{tab:tab8_02} provides a summary of the African language groups that, based on the analysis in §4 and the data presented above, are tentatively identified as being most prominently represented on St. Croix between 1744 and 1832.

% Table 8:
\begin{table}
\begin{tabular}{llrr}
    \lsptoprule
    {Language group} & {“Nation(s)”} & {$N$} & {\%} \\ \midrule
    Mande & Mandingo, Bambara, Sokoto & 909 & 24 \\
    Akan, Kwa & Amina, Ashanti & 804 & 21 \\
    Igboid, Benue-Congo & Ibo, Kalabari & 703 & 18 \\
    Gbe, Kwa & Watyi, Popo & 529 & 14 \\
    Bantu, Benue-Congo & Kongo, Loango & 473 & 12 \\
    − & Akanda, Kwahu & 207 & 5 \\
    Cross River, Benue-Congo & Mokko & 192 & 5 \\
    Nupoid, Benue-Congo & Nupe & 27 & 1 \\ \lspbottomrule
\end{tabular}
\caption{Tentative identification of modern language groups associated with the 15 major African “nations” on St. Croix, 1744–1832}
\label{tab:tab8_02}
\end{table}

The subsample considered above, it should be emphasized, offers no insights regarding the Caribbean-born subset of the population, comprising 2,014 (or 33\% of 6,026) people. The African ``nations'' identified thus far account for 3,844 individuals (64\%) out of 6,026 possible. Thus, the distribution seen above cannot be generalized to the full sample. Moreover, it renders invisible the many ``nations'' that were represented by just a few individuals, such as the ``Bulom'' ($N=2$) or the ``Pu'' ($N=1$), as well as a number of more notable groups, numerically speaking. Such groups include the ``Fulani'' ($N=24$), speakers of Fula lects (North Atlantic, interior Senegambia and West Africa); the ``Kissi'' ($N=15$) and ``Timne'' ($N=20$), speakers of Mel lects (Sierra Leone and environs); the ``Akkran'' ($N=19$), G\~a-Dangme speakers of Ghana; and the ``Kamba'' ($N=19$) and ``Mondonga'' ($N=10$), both presumably speakers of Bantu languages. The remaining African groups all number less than ten individuals, and for the most part less than three. Factoring in these smaller groups only adds further detail to the overall picture of a slave society characterized by considerable ethnolinguistic diversity.

Next, considering the full sample ($N=6{,}026$), the data presented in \tabref{tab:tab9_02} show the distribution of the Friedensthal baptismal candidates' regional origins. Note that I essentially retain \citeauthor{Pope_1970}’s (\citeyear{Pope_1970}) geographical categories, albeit with updated terminology, and with certain changes in terms of the specific composition of the groups. For instance, one of \citeauthor{Sebro_2010}’s (\citeyear{Sebro_2010}: 90) objections to Pope's analysis is that the ``Gold Coast'' category was used in a way where it extended further into the West African interior than this label would otherwise rightly be used to denote. It also encompassed predominantly Gur-speaking areas in the interior, thus reflecting a larger diversity than first assessed. Moreover, Pope placed a number of groups that presumably were Yoruboid-speaking in her ``Nigerian-Cameroon'' (or Bight of Biafra) category, where it would be more accurate to trace these to the Bight of Benin (Pope's ``Slave Coast''). In my analysis, the Gold Coast category has been retained (with the implication that it extends further into the interior than is customary in the literature), but the Yoruboid-speaking groups (comprising 159 individuals) have been counted among the ones originating in the Bight of Benin. Figure \ref{fig:fig5_02} offers a visualization of changes in the distribution over time.

\begin{table}
% \begingroup
\small
% \fontsize{12}{15}\selectfont % Set font size to 12pt with 15pt baseline skip
% \resizebox{\linewidth}{!}{ % Resizes the table to fit the width of the page
\begin{tabularx}{\textwidth}{Q*{9}{r}@{}}  % Added '@{}' to remove extra space on the sides
    \lsptoprule % Top rule for better aesthetics
    Region & 1744–59 & 1760s & 1770s & 1780s & 1790s & 1800s & 1810s & 1820–32 & Totals \\
    \midrule
    Danish West Indies & 91 & 48 & 221 & 222 & 283 & 245 & 182 & 119 & 1,445 \\
    \tablevspace
    Nigeria-Cameroon & 49 & 178 & 277 & 220 & 101 & 104 & 44 & 12 & 985 \\
    \tablevspace
    Gold Coast & 98 & 143 & 173 & 204 & 150 & 100 & 35 & 1 & 904 \\
    \tablevspace
    Senegambia & 6 & 71 & 227 & 203 & 238 & 96 & 30 & 13 & 884 \\
    \tablevspace
    Bight of Benin & 23 & 87 & 122 & 140 & 136 & 129 & 40 & 13 & 690 \\
    \tablevspace
    West Indies (elsewhere) & 25 & 153 & 224 & 111 & 40 & 8 & 5 & 3 & 569 \\
    \tablevspace
    \mbox{West-Central} Africa & 26 & 85 & 106 & 72 & 44 & 100 & 52 & 24 & 509 \\
    \tablevspace
    Sierra Leone, \mbox{Liberia, Ivory Coast} & 0 & 0 & 2 & 3 & 11 & 17 & 5 & 3 & 41 \\ \lspbottomrule % Bottom rule for better aesthetics
\end{tabularx}
% }
% \endgroup
\caption{Regional origins of the enslaved population on St. Croix, 1744–1832 (source: \cite{Pope_1970}: 59–69)}
\label{tab:tab9_02}
\end{table}


% Figure 5 (original order):
\begin{figure}[!ht]
    \centering
    \includegraphics[width=0.95\linewidth]{figures/fig5_02.png}
    \caption{Decade-by-decade and total distribution of the regional origins of Africans and African Caribbeans baptized at Friedensthal, 1744–1832}
    \label{fig:fig5_02}
\end{figure}

Based on these data, the single largest group for the full period covered is the Caribbean-born subset of the population, which includes those born in the Danish West Indies ($N=1{,}445$, 24\%), as well as individuals taken to St. Croix from elsewhere in the West Indies/Caribbean region ($N=569$, 12\%). Nigeria-Cameroon emerges as the African region most clearly represented in the baptismal records ($N=985$, 16\%), followed closely by the Gold Coast ($N=904$, 15\%) and Senegambia ($N=884$, 15\%). Next in line are the Bight of Benin ($N=690$, 9\%), West-Central Africa ($N=509$, 8\%), and the coastal stretch along Sierra Leone, Liberia, and the Ivory Coast ($N=41$, 1\%). These numbers reflect enormous diversity (even disregarding the diversity within these regions), with no groups getting close to a majority.

If one considers diachronic developments in the distribution, it is noteworthy that the locally-born population has a strong presence mainly from the 1770s onward, and a predominance only from the 1790s. This indicates that the overall population became more and more locally born, but also that this development took place later and at a slower pace than on St. Thomas. Alongside the locally-born population, individuals originating in Africa continued to be baptized in large numbers throughout the eighteenth century, and, to a lesser degree, well into the nineteenth century. This is in line with the conclusion of \citet[151]{Simonsen_Olsen_2017} who, writing on overall tendencies in demographic developments in the Danish West Indies, state the following: “For St. Croix, thus, there was a marked Africanization of the island, understood in the sense that the meeting of Africans with different linguistic, ethnic, and religious affiliations came to permeate life on the island” (my translation).


\section{Summary and concluding remarks}

This study has sought to reconstruct a profile of the African ethnolinguistic groups represented in the eighteenth-century Danish West Indies. I examined quantitative data on the transatlantic and intra-Caribbean slave trade to the colony, as well as qualitative and quantitative documentation collected by Moravian missionaries active in the islands. I compared the linguistic data recorded by Oldendorp with modern data on hundreds of languages spoken in West and West-Central Africa, and linked the “nations” identified on this basis with independent data on ethnic origins of the enslaved people on the islands. Those ethnic data were represented in the list of Christianized enslaved individuals from St. Thomas and in baptismal records from St. Croix. I found that it was altogether possible to set out the sources of enslaved people and their languages or language groups represented in the colony.

The different pieces of the puzzle consistently point in the direction of a greater ethnolinguistic diversity than had been assumed to be the case for the Danish West Indies in previous contact linguistic research. The view that a single ethnolinguistic group (Kwa speakers) predominated in the colony in the eighteenth century cannot be upheld, at least not categorically. Enslaved Africans were taken from multiple localities, including from far into the interior. Many of the languages identified are not coastal, hence shipment ports are potentially misleading. In addition to Kwa languages, it appears that Bantu and Nigerian languages – and others yet – had a solid representation in the enslaved population. This indicates, among other things, that the sociohistorical context in which creole languages developed within Virgin Islands society during the colonial period was characterized by significant linguistic diversity among the enslaved. This indicates that the set of potential substrate languages for Virgin Islands Dutch Creole (and, in a more limited, adstratal way, the English-based varieties of the islands) should be expanded. With respect to the question of ethnolinguistic diversity, it is worth considering whether the Danish colony was in any way exceptional – besides the fact that a substantial amount of data has survived. If indeed the overall tendencies discerned for the Danish West Indies apply to other Caribbean islands that share a similar history, then it follows that a number of other territories can look to the Virgin Islands for a more realistic view of the extensiveness of their ethnolinguistic diversity throughout the colonial period.

An important caveat to the conclusions presented here is that the enslaved Africans were generally multilingual, speaking one or more regional lingua francas as well. This study has not addressed this issue, mainly because the data considered offer few insights into it. Another limitation concerns potential underrepresentation of certain groups in the sources considered, which cannot be ruled out. For instance, some African groups may have resisted Christianization, and those would be underrepresented.


\section*{Acknowledgements}
  I wish to thank Peter Bakker and two anonymous reviewers for their valuable feedback on an earlier version of this study. I am also grateful to the audience at my presentation at the 2017 Society for Pidgin and Creole Linguistics’ Summer Meeting in Tampere, Finland, for their comments on my initial work on this topic. When finalizing this study, I had the benefit of a grant from the Carlsberg Foundation (Grant CF23-1162).

%\section*{Contributions}
%John Doe contributed to conceptualization, methodology, and validation.
%Jane Doe contributed to writing of the original draft, review, and editing.

{\sloppy\printbibliography[heading=subbibliography,notkeyword=this]}
\end{document}
