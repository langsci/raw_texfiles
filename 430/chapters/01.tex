\documentclass[output=paper,colorlinks,citecolor=brown]{langscibook}
\ChapterDOI{10.5281/zenodo.14282806}
\title{Introduction} 
\author{Angela Bartens\orcid{}\affiliation{University of Turku;University of Helsinki} and Peter Slomanson\orcid{}\affiliation{Tampere University;University of Turku} and Kristoffer Friis Bøegh\orcid{}\affiliation{Utrecht University;Aarhus University}}

\abstract{}

\IfFileExists{../localcommands.tex}{
   \addbibresource{../localbibliography.bib}
   % add all extra packages you need to load to this file

\usepackage{tabularx,multicol}
\usepackage{url}
\urlstyle{same}

\usepackage{listings}
\lstset{basicstyle=\ttfamily,tabsize=2,breaklines=true}

\usepackage{langsci-basic}
\usepackage{langsci-optional}
\usepackage{langsci-lgr}
\usepackage{langsci-osl}
% \usepackage{./langsci/styles/langsci-lgr}
% \usepackage{./langsci/styles/langsci-osl}
% \usepackage{langsci-gb4e}

\usepackage{tikz}
\usetikzlibrary{patterns,calc}
\pgfdeclarepatternformonly{south east lines}{\pgfqpoint{-0pt}{-0pt}}{\pgfqpoint{3pt}{3pt}}{\pgfqpoint{3pt}{3pt}}{
    \pgfsetlinewidth{0.6pt}
    \pgfpathmoveto{\pgfqpoint{0pt}{3pt}}
    \pgfpathlineto{\pgfqpoint{3pt}{0pt}}
    \pgfpathmoveto{\pgfqpoint{.2pt}{-.2pt}}
    \pgfpathlineto{\pgfqpoint{-.2pt}{.2pt}}
    \pgfpathmoveto{\pgfqpoint{3.2pt}{2.8pt}}
    \pgfpathlineto{\pgfqpoint{2.8pt}{3.2pt}}
    \pgfusepath{stroke}}
    
\usepackage{stmaryrd}
\usepackage{wasysym}
\usepackage{multirow}
\usepackage{caption}
\usepackage{subcaption}
\usepackage{mathrsfs}
\usepackage{qtree}

\usepackage{linguex}


   %pminos do not split footnotes
% \interfootnotelinepenalty=10000 %Footnote in Laporte chapters has to be split SN


%\DeclareIndexNameFormat{default}{%
%\nameparts{#1}%
%\usebibmacro{index:name}%
%{\index[names]}%
%{\namepartfamily}%
%{\namepartgiveni}%
% {}% L1
% {}% L2
%{\namepartprefix}% generates spurious space L3
%{\namepartsuffix}% generates spurious space L4
%}

%  {\DeclareIndexNameFormat{default}{%
%     \usebibmacro{index:name}{\index[names]}{#1}{#3}{#5}{#7}}}

%\DeclareIndexNameFormat{default}{%
%  \usebibmacro{index:name}{\sindex[nom]}{#1}{#3}{#5}{#7}}

%\DeclareIndexNameFormat{default}{%
%  \usebibmacro{index:name}{\sindex[person]}{#1}{#3}{#5}{#7}}
%\DeclareIndexNameFormat{default}{%
%\nameparts{#1} \usebibmacro{index:name}{\sindex[person]]}{\namepartfamily}{‌​\namepartgiven}{\nam‌​epartprefix}{\namepa‌​rtsuffix}}

%\newcommand{\smiley}{:)}

%\renewbibmacro*{index:name}[5]{%
%\usebibmacro{index:entry}{#1}%
%{\iffieldundef{usera}{}{\thefield{usera}\actualoperator}\mkbibindexname{#2}{#3}{#4}{#5}}}

% \newcommand{\noop}[1]{}

%remove for final
%\overfullrule=1mm

\newcommand{\tobi}[2]}}
\renewcommand{\S}[1]{\tobi{#1}{\textsc{*}}}

% this volume references
% puts: [this volume]
% already defined: \citetv
%\newcommand{\citepv}[1]{(\citeauthor{#1} \citeyear*{#1} [this volume])}
\newcommand{\citealtv}[1]{\citeauthor{#1} \citeyear*{#1} [this volume]}

%parentheses around example number
\newcommand{\pref}[1]{(\ref{#1})}

% in-text examples

\newcommand{\lnex}[1]{\textit{#1}} %target lang word
\newcommand{\lnlit}[1]{(lit.: `#1')} %literal reading
\newcommand{\lnlat}[1]{(#1)} % latinization
\newcommand{\lntrans}[1]{`#1'} %translation
\newcommand{\lnexl}[2]%
{\lnex{#1}{} \lnlat{#2}} % ex with latinization
\newcommand{\lnexlat}[3]{\lnex{#1}{} \lnlat{#2}{} \lntrans{#3}} % ex with latinization and tranl.

%ch01
\newcommand{\co}[1]{\mbox{\textbf{#1}}}

%ch09

\newcommand{\cyrbulg}[1]{\begin{otherlanguage*}{bulgarian}#1\end{otherlanguage*}}


%ch10
\newcommand{\nlp}{{\small NLP}}
\newcommand{\mwe}{{\small MWE}}
\newcommand{\rae}{{\small RAE}}
\newcommand{\lvc}{{\small LVC}}
\newcommand{\pos}{{\small P}o{\small S}}
%\newcommand{\todo}[1]{ \textcolor{red}{#1} }

%\renewcommand{\labelenumi}{\theenumi}
%\ainamefmt{{vv}{ll}{, ff}{, jj}} % fullname

\newcommand{\biberror}[1]{{\color{red}#1}}

\newcommand{\osenovaitem}{--~}
   %% hyphenation points for line breaks
%% Normally, automatic hyphenation in LaTeX is very good
%% If a word is mis-hyphenated, add it to this file
%%
%% add information to TeX file before \begin{document} with:
%% %% hyphenation points for line breaks
%% Normally, automatic hyphenation in LaTeX is very good
%% If a word is mis-hyphenated, add it to this file
%%
%% add information to TeX file before \begin{document} with:
%% %% hyphenation points for line breaks
%% Normally, automatic hyphenation in LaTeX is very good
%% If a word is mis-hyphenated, add it to this file
%%
%% add information to TeX file before \begin{document} with:
%% \include{localhyphenation}
\hyphenation{
    Beck-man
    Ngu-yen
    back-chan-nel
    back-chan-nels
    mo-not-o-nous
    ste-reo-typ-i-cal
}

\hyphenation{
    Beck-man
    Ngu-yen
    back-chan-nel
    back-chan-nels
    mo-not-o-nous
    ste-reo-typ-i-cal
}

\hyphenation{
    Beck-man
    Ngu-yen
    back-chan-nel
    back-chan-nels
    mo-not-o-nous
    ste-reo-typ-i-cal
}

   \boolfalse{bookcompile}
   \togglepaper[1]%%chapternumber
}{}

\begin{document}
\maketitle

\noindent\emph{New research on circum-Caribbean creoles and language contact} was first intended as a collection of sociolinguistically and sociohistorically oriented papers presented at the 2017 Summer Meeting of the Society for Pidgin and Creole Linguistics (SPCL), hosted by the Universities of Tampere and Turku. The contributors have since expanded the content of their original papers substantially, which has contributed to the empirical and analytical depth of their submissions.

The volume is organized as follows. The first part features three papers on language contact and creolized languages in the traditional, albeit now frequently debated sense (cf. \cite{Holm_1989,McWhorter_1998,DeGraff2020}). While we acknowledge this ongoing debate (cf. also \cite{McWhorter_2018} for a monograph treatment of the topic), we do not aim to engage with it directly in the present volume. The second part features three papers on partially restructured varieties \citep[cf.][]{Holm_2004} of the circum-Caribbean region \citep{Fleischmann_1986}.

By comparing the sociohistorical and current sociolinguistic outcomes of language contact in the circum-Caribbean region, the present volume introduces new comparative research and constitutes an invitation to further inquiry, both historical and contemporary, along similar lines. What sets of circumstances, past and present, can be claimed to obtain in contact situations, and how do those circumstances contribute to linguistic outcomes and to the ways in which contact varieties are viewed by their speakers?

Part I of the volume starts with \textsc{Kristoffer Friis Bøegh}'s paper “African ethnolinguistic diversity in the colonial Caribbean: The case of the eighteenth-century Danish West Indies”. In this in-depth and well-documented study, Bøegh reconstructs a profile of the African-origin ethnolinguistic groups present in the Danish West Indies (now the US Virgin Islands) during the eighteenth century, based on the triangulation of four bodies of direct and indirect evidence of African languages in the colony. His findings have implications not only for creolistics, but contact linguistics in general, as they shed new light on the sociohistory of high-contact settings, such as the circum-Caribbean region.

The Danish West Indies constitute an interesting case for the study of African origins and ethnolinguistic diversity in the Caribbean. Overall, there is little direct linguistic evidence in the form of quotes or texts in African languages from the Caribbean during the colonial period. In the case of the Danish West Indies, however, documentation on both the transatlantic and intra-Caribbean slave trade enables the author to trace a detailed picture in the sense of \citeauthor{Mufwene_1996}’s (\citeyear{Mufwene_1996}) Founder Principle. Whereas the Trans-Atlantic Slave Trade Database and its different versions have, at times, been criticized  \citep[cf.][]{Inikori_2011}, it constitutes an apt tool for reconstructing transatlantic and intra-Caribbean forced population movements. 

In addition, Bøegh is able to draw on materials compiled by the Moravian missionaries, who were also active in other regions, such as the Nicaraguan Miskito Coast (cf. the establishment of the Herrnhut/Moravian Mission in Bluefields in 1848). These materials include Seidel’s 1753 list of Christianized Africans and African Caribbeans from St. Thomas, baptismal records from the Friedensthal mission station on St. Croix (covering the period 1744–1832), and Oldendorp’s linguistic work on both of these islands \citep{Oldendorp_1777,Oldendorp_2000,Pope_1970,Seidel_1753}.

The deep concern of the Moravian missionaries with language \citep[cf.][17--20]{Holm_1989} has provided us with significant data from the creole-speaking (and other linguistic) regions, best recognized in the field of missionary linguistics where, however, early language descriptions tend to look at the languages in question through the Greco-Latin grammar tradition.

Summarizing, the quantitative and qualitative data collected by Bøegh point to greater ethnolinguistic variety than was previously assumed to have existed in the Danish West Indies. This scenario is potentially applicable to other Caribbean and high-contact regions. The study definitely highlights the significance of the maintenance of both historical and present-day multilingualism as well as lingua francas in the New World context \citep[cf.][]{AlvarezLopez_2004,Palmie_2006}.

\citeauthor{Talmy_1991}'s typology \citeyearpar{Talmy_1991} distinguished between the ways that motion events are framed in different languages, with Germanic languages as satellite-framed and Romance languages as verb-framed in their expression of the path of motion (i.e., the English ``Mary danced into the room'', as opposed to the dispreferred ``Mary entered the room dancing'' in which the path is conveyed by the verb itself). In her paper in this volume, “The expression of motion events in Haitian Creole”, \textsc{Carolin Ulmer} investigates how motion events are expressed by a group of native Haitian Creole speakers living in Germany. Since Kwa languages and French were component languages in the genesis of Haitian, the fact that the Kwa languages use serial verbs to express not just path, but motion as well, raises the question of what these competing diachronic influences have yielded in the grammar of Haitian and in the Haitian of the group of expatriate speakers that Ulmer worked with. This is an interesting and important question, not just for Haitian, but for the Atlantic creoles generally, including the circum-Caribbean group, given substrate influences from African languages with serial verbs and superstrate influence from European languages that are not similarly equipollently framed, the descriptive term employed by \citet{Slobin_2004}, to characterize this third language type.

In ``Postulating Atlantic English Pidgin/Creole as a pluriareal language: A perception study'', \textsc{Angela Bartens}, \textsc{Kwaku Owusu Afriyie Osei-Tutu}, and \textsc{Tamirand Nnena De Lisser} explore the idea of mutual intelligibility involving Atlantic English Pidgin/Creole (EP/C) from the point of view of a pluriareal, as opposed to a pluricentric language. (Note that Atlantic EP/C is distinguished from Atlantic EP/Cs. The authors use the latter to refer to distinct English-lexifier languages of the Atlantic region.) Despite the fact that some scholars, e.g., \citet{Dollinger_2019}, find the distinction unnecessary, the authors argue for a hypothetical step from a pluriareal to a pluricentric language.

To merit the label `pluricentric' a language has to comply with certain criteria, e.g., 1) being used in at least two nations or “interacting centers”; 2) having official status at state or regional level; 3) manifesting linguistic distance in the sense of \citeauthor{Kloss-1967}'s (\citeyear{Kloss-1967}) Abstand languages as a symbol of linguistic and cultural identity; 4) enjoying acceptance as a pluricentric, but also national/regional variety; 5) being codified or in the relevant process and thence disseminated, e.g., through formal education \citep[cf.][]{Muhr_2012,Clyne_1992}. 

The term ``pluriareal languages'', on the other hand, focuses on linguistic differences in these language varieties independent of national and political borders \citep{Niehaus_2015,ElspaB_Durscheid_Ziegler_2017}. The authors argue that Atlantic EP/C might be on its way from a pluriareal to a pluricentric language, a point of view basically shared by \citet{Faraclas_2020}.

In the first part of this study, 56 Ghanaian, Guyanese, and Nigerian informants were presented with audio samples of Ghanaian, Nigerian, Jamaican, and Sranan EP/C. The second part consisted of 20 interviews of Atlantic EP/C speakers conducted in Guyana. Besides researching mutual intelligibility, the authors were interested in the acceptability of a common writing system.

Part II of the volume begins with \textsc{Eliot Raynor}'s paper ``A Gbe substrate model for discontinuous negation in Spanish varieties of Chocó, Colombia: Linguistic and historical evidence''. This study revisits so-called ``double'' or discontinuous negation (henceforward NEG2) and the role of African substrate languages, in this case in Choc\'o Spanish. Choc\'o is a Colombian department on the northwestern Pacific Coast of Colombia mainly inhabited by Afro-descendants (82.1\%). Choc\'o Spanish differs from other Colombian varieties, e.g., in employing the linguistic structure studied in this contribution. In NEG2, an utterance-final negative element occurs in the proposition in addition to the preverbal negator. It is important to note, however, that the use of this construction is not canonical and actually stigmatized, albeit very commonly used, according to Raynor's informants. Despite the fact that no balanced sociolinguistic corpus of Choc\'o Spanish exists, \citet{RuizGarcia_2001} showed in a previous study that the same pragmatic conditioning found in Brazilian Portuguese, i.e., NEG2 in occurrences of the negation of a proposition activated in prior discourse, does not hold for Choc\'o Spanish.

Frequently, substratist approaches to Atlantic Creole and African-influenced varieties of the Americas ignore local adstrate languages. Raynor nevertheless shows that, at least in the case of NEG2, the indigenous languages of the Choc\'o are not candidates for its occurrence in Choc\'o Spanish. The terrain for the discussion of African influence is prepared through a meticulous presentation of the slave groups present in the region during the late 17th and early 18th centuries, the period postulated for the emergence of NEG2 in Choc\'o Spanish. Based on these data, Raynor considers that Gbe speakers, especially Ewe and Gen, were the key agents in its adoption as \emph{bozales}, i.e., African-born slaves, outnumbered locally born \emph{criollo} slaves up to at least 1778. The robustness of his findings on the African origin of the construction versus an Ibero-Romance one \citep[cf.][]{Sessarego2017} is important for the discussion of the same in other Afro-American contact varieties. In addition, recalling that the ethnonym \emph{Mina} should not be categorically associated with Elmina Castle in Ghana opens new perspectives in creolistics. 

In ``The minutes of the \emph{Irmandade Nossa Senhora do Rosário dos Homens Pretos} in Recife: A case study'', \textsc{Fernanda M. Ziober} examines 32 nineteenth-century texts, transcribed by herself following paleographic methodology. 13 of the texts were written by Manoel de Barros, a member of the Brotherhood of Our Lady of the Rosary of Black Men, whereas the remaining texts feature three different calligraphies. Written by persons with limited access to formal education, the texts show interesting deviations from the Portuguese orthography of the time. These deviations reveal, for example, phonetic and morphosyntactic phenomena, among them metathesis, shared by other Portuguese varieties, e.g., Angolan and Mozambican Portuguese \citep{Petter_2009}, cf. \emph{pratido < partido} `broken', and variable agreement within the NP, cf. \emph{dos homem pretos < dos homens pretos} `two black men'.

As the author points out, this type of text constitutes an important source for the historical comparison of writing and, through that lens, Portuguese language as practiced by less literate people both in Brazil and in Africa during the nineteenth century, so crucial for, at least, the formation of Brazilian Portuguese (see \cite{Noll_1999}).

In the third contribution of Part II, “Language contact in Puerto Rico: Documenting an emerging variety of English”, \textsc{Sally J. Delgado} investigates Puerto Rican English as an emerging contact variety spoken in Puerto Rico, one with which its speakers have come to identify, in appreciation of its local character. The author addresses the controversy surrounding the term ``Spanglish'' in previous literature as a cover term for language mixing, a term that has sometimes been used to denigrate the linguistic behavior of Spanish-English bilinguals in the Caribbean and beyond. She introduces the component activities of the piloted data collection project, involving the collection of spoken language, written language, and the documentation of language attitudes. She proposes that involving speakers in the creation of data resources for this variety would serve to increase awareness of its value, its status, and its function. This goal goes beyond the use of native speakers to guarantee the authenticity of the data for the sake of scientific work, but aiming to accomplish an important social goal for the long-term benefit of the speech community.

The awareness that this work and its dissemination can produce is critical to the emotional health of people who should be benefiting from the rich and well-defined array of linguistic resources available to speakers in bilingual cultural communities such as Puerto Rico. This is significant because making use of those resources in such a cultural context, while keeping the associated codes (i.e., Spanish and English) discrete, would be socially and communicatively dysfunctional. Unfortunately, the culturally appropriate non-discrete use of these resources is often impeded by internalized stigma surrounding language mixing and the effects of language contact generally. Delgado concludes that speakers would benefit from the increased presence of translanguaging in formal educational contexts, rather than restricting language mixing to informal contexts in which it is conducive to deeper interpersonal communication, social bonds, and the resulting collective emotional health of the speakers.

The subfield of creolistics has become broader and more inclusive in recent years, so that contact languages that cannot be characterized as pidgins and creoles are of increasing interest to creolists. We are very much in favor of this development. At the same time, we retain an interest in the specific circumstances and linguistic ecologies found in the Caribbean and sociohistorically-related areas in other parts of the Americas, from the period in which these areas were viewed in a sense as core target areas for research on creolization (and decreolization), based on the features of their contact languages, as well as on perceived cross-linguistic parallels between those languages. Those parallels included features such as similar periphrastic tense, mood, and aspect marking systems, verb serialization, syllable structure, and others that were observed across lexical inventories (i.e., English, French, Spanish, Portuguese, and Dutch). This interest is not intended as a prioritization of the circum-Caribbean region above other research areas within contact language research, but rather as a validation of the new work that has continued to appear on the fascinating language varieties that developed in the region and that continue to thrive there.

\section*{Acknowledgements}
\emph{New research on circum-Caribbean creoles and language contact} is a project that started with papers presented at the 2017 Summer Meeting of the Society for Pidgin and Creole Linguistics (SPCL), hosted by the Universities of Tampere and Turku. Its progress was delayed at times by obstacles such as the Covid-19 pandemic and the reorganization of teaching, research, and everyday life that resulted from it. Additionally, a number of technical roadblocks arose that delayed completion of the project.

Consequently, we feel enormous gratitude towards the authors of the papers in this volume. Throughout the process, their faith in it was unwavering and their patience went even further.

This volume could not have been completed without the generous assistance of twelve anonymous reviewers from a geographic area that extends from Haiti, Brazil, the United States, and Ghana to Finland, as well as the mentors the authors have cited in their individual acknowledgments. 

We would also like to thank the community proof readers for their input, Maria Pesola-Gallone for some initial formatting, and Felix Kopecky for final typesetting. The Society for Caribbean Linguistics accepted to partially fund the costs associated with the typesetting which we hereby graciously acknowledge.

Furthermore, we wish to express our gratitude to Language Science Press in general, and first and foremost to Joseph T. Farquharson, Chief Editor of the Studies in Caribbean Languages series, for accepting this volume for publication.

Most importantly, however, we would like to underline the pivotal role played by Kristoffer Friis Bøegh. The volume was conceptualized and primarily edited by Angela Bartens and Peter Slomanson. Kristoffer Friis Bøegh joined as the third editor quite late in the process, substantially contributing to its finalization and ensuring a timely publication of the long-awaited volume.

\printbibliography[heading=subbibliography,notkeyword=this]
\end{document}
