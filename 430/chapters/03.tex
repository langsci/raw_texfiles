\documentclass[output=paper,colorlinks,citecolor=brown]{langscibook}
\ChapterDOI{10.5281/zenodo.14282810}
\author{Carolin Ulmer\orcid{}\affiliation{Freie Universität Berlin}}
%\ORCIDs{}

\title{The expression of motion events in Haitian Creole}

\abstract{This paper investigates the expression of motion events in Haitian Creole. A bipartite typology has been proposed by \citet{Talmy_1991}, sorting languages into verb-framed and satellite-framed languages, depending on where they express the Path component of motion events. \citet{Slobin_2004} expanded the typology by a third type, equipollently-framed languages, to include verb-serializing languages which can express the Path as well as the Manner component in a serial verb construction. Creole languages have so far received little to no attention in regard to this typology. Creole languages are especially interesting because they were formed in a situation of language contact. The investigation of their morphosyntactic features can shed light on the question of which features of the languages involved are passed on and which are not. This can in turn offer clues for the study of the markedness of these features. The languages which were relevant to the formation of Haitian Creole, French and Kwa languages, present different patterns here. In French, verb-framed patterns are predominantly used, but in some cases Manner verbs constitute the main verb of the sentence \citep{Pourcel_Kopecka_2005}. In contrast, Kwa languages can use verb serializations to encode motion events (\citealt{Ameka_Essegbey_2013}; \citealt{LambertBrtire_2009}), a pattern not found in French. In this paper, I describe a small study conducted in Berlin, Germany, in 2017, investigating the expression of Motion events by four native speakers of Haitian Creole. They narrated a picture story and described drawings depicting different combinations of Manner and Path components. A wide range of different morphosyntactic structures encoding motion events was elicited. Verb-framed patterns were frequently used, as well as different Manner-Path verb serializations. Only a few satellite-framed constructions were elicited, but using different Manner verbs and Path-PPs. Further research will need to test the acceptability of different Manner and Path elements in the particular structures.}


\IfFileExists{../localcommands.tex}{
   \addbibresource{../localbibliography.bib}
   \usepackage{langsci-optional}
\usepackage{langsci-gb4e}
\usepackage{langsci-lgr}

\usepackage{listings}
\lstset{basicstyle=\ttfamily,tabsize=2,breaklines=true}

%added by author
% \usepackage{tipa}
\usepackage{multirow}
\graphicspath{{figures/}}
\usepackage{langsci-branding}

   
\newcommand{\sent}{\enumsentence}
\newcommand{\sents}{\eenumsentence}
\let\citeasnoun\citet

\renewcommand{\lsCoverTitleFont}[1]{\sffamily\addfontfeatures{Scale=MatchUppercase}\fontsize{44pt}{16mm}\selectfont #1}
  
   %% hyphenation points for line breaks
%% Normally, automatic hyphenation in LaTeX is very good
%% If a word is mis-hyphenated, add it to this file
%%
%% add information to TeX file before \begin{document} with:
%% %% hyphenation points for line breaks
%% Normally, automatic hyphenation in LaTeX is very good
%% If a word is mis-hyphenated, add it to this file
%%
%% add information to TeX file before \begin{document} with:
%% %% hyphenation points for line breaks
%% Normally, automatic hyphenation in LaTeX is very good
%% If a word is mis-hyphenated, add it to this file
%%
%% add information to TeX file before \begin{document} with:
%% \include{localhyphenation}
\hyphenation{
affri-ca-te
affri-ca-tes
an-no-tated
com-ple-ments
com-po-si-tio-na-li-ty
non-com-po-si-tio-na-li-ty
Gon-zá-lez
out-side
Ri-chárd
se-man-tics
STREU-SLE
Tie-de-mann
}
\hyphenation{
affri-ca-te
affri-ca-tes
an-no-tated
com-ple-ments
com-po-si-tio-na-li-ty
non-com-po-si-tio-na-li-ty
Gon-zá-lez
out-side
Ri-chárd
se-man-tics
STREU-SLE
Tie-de-mann
}
\hyphenation{
affri-ca-te
affri-ca-tes
an-no-tated
com-ple-ments
com-po-si-tio-na-li-ty
non-com-po-si-tio-na-li-ty
Gon-zá-lez
out-side
Ri-chárd
se-man-tics
STREU-SLE
Tie-de-mann
}
   \boolfalse{bookcompile}
   \togglepaper[3]%%chapternumber
}{}

\begin{document}
\maketitle

\section{Introduction}

The expression of motion events in different languages has been of great interest to many linguists since \citet{Talmy_1991} proposed his typology of them, sorting languages into two types depending on whether they typically express the Path of motion in the main verb or a so-called satellite, some other element that is closely associated with the main verb. Romance languages are often cited as typical members of the first group, called verb-framed languages, whereas Germanic languages represent the second group, named satellite-framed languages. Later, \citet{Slobin_2004} proposed a third type that he calls equipollently-framed languages to describe languages expressing both the Path and the Manner component in a verb serialization, a structure that is found e.g. in Mandarin Chinese. Even though many languages have been investigated with regard to the morphosyntactic structures used to encode the different components of motion events, there is still no research on this question for Romance-based Creole languages. As Creole languages were formed in a situation of language contact, their investigation can show which features of the languages involved were passed on. The present paper looks at the morphosyntactic expression of Motion events in Haitian Creole. A small pilot study was conducted with four native speakers of Haitian Creole in Berlin, Germany. After a short sociolinguistic interview to determine their language ideologies and habits of language use (which were deemed necessary as the speakers all lived away from their home country and in a multilingual environment), the speakers completed two different tasks. First, they narrated a picture story about a little bird flying out of its cage and house to explore the outside world. After that, they provided descriptions for single pictures which were assembled in order to control for several combinations of Manner and Path components of motion events. The results show that Haitian Creole possesses a rich inventory of morphosyntactic structures to express motion events. A preference exists for the use of verb-framed constructions, but Manner-Path verb serializations were also used frequently. Satellite-framed constructions were rare, but do not seem to be totally ungrammatical.

The structure of the paper is as follows. \sectref{sec:3:2} first gives a definition of motion events and then describes the three types of motion event encoding mentioned above. Then, the concept of Manner salience, which describes the frequency with which Manner elements are used in different languages, is introduced. Sections 2.3 and 2.4 provide an insight into the expression of motion events in French and some African languages which were relevant to the formation of Haitian Creole. After that, a first light is shed on motion event encoding in Haitian Creole from previous works on the Haitian Creole language. \sectref{sec:3:3} then outlines the study conducted by the author and explains how the data was classified. \sectref{sec:3:4} gives a broad overview of the different structures that were elicited, which are then discussed in \sectref{sec:3:5}.


\section{Motion events}\label{sec:3:2}

In this study, a motion event is understood as defined by \citet[60--61]{Talmy_1991}:

\begin{quote}
    The basic motion event consists of one object (the “Figure”) moving or located with respect to another object (the reference object or “Ground”). It is analyzed as having four components: besides “Figure” and “Ground”, there are “Path” and “Motion”. The “Path” [...] is the course followed or site occupied by the Figure object with respect to the Ground object. […] In addition to these internal components a Motion event can have a “Manner” or a “Cause”, which we analyze as constituting a distinct external event.
\end{quote}


In sentence \REF{ex:3:1}, the components of the motion event are distributed as follows:

\ea\label{ex:3:1}
\gll Tom is running {down the stairs}.\\
     {}  {} Manner  Path\\
\z
		
\textit{Tom} represents the Figure, \emph{running} the Manner, \emph{down} the Path, and \emph{the stairs} the Ground of the motion event described.

Different lexicalization patterns are found in the languages of the world concerning the motion event component expressed in the verb. A first type encodes Motion and Manner in the verb, which is typical e.g. in English, as seen in sentence \REF{ex:3:1}. A second type, which is typically found in Romance languages like French or Spanish, encodes Motion and Path in verbs, such as  \emph{descendre/bajar} (‘go.down’) \citep[62--68]{Talmy_1985}. These differences lead to different patterns of encoding motion events, described in the following.

\subsection{Motion event typology}

In his 1991 paper, Talmy develops a typology of motion event encodings, sorting the languages of the world into two types depending on the morphosyntactic element in which they express the Path component: verb-framed and satellite-framed languages \citep[486--487]{Talmy_1991}. Verb-framed languages, such as Romance languages, express the Path component in the main verb of the sentence.

\ea Spanish \citep[69]{Talmy_1985} \\\label{ex:3:2}
\glll La botella salió de la cueva flotando.\\
      The bottle went.out of the cave floating \\
      {}  {}     Path     {} {} {} Manner\\
\z
	                     	
Other languages, such as English or German, express the Path component in a so-called satellite, a term defined by \citet[102]{Talmy_1985} as “immediate constituents of a verb root other than inflections, auxiliaries, or nominal arguments, [related] to the verb root as periphery (or modifiers) to a head”. These languages are therefore named satellite-framed languages. The English counterpart to \REF{ex:3:2} can be seen in \REF{ex:3:3}.

\ea\label{ex:3:3}
\gll  The bottle floated {out of the cave}.\\
      {}  {}      Manner Path\\
\z
						
\citet{Slobin_2004} revises this binary typology by adding a third type, equipollently-framed languages, suited to describe languages with serial verb constructions where both Path and Manner can be expressed in a verb, as illustrated by the Mandarin Chinese example in \REF{ex:3:4}.

\ea\label{ex:3:4}Mandarin Chinese\\
\glll  Hǎiōu cóng dòng l\u{i} fēi chū.\\
       seagull from hole in fly exit\\
       {}      {}   {}   {}  Manner Path\\
\glt `The seagull flew out of the hole.'
\z

Much work has followed the papers of \citet{Talmy_1991} and \citet{Slobin_2004}, classifying many different languages into the different patterns. Many of these works have used the so-called “Frog Stories” also used in \citet{Slobin_2004}. In sections 2.3 and 2.4, a short overview will be given of the work on motion event encoding in Kwa languages as well as French, which have been relevant to the formation of Haitian Creole.

\subsection{Manner salience}

\citet{Slobin_2004} takes a more detailed look at the expression of Manner, that is to say the frequency with which it is expressed in different languages. To this end, he compares the encoding of one certain event in the Frog Stories, namely an owl flying out of a knothole \citep[224--225]{Slobin_2004}. In the verb-framed languages Spanish, French, Italian, Turkish and Hebrew, virtually no Manner verbs are used, see \REF{ex:3:5} for French.

\ea\label{ex:3:5}French \citep[224]{Slobin_2004}\\
\glll D'un trou de l'arbre sort un hibou.\\
      from.a hole in the.tree  exits  an owl\\
      {}   {}  {} {}     Path {} {} \\ 
\z

Between different satellite-framed languages, more variation can be found regarding the expression of Manner. In German and English, Manner verbs are not used very frequently (only by about 17--32\% of the speakers) to express the motion event in question. This is due to the fact that often deictic verbs are used with Path satellites, as in the German example in \REF{ex:3:6}.

\ea\label{ex:3:6}German\\
\glll  Aus dem Astloch kommt eine Eule raus.\\
      from \textsc{art.def.dat} knothole comes  \textsc{art.indef} owl out \\
      {}   {}  {}     {}    {}   {}  Path\\
\z

In the equipollently-framed languages Mandarin Chinese and Thai, Manner is expressed more frequently (by 40\% of the Mandarin and 59\% of the Thai speakers). In the SF language Russian, the Manner component of this event is expressed by 100\% of the speakers. In all these cases, either a deictic (\emph{pri-letet} `fly here') or a Path-prefix (\emph{vy-letet} `fly out') is added to the Manner verb \emph{letet} `to fly'.

Slobin comes to the conclusion that the frequency with which the Manner component is expressed depends on the language type as well as the morphosyntactic possibilities to encode Manner. He proposes to align languages along a scale of Manner salience, where languages expressing Manner in the main verb typically have a high Manner salience, whereas languages where Manner is subordinate to Path typically have low Manner salience \citep[250]{Slobin_2004}.


\subsection{Motion events in French}

As already mentioned above, French is classified as a verb-framed language, encoding Path in the main verb and Manner in a gerund.

\ea\label{ex:3:7}French\\
\glll  Elle entrait à la  maison en.courant.\\
       \textsc{3sg.f} entered \textsc{prep} \textsc{art.def} house run.\textsc{ger} \\
       {} Path {} {} {} Manner\\
\z


An extensive study of motion event encoding in French can be found in \citet{Pourcel_Kopecka_2005}. They analyze a total of 1800 written and 594 oral descriptions of motion events and, on this basis, describe five different patterns frequently found in French \citep[145--149]{Pourcel_Kopecka_2005}. The most frequent is the verb-framed type, as already shown in \REF{ex:3:7}. Another frequent pattern is the coordination of two verb phrases, one containing a Manner verb and the other containing a Path verb:\footnote{Motion events expressed in a single phrase are the main interest of the study, but because the coordinated pattern is so frequent in the data, it is nevertheless listed here.}

\ea\label{ex:3:8}French \citep[145]{Pourcel_Kopecka_2005}\\
\glll  Il court dans une rue puis rentre dans une maison.\\
       He runs on a street then enters into a house \\
       {} Manner {} {}  Path  \\
\z

The authors find a third pattern which they call “reverse verb-framed pattern” because it is structurally identical with a verb-framed pattern but Manner and Path “switch places”, so that the Manner component is expressed in the main verb and the Path component in a gerund, see \REF{ex:3:9}.

\ea\label{ex:3:9}French \citep[145]{Pourcel_Kopecka_2005}\\
\glll Il court {en traversant} la rue.\\
      He runs crossing the street\\
      {} Manner {}  Path\\  
\z

The fourth type, in which Manner is expressed in the verb and Path in a PP, is also called reverse verb-framed pattern by the authors. This fourth type can also be described as a satellite-framed construction, see \REF{ex:3:10}.

\ea\label{ex:3:10}French \citep[145]{Pourcel_Kopecka_2005}\\
\glll    Il court dans le jardin.\\
         He runs into  the garden\\
         {} Manner {}  Path \\
\z

The fifth pattern is a hybrid type because the verbs here encode Path as well as Manner. There are two subtypes to this pattern: In the first, both elements are expressed in the verb, as in \REF{ex:3:11}; in the second, Path is expressed in an incorporated prefix of the verb, as in \REF{ex:3:12}. 

\ea\label{ex:3:11}French \citep[146]{Pourcel_Kopecka_2005}\\
\glll    Marc {a plongé dans} le lac.\\
         Marc {dived into} the lake\\
         {}    Manner.Path\\
\ex\label{ex:3:12}French \citep[149]{Pourcel_Kopecka_2005}\\
\glll   L'oiseau   s'est en-volé du nid.  \\
        {The bird} has away-flown from the nest  \\
        {}         {}  Path-Manner\\
\z

Besides the description of these five patterns, the authors show by using acceptability judgments that in French it is dispreferred to express the Manner component of a motion event as long as it is inferable from the context. Only when the Manner of motion is not typical for the Figure or Ground of the event, it is acceptable to express it \citep[148]{Pourcel_Kopecka_2005}. This finding is in line with the observation of \citet{Berthele_2013} that Manner is seldomly expressed in French motion event encodings.

\subsection{Motion events in Kwa languages}

Kwa languages form part of the Niger-Congo language family, members of which show a general tendency to lexicalize Path in verbs, such as \emph{enter}, \emph{pass}, or \emph{ascend} \citep[200--202]{Schaefer_Gaines_1997}. As for the expression of Manner, much variation is found between the members of this language family \citep[209]{Schaefer_Gaines_1997}.

A more detailed study on the expression of motion events has been carried out for two different Kwa languages, viz. Ewe \citep{Ameka_Essegbey_2013} and Fon \citep{LambertBrtire_2009}.

In Ewe, serial verb constructions combining a Path verb and a Manner verb can be used to express motion events, see \REF{ex:3:13}.

\ea\label{ex:3:13}
Ewe \citep[24]{Ameka_Essegbey_2013} \\
\gll    Devi-a tá yi xɔ-a me.\\
        child-\textsc{def} crawl go room-\textsc{def} in \\
\glt ‘The child crawled into the room.’
\z

It is possible to combine a Manner verb with more than one Path verb, each indicating movement in respect to a different ground object, see \REF{ex:3:14}.

\ea\label{ex:3:14}Ewe \citep[30--31]{Ameka_Essegbey_2013}\\
\gll  Kofi tá tó ve-a me do yi kpó-á dzí.\\
      Kofi crawl pass ditch-\textsc{def} in exit go hill-\textsc{def} top\\
\glt ‘Kofi crawled through the ditch and emerged  at the top of the hill.’
\z

In Fon, motion events can also be expressed using verb serialization, as in \REF{ex:3:15}.

\ea\label{ex:3:15}Fon \citep[14]{LambertBrtire_2009}\\
\gll xɛ̀ví ɔ̀ zɔ̀n gbɔ̀ tá nǔ é\\
     bird \textsc{def} fly pass head for \textsc{3SG} \\
\glt ‘The bird flew over his head.’
\z

As in Ewe, a Manner verb can be combined with more than one Path verb, as in \REF{ex:3:16}.

\ea Fon \citep[22]{LambertBrtire_2009}\label{ex:3:16}\\
\gll Cùkú ɔ́ lɔ̌n tɔ́n sín xɔ̀ mɛ̀ gbɔ̀n flɛ́tɛ́ ɔ́ nù.\\
     dog \textsc{def} jump exit from room in pass window \textsc{def} edge \\
\glt ‘The dog jumped out of the room through the   window.’
\z


Available for motion event verb serialization is a closed class of ten Path verbs \citep[9]{LambertBrtire_2009}. All of these can also be used outside of verb serializations, but not all Path verbs are available for serialization, like e.g. \emph{xá} ‘go.up’ \citep[16]{LambertBrtire_2009}. Similarly, not all Manner verbs are available for serialization, see \REF{ex:3:17}.

\ea\label{ex:3:17}Fon \citep[15]{LambertBrtire_2009}\\
\gll * yě dǔ-wè tɔ́n sìn xwé ɔ́ mɛ̀  \\
     {} \textsc{3pl} move-dance exit from house \textsc{def} in \\
\glt \phantom{*}‘They danced out of the house.’
\z

Whereas \citet[36]{Ameka_Essegbey_2013} classify Manner-Path verb serializations in Ewe as equipollently-framed constructions, \citet[19]{LambertBrtire_2009} argues that the Fon Manner-Path verb serializations are satellite-framed constructions with the Path verbs acting as satellites. She reaches that conclusion because certain inflectional elements can only appear in front of the Manner verb, which marks them, in her point of view, as the main verb of the sentence.

In fact, the question how Manner-Path verb serializations should be classified in the typology described above is controversial. It depends mainly on the question whether the verbs are co- or subordinated. The discussion of this problem is outside of the scope of the present study. More details on the topic can be found in \citet{Talmy_2009}.

\subsection{Motion events in Haitian Creole}
 
To my knowledge, no study has aimed at investigating the expression of motion events in Haitian Creole\footnote{In the following, all examples are from Haitian Creole, so this will not be indicated in the rest of the paper.} until now. Nonetheless, some insights can be obtained from the literature on Haitian Creole. The language possesses an inventory of Path verbs, many of French origin, like in the example in \REF{ex:3:18}.

\ea\label{ex:3:18} \citep[203]{Fattier_2013}\\
\gll Dlo antre anndan kay. \\
     water enter \textsc{loc} house \\
\glt ‘Water came into the house.’
\z

Besides that, of all French-based Creole languages, Haitian Creole is the one that exhibits the most serial verb constructions \citep[44]{Mutz_2017}. Many of those constructions found in the literature do not express motion events, but a few examples of Manner-Path verb serializations can be found, such as the one in \REF{ex:3:19}.

\ea\label{ex:3:19} \citep[244]{Valdman_2015}\\
\gll  Tidjo kouri ale lakay li. \\
      Tidjo run go home \textsc{poss.pron} \\
\glt ‘Tidjo ran over to his house.’
\z

The dissertation on Haitian Creole verb serialization by \citet{BucheliBerger_2009} does not offer examples of Manner-Path verb serialization, but lists the possible combinations of Manner and Path verbs, a shortened version of which is reproduced here in \tabref{tab:tab1_03} (on the following page).\footnote{Her results are derived from online research. Marked as possible are those combinations for which she could find examples online. If a combination is not marked as possible, this does not necessarily mean that it is impossible but simply that the author could not find an example for it during her research. No acceptability study was carried out. An anonymous reviewer of this paper notes that some combinations, especially the ones with \emph{tonbe}, sound strange to them.}

% Table 1:
\begin{table}
\resizebox{\linewidth}{!}{%
\begin{tabular}{lllllllll}
    \lsptoprule
    & \textit{al(e)} & \textit{vin(i)} & \textit{sòt(i)/} & \textit{antre} & \textit{rive} & \textit{monte/} & \textit{desann} & \textit{(re-) tounen} \\
    & ‘go’ & ‘come’ & \textit{sot(i)} & ‘go in’ & ‘arrive’ & \textit{moute} & ‘go down’ & ‘come back’ \\
     &  &  & ‘go out’ &  &  & ‘go up’ &  &  \\ \midrule
    \textit{kouri ‘run’} & + & + & + & + & + & + & + & + \\
    \textit{mache}  ‘march’ & {+} & {+} & {+} & {+} & {+} & {} & {} & {} \\
    \textit{naje}  ‘swim’& {+} & {} & {+} & {} & {} & {} & {} & {} \\
    \textit{woule} ‘roll’ & + & + & + &  &  &  & + &  \\
    \textit{koule} ‘flow’ &  &  & + &  &  &  & + &  \\
    \textit{vole} ‘fly’ & + & + & + &  &  &  & + &  \\
    \textit{glise} ‘glide’ &  &  & + &  &  &  & + &  \\
    \textit{tonbè} ‘fall’ &  &  & + &  & + &  &  &  \\ 
    \lspbottomrule
\end{tabular}}
\caption{Manner of Motion V1 + Path of Motion V2 in Haitian Creole after \citet[202]{BucheliBerger_2009}}
\label{tab:tab1_03}
\end{table}


\section{Study design}\label{sec:3:3}

The present study investigates the expression of motion events as presented above in Haitian Creole. The main purpose is to describe the morphosyntactic elements used to express the components Manner and Path and the preferences of their use. For this purpose, four Haitian Creole speakers living in Berlin, Germany, took part in interviews that consisted of three parts: an interview on their habits of language use and attitudes towards all their languages, a narration of a picture story, and descriptions of single pictures representing different motion events that were drawn by the author of this study. The entire interviews were held in Creole. One of the participants, P1, helped realize the other three interviews as well as transcribe and translate the language data recorded. She will henceforth be referred to as the main participant. More information on participants and tasks is given in the following sections.


\subsection{Participants}

The four participants were aged between 34 and 56. P1 is female; P2, P3 and P4 are male. P1 is a B.A. student, P2 is a mechanical engineer, P3 is a salesperson and photographer and P4 is a political scientist and educator in development cooperation. They were all born in Haiti and completed most of their education there. P1, P2 and P3 come from the area of Port-au-Prince, P4 moved there from the North of the country when he was ten years old. All four emigrated between 20 and 30 years of age. P2 and P4 regularly work in Haiti. The four participants all speak Haitian Creole, French, German, Spanish, and English. They learned Haitian Creole as their first language from their parents and later learned French in school. They received education almost entirely in French; only P1 had Creole language classes for one year. All four report they are able to converse fluently in Creole but have problems with writing, as they have never learned a norm. As the four participants all live in Germany, they speak German on a daily basis.\footnote{Of course, the fact that the four participants all live in a non-Creole-speaking country and use other languages on a daily basis could influence the Creole they speak causing it to be different from the Creole spoken in Haiti. Because the present study was carried out as an MA thesis, getting fieldwork data was not possible. Possible contact phenomena will not be investigated in the present study, but this has to be kept in mind when interpreting the results.} P1, P3 and P4 report that they speak French often, mostly with their family, especially with their children. P4 also speaks French (as well as Creole) at work. Creole is spoken with friends and family in Haiti and abroad, e.g. with their parents and siblings. P2 is the only participant that reports that he speaks Creole often, mainly with his children but also the rest of the family, as well as when working in Haiti. He is also the only one to name Creole as the language he finds most elegant; for the three others that language is French. When asked what the Creole language means to them, all four replied that it is an important part of their identity and their origin. P3 and P4 also say that they feel that Creole is the most important one of their languages.

\subsection{Tasks}

There were two tasks aiming at eliciting motion events, the narration of a picture story and the description of single pictures drawn for the purpose of this study. The picture story selected was \emph{Die Geschichte vom Vogel} (‘The bird story’) \citep[from][]{RettichRettich1972}.\footnote{Unfortunately, the image of the picture story cannot be reproduced here due to copyright reasons.} Even though the Frog Stories have been used to elicit motion events in many previous studies, they were not used here, first because they were considered difficult to narrate by the author of this study and her supervisor, and second because many of the Frog Story pictures do not contain motion events. The bird story is about a bird that flies out of its cage and then out of its house. Outside, he meets different animals that all chase him away. Finally, he flies back to his house and into his cage. The story was chosen because it contains many different motion events which could help determine how frequent the Manner component would be expressed in order to investigate the Manner salience of Haitian Creole.

The description of single pictures aimed at exploring the morphosyntactic elements that could be used to express different Manner-Path combinations. Therefore, seven Manner elements (\emph{run, swim, fly, jump, crawl, dance, roll}) and ten Path elements (\emph{out, away, to, into, up, down, along, past, after somebody, through}) were used to create a total of 48 motion events, as in \REF{ex:3:20}, which were then portrayed in simple pictures by the author of this study.

\ea\label{ex:3:20}
\gll    He runs out {of the burning house.}\\
        {} Manner Path\\
\z

% Figure 1:
\begin{figure}
    \includegraphics[width=0.6\linewidth]{figures/fig1_03.png}
    \caption{Depiction of the motion event ‘to run out of’}
    \label{fig:fig1_03}
\end{figure}

The combination of the 17 motion event components would have yielded more than 48 events, but it was decided not to overwhelm the participants with too many pictures. The 48 drawings were divided into two groups of 24, which were presented to two participants each. When dividing the pictures, the different Manner and Path components were divided as equally as possible between the two groups. Within the two groups, the pictures were arranged in such a way that two following pictures never contained a component already depicted in the previous picture. During the interview, the participants were told to describe what the person in the picture was doing.

\subsection{Data analysis}

The interviews were transcribed and translated into German by the main participant. Transcriptions and translations were later checked by the author of this paper and revised together.

At the beginning of the data analysis, the number of sentences was counted for the picture story narrations. Every unit containing a subject and (at least) one verb was counted as one sentence. Coordinated clauses were counted as two sentences, but subordinated clauses like complement clauses, relative clauses, causal clauses, temporal clauses and the like were counted as part of the matrix clause.

P1’s narration contains 26 sentences, P2’s contains 46 sentences, P3’s contains 55, and P4’s narration contains 42 sentences. In total, 169 sentences were elicited.

After counting the number of sentences, the number of motion events encoded was determined. Every sentence expressing directional motion was analyzed as a motion event encoding.

P1’s narration contains 18, P2’s 20, P3’s 29, and P4’s 27 motion event encodings. Altogether, the four participants encoded 94 motion events in their picture story narrations. With a total of 169 sentences, more than 50\% of the sentences contained a motion event encoding.

The motion events were then sorted by means of the morphosyntactic structure they used to encode different motion event components. They were sorted into six different categories: Path verbs only, see \REF{ex:3:21}, Path verbs with Ground-PP/NP, see \REF{ex:3:22}, Manner verbs only, see \REF{ex:3:23}, Manner verbs with path elements, see \REF{ex:3:24}, serial verb constructions, see \REF{ex:3:25}, and motion events without a motion verb, see \REF{ex:3:26}. The remaining cases were classified as “Other”.

\ea\label{ex:3:21}
\gll Li rantre.\\
     \textsc{3sg} enter.again \\
\glt ‘He goes back in.’
\ex \label{ex:3:22}
\gll Epi l antre nan kay la. \\
and \textsc{3sg} enter \textsc{loc} house \textsc{def} \\
\glt ‘He enters the house.’
\ex E zwazo a kouri. \\\label{ex:3:23}
    and bird \textsc{def} \textsc{run} \\
\glt ‘And the bird runs/flies fast.’
\ex \label{ex:3:24}
\gll {{\ob}…{\cb}} zwazo a vole sou do yon erison.\\
     {} bird \textsc{def} fly \textsc{loc} back \textsc{indef} hedgehog \\
\glt ‘The bird flies onto the back of a hedgehog.’
\ex \label{ex:3:25}
   \gll Li kouri retounen nan kay {[}kote li te ye a.{]}\\
    \textsc{3sg} run return \textsc{loc} house {[}\textsc{rel.pron} \textsc{3sg} \textsc{pst} \textsc{cop} \textsc{def}{]} \\
\glt `He goes back into the house where he was before.'
\ex \label{ex:3:26}
\gll Epi li kraze rak.\\
     then \textsc{3sg} destroy forest \\
\glt ‘Then he beats loose.’
\z

The results of the analysis will be given in the following section.

The first step of the picture description analysis was to determine the number of descriptions. As 48 pictures were described by two participants each, 96 descriptions should have been elicited, but as one of the participants failed to describe two of the pictures, only 94 descriptions were elicited. Some of the descriptions consist of a simple sentence, whereas others consist of a complex sentence or even more than one sentence. A total of 119 sentences were elicited in both tasks.

If more than one sentence was used for the description of a picture, they were counted and analyzed separately. The same holds for complex sentences if they contained more than one motion event, e.g. a sequence of two relative clauses, as in \REF{ex:3:27}, or sentences with \emph{pou} `in order to', as in  \REF{ex:3:28}.

\ea\label{ex:3:27}
\gll Yon zwazo k ap vole k ap pase bò kot yon pyebwa.\\
\textsc{indef} bird \textsc{rel.pron} \textsc{prog} fly \textsc{rel.pron} \textsc{prog} \textsc{pass} beside side \textsc{indef} tree\\
\glt ‘A bird which is   flying, who is passing next to a tree.’\footnote{Mostly P2, but also P4, described several of the pictures with utterances of the form NP + relative clause. Even though these do not constitute regular sentences, it is possible to analyze them as elliptic versions of sentences like \emph{This is [NP] who is moving} which are also found in some descriptions. They were therefore included in the analysis.}
\ex\label{ex:3:28}
\gll Yon mesye k ap naje sòti nan plaj pou l ale bò rivaj. \\
\textsc{indef} man \textsc{rel.pron} \textsc{prog} swim exit \textsc{loc} beach for \textsc{3sg} go beside coast\\
\glt ‘A man who is swimming away from the beach in   order to swim to the coast.’
\z

Sentences which did not express motion (13 of 119) were not analyzed.

In a few cases, modal verbs were used, see \REF{ex:3:29} and \REF{ex:3:30}. These were ignored for the analysis and the event encodings of the motions were treated as if they did not contain a modal verb.

\ea\label{ex:3:29}
\gll Yon gason ki vle monte sou yon tab.\\
      \textsc{indef} boy \textsc{rel.pron} want ascend \textsc{loc} \textsc{indef} table \\
\glt ‘A boy who wants to go up onto a table.’
\ex\label{ex:3:30} 
\gll Yon mesye ki dwe travèse dyagonal yon chanm. \\
     \textsc{indef} man \textsc{rel.pron} must cross diagonal \textsc{indef} room \\
\glt `A man who has to cross a room diagonally.'
\z

The motion events expressed in the picture descriptions were then sorted into eight categories, seven of which are equivalent to those for the picture story. A new category was established for this part of the data: Manner verbs with Ground elements, as exemplified in \REF{ex:3:31}.

\ea \label{ex:3:31}
\gll  Yon moun k ap rale kote yon mi.\\
     \textsc{indef} person \textsc{rel.pron} \textsc{prog} crawl beside \textsc{indef} wall \\
\glt `A person who is crawling next to a wall.'
\z

The results of the analysis are given in the following section.


\section{Results}\label{sec:3:4}

An overview of the results is given in \tabref{tab:tab2_03}. In the following subsections, the results are discussed in detail.

% Table 2:
\begin{table}
\begin{tabular}{l rr rr rr}
\lsptoprule
                 & \multicolumn{2}{c}{Picture} & \multicolumn{2}{c}{Single} & \\
                 & \multicolumn{2}{c}{story}   & \multicolumn{2}{c}{pictures} &\multicolumn{2}{c}{Total} \\\midrule
{Path verb only} &  10     & 10.6\% & 4      & 3.4\%  & 14     & 6.6\% \\
{Path verb + ground PP/NP} &  25       & 26.6\%   & 29       & 24.4\%   & 54       & 25.4\%  \\
{Manner verb only} & 18      & 19.1\%  & 19      & 16.0\%    & 37      & 17.4\% \\
{Manner verb + ground} &  0      & 0.0\%      & 13     & 10.9\% & 13     & 6.1\% \\
{manner verb + path element} &    2     &  2.1\% &  5     &  4.2\% &  7     &  3.2\%\\
{SVC} &  16     & 17.0\%   & 36     & 30.3\% & 52     & 24.4\%\\
{Motion event without motion verb} &   7      &  7.4\%  &  1      &  0.8\%  &  8      &  3.8\% \\
{Other} &   16      & 17.0\%    & 12      & 10.1\%  & 28      & 13.1\% \\
\midrule
{Total} &  94 & & 119 & & 213 & \\
\lspbottomrule
\end{tabular}
\caption{Motion event expressions in picture story narrations and single picture descriptions}
\label{tab:tab2_03}
\end{table}

Path verbs only were used ten times in the picture story narrations (10.6\% of all occurrences). In most cases, the Ground was mentioned in the preceding or following context but not in the same clause, see \REF{ex:3:32} and \REF{ex:3:33}.

\ea \label{ex:3:32}
\gll E li ouvri pòt kalòj la pou li kapab sòti.\\
     and \textsc{3sg} open door cage \textsc{def} for \textsc{3sg} able.to exit\\
\glt ‘And she opens the door of the cage so that he can go out.’
\ex \label{ex:3:33}
\gll Epi zwazo a tounen. L al nan menm kay la {[}...{]}\\
     and bird \textsc{def} return \textsc{3sg} go \textsc{loc} same house \textsc{def}\\
\glt ‘And he returns. He goes into the same house.’
\z

In one case, no ground is mentioned at all, see \REF{ex:3:34}.

\ea\label{ex:3:34}
\gll Papiyon an ale.\\
     butterfly \textsc{def} go\\
\glt ‘The butterfly goes/flies away.’
\z

At this point, the picture story shows a butterfly flying away from the bird. Therefore, \textit{ale} seems to express not simply `go' but `go away' here.

In the single picture descriptions, Path verbs only occurred in four motion events (3.4\%). As in the narrations, the Ground was usually mentioned in the context, see \REF{ex:3:35}.

\ea\label{ex:3:35}
\gll    Yon moun k ap rale kote yon mi.  L ap pase {[}…{]} \\
        \textsc{indef} person beside \textsc{rel.pron} \textsc{prog} crawl beside \textsc{indef} wall  \textsc{3sg} \textsc{prog} pass \\
\glt ‘A person is crawling next to a wall. He is passing {[}it{]}…’
\z

Again, there was one case where no Ground was mentioned at all, again with the verb \textit{ale}, which seems to mean `go away' \REF{ex:3:36}.

\ea\label{ex:3:36}
\gll Yon zwazo ki sòti nan kalòj pou ale.\\
     \textsc{indef} bird \textsc{rel.pron} exit \textsc{loc} cage for go \\
\glt ‘A bird who leaves the cage in order to go/fly away.’
\z

\subsection{Path verb + Ground NP/PP}

The most frequent pattern used to express motion events in the picture story is a Path verb with a Ground NP or PP, which was used in 25 cases (26.6\%). The verbs \emph{antre} ‘enter’, \emph{pase} ‘pass’, \emph{atèri} ‘land’, \emph{tonbe} ‘fall’, \emph{ale} ‘go’, \emph{poze} ‘sit down’, and \emph{sòti} ‘go out’ were used with PPs \REF{ex:3:37}.

\ea
\label{ex:3:37}
\gll Li atèri sou flè a.\\
     \textsc{3sg} land \textsc{loc} flower \textsc{def}\\
\glt ‘He lands on the flower.’
\z

The verbs \emph{jwenn} ‘reach’, \emph{suiv} ‘follow’, \emph{kite} ‘leave’, \emph{tounen} ‘come back’ were used with object NPs \REF{ex:3:38}.

\ea\label{ex:3:38}
\gll  Li kite do erison an.\\
      \textsc{3sg} leave back hedgehog \textsc{def}\\
\glt ‘He leaves the back of the hedgehog.’
\z

The verb \emph{rive} ‘arrive’ was used with a PP three times (by P2 and P3) and with an object NP once (by P1), see \REF{ex:3:39} and \REF{ex:3:40}. Because of the small number of occurrences, nothing can be said about whether this is simply due to individual preferences.

\ea\label{ex:3:39}
\gll  Lè l rive sou pyebwa {[}…{]} \\
      when \textsc{3sg} arrive on tree  \\
\glt ‘When he arrives on the tree…’
\ex\label{ex:3:40}
\gll    Zwazo a rive lakay li.\\
        bird \textsc{def} arrive home \textsc{poss.pron}\\
\glt ‘The bird arrives at his house.’
\z

In the single picture descriptions, Path verbs with Ground NPs or PPs present the second most frequent pattern with 29 occurrences (24.4\%).

Used with an object NP were the verbs \emph{depase, desann}, and \emph{kite} \REF{ex:3:41}.

\ea\label{ex:3:41}
\gll  Tidjo kite lekòl la. \\
      Tidjo leave school \textsc{def}\\
\glt ‘Tidjo leaves the school.’
\z

The verbs \emph{antre, rantre, pase} and \emph{al(e)} were used with PPs. \emph{Antre} and \emph{rantre} were used with  \emph{nan} ‘into’, \emph{al(e)} with \emph{bò} ‘next to’ and \emph{nan direksyon} ‘in the direction of’, and \emph{pase} also with \emph{bò}, see \REF{ex:3:42}.

\ea\label{ex:3:42}
\gll  L ap pase bò yon kay. \\
      \textsc{3sg} \textsc{prog} pass beside \textsc{indef} house \\
\glt ‘He is passing a house.’
\z

The verbs \emph{monte} and \emph{sòti} were used with both NPs and PPs. \emph{Sòti} was used with three different prepositions, \emph{nan, sou} and \emph{a travè}. See \REF{ex:3:43} for an example with \emph{a travè}, and \REF{ex:3:44} for the use with an NP.

\ea\label{ex:3:43}
\gll Sa se   yon   moun   ki       sòti {a travè} yon fenèt {[}…{]}\\
     \textsc{dem} \textsc{cop} \textsc{indef} person \textsc{rel.pron} exit through \textsc{indef} window\\
\glt ‘That is a person who leaves through a window.’ 
\ex\label{ex:3:44}
\gll    Yon timoun ki sòti lekòl. \\
        \textsc{indef} child \textsc{rel.pron} exit school \\
\glt ‘A child that leaves school.’
\z

As there are only a few occurrences for each verb, often just one but six at the most, it remains unclear whether the use with NP or PP attested here is a general preference of the verb or whether all verbs can appear with both.

\subsection{Manner verb only}

A Manner verb alone cannot, strictly speaking, express a motion event as it is defined above, but because they occur so frequently in both picture story narrations and single picture descriptions, they are taken into account here.

In the picture story task, in the 18 cases counted for this category (19.1\%), only three different Manner verbs were used, \emph{vole} ‘fly’, \emph{kouri} ‘run‘, and \emph{mache} ‘walk’. The last of the three is used only once where a hedgehog continues walking after the bird has landed on his back. The most frequently used of these verbs is \emph{vole}. This is not surprising when taking into account that the story is about a bird and also features other flying animals like owls or butterflies. \emph{Kouri} was used six times to describe the motion of the bird, see \REF{ex:3:45}.

\ea\label{ex:3:45}
\gll Lè chwèt la kouri dèyè zwazo a, sa k pase, zwazo a kouri.\\
     when owl \textsc{def} run behind bird \textsc{def} \textsc{def} \textsc{rel.pron} happen bird \textsc{def} run \\        
\glt ‘When the owl flies behind/after the bird, the bird runs/flies away fast.’
\z

As the story shows several instances of an animal chasing another animal (mostly the little bird) away, the cases where \emph{vole} and \emph{kouri} are used alone always describe a situation where the animal flees. Apparently, in these cases, a directed motion away from the place of action seems to be described. The Path ‘away’ seems to be inferable from the context and is therefore left out. \emph{Kouri} obviously does not express the Manner ‘run’ in these cases, but rather an accelerated manner of movement.

In the single picture descriptions, Manner verbs were used alone in 19 cases (16\% of all occurrences). In seven of these, no further information on the motion event was given, see \REF{ex:3:46}.

% (46)	Yon 	mesye  ki 	     ap 	     naje.
% 		INDEF	man	REL.PRON PROG  swim
% 		‘A man that is swimming.’
\ea\label{ex:3:46}
\gll  Yon mesye ki ap naje. \\
      \textsc{indef} man \textsc{rel.pron} \textsc{prog} swim \\
\glt ‘A man that is swimming.’
\z

These cases do not express a motion event as it is understood here, but an activity.

In the remaining twelve cases, more information on the motion event is given in the preceding or the following context. In five cases, the manner verb is followed by a construction with \emph{pou} ‘to’ in which Path is expressed, see \REF{ex:3:47}.

\ea\label{ex:3:47}
\gll  Tidjo ap naje pou l depase lòt la.\\
      Tidjo \textsc{prog} swim for \textsc{3sg} pass other \textsc{def} \\
\glt ‘Tidjo is swimming in order   to pass the other.'
\z

In three cases, information on the Path is given in the preceding or following sentence, see \REF{ex:3:48} and \REF{ex:3:49}. As the translations show, it is possible to interpret these cases as single complex motion events.

\ea\label{ex:3:48}
\gll Tidjo ap naje. Li kite il la.  \\
     Tidjo \textsc{prog} swim \textsc{3sg} leave island \textsc{def}  \\
\glt ‘Tidjo is swimming. He leaves the island./Tidjo is swimming away from the island.’

\ex\label{ex:3:49}
\gll Tidjo antre nan kay. {[}…{]} L ap danse.\\
     Tidjo enter \textsc{loc} house {} \textsc{3sg} \textsc{prog} dance  \\
\glt ‘Tidjo enters the house. He is dancing./Tidjo is dancing into the house.’
\z

In two cases, Path and Manner are expressed in two precedent relative clauses, see \REF{ex:3:50}. Again, it is possible to interpret this as a single complex motion event.

\ea\label{ex:3:50} 
\gll Yon zwazo k ap vole k ap pase bò yon mi. \\
     \textsc{indef} bird \textsc{rel.pron} \textsc{prog} fly \textsc{rel.pron} \textsc{prog} pass beside \textsc{indef} wall  \\
\glt ‘A bird who is flying, who is passing beside a wall.\slash A bird who is flying past a wall.'
\z

In one case, two main clauses, one containing a Manner and the other a Path verb, are combined with the conjunction \emph{pandan} ‘while’, see \REF{ex:3:51}.

\ea\label{ex:3:51}
\gll L ap danse pandan l ap monte mach eskalye a.\\
     \textsc{3sg} \textsc{prog} dance while \textsc{3sg} \textsc{prog} ascend  step stairs \textsc{def}\\
\glt ‘He dances while he goes up the stairs./He dances up the stairs.’
\z

All these examples seem to represent complex motion events whose components were not expressed in a single clause, meaning that they were not conflated into one event.

\subsection{Manner verb + Ground element}

Similar to the previous category, Manner verbs combined with a Ground element in the same clause do not constitute motion events as defined by \citet{Talmy_1985}, because the Path element obligatory for motion events is not encoded. Such a pattern does not occur in the picture story narrations, but there are 13 such occurrences in the single picture descriptions (10.9\%). Six of the seven manner verbs tested in this task were used in this pattern, see \REF{ex:3:52} for an example with \emph{naje} ‘swim’.

\ea\label{ex:3:52}
\gll    Yon moun k ap naje nan lanmè.\\
        \textsc{indef} person \textsc{rel.pron} \textsc{prog} swim \textsc{loc} sea\\
\glt ‘A person who is swimming in the sea.’
\z

\subsection{Manner verb + Path element}

The least frequent of the patterns is the use of a Manner verb in combination with a Path element. Such cases constitute instances of the satellite pattern described above.

In the picture story descriptions, two such cases (2.1\%) occurred with the verb \emph{vole}, once with the preposition \emph{sou} ‘onto’, see \REF{ex:3:53}, and once with \emph{deyò} ‘out of’, see \REF{ex:3:54}.

\ea\label{ex:3:53}
\gll {[}…{]} zwazo a vole sou do yon erison. \\
      {}  bird \textsc{def} fly \textsc{loc} back \textsc{indef} hedgehog \\
\glt ‘A bird flies onto the back of a hedgehog.’
\ex\label{ex:3:54}
\gll    Li vole deyò. \\
        \textsc{3sg} fly outside  \\
\glt ‘He flies outside.’
\z

In the single picture descriptions, this satellite-framed pattern is used in five cases (4.2\%), three of them with the verb \emph{vole}, see \REF{ex:3:55}, one with \emph{naje}, see \REF{ex:3:56}, and one with \emph{woule} ‘roll’.

\ea\label{ex:3:55}
\gll Yon zwazo k ap vole a travè nyaj yo. \\
     \textsc{indef} bird \textsc{rel.pron} \textsc{prog} fly through cloud \textsc{pl} \\
\glt ‘A bird that is flying through the clouds.’
\ex\label{ex:3:56}
\gll    Tidjo ap naje {[}…{]} sou lòt bò lak la. \\
        Tidjo \textsc{prog} swim  {} \textsc{loc} other side lake  \textsc{def}\\
\glt ‘Tidjo is swimming to the other side of the lake.’
\z

\subsection{Serial verb constructions}

Serial verb constructions are used frequently in both the picture story narrations and the single picture descriptions. As for the story narrations, they present the second most frequent pattern with 16 occurrences (17\%). Most of these consist of a serialization of two Path verbs: \emph{sòti kite} ‘leave go away’, \emph{al(e) poze} `go sit down', \emph{vin poze} `come sit down', \emph{al tonbe} ‘go fall’ and \emph{al antre} ‘go enter’. In two cases, a Manner verb is followed by a Path verb: \emph{vole poze} ‘fly sit down’ and \emph{kouri retounen} ‘run return’. One single verb serialization consists of three verbs: \emph{leve pran kouri} ‘get up take run’. \emph{Pran} most probably acts as an aspectual marker here, encoding inchoativity (see also \cite{Valdman_2015}: 231). Besides the fact that this serial verb construction consists of three verbs, it is also different from the others because the order of the Path and Manner verb is inverted here in regard to the other cases: the Path verb is the first, the Manner the last verb of the serialization.

\hspace*{-2.8pt}The different serial verb constructions occur either with a Ground NP, a Ground PP, or with no Ground element in the same clause. In the last case, Ground is mentioned in the surrounding context.

In four serial verb constructions, \emph{vole poze, vin poze, al(e) poze} and \emph{al tonbe}, the same kind of action is expressed: the bird flying to a certain place and coming to rest. It is not clear whether these are sequential or simultaneous verb serializations, the first meaning ‘he flies and then sits down’ and the latter meaning ‘he flies onto [the tree]’. The same problem exists with \emph{al antre} 'go enter', where it is unclear whether it is said that the bird first goes and then enters or whether he ‘goes into’. 

The case of \emph{sòti kite} is described by \citet[79--81]{BucheliBerger_2009} as a serialization of two verbs that are close to synonyms and which she interprets as a simultaneous serial verb construction expressing one simple motion. It is also possible that the verbs have different meanings here, expressing the Paths ‘out of’ and ‘away’, which would make this a sequential serial verb construction.

In \emph{kouri retounen}, which once again describes the motion of the verb, \emph{kouri} cannot be interpreted as expressing the Manner ‘to run’, but rather a fast way of moving.

\emph{Leve pran kouri} is probably a sequential serial verb construction, expressing that the bird first gets up and then flies away fast, where the Path `away' is left unexpressed and to be inferred from the context.

In the single picture descriptions, serial verb constructions were the pattern most frequently used by the participants to encode a motion event. With 36 occurrences in total (30.3\% of all occurrences), 31 different verb combinations can be described. The different internal structures of these serial verb constructions are summarized in \tabref{tab:tab3_03}.

% Table 3:
\begin{table}[!ht]
\centering
\resizebox{\linewidth}{!}{%
\begin{tabular}{lccccccc}
    \lsptoprule
    \multirow{2}{*}{Type of SVC} & \multirow{2}{*}{manner-path} & \multirow{2}{*}{manner-manner} & \multirow{2}{*}{path-path} & \multirow{2}{*}{path-Other} & \multicolumn{3}{c}{3 verbs} \\ \cmidrule{6-8}
    &  &  &  &  & MMP & PPP & PMP \\ \midrule
    Frequency & 19 & 2 & 1 & 6 & 1 & 1 & 1 \\ \lspbottomrule
\end{tabular}%
}
\caption{Frequency of different serial verb constructions in the single picture descriptions}
\label{tab:tab3_03}
\end{table}

By far the most common verb serializations are those consisting of a Manner and a Path verb. Six different Manner verbs were used in such constructions: \emph{kouri} ‘run’, \emph{naje} ‘swim’, \emph{vole} ‘fly’, \emph{rale} ‘crawl’, \emph{woule} ‘roll’, and \emph{glise} ‘glide/slide’ (used once instead of `roll'). Two of the Manner verbs investigated were not used in Manner-Path serializations, \emph{sote} ‘jump’ and \emph{danse} ‘dance’. Table \ref{tab:tab4_03} shows the different Path verbs that these Manner verbs were used with. Nothing can be said about the possibility to form serial verb constructions other than the ones attested in the data of this study.\footnote{An anonymous reviewer of this paper notes that more combinations than the ones attested in this study should be possible, especially with \emph{rale}. Also, all manner verbs should be able to combine with \emph{ale}.}

% Table 4:
\begin{table}[!ht]
\centering
\resizebox{\linewidth}{!}{%
\begin{tabular}{@{}lllllllllll}
    \lsptoprule
    & \textit{sòti} & \textit{kite} & \textit{antre} & \textit{monte} & \textit{desann} & \textit{pase} & \textit{travèse} & \textit{ale} & \textit{poze} & \textit{suiv} \\ \cmidrule{2-11}
    \textit{kouri} & + & + & + & + & + & + & + &  &  &  \\
    \textit{naje} & + &  &  &  &  &  &  &  &  &  \\
    \textit{vole} & + &  & + &  &  &  &  & + & + & + \\
    \textit{rale} & + &  &  & + &  &  &  &  &  &  \\
    \textit{woule} &  &  &  &  & + & + &  &  &  &  \\
    \textit{glise} & + &  &  &  & + &  &  &  &  & \\ \lspbottomrule
\end{tabular}%
}
\caption{MANNER-PATH-SVC in the single picture descriptions}
\label{tab:tab4_03}
\end{table}

These Manner-Path serial verbs also occur both with NPs and PPs, see \REF{ex:3:57} and \REF{ex:3:58}.

\ea\label{ex:3:57}
\gll Tipyè rale monte mòn nan. \\
     Tipyè crawl ascend mountain \textsc{def} \\
\glt ‘Tipyè crawls up the mountain.’
\ex\label{ex:3:58}
\gll Zwazo  a vole antre nan kizin nan. \\
     bird \textsc{def} fly enter \textsc{loc} kitchen \textsc{def} \\
\glt ‘The bird flies into the kitchen.’
\z

In one case, no ground element is given in the same clause, see \REF{ex:3:59}. As the Path verb is \emph{ale} in this case, the Path can probably again be interpreted as ‘away’ in this case.

\ea\label{ex:3:59}
\gll  Yon zwazo {[}…{]} k ap vole ale. \\
      \textsc{indef} bird   {} \textsc{rel.pron} \textsc{prog} fly go \\
\glt ‘A bird that is flying away.’
\z

In two cases, two Manner verbs were serialized, see \REF{ex:3:60} and \REF{ex:3:61}. In \REF{ex:3:60}, it is obvious that again \emph{kouri} cannot be interpreted as 'run', but most probably means that the motion takes place fast. In \REF{ex:3:61}, it is unclear what the exact meaning of \emph{vole} is. It could possibly be interpreted as having a figurative meaning expressing something like jumping into the water in a high arc. This hypothesis cannot be tested here.\footnote{An anonymous reviewer suggests that the two verbs act like synonyms here.}

\ea\label{ex:3:60}
\gll Tipyè kouri naje dèyè yon lòt.\\
     Tipyè run swim behind \textsc{indef} other \\
\glt ‘Tipye swims fast behind/after another.’
\ex\label{ex:3:61}
\gll Li vole sote nan dlo a.\\
     \textsc{3sg} fly jump \textsc{loc} water \textsc{def} \\
\glt ‘He jumps into the water (in a high arc).’
\z

The only Path-Path serialization in the single picture descriptions is \emph{sòti kite}, which also occurred in the picture story narrations.

In six cases, the participants used serial verb constructions consisting of a Path and a non-motion verb, like in \REF{ex:3:62}.

\ea\label{ex:3:62}
\gll   Yon moun ki ap antre kache nan yon gwòt. \\
      \textsc{indef} person \textsc{rel.pron} \textsc{prog} enter hide \textsc{indef} \textsc{loc} cave \\
\glt ‘A person who is going into a cave in order to hide.’
\z

In three of these six cases, the verb \emph{ale} was used, but cannot be interpreted as a Path verb, see \REF{ex:3:63}. It could be interpreted as an analytical future, but such a structure with this function has not been described for Haitian Creole (see for example, \cite{Valdman_2015,DeGraff_2007}). This structure cannot be further analyzed at this point.

\ea\label{ex:3:63}
    \gll {[Sa se Johana k ap kouri]} pou l al pran bis la  nan stasyon an.\\
          {}                        for \textsc{3sg} go take bus \textsc{def} \textsc{loc} station \textsc{def}\\
\glt ‘{[}That’s Johanna who is running{]} in order to take a bus at the station.’
\z

In three cases, three motion verbs are serialized. The first of them is a Manner-Manner-Path serialization, see \REF{ex:3:64}. As this sentence is about a swimming person, \emph{kouri} probably once again express a fast motion.

\ea\label{ex:3:64}
\gll  Li kouri naje kite il la. \\
      \textsc{3sg} run swim leave island \textsc{def} \\
\glt ‘He swims away from the island fast.’
\z

The second of these serializations consists of three Path verbs \REF{ex:3:65}. \emph{Rive} and \emph{jwenn} are close to synonyms and are therefore interpreted as expressing the same meaning here. Together with \emph{avanse} ‘move forward’ they probably present a sequential serial verb construction.

\ea\label{ex:3:65}
\gll  Li avanse rive jwenn demwazèl la. \\
      \textsc{3sg} advance arrive reach lady \textsc{def} \\
\glt ‘He advances towards and reaches the lady.’
\z

In the last of the three cases, we find a Path-Manner-Path verb serialization, see \REF{ex:3:66}. Interestingly, V1 and V3 are the same verb, \emph{sòti}. As this is the only case where we find such a structure in the data, nothing can be said about whether this is a common pattern of serial verb constructions in Haitian Creole.

\ea\label{ex:3:66}
\gll   Boul la sòti woule sòti nan bwàt katon.\\
       ball \textsc{def} exit roll exit \textsc{loc} box cardboard\\
\glt ‘The ball rolls out of the cardboard box.’
\z

Even though many of the serial verb constructions elicited in this study are single cases that need to be described separately, some patterns can also be found. One frequent strategy to express complex motion events in Haitian Creole is the use of a Manner-Path verb serialization, which in a few cases also occurred in serializations of three verbs. In other cases, two Path verbs or a Path and a non-motion verb were serialized to encode a motion event depicted in one of the drawings. 

\subsection{Motion events without a motion verb}

In the picture story narrations, seven cases occurred where a motion event was expressed without a motion verb (7.4\%). Five of these were uttered by P3, when he used the idiom \emph{kraze rak}, which has the meaning ‘to beat loose’. The two other cases were uttered by P2 using \emph{jwenn direksyon} and \emph{pran direksyon} to say that the bird was going into a certain direction, see \REF{ex:3:67}.

\ea\label{ex:3:67}
\gll  {{[}…{]}} kounye a la      li jwenn direksyon fenèt la \\
      {}        moment \textsc{def} \textsc{dem} \textsc{3sg} reach direction window \textsc{def}\\
\glt ‘Now he takes the direction of the window.’
\z

In the single picture descriptions, there is only one case where a motion event is expressed without a motion verb (0.8\%), also using \emph{direksyon}, see \REF{ex:3:68}.

\ea\label{ex:3:68}
\gll Li pran nan direksyon machin nan. \\
     \textsc{3sg} take \textsc{loc} direction car \textsc{def}\\
\glt ‘He takes the direction of the car.’
\z

\subsection{Other}

The remaining encodings of motion events had to be sorted into a separate category because it was not possible to analyze them as any of the other categories.

For the picture story narrations, all of the 16 cases sorted into this category (17\%) use the preposition \emph{dèyè} ‘behind’ or ‘after’, 13 with \emph{kouri}, see \REF{ex:3:69}, and 3 with \emph{pati}, see \REF{ex:3:70}. In all of these cases, one animal is chasing another.

\ea\label{ex:3:69}
\gll Chwèt la kouri dèyè zwazo  a pou l pran l.\\
     owl \textsc{def} run behind bird \textsc{def} \textsc{3sg} take \textsc{3sg}\\
\glt ‘The owl flies behind/after the bird in order to catch him.’
\ex\label{ex:3:70}
\gll Koukou a pati dèyè zwazo a. \\
     cuckoo \textsc{def} leave behind bird \textsc{def}\\
\glt ‘The cuckoo flies behind/after the bird.’
\z

The problem here is the preposition \emph{dèyè}: If it expresses ‘behind’, the PP can be analyzed to encode the Ground and locate the place where the motion is taking place. If it expresses ‘after’, it can be analyzed as expressing the Path of motion.

The main participant translated all of these cases with \emph{chase away}. This could be the implication of flying fast behind someone. In one case, however, P4 uses \emph{kouri dèyè} where the subject isn’t moving at all. The picture shows a group of birds sitting in a tree chasing away the little bird by screaming at him \REF{ex:3:71}.

\ea\label{ex:3:71}
\gll Zwazo sa yo genlè pa renmen  li. Yo kouri dèyè  li. E lè sa li vole.\\
     bird \textsc{dem} \textsc{pl} seem \textsc{neg} like  \textsc{3sg} \textsc{pl} run behind  \textsc{3sg} and when \textsc{dem} \textsc{3sg} fly\\        
\glt ‘These birds seem to not like him. They chase him away. And then he flies away.’
\z

Another problem is the semantics of \emph{pati}. P1 uses \emph{kouri dèyè} and \emph{pati dèyè} in a similar way. When the owl is chasing the little bird by flying after him, she uses \emph{pati dèyè}. The semantics of \emph{dèyè} are obscure here, because the Path ‘away from X’ is not relevant here. It is neither shown in the pictures nor expressed in the narration.

For the single picture descriptions, twelve descriptions had to be sorted into the category “Other” (10.1\%). Three cases were equivalent to those in the picture story narrations where \emph{dèyè} was used. In two other cases, hybrid verbs were used which could not be clearly identified as Manner or Path verbs as they contain both components: \emph{plonje} ‘dive into’ and \emph{tonbe} ‘fall’.

In \REF{ex:3:72}, a Path verb is combined with a further description of the Path.

\ea\label{ex:3:72}
\gll Yon mesye ki dwe travèse dyagonal {[}…{]} yon chanm.\\
     \textsc{indef} man \textsc{rel.pron} must cross diagonal {} \textsc{indef} room \\
\glt ‘A man who must cross a room diagonally.’
\z

In \REF{ex:3:73}, the Manner component is expressed in a PP, which is the only occurrence of this type.

\ea\label{ex:3:73}
\gll Tidjo ap monte mòn ak kat pat.\\
     Tidjo \textsc{prog} ascend mountain with four paws\\
\glt ‘Tidjo is crawling up the mountain.’
\z

The four remaining occurrences contain a gerund construction, all used by the same participant, P1. In two of these cases, a structure equivalent to the French structure V \emph{en} V.GER is used, see \REF{ex:3:74}. This structure has not been described for Haitian Creole.

\ea\label{ex:3:74}
\gll Li desann eskalye a an dansan. \\
     \textsc{3sg} descend stairs \textsc{def} \textsc{prep} dance.\textsc{ger}\\
\glt ‘He/She goes down the stairs dancing.’
\z

In the other two cases, the Path verb is followed by a gerund form of `to be', \emph{etan}, and then a full sentence consisting of subject, aspect marker and manner verb, see \REF{ex:3:75}. This structure has also not been described for Haitian Creole.\footnote{Both a native speaker present at the talk at the SPCL Meeting in Tampere as well as an anonymous reviewer of this
paper noted that this structure is very uncommon in Haitian Creole and most probably due to other language influence.} 

\ea\label{ex:3:75}
\gll Tipyè antre etan l ap danse nan chanmnam. \\
     Tipyè enter be.\textsc{ger} \textsc{3sg} \textsc{prog} dance \textsc{loc} room \textsc{def} \\
\glt ‘Tipyè is dancing into the room.’
\z


\section{Discussion}\label{sec:3:5}

In the previous section, the morphosyntactic patterns used to express motion events in the present study were described. Most commonly used were three different structures: a Path verb with a Ground NP or DP, a serial verb construction, or a Manner verb only. Path verbs with Ground elements are verb-framed structures in the sense of \citet{Talmy_1991}. In all of those cases, the Manner component of the event was not expressed. Serial verbs often consisted of a Manner and a Path verb, which could be labelled an equipollently-framed construction in the sense of \citet{Slobin_2004}. Other verb serializations were also found, mostly of two Path verbs, but also combinations of Path and non-motion verbs. These are also verb-framed constructions. The third most frequent pattern is the use of a Manner verb only, with no Path element encoded in the same clause. According to \citeauthor{Talmy_1985}'s (\citeyear{Talmy_1985}) definition, the latter is not a motion event. These cases are nevertheless taken into account here, primarily because of their relatively high frequency. Besides that, it is possible that at least some of them do, contrary to Talmy’s definition, express complex motion events, because the Path component is possibly left to be inferred in these cases but implicitly present. This was the case in some examples from the picture story narrations, where the participants uttered sentences like \emph{Zwazo a kouri/vole} ‘The bird flies (fast)’, meaning that the bird is flying away. Most of the uses of a sole Manner verb in the single picture descriptions are probably due to the fact that the depicted motion event was too difficult to recognize as such, as in \emph{swimming along the coast}. In these cases, the participants simply expressed a motion activity instead of the event.

Besides these three main patterns, three further but marginal patterns were found in the data: a Manner verb with a Path element, a Path verb alone and a motion event without any motion verb. Manner verbs with Path elements constitute satellite-framed constructions in the sense of \citet{Talmy_1991}, which were rare but are still attested in the present data. They occurred with few, but different Manner verbs and also different Path elements. When Path verbs were used alone, information on the Ground was usually given in the context. The expression of motion events with idioms or constructions like \emph{pran nan direksyon} ‘take a certain direction’ is most probably not typical for Haitian Creole but occurs in many languages.

The different patterns described above show that Haitian Creole possesses a rich inventory of morphosyntactic structures available to express motion events. It is therefore not classified here as a language of one of the three types described above, VF, SF and EF languages. All of these three patterns are found in the Haitian Creole data, VF and EF patterns being more frequent than SF patterns.

Some problematic cases were also described above which need further investigation. In the cases where \emph{dèyè} was used, it was not clear whether it expresses ‘behind’ or ‘after’ and it could therefore not be decided whether it constitutes a Ground or a Path element. This shows that clear semantic criteria to identify the components of motion events are needed. Such criteria could also help to further investigate hybrid verbs like \emph{plonje} ‘dive’ which are said to express both Manner and Path. Two other cases were also problematic as they presented completely different structures from the ones described earlier. The structures where a gerund of a Manner verb or of the verb `to be' were used, neither of which has been described for Haitian Creole. As the participants of this study all lived outside of Haiti and were using other languages such as German or French on a daily basis, it is possible that these structures are due to language contact, most probably with French. More research needs to be done in this area.

The second aim of this study was to investigate the Manner salience of Haitian Creole, that is the frequency with which the Manner component is encoded in motion event expressions in comparison with other languages. In the data described above, Manner was expressed either in a Manner-Path verb serialization or in a satellite construction.\footnote{As Manner verbs only and Manner verbs with ground expressions are not considered motion events as defined by \citet{Talmy_1985}, they are not included here. This leaves us with 76 motions event expressions in the picture story narrations and 85 motion event expressions in the single picture descriptions.} In the picture story, both Manner-Path verb serializations and satellite constructions were very rare, which means that Manner was often left unexpressed. As the story was about a bird, whose Manner of motion typically is to fly, it is not necessary to encode Manner in every motion event, as it can easily be inferred. In the single picture descriptions, Manner-Path verb serializations are used in 24.7\% (21 of 85) and satellite-framed constructions in 5.9\% (5 of 85) of the motion event expressions. With a total of 30.6\%, the frequency of Manner encodings is much higher here than in the picture story narrations. Considering the fact that all of the pictures showed a specific Manner component, this number is nevertheless rather small. Both the picture story as well as the single picture descriptions therefore indicate that Haitian Creole has low Manner salience.

In comparison to the French motion verb expressions described in the first part of the paper, some similarities and some differences can be observed. Just like French, Haitian Creole possesses a rather large inventory of Path verbs, most of which probably go back to their French counterparts. This is why VF constructions are common in both languages. However, their percentage is larger in French than in Haitian Creole, as the latter possesses another structure not available in French: verb serialization, particularly the serialization of a Manner and a Path verb. This is a feature that Haitian Creole shares with various African languages, Kwa languages in particular, which are said to have played a significant role in the formation of Haitian Creole. Just as in the Kwa languages described above, the Manner verb precedes the Path verb in the Haitian Creole motion verb serializations. Another interesting observation is the fact that in the present data, no serializations with the Manner verb \emph{danse} ‘to dance’ are attested, a Manner-Path serialization which is ungrammatical in Fongbe according to \citet{LambertBrtire_2009}. The (un)grammaticality of such serializations in Haitian Creole needs to be tested in a subsequent study. The preliminary result of the ongoing research on this question is that the morphosyntactic patterns used in Haitian Creole to express motion events seem to be a mixture of the patterns found in the languages that were relevant to its formation.

In further research, more data will be elicited. As some of the drawings used for the single picture descriptions proved to be difficult to interpret, these representations of motion events need to be revised. If possible, videos showing motion events would be preferred for data elicitation. In addition, acceptability judgments will be elicited to investigate which Manner and Path elements are possible in which pattern, mostly in serial verb constructions and satellite-framed constructions. 


\section*{Acknowledgements}
I am grateful for the ongoing support of the supervisor of my thesis, Prof. Judith Meinschaefer. Further thanks go to the native speaker who helped me transcribe and translate the Haitian Creole data. I would also like to thank two anonymous reviewers who have provided very helpful comments on this paper. All remaining errors are my own.

{\printbibliography[heading=subbibliography,notkeyword=this]}

\end{document}
