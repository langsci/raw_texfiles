\documentclass[output=paper,colorlinks,citecolor=brown]{langscibook}
\ChapterDOI{10.5281/zenodo.14282814}
\author{Eliot Raynor\orcid{}\affiliation{Indiana University, Bloomington;Princeton University}}
%\ORCIDs{}

\title[A Gbe substrate model for discontinuous negation]
      {A Gbe substrate model for discontinuous negation in Spanish varieties of Chocó, Colombia: Linguistic and historical evidence}

\abstract{In this paper I highlight the central role of an oft-neglected source language model in the presence of discontinuous negation, also known as NEG2 (e.g. \emph{yo \textbf{no} sé cuál es \textbf{no}} ‘I don’t know which one it is’), in the Spanish of Chocó, Colombia. Namely, I point to the presence of analogous negation patterns in Ewe, a widely-spoken variety of Gbe across modern-day Ghana, Togo, and Benin. Data from the Trans-Atlantic Slave Trade Database as well as a 1759 census of the enslaved African(-descendant) population in Chocó mining camps strongly suggest that Gbe languages were spoken by one-third of enslaved Africans taken to the Pacific lowland region of Colombia via the Caribbean port of Cartagena in the 17th and 18th centuries. Traditionally, non-canonical patterns of negation in Ibero-Romance contact varieties involving discontinuous (i.e. pre-verbal and utterance-final) and strictly utterance-final negator morphemes have been attributed to Bantu languages such as Kikongo, some varieties of which feature similar structures. However, for the years in which enslaved Africans were being trafficked directly from West Africa to Chocó via Cartagena (1650--1800), archival sources indicate that those from Bantu-speaking regions of West Central Africa comprised close to one-sixth of the total African-born population, and thus just around half as many as the Gbe-speaking group. Together these findings lead to the conclusion that speakers of Gbe (especially Ewe) played a foundational role in the development of non-canonical negation patterns in Spanish varieties spoken in Chocó, Colombia.}


\IfFileExists{../localcommands.tex}{
   \addbibresource{../localbibliography.bib}
   \usepackage{langsci-optional}
\usepackage{langsci-gb4e}
\usepackage{langsci-lgr}

\usepackage{listings}
\lstset{basicstyle=\ttfamily,tabsize=2,breaklines=true}

%added by author
% \usepackage{tipa}
\usepackage{multirow}
\graphicspath{{figures/}}
\usepackage{langsci-branding}

   
\newcommand{\sent}{\enumsentence}
\newcommand{\sents}{\eenumsentence}
\let\citeasnoun\citet

\renewcommand{\lsCoverTitleFont}[1]{\sffamily\addfontfeatures{Scale=MatchUppercase}\fontsize{44pt}{16mm}\selectfont #1}
  
   %% hyphenation points for line breaks
%% Normally, automatic hyphenation in LaTeX is very good
%% If a word is mis-hyphenated, add it to this file
%%
%% add information to TeX file before \begin{document} with:
%% %% hyphenation points for line breaks
%% Normally, automatic hyphenation in LaTeX is very good
%% If a word is mis-hyphenated, add it to this file
%%
%% add information to TeX file before \begin{document} with:
%% %% hyphenation points for line breaks
%% Normally, automatic hyphenation in LaTeX is very good
%% If a word is mis-hyphenated, add it to this file
%%
%% add information to TeX file before \begin{document} with:
%% \include{localhyphenation}
\hyphenation{
affri-ca-te
affri-ca-tes
an-no-tated
com-ple-ments
com-po-si-tio-na-li-ty
non-com-po-si-tio-na-li-ty
Gon-zá-lez
out-side
Ri-chárd
se-man-tics
STREU-SLE
Tie-de-mann
}
\hyphenation{
affri-ca-te
affri-ca-tes
an-no-tated
com-ple-ments
com-po-si-tio-na-li-ty
non-com-po-si-tio-na-li-ty
Gon-zá-lez
out-side
Ri-chárd
se-man-tics
STREU-SLE
Tie-de-mann
}
\hyphenation{
affri-ca-te
affri-ca-tes
an-no-tated
com-ple-ments
com-po-si-tio-na-li-ty
non-com-po-si-tio-na-li-ty
Gon-zá-lez
out-side
Ri-chárd
se-man-tics
STREU-SLE
Tie-de-mann
}
   \boolfalse{bookcompile}
   \togglepaper[5]%%chapternumber
}{}


\begin{document}
\maketitle


\section{Introduction}

The specific origins of non-standard linguistic structures in varieties of European languages spoken by majority African-descendant populations in the Americas are the subject of persistent inquiry and much debate among scholars of Atlantic creoles as well as other Caribbean and mainland contact varieties of European languages. The present analysis centers on the presence of one such structure, referred to here as discontinuous negation (abbrev. NEG2), in varieties of Spanish spoken in the Pacific lowland region of northwestern Colombia, exemplified in (\ref{ex:5:1}--\ref{ex:5:4}).


\ea \citep{RuizGarcia_2001}\footnote{All examples from \citet{RuizGarcia_2001} were extracted directly from that author’s corpus of tape-recorded conversational speech collected in and around Tadó, Chocó.}\\\label{ex:5:1}
  \gll Yo \textbf{no} sé un nombre \textbf{no}\\
       \textsc{1sg} \textsc{neg} know a name \textsc{neg}\\
  \glt ‘I don’t know (of) a name.’ 
\ex \citep{RuizGarcia_2001}\\
\gll Yo \textbf{no} voy a subir allá \textbf{no}\\
     \textsc{1sg} \textsc{neg} go to go.up there \textsc{neg}  \\
\glt ‘I’m not going to go up there.’
\ex  \citep{RuizGarcia_2001}\\\label{ex:5:3}
\gll Esse niñito \textbf{no} me habla a mí \textbf{no}\\
     That child-\textsc{dim} \textsc{neg} me talks to me \textsc{neg}  \\
\glt ‘That kid doesn’t talk to me.’
\ex \citep[109]{Schwegler_1991a}\\\label{ex:5:4}
\gll    Mi papá sí \textbf{no} fue a la escuela \textbf{no}\\
        My father \textsc{emph} \textsc{neg} went to the school \textsc{neg}\\
\glt ‘My father definitely didn’t go to any school.’
\z

The prototypical NEG2 pattern consists of the pre-verbal negative morpheme \emph{no} plus a duplicate utterance-final \emph{no} with no perceivable prosodic break. The seamless transition from the end of the propositional content and the utterance-final negator morpheme can be seen in Figure~\ref{fig:fig1_05}.  

% Figure 1:
\begin{figure}
    \includegraphics[width=0.8\linewidth]{figures/fig1_05.png} % Adjust the scaling
    \caption{Pitch contour of NEG2 token from example \REF{ex:5:3}}
    \label{fig:fig1_05}
\end{figure}

\subsection{Discontinous negation in Ibero-Romance}

The use of NEG2 in varieties of Chocó Spanish is of sufficient salience that it has often been reported as a distinguishing feature of this variety, for instance in the work of Colombian dialectologists \citep{Florez_1950,Granda_1988,MontesGiraldo_1974}. Analogous patterns have been documented as well for other Ibero-Romance contact varieties, including Dominican Spanish \citep{Megenney_1990,OrtizLopez_2007,Schwegler_1991b} and Brazilian Portuguese \citep{Schwegler_1991c,Schwenter_2005}. Significant to the present analysis is the fact that discontinuous negation has been discussed extensively in the literature on creole languages, in particular Spanish-based Palenquero \citep{Dieck_2000,Schwegler_1991d,Schwegler_2016,Schwegler_2018} and Portuguese-based Santome \citep{Ferraz_1979,Guldemann_Hagemeijer_2019,Hagemeijer_2008,Hagemeijer_2009}, Angolar \citep{Maurer_1995}, and Fa d’Ambô \citep{Post_2013}.

The use of NEG2 in negated propositions is not categorical in Chocó Spanish lects, however, since it appears alongside and with less frequency than the canonical Spanish pre-verbal negation structure (NEG1), as in (5--7).\footnote{Not discussed here is a third option (NEG3) in which only the utterance-final negative morpheme is used, e.g. \emph{ellos llegan aquí no} ‘they don’t come here’ \citep[102]{RuizGarcia_2001}. While NEG3 is a robust negation strategy in other Ibero-Romance contact varieties such as Brazilian Portuguese, Palenquero, and Principense, this variant is especially infrequent in Chocó Spanish. \citet{Schwegler_1991a} found only one NEG3 example and \citet{RuizGarcia_2001} found just three  in a speech corpus of ~37 hours; thus while its existence is noted, its import is considered marginal.}

\ea\label{ex:5:5}\citep[96]{Schwegler_1991a}\\
\gll \textbf{No} me gustó y me vine\\
     \textsc{neg} me pleased and me came  \\
\glt ‘I   didn’t like it (there), and I came (here).’
\ex \citep[107]{Schwegler_1991a}\\
\gll Yo \textbf{no} me caí \\
     \textsc{1sg} \textsc{neg} me fell \\
\glt ‘I didn’t fall.’
\ex \citep[111]{Schwegler_1991a}\\
\gll \textbf{No} hubo cura para él  \\
     \textsc{neg} was priest for him  \\
\glt ‘There wasn’t a priest for him.’
\z

While a lengthy discussion of the degree of markedness of NEG2 lies outside the scope of the present discussion, informants from my own fieldwork in the capital city of Quibdó described the pattern as stigmatized and yet still \emph{muy común} ‘very common’ and \emph{inconsciente} ‘unconscious’. 

\subsection{Discourse-pragmatic considerations}

The factors governing the variation between NEG1 and NEG2 in Chocó Spanish are at present not well understood and have not been discussed in detail in previous work; this is largely due to the fact that no balanced sociolinguistic corpus of Chocó speech exists at present. Pragmatics-based analyses of variable negation strategies in Brazilian Portuguese \citep[e.g.][]{Schwenter_2005} suggest that NEG2 is licensed only in contexts where the negated proposition is mentioned or “activated” in the immediately preceding discourse. However, in an analysis of 66 tokens of NEG2 extracted from spontaneous conversational data from Chocó, \citeauthor{RuizGarcia_2001} found that only 54.6\% of cases involved negation of a proposition activated in the prior discourse (i.e. old information), while slightly less than half (45.4\%) negated new, “out of the blue” propositions (\citeyear{RuizGarcia_2001}: 109--123). A comparison of these cases is demonstrated in examples (8–9), which reveals the immediate discourse preceding the use of NEG2 from examples (2) and (3) above.

\ea \citep{RuizGarcia_2001}
\begin{xlist}[Speaker A:]
\exi{Speaker A:}{
\gll ¿Y ustedes ya lo fueron a ver a  él allá?\\
      and \textsc{2pl} already him went to see to he there\\
\trans ‘And you already went to see him there?’}
\exi{Speaker B:}{
\gll Yo \textbf{no} voy a subir allá \textbf{no.}\\
     \textsc{1sg} \textsc{neg} go to go.up there \textsc{neg} \\
\glt ‘I’m not going to go up there.’}
\exi{Speaker C:}{
\gll ¿Quién llegó abuela?\\
      who arrived grandmother\\
\glt ‘Who’s here, grandma?’}
\exi{Speaker D:}
{\gll Ese niñito \textbf{no} me habla a mí \textbf{no}\\
     that child-\textsc{dim} \textsc{neg} me talks to me \textsc{neg}\\
\glt ‘That kid doesn’t talk to me.’}
\end{xlist}
\z

In (8), B’s negated proposition ‘I’m not going to go up there’ is activated in A’s question as to whether B ‘already went to see him there’. However, the negation by D relates to the proposition ‘X doesn’t talk to me’, which is not directly activated by C’s question about who has just arrived. Thus, the use of NEG2 in Chocó Spanish does not appear to involve the same pragmatic conditioning that has been proposed for Brazilian Portuguese, but rather constitutes an alternative to NEG1 licensed across a wide range of discourse-pragmatic contexts.

\subsection{Goal and outline}

The objective of the present study is to shed light on the plausible origins of NEG2 in Chocó Spanish, departing from an analysis of the sociodemographic data available regarding the African-descendant population in the Pacific lowlands from the beginnings of significant gold-mining operations in the late 17\textsuperscript{th} c. through the boom and eventual collapse of the slave-labor-based economy in the region in the late 18\textsuperscript{th} century. Section~\ref{sec:5:2}, below, presents a sketch of the sociohistorical context in which Chocó Spanish emerged, which can be broadly characterized as one of vast demographic disproportion, wherein the European(-descendant) population never numbered greater than 2.5\% of the total population. Section 3 then outlines the patterns of negation in each plausible substrate language – that is, any language that the historical record suggests would have been the L1 of non-negligible numbers of speakers during the relevant time period (1650--1800). Finally, Section 4 weighs the contribution of the most likely substrate candidates in light of the historical evidence and the presence or absence of NEG2-like negation patterns in those languages, concluding that the large proportion of L1 Gbe speakers present in the early stages of Chocó’s transformation to a central mining hub of western Colombia provided an analogous and plausible source language model for NEG2. A numerically less prominent yet substantial group of L1 Kikongo speakers likely helped to reinforce the NEG2 pattern, such that it became entrenched in some of the Afro-Hispanic contact varieties that emerged in Chocó. 


\section{Social and historical considerations}\label{sec:5:2}

Chocó is one of the thirty-two \emph{departamentos} ‘provinces/states’ of Colombia, an administrative status it has held since 1944. In the most recent nationwide census, Chocó’s total population numbered 388,476, the vast majority of which (82.1\%) self-identified as \emph{negro o afrocolombiano} ‘black or Afro-Colombian’, while the second largest group (12.7\%) self-identified as \emph{amerindio} ‘Amerindian’; significantly, only 5.2\% of Chocó’s residents self-identified as \emph{blanco o mestizo} ‘white or mixed white/Amerindian ancestry’ \citep{DANE_2010}. These percentages contrast starkly with nationwide demographic figures, in which the \emph{blanco o mestizo} category comprises 86\% of the total Colombian population, dwarfing that of both \emph{negro o afrocolombiano} (10.6\%) and \emph{amerindio} (3.4\%). 


\subsection{Demographics of 18th c. Chocó}

An overview of the history of Chocó shows that similar demographic trends have existed since the earliest large-scale censuses of the region were prepared in the mid- to late-18\textsuperscript{th} century.

% Table 1:
\begin{table}
\centering
\begin{tabular}{cccccc}
    \lsptoprule
    {Year} & {“Whites”} & {“Slaves”} & \textit{{Libres}} & {“Indians”} & {Total} \\ \midrule
    1778 & 332 & 5,756 & 3,160 & 5,414 & 14,662 \\
    1779 & 335 & 5,916 & 3,348 & 5,693 & 15,292 \\
    1781 & 336 & 6,557 & 3,612 & 6,202 & 16,707 \\
    1782 & 359 & 7,088 & 3,899 & 6,552 & 17,898 \\
    1808 & 400 & 4,698 & 15,184 & 4,450 & 24,732 \\ 
    \lspbottomrule
\end{tabular}
\caption{18\textsuperscript{th} c. demographics of Chocó by ethnic group}
\label{tab:tab1_05}
\end{table}

\tabref{tab:tab1_05} above, adapted from \citet[199]{Sharp_1976}, shows that in terms of proportions of the total population, the European-descendants (“Whites”) category reached its maximum in 1778 at just 2.26\%, steadily dropping to 1.62\% by 1808. On the other hand, the percentages of enslaved Africans and African-descendants (“Slaves”) and Amerindians (“Indians”) are roughly equal throughout the same time period, the former reaching a maximum of 39.3\% in 1782 and the latter peaking at 37.2\% in 1779 before both decreased both in terms of raw numbers and percentages in 1808, which saw a massive increase in the number of \emph{libres de todos los colores} ‘freed people of all colors’. While the factors leading to this drastic rise in the number of \emph{libres} remain unclear, contemporary Colombian historian Claudia Leal has suggested some possible explanations:

\begin{quote}
    [T]he Afro-descendant population (slaves plus \emph{libres}) increased from 61.4 percent of the total to 80.6 percent, while Indians decreased from 36.6 percent to 17.8 percent. The extraordinary rise in the number of libres is hard to explain. It could be justified if slaves and Indians changed categories to libres with the same rates of their decrease in numbers. The purchase of freedom could explain how some individuals moved from slaves to libres, but no similar explanation exists for Indians becoming libres. \parencite[50]{Leal_2018}
\end{quote}

The decrease in the Amerindian population very well may reflect an increased tendency towards marronage, a process noted by \citet{WernerCantor_2000} as a common method of escape from semi-captivity employed by the indigenous population of Chocó. This is better understood in light of the fact that the category “Indians” as tabulated in 18\textsuperscript{th} and early 19\textsuperscript{th} century censuses almost certainly referred \emph{only} to tribute-paying Amerindians forced to live in \emph{corregimientos} and provide labor for mine-owners (e.g. riverine transportation, food production, and housing construction), while a proportion of the total Amerindian population maintained their traditional ways of living outside the bounds of Spanish control.

The sociodemographic data from Table~\ref{tab:tab1_05} point to two groups from which any substrate features of Chocó Spanish must be derived: African(-descendant)s and Amerindians. Tracing the linguistic background of the Amerindian population in Chocó is straightforward, as only Emberá- and Wounaan-speaking communities have inhabited this region continuously\footnote{The qualifier “continuously” is important here, since Cuna-speaking communities once resided in Chocó as well. Historical accounts by \citet{WernerCantor_2000,Williams_2005}, and \citet{VargasSarmiento_1993} point out that tensions between the Emberá and Cuna populations of Chocó, exacerbated by early contacts with Spanish and English colonists, caused the  Cuna to permanently relocate north to Darién, Panama, along the border with Colombia, as well as to the San Blas islands off the Panamanian coast.} from the first contacts with Spanish colonists in the early 16\textsuperscript{th} c. through the present day \citep{JimenezMeneses_2004,Williams_2005}. While serious discussion of the Amerindian influences on Chocó varieties of Spanish has unfortunately been omitted in previous accounts, sociohistorical factors including the size of the population, coupled with relative linguistic homogeneity within the Chocoan language family \citep{Loewen_1957,Loewen_1960,Loewen_1963}, lend weight to the importance of considering these languages when discussing issues related to language contact in Chocó.

On the other hand, it must be acknowledged that the vast majority of the current population of Chocó is of African descent, as noted above. Prior analyses and the author’s own fieldwork suggest that this population is currently monolingual in what has been described as a “partially restructured” variety of Spanish \citep{RuizGarcia_2001} reflective of “advanced second language acquisition” at an earlier stage in the development of the variety \citep{Sessarego_2016,Sessarego_2019}. Indeed, even as early as 1816, two entries from the diary of a Spanish military officer on an expedition to root out revolutionaries in Chocó suggest that the African-descendant population of Chocó spoke target-like Spanish: 

\begin{quote}
    May 19: [W]e came across two blacks, and they said the enemies had spent the afternoon prior in the hills above... May 20: [F]rom some blacks that we found along the way we know that the enemies have fled in great haste.\footnote{Transcription and translation here are mine of an original archival document located in the “Latin American mss.— Colombia, 1558--1890” collection at the Lilly Library in Bloomington, Indiana.}
\end{quote}

In order to highlight possible substrate sources of NEG2, a variety of sources are considered below  to discern the plausible linguistic backgrounds of the largest groups within the African-born and African-descendant population that was trafficked to Chocó between the late 17\textsuperscript{th} c. and 18\textsuperscript{th} century.

\subsection{Origins of enslaved Africans arriving in Cartagena, 1650--1800}

Cartagena was the primary port of entry of enslaved Africans into colonial New Granada (modern-day Colombia) and thus is the relevant place to look for archival information concerning their places of origin. Regarding Chocó in particular, William F. Sharp states that “most of the miners in the Chocó during the last decades of the seventeenth century were \emph{bozales} [African-born] and most came from Cartagena by way of the Atrato” (\cite{Sharp_1976}: 111).\footnote{Sessarego argues that “it would be a mistake to assume that blacks in Chocó were all \emph{bozales} who spoke African languages” (2017: 40) and suggests instead that \emph{criollo} [American-born] slaves were more prominent in Chocó. In doing so, he highlights the sales of \emph{criollos} in southeastern city of Popayán, pointing out that most mine (and slave) owners were from there. While these are relevant considerations, it must be noted that a total of only 1,074 \emph{criollos} were sold in Popayán over the course of the entire century (1690--1789) in which mining peaked in Chocó (\cite{Colmenares_1997} [1979]: 36). For comparison, if we set aside local birth and death rates, these \emph{criollos} would have constituted just 12\% of the entire African(-descendant) population of Chocó in 1778.} Sharp also cites archival accounts from as late as 1777 demonstrating mine owners’ preference for “newly imported blacks who were best suited for work in the mines” (117). 

% Table 2:
\begin{table}[!ht]
\centering
\setlength{\tabcolsep}{3.5pt}
\resizebox{\linewidth}{!}{
\begin{tabular}{lrrrrrrrr}
    \lsptoprule
    \multicolumn{1}{c}{{Time}} & {Upper} & {Sierra} & {Gold} & {Bight of} & {Bight of} & {Central} & {Cape} & {Príncipe} \\
    \multicolumn{1}{c}{{Period}} & {Guinea} & {Leone} & {Coast} & {Benin} & {Biafra} & {Africa} & {Verde} & {Island} \\ \midrule
    1651--1675 & 556 &  &  & 966 & 1,088 & 650 &  &  \\
    1676--1700 & 820 &  & 131 & 1,138 &  & 1,078 & 313 & 1,005 \\
    1701--1725 & 245 & 403 & 1,801 & 1,955 & 344 & 720 &  &  \\
    1776--1800 &  &  &  &  & 158 &  &  &  \\ \midrule
    \multicolumn{1}{r}{{Total}} & 1,631 & 403 & 1,932 & 4,059 & 1,590 & 2,448 & 313 & 1,005 \\
    \multicolumn{1}{r}{{\%}} & 12.1\% & 3.0\% & 14.5\% & 30.4\% & 11.9\% & 18.3\% & 2.3\% & 7.5\% \\ \lspbottomrule
\end{tabular}
}
\caption{Broad regions of origin of Africans disembarking in Cartagena (late 17\textsuperscript{th}-18\textsuperscript{th} c.)}
\label{tab:tab2_05}
\end{table}


\tabref{tab:tab2_05} summarizes data from the Trans-Atlantic Slave Trade database showing the provenance of 13,371 enslaved Africans sold in Cartagena during the time period spanning 1650--1800 \citep{VoyagesDatabase_2009}. The range of locations from which Africans were purchased was vast, stretching from the French-held posts of Cacheu and Saint-Louis in Upper Guinea in the north all the way south to West Central Africa (present-day Angola, Congo) and including all major regions of the West African slave trade in between. Notably these include the islands of Cape Verde and Príncipe, which served as entrepôts for the shipment of slaves to the Americas, and where Portuguese-based creole languages had already emerged prior to this time period.\footnote{While slaves arriving from each of these islands comprised relatively smaller groups (especially Cape Verde), their presence is noteworthy, given the claims of \citet{McWhorter_1999,McWhorter_2006} that small numbers of slaves speaking African-origin pidgins or creoles were responsible for catalyzing the creation of creoles across the Atlantic.} On the other hand, the largest percentage (30.4\%) arrived from the Bight of Benin, also often referred to as the Slave Coast, a region which \citet[110--112]{Eltis_Richardson_2010} illustrate would have stretched from contemporary eastern Ghana to western Nigeria and thus would have covered areas occupied by speakers of Gbe languages, especially Ewe and Fon – for a dialect map of contemporary varieties of Gbe, see \citet[xxiii--xxiv]{Capo_1991}. From this broad region the database shows that more than two-thirds (2,899) of enslaved Africans purchased in the Bight of Benin and arriving in Cartagena were purchased from a single location, Whydah,\footnote{“Whydah” corresponds to the modern-day city Ouidah in Benin (for a social history of Ouidah, see \citealt{Law_2004}).} between the years 1676--1725. Those arriving in the earlier period between 1651--1675 primarily came from “Ardra”, also known as “Allada”.\footnote{\citet[Ch.2]{Aboh_2015} provides a helpful overview of the history and significance of the kingdom of Allada, the center of which was a city of the same name situated 50 kilometers inland in modern-day Benin. This is the source of the ethnic denomination Arará, which was applied to enslaved Africans arriving from this location (as seen below).} This half-century span of time corresponds fairly closely with the earliest period in which it is known that large numbers of African-born slaves were purchased in Cartagena and transported to Chocó for work in the nascent gold mining industry.

Considerably less specific are the details provided by the Trans-Atlantic Slave Trade database concerning the places of purchase of the second-largest group (18.3\%) of Africans disembarking in Cartagena, labelled simply “West Central Africa and St. Helena, port unspecified”. Thus, we can state with some degree of certainty only that this group would have been broadly comprised by speakers of Bantu languages, among them certainly Kikongo, which was central to the formation of Palenquero in the maroon community of San Basilio de Palenque just inland from Cartagena (\cite{Schwegler_2002,Schwegler_2011}). Similarly vague is information about Gold Coast arrivals. These comprised the third-largest group (14.5\%) of slaves arriving in Cartagena, for which a specific place of purchase, Cape Coast Castle, is known only for 426 of the 1,801 enslaved Africans arriving between 1701--1725, all others falling under the headings “Gold Coast, port unspecified” and “Gold Coast, Fr. definition”. For the purposes of this analysis, then, we will make the general (and rather commonplace) assumption that many among this population would have been Akan/Twi-speaking.


\subsection{Origins of enslaved Africans in Chocó, 1759}

Some of the uncertainty inherent in the assumptions made above is mitigated by the existence of a census dated 1759 providing information about the enslaved peoples working in fifty-eight mining camps across the region of Chocó at the time.\footnote{Accessed in the online database of the Archivo General de la Nación de Colombia (Sección Colonia, Negros y Esclavos, Cauca, Legajo 4) \url{http://www.archivogeneral.gov.co}. It should be noted that \citet{Granda_1988} analyzed the same document and reached similar conclusions. However, the numbers reported here do not match precisely those summarized in Granda’s article, and do not invoke a priori assumptions, such as general groupings of diverse ethnonyms under a single linguistic family label (e.g. Kwa).} Particularly illuminating from this document are lists of the names of enslaved Africans and African-descendants working in each camp. What follows below is a summary of the data gleaned from a detailed tabulation of the names found in this original handwritten document. 

Out of 2,741 named\footnote{The total population of Chocó in 1759 was 3,918. The difference between the overall total and the 2,741 accounted for here are individuals described collectively as \emph{muleques} ‘children’, \emph{viejos} ‘elderly’, and/or \emph{chusma} ‘unfit for work’.} individuals, 1,535 consist of just a first name, while 1,206 provide a second element consisting of either a surname, an African ethnic denomination, or some type of descriptor. Each of the descriptive second elements of the individuals’ names was labelled with a code designating subtypes of naming conventions. The range of subtypes included African ethnic denominations (e.g. \emph{Bambará, Lucumí}), the generic terms \emph{Criollo} and \emph{Bozal}, surnames of Spanish origin (e.g. \emph{García, Figueroa}), racial descriptors (e.g. \emph{Mulato, Negro}), other descriptors (e.g. \emph{Cabezón} ‘big-headed’, \emph{Bailador} ‘dancer’), toponyms/demonyms (e.g. \emph{Cartagena, Panameño}), marital status (e.g. \emph{Casado} ‘married’, \emph{Soltero} ‘single’), and others which were not immediately discernible (e.g. \emph{Cuco, Pino}). \tabref{tab:tab3_05} gives an overview of the subtypes used in the second elements of names found in the census. Most evident is the largest group (43.9\%) described in terms describing a range of African ethnicities, the most frequent of which are listed in Table~\ref{tab:tab4_05}.

% Table 3:
\begin{table}[b]
\centering
\resizebox{\linewidth}{!}{%
\begin{tabular}{cccccc}
    \lsptoprule
    {African ethnic denomination} & {Criollo}\footnote{It is noteworthy that such a large proportion of individuals are described as \emph{Criollos}, given the evidence discussed in Section 2.2 that this population would have been considerably smaller than the African-born population in Chocó. An important consideration here is that the percentages reported in \tabref{tab:tab3_05} were calculated in reference to the total of just those individuals whose names are accompanied by a second element in the 1759 census. Thus, if we consider all 2,741 named individuals, the names followed by \emph{Criollo/Criolla} comprise a considerably smaller percentage (14.8\%) of the total. Further, since the term \emph{criollo} was typically applied to African-descendant slaves who were born in the colony and spoke Spanish, we may fairly safely assume that few of the remaining 1,535 individuals who were not described explicitly as Criollo would have been in fact \emph{criollos}, since this aspect of their identity would have been apparent to the census preparers upon demonstration of their linguistic abilities.} & {Spanish surname} & {Race} & {Demonym} & {Other}\footnote{Subsumed in the ‘Other’ category are those elements which do not provide any hints as to the provenance of the  person. Besides the elements which were undiscernible as mentioned above (totaling 39 individuals), these also included the term \emph{Bozal} (3 individuals), marital status (8 individuals), and other descriptors (22 individuals).} \\ \midrule
    529 & 405 & 123 & 67 & 10 & 72 \\
    43.9\% & 33.6\% & 10.2\% & 6.0\% & 0.8\% & 5.5\% \\ \lspbottomrule
\end{tabular}
}
\caption{Subtypes of descriptions following names in 1759 census of Chocó}
\label{tab:tab3_05}
\end{table}

% Table 4:
\begin{table}[!ht]
\centering
\setlength{\tabcolsep}{5pt}
\resizebox{\linewidth}{!}{%
\begin{tabular}{cccccccccc}
    \lsptoprule
    {Mina} & {Congo} & {Arará} & {Carabalí} & {Chambá} & {Chalá} & {Setre} & {Mandinga} & {Popó} & {Other}\footnote{‘Other’ subsumes forty-seven distinct ethnic denominations, most appearing only once or twice (e.g. \emph{Fori, Bran}), while a few others, such as \emph{Nangó} (13 individuals) and \emph{Tembo} (11 individuals), appear more frequently.} \\ \midrule
    137 & 85 & 47 & 46 & 28 & 26 & 25 & 21 & 18 & 96 \\
    25.9\% & 16.1\% & 8.9\% & 8.7\% & 5.3\% & 4.9\% & 4.7\% & 4.0\% & 3.4\% & 18.1\% \\ \lspbottomrule
\end{tabular}
}
\caption{African ethnic denominations in 1759 census of Chocó}
\label{tab:tab4_05}
\end{table}

The predominance of the names \emph{Mina} (137 individuals) and \emph{Congo} (85 individuals) suggest that the linguistic backgrounds associated with these ethnic denominations would have been likely candidates as West African substrate languages in 18\textsuperscript{th} c. Chocó. In the past, historians have tended to associate \emph{Mina} with the Gold Coast port of Elmina in modern-day Ghana. However, Hall points out that

\begin{quote}
    [i]n the Portuguese, Spanish, and French colonies, the ‘Mina’ were the \emph{casta} Mina, and they are to be identified with the Gbe language speakers as described by Alonso de Sandoval in both the 1627 and 1647 versions of his book. .... He distinguished the \emph{casta} Mina from the Popos, Fulaos, Ardas or Araraes, although he considered that all were related and they were all one (\emph{que todo es uno}). .... Sandoval makes it clear that the Mina casta is to be identified with Gbe-speaking Africans of the Bight of Benin, and specifically the Ewe, Aja, Fon and others of the Gbe language group, and not immigrants from the Gold Coast. \citep[70--71]{Hall_2003}
\end{quote}

Law has called into question the definitive nature of Hall’s claim about the meaning of \emph{Mina}, pointing out that in many cases 

\begin{quote}
    the aggregation of peoples who were linguistically distinct but geographically adjacent (in Africa) are best explicable on the assumption that many people in these groups were bilingual, so that smaller groups could be assimilated into larger ones in the Americas. \citep[267]{Law2005}
\end{quote}

In light of this, it is relevant to note that \citet[84]{Mosquera_2008} provides archival testimony from a mining settlement in Tadó, Chocó, in 1728 displaying the multilingual abilities of an individual named Antonio Mina, who it appears was able to serve as an interpreter for other enslaved Africans of various ethnolinguistic backgrounds, as indicated by their names: José Nongo, Marcos Chalá, and Francisco Arará. 

In any case, the high frequency of appearance of \emph{Minas} in the 1759 census, coupled with the presence of significant numbers of both \emph{Arará} (47 individuals) and \emph{Popó} (18 individuals), suggests that the Gbe-speaking and/or Gbe-descendant\footnote{This is an important consideration, since second-generation African-descendants were likely among those who still identified as \emph{Mina}. Indeed, the surname \emph{Mina} (as well as others, e.g. \emph{Carabalí}) is still found in African-descendant communities from the Pacific lowlands of Colombia and Ecuador.} population recorded in the document reaches a total of 202 individuals, or 38.2\% of those described in terms of African ethnic denominations.  

In the case of \emph{Congo}, which appears in 18.1\% of the names from the 1759 census, there is little room for debate as to the provenance of these individuals, particularly given the close correspondence with the data in \tabref{tab:tab2_05} showing that 18.3\% of the slaves arriving in Cartagena came from West Central Africa. I will again assume (as in Section 2.2) that members of this group were speakers of Bantu languages, especially Kikongo. As for other ethnic denominations appearing in the 1759 census, in the interest of clarity and a degree of certainty, here no specific claims are made as to their linguistic backgrounds.\footnote{For instance, \emph{Carabalí} appears frequently but can be traced to a range of languages in modern-day Nigeria.}


\section{Negation strategies in candidate substrate languages}

In this section, I present the morphosyntactic structures used to express negation in each of seven likely candidate substrate languages highlighted above in Section~\ref{sec:5:2}: Emberá, Wounaan, Ewe, Gen, Fon, Kikongo, Akan/Twi, and Principense. These negation strategies are illustrated in examples (9–20) below. 

\subsection{Amerindian languages}

I begin with two indigenous languages of the Chocoan language family, given that these (especially Emberá) were the most widely spoken for the longest time period in Chocó, i.e. from initial contacts with Spanish-speaking colonists in the early 16th c. through to the present. As illustrated in examples (9–12), three varieties of Emberá as well as Wounaan express negation in the form of suffix attached to the verb stem, which can optionally be followed by additional suffixes expressing tense, mood, or aspect. 

\ea Emberá Pedee \citep[131]{Harms_1994}\\
\gll  mi warra kʰãi-\textbf{ʔe}  \\
      my son sleep-\textsc{neg}  \\
\glt ‘My son is not sleeping.’
\ex Emberá Katío \citep[98]{Mortenson_1999}\\
\gll ũnũ-\textbf{ẽ}-pa-sʰi-a  \\
     see-\textsc{neg}-\textsc{eq}-\textsc{past}-\textsc{decl}\\
\glt ‘He didn’t see her.’
\ex Emberá Chamí \citep[112]{AguirreLicth_1999}\\
\gll íɖi kúi-\textbf{wẽ}-a\\
     today bathe-\textsc{neg}-\textsc{decl}\\
\glt ‘Today I haven’t taken/won’t take a bath.’ 
\ex Wounaan \citep[129]{Holmer_1963}\\
\gll mə yeka-\textbf{ba}-m  \\
     \textsc{1sg} talk-\textsc{neg}-\textsc{fut}  \\
\glt ‘I’m not going to talk.’
\z

Examples (9–12) demonstrate that each of the Chocoan languages spoken by indigenous peoples in contact with Spanish in Chocó has a similar typological profile with respect to the expression negation, that is, morphologically by way of verbal suffixation. As such, neither Emberá nor Wounaan make particularly good candidates for the introduction of NEG2 into Chocó Spanish, since no pre-verbal element is present in any variety. 

\subsection{West African languages}

Next I turn to Ewe and Fon, the two varieties of Gbe most likely to have been present in Chocó during the early stages of transformation of this region into a major center of the gold mining industry in the 18th century. Examples (13–14), drawn from 19\textsuperscript{th} c. and 20\textsuperscript{th} c. grammars of Ewe, demonstrate that negation in this variety involves the placement of a free morpheme \emph{me} before the verb – as well as before any TMA markers, as seen in (13) – plus the morpheme \emph{o} at the end of the sentence, following any post-verbal complements or adjuncts. Crucially, both pre-verbal \emph{me} and sentence-final \emph{o} are obligatory in the expression of negation in these varieties.

\ea Ewe \citep[234]{Ellis_1890}\\   
\gll Nye \textbf{me} a du \textbf{o}\\
     \textsc{1sg} \textsc{neg} \textsc{fut} eat \textsc{neg}\\
\glt ‘I will not eat.’
\ex Ewe \citep[64]{Ameka_1991}\\
\gll Kofi \textbf{me} va afi sia \textbf{o}\\
     Kofi \textsc{neg} come place this \textsc{neg}\\
\glt ‘Kofi did not come here.’
\z

Similarly, Gen negation requires the expression of both pre-verbal \emph{mú} and utterance-final \emph{ò} negator morphemes \citep[46--47]{Aboh_2004}.

On the other hand, the Fon variety of Gbe can express negation pre-verbally with the morpheme \emph{ma} or post-verbally with the sentence-final morpheme \emph{a}, but according to \citet[45]{Aboh_2004}, the pre-verbal negator \emph{ma} and post-verbal \emph{a} cannot co-occur – unless they are in a conditional clause introduced by \emph{ni} ‘if’, in which case they must both be present. See (15–16) for examples.

\ea Fon \citep[44]{Aboh_2004}\\
\gll Kɔku \textbf{ma} na xɔ asɔn le\\
     Koku \textsc{neg} \textsc{fut} buy crab \textsc{num}\\
\glt ‘Koku will not buy the crabs.’
\ex Fon \citep[45]{Aboh_2004}\\
\gll Kɔku na xo asɔn le \textbf{a}\\
     Koku \textsc{fut} buy crab \textsc{num} \textsc{neg}\\
\glt ‘Koku will not buy the crab.’
\z

Generally, then, while Ewe, Gen, and Fon have some similarities in terms of negation through the use of free morphemes appearing pre-verbally and/or sentence\babelhyphen{hard}finally, only Ewe and Gen appear to have a pattern of negation that could have provided a substrate source for NEG2 in Chocó Spanish.

Continuing within the Niger-Congo language family, examples (17–18) below are from Kikongo. Given the historical data presented above and prior proposals for the origin of NEG2 in Chocó (e.g. \cite{Schwegler_1991a,Schwegler_2018}), this language is considered particularly relevant as a possible substrate. On the other hand, Bantu languages in general, and Kikongo in particular, present the most complex range of possibilities of any of the languages considered here in terms of negation strategies. \emph{The World Atlas of Linguistic Structures} describes Kongo  as the only language\footnote{Here and below names for this language are presented in the sources that have been cited. In all cases it is clear that varieties of the same language (Kikongo) are being described.} “with an obligatory clause-final negative word, a choice between an immediately preverbal negative word or a negative prefix, and an optional suffix” \citep{Dryer_2013}.  

Examples (17–18) present cases in which Kikongo negation partially aligns with that of the NEG2 pattern found in Chocó Spanish, but in both cases the pre-verbal negator is attached as a prefix to the verb and followed by a series of morphemes for person/number and tense/aspect also prefixed to the verb. Furthermore, in (17) the post-verbal negator appears before an adverb and prepositional phrase, whereas in Chocó Spanish the post-verbal negative morpheme in NEG2 follows those same types of constituents (see examples 2 and 4, above). Thus, while NEG2-like constructions are possible, the number of variable strategies of negation in Kikongo and the complexities of its morphosyntactic realization do not make it an ideal candidate as a substrate for the NEG2 structure as used in the Spanish of Chocó.

\ea Kisikongo\\
\gll \textbf{ki}-a-mon-idi o npangi aku \textbf{ko} mazono ku zandu\\
      \textsc{neg}.\textsc{1sg}-\textsc{past}-see-\textsc{prf} \textsc{aug} brother your \textsc{neg} yesterday at market \\
\glt ‘I didn’t see your brother yesterday at the market.’
\ex Kongo \citep[200]{Lumwamu_1973}\\
\gll \textbf{ka-}t-a-dí-di-éti \textbf{ko}\\
     \textsc{neg-1pl-compl}-eat-\textsc{past-rel.compl} \textsc{neg}\\
\glt ‘We have not yet eaten.’ 
\z

While not as likely a candidate in light of the sociohistorical profile of Chocó, Akan/Twi should be considered as well, since languages of the Kwa branch of Niger-Congo are often lumped together and considered typologically similar in language contact work (see, e.g., \cite{Granda_1988}). Akan/Twi negation, as in (19) below, involves the use of a verbal prefix – in this case preceded by an aspectual marker – with no post-verbal marking whatsoever. 

\ea Akan/Twi \citep[104]{Amfo_2010}\\
\gll Papa no a-\textbf{n}-kɔ adwuma nnɛra\\
     man \textsc{def} \textsc{compl-neg}-go work yesterday\\
\glt ‘The man didn’t go to work yesterday.’
\z


Similar to the Chocoan languages, then, Akan/Twi is an unlikely source model for NEG2 in the Spanish of Chocó. 

Finally, to conclude the discussion of possible substrate models for NEG2, example (20) below comes from the Portuguese-based creole Principense. While there is no definitive evidence to suggest that speakers of any Portuguese-based creole were trafficked to Chocó, it is possible that speakers of Principense may have been among the 1,005 enslaved Africans that arrived in Cartagena via Príncipe between 1676--1700, some of whom may have indeed been taken to the Pacific lowland region. 

\ea Principense \citep[58]{Maurer_2009}\\\label{ex:5:21}
\gll Te ninge nhon di pasa lala \textbf{fa}  \\
     have person no of pass there \textsc{neg}  \\
\glt ‘There is nobody who passes by over there.’
\z

However, as the sentence in (20) shows, Principense uses only a sentence-final negator \emph{fa}\footnote{This is in fact unique to Principense as compared to the closely-related Portuguese-based creoles Santome, Angolar and Fa d’Ambô. According to \citet{Maurer_2013}, “[t]he fact that … Principense has no double negation like Santome \emph{na} ... \emph{fa} is probably due to the existence of the validator \emph{na} (epistemic modality) in Principense, which has the same shape and the same position of the negator \emph{na}, namely immediately preceding the tense, aspect, and mood markers”.}. The morpheme \emph{nhon} glossed as ‘no’ above has scope only over the word \emph{ninge} ‘person’, combining to mean ‘nobody’ as seen in the English translation. Thus, Principense is also not a strong candidate as a source for the appearance of NEG2 in Chocó.

\subsection{Summary}

\tabref{tab:tab5_05} summarizes the types of negation strategies found in the seven candidate substrate languages considered most relevant to the context in which Chocó Spanish emerged. The NEG2 pattern presented by Chocó Spanish involves the use of pre-verbal \emph{and} sentence-\slash utterance\babelhyphen{hard}final negator morphemes; therefore, the most likely candidate in purely linguistic terms is Ewe, which presents an analogous pattern. Kikongo may also be of relevance given the \emph{possibility} of analogous structure, with the caveats discussed above. 

% Table 5:
\begin{table}\small
\begin{tabularx}{\textwidth}{llQ}
    \lsptoprule
    Language & Subgroup (Family) & Negation type\\ \midrule
    Emberá   & Chocoan (Amerindian) & Verbal suffix \\
    Wounaan  & Chocoan (Amerindian) & Verbal suffix \\
    Ewe      & Gbe (Niger-Congo) & Pre-verbal and sentence-final morpheme \\
    Gen      & Gbe (Niger-Congo) & Pre-verbal and sentence-final morpheme \\
    Fon      & Gbe (Niger-Congo) & Pre-verbal or sentence-final morpheme \\
    Kikongo  & Bantu (Niger-Congo) & Pre-verbal morpheme or verbal prefix and/or suffix \textit{and clause-final morpheme}\\
    Akan/Twi & Tano (Niger-Congo) & Verbal prefix \\
    Principense & Portuguese-based creole & Sentence-final morpheme \\ \lspbottomrule
\end{tabularx}
\caption{Types of negation among candidate substrate languages}
\label{tab:tab5_05}
\end{table}


\section{Conclusion}

The present analysis is narrow in scope both in terms of the linguistic structure under analysis (discontinuous negation or NEG2) as well as the geographical region of its use (Chocó, Colombia). However, the account offered here is perhaps somewhat more far-reaching insofar as NEG2 has been the subject of some debate regarding its origins in Chocó as either a contact feature derived from a West African source \parencites[514]{Granda_1978}{Granda_1988}{RuizGarcia_2001}{Schwegler_1991a} or one that developed across Ibero-Romance in the 15\textsuperscript{th} c. via Jespersen’s cycle and thus reflecting Spanish varieties spoken by colonists \citep{Sessarego2017}. The latter argument has been challenged and largely discounted by \citet{Schwegler_2018} in part because the pattern of NEG2 in Chocó as seen in examples (1–3) above involves a pre-verbal and a sentence-final negator after objects, adverbs, and other intervening morphemes, while Jespersen’s cycle\footnote{This is the version of the cycle as originally conceived by \citet{Jespersen_1917}. Recently, \citet{vanderAuwera_2009} has offered an expanded version of the cycle that includes the development of patterns with sentence- or utterance-final negator morphemes – notably, this has been invoked primarily to account for these sorts of patterns in West African languages, not European ones.} involves immediately post-verbal negators, as do all of the examples in \citeauthor{Sessarego2017}’s (\citeyear{Sessarego2017}) analysis of 15\textsuperscript{th}--19\textsuperscript{th}~c. Spanish from the \citet{Davies_2002} \emph{Corpus del Español}. There is no existing evidence of NEG2 occurring in colonial varieties of Spanish, nor has any attempt been made to provide data showing its use in western Colombia during the relevant time period. On the other hand, there is an abundance of evidence that NEG2 appears frequently in Ibero-Romance varieties in intense contact with West African languages, including Afro-Hispanic varieties of Chocó, Colombia and the Dominican Republic, as well as Brazilian Portuguese and the creole languages Palenquero, Santome, Angolar, and Fa d’Ambô. The origin and persistence of NEG2 in each of these varieties requires an analysis based on local sociohistorical and linguistic evidence.

In the case of Chocó it has been particularly important to adopt a local perspective, as opposed to a macroscopic or comparative one. The structure of the present paper thus proceeds from a review of original archival documents that shed light on the languages most likely to have been in the right place at the right time to have exerted an influence on Spanish in Chocó – in accordance with Bickerton’s Edict – before considering them as possible substrates. The historical and linguistic evidence I present here indicates that speakers of Gbe varieties, in particular Ewe and Gen speakers enslaved in the Bight of Benin and sold in Cartagena in the late 17\textsuperscript{th} c. and early 18\textsuperscript{th}~c., were in all likelihood the key agents that introduced NEG2 structures found in contemporary varieties of Spanish spoken in Chocó. My account allows room for the contribution of speakers of Bantu languages as well, particularly Kikongo, who were enslaved in West Central Africa and trafficked through Cartagena in the relevant time period, but whose numerical presence was substantially less than that of the Gbe-speaking group and whose language does not present negation patterns strictly analogous to NEG2. Speakers of Chocoan languages (especially Emberá) were the only other group present in Chocó in the 17\textsuperscript{th} and 18\textsuperscript{th} c. with large enough populations to have significantly influenced the development of Chocó Spanish. It is apparent that NEG2, however, did not originate in any of these varieties, all of which express negation by way of verbal suffixation.  


\section*{Acknowledgements}
In addition to two anonymous reviewers and the editors of this volume, several people deserve specific recognition for generously offering their time and insight, from which this analysis has benefited enormously. First and foremost, I have to thank two of my mentors at Indiana University, J. Clancy Clements and Kevin J. Rottet, for their engagement with and encouragement of this line of inquiry since its humble beginnings over three years ago at the time of writing. Secondly, I owe a great deal of credit for the present analysis to Martha Ruíz-García, whose dissertation planted the seeds for this project, and whose invaluable corpus I was given the opportunity to digitize in 2018. I am also indebted to a number of participants who attended my talks at the 2017 and 2019 summer meetings of the Society for Pidgin and Creole Linguistics and Associação de Crioulos de Base Lexical Portuguesa e Espanhola, where I had the opportunity to discuss this and other matters related to the Spanish varieties of Chocó, Colombia. Among many who have been kind enough to correspond and explore\slash debate these questions with me, I’d like to thank in particular Miguel Gutiérrez Maté, Danae Perez, Luana Lamberti, Johan van der Auwera, Armin Schwegler, and Sandro Sessarego. Of course, any inaccuracies herein remain my own.

\printbibliography[heading=subbibliography,notkeyword=this]
\end{document}
