\documentclass[output=paper,colorlinks,citecolor=brown]{langscibook}
\ChapterDOI{10.5281/zenodo.14282816}
\author{Fernanda M. Ziober\orcid{}\affiliation{Vrije Universiteit Amsterdam}}
%\ORCIDs{}

\title[The minutes of the \emph{Irmandade Nossa Senhora do Rosário dos Homens Pretos}]
      {The minutes of the \emph{Irmandade Nossa Senhora do Rosário dos Homens Pretos} in Recife: A case study}

\abstract{This article is a case study of the written texts by Manoel de Barros, a member of the \emph{Irmandade Nossa Senhora do Rosário dos Homens Pretos} ‘Brotherhood of Our Lady of the Rosary of Black Men’ in the 19\textsuperscript{th} century, in the city of Recife, Brazil. The minutes and terms of the brotherhood constitute important sociolinguistic documentation of literacy of  people with limited access to texts and formal education. The texts do not follow the writing standards of 19\textsuperscript{th} century Portuguese. The variations from the standard norm have three main sources: (i) orthographic, (ii) phonetic and (iii) morphosyntactic. Characteristics such as apheresis and epenthesis are compared to those occurring in similar texts written by the African and Afro-Brazilian people of Bahia \citep{Oliveira_2006}, on Bahian Afro-Brazilian Portuguese \citep{Lucchesi_Baxter_Ribeiro_2009} as well as Portuguese in general.}

\begin{document}
\maketitle


\section{Introduction}

This article is a case study of the linguistic characteristics of the historical writings of Judge-Brother Manoel de Barros of the \emph{Irmandade de Nossa Senhora do Rosário dos Homens Pretos} in the 19\textsuperscript{th} century.\footnote{“Brotherhood of Our Lady of the Rosary of Black Men”, a Catholic brotherhood of African men and African descendants.} The \emph{irmandades}, ‘brotherhoods’, were civil organizations that allowed their members to secure a space for Catholic religious worship, a burial place, and other amenities. In view of the fact that these books have been contaminated by bacteria, I want to give visibility to such documentation as a source of historical sociolinguistic data that deserves documentation and digitization. In Brazil, after the publication of Law 10.639/03 which stipulates the mandatory teaching of the history and culture of Afro\babelhyphen{hard}Brazilians throughout the entire educational system, there has been a growing interest in this historical topic and the subsidizing of new pedagogical materials. Related to this issue, there has also been interest in the formation and stabilization of the national variety of Portuguese referred to as Brazilian Portuguese.

According to \citet{Mattos_e_Silva_2004} and \citet{Noll_2008}, the slave trade and the dispersion of enslaved individuals across the Brazilian territory were fundamental factors for the implementation and diffusion of the Portuguese language in Brazil. Up until then, competence in Standard Portuguese was restricted to Portuguese descendants and people who had the financial means to study in the metropolis. Given the process of acquisition of a second language during the colonial period,\footnote{In Brazilian linguistic literature, \citet{Lucchesi_Baxter_Ribeiro_2009} define this phenomenon as an irregular linguistic transmission of the Portuguese language.} the aforementioned authors argue that the African and Afro-Brazilian speakers of Portuguese were agents in the changes and innovations that occurred in Brazilian Portuguese compared to European Portuguese.\footnote{There are characteristics of Brazilian Portuguese that are considered as representing a conservative variety of European Portuguese (cf. \cite{Graebin_2017}, and cf. \cite{Massini_Cagliari_Redenbarger_2016}, on comparative synchronic studies in Portuguese segmental phonology).}  However, there are few linguistic studies based on historical data of Afro-Brazilian and African writers, like manuscripts and large-scale written corpora, to confirm the hypothesis proposed by \citet{Mattos_e_Silva_2004}.

Generally speaking, it is very difficult to locate such manuscripts, whether personal or institutional writings. This is because, in addition to the unfavourable historical socioeconomic conditions in Brazil, the enslaved population was prevented from learning to read and write and, more specifically, they were institutionally forbidden to attend public classes by the Constitution of 1824, although no specific penalty has been reported for offenders. Indeed, there are reports and records of enslaved people present in schools as well as people who knew how to read and write at various levels of proficiency \citep{daSilva_2007}.

In my search for manuscripts, I came across several studies concerning the Catholic brotherhoods \emph{dos homens pretos} ‘of black men’, which were important institutions for the African and Afro-Brazilian population during the slavery period. There are two research areas that study these institutions: History (e.g., \cite{Assis_1998,Levi_2006}, \cite{MacCord_2005}; \cite{Quintao_2002}) and Linguistics \citep[cf.][]{Oliveira_2006,Galves_Lobo_2019}. A number of these studies focus on the states of Rio de Janeiro, Minas Gerais and Bahia. Fortunately, as Pernambuco was one of the states with the highest number of abolitionists and had many freed former slaves already in the 17\textsuperscript{th} and 18\textsuperscript{th} centuries \citep{Klein1969}, a large number of 19\textsuperscript{th} century manuscripts have survived in its brotherhoods. This documentation has proven to be a relevant source for understanding the history of Africans and their descendants, not only in Brazil, but also the rest of the world.\footnote{I also found information about these minutes and terms in Cape Verde and Portugal.}

This is a case study identifying characteristics of popular urban Portuguese, as spoken in Recife, more specifically in the neighbourhood of Santo Antonio. Although this brotherhood functioned in the geographical and institutional centre of the African and Afro-Brazilian community, I do not consider the variety of Portuguese featured in the texts an ethnic variety, but part of a popular and urban regional variety in a multicultural and multi-ethnic context. As the volume of data is not sufficient to allow for a statistical analysis, I present the documentation, the historical context and the fundamentals for using it as a database in this article.

After a brief introduction, the article proceeds with a short history of the brotherhoods, their operational structure and historical importance for the African and Afro-Brazilian population during that period, including their role as an institution that granted access to literacy. Next, I discuss the methodology and the palaeographic criteria essential for the selection of data. After evaluating three books from the brotherhood and the transcription of one thereof, I compared the signatures and the handwriting of the terms and minutes contained therein. In assessing the sociolinguistic nature of these materials, the ethnic background of Manoel de Barros was confirmed because of his position as a judge within the brotherhood, a position that could only be held by significant individuals in the Afro-Brazilian community. The section poses some challenges, such as the classification of the textual genre of the minutes.

In the third section I compare my data to
(1) \citet{Oliveira_2006} mainly for phonological phenomena;
(2) \citet{Lucchesi_Baxter_Ribeiro_2009} for a comparison with the characteristics of Afro-Brazilian Portuguese, such as the optional plural inflection on nouns; and
(3) \citet{Petter_2009} for Portuguese in general. Finally, I present pertinent summary comments and suggest possibilities for future research involving the documentation and transcription of such materials.

\section{Materials and methods}

From the outset of the colonial period, Catholic religious brotherhoods were present in every location where Africans were disembarked. The first records point to the emergence of the Brotherhood of Our Lady of the Rosary in Europe,\footnote{In this case, we are not talking about the brotherhoods \emph{dos homens pretos} ‘of black men’, but the European brotherhoods \emph{da Nossa Senhora do Rosário} ‘of the Rosary’ which gave origin to the brotherhoods \emph{da Nossa Senhora do Rosário dos Homens Pretos e Pardos} ‘of the Rosary of black and brown men’.} the first one being in the German city of Düsseldorf, in 1407. For the black brotherhoods, on the other hand, there are records of the \emph{Confraria da Nossa Senhora do Rosário dos Homens Pretos} ‘Confraternity of Our Lady of the Rosary of Black Men’ in Cape Verde, still in the 15\textsuperscript{th} century and the \emph{Irmandade de Nossa Senhora do Rosário dos Homens Pretos} ‘Brotherhood of Our Lady of the Rosary of Black Men’ in Angola in the 16\textsuperscript{th} century.\footnote{There are records from the 17\textsuperscript{th} century onwards concerning the emergence of both the aforementioned institutions in Brazil. The first brotherhoods \emph{da Nossa Senhora do Rosário dos Homens Pretos e Pardos} ‘of the Rosary of black and brown men’ were the one in Rio de Janeiro in 1639, the one in Belém do Pará in 1682, the ones in Salvador and Recife in 1685 and the one in Olinda and Igarassu in 1688.} The creation of a brotherhood was marked by the creation of a statute. As soon as the brothers began to congregate, they needed their statute to be officially approved by the metropolis (Lisbon). Once approved, they could raise funds to build a church with a cemetery, and for this reason many foundation dates of churches might be considered as the foundation date of a brotherhood.

The brotherhoods were civil organizations that derived from the liberal organizations of artisans of the European Middle Ages. In the colonies, they permitted the enslaved and freed population to secure a space for Catholic religious worship, a burial place (since the cemeteries at that time were private and located in churches attended by the white community), and they provided other amenities, such as cash loans and health care. \citet{Assis_1998} argues that the brotherhoods \emph{dos homens pretos e pardos} ‘of black and brown men’ occupied an ambiguous space in society, because on the one hand they acted as an an agent of assimilation and control, since members could not have participated in revolts or have been arrested, yet, on the other hand, they constituted the only institution of assistance to this population. \citet{Quintao_2002} observes that the greatest advantage of such associations was their autonomy, a fact that contributed to Brazilian religious syncretism.\footnote{Considering that this topic goes beyond the scope of this article, I suggest reading \citet{Jurua_2011}, and the literature on \emph{candomblé}, a syncretic Afro-Brazilian religion, and \emph{Umbanda}, a syncretic religious practice combining Afro-Brazilian and Brazilian Amerindian ritual features.} 

The statute that governed the brotherhoods and confraternities defined the rules of operation and membership. Usually, an initial membership fee was charged, and to occupy positions of responsibility, such as those of judge and clerk, it was necessary to have social prestige. In the case of the brotherhood of the Santo Antonio do Recife neighbourhood, according to the appointment regulation (\emph{compromisso}) in force at the time, it was also necessary for the King of Congo,\footnote{In Recife, the King of Congo was an important public figure who interceded alongside official bodies in the representation of the freed and enslaved population.} elected by the brotherhood itself, to be a descendant of the indigenous people of Angola (Compromisso 1782),\footnote{“[G]entio de Angola” (Compromisso 1782).} which binds this brotherhood to its Angolan origin.\footnote{In 1491, the \emph{manicongo} ‘King of the Congo and Angola’ Nzinga-a-Nkuwu, also called Nkuwu Nzinga, was the first African king to convert to Catholicism, being baptized as John I, having his kingdom recognized by the Pope and even sending one of his sons to be trained in the Vatican \citep{Souza_2002}. Despite this, the same king deconverted himself later.} However, one cannot say that all members were from the Angola region or that this would have been the case for their descendants since people from different parts of Africa were brought to Recife.

An issue that has caused much historical controversy is the fact that, due to limited literacy in general and, possibly, due to the institutional prohibition of access to education for the African and Afro-Brazilian population, the majority of the Catholic brotherhoods \emph{dos homens pretos e pardos} ‘of black and brown men’ in Pernambuco hired white clerks to write their minutes. The justification for Manoel de Barros, the judge of the brotherhood who had already been clerk of the same brotherhood, for writing the deliberations (\emph{termos}), was the temporary financial difficulty of the brotherhood, which prevented it from hiring a clerk, as was the tradition and, in a way, the obligation of such institutions. Moreover, in the documents under consideration, the signatures that were preceded by the expression “Cruz de...” [‘Cross of...’] were actually written by the clerk and the cross functioned as the signature, while those who knew how to write their own names would sign in their own hand, as shown in the two figures below.

% Figure 1:
\begin{figure}[!ht]
    \centering
    \includegraphics[width=.8\linewidth]{figures/fig1_06.png}
    \caption{Signatures made by crosses}
    \label{fig:fig1_06}
\end{figure}

% Figure 2:
\begin{figure}[!ht]
    \centering
    \includegraphics[width=.8\linewidth]{figures/fig2_06.png}
    \caption{Handwritten signature}
    \label{fig:fig2_06}
\end{figure}

\citet{daSilva_2007} does an excellent job of reconstructing the presence of people from the African and Afro-Brazilian community in Pernambucan society. Using data from the 1872 census,\footnote{Cf. \citet{Brazil_1876}.} \citet{daSilva_2007} compares the number of literate people in the enslaved population in Pernambuco, which was 1 literate person for every 999 illiterate ones and 0.6 literate enslaved children for every 999.4 illiterate ones. However, it is possible that some enslaved people did not openly admit their literacy, since that could have been considered dangerous by the authorities.

I analysed the advertisements of “Escravos fugidos”\footnote{‘Runaway slaves’.}, “Avisos diversos”\footnote{‘Miscellaneous notices’.} and “Vendas”\footnote{‘Sales’.} from the newspaper \emph{Diário de Pernambuco}\footnote{‘Pernambuco Daily’.} between the years 1831 and 1848\footnote{Most of them were from the 1840s and all the advertisements are stored in the Jordão Emerenciano State Public Archive.}. In 342 advertisements, I found two references to two literate enslaved men, Agostinho and Joaquim. When discussing alternative forms of literacy, \citet{Moyses_1994} reports that some enslaved people and freedmen used shopping lists, booklets and even a copy of their freedom letters as a way of learning to read and write.

In this way, Manoel de Barros, the clerk who after three years became a judge, was probably a freedman. This allowed him to properly exercise his duties as a person of prestige in the black community and someone who had access to literacy.

With regard to methodology, I adopt, in general, a qualitative analysis. I visited several archives in Pernambuco and found the relevant documentation in possession of IPHAN.\footnote{Instituto do Patrimônio Histórico e Artístico Nacional (lit. National Institute of Historical and Artistic Heritage).} The \emph{Irmandade Nossa Senhora do Rosário dos Homens Pretos} had donated its records to that institution for the purpose of secure storage and conservation. By means of digital photography, copies were made of three books, out of which I transcribed only one, containing information dated between 1829 and 1833. For the transcription, I used the paleographic techniques of \citet{Barbosa_Acioli_Assis_2006}.

Altogether, 32 texts were fully transcribed, 13 of which were written by Manoel de Barros. The remaining texts were in three different calligraphies. Although these very likely were African or Afro-Brazilian clerks, I chose to do a case study of the linguistic characteristics of Manoel de Barros as he held the position of judge and clerk and signed his own name. Consequently, I was sure that the clerk was writing in his own hand (\figref{fig:fig3_06}).\footnote{The excerpt below reads: \emph{Eu Manoel de Barros o fiz e assinei como Escrivão} ‘I Manoel de Barros recorded and signed as clerk’.}

% Figure 3:
\begin{figure}[!ht]
    \centering
    \includegraphics[width=\linewidth]{figures/fig3_06.png}
    \caption{Handwritten signature}
    \label{fig:fig3_06}
\end{figure}

After selecting the texts for analysis, I conducted a paleographic transcription of the documents and identified the main fixed textual structures of the minutes, such as title, place and date. The stylistic rigidity of the texts minimized the syntactic possibilities therein in view of the fact that several phrases were always repeated, for example \emph{Estando reunidos em mesa}... ‘Being together at the table...’ or \emph{Aos três dias do mês}... ‘On the third day of the month...’. Perhaps, because of such repetitions, copying the minutes themselves may have been a means of improving the writing skills of the author.

One major challenge was the lack of comparisons between texts written by African and Afro-Brazilian clerks on the one hand and texts written by other Brazilian clerks, for example \emph{brancos} ‘white men’ and pardos ‘brown men’.\footnote{Although I performed the paleographic transcription of texts by other brothers from the same period, in this work I was unable to compare their characteristics. Moreover, in this research, I only analysed the linguistic characteristics already attributed to the Afro-Brazilian and African population in Brazil.}  Until the beginning of the 19\textsuperscript{th} century, the literate population was minimal, and most clerks were either Portuguese or had lived in Portugal. Even so, I compared our data with previous linguistic analyses of minutes written by Africans and Afro-Brazilians in Salvador, Brazil, and there were several similarities. I also took into consideration the reported characteristics of Afro-Brazilian Portuguese, as described by \citet{Lucchesi_Baxter_Ribeiro_2009},\footnote{The authors described the so-called “Afro-Brazilian Portuguese”, the variety of Portuguese spoken by four rural Afro-Brazilian communities in the state of Bahia.} and Portuguese varieties in general.

Despite the hypothesis presented by \citet{Mattos_e_Silva_2004}, there are characteristics from this last-mentioned dialect that are shared by speakers of Popular Brazilian Portuguese (PBP) in general, so the question remains whether the characteristics described are exclusive to the Afro-Brazilian community or whether these characteristics also extend to PBP regardless of the ethnic origin of the speakers. Therefore, considering that the Portuguese written by Manoel de Barros is not an isolated variety, I treat the Portuguese written by him not as Afro-Brazilian Portuguese, but as part of PBP from Recife, which was written by an Afro-descendant. Hence, I understand the relationship between the two varieties that, although belonging to the same ethnic group, developed in a different way.


\section{Linguistic characteristics of the terms/minutes}

The description of the linguistic characteristics of the texts written by the African\slash Afro\babelhyphen{hard}Brazilian clerk Manoel de Barros is not intended to be a generalization of the work produced by all Afro-Brazilian clerks of the time, but to compare the characteristics of this clerk’s texts to the writings of other Afro-Brazilian and African clerks \citep{Oliveira_2006}, to the linguistic characteristics attributed to the Afro-Brazilian Portuguese \citep{Lucchesi_Baxter_Ribeiro_2009} and to Brazilian Portuguese \citep{Petter_2009}. The main goal of this discussion is to demonstrate the potential of the documents of the brotherhoods and confraternities of \emph{Nossa Senhora do Rosário dos Homens Pretos} in order to reach an understanding of the linguistic interaction between speakers in urban environments in the 19\textsuperscript{th} century.

From a total of 32 texts, I analysed three types of characteristics:
(1) phenomena of an orthographic nature;
(2) phonetic phenomena, which I distinguished between (a) those that occur at the level of the phonological word, and (b) those that occur at the level of more than one phonological word, namely, hypo-and hyper-segmentation; and
(3) morphosyntactic phenomena. I redefine some of the phenomena listed in \citet{Ziober_2014} and below I exemplify these points. Some phenomena are recurrent, such as the elevation of the vowel <e> to <i>, with 79 occurrences, while other occurrences happen only once, such as the rhotacism from <l> to <r>, in <sipruco> \emph{sepulcro} ‘sepulchre’.

\subsection{Phenomena of a graphic nature}

First, I need to acknowledge graphic phenomena as phenomena resulting from variation in spelling. Compare the variation between <s>, <c>, <ç>, <ss> and <sc> for the phoneme /s/:

\ea variation between S, C, Ç, SS and SC\label{ex:6:1}
    \ea \emph{a\textbf{c}ima}         $\rightarrow$ <a\textbf{ç}ima>      \textsc{prep} ‘above’ 
    \ex \emph{\textbf{s}egundo}       $\rightarrow$ <\textbf{ç}egundo>    \textsc{num} ‘second’
    \ex \emph{po\textbf{ss}e}         $\rightarrow$ <po\textbf{ç}ia>      \textsc{nom} ‘the act to take office’
    \ex \emph{fo\textbf{ss}e}         $\rightarrow$ <fu\textbf{s}ce>      \textsc{verb} ‘were/was’
    \ex \emph{e\textbf{ss}a}          $\rightarrow$ <e\textbf{s}a>        \textsc{dem} ‘this’
    \ex \emph{obriga\textbf{ç}ão}     $\rightarrow$ <obriga \textbf{c}om> \textsc{nom} ‘obligation’
    \ex \emph{\textbf{s}enhor}        $\rightarrow$ <\textbf{c}inhor>     \textsc{pron} ‘Sir’
    \ex \emph{do na\textbf{sc}imento} $\rightarrow$ <donacimento>         \textsc{prep-det nom} ‘of the birth’
    \ex \emph{o no\textbf{ss}o}       $\rightarrow$ <ono\textbf{c}o>      \textsc{det pron} ‘our’
\z                                                                    
\z

I found (2) two orthographic possibilities, <ch> and <x>, for the palatal /ʃ/:

\ea\label{ex:6:2a} variation of CH and X
    \ea \emph{se acha} $\rightarrow$ <cia\textbf{x}a>  \textsc{pron verb} ‘find it’
    \ex \emph{Paixão} $\rightarrow$ <Pai\textbf{ch}ao> \textsc{pnom} ‘Paixão’ (lit.Passion)
\z
\z

Because of articulatory reasons, the grapheme <m> is consistent before the bilabials <p> and <b>, and <n> before other consonants. Nevertheless, at that time, such norms were not yet widespread and Manoel de Barros presented variation in the use of these segments, even within the same text.

\ea\label{ex:6:2b} variation of M and N
    \ea \emph{deze\textbf{m}bro} $\rightarrow$ <deze\textbf{n}bro> \textsc{nom} ‘December’
    \ex \emph{co\textbf{n}sistório} $\rightarrow$ <Co\textbf{m}cistorio> \textsc{nom} ‘religious counsel’
\z
\z

Next, Manoel de Barros favoured the use of <u> instead of <v> in consonantal contexts, but never the contrary. Two hypotheses present themselves: first, that cursive writing has similar shapes, making it difficult to recognize them in handwriting; second, that the clerk deliberately wrote forms very close to <u> because of his partial knowledge of the orthographic norm. Thus, writing both almost the same could be a way of not incurring in spelling errors.

\ea\label{ex:6:3} variation of U and V\\
\emph{festi\textbf{v}idade} $\rightarrow$ <festi\textbf{u}idade> \textsc{nom} ‘festivity’
\z

Finally, I found other graphic phenomena occurring only once, namely, variation in the spelling of <h>, <j>, <rr>, <qu> and <f>.

\subsection{Phonetic phenomena}

I believe several phonological phenomena are phonetically conditioned. At first, I present data at the level of the phonological word. The use of <z> in contexts between vowels is related to both graphic and phonetic reasons. Because the spelling is not intuitive to the writer, words are written with <s> although the pronunciation of the sibilant sounds like [z]:

\ea\label{ex:6:4} use of <z> in contexts of [z]
\ea \emph{me\textbf{s}a} $\rightarrow$ <me\textbf{z}a>       \textsc{nom} ‘table’
\ex \emph{Ro\textbf{s}ário} $\rightarrow$ <Ru\textbf{z}ario> \textsc{pnom} ‘Rosary’
    \z
\z

The first phenomenon of purely phonological motivation is metathesis. In their study, \citet{Hora_Telles_Monaretto_2007} define the metathesis of Portuguese in the 18\textsuperscript{th} century as a transposition of consonants in one or two syllables. According to the authors, metathesis involving liquid consonants is of the tautosyllabic perceptual type, because the syllables alternate between one syllable where the liquid is in coda position, or perceptually two syllables where the liquid occurs in a branched onset. This phenomenon has been observed in historical studies of Portuguese \citep{Coutinho_1976,Faria_1970,Oliveira_2006}, but not all the data found by \citet[349]{Oliveira_2006} seemed to be phonetically motivated. Also, \citet[162]{Petter_2009} mentions that the metathesis is common in the Portuguese spoken in Angola\footnote{Cf. \citet{Chavagne_2005}.} and Mozambique,\footnote{Cf. \citet{Laban_1999}.} e.g., \emph{pruguntar} < \emph{perguntar} ‘to ask’. So, metathesis, as well as other phenomena, is not exclusive to the documents or region considered, but it has occurred in many other varieties of Portuguese.

\ea\label{ex:6:5} metathesis
    \ea \emph{\textbf{par}tido} $\rightarrow$ <\textbf{pra}tido> \textsc{adj} ‘broken, split’
    \ex \emph{a\textbf{pro}vada} $\rightarrow$ <a\textbf{por}uada> \textsc{adj} ‘approved’
    \ex \emph{con\textbf{gre}gados} $\rightarrow$  <com\textbf{ger}gados> \textsc{adj} ‘member of a congregation’
\z
\z

Another phenomenon observed is rhotacism, the transition between lateral and vibrant liquids due to their articulatory similarities. In the data of \citet[418]{Oliveira_2006} more than half of the rhotacisms from /r/ to /l/ happened in a coda position. The same occurs in data from Manoel de Barros. In our data, there was only one occurrence of /l/ to /r/, simultaneously with a metathesis, more precisely in the word <sipruco> \emph{sepulcro} ‘sepulchre’.

\ea\label{ex:6:6} rhotacism
    \ea \emph{conta\textbf{r}} $\rightarrow$ <comto\textbf{l}> \textsc{verb} ‘to say/to count’
    \ex \emph{sep\textbf{ul}cro} $\rightarrow$ <sip\textbf{ru}co> \textsc{nom} ‘sepulchre’
\z
\z

\citet[326]{Oliveira_2006} also found aphaeresis, the loss of a word-initial phoneme or syllable. In our data, aphaeresis is found as a result of the graphic omission of some syllables and sounds. In fact, I found an example, still very recurrent in all varieties of Portuguese, \REF{ex:6:8}, i.e. the reduction of the syllable <es> in <estar> \emph{estar} ‘to be’. In addition, I found the total assimilation of the sequence of two vowels [e] between the clitic and the verbal root.\footnote{I chose to classify \REF{ex:6:9} as total assimilation, but since I have only this data, it is hard to judge if it is a case of phoneme loss or assimilation.}

\ea\label{ex:6:8}
    aphaeresis\\
    \emph{\textbf{es}tando} $\rightarrow$ <tondo> \textsc{verb} ‘staying’
\ex\label{ex:6:9}
    total assimilation\\
    \emph{s\textbf{e e}leger} $\rightarrow$ <\textbf{ce}leger> \textsc{pron} \textsc{verb} ‘elect oneself’
\z

A further characteristic observed by \citet[336]{Oliveira_2006} is apocope. He notes that it has been a regular phenomenon throughout the history of Portuguese. \citet[163]{Petter_2009} also cites it from research by \citet{Chavagne_2005} on Angolan Portuguese. The verbal example \REF{ex:6:10} below coincides with a current characteristic of spoken Portuguese in general, deleting /r/ from the infinitive form of the verb. Furthermore, nowadays, in Recife, verb forms ending with /r/ in the final coda may not be realized or are replaced by an extension of the final vowel.

\ea\label{ex:6:10}
    apocope\\\emph{quere\textbf{r}} $\rightarrow$ <quere> \textsc{verb} ‘to want’
\z
         
Epenthesis, the addition of segments in the middle of a word, also occurs in our data, as in example \REF{ex:6:11-1b}, possibly motivated by the assimilation of the /l/ in word-final coda position to the stressed syllable and assimilation of the /s/ from the first to the second syllable. Example \REF{ex:6:11-1c} features the syllabification of an occlusive in coda with the filling of the nuclear vowel, similar to what happens in Brazilian Portuguese nowadays, for example [advogadʊ] > [ad͡ʒɪvogadʊ] ‘lawyer’. \citet[161]{Petter_2009} argues that Brazilian, Angolan and Mozambican Portuguese  have a preference for an open syllable pattern because of their contact with Bantu languages, in which this pattern is standard. Additionally, various Brazilian indigenous languages have the same standard pattern, which also might have contributed to the epenthesis process. In examples \REF{ex:6:11-1a} and \REF{ex:6:11-1b}, assimilation may also have played a role in the epenthetic process through the presence of the phonemes /d/ \REF{ex:6:11-1a} and /ɫ/ \REF{ex:6:11-1b}.

\ea\label{ex:6:11-1} epenthesis
\ea\label{ex:6:11-1a}\emph{reduzida} $\rightarrow$ <rudu\textbf{di}z/ida> \textsc{adj} ‘reduced’
\ex\label{ex:6:11-1b}\emph{resp\textbf{ei}t\textbf{á}vel} $\rightarrow$ <res/-pesto\textbf{l}uel> \textsc{adj} ‘respectable’
\ex\label{ex:6:11-1c}\emph{insí\textbf{g}nia} $\rightarrow$ <inci\textbf{gui}na> \textsc{nom} ‘badge’
    \z   
\z

With regard to vowels, the elevation of pre-tonic and post-tonic mid vowels is common in Brazilian Portuguese, and in the case of post-tonic vowels, the elevation is categorical at the end of the word in many Portuguese varieties. \citet[367]{Oliveira_2006} observes cases of elevation of stressed mid vowels \REF{ex:6:14} but does not find similar cases nor an obvious motivation in the literature despite admitting that phonetic motivation is possible and probable. Cases of hypercorrection due to oscillation between <e> and <i> in Portuguese spelling and phonetics are also found in our data. I have two hypotheses for this: (i) I believe examples \REF{ex:6:12b} and \REF{ex:6:12c} are due to variation among earlier varieties of Portuguese \citep{Graebin_2017} or (ii) they follow developmental paths similar to Mozambican and Angolan Portuguese, in which [e]>[i] occurs in similar contexts to Brazilian Portuguese, e.g. \emph{minino} ‘boy’, \emph{piquinino} ‘small’ (Mozambican Portuguese) and \emph{piqueno} ‘small’ (Angolan Portuguese).\footnote{For Angolan and Mozambican Portuguese examples, cf., respectively, \citet{Chavagne_2005} and \citet{Laban_1999}.}

\largerpage
\ea\label{ex:6:11-2}elevation of pretonic mid vowels
        \ea\label{ex:6:11-2a} \emph{d\textbf{e}} $\rightarrow$ <d\textbf{i}> \textsc{prep} ‘of’
        \ex\label{ex:6:11-2b} \emph{s\textbf{e}rvir} $\rightarrow$ <c\textbf{i}ruir> \textsc{verb} ‘to serve’
        \ex\label{ex:6:11-2c} \emph{c\textbf{o}stume} $\rightarrow$ <c\textbf{u}stume> \textsc{nom} ‘habit’
    \z

    \ex \label{ex:6:12} elevation of post-tonic mid vowels
        \ea\label{ex:6:12a} \emph{aond\textbf{e}} $\rightarrow$ <aond\textbf{i}> \textsc{adv} ‘where’
        \ex\label{ex:6:12b} \emph{quinz\textbf{e}} $\rightarrow$ <quinz\textbf{i}s> \textsc{num} ‘fifteen’
        \ex\label{ex:6:12c} \emph{houv\textbf{e}} $\rightarrow$ <hou\textbf{i}> \textsc{verb} ‘there was’
        \ex\label{ex:6:12d} \emph{junh\textbf{o}} $\rightarrow$ <junh\textbf{u}> \textsc{nom} ‘June’
    \z  

    \ex\label{ex:6:13} elevation of stressed mid vowels\\
     \emph{Pess\textbf{o}a} $\rightarrow$ <Pec\textbf{u}a> \textsc{pnom} ‘Pessoa’ (lit. Person)        
\z

A further variation observed in examples \REF{ex:6:14} is vowel posteriorization. \citet[351]{Oliveira_2006} suggests that assimilation processes contributed to such occurrences, and I add that nasalization possibly has interfered with the perception of the vowel, given that some data from Oliveira, as well as all of our data, constituted contexts of nasality.

\ea\label{ex:6:14} vowel posteriorization
    \ea \emph{M\textbf{a}noel} $\rightarrow$ <M\textbf{o}noel> \textsc{pnom} ‘Manoel’ 
    \ex \emph{S\textbf{a}ntíssimo} $\rightarrow$ <S\textbf{o}mticimo> \textsc{pnom} ‘Santíssimo’ (lit. Sacred)
\z
\z

\citet[371]{Oliveira_2006} attributes the lowering of high vowels to the proximity of <e> in the same word; compare \REF{ex:6:15b} and \REF{ex:6:15c} for assimilation processes. At the same time, in these examples, I encounter the presence of nasality adjacent to the lowered vowel, showing a possible extension of these assimilation processes beyond vowel quality and nasality. Furthermore, in the case of \REF{ex:6:15c}, assimilation occurs by raising the stressed nasal vowel, followed by its lowering in the process of assimilation.

\ea\label{ex:6:15} lowering of high vowels
    \ea\label{ex:6:15a} \emph{ju\textbf{i}z} $\rightarrow$ <ju\textbf{e}s> \textsc{nom} ‘judge’
    \ex\label{ex:6:15b} \emph{circ\textbf{u}nstâncias} $\rightarrow$ <circ\textbf{o}ntoncias> \textsc{nom} ‘circumstances’
    \ex\label{ex:6:15c} \emph{de des\textbf{i}stência} $\rightarrow$ <didez\textbf{e}ntencia> \textsc{prep} \textsc{nom} ‘abdication’
    \z
\z

Afterwards, I observed diphthong processes, like the ones signalled by \citet{Noll_2008} and \citet[409]{Oliveira_2006} in Brazilian Portuguese and \citet[162]{Petter_2009} in Angolan Portuguese. With regard to the contexts of monophthongization, \citet[408]{Oliveira_2006} found a large number of reductions from [ow] to [o], but few cases from [oj] to [o].

\newpage
\ea\label{ex:6:17} diphthongization
\ea \emph{prop\textbf{ô}s} $\rightarrow$ <porp\textbf{oi}s> \textsc{verb} ‘propose’
\ex \emph{t\textbf{o}da} $\rightarrow$ <t\textbf{uo}da> \textsc{quant} ‘all’
\z
\ex\label{ex:6:18} monophthongization
\ea \emph{dez\textbf{oi}to} $\rightarrow$ <dez\textbf{o}to> \textsc{num} ‘eighteen’
\ex \emph{\textbf{ou}tubro} $\rightarrow$ <\textbf{o}tubro> \textsc{nom} ‘October’
    \z
\z

To conclude this discussion of phonetic and phonological variation types, I draw attention to some phonetic phenomena that interact with the morphology and contribute to an understanding of the domain of the morphological and phonological word. Examples \REF{ex:6:19} and \REF{ex:6:20} demonstrate the variation in the segmentation of excerpts involving prepositions and clitic pronouns.

In hyper-segmentation, \REF{ex:6:20a} is a case in which the <a> is segmented from <atual> \emph{atual} ‘current’ because it resembles a definite article \emph{a} ‘the’ before a feminine word, and \REF{ex:6:20c} segments <nosa> in a similar way from \emph{nossa} ‘our’, although there are two words in the excerpt <nós assinamos> ‘we signed’. Despite the fact that the cases involve word segmentation, I interpret the phenomena in \REF{ex:6:19} and \REF{ex:6:20} mainly as graphic ones, demonstrating the phonological character of some morphemes that may contain parts that resemble different words.

\ea\label{ex:6:19}
hypo-segmentation
\ea\label{ex:6:19a} \gll de toda a mesa $\rightarrow$ <dituadameza> \\
         \textsc{prep} \textsc{quant} \textsc{det} table \\
\glt ‘from all the members of the table’

\ex\label{ex:6:19b} \gll e te represento  $\rightarrow$ <eterreperzento> \\
        \textsc{conj} \textsc{pron} represent.\textsc{1sg} \\
\glt ‘and I represent you’

\ex\label{ex:6:19c} \gll com a data $\rightarrow$ <comadata> \\
         \textsc{prep}  \textsc{det}  date \\
\glt ‘on the date’

\ex\label{ex:6:19d} \gll de comum $\rightarrow$ <dicomum> \\
         \textsc{prep} common \\
    \glt ‘in common’

\ex \gll se eleger $\rightarrow$ <celeger> \\
        \textsc{pron} elect \\
    \glt ‘elect himself’
\z
\ex\label{ex:6:20}hyper-segmentation
\ea\label{ex:6:20a} \emph{atual}  $\rightarrow$ <a tual> \textsc{adv} ‘current’
\ex\label{ex:6:20b} \emph{compromisso} $\rightarrow$ <com primico> \textsc{nom} ‘statute’
\ex\label{ex:6:20c} \gll nós assinamos $\rightarrow$ <nosa cinomos> \\
        \textsc{1pl} sign.\textsc{1pl} \\
\glt   ‘we signed’
    \z
\z

\subsection{Morphosyntactic phenomena}

A very widespread phenomenon in Brazilian Portuguese is the variation in plural agreement within the noun phrase, the preference being marking the first element or determiner \citep{Lucchesi_Baxter_Ribeiro_2009}. \citet{Inverno_2005} also gives many examples for Angolan Portuguese in which the noun is not marked for plural, but the demonstratives, determinants and possessives are, e.g., \emph{Estas duas mulher} ‘these two women’ (lit. ‘these two woman’) and \emph{Os meus filho} ‘my sons’ (lit. ‘the my son’).\footnote{These examples are from \citet{Inverno_2005} and \citet{Petter_2009}.} We also observed a similar case in \REF{ex:6:21a}. However, example \REF{ex:6:21b} shows the confusion these markers caused for the clerk, apparently inducing hypercorrection in the word <quinzis> \emph{quinze} ‘fifteen’, yet at the same time he does not mark the plural on the word \emph{dia} ‘day’.

As the volume of documentation is small and there are many fixed textual structures, a more in-depth study would be necessary to yield concrete clues about the behaviour of the agreement rule, since in several cases plural agreement is marked on all items in the noun phrase.

\ea\label{ex:6:21}plural agreement in the noun phrase
\ea \label{ex:6:21a}
\gll dos homen-s preto-s $\rightarrow$ <dos homem preto> \\
    \textsc{prep-det-pl}  man-\textsc{pl}  black-\textsc{pl}   \\
\glt ‘of Black Men’ (lit. of black man) 

\ex \label{ex:6:21b}
\gll ao-s  quinze  dia-s  do $\rightarrow$ <aos quinzis diado> \\
 \textsc{prep-det-pl} fifteen day-\textsc{pl} \textsc{prep-det} \\
\glt ‘On fifteen of...’ (lit. on the fifteen days of...)
\z
\z


\section{Discussion and conclusion}

This study has documented the main linguistic characteristics of the texts produced by the clerk and judge Manoel de Barros of the \emph{Irmandade Nossa Senhora do Rosário dos Homens Pretos}. Since 2014, such documents have been contaminated by fungi and bacteria and have been removed from public access. Therefore, I would like to appeal for the urgency of restoration, documentation and digitization of this data source, not only in Recife, but in the case of brotherhoods throughout Brazil that experience the same conservation problems.

The brotherhoods are an important source of historical data about African and Afro-Brazilian communities. These brotherhoods allowed people to organize themselves in terms of  a civil society and to improve their lives by providing services, loans and amenities at a time when only the upper classes would otherwise have had access to them. Despite the fact that those organizations faced discrimination on a racial basis, there was still a lot of prestige in holding a role as one of the leaders or as a member.

With regard to the linguistic phenomena analysed, four of them were clearly motivated by graphic reasons and sixteen were motivated by phonological aspects of spoken Portuguese. From these, five were related to vowel alteration \xxref{ex:6:11-2}{ex:6:15}, two (\REF{ex:6:6} and \REF{ex:6:11-1}) involved more than one phonological process at the same time and two (\REF{ex:6:4} and \REF{ex:6:14}) were related to orthography in addition to the phonological aspect.

Although I encountered phenomena that are frequently found in other Portuguese varieties, e.g. \REF{ex:6:11-2}, the elevation of pretonic mid vowels, and \REF{ex:6:17}, diphthongization, others are not mentioned in the literature. The elevation of a stressed mid vowel as in \REF{ex:6:14} and the motivation of this process are less clear.

As far as the texts and their organization are concerned, the fixed textual structures should be considered formal language.  For that reason, morphosyntactic phenomena found in them may not constitute as reliable data of spoken Portuguese as those occurring in informal texts, e.g. personal letters and diaries. In that sense, the absence of an official orthography in the late 18\textsuperscript{th} and 19\textsuperscript{th} centuries strengthens the observation of phonologically motivated phenomena and the argument that they are indeed characteristics of spoken Portuguese.

Another important aspect to be mentioned is the fact that these texts were produced in a multicultural environment with many different ethnic backgrounds. Although the institution’s origin can be linked to a specific country, in this case Angola, it is not possible to say that all the members were people of Angolan heritage. The most distinctive social aspect at the time was skin colour, so the only distinction we have is between the black/brown brotherhoods and the white brotherhoods, the latter not featuring any explicit mention of skin colour. That said, we cannot claim that the Portuguese spoken by the black/brown brotherhoods shared necessarily more characteristics with Angolan Portuguese than with PBP in general. In order to investigate which linguistic characteristics are typical of Brazilian Portuguese in general and those  which are typical of a certain region or ethnic group, I suggest that this study should be expanded to include not only black/brown brotherhoods.

In conclusion, these texts were produced by urban people and the absence of an official orthography resulted in more transparency of phonological phenomena than the morphosyntactic ones, since those were limited by text typology. Finally, even considering their limitations, these texts are an important data source to understand the society and the writing patterns of less literate people in the 19\textsuperscript{th} century: they are available all over Brazil and Africa and they are one of the few data sources to allow for comparisons between these varieties from a historical perspective.

\section*{Acknowledgements}
I would like to thank my M.A. thesis supervisors for this research, Stella Telles and Marlos de Barros Pessoa, as well for the members of the brotherhood and for the workers of the archives from Recife, Brazil. I also would like to thank the three anonymous reviewers who helped to significantly improve the text.


\section*{Abbreviations}
\begin{multicols}{3}
  \begin{tabbing}
    \scshape quant\hspace{.33em}\= Quantifier \kill
    \scshape 1sg   \> 1\textsuperscript{st} person singular \\
    \scshape 1pl   \> 1\textsuperscript{st} person plural \\
    \scshape adj   \> Adjective \\
    \scshape adv   \> Adverb \\
    \scshape conj  \> Conjunction \\
    \scshape dem   \> Demonstrative \\
    \scshape det   \> Determinant \\
    \scshape nom   \> Nominal \\
    \scshape num   \> Numeral \\
    \scshape pl    \> Plural \\
    \scshape pnom  \> Proper Noun \\
    \scshape prep  \> Preposition \\
    \scshape pron  \> Pronoun \\
    \scshape quant \> Quantifier \\
    \scshape verb  \> Verb
  \end{tabbing}
\end{multicols}

\section*{Sources}
\begin{itemize}[wide, leftmargin=!,labelindent=0pt,itemindent=-20pt,itemsep=0pt,labelsep=0pt]
\item[] Advertisement of the newspaper \emph{Diário de Pernambuco}. 1831--1846. Transcription of documents from the State Public Archive \emph{Jordão Emerenciano}.

\item[]\emph{Comprimisso} [Statute] \emph{da Irmandade de Nossa Senhora do Rosário dos Homens Pretos da Vila do Recife}. 1782. Arquivo Histórico Ultramarino (AHU–PE), LAPEH, Códice 1303.

\item[]Manuscripts of the \emph{Irmandade Nossa Senhora do Rosário dos Homens Pretos do Recife}. 1829--1832.
\end{itemize}

\printbibliography[heading=subbibliography,notkeyword=this]

\end{document}
