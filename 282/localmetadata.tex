\renewcommand{\lsSeries}{tgdi}
\renewcommand{\lsSeriesNumber}{6}
\renewcommand{\lsID}{282}

\title{Demonstratives in discourse}
\author{Åshild Næss\and Anna Margetts\lastand Yvonne Treis}
\typesetter{Yvonne Treis, Sebastian Nordhoff, Ahmet Bilal Özdemir}

\proofreader{%
Alexia Fawcett,
Amir Ghorbanpour,
Andriana Koumbarou,
Barend Beekhuizen,
Bev Erasmus,
Christian Döhler,
Carmen Jany,
Franny Vandervoort,
Geoffrey Sampson,
Jeroen van de Weijer,
Joseph Lovestrand,
Kalen Chang,
Lachlan Mackenzie,
Madeline Myers,
Natsuko Nakagawa,
Ryan Ka Yau Lai,
Siva Kalyan,
Sune Gregersen,
Tom Bossuyt
}


\BookDOI{10.5281/zenodo.4054814}
\renewcommand{\lsISBNdigital}{978-3-96110-286-0}
\renewcommand{\lsISBNhardcover}{978-3-96110-287-7}

\BackBody{This volume explores the use of demonstratives in the structuring and management of discourse, and their role as engagement expressions, from a crosslinguistic perspective. It seeks to establish which types of discourse-related functions are commonly encoded by demonstratives, beyond the well-established reference-tracking and deictic uses, and also investigates which members of demonstrative paradigms typically take on certain functions. Moreover, it looks at the roles of non-deictic demonstratives, that is, members of the paradigm which are dedicated e.g. to contrastive, recognitional, or anaphoric functions and do not express deictic distinctions. Several of the studies also focus on manner demonstratives, which have been little studied from a crosslinguistic perspective. The volume thus broadens the scope of investigation of demonstratives to look at how their core functions interact with a wider range of discourse functions in a number of different languages. The volume covers languages from a range of geographical locations and language families, including Cushitic and Mande languages in Africa, Oceanic and Papuan languages in the Pacific region, Algonquian and Guaykuruan in the Americas, and Germanic, Slavic and Finno-Ugric languages in the Eurasian region. It also includes two papers taking a broader typological approach to specific discourse functions of demonstratives.}



