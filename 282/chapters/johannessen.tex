\documentclass[output=paper,colorlinks,citecolor=brown]{langscibook} 
\author{Janne Bondi Johannessen\affiliation{University of Oslo}\orcid{}}
\title{Psychologically distal demonstratives in Scandinavian are not “discourse new”}
\abstract{Scandinavian languages have a demonstrative (in the form of the third person pronoun) used with nouns or noun phrases, which is different from more well-known spatial demonstratives. \citet{Johannessen2008} argued that its conditions of use are related to psychological distance, while \citet{Lie2010} argued that its main role is to invoke a referent in discourse. In this chapter, I have gone through empirical data, two short stories and four speech corpus dialogues, and investigated the use of this demonstrative (called the psychologically distal demonstrative or PDD). It is concluded that there are many occurrences of the PDD that would remain unexplained in Lie’s account: It can occur more than once per discourse, it can be used by both interlocutors in the same discourse, and not all referents are denoted by it. Also, it does not point out only key referents. An account based on psychological distance can explain the empirical facts.
}
\IfFileExists{../localcommands.tex}{
  \usepackage{langsci-optional}
\usepackage{langsci-gb4e}
\usepackage{langsci-lgr}

\usepackage{listings}
\lstset{basicstyle=\ttfamily,tabsize=2,breaklines=true}

%added by author
% \usepackage{tipa}
\usepackage{multirow}
\graphicspath{{figures/}}
\usepackage{langsci-branding}

  
\newcommand{\sent}{\enumsentence}
\newcommand{\sents}{\eenumsentence}
\let\citeasnoun\citet

\renewcommand{\lsCoverTitleFont}[1]{\sffamily\addfontfeatures{Scale=MatchUppercase}\fontsize{44pt}{16mm}\selectfont #1}
   
  %% hyphenation points for line breaks
%% Normally, automatic hyphenation in LaTeX is very good
%% If a word is mis-hyphenated, add it to this file
%%
%% add information to TeX file before \begin{document} with:
%% %% hyphenation points for line breaks
%% Normally, automatic hyphenation in LaTeX is very good
%% If a word is mis-hyphenated, add it to this file
%%
%% add information to TeX file before \begin{document} with:
%% %% hyphenation points for line breaks
%% Normally, automatic hyphenation in LaTeX is very good
%% If a word is mis-hyphenated, add it to this file
%%
%% add information to TeX file before \begin{document} with:
%% \include{localhyphenation}
\hyphenation{
affri-ca-te
affri-ca-tes
an-no-tated
com-ple-ments
com-po-si-tio-na-li-ty
non-com-po-si-tio-na-li-ty
Gon-zá-lez
out-side
Ri-chárd
se-man-tics
STREU-SLE
Tie-de-mann
}
\hyphenation{
affri-ca-te
affri-ca-tes
an-no-tated
com-ple-ments
com-po-si-tio-na-li-ty
non-com-po-si-tio-na-li-ty
Gon-zá-lez
out-side
Ri-chárd
se-man-tics
STREU-SLE
Tie-de-mann
}
\hyphenation{
affri-ca-te
affri-ca-tes
an-no-tated
com-ple-ments
com-po-si-tio-na-li-ty
non-com-po-si-tio-na-li-ty
Gon-zá-lez
out-side
Ri-chárd
se-man-tics
STREU-SLE
Tie-de-mann
} 
  \togglepaper[1]%%chapternumber
}{}

\begin{document}
\maketitle
\shorttitlerunninghead{Psychologically distal demonstratives in Scandinavian}

%orphan control
%keep examples together
%The citegen command does not allow page numbers as attributes.

\section{Introduction}\label{sec:johannessen:1}

\citet{Johannessen2008} describes a type of demonstratives found in the Scandinavian languages, which are called “psychologically distal demonstratives” (PDDs). They are mostly found in informal speech, and are not used in written text unless they represent a spoken discourse. 

Formally, the PDD has the form of the 3\textsuperscript{rd} person singular personal pronoun (\textit{han} ‘he’ and \textit{hun} ‘she’), used attributively to (modified) nouns. Plural usage has not been discussed in the literature. Its function, according to \citet{Johannessen2008}, is to signal psychological distance: that the speaker does not personally know the person referred to; or that the addressee does not know the person referred to; or that the speaker has a negative attitude to the person referred to. \citet{Lie2010} contests Johannessen’s account, and instead proposes that these demonstratives are used for background deixis, referring to what \citet{Diessel1999Book} calls “discourse new and hearer old”. A similar understanding is found in \citet[vol. 2: 317]{TelemanEtAl1999} for Swedish: to actualise referents who are not present in a concrete discourse, but who both the speaker and hearer have in their world of concepts. An example of the PDD is given in \REF{ex:johannessen:1}.

\ea\label{ex:johannessen:1}
 \gll Er like gammel som \textbf{hu} \textbf{dama} og kunne ALDRI klart å være sammen med en så gammel mann. \\
     am equally old like she woman.\textsc{def} and could never managed to be together with a such old man\\
\glt ‘I’m as old as that woman and could never have coped being together with a man that old.’\footnote{\url{https://forum.kvinneguiden.no/topic/1275529-hellstr\%C3\%B8m-og-anita/}.}
\z

Example \REF{ex:johannessen:1} is from a comment on a discussion forum, discussing the age difference between an older celebrity cook and his young female partner. The commentator clearly distances herself from the woman.

In this chapter, I will argue against Lie’s point. Using empirical data, specifically dialogues in fictional literature as well as authentic dialogues from speech corpora, I will show that psychologically distal demonstratives are often used repeatedly throughout a dialogue. Also, I will show that the PDD can be used for the same referent by both interlocutors in a dialogue. Both uses would be hard to explain using Lie’s account, while they fit well with \citet{Johannessen2008}. Although the present chapter uses Norwegian data, \citet{Johannessen2008} deals with more Scandinavian languages, and it will be assumed, based on my knowledge of these, that what is said about Norwegian here carries over to the other languages.

The chapter is structured as follows: \sectref{sec:johannessen:2} presents the two different analyses of the PDD and a brief review of relevant accounts in the literature. \sectref{sec:johannessen:3} is a presentation of the empirical material and the method used in this study, while \sectref{sec:johannessen:4} and \sectref{sec:johannessen:5} go through findings from two short stories and two corpora used. In \sectref{sec:johannessen:6}, the results are summarised and the chapter is concluded.

\section{Two different views on the Scandinavian PDD and related discussions}\label{sec:johannessen:2}
\subsection{Psychologically distal demonstratives \citep{Johannessen2008}}\label{sec:johannessen:2.1}

\citet{Johannessen2008} shows that in addition to spatial deixis, the Scandinavian languages (Norwegian, Swedish and Danish, as well as Icelandic) also have demonstratives, with the same form as 3\textsuperscript{rd} person singular pronouns, that express emotional deixis, not previously described in depth. Although they have the form of pronouns, their use is different not only in that they always modify a noun and never replace a noun phrase, but also in that there are special pragmatic conditions for their use (see below). The PDD is also different from the preproprial article found in many dialects, as explained in \citet[169–170]{Johannessen2008}:\footnote{A preproprial article is a prenominal definite article that must appear with proper names in argument positions in many Scandinavian dialects \citep{JohannessenGarbacz2014}.} It is generally not inflected for case; it carries stress; it has a distinct form in many dialects; it adds additional meaning to the noun phrase; it can modify any kind of noun, not just proper nouns; and is not obligatory.

Since the demonstrative co-occurs with a noun phrase, it is possible to ask whether what we see is rather a pronoun followed by an appositive noun phrase. The answer is clearly no. If the noun phrase following the demonstrative were an apposition, one would expect it to be a full noun phrase. In Norwegian, it is easy to see whether this is the case, since definite noun phrases modified by an adjective or a relative clause have obligatory double definiteness, meaning that the noun must be marked by a definite suffix and that an additional definite determiner must occur phrase-initially. This would be the noun phrase used with appositions, which is not what we see with the PDD. Instead, the PDD is an integrated part of the noun phrase (see \citealt[185-187]{Johannessen2008}). A pronoun with an apposition is illustrated in \REF{ex:johannessen:2}, while the PDD is given in \REF{ex:johannessen:3}. Note also that the pronoun is inflected for case, unlike the demonstrative.\footnote{There is, somewhat surprisingly, one single case-inflected demonstrative in the material, see \REF{ex:johannessen:19}. I will not discuss this further here, but simply note its existence.} 

\ea\label{ex:johannessen:2}
 \gll Jeg beundrer \textbf{henne}, \textbf{den} \textbf{flinkeste} \textbf{jenta} i klassen.\\
     I admire her.\textsc{acc} the cleverest.\textsc{def} girl.\textsc{def} in class.\textsc{def}\\
\glt ‘I admire her, the cleverest girl in the class.’
\z

\ea\label{ex:johannessen:3}
 \gll Jeg beundrer \textbf{hun} \textbf{flinkeste} \textbf{jenta} i klassen.\\
     I admire she.\textsc{nom} cleverest.\textsc{def} girl.\textsc{def} in class.\textsc{def}\\
\glt ‘I admire that cleverest girl in the class.’
\z

With spoken language dialogue corpora as empirical evidence, I argue that these demonstratives regulate the degree of psychological distance the speaker wants to convey to the listener about a referent. The conditions for use below and the examples \REF{ex:johannessen:4}-\REF{ex:johannessen:9} illustrating such psychologically distal demonstratives are mainly taken from \citet[164-167]{Johannessen2008}. For information on informant codes, see \sectref{sec:johannessen:3}.\\
\\
\textit{PDD Condition 1: The speaker does not personally know the person referred to.}

\ea\label{ex:johannessen:4}
{(NoTa\_027)}\\
\gll \textbf{Hun} \textbf{dama} hun blei jo helt nerd da. \\
     she woman.\textsc{def} she became yes totally nerd then\\
\glt ‘But that woman became a total nerd, as you know.’\footnote{In both \REF{ex:johannessen:4} and \REF{ex:johannessen:5}, the pronoun after the initial noun phrase shows that the initial noun phrase is left-dislocated. Left-dislocation is very common in speech, and is generally used to lift something into focus \citep[904–908]{FaarlundEtAl1997}.}
\z

\ea\label{ex:johannessen:5}
{(NoTa\_060)}\\
\gll \textbf{Hun} \textbf{von} \textbf{der} \textbf{Lippe} hun hadde lært seg skikkelig.\\
     she von der Lippe she had taught herself really\\
\glt ‘That von der Lippe, she had really taught herself.’
\z

In \REF{ex:johannessen:4}, the speaker refers to some person in charge of a club who, as we can guess from the context, did not grant her access to get in, perhaps because she had a fake ID card. The speaker expresses a distance to this person, she does not know her. In \REF{ex:johannessen:5}, the speaker talks admiringly about an actress, and the use of the demonstrative shows that the speaker needs to say that it is not a person close to her.\\
\\
\textit{PDD Condition 2: The addressee does not personally know the person referred to.}

\ea\label{ex:johannessen:6}
 \gll Du vet han kjørelæreren jeg har? \\
     you know he driving.teacher.\textsc{def} I have\\
\glt ‘You know that driving teacher I have?’ (NoTa\_038)
\z

\ea\label{ex:johannessen:7}
 \gll Men \textbf{hun} \textbf{dattera} \textbf{mi} sa a syns det var så høyt under taket.\\
     but she daughter.\textsc{def} my said she thought it was so high under ceiling.\textsc{def}\\
\glt ‘But that daughter of mine said she thought it was so high under the ceiling.’ (TAUS\_a43)
\z

In \REF{ex:johannessen:6} the speaker is uncertain as to whether the addressee knows her driving instructor, and in \REF{ex:johannessen:7} the speaker uses the PDD to express towards the hearer that she knows that the daughter referred to is not somebody the hearer knows.\\
\\
\textit{PDD Condition 3: The speaker has a negative attitude to the person referred to.} 

\ea\label{ex:johannessen:8}
 \gll Fordi \textbf{han} \textbf{idioten} vi hadde leid som sjåfør ...\\
     because he idiot we had hired as driver\\
\glt ‘Because that idiot we had hired as a driver …’\textbf{ }(NoTa\_079)
\z

\ea\label{ex:johannessen:9}
 \gll Er ikke helt god \textbf{hu} \textbf{mora} mi altså ... \\
     is not quite good she mother.\textsc{def} my you.see\\
\glt ‘That mother of mine is not quite with it, actually …’ (Woman, web)
\z

In \REF{ex:johannessen:8} the speaker refers to a useless bus driver who has destroyed the engine of their bus by driving in the second gear all the time, hence the use of the PDD (and of the word \textit{idiot}). In \REF{ex:johannessen:9} the speaker refers to a close family member (mother) who would presumably usually be referred to in a neutral or even loving way, but here there is a hint of dismay toward her, signalled by the PDD. 

Other linguists have made similar observations on some of the conditions, such as \citet[23]{Delsing2003}, who says that the speaker is uncertain as to whether the listener knows who is intended (originally in Swedish: “talaren är osäker på om lyssnaren vet vem som avses”). The Swedish reference grammar \citep[vol. 2: 274]{TelemanEtAl1999} also points out that the listener’s knowledge of the person referred to can be a condition of the use of this pronominal demonstrative. They say that the speaker and the listener must know of the person referred to, but that this person must not be a close acquaintance or be activated in the discourse.

More generally, both Condition 1 and Condition 2 have an aspect of politeness – what \citet{BrownLevinson1987} call positive politeness. \citet{BrownLevinson1987} suggest 15 politeness strategies divided into three supercategories: claim common ground, convey that S (speaker) and H (hearer) are cooperators, and fulfil H’s want for some X. I suggest that it is the cooperator strategy that is the closest to Conditions 1 and 2, and especially Strategy 9: \textit{Assert or Presuppose S’s knowledge of and concerns for H’s wants} \citep[125]{BrownLevinson1987}. When the speaker does not know the referent personally, it is polite to convey this information to the hearer. Example \REF{ex:johannessen:5} is a good illustration: If the speaker referred to the actress by her name but without the demonstrative, it would give the wrong impression to the hearer that the speaker actually knew the actress personally, or at least knew well the world of theatre and actresses in general. Using the PDD (by Condition 1) the speaker asserts that s/he wants the hearer to know that there is no such knowledge involved, in order not to mislead the hearer. Example \REF{ex:johannessen:6} illustrates Condition 2, in which the speaker actively asks the hearer to direct her attention to the speaker’s driving teacher, and in this way shows the hearer that she is aware of the hearer’s need to know that this is not a person the speaker thinks the hearer knows already.

\subsection{Demonstratives for fetching referents into the discourse \citep{Lie2010}}\label{sec:johannessen:2.2}

\citet{Lie2008} focusses on the discourse activation function of the demonstratives under discussion, and says that their main task is that of bringing a new person into the discourse. \citet{Strahan2007} also stresses the discourse activation aspect. \citet{Lie2010} is more explicit, and says that we would get a better understanding of these demonstratives if we saw them in terms of background deixis, referring to what \citet{Diessel1999Book} calls “discourse new and hearer old”, i.e. that they do not refer to a previously mentioned referent or to a referent in the situational context, but that they are used to activate specific, shared knowledge \citep[63]{Lie2010}. 

\citet[63]{Lie2010} also stresses the importance of grounding, “the shared knowledge, belief and attitudes of the interlocutors” \citep[60]{CroftCruse2004}. A similar understanding to both Lie and Croft \& Cruse can be found in the Swedish reference grammar \citep[vol. 2: 317]{TelemanEtAl1999}: “the function of these words is to actualise referents that are not present in the concrete discourse, but who both speaker and hearer have in their world of concepts” [my translation, JBJ]. 

As we saw above, the Swedish reference grammar also stresses that the demonstrative must not refer to a close acquaintance, and it maintains that the demonstrative is “distance-creating” \citep[vol. 2: 317]{TelemanEtAl1999}. Lie dismisses this possibility: “This I understand as an extra function, not as a primary function” [my translation, JBJ] \citep[64]{Lie2010}. Lie also explicitly opposes Johannessen’s criteria for the use of the PDD: “Her first two points say something about the fact that the speaker or hearer knows what is referred to, but not that the person referred to belongs in a common background. I therefore think it gives a better understanding for these kinds of expressions that we here have background deixis” [my translation, JBJ] \citep[70, fn. 15]{Lie2010}.

\subsection{Accounts in the literature with relevance to the PDD}\label{sec:johannessen:2.3}

\citet{Himmelmann1996} designs a typology of demonstratives in narrative discourse. The first type is situational use, which refers to entities in the utterance situation (\citeyear[219]{Himmelmann1996}), the next is discourse deictic use, referring to events or propositions (\citeyear[224]{Himmelmann1996}), and the third is the tracking use, to keep track of what is happening to whom (\citeyear[226]{Himmelmann1996}). It is the fourth type, recognitional use (\citeyear[230]{Himmelmann1996}), that is most interesting and relevant here, since this type appeals to knowledge of the referent. The intended referent, according to Himmelmann, is to be identified via specific, shared knowledge. 

%There is a bug with the \citegen command, it doesn't allow page numbers as attribute
\citet{Diessel1999Book} is work on which \citet{Lie2010} is based. \citeauthor{Diessel1999Book}'s (\citeyear[106]{Diessel1999Book}) system is developed from Himmelmann’s. He divides demonstratives into exophoric demonstratives (in the speech situation) and endophoric demonstratives, with three subtypes: anaphoric (referring to NPs), discourse deictic (referring to propositions) and recognitional (with the same name as in Himmelmann’s system). This latter type is the one that is relevant for Lie. They are used to activate specific shared knowledge; “[r]ecognitional demonstratives are specifically used to mark information that is \textit{discourse new (i.e.) unactivated)} and hearer old (…)” (\citealt[106]{Diessel1999Book}, my italicization) and are “often used to \textit{indicate emotional closeness}” (\citealt[107]{Diessel1999Book}, my italicization), cf. also emotional deixis \citep{Lakoff1974}. 

It is clear that the demonstrative I am investigating here and the recognitional type have something in common. Indeed, \citet[230]{Himmelmann1996} says that the speaker is uncertain as to whether the hearer can identify the intended referent, which he says can also be seen by the frequently accompanying tags like \textit{you know?} or \textit{remember}? As a matter of fact, \REF{ex:johannessen:6} even contains this kind of appeal to the hearer, with a \textit{you} \textit{know}{}-question. At some level the PDD may appeal to the hearer’s knowledge, as in the obvious case of \REF{ex:johannessen:6}, \textit{han} \textit{kjørelæreren} (he driving.teacher.\textsc{def}) ‘that driving teacher’. Here the speaker may have talked about the driving teacher at some previous point in their common history. Using the demonstrative in addition informs the hearer that the person referred to is one that the speaker knows that the hearer does not know personally. But there are examples where it would be difficult to explain the use of the demonstrative as an appeal to common knowledge, as in \REF{ex:johannessen:9}, with \textit{hu} \textit{mora} \textit{mi} (she mother.\textsc{def} my) ‘that mother of mine’. This phrase would be fully felicitous even if the hearer had never heard about the speaker’s mother before. Indeed, the demonstrative has no function here for the purpose of identification, and could have been dropped. Every human being has a mother, and everybody knows this. Appealing to this kind of knowledge would undermine the whole idea of this function, since as humans we share a lot of common knowledge, which generally is not pointed out by grammatical markers. The demonstrative in \REF{ex:johannessen:9} has only one function, which is that of conveying a negative attitude to the referent. Himmelmann’s and Diessel’s, and hence Lie’s, appeal to shared knowledge does not work here. 

Also, \citet[210, 236]{Himmelmann1996} stresses that the recognitional demonstrative is unique and mentions the referent for the first time and above we saw the characterisation by \citet{Diessel1999Book} (and \citealt{Lie2010}): \textit{discourse} \textit{new} \textit{and} \textit{hearer} \textit{old}. But not only does this not explain examples like \REF{ex:johannessen:9}, it is also descriptively incorrect. We will see many examples in \sectref{sec:johannessen:4} and \sectref{sec:johannessen:5} of the PDD being used repeatedly in short texts or discourses. Furthermore, the PDD can be used by both discourse participants for the same referent. This would also be difficult to explain if activation of common ground was the main purpose.

It seems that \citet{Himmelmann1996}, \citet{Diessel1999Book} and \citet{Lie2008,Lie2010} all see the PDD as discourse structuring elements, while Johannessen sees it is as expressing subjectivity. \citet{Diessel2006} sees such demonstratives as coordinating the interlocutors’ joint focus of attention. Thus, unlike spatial demonstratives, the deictic centre is shifted from the physical world, i.e. the speaker’s location at the time of the utterance, to a particular point in the unfolding discourse \citep[475]{Diessel2006}. Johannessen’s theory instead holds that there is a psychological space in which there is deixis, and where it is crucial to single out the distant psychological distance.\footnote{Proximal or neutral psychological distance are not expressed by a demonstrative, though it could be argued that the preproprial article, in those dialects that have it, has a proximal function (see \citealt{Johannessen2008}: 170).} 

Demonstratives that appeal to the speaker and/or hearer also appear in languages unrelated to the Scandinavian ones. \citet{SchapperRoque2011} describe discourse demonstratives in Papuan languages. \citet{EvansEtAl2018} use the term “engagement” to refer to grammaticalised means for encoding the relative mental directedness of speaker and addressee towards an entity or state of affairs. These systems “express the speaker’s assumptions about the degree to which their attention or knowledge is shared (or not shared) by the addressee” \citep[110]{EvansEtAl2018}. Clearly the PDD in the Scandinavian languages is part of an engagement system, with respect to shared knowledge of a referent. Condition 1 would be characterised as \textsc{–spkr.engag} (the speaker does not know the referent, and it is irrelevant whether the addressee does), while Condition 2 would be \textsc{+spkr–addr.engag} (the speaker knows the referent, but the addressee does not). Condition 3 would be \textsc{+spkr.engag} (the speaker knows the referent, and it is irrelevant if the addressee does). 

\section{Method and material}\label{sec:johannessen:3}

In the following, I will investigate whether \citet{Lie2010} is right, since he explicitly opposes Johannessen’s analysis that the demonstratives are first and foremost used to show psychological distance. The way to do this, I suggest, is to look for discourses in which the PDD is used, and check empirically whether it has been used more than once in a discourse, or vice versa, whether there are contexts in which it ought to have been used as referent-invoking, but was not. If a text or discourse has multiple occurrences of the PDD for the same referent, it would indicate that Lie and also \citet[vol. 2: 317]{TelemanEtAl1999} are wrong, and that the PDD does not have to be “discourse new”. Furthermore, repeated use of the PDD to refer to the same person would probably be annoying for the listener, if indeed its function were to bring a referent into the discourse. If it does turn out that multiple uses of a PDD for the same referent can be found, it will also be central in demonstrating that the PDD expresses some psychological distance, as described by the three conditions in \sectref{sec:johannessen:2.1}. Finally, if both discourse participants use the PDD for the same referent, it would be compatible with an explanation in terms of psychological distance (Condition 3), but not one in terms of invoking a referent from the common ground.

Finding appropriate examples of multiple PDDs has not been very easy. First, they are clearly a very oral phenomenon, regulating the emotional relationship between one or more of the interlocutors and other people mentioned. This means that written sources are only possible as long as they imitate spoken dialogue, such as in fiction. Even in fiction they are rare, because they are such an oral phenomenon; many authors probably do not think of them since they are not used to seeing them in writing. Transcriptions of spoken language are perhaps a better source, but only usable as long as they consist of dialogue, and of course, as long as the interlocutors speak about other people (while it may be possible to use the PDD with non-humans or non-animates, this has not been described in the literature). Second, both written and spoken language have in common the fact that whole noun phrases are not usually repeated over and over in a text or a discourse, since the speakers tend to use pronouns in these cases. 

In spite of these challenges, there are some good sources. There are PDDs in two short stories by the acclaimed author Ingvild H. Rishøi. She has received numerous prizes for her work, including for her able use of the Oslo vernacular (\textit{hverdagsspråket}). I have also found empirical evidence in two spoken language corpora. The NoTa-Oslo Corpus \citep{JohannessenHagen2008} is a corpus of 900,000 words of Oslo speech recorded in 2005, with the web-based user-interface Glossa \citep{NøklestadEtAl2017}, which makes it possible to search in the transcribed speech with audio and video presentations. It consists of carefully selected informants who together form a representative sample of the population, with respect to age, gender, educational and work background and location within Oslo, and contains semi-structured interviews and unstructured dialogues. The Nordic Dialect Corpus (NDC, \citealt{JohannessenEtAl2009, JohannessenEtAl2014}) has a Norwegian section consisting of 2.4 million words of spoken dialect dialogues. This corpus is also available with the Glossa search interface. Both corpora are available to researchers worldwide, and there is a password requirement giving most university employees the option to log in via their own university login system.\footnote{The data from the short stories and the corpora are referred to in the following way:\newline \textit{Rishøi 2014: 66, middle}: A short story by Ingvild Rishøi, published in a volume in 2014, page 66, in the middle of the page \newline \textit{NoTa\_027}: From the NoTa corpus (Oslo), informant no. 027 \citep{JohannessenHagen2008} \newline \textit{TAUS\_a43}: From the TAUS corpus (Oslo), informant no. a43 \citep{HanssenEtAl1978} \newline \textit{NDC\_kvam\_04gk}: From the Nordic Dialect Corpus, at the location Kvam, informant no. 04gk \citep{JohannessenEtAl2009,JohannessenEtAl2014}}

Due to the challenges mentioned, it should be emphasised that there are not masses of examples of multiple PDDs. The examples that are found will each have to be studied and explained, and as we shall see, they provide evidence for determining which account is empirically correct. I should add that, as a native speaker, I find all the dialogues from the corpora and the language in the short stories fully natural.

\section{Investigating two short stories}\label{sec:johannessen:4}

In this section, I will investigate two short stories by Ingvild H. Rishøi in order to find evidence for or against Lie’s claim that it is background deixis that is the main function of the PDD, i.e. that it is used to activate specific background knowledge, fetching referents into the discourse. If this is its main function, we will not expect to find more than one PDD in a discourse.

\subsection{\textit{The} life and death of Janis Joplin}\label{sec:johannessen:4.1}

The short story \textit{The life and death of Janis Joplin} \citep[183-206]{Rishoi2014} is about a 17-year-old girl. The girl, we understand, has problems with both school (she has reading difficulties, has a work practice placement, and wants to quit school permanently) and life. We follow her through her plans and her practice as a lumber jack, her love for her friend David, whom she met through her work practice, to the end, when she dies under a falling tree. The story is communicated through her thoughts and dialogues. 

Central in the girl’s story is an advisor at school, a man she often thinks of. He is referred to in her thoughts, and also with some cited dialogues. We understand from the story that he has made a deep impression on her. For example, his remark that she looks like Janis Joplin makes her want to start dressing like Janis Joplin and buy her CDs, which she listens to virtually all day and night, even as she is dying. Later in the story we learn that she despises this man (he has been unfaithful to his wife, whom he has forced to have an abortion, and probably worst of all, he has forgotten who our girl is). The story is written in an oral, lower register.

Since “discourse” is a central concept in the present chapter, it should be properly defined with respect to the data. This short story is written in a first person narrative (by the girl), and the story vacillates between the present time, which we can call speech time, when she is out in the forest cutting trees, and various episodes in the past at school, involving the advisor directly or indirectly, as well as episodes with other people, especially her friend David. Since the whole story is only 23 pages long, it could be seen as one long discourse (between the girl and us, the readers), but it is also possible to distinguish episodes, defined as sections that belong to the same time and space. In that case, the speech time would also be divided into different discourses. An alternative might be to regard the speech time as one long discourse, while the past episodes are each regarded as separate sub-discourses. The perspective taken is not arbitrary, since it will be decisive in regarding the discourse function of the demonstrative. However, the text itself guides us in how to divide the text into discourses: The girl’s friend David is mentioned multiple times by name, both in the past episodes of the short story and the present speech time. This would have seemed very unnatural had the speech time been one continuous discourse, since it would then have been more natural just to refer to him by a pronoun. I shall therefore regard each stretch of text that belongs to the same time as one discourse. This gives us many small discourses, which will make the task of finding several PDDs with the same referent within one discourse more difficult, and hence also harder to find evidence for \citegen{Johannessen2008} view.

There are 22 discourses, of which five involve the mentioning of the advisor, and 17 other discourses. The advisor is referred to six times by a non-pronominal noun phrase, five of them with a PDD, see \REF{ex:johannessen:10}-\REF{ex:johannessen:14}.

\ea\label{ex:johannessen:10}
 \gll ... men jeg satt på kontoret, og \textbf{han} \textbf{rådgiveren} løfta mappa mi og pusta dypt.\\
     {} but I sat in office.\textsc{def} and he advisor.\textsc{def} lifted file.\textsc{def} my and breathed deeply\\
\glt ‘… but I sat in the office, and that advisor lifted up my file and breathed deeply.’ \citep[186, top]{Rishoi2014}
\z

\ea\label{ex:johannessen:11}
 \gll ... jeg håper det er \textbf{han} \textbf{fyren} med bikkja som får de brevine.\\
     {} I hope it is he guy.\textsc{def} with dog.\textsc{def} who gets those letters\\
\glt ‘… I hope it is that guy with the dog that received those letters.’ \citep[187, bottom]{Rishoi2014}
\z

\ea\label{ex:johannessen:12}
 \gll ... da \textbf{han} \textbf{fyren} kom til tredje vers.\\
     {} when he guy.\textsc{def} came to third verse\\
\glt ‘… when that guy got to the third verse.’ \citep[188, top]{Rishoi2014}
\z

\ea\label{ex:johannessen:13}
 \gll Og da han ringte, \textbf{han} \textbf{rådgiverfyren}, så var det allerede for seint. \\
     and when he phoned, he advisor.guy.\textsc{def} then was it already too late\\
\glt ‘And when that advisor guy phoned, it was already too late.’ \citep[190, bottom]{Rishoi2014}
\z

\ea\label{ex:johannessen:14}
 \gll Og etter tre måneder måtte jeg inn på rådgiverkontoret igjen, og da var \textbf{han} \textbf{fyren} tilbake med skjegget sitt og hunden sin … \\
     and after three months must I in on advisor’s.office\textsc{.def} again, and then was he guy.\textsc{def} back with beard.\textsc{def} his and dog.\textsc{def} his\\
\glt ‘And after three months I had to go back to the advisor’s office again, and that guy was back with his beard and his dog …’ \citep[198, top]{Rishoi2014}
\z

The five occurrences of the advisor mentioned by the demonstrative plus a noun are each found in their own discourse, which, isolated, could mean that they were examples of the function of introducing or invoking a referent in the discourse. But if the demonstrative had this as its main function, we would expect the other protagonists, too, to be introduced by the demonstrative. This does not happen, however. There are many other people that are mentioned by name or by a noun phrase in the story, and they are never introduced by the demonstrative. 

Some of the people referred to in the text are \textit{sjefen} ‘the boss’ and \textit{frøkena} ‘the teacher’, \textit{kattedama} ‘the cat woman’ and \textit{mamma} ‘mummy’. The advisor is also mentioned by his name, \textit{Anders}, without a demonstrative. One important person who is close to the protagonist is referred to several times by a single name, \textit{David} (and by pronouns); the girl obviously has no wish to distance herself from him. Also, others are mentioned, like \textit{Janis} \textit{Joplin} and \textit{Leonard} \textit{Cohen}. David is actually mentioned 25 times, and never introduced by a demonstrative, even though several of these times are in individual discourses, and not even close to each other: The first and second occurrences are in a discourse each on p. 187, but the third is not until p. 193, with eight discourses in between them. 

If the main function of the PDD were to invoke a referent from the interlocutors’ common ground, we would expect this to be done for all the people who take part in this story. This does not happen. If only the important ones were to be invoked, i.e. those who are central in our girl’s story, we would expect not only the advisor, but also the narrator’s close friend David to be referred to by a PDD. This does not happen either. The main themes of the story are those that revolve around the advisor and David, but only one of them is referred to by the PDD. Clearly it is not fetching from the common ground that is the purpose of the PDD, since both characters should then have been introduced by it. What the PDD signals to us is instead that there is something not good about the advisor. It lets us understand that the narrator does not like him. Indeed, the author uses this demonstrative to make us understand what feelings the young girl has towards the advisor. 

\subsection{ \textit{Jimmy} \textit{sa} ‘Jimmy said’}\label{sec:johannessen:4.2}

The short story \textit{Jimmy} \textit{sa} ‘Jimmy said’ \citep[51-77]{Rishoi2014} is about Jimmy, a young man who is looking back on his childhood and youth. It is told in the first person, mostly through spoken dialogue to a female work mate. We learn that his childhood was nice as long as his much older sister lived at home, and that they had a good relationship. She was blonde and they clearly looked different from each other, since he has black eyes, and is, we understand, dark-haired. He thinks he is adopted, but then he finds a photo of himself as a baby, with his sister. He confronts his mother with the idea that he is adopted, and she turns away. Then she confirms it, which leads him to start behaving violently and abusively. Later his sister comes home, and it turns out she is on drugs. Later again, his sister has moved into a flat with a man, Lars Arild, and when Jimmy visits them, she asks if he wants to move in with them, “as their son”, she explains. Jimmy runs out, and is upset for a long time. A couple of weeks later his sister dies of an overdose. And some time later Jimmy realises the truth that his sister was his biological mother. 

Although the story is short, it is useful to divide it into several discourses. The whole story is a monologue, with intermittent questions and answers to the work mate, and with subdialogues within the bigger monologue. I will count each part that belongs to the same temporal unit as a discourse, which gives 70 small discourse units. Again, this way of counting discourses makes the task of finding several PDDs in one discourse difficult, but it also would be a very convincing argument against the “hearer new” perspective.

Three people are referred to by the PDD several times. His teacher, Bente, is mentioned twice as \textit{læreren} \textit{min} ‘my teacher’ (without any demonstrative), and is then introduced by name, \REF{ex:johannessen:15}, after which she is referred to four times, either by name or by work title, but always with the PDD, \REF{ex:johannessen:16}-\REF{ex:johannessen:19}. The first and the last occurrence, \REF{ex:johannessen:16} and \REF{ex:johannessen:19}, are mentioned once each in a discourse unit, while \REF{ex:johannessen:17}-\REF{ex:johannessen:18} are in the same discourse.

\ea\label{ex:johannessen:15}
 \gll Det var en ganske grei lærer egentlig, Bente het hu.\\
     it was a quite nice teacher actually Bente called she\\
\glt ‘It was a quite nice teacher, actually, Bente was her name.’ \citep[53, middle]{Rishoi2014}
\z

\ea\label{ex:johannessen:16}
 \gll Og \textbf{hu} \textbf{Bente} fortalte at dem hadde blitt sjekka flere ganger, …\\
     and she Bente said that they had been checked several times\\
\glt ‘And that Bente said that they had been checked several times ...’ \citep[54, top]{Rishoi2014}
\z

\ea\label{ex:johannessen:17}
 \gll Men jeg snakka med hu læreren jeg fortalte om.\\
     but I talked with she teacher.\textsc{def} I told about \\
\glt ‘But I talked with that teacher I told you about.’ \citep[57, line 2]{Rishoi2014}
\z

\ea\label{ex:johannessen:18}
 \gll Men \textbf{hu} \textbf{Bente} sto oppe ved tavla og fortalte om henne.\\
     but she Bente stood up by blackboard.\textsc{def} and told about her \\
\glt ‘But that Bente stood by the blackboard and told about her.’ \citep[57, line 5]{Rishoi2014}
\z

\ea\label{ex:johannessen:19}
 \gll Så da sparka jeg ned alt mulig og fikk melding med hjem hele tida, av \textbf{henne} \textbf{Bente} også.\\
     so then kicked I down everything possible and got message with home all time.\textsc{def} from her Bente too. \\
\glt ‘So then I kicked down all sorts of things and got a letter to take home, from that Bente, too.’ \citep[61, bottom]{Rishoi2014}
\z

It is clear that Jimmy knows his teacher, Bente, and also that he likes her, which he explicitly says the first time he introduces her by name, in \REF{ex:johannessen:15}. When the PDD is used with this person, it must be by Condition 2. Jimmy knows that the addressee does not know her, is polite and chooses to acknowledge this, and uses the PDD for the sake of the addressee. Since two occurrences, \REF{ex:johannessen:17}-\REF{ex:johannessen:18}, are not only in the same discourse, but only two lines and two small sentences apart (and where both sentences also are about the teacher), \citegen{Lie2010} account cannot explain this use. Invoking a referent that is already activated does not make sense.

The father is referred to twice, both times with the demonstrative, seen in \REF{ex:johannessen:20}-\REF{ex:johannessen:21}.

\ea\label{ex:johannessen:20}
 \gll Jeg drømte om \textbf{han} \textbf{der} \textbf{faren} min i Romania.\\
     I dreamed about he there father.\textsc{def} my in Romania \\
\glt ‘I dreamed about that father of mine in Romania.’ \citep[56, bottom]{Rishoi2014}
\z

\ea\label{ex:johannessen:21}
 \gll … og jeg så for meg \textbf{han} \textbf{der} \textbf{faren} \textbf{min}.\\
     {} and I saw for me he there father.\textsc{def} my\\
\glt ‘… and I imagined that father of mine.’ \citep[65]{Rishoi2014}
\z

Jimmy has never met his father, does not know who he is, and knows nothing about him. He exists only in his imagination, he assumes his father is from Romania, presumably a country where people look like him. Here the PDD is strengthened by the distal adverb \textit{der} ‘there’.\footnote{The presence of the distal adverb \textit{der} ‘there’ is not obligatory, and removing it would not change the truth conditions or the psychological distance of the phrase. Although the interplay between the PDD and the distal adverbs has not been studied in detail, I interpret these adverbs to further strengthen the distal meaning of the PDD. There is also a proximal adverb \textit{her} ‘here’, which could plausibly be used, but it would not make the noun phrase psychologically proximal. There may be dialectal variation as to whether the proximal adverb can be used with PDDs. (For more on complex demonstratives, see \citealt{Vindenes2017}.)} Even though it is true that the addressee does not know his father, it is most likely the narrator’s own distant, actually non-existent, relationship to the father that is in focus. It is a good example of Condition 1 of the PDD: The speaker does not personally know the person referred to.

Notice that the imaginary father is unique in the story. Jimmy only has one father, and there is no other father in the story. This is also the case for the mother and the sister, but in the case of the father, Jimmy chooses to make a point of the distance; that he does not actually know him, otherwise it might seem false, that he would try to make the father closer than he actually is. This explains the use of the PDD. Lie’s account could also be used for the two occurrences of the father – not in the same discourse units – but would fall short of explaining why the other relatives and people in the story are not all invoked this way.

The most problematic figure for Jimmy in this story is his sister’s cohabitant, Lars Arild. He is referred to by Jimmy six times, \REF{ex:johannessen:22}-\REF{ex:johannessen:27}. The first three, \REF{ex:johannessen:22}-\REF{ex:johannessen:24}, are mentioned once each in separate discourse units, but the last three are in the same discourse.

\ea\label{ex:johannessen:22}
 \gll Jo, jeg hata \textbf{han} \textbf{typen} \textbf{hennes} skikkelig.\\
     yes I hated he boy.friend.\textsc{def} her really\\
\glt ‘Yes, I really hated that boyfriend of hers.’ \citep[65, bottom]{Rishoi2014}
\z

\ea\label{ex:johannessen:23}
 \gll Nei, jeg var ikke der så ofte, på grunn av \textbf{han} \textbf{fyren}.\\
     no I was not there so often on reason of he guy.\textsc{def} \\
\glt ‘No, I wasn’t there that often, because of that guy.’ \citep[66, middle]{Rishoi2014}
\z

\ea\label{ex:johannessen:24}
 \gll Jo, jeg likte jo ikke \textbf{han} \textbf{typen}.\\
     yes I liked yes not he guy.\textsc{def} \\
\glt ‘Yes, I didn’t like that guy, as you know.’ \citep[67, top]{Rishoi2014}
\z

\ea\label{ex:johannessen:25}
 \gll “Lars Arild, kan du gi Jimmy litt saft?”, og så henta \textbf{han} \textbf{typen} saft ... \\
     Lars Arild can you give Jimmy some squash and then fetched he guy.\textsc{def} squash \\
\glt ‘“Lars Arild, can you give Jimmy some squash,” and then that guy fetched squash ...’ \citep[68, middle]{Rishoi2014}
\z

\ea\label{ex:johannessen:26}
 \gll ... og så reiste \textbf{han} \textbf{Lars} \textbf{Arild} seg igjen.\\
    {} and then rose he Lars Arild himself again\\
\glt ‘… and then that Lars Arild rose again.’ \citep[68, middle]{Rishoi2014}
\z

\ea\label{ex:johannessen:27}
 \gll Og da blei søstera mi helt tårer i øya, og \textbf{han} \textbf{Lars} \textbf{Arild} snudde seg.\\
     and then became sister.\textsc{def} my totally tears in eyes.\textsc{def} and he Lars Arild turned himself\\
\glt ‘And then my sister became all tears in her eyes, and that Lars Arild turned.’ \citep[69, top]{Rishoi2014}
\z

Lars Arild has been mentioned several times over these pages, and the PDD is used each time Jimmy mentions him. This is clearly because Jimmy does not like him, as he has said explicitly throughout the story. By using the PDD, he is able to indicate this distance every time. So although the PDD would not be strictly logically necessary, given that Jimmy has been explicit about his dislike of Lars Arild, it is used so that we are constantly reminded of his negative feelings. 

It is central to point out that the last three occurrences are not only in the same discourse, but that in \REF{ex:johannessen:25}, the PDD is in fact in the same sentence as another mention of the same person, when Jimmy’s sister speaks directly to Lars Arild, immediately followed by Jimmy’s description of Lars Arild’s action. 

Again, an account of the function of this demonstrative as one invoking a referent into the discourse cannot be right. The referent is already as foregrounded as it is possible to get. 

Jimmy’s use of the PDD with Lars Arild is in sharp contrast to the one time where his sister mentions her boyfriend \REF{ex:johannessen:28}.

\ea\label{ex:johannessen:28}
 \gll Jeg og Lars Arild har tenkt å gifte oss.\\
     I and Lars Arild have thought to marry ourselves\\
\glt ‘Lars Arild and I are planning to get married.’ \citep[68, middle]{Rishoi2014}
\z

His sister can be assumed to love her boyfriend, and naturally does not use the PDD when she talks about him.

Finally, one non-central person is mentioned only in one discourse, but then with the PDD. This is the taxi driver who found him when he had run away from home \REF{ex:johannessen:29}.

\ea\label{ex:johannessen:29}
 \gll Så \textbf{han} \textbf{sjåføren} hadde hørt at han skulle se etter en trettenåring .... \\
     so he driver.\textsc{def} had heard that he should look after a thirteen.year.old\\
\glt ‘So that taxi driver had heard that he should be on the lookout for a thirteen-year-old ...’ \citep[70, middle]{Rishoi2014}
\z

The taxi driver is a peripheral character, and is only mentioned in one discourse. When the PDD is used here, it is due to Condition 1. 

In this story, there are some people who are referred to without the PDD. These are people who we understand Jimmy thinks are nice and close to him. His sister, referred to by him as \textit{søstera} \textit{mi} ‘my sister’ or \textit{Linn}, is referred to 50 times with a noun phrase, and never with a PDD, though there are of course many times that she is mentioned for the first time in a discourse, and sometimes with long stretches between, such as from the middle of p. 58 to the top of p. 60, with six discourses in between, dealing with aspects of his adoption. Another person who is never mentioned with the PDD is his mother, who is mentioned 30 times with a noun phrase: \textit{mora} \textit{vår} ‘our mother’, \textit{mora} \textit{mi} ‘my mother’, and \textit{modern} ‘mum’, and also his work colleague \textit{Freddy}. 

There are three central characters referred to by the PDD, and they represent all three conditions of use. Jimmy’s teacher, Bente, is referred to in accord with Condition 2: The addressee does not personally know the referent. This is done for politeness; Jimmy acknowledges that the addressee does not know her. His father is referred to by Condition 1: The speaker does not personally know the person referred to. This is also a function of politeness, but moreover, of honesty. Normally when people talk about their parents, there is an understanding between the interlocutors that there is a normal parent-child relationship involved. Since Jimmy neither knows his father nor anything about him, it would be almost dishonest to talk about him without the PDD, especially when the addressee is somebody he respects and wants to be honest to. The PDD is therefore necessary in this context. The sister’s boyfriend Lars Arild is a person he dislikes intensely, as we have got to know throughout the story, both explicitly and implicitly by learning about Jimmy’s behaviour when this person is mentioned. Every time the boyfriend is mentioned by Jimmy, it is with the PDD, and this is obviously in accord with Condition 3: The speaker has a negative attitude to the person referred to. Jimmy finds it necessary to distance himself from him every time he mentions him. Crucially, the sister does not use the PDD when referring to Lars Arild, who is a person she likes.

Interestingly, the PDD is used with different functions for the different people, but it is never difficult to understand which is applied with which person. On the contrary, the demonstrative helps us to interpret the relationship between Jimmy and the others. 

If the function of the PDD were to fetch referents from the common ground, we would not expect some characters to be mentioned by it every time and others never (mother, sister, work colleague). Neither would we expect others to have to be fetched from the common ground only two lines apart (as with the teacher, \REF{ex:johannessen:17}-\REF{ex:johannessen:18}). 

\section{The use of the PDD in corpora of spoken conversations}\label{sec:johannessen:5}

Although the use of the PDD in the short stories in \sectref{sec:johannessen:4} shows quite clearly that its function is not to invoke referents from the common ground, it still remains to be seen whether this is also the case in genuine spoken language. To investigate this, it is useful to look at corpora of naturalistic, spoken dialogue. If the fictional dialogues in the short stories reflect the linguistic facts of the language, we should find equivalent patterns in the corpora, i.e. cases in which the PDD is used repeatedly with referents already invoked in the discourse. This is also what we find. We even find discourses where both interlocutors use the PDD for the same referent.

In dialect recording \REF{ex:johannessen:30} an old woman from South Norway is interviewed about school, and the excerpt is from the beginning of a conversation about school times. The two occurrences of \textit{hænn} \textit{lerarn} (he teacher.\textsc{def} \textsc{‘}the teacher’) are the only two mentioning the teacher. 

\ea\label{ex:johannessen:30} {(NDC\_kvam\_04gk)}
\ea {[kvam\_04gk:] visst e skulle lesa lekksa mi ælle no sjlikkt å \textbf{hænn lerarn} va strænng å hænn va nynåssjkær å de bøken våre va itte de}\\
\glt ‘if I was going to read my homework or something, and \textbf{that} \textbf{teacher} was strict, and he was a Nynorsk person and those books of ours were not’
\ex {[interviewer:] nei}\\
\glt ‘no’
\ex {[kvam\_04gk:] næi menn ass åss hadde ei ænaste bok somm hæte “Såga omm fåLLke vårrt” å va på nynåssjk ællers så va de bokmåL åss hadde å \textbf{hænn lerarn} hænn da skull e lessa et stykkji}\\
\glt ‘no, but we had only one book that was called “The story of our people” that was in Nynorsk, otherwise it was Bokmål we had, and \textbf{that teacher} he was going to read a story’\footnotemark{}
\z
\z
\footnotetext{Nynorsk and Bokmål are the two different written standards for Norwegian.}

This excerpt shows that the woman uses the PDD the first time the teacher is mentioned, which could in isolation be interpreted as invoking him from the common ground; both interlocutors know that there is a teacher in a school situation. But then the PDD is used in the next sentence, too, making it very unlikely that there is invoking going on. Instead, the PDD can be understood by Condition 2: The referent is unknown to the addressee, and the speaker is politely acknowledging this fact by using the PDD.

In \REF{ex:johannessen:31}, an old woman (stamsund\_04gk) and an old man (stamsund\_03gm) from North Norway are talking about the old times when they played music in a school band, and the woman then introduces a Dane into the discourse: 

\ea\label{ex:johannessen:31} {(NDC\_stamsund\_04gk and NDC\_stamsund\_03gm)}
\ea {[stamsund\_04gk:] va kje du mæ då \textbf{hann dannsken} va her å starrta opp} \\
\glt ‘were you not there when \textbf{that Dane} was here and started up’
\ex {[stamsund\_03gm:] jo va de ja e sjlutta då hann e va mæ en par år}\\
\glt ‘yes, I was, I stopped when he I was in it for a couple of years’
\ex {[stamsund\_04gk:] ja e va mæ di to føssjte åran ætte att \textbf{hann dannsken} starta opp jænn ætte de ha lie nere}\\
\glt ‘yes, I was in it the first two years after \textbf{that Dane} started up again after it had been down’
\ex {[stamsund\_03gm:] mm}\\
\glt ‘mm’
\ex {[stamsund\_04gk:] de va ennu hann farr din so kåmm inn i i i leiliheta åt åss \# å læmmpa inn en en kornætt å roppt “de herran de e te kjerringa n- må du bærre ha ho te å øv”}\\
\glt ‘it was even your father who came into our flat and dumped a cornet and shouted “this one is for the woman, now you have to get her to practice”’
\ex {[stamsund\_03gm:] mm}\\
\glt ‘mm’
\ex {[stamsund\_04gk:] då bodd vi uttaførr} \\
\glt ‘then we lived further away’
\ex {[stamsund\_03gm:] mm ja}\\
\glt ‘mm yes’
\ex {[stamsund\_04gk:] å førr att eg ha jo kje a me sjøl i de heile tatt å me de va jo mått jo bynn å træn opp, menn dær uttaførr veit du i denn gammle nolannsbanngken jikk jo kje ann å håll på øve de va jo sækks leilighete} \\
\glt ‘and I didn’t choose it myself and I had to start practicing, but over there you know in the old bank one couldn’t practice there as there were six flats there’
\ex {[stamsund\_03gm:] nei \textbf{hann dannsken} hann ee hann va då e va}\\
\glt ‘no, \textbf{that Dane} he ehm he was’
\z
\z

Here, too, the PDD is used when the Dane is first introduced, but then the old woman continues to use the PDD in her next sentence as well, which would be surprising if the main function were to invoke the referent, who has just been invoked. The man also uses the PDD after three utterances by the woman. Again, the Dane is now well established in the discourse, so there is no need to further invoke him. Since they both use the PDD, it is evident that they do not invoke the referent for each other. Instead, the most likely interpretation is that both speakers for some reason distance themselves from this outsider from Denmark, and use the PDD (Condition 3) for this purpose.

In \REF{ex:johannessen:32} we see an excerpt from a conversation between two 18-year-old men from Oslo. One introduces a new person, a German, into the discourse. Because of some laughter and self-interruptions there are some repetitions by this speaker. Crucially, every time the German (a cannibal) is mentioned, he is mentioned with the PDD both by the speaker and by his co-interlocutor. 

\ea\label{ex:johannessen:32}
\ea {[NoTa\_016:] men hva med \textbf{han derre}, leste du om \textbf{han... derre}, \textbf{han tyskeren}, som hadde kuttet av utstyret på en fyr og spist det} \\
\glt ‘but what about \textbf{that there}, did you read about \textbf{that there}, \textbf{that German}, who had cut off the equipment of a guy and eaten it’
\ex {[NoTa\_015:] ja ja ja}\\
\glt ‘yes yes yes’
\ex {[NoTa\_016:] \textbf{han derre kannibalen}}? \\
\glt ‘\textbf{that there cannibal}’
\ex {[NoTa\_015:] \textbf{han kannibalen} ja} \\
\glt ‘\textbf{that cannibal} yes’
\z
\z

The German (the cannibal) is referred to with a PDD each time by both interlocutors. There is no other utterance in between the three times the PDD is used as part of a noun phrase. This would be very surprising if what they are doing is to invoke the referent from their common ground in every utterance and for each other. Instead, the use of the PDD here is to mark distance from themselves to this weird German (Condition 3). (For the function of the adverb \textit{der} ‘there’, please see the comments in \sectref{sec:johannessen:4.2}, on Jimmy’s father.)

In \REF{ex:johannessen:33}, two girls, 18 and 19 years old, from the Oslo area, are talking about exams, and then turn to the topic of one of them, a short story. One of the girls starts discussing the main character in that story, using the PDD. In her next utterance she also uses the PDD about the same character.

\ea\label{ex:johannessen:33}
\ea {[NoTa\_093:] men den var så dum den novellen, \textbf{hu jenta} var så rar} \\
\glt ‘but it was so stupid that short story, \textbf{that} \textbf{girl} was so strange’
\ex {[NoTa\_094:] ja, den var så dårlig altså jeg klarte å skrive to og en halv side til slutt} \\
\glt ‘yes, it was so bad really, I managed to write two and a half pages in the end’
\ex {[NoTa\_093:] ja, \textbf{hu jenta} det handla om var teit} \\
\glt ‘yes, \textbf{that girl} it dealt with was stupid’
\ex {[NoTa\_094:] ja}\\
\glt ‘yes’
\z
\z

Since the PDD is used in two utterances in a row, there seems little reason to argue that invoking the referent is the function of this PDD. Instead, this use is in accordance with Condition 3: The speaker has a negative attitude to the person referred to. The speaker here clearly wants to distance herself from the referent, whom she characterises as strange and stupid. 

In this section, we have investigated four short dialogues, i.e. four discourses, from real conversations, referring to a previous teacher, a Dane, a cannibal and a fictional character. In all of them, a PDD with the same referent was used several times in one discourse. In two of them, the PDD was used for the same referent by both interlocutors. In all cases the use of the PDD could be explained by reference to one of the three conditions in \citet{Johannessen2008}.

\section{Summary and conclusion}\label{sec:johannessen:6}

\citet{Lie2010} claims that the function of the demonstratives \textit{hun} ‘she’ and \textit{han} ‘he’ is to invoke a referent from the common ground of the interlocutors, following \citegen{Diessel1999Book} principle “discourse new and hearer old”, while \citet{Johannessen2008} says that these demonstratives in Scandinavian are used to indicate psychological distance towards some other person referred to, hence the term “psychologically distant demonstratives”. 

The aim of the present chapter was to test whether Lie’s analysis could be correct, and if not, whether Johannessen’s analysis would be. As empirical material, two short stories by the acclaimed author Ingvild Rishøi were chosen: \textit{The life and death of Janis Joplin} and \textit{Jimmy sa}. Both involve several people, of whom one or only a few are referred to by the PDD. In addition, four short dialogues from speech corpora containing spontaneous conversations were chosen (the NoTa-Oslo Corpus and the Nordic Dialect Corpus). Below I summarise the main findings. They all show that \citet{Lie2010} cannot be right.

(i) The PDD are used for more than one mention per discourse: If the PDD were used to invoke a referent from a common ground, we would expect that referent to be mentioned at most once per discourse. Rather than regarding each short story as one discourse, I chose to divide them into several discourses, where each was a stretch of text coherent in space and time. This way of counting gave a high number of discourses per story (23 in the Janis Joplin story, and 70 in the Jimmy story), and therefore a low chance of finding the same character referred to by a PDD several times in one discourse. Even so, we did find this. In the story \textit{Jimmy sa}, Jimmy’s teacher Bente is referred to twice (of four total times) in the same discourse (three lines apart), while Lars Arild is referred to by him six times, with the PDD every time, three occurrences in the same discourse, and one of them even in the same sentence as another mention of the same referent. 

In all four speech corpus dialogues, the PDD is used for the same referent, sometimes in the same sentence (the dialogue about the cannibal), or with one sentence (the dialogue about the Dane), or even a one-word utterance (the dialogue about the teacher, and about the fictional character) apart. 

(ii) The PDD are used by different interlocutors within the same discourse: If the PDD had a purely discourse regulating function, to bring a referent into the discourse from the common ground, we would not expect to see it used with one and the same referent by both interlocutors. However, in two of the naturalistic conversations, both interlocutors use the PDD (about the Dane and about the cannibal).

(iii) Not all referents are referred to by the PDD: If invoking referents into the discourse were the main function of the PDD, we would expect to find it with all characters who are not activated. However, this is not the case in the empirical material we have investigated here. Both the short stories contain many characters not referred to by the PDD: the girl’s important friend David (25 times), and also her boss, mother, teacher, and somebody referred to as the cat woman. These characters are not all present in the discourse all the time, so they ought to have been fetched into the discourse, if this were the role of the PDD. Instead, they are possibly close to the main character or not central, and can be referred to without any particular marker. In the Jimmy story, there are also people who are not marked by the PDD: his sister (referred to more than fifty times), his mother and his work mate Freddy.

(iv) The PDD does not point out only key referents: It could be asked whether the PDD rather than pointing to referents of psychological distance instead points to referents that are central in the discourse. After all, both the girl’s advisor and Jimmy’s sister’s boyfriend are central in the story. However, there are many others that are central or important, who are not introduced by a PDD. Jimmy’s mother is in this category, as well as his sister. The girl’s friend David is also very important, and presumably her mother. And vice versa, the taxi driver introduced by the PDD in Jimmy’s story is not central. His teacher Bente is not very central either, but she does occur with the PDD. The function of the PDD, therefore, is not to point out the most central people. 

Finally, one could ask whether Lie’s function of invoking a referent from the interlocutors’ background knowledge and Johannessen’s PDD Conditions have a common core. Clearly, whenever a PDD is used for the first time in a discourse, it will invoke a referent. From the discussions above, though, this invoking is not the function of the PDD, because the PDD can be used several times in a short discourse (making the hearer quite annoyed in the end, if invoking were the central function), because both interlocutors can use the PDD about the same referent (and it would be comical if both interlocutors had to remind each other of the referent), and because not all referents are referred to by the PDD, even when they are not central in the discourse. 

In conclusion, functions like background deixis and “discourse new and hearer old” do not cover the empirical facts. On the contrary, the referent could well be “discourse old”, and even “hearer new”, as when the speaker does not know the person referred to personally. The two short stories and four spoken dialogues investigated here show that the use of the PDD can instead be accounted for in terms of psychological distance \citep{Johannessen2008}:\\
\\
\textit{PDD Condition 1: The speaker does not personally know the person referred to.}\\
\textit{PDD Condition 2: The addressee does not personally know the person referred to.}\\
\textit{PDD Condition 3: The speaker has a negative attitude to the person referred to.}

\section*{Acknowledgements}

I would like to thank two anonymous reviewers and  Yvonne Treis for their very constructive comments and suggestions that have greatly improved this paper.

\section*{Corpora}

\small{The Nordic Dialect Corpus: \url{http://www.tekstlab.uio.no/nota/scandiasyn/index.html}\\
The NoTa-Oslo Corpus: \url{http://www.tekstlab.uio.no/nota/oslo/english.html}\\
The TAUS Corpus of Oslo Speech: \url{http://www.tekstlab.uio.no/nota/taus/english.html}}

\sloppy\printbibliography[heading=subbibliography,notkeyword=this]
\end{document}
