\documentclass[output=paper]{langscibook}
\ChapterDOI{10.5281/zenodo.5578828}

%This is where you put the authors and their affiliations
\author{Johannes Hein\affiliation{University of Potsdam}}

%Insert your title here
\title{Subject encoding in Limbum}  
\abstract{This paper presents novel data from the understudied Grassfields Bantu language Limbum (Cameroon) showing three interrelated asymmetries within the realm of subject marking. The first is a dependency of overt subject marking on number and category of the subject. The second concerns the apparent absence of subject resumption for third person plural while it is obligatory in all other cases. The third asymmetry is found with focus-marked subjects where subject marking is dependent on the type of focus-marking. It will be argued that the first asymmetry can be understood in terms of differential subject marking, while the second one is due to a weak/strong distinction in pronouns. The last asymmetry is derived from the first in interaction with a structural ambiguity in subject focus constructions.}

\begin{document}
\SetupAffiliations{mark style=none}
\maketitle

\section{Introduction}
\label{sec:johanneshein:intro}

It is well known that syntactic operations and processes do not
necessarily have to be applicable to all kinds of arguments, nor does
one and the same syntactic operation/process have to have the same
effect on different kinds of arguments. In fact, examples of
asymmetric behaviour of distinct kinds of arguments are
abundant. There are subject/object asymmetries with regard to inter
alia \textit{that}-trace effects \citep{perlmutter71}, sub-extraction
\citep{huang82}, resumption \citep{koopman83,mcCloskey90}, and many
more. Direct and indirect objects behave differently with respect to
scope and binding \citep{barss+lasnik86,larson90}, resumption
\citep{stewart01}, and extraction
\citep{bresnan+moshi90,holmbergetal19}. There are also asymmetries
between arguments and non-arguments for island sensitivity and weak
islands \citep{huang82,engdahl86} and reconstruction
\citep{freidin86,lebeaux88}.

Less well known is the fact that there
can be asymmetric behaviour within one kind of argument. Thus, with
focus marking, matrix subjects show one kind of encoding while embedded
subjects employ a different focus marking strategy in Dagbani
\citep{issah+smith18} and in Igbo \citep{amaechi+georgi19}. The most
well known case of such internal asymmetry is possibly differential
object marking, where objects show a different morphological encoding
depending on some inherent (and sometimes also external)
properties. In the realm of subjects, the most prominent asymmetry is
probably the so-called antiagreement effect
(\citealp{ouhalla93,ouhalla05}, see also \citealp{baier18} for a
recent overview and discussion) which distinguishes subjects that have undergone extraction from in situ subjects by a loss of agreement on the verb (antiagreement) or a different
morphological encoding on the verb (alternative agreement).

In this paper, I will present and discuss three subject-internal
asymmetries in Limbum, a Grassfields Bantu language spoken in North
Western Cameroon, that are, to some degree, interdependent. First,
Limbum shows an asymmetry in the presence of a subject marker. While
this marker is obligatory for full NP and plural pronominal subjects,
it has to be absent when the subject is a singular pronoun. Coupled
with the fact that Limbum requires a resumptive pronoun to occur in
the base position of a subject \Abar-dependency, this leads to an
apparent anti-agreement effect \citep[cf.][]{ouhalla93,baier18}.
Second, there is an asymmetry of third person plural subjects vs. all
other person-number combinations with regard to resumption. While,
generally, subject extraction leaves a resumptive pronoun that is
identical in form to the regular personal pronoun, extraction of third
person plural subjects leaves a gap. However, this gap is only
apparent, because, as I argue, the third person plural is the only one
that has a weak pronoun variant which is null. A third asymmetry concerns
the interaction of the particle \textit{cí}, which occurs in focus
constructions, and the choice of subject marker. It is shown that when
\textit{cí} is overt, there has to be a resumptive pronoun \textit{í},
while there is optionality between the resumptive and the subject marker \textit{à} when
\textit{cí} is absent. This optionality is analysed as stemming from
a structural ambiguity between a movement and a non-movement
configuration. 

The data in this paper stem from a number of elicitation sessions with one native speaker from Nkambe, Cameroon, over a period of several months between August 2018 and May 2019. The sessions took place in Leipzig.

\section{Subject agreement}\label{sec:johanneshein:subject-agreement}\largerpage

Limbum, a Grassfields Bantu language (Niger-Congo) is spoken by
about 73\,000--90\,000 \citep[][21]{fransen95} to 130\,000 speakers
\citep[according to a 2005 census,][]{ethnologue} in the Northwest
Region of Cameroon. Its basic word order is SVO with tense-aspect
markers appearing between the subject and the verb. Adverbs always
take the clause-final position \REF{ex:johanneshein:wordorder}.

\ea \label{ex:johanneshein:wordorder}
\gll Njíŋwὲ fɔ̄ à mū yɛ̄ bō fɔ̄ nìŋkòr.\\
woman \textsc{det} \textsc{sm} \textsc{pst2} see children \textsc{det} yesterday\\
\glt `The woman saw the children yesterday.'
\z

\subsection{The data}

In some tenses and aspects (all three past
tenses and, optionally, in the progressive aspect), a subject marker
\textit{à} obligatorily occurs with the subject \REF{ex:johanneshein:SM}.

\ea\label{ex:johanneshein:SM}
\begin{xlist} 
\ex 
\gll Nfòr \textbf{à} mū zhé bzhɨ́.\\
Nfor \textsc{sm} \textsc{pst2} eat food\\
\glt `Nfor ate food.'

\ex
\gll Nfòr \textbf{à} cɨ́ zhé bzhɨ́.\\
Nfor \textsc{sm} \textsc{prog} eat food\\
\glt `Nfor is eating food.'
\end{xlist}
\z
In other tenses and aspects, like the future \REF{ex:johanneshein:NSMa} or the habitual
\REF{ex:johanneshein:NSMb}, no such subject marker occurs. In fact, the presence of a
subject marker renders the sentence ungrammatical.

\ea
\begin{xlist}
\ex\label{ex:johanneshein:NSMa}
\gll Nfòr (*à) bí zhé bzhɨ́.\\
Nfor \hphantom{(*}\textsc{sm} \textsc{fut1} eat food\\
\glt `Nfor will eat food.'

\ex \label{ex:johanneshein:NSMb}
\gll Nfòr (*à) kɨ́ zhé bzhɨ́.\\
Nfor \hphantom{(*}\textsc{sm} \textsc{hab} eat food\\
\glt `Nfor regularly eats food.'
\end{xlist}
\z
In this paper, I will focus on the tenses
and aspects in which the subject marker is found. Interestingly, the subject marker only occurs with full NP subjects
\REF{ex:johanneshein:SM} and plural pronouns \REF{ex:johanneshein:pluralpron}.\largerpage

\ea\label{ex:johanneshein:pluralpron}
\begin{xlist}
\ex 
\gll Wὲr \textbf{*(à)} mū fàʔ.\label{ex:pl-pron-insitu1}\\
\textsc{1pl.exc} \hphantom{(*}\textsc{sm} \textsc{pst2} work\\
\glt `We(excl) worked.'

\ex
\gll Sì \textbf{*(à)} mū fàʔ.\label{ex:pl-pron-insitu2}\\
\textsc{1pl.inc} \hphantom{(*}\textsc{sm} \textsc{pst2} work\\
\glt `We(incl) worked.'

\ex
\gll Yì \textbf{*(à)} mū fàʔ.\label{ex:pl-pron-insitu3}\\
\textsc{2pl} \hphantom{(*}\textsc{sm} \textsc{pst2} work\\
\glt `You(pl) worked.'
\end{xlist}
\z
For third person plural subjects, both pronouns and full NPs, the
subject marker appears in an exclusively plural form \textit{ó}
\REF{ex:johanneshein:pluralSM}.

\ea\label{ex:johanneshein:pluralSM}
\begin{xlist}
\ex 
\gll Wōyè \textbf{*(ó)} mū fàʔ.\label{ex:johanneshein:3pl-pron-insitu}\\
\textsc{3pl} \hphantom{(*}\textsc{3pl.sm} \textsc{pst2} work\\
\glt `They worked.'
\ex
\gll Bō fɔ̄ \textbf{*(ó)} mū zhé bzhɨ́.\label{ex:johanneshein:3pl-np-insitu}\\
children \textsc{det} \hphantom{(*}\textsc{3pl.sm} \textsc{pst2} eat food\\
\glt `The children ate food.'
\end{xlist}
\z 
However, when the subject is a 1st, 2nd, or 3rd person singular
pronoun, the subject marker \textit{à} is ungrammatical \REF{ex:johanneshein:singularSM}. Thus,
singular pronouns and \textit{à} never cooccur in a clause.\largerpage

\ea\label{ex:johanneshein:singularSM}
\begin{xlist}
\ex 
\gll Mὲ \textbf{(*à)} mū fàʔ.\label{ex:johanneshein:singularSMa}\\
\textsc{1sg} \hphantom{(*}\textsc{sm} \textsc{pst2} work\\
\glt `I worked.'
\ex
\gll Wὲ \textbf{(*à)} mū fàʔ.\label{ex:johanneshein:singularSMb}\\
\textsc{2sg} \hphantom{(*}\textsc{sm} \textsc{pst2} work\\
\glt `You(sg) worked.'
\ex
\gll Í \textbf{(*à)} mū fàʔ.\label{ex:johanneshein:singularSMc}\\
\textsc{3sg} \hphantom{(*}\textsc{sm} \textsc{pst2} work\\
\glt `(S)he worked.'
\end{xlist}
\z 
Concerning the tenses that do not show the subject marker for full
NPs, these also lack it if the subject is a pronoun
(singular or plural). Some examples in the future tense are given in (\ref{ex:johanneshein:future}).

\ea\label{ex:johanneshein:future}
\begin{xlist}
\ex 
\gll Wὲr (*{à}) bí fàʔ.\\
\textsc{1pl.exc} \hphantom{(*}\textsc{sm} \textsc{fut1} work\\
\glt `We(excl) will work.'
\ex 
\gll Mὲ (*{à}) bí fàʔ.\\
\textsc{1sg} \hphantom{(*}\textsc{sm} \textsc{fut1} work\\
\glt `I will work.'
\end{xlist}
\z 
Contrary to what is reported in \citet{fransen95}\footnote{The speaker gave the comment that Fransen's data sound archaic to him but admitted that she might also be describing a different dialect of Limbum. Generally, his data diverge from the data presented by \citet{fransen95} for several phenomena, including relativization and focus.}, who restricts \textit{à} and \textit{ó} to subjects of class 1 and 2 respectively, the subject markers for the speaker consulted here are invariant with regard to the noun class of the subject. That is, both \textit{à} and \textit{ó} occur across different noun classes \REF{ex:johanneshein:nounclasses}. The tense restrictions are the same as discussed above.  

\ea\label{ex:johanneshein:nounclasses}
\begin{xlist}
\ex 
\gll Rtāā fɔ̄ à mū gwê.\\
5.cap \textsc{det} \textsc{sm} \textsc{pst2} fall\\
\glt `The cap fell.'
\ex
\gll Mtāā fɔ̄ ó mū gwê.\\
6.cap \textsc{det} \textsc{3pl.sm} \textsc{pst2} fall\\
\glt `The caps fell.'
\ex 
\gll Nà fɔ̄ à mū būmī.\\
10.cow \textsc{det} \textsc{sm} \textsc{pst2} sleep\\
\glt `The cow slept.'
\ex 
\gll Mnà fɔ̄ ó mū būmī.\\
10.pl.cow \textsc{det} \textsc{3pl.sm} \textsc{pst2} sleep\\
\glt `The cows slept.'
\end{xlist}
\z
Concerning the restriction of \textit{à} to past tenses, this is already noted in \citet[][§10.2]{fransen95} albeit as restricted to class 1/1a subjects. As of now, I unfortunately have no explanation for this constraint.

In summary, the distribution of subject markers is quite asymmetric in
Limbum. First, they only occur in a selection of tenses and
aspects. Second, singular NPs and local person plural pronouns pattern
together in requiring the presence of the \textit{à} marker while
singular pronominal subjects demand its absence. Third person plural
subjects obligatorily appear with the exclusive \textit{ó}
marker. The overall pattern is given in \tabref{tab:johanneshein:SMdistro}.

\begin{table}
\caption{Distribution of subject markers in Limbum past tenses and progressive aspect}
\label{tab:johanneshein:SMdistro}
\begin{tabular}{llllcllc}
  \lsptoprule
  &&&\multicolumn{2}{c}{Sg}&&\multicolumn{2}{c}{Pl}\\
  \cmidrule{4-5}\cmidrule{7-8}
  Subjects&&&Pronoun&Marker&&Pronoun&Marker\\
  \midrule
  Pronominal&1.excl&&mὲ&$\varnothing$&&wὲr&à\\
  &1.incl&&&&&sì&à\\
  &2&&wὲ&$\varnothing$&&yì&à\\
  &3&&í&$\varnothing$&&wōyè&ó\\
  \midrule
  Nominal&&&&à&&&ó\\
  \lspbottomrule
\end{tabular}
\end{table}

\subsection{Why is agreement absent for singular pronouns?}

There are some possibilities for why agreement is impossible with
singular pronouns.  First, for Celtic languages, it has been
argued that what looks like an agreement marker is really a pronominal
argument cliticized onto the verb. Thus, in Breton, full DP subjects
never trigger agreement \REF{ex:johanneshein:bretona}, but pronominal subjects are
obligatorily dropped with ``agreement'' showing up on the verb
\REF{ex:johanneshein:bretonb}.

\ea Breton \citep[][1916]{jouitteau+rezac06}
\begin{xlist}
\ex \label{ex:johanneshein:bretona}
\gll Gant o mamm e karf-ent/*karf-e \textit{pro} beza\~n.\\
with their mother \textsc{r} would.love-\textsc{3pl}/*would.love-\textsc{3sg} \textsc{3pl} be.\textsc{inf}\\
\glt `They would like to be with their mother.'
\ex \label{ex:johanneshein:bretonb}
\gll Gant o mamm e karf-e/*karf-ent Azenor ha Iona beza\~n.\\
with their mother \textsc{r} would.love-\textsc{3sg}/*would.love-\textsc{3pl} Azenor and Iona be.\textsc{inf}\\
\glt `Azenor and Iona would like to be with their mother.'
\end{xlist}
\z
This complementarity effect has been taken as evidence that, in fact,
there is no $\phi$-agreement between subject and verb. If the subject
is a pronoun, which is weak enough to cliticize onto the verb, it only
appears as though the verb inflects (see \citealp{anderson82}; see
also \citealp{stump84} who rejects this analysis in favour of an
agreement analysis). The facts are almost identical and have received
an identical analysis in Irish
\citep{pranka83,doron88,ackema+neeleman03} and Scottish Gaelic
\citep{adger00}. Under such an approach, the Limbum subject markers,
would be weak pronouns cliticizing to the verb. Their absence with
pronominal subjects is then due to the fact that these subjects must
be strong pronouns that cannot cliticize onto the verb. In contrast to
the Celtic languages mentioned above, however, Limbum allows the
subject marker to cooccur with a full NP subject. If the subject
marker is indeed a pronoun, one could argue that it is the actual
subject, taking the subject's argument position and theta role,
similar to what has been argued to be the case for polysynthetic
non-configurational languages (see \citealp{jelinek84,baker96}). What
appears to be the full NP subject, then is actually just an adjoined
phrase that is somehow linked to the respective pronominal argument.

However, this analysis would leave unexplained the occurrence of the
subject marker with plural pronominal subjects. In this part of the
paradigm, Limbum behaves more like Welsh, where a (postverbal)
pronominal subject agrees with the verb \REF{ex:johanneshein:welshPRON} while a (postverbal)
full DP subject does not \REF{ex:johanneshein:welshNP}.

\ea Welsh \citep[][227]{borsley09}\label{ex:johanneshein:welshPRON}
\begin{xlist}
\ex 
\gll Gwelodd e/hi ddraig.\\
see.\textsc{pst.3sg} he/she dragon\\
\glt `He saw a dragon.'
\ex 
\gll Gwelon nhw ddraig.\\
see.\textsc{pst.3pl} they dragon\\
\glt `They saw a dragon.'
\end{xlist}
\ex Welsh \citep[][227]{borsley09}\label{ex:johanneshein:welshNP}
\begin{xlist}
\ex 
\gll Gwelodd y bachgen/bechgyn ddraig.\\
see.\textsc{pst.3sg} the boy/boys dragon\\
\glt `The boy/boys saw a dragon.'
\ex
\gll *Gwelon y bechgyn ddraig.\\
see.\textsc{pst.3pl} the boys dragon\\
\end{xlist}
\z
Thus, an account of the absence of the subject marker with singular
pronominal subjects that derives it as a type of complementarity
effect, as found in many Celtic languages, is not feasible.

A second possible explanation is that the subject agreement paradigm simply
contains three markers \textit{à}, \textit{ó}, and $\varnothing$
which are specified such that the zero marker realizes 1st, 2nd, and
3rd person singular. However, in this scenario, the zero marker would
have to explicitly make reference to the (categorial) status of the
subject as a pronoun \REF{ex:johanneshein:VEs}.

\ea Vocabulary entries for agreement markers\label{ex:johanneshein:VEs}
\begin{xlist}
\ex /ó/ $\leftrightarrow$ [$-$1,$-$2,$-$sg]
\ex /$\varnothing$/ $\leftrightarrow$ [pron, +sg]
\ex /à/ $\leftrightarrow$ [~]
\end{xlist}
\z
Now, this requires that subject-verb argeement not only leads to
$\phi$-features being present on the verb/T, but also the categorial
feature of the subject. Agreement for category, however, is a
very uncommon feature in natural languages \citep[cf.][]{weisser19}.

A third option is that the subject marker is not an agreement marker
but a specific past tense marker that displays subject-sensitive
allomorphy. As allomorphy rules are generally able to refer to the
category of an allomorphy-trigger, the fact that pronouns in the
singular require the zero allomorph is easily captured \REF{ex:johanneshein:allomorphs}.

\ea Allomorphs of the subject marker\label{ex:johanneshein:allomorphs}
\begin{xlist}
\ex \textit{ó} / [3pl]\underline{~~~}
\ex $\varnothing$ / [pron, sg]\underline{~~~}
\ex \textit{à}
\end{xlist}
\z
Allomorphy is usually triggered under linear adjacency. Thus, when
material linearly intervenes between the subject and the subject
marker, we would expect that the default allomorph \textit{à}
appears. Unfortunately, adverbs in Limbum always occur clause-finally
making them unusable for testing this prediction. However, we can
employ coordinations where each conjunct requires a different
allomorph. What we find is that the subject marker apparently
references the whole conjunction. Thus, in \REF{ex:johanneshein:conjunct1}, the conjunction
of a full NP \textit{ŋwè rlɔ fɔ̄} `the reverend' and the
pronoun \textit{wὲ} `you (sg.)', which together resolves into a 2nd
person plural subject, triggers the subject marker \textit{à}
despite the singular pronoun \textit{wὲ} being linearly adjacent. Example \REF{ex:johanneshein:conjunct2} shows
the coordination of two different pronominal subjects \textit{wὲ}
`you (sg.)' and \textit{mὲ} `I' each independently requiring the
zero form of the subject marker. However, again \textit{à} appears,
as the whole coordination is a first person plural pronominal
subject. Lastly, \REF{ex:johanneshein:conjunct3} gives the coordination of two singular NPs
each requiring the subject marker \textit{à} in isolation. Instead,
the plural marker \textit{ó} occurs.

\ea\label{ex:johanneshein:coordination}
\begin{xlist}
\ex \label{ex:johanneshein:conjunct1}
\gll [Ŋwὲ rlɔ̄ fɔ̄ bá wὲ]$_{\text{2pl}}$ à mū zhé bā.\\
\hphantom{[}person prayer the and you.\textsc{sg} \textsc{sm} \textsc{pst2} eat fufu\\
\glt `The reverend and you ate fufu.'
\ex \label{ex:johanneshein:conjunct2}
\gll [Wὲ bá mὲ]$_{\text{1pl}}$ à mū zhé bā.\\
\hphantom{[}\textsc{2sg} and \textsc{1sg} \textsc{sm} \textsc{pst2} eat fufu\\
\glt `You(\textsc{sg}) and I ate fufu.'
\ex \label{ex:johanneshein:conjunct3}
\gll [Ŋwὲ rlɔ̄ bá yà bàá]$_{\text{3pl}}$ ó mū zhé bā.\\
\hphantom{[}person prayer and my father \textsc{3pl.sm} \textsc{pst2} eat fufu\\
\glt `The reverend and my father ate fufu.'
\end{xlist}
\z
In sum, the examples in \REF{ex:johanneshein:coordination} behave as if agreement takes
place with the whole coordination rather than with one of its conjuncts. Allomorphy alone can therefore not account
for the pattern of subject marking. In addition, the allomorphy rule
would have to make reference to the feature [pron]. While it is
possible for allomorphy to refer to category features, the general
perspective on pronouns since \citet{postal69} and \citet{abney87} is
that they are elements of category D, i.e. that there is no dedicated
category Pron comprising pronominal elements.

It is thus unclear how to formally analyze the Limbum subject
agreement pattern.  From a functional perspective, it looks like an
instance of complex differential subject marking (DSM,
\citealp{deHoop+malchukov08}). In analogy to differential object
marking (DOM), DSM occurs when the morphological encoding of subjects
varies depending on some properties of the subject with less likely
subjects (according to some hierarchy such as referentiality,
definiteness, or person, \citealp{hale72,silverstein76}) being more
marked than more likely subjects. In the Limbum case, the relevant
property is a combination of definiteness and number. The definiteness
and number scales are given in \REF{ex:johanneshein:DEF} and \REF{ex:johanneshein:NUM}.

\ea Definiteness scale\label{ex:johanneshein:DEF}\\
Pro(noun) $>$ Name (PN) $>$ Def(inite) $>$ Indefinite Specific (Spec) $>$ NonSpecific (NSpec)
\ex Number scale\label{ex:johanneshein:NUM}\\
Plural $>$ Singular
\z 
In effect, when considering these scales for subjects, a pronominal
element turns out to be a more likely/expected subject than a proper
name. The latter, in turn, is a more likely subject than a definite
element, and so on. Now, Limbum draws the line between Pro and PN on
the scale, separating pronouns from all other types of
subjects. Combining the definiteness with the number scale, Limbum
further distinguishes between singular pronominal subjects and plural
pronominal subjects with the former being the most likely/expected
subjects. As such, these do not have to be marked overtly (by an overt
subject marker). In contrast, any subject deviating from the
expectation (i.e. singular pronoun) has to receive a specific encoding
in the form of an overt subject marker. The fact that the Limbum
subject marker is absent with singular pronominal subjects only thus
results from functional considerations where expectations as to what
constitutes a prototypical/likely subject play a role for the
morphological encoding. This, of course, leaves open the question of
why the subject marker only occurs in a handful of tenses/aspect.


\subsection{An apparent anti-agreement effect}

The different behaviour of singular NPs and singular pronouns with
regard to subject agreement gives rise to an interesting effect when
the subject has undergone some form of displacement. When the subject
is questioned \REF{ex:johanneshein:AAE-Limbum-wh}, focussed\footnote{The focus marked by the
  particle \textit{á} here is new information focus. There is at
  least one other focus marking strategy with a particle \textit{bá}
  which encodes contrastive/exhaustive focus
  \citep{beckeretal19,driemel+nformi18}. As the latter does not
  involve displacement to the left periphery, it is of no interest
  here.} \REF{ex:johanneshein:AAE-Limbum-NP}, or relativized \REF{ex:johanneshein:AAE-Limbum-REL}, the \textit{à} marker
 that usually appears with full NP subjects disappears. Instead, there is a different marker \textit{í} occuring
in the clause.\footnote{See \citet{beckeretal19} for arguments that the
  \textit{á} construction is not a biclausal cleft but rather
  involves a monoclausal movement structure.}

\ea \label{ex:johanneshein:AAE-Limbum}
\begin{xlist}
\ex \label{ex:johanneshein:AAE-Limbum-wh}
\gll Á \textbf{ndá}$_1$ cí \textbf{í$_1$} $\varnothing$ mū zhé bzhɨ́ (à)?\\
\textsc{foc} who \textsc{comp} \textsc{3sg.rp} \textsc{sm} \textsc{pst2} eat food \textsc{q}\\
\glt `Who$_{\text{F}}$ ate food?'
\ex \label{ex:johanneshein:AAE-Limbum-NP}
\gll Á \textbf{Nfòr}$_1$ cí \textbf{í$_1$} $\varnothing$ mū zhé bzhɨ́.\\
\textsc{foc} Nfor \textsc{comp} \textsc{3sg.rp} \textsc{sm} \textsc{pst2} eat food\\
\glt `Nfor$_{\text{F}}$ ate food.' (new information focus)
\ex \label{ex:johanneshein:AAE-Limbum-REL}
\gll Mὲ rɨ̀ŋ njíŋwὲ$_1$ [ zhɨ̀ \textbf{í$_1$} $\varnothing$ cɨ́ yɛ̄ ŋgwē fɔ̄ ].\\
\textsc{1sg} know woman ~ \textsc{rel} \textsc{3sg.rp} \textsc{sm} \textsc{prog} see dog \textsc{det}\\
\glt `I know the woman who is seeing the dog.'
\end{xlist}
\z 
This marker \textit{í} is in fact the regular third person singular pronoun as in \REF{ex:johanneshein:PRON}. 

\ea \label{ex:johanneshein:PRON}
\gll Í $\varnothing$ cɨ́ fàʔ mí ŋkàʔ.\\
S/He \textsc{sm} \textsc{prog} work in garden\\
\glt `S/He is working in the garden.'
\z
In light of \REF{ex:johanneshein:PRON}, it is plausible to treat the occuring
\textit{í}-marker in \REF{ex:johanneshein:AAE-Limbum} as a resumptive pronoun taking the place
of the displaced subject. Now at first glance, it appears as though
the \textit{à} marker has gone missing in \REF{ex:johanneshein:AAE-Limbum} as a consequence of
full NP subject displacement. This is reminiscent of the so-called
anti-agreement effect \citep{ouhalla93,baier18}, where subject
agreement is suppressed when the subject undergoes displacement. In
Limbum, however, the pronoun independently cannot cooccur with the
subject marker \textit{à}, which therefore is absent from the
sentence.

That one is not dealing with an anti-agreement effect can immediately
be shown by comparing extraction of singular NP subjects with
extraction of (local) plural pronominal subjects. Both kinds of
subjects obligatorily require the subject marker \textit{à} when in
situ \REF{ex:johanneshein:insituSM}.

\ea\label{ex:johanneshein:insituSM}
\begin{xlist}
\ex \label{ex:johanneshein:insituSMa}
\gll Nfòr \textbf{*(à)} mū zhé bzhɨ́.\\
Nfor \hphantom{(*}\textsc{sm} \textsc{pst2} eat food\\
\glt `Nfor ate food.'
\ex \label{ex:johanneshein:insituSMb}
\gll Wὲr/sì/yì \textbf{*(à)} mū fàʔ.\\
\textsc{1pl.e/1pl.i/2pl} \hphantom{(*}\textsc{sm} \textsc{pst2} work\\
\glt `We(exc)/we(inc)/you(pl) worked.'
\end{xlist}
\z
Now, when the singular subject of \REF{ex:johanneshein:insituSMa} is extracted, it leaves a
singular resumptive pronoun \textit{í} which independently disallows
\textit{à}. Consequently, \textit{à} is absent \REF{ex:johanneshein:extractSM}.

\ea \label{ex:johanneshein:extractSM}
\gll Á \textbf{Nfòr}$_1$ cí \textbf{í$_1$} $\varnothing$ mū zhé bzhɨ́.\\
\textsc{foc} Nfor \textsc{comp} \textsc{3sg.rp} \textsc{sm} \textsc{pst2} eat food\\
\glt `Nfor$_{\text{F}}$ ate food.'
\z
On the other hand, extraction of the subject in \REF{ex:johanneshein:insituSMb} should leave
a plural resumptive pronoun, which requires the presence of
\textit{à}. We would thus expect that no ``antiagreement'' effect
will be observed. As \REF{ex:johanneshein:LDE-local-plural} confirms, this is indeed the case.

\ea
\gll Á \textbf{wὲr/sì/yì} cí \textbf{wὲr/sì/yì} \textbf{*(à)} mū fàʔ.\label{ex:johanneshein:LDE-local-plural}\\
\textsc{foc} \textsc{1pl.exc/1pl.inc/2pl} \textsc{comp} \textsc{1pl.exc/1pl.inc/2pl} \hphantom{(*}\textsc{sm} \textsc{pst2} work\\
\glt `We(exc)/we(inc)/you(pl)$_{\text{F}}$ worked.'
\z
With extraction of singular pronominal subjects, we would expect a
resumptive pronoun to occur but the marker \textit{à} to be absent
as these pronouns never cooccur with \textit{à}
\REF{ex:johanneshein:singularSM}. This expectation is also fulfilled \REF{ex:johanneshein:singularSMextract}.

\ea \label{ex:johanneshein:singularSMextract}
\gll Á \textbf{mὲ/wὲ/í} cí \textbf{mὲ/wὲ/í} \textbf{(*à)} mū fàʔ.\label{ex:LDE-singular}\\
\textsc{foc} \textsc{1sg/2sg/3sg} \textsc{comp} \textsc{1sg/2sg/3sg} \hphantom{(*}\textsc{sm} \textsc{pst2} work\\
\glt `I/you(sg)/(s)he$_{\text{F}}$ worked.'
\z
Third person plural subjects, in contrast, behave in a surprising way
giving rise to yet another asymmetry between different kinds of
subjects. Under the approach sketched so far, we would expect them to
pattern with local person plural subjects, i.e. leaving a resumptive
pronoun plus subject marker, with the difference that this subject
marker is \textit{ó}, not \textit{à}. This is, because like the
latter, a pronominal third plural subject requires the presence of a
subject marker when in situ \REF{ex:johanneshein:3pl-pron-insitu}. However, this is
not what we find. When a third person plural subject is extracted it
obligatorily leaves a gap with the presence of the subject marker
being unaffected by extraction \REF{ex:johanneshein:PL-extract}.

\ea\label{ex:johanneshein:PL-extract}
\begin{xlist}
\ex 
\gll Á \textbf{bō} \textbf{fɔ̄} cí (\textbf{*wōyè}) \textbf{ó} mū zhé bzhɨ́.\\
\textsc{foc} children \textsc{det} \textsc{comp} \hphantom{(*}\textsc{3pl.rp} \textsc{3pl.sm} \textsc{pst2} eat food\\
\glt `The children$_{\text{F}}$ ate food.'
\ex 
\gll Á \textbf{wōyè} cí (\textbf{*wōyè}) \textbf{ó} mū zhé bzhɨ́.\\
\textsc{foc} \textsc{3pl} \textsc{comp} \textsc{3pl.rp} \textsc{sm} \textsc{pst2} eat food\\
\glt `They$_{\text{F}}$ ate food.'
\end{xlist}
\z
The pattern of resumption and subject marking under extraction is
given in \tabref{tab:johanneshein:RPdistro}. As can be seen, to the exception of third person
plural, it reflects the pattern of pronominal in situ subjects and
subject markers in \tabref{tab:johanneshein:PRON}.

\begin{table}
  \begin{floatrow}
  \captionsetup{margin=.05\linewidth}
  \ttabbox{\caption{Resumptive pronouns (RP) and SM}\label{tab:johanneshein:RPdistro}}
  {\begin{tabular}{lcc}
    \lsptoprule
    subject &RP&SM\\
    \midrule
    singular& ✔&---\\
    1st \& 2nd plural& ✔& ✔\\
    3rd plural&---& ✔\\
    \lspbottomrule
  \end{tabular}}
  \ttabbox{\caption{Regular pronouns (Pron) and SM}\label{tab:johanneshein:PRON}}
  {\begin{tabular}{lcc}
    \lsptoprule
    subject&Pron&SM\\
    \midrule
    singular& ✔&---\\
    1st \& 2nd plural& ✔& ✔\\
    3rd plural&  ✔& ✔\\
    \lspbottomrule
  \end{tabular}}
  \end{floatrow}
\end{table}

With the exception of third person plural, it is thus the interaction
between the pattern of agreement on one side and the requirement of
subject displacements to have a resumptive pronoun in their base
position on the other side that gives the impression of an
anti-agreement effect for singular NP subjects.



\section{The third person plural}
\label{sec:johanneshein:third-person}

Turning back to third person plural subjects, recall that they behave
like local person plural pronominal subjects in that they obligatorily
require a cooccuring subject marker when in situ as in \REF{ex:johanneshein:pluralSM}, repeated below as \REF{ex:johanneshein:3pl-insitu}, but differ from these
in that they leave a gap rather than a resumptive pronoun when they
are extracted \REF{ex:johanneshein:PL-extract}.

\ea\label{ex:johanneshein:3pl-insitu}
\begin{xlist}
\ex 
\gll Wōyè \textbf{*(ó)} mū fàʔ.\label{ex:johanneshein:3pl-insitu-a}\\
\textsc{3pl} \hphantom{(*}\textsc{3pl.sm} \textsc{pst2} work\\
\glt `They worked.'
\ex 
\gll Bō fɔ̄ \textbf{*(ó)} mū zhé bzhɨ́.\label{ex:johanneshein:3pl-insitu-b}\\
children \textsc{det} \hphantom{(*}\textsc{3pl.sm} \textsc{pst2} eat food\\
\glt `The children ate food.'
\end{xlist}
\z
% \ea\label{ex:johanneshein:3pl-extracted}
% \begin{xlist}
% \ex 
% \gll Á \textbf{bō} \textbf{fɔ̄} cí \textbf{*wōyè}/\textbf{ó} mū zhé bzhɨ́.\\
% \textsc{foc} children \textsc{det} \textsc{c} \textsc{*3pl.rp/sm} \textsc{pst2} eat food\\
% \glt `The children$_{\text{F}}$ ate food.'
% \ex 
% \gll Á \textbf{wōyè} cí \textbf{*wōyè}/\textbf{ó} mū zhé bzhɨ́.\\
% \textsc{foc} \textsc{3pl} \textsc{c} \textsc{3pl.rp/sm} \textsc{pst2} eat food\\
% \glt `They$_{\text{F}}$ ate food.'
% \end{xlist}
% \z
Given that examples like the ones in \REF{ex:johanneshein:PL-extract} parallel examples of
extraction of other pronominal subjects like in
\REF{ex:johanneshein:LDE-local-plural} and \REF{ex:johanneshein:singularSMextract}, this suggests
that the resumptive pronoun counterpart to the third person plural
pronoun is simply null. The resumptive versions of all other pronouns,
in contrast, are form-identical to the ones used in non-resumptive
contexts as shown in \tabref{tab:johanneshein:regularandresumptive}.

\begin{table}
\caption{Regular and resumptive pronouns}
\label{tab:johanneshein:regularandresumptive}
\begin{tabular}{lcclcc}
  \lsptoprule
  &\multicolumn{2}{c}{Regular}&&\multicolumn{2}{c}{Resumptive}\\
  \cmidrule{2-3}\cmidrule{5-6}
  &Sg&Pl&&Sg&Pl\\
  \midrule
  1.exc&mὲ&wὲr&&mὲ&wὲr\\
  1.inc&--&sì&&--&sì\\
  2&wὲ&yì&&wὲ&yì\\
  3.anim&í&wōyè&&í&$\varnothing$\\
  3.inan&í&bvɨ̄&&í&bvɨ̄\\
  \lspbottomrule
\end{tabular}
\end{table}

Support for this line of analysis comes from subject extraction out of
islands. The island-obviating effect of resumptive pronouns is
well-known by now \citep{mcCloskey79,borer84}. As subject extraction of non-third
person plural subject leaves an overt resumptive pronoun, islands
should not have any degrading effect. Indeed, this is what we find. Examples of subject extraction from a complex NP island are given in
\REF{ex:johanneshein:islandextraction-a} for a second person plural subject and \REF{ex:johanneshein:islandextraction-b} for a third
person singular subject.\footnote{Note that the complementizer \textit{nɛ̄} in \REF{ex:johanneshein:islandextraction} and \REF{ex:johanneshein:islandextraction-plural} shows agreement (in the form of a prefix) with the embedding noun \textit{nsūŋ} `rumour'. This fits the general pattern of complementizer agreement in the language where the complementizer agrees with the matrix subject for person, number, and animacy in case there is no intervener (i.e. a direct object). An exploration of this phenomenon and the interesting intervention effects that are observed with it is beyond the scope of this article. I refer the interested reader to \citet{nformi18}, who documents the pattern in some detail.}

\ea\label{ex:johanneshein:islandextraction}
\begin{xlist}
\ex \label{ex:johanneshein:islandextraction-a}
\gll Á \textbf{yì} cí mὲ $\varnothing$ mū yōʔ nsūŋ zhɨ̌-nɛ̄ yì à mū fàʔ.\\
\textsc{foc} \textsc{2pl} \textsc{comp} I \textsc{sm} \textsc{pst2} hear rumour \textsc{3sg.inan-comp} \textsc{2pl} \textsc{sm} \textsc{pst2} work\\
\glt `I have heard the rumour that you(pl)$_{\text{F}}$ have worked.'
\ex \label{ex:johanneshein:islandextraction-b}
\gll Á \textbf{Nfòr} cí mὲ $\varnothing$ mū yōʔ nsūŋ zhɨ̌-nɛ̄ í mū fàʔ.\\
\textsc{foc} Nfor \textsc{comp} I \textsc{sm} \textsc{pst2} hear rumour \textsc{3sg.inan-comp} \textsc{3sg} \textsc{pst2} work\\
\glt `I have heard the rumour that Nfor$_{\text{F}}$ has worked.'
\end{xlist}
\z
Importantly, the island-obviating effect is also found with extraction
of a third person plural subject despite the lack of an overt
resumptive pronoun \REF{ex:johanneshein:islandextraction-plural}.

\ea \label{ex:johanneshein:islandextraction-plural}
\gll Á \textbf{wōyè} cí mὲ $\varnothing$ mū yōʔ nsūŋ zhɨ̌-nɛ̄ (*wōyè) ó mū fàʔ.\\
\textsc{foc} \textsc{3pl} \textsc{comp} I \textsc{sm} \textsc{pst2} hear rumour \textsc{3sg.inan-comp} \hphantom{(*}\textsc{3pl} \textsc{3pl.sm} \textsc{pst2} work\\
\glt `I have heard the rumour that they$_{\text{F}}$ have worked.'
\z
This parallel behaviour with regard to island-sensitivity suggests
that there is a silent resumptive pronoun present in
\REF{ex:johanneshein:islandextraction-plural}.\footnote{It should be mentioned that this argument loses some
  of its strength as islands in Limbum seem to be quite liberal in
  general (see appendix in section~\ref{sec:johanneshein:appendix} for data). For objects, extraction is possible from regular embedded clauses as well as from inside an island, leaving a gap in both cases. On the other hand, extraction of either the verb
or the verb phrase out of an island is
impossible despite this being grammatical from a simple embedded clause \citep[for details, see][]{hein20:book}. This
indicates that islands still exist in the language and that the
insensitivity of objects towards them might have a different source.
}
If this line of reasoning is correct, Limbum goes against the
cross-linguistically largely valid generalization that the forms of
resumptive pronouns are generally drawn from the set of regular
(personal) pronouns
(\citealp{asudeh11,asudeh12,salzmann17:book,mcCloskey17}, though see
\citealp{adger11} for counter-examples).

However, there is a further qualification to be made. As
\citet[][187]{salzmann17:book} points out, ``[r]esumptives are usually
drawn from the unmarked series of the personal pronoun paradigm, thus
usually the weak/clitic forms''. Now, there is no distinction between
strong and weak pronouns in non-third person contexts. First, in the
various examples throughout this paper the focussed pronoun, which is
arguably strong, has the same form as the arguably weak
resumptive. Second, in a weak pronominal context, such as discourse
anaphora \REF{ex:johanneshein:discourseanaphora}, the anaphoric pronoun is not different from either the
supposedly strong pronoun in focus contexts or the resumptive pronoun as in \REF{ex:johanneshein:LDE-local-plural}.

\ea\label{ex:johanneshein:discourseanaphora}
\begin{xlist}
\ex 
\gll Mὲ bá yà bàá à níŋī. *(Wὲr) à bā kɔ̄nī Nfòr à ŋgàbtfəʔ.\\
I and my father \textsc{sm} arrive \hphantom{*(}\textsc{1pl.exc} \textsc{sm} \textsc{pst1} meet Nfor in morning\\
\glt `Me and my father have arrived. We met Nfor in the morning.'
\ex 
\gll Wὲ bá yà bàá à níŋī. *(Yí) à bā kɔ̄nī Nfòr à ŋgàbtfəʔ.\\
you and my father \textsc{sm} arrive \hphantom{*(}\textsc{2pl} \textsc{sm} \textsc{pst1} meet Nfor in morning\\
\glt `You and my father have arrived. You met Nfor in the morning.'
\end{xlist}
\z
However, the situation is different with third person subjects. 
For third person plural, both in
resumption \REF{ex:johanneshein:3rdpluralresumption} and in discourse anaphoric use \REF{ex:johanneshein:3rdpluraldiscourse} (i.e. in contexts where the pronoun is expected to take the weak form) the form
of the pronoun is null contrasting with the form \textit{wōyè} that appears in focus positions (i.e. a context for a strong form). The only element that appears before the
TAM-marker is the subject marker \textit{ó} in both cases.

\ea
\begin{xlist}
\ex \label{ex:johanneshein:3rdpluralresumption}
\gll Á \textbf{bō} cí (*wōyè) ó mū zhé bzhɨ́.\\
\textsc{foc} children \textsc{comp} \hphantom{(*}\textsc{3pl} \textsc{3pl.sm} \textsc{pst3} eat food\\
\glt `The children$_{\text{F}}$ ate food.'
\ex \label{ex:johanneshein:3rdpluraldiscourse}
\gll Bfər ó níŋī. (*Wōyè) Ó kēʔ ā mʉ̄ʔshɨ̄ mŋkòb.\\
relatives \textsc{3pl.sm} arrive \hphantom{(*}\textsc{3pl} \textsc{3pl.sm} start to open suitcases\\
\glt `The relatives have arrived. (They) have already started unpacking their suitcases.'
\end{xlist}
\z
This suggests that there is a strong/weak distinction for third person plural pronouns and that the weak version has a null realization. In that case, Limbum complies with the abovementioned cross-linguistic generalization.\footnote{There is, of
  course, a very obvious functional explanation for the fact that it
  is just the third person plural which shows a null pronoun. In
  contrast to all other person-number combinations, it has a unique
  subject marker \textit{ó}, which is able to unambiguously identify
  the subject as a third person plural in the absence of an overt
  realization of the subject. The other subject markers $\varnothing$
  and \textit{à} are ambiguous between 1st, 2nd, and 3rd person
  singular and 1st, 2nd person plural as well as 3rd singular NP,
  respectively.}

Interestingly, a third person singular pronominal subject also shows distinct forms for strong and weak contexts. While it occurs as \textit{í} in discourse anaphoric use \REF{ex:johanneshein:3sgdiscourse} and resumption \REF{ex:johanneshein:3rdpron} it takes the form \textit{yé} in focus position \REF{ex:johanneshein:3rdpron}.

\ea
\begin{xlist}
\ex \label{ex:johanneshein:3sgdiscourse}
\gll Nfòr à níŋī. *(Í) $\varnothing$ bā kɔ̄nī wὲr à ŋgàbtfəʔ.\\
Nfor \textsc{sm} arrive \hphantom{(*}he \textsc{sm} \textsc{pst1} meet \textsc{1pl} in morning\\
\glt `Nfor has arrived. He met us in the morning.'
\ex \label{ex:johanneshein:3rdpron}
\gll Á \textbf{yé} cí í $\varnothing$ mū fàʔ.\\
\textsc{foc} \textsc{3sg} \textsc{comp} \textsc{3sg} \textsc{sm} \textsc{pst2} work\\
\glt `S/he$_{\text{F}}$ worked.'
\end{xlist}
\z
In contrast to the third person plural,
however, the weak form for the third person singular is not null.
In addition, for third person singular the strong form is syncretic with the one found in object position \REF{ex:johanneshein:objpron} while there is only a partial identity between the strong subject form \textit{wōyè} and the object form \textit{wō} for third person plural pronouns \REF{ex:johanneshein:objpronpl}.

\ea\label{ex:johanneshein:obj}
\begin{xlist}
\ex \label{ex:johanneshein:objpron}
\gll Nfòr à níŋī. Mὲ $\varnothing$ bā yɛ̄ *(yē) à ŋgàbtfəʔ.\\
Nfor \textsc{sm} arrive I \textsc{sm} \textsc{pst1} see \hphantom{(*}\textsc{3sg.obj} in morning\\
\glt `Nfor has arrived. I saw him in the morning.'
\ex \label{ex:johanneshein:objpronpl}
\gll Bfər ó níŋī. Mὲ $\varnothing$ bā yɛ̄ *(wō) à ŋgàbtfəʔ.\\
relatives \textsc{3pl.sm} arrive I \textsc{sm} \textsc{pst1} see \hphantom{(*}\textsc{3pl.obj} in morning\\
\glt `The relatives have arrived. I met them in the morning.'
\end{xlist}
\z
For local person pronouns, both singular and plural, the forms for subjects and objects are always entirely syncretic. The forms for subject and object pronouns are given in \tabref{tab:johanneshein:subjectobjectpronouns}. 

\begin{table}
\caption{Subject and object pronouns}
\label{tab:johanneshein:subjectobjectpronouns}
\begin{tabular}{lcclcc}
  \lsptoprule
  &\multicolumn{2}{c}{Subject}&&\multicolumn{2}{c}{Object}\\
  \cmidrule{2-3}\cmidrule{5-6}
  &Sg&Pl&&Sg&Pl\\
  \midrule
  1.exc&mὲ&wὲr&&mὲ&wὲr\\
  1.inc&--&sì&&--&sì\\
  2&wὲ&yì&&wὲ&yì\\
  3.anim&í&wōyè&&yé&wō\\
  3.inan&í&bvɨ̄&&zhɨ̄&bvɨ̄\\
  \lspbottomrule
\end{tabular}
\end{table}

Note that pro-drop is not an option in Limbum neither in subject
position \REF{ex:johanneshein:3sgdiscourse} nor in object position \REF{ex:johanneshein:obj}.
The only case in which it looks like the pronoun has been dropped is
when it is a third person plural subject \REF{ex:johanneshein:3rdpluraldiscourse}.
% \ea \label{ex:johanneshein:prodrop3rdplural}
% \gll Bfər ó níŋī. Ó kēʔ ā mʉ̄ʔshɨ̄ mŋkòb.\\
% relatives \textsc{3pl.sm} arrive \textsc{3pl.sm} start to open suitcases\\
% \glt `The relatives have arrived. (They) have already started unpacking their suitcases.'
% \z
Pro-drop is usually not confined exclusively to one specific
person-number combination. Rather, in specific environments all
pronominal elements, independent of their person-number
specifications, are dropped. Thus, I argue that what is special about
the third person is that it is the only person-number combination in
Limbum for which there are distinct strong and weak pronouns. In particular,
the weak form for the third person plural is null, which gives rise to
the apparent surface asymmetry regarding resumption. Additionally, it is also the only person-number combination which exhibits a difference in form for subject and object pronouns.

\section{Focus marking}
\label{sec:johanneshein:focus-movement}

Let me turn to a third asymmetry: focus marking. So far, in examples
with a focussed constituent marked by \textit{á}, this constituent
has consistently been followed by an overt element \textit{cí},
preliminarily glossed as \textsc{comp}.\footnote{This element is very similar to the relative marker \textit{zhɨ̀} used to introduce relative clauses such as \REF{ex:johanneshein:relC}.

\ea \label{ex:johanneshein:relC}
\gll Mὲ rɨ̀ŋ njíŋwὲ [ zhɨ̀ \textbf{í} $\varnothing$ cí yɛ̄ ŋgwē fɔ̄ ].\\
\textsc{1sg} know woman ~ \textsc{rel.p} \textsc{3sg} \textsc{sm} \textsc{prog} see dog \textsc{det}\\
\glt `I know the woman who is seeing the dog.'
\z
However, they are not identical. The relative marker's consonant is pronounced [ʒ] while \textit{cí} is pronounced with a [ʧ]. Also, the former is low toned while the latter bears a high tone. It should thus be clear that focus constructions do not involve a relative clause.
}  This element, however, is in
fact optional and may be left out without a difference in meaning. Interestingly, it interacts with the subject marker
\textit{à} and the resumptive pronouns \textit{í} in the following way. In a regular
declarative focus-less sentence, only \textit{à} is possible and
\textit{cí} has to be absent \REF{ex:johanneshein:nofocus}. In a sentence where a
focussed subject is followed by \textit{cí}, only \textit{í} is
licit, while the presence of \textit{à} renders the sentence
ungrammatical \REF{ex:johanneshein:insitufocus}.  However, if the focussed subject is not
followed by \textit{cí}, both \textit{í} or \textit{à} may occur without any difference in interpretation
\REF{ex:johanneshein:exsitufocus} \citep[for the interpretation of focus in Limbum also see][]{driemel+nformi18,beckeretal19}.

\ea
\begin{xlist}
\ex \label{ex:johanneshein:nofocus}
\gll Nfòr \textbf{*í}/\textbf{à} mū fàʔ.\\
Nfor \textsc{*3sg.rp/sm} \textsc{pst2} work\\
\glt `Nfor worked.'
\ex  \label{ex:johanneshein:exsitufocus}
\gll Á \textbf{Nfòr} cí \textbf{í}/\textbf{*à} mū fàʔ.\\
\textsc{foc} Nfor \textsc{comp} \textsc{3sg.rp/*sm} \textsc{pst2} work\\
\glt `Nfor$_{\text{F}}$ worked.'
\ex \label{ex:johanneshein:insitufocus}
\gll Á \textbf{Nfòr} \textbf{í}/\textbf{à} mū fàʔ.\\
\textsc{foc} Nfor \textsc{3sg.rp/sm} \textsc{pst2} work\\
\glt `Nfor$_{\text{F}}$ worked.'
\end{xlist}
\z
The pattern is summarized in \tabref{tab:johanneshein:focusdistro}.

\begin{table}
    \caption{Pattern of cooccurrence of Focus, \textit{cí}, and the subject marker/resumptive\label{tab:johanneshein:focusdistro}}
\begin{tabular}{ccc}
  \lsptoprule
  \textsc{focus}&\textit{cí}&SM/RP\\
  \midrule
  ---&---&à\hphantom{, í}\\
  ✔&---&à, í\\
  ✔&✔&\hphantom{à, }í\\
  \lspbottomrule
\end{tabular}
\end{table}

We have already seen that, as a resumptive pronoun, \textit{í} only
occurs when the subject has been displaced. In contrast,
\textit{á} is only licit when the subject adjacent to it is not a
singular pronoun. If we now assume that \textit{cí} is the optional
overt realization of the head to whose specifier the focussed subject
is displaced, the pattern in \tabref{tab:johanneshein:focusdistro} falls out straightforwardly.

In \REF{ex:johanneshein:nofocus}, the subject is not focussed and not displaced. As it is a
third person singular NP, it triggers the presence of the subject
marker \textit{à}. The structure of \REF{ex:johanneshein:nofocus} is sketched in \REF{ex:johanneshein:structurenofocus}.

\ea {[}$_{\text{CP}}$ [$_{\text{TP}}$ Nfòr \textbf{à} mū [$_{\text{VP}}$ fàʔ ]]]\label{ex:johanneshein:structurenofocus}
\z
In \REF{ex:johanneshein:exsitufocus}, in contrast, the subject is focussed, as indicated by
it being preceded by the focus particle \textit{á}. Additionally,
the concomitant displacement is indicated by overt material intervening between the
subject and its base position, namely \textit{cí}. As the subject
is unambiguously displaced, the only material that can
appear directly preceding the tense marker \textit{mū} is the
resumptive pronoun \textit{í}. The element \textit{cí} could
either be a realization of the C head, under the assumption that focus
displacement targets SpecCP. It could also be regarded as a realization of
the Focus head, as argued by \citet{beckeretal19}, with the focus
particle \textit{á} heading its own FP projection \citep[see
also][]{horvath07,horvath10,horvath13,cable10}. These structures of
\REF{ex:johanneshein:exsitufocus} are sketched in \REF{ex:johanneshein:structureexsitufocus}.\footnote{An anonymous reviewer suggests that \textit{cí} might also indicate an underlying biclausal cleft structure. In this structure, \textit{á} would serve as a copula and \textit{cí} as a relative marker. The difference between \REF{ex:johanneshein:exsitufocus} and \REF{ex:johanneshein:insitufocus} would then be one between a cleft and a regular fronting/in-situ focus structure. However, as pointed out above, there is no difference in meaning between a structure with \textit{cí} and one without it. In addition, \citet[][§3.1]{beckeretal19} present three arguments against a cleft structure. (i) A focus sentence (with and without \textit{cí}) is compatible with non-exhaustive contexts while clefts typically have an exhaustive meaning component. (ii) The purported copula \textit{á} is not modifiable with tense/aspect markers. Instead, overt tense marking forces the presence of an additional copular element \textit{bā} giving rise to a true cleft sentence. (iii) Clefts contain a relative clause. However, \textit{cí} cannot serve as a relative pronoun. Also, the clause-final demonstrative \textit{nà} that optionally occurs with relative clauses cannot occur with focus sentences. It thus seems very unlikely that focus sentence with \textit{cí} constitute clefts.}

\ea \label{ex:johanneshein:structureexsitufocus}
\begin{xlist}
\ex {[}$_{\text{CP}}$ á Nfòr cí [$_{\text{TP}}$ \textbf{í}
$\varnothing$ mū [$_{\text{VP}}$ fàʔ ]]]
\ex {[}$_{\text{CP}}$ [$_{\text{FocP}}$ [$_{\text{FP}}$ á Nfòr ] cí [$_{\text{TP}}$ \textbf{í}
$\varnothing$ mū [$_{\text{VP}}$ fàʔ ]]]]
\end{xlist}
\z
Turning to the case of optionality, I argue that this is structurally
ambiguous between an in-situ \REF{ex:johanneshein:structurenofocus} and a displacement structure \REF{ex:johanneshein:structureexsitufocus}. In
one case, the subject is focus marked by the particle \textit{á} but
stays in situ in SpecTP \REF{ex:johanneshein:structureinsitufocus}. Here, it is not possible for
\textit{cí} to occur in between the subject and the subject marker
simply because the head which it realizes precedes the subject. The
subject marker \textit{à} occurs as the subject is not displaced.

\ea {[}$_{\text{CP}}$ [$_{\text{TP}}$ á Nfòr \textbf{à} mū [$_{\text{VP}}$ fàʔ ]]]\label{ex:johanneshein:structureinsitufocus}
\z
In the other case, the subject is focus marked by \textit{á} and
displaced to SpecCP or SpecFocP just as in \REF{ex:johanneshein:structureexsitufocus}. However, the C or Foc
head is not overtly realized. Therefore, there is no overt
(configurational) indication of displacement \REF{ex:johanneshein:structureinsitufocus2}. The
resumptive pronoun \textit{í} occurs because the subject is not in its base position. 

\ea \label{ex:johanneshein:structureinsitufocus2}
\begin{xlist}
\ex {[}$_{\text{CP}}$ á Nfòr C$_{\varnothing}$  [$_{\text{TP}}$ \textbf{í} $\varnothing$ mū [$_{\text{VP}}$ fàʔ ]]]
\ex {[}$_{\text{CP}}$ [$_{\text{FocP}}$ [$_{\text{FP}}$ á Nfòr ] Foc$_{\varnothing}$ [$_{\text{TP}}$ \textbf{í}
$\varnothing$ mū [$_{\text{VP}}$ fàʔ ]]]]
\end{xlist}
\z
Both structures \REF{ex:johanneshein:structureinsitufocus} and \REF{ex:johanneshein:structureinsitufocus2} result in the same surface string
with the only difference being that \REF{ex:johanneshein:structureinsitufocus} features the subject marker
\textit{à} and \REF{ex:johanneshein:structureinsitufocus2} contains the resumptive pronoun \textit{í}
instead. 

An indication that the absence of \textit{cí} is not equivalent to
the absence of displacement or the absence of the head that hosts
\textit{cí} comes from object focus. When an object undergoes focus
fronting, \textit{cí} is equally optional as with subject
focussing \REF{ex:johanneshein:objectfocus}.

\ea \label{ex:johanneshein:objectfocus}
\gll Á \textbf{Ngàlá} (cí) mὲ bí kɔ̄nī.\\
\textsc{foc} Ngala \textsc{comp} I \textsc{fut1} meet\\
\glt `I will meet Ngala$_{\text{F}}$.'\hfill\citep[][60]{becker+nformi16}
\z
The object in \REF{ex:johanneshein:objectfocus} clearly appears outside of its base position.
Therefore, there must be a head that provides a specifier to host it
whether \textit{cí} is overt or not. Thus, displacement in \REF{ex:johanneshein:structureinsitufocus2} is
a valid possibility despite the lack of \textit{cí}.

\section{Conclusion}
\label{sec:johanneshein:conclusion}

In this paper, I showcased three subject-internal asymmetries in
Limbum. The first asymmetry is between singular pronominal subjects
and singular full NP\slash plural pronominal subjects. Its interaction with
subject resumption gives rise to what looks like an antiagreement
effect on the surface. As this effect is a direct result of the
interaction, this might lend some support to approaches to antiagreement
effects that attribute it to language-specific properties
\citep{fominyam+georgi19,vanAlem19} rather than some cross-linguistic
general antiagreement rule/mechanism/operation (e.g. antilocality or
\Abar-triggered impoverishment).

The second asymmetry obtains between third person plural
vs. everything else with regard to resumption.  Again, the asymmetry is only apparent as the gap left by
third person plural subject extraction is not a true gap. Only the third person shows a weak vs. strong
distinction in pronouns as evidenced by discourse anaphoricity. The
weak version of the third person plural pronoun used in resumption
contexts simply happens to be null and therefore gives the impression of a gap.

The last asymmetry concerns the cooccurrence of focus marking and the
subject marker/re\-sump\-tive pronoun. It was shown that the absence of
focus marking is paired with the subject marker, while the presence of
full focus marking with \textit{á} and \textit{cí} requires the
resumptive pronoun. Focus marking with only \textit{á} allows for
subject marker or resumptive pronoun to be present. This optionality
can be derived from an underlying structural ambiguity between ex-situ and in-situ focus marking in
interaction with the optionality of overt \textit{cí}. 

Overall, the three subject asymmetries have been argued to be the
result of language-specific peculiarities (i.e. absence of subject marker with singular pronouns, weak-strong
distinction for third person pronouns only, optional overtness
of \textit{cí})
and their interaction with other properties of the language
(e.g. obligatory subject resumption, focus movement).



\section*{Appendix}
\label{sec:johanneshein:appendix}

Nominal object extraction for focus leaves a gap rather than a resumptive pronoun, whether it takes place out of a regular embedded
  clause \REF{ex:johanneshein:haplologypronoun}, or from a complex NP \REF{ex:johanneshein:noRP} or an adjunct clause
  \REF{ex:johanneshein:adj-island}.

\ea \label{ex:johanneshein:haplologypronoun}
\gll Á \textbf{wō(yè)/mὲ/yì} cí Nfòr à mū lìb *wō/*ó/*mὲ/*yì/\gap{}.\\
\textsc{foc} \textsc{3pl/1sg/2pl} \textsc{comp} Nfor \textsc{sm} \textsc{pst2} beat \textsc{3pl.rp}/\textsc{3pl.sm}/\textsc{1sg}/\textsc{2pl}/\gap{}\\
\glt `Them/me/you(pl.)$_{\text{F}}$, Nfor has hit.'
\z

\ea\label{ex:johanneshein:noRP}
\judgewidth{?}
\begin{xlist}
\ex[]{
\gll Á \textbf{ndāp} cí mὲ $\varnothing$ mū yōʔ nsūŋ zhɨ̌-nɛ̄ Nfòr à mū bō *zhī/\gap{}.\\
\textsc{foc} house \textsc{comp} I \textsc{sm} \textsc{pst2} hear rumour \textsc{3sg.inan}-\textsc{comp} Nfor \textsc{sm} \textsc{pst2} build \textsc{3sg.inan.obj}/\gap{}\\
\glt `I have heard a rumour that a house$_{\text{F}}$ Nfor has built.'}
\ex[?]{ 
\gll Á \textbf{wō(yè)} cí mὲ $\varnothing$ mū yōʔ nsūŋ zhɨ̌-nɛ̄ Nfòr à mū kɔ̄nī *ó/*wō/\gap{}.\\
\textsc{foc} \textsc{3pl} \textsc{comp} I \textsc{sm} \textsc{pst2} hear rumour \textsc{3sg.inan}-\textsc{comp} Nfor \textsc{sm} \textsc{pst2} meet \textsc{3pl.sm/3pl/}\gap{}\\
\glt `I have heard a rumour that them$_{\text{F}}$ Nfor has met.'}
\end{xlist}
\ex \label{ex:johanneshein:adj-island}
\gll Á \textbf{wō(yè)/mὲ/yì} (cí) Nfòr à mū būmī káʔ ànjɔ́ʔ í $\varnothing$ mū lìb *ó/*wō/*mὲ/*yì/\gap{}\\
\textsc{foc} \textsc{3pl/1sg/2pl} \textsc{comp} Nfor \textsc{sm} \textsc{pst2} sleep \textsc{neg} because \textsc{3sg} \textsc{sm} \textsc{pst2} beat \textsc{3pl.sm/3pl/1sg/2pl/}\gap{}\\
\glt `Nfor didn't sleep because them/me/you(pl.)$_{\text{F}}$ he hit.'
\z
Extraction for focus of a verbal constituent is generally possible \REF{ex:johanneshein:VPfrontingNonisland}.

\ea
\begin{xlist}
\ex \label{ex:johanneshein:VPfrontingNonisland}
\gll Á r-\textbf{bò} cí Nfòr bí \textbf{bō} ndāp.\\
\textsc{foc} 5-build \hphantom{(}\textsc{comp} Nfor \textsc{fut1} build house\\
\glt `Nfor will build$_{\text{F}}$ a house.'
\ex \label{ex:johanneshein:VPfronting1}
\gll Á r-[\textbf{bò} ndāp] cí Nfòr bí \textbf{gī}.\\
\textsc{foc} \hphantom{[}5-build house \hphantom{(}\textsc{comp} Nfor \textsc{fut1} do\\
\glt `Nfor will [build a house]$_{\text{F}}$.'
\end{xlist}
\z
Extraction for focus of either the verb
or the verb phrase out of an island, here a complex NP, is
impossible, even though arguably, the verb copy in \REF{ex:johanneshein:Vfronting} and the
dummy verb in \REF{ex:johanneshein:VPfronting2} could be regarded as resumptive elements.

\ea
\begin{xlist}
\ex[*]{ \label{ex:johanneshein:Vfronting}
\gll Á r-\textbf{bò} cí mὲ $\varnothing$ mū yōʔ [nsūŋ zhɨ̌-nɛ̄ Nfòr bí \textbf{bō} ndāp].\\
\textsc{foc} 5-build \hphantom{(}\textsc{comp} \textsc{1sg} \textsc{sm} \textsc{pst2} hear \hphantom{[}news \textsc{3sg.inan-comp} Nfor \textsc{fut1} build house\\
\glt `I heard a rumour that Nfor will build$_{\text{F}}$ a house.'}
\ex[*]{ \label{ex:johanneshein:VPfronting2}
\gll Á r-[\textbf{bò} ndāp] cí mὲ $\varnothing$ mū yōʔ [nsūŋ zhɨ̌-nɛ̄ Nfòr bí \textbf{gī}].\\
\textsc{foc} \hphantom{[}5-build house \hphantom{(}\textsc{comp} \textsc{1sg} \textsc{sm} \textsc{pst2} hear \hphantom{[}news \textsc{3sg.inan-comp} Nfor \textsc{fut1} do\\
\glt `I heard a rumour that Nfor will [build a house]$_{\text{F}}$.'}
\end{xlist}
\z


\section*{Acknowledgments}

I would like to thank Doreen Georgi, Imke Driemel, Martin Salzmann, and Philipp Weisser for discussion of various parts of this paper. I am very grateful to Jude Nformi for providing examples and judgements. All errors are my own. This research was funded by the Deutsche Forschungsgemeinschaft (DFG, German Research Foundation) -- Project number 317633480 -- SFB 1287, Project C05.


\section*{Abbreviations}

Below are listed only those abbreviations that do not adhere to or are beyond the scope of the Leipzig Glossing Rules.

\begin{tabbing}
    5, 6, 10+      \=Differential subject marking++ \=Pro(n)+       \=Resumptive pronoun++\kill
    5, 6, 10       \>Noun classes                   \>Pro(n)        \>Pronoun\\
    DOM            \>Differential object marking    \>\textsc{pst1} \>Near past tense \\
    DSM            \>Differential subject marking   \>\textsc{pst2} \>Distant past tense \\
    \textsc{fut1}  \>Near future tense              \>\textsc{sm}   \>Subject marker \\
    \textsc{hab}   \>Habitual                       \>\textsc{r}    \>Breton rannig \\
    \textsc{inan}  \>Inanimate                      \>\textsc{rp}   \>Resumptive pronoun \\
    PN             \>Proper name \\
\end{tabbing}



{\sloppy\printbibliography[heading=subbibliography,notkeyword=this]}
\end{document}
