\documentclass[output=paper]{langscibook}
\ChapterDOI{10.5281/zenodo.5578824}

%This is where you put the authors and their affiliations
\author{Michael Cahill\affiliation{SIL International}}

%Insert your title here
\title{Marking grammatical tone in orthographies: Issues and challenges}  
\abstract{When the concept of how to mark tone in an orthography arises, usually the first thought is to attend to the lexical tone. However, many African languages also employ grammatical tone, and how to indicate this in a practical orthography can involve entirely different approaches. These alternative approaches have significant advantages, but also traps for the unwary. Grammatical tone distinguishes constructions that are otherwise ambiguous, but unlike minimal pairs for lexical tone, contextual clues for disambiguation involving grammatical tone are often absent. Here different strategies for marking are presented, especially semiographic strategies, indicating meaning rather than the phonetics. I also present a warning about using non-Unicode characters.  }

\begin{document}
\SetupAffiliations{mark style=none}
\maketitle

\section{Overview of lexical and grammatical tone}\label{sec:cahill:1}

It is generally accepted that over half the world’s languages are tonal \citep{Yip2002}. In Africa, the percentage is much higher, to the extent that the burden of proof for an Africanist is to show that the language under consideration is \textit{not} tonal. For example, of the 97 Gur languages \citep{Ethnologue2020}, only one, Koromfe, is demonstrably not tonal \citep{Rennison1997}.  

Lexical tone distinguishes one word (lexical item) from another, and this is usually the primary idea that comes to mind when the term “tone language” is used. A few examples are the following:\footnote{For explanation of tone transcription notations used here, see ``Abbreviations and notations" at the end.}

\ea 
Mono [mnh] \citep{Olson2005}

\begin{tabular}{@{}llrclrl@{}} 
 áwá & ‘diarrhea’ & āwā & ‘road’ &  & àwà & ‘fear’ \\ 
\end{tabular}
\ex 
Kɔnni [kma] \citep{Cahill2007}

\begin{tabular}{@{}llrlrc@{}} 
 kpááŋ & ‘oil’ & kpàáŋ & ‘occiput’ & kpá\textsuperscript{!}áŋ  & ‘guineafowl’  \\ 
\end{tabular}
\z
So when the concept of how to mark tone in an orthography arises, usually the first thought is to attend to the lexical tone.


However, many African languages also employ \textit{grammatical} tone, where the tone differentiates different grammatical constructions (e.g. tense and aspect in verbal paradigms, and the pronouns and verbal aspect in the paradigm below: 
\ea 
 Lendu [led] \citep[64]{KutschLojenga2014}, with Mid being unmarked:
 
\begin{tabular}{@{}llll@{}}
 ma rà rǎ  &	‘I went’ & mà rà rǎ  &'we went’  \\ 
 ma rá rá &	‘I should go’ &	mà rá rá & ‘we should go’ \\
 má rǎ rǎ & ‘I am going’	& mǎ rǎ rǎ & ‘we are going’\\
 má ra rá &	‘I will go’ & mǎ ra rá & ‘we will go’\\
\end{tabular}
\z

\begin{sloppypar}
How to indicate grammatical tone in a practical orthography can involve entirely different approaches than marking lexical tone. These alternative approaches have significant advantages, but also traps for the unwary.  The remainder of the paper is structured as follows. \sectref{sec:cahill:2} briefly reviews lexical tone, and some methods that have been used to mark this in practical orthographies. \sectref{sec:cahill:3}, the main section of this paper, first gives a variety of examples of grammatical tone, showing the diversity of functions that can be indicated, and then surveys a multiplicity of ways that such grammatical tone has been marked in orthographies. These include well-known diacritic strategies, but also more imaginative solutions such as unused letters, punctuation, and other non-alphabetic characters. \sectref{sec:cahill:4} brings up some challenges relating to Unicode and non-Unicode compliant characters, and \sectref{sec:cahill:5} concludes with a few practical suggestions.
\end{sloppypar}

The languages discussed in this paper all have Roman-based orthographies. Some, probably all, of the same principles could be applied to Arabic-based orthographies in Africa, but that is beyond the scope of this study. 

\section{Review of marking lexical tone}\label{sec:cahill:2}

\citet{KutschLojenga2014} gives a basic binary typology of tone languages, and notes the types of tone languages which are less likely or more likely to require lexical tone marking for a usable orthography: 

\begin{description}
  \item[Type 1 – low functional load of tone.] This type of language will have few minimal tone pairs, and generally only High and Low tones. It is less likely to need lexical tone marking.
  \item[Type 2 – a high functional load of lexical tone.] This type of language has many minimal tone pairs, often three or more levels of tone, plus multiple contour tones. It is much more likely to need tone marking.
\end{description}

Lexical tone, if marked at all in an orthography, has been marked in a number of ways.

Diacritics are often used. The most common are accent marks, especially the acute accent, e.g. 〈bábá〉. Note that in a practical orthography, a diacritic may not match an IPA-compliant linguistic transcription. An orthographic circumflex accent 〈â〉 could mark a phonetically falling tone, an extra-high tone, or perhaps something else. The principle of one symbol per sound can thus be satisfied in diverse ways. Also, diacritics other than accents, e.g. 〈ä〉, are possible.

Even “full marking” usually marks one tone less than the full tonal inventory (e.g. if H and L are present, only H is marked), as in Budu \citep{Roberts2013}: 

\ea 
Budu [buu] \citep{Roberts2013}, with Low being unmarked

\begin{tabular}{@{}lll@{}}
〈takanaka〉    &  [tàkànàkà]         & dream\\
〈tákánaka〉  & [tákánàkà]&         beg\\
\end{tabular}
\z

Adding otherwise unused alphabetic characters to represent lexical tone is common in Asian languages, but not typically in African ones. In Hani of China, a word-final 〈l〉 marks a high tone, 〈q〉 marks falling, 〈f〉 marks rising, and unmarked is mid. This works better with languages with predominantly monosyllabic words. 

Non-alphabetic characters have been used in some languages with multiple lexical tones, especially in Côte d’Ivoire. Dan would write  〈-kwɛ do 'ka〉 for  [kwɛ̀ dō ká] \citep{Bolli1978}.

If tonal \textit{processes} are identified, the question of underlying vs. surface tone arises (the “levels” question). Though the theory of Lexical Phonology as a total system has been largely abandoned, the division of processes into lexical and post-lexical rules is still referred to. A psycholinguistically real practice which is starting to be increasingly implemented in recent years is to mark the output of the \textit{lexical level} of the phonology \citep{Snider2014,Roberts2013, Roberts2016}.

\section{Marking grammatical tone}\label{sec:cahill:3}
Grammatical tone distinguishes \textit{constructions} that are otherwise ambiguous. But while minimal pairs for lexical tone can often be distinguished by the surrounding context, contextual clues for disambiguation involving grammatical tone are more often absent. Therefore, omitting grammatical tone marking can result in a much more ambiguous and thus less readable orthography. 

\subsection{Examples of grammatical tone}
Cases in which different verb aspects are differentiated solely by tone are not uncommon, and with further typological research, may turn out to be the most common instantiation of grammatical tone. Besides the Lendu examples in (3), a few other cases:

\ea 
Verb aspect     Mbembe, Nigeria [mfn] \citep{Barnwell1969}

\begin{tabular}{@{}llll@{}}
ɔ̀kɔ̂n  &   you sang &	ɔ́kɔ́n &     you should sing\\
ɔ́kɔ̀n &   you have sung & ɔ\textsuperscript{!}kɔ́n   & if you sing\\
\end{tabular}
\ex
Positive vs. negative command	Maa (Maasai) [mas] \citep{Payne2019}

(Low is unmarked here)

\begin{tabular}{@{}ll@{}}
Mɛ́ɪ́sɪsɪ ɔlabánani &	‘you should praise the healer’\\
Mɛɪsɪsɪ́ ɔlabánani &	‘don’t praise the healer’\\
\end{tabular}
\z

Syntactic relations may also be indicated solely by tone:

\ea 
Subject vs. object, Maasai \citep{Tucker1955}

\begin{tabular}{@{}lll@{}}
nominative	&	accusative &\\
èlʊ̀kʊ̀nyá &èlʊ́kʊ́nyá &   ‘head’\\
èncʊ̀màtá &  èncʊ́mátá&   ‘horse’\\
\end{tabular}
\z

Not only verbal forms, but subsets of the nominal system may also be characterized by grammatical tone:

\ea
Singular vs. plural nouns, Mada [mda] \citep{Snider2007}

\begin{tabular}{@{}llll@{}}
\textsc{sg} & \textsc{pl} & { } & { } \\
tsè & tsē&	guineafowl/s&  L/M\\
tʃə̄  & tʃə́& leopard/s& M/H \\
rɛ̀n&  rɛ́n&  pot/s &  L/H\\
gwǎ  &gwá  &snake/s  &L͡H/H\\
\end{tabular}
\ex
Definite vs. indefinite noun,  Bamana Mali [bam] \citep[pc]{Vydrin2016}

Note that the difference shows up on the \textit{following} word:

\begin{tabular}{@{}ll@{}}
jɛ́gɛ́ &  ‘fish’\\
jɛ́gɛ́  tɛ́ yàn&‘\textit{A} fish is not here.'\\
jɛ́gɛ́ \textsuperscript{!}tɛ́ yàn& ‘\textit{The} fish is not here.\\
\end{tabular}
\z
More cases and how they relate to orthography will be introduced in the next section.

\subsection{How grammatical tone has been marked}
\begin{sloppypar}
Marking can be termed either \textit{phonographic} (sound-based) or \textit{semiographic} (mean\-ing-based). See  \citet{Roberts2013} for expansion of these terms. Lexical tone marking is phonographic by definition, but grammatical tone marking may be either phonographic or semiographic. These concepts are independent of what actual symbols are used. Diacritics are one possibility for representing grammatical tone, whether following the phonetics or not:
\end{sloppypar}

\ea
Rangi [lag] \citep{Stegen2005} 

\begin{tabular}{@{}lll@{}}
[adómire] & 〈adómire〉 & `he has gone’ \\
{[}ádómiré{]} & 〈adomiré〉 & `he went’ 
\end{tabular}
\z

Other more imaginative solutions have been used in some languages: using in-line, non-alphabetic characters, e.g. 〈\#baba〉  or  〈//baba〉  or  〈:baba〉 \citep{Roberts2013}. 

\begin{sloppypar}
These non-alphabetic character representations have the advantage of marking the semantics of the construction directly (semiographic representation), thus avoiding the issue of phonological processes and levels altogether. When the reader sees 〈\#baba〉, he knows it is the imperfective, for example, and the phonetics follows naturally. The Attié example below, more extended than most systems, illustrates this in some detail. 
\end{sloppypar}


\ea
Attié [ati] \citep[64]{KutschLojenga2014} (" is extra-High)\\
\begin{tabular}{@{}lllll@{}}
Phonetic	&  & &		Orthographic	 \\
hàn  zè    &  LL 	& &	〈-han  -ze〉	  & ‘we have gone’\\
hàn  zē   &  LM	& &	〈-han   z〉	 &  ‘we are going’\\
hán  zē  &   HM	 &&	〈'han    z〉	   &‘we should go’\\
hán  zè  &   HL	& &	〈'han   -ze〉	   &‘let us go’\\
hàn  ze̋    & LxH	& &	〈-han  "ze〉	  & ‘we didn’t go’\\
\end{tabular}
\z

Shimakonde uses an orthographic 〈h〉 to indicate the low tone of the negative (the /a-/ part of the prefix is often unpronounced and of lesser importance for word recognition, but the tone difference is crucial):

\ea
Shimakonde [kde] \citep[70]{KutschLojenga2014}

\begin{tabular}{@{}rll@{}}
vápáali &	〈vapali〉 &	‘they are present’\\
(a)vapaáli	& 〈havapali〉&	‘they are not present’ \\
\end{tabular}
\z

In Ngangam, 70 percent of verb forms are differentiated only by tone, and the various verb aspects differentiated by tone are indicated by apostrophe or 〈h〉 word-finally:

\ea
Ngangam [gng] \citep{Higdon2000}

\begin{tabular}{@{}lllllll@{}}
 \multicolumn{2}{@{}l}{Imperative} &  \multicolumn{2}{l}{Perfective} &  \multicolumn{2}{l@{}}{Imperfective}\\
 bèré  & 〈bere〉 &   bērè &  〈bere'〉    & bèré	& 〈bereh〉      &  ‘destroy’\\
 ŋɔ̄     & 〈ŋɔ〉 	   & ŋɔ̃́   &   〈ŋɔn'〉  &    ŋɔ̂ &	 〈ŋɔh〉     &  	 ‘dance’\\
 cɔ́kē&   〈cɔke〉&    cɔ́ké&  〈cɔke'〉   &  cɔ́kə̀dē & 〈cɔkedeh〉  &  ‘pierce’
\end{tabular}
\z
In Ngangam above, note that while the symbols 〈'〉 and 〈h〉 consistently represent the relevant syntactic category (the “meaning”), the actual tonal pronunciation varies considerably. This system was accepted by the Ngangam speakers and has aided in their reading.

Budu targets future and past tense by means of punctuation marks in the middle of the written word, after the pronominal prefix:
\ea
Budu [buu] \citep[8]{BamataSubama1997}

\begin{tabular}{@{}lll@{}}
wàɓɛ́n͡dà &		〈wabenda〉 	&	‘you hit’\\
wàɓɛ́n͡dā 	&	〈wa=benda〉& 	‘you will hit’  \\
wǎɓɛ̀n͡dà 	&	〈wa:benda〉&	‘you have hit’\\
\end{tabular}
\z

A proposal for Shilluk, but not implemented, would mark the plurals (which are all L-toned in this language) with a colon:
\ea

Shilluk [shk] \citep{Gilley2004}

\begin{tabular}{@{}llllll@{}}
\textsc{sg} & \textsc{pl} & { } & { } & { } \\ 
líɲ    & lìɲ 	& { } &	〈liny〉  &  〈:liny〉 & ‘war/wars’ \\
pūt̪  &   pùt & { } & 〈pudh〉 & 〈:pudh〉 &‘crippled person/people’    \\
ŋɛ́r &   ŋɛ̀r &	{ } &	〈nger〉  & 〈:nger〉     & ‘antelope/antelopes’ 	\\
\end{tabular}
\z

Tsamakko distinguishes perfective from imperfective by a 〈\sim〉 mark on the perfective construction, while perfective verbs are not marked. The tonal patterns vary in several ways, so the 〈\sim〉 that indicates imperfectivity marks the meaning directly. This was accepted by the community, though reading classes are still beginning, so it is too early to say definitely how this solution affects reading.  
\ea

Tsamakko [tsb]  (Andreas Joswig, pc)

\begin{tabular}{@{}lll@{}}
/ufo vaare vúgí/ &  〈ufo vaare vugi〉  &‘he drank coffee’ \\
/ufo vaare vúgì/ &  〈ufo vaare {\sim}vugi〉  &‘he is drinking coffee’\\
\end{tabular}
\z

Finally, Bafanji has a system of marking several grammatical categories with arbitrary marks to show the category (where the tone is not related to any melody in the utterance). It was introduced and applied to the first published portion of Scripture, the Gospel of Luke, in 2016. Note below that grammatical tone for verb aspect is often marked before the subject noun phrase of the clause. This is because verbs themselves are marked for lexical tone, and these functions of tone are thus clearly separated.
\ea 
Bafanji [bfj] \citep{Hamm2016}
\label{ex:cahill:XlistExample}
\begin{xlist}
\ex \hspace{1ex}Far past/\textsc{p3} (<) placed  before the first element of the subject NP

\textit{Kieʼ \highlight{<}a nchwo ncháŋ nchoo ŋkaʼ ŋgwo chɨʉ la...}

As \highlight{P3}.he was reaching at the gate of the village... 

\ex \hspace{1ex} General future/\textsc{f0} (>) placed as above

\textit{…\highlight{>}meŋ-o ndùu}.

...your \highlight{F0}.child will be well.

\ex  \hspace{1ex} Conditional marker  (\hspace{1ex}̄) placed \textit{on top of} the first letter of the  conditional clause

\textit{ O̅ mbɨ ́Meŋ Mbouʼmbi,...}

\highlight{COND}.If you are the Son of God...

\ex \hspace{1ex} Hortative  marker (=) placed  before the first element of a subject NP

\textit{\highlight{=}A pɨ ́ kieʼ o chú la.}

\highlight{HORT}.Let it be as you said. 
\ex \hspace{1ex} Imperfective  marker (») placed directly after an imperfective verb

\textit{Yiʼ kɨntye no-a ya n:zóʼ\highlight{»} ntye-o? }

What kind of thing am I hearing.\highlight{IPV} about you? 

\end{xlist}
\z

Even though this survey has been brief, we have seen that there is a great variety of strategies for indicating grammatical tone in a practical orthography. While some languages use a phonographic approach to marking grammatical tone, it seems to be increasingly the case that orthography developers are using a semiographic approach, as in the examples above. This takes a variety of forms, ranging from diacritic marks to otherwise unused letters like 〈h〉 to non-alphabetic characters such as punctuation and other marks. The semiographic strategy appears to have advantages over a phonographic representation, in that the reader connects more directly, and presumably with less effort, to the semantics of the construction, not needing to slow down to evaluate the phonetic value of the words.  

\section{The challenge of Unicode}\label{sec:cahill:4}
However, the use of certain non-alphabetic characters can also have drawbacks.\footnote{This section owes much to the 2018 online article \citep{International2018} “Best practice when using non-alphabetic characters in orthographies,” which I wrote together with a technically-informed committee wrote. Much more detail is included there.}

Consider the symbols 〈=  : +   \#  /  '〉 as representative, though not exhaustive. In terms of their characteristics in Unicode, these are not “word-forming characters.”\footnote{“Word-forming character” is not a technical Unicode term, but is convenient and understandable. For more technical detail, see \citet{International2018}.}  That is, one of these symbols is not recognized as part of a word that it is adjacent to; it will function as a word break marker.  Thus if 〈wa:benda〉 from (14) is typed with an ordinary colon, most software programs will split this into two words 〈wa〉 and  〈benda〉, not combining them as one word, as desired. 

Additionally, most software programs will not recognize 〈=baba〉, using an ordinary equals sign, as a word distinct from 〈baba〉. In word searches or any other process in which the user wishes to distinguish these two, an ordinary 〈=, +, \#, :〉, or other character will not appear.  

Thus, ironically, electronic applications may not always work as well in many situations as pen and paper!

However, characters have been developed, proposed, and accepted by the Unicode Consortium which resemble the standard 〈=〉 etc., but do have the property of being word-forming. 

For example, though the characters designated by Unicode as U+A78A 〈꞊〉 and U+A789 〈\char"A789〉 look very similar to the usual equals and colon characters, they are in fact new and different characters which are classified as \textit{modifier letters}, rather than punctuation. They were added to the Unicode standard in version 5.1, and so are available to be used for indicating grammatical tone.

For a more thorough discussion of the issues, challenges, and various alternative strategies for inline non-alphabetic characters, see \citep{International2018}. 


\section{Practical implications}\label{sec:cahill:5}
Many factors enter into the process of developing and using practical orthographies. Acceptability to the local language community is crucial. Actual usability is also crucial. Sometimes these two general principles conflict, as when a hypothetical but realistic language in East Africa has seven vowels and lexical and grammatical tone. The usability criterion indicates that all the vowels and at least some tone should be marked in the orthography, if people are going to be able to use it. However, the language community really admires Swahili, with its five vowels and no tone, and want their orthography to look like Swahili. Usability and acceptability conflict here, and it is not always easy to come to a good solution. The issues addressed in this paper largely deal with usability issues – \textit{can}  people read it? But choice of individual symbols for a particular function, whether grammatical tone or any other orthographic choice, must also take into account the preferences of the local language community. Ideally, they would be in on the decision-making process from Day One. 

The usability criterion can be subdivided into two parts. For ordinary printing and for everyday physical writing on paper or other material, the Unicode considerations are irrelevant. One will not be able to distinguish the normal 〈=〉 from the Unicode word-forming U+A78A 〈{꞊}〉, though if you look closely, you may observe that the latter is slightly shorter. 

And for much cell phone usage, especially casual texting, again the Unicode distinctions between non-alphabetic characters which appear almost identical are irrelevant in most cases.  It is unrealistic and unnecessary to expect the average language user to use the specialized Unicode characters on their unmodified cell phones. In the broader picture, cell phones today are not limited to the normal \textsc{qwertyuiop} English/Roman characters. Keyboards are available that would use many non-Roman characters, and can be downloaded for over 1000 languages at \url{https://keyman.com/}.

However, the other aspect of the usability criterion relates to language-related software which needs to distinguish what is a word and what is not. In particular, SIL’s FLEx software (Fieldworks Lexical Explorer) does better now than it used to in treating apostrophes as potentially word-forming characters, but it does not do so with all non-alphabetic characters. Large digital productions such as dictionaries or Bible translations will benefit from using the word-forming variants of non-alphabetic characters, should they be chosen as part of an orthography. Those working on such projects will be well advised to take the extra time necessary to input the characters that will be compatible with their software’s treatment of the data. Use Unicode word-forming characters rather than the “normal” characters.

Also, the general Unicode principle extends beyond tone marking. Orthography developers and reformers should use Unicode characters whenever and wherever possible, rather than inventing new consonant and vowel graphemes.

We have established that grammatical tone is crucial to represent in the orthographies of many languages. Its very nature, distinguishing closely-related grammatical constructions, means that such tone will very often not be distinguished by context, as is sometimes the case with lexical tone. Rather, some sort of marking must be employed in the orthography to distinguish verb aspects, singular vs. plural nouns, or other categories. Such marking may take an astonishingly wide variety of forms, to accomplish the basic goal of fluent reading by speakers of that language. 



\section*{Abbreviations and notations}\largerpage

Transcriptions used in this paper are the following: 

\begin{multicols}{2}
\begin{tabbing}
 〈 〉 \hspace{1ex}\= orthographic transcription\kill
 a̋ \> extra high tone\\
 á \> high tone\\
 à \> low tone\\
 ā \> mid tone\\
 â \> falling tone\\
 ǎ \> rising tone\\
 \textsuperscript{!}á \> downstepped high tone\\\relax
 [ ] \> phonetic transcription\\
 〈 〉 \> orthographic transcription
\end{tabbing}
\end{multicols}

\noindent Transcriptions without brackets are approximately phonetic.


\section*{Acknowledgements}

Many thanks to the participants at ACAL50 at the University of British Columbia, who gave valuable observations on what started out as a poster. I also appreciate very much my SIL colleagues who provided examples and/or steered me to various data sources.


{\sloppy\printbibliography[heading=subbibliography,notkeyword=this]}

\end{document}
