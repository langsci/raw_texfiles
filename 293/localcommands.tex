\newcommand*{\orcid}{}

\BeforeStartingTOC[toc]{\pagestyle{plain}}
\AfterStartingTOC[toc]{\clearpage}

% % % Shortcuts  (borrowed from JZ, I'm still unsure exactly what xspace requires)
\RequirePackage{xspace}
\xspaceaddexceptions{]\}}

% Useful Ling Shortcuts
\newcommand{\BAR}{$'$\xspace}
\newcommand{\HEAD}{\ensuremath{^\circ}\xspace}
\newcommand{\Lb}[1]{$\text{[}_{\text{#1}}$ } %A more convenient left bracket
\newcommand{\gap}{\underline{\hspace{1.2em}}}
\newcommand{\vP}{\emph{v}P}
\newcommand{\lilv}{\emph{v}}
\newcommand{\Abar}{A$'$-} %A more convenient A-bar notation
\newcommand{\ph}{$\varphi$\xspace} %A more convenient phi
\newcommand{\pro}{\emph{pro}\xspace}
\newcommand{\sub}[1]{\textsubscript{\textit{#1}}} %A more convenient subscript
\newcommand{\hd}{\ensuremath{^\circ}\xspace} %Symbol for printing head / degree symbol
\newcommand{\br}{$'$\xspace} %Symbol for printing head / degree symbol
\newcommand{\spells}{$\Longleftrightarrow$} %spellout arrow for morph spellout rules
%\newcommand{\tr}[1]{\textit{t}\textsubscript{\textit{#1}}} %easy traces with subscript
\newcommand{\subs}[1]{\textsubscript{#1}} %A more convenient subscript
%\newcommand{\hd}{$^{\circ}$\xspace} %Symbol for printing head / degree symbol
\newcommand{\tr}[1]{\textit{t}\textsubscript{\textit{#1}}} %easy traces with subscript
\newcommand{\supers}[1]{\textsuperscript{#1}}
\newcommand{\oden}[1]{⟦#1⟧\subs{$\mbfscrO$}} %requires unicode-math to show up %ordinary denotation
\newcommand{\fden}[1]{⟦#1⟧\subs{$\mbfscrF$}} %requires unicode-math to show up %focus denotation, i.e. gives the focus domain
\newcommand{\regden}[1]{⟦#1⟧} %requires unicode-math to show up %regular denotation marking


% Abbreviations for glossing, based on Leipzig
\newleipzig{hab}{hab}{habitual}
\newleipzig{rem}{rem}{remote}
\newleipzig{fv}{fv}{final vowel}
\newleipzig{nse}{nse}{non-subject extraction}
\newleipzig{pv}{pv}{prothetic vowel}
\newleipzig{se}{se}{subject extraction}
\newleipzig{sm}{sm}{subject marker}
\newleipzig{t}{t}{tense}
\newleipzig{aa}{aa}{anti-agreement}
%\newleipzig{fsg}{\liningnums{1sg}}{first person singular}
%\newleipzig{ssg}{\liningnums{2sg}}{second person singular}
%\newleipzig{fpl}{\liningnums{1pl}}{first person plural}
%\newleipzig{spl}{\liningnums{2pl}}{second person plural}
%\newleipzig{tsg}{\liningnums{3sg}}{third person singular}
%\newleipzig{tpl}{\liningnums{3pl}}{third person plural}
\newleipzig{pron}{pron}{pronoun}
\newleipzig{rec}{rec}{recent}
\newleipzig{om}{om}{object marker}
%\newleipzig{ipfv}{ipfv}{imperfective}
\newleipzig{asp}{asp}{aspect}
\newleipzig{lk}{lk}{linker}
\newleipzig{pcl}{pcl}{particle}
\newleipzig{stat}{stat}{stative}
\newleipzig{ints}{ints}{intensive}
\newleipzig{ascl}{ascl}{assertive subject clitic}
\newleipzig{nascl}{nascl}{non-assertive subject clitic}
\newleipzig{ta}{ta}{tense and/or aspect}
\newleipzig{ni}{ni}{reflex of Proto-Bantu copula \emph{*ni}}
\newleipzig{rm}{rm}{agreeing relative marker}
\newleipzig{assoc}{assoc}{associative marker}
\newleipzig{hon}{hon}{honorific}
%\newleipzig{whprt}{wh}{\wh particle}
\newleipzig{ne}{ne}{(at present) unanalyzed verbal particle occurring across Luyia}
\newleipzig{aug}{aug}{augment}
\newleipzig{sa}{sa}{subject agreement}
\newleipzig{conj}{conj}{conjunction}
%\newleipzig{loc}{loc}{locative}
\newleipzig{nun}{nun}{Korean salience particle}
\newleipzig{expl}{expl}{expletive}
\newleipzig{cj}{cj}{conjoint}
\newleipzig{dj}{dj}{disjoint}
\newleipzig{rcm}{rcm}{reciprocal marker}
\newleipzig{rfm}{rfm}{reflexive marker}
\newleipzig{pers}{pers}{persistive}
\newleipzig{int}{int}{interjection}
\newleipzig{fem}{fem}{feminine}
\newleipzig{aae}{aae}{anti-agreement effect}

%\newleipzig{}{}{}

%% These were removed from Jason Zentz's list. I don't think I need them, but I'm leaving them here for now.
%\newcommand{\Aka}{\Ta{}} %{\Pst{}} %\Rem{}.

%Will draw a circle around a piece of text. Useful for drawing attention to a word in a data example
\newcommand{\circled}[1]{\begin{tikzpicture}[baseline=(word.base)]
\node[draw, rounded corners, text height=8pt, text depth=2pt, inner sep=2pt, outer sep=0pt, use as bounding box] (word) {#1};
\end{tikzpicture}
}


\newcommand{\pil}{\textrightarrow\phantom{m}}

\newcommand{\hlc}[2][yellow]{{\sethlcolor{#1}\hl{#2}}}
 \newcommand{\highlight}[1]{\textbf{#1}}
% \newcommand{\gap}{\underline{\hspace{1.2em}}}


\definecolor{Gray}{cmyk}{0,0,0,0.45}

\newcounter{mycounter}
\newcommand{\myletter}{
        \refstepcounter{mycounter}
        \themycounter}
        \renewcommand{\themycounter}{\alph{mycounter}}
