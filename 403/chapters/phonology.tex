 \chapter{Phonology}\label{ch:2}
\hypertarget{Toc115517752}{}
This chapter begins by introducing the phonemic inventory of Sherbro, first vowels then consonants, discussing some of the variation and analytical problems in their treatment. It next turns to suprasegmental phenomena, tone and syllable structure, and concludes with a presentation of Sherbro's phonological rules.

\section{Phonemic inventory}
\label{sec.2.1}\hypertarget{Toc115517753}{}
The phonemic inventory of Sherbro follows the pattern of other Bolom languages with seven vowels arranged in a symmetrical pattern in the vowel space. Consonants also conform to genetic and areal patterns in the presence of labialvelars and prenasalized stops, though the latter have some distributional peculiarities and are generally voiceless.

\subsection{Vowels}
\label{sec:2.1.1}\hypertarget{Toc115517754}{}
Sherbro vowels are spread out evenly in the accoustic-perceptual vowel space, as represented in the schematic diagram in \tabref{tab:phon:5}, a not uncommon pattern in the area and among Sherbro's closest relatives.

\begin{table}
\caption{\label{tab:phon:5}Sherbro vowels}

\begin{tabular}{lll}
\lsptoprule
 i &  & u\\
 e &  & o\\
 ɛ &  & ɔ\\
& a & \\
\lspbottomrule
\end{tabular}
\end{table}

\subsubsection{/i/}
\label{sec:2.1.1.1}
The high front vowel has no significant allophones in open syllables, being realized invariably as [i]. In closed syllables, as is the case with the other mid to high front vowels, they are lowered and/or centralized; thus there are allophones [ɪ] and [ɨ] in closed syllables. The third example in \tabref{tab:phon:6}, the word for ‘both' is syllabified \textit{li.tiŋ}; only the second /i/ is in a closed syllable and thereby centralized.

\begin{table}
\caption{\label{tab:phon:6}Centralized allophones of /i/}

\begin{tabular}{lll}
\lsptoprule
/bik/ & [bɪk] & ‘a type of mat'\\
/kil/ & [kɪl] & ‘house'\\
/vis/ & [vɨs] & ‘meat, animal'\\
/litiŋ/ & [litɪŋ] & ‘both'\\
\lspbottomrule
\end{tabular}
\end{table}

In borrowings from English, the high vowel [i] is realized as [e] in Sherbro, thus suggesting that the Sherbro vowels /i/ and /e/ are both higher than their English counterparts. Though no acoustic measurements were made, impressionistically the Sherbro [e] does sound higher than its English counterpart.

\ea %14
\label{ex:14}
%(\stepcounter{qwerty}{\theqwerty})  
Kɔfe lɛ hɔ lol.\\
\gll kɔfe    lɛ    hɔ    lol\\
coffee  \textsc{def}  \textsc{ncp}\textsubscript{hɔ}  bitter\\
\glt ‘The coffee is bitter.' (P67 L: 104)\footnote{The material in parentheses indicates the example is from Walter Pichl's 1967\ia{Pichl, Walter} Sherbro-English Dictionary (see \sectref{sec:2.1.1}). The letter indicates the section, and the following number is the specific entry. Thus in \REF{ex:14}, the example sentence is from the entry for the word \textit{lol} which is the 104\textsuperscript{th} entry under the letter L.}
\z

\subsubsection{/e/}
\label{sec:2.1.1.2}
In open syllables, both mid front vowels have higher phonetic values than are indicated by their symbols. In closed syllables, they have centralized allophones [ɪ] and/or [ə], more often the latter, particularly with a coda filled by a liquid or nasal.

\begin{table}
\caption{\label{tab:phon:7}Centralized allophones of /e/}


\begin{tabular}{lll} 
\lsptoprule
/len/ & [lɪn] & ‘thing'\\
/yel/ & [yɪl] & ‘boil' (v.)\\
/kel/ & [kəl] & ‘monkey, bite'\\
\lspbottomrule
\end{tabular}
\end{table}

One example of [ə] in an open syllable and not in a closed one is [peŋkə] ‘first' (also heard as [peŋkɛ]).

Both of the lower and higher mid vowels [e ɛ] and [o ɔ] are higher than their English counterparts, and definitely are not diphthongized. When a word such as \textit{plane} ([pleɪn]) is borrowed into Sherbro, the vowel is reanalyzed as the lower mid vowel /ɛ/ rather than the upper mid vowel /e/.

%(\stepcounter{qwerty}{\theqwerty}) 
\ea %15 
\label{ex:15}
Plɛn dɛ kɔn poto kɛthkɛth hink Kyamp ka.\\
\gll plɛn  dɛ    kɔn  poto    kɛthkɛth    hink  Kyamp    ka\\
plane \textsc{def}  go    Europe  often      from  Freetown  here\\
\glt ‘The plane goes frequently from Freetown to Europe.' (P67 K: 114)
\z

These facts have led to some confusion in the colonial orthography which has persisted to this day. Westerners, including many mapmakers, have confused /u/ and /o/ in particular. For example, the name of the language, here rendered <Bolom>, is often spelled “Bullom,” and [Bom], the name of a river and once the name of a language have both been written “Bum” now recognized as “Bom-Kim\il{Bom-Kim}” (\citealt{Childs2020}). The name of the major girls' initiation society ‘Bondo\il{Bondo}' is spelled <Bundu>, and the port of ‘Tombo' is often rendered as <Tumbu>.

To a lesser extent the front vowels /i/ and /e/ are also confused.

\subsubsection{/ɛ/}
\label{sec:2.1.1.3}
The centralized [ə] occurs in more environments as an allophone of /ɛ/ than it does as an allophone of /e/.

\begin{table}
\caption{\label{tab:phon:8}Centralized allophones of /ɛ/}

\begin{tabular}{lll}
\lsptoprule
/ayɛn/ & [ayən] & ‘truly'\\
/pɛl/ & [pəl] & ‘fishing net, hammock'\\
/pɛmplɛ/ & [pəmp.lɛ] & ‘stalk' (v.)\\
/yɛk/ & [yək] & ‘bedbug' \\
/bɛth/ & [bət̪] & ‘cut' (v.)\\
\lspbottomrule
\end{tabular}
\end{table}

In all cases the schwa allophone is unrounded but, in some cases, can be rhotacized, having almost a retroflex quality: [wantsɚ, wantsɚmi] ‘sister, my sister,' even without a coda consonant.

Diachronic records show that the alternation is possibly a recent phenomenon. \citet{Sumner1921}  gives, for example, ‘earthenware pot' as [bɛl], yet the form is [bəl] in both \citet{Pichl1967} and my own data from 2015-2016. Another explanation would be that Sumner was hearing the sounds phonemically; he was a native speaker.\footnote{Sumner was also a minister in the Methodist Church.}

The vowel [ə] is therefore not considered an independent phoneme, as it is in Mani\il{Mani}. The central vowel allophones range over a considerable part of the acoustic-perceptual vowel space, ranging from a high central allophone [ɨ] to a lower central one [ə], as well as a front centralized [ɪ] and a rhotacized [ɚ]. Despite the many pronunciations, speakers had no problems identifying the sound as one of the phonemic vowels, usually /e/ or /ɛ/. Furthermore, the Sherbro Literacy Committee\is{Sherbro Literacy Committee!} and Methodist missionaries have not used <ə> in the writing system that they have developed and thus do not see it as a separate phoneme.\footnote{Most of the committee's members are native speakers.} \citet{Sumner1921} has included the sound but does not comment on its phonemic status, using a quasi-phonetic system for his transcriptions although \citet{Pichl1967} treats it as an allophone of /ɛ/.

Vowel length is contrastive, and all vowels have long counterparts. However, vowel length is only contrastive for monosyllables or on the first syllable of a disyllabic word. For example, the word for ‘bird' is [vee], but when it appears in a compound, the vowel is shortened: [vebolmin] ‘swallow,' lit. ‘crazy bird.'

On the phonetic side, vowels are compensatorily lengthened before an epenthetic (prenasalized) stop, thus [mbaŋkathoːm ndɛ] from /mbaŋkathom + ɛ /. Below appear some phonemic contrasts in length for which there are a great many minimal pairs.

\ea %16
\label{ex:16} Length contrasts in vowels\\
%(\stepcounter{qwerty}{\theqwerty}) 
\vspace{6pt}
\begin{tabular}[t]{llll}
ha & ‘for' & haa & ‘do' \\
yɛk & ‘bedbug'  & yɛɛk & ‘spoon' \\
ya & ‘\textsc{1sg}'  & yaa & ‘mother'\\
vɛ & ‘that one' & vɛɛ & ‘stone' (v.)\\
ve & ‘be well'   & vee & ‘oyster'\\
ku & ‘call, name'   & kuu & ‘property, estate'\\
gbɛŋ & ‘tomorrow' & gbɛɛŋ & ‘glory'\\
\end{tabular}
\z

Long consonants occur only across morpheme boundaries, typically as a result of syllable restructuring (see \sectref{sec:2.1.2} for some examples of geminates and \sectref{sec:2.3} for syllable restructuring in general).

As with other Bolom languages, there are alternations between front and back vowels: i-u, e-o, and ɛ-ɔ. Speakers accept both front and back alternants for a number of forms, suggesting free variation, despite the phonemic status of the opposition between front and back vowels. For example, the pronunciation of /jo/ ‘eat' was equally acceptable as either [jo] or [je] to some speakers in Shenge\is{Shenge}. As another example, both consultant Adama Mampa and research assistant Abdulai Bendu thought that the pronunciation of ‘tomorrow' as either [gbɛŋ] or [gbɔŋ] was acceptable.

\ea %17
\label{ex:17}
Some front-back alternations in free variation\\
%(\stepcounter{qwerty}{\theqwerty}) 
\vspace{6pt}
\begin{tabular}{ll}
sikɔ / sukɔ & ‘the mast of a ship'\\
wei / woi & ‘fear'\\
thɛm / thɔm & ‘friend'\\
wɛi / wɔi & ‘bad, be ugly'\\
pɛ / pɔ & ‘people'; \textsc{pro}\textsubscript{indef}
\end{tabular}
\z

The variation between the front and back alternants seems to be unconditioned.\footnote{These alternations lead to some variant spellings.} In at least one other closely related language, the alternations have been morphologized; they mark contrasts in the verbal morphology of Kisi\il{Kisi}. In Mani\il{Mani}, there is a trace of vowel harmony\is{vowel harmony!} in the applicative verb extension, perhaps a source for the alternations. In Sherbro, there is a limited case of vowel harmony in the past suffix and in the derivational morphology (see \sectref{sec:4.2}, \sectref{sec:7.1}).

\subsection{Consonants}
\label{sec:2.1.2}\hypertarget{Toc115517755}{}
\tabref{tab:phon:9} presents the consonants of Sherbro. I have used the orthographic symbols of the writing system rather than IPA symbols, as follows. The digraphs represent single phonemes. The symbol “v” is in parentheses because it is an allophone of /w/ but used in the writing system. The symbol “kp” in parentheses, on the other hand, represents a peripheral sound found only in a few words. The voiced prenasalized stops, also in parentheses, similarly have a limited distribution. I make only a few comments on the unusual phonetic and distributional features of Sherbro consonants. 

\begin{table}
\caption{\label{tab:phon:9}Sherbro consonants}

\begin{tabular}{ccccccc}
\lsptoprule
 Bilabial & Dental & Alveolar & Palatal & Velar & Labialvelar & Glottal\\
 \midrule
 m &  & n & ny & ŋ &  & \\
 mp, (mb) &  & nt (nd) &  & ŋk (ŋg) &  & \\
 p, b & th & t, d & ch, j & k & (kp) gb & \\
 f (v) &  & s &  &  &  & h\\
&  & r, l &  &  &  & \\
&  &  & y &  & w & \\
\lspbottomrule
\end{tabular}
\end{table}

\subsubsection{Nasals}
\label{sec:2.1.2.1}
Nasals show little variation except for the nasal assimilation described in \sectref{sec:2.4}. Coda alveolar and velar nasals nasalize preceding vowels. There is also some variation in the lexical form of the final nasal, whether it is individual or dialectal could not be determined. In fact, there is some variation within individuals. The velar nasal alternates with [h] in many forms, as discussed in Orthography and Conventions and below under /h/ (\sectref{sec:1.9} \& \sectref{sec:2.1.2}). The palatal nasal never appears in codas and is a dialectal variant of /h/ before front vowels, e.g., [nyɔl] / [hiɔl] ‘four'. For those interested in “the linguist's delight”, a minimal triplet featuring three of the nasals in word-final position is [pɛŋ] ‘boundary; jump' vs. [pɛn] ‘loud talking' vs. [pɛm] ‘war.'

\subsubsection{Prenasalized stops}
\label{sec:2.1.2.2}
Prenasalized stops in Sherbro form a special category because of their distribution and their phonetics. Although there is some dialectal variation, phonemic prenasalized stops consist of a nasal followed by a homorganic voiceless stop; “voiceless” prenasalized stops are established phonemes. In a few medial contexts, the sequence may be voiced. Voiceless prenasalized stops occur almost exclusively in syllable codas, as exemplified in \REF{ex:18}, but only at four of the six places of articulation for Sherbro stops and three of the four for nasals.


\ea  %18
%\label{bkm:Prenasalizedstopsinsyllablecodas}(\stepcounter{qwerty}{\theqwerty}) 
\label{ex:18}
Prenasalized stops in syllable codas\\
\vspace{6pt}
\TabPositions{1.25cm,2cm,3cm,5cm}

mp \tab \textit{bomp} ‘section,' \textit{kump} ‘helper; plaiting,' \textit{nrɔmp} ‘sickness'\\
nt \tab \textit{tɔnt} ‘creek,' \textit{ntɛnt} ‘near,' \textit{tunt} ‘twist'\\
nth \tab \textit{vunth} ‘push,' \textit{panth} ‘tie' (v.), ‘work' (n.), \textit{santhsanth} ‘grownup'\\
*nch\\
ŋk \tab \textit{tɔŋk} ‘praise' (v.), \textit{thuŋk} ‘deep,' \textit{thɛŋk} ‘put up,' \textit{yeŋk} ‘insect wax'\\
*mŋgb\\
\z

There is some variation between voiceless and voiced prenasalized stops in intervocalic position, e.g., ‘today' recorded as both [nante] and [nande], ‘mango' (a borrowing\is{borrowing}) as both [maŋko] and [maŋgo], with the voiceless variant being the more common one. The voiceless prenasalized stop is sometimes the only variant medially, e.g., [kaŋka] ‘so that,' [peŋka] ‘gun'. Word-initially only a few forms with voiceless prenasalized stops can be found, though they do occur as in \textit{ntent} above in \REF{ex:18} and in the evening greeting \textit{mpikɛ}. Thus, the voiceless prenasalized stops have a stronger claim to phonemic status than voiced ones, particularly in light of the latter's derived status described below.

Only at the beginning of syllables do sequences of voiced prenasalized stops occur, almost all derived. They arise due to the prefixing of a nasal morpheme with a loss of syllabicity and nasal assimilation. Otherwise, voiced prenasalized stops appear only in a few function words, names and borrowings. Voiced prenasalized stops are not considered to be independent phonemes in Sherbro.

\subsubsection{Voiceless stops}
\label{sec:2.1.2.3}
The one unusual feature of voiceless stops is that the language has a contrast between a dental stop and an alveolar one (<th> vs. <t> in the orthography).

\subsubsubsection{/t̪/ (<th>) and /t/}
\label{sec:2.1.2.3.1}
The dental stop /t̪/ has a diagnostic “tinny” sound impressionistically, distinctive from the unaspirated alveolar [t], which is often affricated, e.g., \textit{tu} ‘pound (in a mortar)' is realized as [ʧu] or [ʦu].

\TabPositions{1.25cm,3.75cm,4.75cm,6.75cm,7.75cm,8cm,9cm}
\ea %19
\label{ex:19}
%(\stepcounter{qwerty}{\theqwerty}) 
Dental and alveolar voiceless stops\\
\vspace{6pt}
thetha \tab ‘grandmother' \tab tɛtɛk \tab ‘immature rice' \tab tɛnthe \tab ‘cane stick'\\
thɔk \tab ‘stick' (n.) \tab tɔkɔ \tab ‘about'\\
thu \tab ‘spit' (v.) \tab tu \tab ‘iron pot'\\
thuk \tab ‘be warm' \tab tuk \tab ‘disappear, be lost'\\
\z

\subsubsubsection{/p/ and /k/}
\label{sec:2.1.2.3.2}
There is nothing much to say about the other voiceless stops /p/ and /k/. There are no allophones for the voiceless velar stop. When a voiced counterpart [g] appears in a borrowed word, the borrowing\is{borrowing!} is nativized with a [k], as in \textit{mango} [maŋko] above and [bek]‚ bag,' though some learned borrowings may retain the [g], e.g., \textit{gɔvana} ‘governor'. Similarly there is nothing remarkable about the voiced stops /d/, /b/, /gb/, nor about the affricates /ʧ/ and /ʤ/, although there is some dialectal variation between [ʤi] and [di] for ‘kill' (also reported in \citealt{Pichl1967}).

\subsubsection{Voiceless fricatives}
\label{sec:2.1.2.4}

The voiceless fricatives /f/ and /s/ have no allophones, except for the palatalization of /s/ to [ʃ] before the non-low front vowels.

%(\stepcounter{qwerty}{\theqwerty})   
\ea%20 
\label{ex:20}
\begin{tabular}[t]{lll}
/s/: &  [ʃ] & / {\longrule} V [-lo, -bk]\\
& [s] & elsewhere\\
\end{tabular}

\vspace{6pt}

\begin{tabular}[t]{llll}
/sii/ & [ʃii] & ‘fart' (v.)\\
/setie/ & [ʃeʧie] & ‘Settie' (a chiefdom on Sherbro Island\is{Sherbro Island})\\
/isundɛ/ & [iʃundɛ] & ‘sand' (n.)\\

/biisi/ & [biiʃi] & ‘make tight'\\
/si/ & [ʃi] & ‘know'\\
/silɔ/ & [ʃilɔ] & ‘honey, bee'\\
\end{tabular}
\z

The two examples in \REF{ex:21} come from an early source indicating that [s] / [ʃ] is a long-standing alternation, likely below the level of consciousness of speakers (vs. the dialectal variation of [h] / [w] in \REF{ex:28}) (\citealt{Sumner1921}).

\ea %21
\label{ex:21} [s] / [ʃ] alternation\\
\vspace{6pt}

%\label(\stepcounter{qwerty}{\theqwerty})    
\begin{tabular}[t]{llll}
\relax [simi] \textasciitilde{} [ʃimi]\footnotemark & ‘spoil, become rotten'\\
\text{[yikisi]} \textasciitilde{} [yikiʃi] & \textsc{idph} wiggling gait of a woman (\citealt[21]{Sumner1921})\\
\end{tabular}
\footnotetext{I have changed Sumner's <sh> to [ʃ].}
\z

There are also the alternations [sɔ]{\textasciitilde}[ʃɔ] ‘(in the) morning' and [sɔ]{\textasciitilde}[ʃɔ] ‘till' (a field) (v.), where no conditioning high front vowel appears. A possible explanation comes from the closely related language Kisi\il{Kisi}. The words for ‘morning' and ‘till' in that language are respectively /sìɔ̀/ and /sìɔ̀ɔ́/, where the conditioning [i] vowel is still realized.

Curiously, the alveo-palatal fricative [ʃ] begins the name of the language and the people <Sherbro>. This exonym contrasts with the group's autonym [bolom] (rendered <Bolom> or <Bullom> in various sources), the name now given to the subgroup to which Sherbro belongs (\citealt{Childs2024c}).

In the interests of completeness, I mention the unexpected alternation of [s] with [n] in [si] / [ni], both representing the all-purpose connector ‘with, and,' which also posits temporal and logical relations between clauses. This alternation may represent the diachronic collapse of a former semantic distinction between the two words.

There is also a dialectal alternation in the word for ‘hand' [fi] in the north in Bengeh\is{Bengeh} (also spelled Benge), Bumpeh Chiefdom\is{Bumpeh Chiefdom}  and [sui] in the south in Kagboro Chiefdom. Since [sui] is related to the word for ‘finger' in other related languages, the [s] variant may be older.

\subsubsection{/l/}
\label{sec:2.1.2.5}
The alveolar lateral has no distinct allophones and appears in onsets as well as in codas. Unusually, the lateral can be geminated, as happens also in closely related languages (discussed in \sectref{sec:2.4} as part of a more general process of onset strengthening). As the only long consonant found in the language, the geminate [l] occurs across morpheme boundaries. Epenthesis occurs before the definite marker \textit{ɛ} in \REF{ex:22}. The geminate arises before the question particle \textit{a} in \REF{ex:22} (see also examples in \REF{ex:44}).

\ea %22
%\label{bkm:geminatelallophone}(\stepcounter{qwerty}{\theqwerty})
\label{ex:22}
\ea I amɛn bullɛ ka koŋ wu.\\
\gll hi    a-mɛn    bul  ɛ    ka        koŋ  wu\\
\textsc{1pl}  \textsc{ncm}\textsubscript{ha}{}-five  one  \textsc{def}  \textsc{rem.pst}    \textsc{pfv}  die\\
\glt ‘We are five, one died a while ago.' (007a Agnes J. Simbo: 27)\\
\ex Kɛ mi ŋa mɔ ilella?\\
\gll kɛ    mi      ŋa    mɔ  i-lel        a\\
but  mother  what  \textsc{2sg}  \textsc{ncm}\textsubscript{hɔ}{}-name  \textsc{q}\\
\glt ‘But, Mummy, what is your name?' (007a Agnes J. Simbo: 8)
\z
\z

\subsubsection{/r/}
\label{sec:2.1.2.6}
Phonetically, /r/ is realized as central [ɹ] or even a retroflex approximant [ɻ] in its most common manifestations. \citet{Pichl1967} reports it as a trill [r]. In our work, we heard it as an alveolar tap or trill in pre-vocalic position and as a retroflex central approximant in syllable codas. In Sherbro, just as in Mani, it is a phoneme exhibiting a great deal of variation both phonetically and dialectally\il{Mani}. The phoneme is absent in Bom-Kim\il{Bom-Kim} and in the southern dialect of Kisi\il{Kisi}.

In the coda, /r/ obscures vowel quality in the nucleus as formants are damped and vowels are perceived as more centralized. The same effect occurs with the liquid /l/ or a nasal in the coda, as discussed above. V-/r/ metathesis\is{metathesis} can occur, especially when /r/ is in the coda of a high front vowel, as in the American English alternation, \textit{professor} and the somewhat colloquial or regional \textit{perfessor}. Other variants are possible for such Vr sequences, as illustrated with /tir/ ‘town' and similar words in \tabref{tab:phon:10}. The word for ‘ripe' /dir/ was also pronounced with something like a pharyngeal fricative [ʢ] in place of /r/ in several instances (not shown below), an allophone also found in Mani\il{Mani}.

\begin{table}
\caption{\label{tab:phon:10}Vr/ variation (cf. r/$\emptyset$ alternation in \tabref{tab:phon:11})}

\small

\begin{tabularx}{\textwidth}{lllQQQQl} 
\lsptoprule
& [iɹ] & [ɨɹ] & [əɹ] & [ɚ] / [dɹ̩] & [ɾi] & [ɹə] & [ri]\\
\midrule
/tir/ ‘town' & [tiɹ] & [tɨɹ] & [təɹ] & [tɹ̩] & [tɾi] &  & [tri]\\
/dir/ ‘red, ripe' & [diɹ] &  & [dər] & [dɚ] / [dɹ̩] &  & [dɹə] & \\
/bithir/ ‘bottle' & [bithiɹ] & [bitɨɹ] &  &  &  &  & \\
/kɛntir/ ‘groundnut' & [kɛntiɹ] &  &  &  &  &  & [kɛntri]\\
/kirkir/ ‘round' &  &  &  &  &  &  & [krikri]\\
\lspbottomrule
\end{tabularx}
\end{table}

One dialectal alternation is between [r] and [w], as in the word for ‘push,' pronounced [runt̪] around Shenge\is{Shenge}, shown in \REF{ex:23}. In the Bengeh\is{Bengeh} (also spelled Benge), Bumpeh Chiefdom\is{Bumpeh Chiefdom} dialect to the north, the word is pronounced [wunt̪]. The speaker comes from around Shenge. The sentence is followed by two single-word examples of the alternation.

\TabPositions{3cm,8cm}

\ea%23
%\label{bkm:rwalternation}(\stepcounter{qwerty}{\theqwerty}) 
\label{ex:23}
[r] / [w] alternation\\
\ea \label{ex:23a} 
rɔm / wɔm \tab ‘medicine'\\
rokos / wokos \tab ‘lime'\\
runth / wunth \tab ‘push'

\ex \label{ex:23b}
Lɛ nɔsɛ ha ni gbo kɛkɛ, nrunth gbo, mɔ gbo runth libul, komɔɛ koŋ honi.\\
\gll lɛ  nɔs  ɛ    ha    ni    gbo  kɛkɛ    n    runth    gbo mɔ  gbo  runth    li-bul        komɔ    ɛ    koŋ  honi\\
if  nurse  \textsc{def}  do    \textsc{neg}  just  quickly  \textsc{2sg}  push    just \textsc{2sg}  just  push    \textsc{NCM}\textsubscript{lɔ}{}-one    child    \textsc{def}  \textsc{pfv}  go.out\\
\glt ‘If the nurse does not make it fast, you just push, you just push once, and the baby emerges.' (002a Mabel Lohr, Midwifery: 53)
\z
\z

The “r” in Vr sequences also alternates with “${\emptyset}$,” as shown in \tabref{tab:phon:11}. Sometimes the r-less variant will have a long vowel in the place of the Vr sequence as perhaps a case of compensatory lengthening. Sumner, a native speaker of Sherbro, wrote ‘hoe' as <kar> (\citealt{Sumner1921}), which was recorded in our data with a long vowel [kaa].

\begin{table}
\caption{\label{tab:phon:11}[r] / ${\emptyset}$ alternation (cf. Vr variation in \tabref{tab:phon:10})}
\begin{tabular}[t]{lll}
\lsptoprule
təɹ & tə & ‘waist'\\
her & he & ‘cross'\\
gber & gbe & ‘many, much'\\
pɛr & pɛ & ‘fill'\\
kɛrko & kəko & ‘squirrel'\\
bithir & bithiː & ‘bottle'\\
\lspbottomrule
\end{tabular}
\end{table}

These facts, coupled with the [w] / [r] alternation shown in \REF{ex:23}, underscore the instability of /r/ in Sherbro. Another alternation noticed by a previous writer but not present in our work was between [l] and [r]. \citet{Hanson1979a} noted that both [l] and [r] were heard intervocalically between high vowels.

\ea%24
%(\stepcounter{qwerty}{\theqwerty})
\label{ex:24}
\begin{tabular}[t]{ll}
\relax [čɨrɨŋ] / [čɨlɨŋ] & ‘safe'\\
\relax [pilinni] / [pirinni] & ‘to walk around something' (\citealt[25]{Hanson1979a})\\
\end{tabular}
\z

Neither word appeared in our own data, but an earlier source has both the [l] and the [r] forms for ‘walk around something' (\citealt{Pichl1967}). \citet{Hanson1979a} maintains that [l] “is the actual phoneme.”

The /r/ phoneme exhibits similar instability elsewhere in Bolom-Kisi. For example, the \textit{r/l} contrast has been neutralized in southern dialects of Kisi\il{Kisi} to \textit{l} with significant consequences for the noun class morphology (\citealt{Childs1983}).

Another place where metathesis\is{metathesis} occurs, albeit much less frequently, is with the nasal [n], another resonant. Here the alternation for ‘kneel' is between [bitni] and [bitin] with a perhaps intermediate alternant of [bitəni].

%(\stepcounter{qwerty}{\theqwerty})  

\ea%25
\label{ex:25}
[bitəni] / [bitni] / Pɔ anyaɛ ŋa bitin chɔchai.\\
\gll bitəni    bitni    pɛ      a-nya        ɛ    ŋa    bitni    chɔch-ai\\
kneel    kneel    \textsc{pro}\textsubscript{indef}   \textsc{ncm}\textsubscript{ha}{}-people  \textsc{def}  \textsc{3pl}  kneel    church-in\\
\glt ‘kneel' / ‘kneel' / ‘People kneel in church.' (E14 Albert Yanker: 31)\footnotemark 
\footnotetext{Data citations in the E series are elicitation sessions whose transcriptions can be viewed in the FLEx database for Sherbro lexicon in the Endangered Languages Archive (ELAR).}
\z

\subsubsection{/h/}
\label{sec:2.1.2.7}
Because of heavy nasalization of the vowel after [h], the initial sound is often heard as [ŋ] or as [ɲ] before high front (palatal) vowels (see \sectref{sec:1.9} for ramifications in the orthography).
\clearpage
%(\stepcounter{qwerty}{\theqwerty}) 
\ea%26
\label{ex:26}{[h] / [ŋ] alternation}\\
\vspace{6pt}
\begin{tabular}[t]{ll}
 [ha] / [ŋa]    &  \textsc{subord} \textsc{conj}\\
 {[haa]} / [ŋaa]     & ‘do, make'\\
 {[hɔ]}\footnotemark{} / [ŋɔ] & ‘how'\\
\end{tabular}
\footnotetext{Paramount Chief Madam Doris Lenga-Caulker  Gbabiyor\is{Lenga-Caulker Gbabiyor, Doris} insisted that [hɔ] was the “correct” pronunciation. My suspicion is that she is right that it represents the older form before the advent of the Sherbro Literacy Committee\is{Sherbro Literacy Committee} (see \sectref{sec:1.9} for an account of how the ŋ / h spelling was used to distinguish homonyms.)}
\z

The two orthographic representations <ŋa> and <ha> of the homonymous pair [ha] has been exploited by the Sherbro Literacy Committee to differentiate functionally different forms. Although tone distinguishes some of the pairs, even that, coupled with the distinct spelling, does not differentiate all of the homonymous forms (see \sectref{sec:2.2}).

Another /h/-relevant alternation is between [h] and [w]:

%(\stepcounter{qwerty}{\theqwerty})   
\ea%27
\label{ex:27}{[h] / [w] alternation}\\
\vspace{6pt}

\begin{tabular}[t]{ll}
hɔŋgul / wɔŋɡul & ‘sell'\\
hɔ / wɔ & ‘say'\\
hɔ / wɔ & the class pronoun (see the discussion of /w/)\\
\end{tabular}
\z

The [h] variant seems to be more common in the north in the Bengeh\is{Bengeh} (also spelled Benge), Bumpeh Chiefdom\is{Bumpeh Chiefdom} dialect. The [h] / [w] alternation in ‘say' [hɔ] / [wɔ], sometimes coupled with the front-back alternation [ɛ] / [ɔ] can lead to some confusion for non-native speakers.

A final /h/ alternation is between [h] and ${\emptyset}$, as in [hɔbatokɛ] / [ɔbatokɛ] ‘God,' [hi] / [i] ‘we,' and [hina] / [ina] ‘who.'

\subsubsection{/w/}
\label{sec:2.1.2.8}
Common allophones of /w/ at the beginning of a word are the fully devoiced variant [ʍ] or a partially devoiced one [hw], which might explain the alternation between [h] and [w] discussed above.\footnote{Equally valid transcriptions are [w̥w] and [ʍw].} Some additional examples not presented there can be found in \REF{ex:28}.

\ea%28 
\label{ex:28}{More [h] / [w] alternation}\\ 
\vspace{6pt}
\begin{tabular}[t]{ll}
hwɛ / wɔ & ‘world' (coupled with front-back vowel alternation)\\
hun / wun & ‘come'\\
hu / wu  & ‘die'\\ 
hɔl / wɔl  & ‘eye'\\
\end{tabular}
\z

The labialvelar glide /w/ has an allophone [v], which is treated as a separate letter in the writing system of the Sherbro Literacy Committee\is{Sherbro Literacy Committee}, likely due to the influence of English where <w> and <v> represent distinct phonemes. The Sherbro Literacy Committee\is{Sherbro Literacy Committee} recommendations are followed in this grammar and dictionary, but the alternation is predictable as represented below.\\

\begin{tabular}[t]{lll}
/w/: &  [v] &  / {\longrule} V [-lo, -bk]\\
 & [w] & elsewhere\\
\end{tabular}\\

The distribution of these allophones parallels that found elsewhere in Bolom. Uniquely, however, /w/ has a dialectal variant of [h], as seen in \REF{ex:27} and \REF{ex:28}. Although the precise isogloss of the [w] / [h] alternation cannot be stated, [w] is more often heard with northern and interior speakers, e.g., from Bengeh (also spelled Benge), Bumpeh\is{Bengeh} Chiefdom, rather than with speakers from Shenge\is{Shenge}.

\subsubsection{/y/}
\label{sec:2.1.2.9}
The palatal glide also alternates with [h] in the word for ‘boil' as in \REF{ex:29}.

%(\stepcounter{qwerty}{\theqwerty})
\ea%29
\label{ex:29}
Mɛndɛ ma koŋ yɪl / hɪl.\\
\gll mɛn  ɛ    ma    koŋ   yel / hel\\
water  \textsc{def}  \textsc{ncp}\textsubscript{ma}    \textsc{pfv}  boil / boil\\
\glt ‘The water is boiling (has reached a boiling state).' (E08 Albert Yanker: 14)
\z

Other alternations are in the \textsc{1pl} \textsc{pro} [hi] and [yi], the word for ‘salt' [ihɛl] beside [iyɛl], and the word for ‘four' pronounced both as [yɔl] and [hiɔl] (see the discussion of /ny/ above).

\section{Tone}
\label{sec:2.2}\hypertarget{Toc115517756}{}
On the basis of comparative evidence, historically Sherbro was undoubtedly a tone language; tone was once likely used to mark both lexical contrasts and distinctions in the verbal morphology. Today, because the language has fallen into desuetude, much as is the case with its closest relatives, lexical tone is elusive though grammatical tone is still found in a few environments (\citealt{Childs2002a}).

Motivation for an earlier more tonal state comes from comparative and historical evidence. Both lexical and grammatical tone are found in other Bolom-Kisi languages that are still vital (Mani\is{Mani} and Kisi\il{Kisi}), though tone is less prominent in the most moribund Bolom language Bom-Kim\il{Bom-Kim}. Throughout Mel\il{Mel} in general, the greater group to which the Bolom-Kisi sub-group belongs with Temne\il{Temne}-Baga, tone contrasts have been identified (e.g.,\citealt{Wilson1968}). It is thus likely that tone is reconstructible, as is not the case with Atlantic, the language group to the north with which Mel was once associated but has now been disassociated (\citealt{Childs2004}, \citealt{Childs2024a}). I begin with a characterization of lexical tone in Sherbro.

Sherbro has at least two tones, high and low, as illustrated by the minimal pairs in \REF{ex:30}, representing both grammatical and lexical tone contrasts.

\ea%30
\label{ex:30}Some tonal contrasts\\
\ea \label{ex:30a}
\begin{tabular}[t]{llll}
há / ŋá\footnotemark & ‘you (pl.)' &hà / ŋà & ‘they'\\
háá & ‘do' (optative) & hàà & ‘did' (perfective)\\
kíth & ‘small, short' & kìth & ‘hard to drink'\\
rá & ‘a type of snake' & rà & ‘three'\\
wál & ‘palm leaf' & wàl & ‘resting place'\\
wáŋ & ‘girl' & wàŋ & ‘ten'\\
\end{tabular}
\vspace{6pt}
\ex wáŋ mà àwàŋ\\
\gll wáŋ mà à-wàŋ\\
girl \textsc{ncp}\textsubscript{ma} \textsc{ncm}\textsubscript{ha}-ten\\
\glt ‘ten girls'
\footnotetext{The [h]/[ŋ] alternation, as discussed in \sectref{sec:2.1.2},  represents a spelling rather than a phonemic contrast (see \sectref{sec:1.9}).}
\z
\z

The personal pronoun paradigm contains a minimal pair, first noticed in \citet[13]{Sumner1921} and deemed “important” enough to be marked in his proposed writing system. (Tone was generally not marked in early studies.) The second-person plural pronoun has a high tone, thus \textit{há} (or \textit{ŋá}), and the segmentally identical third-person plural pronoun has a low tone \textit{hà} (or \textit{ŋà}) (see \sectref{sec:1.9} for discussion of the <h/ŋ> variation).

An early study gives the following tonal n-tuplets (\citealt[35]{Sumner1921}; \tabref{tab:phon:12}). Pronouns have been omitted by Sumner for the third-person singular; they often go unexpressed. The main contrast in the verbal morphology is aspectual, which in this book are labelled perfective and imperfective. The particle \textit{ma} is used for both the negative optative and the hypothetical.
\clearpage
\begin{table}
\caption{\label{tab:phon:12}Verbal tone ({\citealt[35]{Sumner1921}})}
\begin{tabular}{ll} 
\lsptoprule
Kɔ́. & ‘He went.'\\
Kɔ̀. & ‘Let him go.' / ‘He should go.'\\
Mà kɔ́. & ‘Let him not go.'\\
Má kɔ̀. & ‘He would have gone.'\\
\tablevspace
À mà kɔ́. & ‘Let me not go.'\\
À má kɔ̀. & ‘I would have gone.'\\
\tablevspace
Yí mà kɔ́. & ‘Let us not go.'\\
Yí má kɔ̀. & ‘We would have gone.'\\
\tablevspace
Hà mà kɔ́. & ‘Let them not go.'\\
Há má kɔ̀. & ‘You (pl.) would have gone.'\\
\lspbottomrule
\end{tabular}
\end{table}

Pronouns can also change their tones depending on context. The 3\textsc{sg} pronoun \textit{wɔ} can be low-toned in subject position and high-toned in object position.\footnote{The high tone on the object may be a consequence of the high tone of the Perfective spreading onto the object. The question was not systematically investigated.}

\ea%31
\label{ex:31}
%(\stepcounter{qwerty}{\theqwerty})  
\ea Wɔ̀ ké mí.\\
‘He saw me.'\\
\ex
Yà ké wɔ́.\\
‘I saw him.'\\
\z
\z

But compare these examples with the tones on \textit{mi} in the examples in \REF{ex:32}. The inflectional marker of past (\textsc{pst}) -ɛ́ has a high tone (see \sectref{sec:4.2}).

\ea%32
%(\stepcounter{qwerty}{\theqwerty})  
\label{ex:32}
\ea Tàmɔ̀ɛ̀ wɔ̀ fɛ̀kiɛ́ mì.\\
\gll tamɔ  ɛ    wɔ    fɛki-ɛ        mi\\
boy  \textsc{def}  \textsc{3sg}  disrespect-\textsc{pst}  \textsc{1sg}\\
\glt ‘The boy has disrespected me.' (E10 Albert Yanker: 9)

\ex Tàmɔ̀ɛ̀   wɔ́ mí fɛ̀kí.\\
\gll tamɔ  ɛ    wɔ    mi    fɛ̀kí\\
boy  \textsc{def}  \textsc{3sg}  \textsc{1sg}  disrespect\\
\glt ‘The boy disrespects me.' (E10 Albert Yanker: 10)
\z
\z

Although nouns generally do not change their tones in different contexts, tone is not stable nor reliably produced, being unpredictably variable for lexical items across speakers and even for individual speakers. Therefore, I have followed the general practice of the Sherbro Literacy Committee\is{Sherbro Literacy Committee} of not marking tone. Full details of the grammatical use of tone are spelled out in Chapter \ref{ch:4} on verbal morphology.

\section{Syllable structure}
\label{sec:2.3}\hypertarget{Toc115517757}{}
Sherbro follows the pattern of other related languages in allowing filled codas (CV(C) syllables are the general pattern), as opposed to the situation in Mende\il{Mende}, the language to which Sherbro speakers are switching. Mande\il{Mande} languages are strictly CV, and Mende is no exception to that generalization (\citealt{Dwyer1989}, \citealt{Vydrin2004}). Long vowels in Sherbro appear in monosyllabic words and in initial syllables of polysyllabic words.

The coda consonant may consist of any of the following (single) segments (prenasalized stops are analyzed as unitary segments). The dental and alveolar stops are privileged in codas, especially when they form part of NC sequences.

\ea%33
%(\stepcounter{qwerty}{\theqwerty}) 
\label{ex:33}
Allowed coda consonants\\

\begin{tabular}[t]{ll}
Liquids & l, r\\
Nasals & m, n, ŋ (never <ny> [ɲ])\\
Prenasalized stops & mp, nth, nt, ŋk (never <nych> [ɲç])\\
Voiceless stops & p, th, t, ch, k\\
\end{tabular}
\z

As mentioned above in the discussion of prenasalized stops, all of which are single segments, there is a skewed distribution of “voiceless” prenasalized stops [mp, nt, ŋk] and voiced prenasalized stops [mb, nd, ŋg]. Namely, the former are found in codas and medially, and occasionally initially, while the latter are found in onsets, usually the result of a (syllabic) nasal prefix being reduced to non-syllabic status.

One exception to the last generalization is when the word following the prefixed nasal does not begin with a voiced stop. In the following example, the nasal prefix of the \textit{ma} class [n-] appears before a number of different consonants: [r t p h], which assimilates to the bilabial in \textit{pakai} ‘papaya'. In addition, the second person subject pronoun [n] assimilates to the velar stop [k] in \textit{kɔ} ‘go'. Thus, virtually any prenasalized sequence is possible initially, though place-of-articulation assimilation seems essential. The voiced prenasalized sequence always involves more than one morpheme.
\clearpage
\ea%34
\label{ex:34}
%(\stepcounter{qwerty}{\theqwerty})
Voiceless prenasalized stops: [ŋk, nt, mp] in initial position\\
\vspace{6pt}
Ŋkɔm lɛnthiɛ nrokos ntiŋ ni mpakai nhiɔl!\\
\gll n    kɔ    mi    lɛnthi  {}-ɛ    n-rokos        n-tiŋ      ni    n-      pakai    n-hiɔl\\
\textsc{2sg}  go    \textsc{1sg}  pluck  {}-\textsc{prt}  \textsc{ncm}\textsubscript{ma}{}-orange    \textsc{ncm}\textsubscript{ma}{}-two and  \textsc{ncm}\textsubscript{ma}{}-  papaya  \textsc{ncm}\textsubscript{ma}{}-four\\
\glt ‘Go pluck me two oranges and four papayas.' (P67 L: 53)
\z

The same generalization holds true for a sequence not shown, [nt̪].

Syllable structure may also vary when /r/ or /n/ is involved (see the discussion in \sectref{sec:2.1.2} for some cases of metathesis\is{metathesis} and epenthesis). A schwa may break up a sequence of disallowed consonants (certainly in compounds but also in stems).

\section{Phonological rules}
\label{sec:2.4}\hypertarget{Toc115517758}{}
Sherbro has both purely phonological rules as well as morphophonological ones. The latter category of rules is treated in the sections on morphology. Here only purely phonological rules are discussed.

\subsection{Nasal assimilation}
\label{sec:2.4.1}
Nasals always agree with the place of articulation of a following obstruent both within words and across morpheme boundaries. Following are examples of the latter phenomenon involving the prefixed second-person singular subject pronoun, which appears only before verbs, as in \REF{ex:35}. Another identical morpheme is the \textit{ma}{}-class prefix (or noun class marker (\textsc{ncm})), as featured in \REF{ex:35b}.

\ea%35
\label{ex:35} Nasal assimilation
%\label{bkm:nasassim}(\stepcounter{qwerty}{\theqwerty})  
\ea \label{ex:35a}
[+nas] $\xrightarrow{}$ [α place] / {\longrule} + C [α place], [m, n, ŋ, m͡ŋ]

\ex\label{ex:35b} \textit{ma}{}-class prefix /n-/\\
\ea Yaaka tallɛ, aka ni ŋaa mpanth ma sobaɛ.\\
\gll ya    a    ka      taa      lɛ    a    ka      ni    ŋaa  n-panth      ma    soba-ɛ\\
\textsc{1sg}  \textsc{1sg}  \textsc{rem.pst}  youth    \textsc{def}  \textsc{1sg}  \textsc{rem.pst}  \textsc{neg}  do    \textsc{ncm}\textsubscript{ma}{}-work  \textsc{ncp}\textsubscript{ma}    sober-\textsc{def}\\
\glt ‘When I was young, I did not do serious work.' (094a Ansu Kagboro:  66)

\ex So lan la ako ha ŋkuath ha ŋɔth.\\
\gll so  lan  la      a    koŋ    ha    n-kuath    ha    ŋɔth\\
so  this  \textsc{pro}\textsubscript{indef}  \textsc{1sg}  \textsc{pfv}    \textsc{opt}  \textsc{ncm}\textsubscript{ma}{}-fear  for    fishing\\
\glt ‘So that is how I became afraid of fishing.' (004a Cyril Manley on Walter Hanson:58)
\z

\ex \label{ex:35c} \textsc{2sg} subject prefix /n-/\\
\ea Nsiɛ tɛm pɛm doki yɛi chaŋ-chaŋdɛ …\\
\gll n    siɛ      tɛm  pɛm  doki  yɛ    yi    chaŋ-chaŋ  yɛ\\
\textsc{2sg}  know    time  war  this  how  \textsc{1pl}  travel      \textsc{prt}\\
\glt ‘You know during the war how we were moving around …' (002a Mabel Lohr, Midwifery: 41)

\ex Mɔm, la ŋka cheni ŋa?\\
\gll mɔm      la    n    ka      che  ni    ŋaa-a\\
\textsc{2sg.emph}  what  \textsc{2sg}  \textsc{rem.pst}  \textsc{prog}  now  do-\textsc{q}\\
\glt ‘You, what have you been doing?' (004a Cyril Manley on Walter Hanson: 45)

\ex Mi mŋgbisiŋɛ?\\
\gll mi      n    gbisiŋɛ\\
Mother  \textsc{2sg}  marry\\
\glt ‘Mummy, are you married?' (007a Agnes J. Simbo: 61)
\z
\z
\z

\subsection{Nasalization}
\label{sec:2.4.2}
An abundance of processes contributes to the ubiquity of nasalization in the language.\footnote{When I played some recordings to a renowned phonetician, he asked, “Don't they ever raise their velums?”} Although there is no contrastive nasalization, the process can be both perseveratory and anticipatory, and can affect consonants as well. Before a nasal consonant, but most dramatically and fulsomely after [h], vowels are nasalized, as has been noticed for related languages, Kisi\il{Kisi}, Mani\il{Mani}, and Bom-Kim\il{Bom-Kim} (\citealt{Childs1995}, \citealt{Childs2011}, \citealt{Childs2020}). When a velar nasal fills a coda, the preceding vowel is (phonetically) nasalized.

\ea%36
%(\stepcounter{qwerty}{\theqwerty})
\label{ex:36}
Anticipatory nasalization\\
\begin{tabular}[t]{lll}
/fuŋfuŋ/ & [f\~uŋf\~uŋ] / [f\~uf\~u] & ‘rice seedlings in a nursery' (P67 F: 38)\\
/wɔŋhul/ & [wɔ̃hũl] & ‘sell'\\
\end{tabular}
\z

But because the velar nasal is prone to disappear in such environments, the only trace of its presence, should it disappear, is the nasalization of the vowel. The vowel can also be compensatorily lengthened with the loss of the velar nasal as in Wong, the name of a sacred island in the Dema\is{Dema} Chiefdom, here with the definite article.

\ea%37
\label{ex:37}
%(\stepcounter{qwerty}{\theqwerty})  
woŋ + \textsc{def}\\
woŋ + ɛ  $\xrightarrow{}$  woŋ + ndɛ  $\xrightarrow{}$  wõːndɛ\\
\z

The colorful term “rhinoglottophilia” is used for the nasalization following glottal [h] and is associated with other glottal sounds (\citealt{Matisoff1975}). The nasalization for [h] is so heavy that the ‘\textsc{1pl} \textsc{pro} we,' /h\~\i/, is sometimes transcribed as [nyĩ]. Note also the two forms for ‘sea.'

\ea%38
%(\stepcounter{qwerty}{\theqwerty})

\label{ex:38}
Rhinoglottophilia\\
\vspace{6pt}
\begin{tabular}[t]{ll}
h\~\i & \textsc{1pl} \textsc{pro} ‘we'\\
hɛ̃l / nyɛ̃l & ‘sea'\\
hãã & ‘make, do'\\
hɔ̃lɛ & ‘whisper'
\end{tabular}
\z

There is also the close association between the velar nasal and [h] represented in such alternations as [ŋa] and [hã] \textsc{2pl} \textsc{pro}, ‘you' discussed above.

As a final nasalization process to mention, there is prenasalization of initial consonants. When a voiced stop begins a word, it can be prenasalized (unpredictably). The name of a chiefdom on Bonthe Island can be pronounced [dema] or [ndema].

For some speakers, nasalization becomes glottalization or creaky voice, affirming the link between nasal and glottal processes as in rhinoglottophilia. Not surprisingly, glottalization again is associated with the “glottal” fricative. The name Kain was sometimes spelled with two syllables and an “h” in the middle <Kahain>.

\ea%39
    \label{ex:39}
     Bia tonkiɛ jali Ka̰ḭn ha kɔnth.\\
    \gll Bia  tonki-ɛ        ja      li-Kain      ha    kɔnth\\
    Bia  summon-\textsc{pst}    matter  \textsc{ncm}\textsubscript{lɔ}{}-Kain    for    seizure\\
    \glt ‘Bia summoned Kain for seizure.' (P67 K: 200)
\z

The same name was used in Bom-Kim country and pronounced the same way (spelled <Kain>). Another word that exhibited glottalization was \textit{kahai} ‘outside' and the name \textit{Mahain}. Thus [kaha̰ḭ] and [maha̰ḭ]. The low back vowel is likely the conditioning factor, even more so when it both precedes and follows [h].\footnote{glottoglottophilia?}

\subsection{Palatalization}
\label{sec:2.4.3}
A number of related processes serve to palatalize alveolar stops. They are summarized here, all described separately in the discussion of the allophones of the respective segments (\sectref{sec:2.1.2}).

\ea%40 
\label{ex:40} Palatalization\\

\begin{tabular}{lllll}
d & $\xrightarrow{}$ & ʤ & / {\longrule} [i, e] & \textit{di/ji}  ‘kill, catch, initiate'\\
s & $\xrightarrow{}$ & ʃ & / {\longrule} [i, e] & \textit{si/shi}  ‘know,' \textit{seko/sheko} ‘fishhook'\\
t, th & $\xrightarrow{}$ & ʧ & / {\longrule} [i, e] & \textit{tii/chii} ‘town,' \textit{the/che} ‘hear, listen' 
\end{tabular}
\z

Related to these phenomena is the labiodental allophone [v] of the labialvelar approximate /w/, occurring in the same environment: w $\xrightarrow{}$ v / {\longrule} [i, e].

I now turn to a phonological rule at the level of the syllable. The syllable structure process described below appears in all languages in Bolom. Generally speaking, the process can be seen as a process of onset strengthening, as described in \citet{Childs1988}. Typically, the onset of a relatively “weak” syllable (featuring a liquid, a glide, or nothing) will borrow phonetic substance from a preceding coda, providing that coda is “stronger” (more prototypically consonantal). How this process plays out in Sherbro is described below. The process affects particles, clitics, suffixes, and other grammatical elements.

It is generally the case that strengthening takes place at morpheme and even word boundaries when the segment on the left is a nasal and the segment on the right is the liquid [l], a glide ([w] or [y]), or an onsetless syllable. Thus:\\


    ${\emptyset}$, [l], [y] and [w] $\xrightarrow{}$ [d] / N + {\longrule}\\


If the preceding coda is empty, no strengthening takes place.

\ea%41
    \label{ex:41}
  Onset strengthening
  \ea\label{ex:41a} Mbolomdɛ\\
\gll n-bolom      ɛ\\
\textsc{ncm}\textsubscript{ma}{}-bolom  \textsc{def}\\
\glt ‘Sherbro language'\\

\ex\label{ex:41b} Nthemdɛ\\
\gll n-them      ɛ\\
\textsc{ncm}\textsubscript{ma}{}-themne  \textsc{def}\\
\glt ‘Themne language'\\

\ex\label{ex:41c} ndoɛ\\
\gll n-loɛ\\
\textsc{ncm}\textsubscript{ma}{}-sleep\\
\glt ‘sleep'
\z
\z

The crucial form in \REF{ex:42} is the locative \textit{lɔ} at the end of the first line of morpheme analysis, which becomes [(n)dɔ] in context after \textit{hun}, as shown in the first line.

\ea%42
\label{ex:42}
Haaŋ mɛŋkɛ ŋɔ apotho aɛ ka hun  dɔ, chal ha pin awok aɛ …\\
\gll haa  mɛŋk    ɛ    ŋɔ      a-potho      a-ɛ        ka      hun  lɔ\\
until  time    \textsc{def}  when    \textsc{ncm}\textsubscript{ha}{}-whites  \textsc{ncm}\textsubscript{ha}{}-\textsc{def}  \textsc{rem.pst}  come  there\\
\gll chal  ha    pin  a-wok      a-ɛ\\
stay  for    buy  \textsc{ncm}\textsubscript{ha}{}-enslaved  \textsc{ncm}\textsubscript{ha}{}-\textsc{def}\\
\glt ‘Until the white man came there and settled to buy enslaved people …' (124aw Yanker, Boy Lost at Sea: 19)
\z
In the following example, there are two instances of Onset Strengthening. The first involves the proximal demonstrative \textit{loki} ‘these' after \textit{tiŋ} ‘two,' and the second is the clause-final binding particle \textit{ɛ} after \textit{thiyeŋ} ‘between.'

\ea%43
\label{ex:43}
    Yɛ   thoŋka ki gbi kɔ haani bɛl siatiŋ doki thiyeŋ dɛ …\\
\gll yɛ    thoŋka  ki    gbi  kɔ    haani    bɛl-si      a-tiŋ      loki  thiyeŋ  ɛ\\
when  arguing  this  all    \textsc{ncp}\textsubscript{kɔ}  happen  rat-\textsc{ncm}\textsubscript{si}  \textsc{ncm}\textsubscript{ha}{}-two  these  between  \textsc{prt}\\
\glt ‘When all this arguing is going on between these two rats …' (123aw Yanker, Rat Wife: 77)
\z

Glide insertion is much less common and even unpredictable. In \REF{ex:44}, [w] is inserted before the clause-final particle -\textit{ɛ} (\textsc{prt}) but not before the definite article \textit{ɛ} (\textsc{def}) in the same sentence after both \textit{de} ‘day' and \textit{lɔkɔ} ‘day.'
\clearpage
\ea%44
%\label{bkm:zerotowy}(44)
\label{ex:44}
\ea Yan deɛ ŋɔ huɛ lɔkɔɛ ŋɔ hu wɛ, aka shilani.\\
\gll ya-n      de    ɛ    wɔ    huɛ  lɔkɔ  ɛ    wɔ    hu  ɛ    a    ka      si    la    ni\\
\textsc{1sg-emph}  day  \textsc{def}  3\textsc{sg}  die  day  \textsc{def}  \textsc{3sg}  die \textsc{prt}  \textsc{1sg}  \textsc{rem.pst}  know  this  \textsc{neg}\\
\glt ‘As for me, the day he died the day he died, I didn't know.' (009--10a Lohr \& Mampa: 317)

\ex  Raiyɛ ŋɔ koŋ tuk.\\
\gll rai      ɛ    hɔ      koŋ  tuk\\
paper    \textsc{def}  \textsc{ncp}\textsubscript{hɔ}    \textsc{pfv}  disappear\\
\glt ‘The document has disappeared.' (002a Mabel Lohr, Midwifery: 41)
\z
\z

In a parallel process, after a left element [l], glides and empty onsets will be strengthened to [l], producing a geminate. Thus:\\

[${\emptyset}$], [y], and [w] $\xrightarrow{}$ [l] / [l] + {\longrule}\\

Again, the process augments the onset of a weak syllable preceded by a stronger one. In \REF{ex:45a}, the \textit{l} at the end of \textit{gbal} ‘line' is geminated before the definite article ɛ. In \REF{ex:45b}, the \textit{y} in the final question particle \textit{ya} is strengthened to \textit{l} and another geminate arises. In \REF{ex:45c}, gemination takes place before the preposition \textit{{}-ai} ‘in.'

\ea%45
    \label{ex:45}
    \ea \label{ex:45a} Inan gballɛ,  ilɔ pɛngipɛngi, ikikkik.\\
    \gll i    nan  gbal  ɛ    i    lɔ    pɛŋgipɛŋgi    i    kikkik\\
    \textsc{1pl}  draw  line  \textsc{def}  \textsc{1pl}  there  jump        \textsc{1pl}  kick\\
\glt ‘We draw the line, we jump there (and) kick.' (005a Jalikatu B. Kumba: 80)

\ex \label{ex:45b}  Kɛ mi ŋa mɔ ilella?\\
\gll kɛ    mi        ŋa    mɔ  i-lel-a\\
but  mother    what  \textsc{2sg}  \textsc{ncm}\textsubscript{hɔ}{}-name-\textsc{q}\\
\glt ‘But, Mummy, what is your name?' (007a Agnes J. Simbo: 8)

\ex \label{ex:45c}  Ŋakɔni fillai ŋa kɔ siŋ.\\
\gll ŋa    kɔn-i    fil    ai    ŋa    kɔ    siŋ\\
\textsc{3pl}  go-then  field  in    \textsc{3pl}  go    play\\
\glt ‘They go to the field and play.' (016a Albert Yanker: 166)
\z
\z

This last set of processes show some unity in that they all serve to strengthen an onset so that it is at least as strong as the coda of a preceding syllable.

\subsection{Vowel harmony}
\label{sec:2.4.4}
A single form in the derivational morphology shows vowel harmony\is{vowel}, the suffix \textit{{}\nobreakdash-il/-ul}, which changes verbs into adjectives. The suffix harmonizes with the [back] specification of the last stem vowel. See \sectref{sec:7.1} for some details and examples.
