\addchap{\lsPrefaceTitle}

\epigraph{Better pass boldly into that other world, in the full glory of some passion, than fade and wither dismally with age.}{James Joyce}

\section*{Foreword}

It is with both honor and regret that we find ourselves writing the foreword to this volume. George Tucker Childs (1948-2021), a prominent African linguist and life-long field worker, passed away on January 26, 2021 due to complications related to legionella. African linguistics has lost an immensely influential contributor in the fields of language documentation, the preservation of endangered West African languages and their cultures, and the documentation and the classification of African ideophones – vivid, often onomatopoeic words that evoke sensory images of the sounds they refer to.

Born in Wayne, Illinois, Childs received his A.B. from Stanford University in English Literature 1970. He continued his studies at Trinity College, University of Dublin, where he earned a diploma in Anglo-Irish literature with  Honours 1975. He subsequently taught secondary school at Woodbury Forest in Virginia, where he concurrently completed his M.Ed. at the University of  Virginia 1980. An emerging interest in linguistics and a burgeoning desire to return to the site of his Peace Corps work in Liberia to document the endangered language Kisi, he began his linguistics career at Georgetown University where he received an M.S. in Sociolinguistics 1982. He moved to University of California, Berkeley to study under the direction of John J. Ohala, where he completed both an M.A. and Ph.D. in Linguistics 1988. His doctoral dissertation was later published, \textit{A Kisi Grammar} (1995), followed by \textit{A Dictionary of the Kisi Language}, in collaboration with Herrmann Jungraithmayr and Norbert  {Cyffer 2000}.

Childs devoted his life research to the preservation of West African languages and culture. He was given the Kisi name “Saa Chakporma” by Fayia McCarthy, his Kisi father (\textit{Kɛ̀kɛ́}): “Saa” for the first-born son and “Chakporma” for his “born-town”, where Fayia was born. During his 40-year career, he produced grammars, dictionaries, readers, and primers, as well as numerous academic articles, on the languages of the Bolom-Kisi group, which includes Kisi (1995, 2000), Mani (2007, 2009), Bom-Kim (2009, 2020), and Sherbro. This current volume, \textit{A Grammar and Dictionary of the Sherbro Language}, was completed in 2020 just before his death. His research has been funded by a number of prestigious institutions: Hans Rausing Endangered Languages Project, Bremer Stiftung für Kultur- und Sozialanthropologie, National Science Foundation (NSF), School of Oriental and African Studies (SOAS) at the University of London (UCL), and Fulbright Research Foundation.

Childs was the consummate fieldworker. In going beyond lexico-grammatical information, his work provides a model for the documentation of endangered African languages by considering the theoretical and methodological issues related to language documentation in its social context, from greater emphasis on naturally occurring conversational data and the adoption of metadata conventions for more nuanced descriptions of socio-cultural settings to considering the impact of language policy and planning. His research always promoted a multifaceted and integrated focus on the community and the people, as much as on the language. He called it \textit{methodological relativism}, “an adaptive and culture-sensitive approach to the field situation” \citep{Childs2016}. He believed that building community cohesion and collaboration is paramount over the fieldworker's research goals and he was not tolerant of fieldworkers who exploited their African collaborators for their own professional ends. He always strove to overcome the way in which Western language ideologies have unjustly influenced language documentation practices in non-Western societies.

As a professor, Childs taught linguistics at a number of prestigious institutions, including San José State University (San José, California), Temple University (Philadelphia), University of Witwatersrand (Johannesburg, South Africa), Universität Freiburg (Germany), University of Toronto (Canada), and ultimately Portland State University (Portland, Oregon). He served tirelessly for many years as the editor for \textit{Studies in African Linguistics,} a public forum for African language scholars to discuss issues in the field of African Linguistics. He was recognized in a 2009 \textit{New York Times} article, “Linguist's Preservation Kit Has New Digital Tools”, regarding his work documenting the Kim language in Tei, Sierra Leone. And in 2017, he was the recipient of the Linguistic Society of America's Kenneth L. Hale Award which recognizes scholars who have done outstanding work on the documentation of a particular language or family of languages that is endangered or no longer spoken. An overview of his fieldwork was documented in a series of video blogs by his brother and Senior Producer Bart Childs of \textit{Voice of America}, called \textit{Lost Voices} (2012).

In addition to his grammars, primers and dictionaries, some of his most influential work includes “African ideophones” (\textit{Sound Symbolism}, 1994), \textit{An Introduction to African Languages} (2003), “Busy intersections: A framework for revitalization” (\textit{Africa's Endangered Languages,} 2015), \textit{Beyond the Ancestral Code: Towards a Model for Sociolinguistic Language Documentation}, with Jeff Good and Alice  {Mitchell 2014}, “Ideophones as a measure of multilingualism” (\textit{Ideophones, Mimetics, and Expressives}, 2019). In “Forty-plus years before the mast” (\textit{Word Hunters: Field Linguists on Fieldwork}, 2018), Childs wrote,

\begin{quote}
 {People ask me why I persist in this research given all the travails. It has become easier than in the past to talk about the satisfaction and the rewards of the job, especially in the past seventeen years when there has been some support for studying languages on the edge. The old people are particularly glad of the attention of the field linguist; they have known a life of marginalization and welcome the interest in their language. They are particularly happy that their descendants will hear their words. People open up, people are friendly, people laugh and they feel valued. That's enough for me (p. 78).}
\end{quote}

In honor of Tucker Childs and to help fulfill his dream of community and educational development, the Sherbro Foundation (\href{http://www.sherbrofoundation.org}{{www.sherbrofoundation.org}}) has set up the \textit{Saa Chakporma Memorial College Scholarship Program}. Each year, the scholarship program starts one student on a four-year path to a bachelor's degree. Upon completion of their degrees, the students are expected to return to the community and work in a development-oriented job for each year of educational support they receive.

Tucker Childs left a prodigious legacy to the fields of African linguistics and endangered language documentation and he will be sorely missed. We are thankful to Language Sciences Press, who is able to bring his final contribution to the field, posthumously.

\hfill Karen Beaman, University of Tübingen

\hfill  Chris Corcoran, independent scholar

\hfill Jedd Schrock, Portland State University

\clearpage
\section*{Foreword}

During the course of his career, George Tucker Childs cast a distinctive presence in the field of African linguistics. His contributions include articles on documentary, descriptive, and theoretical topics, an introductory textbook, and, perhaps most importantly, a series of grammatical descriptions, and associated works, on the Bolom-Kisi group of languages, of which the present book is a much welcome culmination.

Given the importance of Childs' research on Bolom-Kisi languages for his overall scholarship, it seems fitting to focus on this first. As described in the statement made when Childs was awarded the Linguistic Society of America's Kenneth L. Hale Award, “His work provides a model for the documentation of African languages by an American scholar and the value in documenting an entire linguistic subgroup.” Most linguists focusing on underdocumented languages specialize on specific languages, in some cases just one language or, perhaps, two or three that may be related and are chosen for practical reasons, such as the accessibility of speaker communities. It is unusual for a linguist to take on the goal of documenting an entire subgroup, let alone to actually achieve it, as Childs has done. This achievement is especially significant when the results involve not merely basic survey work, but grammars, dictionaries, primers, and archival collections. In this respect, Childs' work on the Bolom-Kisi group is truly remarkable.

Starting with his work on Kisi, including his well-regarded 1995 grammar, Childs later produced a grammar of Mani, as well as articles on Bom and Kim, in the following decades, and, now, with the present book, this description of Sherbro completes his efforts to document Bolom-Kisi. His work began at a time when the level of linguistic interest in endangered and under-documented languages was very different from what it is today, and thus he had to forge his own path in pursuing this direction for his research. This is even more notable given that his linguistic career began in the United States, which, unlike Europe, lacks a strong tradition in the study of African languages from a descriptive and comparative perspective. Moreover, the Bolom-Kisi languages are spoken in countries where few other linguists were working and which do not have a strong local linguistic community, making it all the more impressive that he achieved the results that he did, given the lack of a larger scholarly community to support his efforts.

During the course of his career, as the field of linguistics began to place a higher value on the study of endangered languages, Childs was able to benefit from the increased availability of funding to support his efforts, and this included significant grants from the Endangered Languages Documentation Programme and the US National Science Foundation. This allowed him not only to produce descriptive works but also archival collections and primers which are of value to community members, and his success at receiving grants in the latter part of his career underscores the forward-thinking nature of his work on the Bolom-Kisi group. He was, in effect, a documentary linguist before there was such a thing as documentary linguistics. It is fortunate that the field started to catch up to Childs' view of the kind of work that linguists should be doing with marginalized speaker communities at a time when he could still take advantage of this to complete his project to document the entire Bolom-Kisi subgroup.

A distinguishing feature of Childs' work on Bolom-Kisi is its attention to cultural contexts. In some cases, this was due to necessity. Bom and Kim, for example, were already moribund when he started to document them, and research with small language communities necessarily requires an attention to personal relationships. In other cases, though, this was clearly due to an intense interest in the cultures of the communities whose languages he studied. This led him to focus on areas that have often been neglected by other scholars, despite their clear linguistic significance. For example, starting with his work on Kisi, he conducted pioneering work on the use of ideophones. These words play a crucial role in communication in many African languages but had been neglected in descriptions since they did not fit well with Western notions of the structure of grammar. In the present book, his interest in cultural concerns can be seen in its inclusion of Sherbro proverbs and hymns. Proverbs are a central element of communication in many African communities, though they are not frequently included in grammars written by Western scholars because their importance has not been sufficiently recognized. It is hardly surprising that Childs chose to take a different path than most other linguists in this regard.

If Childs' contributions were limited to his descriptive and documentary work on the Bolom-Kisi group, that would already constitute a very impressive set of scholarly accomplishments. However, his achievements go well beyond.  His 2003 textbook, \textit{An introduction to African languages}, filled an important gap at the time and remains a highly valuable introductory reference for research on African languages. Its coverage of the history of scholarship on African languages, as well as the inclusion of sociolinguistic topics that do not often receive attention in more typical introductory texts, is especially noteworthy, though it is entirely in line with Childs' career-long focus on studying not only languages, but also the social contexts of their speakers.

His additional articles and chapters build on his extensive experience working on specific languages to make more general arguments. These cover grammatical topics such as the development of noun class systems in the Niger-Congo language family or the syntactic structure of clauses, as well as topics of sociohistorical interest, such as language contact in West Africa. All told, these works, along with his descriptive output, have established him as one of the leading scholars of the Atlantic group of Niger-Congo. More recently, much of his work has focused on the practice of language documentation and the study of endangered languages. As a scholar doing documentary work before such research was recognized as a distinct and important endeavor in its own right, this is entirely fitting. Long before the rest of the field, Childs recognized that language loss meant much more than the loss of mere lexico-grammatical codes but also of entire systems of communication. His decades of experience put him in a special position to speak to these issues and transmit his knowledge to both well-established researchers and to junior scholars, in particular. Within African linguistics, there was no other voice like his on these issues. I always especially appreciated the way that his humanity—and the humanity of the community members he worked with—came through in his work on these topics.

Like so many of my colleagues, I regret that we will not have the chance to see many more years of publications by Childs, in particular further studies of Sherbro. At the same time, it gives me great comfort to know that Language Science Press is able to bring this last major piece of his project to document the Bolom-Kisi group of languages to completion.

\hfill Jeff Good

\hfill University at Buffalo
 
 
