\chapter{Introduction}\label{ch:1}\hypertarget{Toc115517742}{}

\begin{exe}
\sn{Ŋhɔbɛ i le ma hɔ haŋ wɔyɛ pi ima lɔ be nwɔk pika gbi, a cheŋ ke gbi.\\
\gll ŋhɔbɛ    i    le    ma    hɔ-ɛ      haŋ  wɔi  ɛ    pi i    ma    lɔ      be      n-wɔk        pika    gbi a    che-ni    ke      gbi\\
even.if  \textsc{1pl}  stay  \textsc{ncp}\textsubscript{ma}    speak-\textsc{prt}  until  day  \textsc{def}  be.dark \textsc{1pl} \textsc{ncp}\textsubscript{ma}    there    \textsc{neg}    \textsc{ncm}\textsubscript{ma}{}-language  other    at.all \textsc{1sg}  \textsc{aux-neg}  tired    at.all\\

\glt ‘Even if we continue speaking it until nightfall using no other language, I would not get tired.'
(093a Alusine Bundu: 84)}\footnotemark
\end{exe}
\footnotetext{The material in parentheses indicates where in the archives the example comes from. Here the recording reference number is “093” with the “a” indicating it is an audio recording (“a” audio, “v” video, “aw” audio of written stories read aloud). The primary speaker is given next, here Alusine Bundu, and then the sentence number in the database (originally FLEx). More than 150 audio and video files as well as a FLEx database for Sherbro are available at the Endangered Languages Archive (ELAR) at SOAS University of London (\url{https://www.elararchive.org/dk0373}), PDX Scholar at Portland State University (\url{https://pdxscholar.library.pdx.edu/sherbro/}), and the library at Fourah Bay College (University of Sierra Leone). Appendix \ref{app:g} provides a cross-reference.}

“Sherbro” is the term used to designate a language and the people speaking the language. When speaking the language, they refer to their language as “Bolom” [mbolomdɛ].\footnote{Because the Sherbro people refer to themselves and their language as Sherbro when speaking English or Krio\il{Krio}, I follow their practice here.} The etymology of Sherbro is uncertain, but a thorough set of possibilities is detailed in \citet{Corcoraninprep}. Sherbro is, undoubtedly, an exonym because <sh> [ʃ] is not a phoneme in the language, post-vocalic [r] is unstable, and consonant clusters with [r] occur only in alternation with post-vocalic [r] as an example of metathesis\is{metathesis}, a not uncommon process in the language where [r] is involved.\footnote{But [sh] is an allophone of /s/ before high front vowels in some dialects.}\footnote{See the discussion of /r/ in \sectref{sec:2.1.2}.} There is also evidence of the term people use themselves when speaking of themselves and their language, i.e., it is not “Sherbro.” Speaker Cyril Manley raves about the language, using the autonym [mbolomdɛ].

\clearpage

\ea \label{ex:1} Yaŋ, a chɔŋ nwɔk mamdɛ len, Mbolomdɛ.\\
    \gll ya-ŋ      a    chɔŋ  n-hɔk          mi ma    ɛ    len    n-bolom    dɛ\\
    \textsc{1sg-emph}  \textsc{1sg}  offer  \textsc{ncm}\textsubscript{ma}{}-language  \textsc{1sg.poss} \textsc{ncp}\textsubscript{ma}  \textsc{def}  thing    \textsc{ncm}\textsubscript{ma}{}-Bolom  \textsc{def}\\
    \glt ‘Me, I love it (the church service) in my language, Bolom.' (004a Cyril Manley on Walter Hanson: 86)\footnote{\textit{chɔŋ...len} is a discontinuous morpheme translated as ‘love'. \textit{Chɔŋ} is a verb meaning ‘pour, dish up, offer' and \textit{len} is an indefinite pronoun that can be translated as ‘thing' or ‘something.'}
\z

What is disturbing about the present situation is the precarious livelihood of its speakers. Aside from the ravages of colonialism and the slave trade, the Sherbro people have recently recovered from a devastating civil war. Their lands are being desolated by multinational mining concerns, leaving a wasteland in their wake, and their fishing grounds have been depleted by multinationals.

One voluble and informed voice is that of Baba Mandela, Plantain Island\is{Plantain Island}'s most successful fisherman. Plantain Island is a small (and shrinking) island off the coast near Shenge\is{Shenge}, the site of the research project Documenting the Sherbro Language and Culture (SLC). Mandela has stated that as goes Plantain Island, so go the fish and so goes the language and culture (142v Baba Mandela, Fishing). He sees the advance of the sea (the island is sinking) and the overfishing and the threat to local livelihoods as epitomizing the threat to Sherbro language and culture.\footnote{See \REF{ex:104} in \sectref{sec:3.10.4} and the surrounding discussion.} Here he talks about the multinationals in their large fishing boats encroaching on the traditional Sherbro fishing areas.

\ea 
\label{ex:2}{Baba Mandela on foreign encroachment}\\
\vspace{6pt}
Bikɔ mɛnkɛ ŋɔ tha ka cheni wun kaɛ, iche sɔthɔ shin yenchɛk. Kɛ ŋke tha wun ka vɛ, aa, la ko che ishin fli-o.\\
\gll bikɔ    mɛnk    ɛ    hɔ    tha    ka      che-ni    hun    ka    ɛ i    che-ni    sɔthɔ    sin      yenchɛk kɛ    n-ke    tha    wun  ka    vɛ aa  la      koŋ    che    i-sin          fli      o\\
because  time    \textsc{def}  \textsc{ncp}\textsubscript{hɔ}  \textsc{ncp}\textsubscript{tha}    \textsc{rem.pst}  \textsc{aux-neg}  come    here  \textsc{prt} \textsc{1pl}  \textsc{aux-neg}  get    shortage  fish(pl) but  2\textsc{sg}{}-see  \textsc{ncp}\textsubscript{tha}    come  here  thus ah  \textsc{pro}\textsubscript{indef}  \textsc{pfv}    be      \textsc{ncm}\textsubscript{hɔ}{}-shortage  really    \textsc{emph}\\
\glt ‘Because when they (the trawlers, powered boats) were not here, we did not have a fish shortage. But you see them coming here, ah, it has now become a real shortage-o!' (142v Baba Mandela, Fishing: 70--72)\\
\z

In addition to these factors and aside from the encroachments of other groups in their historical territory, be it European colonists or indigenous expanding groups, the Sherbro people also face threats from environmental factors (see \sectref{sec:1.7}). There is the general subsidence that has occurred in historical times, observable quite pointedly on Plantain Island\is{Plantain Island}. (Baba Mandela talks about the waves now “licking” its shores \REF{ex:104}.) In addition, there is the seemingly inevitable ocean rise due to climate change.

\section{Previous work on Sherbro}
\label{sec:1.1}
\hypertarget{Toc115517743}{}
A detailed review of the literature on Sherbro can be found in \citet{Corcoraninprep}.\footnote{The author has made her manuscript available to me and I build on her work.} I treat in detail below only one resource of which she was not aware, early work by the abolitionist group, the American Missionary Association, and a few other resources she does not mention.

Early written work on Sherbro was done by missionary Rev. John White\ia{White, John!} with the help of a native speaker, Rev. Barnabas Root\ia{Root, Barnabas} (born Fahma Yahny), who was brought back to the United States by Rev. White after his brief stay (1858--59) in Sherbro country (\citealt{Holmes1877}). White was a member of the American Missionary Association, an abolitionist group.\footnote{Established in 1846 by a network of abolitionists, some of whom took part in the Amistad affair (1839-41). The Association devoted itself to the education of African Americans, for example, in establishing what are now known as HBCUs (“historically black colleges and universities”). Further information can be found at the American Missionary Association archives, 1828-1969, Amistad Research Center at Tulane University (\url{http://amistadresearchcenter.tulane.edu/archon/?p=collections/findingaid & id=19 & rootcontentid=12504}).}  Some thirty pages in length, the first work was a primer “Rai Tammo Lae” (‘My child's book') designed, as stated in a preliminary note, “at giving the Sherbro people religious truth, in their own language” (\citealt{White1860}).\footnote{My best guess as to the translation of the primer's title, “Rai tammo lai”, is ‘Your child's book,' which in the writing system we are using would be <Rai tamɔ mɔ lɛ>, phonetically [rai tammɔ lɛ] with a contraction. I am sure of the first and last words, but not so sure of the middle one. The word for ‘child, boy, son' is \textit{tamɔ}, which is likely cut short (there are examples elsewhere in the data) because of the following word \textit{mɔ} ‘your,' which sometimes cliticizes with the word it possesses, as conditioned by syllable structure. Thus, it would be natural for some elision at the juncture. But mostly the translation seems appropriate because it makes semantic sense. The primer is oriented toward children, as is evidenced by the simple language inside.} The book illustrations, however, feature no Africans and no African scenes, as exemplified in \figref{fig:intro:1} from the frontispiece and title page. The illustrations could represent Victorian parlors as drawn by Sir John Tenniel!

The book begins with a review of the alphabet, illustrating each letter with a scene from the Bible (including <q> and <z>, which are not phonemes in the language). (The book is clearly aimed at proselytizing.) Then follows a few illustrated words, a few words with definitions in English, and the numbers 1-20. The rest of the book (pp. 21-30) is taken up with Christian items such as the Ten Commandments and hymns. Because the writing system is so idiosyncratic and only a few lines of translation are provided, it is not of much use to scholars of the language.

\begin{figure}
\caption{Illustrations from “Rai Tammo Lae” (\citealt{White1860})}
\label{fig:intro:1}
\includegraphics[width=\textwidth]{figures/childsmod-img002.jpg}
\end{figure}

White's second production, “Sherbro and English Book” (\citealt{White1862}), is of more use to linguists, despite the idiosyncratic writing system. White was probably “just a translator since the book was based on the Tract Primer, a popular textbook issued by the American Tract Society" (\url{https://historyleaks.wordpress.com/tag/mendi-mission/} (Accessed 10  {Aug 2018}). The first part (pp. 3-29) hews to a traditional approach comparing Sherbro to English at many points and skipping over some of the finer points of Sherbro grammar, e.g., the noun class system (likely due to being a translation). It begins by introducing the sounds of Sherbro and quickly turns to the parts of speech, illustrating the morphosyntax of both nouns and verbs. Part II constitutes the bulk of the book — sixty pages of “Vocabulary” (pp. 30-90) comprising some 1800 entries (“a copious vocabulary” as announced on the title page).

Following is the preface to the first dictionary of the Sherbro language compiled in the early 1960s and published as \citet{Pichl1967}. Walter J. Pichl(1912-1982) was an Austrian linguist educated at the University of Vienna who did original, ground-breaking work on many languages in West Africa. A brief biography can be found at \url{http://www.afrikanistik.at/pdf/personen/pichl_walter.pdf} (in German).

\begin{quote}
    In compiling this Sherbro dictionary the author has fulfilled the long-felt need of a tribe whose language was considered dead. This will give new life to the tribe in that it will now have a written form of expression.
    
    In this dictionary Dr. Walter J. Pichl, a linguist of Fourah Bay College of international note, has spared no pains in traveling the length and the breadth of the Sherbro land to collect the facts found in this valuable document, in which he has had the assistance of Mr. Charles Walterson-Domingo.
    
    I hope and pray that this dictionary will become the talisman to unity and understanding among all Sherbro people throughout the world.

Shenge\is{Shenge}, June 1964\\
Mme Honoria Bailor Caulker\ia{Caulker, Honoria Bailor}\\
Paramount Chief
\end{quote}

As indicated by Paramount Chief Caulker\ia{Caulker, Honoria Bailor}, he was affiliated with Fourah Bay College, University of Sierra Leone, at the time he did research on Sherbro. I am deeply indebted to his work on Sherbro as well as to his work on Kim (his “Krim”), another language on which he worked and on which I also did research (e.g., \citealt{Childs2020}).

\section{Sherbro Language and Culture Project (SLC)}
\label{sec:1.2}\hypertarget{Toc115517744}{}\is{SLC}
The SLC\is{SLC}, known officially as “Documenting the Sherbro Language and Culture,” was a three-year project (2016-2019) that was extended for a fourth year into 2020. The SLC was designed to document the Sherbro language and culture. Supervised and led by the author, the project relied heavily on Abdulai Bendu, a native speaker of Sherbro from Moyeamoh, Bumpeh Chiefdom\is{Bumpeh Chiefdom}, who began study in linguistics at Fourah Bay College in 2018. The project has documented a great number of activities with both audio and video recordings and photographs. All of this documentation is housed permanently at the Endangered Languages Archive (ELAR) at the School of Oriental and African Studies, University of London.

Outputs of the project include scholarly presentations and articles, as well as the current volume, a Sherbro grammar and a dictionary. In addition, we have developed (and distributed) a primer for developing Sherbro literacy in school children, including several videos. The materials archived include over 130 audio and video recordings, approximately five hours of which have been transcribed and fully analyzed. In addition, we have archived a large number of photographs and written materials.

Most of the linguistic data has been entered into Flex (officially FLEx, a data analysis program developed by SIL\is{SIL} that organizes one's data and makes it suitable for use in writing a grammar and producing a dictionary, among other things).\footnote{I have been through several versions of the software with excellent support from SIL\is{SIL} International. The latest version used is 9.0.9.}

Speakers and other contributors came from several chiefdoms and were both men and women skewed toward the elderly (see \figref{fig:intro:5}). In all cases we sought and obtained permission from speakers to have their words shared with the wider world, following human subject protocols established at Portland State University.\footnote{Portland State University in Portland, Oregon (USA). Some examples of the protocol form several examples below.} See the examples referenced by footnotes 39 and 57 for how permission was sought, granted and recorded. Not all subjects were literate, and we eschewed written contracts due to local fears associated with signatures on written documents.

Below is a bibliographic listing of the archive contents and where all the materials can be found.

\begin{quote}
    2019. Sherbro Archives. Documentation: Audio recordings; photographs; video recordings; digital fieldnotes; databases of people and places (data and metadata); lexicon; texts, transcriptions, and grammatical analyses. Endangered Languages Archive, School of Oriental and African Studies, University of London; Fourah Bay College, Freetown, Sierra Leone; and PDX Scholar, Portland State University, Portland, Oregon.
\end{quote}


\section{Sherbro History}
\label{sec:1.3}\hypertarget{Toc115517745}{}
There have been two major historical forces at work on the Sherbro community. One is pressure from the interior, Temne\il{Temne} and Mende\il{Mende} expansionism (and the national government to a certain extent), and the second is pressure from the exterior, European colonialism and global exploitation. Both pressures are ongoing.

English colonists from the 17th century onward insinuated themselves into Sherbro culture by marrying into already-established royal families and creating their own dynasties (see Appendix \ref{app:b} for a complete list of the Kagboro Chiefdom paramount chiefs). The Sherbro have a long history of being very open to outsiders and a tradition of welcoming “strangers” (outsiders) into their midst (\citealt{Brooks1993}, \citealt{Lowther2011},    \citealt{ShackSkinner1979}). English traders quickly made themselves at home in Sherbro society.

\begin{quote}
    That Englishmen named Corker [a.k.a. Caulker, one of the royal families, see below], Rogers, and Tucker [no relation to the present author] could assimilate themselves into coastal society is hardly surprising. Their control of European trade goods was an obvious advantage. But to achieve and sustain social and political position depended largely on what historian Lynda Rose Day calls “an intriguing pattern of integration characteristic of Sherbro society.” Among the Sherbro, Day writes, “property rights, group identity, and social rights are transmittable through women… who can legitimize the position of the in-marrying foreigner (\citealt[82--84]{Day1983}, as referenced in \citealt[29--30]{Lowther2011}).
\end{quote}

The Sherbro people inhabit a part of the world which has experienced a long period of contact with Europeans and others, where there are Sherbro people with names like Cyril Manley, Virginia Lohr, and Zylette Domingo.

Another factor mentioned by \citet{Day1983} was the “non-exclusivity” of the powerful Poro\is{Poro} initiation society, which all boys would join and which still holds sway in 2020. This trait allowed even outsiders such as the English to join. Adam Jones, who has written extensively on the Gallinas, the southern coastal region of Sierra Leone, describes its power as “a semi-religious constitutional watchdog obliging kings and commoners alike to conform to certain established laws” (\citealt[19]{Jones1983}). Revealing Poro secrets\is{secret societies} in some cases was punishable by death.

The forces from the interior, from the Temne\il{Temne} to the north and the Mende\il{Mende} to the south, have been just as disruptive, likely fragmenting a once-continuous Sherbro area and definitely diluting Sherbro culture. The Temne have also been primarily responsible for the separation of Northern Bolom (Mani) and\il{Mani} Southern Bolom (Sherbro). It is likely the Sherbro people once, previously occupied a much greater expanse of land than they do today, as is suggested by Figures \ref{map:1.3} and \ref{map:1.4}. The Temne\il{Temne} have been making inroads to the north, and the Mende\il{Mende} have taken over many of the towns to the south and east.

Succession in the Kagboro Chiefdom has been relatively smooth since the establishment of the chiefdom in 1896. English traders (the Royal African Company) established themselves in Shenge\is{Shenge} during the 17\textsuperscript{th} century.  {In 1684} Thomas Corker (later spelled Caulker\ia{Caulker}), an employee of the company, married a woman from the Yakumba royal family. With that marriage and the education of their children in England, a dynasty was created. The Sherbro elite has a long tradition of being highly educated. Before the establishment of the Kagboro Chiefdom in 1896, other Caulkers ruled separate parts of the disunited chiefdom (see Appendix \ref{app:b}: Paramount chiefs of the Kagboro Chiefdom). The present (2021) paramount chief, Madam Doris Lenga-Caulker Gbabiyor\is{Lenga-Caulker Gbabiyor, Doris} II, is a descendant of that royal family. Charles Caulker\is{Caulker, Charles}, who is a distantly related cousin from Bumpeh Chiefdom\is{Bumpeh Chiefdom}, another Sherbro chiefdom just to the north of Kagboro, also serves as a Member of Parliament. Throughout the Sherbro area, the Caulkers serve in positions of leadership and authority (see \citealt{Caulker-Burnett2010} for a history).

\section{Classification}
\label{sec:1.4}\hypertarget{Toc115517746}{}
At the family level, the classification of Sherbro has never been in question. Mel\il{Mel} is a closely related group of languages that has likely been fragmented only in historical times. Below appear the Glottolog and Ethnologue codes for the Mel languages and the names by which they have been generally known. (Not included are languages whose status is uncertain, Gola and Limba.)

\ea\label{ex:3}
 Glottolog and Ethnologue codes\\
 
Bom: Glottolog bomk1234; Ethnologue ISO \href{https://www.ethnologue.com/language/bmf}{639-3 bmf}.\\

Kim (Krim): Glottolog krim1238, Ethnologue ISO 639-3 krm.\\

Sherbro: Glottolog sher1258, Ethnologue ISO 639-3 bun.\\

Mani\il{Mani} (Mmani, Bullom So, Northern Bullom): Glottolog \href{https://glottolog.org/resource/languoid/id/bull1247}{bull1247}; Ethnologue ISO 639-3 buy\\

Kisi\il{Kisi} (Kissi, Northern Kissi and Southern Kissi): Glottolog kisi1243, sout2778; Ethnologue ISO kiz, kss.\\
\z

The classification of what was formerly known as (West) Atlantic is still a work in progress. A full discussion of the present-day configuration of Atlantic can be found in \citet{Childs2024a}. What is certain is that the Mel\il{Mel} languages form a coherent entity unrelated to the Atlantic languages further north, an independent family of Niger-Congo.\footnote{In addition to my own articles on Mel\il{Mel}, \citet{Lüpke2020}  contains many up-to-date articles on the language group.} As shown in \figref{figex:intro:2}, Mel consists of two divisions, Temne\il{Temne}{}-Baga and Bolom-Kisi, and two languages whose status is uncertain. This grammar makes reference to languages belonging to the Bolom-Kisi sub-division. Kisi\il{Kisi} is something of an outlier to the group, likely due to its historical separation and isolation, but the other languages are fairly close (\citealt{Childs2024b}). Bom and Kim (Krim) were once considered separate languages but are now considered dialects of the same language, now known as Bom-Kim\il{Bom-Kim}, and likely forming a dialect continuum with Sherbro (\citealt{Childs2020}, \citealt{Childs2024c}).

\begin{figure}
\caption{Current classification of Mel (\citealt{Childs2024a})}
\label{figex:intro:2}
\begin{quote}
    Mel\il{Mel}
\begin{itemize} \leftskip=0.25in
\item[a)] Temne\il{Temne}; Baga languages: Landuma, Baga Koba, Baga Sitemu,\\
Baga Maduri
\item[b)] Bolom: Sherbro, Bom-Kim\il{Bom-Kim}, Mani\il{Mani}; Kisi\il{Kisi}
\end{itemize}
\leftskip=0.6in 
Isolates or status uncertain: Gola, Limba
\end{quote} 
\end{figure}

Previously, Mel\il{Mel} languages had been classified on the basis of geography and typology. They were not Mande\il{Mande} languages and were located in some physical proximity. A review of Sherbro's typology in the following section illustrates the features that motivated the mis-classsification.

\section{Typological overview}
\label{sec:1.5}\hypertarget{Toc115517747}{}
Sherbro has two morphological systems common to the non-Mande\il{Mande} languages of the area: noun classes and verb extensions. These systems are much more robust in what is now known as “Atlantic,” formerly known as “North Atlantic” (\citealt{Segerer2016}). It was these two features that early lumpers considered when they united Mel with Atlantic. There was none of the detailed lexical work usually involved in language classification, a deficiency that has been roundly criticized (e.g., \citealt{Dixon1997}).

Sherbro's phonology contains no uncommon features as compared with other languages in the region. Its segmental inventory includes prenasalized stops, and length is distinctive for vowels. Sherbro was once a fully tonal language, and tone distinctions remain in the verbal morphology, but lexical tone is uncertain. Syllable structure involves codas filled with “voiceless” prenasalized stops as well as several other consonants. There are two liquids, but both often disappear, especially the central approximant /r/.

Basic word order is SVO, but there is a great deal of fronting due to topicalization and focus. One typologically unusual — and perhaps unique —feature is the inviolable unit of tense and object pronouns when tense is marked on an element before the lexical verb. In such cases the object pronouns form a syntactic entity with tense and no longer appear in their usual position after the lexical verb (see \sectref{sec:8.2.3}).

\section{Location} 
\label{sec:1.6}\hypertarget{Toc115517748}{}
Sherbro is currently being abandoned in favor of the more widely spoken Sierra Leone languages, Mende\il{Mende}, Temne\il{Temne}, and Krio\il{Krio} (see \sectref{sec:1.7} for full details). One of the earliest maps showing the Sherbro-speaking area was created by Walter J. Pichl\is{Pichl, Walter}, a researcher from the University of Vienna, who produced a great number of original works describing the less widely documented languages of West Africa. Pichl's map provides some historical perspective on where the language was spoken, but it is no longer an accurate representation of where the language is spoken today; our own assessment shows considerable geographic retreat, as is discussed in the following section (\figref{fig:intro:3}).

\begin{figure}
\caption{Sherbro-speaking area (\citealt{Pichl1967})}
\label{fig:intro:3}
\label{map:1.3}
\includegraphics[width=\textwidth]{figures/childsmod-img003.jpg}
\end{figure}

A useful map that indicates the Bolom languages' rough distribution is a later map in \citet{Hanson1979b} (\figref{fig:intro:4}). Hanson was a missionary based in Shenge\is{Shenge} whose work on Sherbro was ended prematurely with a heart attack in 1980. He was reported as being very comfortable speaking Sherbro (he gave sermons in Sherbro) and produced some pioneering work on the language. After his death it seems as if the mission was abandoned by his sponsors. The incursions by Mende\il{Mende}, Soso\il{Soso}, and Temne\il{Temne} since Hanson's time resulted in much reduced Sherbro-speaking areas on his map. For example, there are only a few hundred speakers of Mani\il{Mani} (his Mmani) today (\citealt{Childs2011}).

\begin{figure}
\caption{Bolom languages (\citealt{Hanson1979b})}
\label{fig:intro:4}
\label{map:1.4}
\includegraphics[width=\textwidth]{figures/childsmod-img004.jpg}
\end{figure}

\citet{IversonCameron1986} contains a virtually identical map (Figure 1.7). A more up-to-date map developed from the SLC\is{SLC} survey fieldwork (2015-2016) appears as \figref{fig:intro:5}. However the Sherbro-speaking areas are mapped, the message is clear: Sherbro speakers are switching to other languages.

\begin{figure}
\caption{Sherbro-speaking chiefdoms ({SLC 2018})}
\label{fig:intro:5}
\includegraphics[width=\textwidth]{figures/childsmod-img005.jpg}
\end{figure}

\section{Language status and vitality} 
\label{sec:1.7}\hypertarget{Toc115517749}{}
The vitality\is{vitality} of the Sherbro language was both more and less than we expected based on our preliminary survey and review of the literature in 2012. On the basis of work begun in  {September 2015}, Sherbro was more widely spoken in terms of geographic area than originally envisioned, but it was also less intensively or frequently spoken within that area, often not spoken at all after children began school. Sherbro is more vital than its closest relatives in Mel\il{Mel}, namely, Bom-Kim\il{Bom-Kim} and Mani\il{Mani}, but is not so vital as Kisi\il{Kisi}, a language from which it is thought to have been separated by the Mane invasions of the sixteenth century (\citealt{Childs1995}, \citealt{Rodney1967}, \citealt{Rodney1970}). “Vitality” is used here in the sense of the United Nations surveys, e.g., (\citealt{UNESCOAdHocExpertGrouponEndangeredLanguages2003}), as a measure of a complex set of factors including demography, language attitudes and ideologies, and institutional support, which combine to index the likelihood that an endangered language will survive.

From a historical perspective, the Sherbro area has decidedly shrunk. The erosion of the Sherbro-speaking area has taken place along the coast adjacent to fishing grounds and to a lesser extent in the interior, although there have been inroads there as well. In the past, one could speak of a Sherbro kingdom extending from Freetown down to well below Sherbro Island\is{Sherbro Island} (e.g., \citealt{Abraham2003},  \citealt{Alie1990}). Today, there is no single coterminous Sherbro region, and what Sherbro region might have once existed is now permeated with holes (towns) and is eroding at the edges, as represented in Figure 1.5, which depicts the status of the language in traditionally Sherbro chiefdoms. Some of the “non-Sherbro” and “formerly Sherbro” areas in the south are where Bom-Kim\il{Bom-Kim} is spoken, and there may be some confusion over the difference between Sherbro and Bom-Kim. The Sherbro area can be extended if Bom-Kim\il{Bom-Kim} is included, though the speakers number only in the hundreds at best (\citealt{Childs2020}).

The most Sherbro-speaking chiefdoms are Kagboro, Bumpeh, and the two chiefdoms on Sherbro Island\is{Sherbro Island}, Dema\is{Dema} and Sittia\is{Sittia}. The map, however, may misrepresent the number of Sherbro speakers in the Sittia Chiefdom on Sherbro Island, where Mende\il{Mende} is rapidly gaining ground, particularly in the southeast, the part of the island nearest the mainland. Not shown on the map are the several trips we made to Plantain Island\is{Plantain Island} and the Dema Chiefdom towns of Chepo and Tissana (part of the Turtle Islands complex) on the west end of Sherbro Island. We also made a trip to Bonthe Town on the east side of the island. Finally, we visited a number of towns in the immediate Shenge\is{Shenge} area, where the SLC\is{SLC} project was based.

Quantifying the vitality\is{vitality} of Sherbro is difficult, not just because no rigorous survey has been performed, but also because speakerhood\is{speakerhood} is a problematic concept in this part of the world, e.g., as explored in \citet{Lüpke2013}. One cannot rely on government statistics, particularly after the disruptions caused by the civil war (1991--2002). Government surveys often record ethnicity rather than language, and respondents graciously provide the answers that their questioners want to hear. Quite simply, no reliable statistics exist. The evolving and dynamic speakerhood of an individual means that a good number of children grow up in Sherbro-speaking households but tend to speak Sherbro less as they age. It may be that their natal town is in a state of transition to one of two or more widely spoken languages, Temne\il{Temne} in the north and Mende\il{Mende} in the south (Krio\il{Krio} everywhere), or it may be that an individual has moved to take advantage of a better school outside the Sherbro area. The Sherbro put a high value on education, and the schools in traditional Sherbro areas may not be highly valued. It may also be that the child orients more toward an urban identity\is{identity}, tilting towards Krio as a language choice.

{The 1978} edition of \textit{Ethnologue} reports 40,000 speakers; in the 2009 edition the number soars to 135,000, an improbable increase; and the online version gives the number as 178,000 in 2016.\footnote{\url{https://www.ethnologue.com/language/bun}, accessed 10  {July 2017}.} At times \textit{Ethnologue} confounds ethnicity and language proficiency or perhaps relies on government figures that do so; it is possible that many people claim a Sherbro heritage with many fewer actually speaking the language. The Sierra Leone census of 1963 reports nearly 75,000 speakers, and the 2004 census just over 65,000 (only ethnicity was counted in the intervening censuses). But because the population doubled during the same timeframe, this represents a decline of more than 60\%.

There are two districts where Sherbro speakers are dominant: Moyamba District (318,064) and Bonthe District (200,730).\footnote{There are also several ethnically-Sherbro fishing villages sprinkled along the coast in the Western Area Rural District.} Bonthe is the least populous district in the country according to provisional statistics of the 2015 census.\footnote{\url{https://www.statistics.sl/wp-content/uploads/2016/06/2015-Census-Provisional-Result.pdf} (accessed 13  {July 2017}).} Within those two districts, there are seventeen chiefdoms that are considered Sherbro \citealt{ReedRobinson2013}). Not all of them are either linguistically or even ethnically Sherbro today because of demographic changes or more organic changes in identity\is{identity}. An initial survey indicated that none of the chiefdoms were exclusively Sherbro. All of them had an admixture of Sherbro and other groups.

For example, a young man from the village of Moyeamoh (183 houses, over 1,000 people according to the 2015 census) in Bumpeh Chiefdom\is{Bumpeh Chiefdom} remembers only a few Temne\il{Temne} houses in the town when he lived there as a child (late 1980s). When he returned there as an adult, he found many more, the town having become half Temne. Moreover, everyone seemed to speak both Temne and Sherbro, except for ethnic Temnes. Even in remote Dema\is{Dema} Chiefdom, the most conservative and isolated Sherbro chiefdom, situated on the west end of Sherbro Island\is{Sherbro Island}, everyone is Sherbro-Mende\il{Mende} bilingual, with many Temne fishermen taking up residence there as well in recent years. Finally, when the research team called on the paramount chief of the Sittia\is{Sittia} Chiefdom on Bonthe Island at his headquarters, we could find only one Sherbro speaker in town. Everyone else spoke only Mende and/or Krio\il{Krio}. As is generally true of Temne speakers relocating into Sherbro areas, Mende speakers do not learn Sherbro when they make a similar move – it is the Sherbro who learn Mende, as was revealed in a study of multilingualism in Shenge\is{Shenge} (\citealt{Childs2019}). The same is true for Temne interlopers.

The multilingual speech economy of the traditional Sherbro area generally features Sherbro as concentrated in the domains of home and village. Krio\il{Krio}, on the other hand, is a widely spoken \textit{lingua franca}. It was formerly the variety of repatriated Africans (Krios) but expanded to become the language of the metropolis Freetown, where most of the Krios lived (also in Bonthe Town on Sherbro Island\is{Sherbro Island}). It is additionally spoken in major up-country cities such as Makeni and Kenema, and functions as a pidgin among rural inhabitants. Moreover, Krio has great appeal to the young as the language of the city, of contemporary life, and of popular culture; it receives considerable governmental support and is taught in the schools.

Mende\il{Mende} and Temne\il{Temne}, two indigenous languages of greater longevity, are both widely spoken languages of great utility, also with governmental support, although more limited geographically. Roughly speaking, Temne is spoken in the north of Sierra Leone and Mende in the south, both qualifying as languages that are displacing others, characterized as indigenous \textit{glottophagic} varieties in the colorful term of \citet{Calvet1974}. Multilingualism is the rule rather than the exception within the traditional Sherbro-speaking area, likely as it has been for centuries, especially in the coastal regions with the advent of European colonialism.

Sherbro Island\is{Sherbro Island} itself and coastal areas are highly multilingual, and children may learn Sherbro at birth but switch early on to Mende\il{Mende}, Krio\il{Krio}, or Temne\il{Temne} once outside the immediate context of the home. Moreover, there is a sizeable Krio community with considerable prestige in Bonthe (Town), the major town on Sherbro Island, with a once glorious past, as mentioned above. Bonthe itself is not part of a Sherbro chiefdom but considered an independent municipal unit, “Bonthe Urban”, comparable to Freetown the capital city and the surrounding Western Area. The large Krio population on Sherbro Island and increasing immigration by Temne and Mende speakers, attracted by the rich but diminishing fishing grounds, mean that the shift to these more widely spoken languages will continue.

Sherbro is more vital to older people, as the opening quote of this chapter indicates. Another anecdote illustrates the pride some speakers have in their language and heritage. The quote comes from Adama Mampa, a community leader who married a Temne\il{Temne} man and lived among them for many years, yet never surrendered her Sherbro identity\is{identity}. Her family name “Mampa” is in fact the Temne word for ‘Bolom,' another exonym that has muddied the waters of scholarship. Adama is an energetic and feisty individual who heads the local Bondo\is{Bondo} Society in Shenge\is{Shenge} and was heavily recruiting for the society in 2016, when we first recorded her.\footnote{This is the girls' initiation society found throughout Atlantic West Africa (e.g., \citealt{Hoffer1975}, \citealt{MacCormack1982}); see full discussion in \sectref{sec:3.6} on society names.} In the interview Adama Mampa stated quite clearly how much she loved her language and culture. Her feistiness is revealed in the following excerpt; she is fiercely proud of her language and culture, even among the disparaging Temne.

\ea\label{ex:4}
Labila jami abolomai, ko gbi lɔa kɔ jami abolomai, nche mi la siŋɛ.\\
\gll labila        ja      mi    a-bolom-ai         ko    gbi  lɔ      a    kɔ\\
that.is.why    matter  \textsc{1sg}  \textsc{ncm}\textsubscript{ha}{}-Sherbro  {}-in    to    all    where  \textsc{1sg}  go\\
\gll ja      mi    a-bolom-ai        n    che  mi   la  siŋ  {}-ɛ\\
matter  \textsc{1sg}  \textsc{ncm}\textsubscript{ha}{}-Sherbro  {}-in    you  be    1\textsc{sg}  it  play-\textsc{prt}\\
\glt ‘That is why in any Sherbro business, anywhere I go, if it concerns the Sherbro, you don't joke with me.' (009-10a Lohr \& Mampa: 192)
\z

Further on in the passage, she explains how she would reply to Temne\il{Temne} greetings – not in Temne but rather in Sherbro. Nonetheless, her speech shows the permeability of linguistic boundaries. In a fairly long discourse on the vitality\is{vitality} and importance of the language, she peppered her speech with Krio\il{Krio} and English words, e.g., \textit{moto} ‘automobile,' \textit{respect} [respɛk] and especially conjunctions such as \textit{so}, \textit{because} [bikɔs], and \textit{then} [dɛn] (009-10a Lohr \& Mampa).

An encouraging fact is that, although Sherbro may have lower prestige than Krio\il{Krio}, Temne\il{Temne}, and Mende\il{Mende}, it does not seem to have the stigmatization of other marginalized languages in the area. A possible explanation is the historical role of the Sherbro people as intermediaries in trade with powerful Europeans, especially in the commerce of slavery, as well as in their close relationship to Krio culture, as represented in the once thriving metropolis of Bonthe. The Sherbro involvement in trade and their traditional matrilineality has led to Afro-British families such as the Tuckers, Caulker\ia{Caulker}s, Rogerses, and Clevelands, many of them installed as royal families within Sherbro chiefdoms (\citealt{Caulker-Burnett2010}, \citealt{ReedRobinsonforthcoming}).\footnote{Coincidentally, an early North American academic researcher was Henry Rogers (\citealt{Rogers1967}, \citealt{Rogers1970}), no relation to the Sherbro Rogers; in addition, the present author is named Tucker Childs, similarly no relation but an attendee at several Tucker family reunions in Sierra Leone.}

\tabref{tab:intro:1} lists all historically Sherbro (or near-Sherbro) chiefdoms and a characterization of the languages used in the chiefdom. Much of the information comes from \citet{ReedRobinson2013}, but also from independent fieldwork. Note that Krio\il{Krio} is spoken in all chiefdoms as a \textit{lingua franca} and therefore is not listed as a language of any one chiefdom. The language listed first when there is a slash between two languages is the dominant one. A comma between languages indicates the languages are of equal prominence.

\begin{table}[t]
\caption{\label{tab:intro:1}Sherbro chiefdoms}

\small
\begin{tabularx}{\textwidth}{llQQ}
\lsptoprule
District & Chiefdom & Language & Comment\\
\midrule
Bo & Bumpeh Ngao & Mende\il{Mende}, Temne\il{Temne} & ethnic Sherbros\\
\tablevspace
Bonthe & Bendu-Cha & mostly Sherbro?/Mende\il{Mende} & unknown\\
& Bum & “Bom”/Mende\il{Mende} & a few old people speak Bom-Kim\il{Bom-Kim}\\
& Dema\is{Dema} & Sherbro/Mende\il{Mende} & a high degree of bilingualism\\
& Imperri & Mende\il{Mende} & \\
& Jong & Mende\il{Mende} & \\
& Kpanda Kemo & unknown & paramount chief is Mende\il{Mende}\\
& Kwamebai Krim & Mende\il{Mende} & formerly Bom-Kim\il{Bom-Kim}\\
& Nongoba Bullom & Mende\il{Mende} & formerly Sherbro\\
& Sittia\is{Sittia} & Mende\il{Mende}/Sherbro & \\
& Yawbeko & Mende\il{Mende} & possibly some Sherbro\\
& Bonthe Urban & Mende\il{Mende}/Sherbro & former major trade center\\
\tablevspace
Moyamba & Bagruwa & Temne\il{Temne} & \\
& Bumpeh & Sherbro/Temne\il{Temne} & \\
& Kagboro & Sherbro/Mende\il{Mende}/Temne\il{Temne} & cultural center of the Sherbro\\
& Kaiyamba & Mende\il{Mende} & historically Sherbro chiefdom\\
& Kongbora & Mende\il{Mende} & no Sherbro speakers\\
& Ribbi & Temne\il{Temne} & possibly some Sherbro\\
& Timdale & Mende\il{Mende}/Sherbro & \\
\lspbottomrule
\end{tabularx}
\end{table}

Although the research is incomplete, one can make a few generalizations. First, there is no one chiefdom where Sherbro is the primary language of all or even the majority of its inhabitants. Second, the languages in competition are all glottophagic varieties, namely, Temne\il{Temne} and Mende\il{Mende}, encroaching on historically Sherbro areas. Finally, a fact not shown in the table is that Krio\il{Krio} is making inroads everywhere, especially among the young, in becoming the primary language of its many users as they grow older.

The growing desuetude of Sherbro, coupled with the growing contact\is{contact} with Krio\il{Krio}, Temne\il{Temne}, and Mende\il{Mende}, has had consequences for the language, primarily in the lexicon. Although the inroads are significant, the changes are not as momentous as those with the moribund Bom-Kim\il{Bom-Kim}. An elderly speaker admitted with regret that Sherbro is no longer spoken in his natal village.


\ea %5
\label{ex:5}
Ko lɔ pɔ gbem miɛ ma lɔ kɔ gbi mbolom mɔ ma lɔ bɔ theɛ.\\
\gll ko    lɔ      pɛ      gbem    mi-ɛ\\
to    where  \textsc{pro}\textsubscript{indef}  bear    \textsc{1sg-prt}\\
ma    lɔ    kɔ    gbi  n-bolom      mɔ  ma    lɔ    bɔ    the-ɛ\\
\textsc{ncp}\textsubscript{ma}    there  go    all    \textsc{ncm}\textsubscript{ma}{}-Bolom  \textsc{2sg}  \textsc{ncp}\textsubscript{ma}    there  can  hear-\textsc{prt}\\
\glt ‘Where I was born, no matter where, it was only Sherbro that you would hear.' (094a Ansu Kagboro: 86)
\z

His feelings about Sherbro are clear: You should speak the language of your village (and it should still be spoken by everyone there).

\ea %6
\label{ex:6}
Wɔkɛ kɔŋ kɔlɔɛ, kɔŋ kɔ mɔ ŋa wɔ lɔɛ.\\
\gll hɔk      ɛ    kɔ-ŋ        kɔ      lɔ-ɛ\\
language  \textsc{def}  \textsc{ncp}\textsubscript{kɔ}{}-\textsc{emph}    \textsc{ncp}\textsubscript{kɔ}    there-\textsc{prt}\\
\gll kɔ-ŋ        kɔ      mɔ  ha      wɔ      lɔ-ɛ\\
\textsc{ncp}\textsubscript{kɔ}{}-\textsc{emph}    \textsc{ncp}\textsubscript{kɔ}    \textsc{2sg}  \textsc{opt}    speak    there-\textsc{prt}\\
\glt ‘The language that is there, it is what you should speak there.' (094a Ansu Kagboro: 96)
\z

But because it is no longer spoken there, as is suggested by the previous quotes taken from the same interview, Ansu concludes,

%\label{bkm:lgshiftnotsweet}

\ea%7
\label{ex:7}
Nle kɔ bo mpɔni nwɔk mpika ntuk maɛ; labi la pɛthi lɛ mini.\\
\gll n    le    kɔ      bo        n    pɔni      n-hɔk          n-pika\\
\textsc{2sg}  leave  \textsc{ncp}\textsubscript{kɔ}    completely  \textsc{2sg}  throw.self  \textsc{ncm}\textsubscript{ma}{}-language  \textsc{ncm}\textsubscript{ma}{}-other\\
\gll n    tuk  ma    ɛ      labi        la  pɛthilɛ  mi    ni\\
\textsc{2sg}  lose  \textsc{ncp}\textsubscript{ma}    \textsc{prt}    that.is.why    it  sweet    \textsc{1sg}  \textsc{neg}\\
\glt ‘If you leave it and throw yourself into another language, you lose it; that is why it is not sweet to me.' (094a Ansu Kagboro: 97)
\z

This opinion, however, was in the minority, or at least does not influence his fellow Sherbro, who continue to abandon their language.

In addition to the inroads made by indigenous groups, neocolonialist endeavors, especially the extractive industries, have had a significant impact on the Sherbro people. Sierra Rutile, for example, a wholly-owned subsidiary of Iluka Resources, an Australia-based resources company, has pretty much destroyed vast swaths of the country, where the Sherbro once had farms. The company produces high quality rutile, ilmenite and zircon from the world's largest natural rutile deposit. The company's operations are located on the border between Moyamba and Bonthe districts, a short distance from the coast in the heart of historically Sherbro country.

Another threat to the livelihood of the Sherbro people is international fishing concerns from China, Korea, EU countries, and Russia who operate on the periphery of legality and have decimated the fishing banks of the Sherbro area (\citealt{Economist2017}). Fishermen have recounted how their catches have depleted in both quality (size and variety of the fish) and number (142v Baba Mandela, Fishing). Huge factory ships lurk offshore, as detailed in reports by Greenpeace Africa, \textit{The Economist}, and the United Nations (\citealt{Joaque2017}). They also tell of sturdy fishing trawlers, owned mostly by Asian and European companies, that drag trawl nets over a large expanse of seabed (\citealt{Ighobor1917}). Overfishing is widespread; some thirty-seven species were classed as threatened with extinction and fourteen more were said to be “near threatened” from Angola in the south to Mauritania in the north, according to the International Union for the Conservation of~Nature (\citealt{Ighobor1917}).

Despite the heroic efforts of speakers like Adama Mampa and others, Sherbro is highly endangered and moreover poorly documented. On these points all analysts agree, although estimates as to the number of speakers vary widely and must be considered uncertain. Today, even if Sherbro is learned by children in interior villages, it is soon abandoned in favor of Mende\il{Mende} (in the south and east), Temne\il{Temne} (in the north and east) or Krio\il{Krio} (everywhere). Its nearest congeners, Bom-Kim and Mani\il{Mani}, are already nearly extinct. This is not to say that the Sherbro culture will disappear or be assimilated to any of the three major interlopers; the Sherbro identity\is{identity} is vital to the autochthonous inhabitants of the area, despite their many ties to the Krios and others (\citealt{Ménard2015}). However, without a change in attitudes and behaviors, the Sherbro language will likely only exist in a few isolated interior villages within a generation or two. The best-case scenario for Sherbro's survival is that it will be as part of a multilingual speech economy in a multiglossic complex.

\section{Dialects}
\label{sec:1.8}\hypertarget{Toc115517750}{}
Previous studies have identified a number of major dialects. The greatest differentiation can be seen in the map in \citet{Hanson1979a} (\figref{fig:intro:6}). He identifies five different dialects: Sitia, Ndema (our “Dema\is{Dema}”), Ronde, North Bolom, and Shenge\is{Shenge} (our “Bumpeh" and “Kagboro"). We found evidence of far fewer Sherbro speakers in the Sittia Chiefdom where the dominant language is Mende. The Dema dialect is also much less widely spoken than is indicated on his map. In general, the Sherbro-speaking area has contracted considerably since Hanson's time. Ronde, if it exists, is used only in a few fishing villages by elderly people. “North Bolom” is likely Mani\il{Mani}, though the actual location is further north in the Samu/Samoun region straddling the border between Sierra Leone and Guinea (the international boundary in the upper left-hand corner of the map). My own work shows that Mani is a distinct language and not a Sherbro dialect (\citealt{Childs2011}).

\begin{figure}
\caption{Sherbro dialects (\citealt{Hanson1979a})}
\label{fig:intro:6}
\includegraphics[width=\textwidth]{figures/childsmod-img006.jpg}
\end{figure}

\citet{Hanson1979a}'s analysis conforms generally to a later, more systematic study done by \citet{IversonCameron1986}, as shown in \figref{fig:intro:7}. What they call “Bullom So” corresponding to Hanson's “North Bolom” is the distinct but closely-related language I have identified as Mani\il{Mani} (\citealt{Childs2011}). They also consider Krim (now “Bom-Kim\il{Bom-Kim}”) to be a separate language, though it likely forms part of a dialect continuum with Sherbro (\citealt{Childs2024a}) (see \sectref{sec:1.3}).

\citet{IversonCameron1986} found intelligibility to be high (90\%) among the four Sherbro dialects they identified. In terms of language vitality\is{vitality}, they found even then that “Krim” had nearly disappeared and Mani\il{Mani} was not far behind (both accurate assessments). Kim is probably the more distantly-related language of the Bolom group according to the pair, since no Kim dialect had an intelligibility percentage higher than 45\% with any Sherbro dialect (not our assessment).\footnote{The name of the group is variously rendered “Bullom” or “Bolom.” We choose “Bolom” reflecting its pronunciation in these languages (see \sectref{sec:2.1.1.2}).} Mani had no rating higher than 69\% and would likely be lower. The four Sherbro dialects, on the other hand, obtained scores as high as 90\% among themselves (\citealt{IversonCameron1986}). My own findings contradict the distinctiveness of Kim, which is the same language as Bom. Moreover, we were able to find only a handful (twenty or so) elderly speakers who actually were fluent in the language. There were many more (several hundred) speakers of Bom (\citealt{Childs2020}).

\begin{figure}
\caption{Bolom language and dialect boundaries (\citealt{IversonCameron1986})}
\label{fig:intro:7}
\includegraphics[width=\textwidth]{figures/childsmod-img007.jpg}
\end{figure}

Tentatively we identified three dialects centering around the modern chiefdoms of Bumpeh, Kagboro, and Dema. Some differences are presented in \tabref{tab:intro:2}. 

\begin{table}
\caption{\label{tab:intro:2}Dialect differences}

\begin{tabularx}{\textwidth}{QQQl}
\lsptoprule
& Bumpeh & Kagboro & Dema\is{Dema}\\
\midrule
‘sell' & wɔŋgul & wɔŋhul & \\
‘kill'  & ji & di & \\
‘medicine'  & nwɔmdɛ & nrɔmdɛ & \\
‘home'  & woŋgo & hoŋko & \\
‘the child'  & tamɔɛ & taamɔ or tamɔɛ & tamɔlɛ or tamɔilɛ\\
‘young woman' & & waantɛ & waantalɛ\\
‘glass, mirror' & mɛmdɛɛ & mɛmdɛɛ & mɛmdɛlɛ\\
‘go' & kɔ & kɔ & ko\\
‘palm oil' & nkuaɛ & nkuaɛ & nkuai\\
‘water' (n.) & mɛn & mɛn & min\\
\lspbottomrule
\end{tabularx}
\end{table}

These correspond to what both \citet{Hanson1979a} and \citet{IversonCameron1986} call Shenge and Ndema. What \citet{Hanson1979a} and \citet{IversonCameron1986} identify as Shenge includes speakers in the contiguous chiefdoms of Bumpeh, Kagboro, and Timdale located along the coast northwest to southeast (\figref{fig:intro:5}). Shenge is the coastal town that serves as the headquarters of Kagboro Chiefdom, but because we identified differences between what was spoken in Bumpeh Chiefdom and Kagboro Chiefdom and did not travel in the Timdale Chiefdom, we have chosen to call the corresponding dialects Bumpeh and Kagboro rather than calling either of them Shenge. We interviewed several people with ties to Timdale Chiefdom without identifying any global differences in their speech from Kagboro speakers, however all of them were living in Kagboro Chiefdom at the time. 

In addition to the examples in \tabref{tab:intro:2}, Bumpeh speakers voice the second part of medial prenasalized stops unlike speakers elsewhere.

\ea %8 
\label{ex:8}
\begin{tabular}[t]{ll}
  wɔŋgul   &  ‘sell' (as in \tabref{tab:intro:2})\\
   maŋgo  & ‘mango' (vs. maŋko)\\
   nandɛ  &   ‘today' (vs. nantɛ)\\
    woŋgo & ‘home' (vs. hoŋko)\\
\end{tabular}
   \z  

   \largerpage

The Dema\is{Dema} dialect is distinctive to the ears of Kagboro speakers, who say that speakers in Dema Chiefdom say \textit{lɛ}, \textit{lɛ}, \textit{lɛ} all the time.\footnote{The pronunciation of the chiefdom's name is [ndema] by speakers of all dialects. We have used the spelling without the prenasalized consonant because this is consistent with the government's official spelling for the chiefdom. See further discussion on orthography in \sectref{sec:2.1.2}.} Many Sherbro speakers make note of the use of [lɛ] for the definite marker in Dema\is{Dema} as opposed to the use of [ɛ] elsewhere. The form \textit{lɛ} is a realization of the definite article found throughout Bolom, except for Kisi\il{Kisi}, where there are only traces (the feature [+\textsc{lateral}]). The form \textit{lɛ} likely represents the older (reconstructible) form of the article (\citealt{Childs2016}). In the Kagboro dialect it has eroded sometimes to just vowel length on the preceding syllable, e.g., \textit{tamɔlɛ} ‘the child' in Dema vs. Kagboro \textit{taamɔ} (see \sectref{sec:3.4} for a characterization of the allophony). Dema thus represents a conservative dialect retaining the older form, an unsurprising fact given their social and geographical isolation.

A second contrast between the two dialects is the way the diphthong /ɔɪ/ is pronounced. Kagboro speakers pronounce it [ɔɪ], but Dema\is{Dema} speakers have raised both ends of the trajectory to [ui] or [wi], e.g., \textit{mɔɪ} ‘afternoon (greeting)' is pronounced as [mɔɪ] in Kagboro and [mui] in Chepo, the chiefdom headquarters of Dema. In fact, the afternoon greeting is used more widely in Dema. People in Dema greet each other with \textit{mui} [mwi] at any time of the day rather than just in the afternoon, as is the case in Shenge and elsewhere in Kagboro Chiefdom (Albert Yanker, p.c., 1  {March 2016}).

Perhaps an additional example of the vowel height contrast is seen in a well-known marker of the dialect which occurs in the pronunciation of the word for ‘water'. In Dema, ‘water' is pronounced [min] making it homophonous with the word for ‘devil' while Bumpeh and Kagboro speakers contrast [mɛn] ‘water' and [min] ‘devil'  (\tabref{tab:intro:2}). Kagboro speakers will say, even when they have been in Dema\is{Dema} for a while, \textit{À bìɛ́ní ‘mìn'} ‘I don't have ‘devil'\is{devils}' to indicate their pronunciation has not assimilated.
 
In Shenge\is{Shenge}, the chiefdom headquarters for Kagboro, speakers use a centralized variant [ə] for /ɛ/. \citet{Pichl1967}whose principal consultants were in Shenge gives [ə] variants for [ɛ] in other dialects, e.g., [nyɛŋkəlɛŋ] for [nyɛŋkɛlɛŋ], as it is in every other related variety, even in Mani\il{Mani} where [ə] exists as an independent phoneme. The use of [ə] extends even to a word such as ‘heart' [gbɔl], pronounced as [gbəl], a variant which also features the front-back alternation (see \sectref{sec:2.1.1}). A comparable example is [pə] for [pɔ] ‘people'. The schwa variant is widespread, used by many Kagboro speakers (see \sectref{sec:2.1.1}). Because the dialect differences illustrated in \tabref{tab:intro:2} are tentative, they are discussed further as linguistic variants in \sectref{sec:2.1.2}.

Set within the Sherbro-speaking area of the Sittia\is{Sittia} Chiefdom on Sherbro Island\is{Sherbro Island} is rumored to be a variety known as “Sei.” Sei is rather a highly divergent dialect. Following is a characterization from Walter Pichl in his dictionary entry for \textit{Se}:
\begin{quote}
    … a language (dialect) of the southern and eastern part of Bonthe Island, including Bonthe [the town].\footnote{The island is known as both Bonthe Island and Sherbro Island\is{Sherbro Island}. The former name comes from its major town and the latter from its inhabitants. The latter is the term used here and is the one featured on most maps.} People say that in olden times a big boat arrived with foreign black people. When they were asked where they came from, they answered, “a se” – ‘I don't understand.' They mixed with the aborigines and became the \textit{Ase} [‘Se people'] (sg. \textit{senɔ}) of our days (\citealt[82]{Pichl1967}).
\end{quote}
    
This somewhat apocryphal story may relate to resettled recaptives brought by the British. Bonthe was once one of the two capitals of the Sierra Leone colony and a bustling center of trade for many years before ceding its importance to Freetown, the other of the two original capitals. Sei no longer seems to be an extant variety distinct from Sherbro.

\section{Orthography and conventions}
\label{sec:1.9}\hypertarget{Toc115517751}{}
The Sherbro Literacy Committee\is{Sherbro Literacy Committee}, which was headed by the late Albert Yanker, has proposed conventions for rendering the Sherbro language, most of which are followed here. I present them below, divided into conventions followed and conventions not followed.

\ea%9
\label{ex:9}
Conventions adopted by the Sherbro Literacy Committee\is{Sherbro Literacy Committee}\\
\begin{itemize}
    \item[A.] Conventions followed\\
\begin{itemize}
\item[]The digraph <th> for dental “t” [t̪] and <t> for its alveolar counterpart

\item[]The digraph <ch> is used for the voiceless affricate /ʧ/.

\item[]<j> is used for the voiced affricate [ʤ]

\item[]<v> and <w>, though allophones of the same phoneme, are represented with different symbols
\end{itemize}

\item[B.] Conventions not followed\\
\begin{itemize}
\item[]Although tone is recognized as a distinctive feature of the language, it is not marked, except occasionally when homophones need to be differentiated, e.g., \textit{ŋá} ‘2\textsc{pl}' vs. \textit{ŋà} ‘3\textsc{pl}.'

\item[]An apostrophe is used to separate clitic-like elements from the units to which they cliticize, e.g., the clause binder <'ɛ>, the definite article <'ɛ>, but only when they follow vowel-final nouns.

\item[]An apostrophe to separate the locative cliticizing preposition <'ai> ‘in,' again, only when they follow vowel-final forms.
\end{itemize}
\end{itemize}
\z


The example in \REF{ex:10} illustrates the many (near) homophones of \textit{ha(a)}.

\ea%10 
\label{ex:10}
The many uses of \textit{ha(a)}\\
\ea 
\label{ex:10a}
Hà tɔ̀nkɔ́ há hà lá ha háá hà yàŋ dɛ̀.\\
\gll hà    tɔ̀nkɔ́    há    hà    lá    hà    háá  hà    yà-ŋ      ɛ̀\\
\textsc{3pl}  praise    \textsc{2pl}  for    \textsc{rel}  \textsc{3pl}  do    for    \textsc{1sg-emph}  \textsc{prt}\\
\glt ‘They praised you (pl) for what they did for me.' (Albert Yanker 15 Mar 2017 p.c.)\\

\vspace{6pt}

\ex
\label{ex:10b}

\begin{tabular}[t]{ll}
hà & 3\textsc{pl} pronoun ‘they'; \textit{ha}{}-class pronoun\\
há & 2\textsc{pl} pronoun ‘you-all'\\
hà & preposition ‘for, etc.'\\
hà & subordinating conjunction ‘that,' etc.\\
hà & optative auxiliary ‘should,' etc.\\
haa & ‘do, make' (tones vary)'\\
hááŋ & ‘on and on, for some time or distance'\\
\end{tabular}
\z
\z

Aside from illustrating many phonological contrasts, e.g. tone, length, as well as some phonological processes (nasalization after [h] (not shown)), this sentence displays some of the confusion the Sherbro Literacy Committee\is{Sherbro Literacy Committee} felt might arise from so many like-sounding words. They felt at least some of the words should be differentiated.

Their solution was quite a sensible one using the orthography and several known phonological processes. The Sherbro Literacy Committee\is{Sherbro Literacy Committee} built on the known vowel nasalization after [h] found throughout Bolom and the perseveratory nasalization following the velar nasal (“rhinoglottophilia” in the learned term from \citet{Matisoff1975}, see \sectref{sec:2.4}). Thus, for the word \textit{há}, the second person plural pronoun ‘you-all,' phonemically /há/ would be pronounced [hã]. Fortunately, nasalization of vowels is particularly prominent on low vowels (\citealt{Ohala1975}); the strategy might not be so successful were the homophony involving high vowels.

The velar nasal is not a common initial segment, present at the beginning of only a few words, but in all cases the vowel is heavily nasalized.\footnote{One analyst says “never”: “<ŋ> is used before <gb> and <k>. This is the only time <ŋ> occurs in this initial position” (\citealt[1]{Hanson1979b}). In a project lexicon of 4,095 words only 10 words began with <ŋ> (excluding those created by the Literacy Committee for names, borrowings, etc.).}

\ea%11
\label{ex:11}
\begin{tabular}[t]{lll}
ŋɛi & [ŋɛ̃\~\i] & ‘open one's mouth'\\
ŋɔthi & [ŋɔ̃thi] & ‘fish' (v.)\\
ŋal & [ŋãl] & ‘elephant grass'\\
\end{tabular}
\z

The velar nasal, with its low spectral intensity\is{intensity}, could be readily interpreted as an [h]-initial form with the predictable following nasalization. Thus, the phonetic similarity works; in the orthography initial [ŋ] vs. [h] is available to contrast similar-sounding forms. Thus, some homophonous items begin with an <h> and some with <ŋ>.

\ea%12 
\label{ex:12}
\begin{tabular}[t]{lll}
<h> & haa & ‘do, make'\\
 & hà & \textit{ha}{}-class pronoun\\
<ŋ> & ŋá & 2\textsc{pl} pronoun\\
 & ŋà & 3\textsc{pl} pronoun\\
 \end{tabular}
 \z

Interestingly, the pronoun for 3\textsc{pl} and the \textit{ha} class is identical, as it is in many class languages, where the noun class system overlaps with the pronominal system. The genius of this approach is that the two may now be differentiated.

The orthographic conventions followed here are summarized in \tabref{tab:intro:3}. 

\begin{table}
\caption{\label{tab:intro:3}Orthographic conventions}
\begin{tabular}{lll}
\lsptoprule
Orthographic & IPA & Comment\\
\midrule
ch & ç / ʧ & \\
sh & ʃ & a dialectal variation\\
ny & ɲ & \\
v & v & /w/\\
w & w & /w/\\
\lspbottomrule
\end{tabular}
\end{table}
Although [v] and [w] are likely allophones of the same phoneme, I follow the practice of the Literacy Committee in representing the two sounds with distinct symbols (see \sectref{sec:2.1.2}).

Tones are not shown unless relevant to the discussion. Syllabic nasals representing either the \textit{ma}\nobreakdash-class prefix /n-/ or the 2\textsc{sg} subject pronoun /n-/ are low-toned and absorbed by the following consonant to become part of a non-tone-bearing prenasalized stop. Nasal prefixes are always homorganic with the following stem-initial consonant.

\ea\label{ex:13}
\begin{tabular}[t]{lllll}
n- & baana & $\xrightarrow{}$ & mbaana & ‘bananas'\\
\textsc{ncm}\textsubscript{ma}   & banana & & \\ 
& & \\
n- & koloŋ & $\xrightarrow{}$ & ŋkoloŋ &   ‘testicles'\\
\textsc{ncm}\textsubscript{ma} & testicles & & &\\
& & \\
n & bi & $\xrightarrow{}$ & mbi &    ‘You have...'\\
\textsc{2sg} & have & & & \\      
& & \\
n & kɔ & $\xrightarrow{}$ & ŋkɔ & ‘Go!'\\
\textsc{2sg} & go & & & \\
\end{tabular}
\z

In addition to the conventions adopted with regard to the language, there are decisions that have been made about variant spellings of place names, whether to opt for the actual pronunciation, a phonemic representation, or the commonly used spelling. I have opted for the last option in most cases, generally following governmental practices, using the spellings in the second column of \tabref{tab:intro:4}.\footnote{My thanks to Solomon Gbani in the Department of Geography at Fourah Bay College in helping me reach these decisions.}

\begin{table}
\caption{\label{tab:intro:4}Variant spellings of place names}
\begin{tabular}{ll}
\lsptoprule
Pronunciation & Official spelling\\
\midrule
Setia, Setie & Sittia\is{Sittia}\\
Tombo & Tumbo\\
Dema\is{Dema} / Ndema & Dema\is{Dema}\\
Tisana & Tissana\\
\lspbottomrule
\end{tabular}
\end{table}

This brief comment illustrates the problems in settling on a single spelling for a place name. There are traditions, such as the colonial legacy of misheard vowels: /e/ and /o/ as [i] and [u] and the lack of an entirely phonemic writing system (see \sectref{sec:2.1.2}).

With this brief introduction to the phonology via the orthography, Chapter \ref{ch:2} turns to the phonology proper.
