\addchap{\lsAbbreviationsTitle}
% \addchap{Abbreviations and symbols}


\begin{tabbing}
unspec. compx.\hspace{1ex} \= subordinating connective\kill
\textsc{1sg} \> first person singular\\
\textsc{1pl} \> first person plural\\
\textsc{2sg} \> second person singular\\
\textsc{2pl} \> second person plural\\
\textsc{3sg} \> third person singular/noun class pronoun for \textit{wɔ}-class\\
\textsc{3pl} \> third person plural/noun class pronoun for \textit{ha}-class\\
\textit{adj} \>  adjective\\
\textit{adp} \>  adposition\\
\textit{adv} \>  adverb \\
\textsc{aux} \> auxiliary\\
\textsc{aux.neg} \> auxiliary whose presence triggers \textsc{neg} interpretation \\
\textsc{c} \> consonant\\
\textit{cf.} \> compare\\
\textsc{consec} \> consecutive\\
\textit{coordconn} \> coordinating connective\\
\textsc{cop} \> copula\\
\textsc{caus} \> causative\\
\textsc{def} \> definite marker\\
\textit{dem} \> demonstrative\\
\textit{disco} \> discourse element\\
\textsc{distr} \> distributive\\
\textsc{emph} \> emphatic\\
\textsc{fut} \> future\\
\textsc{idph} \> ideophone (angle brackets for text)\\
\textit{ifx} \> infix\\
\textsc{ipfv} \> imperfective\\
\textit{indfpro} \> indefinite pronoun\\
\textsc{ins} \> instrumental\\
\textit{interj} \>  interjection\\
lit. \> literal\\
\textit{n} \> noun\\
\textit{nam} \> name\\
\textit{NCM} \> noun class marker\\
\textsc{ncm}\textsubscript{ha} \> noun class marker for \textit{ha}-class\\
\textsc{ncm}\textsubscript{si} \> noun class marker for \textit{si}-class\\
\textsc{ncm}\textsubscript{ma} \> noun class marker for \textit{ma}-class\\
\textsc{ncm}\textsubscript{hɔ} \> noun class marker for \textit{hɔ}-class\\
\textsc{ncm}\textsubscript{tha} \> noun class marker for \textit{tha}-class\\
\textsc{ncm}\textsubscript{lɔ}  \> noun class marker for \textit{lɔ}-class\\
\textit{NCP} \> noun class pronoun\\
\textsc{ncp}\textsubscript{wɔ} \> noun class pronoun for \textit{wɔ}-class\\
\textsc{ncp}\textsubscript{ha} \> noun class pronoun for \textit{ha}-class\\
\textsc{ncp}\textsubscript{kɔ} \> noun class pronoun for \textit{kɔ}-class\\
\textsc{ncp}\textsubscript{ma} \> noun class pronoun for \textit{ma}-class\\
\textsc{ncp}\textsubscript{hɔ} \> noun class pronoun for \textit{hɔ}-class\\
\textsc{ncp}\textsubscript{tha} \> noun class pronoun for \textit{tha}-class\\
\textsc{ncp}\textsubscript{lɔ} \> noun class pronoun for \textit{lɔ}-class\\
\textsc{near.pst} \> near past\\
\textsc{neg} \> negative\\
\textsc{np} \> noun phrase\\
\textit{Numb} \>  number\\
\textsc{o,} \textsc{obj} \> object\\
\textsc{opt} \> optative\\
\textsc{pst} \> past\\
p.c. \> personal communication\\
\textsc{pfv} \> perfective\\
\textit{pfx} \> prefix\\
pl \> plural\\
\textit{post} \> postposition\\
\textit{prep} \>  preposition\\
\textsc{pro} \> pronoun\\
\textsc{pro}\textsubscript{indef} \> indefinite pronoun\\
\textsc{prog} \> progressive\\
\textsc{prt} \> particle\\
\textsc{q} \> interrogative particle\\
\textit{quant} \>  quantifier\\
\textsc{refl} \> reflexive\\
\textsc{rel} \> relative pronoun\\
\textsc{rem.pst} \> remote past\\
\textit{sfx} \>  suffix\\
sg \> singular\\
sth \> something\\
\textsc{s}, \textsc{subj} \> subject\\
\textit{subordconn} \> subordinating connective\\
SLC\index{SLC!} \> Documenting Sherbro Language and Culture Project\\
\textit{temp} \>  temporal adverb\\
\textsc{tmap} \> tense-mood-aspect-polarity\\
\textsc{tns} \> tense\\
\textit{ubd} \>  unbound stem\\
\textsc{v} \> vowel; verb\\
\textsc{vp} \> verb phrase\\
\end{tabbing}

\textbf{Dictionary Only}

\begin{tabbing}
unspec. compx.\hspace{1ex} \= subordinating connective\kill
comp. \> compound\\
der. \> derivational form\\
id. \> idiom\\
unspec. \> unspecified complex form
\end{tabbing}

\begin{tabbing}
unspec. compx.\hspace{1ex} \= subordinating connective\kill
Arabic \> Arabic language\\
Eng \> English language\\
Krio \> Krio language\\
Mandinka \> Mandinka language\\
Mende \> Mende language\\
Port \> Portuguese language\\
Soso \> Soso language (aka Susu)\\
Themne \> Themne language (aka Temne)\\ 
\\
(B dialect) \> dialect spoken in Bumpeh Chiefdom\\ 
(K dialect) \> dialect spoken in Kagboro Chiefdom\\
(Nd dialect) \> dialect spoken in Ndema Chiefdom (aka Dema)\\
\end{tabbing}