\chapter{Verbal morphology}
\label{ch:4}\hypertarget{Toc115517781}{}
There is very little inflectional morphology in Sherbro, primarily because of the peripheral status of tone, which is important for aspectual and modal contrasts in closely related languages. The language may be moving to a more analytic configuration, likely due to language contact\is{contact}, particularly with the analytic extended pidgin Krio\il{Krio} and with Mende\il{Mende}, a highly analytic language typical of the Mande\il{Mande} group (\citealt{Dwyer1989}, \citealt{Dwyer1998}; \citealt{Vydrin2004}, \citealt{Vydrin2012}). In \sectref{sec:3.8}, I discussed the reanalysis of the verb extension \textit{{}-ka} as a preposition, which is one example of the process. The overlapping of functions of inflectional and non-inflectional processes, as shown in \REF{ex:109}, is the type of synchronic variation that could lead to reanalysis in the verbal system.

Aspect dominates the distinctions manifested in the verbal system of Sherbro. The perfective and imperfective are both large umbrellas. The perfective includes ‘perfective' and ‘past' as well as ‘realis'. Imperfective has an even greater semantic range: ‘imperfective,' ‘habitual,' ‘present,' ‘irrealis,' and ‘future' (see detailed discussion in \citealt{Corcoraninprep}). Other distinctions are analytic (separate words) or derivational, verb extensions and compounding.

\ea%109
    \label{ex:109}
    \ea \label{ex:109a} Perfective 
    \ea À jó.\\
    ‘I have eaten, I ate, etc.'\\ 
  
    \ex À kóŋ jò.\\
    ‘I have eaten, I finished eating.'\\
    \z
  
    \ex \label{ex:109b} Imperfective
    \ea Á jò.\\
    ‘I am eating, will eat, etc.'\\
    
    \ex À kɔ́ jò.\\ 
    ‘I will eat, I am going to eat.'
\z
\z
\z

The perfective-imperfective contrast is represented in \REF{ex:109}: perfective in  \REF{ex:109a} and imperfective in \REF{ex:109b}. Only tone differences mark the contrast as seen in the first sentence in \REF{ex:109a} and in \REF{ex:109b}. In the second sentence are substitutes for at least one meaning of the forms in the first sentence. Neither form in the second sentences is an uncommon grammaticalization cline: a verb ‘finish' provides a perfective meaning and a verb ‘go' is used for the future. Notably they are distinct elements separate from the verb, constituting pleonastic constructions, representing the more general trend toward analysis in the language. Other elements are not so transparently related to lexical items.

In describing the verbal morphology of Sherbro, I begin with the basic and central distinctions that are highly grammaticalized and continue on to the more peripheral functions marked by particles.

\section{Aspect}
\label{sec:4.1}\hypertarget{Toc115517782}{}
The examples in \REF{ex:110} show the perfective-imperfective (\textsc{pfv}{}-\textsc{ipfv}) contrast again, which in this particular context is purely tonal. The pronoun in the \textsc{ipfv} has a high tone, and the verb is low. In the \textsc{pfv} the pronoun has a low tone, and the verb is high. (These tonal distinctions were not reliably produced by all consultants in elicitation contexts, and not all forms in this section and elsewhere are marked for tone.)

\ea \label{ex:110}
Perfective-imperfective distinction\\
\ea 
\ea Yá jò.\\   
‘I'm eating.'\\                

\ex Yà jó.\\
‘I ate (have eaten).'\\
\z
   
\ex 
\ea Wɔ́ bàs bòé kò.\\
‘He is sweeping the kitchen.'\\
   
\ex Wɔ̀ bás bòé kò.\\
‘He swept the kitchen.'
\z
\z
\z

It is markings such as these that distinguish verbs from other word categories, which do not change their tones. Additional aspectual distinctions are marked by auxiliaries or verbs in the process of grammaticalization, e.g., \textit{koŋ} ‘finish' as in \REF{ex:109}. Another diagnostic is pronoun movement, as discussed in \sectref{sec:8.2.3}. Object pronouns move to a position between tense and the lexical verb in the imperfective; they appear after the verb in the perfective (see \sectref{sec:8.2} for a full discussion of the verb phrase). This test proved useful when tone distinctions were not readily apparent.

\textsc{pfv} is used to indicate that an event usually punctual has been fully realized. Only the first two examples in \REF{ex:111} have been marked for tone (a high tone on the verb).

\ea%111
    \label{ex:111}
\ea Táàmɔ̀ɛ̀ kɔ́nth bààɛ́.\\
\gll táàmɔ̀    ɛ̀    kɔ́nth    bàà    ɛ́\\
boy    \textsc{def}  catch    squirrel  \textsc{def}\\
\glt ‘The boy caught the squirrel.' (E12 Albert Yanker: 11)

\ex  Nɔ̀sààɛ́ wɔ̀ bɛ́t bàchɛ̀ kà íbáá.\\
\gll nɔ̀sàà    ɛ́    wɔ̀    bɛ́t    bàch  ɛ̀    kà    í-báá\\
tapster  \textsc{def}  3\textsc{sg}  tap  palm  \textsc{def}  with  \textsc{ncm}\textsubscript{hɔ}-curved.knife\\
    \glt ‘The tapster tapped the tree with a knife.' (E12 Albert Yanker: 12)
\ex  Braima wɔe kɔ lɔɔli  pɛl   yɛllɛɛ ni pɛl dukiɛ.\\
\gll Braima  wɔ-i      kɔ    lɔɔli    pɛl    yɛllɛ    ɛ    ni    pɛl    dukiɛ\\
Braima  \textsc{3sg-prt}    go    examine  net  chain    \textsc{def}  and  net  drop\\
\glt ‘Braima went to inspect the net chain, but the net had sunk.' (124aw Yanker, Boy Lost at Sea: 44)
\z
\z

The mark of \textsc{ipfv} is a high tone on the subject pronoun and low tones on the verb. Tense is in this way marked on the subject pronoun, which in all other cases is low toned. In \REF{ex:112a}, the high is on the subject pronoun \textit{ha} (\textsc{3pl}) with a future meaning. In \REF{ex:112b}, the high is again on the subject pronoun, this time \textit{mɔ} (\textsc{2sg}) and has spread onto the object pronoun, which forms a unit with tense marked on the subject pronoun. Example \REF{ex:112c} showcases not just the habitual meaning of the imperfective, but also the past progressive in the second clause and the negative in the third. Example \REF{ex:112d} illustrates the habitual meaning of the imperfective, i.e., what farmers regularly do once the farm-clearing detritus is dry, with a high tone on \textit{hi}, the \textsc{1pl} pronoun. The non-habitual meanings are discussed in greater detail below.

\ea%112
    \label{ex:112}
   \ea \label{ex:112a}  Pɔ̀ há  thònkà gbɛ̀ŋ.\\
    \gll pɔ̀      há    thònkà  gbɛ̀ŋ\\
        people  \textsc{3pl}  judge    tomorrow\\
        \glt ‘They will judge them tomorrow.' (E09 Albert Yanker S8: 2)
    \ex \label{ex:112b} Mɔ́ má thɔ̀k gbèŋ.\\
\gll        mɔ́  má    thɔ̀k    gbèŋ\\
        \textsc{2sg}  \textsc{ncp}\textsubscript{ma}    wash    tomorrow\\
\glt ‘You will wash them tomorrow.' (E09 Albert Yanker S8: 22)
\ex  \label{ex:112c} Atiŋdɛ ŋa kɔ skullai; bullɛ wɔn che pa kɔ skul, kɛ chen pɛ kɔ.\\
\gll a-tiŋ      ɛ    ŋa    kɔ    skul    ai\\
\textsc{ncm}\textsubscript{ha}{}-two  \textsc{def}  \textsc{3pl}  go    school  in\\
\gll bul  ɛ    wɔ{}-n      che  pa      kɔ    skul    kɛ    che-ni    pɛ      kɔ\\
one  \textsc{def}  \textsc{3sg-emph}  \textsc{prog}  formerly  go    school  but  \textsc{prog-neg}  again    go\\
\glt ‘The two go to school, the one was going to school, but he doesn't any more.' (029a Biah Heni: 18)

\ex \label{ex:112d} Lɛ̀ yɔ̀kthà hɔ̀ sɛ̀kìlɛ́ gbó, yí thɛ̀ɛ̀ ìchɛ̀kɛ́.\\
\gll lɛ̀      yɔ̀kthà      hɔ̀      sɛ̀kìl-ɛ́  gbó    yí    thɛ̀ɛ̀  ì-chɛ̀k      ɛ́\\
when    tree.cutting    \textsc{ncp}\textsubscript{hɔ}    dry-\textsc{pst}  quite    \textsc{1pl}  burn  \textsc{ncm}\textsubscript{hɔ}{}-farm    \textsc{def}\\
\glt ‘Once the newly cut brush is completely dry, we burn the farm.' (E10 Albert Yanker: 11)
\z
\z

When tone fails to be distinctive, a sure diagnostic for imperfective is pronoun movement from the post-verbal position to before the lexical verb (close to tense, which is marked on the pronoun). In \REF{ex:113}, the pronoun \textit{ma} references the Sherbro language, \textit{Mbolomdɛ}, topicalized at the beginning of the sentence, and appears after the subject \textit{hi} marked for tense and before the lexical verb \textit{theli} ‘speak.'

\ea%113
    \label{ex:113}
 Mbolomdɛ, hin kagbɔ ka ima theli.\\
\gll n-bolom        ɛ    hi-n      kagbɔ      ka    hi    ma    theli\\
\textsc{ncm}\textsubscript{ma}{}-Sherbro  \textsc{def}  \textsc{1pl}{}-\textsc{emph}  Kagboro    here  \textsc{1pl}  \textsc{ncp}\textsubscript{ma}    speak\\
\glt ‘Sherbro, we here in Kagboro speak it.' (018a Suffian Koroma: 55)
\z


The high tone is present on the pronouns (\textit{wɔ́} in both cases) even when there is no verb, as in the two locative constructions in \REF{ex:114}. The tones on the following nouns are not affected.

\ea%114
    \label{ex:114}
    \ea
    Yàìyɛ́ wɔ́ kótàɛ̀ àlɔ̀.\\
    \gll yàì ɛ́   wɔ́   kótà ɛ̀   àlɔ̀\\
      cat \textsc{def} \textsc{ncp}\textsubscript{wɔ} cloth \textsc{def} under\\
\glt ‘The cat is under the cloth.' (E12 Albert Yanker: 6)\\

\ex Kəllɛ̀ wɔ́ thɔ̀kɛ̀ àtòk.\\
\gll kɛ̀l     ɛ̀  wɔ́   thɔ̀k ɛ̀  àtòk\\
monkey \textsc{def} \textsc{ncp}\textsubscript{wɔ} tree \textsc{def} on.top\\
\glt ‘The monkey is up in the tree.' (E12 Albert Yanker: 8)
\z
\z

Other aspectual distinctions are marked with the auxiliary \textit{che} \REF{ex:112c} and the auxiliary-like \textit{koŋ} \REF{ex:109}. The meaning of \textit{che} is progressive in \REF{ex:115a}, signalling ongoing action and can be additionally marked for tense (e.g., remote past \textit{ka} in \REF{ex:115b}).

\ea%115
    \label{ex:115}
Progressive \textit{che}\\

\ea \label{ex:115a} chala bo che ŋaa beyen.\\
\gll Chala    bo    che  ŋaa  beyen.\\
      seated  just  \textsc{prog}  do    nothing\\
      \glt ‘(She is) just sitting down doing nothing.' (005a Jalikatu B. Kumba: 53)\footnote{Third person pronouns are generally not used in everyday discourse.}

\ex \label{ex:115b} A ka che siŋ bɔllɛ.\\
\gll a    ka      che  siŋ    bɔl  ɛ\\
\textsc{1sg}  \textsc{rem.pst}  \textsc{prog}  play  ball  \textsc{def}\\
\glt ‘I used to play football.' (016a Albert Yanker: 162)
\z
\z

\textit{Che} can also be used as an auxiliary in support of Negation (see \sectref{sec:4.5}).

The verb for ‘finish' (\textit{ko} and \textit{koŋ}) can also mark aspect indicating that an action is complete. Example \REF{ex:116a} illustrates a typical use of the marker, \REF{ex:116b} and \REF{ex:116c} show how it is used with a tense marker, and \REF{ex:116c} and \REF{ex:116d} show how \textit{ko} and \textit{koŋ} are virtually interchangeable.

\ea%116
    \label{ex:116}
    Perfective \textit{koŋ} and \textit{ko}\\
\ea \label{ex:116a} Isɔ bul a koŋ thukuli jomi kusɛ …\\
\gll isɔ        bul  a    koŋ    thukuli  jo    mi    kusɛ\\
morning    one  \textsc{1sg}  \textsc{pfv}    warm    food  \textsc{1sg}  leftover\\
\glt ‘One morning (after) I had warmed my leftover rice …' (002a Mabel Lohr, Midwifery: 32)

\ex  \label{ex:116b} I amɛn bullɛ ka koŋ wu.\\
\gll hi    a-mɛn    bul  ɛ    ka        koŋ  wu\\
\textsc{1pl}  \textsc{ncm}\textsubscript{ha}{}-five  one  \textsc{def}  \textsc{rem.pst}    \textsc{pfv}  die\\
\glt ‘We are five, one died a while ago.' (007a Agnes J. Simbo: 27)
\ex\label{ex:116c}  Komɔ nɔ nse ka koŋ hu?\\
\gll komɔ    nɔ      nse  ka        koŋ    wu\\
child    person  first  \textsc{rem.pst}    \textsc{pfv}    die\\
\glt ‘Has the first child died?' (090a Saidu Netteh: 76)

\ex  \label{ex:116d} Aa, komɔ mi nseyɛ ka ko wu.\\
\gll aa    komɔ    mi    nse  ɛ    ka       koŋ    wu\\
yes  child    my  first  \textsc{def}  \textsc{rem.pst}  \textsc{pfv}     die\\
\glt ‘Yes, my first child is dead.' (090a Saidu Netteh: 77)
\z
\z

As a last point about aspect, it should be noted that the active-stative distinction is relevant for marking aspect. With stative verbs such as ‘ripe' in \REF{ex:117}, the \textsc{pfv}{}-\textsc{ipfv} distinction is irrelevant as is the case also in Bom-Kim\il{Bom-Kim}, Kisi\il{Kisi}, and Mani\il{Mani} (cf. the “factative” in \citealt[14]{NurseEtAl2013}, \citealt{Welmers1973}). All of the sentences in \REF{ex:117} were felt to mean the same thing (i.e., that ‘the mango is ripe').

\ea%117
    \label{ex:117}
    \ea \label{ex:117a} M̀màŋgùɛ̀ (mà) kóŋ dùm.\\
    \gll n-màŋgù      ɛ̀    mà    kóŋ  dùm\\
    \textsc{ncm}\textsubscript{ma}{}-mango  \textsc{def}  \textsc{ncp}\textsubscript{ma}    \textsc{pfv}  ripe\\
    \glt ‘The mango is ripe.' (E13 Albert Yanker, Adj, Lex: 2)

    \ex \label{ex:117b}  Má dùmɔ̀.\\
    \gll má    dùmɔ̀\\
    \textsc{ncp}\textsubscript{ma}    ripe\\
     \glt ‘The mango is ripe.' (E13 Albert Yanker, Adj, Lex: 2)

    \ex \label{ex:117c} Mà dúm.\\
    \gll mà    dúm\\
    \textsc{ncp}\textsubscript{ma}      ripe\\
     \glt ‘The mango is ripe.' (E13 Albert Yanker, Adj, Lex: 2)
    \z
    \z

\section{Inflectional Past}
\label{sec:4.2}\hypertarget{Toc115517783}{}
The only inflectional tense distinction is the simple or general past, which here will be called “past.” It is non-specific about the time something happened in the past, as opposed to the particles near past \textit{na} and remote past \textit{ka}.\footnote{There is also some overlap with the inflectional perfective. One consultant said the two mean the same thing.} The form of the suffix is high-toned and depends for its segmental nature on the preceding vowel. Past always features one of the low vowels [ɛ ɔ a], which harmonizes with the preceding vowel. Non-low front vowels take [ɛ], non-low back vowels take [ɔ], and verbs ending in the low vowel [a] take [a]. Just as in Mani\il{Mani}, a harmonizing high-toned vowel, morphophonemically /ɛ/, changes depending on the backness specification of the stem vowel (\citealt{Childs2011}).

According to \citet[48]{Sumner1921}, some verbs take a vowel at the end when forming the past tense. His examples also include the near past marker \textit{na}. It is clear from his examples that the “extra vowel” is actually one harmonic with the stem vowel.

\begin{table}
\caption{\label{tab:verbmorph:28}The past tense ({\citealt{Sumner1921}})}



\begin{tabular}{llll}
\lsptoprule
chal & ‘sit' & chala na & ‘sat'\\
duk & ‘fall' & dukɔ na & ‘fell'\\
gbal & ‘write' & gbala na & ‘wrote'\\
hin & ‘lie down' & hinɛ na & ‘lay down'\\
sɛm & ‘stand' & sɛmɛ na & ‘stood'\\
vel & ‘call' & velɛ na & ‘called'\\
\lspbottomrule
\end{tabular}
\end{table}

\tabref{tab:verbmorph:29} provides some examples from our own work.

\begin{table}
\caption{\label{tab:verbmorph:29}The extra vowel}



\begin{tabular}{lll}
\lsptoprule
nak & ‘be sick & naka\\
mam & ‘laugh' & mama\\
fɛt / fɛs & ‘sit, be near' & fɛtɛ / fɛsɛ\\
lɛŋ & ‘greet' & lɛnyɛ\\
thɛl & ‘trim' & thɛlɛ\\
the & ‘hear, feel' & theyɛ\\
sil & ‘sting' & stilɛ\\
rɛthi & ‘spread' (wings) & rɛthiɛ\\
mɛmi & ‘be happy' & mɛmiɛ\\
fɛki & ‘disrespect' & fɛkiɛ\\
saki & ‘cease' & sakiɛ\\
sɛkil & ‘dry' & sɛkilɛ\\
gbisiŋ & ‘marry' & gbisiŋɛ\\
chɔŋ & ‘lay egg, pour' & chɔŋɔ\\
tɔŋk & ‘praise' & tɔŋkɔ\\
\lspbottomrule
\end{tabular}
\\\centering\parbox[t]{.6\textwidth}{An exception: [gbem] and [gbemɔ] ‘give birth'}
\end{table}



If the verb already has two syllables, past is added to the stem. The local linguistic expert Ba Yanker was insistent that the disrespect in \REF{ex:118} happened in the past, contrasting it with the imperfective and the perfective.

\ea%118
    \label{ex:118}
    \ea Tamɔɛ fɛkiɛ́ mi.\\
    \gll tamɔ  ɛ    fɛki-ɛ́          mì\\
    boy  \textsc{def}  disrespect\textsc{{}-pst}   me\\
    \glt ‘The boy disrespected me (has done it).' *fèkí (E10 Albert Yanker: 9)
    \z
    \z

It is not clear why only some verbs have the extra vowel with a high tone marking past. It is not dependent on phonology (syllable structure or vowel quality) nor is it conditioned by semantics (the stative-active distinction).

The examples in \REF{ex:119} show typical uses of the morpheme. In \REF{ex:119a}, the past \textit{{}-ɛ} locates the time of the rain as during the time that Mr. Ngobe was coming, and in \REF{ex:119b}, the trimmed material must be dry before it can be burned.

\ea%119
    \label{ex:119}
    \ea \label{ex:119a} A lomani yɛ Ba Ngobɛ ka che hun dɛ hwɛ lɛ hɔ  lelɛ.\\
    \gll a    lomani    yɛ      ba    ŋgobɛ  ka      che  hun  dɛ    hɔɛ    lɛ    hɔ      lel-ɛ\\
    \textsc{1sg}  remember  when    Mr.  Ngobɛ  \textsc{rem.pst}  \textsc{prog}  come  \textsc{prt}  weather  \textsc{def}  \textsc{ncp}\textsubscript{hɔ}    rain-\textsc{pst}\\
    \glt ‘I remember when Mr. Ngobe was coming that it rained.' (P67 L: 114)

    \ex \label{ex:119b} Lɛ̀ yɔ̀kthà (hɔ) sɛ̀kìlɛ́ gbó, yí thɛ̀ɛ̀ ìchɛ̀kɛ́.\\
    \gll lɛ̀    yɔ̀kthà      hɔ      sɛ̀kìl-ɛ́  gbó  yí    thɛ̀ɛ̀  ì-chɛ̀k    ɛ́\\
    when  farming.stage  \textsc{ncp}\textsubscript{hɔ}    dry-\textsc{pst}  quite  \textsc{1pl}  burn  \textsc{ncm}\textsubscript{hɔ}{}-farm  \textsc{def}\\
    \glt ‘Once the newly cut brush is completely dry, we burn the farm.' (E10 Albert Yanker: 11)
\z
\z

In \REF{ex:120}, past \textit{{}-ɛ́} contrasts with remote past \textit{ka} used several times. Moreover, perfective \textit{ko} ‘finish' is used with the past to show the difference in functions: one is aspect, the other is tense.

\ea%120
    \label{ex:120}
    \ea Mishɔnari ka che ka, shenge ka, iko wɔ theɛ, nka shi wɔ?\\
    \gll mishɔnari  ka      che  ka    Sheŋge  ka    hi    ko    wɔ    the-ɛ\\
    missionary  \textsc{rem.pst}  be    here  Shenge  here  \textsc{1pl}  \textsc{pfv}  \textsc{3sg}  hear\textsc{{}-pst}\\
    \gll n    ka        si      wɔ\\
    \textsc{2sg}  \textsc{rem.pst}    know    \textsc{3sg}\\
    \glt ‘There used to be a missionary here, in Shenge\is{Shenge} here, we heard about him, did you know him?' (004a Cyril Manley on Walter Hanson: 61).
\z
\z

I now turn to distinctions marked with verbal particles, as laid out in \tabref{tab:verbmorph:30}.

\begin{table}
\caption{\label{tab:verbmorph:30}Verbal particles (repeated from \tabref{tab:wordcat:27})}


\begin{tabular}{lll} 
\lsptoprule
& Function & Position\\
\midrule
ka & Remote past & Pre-verbal\\
 na & Near past & Post-verbal\\
 ma & Negative, Optative & In the \textsc{aux} slot\\
ha & Optative & In the \textsc{aux} slot\\
 ni & Negative & After the tensed element\\
\lspbottomrule
\end{tabular}
\end{table}

These particles convey temporal, modal, and polarity distinctions; I begin with two and possible three tense particles.

\section{Tense particles}\label{sec:4.3}
\hypertarget{Toc115517784}{}
Tense is also marked with particles. Two and perhaps three past times are distinguished and one future. There are some complications in that perfective often entails past and imperfective is associated with future (among other things). Nonetheless, the inflections and particles can be distinguished formally and semantically.

\subsection{Near Past post-verbal \textit{na}}
\label{sec:4.3.1}
The past closest to the present is signaled by a high-toned \textit{na} (the verb has low tones), which is here referred to as the ‘near past,' as something that has taken place today. Consultants remarked that \textit{na} signaled “today past.” The particle appears immediately after the lexical verb (see \tabref{tab:verbmorph:28}).

Some examples appear in \REF{ex:121}. In \REF{ex:121a} and \REF{ex:121b}, \textit{na} is after the copula \textit{che}. In \REF{ex:121c}, curiously the particle \textit{na} does not appear immediately after the verb \textit{ken} but rather after the demonstrative \textit{ki}.

\ea%121
    \label{ex:121}
    Near past post-verbal \textit{na}\\
    \ea \label{ex:121a} Bímbí bòm kɔ̀ ché ná bóndɔ̀ kò.\\
    \gll bímbí    bòm  kɔ̀      ché  ná        bóndɔ̀  kò\\
    crowd  big  \textsc{ncp}\textsubscript{kɔ}    be    \textsc{near.pst}  wharf    at\\
    \glt ‘There was a big crowd at the wharf.' (E14 Albert Yanker: 10).

\ex \label{ex:121b} Lɛ anya ki ŋa che na boe ko, lɔ amaaɛ che na pos yekeɛ, ni ŋa theeɛ la bɛl siatiŋ dɛ theliɛ…\\
    \gll lɛ  a-nya        ki    ŋa    che  na        boo    ɛ    ko lɔ    a-maa      ɛ    che  na        pos  yeke    ɛ ni    ŋa    thee-ɛ    la      bɛl-si      a-tiŋ      ɛ    theli-ɛ\\
    if  \textsc{ncm}\textsubscript{ha}{}-people  these  \textsc{3pl}  be    \textsc{near.pst}  kitchen  \textsc{def}   to  \textsc{ncp}\textsubscript{lɔ}  \textsc{ncm}\textsubscript{ha}{}-female  \textsc{def}  \textsc{prog}  \textsc{near.pst}  peel  cassava  \textsc{def} and  \textsc{3pl}  hear-\textsc{pst}    \textsc{pro}\textsubscript{indef}  rat-\textsc{ncm}\textsubscript{si}  \textsc{ncm}\textsubscript{ha}-two  \textsc{def}  talk-\textsc{prt}\\
    \glt ‘If these people were there in the kitchen, where these women were peeling the cassava, and heard what the two rats were talking about ...' (123aw Yanker, Rat Wife: 177)

\ex \label{ex:121c} Kenki na isɔki pɔi hɔ ha bas.\\
    \gll ken    ki    na        isɔ        ki    pɛ{}-i      hɔ    ha    bas\\
    be.like  this  \textsc{near.pst}  morning    this  \textsc{pro}\textsubscript{indef}\textsc{{}-prt}  tell  \textsc{opt}  sweep\\
    \glt ‘It was like this in the morning they said to sweep.' (009--10a Lohr \& Mampa: 221)
\z
\z

The examples in \REF{ex:122} show the integration of \textit{na} within the Sherbro verbal system. In the first sentence Mabel Lohr has asked Adama Mampa if they spoke Sherbro in her home. Adama answers affirmatively. Here \textit{na} is used with the imperfective, thus confirming its compatibility with an aspect marker, as is also the case with \textit{ka}, the remote past tense. Tense is marked on the subject \textit{pɔ}, and the pronominal object \textit{ma} moves with tense to appear before the lexical verb. In \REF{ex:122b}, we see the same movement of \textit{na} from after the lexical verb to a spot close to tense, marked on the auxiliary \textit{che}.

\ea%122
    \label{ex:122}
    \ea \label{ex:122a} Aa, pɔ na ma hɔ.\\
    \gll aa    pɔ      na        ma    hɔ\\
    yes  \textsc{pro}\textsubscript{indef}  \textsc{near.pst}  \textsc{ncp}\textsubscript{ma}    speak\\
    \glt ‘Yes, they were speaking it (Sherbro) there.' (009--10a Lohr \& Mampa: 85).

    \ex \label{ex:122b} Ya chen na sɛm ka ŋán chee yaŋ kɛn.\\
    \gll ya    che-ni    na        sɛm    ka    ŋán  chee    ya-ŋ      kɛn\\
    \textsc{1sg}  \textsc{aux-neg}  \textsc{near.pst}  stand    here  \textsc{2pl}  before  \textsc{1sg-emph}  alone\\
    \glt ‘I wouldn't have been standing here before you alone.' (123aw Yanker, Rat Wife: 178)
    \z
    \z

\subsection{Remote Past \textit{ka}}
\label{sec:4.3.2}
The tense marker \textit{ká} ‘remote past' has a high tone and appears before the verb. It is shown twice in \REF{ex:123}, once before the verb \textit{mɔɛ} ‘arrive' and again before the verb \textit{ke} ‘see'. The story took place a long time ago when people could understand the speech of animals. Virtually all verbs in this story are marked in the same way.

\ea%123
    \label{ex:123}
    Remote Past pre-verbal \textit{ka}\\

    \vspace{6pt}
    
    Kaiŋ Taso ka mɔɛ tir bul, lɔ ka ke waaŋmaa kɛlɛŋkɛlɛŋ.\\
    \gll Kaiŋ Taso  ka      mɔɛ    tir      bul  lɔ    ($\emptyset$)   ka       ke    waaŋmaa  kɛlɛŋkɛlɛŋ\\
    Kain Tasso  \textsc{rem.pst}  arrive    village  one  \textsc{ncp}\textsubscript{lɔ}  (3\textsc{sg}) \textsc{rem.pst} see  woman    beautiful\\
    \glt ‘Kain Tasso reached one village, where (he) saw a beautiful young woman.' (123aw Yanker, Rat Wife: 4)
\z

Similarly, in a story of when she was a child, Mabel Lohr uses the \textsc{rem.pst} marker throughout her narrative. In \REF{ex:124}, she uses the progressive marker \textit{che} to indicate that it was done on a regular basis (translated ‘used to').

\ea%124
    \label{ex:124}
    Thetha miyɛ ka che kɔ chɛkaiɛ …\\
    \gll thetha      mi    ɛ    ka        che    kɔ    chɛk  ai    ɛ\\
    grandmother  \textsc{1sg}  \textsc{def}  \textsc{rem.pst}    \textsc{prog}    go    farm  in    \textsc{prt}\\
    \glt ‘When my grandmother used to go to the farm …' (009--10a Lohr \& Mampa: 124)
\z

The examples in \REF{ex:125} suggest an ‘anterior' meaning for \textit{ka} since the first verbs are not marked with the remote past.

\ea%125
    \label{ex:125}
    Anterior \textit{ka}\\
    \ea  Tipik lɛ ye ha bɔnthɛ, ha ka silan lɛ ha  bi ha kantha kil lɛ si mənk lɛ koŋhoni\\
    \gll tipik      lɛ    ye    ha    bɔnthɛ  ha    ka        si      la-n\\
    beginning  \textsc{def}  when  \textsc{3pl}  meet    \textsc{3pl}  \textsc{rem.pst}    know    \textsc{pro}\textsubscript{indef}\textsc{{}-emph}\\
    \gll lɛ    ha    bi    ha    kantha  kil      lɛ    si      mɛnk    lɛ    koŋ    honi\\
    that  \textsc{3pl}  have  to    close    house    \textsc{def}  before  time    \textsc{def}  finish    go.out\\
    \glt ‘At the beginning when they met up, they did not know that they had to close up the house before the time ran out.' (P67 T: 120)
\ex  A theɛ la hin pɔɛ wɔn ka kɔ.\\
    \gll a  the-ɛ      la      nyin    pɛ      ɛ    wɔn  ka        kɔ\\
    I  hear-\textsc{pst}    \textsc{pro}\textsubscript{indef}  people  \textsc{pro}\textsubscript{indef}  \textsc{def}  say  \textsc{rem.pst}    go\\
    \glt ‘I heard it (from) people that they said he had left.' (187v Wong Island: 71)
\z
\z

A possible past tense marker is \textit{pa} / \textit{palɛ}. This form has been translated as ‘formerly' and occupies a position on the past temporal continuum between \textit{na} ‘near past' and \textit{ka} ‘remote past'. It is probably semantically closer to \textit{ka} than to \textit{na} since consultants say it can refer to a time before today up to a few weeks ago. It has variants of [palɛ], [paaɛ], and even [parɛ]. It does not seem so closely integrated into the verbal complex as it is much freer in its distribution. In \REF{ex:126a}, it appears in the same slot as \textit{ka} pre-verbally (and after the auxiliary). In \REF{ex:126b}, it appears in the same slot as \textit{na}, after the verb.

\ea%126
    \label{ex:126}
    \textit{pa} ‘formerly'\\
    \ea \label{ex:126a} Woŋko yɛ ache paa kɔ Dema  koɛ, a yema lɔ kɔ fli abo abo ŋa nkuath ŋa yaŋ kɔlɔ.\\
    \gll woŋko   yɛ       a     che   pa         kɔ    Dema     ko    ɛ\\
    Wong  when    \textsc{1sg}  \textsc{prog}  formerly    go    Dema     to    \textsc{prt}\\
    \gll a    yema    lɔ      kɔ    fli\\
    \textsc{1sg}  want    \textsc{ncp}\textsubscript{lɔ}    go    really\\
    \gll a    bo    a    bo      ŋa    n-kwath      ŋa    ya-ŋ      kɔ    lɔ\\
    \textsc{1sg}  \textsc{emph}  \textsc{1sg}  \textsc{emph}    with  \textsc{ncm}\textsubscript{ma}{}-fear    for    \textsc{1sg-emph}  go    \textsc{ncp}\textsubscript{lɔ}\\
    \glt ‘When I used to go to Dema, I really wanted to go to Wong (Island), (but) I was afraid to go there.' (187v Wong Island: 2--3)

    \ex \label{ex:126b} Ŋa wɔ pa ŋa chi bɔnth, bɔnthɛo ike kɔni, nke.\\
    \gll ha    wɔ    pa        ha    chi    bɔnth\\
    \textsc{3pl}  say  formerly    \textsc{3pl}  bring    help\\
    \gll bɔnth    ɛ    o      hi    ke    kɔ    ni    n    ke\\
    help    \textsc{def}  \textsc{emph}    \textsc{1pl}  see  \textsc{ncp}\textsubscript{kɔ}  \textsc{neg}  \textsc{2sg}  see\\
    \glt ‘They previously said that they would bring help, (but) we have not seen help, you see.' (015a Adama Mampa, Bondo: 11)
\z
\z

Another past temporal adverb is \textit{vɛthiɛ} ‘some time ago,' which comes after the object of the verb but does not seem part of the TMA system, even less so than \textit{pa}. The future marker \textit{ki} is more so.

\subsection{Future \textit{ki}}
\label{sec:4.3.3}
The ‘future' can include the present as well as the future and is marked by the particle \textit{ki} immediately before the verb, as in \REF{ex:127}.

\ea%127
    \label{ex:127}
    Future particle \textit{ki}\\
\ea  Apim ŋamu ki kaŋ, apim ŋa cheni mu ki kaŋ.\\
    \gll a-pim        ŋa    mu  ki    kaŋ    a-pim        ŋa    che-ni    mu  ki    kaŋ\\
    \textsc{ncm}\textsubscript{ha}{}-some  \textsc{3pl}  even  \textsc{fut}   study    \textsc{ncm}\textsubscript{ha}{}-some  \textsc{3pl}  \textsc{aux}{}-\textsc{neg}  even  \textsc{fut}  study\\
    \glt ‘Some will be studying, while others will not be studying.' (090a Saidu Netteh: 69)

\ex  Ya ki hundɛ, pɔ mi buŋ.\\
    \gll ya    ki    hun    lɛ    pɔ      mi    buŋ\\
    \textsc{1sg}  \textsc{fut}  come    after  \textsc{pro}\textsubscript{indef}  \textsc{1sg}  flog\\
    \glt ‘When I come back, they (will) flog me.' (009--10a Lohr \& Mampa: 224.2)
\z
\z

The lexical verb \textit{hun} ‘come' can also precede a verb and indicate a future action or event, usually something of certainty, as in \REF{ex:128}. It thus has an epistemic component to it, functioning as something of a modal.

\ea%128
    \label{ex:128}
    \textit{Hun} ‘come' as incipient
    \ea Ahun yi nɔmaɛ ki ŋa lemɛ mi jali wɔ atokɛ ...\\
    \gll a    hun    yi    nɔmaa  ɛ    ki    ŋa  lemɛ    mi    ja        li-wɔ      atok  ɛ\\
    \textsc{1sg}  \textsc{incip}    ask  woman  \textsc{def}  this  to  explain  \textsc{1sg}  something  \textsc{ncm}\textsubscript{lɔ}{}-\textsc{3sg}  about  \textsc{prt}\\
    \glt ‘I am coming to ask this woman about herself.' (007a Agnes J. Simbo: 2)

\ex ... ni chii chelɛ ya hun sɔthɔ yen ha sɔm, ndikɛ koŋ mi gbɔɔ!\\
    \gll ni    chi    chelɛ    ya    hun    sɔthɔ    yen      ha    sɔm\\
    and  bring    so.that  \textsc{1sg}  come    get    something  for    eat\\
    \gll n-dik          ɛ    koŋ    mi    gbɔ\\
    \textsc{ncm}\textsubscript{ma}{}-hunger    \textsc{def}  finish    \textsc{1sg}  seriously\\
    \glt ‘... and bring it so that I can come eat something, hunger is consuming me!' (123aw Yanker, Rat Wife: 65)
\z
\z

A verb that is like \textit{hun} in designating a probable event is \textit{yema} ‘want,' but with less of the certainty involved in \textit{hun}.

\section{Mood: Optative} \label{sec:4.4}
\hypertarget{Toc115517785}{}
In addition to tense and aspect, mood and polarity can also be marked on verbs. What is here called ‘optative' is the major mood distinction found in Sherbro.\footnote{I choose this term following the practice of the Sherbro Literacy Committee\is{Sherbro Literacy Committee}. “Hortative” is a term used in the discussion of other closely related languages (e.g., \citealt{Childs1995}).} The usual translation of the optative is ‘allow, let' or even ‘should, must'. The example in \REF{ex:129} shows some examples of the optative (affirmative and negative) with the perfective in \REF{ex:129a}. Note that the third-person pronoun ‘he, him' is usually not expressed in such constructions; in all cases it would be just before the verb, as I have indicated in \REF{ex:129a} (see footnote \ref{fn:65}). The second person singular also requires no pronoun.


\ea%129 
\label{ex:129}
    \ea \label{ex:129a} Perfective\\
    (Wɔ̀) kɔ́.\\
    \glt ‘(He) is gone.'\\

    \ex \label{ex:129b} Optative\\
    Há kɔ̀.\\
    \glt ‘Let him go.'\\
\vspace{6pt}
   Á kɔ̀.
    \glt ‘Let me go.'\\
    
    \ex \label{ex:129c}Negative Optative\\
    Mà kɔ́.\\
    \glt ‘Let him not go.'\\
    \vspace{6pt}
    À mà kɔ́.
    \glt ‘Let me not go.'

\ex \label{ex:129d}Mɔ gbo chɔ pu konthoɛ, ha ni pothɛ kɔ kek mɔni.\\
    \gll mɔ  gbo  chɔ  pu    kontho      ɛ    ha    ni    poth  ɛ    kɔ      kek  mɔ  ni\\
    \textsc{2sg}  just  fight  fish  mudskipper    \textsc{def}  let    then  mud  \textsc{def}  \textsc{ncp}\textsubscript{kɔ}    see  \textsc{2sg}  on\\
    \glt ‘If you fight with the mudskipper, then let the mud be seen on you.' (Proverbs: 145)

\ex Yɛ Kaiŋ Taso ka koŋ ŋɔ mɔɛ ha bi nɔmaaɛ.\\
    \gll yɛ      Kaiŋ  Taso    ka    koŋ  ŋɔ      mɔɛ    ha    bi    nɔmaa  ɛ\\
    when    Kain  Tasso    \textsc{past}  \textsc{pfv}  \textsc{ncp}\textsubscript{hɔ}    arrive    \textsc{opt}  have  woman  \textsc{prt}\\
    \glt ‘When Kain Tasso arrived, (they) let him take a wife.' (123aw Yanker, Rat Wife: 2)
\z
\z

A lexical verb that has a similar modal value is \textit{bia} ‘have,' as in ‘have to, should,' as shown in \REF{ex:130}.

\ea%130
    \label{ex:130}
    \ea Labo thibɔm lɔ pɔ bia yukɛ ...\\
    \gll lagbo    thi-bɔm      lɔ    pɛ      bia    yuk-ɛ\\
    if      \textsc{ncm}\textsubscript{tha}{}-mud    there  \textsc{pro}\textsubscript{indef}  have.to  plant-\textsc{prt}\\
    \glt ‘If they had to plant on the mud ...' (006v Abdulai Bendu, Rice Growing: 21)

\ex  Abɛnai bɛ ramdɛ kɔ bia che mɛmiɛ ni, haaŋ ni pɔ hokɔ sakaɛ.\\
    \gll a-bɛn        a-i        bɛ    ram    ɛ    kɔ      biya    che  mɛmiɛ  ni\\
    \textsc{ncm}\textsubscript{ha}{}-parent  \textsc{ncm}\textsubscript{ha}{}-and  self  family  \textsc{def}  \textsc{pro}\textsubscript{indef}  have.to  be    happy  now\\
    \gll haaŋ      ni      pɛ      hokɔ      saaka      ɛ\\
    long.time  when    \textsc{pro}\textsubscript{indef}  take.out    sacrifice    \textsc{def}\\
    \glt ‘The elders themselves, the family, it has to be agreeable up to when they make the sacrifice.' (016a Albert Yanker: 147)
\z
\z

\section{Negation}
\hypertarget{Toc115517786}{}\label{sec:4.5}
Negation is not so much a morphological process as a morphosyntactic one. There are two negative markers \textit{ni} and \textit{ma}, which can be considered part of the morphology (see \sectref{sec:8.2.2} for another). Of the two, the general negator \textit{ni} is linked to tense and undergoes various morphophonological changes in its different positions, invariably, however, associated with a high tone. There is occasionally no segmental material to betray its presence. The optative negator \textit{ma} appears only in this mood and undergoes no changes, though it, too, is associated with tense. The negator \textit{be} has no such association.

In its fullest form the general negator is realized as [ní] (with a high tone) or [ɛ́n] but can be reduced to a simple nasal or just a high tone. Some examples follow in \REF{ex:131}.

\ea%131
    \label{ex:131}
    Negation with \textit{ni} and its phonetic variants\\
    \ea Koŋ wonkru ichɛk wɔ lɛ, hɔ ka heyɛni.\\
    \gll ($\emptyset$)  koŋ    woŋkru    i-chɛk      wɔ    lɛ    hɔ    ka      heiɛ  ni\\
    \textsc{(3sg)}  \textsc{prf}    clear.farm  \textsc{ncm}\textsubscript{hɔ}{}-farm    \textsc{3sg}  \textsc{def}  \textsc{ncp}\textsubscript{hɔ}  \textsc{rem.pst}  burn  \textsc{neg}\\
    \glt ‘He has finished clearing his farm that was never well burnt.' (P67 W: 52)

    \newpage
    \ex A tipɛ lɔ kɔ, kɛ a kɔni livil.\\
    \gll a    tipɛ    lɔ    kɔ    kɛ    a    kɔ    ni    li-vil\\
    \textsc{1sg}  begin    there  go    but  \textsc{1sg}  go    \textsc{neg}  \textsc{ncm}\textsubscript{lɔ}{}-far\\
    \glt ‘I started to go there (school), but I did not go far' (028a Yusuf Fofana: 65)\\

    \ex A che ŋɔ ni pɛ  lonibo lɛ, bikɔs pɔ chiɛmi ka yaŋ taa.\\
    \gll a    che  ŋɔ      ni    pɛ        lonibo    lɛ\\
    \textsc{1sg}  be    \textsc{ncp}\textsubscript{hɔ}    \textsc{neg}  anymore    remember  that\\
    \gll bikɔs    pɛ      chiɛ    mi    ka      ya-ŋ      taa\\
    because  \textsc{pro}\textsubscript{indef}  bring    \textsc{1sg}  here    \textsc{1sg-emph}  child\\
    \glt ‘I cannot remember it, because they brought me here when I was young.' (005a Jalikatu B. Kumba: 82)
    
    \ex Kase che wɔn.\\
    \gll kase    che  wɔ{}-ni\\
    blame  be    \textsc{3sg-neg}\\
    \glt ‘He is blameless.'

    \ex Ya biɛn fɔsa.\\
    \gll ya    bi-ni      fɔsa\\
    \textsc{1sg}  have-\textsc{neg}  strength\\
    \glt ‘I have no strength.'
\z
\z

There are many examples of the auxiliary \textit{che} alone signaling negation as seen in \REF{ex:132}. (Note \textit{che} is with a high tone although it is not transcribed in the discussion.)

\ea%132
    \label{ex:132}
    \ea Hɔbatok che rubani. Kagbɔai chɔygba.\\
    \gll hɔbatok  che      rubani    kagbɔ-ai      chɔygba\\
    God    \textsc{aux.neg}    bless      Kagboro-in    forever\\
    \glt ‘God does not bless Himself. Kagboro forever.' (188 Kagboro anthem: 8)

    \ex Pɛlɛ kɔ che yeɡbe ka hi fi, ken bɛl pothoɛ ki...\\
    \gll pɛlɛ  kɔ      che      yeɡbe    ka    hi    sui    ken  bɛlpotho ɛ     ki\\
    rice  \textsc{ncp}\textsubscript{kɔ}    \textsc{aux.neg}    well    here  \textsc{1pl}  hand  like  coconut  \textsc{def}  this\\
    \glt ‘Rice does not grow well in our hands, like coconut ...' (102v Chernor Ashun: 160)

    \newpage
    \ex A shi ŋɔth kɛ a che kɔ hɛlɛ.\\
    \gll a    si      ŋɔth    kɛ    a    che      kɔ    hɛlɛ\\
    \textsc{1sg}  know    fishing  but  \textsc{1sg}  \textsc{aux.neg}   go    sea\\
    \glt ‘I know how to fish but I don't go out on the sea.' (004a Cyril Manley on Walter Hanson: 55)
\z
\z

There is no identifiable negative in the first clause of \REF{ex:133}, yet \textsc{neg} is present in the second.

\ea%133
    \label{ex:133}
    Bɛ yɛ motoɛ chelɔ bɔ  kɔɛ, ŋa koŋ wɔ ŋa cheŋ bɔ kɔ ...\\
    \gll kɛ    yɛ    moto  ɛ    che      lɔ    bɔ      kɔ-ɛ\\
    but  since  car  \textsc{def}  \textsc{aux.neg}    there  be.able  go-\textsc{prt}\\
    \gll ŋa    koŋ    wɔ    ha    che  ni    bɔ      kɔ\\
    \textsc{3pl}\textsubscript{ha}  finish    say  \textsc{3pl}\textsubscript{ha}  \textsc{aux}  \textsc{neg}  be.able  go\\
    \glt ‘But since vehicles are not able to go there, they have said they are not able to come.' (018a Suffian Koroma: 94)
\z

Negation is mostly morphological but also syntactic, as shown in \REF{ex:134}, sensitive to the \textsc{tns-pro} constituent. In the first clause \textit{ni} is directly after the verb, but in the second clause it comes after the object pronoun, which forms a unit with \textsc{tns}.

\ea%134
    \label{ex:134}
    Aa, a shini ŋɔth gbi, a shi ŋɔ ni gbi.\\
    \gll aa    a    si      ni    ŋɔth    gbi  a    si      ŋɔ      ni    gbi\\
    yes  \textsc{1sg}  know    \textsc{neg}  fishing  all    \textsc{1sg}  know    \textsc{ncp}\textsubscript{hɔ}    \textsc{neg}  all\\
    \glt ‘Right, I don't know fishing at all, I don't know it at all.' (090a Saidu Netteh: 65)
\z

This is another significant fact relevant to the arguments for a tense-object pronoun syntagm, discussed in some detail in \sectref{sec:8.2.3}.

I now turn to the noun class system of Sherbro, a feature common to Bolom-Kisi, Mel\il{Mel}, and Niger-Congo (e.g., \citealt{Childs2003b}).

