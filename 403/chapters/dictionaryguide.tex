\chapter{Dictionary}
\label{app:a}
\section{Guide to the dictionary}

The dictionary is the product of the data documented in the Fieldworks Language Explorer (FLEx) database, which is included in the archive Documenting the Sherbro Language and Culture of Sierra Leone at the Endangered Language Archive (ELAR) (\url{https://www.elararchive.org/dk0373}). The corresponding audio and video recordings are also archived by ELAR, as well as in the project’s Portland State University Archive (\url{https://pdxscholar.library.pdx.edu/sherbro/}). 

These data include transcriptions of 28 interviews, the performances of 20 hymns, three stories, an anthem, and 15 elicitation sessions (see the appendixes). No documentation of this breadth or depth has ever been done for this language. This alone would make the database and dictionary invaluable.

But critically the database Childs produced is more. It is an archive of everything. So it also includes all of the vocabulary and stories from the Reverend A. T. Sumner’s 1921 \textit{Handbook of the Sherbro Language}, Walter J Pichl’s 1967 150-page \textit{Sherbro-English dictionary}, and a collection of 175 proverbs, \textit{Nthaɛ maMbolomdɛ ‘Bolom Proverbs,}’ compiled in 1979 by the Institute for Sierra Leonean Languages (Lutheran Bible Translators) (see Appendix \ref{app:c} for an excerpt). 

Sherbro is understudied and endangered. So this database is genuinely an archive of the data for this language. Having an integrated set of materials provides a more profound appreciation for what these older documents can offer (as well as their limits). Having everything integrated into one project allows for corroboration and the filling out of paradigms that just isn’t possible for many endangered languages. 

Childs’ work on the database was complete for the production of a dictionary of nearly 4100 entries. But his preparation of the dictionary manuscript was cut short by his untimely death in 2021. 

There were some challenges presenting a version of the dictionary that reflected the data. There were some quirks of output that produced a couple hundred headwords without any entry or only a partial entry of related morphological forms. Somewhat ironically this version of the output left the dictionary’s very first entry for \textbf{a\textsubscript{1}} entirely blank and then immediately followed by the related form  \textbf{yɛkia} with its gloss ‘What is this?’ 

Additionally, the output did not provide a ready way to distinguish between sources in either the entries or among examples so additional formatting and labels were added.  


\subsection{Roadmap}
The definition section of an entry may include up to four additional entries: two from elicitation and two from Pichl and Sumner. Each of these four sources is in a separate field and identified with a link to its source in the database. However, when they were incorporated into the dictionary, their source labels did not accompany them and no particular punctuation separated them. So I separate them with a semicolon and have added labels to distinguish them: (K dialect), (B dialect), (\citealt{Pichl1967}), (\citealt{Sumner1921}). Most entries do not contain multiple sources, but here is one example with three. 

\begin{quote}
    \textbf{daŋkɔ} \textit{cf:} \textbf{kuai}. \textit{n} palm kernel oil\\
from the palm nut; [ǹdàŋgò] palm\\
nut oil (B dialect); \textit{ndankɔ} oil from\\
palm kernel (Sumner 1921); \textit{ndankɔ}\\
(ma) palm oil (Pichl 1967).\\
\end{quote}

The layout also distinguishes between elicitations and published sources by putting elicitation data in square brackets and forms from published sources in italic. Note also the (ma) in parentheses is Pichl’s indication of an associated noun class pronoun. Most of his entries include noun class information which he designated with roman numerals. Childs translated these into their corresponding pronouns and these are followed by the plural affix when relevant. 

Nearly 1000 of the 4095 dictionary entries have accompanying elicitation data from two consultants: Albert Yanker and Abdulai Bendu. Albert Yanker (1934–2021) was a Sherbro author and head of the Sherbro Literacy Committee. He was well-known in Shenge as a passionate advocate for the language, and when people were asked who spoke the best Sherbro in Shenge the answer was invariably Albert Yanker. Two of his stories are part of the project’s archive: ‘Kain Tasso’ (Appendix \ref{app:d}) and ‘Boy Lost at Sea.’ He was born in Mofos, Kagboro Chiefdom and lived in Shenge, headquarters of Kagboro Chiefdom, and Freetown, Sierra Leone’s capital, when he was younger. With respect to a formal education, he reached standard six, but once in Freetown, independently pursued educational opportunities to great success. At the time of the elicitation sessions, he had long been based in Shenge and was in his early 80s. He was an L1 speaker of Sherbro, Themne, and Krio, and fluent L2 speaker of English.

The second contributor to the elicitation sessions is Abdulai Bendu, who was born in Moyeamoh, Bumpeh Chiefdom. He spent many early years in Freetown, Sierra Leone’s capital, but after the war, regularly spent time in Moyeamoh and Rotifunk, headquarters for Bumpeh Chiefdom. Bendu is Sherbro, an L1 speaker of Sherbro, Themne, and Krio, and fluent L2 speaker of English. During the project, he was in his 20s and also the project’s chief research assistant. He was instrumental in documentation activities involving both larger groups and one-on-one interviews, an informative guide to cultural practice, and critical aid to transcription and translation. He has a BA (2022) in linguistics from Fourah Bay College, University of Sierra Leone and is pursuing an MPhil in linguistics at the same university. 

Although these two consultants have different biographies and made different contributions to the SLC project, for the purposes of the dictionary, I have labeled their contributions as (K dialect) and (B dialect) based on their being speakers of the Kagboro Chiefdom and Bumpeh Chiefdom dialects respectively. Childs and others have identified different varieties based on these boundaries, and I wanted it to be clear that differences between them were likely rooted in these differences rather than a function of idiolectal variation. Although Childs was eager to choose consultants with different biographies along several dimensions, he did not use these labels in the dictionary. I made that call, based on his research, to clarify what the dictionary makes available. 

Also importantly, my use of dialect labels should not be construed as an indication these forms are unique to those dialects in the way “dialectal” is used with more robustly documented languages. So in most cases whatever form appears in an entry probably has the same form in most other dialects as well. For the two previously published contributions, I have added citations. 

\subsection{Tone}
The L1 Sherbro speakers of the Sherbro Literacy Committee did not recommend marking tone in the orthography (see \sectref{sec:1.9} Orthography and conventions). But for the analysis of the language discussed in the grammar, Childs posits a two tone system but felt he was unable to reliably establish lexical tone unlike other more vital languages of the Bolom-Kisi group like Mani and Kisi. There are approximately 400 entries or just 10\% where he felt confident enough to assign tone, so only these forms appear in square brackets directly following the headword. 

\begin{quote}
    \textbf{baŋ\textsubscript{3}} [bàŋ] n bird species, yel-\\
low and black weaver bird, also\\
called palm bird (K dialect); (wɔ/hã)\\
compact weaver bird, small red-\\
dish or yellowish (Pachyphantes\\
pachyrrhynchus), if sits on the\\
branches near the house, he brings\\
good luck, if he flies away, it means\\
bad luck (Pichl 1967). \textit{Mbàŋsɛ̀ ŋà rɪ́k}\\
\textit{wàɛ̀ tòkɛ̀.} The weaver birds wove\\
(their nests) at the top of the palm\\
tree.
\end{quote}

For other tonal data, they appear after the grammatical designation as in the \textit{daŋkɔ} example above as part of a sub entry for either of the elicitations or as part of an example sentence. 

Pichl’s tone data involves 5 levels of tone along with 5 levels of stress written with capital letters using underlines for stress as well as two more lower case letters to indicate less perceptible syllables. This sort of rich detail was certainly warranted since his was the first to attempt at documenting the language. His data merits review. But because there are so many distinctions, because the underlines designating stress frequently did not survive the scanning process, and proofreading these would have resulted in further delays, I chose to leave it out. It does remain in the database, and of course is still in \citet{Pichl1967}. 

 \subsection{Organization}
To the extent possible the dictionary is organized around monomorphemic forms with related morphologically complex forms appearing as subentries. Whenever a subentry appears out of alphabetical order, there is a stub entry in alphabetical order directing you to the full entry. For example, \textbf{taive} ‘bird nest’ is listed under its headword \textbf{tai} ‘nest’: 
\begin{quote}
    \TabPositions{1cm,2cm,5cm,7cm}
\vspace{6pt}
\small \textbf{taive} (comp. of \textbf{tai\textsubscript{1}}, \textbf{vee\textsubscript{1}}, see \textbf{tai\textsubscript{1}})
\end{quote}

Possible morphological relationships are abbreviated as follows:

\begin{quote}
    \begin{tabular}{ll}
    \small comp. & compound\\
der. & derivative\\
id. & idiom\\
unspec. & unspecified complex forms\\
\end{tabular}
\end{quote}

Usually there are just one or two subentries but in case of multiples, they are organized by type. So morphologically productive forms like \textit{bol} ‘head’ and \textit{nɔ} ‘person’ have subentries that are compounds listed in alphabetical order and they are followed by all derivatives, then idioms, and finally unspecified complex forms. 

Whenever there are multiple levels of morphological complexity within a set of subentries, the intermediate entry has a link to the more morphologically complex form, and that subentry in turn notes its multilevel relationship to the head entry. This inter-indexing among the subentries is the way multiple levels of embedded morphological processes are indicated since formatting does not easily allow for their being subentries of subentries. 

\subsection{Orthography}
The orthography does not mark tone, but otherwise spelling incorporates pronunciation rather than being strictly phonemic. G is only phonemic in borrowings,  sh [ʃ] appears in borrowings or as an allophone of /s/, and v is an allophone of /w/, but because the orthography incorporates them there is a subsection for each; however, the section heads appear in parentheses to mark their non-phonemic status. See below for the complete list of orthographic conventions and corresponding IPA symbols. 

The three special characters ɛ, ɔ, and ŋ are alphabetized following their Latin alphabet counterparts. There are five digraphs: ch [tʃ], gb [g͡b], ny [ɲ], sh [ʃ], and th [t̪]. Because these digraphs all represent single phonemes, they have their own sections which are ordered after the first letter of the digraph; however, the alphabetical order within each section was determined by individual letters.

\subsection{Interventions}
I used whatever I could find in the database to fill in what was missing in the dictionary output. When possible I also used the database to answer whether names were for people or places. Whatever additions of labels, citations, and punctuation has been in an effort to present what was in the database in a more accessible way, and in those efforts I am particularly grateful to Abdulai Bendu, the project’s head research assistant in Sierra Leone, for being accessible to me via WhatsApp in answering questions big and small. Unfortunately, inevitably these interventions have created many more opportunities for introducing error, and that is not the point of using a database. A more perfect solution would reproduce this material directly from the database in a layout that can accommodate it. But we did not want to wait in presenting such an important contribution to future scholarship.  




\begin{flushright}
    Chris Corcoran, 2025
\end{flushright}



%\section*{Abbreviations used in the dictionary}

\subsection{Dictionary orthography and corresponding IPA symbols}

\begin{tabularx}{\textwidth}{llllllll}
a & [a] & h & [h] & o & [o] & (v) & [v]\\
b & [b] & i & [i] ([ɨ], [ɪ]) & ɔ & [ɔ] & w & [w]\\
ch & [tʃ] & j & [ʤ] & p & [p] & y & [j]\\
d & [d] & k & [k] & r & [r] & z & [z]\\
e & [e] ([ɪ], [ə]) & l & [l] & s & [s] &  & \\
{}ɛ & [ɛ] ([ə]) & m & [m] & (sh) & [ʃ] &  ́  & high tone\\
f & [f] & n & [n] & t & [t] &  ̀  & low tone\\
(g) & [g] & ny & [ɲ] & th & [t̪] &  & \\
gb & [g͡b] & ŋ & [ŋ] & u & [u] &  & \\
\end{tabularx}
\medskip

% \begin{quote}
\begin{flushleft}
\citet{Pichl1967} Data 
\end{flushleft}
\begin{tabularx}{.7\textwidth}{llll}
i & (ï) &  & u\\
e & (e̹) & (o̹) & o\\
ɛ & (ə) &  & ɔ\\
æ &  & & a\\
\end{tabularx}

% \end{quote}
