\chapter{Beyond the clause}
\label{ch:9}\hypertarget{Toc115517819}{}
This chapter treats sequences forming a single utterance consisting of links outside the sentence or of what may be analyzed as more than one clause. In the case of information structure, elements in a sentence are identical to ones prominent in the preceding discourse.

\section{Information structure}
\label{sec:9.1}\hypertarget{Toc115517820}{}
Though not systematically investigated, a hypothesis is that information status determines syntactic structure in Sherbro. The comments below are unfortunately not based on a systematic investigation; nonetheless, it is hoped that they will be helpful to other researchers.

The basic structure is topic-comment. The first part of a sentence is taken up with stated, previously established information, referred to usually with a pronoun (or two) in the sentence that follows. Striking is the density of links by means of anaphoric expressions.

In \REF{ex:224}, the topic-comment structure is clear. The Island of Egusi has already been referred to in the previous discourse (see footnote 13 in \sectref{sec:3.6}). Here the full NP is introduced as the continued topic, then followed by the subject (\textit{yel} ‘island'), with the same referent, followed by two demonstratives. The new information is that, many years ago, the island was much bigger, large enough to support farms including the cultivation of egusi.

\ea%224
    \label{ex:224}

    Yel Nsaŋha ko, yel lo kinɔ ka che bomba nɛn thigber tha koŋ chaŋ dɛ.\\
    \gll yel    n-saŋha        ko    yel    lo    kinɔ\\
    island    \textsc{ncm}\textsubscript{ma}{}-egusi    to    island    this  this\\
    \gll ka      che  bomba    nɛn  thi-gber      tha    koŋ  chaŋ-ɛ\\
    \textsc{rem.pst}  be    very.large  year  \textsc{ncm}\textsubscript{tha}{}-many  \textsc{ncp}\textsubscript{tha}    \textsc{pfv}  pass-\textsc{prt}\\
    \glt ‘The Island of Egusi, this island was very big many years ago.' (124aw Yanker, Boy Lost at Sea 12)
\z

In a briefer example \REF{ex:225}, it is the object of the following sentence that is the topic, a group of people. The people being spoken about are outsiders who violated a law of Wong Island, a sacred place to the people of Dema\is{Dema!} Chiefdom. The nominal element fronted, any argument, here the object, is recapitulated in the main clause. There is no gap, as there would be in a clefted construction, as found elsewhere in Bolom (\citealt{Childs1997}).

\ea%225
    \label{ex:225}

    Ŋan ɡbi, pɔ yɛthiɛ   ŋa.\\
    \gll ŋa-n      ɡbi  pɛ      yɛthiɛ    ŋa\\
    \textsc{3pl-emph}  all    \textsc{pro}\textsubscript{indef}  hold    \textsc{3pl}\\
    \glt ‘All of them, the people held them.' (187v Wong Island: 63)
\z

The topic can be an adpositional phrase (\textit{bɔɔko} is a postposition) as in \REF{ex:226}.

\ea%226
    \label{ex:226}
    Killɛ bɔɔko gaadin hɔ lɔ.\\
    \gll kil      ɛ    bɔɔko    gaadin  hɔ      lɔ\\
    house    \textsc{def}  outside  garden  \textsc{ncp}\textsubscript{hɔ}    be\textsubscript{loc}\\
    \glt ‘There is a garden outside the house.' (P67 B: 157)
\z

The topic can form part of a full sentence, as in \REF{ex:227}.

\ea%227
    \label{ex:227}
    Ya bi nrɔm ka, ma mɔ bɔ ramir.\\
    \gll ya    bi    n-rɔm          ka    ma    mɔ  bɔ      ramir\\
    \textsc{1sg}  have  \textsc{ncm}\textsubscript{ma}{}-medicine  here  \textsc{ncp}\textsubscript{ma}    \textsc{2sg}  be.able  cure\\
    \glt ‘‎I have a medicine here, it can cure you.' (P67 B: 151)
\z

The following utterance is part of a story involving much connectedness between its parts, as new information is introduced and then becomes a topic in the following sentence. The story is one of an orphaned daughter, who was taken into a family only to become something of a slave to her adoptive family. Her deceased mother comes back to her in a dream and then later as a tree by a stream which provides her with food, which her new family had denied her. Before the following utterance, the stream was introduced and linked to \REF{ex:228} by the deictic \textit{lɔ} ‘there'. The tree is new information but now the topic, and action has been prescribed by her mother in the dream.

\ea%228
    \label{ex:228}
    Thɔk bomdɛ kɔ lɔ vɛ ni che lɔ kɔ tɔn.\\
    \gll thɔk  bom  lɛ    kɔ    lɔ    vɛ  ni    che  lɔ    kɔ     tɔn\\
    tree  big  \textsc{def}  \textsc{ncp}\textsubscript{kɔ}  there  so  and  \textsc{aux}  there  go   sing\\
    \glt ‘The big tree that is there (by the stream mentioned in the previous utterance), she should go there and sing.' (122a Virginia Lohr: 9)
\z

\section{Comparison}
\label{sec:9.2}\hypertarget{Toc115517821}{}
For simple non-degree comparison \textit{kendɛ} ‘be like, be similar to' is used. A related form \textit{ken} can be used in the same way, and both forms can be used as prepositions to mean ‘like'. (Example \REF{ex:229a} is repeated from \REF{ex:222} in \sectref{sec:8.3}.)

\ea%229
    \label{ex:229}
    \ea\label{ex:229a}  Bulɔ kendɛ handɔ?\\
    \gll bulɔ  kendɛ      handɔ\\
    work  similar.to  what\\
    \glt ‘What kind of work?' (lit. ‘Work similar to what?') (028a Yusuf Fofana: 17)

    \ex\label{ex:229b}  Tamɔ lɛ wɔ dwiye ken top.\\
    \gll tamɔ    lɛ    wɔ    dwiye  ken    top\\
    boy    \textsc{def}  \textsc{3sg}  steal    like    groundhog\\
    \glt ‘The boy is stealing like a groundhog.'\footnotemark {}  (P67 T: 114)

    \ex\label{ex:229c}  Ba Na ka che ayeŋ ha bom kendɛ nvis ha hallɛ, kɛ gbɔlkajo ŋɔ siŋ ka wɔ ayeŋ vɛ.\\
    \gll ba      na    ka      che  ayeŋ    ha    bom  kendɛ    n-vis        ha    halɛ\\
    Mister  spider \textsc{rem.pst}  be    middle  do    big  same    \textsc{ncm}\textsubscript{ma}{}-animal  \textsc{3pl}  other\\
    \gll kɛ    gbɔlkajo    hɔ      siŋ    ka    wɔ    ayeŋ    vɛ\\
    but  gluttony    \textsc{ncp}\textsubscript{hɔ}    play  with  \textsc{3sg}  middle  so\\
    \glt ‘Ba Spider formerly had a big waist equivalent to the other animals, but gluttony played with his middle.' \citep[32]{Sumner1921}
\z
\z
\footnotetext{The groundhog is well known as a problem raider of people's gardens.}

To convey degrees of comparison the verb \textit{chaŋ} ‘pass, surpass' is used. What can vary syntactically is the location of the standard. In \REF{ex:230}, the standard is expressed which is about ‘recruiting girls for the initiation society.'

\ea%230
    \label{ex:230}
    A is X (standard), A surpasses B\\
\vspace{6pt}
    Lagbo ja Bondoɛ la ko che kath ŋa dikil apimaɛ, la chaŋ kacheɛ?\\
    \gll lagbo    ja      bondo    ɛ    la      ko    che  kath  ŋa    dikil    a-pum    a-      ɛ  la      chaŋ    kache      ɛ\\
    whether  matter  bondo  \textsc{def}  \textsc{pro}\textsubscript{indef}  \textsc{pfv}  be    hard to    recruit  \textsc{ncm}\textsubscript{ha}{}-children  \textsc{ncm}\textsubscript{ha}  \textsc{def} \textsc{pro}\textsubscript{indef}  surpass  formerly    \textsc{def}\\
    \glt ‘Has it become harder to recruit girls for Bondo\footnotemark{} than in the old days?' (015a Adama Mampa, Bondo: 3)
    
    \footnotetext{Bondo\is{Bondo} is the girls' initiation society. The speaker is one of the Bondo leaders.}
\z

In \REF{ex:231}, a question \REF{ex:231a} is followed by an answer \REF{ex:231b}, the standard (‘knowing how to speak Sherbro') follows the expression of superiority, ‘surpass.'

\ea%231
    \label{ex:231}
    A passes B Standard (with regard to speaking Sherbro)\\
    \ea\label{ex:231a} Ahina ŋa chan si theli Mbolom dɛ Shenge  ka.\\
    \gll a-hina      ŋa    chaŋ    si      theli    n-bolom        lɛ    Shenge   ka\\
    \textsc{ncm}\textsubscript{ha}{}-who    \textsc{3pl}  surpass  know    speak    \textsc{ncm}\textsubscript{ma}{}-Sherbro  \textsc{def}  Shenge   in\\
    \glt ‘Who speaks Sherbro best in Shenge\is{Shenge}?' (009--10a Lohr \& Mampa: 100)

    \ex\label{ex:231b} Ba Yanka wɔ chaŋ shi theli Mbolomdɛ, wɔ kiban dɛ, wɔ chaŋ si theli mbolomdɛ.\\
    \gll ba      yanka    wɔ    chaŋ    si    theli    n-bolom      lɛ    wɔ    kiban    lɛ\\
    father    Yanker  \textsc{3sg}  surpass  know  speak    \textsc{ncm}\textsubscript{ma}{}-Bolom  \textsc{def}  3\textsc{sg}  expert  \textsc{def}\\
    \gll wɔ    chaŋ    si      theli    n-bolom        lɛ\\
    \textsc{3sg}  surpass  know    speak    \textsc{ncm}\textsubscript{ma}{}-Bolom    \textsc{def}\\
    \glt ‘Ba Yanker knows how to speak Sherbro the best; he is the expert that knows how to speak Sherbro better (than anyone).' (009--10a Lohr \& Mampa: 100-101)
\z
\z

\section{Subordinate clauses}
\label{sec:9.3}\hypertarget{Toc115517822}{}
This section considers subordinate clauses of all types.\footnote{I use the terms “embedded” and “subordinate” interchangeably.} (Some examples of coordinate clauses also appear in \sectref{sec:3.11}). This section first looks at embedded clauses with no overt conjunction, and then at subordinate clauses with subordinating conjunctions. Relative clauses are discussed in the last part of this section.

\ea%232
    \label{ex:232}
    Embedded clauses with no conjunction\\
    \ea Nkela bo la li kɛlɛŋ, ala bɔ yema.\\
    \gll n    ke    la  bo      la        li-kɛlɛŋ    a    la  bɔ      yema\\
    \textsc{2sg}  see  it  \textsc{emph}    something  \textsc{ncp}\textsubscript{lɔ}{}-good  \textsc{1sg}  it  be.able  want\\
    \glt ‘If you see it to be something indeed good, I can agree to it.' (002a Mabel Lohr, Midwifery: 9)

    \ex Labila awɔ ŋa bia kɔlɔ gbɛ, mɔi ke ...\\
    \gll labila      a    wɔ    ŋa    biya    kɔ    lɔ    gbɛ    mɔ-i    ke\\
    therefore    1\textsc{sg}  say  2\textsc{pl}  have.to  go    there  walk    2\textsc{sg-prt}  see\\
    \glt ‘That is why I said you need to go take a walk there, and you'll see ...' (009--10a Lohr \& Mampa: 264)
\z
\z

Embedded clauses without a conjunction show case where it exists. There are only pronouns and just a few of those mark case. The form of the \textsc{1sg} pronoun in subject position is \textit{ya} or \textit{a}; the objective and genitive form is \textit{mi}, as it is in the embedded clause in \REF{ex:233}.

\ea%233
    \label{ex:233}
    Nyema bi mi pinki lɛ wonɔ.\\
    \gll n    yema    bi    mi    pinki      lɛ    wonɔ\\
    \textsc{2sg}  want    have  \textsc{1sg}  transform  be    slave\\
    \glt ‘You want to make me into a slave.' (P67 W: 63)
\z

There is something of an analytical problem with aux-like verbs, e.g., \textit{koŋ} ‘finish', \textit{kɔ} ‘go,' because they behave just like the auxiliary \textit{che}. The problem was first mentioned in \sectref{sec:3.9} (see \REF{ex:93} in \sectref{sec:3.9}). I give some examples in \REF{ex:234}. The question is whether the sequence after such words represents a full embedded clause. In \REF{ex:234}, besides the verb \textit{huŋ} ‘come' used as an incipient marker, there is also an embedded clause with the subordinating conjunction \textit{ŋɔ} ‘how'. In the second example \textit{bi} ‘have' functions like an optative. In the first example, Adama Mampa is asked by the interviewer how to prepare krain-krain, and in the second, she describes the result.\footnote{Krain-krain (crain-crain) is a popular leafy vegetable used in stews.}

\ea%234
    \label{ex:234}
    Auxiliary-like lexical verbs\\
    \ea Wɔŋyi huŋ toŋgi ŋɔ pɔ chɛth keŋkeŋdɛ.\\
    \gll wɔ{}-ŋ      yi    huŋ    toŋgi    ŋɔ    pɛ      chɛth    keŋkeŋ      ɛ\\
    \textsc{3sg-emph}  \textsc{1pl}  come    show    how  \textsc{pro}\textsubscript{indef}  cook    krain-krain    \textsc{def}\\
    \glt ‘She is about to show us how to cook krain-krain.' (012-13a Adama Mampa, Cooking: 2)

    \ex Bikɔs  po  mɔi bia hun jo ni theni yenkɛlɛŋ.\\
    \gll bikɔs    po    mɔ-i      biya    hun    jo    ni    theni    yenkɛlɛŋ\\
    because  man  \textsc{2sg-prt}    have.to  come    eat    and  feel    good\\
    \glt ‘Because your husband has to come eat and feel good.' (012-13a Adama Mampa, Cooking: 57)
\z
\z

Sherbro has a great number of subordinating conjunctions, as presented in \sectref{sec:3.11}. One, the conjunction \textit{lɛ}, has many uses, but its most common use is with the meaning of ‘when' or ‘if'. In \REF{ex:235a}, Adama Mampa is remarking on how the Mende\il{Mende} never reply to her in her own language, even when it involves such a simple exchange such as a greeting well known to all. The example in \REF{ex:235b} is a proverb, again with the subordinate clause introduced with \textit{lɛ}. The example in \REF{ex:235c} is introduced with the conjunction \textit{ŋɔ} ‘when.'

\ea%235
    \label{ex:235}
    \ea \label{ex:235a} Lɛ nwɔ gbo ŋa mɔi, ŋan ŋa wɔ bua.\\
    \gll lɛ  n    wɔ    gbo  ŋa    mɔi      ŋa-n      ŋa    wɔ    bua\\
    if  \textsc{2sg}  say  just  3\textsc{pl}  afternoon  \textsc{3pl-emph}  \textsc{3pl}  say  greetings\\
    \glt ‘If you say to them, \textit{mɔi} (‘Good afternoon' in Sherbro), they will say,  \textit{bua} (‘Greetings' in Mende\il{Mende}).' (009--10a Lohr \& Mampa: 116)

    \ex \label{ex:235b} Nrɔmdɛ ma yemandɛ pɔ bɛ ko thotho mɔɛ, ma ma bɛ ko thotho thɔm mɔ.\\
    \gll n-rɔm          ɛ    ma    yema    lɛ    pɛ      bɛ    ko    thotho  mɔ-ɛ\\
    \textsc{ncm}\textsubscript{ma}{}-medicine  \textsc{def}  \textsc{ncp}\textsubscript{ma}    want    that  \textsc{pro}\textsubscript{indef}  put  to    sore    \textsc{2sg}{}-\textsc{prt}\\
    \gll ma    ma    bɛ    ko    thotho  thom    mɔ\\
    \textsc{neg}  \textsc{ncp}\textsubscript{ma}   put  to    sore    friend    \textsc{2sg}\\
    \glt ‘The medicine that you don't want to be put on your sore, do not put it on the sore of your friend.' (Proverbs: 77)

    \ex \label{ex:235c} Wɔi pɛ muni wɔi hun gbemɔ wantemdɛ ka ŋɔ ba mi ka wuwɛ.\\
    \gll wɔ-i    pɛ      muni    wɔ{}-i    hun    gbemɔ    wante{}-m    ɛ    ka ŋɔ      ba      mi    ka      wu-ɛ\\
    \textsc{3sg-prt}  again    return    \textsc{3sg-prt}  come    give.birth  sister-\textsc{1sg}  \textsc{def}  here  when    father    \textsc{1sg}  \textsc{rem.pst}  die-\textsc{prt}\\
    \glt ‘She came back here to deliver my sister when my father died.' (004a Cyril Manley on Walter Hanson: 37)
\z
\z

In such constructions the subordinate clause usually ends with the particle \textit{{}-ɛ}, as in \REF{ex:236}, which shows the same particle found at the end of relative clauses. The same is true of both clauses in \REF{ex:236}, though the first features no subordinating conjunction. Adama Mampa is talking about why she never enters a specific house: she has seen a dead person standing there. The first clause has no introductory subordinate clause but ends with the particle \textit{{}-ɛ}; the second begins with \textit{yɛ} once again and ends with \textit{{}-ɛ}.

\ea%236
    \label{ex:236}
    Ache lɔŋ kɔ gbi, ya lɔ kɔɛ a ke nɔɛ yɛ sɛmɛ kilɛ koɛ.\\
    \gll a    che  lɔ-ŋ        kɔ     gbi    a    lɔ    kɔ-ɛ\\
    1\textsc{sg}  \textsc{cop}  there{}-\textsc{emph}    go    all      \textsc{1sg}  there  go-\textsc{prt}\\
    \gll a    ke    nɔ      ɛ    yɛ    sɛmɛ    kil      ɛ    ko-ɛ\\
    \textsc{1sg}  see  person  \textsc{def}  as    stand    house    \textsc{def}  in-\textsc{prt}\\
    \glt ‘When I go I see the person standing in the room.' (009--10a Lohr \& Mampa: 211)
\z

I now turn to a consideration of relative clauses in Sherbro, the last sort of subordinate clause to be considered. Relative clauses begin with a pronoun and end with the particle \textit{{}-ɛ}. The pronoun introducing the relative clause is the noun class pronoun of the noun being relativized; even for the personal pronouns \textit{wɔ} and \textit{ha} used for animates there are no signs of case marking.

Example \REF{ex:237} involves a noun from the \textit{tha} class, \textit{siŋthɛ} ‘games', with the relative \textsc{ncp} \textit{tha} introducing the clause and the particle \textit{{}-ɛ} closing it. Jalikatu Kumba, a core member of the research team, is talking about the games she used to play as a little girl.

\ea%237
    \label{ex:237}
    Siŋthɛ thavɛ tha yaŋ   akache siŋdɛ.\\
    \gll siŋ    thi-ɛ      tha    vɛ    tha    ya-ŋ      a    ka      che  siŋ-ɛ\\
    game  \textsc{ncm}\textsubscript{tha}{}-\textsc{def}  \textsc{ncp}\textsubscript{tha}    those  \textsc{ncp}\textsubscript{tha}    \textsc{1sg-emph}  1\textsc{sg}  \textsc{rem.pst}  \textsc{prog}  play\textsc{{}-prt}\\
    \glt ‘Those are the games I used to play.' (005a Jalikatu B. Kumba: 75.2)
\z

In \REF{ex:238}, Mabel Lohr, a sometimes member of the research team, is lamenting the loss of all her medical records during the civil war (she is a midwife). There are two relative clauses in this utterance, both at the end. The first is introduced by the \textit{hɔ}{}-class pronoun \textit{hɔ}, and the second by the \textit{lɔ}{}-class pronoun \textit{lɔ}. The first binding particle seems to have been absorbed by \textit{lɔ} at the end of the first relative clause, but is present at the end of the second.

\ea%238
    \label{ex:238}
    Rai landɛ bul fli ŋɔ ko tuk kɛ rai ɛ ŋalɛ ŋɔ lɔ, lɔ akache makɛ.\\
    \gll rai    lan  ɛ    bul  fli    hɔ    koŋ  tuk\\
    paper  this  \textsc{def}  one  really  \textsc{ncp}\textsubscript{hɔ}  \textsc{pfv}  disappear\\
    \gll kɛ    rai      ɛ    ŋalɛ  hɔ    lɔ    lɔ     a    ka      che  mak-ɛ\\
    but  paper    \textsc{def}  other  \textsc{ncp}\textsubscript{hɔ}  there  where \textsc{1sg}  \textsc{rem.pst}  \textsc{prog}  mark-\textsc{prt}\\
    \glt ‘Every one of these papers has disappeared; it is just the other one that is there, where I used to keep records.' (002a Mabel Lohr, Midwifery: 23)
\z

In \REF{ex:239}, the indefinite pronoun \textit{la} ‘something' is used to relate the two clauses. Rat Husband is warning Rat Wife to do as he says. Soon after, he will try to beat her.

\newpage
\ea%239
    \label{ex:239}
    Bɛl pokan dɛ wɔe gbaki  ni hɔ ko laa wɔɛ, “Ndɛli la mɔm hɔm dɛ, Waata-o!”\\
    \gll bɛl    pokan  ɛ    wɔ  {}-i    gbaki    ni    hɔ    ko    laa    wɔ    ɛ\\
    rat    man    \textsc{def}  \textsc{3sg-prt}  answer  and  say  to    wife  \textsc{3sg}  \textsc{def}\\
    \gll n    lɛli  la    mɔ-m      wɔm  ɛ  wanta[?]-o\\
    \textsc{2sg}  look  what  \textsc{2sg-emph}  say  \textsc{prt} girl[?]\textsc{{}-emph}\\
    \glt ‘Rat Husband answered saying to his wife, “Watch what you are saying, Girly-o!" ' (123aw Yanker, Rat Wife: 66)
\z

An earlier source, gives the form of the binding particle sometimes as \textit{lɛ} (after \textit{minɛ} ‘nose'), as in \REF{ex:240a} (\citealt{Pichl1967}). However, this is not always the case, as is seen in another example in \REF{ex:240b}, also from \citet{Pichl1967}.

\ea%240
    \label{ex:240}
    \ea\label{ex:240a}  Nɔmɔk lɛ kɔ hok wɔn minɛ lɛ kɔ isay.\\
    \gll nɔmɔk  lɛ    kɔ      hok      wɔ-n      minɛ  lɛ    kɔ      i-sai\\
    mucus  \textsc{def}  \textsc{ncp}\textsubscript{kɔ}    come.from  \textsc{3sg}{}-\textsc{emph}  nose  \textsc{prt}  \textsc{ncp}\textsubscript{kɔ}    \textsc{ncm}\textsubscript{hɔ}{}-offensive\\
    \glt ‘The mucus that comes from his nose is offensive.' (P67 N: 75)

    \ex\label{ex:240b}  Kɛ kpɔnko hɔ ka che tri ko ntɛnt, hɔ nɔonɔ ka chen kɔ ayɛ.\\
    \gll kɛ    kpɔnko  hɔ    ka      che  tri    ko n-tɛnt\\
    but  forest    \textsc{ncp}\textsubscript{hɔ}  \textsc{rem.pst}  be    town  to \textsc{ncm}\textsubscript{ma}{}-near\\
    \gll hɔ      nɔ-o-nɔ          ka      che  ni    kɔ    ai      ɛ\\
    \textsc{ncp}\textsubscript{hɔ}    person-\textsc{distr}{}-person  \textsc{rem.pst}  \textsc{aux}  \textsc{neg}  go    inside    \textsc{prt}\\
    \glt ‘But there was a forest near the town, which no one entered.' (P67 K: 166)
\z
\z

In our data, the particle never appeared as \textit{lɛ}.

