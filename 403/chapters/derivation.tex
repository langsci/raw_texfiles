\chapter{Derivational morphology, compounding}
\hypertarget{Toc115517804}{}\label{ch:7}
Establishing derivational directionality sometimes proved challenging, particularly when the two forms, stem and derivative, were formally identical, a phenomenon that has proved problematic in such Mande\il{Mande} languages as Mende\il{Mende}, where there is very little morphology (e.g., \citealt{Dwyer1989}, \citealt{Innes1962}). Fortunately, Sherbro has enough morphology to identify the part of speech but still does not provide much information on directionality. (Problems in determing parts of speech are discussed in the introduction to Chapter \ref{sec:3} on word categories.) Where there was ambiguity, the guidelines adopted here were the following:

\begin{itemize}
\item Verbs generally considered more basic than nouns
\item All other parts of speech considered derived from either verbs or nouns, when related forms exist

\item Adverbs considered more basic than postpositions
\end{itemize}

An example of the challenges that arise is illustrated in \REF{ex:171}. The choice here was between the noun \textit{herka} and the verb \textit{her}. Note that the full form of the instrumental verb extension is \textit{{}-ka}, thus supporting the case for a verb base, but why would a tree not have a name?

\TabPositions{2cm,4cm,6cm,8cm}
\ea%171
    \label{ex:171}
    Derivational challenges\\
herka \tab ‘corkwood tree; canoe made of corkwood, ferry' (n.)\\
\vspace{6pt}
her \tab ‘go across, cross' (e.g., river, stream, stretch of water)\\
herni \tab ‘go across, cross' (e.g., river, stream, stretch of water)\\
herk \tab ‘ferry someone across' (e.g., river, stream, stretch of water)\\
herkɛni \tab ‘ferry oneself across' (e.g., river, stream, stretch of water)\\
\z

This chapter looks at both the highly productive and some of the less productive processes in Sherbro, typically affixes that change the word category of a stem. It also includes highly productive morphemes, e.g., agentive -\textit{nɔ}, \textit{pokan} ‘male,' and \textit{maa} ‘female' that seem more like compound elements (\sectref{sec:7.3}).

\section{Derived adverbs}
\label{sec:7.1}\hypertarget{Toc115517805}{}
Several productive processes exist for creating adverbs from other word categories. Prefixes from the noun class system are used in at least two ways. The \textit{hɔ}{}-class \textsc{ncm} \textit{i-} is a nominalizer of verbs that can then be used adverbially (see \sectref{sec:5.7}). In addition, the \textit{lɔ}{}-class prefix \textit{li-} changes a noun so that it can be used as a locative or as something like an adverb (see \sectref{sec:5.10}). In a similiar fashion adverbs can be derived using \textit{yen} ‘thing' typically followed by an adjective ‘bad' or ‘good', as shown in \REF{ex:172}.

\TabPositions{1.2cm,3cm,6cm,7cm}

\ea%172
    \label{ex:172}  Adverbs from \textit{yen} ‘thing,' indefinite pronoun
    \begin{tabular}{ll}
    yenyen & ideophone underscoring quiet\\
    yeŋwɛi & ‘badly,' lit. ‘thing-bad'\\
    yeŋwɛini & ‘agitatedly,' e.g., of waves during a storm\\
    yeŋkɛlɛŋ & ‘well, carefully' (of task performance)\\
    yeŋkɛlɛŋba & ‘very much'\\
    yeŋkɛlɛŋyeŋkɛlɛŋ & ‘thoroughly'\\
    \end{tabular}
\z

A related class of words is ideophones. These are adverb-like words that may be relatable to other word categories (usually verbs) with no derivational process involved (see \sectref{sec:3.5}, \citealt{Childs1989}). More often, however, they have no obvious language-internal origin.

Locative expressions can also be considered derived. There are at least three ways to create locatives and place names in Sherbro (also \textit{mo-} and \textit{{}-dugu} from Mande\il{Mande}). Names of places are treated in \sectref{sec:3.6}. There are three locativizing suffixes (discussed in \sectref{sec:3.8} on adpositions).

\ea%173
    \label{ex:173}
    Locativizing suffixes and postpositions\\

{}-ai \tab ‘in'\\
{}-ɛ \tab ‘in, on, under'\\
ko \tab ‘to, from, etc.'\\
\z

Other postpositions, e.g., \textit{hɔl} ‘at the entrance or opening' (\sectref{sec:3.7}), are more like the third affix \textit{ko}, which is phonologically independent because of its syllable structure and syntactic versatility. \textit{Ko} may appear either before or after a place name.

\section{Derived adjectives}
\label{sec:7.2}\hypertarget{Toc115517806}{}
The suffix \textit{{}-il/-ul} forms adjectives from verbs. It is the least problematic and most transparent of the derivational morphemes, changing stative verbs into agreeing adjectives. The suffix has a high tone (not marked in the examples) and the vowel is [\textsc{back}] harmonic with the stem vowel, with [a]-stems patterning with the front vowels. They take the suffix [il], and stems with back vowels take the suffix [ul]. In \REF{ex:174}, the first set of stems all have front vowels and the second set have back vowels.\footnote{For another case of vowel harmony,\is{vowel harmony} see the discussion of the past suffix in \sectref{sec:4.2}.}


\TabPositions{1.5cm,5.5cm,7cm,8cm}
\ea%174
    \label{ex:174} Adjectives with \textit{{}-il/-ul}\\
    \vspace{6pt}
    kath \tab ‘be hard, difficult' \tab kathil \tab ‘extremely difficult'\\
    dis \tab ‘be heavy' \tab disil \tab ‘heavy'\\
    dinth \tab ‘gleam, glitter' \tab dinthil \tab ‘pale'\\
    jɛth \tab ‘be weak' \tab jɛthil \tab ‘weak, tasteless'\\
    pɛth \tab ‘be tasty' \tab pɛthil \tab ‘tasty, sweet, good'\\
    sɛk \tab ‘be dry' \tab sɛkil \tab ‘dry'\footnotemark\\ 
  \vspace{10pt}
    nɔth \tab ‘be soft' \tab nɔthul \tab ‘very soft'\\
    bɔs \tab ‘be cold' \tab bɔsul \tab ‘cold'\\
    gbuth \tab ‘be rough' \tab gbuthul \tab ‘unripe, ill-bred'\\
    nyuŋ \tab ‘be blunt' \tab nyuhul \tab ‘blunt'\\
    puth \tab ‘be spoiled, rotten' \tab puthul \tab ‘spoiled, rotten'\\
    thuk \tab ‘be warm' \tab thukul \tab ‘feverish, ill'\\
\z
\footnotetext{cf. sɛkili ‘(make) dry' (v.)}

I now turn to processes that involve independent lexical items, namely, compounding.

\section{Compounding}
\label{sec:7.3}\hypertarget{Toc115517807}{}
Sherbro has productive processes of compounding, not all of which will be detailed here. In all cases compounds involve parts that are independent words. Typically, there is little phonological interaction at the interface between the two parts. I discuss the compounding processes in order of productivity, from the most productive to the least productive. One extremely productive compound involves the agentive \textit{nɔ} ‘person'. Another set is gender- and age-specifying forms. Others create nonce constructions, thus not highly productive morphemes.

\subsection{Agentive \textit{nɔ}}
\label{sec:7.3.1}\hypertarget{Toc115517808}{}
What I call the ‘agentive \textit{nɔ}' represents a highly productive process, close to being a derivational affix. It has an independent existence as ‘person' and an indefinite pronoun ‘someone' or even ‘they'. It can be conjoined to nouns and verbs to designate the person involved with the activity or domain to which it is attached. The bond between \textit{nɔ} and its companion is weak as other material can intervene. Forms with the \textit{nɔ} affix belong to the \textit{wɔ} class and control \textit{wɔ-}class agreement.

\textit{Nɔ} ‘person' can be affixed (prefixed or suffixed) to virtually any content word (excluding adverbs and ideophones). The affix is identical to one found in all other Bolom languages. I first discuss the prefixed forms, which are more common and more productive. In a lexicon of 4,095 entries, there was a total of 70 affixed forms, 56 of which were prefixed.\footnote{The count was performed 20 {Aug 2020}.}

\TabPositions{1.25cm,3.5cm,6.5cm,8cm}
\ea%175
    \label{ex:175} \textit{Nɔ}{}-prefixed forms\\
    \vspace{6pt}
    
    bali \tab  ‘wealth' \tab nɔbalia \tab  ‘rich person'\\
\tab\tab nɔbaliabalia \tab  ‘a very rich person'\\
    bonthɔ \tab ‘help' \tab nɔbonthɔ \tab  ‘helper'\\
    nyun \tab  ‘blindness' \tab nɔinyun \tab  ‘blind person'\\
\z

Productive as the affix is, where it appears, either at the beginning or at the end of the word, is not entirely predictable and may even  be prefixed and suffixed to the same base \REF{ex:176}.

\ea%176
    \label{ex:176}
    ton \tab ‘sing' \tab nɔton / tonnɔ \tab ‘singer'\\
    thom \tab ‘beg' \tab nɔthomɔ / thomnɔ \tab ‘beggar'\\   
\z

The second element of the prefixed forms, if a noun, will occasionally feature its prefixed noun class marker. The suffixed forms of \textit{nɔ} compounds do not \REF{ex:179}, likely because they belong to the \textit{wɔ} class, which never prefixes its \textsc{ncm}s.

\ea%177
    \label{ex:177} \textit{Nɔ} prefixed to nouns with prefixed noun-class markers\\
    
    \ea nɔinyun\\
    \gll nɔ-i-nyun\\
    person-\textsc{ncm}\textsubscript{hɔ}{}-blind\\
    \glt ‘blind person'

    \ex nɔikeche\\
    \gll nɔ-i-keche\\
    person-\textsc{ncm}\textsubscript{hɔ}{}-blind\\
    \glt ‘blind person'

 \ex nɔmpithika\\
    \gll nɔ-n-pithika\\
    person-\textsc{ncm}\textsubscript{ma}{}-rascality\\
    \glt ‘rascal'
    
    \ex nɔŋkwath\\
    \gll nɔ-n-kwath\\
    person-\textsc{ncm}\textsubscript{ma}{}-fear\\
    \glt ‘coward'
\z
\z

Occasionally there are three or more elements in a \textit{nɔ} compound, as illustrated by the examples in \REF{ex:178}.

\ea%178
    \label{ex:178}
    \ea nɔhampanth\\
    \gll nɔ-haa-n-panth\\
    person-do-\textsc{ncm}\textsubscript{ma}{}-work\\
    \glt ‘worker, laborer'

    \ex nɔncheŋwɛy\\
    \gll nɔ-N-che-N-bad\\
    person-\textsc{ncm}\textsubscript{ma}{}-be- \textsc{ncm}\textsubscript{ma}{}-bad\\
    \glt ‘person of bad character'
    \z
\z

There are fewer \textit{nɔ-}suffixed forms (13 of 70), at least some of them having to do with nationality or ethnicity, e.g., \textit{potonɔ} ‘European, outsider,' \textit{Mɛndenɔ} ‘Mende\il{Mende} person,' \textit{Bolomnɔ} ‘Bolom person'. This is an entirely productive process, as seen in \REF{ex:179}.

\TabPositions{1.25cm,4cm,5.75cm,7cm}

\ea%179
    \label{ex:179}
    \textit{Nɔ}{}-suffixed forms\\
    \vspace{6pt}
    chol \tab ‘art' \tab cholnɔ \tab ‘artist'\\
    gbisiŋ \tab ‘marry' \tab gbisiŋnɔ \tab ‘married person'\\
    gboka \tab ‘society' \tab gbokanɔ \tab ‘non-initiate in society' (\citealt{Pichl1967})\\
    kɔisu \tab ‘magic' \tab kɔisunɔ \tab ‘sorcerer'\\
    pɔk \tab ‘country' \tab pɔknɔ \tab ‘country person'\\
    soko \tab ‘society' \tab sokonɔ \tab ‘society leader'
\z

A number of words have no associated base, as in \REF{ex:180}.

\ea%180
    \label{ex:180}
    wonɔ \tab ‘slave'\\
    nyanɔ \tab ‘stranger, outsider'\\
    puinɔ \tab ‘hunter'\\
\z

Example \REF{ex:181} shows that the link between \textit{nɔ} and its accompanying element may not be as tight as other compound elements, since \textsc{poss} intervenes in the following example, where the possessive \textit{mi} (\textsc{1sg}) comes after \textit{nɔ} and before the \textsc{ncm} for ‘hate.'

\ea%181
    \label{ex:181}
    \ea nɔnchenk\\
    \gll nɔ-n-chenk\\
    person\textsc{{}-ncm}\textsubscript{ma}{}-hate\\
    \glt ‘enemy, adversary'

    \ex nɔ mi nchenk\\
    nɔ-mi-n-chen\\
    person\textsc{{}-1sg-ncm}\textsubscript{ma}{}-hate\\
    \glt ‘my enemy'
\z
\z

\noindent The separation here may be part of the general move towards analysis and away from synthesis that was seen already with the former verb extension \textit{{}-ka} becoming the adposition \textit{ka} discussed in \sectref{bkm:Ref48814327}.

Less productive are the gender- and age-denoting compound parts.

\subsection{\textit{pokan} ‘male,' \textit{maa} ‘female,' \textit{taa} ‘young'}
\label{sec:7.3.2}\hypertarget{Toc115517809}{}
The most common of these three compound elements is that of denoting a male-associated entity. The gender-unspecified form \textit{nɔ} ‘person' discussed in the previous section can form a compound with the male morpheme \textit{pokan}, showing its gender neutrality, as well as with other animals to denote the male member of the species.


\ea%182
    \label{ex:182} Derivatives with \textit{pokan}\\
    
    nɔ \tab ‘person' \tab nɔpokan \tab ‘man, husband'\\
    sɔk \tab ‘fowl' \tab sɔkpokan \tab ‘rooster, cock'\\
    na \tab ‘cattle' \tab napokan \tab ‘bull'
\z

Sometimes there is no “male-ness” to the derived form but rather a notion of size or intensity\is{intensity}.

\ea%183
    \label{ex:183} Derivatives with \textit{pokan} not involving ‘male'\\
   su \tab ‘finger' \tab supokan \tab \tab ‘thumb'\\
   rɛm \tab ‘toe' \tab rɛmpokan \tab \tab ‘big toe'\\
   ra \tab ‘snake' \tab rapokan \tab \tab ‘green mamba' (lethal snake)\\
   santhil \tab ‘sword grass' \tab santhilpokan \tab ‘extra sharp sword grass'
\z

The comparable female word is \textit{maa} ‘woman, girl'. There are not nearly as many derived or compound forms.

\ea%184
    \label{ex:184} Derivatives with \textit{maa}\\
    nɔ \tab ‘person' \tab nɔmaa \tab ‘woman'\\
    bɛɛ \tab ‘chief, king' \tab bɛmaa \tab ‘queen'\\
    sɔk \tab ‘fowl' \tab sɔkmaa \tab ‘hen'\\
    na \tab ‘cattle' \tab namaa \tab ‘cow'
\z

Diminutive or ‘young' meanings can be conveyed with \textit{taa} ‘young, youth'. There is also a reduplicated adjectival form \textit{tata} with a syntax similar to the last two forms, which also could be argued to be a noun-adjective construction.

\ea%185
    \label{ex:185}Diminutives with \textit{taa}\\
    
    langban ‘man' \tab taalaŋgbaŋ \tab ‘young man'\\
    pokan \tab ‘man' \tab taapokan \tab \tab ‘young man'\\
    sɔk \tab ‘fowl' \tab taasɔk \tab \tab ‘chick'\\
    rem \tab ‘toe' \tab rɛmta \tab \tab ‘baby toe'\\
    pal \tab ‘net pole' \tab palta \tab \tab ‘inner, smaller net pole'\\
    \z

The next section looks at less productive compound elements.

\subsection{Other compounds}
\label{sec:7.3.3}\hypertarget{Toc115517810}{}
A great number of compounds begin with a generic term followed by a more specific one, much as in a noun-adjective construction. The combination indicates a subset of the larger category, such as the case with agentive \textit{nɔ} compounds discussed in \sectref{sec:7.3.1}.

A great number of compounds begin with the general term for ‘nut' which is \textit{bɛl}.

\ea%186
    \label{ex:186}
    Some compounds in Sherbro\\
    \ea
    bɛlthampel\\
    \gll bɛl-thampel\\
    nut-raptor\\
    \glt ‘a small tree; a grass used for potions'\\

    \ex  bɛlmagbo\\
    \gll bɛl-ma-gbo\\
    nut-\textsc{ncp}\textsubscript{ma}{}-warri.game\\
    \glt ‘seeds for warri game'
\z
\z

\noindent Other compounds beginning with a generic term followed by a more restrictive one are given in \REF{ex:187}.

%\TabPositions{2cm,4cm,6cm,8cm}
\TabPositions{2cm,6cm,7cm,8cm}
\ea%187
    \label{ex:187}
    \ea vee  \tab  ‘bird'\\
        vebolmin \tab ‘swallow' (lit. ‘bird-head-crazy')\footnotemark\\
    \ex thɔk \tab ‘tree'\\
        thɔkbol \tab ‘stick to loosen braids' (lit. ‘stick-head')\\
        thɔkihɔlɔŋ \tab ‘life tree, its bark used for treating malaria\\ 
        \tab (lit. ‘tree-breath/life')\\
    \ex pɛl \tab ‘fishing  net'\\
        pɛlgbampɔ \tab ‘fishing net for \textit{gbampɔ} (mullet)'\\
        pɛlbɔlkek \tab ‘fishing net for \textit{bɔlkek} (beard-beard fish)'
    \z
\z

\footnotetext{So-called because of the way they fly, i.e., “crazily.”}

The syntax is usually determined by the semantics characterized above, but some compounds have variant orderings.

\TabPositions{4cm}

\ea%188
    \label{ex:188}  \textit{bol} ‘head' and \textit{thɛm} ‘hatch' (an egg)\\
    
    thɛmɛbol / bolthɛmɛ \tab ‘severe headache'
\z

Reminiscent of the \textit{nɔ} compounds is the example in \REF{ex:189}, where \textit{ma}, the \textsc{ncp} of the \textit{ma} class, acts as something of a possessive ‘of' with the second element a noun (‘animals', ‘birds', and ‘insects'), the first and the third preceded by their \textsc{ncm}s.

\ea%189
    \label{ex:189}
    Repeated from \REF{ex:145} in \sectref{sec:5.5}\\
    Kaiŋ Taso wɔ thee nhɔk ma nvissɛ, veesɛ, ni ŋkɔlɔŋsɛ.\\
    \gll Kaiŋ    Taso    wɔ    thee      n-wɔk      ma    n-vis-si-ɛ\\
    Kain    Tasso    \textsc{3sg}  understand  \textsc{ncm}\textsubscript{ma}{}-word  \textsc{ncp}\textsubscript{ma}    \textsc{ncm}\textsubscript{ma}{}-animal-\textsc{ncm}\textsubscript{si}{}-\textsc{def}\\
    \gll vee-si-ɛ        ni    n-kɔlɔŋ-si-ɛ\\
    bird-\textsc{ncm}\textsubscript{si}{}-\textsc{def}  and  \textsc{ncm}\textsubscript{ma}{}-insect-\textsc{ncm}\textsubscript{si}{}-\textsc{def}\\
    ‎\glt ‘Kain Tasso understands the words of every animal, bird, and insect.' (123aw Yanker, Rat Wife: 52)
\z

There is also the nominalized form of the verb, the compound nominal \textit{nchɔŋmalen} ‘love', a \textit{ma}\nobreakdash-class noun. The \textit{ma} is used internally to join the two parts of the compound, the verb \textit{chɔŋ} ‘offer' and the pronoun \textit{len} ‘something'. Similar examples are \textit{nkɔsmahuɛ} ‘late night food' (lit. ‘food of the day' and \textit{ndɛthmaboot} ‘ship board' (lit. ‘board of the ship').

\section{Distributive}
\label{sec:7.4}\hypertarget{Toc115517811}{}
The distributive is a productive construction that means something like ‘each' or ‘every' noun. It follows the syntactic pattern of \textsc{noun}{}-\textit{o}{}-\textsc{noun}. The construction is used not just with nouns but also with numbers and pronouns.

\TabPositions{2cm,4cm,6cm,8cm}

\ea%190
    \label{ex:190} \ea lenolen\\
    \gll len-o-len\\
    thing-\textsc{distr}{}-thing\\
    \glt ‘everything'\\
    \vspace{6pt}
   \ex nɔonɔ\\   
    \gll nɔ-o-nɔ\\
    person-\textsc{distr}{}-person\\
    \glt ‘everyone'\\
    \vspace{6pt}
    \ex\label{ex:190c} ndɔndɔ\\
    \gll ndɔ-ndɔ\\
    where-where\\
    \glt ‘everywhere'\\
        \vspace{6pt}
    \ex Lɔkɔɔlɔkɔ hɔ ya hun dɛ, ya bɔnth wɔ ha mpanth.\\
    \gll lɔkɔ-o-lɔkɔ    hɔ    ya     hun     ɛ    ya    bɔnth    wɔ    ha    n-panth\\
    day-\textsc{distr}{}-day  when  \textsc{1sg}  come    \textsc{prt}  \textsc{1sg}  meet    \textsc{3sg}  at    \textsc{ncm}\textsubscript{ma}{}-work\\
    \glt ‘Always when I come, I meet him at work.' (P67 L: 92)\\
        \vspace{6pt}
    \ex Rai o rai hɔ n ke gbo ɛ, n chi hɔ.\\
    \gll rai-o-rai        hɔ      n    ke    gbo    ɛ    n    chi    hɔ\\
    book-\textsc{distr}{}-book  \textsc{ncp}\textsubscript{hɔ}    \textsc{2sg}  see  indeed  \textsc{prt}  \textsc{2sg}  bring    \textsc{ncp}\textsubscript{hɔ}\\
    \glt ‘Bring whatever book you see.' (\citealt[34]{Sumner1921})
\z
\z

The reduplicated form \textit{ndɔ-ndɔ} ‘everywhere' (lit. ‘where-where') without the distributive particle is a comparable form and illustrates one of the uses of reduplication \REF{ex:190c}. Some others are presented below.

\section{Reduplication}
\label{sec:7.5}
Reduplication is a widespread and productive process in the language. The word categories that feature reduplication are limited to content categories: nouns, verbs, adjectives and adverbs, including \textit{ndɔ} mentioned above. The iconic meaning generally involves intensity\is{intensity} / multiplicity / plurality / pluractionality. In the examples in \REF{ex:191}, the reduplicated forms had extra high tones.

\TabPositions{1.5cm,5.5cm,7cm,8cm}
\ea%191
    \label{ex:191}
    vil \tab ‘tall' \tab vilvil \tab ‘very tall'\\
    kith \tab ‘small' \tab kithkith \tab ‘very small'\\
    tɛŋ \tab ‘sour' \tab tɛŋtɛŋ \tab ‘really sour'\\
\z

The simplest form is stem reduplication, though functional morphemes such as noun class markers and verbal morphemes may also be involved.\footnote{This section considers only full reduplication.} For example, with adjectives that show agreement, the agreement marker may be prefixed to both parts of the stem, here the \textsc{ncm} for the \textit{ha} class \textit{a-} in \textit{abomabom} ‘huge'. The adjective in the second sentence, \textit{ŋgbelŋgbel} ‘anxious', also shows full reduplication.

\ea%192
    \label{ex:192}
    \ea \label{ex:192a} Ye koŋ lɛli yɛllɛɛ, wɔe kɔnth ntol abomabom koŋhɔany ara ni yusia tilaŋ.\\
    \gll ye      koŋ  lɛli  yɛl      ɛ    ɛ wɔ{}-i    kɔnth    n-tol      a-bomabom    koŋhɔanya-ra    ni    yu-si        a-tilaŋ\\
    when    \textsc{pfv}  look  chain    \textsc{def}  \textsc{prt} \textsc{3sg-prt}  catch    \textsc{ncm}\textsubscript{ma}{}-fish \textsc{ncm}\textsubscript{ha}{}-huge    twenty-three    and  fish-\textsc{ncm}\textsubscript{si}    \textsc{ncm}\textsubscript{ha}{}-other\\
    \glt ‘After he had checked the \textit{yɛlle}, he caught 23 big \textit{tol} and other fish.' (124aw Yanker, Boy Lost at Sea: 51)\\
   
    \ex \label{ex.192b} Ni wɔe che ŋgbelŋgbel ha kɔ lɛli pɛllɛ.\\
    \gll ni    wɔ{}-i    che  ŋgbelŋgbel  ha    kɔ    lɛli      pɛl        ɛ\\
    then  \textsc{3sg}{}-\textsc{prt}  be    anxious    for    go    look(at)    fishing.net  \textsc{def}\\
    \glt ‘He was very anxious to check the fishing net.' (124aw Yanker, Boy Lost at Sea: 53)
\z
\z

The more typical case is for just the stem to be reduplicated, as is the case with verbs and adjectives. Usually there is some notion of pluractionality.

\TabPositions{1.5cm,4.5cm,7cm,8cm}
\ea%193
    \label{ex:193} Simple stem reduplication\\

bala \tab ‘hug' \tab balabala \tab ‘hug effusively'\\
math \tab ‘hide' \tab mathmathnin \tab ‘multiple acts of hiding' (as\\
\tab \tab \tab during a children's game)\\
gbɛt \tab ‘knock a head' \tab gbɛɛtigbɛɛti \tab ‘strike on the head\\ 
\tab \tab \tab repeatedly'\\
kɛlɛŋ \tab ‘fine, good' \tab kɛlɛŋkɛlɛŋ \tab ‘fine, beautiful'\\
thuk \tab ‘warm' \tab thukthuk \tab ‘very warm, hot'\\
\z

\noindent Some reduplicated forms have no non-reduplicated source counterpart. Most of such forms are nouns.
\TabPositions{2.5cm,4.5cm,7cm,8cm}
\ea%194
    \label{ex:194}
    gbangban \tab ‘pigeon-sized bird'\\
    gbuŋgbuŋ \tab ‘steam launch' (onomatopoeic?)\\
    baba \tab ‘umbrella'\\
    pɛɛpɛɛ \tab ‘shoulder'\\
    kaakaa \tab ‘hermit crab'\\
    fuŋfuŋ \tab ‘rice in the nursery stage of planting'\\
    timitimi \tab ‘weakened, feeble'
\z
