\chapter{Syntax}
\label{ch:8}\hypertarget{Toc115517812}{}
This chapter begins by treating the phrasal grammar of nouns and verbs, then looks at two sentential level processes, questions and focus. The verb phrase of Sherbro, and indeed of Bolom as a whole, is quite complex because it is mixed: there are different word orders depending on the location of tense.

\section{The noun phrase}
\label{sec:8.1}\hypertarget{Toc115517813}{}
The Sherbro \textsc{np} follows a head-initial pattern common to Mel\il{Mel} (\citealt{Childs2024c}). Modifiers follow the noun they modify and sometimes show agreement with the head noun. The definite article \textit{lɛ} is also marked for agreement (see \REF{ex:42} in \sectref{sec:2.4.3}, \REF{ex:136c} in \sectref{sec:5.1}, \REF{ex:151} in \sectref{sec:5.6}, \REF{ex:167b} in \sectref{sec:6.3}, \REF{ex:198} \& \REF{ex:202} in \sectref{sec:8.2}, and \REF{ex:237} in \sectref{sec:9.3}). Adjectives are variable in showing agreement; in fact, whether or not an adjective shows agreement can be used as a criterion for determining core adjectives (see \sectref{sec:3.2}).

The following sentence contains three different noun phrases, illustrating the common structures of \textsc{np}s in Sherbro. The first is a simple noun+article construction, \textit{ndoɛ} ‘(the) day'. The second is a noun followed by an adjective and an article, \textit{nɔmaa bɛndɛ} ‘the old woman'. The third \textsc{np} is a possessive construction, \textit{jajɛl Kaiŋ Tasoɛ} ‘Kain Tasso's mother-in-law' in apposition to \textit{nɔmaa bɛndɛ}. Here possession is indicated only by juxtaposition (and word order). The definite marker appears at the end of the \textsc{np}, as in all other constructions.

\ea%195
    \label{ex:195}
    Ndoɛ muɛkɛ mɛŋtiŋndɛ, ni nɔmaa bɛndɛ, wɔe wu jajɛl Kaiŋ Tasoɛ.\\
    \gll n-loɛ      ɛ    muɛkɛ  mɛŋtiŋ  ɛ    ni    nɔmaa  bɛn  ɛ\\
    \textsc{ncm}\textsubscript{ma}{}-day  \textsc{def}   reach    seven    \textsc{prt}  then  woman  old  \textsc{def}\\
    \gll wɔ{}-i      wu  jajɛl          Kaiŋ  Taso    ɛ\\
    \textsc{3sg-prt}    die  mother-in-law    Kain  Tasso    \textsc{def}\\
    \glt ‘On the seventh day, the old woman died, Kain Tasso's mother-in-law.' (123aw Yanker, Rat Wife: 20)
\z

Possession is indicated simply by juxtaposition, as above, but when pronouns are used also involves elements from the noun class system for some noun classes, namely, agreement shown by the \textit{ma}, \textit{hɔ}, and \textit{tha} classes, parallel to the agreement markers discussed in Chapter \ref{ch:5}. The usual case is for the possessive to be prefixed with the \textsc{ncm} of the head noun, as in \REF{ex:196a}, but the \textit{ma}{}-class nouns unusually feature the \textsc{ncp} \textit{ma} (vs. the \textsc{ncm} \textit{n}{}-) in such constructions \REF{ex:196b}, as already illustrated in \REF{ex:189} in \sectref{sec:7.3.3}.

\ea%196
    \label{ex:196}
    Agreement in possessive constructions with pronouns\\
    \ea \label{ex:196a}  A kɔ viiki bɛŋthim   dɛ.\\
    \gll a    kɔ    viki    bɛŋ  thi-mi      ɛ\\
    \textsc{1sg}  go    stretch  leg  \textsc{ncm}\textsubscript{tha}{}-\textsc{1sg}    \textsc{def}\\
    \glt ‘I'm going to stretch my legs.' (P67 V: 38)

    \ex \label{ex:196b}  Ŋ kɔ thunɔ nyik mam dɛ.\\
    \gll n    kɔ    thunɔ    n-yiik      ma-mi    ɛ\\
    \textsc{2sg}  go    search  \textsc{ncm}\textsubscript{ma}{}-key  \textsc{ncp}\textsubscript{ma}{}-\textsc{1sg}  \textsc{def}\\
    \glt ‘Go look for my keys!' (P67 TH: 168)
\z
\z

\section{The verb phrase}
\label{sec:8.2}\hypertarget{Toc115517814}{}
As mentioned in the introduction to this chapter, the Sherbro \textsc{VP} has an interesting wrinkle found throughout Bolom, an indissoluble unit that results in different word orders in a restricted context. Before entering into that discussion, however, it is necessary to explain some other parts of the VP, which will form several of the arguments advanced there. The first is argument structure.

\ea%197
    \label{ex:197}
    Basic argument structure: SV(O)(O)X\\
\z

In some cases when the subject is pronominal, especially \textsc{3sg}, no subject exists on the surface. As discussed in Chapter \ref{ch:6} on verb extensions, argument structure can be changed (arguments deleted or added) when verbs are affixed. Arguments can also be reordered due to semantics.

Generally speaking, in post-verbal contexts personal possessives are promoted to a position closer to the verb rather than remaining \textit{in situ} following the noun possessed. Undoubtedly animacy\is{animacy} plays a role, as it does elsewhere in the language: semantics overrides morphology in the noun class system (see \sectref{sec:5.1}).\footnote{Animacy also has diachronic importance in the evolution of a prefixing to a suffixing system of marking noun classes (\citealt{Greenberg1977}, \citealt{Greenberg1978}), as has taken place in Mel\il{Mel} (\citealt{Childs1983}).}

In \REF{ex:198}, there are two uses of the 3sg pronoun \textit{wɔ}, both of which illustrate this promotion. In its first appearance it appears next to the verb before ‘saliva,' the thing actually spat. Plausibly \textit{wɔ} could be in a prepositional phrase, certainly after the direct object. In the pronoun's second appearance, it is a possessive pronoun ‘his,' no longer a dependent element following \textit{bol} ‘head'. It has rather been promoted to the immediate post-verbal slot preceding all other arguments (not only \textit{bol} but also \textit{vɛ} ‘thorns').

\ea%198
    \label{ex:198}
    Pɔ thu wɔ ilathɛ, pɔ bɛ wɔ vɛ thɛ bol.\\
    \gll pɛ      thu  wɔ    i-lath        ɛ    pɛ      bɛ    wɔ    vɛ    thi-ɛ      bol\\
    \textsc{pro}\textsubscript{indef}  spit  \textsc{3sg}  \textsc{ncm}\textsubscript{hɔ}{}-saliva  \textsc{def}  \textsc{pro}\textsubscript{indef}  place  \textsc{3sg}  thorn  \textsc{ncm}\textsubscript{tha}\textsc{{}-def}  head\\
    \glt ‘People spat on him and put thorns on his head.' (003a Shenge Youth Choir, Hymns:  102)
\z

In a more complicated example \REF{ex:199}, still involving a possessive pronoun and \textit{bol} ‘head,' the possessive is moved up to the position immediately after the quasi-auxiliary \textit{kɔ} ‘go,' presumably the carrier of tense (see \sectref{sec:8.2.3} for details on the integrity of the tense and object pronoun syntagm).

\ea%199
    \label{ex:199}
    Ŋyemaɛ ŋa kɔ mi pɛl bol?\\
    \gll n    yema-ɛ    ŋa    kɔ    mi    pɛl      bol\\
    2\textsc{sg}  want-\textsc{prt}  3\textsc{pl}  go    1\textsc{sg}  break    head\\
    \glt ‘Do you want them to go and crack my head?' (123aw Yanker, Rat Wife: 70.1)
\z

Animacy also determines which of two post-verbal arguments in ditransitive constructions appears after the verb. Mabel Lohr describes how she is compensated for her midwife services. In the second and third clauses in \REF{ex:200}, as in \REF{ex:199}, the pronoun appears before the lexical verb (\textit{paka} in the first clause, \textit{ka} in the second and third).

\ea%200
    \label{ex:200}
    Apum haŋ che mi paka, apum hamika nsoiɛ, ha mika boyaɛ.\\
    \gll a-pum      ha-ŋ      che  mi    paka\\
    \textsc{ncm}\textsubscript{ha}{}-some  \textsc{3pl-emph}  \textsc{fut}  \textsc{1sg}  pay\\
    \gll a-pum      ha    mi    ka    n-swe        ɛ    ha    mi    ka    boya  ɛ\\
    \textsc{ncm}\textsubscript{ha}{}-some  \textsc{3pl}  1\textsc{sg}  give  \textsc{ncm}\textsubscript{ma}{}-soap    \textsc{def}  \textsc{3pl}  \textsc{1sg}  give  gift  \textsc{def}\\
    \glt ‘Some will not pay me, some will give me soap, others give me a gift.' (002a Mabel Lohr, Midwifery: 16)
\z

Another oddity to the Sherbro \textsc{vp} is what seems to be a split lexical item. An idiomatic construction that is reminiscent of the \textsc{tns-op} syntagma is the discontinuous construction for ‘love'. Here, however, a full \textsc{np} moves in between the two parts, the first of which, \textit{chɔŋ}, means something like ‘pour, dish up, offer.'\footnote{Possibly related to \textit{chɔŋ} ‘lay (eggs).'} The second part, not a verb, is an indefinite pronoun translated as ‘thing, something'. In both examples in \REF{ex:201}, it is a complex \textsc{np} that fills the slot between the two parts.

\ea%201
    \label{ex:201}
    chɔŋ … len ‘love' (lit. ‘pour, offer ... something')
    \ea Yaŋ, a chɔŋ nwɔk mamdɛ len, Mbolomdɛ.\\
    \gll ya-ŋ      a    chɔŋ  n-hɔk          ma mi    ɛ    len    n-bolom    dɛ\\
    \textsc{1sg-emph}  \textsc{1sg}  offer  \textsc{ncm}\textsubscript{ma}{}-language  \textsc{ncp}\textsubscript{ma} \textsc{1sg.poss}  \textsc{def}  thing    \textsc{ncm}\textsubscript{ma}{}-Bolom  \textsc{def}\\
    \glt ‘Me, I love it (the church service) in my own language, Bolom.' (004a Cyril Manley on Walter Hanson: 86)

    \ex A chɔŋ mpanth ma chɛk len kə ma katho.\\
    \gll a    chɔŋ  n-panth      ma-chɛk      len    kɛ    ma    kath-o\\
    \textsc{1sg}  offer  \textsc{ncm}\textsubscript{ma}{}-work  \textsc{ncm}\textsubscript{ma}{}-farm  thing    but  \textsc{ncp}\textsubscript{ma}    hard-\textsc{emph}\\
    \glt ‘I like farmwork, but it is hard!' (P67 K: 65)\footnote{“Ma kath!” is the title of a video, part of the documentation produced by the SLC\is{SLC}, available at all the archive sites mentioned in \sectref{sec:1.2}.}
\z
\z

There is also the nominalized form of the verb, the nominal \textit{nchɔŋmalen} ‘love,' a \textit{ma}{}-class noun, where the two parts are joined into a single word, mediated by the noun class marker and the noun class pronoun: \textsc{ncm}\textsubscript{ma}{}-offer-\textsc{ncp}\textsubscript{ma}{}-thing. The meaning of \textit{len} is still uncertain and is likely not as indefinite as glossed here.

Verb phrases or sentences do not necessarily have to have verbs, as in \REF{ex:202}, with a predicate adjective \textit{bom} ‘big.'

\ea%202
    \label{ex:202}
    Kil thilɛ tha pujoŋ kunɛ tha bom.\\
    \gll kil      thi-lɛ      tha    Pujoŋ    kunɛ    tha    bom\\
    house    \textsc{ncm}\textsubscript{tha}{}-\textsc{def}  \textsc{ncp}\textsubscript{tha}    Pujehun  inside    \textsc{ncp}\textsubscript{tha}    big\\
    \glt ‘The houses in Pujehun are big.' (P67 B: 235)
\z

\subsection{The Consecutive marker \textit{-i}}
\label{sec:8.2.1}\hypertarget{Toc115517815}{}
An unusual morpheme is the suffix \textit{{}-i} (with an \textit{{}-e} allomorph) attached to initial pronouns and found only in the imperfective as it is used in narratives. Its exact function is still uncertain, but it is sensitive to a syntactic unit here called “\textsc{tns} (\nobreakdash-\textsc{pro}\textsubscript{obj})”, as characterized below.

Speakers were not able to easily translate the morpheme nor explain its function. They would typically say it meant something like ‘then' or ‘and,' despite the presence of a lexical conjunction of the same meaning in the same clause. Sentences without the form would be considered ungrammatical. Its function, therefore, seems purely grammatical, yet tied to narratives. Despite the uncertainty over its exact function, in this section it is considered \textsc{consec}, the consecutive marker, with variants of \textsc{consec}\textsubscript{subj}, when it appears on subjects, and \textsc{consec}\textsubscript{obj}, when it appears on objects, and in other sections where this status is important. In other contexts, because of its length, it will be glossed simply as \textsc{prt} for ‘particle.'

Although \REF{ex:203} is somewhat unusual in involving a series of four consecutive uses of the conjunction, it represents well the distribution of the form in simple tenses. It connects sequential actions usually appearing in chronological order from early to late; it was variously translated as ‘and,' ‘and then,' ‘and now,' and ‘next'. It appeared frequently in such process descriptions as the cultivation of rice and also in cooking instructions, in being blown away in a storm, and in growing up. The pronoun \textit{kɔ} is the noun class pronoun for \textit{pɛlɛ} ‘rice.' (The growth of rice is seen as similar to parturition.)

\ea%203
    \label{ex:203}
    Pɛlɛ kɔi pith, kɔi pingi, kɔi bi kun, kɔi gbemɔ.\\
    \gll pɛlɛ  kɔ-i          pith  kɔ-i          pingi\\
    rice  \textsc{ncp}\textsubscript{kɔ}\textsc{{}-consec}\textsubscript{subj}  dark  \textsc{ncp}\textsubscript{kɔ}\textsc{{}-consec}\textsubscript{subj}  change\\
    \gll kɔ-i          bi    kun    kɔ-i            gbemɔ\\
    \textsc{ncp}\textsubscript{kɔ}\textsc{{}-consec}\textsubscript{subj}  have  belly    \textsc{ncp}\textsubscript{kɔ}\textsc{{}- consec}\textsubscript{subj}    give.birth\\
    \glt ‘‎The rice will get dark, and then it will change and swell up and then tiller (shoots appear).' (006v Abdulai Bendu, Rice Growing: 34)
\z

The following example shows the marker on the pronominal object \textit{la} (twice). Virginia Lohr is retelling a folktale about a mistreated young girl who undergoes a great deal of deprivation. Here she talks about how longlasting her mistreatment has been. The pronoun \textit{la} refers to her new family's practice (of abusing her).

\newpage
\ea%204
    \label{ex:204}
    Ŋa lai le kunε haaŋ, ŋa lai le kunε tee.\\
    \gll ŋa    la-i            le        kunε    haaŋ\\
    3\textsc{pl}  \textsc{pro}\textsubscript{indef}\textsc{{}-consec}\textsubscript{obj}  remain    inside    long\\
    \gll ŋa    la-i            le        kunε    tee\\
    3\textsc{pl}  \textsc{pro}\textsubscript{indef}{}-\textsc{consec}\textsubscript{obj}  remain    inside    until\\
    \glt ‘They continued in this way a long time, they remained this way for a while.' (122a Virginia Lohr, Two Mates: 2)
\z

A mixed use of the suffix appears in \REF{ex:205}. In the first clause, the marker is affixed to \textit{kɔ}, the noun class pronoun for the dish Adama Mampa is preparing. It is not the subject but rather the object of the verb \textit{pulipuli} ‘mix' with the affix (‘You mix the food'). In the second clause, the suffix has returned to a spot after the subject pronoun in the absence of an object pronoun.

\ea%205
    \label{ex:205}
    Mɔ kɔi minɛ koŋ pulipuli gbi, joɛ, mɔi gbiŋgith.\\
    \gll mɔ  kɔ-i          minɛ    koŋ  pulipuli  gbi  jo    ɛ    mɔ-i          gbiŋkith\\
    you  \textsc{ncp}\textsubscript{kɔ}{}-\textsc{consec}\textsubscript{obj}  return    finish  mix    all    food  \textsc{def}  \textsc{2sg}{}-\textsc{consec}\textsubscript{subj}  cover\\
    \glt ‘You then mix it all, the food, then you cover it.' (012-13a Adama Mampa, Cooking: 69)
\z

In \REF{ex:206}, like \REF{ex:205}, the example is particularly instructive in showing the marker's two positions. The sentence is part of a narrative provided by Yeabu Bangura in her town of Seaport [sipɔ] at the mouth of the Bumpeh River explaining how one smokes fish.\footnote{She was actually smoking the fish as she explained the process, as can be seen in the video.} Here she speaks of how one procures the fish to smoke, involving negotiation with a fisherman. A regular activity, the collection process is all told in the imperfective, where tense marking is on the subject pronoun. In the first two clauses, the suffix is attached to \textit{wɔ} \textsc{3sg}, ‘he' (the fisherman), but in the third it is attached to the object pronoun \textit{ha} the \textit{ha}{}-class pronoun (‘them,' i.e., the fish).

\ea%206
    \label{ex:206}
    Wɔi kɔni hɛlɛ ko, wɔi kɔ ŋɔth yenchɛkɛ, wɔ ŋai chi.\\
    \gll wɔ{}-i          kɔni  hɛlɛ  ko    wɔ{}-i          kɔ    ŋɔth  yenchɛk    ɛ\\
    \textsc{3sg}{}-\textsc{consec}\textsubscript{subj}  go    sea  to    \textsc{3sg}{}-\textsc{consec}\textsubscript{subj}  go    fish  fish(pl.)    \textsc{def}\\
    \gll wɔ    ha-i          chi\\
    \textsc{3sg}  \textsc{ncp}\textsubscript{ha}{}-\textsc{consec}\textsubscript{obj}  bring\\
    \glt ‘Then he goes to sea, catches the fish, and brings them back.' (184v Fish Smoking Seaport: 30)
\z

In the imperfective, where tense is marked on the subject pronoun, and in compound tenses, with an auxiliary marked with tense, object pronouns appear right after the tense-carrying item and form part of an indissoluble unit \textsc{tns}\nobreakdash-\textsc{pro}\textsubscript{obj}.

In questions when there is no verb nor even a copula, the suffix still attaches to the subject pronoun as seen in \REF{ex:207}.

\ea%207
    \label{ex:207}
    Wante mɔɛ dɔwɔia?\\
    \gll wante    mɔ  ɛ    ndɔ    wɔ-i-a\\
    sister    \textsc{2sg}  \textsc{def}  where  \textsc{3sg-prt}{}-Q\\
‎    \glt ‘Your sister, where is she?' (009--10a Lohr \& Mampa: 78)
\z

What is significant more than the semantics of \textit{{}-i} is its syntax. It appears only after pronominal arguments, after subject pronouns in the imperfective used in narratives, and after the object pronoun before the lexical verb. The same phenomenon occurs in both Bom-Kim and Mani.

\subsection{Negation}
\label{sec:8.2.2}\hypertarget{Toc115517816}
This section deals with the syntactic side of negation. The morphological aspects are treated in \sectref{sec:4.5}. As laid out previously, there are two distinct negators, \textit{ni} and \textit{ma}, with \textit{ni} having many different allomorphs, including its complete absence, and \textit{ma} being more syntactic. Other negators exist. The morpheme \textit{be} is used in a number of different ways, usually as a sentential negator, sometimes with an emphatic marker \textit{{}-o}, \textit{beo} ‘No!,' a strong denial, as in \REF{ex:208d}. The other examples in \REF{ex:208} show different uses.
\TabPositions{1cm,4cm,6cm,8cm}
\ea%208
    \label{ex:208}
    The negator \textit{be} (variant [bo])\\
    \ea\label{ex:208a} benɔ \tab ‘no one' (cf. \textit{nɔ} ‘person')\\
    \ex\label{ex:208b} Tak bahin yɛ wɔ i si, bepɛ nɔ kedɛ wɔn.\\
    \gll tak-ba      hi-n      ɛ    wɔ    i    si      be    pɛ    nɔ      kendɛ    wɔ-n\\
    son-father    \textsc{1pl}{}-\textsc{emph}  \textsc{def}  \textsc{3sg}  \textsc{1pl}  know    \textsc{neg}  again  person  like    \textsc{3sg-emph}\\
    \glt ‘The son of God we know, we know no other like him.' (003a Shenge Youth Choir, Hymns: 119)
    \ex\label{ex:208c} Lɔn lɔ pɔ che ma bo wɔ kɛt-kɛt.\\
    \gll lɔ-n        lɔ      pɛ      che  ma    be    wɔ      kɛtkɛt\\
    there-\textsc{emph}    \textsc{ncp}\textsubscript{lɔ}    \textsc{pro}\textsubscript{indef}  \textsc{aux}  \textsc{ncp}\textsubscript{ma}    \textsc{neg}  speak    regularly\\
    \glt ‘It is only there where people don't speak it regularly.' (018a Suffian Koroma: 58)
    \ex\label{ex:208d} Beo, ŋa che mi bonth ...\\
    \gll be-o      ŋa    che      mi    bonth\\
    no-\textsc{emph}    \textsc{3pl}  \textsc{aux.neg}    \textsc{1sg}  help\\
    \glt ‘No, they do not help me ...' (094a Ansu Kagboro: 34)
\z
\z

Note also the lack of a negative marker on the auxiliary \textit{che}, where it usually appears (see the examples in \REF{ex:132}). The syntax here marks the construction as negative, as is evident in \REF{ex:209}. Here Baba Mandela is talking about climate change and how the rising seas (the waves) are destroying Plantain Island, as they did not do in former times. The fact to note is that there is no negative marker \textit{ni} as part of \textit{che}, yet the meaning is clearly negative. The auxiliary \textit{che} is necessary to support the negative, so if the clause were not negated, there would be no auxiliary.

\ea%209
    \label{ex:209}
    Mɛndɛ ma hun bo ma vɛ ni ma muni, tem lan ma che na pɛ shimi Plantiɛ.\\
    \gll mɛn    lɛ    ma    hun  bo    ma    vɛ    ni    ma    muni\\
    water    \textsc{def}  \textsc{ncp}\textsubscript{ma}    come  only  \textsc{ncp}\textsubscript{ma}    slam  and  \textsc{ncp}\textsubscript{ma}    return\\
    \gll tem  lan  ma    che      na        pɛ      simi    Planti    ɛ\\
    time  that  \textsc{ncp}\textsubscript{ma}    \textsc{aux.neg}    \textsc{near.pst}  again    destroy  Plantain  \textsc{def}\\
    \glt ‘When the water comes, it would just slam and return, (at) that time it would not have destroyed Plantain (Island).' (142v Baba Mandela, Fishing: 33--34)
\z


Two less involved examples follow in \REF{ex:210}. In the first sentence Mabel Lohr is describing her midwifery practice and how she is not paid with money. The example in \REF{ex:210b} is instructive because the first negated clause has no marker but the second does (\textit{ni}). Yeabu Bangura characterizes the marketing of her smoked fish.

\ea%210
    \label{ex:210}
    \ea\label{ex:210a} Ŋɔ koŋ gbo we ha che mi paka.\\
    \gll ŋɔ      koŋ    gbo    we      ha    che      mi    paka\\
    \textsc{ncp}\textsubscript{hɔ}   finish    quite    \textsc{emph}    \textsc{3pl}  \textsc{aux.neg}    \textsc{1sg}  pay\\
    \glt ‘That is it, but they don't pay me.' (002a Mabel Lohr, Midwifery: 17)
    \ex\label{ex:210b} Ŋa ŋa wɔŋɡul, yaŋ min ache ŋa wɔŋɡul, ashini prɛs lan.\\
    \gll ŋa    ŋa    wɔŋɡul  ya-ŋ      mi-n      a    che      ŋa    wɔŋɡul\\
    \textsc{3pl}  \textsc{3pl}  sell    \textsc{1sg-emph}  \textsc{1sg-emph}  \textsc{1sg}  \textsc{aux.neg}    \textsc{3pl}  sell\\
    \gll a    shi-ni      prɛs  lan\\
    \textsc{1sg}  know-\textsc{neg}  price  this\\
    \glt ‘They (the children) sell them (the fish), me I don't sell them, I don't know their price.' (184v Fish Smoking Seaport: 199)
\z
\z

I now turn to the tight bond between tense and object pronouns (\textsc{tns-pro}\textsubscript{obj}) mentioned several times above.

\subsection{Tense and object pronouns}
\label{sec:8.2.3}\hypertarget{Toc115517817}{}
Sherbro word order, as stated at the beginning of this chapter, is normally SVO. In Kisi when the verb is compounded, i.e., consists of an auxiliary and a lexical verb, most of what usually comes after the simple verb can be found between the auxiliary and the verb (\citealt{Childs1995}). But the other Bolom languages follow the pattern of Sherbro, allowing only object pronouns to fill that slot (\citealt{Childs2011}, \citealt{Childs2020}). This section describes the situation in Sherbro.

Just as in Mani and Bom-Kim, pronominal objects form a syntactic unit with imperfective tense that cannot be decomposed.\footnote{In Kisi\il{Kisi} the facts are quite a bit more complicated, but the same basic generalization holds (\citealt{Childs1997}, \citealt{Childs2003a}, \citealt{Childs2005}).} In imperatives, tense is marked on the subject (or a quasi-auxiliary such as \textit{kɔ} ‘go'), where the same bond holds. Thus, when tense is marked to the left of the lexical verb and not on the verb itself, object pronouns are adjacent to tense as marked on the subject pronoun or other tense-carrying element. When there is an auxiliary or the negative marker as part of tense, the object pronouns follow those morphemes.

Here I expand on the discussion introduced in the presentation of \textit{{}-i} in \sectref{sec:8.2.1}. I begin with some examples from imperfective constructions. In \REF{ex:211a}, Chernor Ashun, the newly installed Paramount Chief of the Dema Chiefdom, is explaining how the death of a Poro official is handled. It is the syntax of the final clause that is relevant. The object pronoun \textit{mɔ} appears between the subject pronoun \textit{kɔ} (marked for imperfective) and the lexical verb \textit{koiɛ}. In \REF{ex:211b}, (repeated from \REF{ex:204}), the object \textit{la} is between the subject pronoun \textit{ŋa} and the lexical verb \textit{le.}

\ea%211 
\label{ex:211}

    \ea \label{ex:211a}  Nɔpokandɛ bɛ wu, ɡbo laɡbondɛ mɔlɔ ɡbanabom pɔkɛ— pɔk bomdɛ kɔ mɔ koiɛ.\\
    \gll nɔpokan    ɛ    bɛ    wu  ɡbo    laɡbondɛ  mɔ  lɔ    ɡbanabom pɔk  ɛ    pɔk  bom  ɛ    kɔ      mɔ  koiɛ\\
    man      \textsc{def}  just  die  indeed  if        \textsc{2sg}  be    Poro.official land  \textsc{def}  land  big  \textsc{def}  \textsc{ncp}\textsubscript{kɔ}    \textsc{2sg}  take\\
    \glt ‘(When) a man dies, if you are a Poro official, the country— it is the big country that takes you.' (102v Chernor Ashun: 207)

\newpage
    \ex\label{ex:211b}  Ŋa lai le kunɛ haaŋ, ŋa lai le kunɛ tee.\\
    \gll ŋa    la-i        le      kunɛ    haaŋ  ŋa    la-i        le      kunɛ    tee\\
    \textsc{3pl}    \textsc{pro}\textsubscript{indef}{}-\textsc{prt}    remain  inside    long  3\textsc{pl}  \textsc{pro}\textsubscript{indef}{}-\textsc{prt}    remain  inside    until\\
    \glt ‘They continued in this way a long time, they remained this way for a while.' (122a Virginia Lohr, Two Mates: 2)
\z
\z

The example in \REF{ex:212} illustrates imperfective constructions where tense is marked on the subject pronouns. The utterance is characterizing the sacredness of Wong Island, a place where women are not allowed, no boats can approach it, men wear no clothes, etc. What is significant syntactically is that two object pronouns in the first clause appear between the subject pronouns and the verb. In the first clause, the two pronouns, \textit{kɔ} and \textit{lɔ}, appear between the auxiliary \textit{che} and the verb \textit{bɛth} ‘cut down'. Note also that the second pronoun is a locative. In the second part, it is just the pronoun \textit{kɔ} that appears between the subject pronoun \textit{mɔ} and the verb \textit{kɛnthi}.

\ea%212
    \label{ex:212}
    Thɔkɛ fli nche kɔ lɔ bɛth, ŋa wɔɛ mɔ kɔ kɛnthi ɡbɔŋɡɔ landɛ be.\\
    \gll thɔk  lɛ    fli    n    che  kɔ      lɔ      bɛth\\
    tree  \textsc{def}  even  \textsc{2sg}  \textsc{aux}  \textsc{ncp}\textsubscript{kɔ}    \textsc{ncp}\textsubscript{lɔ}    cut.down\\
    \gll ŋa    wɔɛ  mɔ  kɔ      kɛnthi  ɡbɔŋɡɔ  landɛ    be\\
    \textsc{3pl}  say  \textsc{2sg}  \textsc{ncp}\textsubscript{kɔ}    cut.up  forest    that    just\\
    \glt ‘Even (if it's just) a branch that you cut there, they would say you cut up (have damaged) the (sacred) bush.' (187v Wong Island: 10)\footnote{It is considered a serious violation to “cut” the bush.}
\z

The example in \REF{ex:213} also shows two pronouns being close to tense marked on a subject pronoun. The question asks Yusuf Fofana if he is teaching Bolom (a \textit{ma}{}-class noun) to his children.

\ea%213
    \label{ex:213}
    Mɔ ma ha thoŋki?\\
    \gll mɔ  ma    ha    thoŋki\\
    2\textsc{sg}  \textsc{ncp}\textsubscript{ma} \textsc{3pl}  teach\\
    \glt ‘Are you teaching it (Bolom) to them?' (028a Yusuf Fofana: 80)
\z

The examples in \REF{ex:214} show negated imperative and optative constructions, where the same word order obtains (\textsc{tns-pro}\textsubscript{obj}), now with the negative incorporated into the tense complex. The example in \REF{ex:214a} is a negative imperative, where the object \textit{wɔ} is adjacent to the negative marker \textit{ma}. The example in \REF{ex:214b} illustrates how the object pronoun \textit{wɔ} appears before the lexical verb \textit{pɔkɔni} once again close to the carrier of tense, the negative marker (\textit{ma}), in negative optative constructions.

\ea%214
    \label{ex:214}
    \ea\label{ex:214a} Mma wɔ ka fe-m dɛ.\\
    \gll n    ma  wɔ    ka    fe      mi    lɛ\\
    \textsc{2sg}  \textsc{neg}  \textsc{3sg}  give  money  \textsc{1sg}  \textsc{def}\\
    \glt ‘Don't give him my money!' (P67 M: 2)
    \ex\label{ex:214b} Ha lɛ mma wɔ pɔkɔni, wɔ lɛ nɔdwiyɛ.\\
    \gll ha    lɛ    n    ma  wɔ    pɔkɔni  wɔ    lɛ    nɔdwiyɛ\\
    \textsc{opt}  be    \textsc{2sg}  \textsc{neg}  \textsc{3sg}  forget    \textsc{3sg}  be    thief\\
    \glt ‘You shouldn't forget about him, he's a thief!' (P67 TH: 93)
\z
\z

Locative pronouns also move inside, as first shown in \REF{ex:212}. Jalikatu Kumba explains in \REF{ex:215} a game that she and the other children played when she was young. The locative pronoun \textit{lɔ} appears after the subject pronoun \textit{i} before the lexical verb \textit{pɛŋgipɛŋgi}.

\ea%215
    \label{ex:215}
    Inan gballɛ,  ilɔ pɛngipɛngi, ikikkik.\\
    \gll i    nan  gbal  ɛ    i    lɔ    pɛŋgipɛŋgi    i    kikkik\\
    \textsc{1pl}  draw  line  \textsc{def}  \textsc{1pl}  there  jump        \textsc{1pl}  kick\\
    \glt ‘We draw the line, we jump there (and) kick.' (005a Jalikatu B. Kumba: 80)
\z

In \REF{ex:216}, the pronoun \textit{la} appears after the tensed verb \textit{kɔ}, with \textit{kɔ} functioning as a quasi-auxiliary marking imperative or optative, but crucially marked for tense and thus attracting the object pronoun.

\ea%216
    \label{ex:216}
    Ŋkɔ la hini!\\
    \gll n    kɔ    la        hini\\
    2\textsc{sg}  go    \textsc{pro}\textsubscript{indef}    arrange\\
    \glt ‘Go and arrange it!' (P67 H: 54)
\z

Only “true” pronouns are allowed to be close to tense. Even though \textit{yen} is analyzed as an indefinite pronoun in this book, it is definitely not felt to be of the same status as the personal pronouns. In \REF{ex:217}, \textit{mi} appears after the subject pronoun and before the verb but not \textit{yen}. Agnes Simbo is describing how she is inadequately compensated for the teaching that she does at a local school.

\ea%217
    \label{ex:217}
    Mi, pɔ mi ka yen tonton dɛ.\\
    \gll mi        pɛ      mi    ka    yen      tonton  lɛ\\
    mother    \textsc{pro}\textsubscript{indef}  \textsc{1sg}  give  something  small    \textsc{def}\\
    \glt ‘Mummy, they give me a small something.' (007a Agnes J. Simbo: 55)
\z

The example in \REF{ex:218} shows how it is crucial that it is not the direct object but rather the pronoun. Note how the pronoun moves inside and the noun does not, even when the pronoun is not the direct object.

\ea%218
    \label{ex:218}
    Kɔŋgbɔl wɔ lɛ kɔ duk yɛ pə wɔ ku ilellɛ.\\
    \gll kɔŋgbɔl    wɔ    lɛ    kɔ      duk  yɛ      pɛ      wɔ    ku    i-lel        ɛ\\
    heartbeat  \textsc{3sg}  \textsc{def}  \textsc{ncp}\textsubscript{kɔ}    fall  when    \textsc{pro}\textsubscript{indef}  \textsc{3sg}  call  \textsc{ncm}\textsubscript{hɔ}{}-name  \textsc{def}\\
    \glt ‘His heart beats when they call his name.' (P67 K: 193)
\z

The requirement that object pronouns be close to tense overrides the syntax of an idiom. The object of affection, what is loved, is usually between the verb \textit{chɔŋ} ‘offer' and its discontinuous partner \textit{len} ‘thing,' an idiom meaning ‘love'. In \REF{ex:219}, the pronoun \textit{mɔ} has been moved out of that slot to be adjacent to tense.

\ea%219
    \label{ex:219}
    Aŋaɛ   ŋamɔ chɔŋ len.\\
    \gll a-ŋa        ɛ    ŋa    mɔ  chɔŋ  len\\
    \textsc{ncm}\textsubscript{ha}{}-people  \textsc{def}  \textsc{ncp}\textsubscript{ha}  \textsc{2sg}  offer  something\\
    \glt ‘The people will love you.' (018a Suffian Koroma: 72)
\z

A final process that respects the integrity of the \textsc{tns-pro}\textsubscript{obj} syntagm, as does the consecutive marker discussed in \sectref{sec:8.2.1}, is negation. In both examples of \REF{ex:220}, instead of \textsc{neg} \textit{ni} being right after the tensed element, it appears after the object pronoun, \textit{mi} in \REF{ex:220a} and \textit{mi} again in \REF{ex:220b}. The second sentence is repeated from \REF{ex:7} in \sectref{sec:1.7} and expresses the regret of Ansu Kagboro that Sherbro is being lost.

\ea%220
    \label{ex:220}
    \ea \label{ex:220a}  Chɛliɛ   mi tɛn wɛy, ya che kɔn pɔkɔni.\\
    \gll chɛliɛ    mi    thɛn  wɛi  ya    che  kɔ-n      pɔkɔni\\
    arrange  \textsc{1sg}  affair  ugly  1\textsc{sg}  \textsc{fut}  \textsc{ncp}\textsubscript{kɔ}{}-\textsc{neg}  forget\\
    \glt ‘He created a bad situation for me, I shall not forget it.' (P67 T: 30)

\newpage
    \ex \label{ex:220b} Nle kɔ bo mpɔni nwɔk mpika ntuk maɛ, labi la pɛthi lɛ mini.\\
    \gll n    le    kɔ      bo        m    pɔni      n-hɔk          n-pika\\
    you  leave  \textsc{ncp}\textsubscript{kɔ}    completely  you  throw.self  \textsc{ncm}\textsubscript{ma}{}-language  \textsc{ncm}\textsubscript{ma}{}-other\\
    \gll n    tuk  ma    ɛ      labi        la  pɛthilɛ  mi    ni\\
    you  lost  \textsc{ncp}\textsubscript{ma}    \textsc{prt}    that.is.why    it  sweet    \textsc{1sg}  \textsc{neg}\\
    \glt ‘If you leave it and throw yourself into another language, you will lose it, that is why it is not sweet to me.' (094a Ansu Kagboro: 97)
\z
\z

The data presented in this section show the inviolability of the \textsc{tns-pro}\textsubscript{obj} syntagm, replicating a pattern in Mani and Bom-Kim. The structure is typologically unusual, unknown in both African languages and languages of the world and merits further study (Jeff  {Good 2020} p.c., Greg  {Anderson 2020} p.c.). An earlier paper suggested that it was the source for the more general and extensive S-Aux-O-O-V of Kisi rather than contact with Mande languages (\citealt{Childs2017}).

\section{Questions}
\label{sec:8.3}\hypertarget{Toc115517818}{}
In \sectref{sec:3.3.4}, I discussed question words as a class and the semantics of the individual question words. Here I recapitulate some of what I said there but focus primarily on their syntax and introduce yes/no questions and the variety of possible responses.

Questions featuring a question word require a final question particle \textit{{}-a}. Question words appear initially, and there is a gap where the item questioned would appear in the non-questioned equivalent, much as in English. This is a common pattern across Bolom languages. \tabref{tab:syn:43} provides an exhaustive listing of question words in Sherbro (repeated from \tabref{tab:wordcat:17}).

\begin{table}
\caption{\label{tab:syn:43}Interrogative pronouns (repeated from \tabref{tab:wordcat:17})}

\begin{tabular}{ll}
\lsptoprule
hina / ina & ‘who'\\
yɛ, yɛŋ & ‘what'\\
ndɔ & ‘which, what, where'\\
handɔ & ‘which, what'\\
la & ‘what'\\
ŋɔ / hɔ & ‘how, what'\\
wɔ & ‘how many'\\
\lspbottomrule
\end{tabular}
\end{table}

Some examples appear in \REF{ex:221} (see \sectref{sec:3.3.4} for more examples).

\ea%221
    \label{ex:221}
    Examples of question-word constructions\\
    \ea Tɛm landɛ vɛ ŋɔ mɔi  ya?\\
    \gll tɛm  lan    lɛ    vɛ    hɔ          mɔ-i      a\\
    time  that    \textsc{def}  thus  how.much    \textsc{2sg-prt}    \textsc{q}\\
    \glt ‘At that time how old were you?' (002a Mabel Lohr, Midwifery: 12)

    \ex Mɔm apima awɔ ŋa mbia?\\
    \gll mɔ-n      a-puma        a-wɔ            ŋa    n    bi    a\\
    \textsc{2sg-emph}  \textsc{ncm}\textsubscript{ha}{}-children  \textsc{ncm}\textsubscript{ha}{}-how.many    \textsc{3pl}  \textsc{2sg}  have  \textsc{q}\\
    \glt ‘‎How many children do you have?' (093a Alusine Bundu: 58)

    \ex Ina tongiɛ mɔ ŋa tɔnda?\\
    \gll hina  tongiɛ  mɔ  ŋa    tɔn  a\\
    who  teach    \textsc{2sg}  how  sing  \textsc{q}\\
    \glt ‘Who taught you how to sing?' (005a Jalikatu B. Kumba: 62)
\z
\z

When the question word is in situ, i.e. not fronted, no question particle is used as in \REF{ex:222}.

\ea%222
    \label{ex:222}
    \ea Wɔn gbemni ndɔ?\\
    \gll wɔ-n      gbemni    ndɔ\\
    \textsc{3sg-emph}   born      where\\
    \glt ‘She was born where?' (009--10a Lohr \& Mampa: 41)

    \ex Bulɔ kendɛ handɔ?\\
    \gll bulɔ    kendɛ      handɔ\\
    work    similar.to  what\\
    \glt ‘What kind of work? (Lit. 'The work similar to what?') (028a Yusuf Fofana: 17)
\z
\z

When the question word is understood, as it is in high frequency questions, it may be omitted as in \REF{ex:223}.

\ea%223
    \label{ex:223}
    Ilel mɔa?\\
    \gll i-lel        mɔ-a\\
    \textsc{ncm}\textsubscript{hɔ}{}-name  \textsc{2sg}{}-\textsc{q}\\
    \glt ‘What is your name?' (004a Cyril Manley on Walter Hanson: 8)\\
\z

Yes-no questions are signalled solely by rising intonation. Responses are either an elongated nasal [m, n, ŋ] or a front mid vowel such as [e], optionally nasalized. Affirmative answers (‘yes') are characterized by a falling tone of agreement while negative responses are interrupted by a glottal stop [Ɂ] with a low tone on the first part and a high tone on the second part, e.g., [\`{m}Ɂḿ]. Responding negatively to a negative question, e.g., “Did you not eat all the rice?”, has the same segmental possibilities (with no glottal stop) but this time with a tune of HLH (or fall-rise). An affirmative response is the same as to a question without a negative. These forms are common throughout the area.

