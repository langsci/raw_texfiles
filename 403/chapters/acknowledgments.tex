\addchap{\lsAcknowledgementTitle} 




This grammar and dictionary are the output of the project, “Documenting the Sherbro Language and Culture” (2016-2020), which was funded by the Hans Rausing Endangered Languages Documentation Programme, School of Oriental and African Studies (SOAS), University of London (Major Language Documentation Grant: Project for the Documentation of the Sherbro Language and Culture, \# MDP 0316, officially  {September 2016} to  {June 2019} and extended into 2020). All of the materials are archived at the Endangered Language Archive (ELAR), \href{http://www.elar-archive.org}{{www.elar-archive.org}}.

For a project of this magnitude and duration it is impossible to acknowledge and thank everyone who has supported and guided me over the years. Most important, however, my deepest and sincerest appreciation go to the Sherbro people for generously opening their hearts and warmly welcoming me into their lives. I hope that I am able to repay a small portion of my debt to them with the publication of this grammar and dictionary of their language.

The corpus is considerable for an endangered language and could not have been assembled without the backing of the local Sherbro community. Particularly supportive was Madam Doris Lenga-Caulker\is{Caulker} Gbabiyor\is{Lenga-Caulker Gbabiyor, Doris} II, paramount chief of the Kagboro Chiefdom, where the project was based. Not only did she rent us a house built by her daughter but she also provided us with a varied cuisine and cook, the services of her servants, and many other amenities. Her distantly related cousin, Charles Caulker\is{Caulker, Charles}, paramount chief of Bumpeh Chiefdom\is{Bumpeh Chiefdom}, another Sherbro chiefdom just to the north of Kagboro, and Member of Parliament, was also sympathetic to the project and provided many introductions.

Two colleagues, in particular, lent their hearts, souls, and labor to this project and were indispensable to me in completing this work. First, Abdulai Bendu, a native Sherbro speaker and now a student in Linguistics at Fourah Bay College in Freetown, Sierra Leone, was invaluable to this initiative in opening doors to the community, conducting hundreds of interviews, and tirelessly answering my endless questions. Second, Jedd Schrock's meticulous technical and organizational support kept the project on track. He supported the procurement of equipment before fieldwork began and helped with the final archiving of the SLC materials at Portland State University and ELAR-SOAS. He was also responsible for the layout and printing of the first edition of the Sherbro dictionary which was distributed in the Sherbro Community in 2018. I am deeply indebted to these colleagues, without whom this work could not have been accomplished.

Of course, there are many other members of the research team who helped to make this volume possible. Working alongside of Abdulai Bendu in Shenge (Kagboro Chiefdom) were Jalikatu B. Kumba, Virgina Lohr, and Mabel Lohr, along with Pa Yanker, the tribal linguist, who lent the team his linguistic expertise. I am particularly grateful for the backing of Portland State University (PSU) for providing me with the needed institutional and academic assistance. I also benefited from interactions with Allen Wilson, Jubel Brousseau, and Sasha Kraft, all of whom supported me at PSU in various aspects of the Sherbro project. The personnel at Fourah Bay College, a part of University of Sierra Leone, were also helpful to the project: Professor Sahr P. Thomas Gbamanja (GCOR), Acting Deputy Vice Chancellor, and Prince Kenny, Head of Department, Language Studies, Momoh Taziff Koroma†, Senior Lecturer in Linguistics, and his many colleagues. My thanks also to Solomon Gbani in the Department of Geography at Fourah Bay College for helping me reach decisions on place naming conventions. Finally, I also received excellent assistance from SIL\is{SIL} International in the use of FLEx, a data analysis program that organizes data and makes it suitable for use in writing a grammar and producing a dictionary (among other things).

Finally, while the literature on Sherbro is rather slim, I would like to thank the numerous scholars whose previous work in African linguistics paved the way not only for this volume in particular, but for African linguistics and endangered languages in general. Walter J. Pichl\is{Pichl, Walter}, University of Vienna and Fourah Bay College, with the assistance of Charles Walterson-Domingo, did much original, ground-breaking work on Sherbro as well as many languages in West Africa. Numerous scholars have provided me with academic and collegial support. My thanks to Phillip Cunningham at the Amistad Research Center at Tulane University for his assistance in uncovering the stories of Revs. John White\is{White, John} and Barnabas Root\is{Barnabas Root} and the primer written by Rev. \hyperlink{ENREF92}{White (1860}). And special thanks to Arlene Golembiewski of the Sherbro Foundation for making Rev. White\is{White, John}'s \textit{Sherbro and English Book} (1862) available to me. Chris Corcoran has provided me with considerable collegial support over the years suggesting contacts in Sierra Leone and generously making available a preliminary version of her thesis at the University of Chicago, which focuses on the noun class system of Sherbro and is rich in ethnographic detail (Corcoran, In prep)\is{Corcoran, Chris}. My thanks to Friederike Lüpke\is{Lüpke, Frederike}, University of Helsinki, and Jeff Good, University of Buffalo, for truly satisfying collaborations on the broader questions this work addresses.

My apologies if I have inadvertently overlooked acknowledging any individuals who provided me with support in this effort. Any deficiencies remaining are of course entirely my own.

G. Tucker Childs (1948-2021)

(compiled posthumously from his notes)\\


One final section of acknowledgements is included to thank those that Tucker cannot. First, members of the manuscript- preparation team have written each other’s acknowledgments in Tucker’s stead. Jedd Schrock must be thanked for his dedication to the project providing critical continuity to the posthumous manuscript preparation team and for contributing his considerable FLEx database expertise to produce the most up-to-date version of the dictionary possible. Chris Corcoran graciously, painstakingly, and with fastidious attention to detail, reviewed every word, phrase, transcription, and cross-reference in this grammar to ensure the document was ready for production with the highest quality. Finally, without Karen Beaman this book would not exist. She assembled the team to look for unfinished work, tracked down colleagues and publishing contacts, copyedited, proofread, and managed all the technical details of the assembly of grammar and dictionary. It has been her steadfast shepherding of the manuscript though every stage that has brought this publication to fruition.

We also need to thank Language Science Press. The reviewers and proofreaders made a larger than usual contribution to the final form of this project with their willingness to engage the output of the dictionary at a more preliminary stage. Sebastian Nordhoff and the production team provided months and months of support and considerable patience, and finally we are grateful to series editors Adams Bodomo and Firmin Ahoua. Their belief in the project’s value and their willingness to take on the complications of a posthumous publication are ultimately what made this possible. We thank them for all the time and considerable effort entailed in accommodating these circumstances.



