\chapter{Word categories}
\label{ch:3}\label{sec:3}\hypertarget{Toc115517759}{}
Establishing word categories for some words in Sherbro can be challenging, usually across the categories of nouns, verbs, and adjectives, but also across the categories of nouns, locatives, and adpositions, and between (manner) adverbs and ideophones. Part of the reason for such problems is the relative paucity of morphology in the language. There is verbal and nominal morphology to be sure, but sometimes nouns appear without their noun class markers or without an accompanying article, and verbs can also appear without their subjects or in non-finite forms.

For example, the stem \textit{rithi} ‘darkness' (n.), can also be a verb as well as an adjective. In addition, there are a number of derived forms \textit{kilrithi} ‘prison' (lit. ‘darkness house'), \textit{rithilɛhɔl}  ‘dusk' (lit. ‘darkness mouth (opening, beginning)'), \textit{cholrithi} ‘moonless night' (lit. ‘night darkness'). Given this multifunctionality, identifying the part of speech or identifying a morpheme's category within a compound is problematic.

\section{Nouns}
\label{sec:3.1}\hypertarget{Toc115517760}{}
This category includes only nouns and neither (proper) names nor numbers (see \sectref{sec:3.6} \& \sectref{sec:3.7}). Crucial for distinguishing this word category is morphosyntactic behavior, namely, participation in the noun class system, including prefixes on noun stems for most noun classes and, more criterially, agreement markers on dependent elements (see Chapter \ref{ch:5} for a full treatment of the noun class system).

\tabref{tab:wordcat:13} presents the noun classes of Sherbro. In the first column is the pronoun (\textsc{ncp}) used for each class, the way in which each class will be referenced. In the second column are (usually) prefixed noun class markers (\textsc{ncm}s). The next column presents some examples from each class, and the last gives a very rough semantic characterization. Animacy, humanness, and number (singular, plural, mass) are the criteria distinguishing each class. Configuration (long and thin vs. round) may have once played a role, but is no longer important, although there is a liquid class.

\begin{table}
\caption{Sherbro noun classes}
\label{tab:wordcat:13}
\small
\begin{tabularx}{\textwidth}{llQQ}
\lsptoprule
\textsc{ncp} & \textsc{ncm} & Noun & Semantic characterization\\
\midrule
 wɔ & $\emptyset$ (\textit{wɔ} class) & \textit{thumɔɛ} ‘dog,' \textit{ra} ‘green snake,' \textit{nɔ} compounds & animate singulars\\
 ha & a- (\textit{ha} class) & \textit{abolom} ‘Sherbro people,' \textit{athumɔɛ} ‘dogs' (also \textit{si}\,) (\citet{Sumner1921} gives \textit{awok/siwok} ‘slaves') & animate plurals, animal plurals are often also marked with \textit{si}\\
 (ha) & {}-si (\textit{si} class) & \textit{ramsi} ‘clans' (also \textit{tha}), \textit{bɛlsi} ‘rats,' \textit{fansi} ‘cane rats' (also \textit{ha}), \textit{thumɔɛsi} ‘dogs' (also \textit{ha}) & animate plurals, mostly animals; multiple marking\\
 kɔ & $\emptyset$ (\textit{kɔ} class) & \textit{thɔk} ‘tree,' \textit{pɛm} ‘war,' \textit{kɛfɛ} ‘pepper,' \textit{raka} ‘burweed,' \textit{bon} ‘ceremony' & default class, no prefixes, many singulars of plants\\
 ma & n- (\textit{ma} class) & \textit{mɛn} ‘water,' \textit{ŋkuai} ‘palm oil,' \textit{mfan} ‘palm wine,' \textit{nranth} ‘cane rope' & liquids, some plurals, large things\\
 hɔ & i- (\textit{hɔ} class) & \textit{ibithir} ‘bottle,' \textit{ipan} ‘moon,' \textit{ichak} ‘palm fiber,' \textit{rɔ} ‘shield,' \textit{ihɔlɔŋ} ‘breath' & many singulars\\
 tha & thi- (\textit{tha} class) & \textit{thikil} ‘houses,' \textit{thichala} ‘mats,' \textit{thiram} ‘clans,' \textit{thisabo} ‘diseases' & many plurals, plural inanimates (\textit{hɔ} class)\\
 lɔ & li- (\textit{lɔ} class) & \textit{lipal} ‘sun,' \textit{liken} ‘knife,' \textit{limani} ‘respect,' \textit{lithɛm} ‘love,' \textit{litiŋ} ‘by twos' & a small set of nouns, locatives, converts nouns to adverbs\\
\lspbottomrule
\end{tabularx}
\end{table}

Some examples of agreement are given in \REF{ex:46}. \sectref{sec:3.2} on adjectives provides more examples of agreement.

\ea%46
    \label{ex:46}
    Agreement
    \ea Pɛlɛɛ kɔ dinthɛɛ. / kil dinthɛɛ / kil thidinthɛɛ

        \gll pɛlɛ  ɛ    kɔ    dinthɛ  ɛ    kil      dinthɛ  ɛ    kil      thi-dinthɛ    ɛ\\
        rice  \textsc{def}  \textsc{ncp}\textsubscript{kɔ}  white    \textsc{prt}  house    white    \textsc{def}  house    \textsc{ncm}\textsubscript{tha}{}-white  \textsc{def}\\
    \glt ‘The rice is white.' / ‘white house' / ‘white houses' (E13 Albert Yanker, Adj, Lex: 6)

    \newpage
    \ex kil thithiɛ / kil thisaɛ

    \gll kil      thì-thi      ɛ̀      kil      thi-sa      ɛ\\
    house    \textsc{ncm}\textsubscript{tha}{}-black  \textsc{def}    house    \textsc{ncm}\textsubscript{tha}{}-red  \textsc{def}\\
    \glt ‘black houses' / ‘red houses' (E13 Albert Yanker, Adj, Lex: 6)
    \z
    \z

Details of the morphophonology and morphology are provided in Chapter \ref{ch:5}.

\section{Adjectives}
\label{sec:3.2}\hypertarget{Toc115517761}{}
Sherbro has a relatively large number of adjectives compared to related languages, 172 in a lexicon of 4,095 entries. Adjectives are here understood as being words that can enter into attributive constructions within a noun phrase, showing agreement with the noun they modify. Some examples appear in \REF{ex:47}.

\ea%47
    \label{ex:47} 
    Some Sherbro adjectives\\

    \vspace{6pt}
    
\begin{tabular}[t]{lll}
bom & ‘large'\\
ton & ‘small, fine, little'\\
fai & ‘spicy, hot'\\
charaŋ & ‘clean, clear'\\
\end{tabular}
\z

Some examples of agreement are shown in \REF{ex:46} above. I include two more examples in \REF{ex:48}.

\ea%48
    \label{ex:48}
    \ea Ŋ thɔŋklɔ mi yenchɛk asəkəl!\\
    \gll ŋ    thɔŋklɔ  mi    yenchɛk    a-sɛkil\\
      \textsc{2sg}  keep    \textsc{1sg}  fish(pl)    \textsc{ncm}\textsubscript{ha}{}-dry\\
    \glt ‘Keep the dried fish for me!' (E10 Albert Yanker: 25)
    \ex Ŋ ke mən nthenkil   lɛ!\\
    \gll n    ke    mɛn    n-thenkil    ɛ\\
    \textsc{2sg}  look   water    \textsc{ncm}\textsubscript{ma}{}-clear   \textsc{def}\\
    \glt ‘Look how clear the water is!' (P67 TH: 75)
    \z
    \z

The example in \REF{ex:49a} illustrates simple noun-adjective pairs with a prefixed \textsc{ncm} on the adjective showing agreement with the noun. The example in  \REF{ex:49b} shows the adjective \textit{mɔl} ‘sad' in a predicative construction showing agreement by being prefixed with \textit{i-} the \textsc{ncm} of the noun ‘life'. The examples in \REF{ex:49c} show \textit{kith} ‘short' first in citation form, then in an attributive construction, and finally as a predicate, not showing agreement as is the pattern for the \textit{kɔ} class.

\ea%49
    \label{ex:49}
    \ea \label{ex:49a} panth ŋkathil / huɛ thibɔsul\\
    \gll panth   n-kathil        huɛ   thi-bɔsul\\
    work    \textsc{ncm}\textsubscript{ma}{}-hard      day  \textsc{ncm}\textsubscript{tha}{}-cold\\
    \glt ‘hard work' / ‘cold days' (E13 Albert Yanker, Adj, Lex: 10, 13)
    
    \ex  \label{ex:49b}Ihɔlɔŋ hɔ imɔl.\\
    \gll i-hɔlɔŋ    hɔ      i-mɔl\\
    \textsc{ncm}\textsubscript{hɔ}{}-life  \textsc{ncp}\textsubscript{hɔ}    \textsc{ncm}\textsubscript{hɔ}{}-sad\\
    \glt ‘Life is sad.' (P67 H: 77)
    
    \ex \label{ex:49c} kith / thɔk kith lɛ / Thɔk lɛ kɔ kith.\\
    \gll kith    thɔk  kith  lɛ      thɔk  lɛ    kɔ       kith\\
    short    stick  short  \textsc{def}    stick   \textsc{def}  \textsc{ncp}\textsubscript{kɔ}   short\\
    \glt ‘short' / ‘the short stick' / ‘The stick is short.' (P67 K: 158)
    \z
    \z

Agreement is not always shown. For the \textit{kɔ} class, neither nouns nor modifiers are marked (\ref{ex:49c}, \ref{ex:50}).

\ea%50
    \label{ex:50}
    Bɔn bom kɔɛ, pɔ bia lɛ siŋ haaŋ.\\
    \gll bɔn      bom  kɔ      ɛ    pɛ      biya  lɛ  siŋ    haa\\
    ceremony  big  \textsc{ncp}\textsubscript{kɔ}    \textsc{prt}  \textsc{pro}\textsubscript{indef}  have  be  play  long\\
    \glt ‘If it is a big ceremony, they celebrate for a long time.' (016a Albert Yanker: 146)
\z

Quantifiers such as \textit{pum} ‘some' show agreement similar to the general pattern, so in \REF{ex:51}, where \textit{pum} is a modifier, it also shows agreement.

\ea%51
    \label{ex:51}
    \ea Yɛ ŋ kɔ gbo gadin dai, chiɛ mi mmango mpum.\\
    \gll yɛ    ŋ    kɔ    gbo  gadin    ay    chiɛ  mi    n-mango        n-pum\\
    when  \textsc{2sg}  go    just  garden  in    bring  \textsc{1sg}  \textsc{ncm}\textsubscript{ma}{}-mango    \textsc{ncm}\textsubscript{ma}-some\\
    \glt ‘When you go to the garden, bring me some mangoes.' (P67 P: 239)

    \ex nthɔk mpum\\
    \gll n-thɔk    n-pum\\
    \textsc{ncm}\textsubscript{ma}{}-tree  \textsc{ncm}\textsubscript{ma}{}-some\\
    \glt ‘some trees' (E13 Albert Yanker, Adj, Lex: 18)

    \ex Ŋgbèmàŋ mpùm ma teŋ.\\
    \gll n-gbemaŋ    n-pum      ma    teŋ\\
    \textsc{ncm}\textsubscript{ma}{}-fruit    \textsc{ncm}\textsubscript{ma}{}-some  \textsc{ncp}\textsubscript{ma} sour\\
    \glt ‘Some fruits are sour.' (E13 Albert Yanker, Adj, Lex: 19)
    \z
    \z

Bodily states or conditions can be expressed through predicative constructives with a verb meaning ‘have,' ‘feel,' or as predications with a noun, as in \REF{ex:52}.\footnote{The verb is actually a complex derived form: \textit{the} ‘hear, sense' + the verb extension \textit{{}-ni} which has reflexive meaning (see \sectref{sec:6.2}).}

\ea%52
    \label{ex:52}
    \ea A bi nak.\\
    \gll a    bi    nak\\
      \textsc{1sg}  have  illness\\
      \glt ‘I am sick.' (P67 N: 8)

    \ex Kulmmən hɔ mi.\\
    \gll kul-n-mɛn        hɔ      mi\\
    drink-\textsc{ncm}\textsubscript{ma}{}-water  \textsc{ncp}\textsubscript{hɔ}    \textsc{1sg}\\
    \glt ‘I am thirsty.' (P67 K: 261)

    \ex Wandaɛ bɛ yɛ wɔ ko theni ndikɛ …\\
    \gll wante    ɛ    bɛ    yɛ      wɔ    koŋ    theni    n-dik          ɛ\\
      girl    \textsc{def}  self  when    \textsc{3sg}  \textsc{pfv}    feel    \textsc{ncm}\textsubscript{ma}{}-hunger    \textsc{def}\\
    \glt ‘When the girl felt hungry …' (122a Virginia Lohr, Two Mates: 10)

    \ex Nhɔbɛ i le ma hɔ haŋ wɔyɛ pi ima lɔ be nwɔk pika gbi, acheŋ ke gbi.\\
    \gll nhɔbɛ    i    le    ma    hɔɛ    haŋ\\
      even.if  \textsc{1pl}  stay  \textsc{ncp}\textsubscript{ma}    speak    until\\
    \gll wɔ  i  ɛ    pi    ma    lɔ      be    n-hɔk          pika  gbi\\
      day  \textsc{def}  be.dark  \textsc{ncp}\textsubscript{ma}    there    \textsc{neg}  \textsc{ncm}\textsubscript{ma}{}-language  other  all\\
    \gll a  che-ni    ker  gbi\\
    1\textsc{sg}  \textsc{aux-neg}  tire  all\\
\glt ‘Even if we continue speaking it until nightfall using no other language, I would not get tired.' (093a Alusine Bundu: 84) (repeated from the opening to Chapter \ref{ch:1})
\z
\z

\section{Pronouns}
\label{sec:3.3}\hypertarget{Toc115517762}{}
Sherbro possesses a great variety of pronouns: personal pronouns, noun class pronouns, demonstrative pronouns, interrogative pronouns, and impersonal pronouns. I discuss each category in the sections that follow. 

\subsection{Personal Pronouns}
\label{sec:3.3.1}\hypertarget{Toc115517763}{}

There are some indications of case in the pronominal system across the three cases of nominative, possessive, and objective. A few pronouns have variant forms as indicated in \tabref{tab:wordcat:14}, typically phonologically conditioned by preceding syllable structure.

The possessive pronouns are identical to the subject pronouns except in the first-person singular.

One oddity to the system is the \textsc{2sg} object pronoun \textit{hɔm (cf.} 2\textsc{sg} subject pronoun \textit{mɔ)}. Metathesis is not unheard of in the language. It is also the case that [h] alternations with $\emptyset$ (zero) at various places as well. Metathesis supplemented by an epenthetic [h] would motivate the form (see discussion of /h/ in \sectref{sec:2.1.2}).

Other than the \textsc{2sg} oddity, the pronouns are constant across cases except for the first person (\textit{a, ya} vs. \textit{mi}). The allomorphs of the \textsc{1sg} possessive and object pronouns are phonologically conditioned. The variants \textit{yi} and \textit{si} for the \textsc{1pl} possessive pronoun are dialectal. Tones are marked on the \textsc{2pl} and \textsc{3pl} pronouns since the form a tonal minimal pair. This follows the practice of the Sherbro Literacy Committee, mentioned in \sectref{sec:1.9}.

\begin{table}
\caption{Sherbro personal pronouns}
\label{tab:wordcat:14}

\begin{tabularx}{\textwidth}{XXXXXX}
\lsptoprule
\multicolumn{2}{c}{Subject} & \multicolumn{2}{c}{Possessive} & \multicolumn{2}{c}{Object}\\
\cmidrule(lr){1-2}\cmidrule(lr){3-4}\cmidrule(lr){5-6}
Sg & Pl & Sg & Pl & Sg & Pl\\
\midrule
a, ya & y\`{i} (ɲì) & mi, m & yi, si & mi, m & yi\\
mɔ & ŋá & mɔ & ŋá & hɔm & ŋá\\
wɔ/$\emptyset$ & ŋà & wɔ & ŋà & wɔ & ŋà\\
\lspbottomrule
\end{tabularx}
\end{table}

I also need to make a few comments on the distribution of the pronouns. The third-person singular pronoun is often absent in subject position, especially when there is no expressed NP.

In \REF{ex:53} the subject pronoun \textit{wɔ} does not appear. The subject is well understood from the discourse context. In both cases the speakers are talking about their fathers, and the remarks form part of a larger discourse.

\ea%53
    \label{ex:53}
    \ea Gbem yi hina waŋ.\\
    \gll ($\emptyset$)  gbem    yi    hi-n      a-waŋ\\
      (3\textsc{sg})  bear    \textsc{1pl}  \textsc{1pl-Emph}  \textsc{ncm}\textsubscript{ha}{}-ten\\
    \glt ‘(She) gave birth to ten of us.' (090a Saidu Netteh: 73)

    \ex Kabi ama ayɔl.\\
    \gll ($\emptyset$)  ka      bi    a-maa      a-hiɔl\\
    (3\textsc{sg})  \textsc{rem.pst}  have  \textsc{ncm}\textsubscript{ha}\textsc{{}-}wife    \textsc{ncm}\textsubscript{ha}\textsc{{}-}four\\
    \glt ‘(He) had four wives.' (093a Alusine Bundu: 44)
    \z
    \z

Less commonly the first-person pronoun is omitted in subject position.

Another development is that the subject pronouns are felt to be parts of the verb and are transcribed as such by trained native speakers. These pronouns show some phonological links with the following verb, e.g., the \textsc{2sg} pronoun /n/ assimilating to the following consonant, losing its syllabicity, and becoming a homorganic prenasalized stop (see \sectref{sec:2.4} for some details and examples). Possessive pronouns are not often affixed to the nouns they possess, as would an attributive adjective but rather appear as object pronouns forming a syntactic unit with tense (see the extensive discussion in \sectref{sec:8.2.3}).

Most of the morphophonology of the personal pronouns involves realignment of syllable structure, although there is the unusual case of metathesis\is{metathesis} for the 2\textsc{sg} pronoun just described.

The emphatic pronouns, used typically as topicalized subjects, are almost all formed in the same way, by the addition of a final nasal [ŋ] or [n] to the CV form of the subject pronoun. Thus, the emphatic\is{emphatic} form of the first-person singular pronoun would be \textit{yaŋ}, as in \REF{ex:54a}. In \REF{ex:54b}, it is the third singular pronoun \textit{wɔ}, also fronted, but in \REF{ex:54c}, the emphasized pronoun \textit{ŋa} appears in final position.

\ea%54
    \label{ex:54}
    Use of the emphatic\is{“emphatic”!} pronouns\footnotemark

    \ea\label{ex:54a} Yaŋ kən.\\
    \gll ya-ŋ      kɛn\\
    \textsc{1sg-emph}  alone\\
    \glt ‘I am alone.' (E04 Abdulai Bendu: 4)

    \ex\label{ex:54b} Wɔn wɛ kɔysunɔ lɛ chaŋ atɛma wɔ lɛ.\\
    \gll wɔ-n      wɔ    kɔysunɔ     lɛ    chaŋ     atɛma        wɔ    lɛ\\
    \textsc{3sg-emph}  \textsc{3sg}  sorcerer    \textsc{def}  pass    \textsc{ncm}\textsubscript{ha}{}-mate    \textsc{3sg}   \textsc{def}\\
    \glt ‘He himself was the greatest sorcerer among his peers.' (P67 B: 234)

    \ex\label{ex:54c} Mɔ lɔ bɔnth apuma mɔɛ, han gbi.\\
    \gll mɔ  lɔ    bɔnth    a-puma      mɔ  ɛ    ŋa-n      gbi\\
    \textsc{2sg}  there  find    \textsc{ncm}\textsubscript{ha}{}-child    \textsc{2sg}   \textsc{def}  \textsc{3pl-emph}  all\\
    \glt ‘You will find your children there, all of them.' (P67 B: 169)

    \footnotetext{The English translations are probably more emphatic\is{emphatic!} in English than they would be to Sherbro speakers.}
\z
\z

The one exception is \textsc{2sg} emphatic\is{“emphatic”!} form, namely, [mɔm].

There are variations in the morphosyntax: sometimes the emphasized pronoun is preceded by the nasal, as in \REF{ex:55}.

\newpage
\ea%55
    \label{ex:55}
    Nha hun ha cheɛ alema wɔ lɛ.\\
    \gll n-ha      hun    ha    cheɛ  a-lem          a-wɔ        ɛ\\
    \textsc{emph-3pl}  come    to    be    \textsc{ncm}\textsubscript{ha}{}-follower  \textsc{ncm}\textsubscript{ha}{}-\textsc{3sg}    \textsc{def}\\
    \glt ‘They came to be his followers.' (P67 L: 74)
\z


The first-person object and possessive pronoun /mi/ is often reduced to [m] when the preceding word ends in a vowel, especially in structures of high frequency (a verb in the case of the object pronoun and a family relation, for example, in the case of the possessive pronoun). Some examples appear in \REF{ex:56}.

\ea%56 
    \label{ex:56}
    Reduction of \textit{mi} after a vowel-final verb\\
\ea Object \textit{mi}:\\     ka mi $\xrightarrow{}$ kam \textit{‘give me'}

\ex Possessive \textit{mi}:\\
ba mi $\xrightarrow{}$ bam \textit{‘my father'}
\z
\z

\subsection{Noun Class Pronouns (\textsc{ncp}s)}
\label{sec:3.3.2}\hypertarget{Toc115517764}{}
Sherbro noun class pronouns are given in the first column of \tabref{tab:wordcat:15}. The \textit{wɔ} and \textit{ha} class pronouns are the same as the \textsc{3sg} and \textsc{3pl} personal pronouns. All pronouns can be used anaphorically and as relative pronouns.

\begin{table}
\caption{\label{tab:wordcat:15}Sherbro noun classes (repeated from \tabref{tab:wordcat:13})}
\small
\begin{tabularx}{\textwidth}{llQQ}
\lsptoprule
\textsc{ncp} & \textsc{ncm} & Noun & Semantic characterization\\
\midrule
 wɔ & $\emptyset$ (\textit{wɔ} class) & \textit{thumɔɛ} ‘dog,' \textit{ra} ‘green snake,' \textit{nɔ} compounds & animate singulars\\
 ha & a- (\textit{ha} class) & \textit{abolom} ‘Sherbro people,' \textit{athumɔɛ} ‘dogs' (also \textit{si}\,) (\citet{Sumner1921} gives \textit{awok/siwok} ‘slaves') & animate plurals, animal plurals are often also marked with \textit{si}\\
 (ha) & {}-si (\textit{si} class) & \textit{ramsi} ‘clans' (also \textit{tha}), \textit{bɛlsi} ‘rats,' \textit{fansi} ‘cane rats' (also \textit{ha}), \textit{thumɔɛsi} ‘dogs' (also \textit{ha}) & animate plurals, mostly animals; multiple marking\\
 kɔ & $\emptyset$ (\textit{kɔ} class) & \textit{thɔk} ‘tree,' \textit{pɛm} ‘war,' \textit{kɛfɛ} ‘pepper,' \textit{raka} ‘burweed,' \textit{bon} ‘ceremony' & default class, no prefixes, many singulars of plants\\
 ma & n- (\textit{ma} class) & \textit{mɛn} ‘water,' \textit{ŋkuai} ‘palm oil,' \textit{mfan} ‘palm wine,' \textit{nranth} ‘cane rope' & liquids, some plurals, large things\\
 hɔ & i- (\textit{hɔ} class) & \textit{ibithir} ‘bottle,' \textit{ipan} ‘moon,' \textit{ichak} ‘palm fiber,' \textit{rɔ} ‘shield,' \textit{ihɔlɔŋ} ‘breath' & many singulars\\
 tha & thi- (\textit{tha} class) & \textit{thikil} ‘houses,' \textit{thichala} ‘mats,' \textit{thiram} ‘clans,' \textit{thisabo} ‘diseases' & many plurals, plural inanimates (\textit{hɔ} class)\\
 lɔ & li- (\textit{lɔ} class) & \textit{lipal} ‘sun,' \textit{liken} ‘knife,' \textit{limani} ‘respect,' \textit{lithɛm} ‘love,' \textit{litiŋ} ‘by twos' & a small set of nouns, locatives, converts nouns to adverbs\\
\lspbottomrule
\end{tabularx}
\end{table}

\subsection{Demonstratives}
\label{sec:3.3.3}\hypertarget{Toc115517765}{}
Sherbro has abundant resources for expressing deixis. The sharp distinction between proximal and distal is not always clear and may reference discourse factors rather than physical ones. I will maintain the distinction in the following discussion but warn the reader than in actual discourse their interpretation is more complex.

There is no difference between demonstrative pronouns and their adjectival counterparts. The former have the full distribution of any pronoun, and the latter follow the noun they modify, as do all attributives. I begin with proximal demonstratives, then turn to the distal ones.

The examples in \REF{ex:57} show the variety of the ways in which the two proximal demonstratives can be used. There is no agreement shown on either \textit{lo} or \textit{ki}, neither in class nor number. The example in \REF{ex:57a} shows the simple use of \textit{lo} [do] and the example in \REF{ex:57b} shows the simple use of \textit{ki}. In \REF{ex:57c} the two are used separately in the same sentence, and in \REF{ex:57d} the two are used together. The use of several demonstratives in Sherbro is not uncommon.

\newpage

\ea%57
    \label{ex:57}
    Proximal demonstratives \textit{lo} ‘this' and \textit{ki} ‘this'
    \ea\label{ex:57a} Nɛn do ŋɔ ŋa nɛnthiwaŋnimɛn dɛ.\\
    \gll nɛn  lo    hɔ      ŋaa    nɛn  thi-waŋ    ni    mɛn  ɛ\\
    year  this  \textsc{ncp}\textsubscript{hɔ}    make    year  \textsc{ncm}\textsubscript{tha}{}-ten  and  five  \textsc{def}\\
    \glt ‘This year makes fifteen years.' (017a Boima Samba: 58)

    \ex\label{ex:57b} Ahun yi nɔmaɛ ki ŋa lemɛ mi jali wɔ atokɛ.\\
    \gll a    hun    yi    nɔmaa  ɛ    ki    ŋa  lemɛ    mi    ja        li-wɔ      atok  ɛ\\
    \textsc{1sg}  come    ask  woman  \textsc{def}  this  to  explain  \textsc{1sg}  matter  \textsc{ncm}\textsubscript{lɔ}{}-\textsc{3sg}  about  \textsc{prt}\\
    \glt ‘I am about to ask this woman about herself.' (007a Agnes J. Simbo: 2)
    
    \ex\label{ex:57c} Ŋa  ni  lamgbantho ki ŋa   chalao wɛ, ŋaŋa gbem apumma mɛn do wɛ?\\
    \gll ŋá    ni    langban-tho    ki    ŋá      chala    o      wɛ\\
    \textsc{2pl}  with  man-?    this  \textsc{2pl}    sit      \textsc{emph}    \textsc{prt}\\
    \gll ŋá-n      ŋá    gbem    a-pum        a-mɛn    lo      wɛ\\
    \textsc{2pl}{}-\textsc{emph}  \textsc{2pl}  bear    \textsc{ncm}\textsubscript{ha}{}-children  \textsc{ncm}\textsubscript{ha}{}-five  these    \textsc{prt}\\
    \glt ‘You (pl.) and this man you're living with, are you the ones that have these five (children)?' (017a Boima Samba: 63)

    \ex\label{ex:57d} Huɛɛ ŋɔ ken gbo, Braima wɔ le kɔ lɛliɛ mpɛl lo ki pɛiŋ.\\
    \gll huɛ  ɛ    hɔ      ken    gbo\\
    day  \textsc{def}  \textsc{ncp}\textsubscript{hɔ}    break    just\\
    \gll Braima  wɔ    le    kɔ    lɛliɛ    n-pɛl      loki    pɛiŋ\\
    Braima  \textsc{3sg}  first  go    examine  \textsc{ncm}\textsubscript{ma}{}-net  these    before\\
    \glt ‘Just as day breaks, Braima first goes to inspect these fishing\is{“fishing”!} lines' (124aw Yanker, Boy Lost at Sea: 39)
\z
\z

There is, however, agreement with the distal demonstrative \textit{nɛ}, which is prefixed by the agreeing noun class pronoun, just as is the case with closely related languages (\tabref{tab:wordcat:16}).

\begin{table}
\caption{\label{tab:wordcat:16}Distal demonstratives}

\begin{tabular}{lll}
\lsptoprule
\textsc{ncp} & Demonstrative & Gloss\\
\midrule
wɔ & wɔnɛ & ‘that person'\\
ha & hanɛ & ‘those people'\\
kɔ & kɔnɛ & ‘that'\\
hɔ & hɔnɛ & ‘that'\\
lɔ & lɔnɛ & ‘that'\\
ma & manɛ & ‘that' / ‘those'\\
tha & thanɛ & ‘that' / ‘those'\\
\lspbottomrule
\end{tabular}
\end{table}

There is also the suggestive minimal pair \textit{lanɔ} ‘this (affair, business, matter)' and \textit{lanɛ} ‘that, etc.' \textit{La(n)} is the indefinite pronoun ‘it'. Perhaps \textit{{}-nɔ} was once more productive as the near demonstrative but has been replaced by the invariant (non-agreeing) \textit{lo} and \textit{ki}. In our own data \textit{lanɛ} was ‘that or this (one)' and \textit{lanɛ ki} was used for ‘this (one)'. This usage is reminiscent of other compound demonstratives with \textit{ki} and \textit{ko}. \textit{Ko} has an independent status as an adposition and as a morpheme in many locative expressions (see \sectref{sec:3.8}).

\ea%58
    \label{ex:58} Compound demonstratives\\
    wɔnɛ ki ‘this one'\\      
    hanɛ ki ‘these ones'\\
    wɔnɛ ko ‘that one'\\
    hanɛ ko ‘those ones'
\z

\textit{Ko} is also used in opposition to the locative \textit{ka}, usually translated as ‘here' and used as a proximal demonstrative after a noun \REF{ex:59a}, with \textit{ko} as its opposite, the distal ‘that' \REF{ex:59b}.

\ea%59
    \label{ex:59}
    \ea \label{ex:59a}Wɔ hunɛ lelka.\\                   
    \gll wɔ    hunɛ  lel    ka \\               
    \textsc{3sg}  come  side  this\\             
    \glt ‘He came to this side.' (P67 L: 67)
    
    \ex \label{ex:59b}  A kɔ lelko.\\
    \gll a    kɔ    lel    ko\\
     \textsc{1sg}  go   side  that\\
    \glt ‘I went to that (other) side.' (P67 L: 68)
\z
\z

\subsection{Interrogatives}
 \hypertarget{Toc115517766}{}\label{sec:3.3.4}
The interrogative pronouns are more straightforward than the demonstratives in their usage and meaning. They generally appear initially in a question, though they can remain \textit{in situ}. \textsc{wh}{}-like questions all end with the interrogative particle (\textsc{q}) realized as \textit{a}, except when the question word is in final position, as in Sumner\is{Sumner, A. T.!}'s examples in \REF{ex:63} and our own examples in \REF{ex:64b} and \REF{ex:65b}.

\begin{table}
\caption{\label{tab:wordcat:17}Interrogative pronouns}



\begin{tabular}{ll}
\lsptoprule
Q Pronouns & Gloss\\
\midrule
ina / hina & ‘who'\\
yɛ / yɛŋ & ‘what'\\
handɔ & ‘which, what'\\
ndɔ & ‘when, what, where'\\
la & ‘what'\\
ŋɔ / hɔ & ‘how, what'\\
wɔ & ‘how many'\\
\lspbottomrule
\end{tabular}
\end{table}

The interrogative \textit{hina} (and \textit{ina}) ‘who?' can be pluralized, as in \REF{ex:60c}, by prefixing the noun-class marker \textit{a-} followed by the plural pronoun \textit{ha}, both from the \textit{ha} class (animate plurals).

\ea%60
\label{bkm:who:60} \label{ex:60}
\textit{Ina} / \textit{Hina} ‘who' (sg/pl)
\ea\label{bkm:who:60a}  Ina mɔ ra ichɛka; mɔm fili mɔmɔ yɛthi gbathowɛ ɔ yɛŋ?\\
    \gll ina  mɔ  ra      i-chɛk      a\\
    who  \textsc{2sg}  brush    \textsc{ncm}\textsubscript{hɔ}{}-farm    \textsc{q}\\
    \gll mɔ-m      fili      mɔ-n      mɔ  yɛthi    gbatho  ɛ    ɔ  yɛŋ\\
    \textsc{2sg-}\textsc{emph}  really    \textsc{2sg}-\textsc{emph}  \textsc{2sg}  hold    cutlass  \textsc{def}  or  what\\
    \glt ‘Who does the brushing for you; do you really hold the cutlass yourself or what?' (017a Boima Samba: 53)

    \ex\label{bkm:who:60b}  Ina lɔ ba mɔa?\\
    \gll hina  lɔ    ba    mɔ  a\\
    who  \textsc{cop}  father    \textsc{2sg}  \textsc{q}\\
    \glt ‘Who is your father?' (004a Cyril Manley on Walter Hanson: 12)

    \ex\label{bkm:who:60c}\label{ex:60c}  Ahina ŋa chan shi theli mbolomdɛ Shenge\is{“Shenge”!} ka?\\
    \gll a-hina    ha      chan  si    theli  n-bolom      ɛ    Shenge\is{“Shenge”!}  ka\\
    \textsc{ncm}\textsubscript{ha}{}-who  \textsc{ncp}\textsubscript{ha}    pass  know  speak  \textsc{ncm}\textsubscript{ma}{}-Bolom  \textsc{def}  Shenge\is{“Shenge”!} here\\
    \glt ‘Who (pl) knows how to speak Sherbro best in Shenge\is{“Shenge”!} here?' (009--10a Lohr \& Mampa: 100)
    \z
    \z
    

The question words \textit{yɛ} and \textit{yɛŋ} are identical in their function, asking ‘what?' comparable to ‘who?' but used for inanimates, as in \REF{ex:61}.

\ea%61
    \label{ex:61} \textit{Yɛ} / \textit{Yɛŋ} ‘what' 
    \ea  Yɛ wɔ kache ŋaa?\\
    \gll yɛ    wɔ    ka      che  ŋaa  a\\
    what  \textsc{3sg}  \textsc{rem.pst}  \textsc{aux}  do    \textsc{q}\\
    \glt ‘What did he used to do?' (007a Agnes J. Simbo: 18)

    \ex Yɛŋ ni yɛŋ ŋɔ mɔ bɛ ichɛkɛ vɛ kunɛa?\\
    \gll yɛŋ  ni    yɛŋ  ŋɔ      mɔ  bɛ    i-chɛk      ɛ    vɛ    kunɛ    a\\
    what  and  what  \textsc{ncp}\textsubscript{hɔ}    \textsc{2sg}  put  \textsc{ncm}\textsubscript{hɔ}{}-farm    \textsc{def}  so    inside    \textsc{q}\\
    \glt ‘What and what do you plant on your farm?' (017a Boima Samba: 49)
\z
\z

\newpage
The interrogative \textit{ndɔ} is used more widely and more variously than \textit{handɔ} but has much the same meaning as in \REF{ex:62}. With \textit{handɔ}, a set of possibilities is already established; the question is, which one of the set?

\ea%62
    \label{ex:62}
    \textit{Handɔ} ‘which, what' 
    \ea Mbolomdɛ, Plantɛn ka lɔ mɔi kiɛ, man ni  nthemdɛ handɔ mapɔ chaŋ thelia?\\
    \gll n-bolom      dɛ     Plantɛn  ka    lɔ       mɔ-i       ki-ɛ\\
    \textsc{ncm}\textsubscript{ma}{}-Bolom  \textsc{def}  Plantain  here  \textsc{ncp}\textsubscript{lɔ}    \textsc{2sg}{}-\textsc{prt}    this-\textsc{prt}\\
    \gll ma-n        ni    n-them        dɛ    handɔ    ma    pɛ      chaŋ  theli    a\\
    \textsc{ncm}\textsubscript{ma}{}-\textsc{emph}  and  \textsc{ncm}\textsubscript{ma}{}-Themne  \textsc{def}  which  \textsc{ncp}\textsubscript{ma}   \textsc{pro}\textsubscript{indef}  pass  speak    \textsc{q}\\
    \glt ‘Sherbro, on Plantain (Island) here where you are, Bolom or Themne which do they speak more?' (029a Biah Heni: 65)

    \ex Mi, yɛ laowɛ, mpanthɛ handɔ vili ma mɔ kunɛa?\\
    \gll mi        yɛ    laowɛ    n-panth      ɛ    handɔ    vili    ma    mɔ  kunɛ    a\\
    mother    how  as.it.is  \textsc{ncm}\textsubscript{ma}{}-work  \textsc{def}  which  really    \textsc{ncp}\textsubscript{ma}   \textsc{2sg}  inside    \textsc{q}\\
    \glt ‘Mummy, as it is now, what work are you most involved in?' (017a Boima Samba: 47)
\z
\z

\ea%63
    \label{ex:63} \textit{Handɔ} ‘what' from \citet{Sumner1921}\is{Sumner, A. T.!}\\
    
    \ea
    \gll Nɔ-lɛ handɔ?\\                    
    person-\textsc{def} what\\                
    \glt ‘What person?                 
    \ex
    \gll Anya-lɛ handɔ?\\
    persons-\textsc{def} what\\
    \glt ‘What people?'

    \ex
    \gll Sɔk-ɛ handɔ?\\       
    fowl-\textsc{def} what\\
    \glt ‘What fowl?'                    

    \ex
    \gll Sɔ-si-ɛ handɔ?\\
    fowl-\textsc{ncm}\textsubscript{si}{}-\textsc{def} what\\
    \glt ‘What fowl (pl)?'
    
    \z
    \z

The examples of \textit{ndɔ} in \REF{ex:64} show its more varied uses.

\newpage

\ea%64
    \label{ex:64}
   \textit{Ndɔ} ‘when, what'
    \ea
    \label{ex:64a}Tɛm ndɔ ŋɔ ntipɛ   gbemia?\\
    \gll tɛm  ndɔ    ŋɔ      n    tipɛ    gbemi  a\\
    time  which  \textsc{ncp}\textsubscript{hɔ}    \textsc{2sg}  begin    deliver  \textsc{q}\\
    \glt ‘When (what time) did you start delivering?' (002a Mabel Lohr, Midwifery: 10)

    \ex 
    \label{ex:64b}Yamɔ   wɔ tɔm ndɔ?\\
    \gll yaa    mɔ  wɔ    tɔm    ndɔ\\
    mother  \textsc{2sg}  \textsc{3sg}  number  which\\
    \glt ‘Your mother was what number (wife)?' (093a Alusine Bundu: 45)

 
    \z
    \z

\ea%65
    \label{ex:65}
    \textit{Ndɔ} ‘where'

    \ea 
    \label{ex:65a} Ndɔ mɔ mɛkɛnia?\\
    \gll ndɔ    mɔ  mɛkɛni  a\\
    where  \textsc{2sg}  end    \textsc{q}\\
    \glt ‘Where did you stop?'\footnotemark{} (007a Agnes J. Simbo: 48)
    \footnotetext{Meaning, what class/grade in school did she finish?}
    \ex
    \label{ex:65b}Wɔn gbemni ndɔ?\\
    \gll wɔ{}-n      gbemni  ndɔ\\
    \textsc{3sg}{}-\textsc{emph}  born    where\\
    \glt ‘She was born where?' (009--10a Lohr \& Mampa: 40)    
    \z
    \z

The functions of the form \textit{la} are myriad. In addition to being something of an indefinite pronoun ‘it,' the word can function also as a connective (in the sense used in the Bantu literature), a relative pronoun or a conjunction. Several examples of how it can be used as an interrogative are presented in \REF{ex:66}.

\ea%66
    \label{ex:66} \textit{La} ‘what'
    \ea Ŋa hɔɛ, “La taalaŋgba ki wɔ mama?"\\
    \gll ha    hɔɛ  la    taalaŋgba    ki    wɔ    mam    a\\
    3\textsc{pl}  say  what  young.man    this  \textsc{3sg}  laugh    \textsc{q}\\
    \glt ‘They say, “What is this young man\footnotemark{} laughing about?”' (123aw Yanker, Rat Wife: 130)
\footnotetext{The young man is Kain Tasso who was attending his mother-in-law's funeral (see Appendix \ref{app:d}).}
\newpage
    \ex La nka che ŋa labi pɔ ka che mɔ bundɛa?\\
    \gll la    n    ka      che  ŋaa  labi  pɛ      ka      che  mɔ  bundɛ  a\\
    what  you  \textsc{rem.pst}  \textsc{prog}  do    why  \textsc{pro}\textsubscript{indef}  \textsc{rem.pst}  \textsc{prog}  \textsc{2sg}  beat    \textsc{q}\\
    \glt‘What did you do that you were beaten?' (009--10a Lohr \& Mampa: 220)

    \ex Mɔm, la nka cheni ŋaa?\\
    \gll mɔ-m      la    n    ka      che  ni    haa  a\\
    \textsc{2sg-emph}  what  \textsc{2sg}  \textsc{rem.pst}  \textsc{aux}  now  do    \textsc{q}\\
    \glt ‘You, what have you been doing?' (004a Cyril Manley on Walter Hanson: 45)
\z
\z

Questions are given more treatment in \sectref{sec:8.3}.

\subsection{Other pronouns: locative, impersonal, indefinite} \label{sec:3.3.5}
\hypertarget{Toc115517767}{}
In the “other” category are included the locative pronoun \textit{lɔ} ‘there, where' and the indefinite and impersonal pronouns. The pronoun \textit{lɔ} is identical to the \textit{li}{}-class pronoun but differs in its semantics, always referring to a place or location rather than a noun. In \REF{ex:67a}, the pronoun is used anaphorically to refer to a place already mentioned in the discourse. In \REF{ex:67b}, it is used as a relative pronoun ‘where.'

\ea%67
    \label{ex:67}
   \textit{Lɔ} ‘there, where'

  \ea \label{ex:67a} Ika chelɔ mpaŋ bul.\\
  \gll I    ka      che  lɔ    n-paŋ        bul\\
  \textsc{1pl}  \textsc{rem.pst}  be    there  \textsc{ncm}\textsubscript{ma}{}-month  one\\
  \glt ‘We were there for one month.' (002a Mabel Lohr, Midwifery: 35.1)

  \ex \label{ex:67b} Pɔko, nshi ko lɔ kɔ tipɛ haŋ ko lɔ kɔ ko mɛkindɛ?\\
  \gll pɔko    n    si      ko    lɔ      kɔ      tipɛ\\
  country  \textsc{2sg}  know    to    where  \textsc{ncp}\textsubscript{kɔ}    begin\\
  \gll haŋ  ko    lɔ      kɔ      ko    mɛkin  ɛ\\
  until  to    where  \textsc{ncp}\textsubscript{kɔ}    to    end    \textsc{def}\\
  \glt ‘The country, do you know where it starts and where it ends?' (102v Chernor Ashun: 57)
\z
\z

Relative pronouns, identical to personal and noun class pronouns, are treated in \sectref{sec:9.3}. The reflexive function is conveyed by a verbal suffix \textit{–ni} discussed in \sectref{sec:6.2}. The suffixed emphatic\is{“emphatic”!} particle \textit{–n} and the particle \textit{bɛ}, sometimes in conjunction as in \REF{ex:68}, emphasize a noun or pronoun and though translated ‘self' have no reflexive function.

\ea%68
  \label{ex:68} Locative \textit{lɔ}
  
  Wɔ lɔ Nyambako, wɔn bɛ ko ritaya, yɛlaio wɛ.\\
  \gll wɔ    lɔ    nyamba    ko    wɔ{}-n      bɛ    ko    ritaya    yɛ  laio  wɛ\\
\textsc{3sg}  there  Moyamba  to    \textsc{3sg}{}-\textsc{emph}  self  just  retire    as  now  \textsc{prt}\\
\glt ‘She is in Moyamba, she herself is retired, as of now.' (002a Mabel Lohr, Midwifery: 37)
\z

The indefinite personal pronoun \textit{pɛ} (also [pɔ] and sometimes [pǝ]) has a wide variety of functions. Variously translated as ‘people,' ‘someone,' or ‘they,' it often appears in the sense of impersonal ‘one' or ‘someone,' thus indifferent as to number. In the singular it can be used in the same way as ‘man, person'. It can be used for a sort of passive with verbs such as \textit{gbem} ‘bear,' \textit{hɔ/wɔ}‘say,' and \textit{velu} ‘call' because no passive exists in the language. The example in  \REF{ex:69a} shows \textit{pɛ} used to convey passive, and \REF{ex:69b} shows the impersonal.

\ea%69
    \label{ex:69}
    Impersonal \textit{pɛ} ‘people, they, (some)one'
    
  \ea \label{ex:69a}
  Pɔ gbem wɔ shenge ka.\\
  \gll pɛ      gbem  wɔ    shenge  ka\\
  \textsc{pro}\textsubscript{indef}  bear  \textsc{3sg}  Shenge\is{“Shenge”!}  here\\
\glt ‘She was born in Shenge\is{“Shenge”!}.' (004a Cyril Manley on Walter Hanson: 13)

  \ex \label{ex:69b} Pɔ lɔ pɛ wɔn wɔk mpika?\\
  \gll pɛ      lɔ    pɛ    wɔ      n-hɔk            n-pika\\
\textsc{pro}\textsubscript{indef}  there  also  speak    \textsc{ncm}\textsubscript{ma}{}-language    \textsc{ncm}\textsubscript{ma}{}-other\\
\glt ‘Do they speak other languages there?' (009--10a Lohr \& Mampa: 86)
\z
\z

Another function of \textit{pɛ} is to avoid agency and attribution, using \textit{pɛ} for the impersonal or indefinite, e.g., \textit{pɛ wɔ} ‘People say ….'

The indefinite pronoun \textit{la} (and its emphatic\is{“emphatic”!} form \textit{lan}) has many functions, as first mentioned when discussing \textit{la} used as an interrogative in \REF{ex:66}. Primarily it serves a reference-tracking function in discourse, referring to more than a single person or thing already established in discourse, usually something abstract or general. In addition, it has all the usual functions of a pronoun and is used prolifically.

The first example \REF{ex:70a} illustrates the pronoun is versatile and even has a demonstrative form \textit{lanɛ} that is used as a topicalizer. In all cases it refers to the recorded conversation that Agnes and Jalikatu are about to have.\footnote{This is often the form that “permission” takes since many of our subjects did not know how to read and write.} Jalikatu, a member of the research team, was asking permission from Agnes to have the interview be used as part of the project. The first \textit{la} is used as a relative pronoun and the second is used as a simple anaphor.

In \REF{ex:70b} there are likewise two \textit{la}'s and a \textit{lanɛ}. The first \textit{la} is used as a relative referring to ‘what people say about people,' the demonstrative \textit{lanɛ} refers to what others will say about him, and the second \textit{la} is also a relative referring to the same thing.

\ea%70
    \label{ex:70}
    Indefinite pronoun \textit{la} (and \textit{lan})
    
    \ea \label{ex:70a} Mi ŋa a le yiɛ nɔmaɛ ki, Mi, lanɛ gbi la iba hun thelio we nyema la?\\
    \gll mi      ŋa    a    le    yiɛ    nɔmaa  ɛ    ki\\
    mother  let    \textsc{1sg}  first  ask  woman  \textsc{def}  this\\
  \gll mi        lane  gbi  la      i    ba      hun  theli  o      n    yema    la\\
  mother    that  all    \textsc{pro}\textsubscript{indef}  \textsc{1pl}  \textsc{emph}    come  talk  \textsc{emph}    you  want    \textsc{pro}\textsubscript{indef}\\
\glt ‘Mummy, let me first ask this woman, Mummy, all that we are about to talk about, do you want that?' (007a Agnes J. Simbo: 6)

\ex \label{ex:70b} Nɔɛ wɔ chal ha lɔŋ nui ko la pɔ hɔ ha yindɛ, bi ha theɛ lanɛ la biɛn ha pɛthil wɔɛ.\\
\gll nɔ      ɛ    wɔ    chal  ha    lɔŋ  nui  ko  la      pɛ          hɔ    ha    yin    ɛ\\
person  \textsc{def}  \textsc{3sg}   sit    for    set    ear  to  \textsc{pro}\textsubscript{indef}  \textsc{pro}\textsubscript{indef.person}  say  about  people  \textsc{def}\\
\gll bi    ha    theɛ  lane  la      bi-ɛn      ha      pɛthil      wɔ    ɛ\\
have  to    hear  that  \textsc{pro}\textsubscript{indef}  have-\textsc{neg}  for       pleasant    \textsc{3sg}  \textsc{prt}\\
\glt ‘The person that sits listening to the gossip of others will hear that which displeases him.' (Proverbs: 34)
\z
\z

\section{The definite article} \label{sec:3.4}\hypertarget{Toc115517768}{}
The definite article (\textsc{def}) shows a great deal of predictable phonological variation in both Sherbro itself and in Bolom generally, as well as some dialectal variation. Because the general variation in Sherbro replicates the variation found more generally in Bolom, this section incorporates some comparative statements, as first presented in \citet{Childs2016}.

In Sherbro, as in Bolom, \textsc{def} ranges phonetically from a lengthened preceding vowel to a robust fully formed syllable with a prenasalized stop [ndɛ]. In Mani\il{Mani} the article has the phonetic form of [ʧɛ] with little variation in its form, except for its polar tone. In Mani the tone on the article is a polar tone, the opposite of the final tone of the stem, but in Sherbro the tone on the article is unpredictable (usually low) and inconsistently produced; it is therefore not marked  (\citealt{Childs2011}). In Kisi\il{Kisi} it is always a high tone. The article's ubiquitous presence in the language has caused outsiders to call Mani the [ʧɛ-ʧɛ{}-ʧɛ] language. In the other Bolom languages the article is just as ubiquitous but not nearly so salient because of its reduced forms. Kisi has no definite article to speak of.

One distinguishing mark of \textsc{def} is its full form as the definite article in the Dema\is{“Dema”!} dialect, as opposed to more reduced forms in Shenge\is{“Shenge”!} and the other Sherbro dialects. For speakers in the Dema Chiefdom, the article has the form [lɛ] strengthening to [ndɛ] after nasals, as in the other dialects, but always [lɛ] elsewhere. The basic form of the article in the Shenge dialect is [ɛ] with the set of allomorphs described in \REF{ex:71}.\footnote{The dialectal \textit{ɛ{\textasciitilde}lɛ} alternations leads to some variation in the morpheme analysis line.} The noun stem is given in parentheses.

\TabPositions{1cm,4cm}

\ea%71
\label{ex:71}
Allomorphs of the definite article in Sherbro

[dɛ] After nasals:\\
\tab paŋdɛ ‘month' (paŋ)\\
\tab kundɛ ‘pregnancy' (kun)\\
\tab apimdɛ ‘some' (pim)\\

[lɛ] After [l]:\\
\tab gballɛ ‘scar' (gbal)\\
\tab sɛllɛ ‘wood chips' (sɛl)\\
\tab ntollɛ ‘assault' (tol)\\

[a] After [a] (for most speakers):\\
\tab bokaa ‘cutlass' (boka)\\
\tab jaa ‘matter' (ja)\\
\tab sabaa ‘law' (saba)\\

[ɛ] Elsewhere\\
\z

As to the article's distribution, nouns generally appear with the definite article, as in \REF{ex:72}.

\ea%72
    \label{ex:72}
    Tamɔ lɛ wɔ dwiye ken top.\\
    \gll tamɔ    lɛ    wɔ    dwiye  ken    top\\
    boy    \textsc{def}  3\textsc{sg}  steal    like    groundhog\\
  \glt ‘The boy is stealing like a groundhog.' (P67 T: 114)
\z

If the noun has dependent elements, the article appears at the end of the noun phrase, after the last dependent element, be it a possessive as in \REF{ex:73a} or an attributive adjective as in \REF{ex:73b}.

\ea%73
    \label{ex:73}
    \ea \label{ex:73a} Bɛŋ miɛ bo kɔ nɛki.\\
    \gll bɛŋ  mi    ɛ     bo      kɔ      nɛki\\
    foot  \textsc{1sg}   \textsc{def}  \textsc{emph}    \textsc{ncp}\textsubscript{kɔ}    painful\\
    \glt ‘My foot is hurting me.' (E10 Albert Yanker: 24)
    
    \ex \label{ex:73b} kil thidinthɛ / kil thithiɛ /  kil thisaɛ\\
    \gll kil      thi-dinth      ɛ    kil      thi-thi      ɛ    kil     thi-sa      ɛ\\
    house    \textsc{ncm}\textsubscript{tha}{}-white  \textsc{def}  house    \textsc{ncm}\textsubscript{tha}{}-black  \textsc{def}  house    \textsc{ncm}\textsubscript{tha}{}-red  \textsc{def}\\
    \glt ‘white houses' / ‘black houses' / ‘red houses' (E13 Albert Yanker Adj, Lex: 6)
\z
\z

In terms of function \textsc{def} covers a much greater semantic range than its name indicates. It is the default form attached to nouns in elicitation and in many more contexts than the noun without the definite article. Although the noun stem appears without the article when modified (the article appears at the end of the phrase), there are few other contexts where the noun appears without the article.

Another context where nouns appear without \textsc{def} is in proverbs and folktales. In the proverbs below, the nouns in both \REF{ex:74a} (\textit{lɛin} ‘greeting' and \textit{n-chɔnmalen} ‘love') and \REF{ex:74b} (\textit{bɔk} ‘tortoise' and \textit{pia} ‘arm') do not have an article.

\ea%74
\label{ex:74}
\ea \label{ex:74a} Lɛin kɛtkɛt kɔ cheni   nchɔnmalen.\\
\gll lɛin    kɛtkɛt      kɔ      che-ni    n-chɔnmalen\\
  greeting  frequently  \textsc{ncp}\textsubscript{kɔ}    \textsc{aux-neg}  \textsc{ncm}\textsubscript{ma}{}-love\\
  \glt ‘Frequent greetings is not love.' (Proverbs: 2)

  \ex \label{ex:74b} Bɔk yema fɔs, kɛ pia wɔ kɔ kith.\\
  \gll bɔk    yema    fɔs      kɛ    pia  wɔ    kɔ      kith\\
  tortoise  want    strike    but  arm  \textsc{3sg}  \textsc{ncp}\textsubscript{kɔ}    short\\
 \glt ‘The tortoise wants to punch, but its arm is short.' (Proverbs: 4)
\z
\z

\largerpage
In folktales, animals are presented as names, sometimes even with titles, e.g., \textit{Ba,} in \REF{ex:75}.

\ea%75
    \label{ex:75}
    \ea Ba Na lee mathini.\\
    \gll ba      na      le    mathini\\
    mister  spider  stay  hide.oneself\\
   \glt ‘Mr. Spider\footnotemark{} stayed behind to hide himself.' (P67 L: 61)
\footnotetext{The spider, sometimes known as Ananse, is prominent in Akan culture and elsewhere in West Africa and the West Indies as a hero or at least a character in folktales (e.g., \citealt{BadoeDiakité2001}).}
  \ex Ba  Na ni gbɔlkajo wɔɛ yema ŋa jo tri thɛai than gbi.\\
    \gll ba      na      ni    gbɔlkajo    wɔ    ɛ\\
    mister  spider  with  gluttony    \textsc{3sg}  \textsc{prt}\\
    \gll yema    ŋa    jo    tri    thi-ɛ    ai    tha-n        gbi\\
    want    for    eat    town  \textsc{ncm}\textsubscript{tha}  in    \textsc{ncp}\textsubscript{tha}{}-\textsc{emph}  all\\
  \glt ‘The spider with his gluttony wants to eat in all the towns.' (\citealt{Sumner1921} txt:7)
\z
\z


\section{Ideophones}
\label{sec:3.5}\hypertarget{Toc115517769}{}
Ideophones constitute a distinct word category in many African languages but tend to disappear in situations of language contact\is{“contact”!} and language death, particularly when a sense of local identity\is{“identity”!} is lost (\citealt{Childs1994}, \citealt{Childs1998}). Such is the case in Sherbro where only a few ideophones were identified.\footnote{44 in a lexicon of 4,095 entries.}

Ideophones typically have an aberrant phonetic form not in accord with, e.g., the segmental or phonotactic constraints of the language. They have no morphology except for expressive lengthening and repetition. They are set off prosodically by being in a distinctly higher or lower register and syntactically by being at the end of a sentence forming a separate constituent. Ideophones generally underscore or add emphasis to a sensation. They are not, strictly speaking, essential except in a performative sense (e.g., \citealt{Dingemanse2009}). They may also feature sound symbolism and sub-morphemic partials.\footnote{There are many more features than these and much exemplification in a chapter I wrote on African ideophones and in another on sound symbolism (\citealt{Childs1994}, \citealt{Childs2015}).} Some examples from Sherbro appear in \REF{ex:76}.

\ea%76 
\label{ex:76}
\ea Wɔ lɛli teen.\\
\gll wɔ  lɛli    teen\\
\textsc{3sg}  look    \textsc{idph}\\
\glt ‘She observed very closely.' (Albert Yanker p.c.)

\ex Ayeŋ wɔ lɛ   che bisiɛ peŋ.\\
\gll ayeŋ  wɔ    lɛ    che  bisiɛ  peŋ\\
middle  \textsc{3sg}  \textsc{def}  \textsc{aux}  tight  \textsc{idph}\\
\glt ‘His waist is tight! (He has a narrow waist.)' (P67 B: 144)

\newpage
\ex Tha sɔikɛ yenchɛkɛ, yɛ ŋa the tiŋ yɛ vɛ ŋa ɡbikini wa!\\
\gll tha    sɔikɛ    yenchɛk  ɛ    yɛ    ŋa    the   tiŋ    yɛ    vɛ\\
\textsc{ncp}\textsubscript{tha}  scare    fish(pl)  \textsc{def}  \textsc{prt}  \textsc{3pl}  hear  noise  \textsc{prt}  thus\\
\gll ŋa  ɡbikini  wa\\
\textsc{3pl}  flee    \textsc{idph}\\
\glt ‘They (trawlers) scare the fish away, when they (the fish) hear the noise, they flee in a panic!' (142v Baba Mandela, Fishing: 73)
\z
\z

Ideophones do not transfer well to paper, but if they were heard, the unusual phonetic features would be obvious. At the least it is clear how they are set apart from the matrix sentence.

\section{Names}
\label{sec:3.6} \hypertarget{Toc115517770}{}
Names function importantly in Sherbro culture. They signify gender and birth order but also society membership, i.e., there are names given to boys and girls when they are initiated\is{initiation!}. There is the usual assortment of nicknames as well. For example, one of the paramount chief's favorites was a famous dancer and quite strong. He was known as \textit{Trongman}, a borrowing\is{“borrowing”!} from Krio\il{Krio}. The first name given to a child is determined by gender and birth order, as given in \tabref{tab:wordcat:18}. These names are virtually identical to those used by the Bom-Kim\il{Bom-Kim} people.

\begin{table}
\caption{\label{tab:wordcat:18}Birth-order names}

\begin{tabular}{lll}
\lsptoprule
& Male & Female\\
\midrule
1\textsuperscript{st} & Cho & Bɔi\\
2\textsuperscript{nd} & Tɔmi & Yema\\
3\textsuperscript{rd} & Sɔba & Kɔni\\
4\textsuperscript{th} & Baki & Mahan [ma̰ḭn]\\
5\textsuperscript{th} & Baiyikɛ & Chɔkɔ\\
6\textsuperscript{th} & Boka & Manɛ\\
7\textsuperscript{th} & Puluk & Yɔki\\
\lspbottomrule
\end{tabular}
\end{table}

Should one woman have an eighth child, a relatively uncommon occurrence, the choice of a name is open.\footnote{Among the Kisi\il{Kisi}, the solution is to follow the name with ‘small'; a \textit{Saa}, usually used for the first-born male, would also be the name given to the male child after the names were exhausted, but he would be known as \textit{Sàà Pòómbɔ̀} ‘Small Saa.'}

A second naming takes place when children are initiated\is{initiation!} or “join society” in the local idiom, \textit{Bondo}\is{“Bondo”!} for the girls and \textit{Poro}\is{“Poro”!} for the boys. There are many other secret societies\is{“secret!}, e.g., Yassay (Yase), Thoma, Nthun-Nthun, Gbaa, and the disbanded Leopard Society of Dema\is{“Dema”!} Chiefdom, but Poro and Bondo are the most widespread (\textit{Kɔɔɔli} in Mende\il{Mende}). When boys and girls emerge from society, they have a new name bestowed upon them during the initiation process, a name not widely known outside the society itself. An exception is the name of the Paramount Chief of Kagboro\is{“Kagboro”!} Chiefdom, Doris Lenga-Caulker\is{Caulker!} Gbabiyor\is{Lenga-Caulker Gbabiyor, Doris!} II, where \textit{Gbabiyor} is the chief's society name. The society name is often determined by an initiate's status in the school. For example, the first girl to emerge is known as \textit{Kema}. Girls more than boys will keep their society names.

\begin{table}
\caption{\label{tab:wordcat:19}Some Sherbro society names}


\begin{tabular}{ll}
\lsptoprule
Male & Female\\
\midrule
Gbanabom & Njabu\\
Kaare & Kolone\\
Balaka & Pondo\\
Yamba & Gbatewa\\
Biaheni & Njopojo\\
\lspbottomrule
\end{tabular}
\end{table}

These names are usually not publicly used. In earlier generations, these names were a combination of your society name and your mother's name. Consultant Albert Yanker's name was \textit{Bia Bue}, where \textit{Bue} is his mother's society name. The naming practice may be a residue of the former matrilineality of the Sherbro, found also in the practice of having female paramount chiefs. The practice is still known among the Mani\il{Mani}, people being known as ‘child of (mother's name)' (\citealt{Childs2011}).

The language of Poro\is{“Poro”!} is quite metaphorical, and everyday words take on a new significance. Typically, everyday language is used in non-everyday ways, e.g., \textit{bɛthpɔɔ} ‘cut Poro' means to be summoned in society. Leaders are known as ‘grandfather,' boys are said to be ‘eaten' and even ‘die' as part of the Poro ritual. Everyday words such as ‘thing,' ‘big' and ‘bush' take on new significance. Although these two societies are gendered, women are able to join the men's society, but only once they have gone through menopause. For example, District Chief Koŋchaŋmaa (lit. ‘finish-pass-woman') of Samu District, Bumpeh Chiefdom\is{Bumpeh Chiefdom!} joined Poro before becoming chief.

Some names are taboo\is{“taboo”!}, primarily those associated with secret society\is{secret societies!} personnel, the “masks” or “devils\is{devils!}”, leaders in the society.\footnote{“Masks” or “devils” are the common names for important, powerful, and sometimes frightening masked personalities of the initiation societies for boys and girls (a.k.a. “bush school”). Some can be seen by outsiders, especially those belonging to female societies, but some cannot.} For example, the real name of one's mask cannot be spoken; it is known only as \textit{jalimatha} ‘the hidden thing'. Other taboos exist. The name for ‘leopard' \textit{gbel} is usually not pronounced; the leopard is referred to by a euphemism \textit{yentho} ‘bush thing' or \textit{hathog} or \textit{hãtoɛ} (\citealt[83]{Pichl1967}).

Many town names have a Temne\il{Temne} prefix \textit{mo-} [mɔ], e.g., \textit{Moyeamoh} (Bumpeh Chiefdom\is{Bumpeh Chiefdom!}), \textit{Mokainsumana} (Kagboro\is{“Kagboro”!} Chiefdom). Some of them have a Sherbro equivalent: \textit{Mokornbeti} is known as \textit{Nkɔŋbɛti} in Sherbro. The traditional equivalent prefix for a Sherbro town is \textit{ko}, a locative preposition translated roughly as ‘to' but with many other locative meanings, as in \REF{ex:77}, where in its first realization it means something like ‘from' and its second something like ‘to' (see discussion of \textit{ko} as an adposition in \sectref{sec:3.8}).

\ea%77
    \label{ex:77}
    Awokɔ gbo ko mɔ ko yai hun ko Mi Adama.\\
\gll a    wokɔ    gbo  ko    mɔ  ko    ya-i  hun    ko    mi      adama\\
\textsc{1sg}  leave    just  from  \textsc{2sg}  to    \textsc{1sg}  come    to    Mother  Adama\\
\glt ‘After leaving you, I will go to Mami Adama.' (009--10a Lohr \& Mampa: 11)
\z

As discussed in \sectref{sec:3.8}, the adposition \textit{ko} can appear as both a preposition and as a postposition, but in names it most often appears finally as in \textit{Woŋko}, the name for Wong Island, and \textit{Nyambako}, the Sherbro name for the town more generally known as \textit{Moyamba} (with the Temne\il{Temne} prefix \textit{mo-} mentioned above).

Town names sometimes originate in the names of their founders. Plantain Island\is{“Plantain!} is not named after the fruit but rather after a slaver and pirate, an Englishman named John Plantain, who used the island as a base for his slaving operations at the beginning of the eighteenth century.\footnote{This is despite a nearby island being named Banana Island, a more relevant appellation since the island is shaped like a banana. There is also a Monkey Island in the same archipelago, where, indeed, monkeys rule.} The island was earlier known as \textit{Yelsaha} (Egusi Island) when it was famous for its extensive cultivation of the plant \textit{saha} ‘egusi' on the island.\footnote{Egusi is a melon-like plant (\textit{Cucumeropsis mannii}), whose seeds are dried and used as a thickening agent in soups and gravies. It is widely used in West Africa but is not native to the Sherbro area.} Farming, however, is no longer possible because of the island's subsidence; there are now few wells with potable water, requiring fresh water be brought from the mainland.

Other names are more prosaic in their origin. Another island in the same archipelago is \textit{Bompetok}, literally ‘on top of the island'. \textit{Bachelor}, a species of palm, is a town named after a prominent short palm tree. \textit{Seaport} (pronounced [sipɔt] in Sherbro) is a town near the mouth of the Bumpeh River that features commerce with both sea and river travellers. \textit{Tissana} or ‘new town,' is a common name for towns in the Sherbro-speaking area.

\section{Numbers}
\label{sec:3.7}\hypertarget{Toc115517771}{}
Sherbro uses a 5-and-20-based system of numbers (quinary-vigesimal), the same system found in all closely related languages and widely throughout the world (\citealt{Nykl1926}).\footnote{See Chapter 6 in \citet{Harrison2007} for the complexities of other numbering system and the extensive database of Eugene D.L. Chan, which contains numeral systems of 4,380 languages (\url{https://mpi-lingweb.shh.mpg.de/numeral/}, 2020-06-22).} A quick glance at \tabref{tab:wordcat:20} reveals that the numbers 6-9 are simply ‘5+1,' ‘5+2,' etc., and the numbers above ‘20' use a base of twenty, i.e., ‘30' is ‘20 + 10,' ‘40' is ‘two twenties,' etc. The word for ‘100' is borrowed from a Mande\il{Mande} language, as is the case in other closely related languages. This fact is not surprising in that the Mande peoples have historically been much more aligned with trade and commerce, as is shown by patterns of borrowing\is{“borrowing”!} (\citealt{Childs2002a}).

\begin{table}
\caption{\label{tab:wordcat:20}Numbers}

\begin{tabular}{lllll}
\lsptoprule
bul & ‘one' & waŋ-ni-bul & ‘eleven'\\
tiŋ & ‘two' & waŋ-ni-tiŋ & ‘twelve'\\
ra & ‘three' &  … & \\
hiɔl & ‘four' &  & \\
mɛn & ‘five' &  & \\
mɛnbul & ‘six' & waŋ-ni-mɛnbul & ‘sixteen'\\
mɛntiŋ & ‘seven' & waŋ-ni-mɛntiŋ & ‘seventeen'\\
mɛnra & ‘eight' & …\\
mɛnhiɔl & ‘nine'\\
waŋ & ‘ten' & kuaŋa & ‘twenty'\\
\tablevspace
kuaŋa-ni-waŋ & ‘thirty'\\
kuaŋa-tiŋ & ‘forty'\\
kuaŋa-tiŋ-ni-waŋ & ‘fifty'\\
… \\
kɛmɛ & ‘hundred'\\
\lspbottomrule
\end{tabular}
\end{table}

Despite the availability of this system, merchants and their patrons use English or Krio\il{Krio} words for their market interactions, as do litigants in court cases, especially, as the Sierra Leone currency loses value, for large transactions. It is likely that the traditional system will soon be lost, although the low numbers are still in use, particularly in bartering transactions.

\section{Adpositions}
\label{sec:3.8}\hypertarget{Toc115517772}{}
Sherbro has a relatively rich set of adpositions, not all of the permutations of which will be discussed here.\footnote{In a lexicon of 4,095 entries, there are 29 entries that function as prepositions and 21 as postpositions.}

Adpositions vary in both their syntax and their lexical status. They may appear both before and after their objects, sometimes both, and sometimes together as a circumposition. Some adpositions may function also as simple locatives without an object. They may be phonologically dependent as well as independent words. Generally speaking, prepositions are more grammatical (a closed class) and postpositions more lexical (an open class). The former can indicate that a relation exists between the object and the rest of the sentence; the latter gives the details and is the place where variety and innovation are situated.

The most versatile of adpositions is \textit{ko} meaning ‘to' with other generally locative meanings. The examples in \REF{ex:78} show \textit{ko} as a preposition. In \REF{ex:78b}, \textit{ko} is before the fronted relative pronoun \textit{lɔ}.

\ea%78
    \label{ex:78}
    \textit{Ko} ‘to, etc.' as a preposition

\ea\label{ex:78a} Ya  hink ko Ba Yanka.\\
\gll ya    hink    ko     ba      Yanka\\
  1\textsc{sg}  come    from    mister  Yanker\\
  \glt ‘I came from (seeing) Mr Yanker.' (E10 Albert Yanker: 9)

\ex\label{ex:78b} …ko lɔ Kaiŋ Taso hinɛ pɛllɛaiɛ.\\
\gll ko    lɔ      Kaiŋ  Taso    hinɛ  pɛl        lɛ    ai    ɛ\\
to    \textsc{ncp}\textsubscript{lɔ}    Kain  Tasso    lie    hammock  \textsc{def}  in    \textsc{prt}\\
\glt ‘…to where Kain Tasso was lying in the hammock.' (123aw Yanker, Rat Wife: 54)
\z
\z

In the first three examples in \REF{ex:79}, \textit{ko} is used as a postposition meaning ‘to, in, at' after place names (cf. the discussion around \REF{ex:82}). The example in \REF{ex:79c} shows its versatility. In \REF{ex:79d}, it is used after \textit{lala} ‘fire, hearth' to mean ‘on.'

\ea %79
\label{ex:79} \textit{Ko} ‘to, etc.' as a postposition

\ea \label{ex:79a} Wɔn pɔ gbem wɔ Nra ko.\\
\gll wɔ{}-n      pɛ      gbem      wɔ    n-ra        ko\\
\textsc{3sg-emph}  \textsc{pro}\textsubscript{indef}  give.birth  \textsc{3sg}  \textsc{ncm}\textsubscript{ma}{}-ra    to\\
\glt ‘She was born in Ra village.' (005a Jalikatu B. Kumba: 43)

\ex  \label{ex:79b} Nkeni ko ntɛnt?\\
\gll nkeni    ko    n-ntɛnt\\
Makeni  to    \textsc{ncm}\textsubscript{ma}{}-near\\
\glt ‘Is it near Makeni?' (005a Jalikatu B. Kumba: 44)

\ex \label{ex:79c} A-a, Themdel ko, tikowɔko lɔ Nsanda ko.\\
\gll aɂa  Themdel    ko    tii    ko    wɔ    ko    lɔ      Nsanda  ko\\
no    Themdel    to    town  to    \textsc{3sg}  to    \textsc{ncp}\textsubscript{lɔ} Nsanda  to\\
\glt ‘No, in Timdale [Chiefdom], his town is called “Nsanda.” '(009--10a Lohr \& Mampa: 83)

\ex \label{ex:79d} …mɔi huŋ bɛ lala ko.\\
\gll mɔ-i      huŋ    bɛ    lala  ko\\
you-\textsc{prt}    come    put  fire  to\\
\glt ‘…and then put it on the fire.' (012-13a Adama Mampa, Cooking: 64)
\z
\z

The use of \textit{ko} after a proper name locative may be a parallel to the Mande\il{Mande} suffixed locative \textit{du} or \textit{dugu} as in the place name Kissidougou, a Forest Region (Guinea) city of the Kisi\il{Kisi} people and the name of a town of Kisi workers in the Guinean Samou region near the border with Sierra Leone.

The versatility of \textit{ko} does not end here. A common construction features \textit{ko} both before and after a personal pronoun or name with a meaning of ‘to my place' / ‘to me,' ‘to Adama's place,' etc. (cf. \REF{ex:79} above). The sequence is \textit{ko mɔ ko} ‘to you to' in \REF{ex:80a} and \textit{ko mi ko} ‘to me to' in \REF{ex:80b}. (There is also another example of \textit{ko} as a preposition, i.e., before \textit{Mi Adama} ‘Mami Adama' in \REF{ex:80a}.) The meaning of the first locative is more like ‘to,' while the second is more like ‘home, place,' as in ‘from my place,' especially evident in \REF{ex:80b}.

\ea %80
\label{ex:80}
Use of \textit{ko} … \textit{ko} construction\\

\ea \label{ex:80a} Awokɔ gbo ko mɔ ko yai hun ko Mi Adama.\\
\gll a    wokɔ    gbo  ko    mɔ  ko    ya-i    hun    ko    mi        adama\\
  1\textsc{sg}  leave    just  to    \textsc{2sg}  to    \textsc{1sg-prt}  come    to    Mother    Adama\\
  \glt ‘After leaving you, I go to Mami Adama.' (009--10a Lohr \& Mampa: 11, repeated from \REF{ex:76})

\ex \label{ex:80b} Yɛ ya wokɔ tikomiko a kɔni yena livil we…\\
\gll yɛ    ya    wokɔ    tii      ko    mi    ko    a    kɔni  yena    li-vil      we\\
  if    \textsc{1sg}  leave    town    to    \textsc{1sg}  to    \textsc{1sg}  go    place    \textsc{ncm}\textsubscript{lɔ}{}-far  \textsc{emph}\\
\glt ‘If I travel from home to somewhere far away…' (003a Shenge Youth Choir, Hymns: 9)
\z
\z

There are other unusual features to the adpositional system of Sherbro also involving the versatile \textit{ko}. In \REF{ex:81}, the noun \textit{mɛnɛ} has both a preposition (\textit{hink}) and a postposition (\textit{ko}), instead of \textit{ko} being both preposition and postposition (see discussion of \textit{mɛnɛ} below). The example in \REF{ex:81b} shows the common use of \textit{ko} as a postposition after \textit{mɛnɛ} without the use of \textit{hink} as preposition.

\ea%81
\label{ex:81}
\ea \label{ex:81a} Ka hok hink mɛnɛ ko.\\
\gll ($\emptyset$) ka      hok      hink  mɛnɛ    ko\\
\textsc{(3sg)} \textsc{rem.pst}  come.from  from  grave to\\
\glt ‘(He) came from the grave.'


\ex \label{ex:81b} Wɔ mɛnɛ ko\\
\gll wɔ    mɛnɛ    ko\\
\textsc{3sg}  grave    to\\
\glt ‘He is in the grave.' (P67 M: 52)
\z
\z

In \REF{ex:82} appears another example of \textit{ko} serving as a locativizing suffix, but in this case it has become something of a derivational morpheme inside the NP, i.e., before \textit{(l)ɛ} the definite marker. It thus forms part of the noun itself since the definite article is never used after an adpositional phrase (see Chapter \ref{ch:7}).


\ea %82
\label{ex:82}
Locativizing \textit{ko}\\
Yɛ mɔ bɛ lalakoɛ jɛmdɛ lɔlɔ bo shi che kɔ ma ki hei.\\
\gll yɛ    mɔ  bɛ    lala-ko    ɛ    jɛm  ɛ    lɔ    lɔ\\
after  \textsc{2sg}  put  hearth-to  \textsc{def}  fire  \textsc{def}  \textsc{ncp}\textsubscript{lɔ}  there\\
\gll si      che  kɔ      ma  ki    hei\\
so.that  be    \textsc{ncp}\textsubscript{kɔ}    \textsc{neg}  this  burnt\\
\glt ‘After you put it on the hearth, the fire would be just so, so that it (rice) would not burn.' (012-13a Adama Mampa, Cooking: 65)
\z

The examples in \REF{ex:83} show that \textit{ko} has a status different from the postposition \textit{ai} ‘in' (see \REF{ex:87}). The latter, likely because of its phonology, always attaches to the noun it follows; it also has a more precise meaning than \textit{ko}. It appears inside the PP, closer to the noun than \textit{ko}, as is clear in the following examples where they are both used.

\ea%83
    \label{ex:83}
    \ea Ni wɔ ye kɔ killɛai wɔ ko.\\
    \gll ni    wɔ    ye    kɔ    kil      lɛ    ai      wɔ    ko\\
    and  \textsc{3sg}  then  go    house    \textsc{def}  inside    \textsc{3sg}  to\\
    \glt ‘And then he went into his house (his place).' (P67 K: 211)

    \ex Ponk pia lallɛai ko.\\
  \gll ($\emptyset$) ponk  pia  lal    lɛ    ai    ko\\
(\textsc{3sg})  put  hand  fire  \textsc{def}  in    to\\
\glt ‘He put his hand into the fire.' (P67 P: 11)
\z
\z

The preposition \textit{ka}, strictly a preposition meaning ‘with,' is likely a reanalysis of the verb extension \textit{{}-ka}, which has also taken place in closely related Bom-Kim\il{Bom-Kim} {\citealt{Childs2020}}). Bom-Kim has a limited number of verb extensions, only one of which, the instrumental \textit{\nobreakdash-ka}, is currently productive, surviving perhaps because it has been reanalyzed as a preposition and is sometimes ambiguous as to its status.

\ea%84
    \label{ex:84}
    Bom-Kim instrumental \textit{{}-ka} [ga] (as verb extension)\\
    \vspace{6pt}  
    Ha bɛmpaga blɔklɛ isunndɛ.\\
    \gll ha    bɛmpa-ka    blɔk    ɛ    i-sun        ɛ\\
    3\textsc{pl}  make-\textsc{ins}    block    \textsc{def}  \textsc{ncm}\textsubscript{hɔ}{}-sand    \textsc{def}\\
    \glt ‘They make blocks with sand.' (\citealt{Childs2020})
  \z

For some Bom-Kim\il{Bom-Kim} speakers, \textit{{}-ka} has been reanalyzed as a preposition, however. This is probably due to the influence of Mende\il{Mende}, a highly analytic language with adpositions and the language to which all speakers have shifted. The example in \REF{ex:85} shows \textit{ka} as a separate word after the direct object (cf. \REF{ex:84}), thus affirming its separation and independence from the verb.

\ea%85
    \label{ex:85}
    Bom-Kim\il{Bom-Kim} preposition \textit{ka}\\
    \vspace{6pt}   
    A kɔn kɛti tɔgilɛ ka gbɛlalɛ.\\
    \gll a     kɔn   kɛti   tɔgi  lɛ    ka    gbɛla    lɛ\\
    1\textsc{sg}  go    cut  tree  \textsc{def}  with  axe    \textsc{def}\\
    \glt ‘I cut the tree with the axe.' (\citealt[445--46]{Childs2020})
\z

In Sherbro, the same reanalysis is underway. There is both a verb extension \textit{{}-ka} \REF{ex:86a}, as well as a preposition \textit{ka} \REF{ex:86b}.

\newpage
\ea%86 
\label{ex:86} 
    \ea \label{ex:86a} Verb extension \textit{{}-ka} (allomorph [ka] and [k])\\
    \ea Fe wullɛ lɔ pɔ bɛmpaka wullɛ.\\
    \gll fe      wul    ɛ    lɔ      pɛ      bɛmpa-ka      wul    ɛ\\
    money  funeral  \textsc{def}  \textsc{ncp}\textsubscript{lɔ}    \textsc{pro}\textsubscript{indef}  prepare-\textsc{ins}    funeral  \textsc{def}\\
    \glt ‘It is the funeral money that will be used for the funeral.' (Proverbs: 137)\\
    
	\ex Ba Amadu Kamara wɔe herk yagbe wɔɛ Braima Nsheŋke ka.\\
	\gll ba       Amadu Kamara wɔ-ɛ    her-ka       yagbe  wɔ    ɛ    Braima n-sheŋke      ka\\
	mister  Amadu Kamara \textsc{3sg-prt} cross-\textsc{ins}  nephew  \textsc{3sg}   \textsc{def}  		Braima \textsc{ncm}\textsubscript{ma}{}-Shenge  here\\
	\glt ‘Mr. Amadu Kamara then takes his nephew Braima across to Shenge.' (124aw Yanker, Boy Lost at Sea: 252)\\
 \z

 \ex \label{ex:86b} Preposition \textit{ka}\\
 \ea Nɔsaa ɛ wɔ kɔ bɛt bachɛ ka ibaa.\\
	\gll nɔsaa    ɛ    wɔ    kɔ    bɛt    bach  ɛ    ka    i-baa\\
	tapster  \textsc{def}  \textsc{3sg}  go    tap  palm  \textsc{def}  with  \textsc{ncm}\textsubscript{hɔ}{}-curved.knife\\
	\glt ‘The palmwine tapster tapped the palm tree with a knife.' (E12 Albert Yanker: 12)\\
 
	\ex Pɔ kɔ yuk ka tɛnthe.\\
	\gll pɛ      kɔ      yuk    ka      tɛnthe\\
	\textsc{pro}\textsubscript{indef}  \textsc{ncp}\textsubscript{kɔ}    plant    with    stick\\
 \glt ‘They plant it with a split cane stick.' (006v Abdulai Bendu, Rice Growing: 28)\\

	\ex Laŋgbaɛ the nɛki ka billɛ.\\
 \gll laŋgba  ɛ    the  nɛki  ka    bil    ɛ\\
	man    \textsc{def}  feel  pain  with  yaws  \textsc{def}\\
	\glt ‘The man was in pain due to yaws.' (E14 Albert Yanker: 7)
\z
\z
\z

That the same reanalysis has taken place in Sherbro and Bom-Kim points to their close relationship and a shared ancestry.

A postposition that behaves much like a suffix is \textit{{}-ai}, which has a meaning of something like ‘in' or ‘on,' as in \REF{ex:87}.

\newpage
\ea%87
    \label{ex:87}
    Postposition \textit{{}-ai}

  \ea \label{ex:87a} Wɔ theli mbolomdai, wɔ theli mpothoai.\\
  \gll wɔ    theli    n-bolom      {}-ai    wɔ    theli    n-potho        {}-ai\\
  \textsc{3sg}  speak    \textsc{ncm}\textsubscript{ma}{}-Bolom  in    \textsc{3sg}  speak    \textsc{ncm}\textsubscript{ma}{}-European  in\\
   \glt ‘He spoke in Bolom, he spoke in English.' (004a Cyril Manley on Walter Hanson: 84)

  \ex \label{ex:87b}  Atiŋdɛ ŋa kɔ skullai, bullɛ wɔn chepa kɔ skul kɛ chen pɛ kɔ\\
  \gll a-tiŋ      ɛ    ŋa    kɔ    skul    {}-ai\\
    \textsc{ncm}\textsubscript{ha}{}-two  \textsc{def}  \textsc{3pl}  go    school  in\\
  \gll bul  ɛ    wɔ{}-n      che  pa      kɔ    skul    kɛ    che-ni    pɛ      kɔ\\
  one  \textsc{def}  \textsc{3sg-emph}  \textsc{aux}  formerly  go    school  but  \textsc{aux-neg}  again    go\\
\glt ‘The two go to school, the one was going to school but he doesn't anymore.' (029a Biah Heni: 18)
\z
\z

The locative \textit{{}-ai} conditions the same syllable building seen with the definite article \textit{ɛ} discussed in \sectref{sec:3.4}. After nasals, the surface form is [ndai] \REF{ex:87a} and, after [l], it is [lai] \REF{ex:87b}. Note also in \REF{ex:87b}, the gemination involving the definite article after \textit{bul} ‘one.'

The locative \textit{{}-ɛ} works in a similar way with a more general meaning than \textit{{}-ai}. It locativizes any noun, as illustrated in \REF{ex:88}, and so may be on its way to becoming a derivational suffix.

\ea%88
    \label{ex:88}
  Locative \textit{{}-ɛ} ‘in, at, on, etc.'
  \ea \label{ex:88a} njok / pia njok / njokɛ\\
  \gll n-jok        pia  n-jok          n-jok        {}-ɛ\\
\textsc{ncm}\textsubscript{ma}{}-right  hand  \textsc{ncm}\textsubscript{ma}{}-right    \textsc{ncm}\textsubscript{ma}{}-right  on\\
\glt ‘right, right side' / ‘right hand' / ‘on the right side' (P67 J: 31)

\ex \label{ex:88b} mɛndɛ / mɛnɛ / mɛndaiɛ\\
\gll mɛn  ɛ      mɛn-ɛ                mɛn    {}-ai    ɛ\\
water  \textsc{def}    water-in                water    in    \textsc{def}\\
\glt ‘the water' / ‘in water, under (e.g., water, earth)' / ‘in the water'

\ex \label{ex:88c} La mi bolɛ. Ma mi bɛnbolɛ.\\
\gll la      mi    bol    {}-ɛ      ma  mi    bɛnbol  {}-ɛ\\
\textsc{pro}\textsubscript{indef}  \textsc{1sg}  mind    in      \textsc{neg}  \textsc{1sg}  mind    in\\
\glt ‘It is in mind.' ‘Do not worry about me.' (lit. ‘Do not (have) me in mind.') (E13 Albert Yanker, Adj, Lex: 67)
\z
\z

What is curious about the postposition \textit{{}-ɛ} is that it does not trigger any of the onset-building processes seen with the definite article \textit{ɛ} and with the postposition \textit{ai}, as seen in the minimal triplet in \REF{ex:88b} (see \sectref{sec:3.4}). In \REF{ex:88c}, there is similarly no geminate “l,” as would be expected were \textit{ɛ} the definite article (\textit{bollɛ} ‘the head') or the postposition (\textit{skullai} ‘to school'). Thus, the rule described in \sectref{sec:3.4} is much less phonological than it is morphophonological, i.e., restricted to distinct morphological contexts. In Kisi\il{Kisi}, a similar process is purely phonological, motivated by considerations of syllable structure.

Body parts can also be used as adpositions in the more lexical slot following an NP. The first example \REF{ex:89a} shows the word for ‘head' used to mean something like ‘ahead.'

\ea%89
    \label{ex:89}
    Body parts as adpositions: ‘head,' ‘mouth,' ‘belly'\\

    \vspace{6pt}
    
	\ea \label{ex:89a} \textit{Bol} ‘head,' ‘ahead, in front of'\\

 \vspace{6pt}
    
	Kɔ  mathin yaŋ che næ lɛ ibol ha pakali mi.\\
	\gll ($\emptyset$)  kɔ    mathin  ya-ŋ      che  nai  lɛ    i-bol        ha    pakali  mi\\
	(\textsc{3sg})  go    hide    \textsc{1sg}{}-\textsc{emph}  be    road  \textsc{def}  \textsc{ncm}\textsubscript{hɔ}{}-head    for    scare    1\textsc{sg}\\
 \glt ‘He went to hide ahead of me on the road, in order to scare me.' (P67 P: 9)\\

\vspace{6pt}
    
  \ex \label{ex:89b}  \textit{Hɔl} ‘mouth,' ‘inside (with \textit{ko})'\\

  \vspace{6pt}
  
Pəŋ hu lɛ ni kɔni kil lɛ hɔl ko.\\
  \gll ($\emptyset$)  pɛŋ  hu    lɛ    ni    kɔni  kil      lɛ    hɔl    ko\\
  (\textsc{3sg})  jump  fence  \textsc{def}  and  go    house    \textsc{def}  mouth  to\\
  \glt ‘He jumped over the fence and went into the house.' (P67 P: 78)


    
  \ex \label{ex:89c}  \textit{Kun} ‘stomach, belly'\\

  \vspace{6pt}
  
  Mɔi bɛ ituɛ kunɛ.\\
  \gll mɔ-i    bɛ    i-tu      ɛ    kunɛ\\
  \textsc{2sg-prt}  put  \textsc{ncm}\textsubscript{hɔ}{}-pot  \textsc{def}  inside\\
  \glt ‘You put it in the pot.' (012-13a Adama Mampa, Cooking: 21)
\z
\z

There is a possible relic \textit{a-} which has combined with other forms to produce a set of postpositions as in \REF{ex:90}. The speculation is that \textit{a-} is related to the general adposition \textit{a} (a preposition in Kisi\il{Kisi}) to form compound postpositions since no other lexical items begin with [a].

\newpage
\ea%90
    \label{ex:90}
Postpositions beginning with \textit{a-}\\

\begin{tabular}[t]{lll}
ahɔl & ‘mouth,' ‘inside'\\

alɔ & ‘lower, under'\\

atok & ‘on top of, above'\\

ayeŋ & ‘in the middle' (cf. \textit{thiyeŋ} ‘between, among')
\end{tabular}
\z

This brief treatment of Sherbro adpositions will hopefully lead to a more extensive survey of their function and form. I turn now to verbs which have only a few derivational relations to adpositions.

\section{Verbs}
\label{sec:3.9}\hypertarget{Toc115517773}{}
The easiest way to identify a verb is by its morphosyntax. Morphologically, verbs are most centrally inflected for aspect, and syntactically, they form the head of a verb phrase. Semantically, they cover the same ground as verbs in other languages, but considerably more ground than in Western languages, such as English and French, where stative concepts are often expressed as adjectives. Stative concepts such as ‘dry,' ‘heavy,' and ‘tasty' are expressed by verbs, respectively, \textit{sɛk}, \textit{dis}, and \textit{pɛth}. Sherbro has a sizeable number of “true” adjectives and a productive process for forming adjectives from stative verbs (see \sectref{sec:3.2} \& \sectref{sec:7.1}).

Phonologically, verb roots take the form CVC(V) with inflectional and derivational material either directly changing the form or adding on material to the right. Verbs rarely exceed three syllables in length, except when reduplicated. Other prosodic limits on the verb are phonotactic constraints common throughout the language. It appears that as one moves rightward in the verb, however, the inventory of contrasts decreases. That is, fewer contrasts exist in rightward syllables than in syllables beginning the word. Very few trisyllabic verbs end in a vowel other than [i] or [a], the phenomeon echoing similar facts in Bantu verbs (\citealt{Hyman1993},  \citealt{Hyman2004}).

Morphological distinctions are treated in detail in    \ref{sec:4.1}. Here I outline what distinctions may be marked on verbs. Aspect is the most important contrast to be marked in Sherbro, a contrast that can be roughly characterized as the difference between imperfective and perfective meaning and is probably the best example of a true inflection. Mood distinctions, as well as polarity, can also be marked on verbs.

Tense is marked periphrastically by pre- and post-verbal particles to mark near past (post-verbal \textit{na}) and distant past (pre-verbal \textit{ka}). Speakers say that \textit{na} represents “past today time” while \textit{ka} designates something further in the past, even a few years ago.

Negation is not so much a morphological process as a morphosyntactic one, with two negative markers \textit{ni} and \textit{ma}, the former attaching to the verb, the latter a pre-verbal particle.

In addition to these morphological and morphosyntactic processes, verbs exhibit a distinctive syntax. They form the head of a verb phrase, which consists of an optional subject pronoun, optional auxiliaries, and optional pre- and post-verbal \textsc{tmap} markers, with up to two optional non-subject arguments (see \sectref{sec:8.2}). The argument structure of a verb may be altered by the affixation of suffixed verb extensions (see Chapter \ref{ch:6}).

\ea%91
    \label{ex:91}
  Verb phrase in Sherbro\\
(SM) (Particles) (Auxiliaries) Tense V (-Extensions) (Particles) (Objects)
\z

There is a variation in this syntax when tense is not on the lexical verb. In these cases, if the verb has any pronominal arguments, these appear before the verb in an order reminiscent of the split predicate found in Kisi\il{Kisi} and throughout Mande\il{Mande} (\citealt{Childs2017}).

\ea%92
    \label{ex:92}
    Two basic word orders of Sherbro\\

    
	Subject-(Aux-)Verb-Object (SVO)\\
	Subject-Aux-Object\textsc{\textsubscript{pro}}{}-Verb-Other (SAuxO\textsc{\textsubscript{pro}}V)
\z

Note that this second word order occurs only when tense has moved off the lexical verb and only when the objects are pronouns.

Auxiliary verbs are those which precede the lexical verb and convey contrasts beyond the inflectional marking of the perfective-imperfective contrast on the verb itself. In addition, the auxiliary will often be marked for tense, which in the Africanistic tradition refers generally to all verbal distinctions: tense, mood, aspect, and polarity (\textsc{tmap}), although these distinctions may also be marked by pre- and post-verbal particles.

The aspectual auxiliary \textit{che} marks at least the progressive, signaling that the action is ongoing, while the modal auxiliaries (\textit{ha} (or \textit{ŋa}) ‘should,' \textit{bi} ‘have to,' and \textit{bɔ} ‘be able') indicate willingness, obligation, or possibility.

Many verbs can perform an auxiliary-like function and condition the movement of objects, here not just pronominal objects, but also full noun phrases into the slot between the two verbs. All of the verbs in this category function elsewhere as verbs with full lexical meaning, but in a pre-verbal slot they do not have that same lexical meaning but rather have a meaning more commonly associated with verbal inflections such as tense, aspect, and mood. They express distinctions similar to the inflectional categories discussed in Chapter \ref{ch:4}. Some examples appear in \REF{ex:93}.

\ea%93
    \label{ex:93} Auxiliary like lexical verbs\\
    \begin{tabular}[t]{lll}
Tense & future conveyed by the verb \textit{kɔ} ‘go'\\
Aspect & perfective by the verb \textit{ko/koŋ} ‘finish'; incipient by \textit{tipɛ}\\ 
&  ‘begin, start' and \textit{hun} ‘come' (with more of a commitment)\\
Mood & epistemic by \textit{yema} ‘want'; optative by \textit{ha/ŋa} ‘should'\\
\end{tabular}
\z

The last verb-like word category to be discussed is the copula, both when it is present and when it is not. In some constructions there is no need for a copula, but at least two copulas exist in Sherbro \textit{che} and \textit{lɔ} (alternate [lɛ]), just as in Mani\il{Mani} and Bom-Kim\il{Bom-Kim!}; in Kisi\il{Kisi} the copula is \textit{co} [ʧo].

\ea%94
    \label{ex:94}
  Zero copula\\

  Baki  wɔ  ŋkil.\\
\gll baki    wɔ      n-kil\\
Baki    \textsc{3sg}    \textsc{ncm}\textsubscript{ma}\textsc{{}-}rascal\\
\glt ‘Baki is a rascal.' (P67 K: 145)
 \z

\ea%95
    \label{ex:95}
  Copula \textit{che}\\

\ea Ya koŋ che boeo tokɛ ka ha nduɛ ŋra gbi.\\
\gll ya    koŋ    che  boo-o        tokɛ    ka    ha    n-loɛ      n-ra        gbi\\
  \textsc{1sg}  finish    \textsc{cop}  kitchen-\textsc{emph}  above    here  for     \textsc{ncm}\textsubscript{ma}{}-day  \textsc{ncm}\textsubscript{ma}{}-three  all\\
\glt ‘I have been here above this kitchen for three whole days.' (123aw Yanker, Rat Wife: 107)\\

\ex Mbolom   ŋwɛi   ma che\footnotemark{}  palɛ bay ko, anya atiŋ dɛ ha lol.\\
\gll n-bolom      n-wɛi      ma    che  palɛ          bai    ko\\
\textsc{ncm}\textsubscript{ma}{}-case    \textsc{ncm}\textsubscript{ma}{}-bad  \textsc{ncp}\textsubscript{ma}    \textsc{cop}  three.days.ago    court    to\\
\gll a-nya        a-tiŋ      ɛ    ha    lol\\
\textsc{ncm}\textsubscript{ha}{}-people  \textsc{ncm}\textsubscript{ha}{}-two  \textsc{def}  \textsc{3pl}  free\\
\footnotetext{Here the copula is used for an event that happened in the past, anterior to the main event. It is surprising that the remote marker \textit{ka} is not used.}
\glt ‘In the bad case that was before the court three days ago, the two men were freed.' (P67 L: 106.1)

\ex Ni Ba Na Che Tə Tondɛ\\
\gll ni    ba      na      che  ter    ton    ɛ\\
why  mister  spider  \textsc{cop}  waist  small    \textsc{def}\\
\glt ‘Why Mr Spider Has Such a Small Waist.' (\citealt{Sumner1921} txt:1)

\ex  …ni yekeɛ che wɔn   nyɔŋhɔl.\\
\gll ni    yeke    ɛ    che  wɔ{}-n      nyɔŋhɔl\\
with  cassava  \textsc{def}  \textsc{cop}  \textsc{3sg-emph}  mouth\\
\glt ‘…with the cassava in her mouth.' (123aw Yanker, Rat Wife: 82)
\z
\z


\ea%96
    \label{ex:96}
Copula \textit{lɔ}

\ea Wɔ lɔ nyambako.\\
\gll wɔ    lɔ    nyamba    ko\\
\textsc{3sg}  \textsc{cop}  Moyamba  to\\
\glt ‘She is in Moyamba.' (002a Mabel Lohr, Midwifery: 37)

\ex Ina lɔ ba mɔa?\\
\gll hina  lɔ    ba      mɔ  {}-a\\
who  \textsc{cop}  father    \textsc{2sg}  \textsc{q}\\
\glt ‘Who is your father?' (004a Cyril Manley on Walter Hanson: 12)

\ex  Ya lɔ bɛɛ pɔkɛ.\\
\gll ya      lɔ    bɛɛ    pɔk    ɛ\\
\textsc{1sg}    \textsc{cop}  chief    country  \textsc{def}\\
\glt ‘I am the chief of the area.' (102v Chernor Ashun: 23)
\z
\z

After this brief look at verbs, auxiliaries, and copulas in Sherbro, the reader is encouraged to look at the morphology of verbs in Chapter \ref{ch:4} and at their syntax in \sectref{sec:8.2}.

\section{Adverbs}\label{3.10}
\hypertarget{Toc115517774}{}
The section treats three types of words that may be considered adverbs, which is not a large word category in Sherbro.\footnote{Adverbs numbered 64 out of a lexicon of 4,095 entries, a number of them borrowed from English, e.g., [fainali] ‘finally.'} These are words with no morphology except reduplication, although they may show a derivational history marking their origin in other word categories and can consist of compounds (see \sectref{sec:7.1}). Their syntax is relatively straightforward: they can appear within a verb phrase but generally form a separate constituent. The first of the three categories is manner adverbs, a notionally familiar category, generally characterizing how something was done or qualifying a state. I next turn to locatives and then to temporal expressions. I also include a few words on intensifying adverbs. Ideophones have been excluded from treatment here, constituting a category of their own, although they do show some overlap with manner adverbs (see \sectref{sec:3.5} above).

\subsection{Manner adverbs}\label{sec:3.10.1}
\hypertarget{Toc115517775}{}
Some examples of manner adverbs appear in \REF{ex:97}.

\ea%97
    \label{ex:97}
 \ea \label{ex:97a}
 \textit{Wai} ‘quietly'\\
 \vspace{6pt}
Pɔ wɔ bo kɔ kɔŋ wai, pɔ sɛŋyɛ lɔni.\\
\gll pɛ      wɔ    bo    kɔ    kɔŋ  wai    pɔ      sɛŋyɛ    lɔ    ni\\
\textsc{pro}\textsubscript{indef}  \textsc{3sg}  only  go    bury  quietly  people  leave    there  then\\
\glt ‘They would just bury him quietly, then everybody would go away.' (016a Albert Yanker: 144)\footnote{If the deceased does not have a ‘clean belly,' i.e., shows no evidence of witchcraft, he will be buried with great ceremony. This is what happens to people who do not have a clean belly.}
 \vspace{6pt}
\ex \label{ex:97b}
\textit{Kilia} ‘clearly' (< English \textit{clear}) and \textit{charaŋ} ‘cleanly'\\
\vspace{6pt}
 Mbolom dɛ ma wɔni kilia ni charaŋ.\\
\gll n-bolom      ɛ    ma    wɔ      ni    kilia    ni    charaŋ\\
\textsc{ncm}\textsubscript{ma}{}-Bolom  \textsc{def}  \textsc{ncp}\textsubscript{ma}    speak    then  clearly  and  cleanly\\
\glt ‘The Sherbro language is being spoken clearly and cleanly.' (017a Boima Samba: 70)


 
\ex \textit{Leiŋ} ‘openly'\\
\vspace{6pt}
 Pɔ  tɔm feɛ, pɔ ŋɔ dikil mɛsa bom dɛ atok leiŋ.\\
\gll pɛ      tɔm    fe      ɛ    pɔ      hɔ      dikil    mɛsa  bom  ɛ    atok  leiŋ\\
\textsc{pro}\textsubscript{indef}  count    money  \textsc{def}  people  \textsc{ncp}\textsubscript{hɔ}    gather  table  big  \textsc{def}  top  openly\\
\glt ‘They are counting the money, gathering it openly on the big table.' (123aw Yanker, Rat Wife: 144)
\z
\z

Some adverbs are compounds \textit{yeŋkɛlɛŋ} ‘well' and \textit{yenwɛi} ‘badly, poorly,' both built on the word \textit{yen} ‘thing' + respectively, \textit{kɛlɛŋ} ‘good' and \textit{wɛi} ‘bad.'

\subsection{Locatives}\label{sec:3.10.2}
\hypertarget{Toc115517776}{}
Many place names are accompanied by a locative marker, be it \textit{ka} ‘here,' \textit{lɔ} ‘there,' or \textit{ko} ‘to,' as illustrated in \REF{ex:98} (remarked on in \citealt{Hanson1979a}). (See previous examples in \REF{ex:76}\ in the discussion of names). The function of the adposition \textit{ko} is discussed in \sectref{sec:3.8}. The second example features \textit{ko} as a simple locative meaning ‘yonder' \REF{ex:98b}.

\ea%98
    \label{ex:98}
    \ea \label{ex:98a} Anyaɛ kani gbo che vel yelloɛ, “Yelnsaŋhako.”\\
	\gll a-nya        ɛ  ka      ni    gbo  che  vel  yel    lo    ɛ   yel-nsaŋha-ko\\
	\textsc{ncm}\textsubscript{ha}{}-people \textsc{def}  \textsc{rem.pst}  then  just  \textsc{aux}  call  island    this  \textsc{def} island-egusi-\textsc{loc}\\
	\glt ‘The people were only now calling this island “Island of Egusi.”' (124aw Yanker, Boy Lost at Sea: 18)

	\ex \label{ex:98b} Kɛ haŋa pim nke ŋa ko nyuni, ŋa ye ma ni bɛ pɛ hɔ Mbolom.\\
	\gll kɛ    ha-ŋ      a-pum      n    ke    ŋa    ko      nyuni\\
	but    \textsc{3pl-emph}  \textsc{ncm}\textsubscript{ha}{}-some  \textsc{2sg}  see  \textsc{3pl}  yonder  move\\
	\gll ha    ye    ma    ni    bɛ    pɛ      hɔ      n-bolom\\
	\textsc{3pl}    then  should  \textsc{neg}  even  again    speak    \textsc{ncm}\textsubscript{ma}{}-bolom\\
	\glt ‘But some people you see them move to other places, they don't even speak Sherbro anymore.' (009--10a Lohr \& Mampa: 325)\\
\z
\z

Locative expressions are multiple and varied, some more deictic than others.\footnote{Roughly 41 expressions in a lexicon of 4,095 entries.}

\begin{table}
\caption{\label{tab:wordcat:21}Sherbro locative expressions}


\begin{tabular}{ll}
\lsptoprule
kahai & ‘outside'\\
veleŋ & ‘behind, outside'\\
poloŋ & ‘far away'\\
tokɛ & ‘up, above, high'\\
(n)tɛnt & ‘nearby'\\
lel & ‘across, on the other side'\\
chɛthlipalkɔ & ‘west' (lit. ‘sunset-sun-go')\\
\lspbottomrule
\end{tabular}
\end{table}

The words for ‘left hand' and ‘right hand' are both compounds, the first part being ‘hand,' as in \REF{ex:99}. The word for ‘left' is related to the word for ‘spirit, ghost, devil\is{devils!},' and ‘right' has to do with the word for ‘eat' since everyone eats with their right hand.\footnote{Lefthanders as children are discouraged from using their left hand for anything, but especially for eating, since the left hand is used in the toilet.}

\ea%99
    \label{ex:99} Left and right\\
  \ea \label{ex:99a} piamindɛ\\
  \gll pia{}-min{}-ɛ\\
  hand-spirit-\textsc{def}\\
  \glt ‘the left (hand)'

\ex \label{ex:99b} pianjokɛ\\
  \gll {pia}{ -n}{-jo}{ -ka}{ -ɛ}\\
  {hand}{-\textsc{NCM}}{\textsubscript{ma}}{-eat}{-\textsc{ins}}- \textsc{def}\\
  \glt ‘the right (hand)'
\z
\z

A word that is something like an ideophone and spans the divide between locative and temporal expressions is the form \textit{haaa} <haaŋ>, which is written with three vowels to represent the expressive lengthening that so often characterizes the word. Its vowel is heavily nasalized. Its meaning is ‘going on for some time or distance' and usually appears in final position. It is an areal word used not only in totally unrelated languages but also in the pidgins and creoles of West Africa. In \REF{ex:100a}, it is used to characterize the extension of an activity (fish smoking); in \REF{ex:100b} it refers to the extended period of growing up.


\ea%100
  \label{ex:100}
  \textit{Haaa} as both locative and temporal\\
  \ea \label{ex:100a}  Mpanthɛ ɡbi ma mɔ ŋaɛ, wok ka ko pindɛ haŋ...\\
  \gll n-panth      ɛ    ɡbi  ma    mɔ  ŋaa-ɛ    wok  ka    ko    pin  ɛ    haaa\\
  \textsc{ncm}\textsubscript{ma}{}-work  \textsc{def}  all    \textsc{ncp}\textsubscript{ma}    \textsc{2sg}  do-\textsc{prt}  start  here  to    buy  \textsc{def}  extend\\
  \glt ‘All the work you do, starting from the buying going on...' (184v Fish Smoking Seaport: 24)

\ex \label{ex:100b} Ka lɔ pɔ dumɔ mɔ haŋ nko gbako?\\
  \gll ka    lɔ    pɛ      dumɔ    mɔ  haaa  n    ko      gbako\\
  here  \textsc{ncp}\textsubscript{lɔ}  \textsc{pro}\textsubscript{indef}  raise    \textsc{2sg}  until  \textsc{2sg}  finish    grow\\
  \glt ‘Did they raise you here until you grew up?' (004a Cyril Manley on Walter Hanson: 16)
  \z
  \z

A similar areal word is \textit{tee} ‘until,' which can also be expressively lengthened to iconically convey extensive duration.

\subsection{Temporal expressions}
\label{sec:3.10.3}\hypertarget{Toc115517777}{}
Sherbro has a rich set of temporal distinctions, particularly for the future (the first group in \tabref{tab:wordcat:22}). The language divides the day into segments with reference to the sun \textit{li-pal} or \textit{pal-(l)i}, as illustrated by the third group.

\begin{table}
\caption{\label{tab:wordcat:22}Temporal expressions}



\begin{tabular}{ll}
\lsptoprule
nante & ‘today'\\
gbɛŋ & ‘tomorrow'\\
jɛk & ‘day after tomorrow, next tomorrow'\\
jith & ‘next next tomorrow, day after the day after tomorrow'\\
joth & ‘four days hence'\\
paaɛ & ‘in two to four weeks'\\
vɛɛthɛɛ & ‘in one to six months'\\
\tablevspace
chencha & ‘yesterday'\\
palɛ & ‘three or more days ago' (\textit{pa} ‘in the past')\\
nɛnveleŋ & ‘last year' (lit. ‘year behind')\\
kache & ‘after twenty years and beyond, formerly'\\
\tablevspace
isɔ & ‘in the morning'\\
palisɔ & ‘morning'\\
palpal & ‘noon'\\
palikasabul & ‘afternoon'\\
chɛthɛ & ‘at sunset'\\
palichɛthɛ & ‘sunset'\\
\lspbottomrule
\end{tabular}
\end{table}

There are a number of time words that have been considered part of the verbal morphology and only mentioned here, namely, \textit{na} ‘recent past' and \textit{ka} ‘remote past'. The form \textit{pa} also means ‘in the past' but is not so widely used and is not fully integrated into the verbal system (cf. \textit{palɛ} in \tabref{tab:wordcat:22}).

Some words for months of the year were identified, but they do not correspond to Western months and seem artificial. The main yearly divisions are between the dry and rainy seasons and vary as to whether people fish (coast) or farm (interior).\footnote{A recent development, likely due to climate change, is a very short (less than two weeks) rainy interlude on the coast during the dry season, accompanied by some fierce winds.} The exact times of the “months” also vary between the coast and the interior. Roughly speaking, and with these qualifications in mind, the rainy season begins in May or June and lasts through August or September. In farming areas, people will speak of a planting time and a harvest time, and also a time for brushing and clearing a farm and a time for burning the fields after they have been brushed. The practices vary as to the type of rice being cultivated, swamp rice or the upland variety. The word \textit{hɔl} used in some of the names in the following two tables comes from a word that means ‘mouth, door, start.'

\begin{table}
\caption{\label{tab:wordcat:23}The major seasons}



\begin{tabular}{ll}
\lsptoprule
sal (lisal) & ‘rainy season'\\
sai & ‘dry season'\\
sirɔkɔhɔl & ‘harvest time' (around September)\\
\lspbottomrule
\end{tabular}
\end{table}

The correspondences to Western months are inexact, as mentioned above. For example, \textit{poto} was said to be ‘April-May' or ‘summertime,' and \textit{potohɔl} was said to be ‘springtime' as well as the ‘beginning of summer,' ‘the end of March,' and ‘June' (\citealt{Pichl1967}). What follows are some very rough correspondences.

\begin{table}
\caption{\label{tab:wordcat:24}Some months}



\begin{tabularx}{\textwidth}{lQ}
\lsptoprule
vɛlvɛl & ‘January'\\
báŋkèlèn & ‘March'\\
poto & ‘April-May'\\
pothɔhɔl & ‘June'\\
gbiminte & ‘July'\\
basmanchin & ‘August' (lit. ‘sweep the fields,' when the rains sweep the fields clean) (E13 Albert Yanker, Adj, Lex: 26)\\
saa & ‘September' (‘escape' from the rainy season)\\
saihɔl & ‘December' (lit. \textit{sai} ‘dry season' + \textit{hɔl} ‘mouth, opening')\\
\lspbottomrule
\end{tabularx}
\end{table}

\subsubsection{Emphatics and intensifying adverbs}
\label{3.10.3.1}
\hypertarget{Toc115517778}{}
A number of words qualify as emphatics or intensifying adverbs distinct from ideophones and other adverbs. Their semantics are less specific, and there are fewer selectional restrictions on where they may occur. They usually denote the intensity\is{intensity!} of a phenomenon or its quantity. One such morpheme has already been introduced in the discussion of mostly personal pronouns, the suffix \textit{{}-n} (see \sectref{sec:3.3.1}). The list in \tabref{tab:wordcat:25} represents those that have full lexical status.

\begin{table}
\caption{\label{tab:wordcat:25}Emphatics and intensifiers}



\begin{tabular}{ll}
\lsptoprule
ba & emphatic\is{emphatic!}\\
be & ‘just, only, indeed'\\
bɛ & ‘even, also, just'\\
bo & emphatic\is{emphatic!} or intensifying adverb\\
gbɔ & ‘excessively'\\
ɡbɛt & ‘only'\\
gbi & ‘all'\\
gbo & ‘indeed, quite, just'\\
vuli & ‘very'\\
\lspbottomrule
\end{tabular}
\end{table}

The phonological and semantic similarities between many of the items in \tabref{tab:wordcat:25} suggests there may be some unidentified overlap.

\subsection{Conjunctions}
\label{sec:3.10.4}\hypertarget{Toc115517779}{}
Sherbro has a wide range and variety of conjunctions, not all of which will be discussed here.\footnote{In a lexicon of 4,095 entries, there were roughly 27 coordinating and 40 subordinating conjunctions.} The flexibility or versatility of Sherbro conjunctions applies at least to the distinction between subordinating and coordinating conjunctions. The conjunction \textit{ni} is one such element, serving at times to join coordinate clauses with a meaning of ‘and,' at others to subordinate one clause to another ‘that, so that,' especially after verbs such as \textit{yema} ‘want' that take full sentence complements. The conjunction \textit{ni} can also be used to connect smaller syntactic units. Curiously, \textit{ni} has a counterpart \textit{si}, which serves exactly the same functions. Both can be used by the same speaker in apparent free variation.

\ea%101
  \label{ex:101}
  Coordinating \textit{ni}\\
  \ea Wanthɛmdɛ ka le blid te-e-e ni hu.\\
  \gll wanthɛm      ɛ    ka      le    blid      tee        ni    hu\\
  young.woman  \textsc{def}  \textsc{rem.pst}  stay  bleeding    on.and.on  and  die\\
  \glt ‘The woman kept on bleeding until she died.' (002a Mabel Lohr, Midwifery: 91)

\newpage
  \ex Mbolomdɛ … ni nthemdɛ handɔ mapɔ chaŋ thelia?\\
  \gll n-bolom        ɛ     ni    n-them        ɛ    handɔ    ma    pɛ      chaŋ  theli a\\
  \textsc{ncm}\textsubscript{ma}{}-Sherbro  \textsc{def}  or    \textsc{ncm}\textsubscript{ma}{}-Themne  \textsc{def}  which  \textsc{ncp}\textsubscript{ma}   \textsc{pro}\textsubscript{indef}  pass  speak  \textsc{q}\\
  \glt ‘Sherbro … or Themne which do they speak more?' (029a Biah Heni: 65)

  \ex Wɔn ni nɔmaɛ ŋa gbem?\\
  \gll wɔ{}-n      ni    nɔmaa  ɛ    ŋa    gbem\\
  \textsc{3sg}{}-\textsc{emph}  and  woman  \textsc{def}  \textsc{3pl}  give.birth\\
  \glt ‘He and the woman do they have children?' (007a Agnes J. Simbo: 76)
\z
\z

\ea%102
  \label{ex:102}
  Subordinating \textit{ni}\\
  \ea Nthɛkɛsiɛ wɔ   ni san la ntenɛ.\\
  \gll n    thɛkɛsiɛ  wɔ    ni      san  la      n-ten        ɛ\\
  \textsc{2sg}  clarify  \textsc{3sg}  so.that  get  \textsc{pro}\textsubscript{indef}  \textsc{ncm}\textsubscript{ma}{}-sense  \textsc{def}\\
  \glt ‘You clarify things for him to get an understanding.' (009--10a Lohr \& Mampa: 299.1)

  \ex Kenda ŋɔ awɔmɔ mɔ boɛ, awɔ ŋalmɔ, wɔlɔŋ mɔɛ, lagbo mɔla yema ni nɔ ndɔndɔ thela.\\
  \gll kend    la      hɔ      a    wɔ    mɔ  mɔ  bo    {}-ɛ\\
  be.like  \textsc{pro}\textsubscript{indef}  \textsc{ncp}\textsubscript{hɔ}    \textsc{1sg}  tell  \textsc{2sg}  \textsc{2sg}  just  \textsc{prt}\\
  \gll a    wɔ    ŋal    mɔ    wɔlɔŋ    mɔ  ɛ\\
  1\textsc{sg}  say  about    \textsc{2sg}    life    \textsc{2sg}  \textsc{def}\\
  \gll lagbo  mɔ  la      yema    ni    nɔ      ndɔndɔ    the  la\\
  if    \textsc{2sg}  \textsc{pro}\textsubscript{indef}  want    for    person  everyone  hear  \textsc{pro}\textsubscript{indef}\\
  \glt ‘As I just told you, I [will] ask about you, your life, if you want that and for people to hear that.' (018a Suffian Koroma: 7)\footnote{This is another example of the permission that would be sought for recording, transcribing, and disseminating the data.}
\z
\z

The conjunction \textit{yɛ} is exclusively a subordinating conjunction. It usually appears at the beginning of the clause, and the clause has a final particle \textit{ɛ} (designated \textsc{prt} in \REF{ex:103}). The conjunction itself can appear elsewhere at the beginning of the clause, but always before the verb or tense-bearing element, and the final particle will sometimes be absent. Nonetheless, the initial position and final particle mark it as distinctive within the set of words that could be interpreted as subordinating conjunctions.

\ea%103
    \label{ex:103}
  \ea Yɛmɔ kɔni hɛlɛ koɛ, mɔ lɔ kɔ lɔl?\\
  \gll yɛ      mɔ  kɔni  hɛlɛ  ko-ɛ    mɔ  lɔ    kɔ    lɔl\\
  when    \textsc{2sg}  go    sea  to-\textsc{prt}  \textsc{2sg}  there  go    sleep\\
  \glt ‘When you go out to sea, do you sleep there (on the boat)?' (029a Biah Heni: 95)

\ex Yɛ  nkache ko tallɛ, nkache siŋ?\\
    \gll yɛ      n    ka      che  ko    taa      ɛ    n    ka      che  siŋ\\
    when    \textsc{2sg}  \textsc{rem.pst}  \textsc{cop}  to    youth    \textsc{prt}  \textsc{2sg}  \textsc{rem.pst}  \textsc{aux}  play\\
    \glt ‘When you were young, did you used to play?' (029a Biah Heni: 75)
\z
\z

A widely used coordinating conjunction is \textit{kɛ}, roughly translated as ‘but,' although it has more functions than its equivalent in English. It sometimes occurs at the beginning of an utterance with no reference to what has been said before. In \REF{ex:104}, a famous fisherman talks about the subsidence of Plantain Island\is{Plantain Island!} and the encroaching sea.

\ea%104
    \label{ex:104}
    Che wɔiowɔi-o, kɛ yɛ helendɛ ŋɔ che vɛ, ŋɔ mɛndɛ ma thaŋ tokɛtokɛ, mai, nyathi lɛllɛ.\\
    \gll che  wɔi-o-wɔi    o      kɛ    yɛ      hɛliŋ      ɛ    hɔ      che  vɛ\\
    be    day-\textsc{distr}{}-day  \textsc{emph}    but  when    high.tide    \textsc{def}  \textsc{ncp}\textsubscript{hɔ}    \textsc{cop}  so\\
    \gll ŋɔ    mɛn    dɛ    ma    thaŋ    tokɛtokɛ    ma-i      nyathi  lɛl    ɛ\\
    how  water    \textsc{def}  \textsc{ncp}\textsubscript{ma}    climb    high      \textsc{ncp}\textsubscript{ma}{}-\textsc{prt}  lick    land  \textsc{def}\\
  \glt ‘It is not every day-o, but when it is high tide, the water climbs high and licks the land' (142v Baba Mandela, Fishing: 45–46)
\z

The relation between independent clauses may go unstated and must be inferred, as in \REF{ex:105}.

\ea%105
    \label{ex:105}
    Puinɔ lɛ chala tholɛai wɔ mire   challɛ.\\
    \gll puinɔ    lɛ    chala    tho  lɛ    ai    wɔ    mire        chal  lɛ\\
    hunter  \textsc{def}  sit      bush  \textsc{def}  in    he    watch.closely  deer  \textsc{def}\\
    \glt ‘The hunter sits in the bush (and) watches the deer.' (P67 M: 79)
\z

Here the two clauses are simultaneous, with both contributing equally to a characterization of the scene; an indication of coordinate structures would be expected.

\section{Particles}
\label{sec:3.11}\hypertarget{Toc115517780}{}
\textit{Particle} is something of a catch-all category containing two sets of items. Those in the first set generally have a discourse function and cannot be assigned a function more deeply embedded in the grammar. Typically, they are phonologically dependent, behaving something like clitics appearing utterance finally but sometimes within a clause (\tabref{tab:wordcat:26}).

\begin{table}
\caption{\label{tab:wordcat:26}Sherbro discourse particles}


\begin{tabular}{lll} 
\lsptoprule
& Function & Position\\
\midrule
 {}-e & vocative & after child's name, replacing final V\\
 {}-o & emphatic & after clause or item emphasized\\
 {}-we & emphatic & after clause or item emphasized\\
 {}-i &  & after \textsc{tns} (\textsc{pro}\textsubscript{obj}), see \sectref{sec:8.2.1}\\
\lspbottomrule
\end{tabular}
\end{table}


Another particle appears commonly at the end of a clause and is more accurately characterized as an areal phenomenon rather than a feature solely of Sherbro. This is the emphatic\is{emphatic!} particle \textit{{}-o} which is used for emphasis and contrast, as in \REF{ex:106} (see \citealt{Singler1988b}).\footnote{Emphatic –o should be differentiated from euphonic [o], both of which appear prolifically in Christian hymns.}

\ea%106
    \label{ex:106}
    West African emphatic\is{emphatic!} \textit{{}-o}\\
  \ea Beo,  a  bo  pin  agbaŋ    ŋa.\\
  \gll be    o      a    bo    pin  a    gbaŋ    ŋa\\
  no    \textsc{emph}    \textsc{1sg}  only  buy  \textsc{1sg}  spread  \textsc{3pl}\\
  \glt ‘No, I just buy and smoke them.' (004a Cyril Manley on Walter Hanson: 52)

  \ex Ya wɔ hin, ya wɔ pabondɛ Mɔmi Prat wɔɛ, a cheŋ kɔo.\\
  \gll ya    wɔ    nyin    ya    hɔ    pabondɛ    mɔmi      Prat  wɔ    ɛ\\
  \textsc{1sg}  say  people  \textsc{1sg}  say  if        Mommy    Pratt  3\textsc{sg}  \textsc{prt}\\
  \gll a    che-ni      kɔ    o\\
  \textsc{1sg}  \textsc{aux-neg}  go    \textsc{emph}\\
  \glt ‘I said to him, I said that if it is Mummy Pratt, I'm not going.' (002a Mabel Lohr, Midwifery: 65)
\z
\z

The particle \textit{wei} can be attached to politeness items such as salutations and leave-takings to add a note of “friendliness.” It has a number of variants: [we], [wei], and [wɛi]. Its source is probably Mende\il{Mende}.\footnote{A co-worker and native speaker of Mende\il{Mende} said this was something friendly that had to be added to an utterance (Taziff Koroma 2015, personal communication).} It is also found in Soso\is{Soso!}, another Mande\il{Mande} language, and in Mani\il{Mani}, whose speakers are switching to Soso.

\ea%107
    \label{ex:107}
    Friendly \textit{{}-we} (\textit{ex} Mende\il{Mende}?)\\
  \ea wɔsowei\\
  \gll wɔso-wéí\\
  goodbye-\textsc{emph}\\
  \glt ‘goodbye-o' (E01 Abdulai Bendu: 15)

  \ex Kɛ, apa lagbowɛwe.\\
  \gll kɛ    pa      lagbowɛ    we\\
  well  father    goodbye    \textsc{emph}\\
  \glt ‘Well, Pa, goodbye.' (028a Yusuf Fofana: 98)

	\ex  Sɛkɛ, sɛkɛ we ŋa yɛ mɔ luŋnui koniko wɛ.\\
  	\gll sɛkɛ    sɛkɛ    we      ŋaa  yɛ    mɔ  luŋnui  kohiko  wɛ\\
  thanks  thanks  \textsc{emph}    \textsc{for}  how  \textsc{2sg}  listen    to.us    \textsc{prt}\\
  \glt ‘Thanks, thanks very much for listening to us.' (028a Yusuf Fofana: 99)
\z
\z

The vocative particle, often used for calling small children, is present here as it is in Kisi\il{Kisi} and likely is an areal feature (\citealt{Childs1995}). It is suffixed to names and can replace a final vowel.

\ea%108
  \label{ex:108}
  Vocative\\
A mother calling \textit{Marco} [maako]\textit{, Marco-e!} [maakoˈee] / [maakˈee]\\
A mother calling \textit{Augusta} [agusta]\textit{, August-e!} [agusˈtee]
\z

The particle \textit{{}-i} is discussed in \sectref{sec:8.2.1}, as a discourse phenomena.\\

The set of particles in \tabref{tab:wordcat:27} is more grammatically integrated. The first sub-division contains non-verbal particles, and most of the second sub-division are phonologically independent morphemes syntactically bound within the verb phrase. They are discussed in detail in Chapter \ref{ch:4} on verbal morphology.

\begin{table}
\caption{\label{tab:wordcat:27}Sherbro grammatical particles}



\begin{tabular}{llll}
\lsptoprule
& & Function & Position\\
\midrule
& n/ŋ & emphatic & suffixed to pronouns (\sectref{sec:3.3.1})\\
& a & question particle & question finally (\sectref{sec:3.3.4})\\
& ɛ & binding particle & clause finally (\sectref{sec:3.11})\\
& a & quotative & before quoted material\\
\tablevspace
% \ex
& na & near past & post-verbal (\sectref{sec:4.3.1})\\
& ka & remote past & pre-verbal (\sectref{sec:4.3.2})\\
& ha & optative & in \textsc{aux} slot (\sectref{sec:4.4})\\
& ma & negative optative & in \textsc{aux} slot (\sectref{sec:4.4})\\
& ni & negative & after tensed element (\sectref{sec:4.5})\\
\lspbottomrule
\end{tabular}
\end{table}

The next chapter discusses these verbal particles, as well as verbal inflections.
