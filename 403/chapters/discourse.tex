\chapter{Discourse, pragmatics}
\label{ch:10}\hypertarget{Toc115517823}{}
This chapter consists of a set of observations falling into the general domains of discourse and pragmatics. I start off with some observations about face-to-face interactions. The issues were not investigated systematically and thus stand as an area for future investigation. I then discuss the use of discourse particles and formulaic expressions.

Politeness and indirection dominate interpersonal interaction, especially if it takes place in public. Face-threatening actions are couched in oblique terms. For example, one rarely asks for anything directly. One might make reference to the high cost of medical treatment for a member of one's family, for example, rather than ask for assistance when a relative needs treatment and there is no money to pay for it. Another area where politeness controls interaction is in attributing agency or blame to bad events. If someone accidentally hurts themselves, it is the instrument that is blamed, as in \REF{ex:241}. Although it is stated very generally that Sese hurt himself, the specific cause is blamed on the adze.

\ea%241
    \label{ex:241}
      Sese theyɛn-nɛki; thɔ lɛ kəth wɔ   yenwɛy.\\
      \gll Sese  theyɛn-nɛki    thɔ  lɛ    kɛth  wɔ   yenwɛy\\
      Sese  hurt-oneself    adze  \textsc{def}  cut  \textsc{3sg}  badly\\
      \glt ‘Sese hurt himself; the adze badly cut him.' (P67 TH: 81)
\z

A second general observation affecting discourse is that children are not held responsible for their behavior and are allowed a great deal of freedom. Children were always interested in what the research team was doing, as members attested and sometimes complained about. They definitely helped us in our language learning when they spoke Sherbro. Unfortunately, many of them in the town where we lived did not speak Sherbro, preferring Krio\il{Krio}.

Children are often thought of as something akin to chickens that run about acting crazily and totally irresponsible. A young girl “stole” a watch from a member of the team, an act which appalled the Westerners, but we found the community and the little girl totally nonplussed. In fact, she had worn it to an all-female weeding session out in the fields, saying that the researcher had given it to her. When I asked the parents about it, they did not seem to care. They said it was what was expected from children. Responsibility did not settle in until children had graduated from bush school. Thus, children were allowed a great deal of latitude in their behavior generally, but specifically in how they interacted with us, sometimes having to be shooed away.

The expectations for children who had gone through society education were much higher. One very bright girl who had been recommended by the school principal quickly learned how to write her language and helped with the transcription.

\section{Discourse particles}
\label{sec:10.1}\hypertarget{Toc115517824}{}
\-The particle \textit{{}-o} is likely related to the same emphatic\is{emphatic} particle found throughout West Africa and discussed in \sectref{sec:3.11} (\citealt{Childs1995}, \citealt{Singler1988b}). There are two emphatic articles used at the end of an utterance: \textit{{}-e} and \textit{{}-o}.\footnote{The front-back alternation is common in the language, especially with the upper mid and high vowels, as first discussed in \sectref{sec:2.1.1}.} The two particles seem identical in function and exist in apparent free variation, as shown in a hymn from the Shenge\is{Shenge} Youth Choir.

\ea%242
    \label{ex:242}
      \ea  Velia mi yo! Jizɔs velia mi we!\\
      \gll velia  mi    yo      Jizɔs    velia  mi    we\\
      save  \textsc{1sg}  \textsc{emph}    Jesus    save  \textsc{1sg}  \textsc{emph}\\
      \glt ‘Save me-o! Jesus, save me-e!' (003a Shenge Youth Choir, Hymns: 101)

      \ex Velia mi we! Jizɔs velia mi yo!\\
      \gll velia  mi    we      Jizɔs    velia  mi    yo\\
      save  \textsc{1sg}  \textsc{emph}    Jesus    save  \textsc{1sg}  \textsc{emph}\\
      \glt ‘Save me-e! Jesus, save me-o!' (003a Shenge Youth Choir, Hymns: 104)
\z
\z

There is also the “friendliness” morpheme \textit{wei}, likely borrowed from a Mande\il{Mande} language, used throughout the region (\citealt{Childs2011}). It is also glossed as “\textsc{emph}.”

\ea%243
    \label{ex:243}
    Emphatic\is{emphatic} friendliness morpheme \textit{wei}\\
   \ea Wɔsowei!\\
   \gll wɔso-wei\\
   goodbye-\textsc{emph}\\
 \glt ‘Goodbye!' or ‘Good evening!' (with emphasis)\\
      
    \ex  So sɛkɛ wei, Abatokɛ chema mɔni.\\
      \gll so    sɛkɛ    wei    Abatokɛ    che  ma  mɔ-n-i\\
      so    thanks  \textsc{emph}    God      be    with  \textsc{2sg-emph-prt}\\
      \glt ‘So thank you, may God be with you.' (002a Mabel Lohr, Midwifery: 111)
\z
\z

Another emphatic\is{emphatic} particle is \textit{bo} ‘indeed, just.'

\ea%244
    \label{ex:244}
    Yaŋ bɛ agbem bo apumma mɛn.\\
      \gll ya-ŋ      bɛ    a    gbem    bo      a-pum        ma-mɛn\\
      \textsc{1sg-emph}  self  \textsc{1sg}  bear    \textsc{emph}    \textsc{ncm}\textsubscript{ha}{}-children  \textsc{ncm}\textsubscript{ma}{}-five\\
      \glt ‘Myself I gave birth to five children.' (017a Boima Samba: 62)
\z

As mentioned in \sectref{sec:3.3}, on pronouns and illustrated by the two preceding examples, all pronouns may be suffixed with an “emphatic\is{emphatic}” nasal.

A number of discourse markers, all with the same function are listed in \REF{ex:245}.

\TabPositions{1cm,4cm,6cm,8cm}

\ea%245
    \label{ex:245}
    Discourse markers\\

ayo  \tab  ‘okay'

awa  \tab  ‘okay' (from Mande\il{Mande}, e.g., Soso\il{Soso}, Mende\il{Mende})

oke  \tab  ‘okay' (from English, Krio\il{Krio})
\z

There is a tag question, \textit{nyɛ} ‘right? not so?,' appearing at the end of sentences that is likely an areal phenomenon. A number of discourse particles are indeed borrowed not just from Mande\il{Mande} languages. Whether English-sounding ones come from Krio\il{Krio}, an English-based creole, or from English itself can be difficult to determine. Because the Bolom substrate allows for closed syllables, unlike Mande and Kru, less obvious phonological differences exist between the source form and its realization than between such pairs in the French of Guinea and Liberian English (\citealt{Childs1999}, \citealt{Childs2002b}, \citealt{Singler1988a}).

\section{Formulaic expressions: Greetings and other common phrases}
\label{sec:10.2}\hypertarget{Toc115517825}{}
The importance of greetings in West Africa cannot be exaggerated. The list in \REF{ex:246} is an abbreviated one for there are many expressions inquiring about one's health, one's family, etc. It is considered extremely rude not to at least greet someone on first sighting, if not inquire as to the interlocuter's health. Responses and follow-up are equally as important. One of our recordings features a court case in Moyeamoh, Bumpeh Chiefdom\is{Bumpeh Chiefdom}, where the first ten minutes is taken up with greetings and small talk, and only after those exchanges can the court case begin.

\TabPositions{2cm,3cm,6cm,8cm}

\ea%246
    \label{ex:246}
    Greetings\\
      nsaka \tab morning greeting\\
      mɔɛ \tab afternoon greeting (used any time of day in Dema Chiefdom)\is{Dema}\\ \tab \\
      mɔlɔ \tab daytime greeting\\
      mpikɛ \tab late afternoon\\
      wɔsowei! \tab ‘Goodbye!' or ‘Good evening!'\\
      lagbo \tab ‘Goodbye'\\
      sakao \tab a response to all greetings, something like ‘thank you'\\
      sɛkɛ(-sɛkɛ) \tab ‘thanks, thank you'\\
\z

The single word greetings in \REF{ex:246} that vary as to the time of the day can be replaced or accompanied by longer enquiries as in \REF{ex:247}.

\ea%247
    \label{ex:247}
    Expanded greetings and follow-ups
    
\ea N jallɛ a?\\
‘How is the body?' (in Krio\il{Krio} \textit{Aw di bɔdi?}, a common greeting)\\
\vspace{6pt}
\ex N lɔllɔ ɲɛŋkɛlɛŋ?\\
‘Did you sleep well?'\\
\vspace{6pt}
\ex A chɔŋɔ Hobatokɛ.\\
‘I give thanks to God.' (response to either one of the preceding questions)\\
\vspace{6pt}
\ex A chɔŋɔ Bɛi bullɛ sɛkɛ ya po ni velɛ.\\
‘I give thanks to God that I wake up healthy.' (ditto)\\
\vspace{6pt}
\ex Hɔbatokɛ ŋɔ che mɔ.\\
‘May God be with you.' (common leave-taking)
\z
\z


Folktales have set features as well. The standard opening of a story, \textit{thɛnothɛn} (the distributive form of \textit{thɛn} ‘each story, every story'), can be followed by another \textit{thɛn} after a pause, as in the second line, as in \REF{ex:248}. Note the use of the emphatic\is{emphatic} particle \textit{{}-o} in the following phrase.

\ea%248
    \label{ex:248}
    Formulaic story-telling expressions\\
    
\ea Thɛnothɛn… \\
‘Once upon a time…' lit. story-o-story (standard opening to a story)\\

\ex Thɛnothɛn thɛn po mbawom-o. \\
‘Once upon a time, a story arose from the ancestors-o.' (possible following phrase)\\

\ex Lɔntha ya mɛkɛni. \\
‘I have reached the end.' (standard closing to a story)\\

\ex rɛkɛ-rɛkɛ / lɛkɛ-lɛkɛ gbut \\
‘The end.' (another way to end a story)\\
\vspace{6pt}
\ex La boɛ lɛkɛ-lɛkɛ mgbut.\\
\gll la      bo-ɛ      lɛkɛlɛkɛ    n-gbut\\
\textsc{pro}\textsubscript{indef}  just-\textsc{prt}    over      \textsc{ncm}\textsubscript{ma}\textsc{{}-}end\\
\glt ‘That's it, the story ends.' (123aw Yanker, Rat Wife: 189)
\z
\z

I conclude this section with some formulaic expressions that are either idiomatic and/or involve taboo\is{taboo}s.

A polite way of saying ‘to relieve oneself' is \textit{kɔnaibol}, lit. ‘go to the front of the road.' The term \textit{kunɛdinthɛ} ‘clean belly' is the descriptor for someone deceased who did not practice witchcraft or cannibalism. The verdict is rendered by leaders of Poro\is{Poro} after they have examined the entrails of the corpse (see 016a Albert Yanker: 129–149 for details and Appendices \ref{app:f} and \ref{app:g} for a list of speakers and recordings).

Another expression referencing the stomach is \textit{kunputhul}, which means ‘gluttonous, eating more than one's fair share' lit. ‘rotten belly'. Compare this with \textit{gbɔlkajo} ‘gluttonous' lit. ‘heart (obsessed?) with food'. A great number of idioms reference the heart, listed in \REF{ex:249}:

\ea%249
    \label{ex:249} Idioms involving \textit{gbɔl} ‘heart'\\
\vspace{6pt}
gbɔlthukul \tab ‘quick to anger,' lit. ‘hot heart'\\
\tab (cf. \textit{mgbɔlnthuk} ‘madness' lit. ‘feverish heart')\\
gbɔlbom \tab ‘a proud person,' lit. ‘big heart'\\
gbɔlmafe \tab ‘avaricious,' lit. ‘heart (concerned) with money'\\
hiŋgbɔl \tab ‘be satisfied,' lit. ‘lie down heart'\\
hinigbɔl \tab ‘satisfaction,'  lit. ‘make lie down heart'\\
lanthgbɔl \tab ‘anxiety,'  lit. ‘hang heart'\\
mingbɔl \tab ‘dead,' lit. ‘swallow heart'\\
simgbɔljɛm \tab ‘discourage,' lit. ‘stand heart fire'
\z
