\chapter{Verb extensions}
\label{ch:6}\hypertarget{Toc115517799}{}
Verb extensions are suffixed to verb stems, following the Proto-Niger-Congo pattern of the “inflectional verb” as outlined in \citet{Nurse2007} and \citet{Nurse2008}, an amazingly stable pattern throughout Niger-Congo (\citealt{Hyman2007}). Though the system of verb extensions exists and is somewhat productive in Sherbro, it is a reduced version of what exists even in closely related Kisi\il{Kisi}.

This chapter introduces the more productive verb extensions of Sherbro, treating them in the order of most productive to less productive. The causative \textit{{}-i} is the most productive, followed by the intensiver/reflexive \textit{{}-ni} and the instrumental \mbox{\textit{{}-ka}}, which has been reanalyzed as a preposition. Others are less well established. I first give an overview of their form and semantics with some examples, followed with some details of each extension. The chapter concludes with a comment on the morphotactics of the verb extension system, which extensions can be used together and in what order they appear, and a brief summary.

\section{Causative \textit{-i}}
\label{sec:6.1}\hypertarget{Toc115517800}{}
The causative remains intact from what was likely once a more extensive and productive system of verb extensions than exists today. The form of the extension, \textit{{}-i} is identical to what is found in other related languages (e.g., \citealt{Childs1987}). The causative adds to the base form an additional argument, an agent or a causee. The function is most apparent when the base verb is stative.

\TabPositions{1.5cm,5.5cm,7cm,8cm}

\ea%158
  \label{ex:158}
  Causative extension

  \ea
  hin \tab ‘lie down' \tab hini \tab ‘lay down, arrange'\\
  dinth \tab ‘gleam' \tab dinthi \tab ‘whiten'\\
  lap \tab ‘be ashamed, shy' \tab lepi\footnotemark \tab ‘disgrace'\\
  duk \tab ‘fall' \tab duki \tab ‘fell (tr.), throw down'\\
  lol \tab ‘be safe, saved' \tab loli \tab ‘save'\\
\vspace{6pt}
  \ex Wɔ koŋ mi loli.\\
  \gll wɔ    koŋ  mi    lol-i\\
  \textsc{3sg}  \textsc{pfv}  \textsc{1sg}  safe-\textsc{caus}\\
  \glt ‘He saved me.' (P67 L: 108)
\z
\z
\footnotetext{This is one of the few places where there is a stem vowel change in the verb extension system. In Kisi,\il{Kisi} it is quite common, and in both Mani\il{Mani} and Kisi, the verb extensions harmonize with the stem vowel. There is no vowel harmony\is{vowel harmony} here in Sherbro.}

The causative may also have a pluractional sense, for example, changing \textit{kɛnth} ‘break' into \textit{kɛnthi} ‘break into pieces'. The pluractional sense also includes marking greater intensity\is{intensity}, as in \REF{ex:159b}.\footnote{As I mentioned in an earlier paper, Kuryłowicz has pointed out the close relationship among the iterative, the intensive, and the causative in Indo-European. In many languages the source of the iterative is often the causative with an intermediate stage ((\citealt[91, fn3]{Childs1987}, \citealt[86f]{Kuryłowicz1964}). Thanks to Johanna Nichols for pointing out this discussion to me.}



\ea%159
    \label{ex:159}Causative as pluractional
    
    \ea 
    \label{ex:159a} \textit{nan} ‘draw, pull'\\

    Ha  ke bondɔ ko ni ha nan wɔmdɛ chiɛ ko.\\
    \gll ha    ke    bondɔ  ko    ni    ha    nan  wɔm-dɛ    chiɛ  ko\\
    \textsc{opt}  see  wharf    to    and  \textsc{opt}  pull  canoe{}-\textsc{def}  shore  to\\
  \glt ‘Look for the wharf and pull the canoe on shore.' (P67 N: 11)\\
  \vspace{6pt}

  \ex
  \label{ex:159b} \textit{nani} ‘draw, pull with force'\\
  
    Wɔn wɔ gbo nani, aha lɛ ha jɛthɛli ha ma ha mbank lɛ.\\
  \gll wɔ{}-n      wɔ    gbo    nan-i    a-ha      lɛ     ha    jɛthɛli\footnotemark{}    ha    ma   ha  n-bank      lɛ\\
  \textsc{3sg-emph}  \textsc{3sg}  indeed  pull-\textsc{caus}  \textsc{ncm}\textsubscript{ha}{}-3\textsc{pl}  \textsc{def}    \textsc{3pl}    slacken    \textsc{3pl}  \textsc{aux} \textsc{opt} \textsc{ncm}\textsubscript{ma}{}-rope    \textsc{def}\\
  \glt ‘While he is pulling hard, the others should slacken their ropes.' (P67 J: 19)
\z
\z
 \footnotetext{This form is also complex, likely also involving the causative extension \textit{{}-i}.}
 
In some cases, particularly with active verbs, the causative may have the same meaning as the verb without the extension, e.g., both \textit{rok} and \textit{roki} mean ‘harvest.'

\section{Reflexive \textit{-ni}}
\label{bkm:Ref521779843}\hypertarget{Toc115517801}{}\label{sec:6.2}
The term “reflexive” (\textsc{refl}) is meant to cover a wide semantic range, as detailed below. The term was chosen because it is often the most obvious and accessible one. Its semantics range from a strict reflexive through a passive and on to an intensive and perhaps something more disagreeable than the stem without the suffix. In its fullest form the \textsc{refl} morpheme takes the form [ni] but it can be reduced to a simple nasal [n] and be metathetized to [in]. In \REF{ex:160}, the stems are followed by their derived forms.


\ea%160
    \label{ex:160} Derived forms with \textit{{}-ni}
    \ea bos \tab  ‘shave' \tab bosni \tab  ‘shave oneself'\\
    bɔs \tab  ‘be cold, wet' \tab bɔsɔlin \tab  ‘quench thirst, cool'\\
    gbani \tab  ‘lean (sth) against'  \tab gbalɔni \tab  ‘lean (oneself) against'\\
    herk \tab ‘ferry' \tab herkɛni \tab ‘ferry oneself' \\
  
    tɛntɛn \tab ‘think' \tab tɛnini \tab ‘think, remember'\\
    tipɛ \tab ‘begin' \tab  tipɛni \tab ‘begin (more extensively\\ 
    \tab\tab\tab or intensely)'\\
    
\vspace{6pt}
 
  \ex Ya herkɛni.\\
    \gll ya  herkɛ-ni\\
    1\textsc{sg} cross\textsc{{}-refl}\\
    \glt ‘I ferried myself.' (e.g., across the river) (E10 Albert Yanker: 9--10)
  \z
\z

\ea%161
\label{ex:161}Some pejorative meanings
\ea \textit{bus} ‘skin' / \textit{busni} ‘(de-)skin oneself (shed like a snake), undress, erupt'
  \ea Ŋkɔ bus vis lɛ.\\
  \gll ŋ-kɔ     bus   vis      lɛ\\
  \textsc{2sg}{}-go  skin  animal \textsc{det}\\
  \glt ‘Go skin the animal!' (P67 B: 277)

  \ex Pəmdɛ kɔ busni Mpelɛ ko.\\
  \gll pɛm  ɛ    kɔ    busni     mpelɛ  ko\\
  war  \textsc{def}  go    break.out  Mpele  to\\
  \glt ‘War has broken out at Mpele.' (P67 B: 278)
\z

\ex \textit{bɔi} ‘have enough, be satisfied' / \textit{bɔyni} ‘be disgusted with'

\ea Ya koŋ bɔi.\\
\gll ya    koŋ  bɔi\\
\textsc{1sg}  \textsc{pfv}  have.enough\\
\glt ‘I have had enough.'

\ex Ya koŋ bɔyni jali mɔ.\\
  \gll ya    koŋ  bɔyni      ja      li-mɔ\\
  \textsc{1sg}  \textsc{pfv}  disgusted  affairs  \textsc{ncm}\textsubscript{lɔ}{}-\textsc{2sg}\\
  \glt ‘I am disgusted with you.' (lit. with your affairs or actions)
\z
\z
\z
The meaning of the extension can also be passive, as in \REF{ex:162}.

  \ea%162
  \label{ex:162}
  \ea chɛth \tab ‘boil' \tab chɛthni \tab ‘be boiled'\\
  bɛmpa \tab ‘make' \tab bɛmpani \tab ‘be made'\\
  
  \ex Mbɔŋ ma pipɛ ma bɛmpani iwɔm.\\
  \gll n-bɔŋ    ma    pipɛ  ma    bɛmpa-ni  i-wɔm\\
  \textsc{ncm}\textsubscript{ma}  \textsc{ncp}\textsubscript{ma}    cask  \textsc{ncp}\textsubscript{ma}     make-\textsc{refl}  \textsc{ncm}\textsubscript{hɔ}{}-wood\\
  \glt ‘Barrel bungs are made of wood.' (P67 B: 165)
\z
\z

\noindent The suffix can be added to adjectives as well, here with its intensifying function, as in \REF{ex:163}.

\ea%163
  \label{ex:163}
  \ea wɛi \tab ‘bad, ugly'\\
  wɛini \tab ‘very bad, terrible, dreadful'\\
  \vspace{6pt}
  \ex yenwɛini lɛ\\
  \gll yen    wɛi-ni     lɛ\\
  thing    bad-\textsc{refl}  \textsc{def}\\
  \glt ‘horrible thing' (women's name for men's Poro\is{Poro} devil\is{devils}) (P67 W: 24)\footnote{It is forbidden for women to say the devil\is{devils}'s real name.}
\z
\z

And the rather curious example in \REF{ex:164}, where it is used as a simple intensifier with a meaning reversal of its stem or at least stripped of the ‘bad,' parallel to English \textit{awfully}.

\ea%164
    \label{ex:164}
    Wɔ kɛleŋ wɛini.\\
    \gll wɔ   kɛleŋ wɛi-ni\\
    3\textsc{sg}  good  bad-\textsc{refl}\\
    \glt ‘He is awfully nice.'
\z

The next verb extension will also be shown to have extended its domain, this time in its morphosyntax, where a verb extension has been reanalyzed as a preposition, a case of demorphologization.

\section{Instrumental \textit{-ka}}
\label{sec:6.3}\label{bkm:Ref48814327}\hypertarget{Toc115517802}{}
The instrumental (\textsc{ins}) \textit{{}-ka}, as it is labelled here, is sometimes known as a sub-category of the “applicative” in the literature on verb extensions, but its usage is more limited here, and I therefore use the narrower term (e.g., \citealt{Hyman2014}). The examples in \REF{ex:165} show the varied uses with a single verb.

\newpage
\ea%165
    \label{ex:165}
    Extended forms of \textit{bɛmpa} ‘do, make'\\
  \ea Bɔ wɔ lɛ hɔ bɛmpaka lithu.\\
  \gll bɔ    wɔ    lɛ    hɔ      bɛmpa-ka    li-thul\\
  hat    \textsc{3sg}  \textsc{def}  \textsc{ncp}\textsubscript{hɔ}    make-\textsc{ins}    \textsc{ncm}\textsubscript{lɔ}{}-straw\\
  \glt ‘His hat is made of raphia-straw.' (P67 B: 83)
\vspace{6pt}
  \ex Fe wullɛ lɔ pɔ bɛmpaka wullɛ.\\
  \gll fe      wul    lɛ    lɔ    pɛ      bɛmpa-ka    wul    lɛ\\
  money  funeral  \textsc{def}  \textsc{ncp}\textsubscript{lɔ}  \textsc{pro}\textsubscript{indef}  make-\textsc{ins}    funeral  \textsc{def}\\
  \glt ‘It is the funeral money that will be used for the funeral.' (Proverbs: 137)
\z
\z

The form of the extension is usually reduced to [k], as illustrated by the following pairs in \REF{ex:166}.

\ea%166
  \label{ex:166}
  Reduced forms of -\textit{ka}\\

  \ea bim \tab ‘cover, thatch' (v.) \tab bimik \tab ‘cover, close' (v.)\\
  ho \tab ‘emerge, come out' \tab hok \tab ‘come from or out'\\
  kee \tab ‘see' \tab kek \tab ‘see (with)'\\
  pɔŋ \tab ‘throw' \tab pɔŋki \tab ‘throw (+\textsc{caus}?)'\\
  tipɛ \tab ‘begin, start' \tab tipik \tab ‘beginning' (n.)\\
  
\vspace{6pt}
\ex Nɔonɔ yellɛ ko ŋa hun ha kek Braima thihɔl.\\
  \gll nɔ-o-nɔ      yel    lɛ    ko    ha    hun    ha    kek  Braima  thi-hɔl\\
  one-\textsc{distr}{}-one  island    \textsc{def}  to    3\textsc{pl}  come    for    see  Braima  \textsc{ncm}\textsubscript{tha}{}-eye\\
  \glt ‘Everyone on the island came to see Braima with their own eyes' (124aw Yanker, Boy Lost at Sea: 249)
\z
\z

Another example in \REF{ex:167} suggests an intensive meaning comparable to what was seen with reflexive \textit{{}-ni}.

\ea%167
    \label{ex:167}
   \ea \label{ex:167a} pɛn \tab ‘talk loud and authoritatively' (P67 P: 76)\\
    pɛnɛk \tab ‘scream, shout' (P67 P: 77)\\
\vspace{6pt}
  \ex \label{ex:167b} Min wɔ pənɛk ama alɛ.\\
  \gll min  wɔ    pɛnɛk    a-ma      a-lɛ\\
  devil\is{devils}  \textsc{3sg}  scream  \textsc{ncm}\textsubscript{ha}{}-female  \textsc{ncm}\textsubscript{ha}{}-\textsc{def}\\
  \glt ‘The devil\is{devils} shouts at the women.' (P67 P: 77)
\z
\z

A transition from verb extension to preposition is apparent in the set of forms in \REF{ex:168}. The examples in \REF{ex:168a} show a similar relationship between the two verb forms in \REF{ex:167a}, but here both are stative. Next appears a sentence with the unextended form \textit{pɛɛ} ‘be full' followed by \textit{ka} ‘with' as a preposition \REF{ex:168b}. Then in \REF{ex:168c}, the extended form allows another argument, \textit{iwɛi} ‘evil,'  after the source form, a stative verb.

\ea%168
  \label{ex:168}
\ea  \label{ex:168a} pɛɛ \tab ‘be full'\\
  pɛkɛ \tab ‘be overfull, filled to the brim'\\
  \vspace{6pt}
  \ex \label{ex:168b}  Boiɛ hɔ pɛɛ ka mɛn.\\
  \gll boi   ɛ     hɔ       pɛɛ  ka    mɛn\\
  dish  \textsc{def}  \textsc{ncp}\textsubscript{hɔ}    be.full  with  water\\
  \glt ‘The dish is filled with water.' (P67 P: 53)
\vspace{6pt}
\ex \label{ex:168c} Yaŋ ya pəkɛ iwɛi.\\
  \gll ya-ŋ      ya    pɛkɛ      i-wɛi\\
  \textsc{1sg-emph}  \textsc{1sg}  be.overfull    \textsc{ncm}\textsubscript{hɔ}{}-bad\\
  \glt ‘I am (truly) filled with evil.' (P67 P: 54)
\z
\z

The use of [ka] as a preposition is more common certainly than the full form of the extension \textit{{}-ka} and probably the reduced form. In both of the examples in \REF{ex:169}, it is clear that \textit{ka} is no longer part of the verb since it is separated from the verb by the direct object. This development is identical to the extension \textit{{}-ka} becoming a preposition in Mani\il{Mani}, claimed to be the result of contact\is{contact} with Soso\il{Soso} and Mande\il{Mande} more generally (\citealt{Childs2011}, \citealt{Childs2017}). See also the examples and discussion in \sectref{sec:3.8} for a cross-linguistic discussion.

\ea%169
  \label{ex:169}
  \textit{{}-ka} reanalyzed as a preposition\\
  
  \ea Ha buŋ wɔ ka thɔk.\\
  \gll ŋa    buŋ  wɔ    ka    thɔk\\
  \textsc{3pl}  flog  \textsc{3sg}  with  stick\\
  \glt ‘They flogged him with a stick.' (P67 B: 271)

  \ex A bɛth thɔk lɛ ka bɛrɛ.\\
  \gll a    bɛth  thɔk  lɛ    ka    bɛrɛ\\
  \textsc{1sg}  cut  tree  \textsc{def}  with  axe\\
  \glt ‘I cut the tree with an axe.' (P67 K: 2)
\z
\z

Thus, there has been some erosion in the form of the instrumental extension, but at the same time there has been a resuscitation of its function in the adposition \textit{ka}.

\section{Final comments on verb extensions}
\label{sec:6.4}\hypertarget{Toc115517803}{}
First, a statement of the morphotactics of Sherbro verb extensions is confined to the three relatively productive extensions, causative \textit{{}-i}, reflexive \textit{{}-ni}, and instrumental \textit{{}-ka}. The proposed order of the morphemes is given in \REF{ex:170}.

\ea%170
  \label{ex:170}
  Morphotactics of Sherbro verb extensions\\
  \ea Stem-\textsc{caus-refl-ins}\\
  \begin{tabular}{lll}
  bɔs & Stem & ‘be cold or wet'\\
  bɔsɔl-i & Stem-\textsc{caus} & ‘make wet, soak'\\
  bɔsɔl-i-n &  Stem-\textsc{caus-refl} & ‘quench, cool, satisfy one's thirst'\\
  \end{tabular}
  
  \ex Ha bɔsɔlin gbɔl\footnotemark{} lɛ hĩ kul mən dɛ.\\
  \gll ha    bɔsɔl-i-n      gbɔl    lɛ    hi    kul    mɛn    ɛ\\
  \textsc{opt}  wet-\textsc{caus-refl}  heart  \textsc{prt}  1\textsc{pl}  drink    water    \textsc{def}\\
  \glt ‘To quench our thirst we drink water.' (P67 B: 174)
  \z
  \z

\footnotetext{‘Satisfying one's heart' is a common idiom for ‘being full or satiated' in this part of West Africa.}
\noindent There were few examples of \textit{{}-ka} used as a verb extension in the data, particularly used with other extensions.

In summary, the system of verb extensions has undergone some attrition similar to what has happened to the noun class system. Neither is a surprising development given the threatened state of the language (e.g., \citealt{Childs2009}, \citealt{Craig1997},  \citealt{Dorian1978}). The changes have taken place both in form and in function. In most cases the verb extension morphemes are no longer identifiable, and their number has been considerably reduced as compared to their closest relatives. The functions have similarly become more diffuse, sometimes resolving into an emphatic\is{emphatic} function of ‘intensity\is{intensity}.'

In the derivational morphology there is greater transparency and regularity.

