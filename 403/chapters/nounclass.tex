\chapter{Noun class system}
\label{ch:5}\hypertarget{Toc115517787}{}
This chapter provides more detail on the noun class system of Sherbro than was given in \sectref{sec:3.1}, where it was introduced as a means of identifying nouns in the language. As a reminder, two facts need to hold true for a language to be characterized as a noun class language: all nouns must be exhaustively assigned to a noun class, and all nouns must control agreement on dependent elements (e.g., \citealt{Greenberg1977}). It is not necessary for the nouns themselves to bear marks of their noun-class membership; in fact, many do not.

After some general comments about the system itself, I discuss agreement patterns and talk about the pairings and non-pairings of various classes. I then talk about each noun class individually, characterizing the form, the semantics, and the special distribution of each noun class.

As stated in \sectref{sec:3.1}, Sherbro is a noun-class language that generally prefixes its noun-class markers (\textsc{ncm}s) and usually prefixes dependent elements with the same \textsc{ncm}. Pronouns (\textsc{ncp}s) also show agreement, formally distinct from the \textsc{ncm}s (see \tabref{tab:nounclass:31}). Essential to the system is agreement controlled by the head noun and manifested on dependent elements, such as adjectives, low numbers, and the definite article, typically in the form of prefixed \textsc{ncm}s. Nouns sometimes, but not always, feature the prefixed \textsc{ncm} themselves. All \textsc{ncp}s and \textsc{ncm}s have low tones.

Sherbro has not kept as full a system of classes as closely related Bom-Kim\ij{Bom-Kim} nor as full as the systems of more distantly related Kisi\il{Kisi} and Mani\il{Mani}, though there are many correpondences, both semantic and formal. \tabref{tab:nounclass:31} summarizes the formal and semantic features of the Sherbro noun classes. Each class will be referenced by its pronoun.

\begin{table}
\caption{\label{tab:nounclass:31}Sherbro noun classes (repeated from \tabref{tab:wordcat:13})}
\small
\begin{tabularx}{\textwidth}{llQQ}
\lsptoprule
\textsc{ncp} & \textsc{ncm} & Noun & Semantic characterization\\
\midrule
 wɔ & $\emptyset$ (\textit{wɔ} class) & \textit{thumɔɛ} ‘dog,' \textit{ra} ‘green snake,' \textit{nɔ} compounds & animate singulars\\
 ha & a- (\textit{ha} class) & \textit{abolom} ‘Sherbro people,' \textit{athumɔɛ} ‘dogs' (also \textit{si}\,) (\citet{Sumner1921} gives \textit{awok/siwok} ‘slaves') & animate plurals, animal plurals are often also marked with \textit{si}\\
 (ha) & {}-si (\textit{si} class) & \textit{ramsi} ‘clans' (also \textit{tha}), \textit{bɛlsi} ‘rats,' \textit{fansi} ‘cane rats' (also \textit{ha}), \textit{thumɔɛsi} ‘dogs' (also \textit{ha}) & animate plurals, mostly animals; multiple marking\\
 kɔ & $\emptyset$ (\textit{kɔ} class) & \textit{thɔk} ‘tree,' \textit{pɛm} ‘war,' \textit{kɛfɛ} ‘pepper,' \textit{raka} ‘burweed,' \textit{bon} ‘ceremony' & default class, no prefixes, many singulars of plants\\
 ma & n- (\textit{ma} class) & \textit{mɛn} ‘water,' \textit{ŋkuai} ‘palm oil,' \textit{mfan} ‘palm wine,' \textit{nranth} ‘cane rope' & liquids, some plurals, large things\\
 hɔ & i- (\textit{hɔ} class) & \textit{ibithir} ‘bottle,' \textit{ipan} ‘moon,' \textit{ichak} ‘palm fiber,' \textit{rɔ} ‘shield,' \textit{ihɔlɔŋ} ‘breath' & many singulars\\
 tha & thi- (\textit{tha} class) & \textit{thikil} ‘houses,' \textit{thichala} ‘mats,' \textit{thiram} ‘clans,' \textit{thisabo} ‘diseases' & many plurals, plural inanimates (\textit{hɔ} class)\\
 lɔ & li- (\textit{lɔ} class) & \textit{lipal} ‘sun,' \textit{liken} ‘knife,' \textit{limani} ‘respect,' \textit{lithɛm} ‘love,' \textit{litiŋ} ‘by twos' & a small set of nouns, locatives, converts nouns to adverbs\\
\lspbottomrule
\end{tabularx}
\end{table}

There are some fixed meanings to the noun classes. Two important features are [\textsc{animacy}\is{animacy}] and [\textsc{number}], with the latter also involving mass or collective nouns as opposed to individuating ones. Other factors are important and likely trace back to the mostly configurational basis for class membership (\citealt{CreiderDenny1975}, \citealt{DennyCreider1986}). For example, implements and tools and body parts, all of an elongated shape generally belong to a single class (\citealt{Childs1983}). Clusters of such items occur in a noun class, but these generalizations rarely apply to all members of a class since there has been some collapse of the system.

To illustrate how arbitrary some of the class memberships can be, Sherbro has a number of words for ‘knife,' as indicated in \tabref{tab:nounclass:32}. The columns are headed by the name of the noun class. The first three columns are the singulars; the last two are the plurals.

\begin{table}
\caption{\label{tab:nounclass:32}Words for ‘knife'}
\begin{tabularx}{\textwidth}{llllll}
\lsptoprule
Kɔ & Hɔ & Lɔ & Ma & Tha & \\
\midrule
balmaa &  &  & balmaa & & ‘two-edged knife' \\
saki &  &  & saki & & ‘war knife' \\
& gbosa &  &  & gbosa & ‘palm-cabbage knife' \\
& hɔlɔŋ &  &  & hɔlɔŋ & ‘curved knife' \\
& baa &  & baa & & ‘curved knife' \\
&  & ken & ken & & ‘knife' \\
\lspbottomrule
\end{tabularx}
\end{table}

The irregularities in the system point to a different organization than the one currently extant. On the basis of comparative evidence, Sherbro once had a more extensive system with more noun classes, more transparent semantics, and tighter agreement. Semantics today overrides morphology in that [\textsc{animacy}\is{animacy}] commonly determines agreement rather than morphology, as shown in \REF{ex:135}.

\newpage
\ea%135
    \label{ex:135}
    Mpɛnte ŋamiyɛ gbi ko ba bullɛ.\\
    \gll n-pɛnte      ha-mi      ɛ    gbi  ko    ba      bul  ɛ\\
    \textsc{ncm}\textsubscript{ma}{}-brother  \textsc{ncp}\textsubscript{ha}{}-\textsc{1sg}  \textsc{def}  all    from  father    same  \textsc{def}\\
    \glt ‘My brothers come from the same father.' (001a Abdulai Bendu: 18)
\z

Here the plural form of \textit{pɛnte} ‘brother' belongs to the \textit{ma} class whose \textsc{ncm} is a homorganic nasal, prefixed on the noun. Instead of the expected coresponding pronoun, \textit{ma}, the \textit{ha}{}-class pronoun, [ŋa-], is prefixed to the \textsc{1sg} possessive \textit{mi}.

That the noun classes and even their pairings retain some psychological reality is shown by what happens to noun borrowings, i.e., which classes they are incorporated into, as will be discussed below. They tend to join classes with semantically similar features.

\section{Agreement}
\hypertarget{Toc115517788}{}\label{sec:5.1}
Elements which show agreement, however desultory, include the following.

\begin{itemize}
\item Definite article (\sectref{sec:3.4})
\item Some adjectives (\sectref{sec:3.2})
\item Possessives (\sectref{sec:3.3.1})
\item Some demonstratives (\sectref{sec:3.3.3})
\item Numbers (\sectref{sec:3.7})
\end{itemize}

Not all noun classes show agreement; the \textit{wɔ} and the \textit{kɔ} classes do not (nor do they feature prefixes on the noun itself, as shown in \REF{ex:135}, and the \textit{hɔ} class shows agreement only some of the time (nouns are sometimes prefixed with the \textsc{ncm} \textit{i-}).

The example in \REF{ex:136a} illustrates the typical agreement pattern with numbers, \textit{tiŋ} ‘two' and \textit{hiɔl} ‘four,' determined by the head nouns \textit{rokos} ‘orange' and \textit{pakay} ‘papaya,' both of them \textit{ma}{}-class nouns whose \textsc{ncm} is \textit{n-}. (Note also how both nouns are prefixed with the same \textsc{ncm}.) The example in \REF{ex:136b} shows adjectival agreement (on \textit{kɛlɛŋ} ‘good') here controlled by \textit{panth} ‘work,' another \textit{ma}{}-class noun with an identical agreement pattern. The example in \REF{ex:136c} shows \textit{tha}{}-class agreement, with the \textsc{ncm} \textit{thi}{}- prefixed to first the definite marker \textit{lɛ}, secondly to the partitive \textit{pum} ‘some' and finally to the adjective \textit{kɛlɛŋ} ‘good'. The example in \REF{ex:136c} shows the \textit{tha}{}-class concordant pronoun \textit{tha}. The example in \REF{ex:136d} shows a possessive construction with the \textsc{3pl} pronoun \textit{ŋa} ‘their' prefixed with the \textit{ta}{}-class \textsc{ncm} \textit{thi}{}-. (Both the \textit{ma} and the \textit{tha} classes show robust and regular agreement patterns.)

\largerpage
\ea%136
    \label{ex:136}
    Agreement patterns

\ea \label{ex:136a} Ŋkɔm lɛnthiɛ nrokos ntiŋ ni mpakay nhiɔl.\\
\gll n    kɔ    mi    lɛnthiɛ  n-rokos        n-tiŋ      ni    n-pakay        n-hiɔl\\
you  go    1\textsc{sg}  pluck    \textsc{ncm}\textsubscript{ma}\textsc{{}-}orange    \textsc{ncm}\textsubscript{ma}\textsc{{}-}two  and  \textsc{ncm}\textsubscript{ma}\textsc{{}-}papaya    \textsc{ncm}\textsubscript{ma}\textsc{{}-}four\\
\glt ‘Go pluck me two oranges and four papayas.' (P67 L: 53)

\ex \label{ex:136b} Mpanth ma ichɛk ma ɛ mpanth ŋkəlɛŋ.\\
    \gll n-panth      ma    i-chɛk      ma    lɛ    n-panth      n-kɛlɛŋ\\
    \textsc{ncm}\textsubscript{ma}\textsc{{}-}work  \textsc{ncp}\textsubscript{ma}    \textsc{ncm}\textsubscript{hɔ}\textsc{{}-}farm    \textsc{ncp}\textsubscript{ma}    be    \textsc{ncm}\textsubscript{ma}\textsc{{}-}work  \textsc{ncm}\textsubscript{ma}\textsc{{}-}good\\
    \glt ‘Farmwork is fine work.' (P67 P: 42)

\ex \label{ex:136c} Næ thilɛ thipum tha thikəlɛŋ.\\
    \gll nai    thi-lɛ        thi-pum      tha     thi-kɛlɛŋ\\
    road    \textsc{ncm}\textsubscript{tha}{}-\textsc{def}    \textsc{ncm}\textsubscript{tha}{}-some  \textsc{ncp}\textsubscript{tha}    \textsc{ncm}\textsubscript{tha}{}-good\\
    \glt ‘Some roads are fine.' (P67 K: 88)

\ex  \label{ex:136d} Lɛ ŋke yɛ amaaɛ ŋa koŋ nuik tɔn thiŋaɛ, ...\\
    \gll lɛ  n    ke    yɛ    a-maa      ɛ    ŋa    koŋ  nuik    tɔn  thi-ŋa      ɛ\\
    if  \textsc{2sg}  see  how  \textsc{ncm}\textsubscript{ha}{}-female  \textsc{def}  \textsc{3pl}  \textsc{pfv}  amuse  song  \textsc{ncm}\textsubscript{tha}{}-\textsc{3pl}  \textsc{def}\\
    \glt ‘If you see how the women amused themselves with their songs, ...' (123aw Yanker, Rat Wife: 49)
\z
\z

Agreement also occurs on numbers 1--10 both in isolation or as part of a larger number but usually not the number ‘twenty'. Sherbro uses a 5-and-20-based system of numbers (quinary-vigesimal) and thus most numbers will show agreement. The example in  \REF{ex:137} shows agreement on ‘twelve.'

\ea%137
    \label{ex:137}
    Mpaŋ nwaŋ ni tiŋ man ma nɛn bulaiɛ\\
    \gll n-paŋ        n-waŋ    ni    tiŋ    ma-n       ma    nɛn  bul  ai      ɛ\\
    \textsc{ncm}\textsubscript{ma}{}-month  \textsc{ncm}\textsubscript{ma}{}-ten  and  two  \textsc{ncp}\textsubscript{ma}\textsc{{}-emph} \textsc{ncp}\textsubscript{ma}  year  one  inside    \textsc{def}\\
    \glt ‘There are twelve months in a year.' (P67 N: 22)
\z

To say that Sherbro is a prefixing language is a little misleading. First of all, not all nouns have prefixes (noun-class markers, \textsc{ncm}s). Nouns belonging to the \textit{wɔ} and \textit{kɔ} classes never do, and not all \textit{hɔ}\nobreakdash-class nouns are prefixed with the class's \textsc{ncm} \textit{i-}.\footnote{The differential erosion of the prefixed \textsc{ncm}s may have to do with syllable structure. The prefixes \textit{li-}, \textit{N-}, \textit{thi-} are relatively stable, i.e., all consonant-involved prefixes.} Prefixes, however, generally do appear on concordant elements. Other irregularities exist and are discussed in the sections on the individual noun classes.

Another sign of desuetude, one that may be related to language death, is the occasional agreement “error” made by speakers. The example in \REF{ex:138} illustrates one such instance. Instead of the \textit{hɔ}{}-class \textsc{ncm} on the dependent \textsc{3pl} possessive \textit{ŋa} in accord with the head noun, \textit{hɔ}{}-class \textit{lel} ‘name,' the speaker has used a form from the \textit{ma} class, which is not associated with ‘name.'

\ea%138
    \label{ex:138}
    Ntoŋgi mi ilel maŋaɛ.\\
    \gll n    toŋgi    mi    i-lel        ma-ŋa    ɛ\\
    \textsc{2sg}  show    \textsc{1sg}  \textsc{ncm}\textsubscript{hɔ}{}-name  \textsc{ncp}\textsubscript{ma}{}-\textsc{3pl}  \textsc{def}\\
    \glt ‘Show me their names.' (001a Abdulai Bendu: 13)
\z

Here the conflict is not between semantics and morphology – no explanation is available save the change in the language due to its desuetude.

\section{Size and pairings}
\hypertarget{Toc115517789}{}\label{sec:5.2}
The largest noun class is the \textit{hɔ} class.\footnote{The generalizations and numbers are based on a consideration of 1,176 Sherbro nouns in a lexicon of 4,095 entries. We did not have noun-class information on all of the nouns in the database but believe the generalizations are nonetheless valid.} Either by itself or paired with another class, it contains roughly 35\% of the nouns. The next largest class is the \textit{kɔ} class with 26\% of the total nouns, then the \textit{wɔ} class (animates) with 25\%, and the \textit{ma} class with 10\%. The \textit{lɔ} class constitutes 4\% of the total. The strictly plural classes, \textit{tha}, \textit{ha}, and \textit{si}, have very few unpaired members and are therefore not listed in \tabref{tab:nounclass:33}. The \textit{tha} class has only 10 independent members (many plurals of \textit{kɔ} and \textit{hɔ} classes). The \textit{si} and \textit{ha} classes exist only as plurals of the \textit{wɔ} class with only three members when not paired all in the \textit{ha} class. \tabref{tab:nounclass:34} presents the numbers for the larger classes.

\begin{table}
\caption{\label{tab:nounclass:33}The size of Sherbro noun classes}



\begin{tabularx}{\textwidth}{rrllQ}
\lsptoprule
 N & \% & \textsc{ncp} & \textsc{ncm} & Semantic characterization\\
 \midrule
 329 & 35\% & hɔ & i- (\textit{hɔ} class) & singulars, body parts, tools and implements, diseases\\
 242 & 26\% & kɔ & $\emptyset$ (\textit{kɔ} class) & singulars of plants, trees, and fruits, foods\\
 235 & 25\% & wɔ & $\emptyset$ (\textit{wɔ} class) & animate singulars\\
 94 & 10\% & ma & n- (\textit{ma} class) & liquids, fruits, some plurals, large things, languages\\
 34 & 4\% & lɔ & li- (\textit{lɔ} class) & derivative abstractions\\
\lspbottomrule
\end{tabularx}
\end{table}

There are significant pairings, beginning with the singular-plural \textit{wɔ}/\textit{ha} association of animate nouns. The second most common is the singular-plural \textit{hɔ}/\textit{tha} pairings, where \textit{tha} contains only plurals. Some nouns in the \textit{hɔ} class are collectives and have no plurals. Many nouns belong to the \textit{kɔ}/\textit{ma} pairing, where the \textit{ma} class, normally the liquid class, contains plurals, though the \textit{kɔ} class has many nouns without plurals.

\begin{table}
\caption{\label{tab:nounclass:34}The most numerous noun-class pairings}
\begin{tabular}{rl}
\lsptoprule
 \textsc{N} & \textsc{ncp}s\\
 \midrule
 181 & wɔ / ha (\:/ si)\\
 177 & hɔ / tha\\
 148 & kɔ / ma\\
 23 & kɔ / tha\\
 18 & hɔ / ma\\
\lspbottomrule
\end{tabular}
\end{table}

Many nouns (332), however, have no paired counterpart. The largest are: \textit{hɔ} with 134, \textit{ma} with 92, \textit{kɔ} with 66, and \textit{lɔ} with 29. A few other classes have negligible numbers of unpaired singles.

\section{\textit{Wɔ} class}
\hypertarget{Toc115517790}{}\label{sec:5.3}
The \textit{w}ɔ class contains human and animal singulars. Many of its plurals fall in the \textit{ha} class and animals also in the \textit{si} class. The \textit{wɔ} class contains the noun \textit{nɔ} ‘person' as well as its many derivatives, e.g., \textit{cholnɔ} ‘artist' (cf. \textit{chol} ‘art'), \textit{nɔbonthɔ} ‘helper' (cf. \textit{bonthɔ} ‘help'). Precictably the plural of \textit{nɔ} ‘person' belongs to the \textit{ha} class, but the forms are irregular, \textit{anyin} or \textit{anya}.

As mentioned above and illustrated in \REF{ex:135}, \textit{wɔ}{}-class nouns have no marking, and agreement marking is absent, as seen in \REF{ex:139}.

\ea%139
    \label{ex:139}
    \textit{Wɔ}{}-class nouns without a prefix\footnotemark\\

    \vspace{6pt}

    Sɔk lɛ wɔ mu hel.\\
    \gll sɔk  lɛ    wɔ    mu  hel\\
    fowl  \textsc{def}  \textsc{3sg}  still  boil\\
    \glt ‘The fowl is still boiling.' (P67 M: 100)
\z
\footnotetext{The \textsc{ncm} for the class in related languages is \textit{o-}. It appears variably in Bom-Kim\ij{Bom-Kim} and Mani\il{Mani}, more often in the latter. In Kisi\il{Kisi}, where it is a suffix, it is always present except in proverbs, negations, and the like.\label{fn:65}}

Borrowings into the \textit{wɔ} class show the psychological reality of at least the [\textsc{human}] and [\textsc{animate}] features. The examples in \REF{ex:140} come from a variety of languages. In the first column is the \textit{wɔ/nɔ} class form and in the second is the \textit{ha} class with the prefixed \textsc{ncm} \textit{a-}.

\ea%140
\label{ex:140} Singular/plural pairings for [\textsc{animate}] borrowings\\
\vspace{6pt}

\begin{tabular}{lll}
    mɛknɔ & amɛk & ‘American'\\
    mulat & amulat & ‘mulatto'\\
    bolnafali & abolnafali & ‘a Mende\il{Mende} “play” mask'\footnotemark\\
    bolkoŋgoli & abolkoŋgoli & ‘Kongoli mask'\\
    potho & apotho & ‘European' (< Portuguese), areal word
\end{tabular}

\z
\footnotetext{The Nafali mask is danced and explained in a short YouTube video, \url{https://www.youtube.com/watch?v=r4jNq2gxNag} (accessed 12 Aug 2020)}

New words for animals are not so common, but at least one exists. The word for ‘whale' belongs to three classes (\textit{wɔ}, \textit{ha}, and \textit{ma}), as do many other sea creatures.

\TabPositions{2cm,4cm}

\ea%141
    \label{ex:141}
    klampis{\textasciitilde}krampis \quad aklampis \quad nklampis \quad ‘whale' < English \textit{grampus}
\z




% \ea%141
%     \label{ex:141}
%     \begin{tabular}{cccc}
%     klampis{\textasciitilde}krampis & aklampis & nklampis & ‘whale' < English \textit{grampus}.\\
%     \begin{tabular}
% \z

\section{\textit{Ha} class}
\largerpage
\hypertarget{Toc115517791}{}\label{sec:5.4}
Virtually all plurals of \textit{wɔ}{}-class nouns can be found in the \textit{ha} class, if not as the only plural as at least one of the plurals. Other possibilities for \textit{wɔ}{}-class plurals are the \textit{si} class and the \textit{ma} class. The affix \textit{nɔ} can be used to create agentive (singular) nouns from both nouns and verbs (see \sectref{sec:7.3.1}); these derived forms always have plurals in the \textit{ha} class. The \textit{nɔ}{}-derived forms were not included in the analysis presented above.

In addition to the examples already given are the forms in \tabref{tab:nounclass:35}. I give both singulars and plurals and a second plural in the \textit{si} (\textsc{ncm} -\textit{si} ) or \textit{ma} (\textsc{ncm} \textit{n-}) class if it exists. At times a plural will be doubly marked as is the plural \textit{a-wok-si} ‘slaves' (see Table 5.6). There are four nouns that have plurals in all three plural classes, e.g., \textit{tɔmbɔ, tɔmbɔsi, ntɔmbɔ} ‘jumper mullet.'

\begin{table}
\caption{\label{tab:nounclass:35}Animate plurals}


\begin{tabular}{lllll}
\lsptoprule
Wɔ & Ha & Si & Ma &\\
\midrule
dip & adip &  &  & ‘porcupine'\\
sampa & sampa &  &  & ‘women's summoner'\\
to & ato &  &  & ‘snail'\\
bɛl & abɛl & bɛlsi &  & ‘rat'\\
tun & atun & tunsi &  & ‘time bird, coucal'\\
\tablevspace
chanth & achanth &  & nchanth & ‘baby'\\
koluŋ & akoluŋ &  & nkoluŋ & ‘cockroach'\\
samak & asamak &  & nsamak & ‘guinea fowl'\\
\lspbottomrule
\end{tabular}
\end{table}

In \REF{ex:142}, the noun-class marker \textit{a-} can be seen on the nouns themselves as well as on the dependent element \textit{tata} ‘young'. The pronoun \textit{ha} of the class is used as both object and subject.

\ea%142
    \label{ex:142}
    Anyindɛ kache, ŋɔ pɔ kache ŋa trit a, apima atata ŋa ka bi rɛspɛkt ŋa ayin?\\
    \gll a-nyin      ɛ    kache      ŋɔ    pɛ      ka      che  ha    trit    a\\
    \textsc{ncm}{}-\textsubscript{ha}{}-people  \textsc{def}  formerly    how  \textsc{pro}\textsubscript{indef}  \textsc{rem.pst}  \textsc{prog}  \textsc{3pl}  treat  \textsc{q}\\
    \gll a-puma        a-tata        ha    ka      bi    rɛspɛkt  ŋa    a-nyin      a\\
    \textsc{ncm}\textsubscript{ha}{}-children  \textsc{ncm}\textsubscript{ha}{}-young  \textsc{3pl}  \textsc{rem.pst}  have  respect  for    \textsc{ncm}\textsubscript{ha}{}-people  \textsc{q}\\
    \glt ‘The people in those days, how were they treated? The children did they have respect for people?' (009--10a Lohr \& Mampa: 233)
\z

The example in \REF{ex:143} illustrates once again how [\textsc{animacy}\is{animacy}] overrides morphology with \textit{ma}{}-class \textsc{ncm} on ‘young woman' but \textit{ha}{}-class agreement \textit{a-} on ‘many'. More importantly there is agreement on the predicate adjective \textit{kɛlɛŋkɛlɛŋ} ‘beautiful.'

\ea%143
    \label{ex:143}
    Nwantɛm agber ha tri ka ni ha akəlɛŋkəlɛŋ.\\
    \gll n-wantɛm          a-gber      ha    tri    ka    ni    ha    a-kɛlɛŋkɛlɛŋ\\
    \textsc{ncm}\textsubscript{ma}{}-young.woman  \textsc{ncm}\textsubscript{ha}{}-many  \textsc{3pl}  town  in    and  \textsc{3pl}  \textsc{ncm}\textsubscript{ha}{}-beautiful\\
    \glt ‘There are many young women in this town and they are very beautiful.' (P67 W: 15)
\z

There are some irregularities. The first line in \tabref{tab:nounclass:36} shows the highly unusual plural of \textit{yu} ‘fish' (\textit{wɔ} class, singular) which is \textit{yenchɛk} ‘fish pl.' and takes agreement in the \textit{ha} class. The next three lines show the somewhat jumbled forms for family relations and for their gender- and age-based distinctions. In addition, there are at least three terms for ‘husband,' \textit{po}, \textit{pokan}, and \textit{nɔpokan}, all of which can be used for ‘man' as well as \textit{nɔ}, \textit{langban}, and \textit{langbanɔ}. The next two lines continue the double marking of the plural, and the last example shows double marking with \textsc{ncm}s from two different classes.

\begin{table}
\caption{\label{tab:nounclass:36}Unusual plural marking}

\begin{tabular}{llll}
\lsptoprule
Wɔ & Ha & Si &\\
\midrule
yu & yenchɛk &  & ‘fish'\\
\tablevspace
la & ama &  & ‘woman, wife'\\
tamɔ & apuma pokan &  & ‘boy, son'\\
wanta & apuma ma &  & ‘girl, daughter'\\
\tablevspace
bɛknɔ & abɛka &  & ‘Krio (person)'\\
potonɔ & apotoa &  & ‘white person'\\
\tablevspace
wonɔ & awok & awoksi & ‘slave'\\
\lspbottomrule
\end{tabular}
\end{table}

There are a number of words that begin with [a] – few words begin with a vowel – but none of them controls \textit{ha}{}-class agreement, e.g., \textit{ayenal} ‘place' with an optional initial vowel. Many of the \textit{a-}initial words function as postpositions: \textit{ahɔl} ‘at the mouth of,' \textit{atok} ‘at the top of,' and perhaps these “prefixes" represent the remnant of a locative class.


\section{\textit{Si} class}
\hypertarget{Toc115517792}{}\label{sec:5.5}
The \textit{si} class is likely the remnant of a more highly differentiated noun class system, in which the \textit{si} class may have once distinguished non-human animates from humans. Today it exists as a usually suffixed marker of plural for animals, as in \REF{ex:144}.

\ea%144
    \label{ex:144}
    \textit{si}{}-class nouns in context\\
    \ea Huksi atiŋ ha che kil lɛ kunɛ.\\
    \gll huk-si      a-tiŋ      ha      che  kil      lɛ    kunɛ\\
    spider-\textsc{ncm}\textsubscript{si}  \textsc{ncm}\textsubscript{ha}{}-two  \textsc{ncp}\textsubscript{ha}    be    house    \textsc{def}  inside\\
    \glt ‘There are two bush spiders in the house.' (P67 H: 106)

    \ex Yɛ thoŋka ki gbi kɔ haani bɛlsi atiŋ   doki thiyeŋ dɛ ...\\
    \gll yɛ    thoŋka  ki    gbi  kɔ    haani    bɛl  {}-si      a-tiŋ      loki  thiyeŋ-ɛ\\
    when  arguing  this  all    \textsc{ncp}\textsubscript{kɔ}  happen  rat-\textsc{ncm}\textsubscript{si}  \textsc{ncm}\textsubscript{ha}{}-two  these  between-\textsc{prt}\\
    \glt ‘When all this arguing is going on between these two rats …' (123aw Yanker, Rat Wife: 77)
\z
\z
In both examples, \textit{si} appears as a suffix on the noun stems \textit{huk} ‘spider' and \textit{bɛl} ‘rat'; normally agreement markers appear only prefixed to dependent elements such as \textit{tiŋ} ‘two,' just as the agreement marker for the \textit{ha} class \textit{a-} does. Furthermore, the \textsc{ncm} \textit{si} unusually appears as a suffix.

The suffix \textit{{}-si}, as indicated in the parenthesized pronoun \textit{ha} in \tabref{tab:nounclass:31}, has no corresponding pronoun. When a pronoun is required, \textit{ha} is used.

In citation forms, which feature the definite marker \textit{{}-ɛ}, the suffix also appears. When asked to give the plurals for animal nouns taking the \textit{si}{}-class marker, speakers give a stem with a following \textit{sɛ} (\textit{si} + the definite marker [ɛ], which generally shows agreement with the noun it follows (see \sectref{sec:3.4}). In the first column of \tabref{tab:nounclass:37} appear the singulars, in the second the plurals, and in the third a gloss.

\begin{table}
\caption{\label{tab:nounclass:37}\textit{Si}{}-class citation forms}

\begin{tabular}{lll}
\lsptoprule
Singular & Plural & \\
\midrule
nɔŋgbɛ & nɔŋgbɛsɛ & ‘sheep'\\
thumɔɛ & thumɔisɛ & ‘dog'\\
vee & veesɛ & ‘bird'\\
thɛthɛl & thɛthɛlsɛ & ‘grasshopper'\\
gbɛgbɛ & gbɛgbɛsɛ & ‘frog'\\
bɔkɛ & bɔksɛ & ‘turtle'\\
pio & piosɛ & ‘pig'\\
\lspbottomrule
\end{tabular}
\end{table}

In the examples from a narrative, the affix \textit{si} seems more plausibly analyzed as a prefix \REF{ex:145}, as opposed to its analysis as a suffix in \REF{ex:144}, where the number ‘two' has the \textit{ha}{}-class agreement marker \textit{a-}. Note that the generic words for both ‘animal' and ‘insect' have their prefixed \textit{ma}\nobreakdash-class \textsc{ncm}s.

\newpage
\ea%145
    \label{ex:145}
    Kaiŋ Taso wɔ thee ŋhɔk ma ŋvissɛ, veesɛ, ni ŋkɔlɔŋsɛ.\\
    \gll Kaiŋ  Taso    wɔ    thee      n-wɔk      ma    n-vis-si-ɛ\\
    Kain  Tasso    \textsc{3sg}  understand  \textsc{ncm}\textsubscript{ma}{}-word  \textsc{ncp}\textsubscript{ma}    \textsc{ncm}\textsubscript{ma}{}-animal-\textsc{ncm}\textsubscript{si}{}-\textsc{def}\\
    \gll vee-si-ɛ        ni    n-kɔlɔŋ-si-ɛ\\
    bird-\textsc{ncm}\textsubscript{si}{}-\textsc{def}  and  \textsc{ncm}\textsubscript{ma}{}-insect-\textsc{ncm}\textsubscript{si}{}-\textsc{def}\\
    \glt ‘Kain Tasso understands the words of every animal, bird, and insect.' (123aw Yanker, Rat Wife: 52)
\z

It is not just animal plurals that belong to the class, it can also be human beings such as ‘twins' and ‘enslaved person,' as in \REF{ex:146}.

\ea%146
    \label{ex:146}
    \textit{Si} plurals for human beings\\
    \ea Pɛnthsɛ ŋan ŋanpɛ, ŋa bi ŋa bi ilel ɡba?\\
    \gll pɛnth-si-ɛ      ha-n      ha-n      pɛ    ha    bi    ha  bi    i-lel        ɡba\\
    twin-\textsc{ncm}\textsubscript{si}{}-\textsc{def}  \textsc{3pl}{}-\textsc{emph}  \textsc{3pl}{}-\textsc{emph}  also  \textsc{3pl}  have  to  have  \textsc{ncm}\textsubscript{hɔ}{}-name   different\\
    \glt ‘Twins they themselves, they have to have separate names?' (102v Chernor Ashun: 238.1)

\ex  wonɔ ‘slave' / awok, siwok ‘slaves' (\citealt{Sumner1921})\footnotemark
\footnotetext{The \textit{si} form for ‘slaves' did not appear in our own data.}
\z
\z

Likely many animates once belonged to a non-\textit{wɔ}/\textit{ha} class pairing, signaled by agreement patterns still in accord with the \textit{hɔ} and \textit{ma} classes in the plural. Such nouns, typically animals such as fish, insects, birds, mammals, and family relations, show only plural accord with these classes; singular accord is with the animate wɔ class and such nouns bear the prefix of the \textit{ha} animate class. The \textit{tha} class, the default plural class for non-animates, contains no such plurals. Clearly [\textsc{animacy}\is{animacy}] plays a role in determining plural agreement patterns. The one clear piece of evidence for at least one more class than the seven is the curious plural affix and agreement marker for some non-human animates.

\section{\textit{Ma} class}
\hypertarget{Toc115517793}{}\label{sec:5.6}
The \textit{ma}{}-class is probably the easiest to characterize and one of the most formally stable throughout Bolom-Kisi and Mel\il{Mel} in general. It includes liquids, language names, and many plurals. The “plural” of several other classes is not quite the same as the plural of a count noun; it is more of an augmentative since many of the nouns with which it is paired are collective or mass nouns. Its prefixed \textsc{ncm}, a nasal homorganic to the initial consonant of the noun stem, is highly regular and highly present.

\ea%147
    \label{ex:147}
    Some typical \textit{ma}{}-class nouns\\
    \begin{tabular}{lll}
    Liquids & mɛn & ‘water'\\
            & ŋkuai & ‘palm oil'\\
            & mmanteŋka & ‘butter' (< Portuguese \textit{manteiga})\\
            & ŋkalom & ‘palm wine'\\
Juicy fruits & nsupsap & ‘soursop'\\
             & mplɔm & ‘plum'\\
Other foods & mmalɔ & ‘ground nut'\\
            & nsaha & ‘egusi'\\
            & mbinch & ‘beans'\\
Abstractions & nsɔn & ‘dream'\\
             & mbɔs & ‘peace'\\
             & ŋkuath & ‘fear'\\
            & nlap & ‘shame'\\
\end{tabular}
\z

In addition to the nouns above, other possible semantic groupings could be ‘sicknesses' and ‘languages'. Sicknesses are numerous and found in other classes as well, but all language names are found in the \textit{ma} class.

Definitely \textit{plɔm} ‘plum' and \textit{mmanteŋka} ‘butter' are borrowings and likely \textit{nsupsap}. That they belong to the \textit{ma} class shows that [\textsc{liquidity}] is still a feature of the class in speakers' minds. Nonetheless, it is impossible to identify any cohesive notion uniting the entire set of nouns belonging to the class.

One semantic anomaly to the \textit{ma} class is with family relations. Some of them with animate singulars as expected in the \textit{wɔ} class, have plurals in the \textit{ma} class rather than in the expected \textit{ha} class (animate plurals), as first illustrated in \tabref{tab:nounclass:35}. A few more examples appear in \REF{ex:148}.

\ea%148
    \label{ex:148} jajɛl \tab njajɛl \tab ‘mother-in-law'\\
gbɛnɔ \tab mgbɛnɔ \tab ‘sister-in-law'\\ 
komnɛ \tab ŋkomnɛ \tab ‘father- or brother-in-law'\\
kɛɲa \tab ŋkɛɲa \tab ‘uncle'\\
\z

In addition to it being a “plural” for some nouns in the \textit{wɔ} class it is also the plural for many singulars in the \textit{kɔ} and \textit{hɔ} classes. This has perhaps led to its use as an “augmentative" when these nouns have other plurals. The augmentative function is clear with a stem such as \textit{wɔm} ‘boat, canoe,' which belongs to the \textit{hɔ}/\textit{thi} (sg/pl) pairing. It is the largest pairing in the language and contains many nautical terms as well as many body parts and items of daily use. It is also the destination of some borrowings including some from English and Portuguese.  Thus, the pairing is one of some productivity. However, when a boat or canoe is overlarge and is propelled by oars rather than with a paddle, it belongs to the \textit{ma} class.

\ea%149
    \label{ex:149}
    wɔm \tab hɔ / thi \tab ‘boat, canoe'\\
    wɔmmbɔkul ma \tab ‘large canoe or boat, weighing up to 3 tons,\\
    \tab\tab propelled by oars (rather than paddles)'
\z

There are some formal irregularities in the agreement patterns. The expected pattern with adjectives showing agreement is noun \textsc{ncm}{}-adjective, i.e., the adjective prefixing the \textsc{ncm} of the \textit{ma}{}-class head noun, namely, a homorganic nasal. Instead of the \textsc{ncm}, however, speakers use the \textsc{ncp} \textit{ma} in the \textsc{ncm}'s place. In \REF{ex:150a}, it is between ‘men' and ‘uncle,' and in \REF{ex:150b}, between ‘work' and ‘farm'. Strangely, \textit{ichɛk} ‘farm' retains its \textit{hɔ}{}-class prefix \textit{i-}.

\ea%150
    \label{ex:150}
    \textit{ma} as an \textsc{ncm} prefixed to dependent elements\\
    \ea\label{ex:150a} ... ndaŋgbaŋ ma kenyaa wɔɛ, kenyaa wɔɛ Ba Amadu Kamara ...\\
    \gll n-laŋgbaŋ    ma-kenyaa    wɔ    ɛ    kenyaa  wɔ    ɛ    Ba      Amadu  Kamara\\
    \textsc{ncm}\textsubscript{ma}{}-man    \textsc{ncm}\textsubscript{ma}{}-uncle  3\textsc{sg}  \textsc{def}  uncle    \textsc{3sg}  \textsc{def}  Mister  Amadu  Kamara\\
    \glt ‘... his uncle's men, his uncle, Mr Amadu Kamara...' (124aw Yanker, Boy Lost at Sea: 210)

\ex\label{ex:150b} Mpanth ma ichɛk ma lɛ mpanth ŋkəlɛŋ.\\
    \gll n-panth      ma    i-chɛk      ma    lɛ    n-panth      n-kɛlɛŋ\\
    \textsc{ncm}\textsubscript{ma}\textsc{{}-}work  \textsc{ncp}\textsubscript{ma}    \textsc{ncm}\textsubscript{hɔ}\textsc{{}-}farm    \textsc{ncp}\textsubscript{ma}    be    \textsc{ncm}\textsubscript{ma}\textsc{{}-}work  \textsc{ncm}\textsubscript{ma}\textsc{{}-}good\\
    \glt ‘Farmwork is fine work.' (P67 P: 42, repeated from \REF{ex:136b}.)
\z
\z

\textit{Ma}{}-class nouns are pluralized by using an affiliation with the unambiguously plural \textit{thi} class. The plural of \textit{mputh} ‘intestines' (\textit{ma} class) is in the \textit{tha} class retaining its prefixed homorganic nasal as shown in \REF{ex:151}. Despite such possibilities, it is hard to imagine the context in which such \textit{ma} plurals, such as that of ‘intestines' would be used, i.e., ‘multiple intestines'?, even less likely when dealing with a cow's intestines rather than those of a small fish or squirrel. The following example comes from an elicitation context. ‘Intestines (pl)' takes agreement in the \textit{tha} class.

\ea%151
    \label{ex:151}
    mputh / mputh thɛ\\
    \gll n-puth            n-puth           thi-ɛ\\
    \textsc{ncm}\textsubscript{ma}{}-intestines      \textsc{ncm}\textsubscript{ma}{}-intestines    \textsc{ncm}\textsubscript{thi}{}-\textsc{def}\\
    \glt ‘intestines' / ‘the intestines(pl)' (E04 Abdulai Bendu: 11)
\z

In sum, the \textit{ma} class shows little semantic unity and some formal irregularities.

\section{\textit{Hɔ} class}
\label{sec:5.7}\hypertarget{Toc115517794}{}
As the noun class with the greatest number of members, the \textit{hɔ} class resists easy semantic characterization as shown in \tabref{tab:nounclass:38}. It includes a great variety of singulars (implements), collectives, and abstractions. Many words relate to the nautical domain: boat parts, sailing, and fishing. The class contains items of everyday use, body parts, geographic features, sicknesses, foods, and abstractions. When there is a related form in another noun class, the noun class is either \textit{ma} or \textit{tha}, with the vast majority of related forms being plurals in the \textit{tha} class. Because forms sometimes do and sometimes do not have a prefix, I have not included any prefixes in the examples. The “singular” is sometimes a mass or collective noun where there is no plural. I have left the cell blank in \tabref{tab:nounclass:38} where there is no related form. I have only indicated the noun class when a paired form exists.

\begin{table}
\caption{\label{tab:nounclass:38}Representative examples from the \textit{hɔ} class}


\begin{tabularx}{\textwidth}{lllQl}
\lsptoprule
Domain & Singular & Plural & Gloss & Comment\\
\midrule
Sea-oriented & yɔŋkɔ & tha  & ‘fishingbasket'\\
&  chocho & & ‘sea shell'  &\\
&  lel & & ‘ocean, sea' & some derived forms\\
\tablevspace
Everyday items & gbap & ma & ‘mat' & \\
&  pɛl & ma & ‘net, hammock' &\\ 
& pɛpɛ & tha & ‘calabash'\\
\tablevspace

Body parts & paka & tha & ‘spine' &\\
& kun &  & ‘stomach' &\\
& thath &  & ‘eye mucus' &\\

\tablevspace
Geographic & boŋ & tha & ‘low cliff' &\\
& gbɔthɔ & tha & ‘valley' &\\
& bian &  &‘area'  &\\
\tablevspace
Sicknesses  & gbokoth &  & ‘cowpox'  &\\
& hɔkɔ &  & ‘goiter' &\\
& lua &  & ‘hernia' &\\
\tablevspace
Food  & boo & tha & ‘bread'\\
& sek &  & ‘rice flour' &\\
& yeke & & ‘cassava' & some derived forms\\
\tablevspace
Misc & fɔn &  & ‘secret, society'  &\\
& ŋɔi &  & ‘happiness' &\\
& rɔ &  & ‘debt' &\\
\lspbottomrule
\end{tabularx}
\end{table}

Nearly all maladies and diseases fall into this class, e.g., \textit{sɔkul} ‘scabies,' with usually no related forms. The most common pairing by far is \textit{hɔ/tha}, as mentioned above.

The noun-class marker \textit{i-} when prefixed to a noun can nominalize a verb, as shown by the examples in \tabref{tab:nounclass:39}.

\begin{table}
\caption{\label{tab:nounclass:39}Verbs nominalized in the \textit{hɔ} class}
\begin{tabular}{llll}
\lsptoprule
\multicolumn{2}{c}{Verb} & \multicolumn{2}{c}{Noun}\\
\cmidrule(lr){1-2}\cmidrule(lr){3-4}
dui & ‘steal' & idui & ‘theft, stealing'\\
 chal & ‘sit' &ichɛli & ‘sitting'\\
luɛi & ‘enter' & luɛi & ‘hole, well' \\
\lspbottomrule
\end{tabular}
\end{table}

Some recent borrowings into the class include the items shown in \tabref{tab:nounclass:40}. A number of nautical terms come from Portuguese and English. Note how there are two different borrowings for ‘table.'

\begin{table}
\caption{\label{tab:nounclass:40}Borrowings into the \textit{hɔ} class}



\begin{tabular}{lll}
\lsptoprule
skuna & ‘schooner' & < Eng \textit{schooner}\\
sithir & ‘line on boat' & < Eng \textit{sheet}\\
waya & ‘fishingenclosure' & < Eng \textit{wire}\\
suga & ‘sugar' & < Eng \textit{sugar}\\
bias & ‘trip' & < Port \textit{viaje}\\
mɛsa & ‘table' & < Port \textit{mesa}\\
tɛbul & ‘table' & < Eng \textit{table}\\
mɛsɛi & ‘needle' & < Arabic via Mandinka (\citealt{Pichl1967})\\
\lspbottomrule
\end{tabular}
\end{table}

The \textit{hɔ} class is a large and unwieldy class that likely represents the collapse of several classes. At its core, however, it is the \textit{hɔ/tha} sg/pl pairing that contains items of daily use.

\section{\textit{Kɔ} class}
\label{sec:5.8}\hypertarget{Toc115517795}{}
The second largest class in the language, the \textit{kɔ} class has a disparate set of members. The largest semantic grouping contains trees, plants, and grasses in both the, kɔ/ma, pairing sub-class and in the \textit{kɔ} class with no pairing. Much smaller sub-groups are foods, everyday activities and implements, and abstractions. Like nouns in the \textit{wɔ} class, \textit{kɔ}{}-class nouns have no prefix.

\tabref{tab:nounclass:41} contains representative examples from the major semantic categories of the \textit{kɔ} class. As in \tabref{tab:nounclass:38}, I have indicated the companion class “Plural” when one exists and for mass or collective nouns with no plural, I have left the cell blank. I have only indicated the noun class when a paired form exists.

\begin{table}
\caption{\label{tab:nounclass:41}Representative examples from the \textit{kɔ} class}

\begin{tabularx}{\textwidth}{lllQl}
\lsptoprule
Domain & Singular & Plural & Gloss & Comment\\
\midrule
Trees & chu & ma & ‘mangrove' &\\
&  thɛ & ma & ‘sandpaper tree' &\\
& thok & ma & ‘tree (generic)' & \\
\tablevspace
Plants & luba & ma & ‘ringworm shrub' & \\
& te & ma & ‘tea bush' & \\
& pui &  & ‘grass (generic)' & many derived forms\\
\tablevspace
Foods & gbam & ma & ‘potato, yam' & many derived forms\\
& santh &  & ‘shrimp' & \\
\tablevspace
Body parts & bɔŋk & ma & ‘scrotum' & \\
& mɔ & ma & ‘breast' & \\
& su & ma & ‘finger' & \\
\tablevspace
Fishing \& & gbit & ma & ‘fishingnet pole' & \\
Hunting & pɛl & ma & ‘net' & many derived forms\\
& tokot & ma & ‘animal trap' & \\
\tablevspace
Misc & fai & tha & ‘Poro bush' & \\
& hu & tha & ‘yard, enclosure' & \\
& lɛka &  & ‘juju' & \\

\lspbottomrule
\end{tabularx}
\end{table}

The \textit{kɔ} class both with and without its \textit{ma}{}-class pairing is a frequent landing spot for borrowings, as illustrated in \tabref{tab:nounclass:42}.

\begin{table}
\caption{\label{tab:nounclass:42}Borrowings into the \textit{kɔ} class}

\begin{tabular}{lll}
\lsptoprule
sɛli & ‘prayer' & < Arabic \textit{salaa} ‘prayer'\\
sɔbul & ‘shovel' & < English \textit{shovel}\\
chumbu & ‘lead' & < Port \textit{chombo} ‘tool'\\
gwava & ‘guava' & < Port \textit{goiba} or Eng \textit{guava} (<Taino?)\\
kɛntri & ‘groundnut' & < Mandinka \textit{kantiga} ‘groundnut'\\
\lspbottomrule
\end{tabular}
\end{table}

\section{\textit{Tha} class}
\label{sec:5.9}\hypertarget{Toc115517796}{}
The \textit{tha} class has most of the plurals in Sherbro; it is prominently the plural for things that can be counted, as in \REF{ex:152}. Typically \textit{tha}{}-class nouns are paired with nouns from the generally singular \textit{kɔ} and \textit{hɔ} classes, as in \REF{ex:153}, which may have plurals in other classes as well, but in some cases the \textit{tha} class is the only class to which a noun belongs, as illustrated in \REF{ex:154}.

An example of the \textit{tha}{}-class morphology and agreement patterns is given in \REF{ex:152}. The definite article \textsc{def} \textit{lɛ}, the partitive \textit{pum} ‘some,' and the adjective \textit{kɛlɛŋ} ‘good' are all prefixed with the \textsc{ncm} \textit{thi} and the concordant pronoun \textit{tha} is used.

\ea%152
    \label{ex:152}
    Næthi lɛ thipum tha thikəlɛŋ.\\
    \gll nai    thi-lɛ        thi-pum      tha     thi-kɛlɛŋ\\
    road    \textsc{ncm}\textsubscript{tha}{}-\textsc{def}    \textsc{ncm}\textsubscript{tha}{}-some  \textsc{ncp}\textsubscript{tha}    \textsc{ncm}\textsubscript{tha}{}-good\\
    \glt ‘Some roads are fine.' (P67 K: 88, repeated from \REF{ex:136c})
\z

\ea%153
    \label{ex:153}
    \textit{tha}{}-class nouns paired with (singular) \textit{kɔ} and \textit{hɔ} classes

    \vspace{6pt}
    
    \begin{tabular}{lll}
    \textit{kɔ} plurals & taŋka & ‘crab pincer'\\
                        &  kumba & ‘shirt, gown'\\
                        & hɛlɛ & ‘raphia basket'\\
                        & hu & ‘yard'\\
    \textit{hɔ} plurals & pis & ‘rag, cloth'\\
                        & lathaŋ & ‘thigh'\\
                        & bes & ‘ladder'\\
                        & boŋ & ‘low cliff'\\
    \end{tabular}
\z

In most cases \textit{tha}{}-class nouns have a singular counterpart, but in some cases they do not. I give some examples in \REF{ex:154}.

\ea%154
    \label{ex:154}
    Unpaired \textit{tha}{}-class nouns\\

     \vspace{6pt}
    
    \begin{tabular}{ll}
    thigbiikan & ‘race'\\
    thigbo & ‘children's top'\\
    thigbu & ‘jaws'\\
    lomthibul & ‘unanimity' (lit. ‘voices one')\\
    thikran & ‘pile'\\
    tɔnthi & ‘song, singing'\\
\end{tabular}
\z

\section{\textit{Lɔ} class}
\label{sec:5.10}\hypertarget{Toc115517797}{}
The \textit{lɔ} class is the smallest noun class in Sherbro.\footnote{Thirty-four members as of 4 Sept 2020. The count includes some derived members.} Though small in terms of the number of underived nouns belonging to it, the prefix of the \textit{lɔ} class, \textit{li-}, can be used to derive new nouns from nouns and other classes. Another unusual feature to the class is that \textit{li}{}-prefixed forms can be used adverbially. In terms of its semantics, the class generally contains abstracts, e.g., \textit{lichol} ‘art,' \textit{limani} ‘respect,' \textit{live} ‘health,' but also a few objects of daily use, \textit{liken} ‘knife,' \textit{limenth} ‘broomstick,' \textit{thul} ‘raffia, and even the word for ‘day' itself \textit{lipal} (also the word for ‘sun,' likely its basic meaning). There's even a borrowing\is{borrowing} in the class, \textit{libushɛl} from English \textit{bushel}.

An example of an abstraction derived from a noun is the word \textit{libɛɛ} ‘chieftiancy' from the word \textit{bɛɛ} for ‘chief'. Both \textit{lithɛm} ‘love' and \textit{lithemba} come from \textit{thɛm} ‘friend, companion'. The word for ‘age' \textit{libɛn} comes from the noun \textit{bɛn} or the adjective \textit{bɛn} ‘old,' and \textit{likith} ‘shortness' from the adjective ‘short'. Abstractions can also be derived from verbs: \textit{lisei} ‘evidence' from \textit{sei} ‘testify' and \textit{live} ‘health' from \textit{ve} ‘be well or healthy'; \textit{libaŋ} ‘laziness' from \textit{baŋ} ‘be lazy.'

The most common derivational process is the formation of adverbs from nouns (and here numbers) (see \sectref{sec:7.1} on adverbialization). Like the “multiplicative” described below, an adverb is formed from the word \textit{tiŋ} ‘two' as in \REF{ex:155}.

\ea%155
    \label{ex:155}
    Bálmá lúɛ́ lítìŋ.\\
    \gll bálmáá        lúɛ́      lí-tìŋ\\
    two.edged.knife  sharp    \textsc{ncm}\textsubscript{lɔ}{}-two\\
    \glt ‘The \textit{balmaa} knife is sharp on both sides.' (lit. ‘The sharp \textit{balmaa} is doubly (sided).') (E12 Albert Yanker: 27)
\z

Though the usage did not appear in our own data Sumner  reports \textit{li-} can be used for the “multiplicative”, as in \textit{li-tiŋ}, ‘twice,' \textit{li-ra} ‘thrice' (\citealt[34]{Sumner1921}).

There are also forms with no derivational history that function as adverbs (see  \REF{ex:155} for one that does). In \REF{ex:156}, \textit{lifĩk} means something like ‘randomly' but there is no noun \textit{fik} from which it could claim descent.

\ea%156
    \label{ex:156}
    Tamɔ lɛ wɔ gbo ha len lifĩk, chen tɛnini.\\
    \gll tamɔ  lɛ    wɔ    gbo  haa  len    lifik      che-ni    tɛnini\\
    boy  \textsc{def}  \textsc{3sg}  just  do    thing    randomly  \textsc{aux-neg}  think\\
    \glt ‘The boy just does things at random, he doesn't think.' (P67 F: 16)
\z

The form may not be derived but it fits well the profile of adverbs with the prefix \textit{li-}.

\newpage
\ea%157
    \label{ex:157}
    Mpanth ma lifamalifama, la a  ni kunɛ ko ŋami ichɛliɛ   kunɛ.\\
    \gll n-panth      ma    li-fama-li-fama\\
    \textsc{ncm}\textsubscript{ma}{}-work  \textsc{ncp}\textsubscript{ma}    \textsc{ncm}\textsubscript{lɔ}{}-farmer- \textsc{ncm}\textsubscript{lɔ}{}-farmer\\
    \gll la      a    ni        kunɛ    ko    ŋa      mi    i-chɛli        ɛ    kunɛ\\
    \textsc{pro}\textsubscript{indef}  \textsc{1sg}  presently  within  in    people  \textsc{1sg}  \textsc{ncm}\textsubscript{hɔ}{}-household  \textsc{def}  inside\\
    \glt ‘Farming work, that is what I am presently doing in my household.' (090a Saidu Netteh: 59)
\z

A curious use of the prefix is in a reduplicated form, as in \REF{ex:157}, functioning as the second part of a compound, meaning ‘farming work'. The second part of the prefixed item of focus is based on \textit{fama}, a borrowing\is{borrowing} from English \textit{farmer}.

\section{Summary}
\label{sec:5.11}\hypertarget{Toc115517798}{}
The Sherbro noun class system has shown some regularities that lend themselves to generalizations but also a number of irregularities. Many of the irregularities arise in the area of agreement, and the primary agent of disruption is the importance of [\textsc{animacy}\is{animacy}], which overrides morphological considerations. It is possible the puissance of this feature has led to the demise of a distinct and fully operant class surviving today only in the \textit{si} affix of ‘animals'. It has been absorbed or is in the process of being absorbed by the \textit{ha} class, which is no longer just ‘human plurals' but also ‘animate plurals.'

Another generalization that emerges from the facts adduced above is that the system is in decay, a condition common to a dying language and to closely related languages as well (\citealt{CampbellMuntzel1989}, \citealt{Childs2009}, \citealt{Sasse1992}).
