
\begin{letter}{M}

\TCheadword[1]{ma} \textit{cf}: \TClink[2]{bi}, \TClink[2]{ha}, \TClink[3]{lɔi}, \TClink{mɔs}, \TClink[2]{ŋa}. \textit{Aux} \textbf{1)} should. \textit{Ya bi nrɔm ka, ma mɔ bɔ ramir.} I have a medicine here, it should cure you (\citealt{Pichl1967}). \textit{Wɔn wɔ gbo nani, ahã lɛ hã jɛthɛli hã ma hã mbank lɛ.} While he is pulling hard, the others should slacken their ropes (\citealt{Pichl1967}). \textbf{2)} ought (\citealt{Sumner1921}). \textbf{3)} could. \textit{Mma pakali lɛɛ thɔk lɛ thɔm mɔ lɛ ma ki duk.} Don't shake the tree branch lest your companion fall (\citealt{Pichl1967}).\textbf{4)} may.

\TCsubword{maha} (comp.) \textit{prt} should; ought. der. \TClink{manha} (see \TClink[1]{ma})

\TCsubword{manha} (comp.), (der. of \TClink{maha}) \textit{Aux} should not; ought not.

\TCheadword[2]{ma} \textit{cf}: \TClink[2]{ni}. \textit{prt} \textbf{1)} negation particle used to negate hortatives, including imperatives. \textit{Mma buŋ ba mɔ sua!} Don't oppose your father! (\citealt{Pichl1967}). \textit{Hí má kɔ̀.} Let us not go. \textit{Yi ma yom kil l'ay thi-hi ko lɔn che igbɛth.} We shouldn't allow dirt in our houses (\citealt{Pichl1967}). \textit{Mma wɔ ka fem dɛ!} Do not give him my money! (\citealt{Pichl1967}). \textit{Hã lɛ mma wɔ pɔkɔni, wɔ lɛ nɔdwiyɛ!} You should not forget about him, he's a thief! (\citealt{Pichl1967}). \textbf{2)} would have. \textit{Há mà kɔ́‚ sí kòŋ bɛ̀ kɔ̀nìɛ́ [yɛ].} He would have gone‚ but he was waiting for something. \textit{Yí má kòŋ kɔ́nì.} We would have gone. \textit{A má ná bɛ́ kòŋ kɔ̀nì.} I would have gone.

\TCheadword[3]{ma} \textit{NCP} \textbf{1)} they. \textit{Mbɔŋ ma pipɛ ma bɛmpani iwɔm.} Barrel bungs are made of wood (\citealt{Pichl1967}). \textit{Ikoni shiɛlɛ Mbolomdɛ ma chaŋ theli ka, nye?} We now know that they speak Sherbro here more, right? \textbf{2)} them. \textbf{3)} it. \textit{Thetha mi ka che ŋa mpanth ma landɛ pɛŋ bifo wɔ mmu hu.} My grandmother used to do the work before she died. \textit{Jizɔs ŋa ja bom ba ŋa yaŋ, yɛ peyɛ nkɔŋ ma wɔlɛ.} Jesus has done a big thing for me when He shed his blood. \textit{Kɛ wɔ theli Mbolomdɛ ni wɔ ma pɛ gbal?} But he speaks Sherbro as well as writes it? \textbf{4)} which. comp. \TClink{gbɔlmafe} (see \TClink{gbɔl}), \TClink{ndɛthmaboot} (see \TClink{dɛth}), id. \TClink{muŋkma} (see \TClink{muŋk}) 

\TCsubword{manante} (der.) \textit{temp} until the present; until now; up to this day. \textit{Mɔm, frɔm yɛpɔka gbem mɔ haŋ ma nandɛ, yɛ nko ke wɔlɔɛ frɔm kache haŋ ma nande, ŋɔ nkeni wɔlɔa?} From since you were born until today, since you have seen the world in the past up until now, how do you see the world? \textit{Anyaɛ ŋani gbo vel yel lo ɛ Planti ko haaŋ ni manante.} People have been calling it Plantain ever since.

\TCsubword{manɛ} (unspec.) \textbf{1)} \textit{dem} those. \textit{I koŋ gbo, iban mthɔkɛ manɛ malɔ, man gbi.} When we have finished, we have to gather all of those sticks, all of them. \textit{Aa, ashila manɛ maŋa chiɛ maa kritikallɛ…} Yes, I know that, those they bring to us that are critical… \textbf{2)} \textit{dem} that.

\TCheadword[4]{ma} \textbf{1)} \textit{adp} with. \textit{Aa, Mbɛkɛ ki ma pɔ hɔ ma apumaɛ.} Yes, it is this Krio they speak with the children. \textit{Sɛkɛsɛkɛ we, Hɔbatokɛ che ma mɔ.} So thank you for that, may God be with you. \textit{Sɛkɛwei, Hɔbatokɛ chema mɔ ni.} Thank you, may God be with you. \textbf{2)} \textit{post} for. \textit{Yɛ yi ka che ko tallɛ, yi yukɔ wɔ ma.} When we were younger, we planted it for him. \textbf{3)} \textit{post} at. comp. \TClink{kɔsmahwɛ} (see \TClink{kɔs}), \TClink{lɔlma} (see \TClink{lɔl}), der. \TClink{gbɛmani} (see \TClink{gbɛ}), \TClink[1]{kɔma} (see \TClink[2]{kɔ}) 

\TCsubword{-mani} (comp.) \textit{cf}: \TClink{-kani} (der. of \TClink{-k}, \TClink{-ni}) \textit{v > v} \textit{sfx} combination of two verb extensions.

\TCsubword{binthima} (unspec.) \textit{cf}: \TClink{sɔima}. \textit{v} \textbf{1)} [bìnthìmà] mix (K dialect); \textit{binthma} mix (\citealt{Sumner1921}). \textit{Wɔ́ bìnthìmà bòɛ̀. Hí bìnthìmà jóɛ̀.} She mixed the rice flour (with water). Let us mix the food. \textbf{2)} \textit{binthma} bring together (syn. \textit{sɔyma}) (\citealt{Pichl1967}). \textbf{3)} confuse.

\TCsubword{binthimani} (unspec.) \textit{v} join together. \textit{Ni anya tilaŋ Planti ka, ŋa binthimani ha lɛlie wo hɛllɛ ko.} And other people from Plantain (Island) joined together to look for him on the sea. \textit{Ni Nsheŋke ka bɛ, anya gber binthimani Mma Niomai Sɔmna wa ka Braima mbɛoya.} In Shenge here, too, many people joined Ma Naomi Sumner to give Braima gifts. 

\TCsubword{binthmabinthma} (unspec.), (der. of \TClink{binthima}) \textit{v} mix. \textit{Gbi ni ngefeyɛ, mɔi binthmabinthma mpuliɛpuliɛ mɔi nɛmil labo iyɛllɛ ŋɔ shilɔ che.} Together with the pepper, you mix it up, and then you taste it to know if the salt is okay.

\TCheadword[5]{ma} \textit{subordconn} subordinating particle. \textit{Itɔnk wa, ŋa mpanth ma wɔ kɛlɛn dɛ.} Celebrate for the wonderful work he has done. \textit{A chen duki pɛl, nhukɛ ma a dukiɛ.} I do not use a net, I use hooks. \textit{Ipulukɛ gbi ma lɔɛ pɔ ma lɔ koŋ hok.} All the piles (of branches and leaves) that are there are taken out. \textit{Bikɔ pomdɛ wɔ mi ni yɛthi sɔŋgɔ ma ŋɔ nɔpikan wɔ ŋa yɛthi nɔma wɔi.} Because my husband is really treating me as a husband should treat his wife.

\TCheadword[6]{ma} \textit{n} forgiveness.

\TCheadword[7]{ma} \textit{NCM} noun class marker (ma). \textit{Ndaŋgbaŋ ma kenyaa wɔɛ, kenyaa wɔɛ Ba Amadu Kamara wɔ ayɛn.} His uncle's men, even his uncle, Mr. Amadu Kamara.

\TCheadword{ma-} \textit{cf}: \TClink{n-}. \textit{NCM} \textit{pfx} noun class marker (ma).

\TCheadword{maa} \textit{cf}: \TClink[2]{laa}, \TClink{wante} (der. of \TClink[1]{waŋ}), \TClink{uman}. \textit{n} \textbf{1)} female. \textit{Wa maɛ, wɔ tika, Mɔmi Prat ki wante wɔi, wɔ tika.} A girl, she is in this town, Mummy Pratt's sister. \textit{Ama awɔ apokan awɔ?} How many girls, how many boys? \textbf{2)} woman. \textit{Pɔ koŋ gbo pɔi gbɛki amaɛ ŋai kɔni ko futh pɛlɛ.} When they have finished, they hire the women to go and uproot the rice. \textit{Ama ŋa Kadiatu Bɛndu, Isata Bɛndu, Ramatu Bɛndu ni Aminata Bɛndu.} The women are Kadiatu Bendu, Isata Bendu, Ramatu Bendu, and Aminata Bendu. \textit{Amaa ki, apum ŋa pos gbam dɛ, apum ŋa pos yekeɛ.} These women, some were peeling potatoes, others peeling cassava. \textbf{3)} wife. \textit{Aa, ba mi bi ama ara.} Yes, my father had three wives. comp. \TClink{bɛmaa} (see \TClink[3]{bɛ}), \TClink{mɔmana} (see \TClink[1]{mɔ}), \TClink{nama} (see \TClink[1]{na}), \TClink{nɔmaa} (see \TClink{nɔ}), \TClink{nɔmachondal} (see \TClink{nɔ}), \TClink{ramaa} (see \TClink[3]{ra}), \TClink{sɔkma} (see \TClink{sɔk}), \TClink{wantama} (see \TClink[1]{waŋ}), \TClink{waŋmaa} (see \TClink[1]{waŋ}), \TClink{wɔkmmɔ} (see \TClink[1]{wɔk}), der. \TClink{nyama} (see \TClink{nɔ}) 

\TCheadword{Mach} \textit{nam} March.

\TCheadword{Madam} \textit{nam} Madam. \textit{Veɛni ka che Bachalɔ ko, ko kil Madamdɛ Bachalɔ ko.} He did not stay long and he was staying at Bachalor, at Madam's house at Bachalor.

\TCheadword{Magba} \textit{nam} Magba, name given by Yase Society.

\TCheadword{magi} \textit{n} Maggi seasoning sauce. \textit{Yɛ mɔ pin yuwɛ, nchiɛ nkuaiɛ, ngefeɛ, yabasɛ, ni magi mɔɛ mɔi bɛ lalako.} After buying the fish, you bring the palm oil, the pepper, the onion, and the Maggi, then you put (it) on the fire (to cook). 

\TCheadword{maha} (comp. of \TClink[1]{ma}, \TClink[2]{ha}, see \TClink[1]{ma})

\TCheadword{Mahan} \textit{nam} Mahan, name given to 4\textsuperscript{th} daughter.

\TCheadword{mai} \textit{cf}: \TClink{gbogbotok} (unspec. of \TClink[3]{gbogbo}), \TClink{kɔm}, \TClink{tom}, \TClink[2]{wo}. \textit{n} \textit{mæ} (ma) female sexual organ (polite) (\citealt{Pichl1967}). 

\TCheadword{mak} \textit{v} mark. \textit{Rai landɛ bul fli ŋɔ ko tuk kɛ rai ɛ ŋalɛ ŋɔ lɔ, lɔ a kache makɛ.} Every one of these papers has disappeared; it is just the other one that is there, where I used to keep records. 

\TCheadword{makɛt} (Eng \textit{market}) \textit{n} market. \textit{Labo keŋkeŋdɛ kɔ mɔ chɛth, mɔi kɔ makitai, laboɛ yuwɛ mɔ wɔi pin.} If it is krain-krain you are cooking, you go to the market, if there is fish you buy it.

\TCheadword{makintɔsh} (Eng \textit{Macintosh}) \textit{n} Mac computer. \textit{Abiɛ lɔni bopɛ sizɔs kunɛ abiɛ lɔni Makintɔsh kunɛ.} I do not have the scissors in it (the kit), nor do I have the Macintosh in it.

\TCheadword{Malama} \textit{nam} Malama, male name given to a person. \textit{Laŋgba dɛ fli wɔ ŋa fɛtɛndɛ Lɔmli, Malama, Bolomnɔ.} Even the man they are close with at Lumley, Malama, is Sherbro.

\TCheadword{malaŋ} \textit{n} scar.

\TCheadword{malka} (Arabic {\textarab{ملاك} }\textit{malak} ‘angel') \textit{cf}: \TClink{pɔkdinthɛ} (comp. of \TClink[3]{pɔk}, \TClink{dinthɛ}). \textit{n} \textbf{1)} (wɔ/hã, N) angel (\citealt{Pichl1967}). \textbf{2)} (wɔ/hã, N) egret, at Shenge used for white cattle egret -- \textit{pɔɔk dinthɛ} (\citealt{Pichl1967}). 

\TCheadword{malɔ} \textit{cf}: \TClink{kɛntri}. \textit{n} groundnut. \textit{Ŋgɛtiɛ malɔ gbo mɔ bɛ nton.} If you have groundnut, add a little.

\TCheadword[1]{mam} \textit{n} tears. \textit{Ligber yi pei imam hi lɛ.} Often we shed our tears (\citealt{Pichl1967}). 

\TCheadword[2]{mam} \textit{v} laugh. \textit{Kɛ tirɛ lɔ Kaiŋ Taso kɔ thunɔ laa wɔɛ, lɛ nɔ wu lɔ gbo, pɔ chelɔn maam tokɛ kathba.} But this town where Kain Tasso found his wife, if somebody dies there, no one will laugh loudly.

\TCsubword[3]{mam} (der.) \textit{n} laughter.

\TCheadword[3]{mam} (der. of \TClink[2]{mam})

\TCheadword{mama} \textit{nam} head of Yase society.

\TCheadword{Mamu} \textit{nam} Mamu, name given to a place located in Bumpeh Chiefdom, Moyamba District. \textit{Ya gbemni Nyemɔko, Mamu Sɛkshɔn, Bompɛ Chifdɔm, Mɔyamba Distrikt.} I was born in Moyeamoh, Mamu Section, Bumpeh Chiefdom, Moyamba District.

\TCheadword[1]{man} \textit{v} burn. \textit{I kon gbo iban mbithiɛ manɛ malɔ, man gbi.} Once we have finished, we gather those sharp sticks that are there and burn them.

\TCheadword[2]{man} \textit{cf}: \TClink[1]{mɛkin} (der. of \TClink[1]{mɛk}, \TClink[1]{-n}), \TClink{thapa}. \textit{v} stop; leave off.

\TCsubword[1]{mani} (der.) \textit{v} stop; leave off.

\TCheadword{manante} (der. of \TClink[3]{ma}, \TClink{nante}, see \TClink[3]{ma}) 

\TCheadword{manawa} (Eng \textit{man-o-war}) \textit{n} warship.

\TCheadword{mane} \textit{cf}: \TClink{bɔfima}, \TClink[1]{fɔnwɛi} (comp. of \TClink[1]{wɛi}), \TClink{humoe}, \TClink[3]{wɔm}, \TClink{yasi}. \textit{n} cleansing medicine similar to humoe – Mende or Themne in origin (\citealt{Hall1938}). 

\TCheadword{manej} (Eng \textit{manage}) \textit{v} manage. \textit{Nɛva maind yɛ ibiyɛn dɛ, kɛ stil ai maneg bikɔs pomdɛ che ŋa mpanth, biyɛni.} Never mind that he does not have (money), but we are managing because my husband is not working, he does not have (any money). 

\TCheadword{Manɛ} \textit{nam} Maneh, name given to 6\textsuperscript{th} daughter. \textit{Tɔmi, Manɛ, Tisana, ŋai gbemɔ wanteyi.} He was Tommy, Mane, Tisana, then they gave birth to our sister.

\TCheadword{manɛ} (unspec. of \TClink[3]{ma}) 

\TCheadword{manha} (der. of \TClink{maha} (comp. of \TClink[1]{ma}, \TClink[2]{ha}), \TClink[2]{ni}, see \TClink[1]{ma}) 

\TCheadword[1]{mani} (der. of \TClink[2]{man}, \TClink[1]{-i}, see \TClink[2]{man}) 

\TCheadword[2]{mani} \textit{cf}: \TClink{rɛspɛkt}, \TClink[1]{yiki}. \textit{n} \textbf{1)} respect. \textit{I bo ŋa ka ha limani.} We just need to give them respect. \textit{Kɛ boŋgo pɔ che pɛ ka ha ŋabɛn limani gbi?} But now they do not give elders respect at all? \textbf{2)} consent. comp. \TClink[1]{yimani} (see \TClink[1]{yi}) 

\TCheadword{-mani} (comp. of \TClink[4]{ma}, \TClink{-ni}, see \TClink[4]{ma}) 

\TCheadword{Maniŋka} \textit{nam} Maninka people. \textit{Mɛndenɔ gbi wɔ Koromanɔ wɔ Maniŋka.} All the Mende people who are Koroma people are Maninka people.

\TCheadword{Maniŋkanɔ} (der. of \TClink{Maniŋka}, \TClink{nɔ}, see \TClink{nɔ}) 

\TCheadword{Manli} \textit{nam} Manley, name given to a person. \textit{Ya a Siril Manli.} I am Cyril Manley.

\TCheadword{manteŋka} (Port \textit{manteiga} ‘butter') \textit{n} butter.

\TCheadword{Manu} \textit{nam} Manu, name given to person. \textit{Choo Manu ŋɔ pɔ gbemka miɛ.} Cho Manu is the name I was born with. \textit{Kɛ ilel mi sendɛ gbi ŋɔ Choo Manɔ.} But my first name is Cho Mano.

\TCheadword{maŋgi} \textit{n} juju type.

\TCheadword{maŋko} \textbf{1)} \textit{n} mango. \textit{Yɛ ŋ kɔ gbo gadin dai, chiɛ mi mmango mpum.}	When you go to the garden, bring me some mangoes (\citealt{Pichl1967}). m̀màŋgùɛ̀ (mà) kóŋ dùm / má dùmɔ̀ / mà dúm. The mango is ripe (all three have same interpretation). \textbf{2)} \textit{nam} name of an island. \textit{Aa, yel bul ŋɔ lɔ bɛ ko, ŋɔ pɔ velɛ Gɔvna Maŋɡo.} Yes, there is even an island there they call Governor Mango. \textit{Lɔ pɔ velɛ Gɔvna Maŋɡoɛ vɛ.} There they call Governor Mango. 

\TCheadword{mared} (Eng \textit{married}) \textit{adj} married. \textit{Wɛl, wɔn bɛpɛ ka cheɛ mared uman, wɔi pɛ cheɛ sokonɔ Bondo.} Well, she herself was a housewife, and she was also the head of the Bondo Society. \textit{Yɛlaio wɛ, yɛ jaɛ ma ko ŋani mgbeɛ ŋɔ maredɛ kɔ bi ni prɔblɛm thɛ.} Nowadays, when things are abundant, all the marriages are full of problems.

\TCheadword{Mari} \textit{nam} Mary, female name given to a person. \textit{Ka gbemɔ ya wɔ Mari lɛ.} Who was born of his mother Mary (\citealt{Pichl1967}).

\TCheadword{mashin} \textit{n} machine.

\TCheadword{maso} \textit{n} head of Bondo Society.

\TCheadword{math} \textit{v} \textbf{1)} hide. \textbf{2)} safekeep. \textit{Ila bia yɔk Potho ko, ila lɔ ko math ŋal mɔ.} We are taking it abroad, we are going to keep it there for you. \textit{Mbolomdɛ ki ma ma iyema math.} This Bolom is what we want to keep. \textbf{3)} keep.

\TCsubword{mathbele} (comp.) \textit{n} children's game hide-the-disk, involves a disk of straw, 2 inch diameter, with a hole in its center, covered with sand, child has to try to stick [a straw?] through the hole. Child who succeeds first is the winner (\citealt{Pichl1967}). 

\TCsubword{mathboni} (comp.) \textit{cf}: \TClink[2]{boni}. \textit{n} children's game hide-and-seek. \textit{Chaŋ bo yɛ ikache math boni ɛ.} Only when we used to play hide-and-seek.

\TCsubword{mathmathnin} (der.) \textit{v} hide. \textit{Yɛ imath-mathnindɛ apikandɛ ŋani thoŋi-thoŋi siŋthɛ vɛ…} When we would hide and the boys would run after us, (in) those games…

\TCsubword{mathni} (der.) \textit{v} hide oneself. \textit{Ba Na lee mathini.} Mr. Spider stayed behind to hide himself (\citealt{Pichl1967}).

\TCsubword{mathui} (der.) \textit{cf}: \TClink{pikɛ} (der. of \TClink[1]{pi}, \TClink{-k}). \textit{v} be hidden. \textit{Ba Thəngbəŋ lee mathui bach lɛ veleŋ che-lɛ mɔ hunki gbo...} Mr. Bat remained hidden behind a young palm tree so that if somebody came there... (\citealt{Pichl1967}). 

\TCsubword[2]{mathin} (unspec.) \textit{v} hide. \textit{Kɔ mathin yaŋ che nai lɛ ibol hã pakali mi.} He went to hide ahead of me on the road, in order to scare me (\citealt{Pichl1967}). der. \TClink[1]{mathin} (see \TClink{math})

\TCsubword[1]{mathin} (unspec.), (der. of \TClink[2]{mathin}) \textit{n} \textbf{1)} shelter. \textit{Mathin hĩ lɛ hink ŋgbathïl gbi.} Our shelter from all the troubles (\citealt{Pichl1967}). \textbf{2)} hiding place. \textit{Ya koŋ sotho mathin kəlɛŋ.} I have (got) a good hiding place (\citealt{Pichl1967}). 

\TCheadword{mathbele} (comp. of \TClink{math}, \TClink{bele}, see \TClink{math}) 

\TCheadword{mathboni} (comp. of \TClink{math}, \TClink[1]{boni} (der. of \TClink[1]{bo}, \TClink{-ni}), see \TClink{math}) 

\TCheadword[1]{mathin} (der. of \TClink[2]{mathin} (unspec. of \TClink{math}), see \TClink{math}) 

\TCheadword[2]{mathin} (unspec. of \TClink{math})

\TCheadword{mathmathnin} (der. of \TClink{math}, \TClink{-ni}, \TClink[1]{-n}, see \TClink{math}) 

\TCheadword{mathni} (der. of \TClink{math}, \TClink{-ni}, see \TClink{math}) 

\TCheadword{mathui} (der. of \TClink{math}) 

\TCheadword{may} \textit{cf}: \TClink[2]{kɔnɛ}. \textit{v} forgive.

\TCheadword{Mayeni} \textit{nam} Mayeni, female name given to a person. \textit{Yami wɔɔ Mayeni Laŋgo, Manɔ Dodo.} My mother is Mayeni Lango, Mano Dodo.

\TCheadword{Mayeŋka} \textit{nam} Mayenka. Name of a famous cotton tree on York Island. When the Poro chief of Bomplik pointed at it with his horn, the tree bowed down to the water and rose again (\citealt{Pichl1967}).

\TCheadword{Mayma} \textit{nam} diminutive of Miriam, female name given to a person.

\TCheadword{mba} \textit{cf}: \TClink{tobae}. \textit{n} comrade.

\TCheadword{Mbonan} \textit{nam} Morbondan, name given to a place. \textit{Mbonan ko.} At Morbondan.

\TCheadword{Mbontɛ} \textit{nam} Mbonte, female name given to a person.

\TCheadword{Mboŋka} \textit{nam} Mbonka, name given to a place. 

\TCheadword{Mbuɛ} \textit{nam} Mbueh, name given to a place. \textit{Mbuɛ ko.} At Mbueh.

\TCheadword{Mebɛl} \textit{nam} Mabel, female name given to a person. \textit{Yaŋ ilel miɛ hɔn Mabɛl Lɔ.} My name is Mabel Lohr. \textit{Mi Mebɛl, yɛ nka che ko tallɛ, nkache sin?} Mammy Mabel, when you were young, did you used to play?

\TCheadword{meni} \textit{cf}: \TClink[2]{pot}, \TClink{woso}. \textit{n} clay.

\TCheadword[1]{mɛ} \textit{subordconn} as. \textit{Tamɔ tondɛ wɔ gbaŋkthani kotha kathil bom mɛ nɔ bɛn.} The small boy wrapped the big Kente cloth around himself as if he were a big man (\citealt{Pichl1967}).

\TCheadword[2]{mɛ} \textit{prep} like. \textit{I koi pisthɛ iraparapa tha iŋakɔ mɔi bɔl.} We would take small pieces of cloth and make it like a ball.

\TCheadword{mɛi} \textit{v} warn.

\TCheadword[1]{mɛk} \textit{cf}: \TClink{lɛkɛlɛkɛ}. \textit{v} finish.

\TCsubword[1]{mɛkin} (der.) \textit{cf}: \TClink[2]{man}, \TClink{mɛkni} (der. of \TClink[1]{mɛk}, \TClink{-ni}), \TClink{thapa}. \textit{v} \textbf{1)} stop. \textit{Ndɔ mɔ mɛkɛnda?} Where did you stop? \textbf{2)} end. comp. \TClink{pɔkmɛkin} (see \TClink[1]{pɔk}) 

\TCsubword[2]{mɛkin} (der.) \textit{adv} finally; lastly. \textit{Mɛkin dɛ ya kɔ ni sit Wasi ɛ, Kiamp ka pɛ, ni mpɛnteŋamiyɛ gbi…} Lastly, after I sit the WASSCE (West African Senior School Certificate Exam), again here in Freetown, and all my brothers…

\TCsubword[3]{mɛkin} (der.) \textit{adj} last. \textit{Beo, yalɔ komɔ mɛkindɛ ni gbi ko hin.} No, I am the very last child of all of us. \textit{Ha asanth kɛ a gbe yaŋ ya veleŋ thimɛkin ni.} The older ones are numerous but I am after the last ones.

\TCsubword[4]{mɛkin} (der.) \textit{n} end. \textit{Kɛ mɛkindɛ achɔŋɔ Hɔbatokɛ sɛkɛ.} But in the end I thank God.

\TCsubword{mɛkni} (der.) \textit{cf}: \TClink[1]{mɛkin} (der. of \TClink[1]{mɛk}, \TClink[1]{-n}). \textit{v} \textbf{1)} finish. \textbf{2)} end. \textbf{3)} stop. \textit{Iko mɛkni nande?} Have we stopped today? \textit{Dɔndɔ amɛkɛndɛ mi yo.} It is there I stopped.

\TCheadword[2]{mɛk} \textit{cf}: \TClink{Amɛrika}. \textit{adj} \textit{mək} American (\citealt{Pichl1967}).

\TCsubword{mɛknɔ} (comp.) \textit{n} \textit{məknɔ} (wɔ/hã) American (\citealt{Pichl1967})

\TCheadword[1]{mɛkin} (der. of \TClink[1]{mɛk}, \TClink[1]{-n}, see \TClink[1]{mɛk})

\TCheadword[2]{mɛkin} (der. of \TClink[1]{mɛk}, \TClink[1]{-n}, see \TClink[1]{mɛk}) 

\TCheadword[3]{mɛkin} (der. of \TClink[1]{mɛk}, \TClink[1]{-n}, see \TClink[1]{mɛk}) 

\TCheadword[4]{mɛkin} (der. of \TClink[1]{mɛk}, \TClink[1]{-n}, see \TClink[1]{mɛk}) 

\TCheadword{mɛkni} (der. of \TClink[1]{mɛk}, \TClink{-ni}, see \TClink[1]{mɛk}) 

\TCheadword{mɛknɔ} (comp. of \TClink[2]{mɛk}, \TClink{nɔ}, see \TClink[2]{mɛk}) 

\TCheadword{mɛl} \textit{v} \textbf{1)} leave. \textbf{2)} stop.

\TCsubword{mɛlkɛni} (der.) \textit{v} surrender; give up; abandon oneself.

\TCsubword{mɛlni} (der.) \textit{v} let oneself go.

\TCheadword{mɛlkɛni} (der. of \TClink{mɛl}, \TClink{-ni}, see \TClink{mɛl}) 

\TCheadword{mɛlni} (der. of \TClink{mɛl}, \TClink{-ni}, see \TClink{mɛl}) 

\TCheadword{mɛmba} (Eng \textit{remember}) \textit{cf}: \TClink{lomani}, \TClink{lonibolɛ}, \TClink[2]{tɛn} (der. of \TClink[1]{tɛn}). \textit{v} remember. \textit{Kɛ ŋanɛ ŋa wuɛwuɛ ni ache pɛ mɛmba hin awɔ ile lɔ, hin awɔ ile lɔɛ... yi abaot amɛnbul.} But some have died so I do not remember how many of us remain, how many of us remain there... we are about six.

\TCheadword{mɛmi} \textit{v} \textbf{1)} be happy. \textit{Yaa mɛmiɛ ni ŋa mɔm shi nɛndɛ ŋɔ pɔ gbem mɔ.} I am happy that you know your date of birth. \textit{Lɛ ŋa theɛ la gbo, ŋa bia che amɛmiɛ ni.} If they hear that, then they would be happy. \textbf{2)} be joyful. La che le mɛmin. It will be joyful (\citealt{Pichl1967}). \textbf{3)} be glad.

\TCsubword{mɛmiɛni} (unspec.) \textit{v} \textbf{1)} rejoice; be glad. \textbf{2)} be happy. \textit{Ya la mɛmiɛni fli ha haŋ mpanth haŋ pɔkimdɛ.} I am happy about that, to really work for my country. \textit{Wɛl imɛmiɛni ŋa hin sɔthɔ mɔ.} Well we are happy to have you.

\TCsubword{mɛmilni} (unspec.) \textit{cf}: \TClink{ŋɛi}. \textit{v} smile.

\TCsubword{mɛmin} (der.) \textit{cf}: \TClink{ŋɔi}. \textit{n} joy.

\TCheadword{mɛmiɛni} (unspec. of \TClink{mɛmi}, \TClink{-ni}, see \TClink{mɛmi}) 

\TCheadword{mɛmilni} (unspec. of \TClink{mɛmi}, \TClink{-ni}, see \TClink{mɛmi}) 

\TCheadword{mɛmin} (der. of \TClink{mɛmi}) 

\TCheadword{Mɛmorial} (Eng \textit{memorial}) \textit{nam} Memorial. \textit{Shenge ka fli skullɛ ŋɔ pɔ wɔ Hawɔd Mɛmorial vɛ.} It is in Shenge here in that school called Howard Memorial.

\TCheadword{Mɛmuna} \textit{nam} Memuna, female name given to a person. \textit{Ŋgɔ Mɛmuna wɔi gbemɔ atiŋ.} Aunty Memuna gave birth to two.

\TCheadword[1]{mɛn} \textit{Numb} five. \textit{I ka che amɛŋra kɛ ile nibo amɛn.} We were eight but only five of us are left. \textit{I amɛn bullɛ ka koŋ wu.} We are five, one died a while ago. \textit{Ika gbem apima amɛn, ayɔl ŋa lɔɛ, ama tiŋ, apikan atiŋ.} We had five children, there are four left, two girls and two boys.

\TCsubword{mɛnbul} (comp.) [mɛ́ːmbùl] \textit{Numb} six (lit. five-one). \textit{Wɛl gbem apima amɛnbul.} Well, he gave birth to six children. \textit{Mpang mən-bul bɛlɛng buli, mən-bul bɛlɛng hãlɛ.} Six months on the one side, six months on the other side (\citealt{Pichl1967}). 

\TCsubword{mɛnhiɔl} (comp.) \textit{Numb} nine. \textit{mɛ́ɛ́yɔ̀l} nine. \textit{Pɔ gbem mi paŋdɛ ŋɔ pɔ wɔ Sɛptɛmbaɛ, paŋ mɔikɛ mɛnyɔllɛ.} I was born in the month that they call September, the ninth month. \textit{Abi apima mɛnyɔl.} I have nine children.

\TCsubword{mɛnra} (comp.) \textit{Numb} eight. \textit{t̪s̪ə́n ɹà} eight. \textit{I ka che amɛŋra kɛ ile nibo amɛn.} We were eight but only five of us are left. comp. \TClink{waŋnimɛnra} (see \TClink[2]{waŋ})

\TCsubword{mɛntiŋ} (comp.) \textit{Numb} seven. \textit{mɛ́ːnt̪s̪ə̀n} seven. comp. \TClink{waŋnimɛŋtiŋ} (see \TClink[2]{waŋ})

\TCsubword{mɛŋhiɔlniwaŋ} (comp.) \textit{Numb} nineteen. \textit{Wul bul kɛmɛ koŋhɔanya mɛŋhiɔlniwaŋ, koŋhɔanya hiɔl ni mɛŋbul.} nineteen hundred eighty-six (1986) (lit. one thousand hundred completed-person five-four-and-ten, completed-person-four and five-one) (K dialect). 

\TCheadword[2]{mɛn} \textit{post} under. \textit{Nrebɛllɛ ŋa hun, ŋa hun tho, ikɔni mɛn ko.} The rebels came, then they drove us out, and we went to the countryside. comp. \TClink{lemmɛnnhyɛl} (see \TClink[1]{le}), \TClink{nuimɛn} (see \TClink{nui}) 

\TCsubword{mɛnɛi} (der.) \textit{Loc} under.

\TCheadword[3]{mɛn} [mɛ̀n] \textit{cf}: \TClink{halthe}, \TClink[2]{hɛlɛ}, \TClink[5]{lel}. \textit{n} \textbf{1)} water. \textit{Ye lɛ kulɔ gbo ni mən bɔsul, mɔ bi ipula mɔm kunɛ.} Then if you drink unboiled water, you (will) have worms in the belly (\citealt{Pichl1967}). \textit{Tɛm hɔ̃ gbo ke̹n mɛn nsoso lɛ hɔ̃ chenk anyathi gbi}. Time is like running water, it carries people away (\citealt{Pichl1967}). \textbf{2)} sea. \textit{Mən dɛ kong gbəta.} The sea has ebbed completely (\citealt{Pichl1967}). \textit{Kɛ be, kilikɛ ŋɔ ton ha bɔɔ yɛthi wɔm dɛ mmɛn nyamban dɛai huɛ vɛ.} But no, the anchor was (too) small to hold the canoe in the rough sea that day. comp. \TClink{kulmmɛn} (see \TClink[1]{kul}), \TClink{puluk-mɛn} (see \TClink{puluk}) 

\TCsubword{katamɛn} (comp.) \textit{n} \textit{kata-mən} (wɔ/hã, N) worm species, wormlike animal (clamworm - Nereis?) found at lower parts of the beach (\citealt{Pichl1967}).

\TCsubword{kobo-mɛn} (comp.) \textit{n} \textit{kobo mən} (hɔ̃/tha) jar for water (\citealt{Pichl1967}).

\TCsubword{mɛnnhyɛl} (comp.) \textit{n} \textit{mən nhyɛl} (ma) salt water, sea-water (\citealt{Pichl1967}).

\TCsubword{mɛnnjal} (comp.) \textit{n} \textit{mən njal} (ma) living water, i.e., water not boiled or filtered or water which was not left in a vessel overnight. Also water fetched from the well if the latter was not disturbed during the night. For certain medicines only living water must be used (\citealt{Pichl1967}). 

\TCsubword{mɛnŋgeta} (comp.) \textit{n} \textit{mən ŋge̹ta} dry shore at ebb-tide (\citealt{Pichl1967}).

\TCsubword{mɛnŋkus} (comp.) \textit{n} \textit{mən ŋkus} (ma) water left standing overnight (\citealt{Pichl1967}). 

\TCsubword{mɛnpɛyɛ} (comp.) \textit{cf}: \TClink[4]{gbi}, \TClink[2]{hɛlɛ}, \TClink{hɛliŋ}. \textit{n} \textit{mən pɛyɛ} high tide; the water is high (\citealt{Pichl1967}).

\TCheadword{mɛnbul} (comp. of \TClink[1]{mɛn}, \TClink[3]{bul}, see \TClink[1]{mɛn}) 

\TCheadword{Mɛnde} \textit{nam} \textbf{1)} Mende people. \textit{Kɛ ayema ni ncheŋa piŋgiyɛ Mbolomdai, bikɔs lɛ nkɔgbo Mɛndɛ ko…} But I want you to be replying to them in Sherbro, because if you go to the Mendes… \textbf{2)} Mende language. \textit{Nwɔk nra ma pɔ chaŋ theliɛ, Mbolomdɛ, Mmɛndeɛ ni Nthemdɛ.} They speak three languages here: Sherbro, Mende, and Themne.

\TCheadword{Mɛndenɔ} (comp. of \TClink{Mɛnde}, \TClink{nɔ}, see \TClink{nɔ}) 

\TCheadword{mɛndɛ} (men, mɛmɛ) \textit{n} mirror; glass. \textit{Wɔ̀ pɛ́l mɛ̀ndɛ̀ɛ̀.} He broke the glass.

\TCheadword[1]{mɛnɛ} \textit{cf}: \TClink[2]{tɔ}. \textit{n} \textbf{1)} grave. \textit{Ka hok hiŋk mɛnɛ ko. Wɔ mɛnɛ ko.} He came from the grave. He is in the grave (\citealt{Pichl1967}). \textbf{2)} bottom. \textit{Ira thoɛ yeŋkɛlɛŋ-yeŋkɛlɛŋ mɛnɛ ko.} We brush the bush well, right under.

\TCheadword[2]{mɛnɛ} (der. of \TClink[3]{mɛnɛ}) \textit{Loc} bottom of the sea.

\TCheadword{mɛnɛi} (der. of \TClink[2]{mɛn}, \TClink[1]{ɛ}, see \TClink[2]{mɛn}) 

\TCheadword{mɛnhiɔl} (comp. of \TClink[1]{mɛn}, \TClink{hiɔl}, see \TClink[1]{mɛn}) 

\TCheadword{mɛni} \textit{cf}: \TClink[2]{thɛki}. \textit{v} kindle.

\TCheadword{mɛnnhyɛl} (comp. of \TClink[3]{mɛn}, \TClink[2]{hɛlɛ}, see \TClink[3]{mɛn}) 

\TCheadword{mɛnnjal} (comp. of \TClink[3]{mɛn}) 

\TCheadword{mɛnŋgeta} (comp. of \TClink[3]{mɛn}) 

\TCheadword{mɛnŋkus} (comp. of \TClink[3]{mɛn}, \TClink[2]{kus}, see \TClink[3]{mɛn}) 

\TCheadword{mɛnpɛyɛ} (comp. of \TClink[3]{mɛn}, \TClink[1]{pɛ}, see \TClink[3]{mɛn}) 

\TCheadword{mɛnra} (comp. of \TClink[1]{mɛn}, \TClink[1]{ra}, see \TClink[1]{mɛn})

\TCheadword[1]{mɛntɛ} \textit{cf}: \TClink[2]{thon}. \textit{n} outer sheath of bamboo used to make baskets (\citealt{Pichl1967}). 

\TCheadword[2]{mɛntɛ} \textit{n} arrow. \textit{Mɛntɛ so lɛ kɔ sonthul.} The arrow of the bow is sharp (\citealt{Pichl1967}). 

\TCheadword{mɛnth} \textit{n} broomstick. \textit{Ya koŋ kɛthi mɛnthɛ lɔi gbeŋgbeŋdɛ ahɔl.} I have broken the broom stick in the mouth of the ant's hole.

\TCheadword{mɛntiŋ} (comp. of \TClink[1]{mɛn}, \TClink[1]{tin}, see \TClink[1]{mɛn}) 

\TCheadword{mɛŋhiɔlniwaŋ} (comp. of \TClink[1]{mɛn}, \TClink{hiɔl}, \TClink[1]{ni}, \TClink[2]{waŋ}, see \TClink[1]{mɛn}) 

\TCheadword[1]{mɛŋk} \textit{cf}: \TClink{bonk}, \TClink{lɔkɔ}, \TClink[2]{nɛn}, \TClink[1]{tɛm}, \TClink{yiars}. \textit{n} \textbf{1)} time. \textbf{2)} year. \textit{Wɛl atipɛ tɔn nɛndɛ ŋɔ Apothoɛ ŋa wɔ 2013, te mɛŋko ki amu tɔndai.} Well, I started singing in the year that white people call 2013, up to this year I'm still singing. \textbf{3)} day.

\TCsubword{mɛŋkoki} (unspec.) \textit{temp} this time.

\TCheadword[2]{mɛŋk} \textit{Temp.} o'clock. \textit{Pal thipaŋ dɛ, mɛŋk hiɔl-lɛ yɛ pɔ koŋ hok saka jajɛl wɔɛ.} Four days later, this man left the ceremony for his mother-in-law.

\TCheadword{Mɛŋki} \textit{nam} Menki, male name given by a society.

\TCheadword{mɛŋkilɛn} \textit{v} seek refuge.

\TCsubword{mɛŋklɛni} (der.) \textit{v} seek refuge.

\TCheadword{mɛŋklɛni} (der. of \TClink{mɛŋkilɛn}, \TClink[1]{-i}, see \TClink{mɛŋkilɛn}) 

\TCheadword{mɛŋkoki} (unspec. of \TClink[1]{mɛŋk}, \TClink[1]{ko}, \TClink[1]{ki}, see \TClink[1]{mɛŋk}) 

\TCheadword{mɛsa} (Port \textit{mesa} ‘table') \textit{cf}: \TClink{tebul}. \textit{n} table. \textit{Bɛlsɛ ŋa lɔ baiɛ tokɛ, ŋa ke feɛ ŋɔ pɔ koŋ dikil mesaɛ atokɛ.} The rats are there on top of the bari, they saw how they gathered the money on the table.

\TCheadword{mɛsɛi} \textit{n} (-/ma) needle (\citealt{Sumner1921}); \textit{mɛsɛy} (hɔ̃/tha) needle (\citealt{Pichl1967}).

\TCheadword{mɛt} \textit{n} (hɔ̃/-, i) rocky area, cliffs or rocks at a distance from the shore where many crabs and shells are to be found and where it is difficult to walk barefoot (\citealt{Pichl1967}).

\TCheadword{Mgbaŋguma} \textit{nam} Mgbanguma, name given to a place. \textit{Wɔn mgbaŋgmako lɔ pɔ gbem wɔɛ.} He was born in Mgbanguma.

\TCheadword[1]{mi} \textit{cf}: \TClink{ya}. \textit{pers} \textbf{1)} me. \textbf{2)} my, [wàntsə́mí] my sister (B dialect). \textit{Ba mi koŋ kɔn bias ai nante.} My master went on a journey today (\citealt{Pichl1967}). comp. \TClink{bami} (see \TClink[1]{ba}), \TClink{kiminmi} (see \TClink{kii}) 

\TCheadword[2]{mi} \textit{nam} \textbf{1)} mother. \textit{Mii baa wɔ ka che ha mpanth ma yenchek.} Mother's father used to do fishing work. \textit{Yɛ mpanth ma mi wɔ ni ha?} What work was your mother doing? \textbf{2)} title of respect for a woman, may be translated as mother, ma, madam, ma'am, but also used in Sierra Leone English. \textit{Mi, pɔ mi ka yen tontondɛ.} Ma, they give me a little something. \textit{Mi ŋa a le yiɛ nɔmaɛ ki mi.} Mi, let me first ask this woman. \textit{Awokɔ gbo ko mɔ ko yai hun ko Mi Adama.} After leaving you, I will go to Mi Adama.

\TCsubword{yemi} (comp.) \textbf{1)} \textit{n} mother. \textbf{2)} \textit{nam} lady. \textit{Nɔmaa chaɛ a: Yemi, ni ntɛniɛ mini o-o-o.} The woman sang: My lady, remember me. \textit{Amaaɛ ŋae yom: Yeee mi-i-i.} The others: My lady.

\TCheadword{miks} \textit{adj} mixed.

\TCheadword{miliŋ} \textit{n} (lɔ/-) tongue, \textit{limilïng/ limilïngdi-m dɛ} tongue/ my tongue (\citealt{Pichl1967}). \textit{Hã milïngdi gbe̹r.} They have many tongues, i.e. they are not reliable (\citealt{Pichl1967}).

\TCsubword{miliŋdibil} (comp.) \textit{n} (kɔ/-) plant species, prickly hairy climber with solitary yellow, chocolate-centered flowers (Hibiscus surratensis) (\citealt{Pichl1967}). 

\TCsubword{miliŋdithumɔɛ} (comp.) \textit{n} (kɔ/-) plant species, erect, smooth plant with a few pale-yellow flowers (Crotalaria glauca) (\citealt{Pichl1967}).

\TCheadword{miliŋdibil} (comp. of \TClink{miliŋ}, \TClink[1]{li-}, \TClink[1]{bil}, see \TClink{miliŋ}) 

\TCheadword{miliŋdithumɔɛ} (comp. of \TClink{miliŋ}, \TClink{thumɔɛ}, see \TClink{miliŋ}) 

\TCheadword[1]{min} \textit{v} swallow. comp. \TClink{kiminmi} (see \TClink{kii}) 

\TCsubword{min-gbɔl} (id.) \textit{v} die. \textit{Wɔ koŋ min-gbɔl.} He has swallowed his heart, i.e., He is dead (\citealt{Pichl1967}). 

\TCheadword[2]{min} \textit{n} nose, [míndɛ̀]/[mínthɛ̀] nose/ noses (K dialect). \textit{Nɔmɔk lɛ kɔ hok wɔn minɛ lɛ kɔ isay.} The mucus that comes from his nose is offensive (\citealt{Pichl1967}). 

\TCsubword{minɛ-hɔl} (comp.) \textit{n} entrance to the nostrils (\citealt{Pichl1967}).

\TCsubword{yaŋminɛ} (comp.) \textit{n} nostril (\citealt{Pichl1967}). 

\TCheadword[3]{min} \textit{cf}: \TClink{Mtoin}. \textit{n} spirit associated with a society who appears as a dancing masquerade, [míndɛ̀]/[mínsɛ̀] spirit/spirits (K dialect). \textit{Min dɛ hothkɔ wɔ.} The spirit took her unawares (\citealt{Pichl1967}). \textit{Cho koŋ kəthani, wɔ lɛ gboka-nɔ, chen bo chaŋ fay-hɔl ko yɛ theɛ min dɛ wɔ hɔ lɛ.} Cho is perplexed, he is a non-initiate, he cannot pass in front of the Poro bush when he hears the (Poro) spirit is talking (\citealt{Pichl1967}). comp. \TClink{baŋkmin} (see \TClink[2]{baŋk}), \TClink{bolmin} (see \TClink[1]{bol}), \TClink{leemin} (see \TClink[1]{lee}), \TClink{piamin} (see \TClink[1]{pia}), \TClink{poŋ … nin} (see \TClink[1]{poŋ}), \TClink{vebolmin} (see \TClink[1]{vee}) 

\TCsubword{min-pem} (comp.), (id.) \textit{n} ghost (\citealt{Pichl1967}).

\TCsubword{mindo} (der.) \textit{adj} holy. \textit{Mindo, mindo, mindo} Holy, holy, holy (hymn).

\TCheadword[4]{min} \textit{cf}: \TClink{kɛmɛkɛ}, \TClink[4]{ko}, \TClink{tɛnin} (der. of \TClink[2]{tɛni}, \TClink[2]{-n}). \textit{v} \textbf{1)} mean. \textit{La minɛ yɛpɔ lɔik wanda Bondoɛ…} It means when a girl is initiated into the Bondo Society… \textit{So la minɛ skul buli ŋɔɛ?} So it means Bondo is a whole school? \textbf{2)} think. \textit{A minɛ pɛl kɔ mɔ kɔ woɛ.} I thought it was a net that you would throw.

\TCheadword[5]{min} \textit{cf}: \TClink[1]{hun}, \TClink{muni}, \TClink{muŋk}. \textit{v} return. \textit{Yɛ ŋa ni joɛ, ŋa koni gbo jo,mɔi kɔ thɔk panthɛ gbi m'minɛ tha koŋ sɛmi.} As you are now eating, after eating, you wash all the dishes and return them. \textit{Yɛ mɔ ni hun minɛ puli vɛ, lɛ nke bo yabasɛ kɔ bɔ, ni mɔi bɛrɛ.} When you are coming back to mix it, if you see that the onion is not enough, you add (some).

\TCheadword[6]{min} \textit{n} left side.

\TCheadword{Min-Charaŋ} \textit{nam} Holy Spirit. \textit{Chaŋbo athɔni ka Min Charaŋ dɛ we...} Unless I cleanse myself with the Holy Spirit...

\TCheadword{min-gbɔl} (id. of \TClink[1]{min}, \TClink{gbɔl}, see \TClink[1]{min}) 

\TCheadword{min-pem} (id. of, comp. of \TClink[3]{min}, \TClink[1]{pem}, see \TClink[3]{min}) 

\TCheadword{mina} \textit{cf}: \TClink[3]{pɛ}. \textit{adv} \textbf{1)} again. \textit{Be, che ŋɔ huɛ, ka mina muni ka 1980.} No, that was not the time he died, he returned here again in 1980. \textit{Po mɔɛ bɛ wɔ gbo yema jo, mɔi minɛ ko wok ŋa wɔn joɛ.} If your husband also said he wants to eat, you go and take the rice out again. \textit{Ye koŋ vɛ m'minɛ dikil panthɛ gbɛlɛ nkɔŋtha thɔk, m'minɛ tha thɔŋgul,yɛ pɔ ŋaɛ.} When he is finished eating, you then gather all the pans, wash them, you keep them again, that is how they do [it]. \textbf{2)} also.

\TCheadword{mindo} (der. of \TClink[3]{min}) 

\TCheadword{minɛ-hɔl} (comp. of \TClink[2]{min}, \TClink[1]{ahɔl}, see \TClink[2]{min}) 

\TCheadword{mini} \textit{v} go back.

\TCheadword{minnɔ} (unspec. of \TClink{nɔ}) 

\TCheadword{minth} \textit{v} bear, be patient (\citealt{Pichl1967}). id. \TClink{mintha-gbɔl} (see \TClink{gbɔl}) 

\TCheadword{mintha} \textit{n} fear.

\TCheadword{mintha-gbɔl} (id. of \TClink{minth}, \TClink{gbɔl}, see \TClink{gbɔl}) 

\TCheadword{mirɛ} \textit{v} pay close attention; watch intently. \textit{Pui-nɔ lɛ chala tho l'ay wɔ mïrə chal lɛ.} The hunter sits in the bush and watches the deer (\citealt{Pichl1967}).

\TCsubword{mirmir} (der.) \textit{v} pay close attention, watch intently. \textit{Bɛl Maaɛ wɔe tipɛ mir-mir, wɔ mukumuku ton, ton, tokɛ ko.} Rat Wife began to watch intently, she crept little by little from above.

\TCheadword{mishɔnariɛ} (Eng \textit{missionary}) \textit{n} missionary. \textit{Wɔi pɛ toŋgi ŋa mishɔnariɛ wɔ ka che ŋa, Mista Wɔlta Hanson.} To show us again about the missionary that was here, Mr. Walter Hanson. \textit{Mishɔnari ka che ka, Shenge ka iko wɔ theɛ, nka shi wɔ?} There used to be a missionary here, in Shenge here, we have heard about him, did you know him?

\TCheadword{Mista} (Eng \textit{mister}) \textit{nam} Mister. \textit{Ya Mista Alusain.} I am Mr. Alusine. \textit{Mista, laŋgba landɛ koŋ pa hu, wɔi hun wɔ ŋai hun hɔm lɛ ŋa ma blem wanthɛm dɛ vɛo.} Mister, the person is dead, he came and he told them that you should not blame that woman.

\TCheadword[1]{mith} \textit{v} \textbf{1)} hate. \textit{À míthəmɔ̀.} I hate you. \textit{À mɔ̀ míth.} I will hate you. \textbf{2)} dislike.

\TCsubword[2]{mith} (der.) \textit{n} \textbf{1)} hatred. \textit{Mith lɛ ko che gbe we.} Hatred is plentiful now. \textbf{2)} enemy. \textit{Jizɔs wu aŋai wɔ mithɛ thiyeŋ.} Jesus died among his enemies.

\TCheadword[2]{mith} (der. of \TClink[1]{mith}) 

\TCheadword{mithil} \textit{v} \textbf{1)} glow. \textbf{2)} shine dimly.

\TCheadword{mm} \textit{disco} mm.

\TCheadword{moɛkɛ} \textit{prep} until.

\TCheadword[1]{mɔ} \textit{cf}: \TClink{kunɔlɔ} (comp. of \TClink{kun}, \TClink[7]{lɔ}). \textit{n} breast. \textit{Mɔ wɔ bala-bala ni, wɔn bɛ wɔ mɔ balani, mɔ wɔ kis-kis yɛŋ bɛ, wɔi po ha yɛthi mmɔ ma mɔɛ.} You hug him, he hugs you, you kiss him all over, then he begins to hold your breast. comp. \TClink{bolmɔ} (see \TClink[1]{bol}) 

\TCsubword{mɔmana} (comp.) \textit{n} cow milk.

\TCheadword[2]{mɔ} \textit{cf}: \TClink{n}. \textit{pers} \textbf{1)} you, your. \textit{Ba mɔ nɛndɛ ŋɔ wuwɛ nshiŋɔ pɛ?} Do you know now what year your father died? \textit{Ima koi mɛŋk mɔ livil.} I do not want to take much of your time. \textit{So ilel mɔa?} What is your name? \textit{Yi koni shi temdɛ ŋɔ pɔ gbem mɔ, ko lɔ pɔ gbem mɔ.} We already know when and where you were born. \textit{Kɛ ayema mɔ yi yi bul.} But I just want to ask you a question. \textit{So labi ichɔŋ len ŋa hin chemɔ vel.} So that is why we like to call you. \textit{Mɔ wɔ ni ŋa yen-o-yen.} You have to give her everything. \textit{Yɛ mɔ hɔ Mbolomdɛ motoɛ kunɛ, nɔnɔ wɔ thimni wɔi yi mɔ Bolomnɔ?} When you speak Sherbro in a vehicle, everybody will turn and ask, Are you Sherbro? \textit{Kɛ mɔm n shini ŋɔth gbi?} But you do not know how to fish at all? \textbf{2)} who.

\TCheadword[3]{mɔ} \textit{v} sink.

\TCheadword[1]{mɔɛ} \textit{n} \textbf{1)} [mɔ̀ìyɛ̀] palm wine (K dialect); \textit{mmõɛ} (ma) rum, win, alcoholic drink (\citealt{Pichl1967}). \textit{Ŋa kul mɔi ma sɔisɔi gbi ŋa koi piŋiɛni.} They drink tasty drinks and they turn against us. \textbf{2)} rum. comp. \TClink{mɔɛŋkɛn} (see \TClink[3]{kɛn}) 

\TCsubword{mɔɛŋkalom} (comp.) \textit{n} palm wine.

\TCheadword[2]{mɔɛ} (der. of \TClink{muɛ})

\TCheadword[3]{mɔɛ} (id. of \TClink[5]{mɔɛ}) 

\TCheadword[4]{mɔɛ} (der. of \TClink[2]{mɔɛ} (der. of \TClink{muɛ}), see \TClink{muɛ}) 

\TCheadword[5]{mɔɛ} \textit{n} [mɔ̀ì] afternoon (B dialect). 

\TCsubword[3]{mɔɛ} (id.) \textit{cf}: \TClink{mɔlɔ}, \TClink{mpikɛ}. \textit{disco} afternoon greeting, [mɔ̀í] good afternoon (B dialect). \textit{Lɛ nwɔ gbo, ŋa mɔi, ŋan ŋa wɔ “bua.”} If you say to them, \textit{mɔi} (‘good afternoon' in Sherbro), they will say, \textit{bua} (‘greetings' in Mende).

\TCheadword{mɔɛktu} \textit{n} \textbf{1)} perplexity. \textbf{2)} dilemma.

\TCheadword{mɔɛŋkalom} (comp. of \TClink[1]{mɔɛ}, \TClink{kalom}, see \TClink[1]{mɔɛ}) 

\TCheadword{mɔɛŋkɛn} (comp. of \TClink[1]{mɔɛ}, \TClink[3]{kɛn}, see \TClink[3]{kɛn}) 

\TCheadword{Mɔfɔs} \textit{cf}: \TClink{Nfɔs}. \textit{nam} Mofos, name given to a place – a town in Kagboro Chiefdom, upriver from the coast above Shenge. \textit{Mɔfɔs Sɛkshɔn.} Mofos Section.

\TCheadword{Mɔkɛlɛ} \textit{n} Toma Society spirit who appears as a dancing masquerade (\citealt{Pichl1967}).

\TCheadword[1]{mɔl} (der. of \TClink[2]{mɔl}) 

\TCheadword[2]{mɔl} \textit{n} sorrow.

\TCsubword[1]{mɔl} (der.) \textit{adj} sad. \textit{Ihɔlɔŋ hɔ imɔl.} Life is sad (\citealt{Pichl1967}).

\TCheadword{mɔlen} \textit{n} fishing rope with hooks tied along its length (\citealt{Sumner1921}). \textit{Yi pɛ hõth ka mɔlen hialsi ay.} Again we are fishing in rivers with a rope with hooks tied along its whole length (\citealt{Pichl1967}). 

\TCheadword{mɔlɔ} \textit{cf}: \TClink[3]{mɔɛ} (id. of \TClink[5]{mɔɛ}), \TClink{mpikɛ}. \textit{disco} daytime greeting.

\TCheadword{mɔmana} (comp. of \TClink[1]{mɔ}, \TClink{maa}, \TClink[1]{na}, see \TClink[1]{mɔ}) 

\TCheadword{mɔmi} (Eng \textit{mommy}) \textit{nam} Mommy. \textit{Ya wɔ hin, ya wɔ pabondɛ Mɔmi Prat wɔɛ, a cheŋ kɔo.} I said to him, I said that if it is Mummy Pratt, I'm not going.

\TCheadword[1]{mɔn} \textit{cf}: \TClink{lɔlma} (comp. of \TClink{lɔl}, \TClink[4]{ma}), \TClink{twɛ}. \textit{v} have sex.

\TCheadword[2]{mɔn} \textit{cf}: \TClink{sin}. \textit{n} poverty. \textit{Jizɔs, ya mɔnɛ ni mbali mi.} Jesus, I am poor so make me rich.

\TCheadword{Mɔnde} (Eng \textit{Monday}) \textit{nam} Monday. \textit{Nante mɔnde ndɔi waŋ ni mɛn dɛ, paŋdɛ ŋɔ pɔ wɔɛ Fɛbruari.} Today is Monday the 15\textsuperscript{th} of February.

\TCheadword{mɔŋa} \textit{v} [mɔ́ŋá] be (B dialect). \textit{Mɔ́ŋá ŋálá wàì!} Be patient!

\TCheadword{mɔs} (Eng \textit{must}) \textit{cf}: \TClink[2]{bi}, \TClink[2]{ha}, \TClink[3]{lɔi}, \TClink[1]{ma}, \TClink[2]{ŋa}. \textit{Aux} must. \textit{Achɔn ma len eh, bikɔs amɔs wɔni ɛ nwɔkɛ ma pɔ yemaɛ mavɛ Mbɛkɛ vɛ.} I like it, because I must say the language they want, it is that Krio.

\TCheadword{mɔt} \textit{n} tight short trousers (\citealt{Pichl1967}). 

\TCheadword{mɔthiani} \textit{v} manage.

\TCheadword{mɔtɔ} (Eng \textit{motor}-) \textit{n} automobile; car; vehicle. \textit{Kɛ yɛ motoɛ chelɔ bɔ kɔɛ, ŋa koŋ wɔ ŋa cheŋ bɔ kɔ.} But since vehicles cannot go there, they said they would not be able to come.

\TCheadword{Mɔy} \textit{n} Muslim.

\TCsubword{Mɔynɔ} (comp.) \textit{n} Muslim.

\TCheadword{Mɔya} \textit{nam} Moya, name given to a place. \textit{Pɔ gbem mi Nkainsumana ko, Mɔya Sɛkshɔn.} I was born in Mokainsumana, Moya Section.

\TCheadword{Mɔynɔ} (comp. of \TClink{Mɔy}, \TClink{nɔ}, see \TClink{Mɔy}) 

\TCheadword{mpa} \textit{disco} emphatic ‘do.'

\TCheadword{mpanth-o-mpanth} (der. of \TClink[1]{panth}, \TClink{-o-}, see \TClink[1]{panth})

\TCheadword{Mpelɛ} \textit{nam} Mpele, name given to a place.

\TCheadword{mpikɛ} \textit{cf}: \TClink[3]{mɔɛ} (id. of \TClink[5]{mɔɛ}), \TClink{mɔlɔ}. \textit{disco} evening greeting. \textit{\`{m}pìkɛ́ sàkàò ŋɔ̌mpìù} evening or night greeting, replies (B dialect). 

\TCheadword{Mpithɛ} \textit{nam} Mpithe, name given to a place.

\TCheadword{Mpondo} \textit{nam} Mpondo, male name given to a person. 

\TCheadword{Mtoin} \textit{cf}: \TClink[3]{min}. \textit{nam} Society spirit who appears as a dancing masquerade (\citealt{Pichl1967}).

\TCheadword[1]{mu} \textit{cf}: \TClink[1]{bɛ}, \TClink{fili}, \TClink{huɛŋ}, \TClink{ivin}, \TClink[1]{ni}, \TClink[3]{pɛ}, \TClink{stil}. \textbf{1)} \textit{temp} still. \textit{Sɔk lɛ wɔ mu hel.} The fowl is still boiling (\citealt{Pichl1967}). \textit{Sistha Kɔba lan wɔ lɔ mu haŋ ma nantɛ?} That Sister Koba, is she still there up to this day? \textbf{2)} \textit{temp} ever. \textit{Wan day asɔthɔni mu prɔblɛm ya lɔ gbemiɛ.} Not once have I ever had a problem delivering. \textbf{3)} \textit{temp} yet. \textit{Lɛ ŋa yema bo won lɛŋ ko ŋanɛ ha hunɔn muɛ, ko nrekiaɛ ŋa pɔ gbemɛn muɛ.} What greeting would you want to send to those that have not come yet, the grandchildren, those that have not been born yet. \textit{Nɛn lan agbemenimu.} That year I had not started having children yet. \textbf{4)} \textit{temp} presently. \textit{Aa miyo amu ŋa mpanth ma chichindɛ.} I am presently doing teaching work. \textbf{5)} \textit{adv} even. \textit{Kɛ nyanmu joɛ.} But you have not even cooked the rice. \textit{Yi yema ni nwɔmyi lanɛ la ŋa yɛ ntipɛ ni mu kɔ skul, siŋ thɛ gbi tha ŋsiŋdɛ.} Please tell us about your early life, before you even went to school, the games you played.

\TCheadword[2]{mu} \textit{cf}: \TClink{kɛth}. \textit{v} cut down; fell.

\TCheadword{muɛ} \textbf{1)} \textit{v} arrive; reach a destination. \textbf{2)} \textit{v} come. \textit{Chaŋbo paŋdɛ ŋɔ mɔi bo pɔ hiŋ ka ja tuthɛ, than bo tha ika che kunɛ.} Except when evening came, we would be given rice pounding work, that was the work we were engaged in. \textit{Yɛ nkoyɛ mɛŋk mɔɛ nwun, nwun theli yithɛ tha hun mɔ yiɛ, Abatokɛ che mamɔ.} That you have taken your time to come and respond to the questions I have asked you, may God be with you. \textbf{3)} \textit{v} come to an end. \textbf{4)} \textit{v} reach. \textit{Kaiŋ Taso ka mɔɛ tir bul, lɔ ka ke waaŋmaa kɛlɛŋ-kɛlɛŋ.} Kain Tasso reached a village where he saw a fine young woman. \textit{Wɔ gbo chɔ ha muɛ ko pɛllɛ.} He was just fighting to reach the chain. \textbf{5)} \textit{temp} yet.

\TCsubword[2]{mɔɛ} (der.) \textbf{1)} \textit{v} be ripe, can be used not of just fruit but also of people, if a child is not ‘ripe,' the child is not fully grown, not able to carry a load (K dialect). \textit{Mà kóŋ hɔ̀ mɔ̀ɛ̀./Mà kóŋ mɔ̀ɛ̀.} (The fruit) is ripe. \textbf{2)} \textit{v} be mature. \textbf{3)} \textit{n} being old. \textit{Dɛ mbɔnthɔ bo nɔ wɔ chaŋ mɔ mɔ…} If you meet someone that is older than you… 

\TCsubword[4]{mɔɛ} (der.), (der. of \TClink[2]{mɔɛ}) \textit{Aux} incipient modal. \textit{Wɛl tɛm dɛ vɛ yɛ pɔ kɔ hun lɛli labo kɔ ko mɔi futhɛ.} At that time they will come to see if it has formed roots. 

\TCsubword[1]{muɛkɛ} (der.) \textbf{1)} \textit{prep} ordinal particle. \textit{Mɔikɛ tindɛ, mii gbemɛni komɔ pokan, i gbo ama.} The second thing is mother did not have male children, we are just female. \textit{Nduɛ muɛkɛ mɛŋtiŋndɛ, ni nɔmaa bɛn dɛ, wɔe wu jajɛl Kaiŋ Tasoɛ.} On the seventh day, the old woman died, Kain Tasso's mother-in-law. \textbf{2)} \textit{prep} almost; ‘going on.' \textit{Yɛ yɔk mi Kiamp koɛ, nen bul mɔikɛ tiŋ, wɔ mi bɛ skullai.} When she took me to Freetown, one year going on two, she sent me to school. \textit{Pɔ telɛ wik bul mɔike tindɛ pɔi kutha.} They will wait one or two weeks to plow the land. \textbf{3)} \textit{n} time.

\TCsubword[2]{muɛkɛ} (der.) \textit{indfpro} pronoun for clauses involving numbers. \textit{Pɔ gbem mi paŋdɛ ŋɔ pɔ wɔ Sɛptɛmbaɛ, paŋ mɔikɛ mɛnyɔllɛ.} I was born in the month that they call September, the ninth month. \textit{Wɛl, ara ŋaa kandaɛ bul thamura mɔikɛ yɔllɛ.} Well, three are in school and one dropped out which makes it four. \textit{Beo mɔikɛ mɛmbul.} No, the sixth one.

\TCheadword[1]{muɛkɛ} (der. of \TClink{muɛ}, \TClink{-k}, see \TClink{muɛ}) 

\TCheadword[2]{muɛkɛ} (der. of \TClink{muɛ}, \TClink{-k}, see \TClink{muɛ}) 

\TCheadword{Muhamɛd} \textit{nam} Mohammed, male name given to a person. \textit{Abi Suleman Bɛndu, Usman Bɛndu, Abas Bɛndu ni Muhamɛd Bɛndu.} I have Sulaiman Bendu, Usman Bendu, Abass Bendu, and Mohamed Bendu.

\TCheadword[1]{muk} \textit{n} forehead (\citealt{Pichl1967}).

\TCsubword{mukɔhɔl} (comp.) \textit{n} forehead (K dialect). (\textit{mukɔhɔlɛ} the forehead, example from AY, but would anticipate geminate l in this position.)

\TCheadword[2]{muk} \textit{adv} completely, entirely. \textit{Koŋ hã jo muk.} He has eaten (his money), i.e., he has wasted all his money (\citealt{Pichl1967}). 

\TCheadword{mukɔhɔl} (comp. of \TClink[1]{muk}, \TClink[1]{ahɔl}, see \TClink[1]{muk}) 

\TCheadword{muku} \textit{cf}: \TClink{thal}. \textit{v} \textbf{1)} creep. \textbf{2)} crawl.

\TCsubword{mukumuku} (der.) \textit{v} creep. \textit{Bɛl Maaɛ wɔe tipɛ mir-mir, wɔ mukumuku ton, ton, tokɛ ko.} Rat Wife began to watch intently, she crept little by little from above.

\TCheadword{mukumuku} (der. of \TClink{muku}) 

\TCheadword{mulat} (Port, Eng \textit{mulatto}) \textit{n} mulatto.

\TCheadword{mumu} \textit{cf}: \TClink[3]{bobo}. \textit{n} deaf mute.

\TCheadword{muni} \textit{cf}: \TClink[1]{hun}, \TClink[5]{min}, \TClink{muŋk}. \textit{v} return. \textit{Wɔi pɛ muni wɔi hun gbemɔ wantemdɛ ka ŋɔ ba mi ka wuwɛ.} She came back here to deliver my sister when my father died. \textit{Muni 1980, wɔi huŋ che ka haŋ, wɔi huɛ.} He returned in 1980, came back and stayed for a very long time, and then died.

\TCsubword{munini} (der.) \textit{v} return. \textit{Ya kɔnth bo vel bomdɛ bul yai munini.} I just caught a big grouper and returned.

\TCsubword{muniya} (der.) \textit{temp} on return.

\TCheadword{munini} (der. of \TClink{muni}, \TClink{-ni}, see \TClink{muni}) 

\TCheadword{muniya} (der. of \TClink{muni}) 

\TCheadword{muŋk} \textit{cf}: \TClink[1]{hun}, \TClink[5]{min}, \TClink{muni}. \textit{v} refund; return. comp., id. \TClink{muŋkofe} (see \TClink{fe}), \TClink{muŋkokol} (see \TClink{kol})

\TCsubword{muŋkma} (id.) [mùŋkmá] \textit{n} snake species, small and deadly like the mamba found in the Shenge area but scarce; people die quickly, that's why they say “Take it back!” (lit. return it) (K dialect). 

\TCheadword{muŋkma} (id. of \TClink{muŋk}, \TClink[3]{ma}, see \TClink{muŋk}) 

\TCheadword{muŋkofe} (comp. of, id. of \TClink{muŋk}, \TClink{fe}, see \TClink{fe}) 

\TCheadword{muŋkokol} (id. of, comp. of \TClink{muŋk}, \TClink{kol}, see \TClink{kol}) 

\TCheadword{Musa} \textit{nam} Musa, male name given to a person. \textit{Ba mi ilel wɔ ŋɔ ka cheɛ Musa Sise.} My father, his name was Musa Sesay.

\TCheadword{Muslim} \textit{adj} Muslim. \textit{Patikulali hi Amɔyaɛ ko a wokɛ lɔ pridɔminantli Muslim.} Particularly we Muslims, where I came from is predominately Muslim.

\TCheadword{mutmut} \textit{cf}: \TClink[2]{hul}, \TClink{kuum}. \textit{n} insect species, small mosquito species (K dialect). 

\TCheadword{Muuli} \textit{nam} Muuli, name given to a place. 

\TCheadword[1]{muyu} \textit{v} \textbf{1)} [múyù] be patient (K dialect). \textit{Lɛ m muyu gbo ni mbɛ komɔ kaŋdai, ni wɔnbɛ bɛlɔ bolwɔi, mɛkindɛ ŋɔ vɛ.} If you are patient and put your children in school, and he pays attention there, that is the end. \textit{Hɔɛ-o-hɔɛ à tɔnk PY, hàlìwɔ́ wɔ́ múyù / bɛ̀má mì nì.} Every day I praise PY, because he is patient / helps me. \textbf{2)} endure. \textit{Ya bɔ muyu hã ndoɛ nra.} I can endure doing it for three days (\citealt{Pichl1967}). 

\TCheadword[2]{muyu} \textit{cf}: \TClink[1]{ŋala}. \textit{n} patience. \textit{Wɔ hĩ telɛ ka mũyu.} He is patiently waiting for us (\citealt{Pichl1967}). \textit{Mbi muyu, mma pakni!} have patience, don't tremble! (\citealt{Pichl1967}). 

\TCheadword[3]{muyu} (der. of \TClink[2]{muyu}) \textit{adj} patient.


\end{letter}
\begin{letter}{N}

\TCheadword{n} \textit{cf}: \TClink[2]{mɔ}. \textit{pers} you (subject). \textit{Tɛm ndɔ ŋɔ ntipɛ gbemia?} When did you start delivering? \textit{Nsiɛ tɛm pɛm doki yɛi chaŋ-chaŋdɛ.} You know during the war how we were moving around. \textit{Mɔm, la nka cheni ŋaa?} You, what have you been doing? \textit{Mi mgbisiŋɛ?} Mummy, are you married? \textit{Bikɔs lɛ nkɔgbo Mɛndɛ ko…} Because if you go to the Mendes…

\TCheadword[1]{-n} \textit{cf}: \TClink{-ni}, \TClink[1]{bɛ}, \TClink{delma}. \textit{pro} \textit{sfx} \textbf{1)} emphatic pronoun suffix. \textit{Mɔŋ a chichi.} You are jealous. \textit{Yin gbi hwɛlo ay liting ay.} We all in the world are in two conditions (\citealt{Pichl1967}). \textit{La hini ha, ŋa sɔthɔ hini-gbɔl?} What are we to do, to have peace of mind? \textit{Ha wɔn gbi, nɔ gbi cheni.} Above him there is no other. \textbf{2)} self. \textit{Be o, wɔn bɛ pɛ, yɛ pɔ hokkɔ wɔ ifɔndaiɛ.} No, he himself, when he was taken out from initiation. \textit{Wɔn ilel wɔa?} What is her own name? \textit{Wɛl, wɔn bɛpɛ ka cheɛ mared uman, wɔi pɛ cheɛ sokonɔ Bondo.} Well, she herself was a housewife, and she was also the head of the Bondo Society. der. \TClink[2]{mɛkin} (see \TClink[1]{mɛk}), \TClink[3]{mɛkin} (see \TClink[1]{mɛk}), \TClink[4]{mɛkin} (see \TClink[1]{mɛk}), \TClink{theɛn} (see \TClink{the}) 

\TCheadword[2]{-n} \textit{cf}: \TClink{-ni}. \textit{v} \textit{sfx} verbal suffix. \textit{Achɔŋɔ Abatokɛ sɛkɛ ya chalan dɛ ka.} I give thanks to God to be sitting here. der. \TClink{tɛnin} (see \TClink[1]{tɛn}), \TClink{thɔn} (see \TClink[1]{thɔk}) 

\TCheadword{n-} \textit{cf}: \TClink{ma-}. \textit{NCM} noun class marker (ma). \textit{Mpanth ma apuma maɛ, akɔ pɔɛ, atu, ko gbi lɔ yema mi bo womdɛ.} The jobs of the girl children, I go to fetch water, I pound, wherever she wants to send me. \textit{Ya aka tallɛ aka ni ŋa mpanth ma sobaɛ, mpanthɛ ma pɔ chemi kaɛ ma aŋa.} When I was young, I did not do serious work, the work that was given to me is what I did. \textit{Yaŋ, a chɔŋ nwɔk mamdɛ len, Mbolomdɛ.} Me, I love it (the church service) in my language, Bolom. \textit{Thetha mi ka che ŋa mpanth ma landɛ pɛŋ bifo wɔ mmu hu.} My grandmother used to do the work before she died. \textit{Huŋ kaŋ Mbolomdɛ.} He came to learn Bolom. \textit{So lan la ako ha nkuath ha ŋɔth.} So that is how I became afraid of fishing. comp. \TClink{tismabue} (see \TClink{boe}) 

\TCheadword[1]{na} \textit{n} \textbf{1)} cow, [nàà]/[nààsɛ̀] cow/cows (B dialect); \textit{ná} (/si) cow (\citealt{Sumner1921}). \textit{Naa lɛ wɔ te̹m bo̹m.} The cow has a big stomach (\citealt{Pichl1967}). \textbf{2)} cattle. comp. \TClink{mɔmana} (see \TClink[1]{mɔ})

\TCsubword{nama} (comp.) \textit{n} cow. comp. \TClink{wɔkmmɔ} (see \TClink[1]{wɔk})

\TCsubword{napokan} (comp.) \textit{n} bull.

\TCheadword[2]{na} \textit{cf}: \TClink[5]{ka}, \TClink[1]{pa}. \textit{prt} particle marking recent past tense. \textit{ká, ná} after some years, long past, \textit{ná} is much more recent than \textit{ká}. \textit{A má ná bɛ́ kòŋ kɔ̀nì.} I would have gone. \textit{Hã kɔ chæ thɔk lɛ kɔ bikɛɛ lɛ duki chɔl na næ lɛ 'hɔl lɛ.} Go and lift the tree that the storm felled on the road last night (\citealt{Pichl1967}). der. \TClink{sɔna} (see \TClink[2]{isɔ}) 

\TCheadword[3]{na} \textit{post} with. \textit{I ko vei ina pomdɛ o, iko bɛ chaŋ nɛnthi waŋdɛ.} We have stayed together me and my husband, now more than ten years.

\TCheadword[4]{na} \textit{prt} Neg.

\TCheadword{naa} \textit{n} spider. \textit{Ba Na lee mathini.} Mr. Spider stayed behind to hide himself (\citealt{Pichl1967}).

\TCheadword[1]{nai} \textit{n} \textbf{1)} road. \textbf{2)} way. comp., id. \TClink{kɔnaibol} (see \TClink[2]{kɔ})

\TCsubword{naiahol} (comp.) \textit{n} farmhouse road.

\TCsubword{naibol} (comp., id.) [náìbòl] \textit{cf}: \TClink{fol}, \TClink{kɔnaibol} (id. of, comp. of \TClink[2]{kɔ}, \TClink[1]{nai}, \TClink[1]{bol}). \textit{v} defecate (lit. go to the head of the road). \textit{Bàŋkáɛ́ wɔ̀ kɔ́ náɛ̀bòl áyéná bùl.} The \textit{baŋka} civet always defecates in the same place.

\TCsubword{naibom} (comp.) \textit{n} street.

\TCsubword{naibɔl} (comp.) \textit{Loc} along the way.

\TCsubword{naithisɛŋki} (comp.) \textit{n} crossroads.

\TCheadword[2]{nai} \textit{n} children. \textit{Nɔ mɔ gbem nai, gbem ha mɔ.} If your relatives have children, give birth to your own (proverb) (\citealt{TISLL1979}).

\TCheadword{naiahol} (comp. of \TClink[1]{nai}, \TClink{ahel}, see \TClink[1]{nai}) 

\TCheadword{naibol} (comp. of, id. of \TClink[1]{nai}, \TClink[1]{bol}, see \TClink[1]{nai}) 

\TCheadword{naibom} (comp. of \TClink[1]{nai}, \TClink{bom}, see \TClink[1]{nai}) 

\TCheadword{naibɔl} (comp. of \TClink[1]{nai}, \TClink[2]{ibɔl}, see \TClink[1]{nai}) 

\TCheadword{naithisɛŋki} (comp. of \TClink[1]{nai}) 

\TCheadword[1]{nak} \textit{cf}: \TClink{rɔmp}. \textit{n} \textbf{1)} illness. \textbf{2)} sickness. \textit{A bi nak.} I have sickness / I am sick (\citealt{Pichl1967}). comp. \TClink{nɔnaka} (see \TClink{nɔ})

\TCsubword[2]{nak} (comp.) \textit{v} be ill. \textit{Ya che ko taallɛ, acheni ve, ya naka naka, pɔ mi yɔk hɔspithai ni asoŋ.} When I was young, I was not well, they took me to the hospital to get well. der. \TClink{nɛkɛli} (see \TClink[1]{nak}), \TClink{nɛki} (see \TClink[1]{nak})

\TCsubword{nakibɛl} (comp.) \textit{n} sleeping sickness (Trypanosoma brucei) (\citealt{Pichl1967}).

\TCsubword{naknchɛs} (comp.) \textit{n} leprosy. \textit{Nɔ-pokan bən do bi nak-nchɛs.} This old man has leprosy (\citealt{Pichl1967}). 

\TCsubword{nɛkɛli} (comp.), (der. of \TClink[2]{nak}) \textit{v} cause sickness.
\TCsubword{nɛki} (comp.), (der. of \TClink[2]{nak}) \textit{v} hurt; be painful. \textit{Wɔ nɛkiɛ lɛ wɔ kuyɛ yu ihuk lɛ.} He hurt himself when he took a fish from the hook (\citealt{Pichl1967}). \textit{Bɛŋ miɛ bó kɔ̀ nɛ̀kí.} My leg hurts. comp. \TClink{theyɛn-nɛki} (see \TClink[1]{nak})

\TCsubword{theyɛn-nɛki} (comp.), (comp. of \TClink{nɛki}) \textit{cf}: \TClink{thɛmni} (unspec. of \TClink{-ni}). \textit{v} hurt oneself. \textit{Sese theyɛn-nɛki, thɔ lɛ kəth wɔ yenwɛy.} Sese hurt himself, the adze badly cut him (\citealt{Pichl1967}).

\TCsubword{naka} (der.) [nàká] \textit{n} \textbf{1)} pain. \textit{Nak-naka bí bɛ̀ŋ nàká.} Her leg is hurting (her). \textbf{2)} sick. der. \TClink{naknaka} (see \TClink[1]{nak})

\TCsubword{naknaka} (der.), (der. of \TClink{naka}) \textit{n} pain. \textit{Nak-naka bí bɛ̀ŋ nàká.} Her leg is hurting (her). 

\TCheadword[2]{nak} (comp. of \TClink[1]{nak})

\TCheadword{naka} (der. of \TClink[1]{nak})

\TCheadword{nakibɛl} (comp. of \TClink[1]{nak}, \TClink[3]{bɛl} (der. of \TClink[2]{bɛl}), see \TClink[1]{nak}) 

\TCheadword{naknaka} (der. of \TClink{naka} (der. of \TClink[1]{nak}), see \TClink[1]{nak}) 

\TCheadword{naknchɛs} (comp. of \TClink[1]{nak}) 

\TCheadword{nal} \textit{cf}: \TClink{poth}. \textit{n} \textbf{1)} soil. \textit{Pɔ koŋ gbo raa pɔi piŋgi kaŋka inallɛ lɔ ŋa ni kɛlɛn.} After brushing, they have to turn over the soil so that it becomes clean. \textbf{2)} place.

\TCheadword{nama} (comp. of \TClink[1]{na}, \TClink{maa}, see \TClink[1]{na})

\TCheadword{nan} \textit{v} \textbf{1)} pull. \textit{A bi huk bul ŋɔ a dukiɛ yuɛ betɛ gbo koi gbo hukɛ, a wɔi nan.} I have a hook that I use, if the fish comes for the bait on the hook, I pull it up. \textbf{2)} draw. \textit{Ha ke bondɔ ko ni ha nan wɔmdɛ chiɛ ko.} Look for the wharf and pull the canoe on shore (\citealt{Pichl1967}). \textit{Inan gballɛ, ilɔ pɛŋgipɛŋgi, i kikkik.} We draw the line, we jump there (and) kick.

\TCsubword{nanihɔlɔŋ} (comp.) \textit{v} breathe. \textit{Wɔ mu nan nanihɔlɔŋ.} He is still breathing. 

\TCsubword{nani} (der.) \textit{v} pull, draw with force. \textit{Wɔn wɔ gbo nani, aha lɛ ha jɛthɛli ha ma ha mbank lɛ.} While he is pulling hard, the others should slacken their ropes (\citealt{Pichl1967}). 

\TCsubword{naŋkani} (der.) \textit{v} pull together.

\TCheadword{nani} (der. of \TClink{nan}, \TClink[1]{-i}, see \TClink{nan}) 

\TCheadword{nanihɔlɔŋ} (comp. of \TClink{nan}, \TClink[2]{hɔlɔŋ} (comp. of \TClink[2]{hɔl}), see \TClink{nan}) 

\TCheadword{Nanɔ} \textit{nam} July. \textit{Pàŋ Nanɔɛ ŋɔ pɔ vellɛ Julai Mpothoaiɛ.} \textit{Nanɔ} that they call July in English. 

\TCheadword{nante} \textit{cf}: \TClink{chencha}, \TClink{gbɛŋ}, \TClink{jɛk}. \textit{temp} today. \textit{Ya lɔ kɔ nantɛ.} I am going there today. der. \TClink{manante} (see \TClink[3]{ma}) 

\TCheadword{naŋkani} (der. of \TClink{nan}, \TClink[3]{ka}, \TClink{-ni}, see \TClink{nan}) 

\TCheadword{napokan} (comp. of \TClink[1]{na}, \TClink{pokan} (unspec. of \TClink[5]{po}), see \TClink[1]{na}) 

\TCheadword{nashɔn} (Eng \textit{nation}) \textit{n} nation. \textit{Nashɔndɛ gbi ŋɔ lɔ kɔ.} All the other nations go there (behave the same way).

\TCheadword{naw} (Eng \textit{now}) \textit{temp} now. \textit{Te naw laa mu kunɛ.} Up until now that is what I am in. \textit{Wɛl rait naw ilɔ amɛŋra, araɛ ŋa ka koŋ wu.} Well, right now we are eight, three have died. \textit{Rait naw mpanth ma lifamalifama.} Right now I am involved in farming work.

\TCsubword{rait naw} (comp.), (id.) (Eng \textit{right now}) \textit{temp} right now. \textit{Wɛl rait naw ilɔ amɛŋra,varaɛ ŋa ka koŋ wu.} Well, right now we are eight, three have died. \textit{Rait now isɔloki pɔ ko mi bɛ fon ŋa hanya tiŋ.} Right now this morning they have called me for two people. \textit{Rait naw mpanth ma lifamalifama.} Right now I am involved in farming work.

\TCheadword{nchembul} (comp. of \TClink[1]{che}, \TClink[2]{bul} (comp. of, der. of \TClink[3]{bul}), see \TClink[1]{che}) 

\TCheadword{Ndema} \textit{nam} Dema Chiefdom (pronounced \textit{Ndema} with a pre-nasalized stop but spelled locally without) (Nd dialect). \textit{Ndema ko lɔ pɔ gbem mi.} I was born in Dema (Chiefdom). 

\TCheadword[1]{ndɛ} (der. of \TClink[1]{lɛli} (comp. of \TClink[3]{lɛ}), see \TClink[3]{lɛ}) 

\TCheadword[2]{ndɛ} \textit{cf}: \TClink[2]{ɛ}. \textit{def} the. \textit{Bɔ̀mndɛ́ ɔ̀ gbɛ́gbɛ́yɛ̀? Gbɛ́gbɛ́yɛ̀ wɔ̀ pɛ́ŋhɛ̀.} Toad or frog? It's the frog who jumps. \textit{Hɛ̀ŋndɛ́ ŋɔ́ [hɔ̃] bɔ̀s.} The wind is cold. \textit{Bìmndɛ́ wɔ̀ chɔ́ má wɔ̀mdɛ̀.} The porpoise fought the boat.

\TCheadword{ndɛthmaboot} (comp. of \TClink{dɛth}, \TClink[3]{ma}, \TClink{bot}, see \TClink{dɛth}) 

\TCheadword{ndɔ} \textit{cf}: \TClink{handɔ}, \TClink[5]{hɔ}, \TClink[1]{la}, \TClink[5]{ŋa}. \textit{interrog} \textbf{1)} what. \textit{Shenge ka nwɔk ndɔ ma pɔ chan thelia?} In Shenge here, what language do they speak more? \textit{Nɛn ndɔ?} What year? \textit{Yamɔ wɔ tɔm ndɔ?} Your mother was what number (wife)? \textbf{2)} where. \textit{Wɔn gbemni ndɔ?} She was born where? \textit{Ndɔ mɔ ni yai?} Where are you now? \textit{Ndɔ mɔ mɛkɛnia?} Where would you stop? \textit{Ndɔ mmɛkɛnia?} Where did you stop? \textbf{3)} when. \textit{Ndɔ mɛkɛnia?} When did you stop? \textit{Mɛŋk handɔ ŋɔ mɔ mɛknia?} When would you stop? \textit{Tɛm ndɔ ŋɔ ntipɛ gbemia?} When did you start delivering?

\TCsubword{ndɔlɔ} (comp.) \textit{interrog} where.

\TCsubword[1]{ndɔndɔ} (der.) \textit{cf}: \TClink{nɔonɔ} (der. of \TClink{nɔ}, \TClink{-o-}), \TClink[4]{ŋa}. \textit{indfpro} \textbf{1)} anybody; whoever. \textit{Nɔ ndɔndɔ wɔ yema ŋa thelaɛ wɔla the, wɔlɔka gbi.} Whoever wants to hear it hears it, throughout the whole world. \textit{Nɔ ndɔndɔ wɔ yema ŋa the laɛ wɔ la bia the wɔlɔ ka.} Whoever wants to hear it gets to hear it, throughout the world. \textbf{2)} everyone. \textit{Lɛ nɔ shi la bo lɛ mɔ Bolomnɔ, nɔ ndɔndɔ wɔ mɔ ka limani.} If a person knows that you are Sherbro, everybody gives you respect. \textit{I koŋ gbo siŋ, nɔ ndɔ-ndɔ ko kɔni woŋgo wɔ ko.} When we have finished playing, everyone goes to his house.

\TCsubword[2]{ndɔndɔ} (der.) \textit{indfpro} \textbf{1)} everywhere. \textit{Ŋa hɛthhɛthni ŋa dukduk hiŋk ndɔndɔ, ŋa gbundagbunda feɛ hiŋk mɛsaɛ atok.} They slipped in from all directions and grabbed the money from the table. \textbf{2)} anywhere.

\TCheadword{ndɔi} \textit{indfpro} it (is) \textit{Nandɛ ndɔi nwaŋ ni ra, paŋdɛ ŋɔ pɔ wɔ Fɛbuari, nɛndɛ ŋɔ pɔ wɔ 2016.} Today (is) the thirteenth of the month called February, in the year called 2016. \textit{Tɛmpim la koi ndɔi ntiŋ pɔ che wɔ kɔŋ, chaŋ pɔ koŋla.} Sometimes it would take two days without being buried, until the process is done.

\TCheadword{ndɔlɔ} (comp. of \TClink{ndɔ}, \TClink[5]{lɔ}, see \TClink{ndɔ}) 

\TCheadword[1]{ndɔndɔ} (der. of \TClink{ndɔ}) 

\TCheadword[2]{ndɔndɔ} (der. of \TClink{ndɔ}) 

\TCheadword{ndumabe} \textit{n} [ndùmàbé] someone who has a heart of punishing (K dialect). 

\TCheadword[1]{nɛ} \textit{cf}: \TClink[2]{bɛŋ} (comp. of,der. of \TClink[3]{bɛŋ}). \textit{n} sole of the foot.

\TCsubword[2]{nɛ} (unspec.) \textit{v} \textbf{1)} tread on (\citealt{Pichl1967}). \textbf{2)} step on. (\citealt{Sumner1921}). \textbf{3)} creep in and steal. (\citealt{Pichl1967}). \textit{Nɛ kufə thɔm wɔ lɛ kɔ na lẽy lɛ.} He furtively stole the trousers of his friend while paying him a visit (\citealt{Pichl1967}).

\TCheadword[2]{nɛ} (unspec. of \TClink[1]{nɛ}) 

\TCheadword{Nɛbaŋ} \textit{nam} Nebang, name given to 7\textsuperscript{th} daughter.

\TCheadword{nɛbaŋ} \textit{n} [nɛ̀bàŋ] tree species, lily-like tree, its broad leaf sometimes used like a cocoa leaf for wrapping (K dialect). 

\TCheadword{nɛkɛli} (der. of \TClink[2]{nak} (comp. of \TClink[1]{nak}), \TClink[1]{-i}, see \TClink[1]{nak}) 

\TCheadword{nɛki} (der. of \TClink[2]{nak} (comp. of \TClink[1]{nak}), \TClink[1]{-i}, see \TClink[1]{nak})

\TCheadword{nɛmil} \textit{cf}: \TClink{theki} (der. of \TClink{the}, \TClink{-k}, \TClink[1]{-i}). \textit{v} taste. \textit{Gbi ni ngefeyɛ, mɔi binthmabinthma mpuliɛpuliɛ mɔi nɛmil labo iyɛllɛ ŋɔ shilɔ che.} Together with the pepper, you mix it up, and then you taste it to know if the salt is okay. \textit{Mɔi nɛmil hɔŋ shi gbo che hɔŋ nyemɔɛ, mɔi thiŋgi hɔ koŋ gbo lɔ, mɔi thiŋgi.} You taste it if it is exactly as you want it, then you put it down if it has finished cooking.

\TCheadword[1]{nɛn} \textbf{1)} \textit{n} guy, man, companion, friend, bloke. \textit{Nɛn doki wɔe hun chɔŋ waaŋmaa len yeŋkɛ-lɛŋba.} This man came to (began to) love this woman very much. \textbf{2)} \textit{interj} you (in addressing one's equal) \textit{Ŋa wɔe yii-ni ŋa hɔɛ, “Nɛn mbi len gbi ha hɔ, ha la pɔ ka mɔ ŋhɔɛ?”} They asked him, “Young man, do you have anything to say about this accusation?”

\TCheadword[2]{nɛn} \textit{cf}: \TClink[1]{mɛŋk}, \TClink{yiars}. \textit{n} year. \textit{Nɛn thi wɔ?} How many years?

\TCsubword{nɛnveleŋ} (comp.) \textit{temp} last year, year behind (K dialect). 

\TCsubword{nɛnonɛn} (der.) \textit{temp} every year \textit{Nɛn-o-nɛn yɛi la ŋa.} Every year that is how we do. \textit{Mɔ lɔ kɔ nɛn-o-nɛn?} Do you go there every year?

\TCheadword{Nɛnɛ} \textit{nam} Nene, female name given to a person. \textit{Mi Nɛnɛ, nɔ wɔ sɛmɛlɔ kilɛko.} Mama Nene, someone is standing in the house.

\TCheadword{nɛnonɛn} (der. of \TClink[2]{nɛn}, \TClink{-o-}, see \TClink[2]{nɛn}) 

\TCheadword{nɛnveleŋ} (comp. of \TClink[2]{nɛn}, \TClink[1]{veleŋ}, see \TClink[2]{nɛn}) 

\TCheadword{nɛŋkoŋ} \textit{n} tree species, good for firewood (\citealt{Pichl1967}).

\TCheadword{nɛs} [nɛs] (Port \textit{ananás} ‘pineapple') \textit{n} pineapple. \textit{A yuk pɛlɛ, a yuk ikonatɛ, a yuk inɛsɛ.} I plant rice, I plant coconut, I plant pineapple.

\TCheadword{Nɛtɛ} \textit{nam} Netteh, name given to a person. \textit{Bami wɔlɔ Jɔn Nɛtɛ.} My father is John Netteh.

\TCheadword{nɛvamaind} (Eng \textit{nevermind}) \textit{disco} nevermind. \textit{Nɛvamaind yɛ ibiyɛn dɛ kɛ stil ai maneg bikɔs pomdɛ che ŋa mpanth, biyɛni.} Never mind that we do not have anything, but still I manage although my husband does not have work and does not have anything.

\TCheadword{nfinɔthomɔ} (comp. of \TClink{nɔ}, \TClink{thom}, see \TClink{nɔ}) 

\TCheadword{Nfɔs} \textit{cf}: \TClink{Mɔfɔs}. \textit{nam} Mofos, name given to a place – a town in Kagboro Chiefdom, upriver from the coast above Shenge (K dialect). \textit{Aa, a tipɛ kilkaŋdɛ Nfɔs ko.} Yes, I started school in Mofos.

\TCsubword{ŋgbɔl} (der. of \TClink{gbɔl})

\TCheadword[1]{ni} \textit{cf}: \TClink{huɛŋ}, \TClink[1]{mu}, \TClink[3]{pɛ}, \TClink[4]{si}. \textit{temp} \textbf{1)} then. \textit{Yɛ mɔ theli wɔk ni nɔɛ kɔ ke sampullɛ wɔi si kɛ nɔɛ ki wɔ tintin, n thambas ɛ.} When you say something, let the person see the sample, then the person knows that this person is straightforward. \textit{Yɛ mɔ ni bɛ yabasɛ atok, mɔi gbiŋgith.} After putting the onions in, then you cover it. \textbf{2)} now. \textit{Awa kɛ mi ŋɔ mɔ ni ŋa ja ramɔa?} How do you now do things for the family? \textit{Ok, a wɔni yɛ nɛnthi mɛn dɛ kunɛ lɔni yɛ.} Ok, I (would) say that it is five years I am in it now. comp. \TClink{mɛŋhiɔlniwaŋ} (see \TClink[1]{mɛn}) 

\TCsubword{buleŋni} (der.) \textit{prep} together with.

\TCheadword[2]{ni} \textit{cf}: \TClink[2]{ma}. \textit{prt} negation particle; no; not; none. \textit{Apim ashiŋa, apim achehaŋ pɛ koŋshi.} I know some of them, some I would not know anymore. \textit{Ha wɔn gbi, nɔ gbi cheni.} Above him there is no other. \textit{A-a, i cheni wɔi hin waŋ ni tindɛ, hin wan iko leɛ.} No, we twelve are not all alive, there are ten of us remaining. comp. \TClink{pɔsɔni} (see \TClink{pɔs}), der. \TClink{manha} (see \TClink[1]{ma}), \TClink{yɛni} (see \TClink[1]{yen}), unspec. \TClink[1]{biyɛni} (see \TClink[1]{bi}) 

\TCheadword[3]{ni} \textit{coordconn} \textbf{1)} and. \textit{Ka hin hɔlɔŋ ni ŋjeyɛ ŋa hin jo.} He gave us life and food for us to eat. \textbf{2)} but. comp. \TClink{mɛŋhiɔlniwaŋ} (see \TClink[1]{mɛn}), \TClink{waŋnibul} (see \TClink[2]{waŋ}), \TClink{waŋnihiɔl} (see \TClink{hiɔl}), \TClink{waŋnimɛnra} (see \TClink[2]{waŋ}), \TClink{waŋnimɛŋtiŋ} (see \TClink[2]{waŋ}), \TClink{waŋnitiŋ} (see \TClink[2]{waŋ})

\TCheadword[4]{ni} \textit{cf}: \TClink[5]{che}, \TClink[3]{hɔ}, \TClink[2]{la}, \TClink[2]{lɛ}, \TClink[3]{ŋa}, \TClink[2]{si}, \TClink[1]{yɛ}. \textit{subordconn} \textbf{1)} that. \textit{I yema ni wun ko ja tɔntho, la ivelɛmɔ teŋga-teŋgaɛ.} We want to now come to the singing aspect that we actually called you for. \textit{A hun yi lamŋan dɛ ki ni lemɛmi jaliwɔ atokɛ, lenolen la wɔ si ŋa wɔndɛ.} I have come to ask this man to talk about himself, everything that he knows about himself. \textbf{2)} in order to; so that. \textit{Nɔthiɛ nthɛkɛsiɛ wɔ ni san la ntenɛ.} Human beings clarify in order to understand things. \textbf{3)} for. \textit{Awɔ ŋalmɔ, wɔlɔŋ mɔɛ, labo mɔla yema ni nɔ ndɔ-ndɔ thela.} I said it's about you, your life, if you are ok with everyone hearing (about it). \textbf{4)} when. \textit{Ŋɔ pɔ ni tipɛ rɔkɛ.} That is the time harvesting begins. \textbf{5)} why. \textbf{6)} if. comp. \TClink{yɛbini} (see \TClink[3]{yɛ}) 

\TCheadword[5]{ni} \textit{cf}: \TClink{kendɛ}, \TClink[4]{ken}, \TClink{ŋɛ}. \textit{prep} \textbf{1)} like. \textit{Pabondɛ fli ni ŋɔ rɛdi ha hun, hɛ hɔ ha ni ki.} If really it is ready to come out, it will make like this. \textbf{2)} with. \textit{Ba Na ni gbɔlkajo wɔ ɛ yema ŋa jo tri thɛai than gbi.} The spider with his gluttony wants to eat in all the towns (Sumner 1921: txt 7). \textit{Nɛn-o-nɛn yɛ kɔ ko baŋiɛ, iwɔ kɔ bɔnth mpanthɛ ni mpɛntɛ ŋa mi yɛ gbi.} Every year when we go to our father, we help him in the work with all my brothers. \textbf{3)} around. \textbf{4)} from. \textbf{5)} to.

\TCheadword[6]{ni} \textit{cf}: \TClink{atok}. \textit{post} about. \textit{Wɛl kenɛki nia?} What about now? \textit{Wɛl, kɔŋdɛ nia?} Well, what about burials?

\TCheadword{-ni} \textit{cf}: \TClink[1]{-n}. \textit{v} \textit{sfx} reflexive, involves more than one actor, a group action, plural? \textit{Nrokɛ ŋanɛ ŋahunɔni-muɛ, nrekia ɛ.} The grandchildren that have not come yet, the great grandchildren. \textit{Lɛlini àtok wɔ́ŋ pəŋ.} Be careful. (how to say ‘Watch out!' without offending). comp. \TClink{bosni} (see \TClink[1]{bos}), \TClink{honi} (see \TClink[1]{ho}), \TClink{lathni-nser} (see \TClink[1]{lath}), \TClink{-mani} (see \TClink[4]{ma}), der. \TClink{bakni} (see \TClink[1]{bak}), \TClink{balani} (see \TClink{bala}), \TClink{bɛrɛlɔni} (see \TClink[2]{bɛ}), \TClink[1]{boni} (see \TClink[1]{bo}), \TClink{chaini} (see \TClink[1]{chai}), \TClink{chɛthni} (see \TClink[1]{chɛth}), \TClink[1]{gbemni} (see \TClink{gbem}), \TClink[2]{gbemni} (see \TClink{gbem}), \TClink{gbɛmani} (see \TClink{gbɛ}), \TClink{gbiŋkithni} (see \TClink[1]{bim}), \TClink{hosni} (see \TClink{bus}), \TClink{jɛthɛlini} (see \TClink[2]{jɛth}), \TClink{keni} (see \TClink{ke}), \TClink{kenin} (see \TClink{ke}), \TClink{lathni-nser} (see \TClink[1]{lath}), \TClink{leni} (see \TClink[2]{le}), \TClink{mathni} (see \TClink{math}), \TClink{mɛlkɛni} (see \TClink{mɛl}), \TClink{mɛlni} (see \TClink{mɛl}), \TClink{piŋkini} (see \TClink[1]{piŋki}), \TClink{raŋkani} (see \TClink{raŋka}), \TClink{sɛini} (see \TClink{sɛin}), \TClink{tɛmɛni} (see \TClink{tɛmɛ}), \TClink{tɛnini} (see \TClink[1]{tɛn}), \TClink{thɛkɛsini} (see \TClink{thɛkɛsi}), \TClink{thɛkini} (see \TClink{thak}), 
\TClink{thɔni} (see \TClink[1]{thɔk}), \TClink{tipɛni} (see \TClink{tipɛ}), \TClink{tuntni} (see \TClink{tunt}), \TClink[1]{wɛini} (see \TClink[2]{wɛi}), \TClink[2]{wɛini} (see \TClink[2]{wɛi}), \TClink[2]{wɔni} (see \TClink[1]{hɔ}), \TClink{wɔŋni} (see \TClink{wɔŋ}), \TClink{yɛthini} (see \TClink{yɛthi}), unspec. \TClink{bɔini} (see \TClink[2]{bɔi}), \TClink{busni} (see \TClink{bus}), \TClink{mɛmiɛni} (see \TClink{mɛmi}), \TClink{mɛmilni} (see \TClink{mɛmi}) 

\TCsubword{gbɛlɛni} (der.) \textit{v} visit one another. \textit{No, ashila fli lɛ wɔ Bolomnɔ bikɔs ikache gbɛlɛni.} No really, I know that he is Sherbro because we used to visit each other.

\TCsubword{-kani} (der.) \textit{cf}: \TClink{-mani} (comp. of \TClink[4]{ma}, \TClink{-ni}) \textit{v > v} \textit{sfx} action against oneself. der. \TClink{yakani} (see \TClink[1]{ya})

\TCsubword{ruŋklani} (der.) \textit{v} clutch oneself, clutch the hands around the body indicating sadness (\citealt{Pichl1967}). 

\TCsubword{thimni} (der.) \textit{v} turn around. \textit{Cho thimni ka mma-m thim kɔk!} Cho, turn round, don't show me your back! (\citealt{Pichl1967}). 

\TCsubword{thoŋkini} (der.) \textit{v} show oneself. \textit{Wantɛm do kong gbo toɛɛ, wɔ yɛ hɔni street lɛ ibɔl hã kɔ thonkini.} This young woman, when she only has dressed up, she goes out in the the street to show herself (\citealt{Pichl1967}). 

\TCsubword{tilɛni} (der.) \textit{v} not let someone have their rightful share, distribute unfairly (\citealt{Pichl1967}). 

\TCsubword{thɛmni} (unspec.) \textit{cf}: \TClink{theyɛn-nɛki} (comp. of \TClink{theni}, \TClink{nɛki}). \textit{v} stub one's toe. \textit{Bia bɛth rəm wɔ lɛ thɛmni yenwɛy nai lɛ bol.} Bia has cut his toe, he stubbed it badly on the way (\citealt{Pichl1967}). 

\TCheadword{nimonia} \textit{n} pneumonia.

\TCheadword{niŋgbi} \textit{cf}: \TClink{bɔkon}. \textit{n} owl species, smaller than giant \textit{bɔ̀kón} (B dialect); (wɔ/hã, si) owl (\citealt{Pichl1967}). \textit{Nïŋgbi lɔ wɔ lɛ ve fɔnwɛy, vɛ anyin hã hɔ}. The owl is the bird of witches, so people say (\citealt{Pichl1967}). 

\TCheadword{niŋka} \textit{cf}: \TClink{swɛ}. \textit{n} \textbf{1)} (kɔ/-) charcoal (\citealt{Pichl1967}).

\TCheadword{niŋkta} \textit{cf}: \TClink{yubom} (comp. of \TClink{yu}, \TClink{bom}). \textit{n} fish species, electric ray, torpedo fish (Torpedo spp.) (also \textit{yu bo̹m}) (\citealt{Pichl1967}). 

\TCheadword{Njabu} \textit{nam} Njabu, female name given by a society. 

\TCheadword{Njefe} \textit{nam} Njefe, female name given to a person (\citealt{Pichl1967}).

\TCheadword{njok} \textit{Loc} right.

\TCheadword{Njopojo} \textit{nam} Njopojo, female name given by a society. 

\TCheadword{nn} \textit{cf}: \TClink{aa}, \TClink{ayo}, \TClink{ee}, \TClink{yɛs}. \textit{disco} yes. \textit{Nn, aka che siŋ.} Yes, I used to play. 

\TCheadword[1]{no} \textit{cf}: \TClink{gbe}, \TClink{pɔs}. \textit{quant} much; a lot. \textit{Apuma mɔi ŋa bɛŋsin no we.} Your children are suffering a lot.

\TCheadword[2]{no} (Eng \textit{no}) \textit{cf}: \TClink{a-a}, \TClink{bɛaan}, \TClink{sakoo}. \textit{disco} no. \textit{No, ashila fli lɛ wɔ Bolomnɔ bikɔs ikache gbɛlɛni.} No really, I know that he is Sherbro because we used to visit each other.

\TCheadword{Noro} \textit{nam} [nor'o] meaning the Bondo mask, there are black ones and white ones (B dialect). 

\TCheadword{Novɛmba} \textit{nam} November. \textit{Tɛmdɛ ni ŋɔ kɔi ni hun sɛkillɛ, tɛm Novɛmba ŋa bɔnth ni Disɛmbaɛ.} The time for drying comes between November and December.

\TCheadword{nɔ} \textbf{1)} \textit{n} person; agentive element in compounds. \textit{Ahindɛ ha hun ha hayema nɔ pɔ koi nɔ bul pɔ wɔ wom Nyamba ko.} Then the people came, they said they want one person to send her to Moyamba. \textit{Che nɔ pika wɔ dumɔni yɛ, yanyi wɔn wɔ dumɔniye.} There is no one else who raised us, it is our mother who raised us. \textit{Kɛ nsi nɔ lan?} But do you know the person? \textbf{2)} \textit{n} man. \textit{Tamɔ tondɛ wɔ gbaŋkthani kotha kathil bom mɛ nɔ bɛn.} The small boy wrapped the big Kente cloth around himself as if he were a big man (\citealt{Pichl1967}). \textbf{3)} \textit{indfpro} someone. \textbf{4)} \textit{n} relative. comp. \TClink{Bolomnɔ} (see \TClink[1]{Bolom}), \TClink{bulnɔbul} (see \TClink[3]{bul}), \TClink{kɔysunɔ} (see \TClink{kɔysu}), \TClink{kugbanɔ} (see \TClink{kugba}), \TClink{lemnɔ} (see \TClink[1]{lem}), \TClink{mɛknɔ} (see \TClink[2]{mɛk}), \TClink{nɔyegbe} (see \TClink{yegbe}), \TClink{Pothonɔ} (see \TClink{Potho}), \TClink{sokonɔ} (see \TClink{soko}), \TClink{Thɛmnɔ} (see \TClink{them}), \TClink{thiŋnɔ} (see \TClink{thiŋ}), \TClink{womnɔ} (see \TClink{wom}), \TClink{wonɔ} (see \TClink[2]{wok}), \TClink{yasenɔ} (see \TClink{yase}), der. \TClink{kinɔ} (see \TClink[1]{ki}), \TClink{thomnɔ} (see \TClink{thom})

\TCsubword{gbokanɔ} (comp.) Cf: \TClink{Boka}, \TClink{Gboka}, \TClink{Gbokathoŋthoŋ}. \textit{n} (wɔ/hã) one who is not a member of a society (\citealt{Pichl1967}). 

\TCsubword{Mɛndenɔ} (comp.) \textit{n} Mende person.

\TCsubword{nfinɔthomɔ} (comp.) \textit{cf}: \TClink{thomnɔ} (der. of \TClink{thom}, \TClink{nɔ}). \textit{n} beggar.

\TCsubword{nɔbalia} (comp.) \textit{n} rich man. \textit{Nɔ́bàlìà wɔ́ kì.} This is a rich man. der. \TClink{nɔbaliabalia} (see \TClink{nɔ}) 

\TCsubword{nɔbaliabalia} (comp.), (der. of \TClink{nɔbalia}) \textit{n} very rich man. \textit{Laŋgba ki ka cheɛ nɔbaliabalia: ka che pin anyín si wɔ ŋa wom pɔk Pothoɛ.} This man was a very rich man: he bought people and sent them to the white people's country.

\TCsubword{nɔbɛn} (comp.) \textit{n} \textbf{1)} old person. \textit{Bikɔs nɔbɛndɛ koŋ gbo tham, ko piŋgindɛ tamɔ.} Because if an old person has become old enough, she has turned into a baby. \textbf{2)} elder. \textit{Yaŋ yalɔ bɛ nɔbɛndɛ ni.} I myself am even the older one (now).

\TCsubword{nɔbonthɔ} (comp.) \textit{cf}: \TClink[1]{kump}, \TClink{nɔbɛma} (comp. of \TClink{nɔ}, \TClink{bɛmpa}). \textit{n} helper. \textit{A-a, be nɔbonthɔ nɔ cheni pɛ.} No, there are no helpful people anymore.

\TCsubword{nɔbulɔ} (comp.) \textit{n} working man. \textit{Wɔn wɔ nɔbulɔ.} He is a working man.

\TCsubword{nɔchanchaa} (comp.) \textit{cf}: \TClink{nɔpili} (comp. of \TClink{nɔ}, \TClink{pili}). \textit{n} wanderer (\citealt{Pichl1967}).

\TCsubword{nɔdiɛnɔ} (comp.) \textit{n} murderer (\citealt{Pichl1967}).

\TCsubword{nɔdwiyɛ} (comp.) \textit{n} thief. \textit{Ha lɛ mma wɔ pɔkɔni, wɔ lɛ nɔdwiyɛ!} You should not forget about him, he's a thief! (\citealt{Pichl1967}).

\TCsubword{nɔfɔnwɛi} (comp.) \textit{cf}: \TClink{kɔysunɔ} (comp. of \TClink{kɔysu}, \TClink{nɔ}). \textit{n} witch (\citealt{Pichl1967}). \textit{Kacheɛ nɔwɔi, nɔwɔiɛ wɔ nɔfɔnwɔiyɛ.} He was a bad person, a bad person is a witch person.

\TCsubword{nɔhampanth} (comp.) \textit{cf}: \TClink{hɔima}, \TClink[1]{kump}. \textit{n} worker, workman (\citealt{Pichl1967}).

\TCsubword{nɔhinyɛchɛk} (comp.), (der.) \textit{cf}: \TClink[2]{fama}, \TClink{nɔra} (comp. of \TClink{nɔ}, \TClink[2]{ra}). \textit{n} farmer (\citealt{Pichl1967}).

\TCsubword{nɔhɔnthɛ} (comp.) \textit{n} fisherman (\citealt{Pichl1967}).

\TCsubword{nɔikeche} (comp.) \textit{n} blind man (\citealt{Pichl1967}).

\TCsubword{nɔkafa} (comp.) \textit{n} sinner (\citealt{Pichl1967}).

\TCsubword{nɔkil} (comp.) \textit{n} housewife. \textit{Yami wɔn kacheɛ nɔkilɛ ka ko baa mi.} My mother was a housewife (lit. My mother was at the house person to my father).

\TCsubword{nɔkith} (comp.) \textit{n} little person (\citealt{Pichl1967}). 

\TCsubword{nɔkɔmbɛl} (comp.) \textit{n} palm tree climber (\citealt{Pichl1967}).


\TCsubword{nɔlimɛnsɔn} (comp.) \textit{n} dream interpreter (\citealt{Pichl1967}).

\TCsubword{nɔloliɛ} (comp.) \textit{n} savior (\citealt{Pichl1967}). \textit{Wɔ, wɔ lɔ nɔloliɛ.} He has been the savior.

\TCsubword{nɔmpithika} (comp.) \textit{cf}: \TClink[3]{kil}, \TClink{nɔsukusɛkɛ} (comp. of \TClink{nɔ}, \TClink[1]{sukusɛkɛ}). \textit{n} rascal (\citealt{Pichl1967}).

\TCsubword{nɔmpɔm} (comp.) \textit{cf}: \TClink{nɔramda} (comp. of \TClink{nɔ}, \TClink[2]{ramil}). \textit{n} herbalist (\citealt{Pichl1967}).

\TCsubword{nɔnaka} (comp.) \textit{n} sick person (\citealt{Pichl1967}).

\TCsubword{nɔncheŋwɛi} (comp.), (der.) \textit{n} bad character (\citealt{Pichl1967}).

\TCsubword{nɔpaa} (comp.) \textit{n} guardian, protector (\citealt{Pichl1967}). \textit{Ncheɛ nɔ paa hĩ ɛ ihɔlɔng do ay.} Be our protector in this life (\citealt{Pichl1967}). 

\TCsubword{nɔpili} (comp.) \textit{cf}: \TClink{nɔchancha} (comp. of \TClink{nɔ}). \textit{n} wanderer (\citealt{Pichl1967}).

\TCsubword{nɔpokan} (comp.) \textit{n} \textbf{1)} man (\citealt{Pichl1967}). \textbf{2)} husband. \textit{Aa, nɔ gbisiŋɛ, abi nɔpokan.} Yes, I am married, I have a husband. \textit{Bikɔ pomdɛ wɔ mi ni yɛthi sɔŋgɔ ma ŋɔ nɔpikan wɔ ŋa yɛthi nɔma wɔi.} Because my husband is treating me as a husband should treat his wife.

\TCsubword{nɔra} (comp.) \textit{cf}: \TClink[2]{fama}, \TClink{nɔhinyɛchɛk} (comp. of, der. of \TClink{nɔ}, \TClink[1]{hini}, \TClink{chɛk}). \textit{n} farmer. \textit{Bami ka cheɛ nɔraa nchɛk.} My father was a person that brushes farms. \textit{Wɔ nɔra, wɔ ra ichɛkɛ?} He used to brush, is he brushing a farm? comp. \TClink{nɔrachɛk} (see \TClink{nɔ}) 

\TCsubword{nɔrachɛk} (comp.), (comp. of \TClink{nɔra}) \textit{n} farmer (\citealt{Pichl1967}).

\TCsubword{nɔramda} (comp.) \textit{cf}: \TClink{nɔmpɔm} (comp. of \TClink{nɔ}, \TClink[1]{pɔm}). \textit{n} doctor (\citealt{Pichl1967}).

\TCsubword{nɔsanth} (comp.) \textit{n} elder (\citealt{Pichl1967}).

\TCsubword{nɔsɔnthɔ} (comp.) \textit{cf}: \TClink{telɔ}. \textit{n} tailor (\citealt{Pichl1967}).

\TCsubword{nɔsukusɛkɛ} (comp.) \textit{cf}: \TClink[3]{kil}, \TClink{nɔmpithika} (comp. of \TClink{nɔ}, \TClink{pithika}). \textit{n} troublemaker (\citealt{Pichl1967}).

\TCsubword{nɔthikla} (comp.) \textit{n} trader, merchant (\citealt{Pichl1967}).

\TCsubword{nɔthoŋka} (comp.) \textit{cf}: \TClink{nɔŋhɔ} (der. of, comp. of, id. of \TClink{nɔ}, \TClink[1]{hɔ}). \textit{n} lawyer (\citealt{Pichl1967}).

\TCsubword{nɔtolɔ} (comp.) \textit{n} exhibitionist (\citealt{Pichl1967}).

\TCsubword{nɔtɔnɔ} (comp.) \textit{n} singer (\citealt{Pichl1967}).

\TCsubword{nɔwu} (comp.) \textit{cf}: \TClink[2]{bobo}, \TClink[2]{pɔm}. \textit{n} corpse. \textit{Yɛ nɔhuɛ hinɛ muɛ, te wɔi mɔi…} When the corpse is still lying down, until the time the man would come… \textit{Ì kɔ́ŋ nɔ́éwɛ̀.} We buried the corpse. [first syllable of corpse sounds like it's stressed]

\TCsubword{nɔyeŋkes} (comp.) \textit{cf}: \TClink{Potho}. \textit{n} English person (\citealt{Pichl1967}).

\TCsubword{nɔyes} (comp.) \textit{n} dancer (\citealt{Pichl1967}).

\TCsubword{nɔyiɛnthiŋ} (comp.) \textit{cf}: \TClink{nɔyiɛyibaw} (comp. of \TClink{nɔ}, \TClink[1]{yi}, \TClink{yibaw}), \TClink{thiŋnɔ} (comp. of \TClink{thiŋ}, \TClink{nɔ}). \textit{n} fortune teller (\citealt{Pichl1967}).

\TCsubword{nɔyiɛyibaw} (comp.) \textit{cf}: \TClink{nɔyiɛnthiŋ} (comp. of \TClink{nɔ}, \TClink[1]{yi}, \TClink{thiŋ}), \TClink{thiŋnɔ} (comp. of \TClink{thiŋ}, \TClink{nɔ}). \textit{n} diviner (\citealt{Pichl1967}).

\TCsubword{nɔyilɛ} (comp.) \textit{cf}: \TClink[2]{bɛŋk} (id. of \TClink[1]{bɛŋk}), \TClink[2]{koŋkbo} (id. of \TClink[1]{koŋkbo}), \TClink[2]{thɔŋpaŋ} (id. of \TClink[1]{thɔŋpaŋ}). \textit{n} drunkard (\citealt{Pichl1967}).

\TCsubword{puinɔ} (comp.) \textit{n} hunter (\citealt{Pichl1967}). \textit{Pui-nɔ lɛ chala tho l'ay wɔ mïrə chal lɛ.} The hunter sits in the bush and watches the deer (\citealt{Pichl1967}).

\TCsubword{nɔlɛli} (der.), (comp.) \textit{n} observer (K dialect). \textit{Nɔlɛliɛ cheni bɛ pɛ.} There aren't any examiners anymore. 

\TCsubword{nɔmaa} (der.), (comp.) \textit{n} \textbf{1)} woman. \textbf{2)} wife. \textit{Bikɔ pomdɛ wɔ mi ni yɛthi sɔŋgɔ ma ŋɔ nɔpikan wɔ ŋa yɛthi nɔma wɔi.} Because my husband is treating me as a husband should treat his wife. \textit{Abiɛni pɛ nɔma.} I do not have a wife anymore. \textbf{3)} female. \textit{Aa, nɔsendɛ, nɔma wɔn wɔ mɔikɛ tiŋdɛ ko bami.} I am the first one, it is a female that is the second one to my father. comp. \TClink{nɔmachondal} (see \TClink{nɔ}), \TClink{pumaama} (see \TClink[3]{pum}) 

\TCsubword{nɔmachondal} (der.), (comp.), (comp. of \TClink{nɔmaa}) \textit{n} lewd woman, prostitute (\citealt{Pichl1967}). 

\TCsubword{nɔncheŋk} (der.), (comp.) \textit{n} enemy (\citealt{Pichl1967}).

\TCsubword{nɔnse} (der.), (comp.) \textit{n} first child (\citealt{Pichl1967}). \textit{Mɔmɔ nɔnse ko bamɔ?} Are you your father's first child?

\TCsubword{nɔŋhɔ} (der.), (comp.), (id.) \textit{cf}: \TClink{nɔthoŋka} (comp. of \TClink{nɔ}, \TClink[1]{thoŋka}). \textit{n} lawyer.

\TCsubword{ŋɔhɔlpok} (der.), (comp.) \textit{n} judge. \textit{Ŋɔ́hɔ́lpòkɛ̀ wɔ̀ thɛ́kɛ́sí sàbàɛ́.} The judge interprets the law. \textit{Ŋɔ́hɔ́lpòkɛ̀ wɔ̀ thɛ́kɛ́sí sàbàɛ́ yèŋwɛ̀í/yèŋkɛ̀lɛ́ŋ.} The judge interpreted the law badly/well.

\TCsubword{Maniŋkanɔ} (der.) \textit{nam} Maninka person or people. \textit{Koromanɔ aida ɔrijin wɔɛ wɔ Maninkanɔ... che Themnɔ wɔɛ.} Koroma, either the origin is Maninka... it is not Themne. 

\TCsubword{nɔonɔ} (der.) \textit{cf}: \TClink[1]{ndɔndɔ} (der. of \TClink{ndɔ}), \TClink[4]{ŋa}. \textit{indfpro} \textbf{1)} anyone. \textit{Kɛ kpɔnko hɔ ka che trï ko ntɛnt, hɔ nɔonɔ ka chen kɔ ai ɛ.} But there was a forest near the town, which no one entered (\citealt{Pichl1967}). \textit{Wɔlɔɛ ki ɡbi, nɔ-nɔ, wɔ bia yema ŋa ke ŋɔɛ wɔ ŋɔ ke.} Throughout the world, anyone who wants to see it, sees it. \textbf{2)} everyone. \textit{Nɔonɔ nten ma wɔɛ ma gbo ko feɛ mesaɛ atok.} Everyone in the court bari focused their minds on the money on the table.

\TCsubword{nɔthi} (der.) \textit{n} \textbf{1)} mankind. \textbf{2)} human being. \textit{Nɔthiɛ nthɛkɛsiɛ wɔ ni san la ntenɛ.} Human beings clarify in order to understand things. 

\TCsubword[1]{nya} (der.) \textit{cf}: \TClink{nyin} (der. of \TClink{nɔ}). \textit{n} \textbf{1)} people. \textit{Kache yɛ n yema bo nɔma, ni anyamɔɛ kɔlɔ...} In the past if you wanted a woman, then your people would go there... \textit{So yan aka bo minɛ mɔ gbemi kilɛ ko ni pɔmthɛ ken aŋa bɛndɛ ŋa ŋaɛ.} So me, I always thought you just delivered in the home, with the leaves, like our first people did it. \textit{Oo aŋa mi isi yɛ lɛ kɛ Kraist ka wu ŋa hin.} Oh, my people, let us realize that Christ died for us. \textit{Aŋa pum ŋa mɔ mith we.} Some people will hate you. \textbf{2)} men. \textbf{3)} population. comp. \TClink{nyabɔn} (see \TClink{nɔ}), \TClink{nyanɔ} (see \TClink{nɔ}), der. \TClink{nyama} (see \TClink{nɔ}), id. \TClink{Nyemɔ} (see \TClink{nɔ})

\TCsubword{nyabɔn} (der.), (comp. of \TClink[1]{nya}) \textit{cf}: \TClink{gboka}. \textit{n} cannibal (\citealt{Pichl1967}).

\TCsubword{nyama} (der.), (der. of \TClink[1]{nya}) \textit{cf}: \TClink{sumoŋ}. \textit{n} Bondo initiates.

\TCsubword{nyanɔ} (der.), (comp. of \TClink[1]{nya}) \textit{n} stranger.

\TCsubword{Nyemɔ} (der.), (id. of \TClink[1]{nya}) \textit{nam} Moyeamoh, name given to place located in Mamu Section, Bumpeh Chiefdom, Moyamba District (lit. 'agreed place' – people agreed to stay at the place after looking for a place to settle). \textit{Ya gbemni Nyemɔko, Mamu Sɛkshɔn, Bompɛ Chifdɔm, Mɔyamba Distrikt.} I was born in Moyeamoh, Mamu Section, Bumpeh Chiefdom, Moyamba District.

\TCsubword{nyin} (der.) \textit{cf}: \TClink[1]{nya} (der. of \TClink{nɔ}). \textit{n} \textbf{1)} people. \textbf{2)} humans. der. \TClink{nyina} (see \TClink{nɔ}) 

\TCsubword{nyina} (der.), (der. of \TClink{nyin}) \textit{n} soul.

\TCsubword{minnɔ} (unspec.) \textit{n} person. \textit{Bolomnɔɛ minnɔ ndum wɔɛ.} The Sherbro man is a person with good character.

\TCheadword{nɔbalia} (comp. of \TClink{nɔ}, \TClink[1]{bali}, see \TClink{nɔ})

\TCheadword{nɔbaliabalia} (der. of \TClink{nɔbalia} (comp. of \TClink{nɔ}, \TClink[1]{bali}), see \TClink{nɔ}) 

\TCheadword{nɔbɛma} (comp. of \TClink{nɔ}, \TClink{bɛmpa}, see \TClink{bɛmpa}) 

\TCheadword{nɔbɛn} (comp. of \TClink{nɔ}, \TClink[3]{bɛn}, see \TClink{nɔ}) 

\TCheadword{nɔbonthɔ} (comp. of \TClink{nɔ}, \TClink{bɔnth}, see \TClink{nɔ}) 

\TCheadword{nɔbulɔ} (comp. of \TClink{nɔ}, \TClink[1]{bulɔ}, see \TClink{nɔ}) 

\TCheadword{nɔchancha} (der. of \TClink{nɔ}, \TClink[1]{chaŋchaŋ} (der. of \TClink[1]{chaŋ}), see \TClink{nɔ})

\TCheadword{nɔdiɛnɔ} (comp. of \TClink{nɔ}, \TClink[1]{di}, see \TClink{nɔ}) 

\TCheadword{nɔdwiyɛ} (comp. of \TClink{nɔ}, \TClink[1]{dui}, see \TClink{nɔ}) 

\TCheadword{nɔfɔnwɛi} (comp. of \TClink{nɔ}, \TClink[1]{fɔnwɛi} (comp. of \TClink[1]{wɛi}), see \TClink{nɔ}) 

\TCheadword{nɔhampanth} (comp. of \TClink{nɔ}, \TClink{haa}, \TClink[1]{panth}, see \TClink{nɔ}) 

\TCheadword{nɔhinyɛchɛk} (comp. of, der. of \TClink{nɔ}, \TClink[1]{hini} (der. of \TClink{hin}, \TClink[1]{-i}), \TClink{chɛk}, see \TClink{nɔ}) 

\TCheadword{nɔhɔnthɛ} (comp. of \TClink{nɔ}, \TClink[2]{hɔth}, see \TClink{nɔ}) 

\TCheadword{nɔi} \textit{Aux} would. \textit{Pɔ nɔi koŋ ka inshɔ, tɛmdɛ vɛ pɔ nɔi hɔm lɛ, haŋ ha thunɔ thaozin waŋ.} They would have given assurances, when they tell you the bride price is ten thousand.

\TCheadword{nɔikeche} (comp. of \TClink{nɔ}, \TClink{keche} (unspec. of \TClink{ke}), see \TClink{nɔ}) 

\TCheadword{nɔkafa} (comp. of \TClink{nɔ}, \TClink[1]{kafa}, see \TClink{nɔ}) 

\TCheadword{nɔkil} (comp. of \TClink{nɔ}, \TClink[1]{kil}, see \TClink{nɔ}) 

\TCheadword{nɔkith} (comp. of \TClink{nɔ}, \TClink[1]{kith}, see \TClink{nɔ}) 

\TCheadword{nɔkɔmbɛl} (comp. of \TClink{nɔ}, \TClink[2]{kɔ}, \TClink[2]{bɛl}, see \TClink{nɔ}) 

\TCheadword{nɔlɛli} (comp. of, der. of \TClink{nɔ}, \TClink[1]{lɛli} (comp. of \TClink[3]{lɛ}), see \TClink{nɔ}) 

\TCheadword{nɔlimɛnsɔn} (comp. of \TClink{nɔ}, \TClink[1]{sɔn}, see \TClink{nɔ}) 

\TCheadword{nɔloliɛ} (comp. of \TClink{nɔ}, \TClink{loli} (der. of \TClink[2]{lol}, \TClink[1]{-i}), see \TClink{nɔ}) 


\TCheadword{nɔma} \textit{n} \textit{inɔma} (hɔ̃/-) cotton thread (\citealt{Pichl1967}). 

\TCsubword{nɔmafuuŋk} (comp.) \textit{cf}: \TClink{kɔtin}. \textit{n} \textit{nɔma fuunk} (hɔ̃/-) cotton (\citealt{Pichl1967}).

\TCheadword{nɔmaa} (der. of, comp. of \TClink{nɔ}, \TClink{maa}, see \TClink{nɔ})

\TCheadword{nɔmachondal} (comp. of \TClink{nɔmaa} (der. of, comp. of \TClink{nɔ}, \TClink{maa}), \TClink{chondal}, see \TClink{nɔ}) 

\TCheadword{nɔmafuuŋk} (comp. of \TClink{nɔma})

\TCheadword{nɔmɔk} \textit{n} \textbf{1)} [nɔ̀mɔ̀k] mucus, snot (K dialect); (kɔ/ma) mucus of the nose (\citealt{Pichl1967}). \textit{Nɔmɔk lɛ kɔ hok wɔn minɛ lɛ kɔ isay.} The mucus that comes from his nose is offensive (\citealt{Pichl1967}).

\TCsubword{nɔmɔkhuth} (comp.) \textit{n} [nɔ̀mɔ̀khùth] tree species, ‘snot-sneeze' tree (idph?), leaves have a harsh scent, they are crushed and snorted as medicine (K dialect); (kɔ/ma) leaves of a tree (Allophyllus africanus) as well as the tree itself. The leaves are ground and snuffed as a medicine against cold (\citealt{Pichl1967}). 

\TCheadword{nɔmɔkhuth} (comp. of \TClink{nɔmɔk}, \TClink{huth}, see \TClink{nɔmɔk}) 

\TCheadword{nɔmpithika} (comp. of \TClink{nɔ}, \TClink{pithika}, see \TClink{nɔ}) 

\TCheadword{nɔmpɔm} (comp. of \TClink{nɔ}, \TClink[1]{pɔm}, see \TClink{nɔ}) 

\TCheadword{nɔnaka} (comp. of \TClink{nɔ}, \TClink[1]{nak}, see \TClink{nɔ}) 

\TCheadword{nɔncheŋk} (der. of, comp. of \TClink{nɔ}, \TClink[1]{cheŋk}, see \TClink{nɔ}) 

\TCheadword{nɔncheŋwɛi} (comp. of, der. of \TClink{nɔ}, \TClink[1]{che}, \TClink[1]{wɛi} (der. of \TClink[2]{wɛi}), see \TClink{nɔ}) 

\TCheadword{nɔnse} (der. of, comp. of \TClink{nɔ}, \TClink[1]{nse}, see \TClink{nɔ}) 

\TCheadword{nɔŋgbɛ} \textit{n} sheep, [nɔ̀ŋgbɛ̀]/[nɔ̀ŋgbɛ̀sɛ̀] sheep/sheep (pl) (B dialect); \textit{nɔnkbə} (wɔ/hã, si) sheep (\citealt{Pichl1967}). 

\TCheadword{nɔŋhɔ} (der. of, comp. of, id. of \TClink{nɔ}, \TClink[1]{hɔ}, see \TClink{nɔ}) 

\TCheadword{nɔŋka} \textit{n} (wɔ/hã) bird species (\citealt{Pichl1967})(.

\TCheadword{Nɔŋkɔbɛ} \textit{nam} (wɔ/-) Toma Society spirit who appears as a dancing masquerade (\citealt{Pichl1967}).

\TCheadword{nɔŋkwath} (comp. of \TClink{nɔ}, \TClink{kuath}, see \TClink{kuath}) 

\TCheadword{nɔonɔ} (der. of \TClink{nɔ}, \TClink{-o-}, see \TClink{nɔ}) 

\TCheadword{nɔɔmi} \textit{cf}: \TClink[1]{boni} (der. of \TClink[1]{bo}, \TClink{-ni}), \TClink{keni} (der. of \TClink{ke}, \TClink{-ni}), \TClink[1]{lɛli} (comp. of \TClink[3]{lɛ}). \textit{v} find something that was lost (\citealt{Pichl1967}). \textit{Ya kong nɔɔmi fe̹ kïl lɛ mɔ ka.} I have found some money in your house there (\citealt{Pichl1967}). \textit{Lɛ nɔɔmiɛ gbo kotha lɛ hɔ̃ thuk lɛ, ya bi hã paka mɔ.} If you should find the cloth which was lost, I shall pay you (a reward) (\citealt{Pichl1967}).

\TCheadword{nɔpaa} (comp. of \TClink{nɔ}, \TClink{paa}, see \TClink{nɔ}) 

\TCheadword{nɔpili} (comp. of \TClink{nɔ}, \TClink{pili}, see \TClink{nɔ}) 

\TCheadword{nɔpokan} (comp. of \TClink{nɔ}, \TClink{pokan} (unspec. of \TClink[5]{po}), see \TClink{nɔ}) 

\TCheadword{nɔra} (comp. of \TClink{nɔ}, \TClink[2]{ra}, see \TClink{nɔ})

\TCheadword{nɔrachɛk} (comp. of \TClink{nɔra} (comp. of \TClink{nɔ}, \TClink[2]{ra}), \TClink{chɛk}, see \TClink{nɔ}) 

\TCheadword{nɔramda} (comp. of \TClink{nɔ}, \TClink[2]{ramil}, see \TClink{nɔ}) 

\TCheadword{nɔs} (Eng \textit{nurse}) \textit{n} nurse. \textit{Nɔs gbi ŋa ka cheni eriaio ai, hɔspitalai fli nɔs ka che ŋa ni.} There was no nurse in that whole area, even in the hospital there was no nurse.

\TCheadword{nɔsaa} \textit{n} palm wine tapster (K dialect). \textit{Nɔ̀sààɛ́ wɔ̀ bɛ́t bàchɛ̀ kà íbáá.} The tapster tapped the tree with a knife.

\TCheadword{nɔsanth} (comp. of \TClink{nɔ}, \TClink[2]{santh}, see \TClink{nɔ}) 

\TCheadword{nɔsɔnthɔ} (comp. of \TClink{nɔ}, \TClink[1]{sɔnth}, see \TClink{nɔ}) 

\TCheadword{nɔsukusɛkɛ} (comp. of \TClink{nɔ}, \TClink[1]{sukusɛkɛ}, see \TClink{nɔ}) 

\TCheadword{nɔth} \textit{adj} soft, tender. \textit{Yu lɛ kong puthul, lɛ ŋgbəŋ wɔ gbo hinɛ gbo nɔth.} The fish is rotten already, if you touch it, you will find it quite soft (\citealt{Pichl1967}).

\TCsubword{nɔthnɔth} (der.) \textit{adj} very soft.

\TCsubword{nɔthul} (der.) \textit{adj} very soft. \textit{Gbam dɛ kɔ cho gbïlɛ na lɛ kong nɔthul, kɔ kong lɔɔ.} The potato which you put (on) to roast is soft already, it is roasted (\citealt{Pichl1967}).

\TCheadword{nɔthi} (der. of \TClink{nɔ}, \TClink{thi-}, see \TClink{nɔ}) 

\TCheadword{nɔthikla} (comp. of \TClink{nɔ}, \TClink{thikla}, see \TClink{nɔ}) 

\TCheadword{nɔthnɔth} (der. of \TClink{nɔth}) 

\TCheadword{nɔthoŋka} (comp. of \TClink{nɔ}, \TClink[1]{thoŋka}, see \TClink{nɔ}) 

\TCheadword{nɔthul} (der. of \TClink{nɔth}) 

\TCheadword{nɔtolɔ} (comp. of \TClink{nɔ}, \TClink[1]{tol}, see \TClink{nɔ}) 

\TCheadword{nɔtɔ} \textit{cf}: \TClink{bolo}, \TClink{chocho}, \TClink{kɔŋko}, \TClink{suk}, \TClink{thoŋku}. \textit{n} shell.

\TCheadword{nɔtɔnɔ} (comp. of \TClink{nɔ}, \TClink[2]{tɔn}, see \TClink{nɔ}) 

\TCheadword{nɔwɔi} (comp. of \TClink{nɔ}, \TClink[1]{wɛi} (der. of \TClink[2]{wɛi}), see \TClink[2]{wɛi}) 

\TCheadword{nɔwu} (comp. of \TClink{nɔ}, \TClink[1]{wu}, see \TClink{nɔ}) 

\TCheadword{nɔyegbe} (comp. of \TClink{nɔ}, \TClink{yegbe}, see \TClink{yegbe}) 

\TCheadword{nɔyeŋkes} (comp. of \TClink{nɔ}, \TClink[1]{yeŋkes}, see \TClink{nɔ}) 

\TCheadword{nɔyes} (comp. of \TClink{nɔ}, \TClink[1]{ye}, see \TClink{nɔ}) 

\TCheadword{nɔyiɛnthiŋ} (comp. of \TClink{nɔ}, \TClink[1]{yi}, \TClink{thiŋ}, see \TClink{nɔ}) 

\TCheadword{nɔyiɛyibaw} (comp. of \TClink{nɔ}, \TClink[1]{yi}, \TClink{yibaw}, see \TClink{nɔ}) 

\TCheadword{nɔyilɔ} (comp. of \TClink{nɔ}, \TClink[1]{yil}, see \TClink{nɔ}) 

\TCheadword{Nra} \textit{nam} Ra, name given to a place. \textit{Wɔn pɔ gbem wɔ Nra ko.} She was born in Ra (village).

\TCheadword{nsaka-bunthul} (comp. of \TClink[1]{saaka}) 

\TCheadword[1]{nse} \textit{cf}: \TClink[2]{nse}, \TClink{sen}. \textit{adj} first. \textit{Kɛ gbemɔ nseiɛ primi, ye pɔ hɔ primiɛ vɛ, aagbemɔ landɛ kɔ kath.} But they say that giving birth first to a preemie is difficult. comp., der. \TClink{nɔnse} (see \TClink{nɔ})

\TCsubword{nser} (der.) \textit{n} first stage of farming after clearing and before felling the trees (\citealt{Pichl1967}). 

\TCheadword[2]{nse} \textit{cf}: \TClink[1]{nse}. \textit{temp} early. \textit{Ashiɛlɛ nkɔ pɛ Kiamp ko nsheɛ, so nwɔm yi len ŋa lan.} And I know you went to Freetown early on, so tell us something about that.

\TCheadword{nser} (der. of \TClink[1]{nse}) 

\TCheadword{Nsɔnwe} \textit{nam} Somwe, name given to a place. \textit{Yaa wɔ ka che sokonɔ Bondo, ɛn apima wɔ agbimɛ, apim ha ka che hɔth, Nsɔnwe ko.} Her mother was a Bondo leader, and the children she gave birth to, some were fishing in Somwe.

\TCheadword[1]{ntɛnt} (der. of \TClink{tɛnt}) 

\TCheadword[2]{ntɛnt} (der. of \TClink{tɛnt}) 

\TCheadword{Nthumba} \textit{nam} Mothumba, name given to a place. \textit{Nthumba ko, ntɛnt.} near Mothumba.

\TCheadword{ntɔɔli} \textit{disco} [ǹtɔ̀ɔ̀lí] sorry, expression of sympathy (B dialect).

\TCheadword{nu} \textit{cf}: \TClink{nuka}. \textit{v} fold. \textit{Pɔ gbaŋga wɔ bo pothɛ atok, pɔi nu bikɛ pɔ bim wɔ lɔ atok.} After he would be put in the ground, they would fold the mat, then they would put the corpse on it.

\TCheadword{nui} \textit{n} ear, [nuiɛ] the ear (K dialect). \textit{Wɔn nui bo̹mbo̹m.} His ears are very large (\citealt{Pichl1967}). \textit{Nɔɛ wɔ chal ha lɔŋ nui ko la pɔ hɔ ha yindɛ, bi ha theɛ lanɛ la biɛn ha pɛthil wɔɛ.} The person that sits listening to the gossip of others (lit. sits to set an ear to what people say) will hear that which displeases him (proverb) (\citealt{TISLL1979}). comp. \TClink{gbɛtnui} (see \TClink[2]{gbɛt}) 

\TCsubword{nuikel} (comp.) \textit{n} [núíkél] plant species, ear-monkey plant, leaves used for medicine (K dialect). 

\TCsubword{nuimɛn} (comp.) \textit{n} earlobe.

\TCsubword{lɔŋnui} (unspec.) \textit{cf}: \TClink[1]{si}, \TClink{the}. \textit{v} listen. \textit{Sɛkɛ, sɛkɛ we ŋa yɛ mɔ luŋnui konikowɛ.} Thank you very much for listening to us.

\TCheadword{nuik} \textit{v} amuse oneself. \textit{Lɛ ŋke yɛ amaaɛ ŋa koŋ nuik tɔn thiŋaɛ; haliwɔ yɛ ŋa tɔn dɛ, vɛ ŋa yeek bol thiŋaɛ.} If you see how the women amuse themselves with their songs; because when they sing, so do they dance with their heads.

\TCheadword{nuikel} (comp. of \TClink{nui}, \TClink[1]{kel}, see \TClink{nui}) 

\TCheadword{nuimɛn} (comp. of \TClink{nui}, \TClink[2]{mɛn}, see \TClink{nui}) 

\TCheadword{nuka} \textit{cf}: \TClink{nu}. \textit{n} elbow, [nukaɛ] the elbow (K dialect); \textit{nukaa}, \textit{nukraa} (hɔ̃/tha) elbow (\citealt{Pichl1967}).

\TCheadword{numu} \textit{n} (wɔ/hã, si) hippopotamus (\citealt{Pichl1967}).

\TCheadword{nuŋki} \textit{v} be virile (\citealt{Pichl1967}). 

\TCheadword{nuputha} \textit{v} mix. \textit{Apum ŋa nuputha mbana ndriɛ ni gbɛrɛ ha thóŋ bo.} Others mix ripe bananas with flour to fry.

\TCheadword{nus} \textit{n} Bondo mask.

\end{letter}
\begin{letter}{Ny}

\TCheadword[1]{nya} (der. of \TClink{nɔ})

\TCheadword[2]{nya} \textit{cf}: \TClink[1]{bɔk}, \TClink[1]{kɛk}, \TClink[2]{koŋ}. \textit{n} (wɔ/hã, si) turtle species, shell used to make finger rings that are believed to prevent drowning (\citealt{Pichl1967}). comp. \TClink{koŋkonya} (see \TClink{kɔŋko}) 

\TCheadword[3]{nya} \textit{cf}: \TClink{thubi}. \textit{adj} \textbf{1)} thin. \textit{Yay lɛ wɔ nya.} The cat is thin (\citealt{Pichl1967}). \textbf{2)} meager. comp. \TClink{nyamkoŋ} (see \TClink[4]{koŋ}) 

\TCheadword{nyabɔn} (comp. of \TClink[1]{nya} (der. of \TClink{nɔ}), \TClink[2]{bɔn}, see \TClink{nɔ}) 

\TCheadword{nyai} \textit{v} \textbf{1)} bring. \textbf{2)} fetch. \textit{Ayi kɔ ŋyai mɛndɛ ko yami, ayi ya ayi chɔŋ-chɔŋ.} And then I go fetch water for my mother, then I dish it out. comp. \TClink[1]{nyamban} (see \TClink[1]{ban}) 

\TCheadword[1]{nyam} \textit{n} (hɔ̃/-) poison (\citealt{Pichl1967}). \textit{Du gbokbo lɛ bi nyam.} The fins of the catfish are poisonous (\citealt{Pichl1967}). 

\TCheadword[2]{nyam} \textit{n} \textbf{1)} fear. \textbf{2)} horror. \textit{Liwu koŋ tuki inyam wɔ lɛ.} Death has lost its horror (\citealt{Pichl1967})

\TCheadword{nyama} (der. of \TClink[1]{nya} (der. of \TClink{nɔ}), \TClink{maa}, see \TClink{nɔ}) 

\TCheadword{Nyamaina} \textit{nam} Nyamaina, name given to a place. \textit{Lel ko, Nyamaina ko.} Over the river at Nyamaina.

\TCheadword{Nyamba} \textit{nam} Moyamba, name given to a place – refers to both a town and a district; Sherbro name for town more generally known as \textit{Moyamba} that has the Temne prefix \textit{mo-} (B dialect). \textit{Wɔ lɔ Nyambako.} She is there in Moyamba. \textit{Pɔ ni vel Yɛlaioɛ Planti koɛ, Bomp Thasɔ-Sheŋke, Pɔk Kagbɔɔɛ, Pɔk Nyambaɛ.} They now call it Plantain Island, Shenge Section, Kagboro Chiefdom, Moyamba District.

\TCheadword[1]{nyamban} (comp. of \TClink{nyai}, \TClink[1]{ban}, see \TClink[1]{ban}) 

\TCheadword[2]{nyamban} \textit{adj} rough. \textit{Kɛ be, kilikɛ ŋɔ ton ha bɔɔ yɛthi wɔm dɛ mmɛn nyamban dɛai huɛ vɛ.} But no, the anchor was (too) small to hold the canoe in the rough sea that day.

\TCheadword{nyamkoŋ} (comp. of \TClink[3]{nya}, \TClink[4]{koŋ}, see \TClink[4]{koŋ}) 

\TCheadword{nyanɔ} (comp. of \TClink[1]{nya} (der. of \TClink{nɔ}), \TClink{nɔ}, see \TClink{nɔ}) 

\TCheadword{nyaŋ} \textit{n} (wɔ/hã, si) fish species, ninebone (Elops lacerta) (\citealt{Pichl1967}). 

\TCheadword{nyaŋa} \textit{v} be fond of pleasure (\citealt{Pichl1967}). 

\TCheadword{nyaŋgbɛ} \textit{n} (wɔ/hã, si) mongoose (drwarf mongoose?), by my informants called “fox” (\citealt{Pichl1967}). 

\TCheadword{nyaŋktha} \textit{n} insect species, long stick-like legs, rarely seen, for some people its appearance a sign or warning (K dialect). 

\TCheadword{nyathi} \textit{v} \textbf{1)} lick. \textit{Lɛ siŋkɛ go thumɔs ta, wɔ mɔ yema nyathi sumɔhɔl.} If you play with a young dog, he will lick your mouth (proverb) (\citealt{Pichl1967}). \textbf{2)} lap.

\TCsubword{nyathia} (der.) \textit{v} be taken or caught by the \textit{Labɛŋ} devil. \textit{Labɛŋ dɛ nyathia wɔ.} The Labeng has caught him (\citealt{Pichl1967}).

\TCsubword{nyathini} (der.) \textit{v} lick. \textit{Pia sɛkil hɔ chen nyathini.} You cannot lick a dry hand (proverb). (Food is eaten with the hand. Since the hand gets gooey it is licked. If the hand has not had food in it, there is no reason to lick it.) (\citealt{TISLL1979}). 

\TCheadword{nyathia} (der. of \TClink{nyathi}) 

\TCheadword{nyathini} (der. of \TClink{nyathi}, \TClink{-ni}, see \TClink{nyathi}) 

\TCheadword{nye} \textit{disco} \textbf{1)} pan-West African confirmatory particle. \textit{Kaŋga kɔ kaŋ ŋa huŋgbemiɛ, nthela, nye?} To go and teach me how to deliver, you hear that, right? \textit{Laŋgbando akoŋ gbo pɔkɔni ilel wɔɛ, Sijismɔn, Sijismɔn wɔ ka che as bɛiyɛ, nthela, nye?} This man I've just forgotten his name, Sigismund, Sigismund was the chief, you hear that, right? \textit{Kɛ bamɔ ni yamɔ gbi ŋa koŋ wu, nye?} But your father and mother had died, right? \textbf{2)} what.

\TCheadword{Nyemɔ} (id. of \TClink[1]{nya} (der. of \TClink{nɔ}), \TClink[1]{yema}, see \TClink{nɔ}) 

\TCheadword{nyɛ} \textit{cf}: \TClink[1]{hoth}. \textit{iñɛ} \textit{n} (hɔ̃/-) palm nut chaff often dried and used as fuel (\citealt{Pichl1967}). 

\TCheadword{Nyɛkɛ} \textit{nam} Poro subgroup who are mainly concerned with soothsaying and healing, sometimes they practice bush washing (cleansing or purifying) (\citealt{Pichl1967}).

\TCheadword{nyɛnyɛ} (Mende \textit{nyɛnyɛ}) \textit{n} (hɔ̃/-) chicken pox (\citealt{Pichl1967}). 

\TCheadword{nyi} \textit{v} ? be cleft (\citealt{Pichl1967}).

\TCheadword{nyikith} \textit{n} intestinal worm, esp hookworm (\citealt{Pichl1967}). 

\TCheadword{nyin} (der. of \TClink{nɔ})

\TCheadword{nyina} (der. of \TClink{nyin} (der. of \TClink{nɔ}), see \TClink{nɔ}) 

\TCheadword{nyith} \textit{n} (kɔ/ma) vein, blood vessel (\citealt{Pichl1967}). 

\TCheadword{nyithi} \textit{v} feed (\citealt{Pichl1967}).

\TCheadword{Nyogbako} \textit{nam} Moyogba, name given to a place. \textit{Nyogbako lɔ pɔ gbem wɔ?} Is it in Moyogba that she was born?

\TCheadword{nyoŋkni} \textit{cf}: \TClink{vila}. \textit{v} shrink; wither. \textit{Kəfɛ lɛ kɔ nyonkni.} The peppers are shrinking (as they dry) (\citealt{Pichl1967}).

\TCheadword{Nyoro} \textit{nam} Nyoro, name given to a place. \textit{Aa, wɔnbɛ wɔɔ nyoroko, tiko bami, ha ha le kilɛ wɔl ko.} Yes, She herself is in Nyoro, my father's village, they are the ones she left in the house.

\TCheadword{nyɔhɔl} (comp. of \TClink[1]{ahɔl}) 

\TCheadword{nyɔŋkɔth} (unspec. of \TClink{nyuni}) 

\TCheadword{nyɔŋpɔ-nyɔŋpɔ} \textit{Idph} of softness. \textit{<Nyɔŋpɔ-nyɔŋpɔ> diɛn bɔk.} <Nyɔŋpɔ-nyɔŋpɔ> (The softness) of a tortoise does not kill it (proverb) (\citealt{TISLL1979}).

\TCheadword{nyuhul} (der. of \TClink[1]{nyuŋ}, \TClink{-ul}, see \TClink[1]{nyuŋ}) 

\TCheadword[1]{nyum} \textit{cf}: \TClink[1]{wu}. \textit{v} \textbf{1)} be extinguished; die out (fire) (\citealt{Pichl1967}). \textit{Ŋkɔ gbïl iwɔm dɛ lal l'ay kɔ, jɛmdi lɛ lɔ yema nyum.} Go put wood on the fire, the fire is about to go out (\citealt{Pichl1967}). \textbf{2)} close. \textit{Ñum thihɔl}. Close the eyes (\citealt{Pichl1967}). 

\TCsubword{nyumi} (der.) \textit{v} put out (the fire) (\citealt{Pichl1967}). \textit{Ŋkɔ nyumi jɛmdi lɛ.} Go put out the fire (\citealt{Pichl1967}). 

\TCheadword[2]{nyum} \textit{cf}: \TClink{biŋk}. \textit{n} \textit{iñum} (hɔ̃/-) blindness (\citealt{Pichl1967}). \textit{Nɔ iñum dɛ kong keche.} The blind man finally was able to see (\citealt{Pichl1967}). 

\TCheadword{nyumi} (der. of \TClink[1]{nyum}, \TClink[1]{-i}, see \TClink[1]{nyum}) 

\TCheadword{nyumpɔ} \textit{v} look at somebody with evil eyes; to wink scornfully (\citealt{Pichl1967}).

\TCheadword{nyun} \textit{v} drown. \textit{A ka bi pɛl kɔ a ka che yɔk hɛlɛ koɛ, kɛ iŋɛŋdɛ ka bɔnth mi lɔ yay ŋyun.} I had a net I used to go out with to sea, but the wind met me there and I almost drowned.

\TCheadword{nyuni} \textit{v} move. \textit{Kɛ haŋa pim nke ŋa ko nyuni, ŋa ye ma ni bɛ pɛ hɔ Mbolom.} But some people you see them move to other places, they do not even speak Sherbro anymore.

\TCsubword{nyɔŋkɔth} (unspec.) \textit{v} walk or dance in a stylish manner with a peculiar break and twist of neck and body (\citealt{Pichl1967}). \textit{Wantɛm dɛ wɔ ñanga, wɔ ñɔnkɔth lɛ wɔ gbo ye ɔ gbɛɛ.} The woman is fond of pleasure, she is stylish when dancing or walking (\citealt{Pichl1967}). 

\TCheadword[1]{nyuŋ} [wu] \textit{v} \textbf{1)} (v. \textit{ñuhul}) be blunt (\citealt{Pichl1967}). \textit{Kendi lɛ lɔ ñung.} The knife is blunt (\citealt{Pichl1967}). \textbf{2)} [wù] be dull, ‘dead' (B dialect); \textit{nyu} dull (\citealt{Sumner1921}). \textit{Ká kɔ́ chènì wù.} The hoe is not dull/dead.

\TCsubword{nyuhul} (der.) \textit{v} be blunt. \textit{Kendi lɛ ñuhul lɛ.} The blunt knife (\citealt{Pichl1967}).

\TCheadword[2]{nyuŋ} \textit{v} \textbf{1)} capsize. \textit{Braima wɔe tɛnthil ni keɛ mmɛn dɛ yema bɛ pɛr wɔm dɛ thiiŋ mɛŋkoki, ni ŋɔ yema nyuŋ.} Braima wakes and sees the water is about to fill the canoe full this time, and it will capsize. \textbf{2)} drown.

\TCheadword{nyɛŋkin} \textit{cf}: \TClink{kisik}. \textit{temp} in the end; finally. \textit{Nyənkin dɛ hɔbatokɛ bɛmpa nɔthi.} At the end God made man (\citealt{Pichl1967}). 


\end{letter}
\begin{letter}{Ŋ}

\TCheadword{-ŋ} \textit{cf}: \TClink{-ni}, \TClink[2]{-n}. \textit{sfx} reflex of mid? der. \TClink{thoiŋ} (see \TClink{thoi})

\TCheadword[1]{ŋa} \textit{pers} \textbf{1)} they; them, 3\textsc{pl} (ha class). \textit{Ŋa kaŋ Mbolomdɛ.} They were learning Bolom. \textit{Ŋan bɛ lɛ lagbandɛ wɔ gbo hun nɛn veleŋ ni ŋan bɛ ŋa shiɛ lɛ ahin ŋa lɔ ka ŋa ŋan.} When the man comes next year, let them know there are people here for them. \textit{Aŋa pum ŋa mɔ mith we.} Some people will hate you. \textit{Pɛnte maiɛ ŋa lɔ we; ŋa koi piŋiɛni.} Our brothers are all there; they have turned against us. \textit{Ŋa bia the la, labi imɔ le yiyɛ labo nyema la.} They would have to hear it, that is why we are asking for your permission. \textbf{2)} their, 3\textsc{pl} (ha class). \textit{Pomdɛ pɛntewɔ bɛndɛ wɔɛ ba bullɛ, kɛ ya ŋa ŋa ka che li thɛmko.} My husband, it is his elder brother of the same father, but their mothers were mates. \textit{Ŋa bi peŋa ŋan sui o.} They have guns in their hands. \textit{Neshɔn ŋaɛ ŋɔ ŋa lɔ theli Nthemdɛ.} It is their language (nation) that they speak there – Themne. \textbf{3)} those (ones), Rel 3\textsc{pl} (ha class). \textit{Nrokɛ, nrekiaɛ, ŋanɛ ŋa bia kɔ hundɛ.} The grandchildren, the great-grandchildren, those that are going to come. \textit{Pɛlɛ bɛ, haŋaɛ kuthai gbo, hanɛ ha han nchɛkɛ han ha kuthaɛ.} Even rice, indeed let them plow, those that make a farm must plow. \textit{I lɛŋ ŋanɛ vɛ ŋa mɛndɛ veleŋkoɛ, ŋanɛ ha sihindɛ, ŋanɛ ŋasihiŋɛ.} We are sending greetings to the ones that are behind the water, the ones that do not know us. \textbf{4)} you (pl). \textit{Laŋgba landɛ koŋ pa hu, wɔi hun wɔ ŋai hun hɔm lɛ ŋa ma blem wanthɛm dɛ vɛo.} The person is dead, he came and he told them that you should not blame that woman \textit{Ŋan awɔ ŋa gbemda?} How many of you did she give birth to? \textit{Awɔ ŋani wɔi ka?} How many of you are alive? \textbf{5)} your (pl). 

\TCheadword[2]{ŋa} \textit{cf}: \TClink[2]{bi}, \TClink[2]{ha}, \TClink[3]{lɔi}, \TClink[1]{ma}, \TClink{mɔs}. \textit{Aux} \textbf{1)} subordinating modal. \textit{A yiyɛ Bahin ŋa toŋi mi nai wɛ we.} I ask the Lord to show me the way. \textit{La hini ha ŋa sɔthɔ hini-gbɔl?} What must we do to have peace of mind? \textit{Nɔ gbi sini mɛŋkɛ ŋɔ bahin bi ŋa hun.} No one knows when our father is going to come. \textbf{2)} should. \textit{I bo ŋa ka ha limani.} We just need to give them respect. \textit{Iŋa tɔnk wɔ wɛ yo wɛ.} We should pray to you every day. \textit{Mɔ ŋa koi ndumma mɔe ma pɔ dumɔ mɔi.} You should take the character you were raised up with.

\TCheadword[3]{ŋa} \textit{cf}: \TClink[5]{che}, \TClink[3]{hɔ}, \TClink[2]{la}, \TClink[2]{lɛ}, \TClink[4]{ni}, \TClink[1]{yɛ}. \textit{subordconn} \textbf{1)} how. \textit{Ina toŋgiɛ mɔ ŋa tɔnda?} Who taught you how to sing? \textbf{2)} that. \textit{Paŋdo ki ŋɔ chaŋ paoɛ Januari.} That month that is just past January. \textbf{3)} to. \textit{Ahun yi nɔmaɛ ki ŋa lemɛ mi jali wɔ atokɛ.} I am coming to ask this woman about herself. \textit{Yaŋ Abdulai Bɛndu, nandɛ ako vel laŋgbaŋ bul ŋa hun wɔ yi ŋalwɔ atokɛ.} I, Abdulai Bendu, today have called on a man to come to ask him about himself. \textbf{4)} for. \textit{ŋa yaŋ toŋgi ŋɔ pɔ yuk pɛlɛ.} For me to show how to plant rice.

\TCheadword[4]{ŋa} \textit{cf}: \TClink[1]{ndɔndɔ} (der. of \TClink{ndɔ}), \TClink{nɔonɔ} (der. of \TClink{nɔ}, \TClink{-o-}), \TClink[2]{pɛ}. \textit{indfpro} \textbf{1)} one. \textit{Nɔ shini che ko labi yendɛ yɛ mɔ la ŋa ncheyi ni nshila thiyen, ni la saŋ mɔ ntenɛ.} One does not know the future that is why when doing something you should ask so you can know it and understand it better. \textbf{2)} somebody; someone. \textit{I taŋ ŋa loli benɔ ŋa bɔnth.} We cry for rescue, no one to help. \textbf{3)} anybody; anyone \textit{Nɔ halɛ wɔe hɔɛ, “Bami, yaŋ bɛ ya theeɛ la bɛlsɛ hɔɛɛ, kɛ pɔ chen laanɛ nɔ ka kakeiŋ.} One person then said, “Mister, I, too, heard what the rats said, but they will not believe anybody else.”

\TCheadword[5]{ŋa} \textit{cf}: \TClink{handɔ}, \TClink[5]{hɔ}, \TClink[1]{la}, \TClink{ndɔ}. \textit{interrog} what. \textit{Mi, ŋa mɔ ilel la?} Mami, what's your name? \textit{Lɛ ŋa yema bo won lɛŋ ko ŋanɛ ha hunɔn muɛ, ko nrekiaɛ ŋa pɔ gbemɛn muɛ?} What greeting would you want to send to those that have not come yet, the grandchildren, those that have not been born yet?

\TCheadword{ŋaiŋai} \textit{cf}: \TClink{teŋ}. \textit{adj} sour (\citealt{Pichl1967}). \textit{Rokos lɛ kɔ ŋayŋay.} The lime is sour (\citealt{Pichl1967}).

\TCheadword[1]{ŋal} \textit{n} (kɔ/ma) grass species, elephant grass (?) (\citealt{Pichl1967}). 

\TCheadword[2]{ŋal} \textit{cf}: \TClink[1]{ha}, \TClink{tɔkɔ}. \textit{prep} about. \textit{Ibi jaa ki la iŋaɛ, ŋa hun mɔ koi, lomɔɛ, yen-o-yen ŋal mɔ.} We have this thing we are doing, to come and take you, your voice, everything about you. \textit{Nande a ko vel laŋgba bul ni ŋa hun wɔ yi ŋalwɔ atokɛ.} Today I have called on a man to come, to ask him about himself.

\TCheadword[1]{ŋala} \textit{cf}: \TClink[2]{muyu}. \textit{n} patience. \textit{Mɔ́ŋá ŋálá wàì!} Be patient!

\TCheadword[2]{ŋala} \textit{Loc} here. \textit{Kàá kó ŋàlà; kàá kó lɔ̀kò.} The hoe is here; the hoe is there.

\TCheadword{ŋɛ} \textit{cf}: \TClink{kendɛ}, \TClink[4]{ken}, \TClink[5]{ni}. \textit{prep} like. \textit{Hɔ hani ki, hɔ chaini fli ŋɛ chanthɛ.} Make like this, it rises up again like a baby.

\TCheadword{ŋɛi} \textit{cf}: \TClink{mɛmilni} (unspec. of \TClink{mɛmi}, \TClink{-ni}). \textit{v} open one's mouth, separate one's teeth, different from \textit{mɛmilni} ‘smile' (B dialect); \textit{ŋyẽy} show the teeth, grin, smile (\citealt{Pichl1967}). \textit{Nɔ chen ŋyẽy thanthɛn, pum ke̹ ja kɛlɛng ɔ the ikɛlɛng wɔ hunɛ hã wɔn, là bi ni che mɛmilni.} One does not smile for nothing, perhaps he sees something good or hears of some good news in store for him, hence he smiles (\citealt{Pichl1967}). 

\TCheadword{Ŋɡabe} \textit{nam} Ngabe, name given to a person.

\TCheadword{Ŋgamaŋga} \textit{nam} Ngamanga, name given to a person.

\TCheadword{Ŋgasumana} \textit{nam} Mokainsumana, name given to a place. \textit{Ka lɔ pɔ bɛ bia huŋa sakaɛ, lel ko, Ŋgasumana ko.} It is here that they have to come and make his sacrifice (tithe), at Mokainsumana.

\TCheadword{ŋgbelŋgbel} \textit{cf}: \TClink[2]{lanthgbɔl}. \textit{adj} anxious. \textit{Wɔɛ che ŋgbel-ŋgbel ha kɔ lɛli pɛl dukiɛɛ kɔ, kɔ chencha lɔɔliɛ.} He was very anxious to look at the lego (fishing) chain he had seen yesterday.

\TCheadword{Ŋgendema} \textit{nam} Gendema, name given to a place. \textit{Ka koŋ che Ŋgendema ko.} He had been (had lived) at Gendema.

\TCheadword{Ŋgewa} \textit{nam} Ngewa, name given by Yase Society (\citealt{Pichl1967}). 

\TCheadword{Ŋgobɛ} \textit{nam} Ngobe, name given to a person. \textit{A lomani yɛ Ba Ngoba ka che hun dɛ hwɛ lɛ hɔ̃ le̹l̦ɛ.} I remember when Mr. Ngobe was coming that it rained (\citealt{Pichl1967}).

\TCheadword{ŋgɔ} \textit{nam} \textbf{1)} Auntie. \textit{Yami gbem ara; Ŋgɔ Mɛmuna wɔi gbemɔ atiŋ.} My mother gave birth to three; Aunty Memuna gave birth to two. \textbf{2)} older sibling. \textbf{3)} term of address, title.

\TCheadword{Ŋgubɛ} \textit{nam} Ngube, name given to a person. \textit{To̦k lɛ kɔ pɛn parɛ hwɛ lɛ hɔ̃ ba Ngubɛ wuɛ.} The thunder cracked the other day, they say it was (when) Mr. Ngube died (\citealt{Pichl1967}). 



\TCheadword{ŋhie} \textit{coordconn} so; hence. \textit{Bɛl Maaɛ wɔe hɔ ko poo wɔɛ, “M-m-m, ŋhie ŋhɔɛ chen kɔ?”} Rat Wife said to her husband, “Hm-m-m, so you say you are not going?”

\TCheadword{ŋhɔbɛ} \textit{subordconn} even if. \textit{Ŋhɔbɛ ilema hɔ haŋ wɔyɛ pi ima lɔ be nwɔk pika gbi, acheŋ ke gbi.} Even if we keep speaking it until nightfall using no other language, I would not get tired.

\TCheadword{Ŋkatha} \textit{nam} Katha, name given to a place. \textit{Kɛ kɔo ki bɛ mmɔi gbo ŋkatha ko, wɔmthɛ tha ko tipɛ tik hin isɔ loki bɛ.} Just now if you reach Katha, the boats have started coming to my village this early morning.

\TCheadword{Ŋkenikoɛ} \textit{nam} Makeni, name given to a place. \textit{Pɔ gbem mi pɔkɛ lɔɔ Ŋkenikoɛ.} I was born in Makeni.

\TCheadword{Ŋkɔŋbɛti} \textit{nam} Mokornbeti, name given to a place. \textit{Ŋkɔŋbɛti ko.} At Mokornbeti.

\TCheadword{ŋɔhɔlpok} (der. of, comp. of \TClink{nɔ}, \TClink[1]{hɔ}, \TClink{pokan} (unspec. of \TClink[5]{po}), see \TClink{nɔ}) 

\TCheadword{ŋɔi} \textit{cf}: \TClink{mɛmin} (der. of \TClink{mɛmi}). \textit{n} \textbf{1)} gladness. \textit{Hi ma lemil inui hɔlɔɛ, hai bi na nɔloliɛ.} Let us not follow the gladness of the world and we should have got a saviour. \textit{Ke ko gbɛ nai arijana lɔ wɔ che iŋɔi bomai.} He has walked the heaven road where we will be with gladness. \textbf{2)} joy. \textit{Kɔnɛ ka hin iŋuɛ gbɔliai yai.} Please give us joy in our hearts. \textbf{3)} merriment. \textit{Tɛmdɛ vɛ ŋɔ ha yindɛ ha ŋa iŋɔi...} When people will celebrate...



\end{letter}

\begin{letter}{O}

\TCheadword[1]{o} \textit{cf}: \TClink[2]{-i}, \TClink{ɛn}, \TClink[1]{kɛ}, \TClink[4]{la}, \TClink{ɔ}. \textit{coordconn} \textbf{1)} and (if repeated several times). \textbf{2)} either; or. \textit{o … o}; neither … nor. \textit{Si la vɛ o si la chen vɛ o, a sini.} Whether it is so or not, I don't know (\citealt{Pichl1967}). \textit{Tano lɛ o ya wo o hã kani.} Neither the boy nor his mother went (\citealt{Pichl1967}). 

\TCheadword[2]{o} \textit{disco} oh. \textit{Oo, Bahin, lahi cha ba ha ba?} Oh, Our Father, what are we doing? \textit{O, n ka che siŋ bɔllɛ?} Oh, you used to play?

\TCheadword[3]{o} \textit{interj} emphatic particle. \textit{Velia mi we Jizɔs velia mi yo.} Rescue me, Jesus, redeem me o! \textit{La Bahin ko ŋa ha yan dɛ oo.} What our Father has done for us-o.

\TCheadword{-o-} \textit{n > ???} \textit{ifx} Distributive, reduplication coordinating particle. \textit{Iyema mɔ wɛyowɛ.} We need you everyday. der. \TClink{lenolen} (see \TClink{len}), \TClink{lɔkɔolɔkɔ} (see \TClink{lɔkɔ}), \TClink{nɔonɔ} (see \TClink{nɔ}), \TClink[1]{tɛmotɛm} (see \TClink[1]{tɛm}), \TClink[2]{tɛmotɛm} (see \TClink[1]{tɛm}), \TClink{yenoyen} (see \TClink[1]{yen}) 

\TCsubword{wɔiowɔi} (der.) \textit{cf}: \TClink{lɔkɔolɔkɔ} (der. of \TClink{lɔkɔ}, \TClink{-o-}). \textit{temp} everyday. \textit{La mɔ tɛniɛn wɔiyowɔ ɛ?} What are you thinking everyday? \textit{Mɔ ya wɔiowɔi?} You cook everyday? \textit{A sɔthɔ gbo aya wɔiowɔi, a sɔthɔni gbo, ai bya ŋa wɔi ŋallɛ.} If I have (something) everyday, I cook; if I do not, I am patient for another day. \textit{Iyema mɔ wɛyowɛ.} We need you everyday.

\TCheadword{ogiri} \textit{n} a flavoring made of fermented oil seeds (Wikipedia. \textit{Nkoŋ gbo, labo nbi ogiri mɔ hɔ lɔi bɛ, ŋɔi yel yeŋkɛlɛŋ.} When you have finished, if you have the ogiri you put it in, then boil it properly. \textit{Nsɔthɔni gbo ŋgɛtiɛ mɔi bɛ ogiɛ.} If you do not have groundnut, you put in ogiri.

\TCheadword{oke} (Eng \textit{okay}) \textit{cf}: \TClink{awa}, \TClink{ayo}. \textit{disco} okay. \textit{Oke, wɔ nɔ ntɛnt ka?} Okay, is she near you here? \textit{Ɔke, mi, sɛkɛ, sɛkɛwe, Abatokɛ ŋɔ chema m.} Okay, Ma, thank you very much, may God be with you. \textit{Oke, mɔm pɛ sɛkɛo.} Okay, thanks to you once again.

\TCheadword{Omɛga} \textit{nam} Omega. \textit{Aaa, Bahin mɔ, mɔlɔ Alfa ni Omɛga.} Yes, Lord, you are the Alpha and Omega.

\TCheadword{osi} \textit{nam} Officer in Charge (OC). \textit{Aa ha ka che theli Mbolomdɛ, wɔnɛ fli ka che osi pɔlis, Hestins.} Yes, they used to speak Sherbro, even the one who was OC Police, Hastings.


\end{letter}
\begin{letter}{Ɔ}

\TCheadword{ɔ} \textit{cf}: \TClink[4]{la}, \TClink[1]{o}. \textit{coordconn} or. \textit{Pɔ koŋ gbo kutha, pɔi chi pɛlɛ ken bushɛl libul ɔ litiŋ ɔ limɛn bɛ ɔ waŋ bɛ.} After the plowing, they would have to bring the rice, like one or two bushels or five or even ten. \textit{Wɔ mu wɔɛ ɔ cheni pɛ wɔɛ?} Is he still alive or not? \textit{Ŋa wɔɛ, Mbɛkɛ ma pɔ chan theli ɔ Mbolomdɛ?} Per day, is it Krio they speak more or Sherbro?

\TCheadword{ɔf} (Eng \textit{of}) \textit{prep} of. \textit{Naintin fɔti tu fɔst ɔf Januari.} 1942, first of January. \textit{Tɛm landɛ ejimdɛ ŋɔ ej ɔf fɔti sɛvin yias.} At that time, I was 47 years old.

\TCheadword{ɔnfɔtinetli} (Eng \textit{unfortunately}) \textit{adv} unfortunately. \textit{Ɔnfɔtinetli yai gbem hin waŋ ni tin, ile bo hina tiŋ.} Unfortunately our mother gave birth to twelve, only two of us remain.

\TCheadword{ɔrijin} (Eng \textit{origin}) \textit{n} origin. \textit{Koromanɔ aida ɔrijin wɔɛ wɔ Maninkanɔ... che Themnɔ wɔɛ.} Koroma, either the origin is Maninka... it is not Themne. 

\TCheadword{ɔrijinali} (Eng \textit{originally}) \textit{adv} originally. \textit{Kɛ ɔrijinali ŋan ŋa Kamara, Sise, dis, dat.} But originally they were Kamara, Sesay, this, that.

\TCheadword{ɔpreshɔn} (Eng \textit{operation}) \textit{n} operation. \textit{Velen thilandɛ dɔktaɛ wɔ ka ŋa wɔ ɔpreshɔndɛ ka hun.} After all that, the doctor that did his operation came.

\end{letter}
\begin{letter}{P}

\TCheadword[1]{pa} \textit{cf}: \TClink[5]{ka}, \TClink[2]{na}. \textit{temp} in the past. \textit{Ŋa wɔ pa ŋa chi bɔnth, bɔnthɛo ike kɔni, nke.} They said previously that they would bring help, (but) we have not seen help, you see. \textit{Atiŋdɛ ŋa kɔ skullai bullɛ wɔn chepa kɔ skul kɛ chen pɛ kɔ.} The two are going to school, the one was going to school but he does not go anymore. \textit{Be, apa ni loman ja ŋɔth.} No, I do not remember knowing how to fish. \textit{Tɛm lan ikɔlɔ bɛ pa, bikɔs kil hinyɛ ŋɔ fɛtɛni bo.} Even that time we went there, because our house is just close. \textit{Ikɔlɔ paɛ.} In the past, we went there. \textit{Anyaɛ nke si nakɛ kɔ koni pa wun pɔkaiɛ…} The people, you see, understand in the past the sickness (ebola) had come into the country…

\TCheadword[2]{pa} \textit{cf}: \TClink{thotho}. \textit{n} large sore that takes a long time to heal, as opposed to a \textit{thotho} ‘a small sore' (K dialect). 

\TCheadword{paa} \textit{v} protect. \textit{Mpaa mi hink nak.} Protect me from sickness (\citealt{Pichl1967}). comp. \TClink{nɔpaa} (see \TClink{nɔ}) 

\TCheadword{pabondɛ} \textit{cf}: \TClink{lagbo} (comp. of \TClink[2]{la}, \TClink[1]{gbo}), \TClink[2]{lɛ}, \TClink[2]{si}, \TClink[1]{yɛ}. \textit{subordconn} if. \textit{Pabondɛ fli ni ŋɔ rɛdi ha hun, hɛ hɔ ha ni ki.} If it is really ready to come out, it will make like this. \textit{Paali pagbondɛ akɔni pɔiko, ale sɛmi kɛmdɛ akoŋ kɔni ale kɔ siŋɛ.} The whole day, if I go to fetch water, I will leave the bucket then I go play.

\TCheadword[1]{pak} \textit{n} bone, [pàk]/[pàkthɛ́] bone/the bones (B dialect); (hɔ̃/tha) bone (\citealt{Pichl1967}). \textit{Wɔ tɔth pak lɛ}. He sucks (the marrow out of) the bone (\citealt{Pichl1967}).

\TCheadword[2]{pak} (Eng \textit{park}) \textit{n} park.

\TCheadword[3]{pak} (Eng \textit{park}) \textit{v} park. \textit{Pɔ koŋ gbo rɔk, pɔi pak.} After havesting, they will then park the rice. \textit{Iwoɛ, iwo itataɛ pɔ ŋɔ pak ayen, pɔ ŋɔ pɛ bia buŋ.} The rice grass stalks, the immature stalks are parked somewhere, people thresh them again. 

\TCheadword[4]{pak} \textit{v} shake.

\TCheadword[1]{paka} \textit{n} \textbf{1)} reward. \textbf{2)} payment. \textbf{3)} strength. \textit{Ni ŋɔ chaŋ wɔ thipakaɛ.} And it was more than his strength (could handle). 

\TCheadword[2]{paka} (Port \textit{pagar} ‘pay') \textit{cf}: \TClink[6]{kɔ}, \TClink[1]{pin}. \textit{v} \textbf{1)} pay. \textit{Ni ŋa pa[ka] thunɔ waaŋmaaɛ huɛ bullɛ vɛ gbi.} And they paid the dowry at once. \textbf{2)} repay. \textit{Ye sɔlɛmaɛ yɛ mɔ chai iroɛ, mbɔni ha paka ŋɔ?} What a hassle (it is) when you borrow something and you cannot pay it back.

\TCheadword[3]{paka} \textit{n} spine; backbone.

\TCheadword{pakai} \textit{n} papaya; pawpaw. \textit{Ŋkɔm lɛnthiɛ nrokos ntiŋ ni mpakai nhiɔl!} Go pluck me two oranges and four papayas (\citealt{Pichl1967}).

\TCheadword{pakali} (der. of \TClink{pakil}, \TClink[1]{-i}, see \TClink{pakil}) 

\TCheadword{pakil} \textit{cf}: \TClink{pakni}, \TClink{yikitha}. \textit{v} tremble. \textit{Ni Braima chal ŋɔ kunɛ ni che pakil.} With Braima sitting inside it and trembling.

\TCsubword{pakali} (der.) \textit{cf}: \TClink{hothɔk}, \TClink{jɔhɔ}, \TClink{sɔyɛ}, \TClink{woli} (der. of \TClink[1]{woi}, \TClink[1]{-i}). \textit{v} \textbf{1)} scare. \textbf{2)} make shake. \textit{Nha yenkəlɛŋ thɔk lɛ tok ɛ, mma pakali lɛɛ thɔk lɛ thɔm mɔ lɛ ma ki duk.} Be careful you there up in the tree, don't make the tree branch shake lest your companion fall (\citealt{Pichl1967}). 

\TCheadword{pakni} \textit{cf}: \TClink{pakil}, \TClink{yikitha}. \textit{v} tremble.

\TCheadword[1]{pal} \textit{n} \textbf{1)} [pàl] sun (B dialect). \textit{Palli lɛ yema duk.} the sun is about to set (\citealt{Pichl1967}). \textbf{2)} day. \textit{Pal thipaŋ dɛ, mɛŋk hiɔl-lɛ yɛ pɔ koŋ hok saka jajɛl wɔɛ.} Four days later, this man left the ceremony for his mother-in-law. \textbf{3)} midday; noon. \textit{Gbeng ipal.} Tomorrow noon (\citealt{Pichl1967}). \textit{A chen che ka gbəng ipal; lɛ nyema-m gbo bɔnthi gbəng boa.} I shall not be here tomorrow at midday; if you want to meet me, come early tomorrow (\citealt{Pichl1967}). comp. \TClink{babalipal} (see \TClink[2]{baba}), unspec. \TClink{buŋklipal} (see \TClink[1]{li-}) 

\TCsubword{palli-chɛthɛ} (comp.) \textit{temp} at sunset (\citealt{Pichl1967}).

\TCsubword{palli-kasa-bul} (comp.) \textit{temp} afternoon, to about 4 o'clock (\citealt{Pichl1967}). 

\TCsubword{palli-paŋ} (comp.) \textit{temp} toward evening (\citealt{Pichl1967}).

\TCsubword{ipal} (der.) \textit{temp} \textbf{1)} during the day. \textbf{2)} [ìpàl] in the afternoon, afternoon (B dialect). 

\TCsubword{pali} (der.) \textit{temp} whole day (B dialect); \textit{paali} (hɔ̃/-) the whole day (\citealt{Pichl1967}). \textit{...paliioki tɛmpim te ki et-o-klɔk ichɔl wɔni huŋ gbemɔ.} ...the whole day, sometimes (not) until eight o'clock in the evening before giving birth. \textit{Pi kəbe̹l ko pali lo.} He was on the farm the whole of today (\citealt{Pichl1967}). der. \TClink{paalio} (see \TClink[1]{pal}) 

\TCsubword{paalio} (der.), (der. of \TClink{pali}) \textit{temp} the whole of today (\citealt{Pichl1967}). 

\TCsubword{palpal} (der.) \textit{temp} noon. \textit{Nduɛ waŋnimɛŋtiŋ dɛ, palpal lɛ, mɛŋkɛ ŋɔn waŋnibul lɛ.} The seventeenth day, noon, the eleventh hour (\citealt{Pichl1967}).

\TCsubword{parɛ} (der.) \textit{cf}: \TClink[2]{kan}. \textit{temp} \textbf{1)} other day. \textit{Pə hɔmɔ-m parɛ lɛ ŋkɔ vɛthiɛ Themdel ko ni Krim ko.} I was told the other day you went to Timdale and Krim some time ago (\citealt{Pichl1967}). \textit{Tok lɛ kɔ pɛn parɛ hwɛ lɛ hɔ ba Ngubɛ wuɛ.} The thunder cracked the other day, they say it was (when) Mr. Ngube died (\citealt{Pichl1967}). \textbf{2)} recently. \textit{Boon dɛ kɔ che parɛ Furabee Kɔlɛj kɔ koŋ sẽyni.} The meeting which was recently at Fourah Bay College has dispersed (\citealt{Pichl1967}). 

\TCsubword{chɛtlipalkɔ} (unspec.) \textit{cf}: \TClink[3]{lɔ}. \textit{Loc} west.

\TCheadword[2]{pal} (Eng \textit{pearl}) \textit{n} (wɔ/hã, N) pearl (of oysters) (\citealt{Pichl1967}).

\TCheadword[3]{pal} \textit{cf}: \TClink[2]{gbit}. \textit{n} (hɔ̃/tha) pole of fishing net or chain (\citealt{Pichl1967}).

\TCsubword{palpɛl} (comp.) \textit{n} (hɔ̃/tha) any kind of pole of fishing net or chain (\citealt{Pichl1967})

\TCsubword{palta} (comp.) \textit{cf}: \TClink{palthon} (comp. of \TClink[3]{pal}, \TClink[1]{ton}). \textit{n} (hɔ̃/tha) inner and smaller pole of fishing net (\citealt{Pichl1967}).

\TCsubword{palthon} (comp.) \textit{cf}: \TClink{palta} (comp. of \TClink[3]{pal}, \TClink{taa}). \textit{n} (hɔ̃/tha) inner and smaller pole of fishing net (\citealt{Pichl1967}).

\TCsubword{palbom} (der.) \textit{n} (hɔ̃/tha) big outer pole of fishing net or chain (\citealt{Pichl1967}). 

\TCheadword{palbom} (der. of \TClink[3]{pal}, \TClink{bom}, see \TClink[3]{pal}) 

\TCheadword{palemɛnt} (Eng \textit{parliament}) \textit{nam} Parliament. \textit{Ŋkɔ kïl pale̹mɛnt lɛ ni nlɔng-nui.} Go to the parliament house and listen (carefully) (\citealt{Pichl1967}). 

\TCheadword{palɛ} \textit{cf}: \TClink{vɛthiɛlɛ} (unspec. of \TClink{vei}). \textit{temp} \textbf{1)} 3+ days ago. \textit{Mbo̹lo̹m ŋwɛi ma che paalɛ bai ko, anya atïŋ dɛ hã lo̹l.} In the bad case that was recently before the court, the two men were set free (\citealt{Pichl1967}). \textbf{2)} [paaɛ] after two to four weeks (K dialect). \textit{Làŋgbàɛ́ kɔ̀ bə́mə́k, làŋgbàɛ́ ché pàɛ̀ kə́ gbér, kɛ́, yɛ̀làìò kóŋ bə́mə́k.} The man is blind, the man once was seeing well, but now he is blind. \textit{Nɔ̀ mí chéŋk wɔ che paaɛ, kɛ yɛ laio, chɔ́ŋ mì lèn.} He hated me some time ago, but now he likes me.

\TCheadword{pali} (der. of \TClink[1]{pal})

\TCheadword{palli-chɛthɛ} (comp. of \TClink[1]{pal}, \TClink{kɛth}, see \TClink[1]{pal}) 

\TCheadword{palli-kasa-bul} (comp. of \TClink[1]{pal}, \TClink[3]{bul}, see \TClink[1]{pal}) 

\TCheadword{palli-paŋ} (comp. of \TClink[1]{pal}, \TClink[3]{paŋ}, see \TClink[1]{pal}) 

\TCheadword{paalio} (der. of \TClink{pali} (der. of \TClink[1]{pal}), see \TClink[1]{pal}) 

\TCheadword{palpal} (der. of \TClink[1]{pal}) 

\TCheadword{palpɛl} (comp. of \TClink[3]{pal}, \TClink[2]{pɛl}, see \TClink[3]{pal}) 

\TCheadword{palta} (comp. of \TClink[3]{pal}, \TClink{taa}, see \TClink[3]{pal}) 

\TCheadword{palthon} (comp. of \TClink[3]{pal}, \TClink[1]{ton}, see \TClink[3]{pal}) 

\TCheadword{pamishɔn} (Eng \textit{permission}) \textit{cf}: \TClink[2]{yema}. \textit{n} permission. \textit{Aa ŋa le koiɛ pamishɔn.} I should get your permission first.

\TCheadword[1]{pampa} \textit{cf}: \TClink{bot}, \TClink[2]{wɔm}. \textit{n} \textbf{1)} [pámpá] general name for boat (K dialect). \textbf{2)} launch (K dialect). 

\TCheadword[2]{pampa} \textit{n} [pàmpà] tree species, type of lily tree, grows by river, leaves used for making mats (K dialect). 

\TCheadword{pan} (Eng \textit{pan}) \textit{n} pan. \textit{Ya dikil panthe, panthe bɛnbɛndɛ.} I gather the pans, the old old pans.

\TCheadword[1]{panth} \textit{cf}: \TClink[1]{bulɔ}, \TClink[1]{ja}. \textit{n} \textbf{1)} work. \textit{Pánthɛ̀ mà dìs / Mà dìsìl [dəsəl].} The work is heavy. \textit{Lɔkɔɔlɔkɔ hɔ ya hun dɛ, ya bɔnth wɔ ha mpanth.} Always when I come, I meet him at work (\citealt{Pichl1967}). \textit{Thetha mi ka che ŋa mpanth ma landɛ pɛŋ bifo wɔ mmu hu.} My grandmother used to do the work before she died. \textbf{2)} job. comp. \TClink{nɔhampanth} (see \TClink{nɔ})

\TCsubword{mpanth-o-mpanth} (der.) \textit{cf}: \TClink{yenoyen}. \textit{n} any work. \textit{Planti ka, mpanth handɔ, ma ayindɛ ŋaa ma chaŋ, ŋa la chaŋ mpanth-o-mpanth a?} On Plantain [Island] here, what work do people do more, that is more than any other job? 

\TCsubword[2]{panth} (der.) \textit{cf}: \TClink{sik}. \textit{v} \textbf{1)} [pánth] tie (K dialect). \textit{I huni ko ja gbisiŋdɛ, yɛ pɔ panth li thɛmdɛ, ŋɔ nkela ja kache ɛ ni kenɛkiɛ?} Let us now come to the tying of love (i.e., marriage), how they used to engage couples, what was it like in the past, and nowadays? \textit{Ŋ kɔ panth dik iwɔm dɛ.} Go tie the bundle of wood (\citealt{Pichl1967}). \textbf{2)} bind. der. \TClink{panthini} (see \TClink[1]{panth})

\TCsubword{panthini} (der.), (der. of \TClink[2]{panth}) \textit{v} tie. \textit{Ndik ma chen panthini lithɛmba.} Hunger does not bind friendship.

\TCheadword[2]{panth} (der. of \TClink[1]{panth})

\TCheadword{panthini} (der. of \TClink[2]{panth} (der. of \TClink[1]{panth}), see \TClink[1]{panth}) 

\TCheadword{Panya} (Eng \textit{Spaniard}) \textit{n} Spaniard.

\TCheadword[1]{paŋ} \textit{cf}: \TClink{pemple}. \textit{n} fishing method for jumping fish on the mudbanks. A dead crab is wrapped in a leaf and attached to the end of a stick, which is then thrust into the fish's hole. When the fish eats the crab, it can be dragged out and caught (\citealt{Pichl1967}). 

\TCsubword{chuŋpaŋ} (comp.) \textit{n} stick used to catch jumping fish (\citealt{Pichl1967}). 

\TCheadword[2]{paŋ} \textit{n} \textbf{1)} [pàŋ] moon (K dialect); \textit{ipaang} (hɔ̃/-) moon (\citealt{Pichl1967}). \textbf{2)} month. \textit{Hɔ̃ poɔni thiyeng, mpang mən-bul bɛlɛng buli, mən-bul bɛlɛng hãlɛ.} It (the year) is divided in the middle, six months on one side, six months on the other side (\citealt{Pichl1967}). \textit{Pɔ gbem mi paŋdɛ ŋɔ pɔ wɔ Sɛptɛmbaɛ, paŋ mɔikɛ mɛnyɔllɛ.} I was born in the month that they call September, the ninth month.

\TCsubword{paaŋkith} (comp.) \textit{cf}: \TClink{paaŋtriayeŋ} (comp. of, id. of \TClink[2]{paŋ}, \TClink{tri}, \TClink{ayeŋ}). \textit{n} half moon (\citealt{Pichl1967}).

\TCsubword{paaŋpɛɛ} (comp.) \textit{n} full moon (\citealt{Pichl1967}).

\TCsubword{paaŋpikɛ} (comp.) \textit{n} hidden moon (\citealt{Pichl1967}).

\TCsubword{paŋsaa} (comp.) \textit{n} September moon (\citealt{Pichl1967}). \textit{Pang saa lɛ hɔ̃ sirɔkɔ-hɔl.} The month of September is the harvest time (\citealt{Pichl1967}). 

\TCsubword{paaŋsana} (comp.) \textit{n} new moon (\citealt{Pichl1967}).

\TCsubword{paaŋtriayeŋ} (comp.), (id.) \textit{cf}: \TClink{paaŋkith} (comp. of \TClink[2]{paŋ}, \TClink[1]{kith}). \textit{n} half moon (\citealt{Pichl1967}).

\TCsubword{paŋopaŋ} (der.) \textit{temp} \textbf{1)} every month. \textit{Ivin paŋ-o-paŋ.} Even every month. \textbf{2)} every evening. \textit{Paŋopaŋ gbi, Braima wɔ kɔ lɔɔli pɛl dukiɛ ni yɛllɛ'ɛ.} Every evening, Braima goes to inspect the leggo chain and the yɛllɛ chain. \textbf{3)} monthly.

\TCsubword{paŋpaŋ} (der.) \textit{temp} \textbf{1)} evening. \textbf{2)} late afternoon. \textit{Paŋ-paŋ dɛ sɔŋkɔma mɛŋk mɛn dɛ, wɔe tipɛ ha taŋ yɛ wɔ bosi mmɛn dɛ hiŋk wɔm dɛai.} Late in the afternoon something like 5pm, he began to cry as he was bailing out the boat.

\TCheadword[3]{paŋ}\textit{temp} [páŋ] in the evening (K dialect); (hɔ̃/-) evening (\citealt{Pichl1967}). \textit{Tondɛ kɔ lɛ ituɛ kunɛ, mɔ kɔi kɔ thɔŋgul ŋa paŋdɛ.} The small bit that remains in the pot, you reserve it for the evening. \textit{Chaŋbo paŋdɛ ŋɔ mɔi bo pɔ hiŋ ka ja tuthɛ, than bo tha ika che kunɛ.} Except when evening came, we would be given rice pounding work, that was the work we were engaged in. comp. \TClink{palli-paŋ} (see \TClink[1]{pal}) 

\TCheadword{paaŋkith} (comp. of \TClink[2]{paŋ}, \TClink[1]{kith}, see \TClink[2]{paŋ}) 

\TCheadword{paŋopaŋ} (der. of \TClink[2]{paŋ}, \TClink{-o-}, see \TClink[2]{paŋ}) 

\TCheadword{paŋpaŋ} (der. of \TClink[2]{paŋ}) 

\TCheadword{paaŋpɛɛ} (comp. of \TClink[2]{paŋ}, \TClink[1]{pɛ}, see \TClink[2]{paŋ}) 

\TCheadword{paaŋpikɛ} (comp. of \TClink[2]{paŋ}, \TClink{pikɛ} (der. of \TClink[1]{pi}, \TClink{-k}), see \TClink[2]{paŋ}) 

\TCheadword{paŋsaa} (comp. of \TClink[2]{paŋ}, \TClink{saa}, see \TClink[2]{paŋ}) 

\TCheadword{paaŋsana} (comp. of \TClink[2]{paŋ}, \TClink{sana}, see \TClink[2]{paŋ}) 

\TCheadword{paaŋtriayeŋ} (comp. of, id. of \TClink[2]{paŋ}, \TClink{tri}, \TClink{ayeŋ}, see \TClink[2]{paŋ}) 

\TCheadword{parat} \textit{cf}: \TClink{kimɔ}. \textit{v} flee; run away.

\TCheadword{parɛ} (der. of \TClink[1]{pal}) 

\TCheadword[1]{path} (comp. of \TClink[2]{path}) 

\TCheadword[2]{path} \textit{cf}: \TClink[2]{gbisiŋ}, \TClink{thuka}. \textit{v} [páth] marry (K dialect)

\TCsubword[1]{path} (comp.) \textit{n} bride price. \textit{Abɛna wɔɛ ŋae bɛmpani yeŋkɛlɛŋ ba ni ŋae kɔ pat.} His parents prepared themselves well and paid the bride price (to make the engagement). 

\TCheadword{patikulali} (Eng \textit{particularly}) \textit{cf}: \TClink{teŋka}. \textit{adv} particularly. \textit{Patikulali hi Amɔyaɛ ko a wokɛ lɔ pridɔminantli Muslim.} Particularly we Muslims, where I come from is predominantly Muslim.

\TCheadword{patikulas} (Eng \textit{particulars}) \textit{n} particulars, particular choices. \textit{Ŋa koŋ gbo koi patikulas.} They had just taken what they had picked out.

\TCheadword{pe} \textit{cf}: \TClink[1]{gbi}. \textit{adv} together.

\TCheadword{peayɛn} (comp. of \TClink{pe}, \TClink{ayɛn}, see \TClink{pe}) 

\TCheadword{pebɔhulka} (comp. of \TClink{pe}, \TClink{boŋhul}, see \TClink{pe}) 

\TCheadword{pe} \textit{cf}: \TClink[2]{thɛrɛŋ}. \textit{n} \textbf{1)} stone, [pèɛ̀]/[pɛ̀tɛ̀] stone/stones (B dialect); \textit{pe} (hɔ̃/tha) stone (\citealt{Pichl1967}). \textit{I kɔ sɛm pethɛ atok.} We go and stand on the stones. \textbf{2)} rock. \textit{Pe rɛnthɛ, Laɔn ɔf Juda.} Rock of ages, Lion of Judah. \textit{Nthim bot lɛ njok ɛ, thipe tha che ko!} Turn the boats to the right side, there are rocks ahead (\citealt{Pichl1967}.

\TCsubword{peayɛn} (comp.) \textit{n} (hɔ̃/tha) hard stone (lit. real stone) (\citealt{Pichl1967}).

\TCsubword{pebɔhulka} (comp.) \textit{n} (hɔ̃/tha) whetstone (\citealt{Pichl1967}).

\TCsubword{pekɔl} (comp.) \textit{n} (hɔ̃/tha) red stone, ground and used for paint (\citealt{Pichl1967}). 

\TCsubword{pethimbɔs} (comp.) \textit{n} (hɔ̃/tha) rocks on or near the shore where many cowries are to be found (\citealt{Pichl1967}). 

\TCheadword{peem} \textit{cf}: \TClink[1]{bɔs}, \TClink[1]{pem}, \TClink[2]{pem}. \textit{Idph} [pèèm] of quiet, still, stagnant, given as the Sherbro equivalent of a Mende ideophone meaning ‘quiet' (K dialect). 

\TCheadword{peenɛ} \textit{cf}: \TClink{soo}. \textit{n} [péénɛ̀] millet (K dialect); peenɛ (kɔ/ma) fundi, hungry millet (Digitaria exilis) In the sandy parts of Sherbro country, the oldest cultivated plant (\citealt{Pichl1967})

\TCsubword{peenɛmboŋ} (comp.) \textit{cf}: \TClink{puympene} (comp. of \TClink[1]{puy}). \textit{n} (kɔ/ma) grass species with small white seeds eaten only by birds (\citealt{Pichl1967}). 

\TCheadword{pei} \textit{v} \textbf{1)} shed. \textit{Jizɔs ŋa ja bom ba ŋa yaŋ, yɛ peyɛ nkɔŋ ma Wɔlɛ.} Jesus has done a big thing for me when He shed his blood. \textbf{2)} spill. \textbf{3)} throw away.

\TCsubword{peini} (der.) \textit{v} pour. \textit{Mɛndɛ ma peini, kɛ pɛpɛ hɔ pɛlɛni.} The water was poured but the calabash was not broken.

\TCheadword{peini} (der. of \TClink{pei}, \TClink{-ni}, see \TClink{pei}) 

\TCheadword{pekɛ} \textit{cf}: \TClink{dugbu}. \textit{n} \textbf{1)} healing place. \textbf{2)} treatment. comp. \TClink{kilpekɛ} (see \TClink[1]{kil}) 

\TCheadword{pekɔl} (comp. of \TClink{pe}) 

\TCheadword[1]{pel} \textit{v} load.

\TCheadword[2]{pel} [pel] \textit{cf}: \TClink[2]{chɔŋ}. \textit{n} egg. \textit{pəlthɛ} eggs. comp. \TClink{bolpel} (see \TClink[1]{bol}), id. \TClink{bolpel} (see \TClink[1]{bol}) 

\TCsubword{pelbol} (comp.), (id.) \textit{cf}: \TClink{bolpel} (comp. of, id. of \TClink[1]{bol}, \TClink[2]{pel}). \textit{adj} bald.

\TCheadword[3]{pel} \textit{n} [pèl] tree species, resembles \TClink{pelapela} but with larger leaves, also used as medicine, e.g., on snake bites (K dialect). 

\TCsubword{pelapela} (der.) \textit{cf}: \TClink{pelbɔ} (unspec. of \TClink[3]{pel}). \textit{n} tree species whose leaves used for medicine (K dialect); tree species, variety of \TClink{pelbɔ} with smaller leaves (\citealt{Pichl1967}).

\TCsubword{pelbɔ} (unspec.) \textit{cf}: \TClink{pelapela} (der. of \TClink[3]{pel}). \textit{n} tree species, blood tree (\citealt{Pichl1967}). 

\TCheadword{pelapela} (der. of \TClink[3]{pel}) 

\TCheadword{pelbol} (comp. of, id. of \TClink[2]{pel}, \TClink[1]{bol}, see \TClink[2]{pel}) 

\TCheadword{pelbɔ} (unspec. of \TClink[3]{pel}) 

\TCheadword[1]{pem} \textit{cf}: \TClink[1]{bɔs}, \TClink{peem}. \textit{n} quiet. comp., id. \TClink{min-pem} (see \TClink[3]{min})

\TCheadword[2]{pem} \textit{cf}: \TClink{peem}, \TClink{thɔli}, \TClink{thɔnthɔ}. \textit{v} be quiet.

\TCheadword{pemple} \textit{cf}: \TClink[1]{paŋ}. \textit{n} [pémplé] fishing method, type of fishing done by both men and women (K dialect); kind of fishing used for \TClink{gbuluŋk}, using a short line to which a periwinkle is attached (\citealt{Pichl1967}). 

\TCheadword{pen} \textit{cf}: \TClink[2]{tiŋ}. \textit{Idph} of tightness.

\TCheadword{peenɛmboŋ} (comp. of \TClink{peenɛ}, \TClink[2]{boŋ}, see \TClink{peenɛ}) 

\TCheadword{peŋ} \textit{n} headpad.

\TCheadword{peŋka} (Port \textit{espingarda} ‘shotgun') \textit{cf}: \TClink{chakabulla}. \textit{n} (kɔ/ma) gun (\citealt{Pichl1967}). \textit{Ŋa bi peŋa ŋan sui o.} They have guns in their hands.

\TCheadword[1]{peŋkɛ} \textit{v} be first. \textit{Peŋkɛ kɔni hɔ chendɛ peŋkɛ mɔɛ.} To go first does not mean you will be the first to arrive.

\TCsubword[2]{peŋkɛ} (der.) \textit{temp} first. \textit{Yɛ Braima muɛ bondɔ koɛ, wɔe peŋkɛ kɔ ko yɛllɛɛ.} When Braima got to the wharf, he went to look at the \textit{yɛllɛ} chain first. \textit{Yɛ mɔni hun chɔŋ vɛ boi po mɔɛ nse ŋɔ mɔ piŋgɛ chɔŋ.} As you are coming to serve (food), your husband's basin should be the first one to dish out.

\TCheadword[2]{peŋkɛ} (der. of \TClink[1]{peŋkɛ}) 

\TCheadword{peshɛnt} (Eng \textit{patient}) \textit{n} patient. \textit{Kɛ wanta bul ka che lɔ woŋga ka Tɔmi Tɔka ka kɛ ka che peshɛnt siza.} A girl used to be in this house of Tommy Tucker's, but she was a Cesarean-section patient.

\TCheadword{pethimbɔs} (comp. of \TClink{pe}, \TClink{thimbɔs}, see \TClink{pe}) 

\TCheadword{peyɛni} \textit{v} catch a cold.

\TCheadword[1]{pɛ} \textit{v} \textbf{1)} be full. \textit{Boiɛ hɔ pɛɛ ka mɛn.} The dish is filled with water (\citealt{Pichl1967}). \textbf{2)} fill. \textit{Ŋ kɔ pɛy bithir lɛ ka mən}. Go fill the bottle with water (\citealt{Pichl1967}). comp. \TClink{mɛnpɛyɛ} (see \TClink[3]{mɛn}), \TClink{paaŋpɛɛ} (see \TClink[2]{paŋ}) 

\TCsubword{pɛkɛ} (unspec.) \textit{v} \textbf{1)} be filled. \textit{Yaŋ ya pəkɛ gbo iwɛi.} I am (truly) filled with evil (\citealt{Pichl1967}). \textbf{2)} be overfull.

\TCheadword[2]{pɛ} \textit{cf}: \TClink[4]{ŋa}. \textit{indfpro} \textbf{1)} they. \textit{Pɛ velɛ bul-nɔ-bul.} They called one after the other. \textit{Abiɛ lɔni nɔndo ŋɔ pɔ lɛli kunthɛ.} I do not have that thing they use to look inside pregnant women. \textit{Pɔ telɛ wik bul mɔike tindɛ pɔi kutha.} They wait one to two weeks (before) they plow. \textbf{2)} people; person. \textit{Ahindɛ ha hun ha hayema nɔ pɔ koi nɔ bul pɔ wɔ wom Nyamba ko.} Then the people came, they said they want one person to send her to Moyamba. \textit{Ilellɛ vɛ ŋɔ pɔ gbem ka wɔ?} The name is what people gave him at birth. \textit{Wɔnpɛ aŋa wɔɛ ka lɔ pɔ ka gbem ŋa?} She herself her people were born here? \textbf{3)} someone; one.

\TCheadword[2]{pɛ} \textit{cf}: \TClink{halɛ}, \TClink[1]{pika} (der. of \TClink[2]{pika}), \TClink[2]{pim}, \TClink{tilaŋ}. \textit{adj} other. \textit{Be pɛ nɔ cheni wɔ mɔilɛ Jizɔs Kraist.} There is no other person who compares to Jesus Christ. id. \TClink{pɛpɛnthe} (see \TClink{pɛnthe}) 

\TCheadword[3]{pɛ} \textit{cf}: \TClink{huɛŋ}, \TClink{mina}, \TClink[1]{mu}, \TClink[1]{ni}. \textit{adv} \textbf{1)} again. \textit{Ba na pɛ wɔ ye gbïkni.} Mr. Spider ran away again (\citealt{Pichl1967}). \textbf{2)} also. \textit{Yamɔ pɛ wɔn Shenge ka lɔ pɔ gbem wɔ hinye?} Your mother was also born here at Shenge, right? \textit{Kɛ wɔ theli Mbolomdɛ ni wɔ ma pɛ gbal?} But he speaks Sherbro as well as writing it? \textit{Mi Adama, i yema pɛ ni nwun wom lɛŋ ko ŋanɛ ŋa hunɔnimuɛ.} Mami Adama, I also want for you to come and send greetings to your descendants. \textit{So yɛ nwuni Shenge ka, nkaŋa ŋa pɛ?} So when you came to Shenge here, did you study here as well? \textbf{3)} at all. \textit{Kɔŋdɛ kɔ akekɛ thiwɔllɛ yɛ laiyoɛ hɔ cheni pɛ bul.} The burial that I have seen (with my) eyes, it is not at all the same. \textbf{4)} back. \textbf{5)} now.

\TCheadword[4]{pɛ} [pɛ], [pɛ] \textit{temp} \textbf{1)} still. \textit{Hin gbi icheni pɛe, ilɔ amɛn.} Not all of us are still alive, we are five. \textit{Mɔ pɛ loni bolɛ siŋ thi landɛ?} Would you remember those games? \textbf{2)} anymore. \textit{Pa ni pɛ che wɔe mi.} He is no longer alive. \textit{Pɔ che bonth nɔ pɛ?} They do not help people anymore?

\TCheadword{pɛɛ} [pɛ̀ɛ̀] \textit{n} \textbf{1)} snake species, boa constrictor, found around Shenge, brown with black spots (K dialect). \textbf{2)} \textit{pɛɛ} (wɔ/hã, si) python (\citealt{Pichl1967}).

\TCheadword{pɛɛpɛɛ} \textit{n} [pɛɛpɛɛ] the shoulder (K dialect); \textit{pɛpɛɛ} (hɔ̃/tha) shoulder (\citealt{Pichl1967}). \textit{A che gbo pɔng silal yɛ ya fɔs mɔ thipɛpɛ lɛ, mma silini.} I was only joking when I tapped your shoulders, don't be annoyed (\citealt{Pichl1967}).

\TCheadword{pɛiŋ} (Eng \textit{paying}) \textit{n} [pɛ́íŋ] pre-payment, before I give you something you give me something (K dialect). 

\TCheadword{pɛkɛ} (unspec. of \TClink[1]{pɛ}) 

\TCheadword[1]{pɛl} \textit{v} \textbf{1)} break. \textit{Ni hɔ̃ wɔ ye hɛ̃thin sui ni hɔ̃ ye duk ni hɔ̃ pɛl.} And it slipped out of his hand and it fell down and broke (\citealt{Pichl1967}). \textit{M pəl pel lo!} Break this egg! (\citealt{Pichl1967}). \textit{Ikoi bithi thiseko ki, thanɛ thakoŋ pɛli vɛ.} We take the bottle of hooks, those broken ones. \textbf{2)} cease. \textit{Hɔɛ hɔ pɔ pɛl taŋdɛ...} The day that people would cease the mourning... (\citealt{Sumner1921}). \textbf{3)} announce. \textit{Pə koŋ pəl taŋ bɛɛ lɛ.} They have announced the mourning for the chief (\citealt{Pichl1967}). 

\TCsubword{pɛlmahɔl} (comp.) \textit{n} insect species, the spit that comes from its mouth is the foam where it lays its eggs, warn children to be wary because it causes blindness (lit. eye break) (K dialect). 

\TCsubword{pɛltaan} (comp.) \textit{n} [pɛ́ltààn] when mourning and wailing can begin, e.g., after the death of a paramount chief (lit. break(out?)-crying) (K dialect); \textit{pɛltang} ‘outbreak of crying,' formal announcement of the death of an important person by the Gbanabom, usually after the funeral ceremonies. This announcement is made especially for the women, who are then allowed to cry (\citealt{Pichl1967}). 

\TCheadword[2]{pɛl} \textit{n} \textbf{1)} fishing net, [pəllɛ]/[pɛ̀lǹlɛ́] net/nets (K dialect). \textit{A thók ma pɛllɛ.} I hunt with a net. \textit{A chen duki pɛl, nhukɛ ma a dukiɛ.} I do not use a net, I use hooks. \textit{Braima wɔe kɔ lɔɔli pɛl yɛllɛɛ ni pɛl dukiɛ.} Braima went to inspect the net chain but the net had sunk. \textbf{2)} hammock (made of fishing nets). \textit{Kaiŋ Taso wɔ jajɛl wɔɛ wuɛ, hinɛ lɔ pɛllɛai amaaɛ ntɛnt.} Kain Tasso, whose mother-in-law died, lay down in the hammock near the women. comp. \TClink{hanthpɛl} (see \TClink{hantha}) 

\TCsubword{pɛlbom} (comp.) \textit{n} (kɔ/ma) fishing net with long poles on the outside, used in neck-deep water (\citealt{Pichl1967}). 

\TCsubword{pɛlbɔlkek} (comp.) \textit{cf}: \TClink{bolkek}. \textit{n} (kɔ/ma) fishing net or chain used to catch the beard-beard (a type of fish) (\citealt{Pichl1967}.) 

\TCsubword{pɛlchal} (comp.) \textit{n} (kɔ/ma) large hunting chain for deer and other bigger animals (\citealt{Pichl1967}). 

\TCsubword{pɛlgbampɔ} (comp.) \textit{cf}: \TClink{kamanthi}. \textit{n} (kɔ/ma) casting net (\citealt{Pichl1967}). 

\TCsubword{pɛlgbokbo} (comp.) \textit{n} (kɔ/ma) catfish net, used in waist-deep water (\citealt{Pichl1967}). 

\TCsubword{pɛlkuukuu} (comp.) \textit{n} fishing net, large float net with corks, used for fishing during the night (\citealt{Pichl1967}). 

\TCsubword{pɛlmfan} (comp.) \textit{n} small hunting chain for smaller animals like cutting grass [ground hog] (\citealt{Pichl1967}).

\TCsubword{pɛlnaa} (comp.) \textit{n} cobweb (\citealt{Pichl1967}).

\TCsubword{pɛlnsɛk} (comp.) \textit{n} net used for catching mullet in shallow water (\citealt{Pichl1967}). 

\TCsubword{pɛlthook} (comp.) \textit{n} hunting net, hunting chain (\citealt{Pichl1967}). 

\TCheadword{pɛlbom} (comp. of \TClink[2]{pɛl}, \TClink{bom}, see \TClink[2]{pɛl}) 

\TCheadword{pɛlbɔlkek} (comp. of \TClink[2]{pɛl}, \TClink{bɔlkek}, see \TClink[2]{pɛl}) 

\TCheadword{pɛlchal} (comp. of \TClink[2]{pɛl}, \TClink[1]{chal}, see \TClink[2]{pɛl}) 

\TCheadword{pɛlɛ} \textit{n} \textbf{1)} uncooked, husked rice; seed rice. \textit{Ya koŋ hɛy pəlɛ lɛ ha ko hɛ̃thi ibənkɛ lɛ.} I have fanned the rice, you go and pick out the husk (\citealt{Pichl1967}). \textit{Ŋkɔ lath pəlɛ lɛ kãhãy ko.} Go spread the rice outside (to dry) (\citealt{Pichl1967}). \textbf{2)} red rice. \textit{Pəl lɛ kɔ ko̹y chaŋ buthba lɛ.} The reddish rice increases more than the dark one (\citealt{Pichl1967}). comp. \TClink{pɛlɛsɔi} (see \TClink{sɔi}), \TClink{pɔmpɛlɛ} (see \TClink[1]{pɔm}) 

\TCsubword{pɛlɛdinthɛ} (comp.) \textit{n} rice cleaned and free of husks (\citealt{Pichl1967}). 

\TCsubword{saŋpɛlɛ} (comp.) \textit{cf}: \TClink{tɛtɛk}. \textit{n} young rice before being planted (lit. sow rice) (\citealt{Pichl1967}). 

\TCheadword{pɛlɛdinthɛ} (comp. of \TClink{pɛlɛ}, \TClink{dinth}, see \TClink{pɛlɛ}) 

\TCheadword{pɛlɛsɔi} (comp. of \TClink{pɛlɛ}, \TClink{sɔi}, see \TClink{sɔi}) 

\TCheadword{pɛlgbampɔ} (comp. of \TClink[2]{pɛl}, \TClink{gbampɔ}, see \TClink[2]{pɛl}) 

\TCheadword{pɛlgbokbo} (comp. of \TClink[2]{pɛl}, \TClink{gbokbo}, see \TClink[2]{pɛl}) 

\TCheadword{pɛlkuukuu} (comp. of \TClink[2]{pɛl}) 

\TCheadword{pɛlmahɔl} (comp. of \TClink[1]{pɛl}, \TClink[3]{hɔl}, see \TClink[1]{pɛl}) 

\TCheadword{pɛlmbos} (comp. of \TClink[3]{bos}) 

\TCheadword{pɛlmfan} (comp. of \TClink[2]{pɛl}, \TClink[1]{fan}, see \TClink[2]{pɛl}) 

\TCheadword{pɛlnaa} (comp. of \TClink[2]{pɛl}, \TClink{naa}, see \TClink[2]{pɛl}) 

\TCheadword{pɛlnsɛk} (comp. of \TClink[2]{pɛl}, \TClink[1]{sɛk}, see \TClink[2]{pɛl}) 

\TCheadword{pɛltaŋ} (comp. of \TClink[1]{pɛl}, \TClink[2]{taŋ} (der. of \TClink[1]{taŋ}), see \TClink[1]{pɛl}) 

\TCheadword{pɛlthook} (comp. of \TClink[2]{pɛl}, \TClink{thok}, see \TClink[2]{pɛl}) 

\TCheadword{pɛm} \textit{n} war. \textit{Nsiɛ tɛm pɛm doki yɛi chaŋ-chaŋdɛ.} You know during the war how we were moving around. comp. \TClink{wɔmpɛm} (see \TClink[2]{wɔm}) 

\TCsubword{lakapɛm} (unspec.) \textit{n} company or regiment of soldiers.

\TCheadword[1]{pɛmplɛ} \textit{cf}: \TClink[1]{lɛli}, \TClink[1]{tok}. \textit{v} [pɛ́mplɛ́] watch (K dialect). 

\TCheadword[2]{pɛmplɛ} \textit{v} \textbf{1)} stalk, waiting to ambush someone or something, e.g., someone who owes you money (B dialect). \textit{Ya kɔ pɛmplɛ wɔ.} I'm going to wait (in order to ambush) him. \textbf{2)} \textit{pəmplɛ} stumble, stagger (\citealt{Pichl1967}). 

\TCheadword[1]{pɛn} (der. of \TClink[2]{pɛn})

\TCheadword[2]{pɛn} [pɛ́n] \textit{cf}: \TClink[2]{tok}. \textit{n} thunder, crack of thunder (K dialect). \textit{Tok lɛ kɔ pɛn parɛ hwɛ lɛ hɔ ba Ngubɛ wuɛ.} The thunder cracked the other day, they say it was (when) Mr. Ngube died (\citealt{Pichl1967}). 

\TCsubword[1]{pɛn} (der.) v shout, talk loudly and authoritatively, e.g., as the Poro devil does to women (\citealt{Pichl1967}). der. \TClink{pɛnɛk} (see \TClink[2]{pɛn}), \TClink{pɛni} (see \TClink[2]{pɛn})

\TCsubword{pɛnɛk} (der.), (der. of \TClink[1]{pɛn}) \textit{v} shout. \textit{Min wɔ pənɛk amaa lɛ}. The devil shouts at the women (\citealt{Pichl1967}). 

\TCsubword{pɛni} (der.), (der. of \TClink[1]{pɛn}) \textit{v} \textbf{1)} shout (\citealt{Pichl1967}). \textbf{2)} practice. \textit{So ni ikancheya pɛni tɔnthe kaŋga chɔche ŋɔ kɔ che ni ithe Mbolomdɛ yeŋkɛlɛŋ-yeŋkɛlɛŋ.} So we should be practicing singing for the church and for us to know Sherbro really well. \textit{Braima wɔe pɛrni ha che kɔ duki mpɛl lo ki ndelma wɔ.} Braima then practiced going to leave the nets at sea.

\TCheadword{pɛnɛk} (der. of \TClink[1]{pɛn} (der. of \TClink[2]{pɛn}), -\TClink{k}, see \TClink[2]{pɛn})

\TCheadword{pɛni} (der. of \TClink[1]{pɛn} (der. of \TClink[2]{pɛn}), \TClink[1]{-i}, see \TClink[2]{pɛn}) 

\TCheadword{pɛnsil} (Eng \textit{pencil}) \textit{n} pencil.

\TCheadword{pɛnth} \textit{n} twin. \textit{Pɛnthsɛ ŋan ŋanpɛ, ŋa bi ŋa bi ilel ɡba?} Twins, they themselves, they have to have separate names?

\TCheadword{pɛnthe} \textit{n} brother, [pə̀ntsə́]/[pə̀ntsə́mì]/ [pə̀ntsə́nɔ̀] brother/my brother/ your (pl) brother (B dialect). \textit{Mpɛntɛ ha mɔɛ ha ba mɔ gbemdɛ, ha wɔi?} Your brothers born of the same father, are they alive? \textit{Hɔmɔ mi ja pɛnthe wɔ lɛ.} He told me about his brother (\citealt{Pichl1967}). \textit{Pɛnthe mi nlə-m lanɔ la bɔnthə-m dɛ.} Brother, look at what has happened with me (lit. what met me) (\citealt{Pichl1967}). 

\TCsubword{pɛpɛnthe} (id.) \textit{n} close friend.

\TCheadword[1]{pɛŋ} \textit{n} \textbf{1)} boundary. \textbf{2)} border.

\TCheadword[2]{pɛŋ} \textit{v} jump; jump over. \textit{Wɔ́ pɛ̀ŋ, wɔ̀ pɛ́ŋhɛ̀, wɔ̀ŋndɛ́ kóŋ pɛ̀ŋ.} It (the frog) jumps. It is it (the frog) who jumps. \textit{Pəŋ hu lɛ ni kɔni kil lɛ hɔl ko.} He jumped over the fence and went into the house (\citealt{Pichl1967}). 

\TCsubword{pɛŋgipɛŋgi} (der.) \textit{v} jump. \textit{Inan gballɛ, ilɔ pɛŋgipɛŋgi, i kikkik.} We draw the line, we jump there (and) kick.

\TCsubword{pɛŋka} (der.) \textit{v} jump. \textit{I koi baŋg li thanthɛndoki i kɔ pɛŋka.} We take this ordinary rope and jump with it.

\TCsubword{pɛŋchanth} (id.) \textit{v} wean (a child).

\TCsubword{pɛŋkiyɔ} (unspec.) \textit{v} jump.

\TCheadword{pɛŋchanth} (id. of \TClink[2]{pɛŋ}, \TClink{chanth}, see \TClink[2]{pɛŋ}) 

\TCheadword{pɛŋgipɛŋgi} (der. of \TClink[2]{pɛŋ}) 

\TCheadword{pɛŋka} (der. of \TClink[2]{pɛŋ}) 

\TCheadword{pɛŋke} \textit{v} give up. \textit{Man pɛŋke, ŋa tɔnk Bahin yɛ.} Do not give up serving the Lord.

\TCheadword{pɛŋkiyɔ} (unspec. of \TClink[2]{pɛŋ}, \TClink{-k}, see \TClink[2]{pɛŋ}) 

\TCheadword{pɛpɛ} [pɛ̀pɛ̀] \textit{n} \textbf{1)} calabash, [pə̀pə̀]/ [pə̀pə̀thɛ́] calabash/calabashes (B dialect); \textit{pəpə} (hɔ̃/tha) calabash made of a gourd (Crescentia cujete) (\citealt{Pichl1967}). \textbf{2)} [pɛ̀pɛ̀] a cup for measuring or drinking made by cutting a calabash, (\textit{gbùlù}), in half (K dialect).

\TCheadword{pɛpɛnthe} (id. of \TClink[2]{pɛ}, \TClink{pɛnthe}, see \TClink{pɛnthe}) 

\TCheadword{pɛr} \textit{n} kinds.

\TCheadword{pɛri} \textit{v} fill.

\TCheadword{pɛth} \textit{cf}: \TClink{sɔisɔi} (der. of \TClink{sɔi}). \textit{v} \textbf{1)} taste good; be delicious. \textit{Yekə lɛ hɔ̃ pəth.} The cassava is good (\citealt{Pichl1967}). \textit{Yenchɛk ha rɔnka lɛ ha pɛth.} Fish prepared in the manner of \textit{rɔnka} are very good (\citealt{Pichl1967}). \textit{Ja ɛ la pɛth hɛ laŋ Jisɔs.} It is sweet to believe in Jesus (\citealt{Pichl1967}). \textbf{2)} please. \textit{Kɛ la wɔ pɛth yɛ jajɛl wɔɛ wuɛ?} Does it please him that his mother-in-law died?

\TCsubword{pɛthil} (der.) \textit{v} \textbf{1)} taste good. \textbf{2)} be sweet. \textit{Nle kɔ bo mpɔni nwɔk mpika ntuk maɛ; labi la pɛthilɛ mini.} If you leave it and throw yourself into another language, you lose it; that is why it is not sweet to me. \textbf{3)} be pleasing. \textit{Nɔɛ wɔ chal ha lɔŋ nui ko la pɔ hɔ ha yindɛ, bi ha theɛ lanɛ la biɛn ha pɛthil wɔɛ.} The person that sits listening to the gossip of others will hear that which displeases him (proverb). 

\TCsubword{pɛthpɛthɛ} (der.) \textit{n} \textbf{1)} tastiness. \textit{Pɔmthi gbamdɛ lɛ ye ma kɔ gbo chɛth yeŋkɛlɛŋ ni ntheki kɔni pɛth-pɛthɛ…} Potato leaves, if you want to cook them nicely so that they taste good… \textbf{2)} pleasantness. \textit{So, ŋɔ ke bila, pɛth-pɛth ŋɔ lɔ?} So, how is your marriage, is it good? \textit{Ee, pɛth-pɛth ŋɔ lɔ.} Yes, the marriage is good. 

\TCheadword{pɛthɛli} \textit{v} pet (\citealt{Pichl1967}). 

\TCheadword{pɛthɛpɛthɛ} \textit{n} vine species, leaves used for medicine, no fruit or flowers, when put on fire makes a noise, ‘pɛthɛpɛthɛ,' like popcorn (K dialect).

\TCheadword{pɛthil} (der. of \TClink{pɛth}, \TClink{-il}, see \TClink{pɛth}) 

\TCheadword{pɛthpɛthɛ} (der. of \TClink{pɛth}) 

\TCheadword{pɛyɛ} \textit{v} accompany. \textit{Kaiŋ Taso wɔe munini tir ko wɔ ko ni anya wɔɛ ŋa hun wɔ pɛyɛ ko wul lijajɛl wɔɛ.} Kain Tasso returned to his town and his people came to welcome him from his mother-in-law's funeral.

\TCheadword[1]{pi} \textit{v} \textbf{1)} become dark. \textit{nhɔbɛ ilema hɔ haŋ wɔyɛ pi ima lɔ be nwɔk pika gbi, acheŋ ke gbi.} Even if we keep speaking it until nightfall using no other language, I would not get tired. \textbf{2)} spend day. \textit{A pi chɛk lɛ ko.} I spent the whole day on the farm. comp. \TClink{hwɛpi} (see \TClink[2]{hu}) 

\TCsubword{pithi} (comp.) \textit{cf}: \TClink[2]{rithi} (der. of \TClink[1]{rithi}). \textit{v} \textbf{1)} be black. \textbf{2)} be dark. \textit{Pɛlɛ kɔi pith kɔi piŋgi, kɔi bi kun, kɔi gbemɔ.} The rice will get dark, and then it will change and swell up (lit. have a belly, i.e. be pregnant) and then tiller. \textbf{3)} dye dark.

\TCsubword{pikɛ} (der.) \textit{cf}: \TClink{mathui} (der. of \TClink{math}). \textit{v} be hidden. comp. \TClink{paaŋpikɛ} (see \TClink[2]{paŋ})

\TCsubword{piki} (id.) \textit{v} greet in evening.

\TCheadword[2]{pi} \textit{n} beauty.

\TCheadword[3]{pi} \textit{temp} evening. \textit{...wɔyɛ ŋɔ pi gbo yendɛ ŋɔ hi jo wɛ} ...in the evening what we are to eat.

\TCheadword[1]{pia} [pìà] \textit{cf}: \TClink[1]{han}, \TClink{sui}. \textit{n} \textbf{1)} hand \textit{Ya ke wɔ ma hɔl thimdɛ, ni ya bɛŋ ma wɔ pia mi njokɛ, ni ya theli ko wɔ ko.} I saw him with my eyes, and I touched him with my right hand, and I talked to him. \textbf{2)} arm.

\TCsubword{piamin} (comp.) \textit{n} left hand. comp. \TClink{bɛŋpiamin} (see \TClink[2]{bɛŋ}), der. \TClink{piaminɛ} (see \TClink[1]{pia})

\TCsubword{piaminɛ} (comp.), (der. of \TClink{piamin}) \textit{Loc} [pìàmíndɛ̀] on the left (B dialect). comp. \TClink{bɛŋpiamin} (see \TClink[2]{bɛŋ})

\TCsubword{pianjok} (comp.) \textit{n} right hand (\citealt{Pichl1967}). comp. \TClink{bɛŋpianjok} (see \TClink[2]{bɛŋ}), der. \TClink{pianjokɛ} (see \TClink[1]{pia})

\TCsubword{pianjokɛ} (comp.), (der. of \TClink{pianjok}) \textit{Loc} [pìànjók] on the right (B dialect). comp. \TClink{bɛŋpianjok} (see \TClink[2]{bɛŋ})

\TCheadword[2]{pia} (Eng \textit{pear}) \textit{n} [píà] pear (K dialect). 

\TCheadword{piamin} (comp. of \TClink[1]{pia}, \TClink[3]{min}, see \TClink[1]{pia})

\TCheadword{piaminɛ} (der. of \TClink{piamin} (comp. of \TClink[1]{pia}, \TClink[3]{min}), \TClink[1]{ɛ}, see \TClink[1]{pia}) 

\TCheadword{pianjok} (comp. of \TClink[1]{pia}, \TClink[1]{jo}, see \TClink[1]{pia}) 

\TCheadword{pianjokɛ} (der. of \TClink{pianjok} (comp. of \TClink[1]{pia}, \TClink[1]{jo}), \TClink[1]{ɛ}, see \TClink[1]{pia}) 

\TCheadword{piath} \textit{n} (wɔ/hã, si) fish species, Spanish (Polydactylus quadrifilis (\citealt{Pichl1967}).

\TCheadword{Piɛ} \textit{nam} Pieh, name given by Poro Society.

\TCheadword{piɛ} \textit{n} elephant. In Banta area near Mokele, people believe that if old people die, their souls go into the bush and turn into elephants (\citealt{Pichl1967}). comp. \TClink{riŋpiɛ} (see \TClink{riŋ}) 

\TCheadword{pii} \textit{n} (wɔ/hã, si) glow worm (\citealt{Pichl1967}).

\TCheadword[1]{pika} (der. of \TClink[2]{pika}) 

\TCheadword[2]{pika} \textit{n} remainder; the rest. \textit{Lomthinɔo, pikchɔthinɔo, lanɛ gbi wɔ tha chi, lipikaɛ pɔ lai ni be ki buk.} Your voice (recordings), your pictures, he will bring all of that, the rest will be put in books.

\TCsubword[1]{pika} (der.) \textit{cf}: \TClink{halɛ}, \TClink[2]{pɛ}, \TClink[2]{pim}, \TClink{tilaŋ}. \textit{adj} \textbf{1)} other. \textit{Che risen pika ŋɔ gbi.} It is for no other reason. \textit{Be thipika thalɔ kɛ ache tha bɔ ku gbe.} No, there are other ones but there are too many. \textit{Yɔ pɛ bia kɔ hundɛ, wɔ pɛ bia koi li pika.} When he is going to come, he will also take other things. \textbf{2)} another. \textit{Dɛn yami wokɔ pɛ ko ba yi yɛ, wɔi bi nɔ pokan pika.} Then when mother left our father, she had another husband. \textit{So, mɔm ni po mɔ ŋaŋa ka tipɛn dɛ ɔ mɔm ni nɔ peka ŋa ni yɛ?} So, you and your first husband, or you are now with another person? \textit{I na pomdɛ i tipɛ i cheni pɛ, i na nɔ peka ini yɛ.} Me and my husband that started, we are no more, I am now with another person.

\TCheadword{pikchɔ} (Eng \textit{picture}) \textit{n} picture. \textit{Lomthinɔo, pikchɔthinɔo, lanɛ gbi wɔ tha chi, lipikaɛ pɔ lai ni be ki buk.} Your voice (recordings), your pictures, he will bring all of that, the rest will be put in books.

\TCheadword{pikɛ} (der. of \TClink[1]{pi}, \TClink{-k}, see \TClink[1]{pi})

\TCheadword{piki} (id. of \TClink[1]{pi}, \TClink[1]{ki}, see \TClink[1]{pi}) 

\TCheadword{pikith} \textit{cf}: \TClink{chok}, \TClink{thim}, \TClink{yikitha}. \textit{v} \textbf{1)} shake. \textbf{2)} wag.

\TCsubword{pikith-bol} (comp.) \textit{v} shake head.

\TCheadword{pikith-bol} (comp. of \TClink{pikith}, \TClink[1]{bol}, see \TClink{pikith}) 

\TCheadword{pil} \textit{cf}: \TClink{gbɔnthɔ}. \textit{n} palm wine dregs.

\TCheadword{pili} \textit{v} walk about. comp. \TClink{nɔpili} (see \TClink{nɔ})

\TCsubword{piliŋni} (der.) \textit{v} go around.

\TCheadword{piliŋni} (der. of \TClink{pili}, \TClink[2]{-n}, \TClink{-ni}, see \TClink{pili}) 

\TCheadword[1]{pim} \textit{n} dolphin.

\TCheadword[2]{pim} \textit{cf}: \TClink{halɛ}, \TClink[2]{pɛ}, \TClink[1]{pika} (der. of \TClink[2]{pika}), \TClink{tilaŋ}. \textit{adj} other. \textit{Kɛ ayenal pim Mbolom dɛ ma pɔ hɔ.} But in other places, it is Bolom they speak. \textit{Pimdɛ kɔnɛ kɔ pɔ bia joɛ, pɔ kɔi bɛ stɔ thai kunɛ.} The remainder will be put aside for food, will be kept in storage.

\TCheadword{pimpi} \textit{n} [pímpí] tree species, black tumbler that bears black fruit in clumps, some are sour, some sweet, one kind of ochre, the other straw-colored, eaten during brushing time (March), suck the seeds, can be put in a tumbler of water (K dialect); (kɔ/ma) tree species, black tumbler (Dialium guineense) (\citealt{Pichl1967}). 

\TCheadword[1]{pin} \textit{cf}: \TClink[6]{kɔ}, \TClink[2]{paka}. \textit{v} \textbf{1)} buy. \textbf{2)} pay. \textit{Hálíwɔ̀ hìn má Yèmà, wɔ̀ pín bállɛ̀ kò Chó.} Because he slept with Yema, he paid \textit{bal} to Cho.

\TCheadword[2]{pin} \textit{n} fly. \textit{Nɔ biɛn gbo thotho, chen siɛ pindɛ bi nchaŋ.} A man without sores will not know that the fly has teeth.

\TCheadword{pinthaŋ} \textit{cf}: \TClink{yereŋ}. \textit{v} [pìnthàŋ] be confused (K dialect). 

\TCheadword[1]{piŋ} \textit{adj} empty.

\TCheadword[2]{piŋ} \textit{n} [piŋ] insect species, fly (K dialect); \textit{pïng} (wɔ/hã, N) fly (\citealt{Pichl1967}).

\TCsubword{piŋbok} (comp.) \textit{n} (wɔ/hã, N) busybody, someone who meddles in all kind of affairs which don't concern him (\citealt{Pichl1967}). 

\TCheadword{piŋɡɛ} \textit{temp} first. \textit{Wɔnɛ fɔs wɔ piŋɡɛ yɛthi chukalɛ?} The first person that held the staff?

\TCheadword{piŋin} \textit{v} turn against; oppose. \textit{Ŋa jo ŋje ma sɔisɔi gbi ŋa piŋini gbo we.} They eat nice food, yet still they turn against us. \textit{Wante maiɛ ŋa lɔ we ŋa koi piŋiɛni.} Our sisters are all there; they have turned against us. \textit{Ŋa kul mɔi ma sɔisɔi gbi ŋa koi piŋiɛni.} They drink tasty drinks and they turn against us.

\TCheadword[1]{piŋki} \textit{v} \textbf{1)} transform. \textbf{2)} turn over. \textit{Labo thibɔm lɔ pɔ bia yukɛ, pɔ kɔ ni bɔm thai pɔi kɔ piŋgi bɔmdɛ ɔ pɔi gbusa.} If people have to plant where it is muddy, they will then turn the mud over or then they dig. \textit{Pɔ koŋ gbo raa pɔi piŋgi kaŋka inallɛ lɔ ŋa ni kɛlɛn.} After brushing, they have to turn over the soil so that it becomes clean. \textbf{3)} change. \textit{Pɛlɛ kɔi pith kɔi piŋgi, kɔi bi kun, kɔi gbemɔ.} The rice will get dark, and then it will change and swell up (lit. have a belly, i.e. be pregnant) and then tiller. \textbf{4)} turn into. \textit{Mɛŋkɛ ŋɔ Apotho aɛ ka che pin anyaɛ hiŋk Afrikaɛ, ŋà ŋá kɔ piŋkiɛ awokɛ.} The time when the white man was buying people from Africa, they went and turned them into enslaved people. \textit{Bikɔs nɔbɛndɛ koŋ gbo tham, ko piŋgindɛ tamɔ.} Because if an old person has become old enough, she has turned into a baby. \textbf{5)} become.

\TCsubword{piŋkini} (der.) \textit{v} \textbf{1)} turn to. \textbf{2)} turn into. \textit{Kel lɛ pinkiɛni nken lɛ nɔ.} The monkey turned himself into a person (\citealt{Pichl1967}).

\TCsubword{piŋki-piŋki} (der.) \textit{adj} variable, unreliable.

\TCsubword{piŋkilini} (unspec.) \textit{v} \textbf{1)} roll around. \textit{Tamɔ lɛ wɔ taŋ ni che pinkilini lɛ ko.} The child is crying and rolling around on the ground (\citealt{Pichl1967}). \textbf{2)} roll down. \textit{Pe lɛ pinkiliɛni hink rɔŋ dɛ atok.} The stone rolled down from the height of the mountain (\citealt{Pichl1967}). 

\TCheadword[2]{piŋki} \textit{v} reply. \textit{A ŋa leŋyi Nthemdai, lɛ ha lɛyɛmigbo Nthemdai aha piŋgiyɛ.} Yes, I greet them in Themne; if they greet me in Themne, I will reply the same.

\TCheadword{piŋkilini} (unspec. of \TClink[1]{piŋki}) 

\TCheadword{piŋkini} (der. of \TClink[1]{piŋki}, \TClink{-ni}, see \TClink[1]{piŋki}) 

\TCheadword{piŋki-piŋki} (der. of \TClink[1]{piŋki}) 

\TCheadword{piŋkliŋ} \textit{adv} aloud; out loud.

\TCheadword{piŋkta} \textit{v} stir up.

\TCheadword{pio} \textit{n} \textbf{1)}pig, [píó]/[píósɛ̀] pig/pigs (B dialect). \textbf{2)} hog.

\TCsubword{piɔm} (der.) \textit{n} manatee.

\TCheadword{piɔm} (der. of \TClink{pio}) 

\TCheadword{pipa} \textit{n} measles.

\TCheadword{pipɛ} \textit{n} cask; barrel. \textit{Mbɔŋ ma pipɛ ma bɛmpani iwɔm.} Barrel bungs are made of wood (\citealt{Pichl1967}).

\TCheadword{pir} \textit{n} (wɔ/hã, si) monkey species, has a white mouth (\citealt{Pichl1967}). 

\TCheadword{piriŋ} \textit{adv} around. \textit{Yɛ pɔ ka ka na, ken ŋɔ pɔ ŋa, pɔ bɛ lɔ simɛntɛ haŋ pɔ piriŋɛ ni...} If they had given here, like they did, they put cement there right around...

\TCheadword{pis} (Eng \textit{piece}) \textit{n} piece of cloth.

\TCheadword{pisa} \textit{v} [pìsà] be better, improve (K dialect). \textit{Ŋa pisa..} It is better... \textit{Tipɛni fisa.} He begins to be (or to feel) better (\citealt{Pichl1967}). 

\TCheadword{pithi} (comp. of \TClink[1]{pi}, \TClink[1]{thi}, see \TClink[1]{pi}) 

\TCheadword{pithika} \textit{n} rascality. comp. \TClink{nɔmpithika} (see \TClink{nɔ})

\TCheadword{pithilin-tholɛ} \textit{v} frown.

\TCheadword{piyaŋ} [pìyàŋ] \textit{v} not normal, when someone is not doing something correctly, not able to talk or to say anything you understand (K dialect). 

\TCheadword{piyɛtpiyɛt} \textit{cf}: \TClink[1]{tata} (der. of \TClink{taa}), \TClink{tonton} (der. of \TClink[1]{ton}). \textit{adj} [píyɛ́tpíyɛ́t] very small (K dialect). 

\TCheadword{piylɛ} \textit{cf}: \TClink{lepi} (der. of \TClink[1]{lap}, \TClink[1]{-i}). \textit{n} disgrace.

\TCheadword{Plantin} \textit{nam} Plantain Island, name given to a place. \textit{Mbolomdɛ, Plantin ka lɔ mɔi kiɛ, man ni Nthemdɛ handɔ mapɔ chaŋ thelia?} The Sherbro, on Plantain (Island) here where you are, Bolom or Themne, which do they speak more? \textit{Ŋha ya lemɛ nɔ len la haani, rɔŋ ayén Planti ko.} Let me tell you something that happened, a true story at Plantain (Island).

\TCheadword{ple} (Eng \textit{play}) \textit{cf}: \TClink[2]{siŋ}. \textit{v} play. \textit{Abibo tep, akɔ ŋɔ hok a ple.} If I have a tape, I take it out and play (it). \textit{Wɛl i ka che ple han tɛnis bɔl, ni iple chɔch, ni thipika.} We used to play hand tennis ball, and we play church, and other ones.

\TCheadword{plet} (Eng \textit{plate}) \textit{cf}: \TClink{boi}, \TClink{chɛnchi}. \textit{n} plate. \textit{Mɔi thɔk sɛyɛ ni pletɛ lɔ po mɔɛ bia huŋ bɛth joɛ.} You wash the spoon and plate where you have to come and cut the rice. \textit{Ni mbɛthɛwɔ pletɛ kunɛ mɔ wɔi ka.} And cut for him on the plate and give it to him.

\TCheadword{plɛn} (Eng \textit{plane}) \textit{cf}: \TClink[2]{balon}, \TClink{wɔmtokɛ} (comp. of, id. of \TClink[2]{wɔm}, \TClink[1]{tokɛ}). \textit{n} airplane. \textit{Plɛn dɛ kɔn poto kɛthkɛth hink Kyamp ka.} The plane goes frequently from Freetown to Europe (\citealt{Pichl1967}). 

\TCheadword{plɔm} (Eng \textit{plum}) \textit{n} plum. \textit{A yuk pɛlɛ, a yuk ikonatɛ, a yuk inɛs ɛ, mpanth vɛ maa kunɛ, a yuk mplɔmdɛ.} I plant rice, I plant coconut, I plant pineapple, that is the work I am into, I plant plums too.

\TCheadword[1]{po} \textit{v} \textbf{1)} arise. \textit{Yɛ ya lɔl ya po yɛ wɔɛ ŋɔ keni we…} When I sleep and wake up early in the morning… \textit{Achɔŋɔ Bɛi bullɛ sɛkɛ ya po ni velɛ.} I give the only one God thanks that I wake up healthy. \textbf{2)} get up. \textbf{3)} awaken. \textbf{4)} [yóó] grow up (K dialect). \textit{Yoo, nɔmaaɛ wɔ yoo} Grownup, the woman is grown. \textit{Lɛ awokɔlɔ gbopɛ, yɛ laio wɛ, yɛ ŋa ko ni po kinɛi yɛ mi chala ni}… If I leave that path, as it is, when you had grown and your mother was still there… \textit{Apumahiyɛ bɛ ŋa po bo ŋa labi ŋa the la.} Our children also, when grown up, they will hear it. \textbf{5)} begin.

\TCsubword{pokɔ} (comp.) \textit{v} grow up. \textit{A-a apokɔni thi tɔn.} No, I did not grow up knowing how to sing.

\TCsubword{pɔl...len} (unspec.) \textit{v} grow; thrive. \textit{pɔl…len} discontinuous form.

\TCheadword[2]{po} \textit{n} beach.

\TCheadword[3]{po} \textit{cf}: \TClink{prim}. \textit{n} [póɛ̀] pigeon, different from dove, \textit{prim} [r trilled, V central] (K dialect). 

\TCheadword[4]{po} \textit{cf}: \TClink{kosi}, \TClink{sɛin}. \textit{v} \textbf{1)} share. \textit{Ya bɔnthɔ wɔ poo yekə, ya thom wɔ ni kənklɛni.} I met him sharing cassava; I begged him (for some), but he refused (\citealt{Pichl1967}). \textbf{2)} separate; divide. \textit{Poo pɔk lɛ.} To divide the country (\citealt{Pichl1967}). \textit{M poo shiliŋ bul ndel nra.} Divide one shilling into three parts (\citealt{Pichl1967}). 

\TCsubword{poni} (der.) \textit{v} separate; divide. \textit{Yema ni əpook Kãy ha koŋ pooni.} Yema and her husband Kay are divorced (separated) (\citealt{Pichl1967}). 

\TCheadword[5]{po} \textit{n} husband. \textit{I ko vei ina pomdɛ o, iko bɛ chaŋ nɛnthi waŋdɛ.} We have stayed together me and my husband, now more than ten years. unspec. \TClink{nɔpokan} (see \TClink{nɔ}), \TClink{ŋɔhɔlpok} (see \TClink{nɔ}) 

\TCsubword{pokan} (unspec.) \textit{n} \textbf{1)} man. \textit{Ya wokɛ ko kaŋdɛ ai munini ko ichɛli ba mi bikɔ nɔ pikan pika che ŋa ni.} When I finished learning, I had to return to my father's seat because there was no other man there. \textbf{2)} male. \textit{Apokandɛ ŋan gbi ŋa ka koŋ wu?} All the males were dead? \textbf{3)} husband. \textit{Dɛn yami wokɔ pɛ ko ba yi yɛ, wɔi bi nɔ pokan pika.} Then when mother left our father, she had another husband. \textit{Bɛl Pokan dɛ: “Mba yaŋ ya mɔ hɔm vɛ?”} Rat Husband: “Woman, is it me you are abusing like that?” \textbf{4)} boy. \textit{Yɛ imath-mathnindɛ apikandɛ ŋani thoŋi-thoŋi siŋthɛ vɛ…} When we would hide and the boys would run after us, (in) those games… \textbf{5)} person. \textit{Haaŋ ni la muɛ ko apokana tirɛ} Until it then reached the townspeople. \textit{Apokana tirɛ ŋae hɔɛ, Taalaŋgba ki koŋ simi saba tirɛ njɛm.} The townspeople then said, this man has spoiled the town law. comp. \TClink{nɔpokan} (see \TClink{nɔ}), \TClink{ŋɔhɔlpok} (see \TClink{nɔ}), \TClink{rapokan} (see \TClink[3]{ra}), \TClink{rɛmpokan} (see \TClink{rɛm}), \TClink{rɛmsupokan} (see \TClink{rɛm}), \TClink{santhilpokan} (see \TClink{santhil}), \TClink{sɔkpokan} (see \TClink{sɔk}), \TClink{taapokan} (see \TClink{taa}), \TClink{tamɔpokan} (see \TClink{taa}) 

\TCheadword[6]{po} \textit{cf}: \TClink{tipɛ}. \textit{v} start. \textit{Wɔi kɔni pɔyko, yɛ kɔni yɛ wɔi ko sɛm ko thɔkɛ, wɔi po ŋa tɔn.} And then she goes to the stream, when she went to the stream, she stood by the tree, and then she started to sing.

\TCheadword{poa} \textit{v} snatch.

\TCheadword[1]{poepoe} (der. of \TClink[1]{poi}) 

\TCheadword[2]{poepoe} (der. of \TClink[2]{poi}) 

\TCheadword{poɛpoɛ} (der. of \TClink[1]{pɔ}) 

\TCheadword[1]{poi} \textit{adj} lightweight.

\TCsubword[1]{poepoe} (der.) \textit{adj} lightweight.

\TCheadword[2]{poi} \textit{temp} early. \textit{Chencha bɛ ya kɔɛ akɔni poi.} Even yesterday when I went, I didn't go early.

\TCsubword[2]{poepoe} (der.) \textit{temp} [póépóé] early (K dialect). 

\TCheadword{poiŋ} \textit{v} [póíŋ] raise, e.g., raise from the dead (K dialect). 

\TCheadword[1]{pok} \textit{v} leave. \textit{Anyaɛ bai ko bul wɔe gbaki ni hɔɛ, “Bɛra, ŋa pokɔ mi lɔ ka.”} Of the people in the bari, one said, “Gentlemen, get out of here.”

\TCheadword[2]{pok} \textit{n} medicine people swear on (\citealt{Hall1938}).

\TCheadword{pokan} (unspec. of \TClink[5]{po})

\TCheadword{pokɔ} (comp. of \TClink[1]{po}, \TClink[4]{kɔ}, see \TClink[1]{po}) 

\TCheadword[1]{pol} \textit{n} serenade.

\TCheadword[2]{pol} \textit{v} be foolish.

\TCheadword[1]{poloŋ} \textit{cf}: \TClink[2]{poŋ}. \textit{Loc} [pólóŋ] far away (K dialect). \textit{Kɔn gbes ko poloŋ.} He is gone far away to the east (\citealt{Pichl1967}). \textit{Wɔ gbo chaŋchaŋ poloŋ sin la wɔ ha lɛ.} He only goes about from place to place and does not know what to do (\citealt{Pichl1967}). 

\TCheadword[2]{poloŋ} \textit{n} [pòlòŋ] tree species, cotton tree used for canoes, leaves used for sauce and medicine (K dialect). \textit{Mɔ-m kɔ bɔnth che-ko ko poloŋ dɛ.} You will meet me before the cotton tree. (\citealt{Pichl1967}). \textit{Poloŋ dɛ kɔ gbo kil mi lɛ ntɛɛnt.} The cotton tree is right near my house (\citealt{Pichl1967}). unspec. \TClink{yekɛpoloŋ} (see \TClink{yekɛ}) 

\TCheadword{Pondo} \textit{nam} Pondo, female name given by a society. 

\TCheadword{poni} (der. of \TClink[4]{po}, \TClink{-ni}, see \TClink[4]{po}) 

\TCheadword[1]{poŋ} \textit{v} feed.

\TCsubword{poŋ … nin} (comp.) \textit{v} feed the Poro devil.

\TCheadword[2]{poŋ} \textit{cf}: \TClink[1]{poloŋ}. \textit{Idph} \textbf{1)} of being far away. \textit{Wɔe kɔni pɔk livil poŋ ha kɔ lɛliɛ waaŋmaa.} He went far away to find (look for) a woman. \textit{La koŋ wɔ yɔk poŋ, koŋ yereŋ gbi.} He was carried far away and was completely confused. \textbf{2)} of disappearing. \textit{Ni ŋa muni thaŋni, kara-kara, kara-kara, kara-kara poŋ! baiɛ tokɛ tɔrɔth.} And they return to climbing up… gone! up the bari \textit{tɔrɔth} (idph of emphasis).

\TCheadword{poŋ … nin} (comp. of \TClink[1]{poŋ}, \TClink[3]{min}, see \TClink[1]{poŋ}) 

\TCheadword[1]{poŋk} \textit{v} put. \textit{Poŋk pia lal lɛ ai ko.} He put his hand into the fire (\citealt{Pichl1967}).

\TCheadword[2]{poŋk} \textit{Idph} of being very red.

\TCheadword{pool} \textit{cf}: \TClink[2]{jɛth} (der. of \TClink[1]{jɛth}), \TClink{jobɔi}. \textit{adj} [póól] not strong (K dialect).

\TCheadword{pos} \textit{cf}: \TClink{pɔŋk}. \textit{v} peel. \textit{Amaa ki, apum ŋa pos gbam dɛ, apum ŋa pos yekeɛ.} These women, some were peeling potatoes, others peeling cassava.

\TCheadword[1]{pot} \textit{n} \textbf{1)} [pòt] palm species, swamp palm, like \textit{kɛn} but has no palm wine or raffia, get \textit{chak} ‘fiber,' which is fermented, then beaten (K dialect). \textbf{2)} thatch (\citealt{Pichl1967}).

\TCheadword[2]{pot} \textit{cf}: \TClink{meni}, \TClink{woso}. \textit{n} clay.

\TCheadword{poth} \textit{cf}: \TClink{nal}. \textit{n} \textbf{1)} earth. \textit{Pɔ gbaŋga wɔ bo pothɛ atok, pɔi nu bikɛ pɔ bim wɔ lɔ atok.} After he would be put in the ground, they would fold the mat then they would put the corpse on it. \textit{Pɔi tholi ni pɔ yɛthiɛ ŋɔ, pɔi bɛ pothɛ.} They put it down and would lower it, and then they add the dirt. \textbf{2)} mud. comp. \TClink{thullipoth} (see \TClink{thul}) 

\TCheadword{Potho} (Port \textit{português}) \textit{cf}: \TClink{nɔyeŋkes} (comp. of \TClink{nɔ}, \TClink[1]{yeŋkes}) \textbf{1)} \textit{n} white people. \textit{Wɛl atipɛ tɔn nɛndɛ ŋɔ Apothoɛ ŋa wɔ 2013, te mɛŋko ki amu tɔndai.} Well, I started singing in the year that white people call 2013, up to this year I'm still singing. \textit{Kenɛki-kenɛki wantɛ yi bɛndɛ wɔ pɔk Potho wɔ yi sɔpɔt.} This time now, we have our sister in the whiteman's country who supports us. \textbf{2)} [pòthò] \textit{n} English language (K dialect). \textit{Lɛ mbɔn gbo hɔ mpootoo lɛ koot l'ay, mɔ lɛ Bolomnɔ, Themnɔ, Mendenɔ.} If you do not speak English at the court, there is someone who will interpret for you in your language, be you Bolom, Themne, or Mende (\citealt{Pichl1967}). \textbf{3)} \textit{n} European. \textit{Gbemni abəka lɛ ni nche ma ha lɛ ma fəsɛ ha ma apotoa.} The inheritance and the way of the life of the Krios resemble those of the Europeans (\citealt{Pichl1967}). \textit{A si rai lɛ pootoo.} I know the book of the Europeans, i.e., I am literate (\citealt{Pichl1967}). \textbf{4)} \textit{nam} Europe, name given to a place. \textit{Plɛn dɛ kɔn poto kɛthkɛth hink Kyamp ka.} The plane goes frequently from Freetown to Europe (\citealt{Pichl1967}). comp. \TClink{bɛlpotho} (see \TClink[2]{bɛl}) 

\TCsubword{Pothonɔ} (comp.) \textit{n} white man. \textit{Ye lai bikɔs ivin Pothonɔ ki yɔ hun ke nɔ ndɔndɔ ko wɔko, lɔ yen-yen, pɔ che diskres nɔ.} That is it, because even when this white man came here, he saw everybody in his place, the place is quiet, they do not disgrace people. \textit{Ilɛllɛ Planti ko: Pothonɔ bul wɔ ka chal yel nsaŋha ko, wɔ ilel wɔɛ ka cheɛ Jɔn Planten dɛ.} The name Plantain: A white man who resides on Egusi was named John Plantain.

\TCheadword{Pothonɔ} (comp. of \TClink{Potho}, \TClink{nɔ}, see \TClink{Potho}) 

\TCheadword{pothɔhɔl} (der. of \TClink{potɔhɔl} (comp. of \TClink{poto}, \TClink[1]{ahɔl}), see \TClink{poto}) 

\TCheadword{poto} \textit{temp} \textbf{1)} April-May (\citealt{Pichl1967}). \textbf{2)} summertime (\citealt{Pichl1967}). \textit{Poto lɛ koŋ tipɛ, ipuluk lɛ tipɛ puuki.} Summer has begun, the grass begins to blossom (\citealt{Pichl1967}).

\TCsubword{potɔhɔl} (comp.) [póthɔ̀hɔ́l] \textbf{1)} \textit{nam} June (K dialect). \textbf{2)} \textit{n} end of March (\citealt{Pichl1967}). \textbf{3)} \textit{n} springtime (\citealt{Pichl1967}). \textit{Potɔhɔl lɛ koŋ moɛy, ngbemaŋ dɛ tipɛ wantiŋ.} When springtime has come, the fruit trees begin to blossom (\citealt{Pichl1967}).

\TCsubword{pothɔhɔl} (der.), (der. of \TClink{potɔhɔl}) \textit{n} [póthɔ̀hɔ́l] insect species, like black ants, given the name because they come out in June, have many arms like millipede, but much smaller (K dialect). 

\TCheadword{potogi} (Port \textit{português}) \textit{n} Portuguese language.

\TCheadword{potɔhɔl} (comp. of \TClink{poto}, \TClink[1]{ahɔl}, see \TClink{poto})

\TCheadword{poyok} \textit{n} plant species, Afrolicana elaerpermum (\citealt{Pichl1967}). 

\TCheadword{Pɔ} \textit{nam} Poro Society. comp. \TClink{walpɔ} (see \TClink[2]{wal}) 

\TCsubword[1]{pɔŋchaŋchaŋ} (comp.) \textit{nam} Poro ceremony location.

\TCsubword[2]{pɔŋchaŋchaŋ} (comp.) \textit{v} complete final stage of Poro.

\TCheadword[1]{pɔ} \textit{v} be fresh.

\TCsubword{poɛpoɛ} (der.) \textit{v} be very fresh.

\TCheadword[2]{pɔ} \textit{v} fetch water. \textit{Mpanth ma apuma maɛ, a kɔ pɔɛ, atu, ko gbi lɔ yema mi bo womdɛ.} The work of the girl children, I go to fetch water, I pound, where ever she wants to send me. \textit{Paali pagbondɛ akɔni pɔiko, ale sɛmi kɛmdɛ akoŋ kɔni ale kɔ siŋɛ.} The whole day if I go to fetch water, I will leave the bucket then I go play.

\TCheadword{pɔba} (Port \textit{pólvora} ‘gunpowder') \textit{n} gunpowder.

\TCheadword{pɔɛ} \textit{n} waterside.

\TCheadword{pɔhɔ} \textit{v} give.

\TCheadword{pɔi} \textit{n} brother-in-law.

\TCheadword[1]{pɔk} \textit{n} \textbf{1)} country. \textit{Pɔki Salon dɛ, pɔ ko ha jagbe.} In our country Sierra Leone, they have done a lot. \textit{Kɛ pɔk pim kɔlɔ nyanɔɛ pɔ cheŋ wɔ ka fɔsa, hin ka gbo.} But in other countries if a stranger goes there, they would not give him power, only we here. \textbf{2)} land. \textbf{3)} region, district. \textit{Nɛn thiwaŋnihiɔl, gbemni Fuŋk ko, Pɔk Bompɛɛ, Pɔk Nyambaɛ.} Fourteen years old, born in Rotifunk, Bumpeh Chiefdom, Moyamba District. \textbf{4)} chiefdom. \textit{Sundu ko Pɔk Kagbɔɛ ki?} Sundu in Kagboro Chiefdom?

\TCsubword{pɔkmɛkin} (comp.) \textit{n} end of the world.

\TCsubword{Pɔkpoto} (comp.) \textit{nam} Europe, name given to a place. 

\TCheadword[2]{pɔk} \textit{n} secret society.

\TCheadword[3]{pɔk} [pͻ̀k] \textit{n} \textbf{1)} \textit{pɔɔk} (wɔ/hã, si) heron (\citealt{Pichl1967}). \textbf{2)} bird species, seabird, thin, white, some dark grey, egret? (K dialect). 

\TCsubword{pɔkdinthɛ} (comp.) \textit{cf}: \TClink{malka}. \textit{n} cattle egret.

\TCsubword{pɔkyagba} (comp.) \textit{n} \textbf{1)} blue heron. \textbf{2)} [pͻ̀kyàŋgbà] seabird species, same as \textit{pɔk} but bigger (K dialect). 

\TCheadword{pɔkdinthɛ} (comp. of \TClink[3]{pɔk}, \TClink{dinthɛ} (der. of \TClink{dinth}, \TClink{-ɛ}), see \TClink[3]{pɔk}) 

\TCheadword{pɔkmɛkin} (comp. of \TClink[1]{pɔk}, \TClink[1]{mɛkin} (der. of \TClink[1]{mɛk}, \TClink[1]{-n}), see \TClink[1]{pɔk}) 

\TCheadword{pɔkɔn} \textit{v} forget.

\TCsubword{pɔkɔni} (der.) \textit{v} forget. \textit{Chɛliɛ mi tɛn wɛy ya che kɔn pɔkɔni.} He created a bad situation for me, I shall not forget it (\citealt{Pichl1967}). 

\TCheadword{pɔkɔni} (der. of \TClink{pɔkɔn}, \TClink[1]{-i}, see \TClink{pɔkɔn}) 

\TCheadword{Pɔkpoto} (comp. of \TClink[1]{pɔk}, \TClink{Potho}, see \TClink[1]{pɔk}) 

\TCheadword{pɔkyagba} (comp. of \TClink[3]{pɔk}) 

\TCheadword{Pɔl} \textit{nam} Paul, male name given to a person.

\TCheadword{pɔl} \textit{cf}: \TClink{sɔnthɔ}, \TClink{yɔŋ}. \textit{n} \textbf{1)} [pɔ́l] fish trap (K dialect). \textbf{2)} weir basket (\citealt{Pichl1967}). 

\TCheadword{pɔl...len} (unspec. of \TClink[1]{po}) 

\TCheadword{pɔli} (Eng \textit{Polly}) \textit{n} [pͻ̀lí] parrot (K dialect). 

\TCheadword{pɔlis} (Eng \textit{police}) \textit{n} police. \textit{Aa ha ka che theli Mbolomdɛ, wɔnɛ fli ka che OC police, Hestins.} Yes, they used to speak Sherbro, even the one (who) was an OC Police, Hastings.

\TCheadword{pɔllen} \textit{n} height.

\TCheadword[1]{pɔm} \textit{n} leaf. \textit{A si pɔmthɛ.} I know the leaves. \textit{Nshi pɔmthɛ?} Do you know (how to use) leaves? comp. \TClink{nɔmpɔm} (see \TClink{nɔ}), \TClink{sithapɔm} (see \TClink{sinthil}) 

\TCsubword{pɔmpɛlɛ} (comp.) [pͻ̀mpɛ̀lɛ̀] \textit{n} snake species, ‘leaf rice' called so because of its color, snake is green – people say it is poisonous but Ba Yanker has not seen any harm; not that afraid of people, a small snake of finger thickness, found in the bush (K dialect).

\TCsubword{pɔmthaba} (comp.) \textit{n} tobacco leaf. \textit{Bikɔs hin abena hiɛ pɔ thuka ŋa bo pɔm thaba.} Because our (emph.) parents were just married with tobacco leaf.

\TCheadword[2]{pɔm} \textit{cf}: \TClink[2]{bobo}, \TClink{nɔwu} (comp. of \TClink{nɔ}, \TClink[1]{wu}). \textit{n} corpse.

\TCsubword{pɔmul} (der.) \textit{n} dead person.

\TCheadword{pɔmpɛlɛ} (comp. of \TClink[1]{pɔm}, \TClink{pɛlɛ}, see \TClink[1]{pɔm}) 

\TCheadword{pɔmthaba} (comp. of \TClink[1]{pɔm}, \TClink{thaba}, see \TClink[1]{pɔm}) 

\TCheadword{pɔmul} (der. of \TClink[2]{pɔm}, \TClink{-ul}, see \TClink[2]{pɔm}) 

\TCheadword{pɔn} \textit{prep} on. \textit{Lɛ nɔ koyɛni gbo ha pɔn bɛmpa la, makɔni kɔtai, lokal kɔt.} If the person does not accept the settlement, they go to the court, the local court.

\TCheadword{pɔni} (der. of \TClink[2]{pɔŋ}, \TClink{-ni}, see \TClink[2]{pɔŋ}) 

\TCheadword[1]{pɔnth} \textit{n} [pɔ̀nth] parcel, something knotted (K dialect). \textit{Wuthi, ŋ kɔ wuthiɛ pɔnth lɛ vɛ hɔ hinkɔ Pootoo.} Untie or open something knotted, go untie that parcel for me that has come from Europe (\citealt{Pichl1967}). \textbf{2)} wetlands, swamp. \textit{Agbole chal pɔnthɛ ŋɔ hun mi bɔnth, a chɔŋɔ Hɔbatokɛ sɛkɛ halan.} I would just sit and see the wetlands near me, I thank God for that. \textit{Pɔnth lɛ hɔ trï bɔko.} The swamp is outside town (\citealt{Pichl1967}). \textit{Ba yentho bi lɔ hantha ka pənth lɛ ay.} There was a Mr. Leopard who had a fishing fence here in the swamp (\citealt{Pichl1967}). 3) \textit{n} food dish consisting of fish, peppers, onions, salt, etc., rolled in a leaf and baked in the fire (\citealt{Pichl1967}).

\TCheadword{pɔnthlɔŋ} \textit{n} [pͻnthlͻŋ] bird species, a small grey bird found everywhere, distinctive sound in the morning (K dialect).

\TCheadword{pɔnthpɔnth} \textit{n} (kɔ/ma) plant species, shrub in swamp used for manufacturing chairs and baskets (\citealt{Pichl1967}). 

\TCheadword[1]{pɔŋ} \textit{n} pound (monetary unit). \textit{Pɔn waŋ tɛmdɛ ibiɛn mu thauzin.} Ten pounds, at that time we did not have thousands.

\TCheadword[2]{pɔŋ} \textit{v} \textbf{1)} discard. \textbf{2)} cast. \textit{Ntolɛ, i pɔŋ hukɛ. Ihukɛ ŋɔi pɔŋɛ, aji.} You used tricks, we threw hooks. It is the hooks that we throw, (and) we caught (fish)! \textbf{3)} give out.

\TCsubword{pɔni} (der.) \textit{cf}: \TClink{pufuth}. \textit{v} \textbf{1)} get involved. \textit{Wɛl yami ka bɛmi skul kɛ akɔni livil, ŋɔ aka mɛl ayi pɔni ŋɔthɛ kunɛ.} Well, my mother sent me to school but I didn't go far, then I left and I involved myself in fishing. \textbf{2)} throw oneself into something. \textit{Nle kɔ bo mpɔni nwɔk mpika ntuk maɛ; labi la pɛthilɛ mini.} If you leave it and throw yourself into another language, you lose it; that is why it is not sweet to me. \textbf{3)} pour.

\TCsubword{pɔŋki} (der.) \textit{v} throw. der. \TClink{pɔŋkiɛn} (see \TClink[2]{pɔŋ})

\TCsubword{pɔŋkiɛn} (der.), (der. of \TClink{pɔŋki}) \textit{v} exchange words. \textit{Lanɔ ki gbi la bɛl siatiŋ dɛ ŋa pɔŋkiɛn thiyɛŋ dɛ...} This affair between the two rats exchanging words...

\TCsubword{pɔŋpɔŋ} (der.) \textit{v} throw away. \textit{Ibom-bom dɛ, pɔ ŋɔ pɔŋpɔŋ, pɔ che ŋɔ pɛ bia buŋ, pɔ pɔkɔni ŋa ŋɔn.} The big ones will be thrown away; they forget to do anything about it.

\TCheadword[1]{pɔŋchaŋchaŋ} (comp. of \TClink{Pɔ}, \TClink[1]{chaŋ}, see \TClink{Pɔ}) 

\TCheadword[2]{pɔŋchaŋchaŋ} (comp. of \TClink{Pɔ}, \TClink[1]{chaŋ}, see \TClink{Pɔ}) 

\TCheadword{pɔŋk} \textit{cf}: \TClink{pos}. \textit{v} peel.

\TCheadword{pɔŋki} (der. of \TClink[2]{pɔŋ}, \TClink{-k}, \TClink[1]{-i}, see \TClink[2]{pɔŋ})

\TCheadword{pɔŋkiɛn} (der. of \TClink{pɔŋki} (der. of \TClink[2]{pɔŋ}, \TClink{-k}, \TClink[1]{-i}), see \TClink[2]{pɔŋ}) 

\TCheadword{pɔŋpɔŋ} (der. of \TClink[2]{pɔŋ}) 

\TCheadword{pɔɔ} \textit{cf}: \TClink{hoɛ}. \textit{n} [pɔ̀ɔ̀] rain (same as Poro) (B dialect). \textit{Wɔ̀ìyɛ́ kò hṹn.} The rain is coming.

\TCheadword{pɔɔbɛl} \textit{n} [pɔ̀ɔ̀bɛ́l] grass species (K dialect). 

\TCheadword{pɔs} \textit{cf}: \TClink{gbe}, \TClink[1]{no}. \textit{quant} \textbf{1)} much. \textbf{2)} many.

\TCsubword{pɔsɔni} (comp.) \textit{v} not be much.

\TCheadword{pɔsɔni} (comp. of \TClink{pɔs}, \TClink[2]{ni}, see \TClink{pɔs}) 

\TCheadword{pɔthi} \textit{n} cup.

\TCheadword{pɔthkɔlɔ} \textit{n} \textbf{1)} [pɔ̀thkɔ̀lɔ̀] a sickness, smallpox (K dialect). \textbf{2)} cowpox (\citealt{Pichl1967}). 

\TCheadword{pɔti} (Eng \textit{pot}) \textit{n} mug.

\TCheadword{pɔy} \textit{n} stream. \textit{Yɛ kɔni yɛ wɔi ko sɛm ko thɔkɛ, wɔi po ŋa tɔn.} When she went to the stream, she stood by the tree, and she started to sing. \textit{Yawɔ wɔ wɔi wɔm tɛm dɛ ŋbi ŋɔ wɔ theni bo ndik ni kɔ ni pɔyko.} Her mother told her that anytime she is hungry, she should go to the stream.

\TCheadword{Prat} \textit{nam} Pratt, name given to a person. \textit{Wa maɛ, wɔ tika, Mɔmi Prat ki wante wɔi, wɔ tika.} A girl, she is in this town, Mummy Pratt's sister.

\TCheadword{prɛs} (Eng \textit{price}) \textit{cf}: \TClink{sɔŋkɔ}. \textit{n} price.

\TCheadword{pridɔminantli} (Eng \textit{predominantly}) \textit{adv} predominantly. \textit{Patikulali hi Amɔyaɛ ko a wokɛ lɔ pridɔminantli Muslim.} Particularly, we the Muslims, where I came from is predominantly Muslim.

\TCheadword{prim} \textit{cf}: \TClink[3]{po}. \textit{n} dove species. \textit{prim} dove, makes a low gurgling sound (pigeon louder and sharper, can be heard far away). 

\TCheadword{primi} (Eng \textit{preemie}) \textit{n} premature baby. \textit{Ye pɔ hɔ primiɛ vɛ, aagbemɔ landɛ kɔ kath.} When they say preemie, that (kind of) giving birth is difficult.

\TCheadword{prizina} \textit{n} imprisoned.

\TCheadword{prɔblɛm} (Eng \textit{problem}) \textit{n} problem. \textit{Wɔn kɛndɛ vɛ wɔ asɔthɔ bo prɔblɛm.} That is the only problem I had. \textit{Yɛlaio wɛ, yɛ jaɛ ma ko ŋani mgbeɛ ŋɔ maredɛ kɔ bi ni prɔblɛm thɛ.} Nowadays, when things are abundant, all the marriages are full of problems.

\TCheadword{Prɔf} \textit{nam} Professor. \textit{Yɛ bilaɛ Prɔf wɔn pɛ yema kɔ toŋgi lawɔɛ yɛ wɔ bia muniniɛ.} The reason is because Prof himself would want to go and show his wife after he has returned.

\TCheadword{prɔfit} \textit{n} profit.

\TCheadword{prpr} \textit{n} fishing chain.

\TCheadword[1]{pu} \textit{Idph} of being white. \textit{Wɔ dinthɛ <pu>.} He is very white.

\TCheadword[2]{pu} \textbf{1)} \textit{v} plunder. \textbf{2)} \textit{n} fight. \textbf{3)} \textit{n} war.

\TCheadword[3]{pu} \textit{n} fish species.

\TCheadword{pufuth} \textit{cf}: \TClink{pɔni} (der. of \TClink[2]{pɔŋ}, \TClink{-ni}, der. of \TClink[2]{pɔŋ}, \TClink{-ni}). \textit{v} [púfúth] jump into something, join an argument (K dialect). 

\TCheadword{puhapuha} \textit{n} sauce type, made of finely cut krenkren cooked together with rice and other ingredients.

\TCheadword{pui}\textit{n} [púí] first son who dies, the person buried only with leaves (K dialect). 

\TCheadword{puinɔ} (comp. of \TClink{nɔ}) 

\TCheadword{Pujoŋ} \textit{nam} Pujehun, name given to a place. \textit{Kilthi lɛ tha Pujoŋ kunɛ tha bom.} The houses in Pujehun are big (\citealt{Pichl1967}).

\TCheadword{puk} \textit{cf}: \TClink{bolmɔ} (comp. of \TClink[1]{bol}, \TClink[1]{mɔ}). \textit{n} \textbf{1)} navel. \textbf{2)} nipple. \textit{pukɛ} the nipple.

\TCsubword{pukhɔl} (comp.) \textit{Loc} around the navel.

\TCheadword{pukhɔl} (comp. of \TClink{puk}, \TClink[1]{ahɔl}, see \TClink{puk}) 

\TCheadword{puki} \textit{cf}: \TClink[2]{wantiŋ} (der. of \TClink[1]{wantiŋ}). \textit{v} blossom; bloom.

\TCheadword[1]{pukɔ} (der. of \TClink[2]{pukɔ}) 

\TCheadword[2]{pukɔ} \textit{cf}: \TClink{hwe}. \textit{n} foam.

\TCsubword[1]{pukɔ} (der.) \textit{v} foam.

\TCheadword[1]{pul} \textit{cf}: \TClink[1]{boo}, \TClink[2]{gber}. \textit{n} rice flour.

\TCheadword[2]{pul} \textit{n} ashes.

\TCheadword{pula} \textit{n} \textbf{1)} just a worm, one in the stomach is smaller, in the ground – dies quick due to insects and maybe heat, pigs love them (K dialect). \textbf{2)} stomach worms.

\TCheadword{puli} \textit{v} stir. \textit{Yɛ mɔ kɔ ni puliɛ, mɔ koi yabasɛ nbɛlɔ atok.} As you are mixing it, you take the onion and add it in.

\TCsubword{pulijo} (comp.) \textit{v} stir food.

\TCsubword{pulipuli} (der.) \textit{v} mix. \textit{Mɔ kɔi minɛ koŋ pulipuli gbi, joɛ, mɔi gbiŋgith.} You then mix it all, the food, then you cover it. \textit{Gbi ni ngefeyɛ, mɔi binthmabinthma mpuliɛpuliɛ mɔi nɛmil labo iyɛllɛ ŋɔ shilɔ che.} Together with the pepper, you mix it up, and then you taste it to know if the salt is okay.

\TCheadword{pulijo} (comp. of \TClink{puli}, \TClink[2]{jo}, see \TClink{puli}) 

\TCheadword{pulipuli} (der. of \TClink{puli}) 

\TCheadword{Puluk} \textit{nam} Puluk, name given to 7\textsuperscript{th} son. 

\TCheadword{puluk} [pùlùk] \textit{n} grass species. \textit{Ipuluk ɛ bɛ hɔ̃ tipɛ ho.} The grass also begins to sprout (\citealt{Pichl1967}). \textit{Pùlùkɛ́ bə̀mə̀kɛ́ lɛ́llɛ̀.} Grass covered the ground. \textit{Poto lɛ koŋ tipɛ, ipuluk lɛ tipɛ puuki.} Summer has begun, the grass begins to blossom (\citealt{Pichl1967}). 

\TCsubword{puluk-mɛn} (comp.) \textit{n} sea flora. 

\TCheadword{puluk-mɛn} (comp. of \TClink{puluk}, \TClink[3]{mɛn}, see \TClink{puluk}) 

\TCheadword{pulukɛ} [pùlùk] \textit{cf}: \TClink{kran}, \TClink[2]{tuntun}. \textit{n} a pile of leaves or trash (B dialect). \textit{Akoŋ gbo bas, adikilɛ gbo ipulukɛ ai le yini achaŋ-chaŋ tiko.} After sweeping, I will gather the dirty clothes and leave them there and travel about town. \textit{Ipulukɛ gbi ma lɔɛ pɔ ma lɔ koŋ hok.} All the piles (of branches and leaves) that are there are taken out.

\TCheadword[1]{pum} \textit{quant} some. \textit{Næthi lɛ thipum tha thikəlɛŋ.} Some roads are fine (\citealt{Pichl1967}). \textit{Yɛ ŋ kɔ gbo gadin dai, chiɛ mi mmango mpum.}	When you go to the garden, bring me some mangoes (\citealt{Pichl1967}).

\TCheadword[2]{pum} \textit{cf}: \TClink{tɛŋka}. \textit{adv} \textbf{1)} perhaps. \textbf{2)} maybe. \textit{Lɛ pɔ yiɛ wɔ gbo, pim wɔ bia wɔ, hok nɔ ntɛnt ni kɔni ayenal pika ha ko lɔ chal.} If they ask him, maybe he would say that they go far away to another place and stay there. \textit{Pim nɔ wɔ sɔtha nten Inglan la atheliɛ komɔko.} Maybe someone in England will understand what I said to you.

\TCsubword[1]{tɛmpum} (comp.) \textit{adv} \textbf{1)} maybe. \textbf{2)} perhaps.

\TCsubword[2]{tɛmpum} (comp.) \textit{temp} sometimes. \textit{Tɛmpim la koi ndɔi ntiŋ pɔ che wɔ kɔŋ, chaŋ pɔ koŋla.} Sometimes it would take two days without being buried, until the process is done.

\TCheadword[3]{pum} \textit{cf}: \TClink[1]{pumɔ}. \textit{n} children. \textit{Mɔ lɔ bɔnth apuma mɔ ɛ han gbi}. You will meet all your children there (\citealt{Pichl1967}). \textit{Apuma lɛ ha chɔ' yenwɛy, ha kɔ koosi.} The children are fighting badly; do go part them (\citealt{Pichl1967}). 

\TCsubword{pumaama} (comp.) \textit{cf}: \TClink[1]{waŋ}. \textit{n} daughters (pl. of \TClink[1]{waŋ}).

\TCsubword[2]{pumɔ} (der.) \textit{cf}: \TClink[1]{tata} (der. of \TClink{taa}). \textit{adj} young. 

\TCheadword{puma}\textit{v} [pùmá] being cuckolded, catch a man in another man's house with your wife (K dialect). 

\TCheadword{pumaama} (comp. of \TClink[3]{pum}, \TClink{nɔmaa} (der. of, comp. of \TClink{nɔ}, \TClink{maa}), see \TClink[3]{pum}) 

\TCheadword{pumaŋ} \textit{n} \textbf{1)} satisfaction. \textit{Yɛ wɔ ko joɛ wɔ yɛ pumaŋ ko bo pumaŋ.} After eating she became satisfied. \textbf{2)} full belly.

\TCheadword[1]{pumɔ} \textit{cf}: \TClink[3]{pum}. \textit{n} child. 

\TCheadword[2]{pumɔ} (der. of \TClink[3]{pum}) 

\TCheadword{pun} \textit{n} tree species, sumac (\citealt{Pichl1967}). 

\TCheadword{punth} \textit{n} \textit{ipunth} (hɔ̃/-) oyster or cockle shell (\citealt{Pichl1967}).

\TCsubword{punththɛ} (comp.) \textit{n} \textit{ipunth-thɛ} quicklime made of cockle shells (\citealt{Pichl1967}). 

\TCheadword{punththɛ} (comp. of \TClink{punth}) 

\TCheadword[1]{puŋ} \textit{n} [púŋ] boil, swelling (K dialect). 

\TCheadword[2]{puŋ} \textit{v} [pùŋ] ignite, catch something on fire (K dialect). 

\TCheadword{puŋki} \textit{n} water connection.

\TCheadword{pupende} \textit{n} [pùpéndé] grass species that grows in a stream, used for medicine, never completely submerged, used widely by herbalists (K dialect).

\TCheadword{pupun} \textit{cf}: \TClink{senthetha}. \textit{n} duck species, water ducks, large (larger than \TClink{senthetha}), slight horns on shoulders with which they can fight (K dialect). 

\TCheadword{puraŋ} \textit{Idph} [pùràŋ] of jumping into water (K dialect). 

\TCheadword[1]{puth} (der. of \TClink[2]{puth} (der. of \TClink[3]{puth}), see \TClink[3]{puth}) 

\TCheadword[2]{puth} (der. of \TClink[3]{puth})

\TCheadword[3]{puth} \textit{n} \textbf{1)} intestines, [\`{m}pút] [\`{m}pùt thɛ́] intestines/ the intestines (pl) (B dialect). \textbf{2)} guts.

\TCsubword[1]{puth} (der.), (der. of \TClink[2]{puth}) \textit{Idph} very stinky.

\TCsubword[2]{puth} (der.) \textit{v} be rotten. der. \TClink[1]{puth} (see \TClink[3]{puth}), \TClink{puthi} (see \TClink[3]{puth}), \TClink{puthul} (see \TClink[3]{puth})

\TCsubword{puthi} (der.), (der. of \TClink[2]{puth}) [púthí] \textit{v} burst (K dialect).

\TCsubword{puthul} (der.), (der. of \TClink[2]{puth})\textit{v} \textbf{1)} be rotten. \textit{Yu lɛ kong puthul, lɛ ŋgbəŋ wɔ gbo hinɛ gbo nɔth.} The fish is rotten already, if you touch it, you will find it quite soft (\citealt{Pichl1967}). \textit{Yu lɛ koŋ puthul, hɔ thuŋ puth.} The fish is rotten; it stinks awfully (\citealt{Pichl1967}). \textbf{2)} be spoiled. \textbf{3)} [pùthùl] be smelly (K dialect). comp. \TClink{kunputul} (see \TClink{kun}), \TClink{puthuli} (see \TClink[3]{puth}) 

\TCsubword{puthuli} (der.), (der. of \TClink{puthul}) \textit{v} make rotten. \textit{Mma puthuli komo lɛ wɔ ma choŋ lɛɛpi.} Don't spoil the child; it will make you ashamed in the future (\citealt{Pichl1967}). 

\TCheadword{puthi} (der. of \TClink[2]{puth} (der. of \TClink[3]{puth}), \TClink[1]{-i}, see \TClink[3]{puth}) 

\TCheadword{puthul} (der. of \TClink[2]{puth} (der. of \TClink[3]{puth}), \TClink{-ul}, see \TClink[3]{puth})

\TCheadword{puthuli} (der. of \TClink{puthul} (der. of \TClink[2]{puth} (der. of \TClink[3]{puth}), \TClink{-ul}), \TClink[1]{-i}, see \TClink[3]{puth})

\TCheadword{puthun} \textit{n} \textit{mputhun} (ma) by surprise, unexpected, unaware (\citealt{Pichl1967}).

\TCsubword{puthuni} (der.) \textit{v} [púthúnì] surprise (someone) (K dialect). 

\TCheadword{puthuni} (der. of \TClink{puthun}, \TClink[1]{-i}, see \TClink{puthun}) 

\TCheadword[1]{puy} \textit{n} (kɔ/-) grass species, kind of fieldgrass used to thatch a roof (Anadelpha etecta and other spp.) (\citealt{Pichl1967}). 

\TCsubword{puympene} (comp.) \textit{cf}: \TClink{peenɛmboŋ} (comp. of \TClink{peenɛ}, \TClink[2]{boŋ}). \textit{n} grass species.

\TCsubword{puysa} (comp.) \textit{n} (kɔ/-) grass species, (Ctenium newtonii, Andropagon gayanus and similar ones) (\citealt{Pichl1967}). [red grass?] 

\TCheadword[2]{puy} \textit{v} blow on fire (\citealt{Pichl1967}). 

\TCheadword{puympene} (comp. of \TClink[1]{puy}) 

\TCheadword{puysa} (comp. of \TClink[1]{puy}, \TClink[1]{sa}, see \TClink[1]{puy}) 

\end{letter}
\begin{letter}{R}

\TCheadword[1]{ra} \textit{cf}: \TClink{thri}. \textit{Numb} three; \textit{rà} three (\citealt{Sumner1921}). \textit{Aa, ba mi bi ama ara.} Yes, my father had three wives. comp. \TClink{mɛnra} (see \TClink[1]{mɛn}) 

\TCheadword[2]{ra} \textit{cf}: \TClink[1]{fama}, \TClink{woŋkru}. \textit{v} \textbf{1)} first-stage farm clearing; \textit{rá} clear bush for farm (\citealt{Sumner1921}). \textbf{2)} [rá] brush (K dialect). \textit{Wɔ ra ichɛkɛ, wɔ telɔ, kɛ wɔ ra.} He is a farmer, and also a tailor, but he brushes. \textbf{3)} farm. \textit{Wɔn pɛ mpanth bul lɛ ma bo wɔɛ wɔ ra.} She also does the same thing, farming. \textit{Mpanth ma thoɛ ma ka che ŋaɛ, ka che ra.} He used to do bush work, he used to farm. comp. \TClink{nɔra} (see \TClink{nɔ}), \TClink{nɔrachɛk} (see \TClink{nɔ}) 

\TCheadword[3]{ra} \textit{n} [rá] snake species, big snake, like the \textit{kɔɔ}, found in the bush, very poisonous, especially the female, male is bigger, two are different in color, male is bright yellow, both very swift – mambas – larger than the \textit{muŋka} – hard to see – female is quick to anger, may even chase a person (K dialect); (wɔ/hã, si) any kind of green snake (\citealt{Pichl1967}).

\TCsubword{ramaa} (comp.) \textit{n} (wɔ/hã, si) snake species, a green snake not said to be dangerous, emerald snake or green-lined snake (\citealt{Pichl1967}). 

\TCsubword{rapokan} (comp.) \textit{n} (wɔ/hã, si) snake species, a green snake said to be very dangerous, green mamba? (\citealt{Pichl1967}). 

\TCheadword{rai} \textit{n} \textbf{1)} book. \textit{A si ræ lɛ.} I am literate (know book) (\citealt{Pichl1967}). \textbf{2)} paper. \textit{Rælɛ hɔ̃ gbo ləm}. The paper is very thin (\citealt{Pichl1967}). \textit{Ræ hɔ̃ pə gbal ka thənkɔ.} It is on paper one writes with a pen (\citealt{Pichl1967}). \textbf{3)} letter. \textit{A yema wɔ ke̹ ni yɔk ræ lo ko ba wɔ.} I want to see him so that he may take this letter to his father (\citealt{Pichl1967}). \textbf{4)} studies. \textit{Lɛ mɛlɛn gbo ŋke̹n, thoma mɔ lɛ ve̹lɛng ræ lɛ.} If you let yourself go, your companion will excel you in the studies (\citealt{Pichl1967}). \textbf{5)} class. \textit{A mɛkɛni rai thimɛn.} I stopped at class five.

\TCheadword{rait} \textbf{1)} \textit{adv} right. \textbf{2)} \textit{n} right.

\TCheadword{rait naw} (comp. of, id. of \TClink{naw}) 

\TCheadword{raith} \textit{n} right.

\TCheadword{rak} \textit{n} (kɔ/ma) tree species, African oak (Lonchocarpus sericeus and Terminalia scutifera) (\citealt{Pichl1967}). 

\TCheadword{raaka} \textit{cf}: \TClink[1]{tuntun}. \textit{n} (kɔ/ma) plant species, burweed (Triumfetta cordifolia), its leaves are used to prepare a sauce (\citealt{Pichl1967}). 

\TCheadword{ram} \textit{cf}: \TClink{abɛna} (der. of \TClink[1]{bɛn}). \textit{n} \textbf{1)} family. \textit{Awa kɛ mi ŋɔ mɔ ni ŋa ja ramɔa?} How do you now do things for the family? \textit{Kɛ che yi koŋ sɔpɔt bikɔs ramdɛ kɔ bom che yi koŋ sɔpɔt gbi.} But she does not support us all, because the family is big. \textbf{2)} (kɔ/tha, si) clan (\citealt{Pichl1967}). \textbf{3)} kinship group in which descent is normally reckoned in the female line (\citealt{Hall1938}: 2). \textbf{4)} \textit{ramsi} (-/tha) generation (\citealt{Pichl1967}). comp. \TClink{gbutaram} (see \TClink{gbuta}) 

\TCheadword{ramaa} (comp. of \TClink[3]{ra}, \TClink{maa}, see \TClink[3]{ra}) 

\TCheadword{Ramani} \textit{nam} Ramani, name given to a person.

\TCheadword{Ramatu} \textit{nam} Ramatu, female name given to a person. \textit{Ama ŋa Kadiatu Bɛndu, Isata Bɛndu, Ramatu Bɛndu ni Aminata Bɛndu.} The women are Kadiatu Bendu, Isata Bendu, Ramatu Bendu and Aminata Bendu.

\TCheadword[1]{ramil} \textit{v} sacrifice food.

\TCheadword[2]{ramil} \textit{cf}: \TClink{soŋki} (der. of \TClink[1]{soŋk}, \TClink[1]{-i}). \textit{v} cure. \textit{Ya bi nrɔm ka, ma mɔ bɔ ramir.} I have a medicine here, it should cure you (\citealt{Pichl1967}). comp. \TClink{nɔramda} (see \TClink{nɔ}) 

\TCheadword[1]{ranth} \textit{n} bamboo cabbage.

\TCheadword[2]{ranth} \textit{n} [rànth] cane rope, kind of long rattan switch, woven, fiber also used to make fanners (winnowing baskets) (K dialect). 

\TCheadword[3]{ranth} \textit{v} [ránth] whip someone (K dialect). 

\TCheadword{raŋka} \textit{v} curse. \textit{Chen ndik ma mɔɛ, tama ni raŋka ŋɔ mɔɛ.} You are not hungry, it is foolishness and a curse upon you.

\TCsubword{raŋkani} (der.) \textit{v} be cursed.

\TCheadword{raŋkani} (der. of \TClink{raŋka}, \TClink{-ni}, see \TClink{raŋka}) 

\TCheadword{raparapa} \textit{n} wrapping. \textit{I koi pisthɛ iraparapa tha iŋakɔ mɔi bɔl.} We would take small pieces of cloth; we make it like ball.

\TCheadword{rapokan} (comp. of \TClink[3]{ra}, \TClink{pokan} (unspec. of \TClink[5]{po}), see \TClink[3]{ra}) 

\TCheadword{rasa} \textit{n} [rásá] kind of medicine, something they peel from a fig tree, useful for treating burning chest (K dialect). 

\TCheadword{re} \textit{cf}: \TClink[1]{chal}, \TClink{gbɛma}. \textit{n} antelope species, hunted and eaten (B dialect); (wɔ/hã, si) antelope species, small grey antelope, grey duiker (Philantomba maxwelli) (\citealt{Pichl1967}). 

\TCheadword{rebɛl} (Eng \textit{rebel}) \textit{n} rebel. \textit{Nrebɛllɛ ŋa hun, ŋa hun tho, ikɔni mɛn ko.} The rebels came, then they drove us into the countryside.

\TCheadword{ree} \textit{n} (kɔ/ma) bush species, teabush (Ocimum viride) (\citealt{Pichl1967}). 

\TCheadword{rek} \textit{v} build. \textit{Kɔ ha rek kil.} He went to build a house (\citealt{Pichl1967}). \textit{Pəlɛ Kɔŋ dɛ koŋ soth, ya bɔnthɔ wɔ rek tənkɛ chɛk lɛ, wɔ ko, apuma lɛ ha tipɛ yo loko tiŋ dɛ.} Kong's rice is coming out (read for harvesting); I found him erecting a platform on his farm; the children will begin to drive away the birds in two days (\citealt{Pichl1967}). 

\TCsubword{rekni} (der.) \textit{v} build for oneself. \textit{Yɛ kɔ lɔ rekni, nɔma wɔ ki sin dɛ ke vɛ kel.} When he went there to build himself (a house), the woman didn't know that he was a monkey (\citealt{Pichl1967}). 

\TCheadword{rekia} [rèkíá] \textit{n} great-grandchild, [rèkíá]/ [ǹrèkyá] great-grandchild/ great-grandchildren (B dialect). \textit{Lɛ ŋa yema bo won lɛŋ ko ŋanɛ ha hunɔn muɛ, ko nrekiaɛ ŋa pɔ gbemɛn muɛ.} What greeting would you want to send to those that have not come yet, the grandchildren, those that have not been born yet.

\TCheadword{rekni} (der. of \TClink{rek}, \TClink{-ni}, see \TClink{rek}) 

\TCheadword{remda} \textit{n} [remda] a viper like a boa, short and fat (K dialect); \textit{re̹mda} (wɔ/hã, si) snake species, (fabulous?) snake, said to have one horn on its nose and when attacking to jump very high and far (\citealt{Pichl1967}). \textit{Mɔm komɔ rɛmda ki, ya chen lan haa gbi.} You child of a viper, I will not do it--at all. 

\TCheadword{rɛ} \textit{n} shield.

\TCheadword{rɛdi} (Eng \textit{ready}) \textit{cf}: \TClink{woŋki}. \textit{adj} ready. \textit{Pabondɛ fli ni ŋɔ rɛdi ha hun, hɛ hɔ ha ni ki.} If really it is ready to come out, it will make like this. \textit{Wɛl tɛmdɛ gbi ŋɔa rɛdiɛ akɔ hɛlɛ ko, ŋɔn ŋɔ biɛni standad taim.} Well at any time I am ready and will go out to sea, it does not have a standard time.

\TCheadword{rɛdilɛ} \textit{n} \textbf{1)} redileh. \textbf{2)} cannibalism.

\TCheadword{rɛkɔd} (Eng \textit{record}) \textit{n} recordings. \textit{Pɔ ple rɛkɔd mɔɛ, pɔi theɛ lom mɔɛ ŋɔ nche pa theliɛ.} They will play your recordings, then they will hear your voice, how you used to talk.

\TCheadword{rɛkɔdɛ} (Eng \textit{record}) \textit{v} record. \textit{Pɔ ple rɛkɔd mɔɛ, pɔi theɛ lom mɔɛ ŋɔ nche pa theliɛ.} They will play your recordings, then they will hear your voice, how you used to talk. 

\TCheadword{rɛm} \textit{n} (hɔ̃/tha) toe (\citealt{Pichl1967}). \textit{Bia bɛth rəm wɔ lɛ thɛmni yenwɛy nai lɛ bol.} Bia has cut his toe, he stubbed it badly on the way (\citealt{Pichl1967}). 

\TCsubword{rɛmbom} (comp.) \textit{n} (hɔ̃/tha) big toe (\citealt{Pichl1967}).

\TCsubword{rɛmpokan} (comp.) \textit{n} (hɔ̃/tha) big toe (\citealt{Pichl1967}).

\TCsubword{rɛmsupokan} (comp.) \textit{n} (hɔ̃/tha) middle toes (3d and 4th toe) (\citealt{Pichl1967}).

\TCsubword{rɛmtaa} (comp.) \textit{n} (hɔ̃/tha) last toe (\citealt{Pichl1967}).

\TCheadword{rɛmbom} (comp. of \TClink{rɛm}, \TClink{bom}, see \TClink{rɛm}) 

\TCheadword{rɛmpokan} (comp. of \TClink{rɛm}, \TClink{pokan} (unspec. of \TClink[5]{po}), see \TClink{rɛm}) 

\TCheadword{rɛmsupokan} (comp. of \TClink{rɛm}, \TClink[1]{su}, \TClink{pokan} (unspec. of \TClink[5]{po}), see \TClink{rɛm}) 

\TCheadword{rɛmtaa} (comp. of \TClink{rɛm}, \TClink{taa}, see \TClink{rɛm}) 

\TCheadword{rɛn} \textit{n} age. \textit{Pe rɛnthɛ, Laɔn ɔf Juda.} Rock of ages, Lion of Judah.

\TCheadword{rɛnth} \textit{cf}: \TClink{kilɛihɔl} (comp. of \TClink[1]{kil}, \TClink[1]{ɛ}, \TClink[1]{ahɔl}). \textit{n} door.

\TCheadword{Rɛnu} \textit{nam} Renu Society.

\TCheadword{rɛs} \textit{n} (hɔ̃/tha) kerchief or piece of cloth sewn in the form of a bikini or bathing trunk and worn by boys and girls, and for the latter, a sign of virginity (\citealt{Pichl1967}). comp. \TClink{waŋmarɛs} (see \TClink[1]{waŋ}) 

\TCheadword{rɛspɛkt} (Eng \textit{respect}) \textit{cf}: \TClink[2]{mani}, \TClink[1]{yiki}. \textit{n} respect. \textit{Anyindɛ kache, ŋɔ pɔ kache ŋa trit a? Apima atata ŋa ka bi rɛspɛkt ŋa ayin?} The people in those days, how were they treated? Did the children have respect for people?

\TCheadword{rɛsth} (Eng \textit{rest}) \textit{cf}: \TClink{hin}, \TClink[2]{hɔl}. \textit{v} rest. \textit{Aftabakɛ vɛ ŋɔ le rɛsthɛ.} The afterbirth rests a bit.

\TCheadword{rɛth} \textit{v} \textbf{1)} wide. \textbf{2)} broad.

\TCsubword{rɛthi} (der.) \textit{cf}: \TClink[1]{lath}, \TClink[1]{lath}, \TClink[1]{sak}. \textit{v} \textbf{1)} spread out. \textbf{2)} reduce. \textit{Haa yɛ mɔ kɔ yiɛ mɛndɛ ma shi gbo che, mɔi rɛthi jɛmdɛ ton-ton.} Then you open (the pot), if the water is just as it should be, you reduce the fire a little.

\TCheadword{rɛthi} (der. of \TClink{rɛth}, \TClink[1]{-i}, see \TClink{rɛth}) 

\TCheadword{rigbɛrigbɛ} \textit{v} thick (of liquids) (\citealt{Pichl1967}). \textit{Sup lɛ hɔ̃ rïgbɛrïgbɛ.} The soup is thick (\citealt{Pichl1967}). 

\TCheadword{rik} [rɪ́k] \textit{cf}: \TClink[1]{kan}, \TClink[2]{lo}. \textit{v} weave. \textit{Mbàŋsɛ̀ ŋà rɪ́k wàɛ̀ tòkɛ̀.} The weaver birds wove (their nests) at the top of the palm tree.

\TCheadword{Rikɛ} \textit{nam} Rike, name given to 5\textsuperscript{th} son. 

\TCheadword{rikisi} \textit{n} duplicity.

\TCheadword{rilijɔn} (Eng \textit{religion}) \textit{n} religion. \textit{Ligbe ba la hun ni ŋɔ pɔ vellɛ, ŋɔi hɔni Mpothoai ɛ rilijɔndɛ la ko hundɛ, Kristiandɛ} Many things have happened in what we called in English religion, Christianity.

\TCheadword[1]{rim} \textit{cf}: \TClink{suŋkuthani} (der. of \TClink[1]{suŋkutha}, \TClink{-ni}). \textit{v} \textbf{1)} [rím] perish (K dialect). \textbf{2)} be destroyed (\citealt{Pichl1967}). \textit{Trï lɛ hɔ̃ koŋ rim.} The town has been destroyed (\citealt{Pichl1967}). \textit{Anya lɛ koŋ rim.} The population was annihilated (by a catastrophe, etc.) (\citealt{Pichl1967}). 

\TCsubword{rimi} (der.) \textit{cf}: \TClink{simi}. \textit{v} destroy.

\TCheadword[2]{rim} \textit{Idph} of pitch blackness (\citealt{Pichl1967}).

\TCheadword[3]{rim} \textit{n} \textbf{1)} [rìm] steam (K dialect). \textit{Rïm dɛ kɔ hok tii-kɛtïl l'ay.} The steam comes out of the tea kettle (\citealt{Pichl1967}). \textbf{2)} cloud (\citealt{Pichl1967}). 

\TCheadword[4]{rim} \textit{v} be boring. \textit{I chala boɛ ni iŋa group, iwɔ kaŋga gbo chɔchɛ lɔma rim.} We just sat and decided to form a group, we say just for the church not to be boring.

\TCheadword{rimi} (der. of \TClink[1]{rim}, \TClink[1]{-i}, see \TClink[1]{rim}) 

\TCheadword{riŋ} \textit{n} tusk.

\TCsubword{riŋpiɛ} (comp.) \textit{n} ivory.

\TCheadword{riŋpiɛ} (comp. of \TClink{riŋ}, \TClink{piɛ}, see \TClink{riŋ}) 

\TCheadword{ripɔt} \textit{n} report.

\TCheadword{risen} (Eng \textit{reason}) \textit{cf}: \TClink{bila}, \TClink[1]{ja}, \TClink[2]{yen}. \textit{n} reason. \textit{Che risen pika ŋɔ gbi.} It is for no other reason.

\TCheadword{ritaya} (Eng \textit{retire}) \textit{v} retire. \textit{Wɔn bɛ ko ritaya, yɛlaio wɛ.} She herself has retired, as it is.

\TCheadword[1]{rithi} \textit{n} darkness. \textit{Cheche yɛ rithi yai yo.} The light in darkness-o.\textit{ Nɔɛ wɔ kil rithiaiɛ wɔ mɔ tonki ichɛli.} It is the person in the dark house that shows you where to sit (proverb). comp. \TClink{kilrithi} (see \TClink[1]{kil}) 

\TCsubword[2]{rithi} (der.) \textit{cf}: \TClink{pithi} (comp. of \TClink[1]{pi}, \TClink[1]{thi}). \textit{v} be dark. comp. \TClink{chol-rithi} (see \TClink[1]{chɔl}), der. \TClink{rithilɛhɔl} (see \TClink[1]{rithi})

\TCsubword{rithilɛhɔl} (der.), (comp. of \TClink[2]{rithi}) \textit{cf}: \TClink{hwɛpi} (comp. of \TClink[2]{hu}, \TClink[1]{pi}). \textit{temp} dusk. 

\TCheadword[2]{rithi} (der. of \TClink[1]{rithi}) 

\TCheadword[3]{rithi} \textit{adj} dark.

\TCheadword{rithilɛhɔl} (comp. of \TClink[2]{rithi} (der. of \TClink[1]{rithi}), \TClink[1]{ahɔl}, see \TClink[1]{rithi}) 

\TCheadword{rok} \textit{n} grandchild, [ròk]/[ǹròk] grandchild/ grandchildren (B dialect). \textit{Anyaiɛ, apima iyɛ, nrokɛ, nrekiaɛ ŋa bia hundɛ.} Our people, our children, the grandchildren, the great-grandchildren that are going to come.

\TCheadword{rokos} \textit{cf}: \TClink{dembe}, \TClink{gbogbɔth}, \TClink{lembe}. \textit{n} \textbf{1)} orange. \textit{Ŋkɔm lɛnthiɛ nrokos ntiŋ ni mpakai nhiɔl!} Go pluck me two oranges and four papayas (\citealt{Pichl1967}). \textit{Ŋ kwey ndembe lo ni rokos lɛ ni ŋkɔ ma wɔk ni nsas mɔ.} Take the limes and the orange and go and squeeze them and then strain them (\citealt{Pichl1967}). \textbf{2)} lime. \textit{Rokos lɛ kɔ ŋayŋay.} The lime is sour (\citealt{Pichl1967}). 

\TCheadword{rokosthoɛ} \textit{n} [rókósthòɛ̀] tree species, tree that never grows very tall, used for snake bites (K dialect). 

\TCheadword{romba} \textit{nam} Yase drummer. The main drummer in the Yase society, usually addressed as Ba Romba (\citealt{Pichl1967}). 

\TCheadword{roŋkɛ} \textit{n} \textit{ro̹nkə} (kɔ/ma) stilts as used by the Mamakpara and others. There are also some dancers who dance on stilts (Liberian origin) (\citealt{Pichl1967}). 

\TCheadword{Roŋko} \textit{nam} Ronko, name given to a place. 

\TCheadword{roŋkoo} \textit{nam} \textit{ro̹nkoo} (kɔ/-) dance of the Taso or Kase (\citealt{Pichl1967}). 

\TCheadword{Roshia} (Eng \textit{Russia}) \textit{nam} Russia, name given to a place. 

\TCheadword{roth} \textit{cf}: \TClink{gbɔɔ}. \textit{n} [ròth] vegetable species, garden plant something like an eggplant, white in color (K dialect). 

\TCheadword{Rotifuŋk} \textit{nam} Rotifunk, name given to a place. \textit{Yami pɔ gbem wɔ pɔk Rotifuŋgɛ, lɔ pɔ vel Bompɛɛ, Nyogbako.} My mother was born in the country of Rotifunk, (which) they used to call Bumpeh, Moyogba.

\TCheadword{rɔ} \textit{n} debt.

\TCheadword{rɔk} \textit{cf}: \TClink[2]{bɛth}, \TClink[2]{kɛn}, \TClink{kɛth}, \TClink{thak}. \textit{v} [rɔ́ɔ́k] cut rice, harvest rice (K dialect). \textit{Yi gbɛki kump ha bɔnth hi ha rɔk.} We hire helpers to help us to harvest (rice) (\citealt{Pichl1967}). 

\TCsubword{rɔki} (der.) \textit{v} cut rice, harvest rice. \textit{Yi koŋ gbo rɔki si yi ɛ thɔnk pəlɛ bɛl l'ay.} After having harvested it, we put up the rice in the farmhouse (\citealt{Pichl1967}). 

\TCsubword{sirɔkɔ-hɔl} (unspec.) \textit{n} harvest time. \textit{Paŋ-saa lɛ hɔ sirɔkɔ-hɔl.} The month of September is harvest time (\citealt{Pichl1967}). 

\TCheadword{rɔki} (der. of \TClink{rɔk}, \TClink[1]{-i}, see \TClink{rɔk}) 

\TCheadword{rɔmp} \textit{cf}: \TClink[1]{nak}. \textit{n} sickness.

\TCheadword{rɔnka} \textit{n} fish or meat cooked in a leaf.

\TCheadword{rɔntma} (Themne) \textit{n} [rɔ̀nchmá] nail (B dialect). 

\TCheadword[1]{rɔŋ} \textit{cf}: \TClink[2]{tintin}. \textit{n} truth.

\TCheadword[2]{rɔŋ} \textit{n} (kɔ/ma) mountain (\citealt{Pichl1967}). \textit{Rɔng dɛ tokɛ-tokɛ.} The mountain is very high (\citealt{Pichl1967}).

\TCsubword{rɔŋjɛmdi} (comp.) \textit{n} (kɔ/ma) volcano (\citealt{Pichl1967}).

\TCsubword{rɔŋkasilan} (comp.) \textit{nam} (kɔ/ma) mountain of the Sherbro guardian spirit Kashilan, near Gbagru (\citealt{Pichl1967}). 

\TCheadword{rɔŋjɛmdi} (comp. of \TClink[2]{rɔŋ}, \TClink{jɛm}, see \TClink[2]{rɔŋ}) 

\TCheadword{rɔŋkasilan} (comp. of \TClink[2]{rɔŋ}, \TClink{Kasilan}, see \TClink[2]{rɔŋ}) 

\TCheadword{ruba} \textit{n} blessing. \textit{Kɛ mi yaŋbɛ achɔŋɔmɔ sɛkɛe ŋa yi the tha nyiyɛ mi ɛ, Abatokɛ bɛ lɔ ruba.} But myself I thank you for the questions you have asked me, may God be with you.

\TCsubword{ruban} (der.) \textit{n} blessed one. \textit{Itɔnk Bahin ruban dɛ.} Let us praise our Father, the blessed one.

\TCsubword{rubani} (der.) \textit{v} be blessed.

\TCheadword{ruban} (der. of \TClink{ruba}, \TClink[1]{-n}, see \TClink{ruba}) 

\TCheadword{rubani} (der. of \TClink{ruba}, \TClink{-ni}, see \TClink{ruba}) 

\TCheadword{rum} (Eng \textit{room}) \textit{n} room.

\TCheadword{runth} \textit{v} [rúnth] push (K dialect). \textit{Lɛ nɔsɛ ha ni gbo kɛkɛ nrunth gbo mɔ gbo runth li bul komɔɛ koŋ honi.} If the nurse does not make it fast, you just push, you just push once, and the baby is out.

\TCsubword{vunthu} (der.) \textit{cf}: \TClink{thimkɔk} (comp. of \TClink{thim}, \TClink{kɔk}). \textit{n} retreat.

\TCheadword{ruŋklani} (der. of \TClink{-ni}) 

\end{letter}
\begin{letter}{S}

\TCheadword[1]{sa} \textit{adj} red, [kìl thìsáɛ̀] red houses (K dialect). comp. \TClink{gbamsa} (see \TClink{gbam}), \TClink{kɛmsa} (see \TClink[2]{kɛm}), \TClink{puysa} (see \TClink[1]{puy}), \TClink{velsa} (see \TClink[2]{vel}) 

\TCheadword[2]{sa} \textit{n} (kɔ/ma) shrub similar to alligator pepper (Cuviera acutiflora) (\citealt{Pichl1967}). 

\TCheadword[3]{sa} (Eng \textit{saw}) \textit{n} (hɔ̃/tha) saw (\citealt{Pichl1967}).

\TCheadword[1]{saa} \textit{v} \textbf{1)} go through. \textbf{2)} escape. \textit{Təm dɛ kɔ ka chɔni Pəm Taks ɛ, pə ka di Abək agbe̹r abul-abul gbo hã ka saa.} During the time of the Hut Tax War, many Krios were killed, only a few escaped (\citealt{Pichl1967}). 

\TCheadword[2]{saa} \textit{nam} [sàà] September (K dialect). comp. \TClink{paŋsaa} (see \TClink[2]{paŋ}) 

\TCheadword{saagbi} \textit{n} (kɔ/-) grass species, plant similar to sugar cane but not as high (Palisota hirsuta) (\citealt{Pichl1967}). 

\TCheadword[1]{saaka} \textit{cf}: \TClink[2]{hu}, \TClink[2]{isɔ}. \textit{n} morning (B dialect). 

\TCsubword{nsaka-bunthul} (comp.) \textit{temp} very early.

\TCsubword{sakahɔl} (comp.), (id.) \textit{n} early morning; \textit{nsaka-hɔl} (ma) toward dawn, early morning (\citealt{Pichl1967}). comp. \TClink{lensakahɔl} (see \TClink[1]{le}) 

\TCsubword[2]{saka} (der.) \textit{disco} morning greetings, [ǹ sàkà]/ [ŋá sákà]/ [sàká sàkáò] Good morning (you, sg)/ good morning (you, pl)/ good morning (thank you)! (B dialect). \textit{Ba mi ya hun mɔ sɛɛki, mɔ ve?} Mister, I come to wish you good morning, are you well? (\citealt{Pichl1967}). der. \TClink{sakasaka} (see \TClink[1]{saaka})

\TCsubword{sakasaka} (der.), (der. \TClink[2]{saka}) \textit{disco} salutations.

\TCheadword[2]{saaka} \textit{n} [sááká] tree species, tree with very hard wood used for boards (K dialect).

\TCheadword[3]{saaka} \textit{cf}: \TClink[4]{saaka}. \textit{n} \textbf{1)} thanks. \textit{N lɔ̀llɔ́ ɲɛ̀ŋkɛ̀lɛ́ŋ? À chɔ̀ŋá Àbátùkɛ́ sàkà.} Did you sleep well? I give thanks to God. \textit{I chɔŋɔ Abatokɛ sɛkɛ yɛ ŋa hundɛ.} We thank God that you came. \textbf{2)} [sáákà] sacrifice (K dialect). \textit{Ya koŋ kwey saaka thigber ha nrɔmp lɔ kə ya sonkɔni.} I have made so many sacrifices for this sickness, but I have not gotten well (\citealt{Pichl1967}). \textit{Amaaɛ ŋa bɛmpa ŋjeeɛ ha sakaɛ ŋae thee yɛ Kaiŋ Taso mam kaathbaɛ.} The women who were preparing the food for the sacrifice heard Kain Tasso laughing loudly. \textbf{3)} charity. \textit{A kɛ lokimdɛ wɔi pɔ bi bɛ ha hu ŋ saka wɔi, Ŋgasumana ko, fakai ko.} Because he is my in-law, we even have to make his sacrifice (tithe) in Mokainsumana, in the village.

\TCsubword{sɛkɛ} (der.) \textit{n} thanks. der. \TClink{sɛkɛ-sɛkɛ} (see \TClink[3]{saaka})

\TCsubword{sɛkɛ-sɛkɛ} (der.), (der. of \TClink{sɛkɛ}) \textit{cf}: \TClink{sakao}. \textit{disco} thank you. \textit{Sɛkɛ-sɛkɛ we.} Thanks! \textit{Sɛkɛ-sɛkɛ we, Abatokɛ che mamɔ.} Thank you very much, may God be with you. \textit{So sɛkɛ-sɛkɛ we, so womdɛki ŋanɛ ŋa hunɔ ni muɛ} So thank you very much, so this greeting those that have not come yet.

\TCheadword[4]{saaka} \textit{cf}: \TClink[3]{saaka}. \textit{v} sacrifice, often involves staying up all night. \textit{A ma mɔ saka, ni nyiɛ mi yɛ driɛ mɔ thihɔlla?} I should stay awake (sacrifice) for you and then have you ask me why my eyes are red?

\TCheadword{saaki} [sààkì] \textit{n} snake species said to have two heads because of the way it moves both backward and forward, bright black shiny color, difficult to see, can be a sign or warning when it appears, lives on the ground (K dialect).

\TCheadword{Saayalɛ} \textit{nam} Saya, name given to a place. 

\TCheadword[1]{saba} \textit{n} law. \textit{Ŋɔ́hɔ́lpòkɛ̀ wɔ̀ thɛ́kɛ́sí sàbàɛ́.} The judge interprets the law. \textit{Lɔn la saba ko ki mɔilɛ, pɔ cheŋ vei hini nɔ.} That is a law for Muslims, they do not keep the corpse for a long time.

\TCheadword[2]{saba} \textit{cf}: \TClink{wini}. \textit{n} Poro dance.

\TCheadword{sabɛ-bɔs-wɛy} \textit{n} stinging leaf.

\TCheadword[1]{sabo} \textit{n} \textbf{1)} [sàbò] twins (K dialect). \textbf{2)}. (hɔ̃/tha) twin society (\citealt{Pichl1967}).

\TCsubword[2]{sabo} (id.) \textit{n} (hɔ̃/tha) disease associated with twins, disease that can only be healed by twin society. It is said that twins can make the ears of a person rot and fall off without touching the person, but otherwise they know herbs to cure this sickness. The snail is an emblem of twins (\citealt{Pichl1967}).

\TCheadword{sabu} \textit{n} [sàbù], [sàbàbù] luck (K dialect). 

\TCheadword[1]{saɛ} \textit{cf}: \TClink{chɛrchɛr}, \TClink{fiyoŋfiyoŋ}. \textit{n} [sàɛ̀] bird species, very small bird that can foretell the future with two distinctive cries: \textit{chɛrchɛr}, \textit{fiyoŋfiyoŋ} (K dialect). 

\TCheadword[2]{saɛ} \textit{n} [sàɛ̀] beard (K dialect). \textit{Sae wɔ lɛ kɔ dinthe.} His beard is white (\citealt{Pichl1967}).

\TCheadword{sagbana} \textit{n} [sàgbáná] bird species that builds nest with feathers plucked from other living birds (K dialect). 

\TCheadword{sagbe} \textit{n} [sàgbé] tree species, bark used for medicine, bitter (K dialect); \textit{sagbə} (kɔ/ma) tree species, roots have a bitter taste and are used for toothbrushes (\citealt{Pichl1967}). 



\TCheadword{saha} \textit{n} [sáŋhá] plant species, egusi, garden plant like watermelon, seeds used to sweeten soups, very thick, first parched and pounded, then soaked in water before used in cooking (K dialect); \textit{nsahã} (ma) fruit similar to watermelon and cucumber, whose dried, crushed seeds are used for cooking (\citealt{Pichl1967}). comp. \TClink{Yelsaha} (see \TClink[3]{yel}) 

\TCsubword{Saihɔl} (comp.) \textit{nam} [sàìhɔ́l] December (K dialect).

\TCsubword{saihɔl} (comp.) \textit{temp} approach of rice harvest (\citealt{Pichl1967}). 

\TCheadword{Saidu} \textit{nam} Saidu, male name given to a person. \textit{Ya lɔ Saidu Nɛtɛ.} I am Saidu Netteh.

\TCheadword{Saihɔl} (comp. of \TClink{sai}, \TClink[1]{ahɔl}, see \TClink{sai}) 

\TCheadword{saihɔl} (comp. of \TClink{sai}, \TClink[1]{ahɔl}, see \TClink{sai}) 

\TCheadword{saɛ} \textit{n} [sàɛ̀] dry season (K dialect). \textit{Sæ lɛ kɔ kath, pɔmthi ŋkəfe lɛ koŋ vila.} The dry season is hard, the leaves of the peppers have withered (\citealt{Pichl1967}). 

\TCheadword[1]{sak} \textit{cf}: \TClink[1]{lath}, \TClink{rɛthi} (der. of \TClink{rɛth}, \TClink[1]{-i}). \textit{v} \textbf{1)} spread out. \textbf{2)} stay. \textit{Lɛ hɛn gbo lom tendɛ, mbi ha sak ndɔɛ.} If you ignore the song of the bird, you will oversleep. \textbf{3)} make the bed.

\TCheadword[2]{sak} \textit{cf}: \TClink[1]{wul} (der. of \TClink[1]{wu}). \textit{n} \textbf{1)} (kɔ/-) feast or dance lasting the whole night (\citealt{Pichl1967}). \textbf{2)} wake. \textit{Haaŋ ni nante bɛ, pɔ mu tɔn tontho ki chɔl sakɛ ha hok saka wul-lɛ.} Even up to the present day, people still sing these songs the night of the wake. \textbf{3)} festival.

\TCsubword[2]{sakil} (der.) \textit{n} dancers' morning call; \textit{səkil} a call at morning by dancers, usually to ask for a present (\citealt{Pichl1967}).

\TCheadword[1]{saka} \textit{n} \textit{isaka} (hɔ̃/ma) plant species, shrub similar to \textit{ithɛkɛn} but with larger leaves (\citealt{Pichl1967}). 

\TCheadword[2]{saka} (der. of \TClink[1]{saaka}) 

\TCheadword{sakahɔl} (comp. of, id. of \TClink[1]{saaka}, \TClink[1]{ahɔl}, see \TClink[1]{saaka})

\TCheadword{sakao} \textit{cf}: \TClink{sɛkɛ-sɛkɛ} (der. of \TClink{sɛkɛ}). \textit{disco} thank you. \textit{Mɔ̀í, ì sàkáò.} Good afternoon, fine thank you. \textit{\`{m}pìkɛ́ sàkàò ŋɔ̌mpìù.} Evening or night greeting, replies.

\TCheadword{sakasaka} (der. of \TClink[2]{saka} (der. of \TClink[1]{saaka}), see \TClink[1]{saaka})

\TCheadword[1]{saki} \textit{v} cease soon (\citealt{Pichl1967}). \textit{Pɔɔ lɛ hɔ̃ sakia.} The rain will soon cease (\citealt{Pichl1967}). 

\TCheadword[2]{saki} \textit{cf}: \TClink{balmaa}. \textit{n} (kɔ/ma) two-edged knife for self defence (\citealt{Pichl1967}).

\TCheadword[3]{saki} \textit{n} cassava leaf. \textit{Sakiɛ kɔn ache bɔ yuk bikɔs kulunsɛ ŋa kɔ sɔm.} The cassava leaves are what I do not plant because the goats would eat them.

\TCheadword[4]{saki} \textit{n} earthworm.

\TCheadword[1]{sakil} \textit{v} swim. \textit{Sakil bunkluŋ dɛ atok.} (He) swam on the waves (\citealt{Pichl1967}). \textit{Yai sakil, yai hunni chɛka.} I swam to the land and I came onto it.

\TCheadword[2]{sakil} (der. of \TClink[2]{sak}, \TClink{-il}, see \TClink[2]{sak}) 

\TCheadword{sakoo} \textit{cf}: \TClink{a-a}, \TClink{bɛaan}, \TClink[2]{no}. \textit{disco} no! \textit{} No! No I don't! Not at all! (\citealt{Pichl1967}).

\TCheadword[1]{sal} \textit{cf}: \TClink{hip}. \textit{n} heaps made after brushing and burning (K dialect); heap of wood, weeds, etc. from clearing the farm, heaps are then burnt (\citealt{Pichl1967}). comp. \TClink{thɛsal} (see \TClink[1]{thɛ}) 

\TCheadword[2]{sal} \textit{n} [sàl]/[sààlɛ́] rainy season/ the rainy season (B dialect); \textit{lisal} (lɔ/-) rainy season (\citealt{Pichl1967}). \textit{Salli lɛ lɔn tipɛ.} The rainy season begins (\citealt{Pichl1967}). 

\TCheadword{Salematu} \textit{nam} Salaymatu, female name given to a person. \textit{Ya mi wɔ lɔ Salematu Bundu.} My mother is Salaymatu Bundu.

\TCheadword{saleŋka} (Port \textit{salgar} ‘salt') \textit{v} salt to preserve (\citealt{Pichl1967}). \textit{Ŋ kɔ salenka gbokbo lɔ!} Go salt this catfish! (\citealt{Pichl1967}).

\TCheadword{Salima} \textit{nam} Salima, name given to a place. \textit{Salima ko lɔ pɔ gbem mi.} I was born in Salima.

\TCheadword{Salon} \textit{nam} Sierra Leone, name given to a place. \textit{Pɔki Salon dɛ, pɔ ko ha jagbe.} In our country Sierra Leone, they have done a lot.

\TCheadword{samak}\textit{cf}: \TClink{sɔk}. \textit{n} \textbf{1)} [sàmàk] type of fowl, guinea fowl (K dialect). \textbf{2)} type of fowl, large bush fowl (\citealt{Pichl1967}). 

\TCheadword{Samba} \textit{nam} Samba, name given to a person. \textit{Yaa Bɔima Samba.} I am Boima Samba.

\TCheadword{samba} \textit{n} Bondo messenger, \textit{sambaa} (wɔ/hã) official messenger of Bondo who tells people that somebody has died and that they should come to the funeral (\citealt{Pichl1967}). 

\TCheadword{samdɛ} \textit{v} pursue evilly; \textit{səmdɛ} be after somebody for an evil purpose (\citealt{Pichl1967}). 

\TCheadword[1]{sampa} \textit{cf}: \TClink[1]{kasa}. \textit{n} basket. \textit{Yema si kump sampa chang awante Bue.} Yema knows better than her sister Bue how to finish a basket (\citealt{Pichl1967}). comp. \TClink{kothasampa} (see \TClink{kotha}) 

\TCheadword[2]{sampa} \textit{n} [sàmpà] position within Bondo Society, women's summoner, takes messages out and brings messages back, sounds alarms (K dialect). 

\TCheadword{sampamani} \textit{v} [sàmpàmàní] leave alone, i.e., not punish (K dialect). 

\TCheadword{sampi} \textit{n} (hɔ̃/tha) horn filled with medicines (\citealt{Pichl1967}). 

\TCheadword{sampul} (Eng \textit{sample}) \textit{n} sample. \textit{Yɛ mɔ theli wɔk ni nɔɛ kɔ ke sampullɛ wɔi si kɛ nɔɛ ki wɔ tintin, n thambas ɛ.} When you say something, let the person see the sample, then the person knows that this person is straightforward.

\TCheadword{Samuɛl} \textit{nam} Samuel, male name given to a person.

\TCheadword[1]{san} [sàn] \textit{cf}: \TClink{sinthil}. \textit{n} ant species, black driver ant that resembles \textit{sinthil} but is black and more numerous (K dialect). 

\TCheadword[2]{san} \textit{n} otter. \textit{San dɛ koŋ lo nthɪn.} The otter has delivered the judgment (proverb). Once upon a time the cat came to the otter to complain that people falsely accused her of stealing fish from the platform where they were put out to dry. The otter asked the cat where she lived and when she answered that it was near the platform and no one else was allowed to go there, the otter found the cat guilty of theft. Therefore, the meaning of the proverb is about, “Qui s'excuse s'accuse” or “This is a final decision.” (\citealt{Pichl1967}). 

\TCheadword[3]{san} \textit{cf}: \TClink{sɔthɔ}. \textit{v} \textbf{1)} get. \textit{Nɔthiɛ nthɛkɛsiɛ wɔ ni san la ntenɛ.} Human beings clarify in order to understand things. \textbf{2)} achieve. \textit{Nɔ shini che ko labi yendɛ yɛ mɔ la ŋa ncheyi ni nshila thiyen, ni la saŋ mɔ ntenɛ.} One does not know the future that is why when doing something you should ask so you can know it and understand it better.

\TCheadword{Sana} \textit{nam} Sana, female name given to a person.

\TCheadword{sana} \textit{adj} [sànà] new (K dialect). \textit{Kisik lɛ ha hini lɛ ha pɛ bɔni nɛn sana lɛ.} At the end, they decided they would meet again in the new year (\citealt{Pichl1967}). comp. \TClink{paaŋsana} (see \TClink[2]{paŋ}) 

\TCheadword{Sanda} \textit{nam} Sanda, town located in Timdale Chiefdom. \textit{Bentu, wɔn wɔ Nsanda ko.} Bentu, she is in Sanda (Timdale Chiefdom).

\TCheadword{Sanduku} \textit{nam} Sanduku, name given to a person. \textit{Sanduku koŋ trai inallɛ ki.} Sanduku has tried in this place.

\TCheadword{saŋkath} \textit{v} [sáŋkáth] rinse (K dialect). \textit{Ŋ kɔ sankath boy lɔ, hɔ chen charaŋ.} Go rinse the plate there, it is not clean (\citealt{Pichl1967}). 

\TCheadword{Saŋkɔ} \textit{nam} Sanka, name given by Toma Society. 

\TCheadword[1]{santh} \textit{cf}: \TClink[2]{tata}. \textit{n} shrimp. \textit{Santh bom-bom dɛ kɔ mən njɛthil l'ay kə santh ta-ta lɛ kɔn dinthɛni kɔ hɛlɛɛ ko.} The big shrimp are found in freshwater but the small and white shrimp are to be found in the sea (\citealt{Pichl1967}). 

\TCheadword[2]{santh} \textit{n} \textbf{1)} older one. \textit{Ha asanth kɛ a gbe yaŋ ya veleŋ thimɛkin ni.} The older ones are numerous but I am after the last ones. \textit{Nɔsanth wɔ ki, m ma wɔ lepi.} He is an elder, don't disgrace him (\citealt{Pichl1967}). \textit{Apuma lɛ ha cho', santh lɛ tunt thɔm wɔ lɛ yenwɛy, ha kɔ ha koosi.} The children are fighting, the older one has badly twisted his companion, go and separate them (\citealt{Pichl1967}). \textbf{2)} adult; grownup. \textit{Ya ka ni hani santhɛ...} When I had grown up... \textit{Kɛ yɛ laiyoɛ tamɔ ta kani nɔ santh limani.} But as it is, a young boy does not give adults respect. comp. \TClink{nɔsanth} (see \TClink{nɔ}) 

\TCsubword{santhsanth} (der.) \textit{n} \textbf{1)} elder. \textit{A a che yaŋ ya nɔsein dɛ ko yami, asanth-santhɛ ŋalɔ.} No, I am not the first one of my mother's, the elder ones are there. \textbf{2)} grownup. \textit{Aa, ŋa gbem, ŋa bi apuma santh-santh.} Yes, they have children, they have grown children.

\TCheadword{santhil} \textit{n} sword grass; \textit{santhil} (kɔ/ma) kind of grass with sharp cutting leaves (\citealt{Pichl1967})

\TCsubword{santhilpokan} (comp.) \textit{n} extra sharp sword grass; \textit{santhil-pokan} sharper type of sword grass (\citealt{Pichl1967}). 

\TCheadword{santhilpokan} (comp. of \TClink{santhil}, \TClink{pokan} (unspec. of \TClink[5]{po}), see \TClink{santhil}) 

\TCheadword{santhoŋ} \textit{n} bush species, used like Maggi for flavor (K dialect). \textit{Sànthóŋ kɔ́ tèŋ.} \textit{Santhoŋ} bush is sour.

\TCheadword{santhsanth} (der. of \TClink[2]{santh}) 

\TCheadword{santhuŋ} \textit{n} Jamaican sorrel; \textit{santhuŋ} (kɔ/-) herb species, Jamaican sorrel or sour-sour (\citealt{Pichl1967}). 

\TCheadword{saŋ} \textit{cf}: \TClink[3]{sei}. \textit{v} \textbf{1)} [sáŋ] sow, broadcast seeds, e.g., rice (K dialect). \textit{Hin lɛ pɛ sallɛ mɔi gbo asaŋ keŋkendɛ a yuk gbamdɛ.} For us, when rainy season comes, I plant krain-krain, (and) I plant potato leaves. \textit{Pɔi hun saŋ pɛlɛ.} Then they come and scatter (sow) the rice. \textbf{2)} scatter. \textit{Pɔ koŋ gbo pɔ chi fatalaisaɛ pɔi saŋ.} When they have finished, they will bring the fertilizer and scatter it. \textbf{3)} pour. comp. \TClink{saŋpɛlɛ} (see \TClink{pɛlɛ}) 

\TCheadword[1]{saŋgba} \textit{n} drum type about two feet high, one skin, beaten with the hands (\citealt{Pichl1967}). 

\TCheadword[2]{saŋgba} \textit{cf}: \TClink{baŋkgbɔl} (comp. of \TClink[2]{baŋk}). \textit{n} string.

\TCheadword{saŋk} \textit{n} [sàŋk] ginger, grown in gardens or around house, used for medicine (K dialect).

\TCsubword{saŋkntonton} (comp.) \textit{n} alligator pepper; \textit{nsanknto̹nto̹n} (ma) alligator pepper (\citealt{Pichl1967}). 

\TCheadword{saŋka} \textit{cf}: \TClink{gbaŋgbaŋsasa} (comp. of \TClink{gbaŋgbaŋ}). \textit{n} bird species; sanka (wɔ/hã, N) Senegal kingfisher (\citealt{Pichl1967}). comp. \TClink{baŋsakɔ} (see \TClink[3]{baŋ}) 

\TCheadword{saŋkntonton} (comp. of \TClink{saŋk}, \TClink{tonton} (der. of \TClink[1]{ton}), see \TClink{saŋk}) 

\TCheadword{saŋpɛlɛ} (comp. of \TClink{saŋ}, \TClink{pɛlɛ}, see \TClink{pɛlɛ}) 

\TCheadword{saŋthoŋ} [sànthóŋ] \textit{n} leafy vegetable, used for sauce, grown in gardens (K dialect).

\TCheadword{sap} \textit{cf}: \TClink{thontha}. \textit{v} catch something thrown (\citealt{Pichl1967}). 

\TCheadword{sapo} \textit{n} sponge; \textit{sapo̹} (kɔ/-) ordinary sponge (\citealt{Pichl1967}). 

\TCheadword{sas} \textit{cf}: \TClink{vɛkɛth} (der. of \TClink[1]{wɔk}), \TClink[1]{wɔk}. \textit{v} \textbf{1)} strain. \textit{Ŋ kwei ndembe lo ni rokos lɛ ni ŋkɔ ma wɔk ni nsas mɔ.} Take the limes and the orange and go and squeeze them and then strain them (\citealt{Pichl1967}). \textbf{2)} squeeze. \textit{Ŋ kwey ndembe lo ni ŋ kɔ ma wɔk ni nsas ma.} Take these limes and go squeeze them (\citealt{Pichl1967}). 

\TCheadword{sasi} \textit{adj} [sàsí] unappealing, something you do not want to touch, e.g., a dirty cloth (K dialect). 

\TCheadword{sathaŋ} [sàthàŋ] \textit{n} centipede species, brown in color, very poisonous, can move quickly in either direction, some people say [yathaŋ] (K dialect). 

\TCheadword{Sathia} \textit{nam} Sathia, female name given to a person. \textit{Sathia chanth lɛ koŋ bɔy mɔ lɛ, mma wɔ pɛ kuli.} Sathia's child has suckled enough, don‘t give it more to drink.

\TCheadword{Satia} \textit{nam} Satia, female name given to a person.

\TCheadword{Satide} \textit{nam} Saturday.

\TCheadword{satok} \textit{prep} on account of; for. \textit{Wɔn ŋkɔŋ ma wɔ ɛ hã satok yaŋ.} He (gave) his blood on account of me (\citealt{Pichl1967}). \textit{Nloli mi hã satok ilɛl mɔ ɛ.} Save me for thy name's sake (\citealt{Pichl1967}). 

\TCheadword{Satɔde} \textit{nam} Saturday.

\TCheadword{say} \textit{n} \textbf{1)} offensive thing. \textit{Nɔmɔk lɛ kɔ hok wɔn minɛ lɛ kɔ isay.} The mucus that comes from his nose is offensive (\citealt{Pichl1967}). \textbf{2)} \textit{isay} (hɔ/-) filth, dirt (\citealt{Pichl1967}). \textit{Bondo ka lɔ thuŋ puth, isay igbe̹r lɔ ka.} It stinks very much at the wharf; there is a lot of filth there (\citealt{Pichl1967}). 

\TCheadword{sayom} \textit{cf}: \TClink{gboso}, \TClink{hakla}, \TClink{tokoth}. \textit{n} animal trap; \textit{sayo̹m} (kɔ/ma) bush trap (\citealt{Pichl1967}). 

\TCheadword{Sayprɔs} \textit{nam} Cyprus, name given to a place. \textit{Simi-njɛm bo̹m hɔ̃ kong duk Sayprɔs Agriika lɛ thiye̹ng aña Thɔɔki lɛ.} A big misunderstanding has been created (befallen) in Cyprus between the Greeks and the Turks (\citealt{Pichl1967}). 

\TCheadword{Se} \textit{nam} Sei, language (dialect) of the southern and eastern part of Bonthe Island, including Bonthe. 

\TCsubword{Sechiɛ} (comp.) \textit{nam} Sitia; \textit{Sechiɛ} (hɔ̃/-) name given to a place located along the shore of Se (\citealt{Pichl1967}). 

\TCheadword[1]{se} \textit{n} \textit{nse} (ma) pus, gleet, \textit{lwɛ nse} suppurate (\citealt{Pichl1967}). \textit{Chanth lɛ bi puŋ, hɔ koŋ lwɛ nse.} The child has a boil, it suppurates (\citealt{Pichl1967}). \textit{Mma vəkɛth su-m dɛ, kɔ hinth ni lwɛ nse, mma ki-m nɛki!} Don't squeeze my finger, it will swell and suppurate, don't hurt me! (\citealt{Pichl1967}). 

\TCheadword[2]{se} \textit{v} say.

\TCheadword{Sebe} \textit{nam} Sabay, name given to a person. \textit{Wɔlɔ Piɛ Sebe.} He is Pieh Sabay.

\TCheadword{Sechiɛ} (comp. of \TClink{Se}, \TClink{chiɛ}, see \TClink{Se}) 

\TCheadword[1]{sei} \textit{n} \textbf{1)} witness. \textit{Nsey lɛ ha koŋ sey mbolom dɛ.} The witnesses have given evidence in the case (\citealt{Pichl1967}). \textbf{2)} evidence. \textit{Sese kɔ wɔŋ lisey mbolom dɛ ay yeŋ thi Bia ni Koŋ.} Sese went to give evidence in the case between Bia and Kong (\citealt{Pichl1967}). 

\TCheadword[2]{sei} \textit{v} testify; witness. \textit{Nsey lɛ ha koŋ sey mbolom dɛ.} The witnesses have given evidence in the case (\citealt{Pichl1967}). 

\TCheadword{sein} \textit{v} [séín] cleanse, purify, wash (K dialect). 

\TCheadword{Seiŋyɛ} \textit{nam} Seinyeh (Friday). \textit{Huɛ Seiŋyɛ ŋɔ pɔ vel lɛ Flaideɛ Mpothoaiɛ, nduɛ waŋnimɛŋraɛ.} On Seinyeh, which they call Friday in English, the eighteenth.

\TCheadword{sek} \textit{n} \textbf{1)} \textit{se̹k} (?/ma) piece, slice (\citealt{Sumner1921}); \textit{sek} a broken piece (\citealt{Sumner1921}). \textit{Ŋ ka mi sɘk brɛdi!} Give me a slice of bread! (\citealt{Pichl1967}). \textbf{2)} \textit{isək} (hɔ̃/-) broken grains of rice (\citealt{Pichl1967}).

\TCsubword{sekitini} (der.) [sèkítìnì] \textit{cf}: \TClink{kɛnthi}, \TClink[1]{pɛl}. \textit{v} shatter (more sophisticated speakers use \textit{sekitini} ‘shatter' instead of \textit{kɛnth} or \textit{pɛl}) (K dialect). \textit{Hɔ̀ sèkítìnì.} It shattered. comp. \TClink{bɛlsekiɛni} (see \TClink[2]{bɛl}) 

\TCheadword{sekitini} (der. of \TClink{sek}, \TClink{-ni}, \TClink[1]{-i}, see \TClink{sek})

\TCheadword{seko} \textit{cf}: \TClink[1]{huk}. \textit{n} fishhook. \textit{Ikoi bithi thiseko ki, thanɛ thakoŋ pɛli vɛ.} We take the bottle of hooks, those broken ones.

\TCheadword{seminji} \textit{n} [sémínjí] salve, sweet smelling, used as a body salve, brought by Nigerians for sale (K dialect). 

\TCheadword{sen} \textit{cf}: \TClink{fɔst}, \TClink[1]{nse}. \textit{adj} first. \textit{Hun sendɛ ŋɔ hundɛ, hun 1978.} The first time he came was in 1978. \textit{Mɔmɔ nɔ sendɛ ko bamɔ?} Are you your father's first child? \textit{Aa, ya nɔ sendɛ ko ba mi.} Yes, I am my father's first one.

\TCheadword{sent} (Eng \textit{saint}) \textit{nam} saint. \textit{Baybul lɛ hɔ lɛ Sent Pɔl ka che-lɛ ni ke ka thihɔl yɛ pə ka vɛy Sent Stivɛn.} The Bible says that St. Paul was present and saw with his eyes when they stoned St. Stephen.

\TCheadword{senthetha} \textit{n} [sénthéthà] duck species, water ducks, smaller than \textit{pupun}, move in flocks of as many as a hundred (K dialect). 

\TCheadword{seŋgbɛŋ} (comp. of \TClink{sɛŋ}, \TClink[1]{gbɛlaŋ}, see \TClink[1]{gbɛlaŋ}) 

\TCheadword{seŋka} \textit{v} draw in. \textit{seŋka} draw in tightly at the waist, esp. women for beauty's sake when dressing.

\TCheadword{Sese} \textit{nam} [sésé] Sese, male name given by Poro Society (K dialect). \textit{Sese theyɛn-nɛki, thɔ lɛ kəth wɔ yenwɛy.} Sese hurt himself, the adze badly cut him (\citealt{Pichl1967}).

\TCheadword{sese} \textit{n} (wɔ/hã, N) fish species, rainbow fish (Upeneus prayensis, Xyrichthys novacula) (\citealt{Pichl1967}).

\TCheadword{seth} \textit{n} caterpillar; seth (wɔ/hã, i) caterpillar and all kinds of worms similar to it (\citealt{Pichl1967}). 

\TCheadword{Sethana} \textit{nam} Satan; \textit{sethana} (wɔ/-) devil, satan (\citealt{Pichl1967}). 

\TCheadword{Setie} \textit{nam} Sittia, name given to chiefdom located on Sherbro Island. \textit{…Tetima ko, so dat ka ko lɔ, le niɛ lɔ koni leɛ Shechiɛ.} …to Tetima, so that the remaining section is Sittia (Chiefdom). 

\TCheadword{-sɛ} \textit{NCM} noun class marker (si). \textit{Ko lɔ anyaɛ dikleni bai koɛ, anyin ŋa lɔ ŋan thiyeŋ ŋa thee ŋhɔk ma ŋvisɛ ni veesɛ.} Where the people gathered in the bari, there are people among them who hear what the animals and the birds speak. \textit{Kaiŋ Taso wɔ thee ŋhɔk ma ŋvissɛ, veesɛ, ni ŋkɔlɔŋsɛ.} Kain Tasso understands the words of every animal, bird, and insect. \textit{Yà kɔ̀ bón véésɛ̀.} I go harvest oysters. \textit{Huksi atïŋ hã che kïl lɛ kunɛ.} There are two bush spiders in the house (\citealt{Pichl1967}).

\TCheadword{Sɛbura} \textit{nam} Sebura. Title of the paramount chief of Sherbro. (The word “Sherbro” comes from “Sebura,” presumably an abbreviation for “the people or subjects of S” (\citealt{Pichl1967}).

\TCheadword{sɛɛ} \textit{n} spoon. \textit{Yɛ mɔ koŋ thɔk boithɛ gbi ni sɛiyɛ,mɔi bɛ tebullɛ atok.} After washing the dishes with the spoon, then you put it on the table.

\TCsubword{sɛɛbom} (comp.) \textit{cf}: \TClink{yɛɛk}. \textit{n} [sɛ̀ɛ̀bòm] big wooden spoon (K dialect). 

\TCsubword{sɛɛthɔk} (comp.) \textit{n} (kɔ/ma) wooden spoon (\citealt{Pichl1967}). 

\TCsubword{sɛɛwai} (comp.) \textit{n} (kɔ/ma) iron or metal spoon of any sort (\citealt{Pichl1967}).

\TCheadword{sɛɛbom} (comp. of \TClink{sɛɛ}, \TClink{bom}, see \TClink{sɛɛ}) 

\TCheadword{Sɛi} \textit{nam} [sɛ́í] Sei, male name given by Poro Society (K dialect). 

\TCheadword{sɛin} \textit{cf}: \TClink{kosi}, \TClink[4]{po}, \TClink{saŋ}. \textit{v} \textbf{1)} [sɛ̀ìn] broadcast, scatter (K dialect). \textit{Tɔŋ chiɛ pəlɛ lɛ sampa l'ay, kə koŋ kɔ sẽy kïl lɛ ko.} Tong brought the rice in the basket, but he has scattered it in the house (\citealt{Pichl1967}). \textbf{2)} separate. \textit{Kɛ ŋani po wɛ ŋa bi mu nwɔ ton-ton, kɛ ŋa sɛiɛ ni mu o, ŋalɔ mu.} Though she and her husband had a small quarrel, they have not separated, they are still there.

\TCsubword{sɛini} (der.) \textit{v} \textbf{1)} be dispersed. \textit{Boon dɛ kɔ che parɛ Furabee Kɔlɛj kɔ koŋ sẽyni.} The meeting which was recently at Fourah Bay College has dispersed (\citealt{Pichl1967}). \textbf{2)} be scattered. \textit{Yema kɔ gboth awante l'ay chena lɛ lɛliɛ yen koŋ wusi gboth l'ay lɔn gbi nyək lɛ ma gbo seyɛni hinth l'atok.} Yema went into her sister's box to find that the box had been ransacked and all the things were scattered about on the bed (\citealt{Pichl1967}). \textbf{3)} be separated. \textit{Ŋan lamɔ ŋako sɛini?} You and your wife are separated? der. \TClink{sɛinsɛinia} (see \TClink{sɛin})

\TCsubword{sɛinsɛinia} (der.), (der. of \TClink{sɛini}) \textit{v} scatter. \textit{Anyalɛ ŋae gbaki ŋa hɔɛ, “Awa la likɛlɛŋ; hi sɛiŋsɛiŋnia.”} The others answered and said, “Okay, it is good; let us scatter.”

\TCheadword{sɛini} (der. of \TClink{sɛin}, \TClink{-ni}, see \TClink{sɛin})

\TCheadword{sɛinsɛinia} (der. of \TClink{sɛini} (der. of \TClink{sɛin}, \TClink{-ni}), see \TClink{sɛin}) 

\TCheadword[1]{sɛk} \textit{n} mullet species central to Mani culture; fish species, long, bony fish, [sɛ̀k]/[ǹsɛ̀kɛ́] fish/fish (pl) (B dialect); (wɔ/hã, N) mullet (Mugil, Liza spp.) (\citealt{Pichl1967}). comp. \TClink{pɛlnsɛk} (see \TClink[2]{pɛl}) 

\TCsubword{sɛkbom} (comp.) \textit{n} (wɔ/hã, N) big mullet (Liza, Mugil spp.) (\citealt{Pichl1967}). 

\TCsubword{sɛkbɔ} (comp.) \textit{n} (wɔ/hã, N) bigger kind of mullet, jumper (\citealt{Pichl1967}). 

\TCheadword[2]{sɛk} \textit{cf}: \TClink[2]{wai}. \textit{adj} dry. \textit{Thɔk lɛ kɔ sɛk.} The tree is dry.

\TCsubword{sɛkɛli} (der.) \textit{v} dry. \textit{pɔ koŋ kɔ gbo bɛ bɛkthai, pɔ ye ma gbo jo, pɔ kɔ sɛkɛli.} After putting it in bags, if they (want to) eat it, they first dry it (in the sun). \textit{Palli kɔni lɔ che sɛkɛli pɛlɛ.} A setting sun does not dry rice (proverb). \textit{Kani yom ŋa yin, chaŋ yenchɛkoki ŋa sɛkɛliɛ.} She never allow things about us, it was only this fish that (she) dries.

\TCsubword[1]{sɛkil} (der.) \textit{v} be dry. \textit{Lɛ yɔktha sɛkilɛ gbo yenkəlɛŋ yi lo he̹r charaŋ.} When the farm with felled trees is quite dry, we burn it clean (\citealt{Pichl1967}). \textit{Ŋ thɔ́ŋklɔ̀ mí yènchɛ́k àsəkəl.} Keep the dried (smoked) fish for me. der. \TClink[2]{sɛkil} (see \TClink[2]{sɛk})

\TCsubword[2]{sɛkil} (der.), (der. of \TClink[1]{sɛkil}) \textit{adj} dry.

\TCheadword{sɛkbom} (comp. of \TClink[1]{sɛk}, \TClink{bom}, see \TClink[1]{sɛk}) 

\TCheadword{sɛkbɔ} (comp. of \TClink[1]{sɛk}) 

\TCheadword{sɛkɛ} (der. of \TClink[3]{saaka})

\TCheadword{sɛkɛ-sɛkɛ} (der. of \TClink{sɛkɛ} (der. of \TClink[3]{saaka}), see \TClink[3]{saaka}) 

\TCheadword{sɛkɛli} (der. of \TClink[2]{sɛk}) 

\TCheadword[1]{sɛkil} (der. of \TClink[2]{sɛk}, \TClink{-il}, see \TClink[2]{sɛk}) 

\TCheadword[2]{sɛkil} (der. of \TClink[1]{sɛkil} (der. of \TClink[2]{sɛk}, \TClink{-il}), see \TClink[2]{sɛk}) 


\TCheadword{sɛkɔn} (Eng \textit{second}) \textit{cf}: \TClink[2]{tin} (der. of \TClink[1]{tin}). \textit{adj} second. \textit{Ya nɔ sɛkɔndɛ, nɔ mɛkɛ tiŋ?} I am the second, the second person?

\TCheadword{sɛkshɔn} (Eng \textit{section}) \textit{n} section, district. \textit{Pɔ gbem mi Nkainsumana ko, Mɔya Sɛkshɔn.} I was born in Mokainsumana, Moya Section.

\TCheadword{sɛl} \textit{n} woodchips; \textit{isɛl} (hɔ/-) chips of wood (\citealt{Pichl1967}). 

\TCheadword[1]{sɛli} (Arabic {\textarab{صلى} } \textit{salaa} ‘pray') \textit{cf}: \TClink{tɔŋk} (der. of \TClink[2]{tɔn}, \TClink{-k}). \textit{v} pray. \textit{A the lɛ amɔya lɛ hã sɛli gbeŋ, vɛ la yɛ? }I hear that the Muslims are praying tomorrow, isn't it so? (\citealt{Pichl1967}). \textit{Oo, i mbo sɛli we ŋa alema iyɛ.} Oh, we are praying for our disciples.

\TCsubword[2]{sɛli} (der.) (Arabic {\textarab{صلى} }\textit{salaa} ‘pray') \textit{n} prayer; \textit{sɛli} (kɔ/ma) prayer (\citealt{Pichl1967}). 

\TCheadword[1]{sɛm} [sɛm] \textit{v} \textbf{1)} stand. \textit{Ndɛm ya sɛmɛ kîl lɛ ahɔl!} Look at me standing at the door! \textit{I kɔ sɛm pethɛ atok.} We go and stand on the stones. \textit{Ache lɔŋ kɔ gbi, ya lɔ kɔɛ a ke nɔɛ yɛ sɛmɛ kilɛ koɛ.} I will not go there at all, when I see the person standing in the room. \textbf{2)} rise. \textit{Lɔ Jizɔs sɛmɛ ŋa loli aŋa wɔ.} Lord Jesus rises to save his people. \textbf{3)} be situated. \textbf{4)} stay. \textit{Wɔm dɛ ŋɔ bi ha sɛm.} The canoe would stay (in one place).

\TCsubword{sɛmith} (comp.) \textit{n} \textbf{1)} standing. \textbf{2)} stature. \textbf{3)} position. \textit{Apa, mɔm yɛlɔ sɛmith mɔɛ ko pɔkoa?} Father, what is your position in this region? \textbf{4)} role. \textbf{5)} status. \textit{Ama ko pɔk o, yɛlɔ sɛmith ŋaa?} What is the status of women in this country?

\TCsubword{sɛmka} (comp.) \textit{v} stand. \textit{Labondɛ ŋɔ kɔ lɔ, ŋɔ kɔ sɛmka ko.} If it (a boat) goes there, it will stand (moor?) there. 

\TCsubword{sɛmɛkni} (der.) \textit{cf}: \TClink{kɔnaibol} (id. of, comp. of \TClink[2]{kɔ}, \TClink[1]{nai}, \TClink[1]{bol}), \TClink{thil}. \textit{v} urinate (polite) (lit. stand alone) (K dialect). \textit{A kɔ sɛmɛkəni.} I have to urinate.

\TCsubword{sɛmi} (der.) \textit{v} \textbf{1)} erect. \textit{Thitənkə tha yi sɛmi ichɛk ai.} The scaffolds which we erect on a farm. \textbf{2)} stand. \textit{Pɔ bɛ wɔ ŋgbekteɛ ni pɔ sɛmi wɔ bai ko anyaɛ gbi chee lɔ pɔ bi ha thoŋka wɔ.} They put him in handcuffs and brought him to the bari in front of all the people where they will judge him. \textbf{3)} set. \textit{Yɛ wɔ koŋ thɔkɛ pagbondɛ chiɛ nyekma lan ni sɛmiyɛ ma kilɛ ko.} When she has washed (the corpse), if (she) brought those things and set them inside the house.

\TCsubword{sɛmil} (der.) \textit{v} stand near, persist in, stand by (\citealt{Pichl1967}). \textit{Sɛmil mi.} He stood by me (\citealt{Pichl1967}).

\TCheadword[2]{sɛm} \textit{n} [sɛ̀m] tree species, with white rubber-like sap that hurts if it gets in the eyes, can blind someone (K dialect). 

\TCsubword{sɛmplɛn} (comp.) \textit{n} [sɛ̀mplɛ̀n] tree species, stripped bark used for weaving mats, has a nice scent, branches used to keep away snakes, also used for hoe and axe handles (K dialect). 

\TCheadword[3]{sɛm} \textit{cf}: \TClink{wothkun} (comp. of \TClink[1]{woth}, \TClink{kun}). \textit{v} be in the first months of pregnancy (\citealt{Pichl1967}).

\TCheadword{sɛmɛkni} (der. of \TClink[1]{sɛm}, \TClink{-k}, \TClink{-ni}, see \TClink[1]{sɛm}) 

\TCheadword{sɛmi} (der. of \TClink[1]{sɛm}, \TClink[1]{-i}, see \TClink[1]{sɛm}) 

\TCheadword{sɛmil} (der. of \TClink[1]{sɛm}, \TClink{-il}, see \TClink[1]{sɛm}) 

\TCheadword{sɛmith} (comp. of \TClink[1]{sɛm})

\TCheadword{sɛmka} (comp. of \TClink[1]{sɛm})

\TCheadword{sɛmplɛn} (comp. of \TClink[2]{sɛm}) 

\TCheadword{sɛmplɛŋ} \textit{cf}: \TClink{chencha}. \textit{temp} yesterday (K dialect).

\TCheadword{sɛnow} \textit{v} welcome someone on arrival after a long journey by shaking hands (\citealt{Pichl1967}).

\TCheadword{sɛnthɛŋ} \textit{n} \textbf{1)} (hɔ̃, i) fingernail (\citealt{Pichl1967}). \textbf{2)} (hɔ̃, i) toenail (\citealt{Pichl1967}).

\TCheadword{sɛŋ} \textit{v} \textbf{1)} leave. \textit{Pɔ wɔ bo kɔ kɔŋ wai, pɔ sɛŋyɛ lɔni.} They would just bury him quietly, then everybody would go away. \textbf{2)} go away. comp. \TClink{seŋgbɛŋ} (see \TClink[1]{gbɛlaŋ}) 

\TCheadword{sɛthɔk} (comp. of \TClink{sɛɛ}, \TClink[2]{thɔk}, see \TClink{sɛɛ}) 

\TCheadword{sɛvintin} \textit{Numb} seventeen.

\TCheadword{sɛwai} (comp. of \TClink{sɛɛ}, \TClink[1]{wai}, see \TClink{sɛɛ}) 

\TCheadword[1]{si} \textit{cf}: \TClink{lɔŋnui} (unspec. of \TClink{nui}), \TClink{the}. \textit{v} \textbf{1)} know. \textit{Mɔm mbi ja gbe ŋa ŋanɛ ŋa hunɔni muɛ ŋa ŋan si.} You have many things for those that have not come yet to know. \textit{Kɛ mɔm nshini ŋɔthi?} But you do not know how to fish? \textbf{2)} understand. \textit{si} understand. \textit{Labila awɔ ŋa bia kɔlɔ gbɛ, mɔi ke, bikɔs nɔ mɔ gbo lemɛ Mbolomdai, ŋa ni shila.} That is why I said you need to go and take a walk there, and you see, because someone explains to you in Bolom, you just understand it. \textbf{3)} realize. \textit{Oo aŋa mi isi yɛ lɛ kɛ Kraist ka wu ŋa hin.} Oh, my people, let us realize that Christ died for us.

\TCsubword[3]{si} (der.) \textit{n} knowledge. \textit{M bi shi lan?} Are you aware of it? (Do you have that knowledge?) \textit{I koni sɔtha shiɛ lɛ Mbolomdɛ ma yema tuk ayenal gbe ko lɔ pɔ kache theli Mbolomdɛ.} We know that Bolom is disappearing in many places where they used to speak Bolom.

\TCheadword[2]{si} [si] \textit{cf}: \TClink{lagbo} (comp. of \TClink[2]{la}, \TClink[1]{gbo}), \TClink[2]{lɛ}, \TClink[4]{ni}, \TClink{pabondɛ}, \TClink[1]{yɛ}. \textit{subordconn} \textbf{1)} if. \textit{Ŋɔi ni ŋa fili si i mɔla chaŋ gbo ka Jizɔs sɛ?} How are we to go there, only if we pass through Jesus? \textit{Sila vɛ o sila chen vɛ o, a sini.} Whether it is so or not, I do not know. \textit{Laa mi, si ŋcha thol hiŋk ka ni ŋkɔ chii yeke hiŋk ŋken dɛ ma luɛ vɛ…} My wife, if you descend from here and bring back cassava from those sharp knives… \textbf{2)} before. \textit{Pɔ yuk mansaŋhaɛ nseen si pɔ wɔm bɛ kutha pɛlɛɛ ni nyiki ntilaŋ.} They plant this egusi together with it first, before they plant rice or any other seeds. \textit{Tipik lɛ ye ha bɔnthɛ, ha ka silan lɛ ha	bi ha kantha kil lɛ si mənk lɛ koŋhoni.} At the beginning when they met up, they did not know that they had to close up the house before the time ran out (\citealt{Pichl1967}). \textbf{3)} whether. \textit{Sila vɛ-o sila chen vɛ-o ya sini.} Whether it is or not, I do not know. \textbf{4)} when. \textit{Təm ra lɛ moɛ gbo si ɛ yi yɔk ŋgbatho ma hĩ ɛ.} When the time for clearing the bush arrives, we grab our cutlasses (\citealt{Pichl1967}). \textbf{5)} after. \textit{Yi koŋ gbo rɔki si ɛ yi thɔnk pəlɛ be̹l l'ay.} After having harvested it, we put up the rice in the farmhouse (\citealt{Pichl1967}). 

\TCsubword{sila} (comp.) \textit{subordconn} whether… or. \textit{Sila vɛ-o sila chen vɛ-o ya sini.} Whether it is or not, I do not know.

\TCheadword[3]{si} (der. of \TClink[1]{si}) 

\TCheadword[4]{si} \textit{cf}: \TClink[1]{ni}. \textit{temp} then. \textit{Than tha yi hɛ̃y ay si yi yatha si yi kɔ trï lɛ.} In these (canoes) we embark, then we pull the oars and then we go to town (\citealt{Pichl1967}). \textit{Ŋ kɔ thɛki iwɔm dɛ si ŋ kɔ yeki thɔk lɛ.} Go split the wood and then widen the split in the wood (\citealt{Pichl1967}). (\textit{Bɛlsa ŋɔɛ handɔ ŋa hɔ si ŋa thee la?} What rat will speak and (then) you understand it?

\TCheadword{sibɔla} (Port \textit{cebola} ‘onion') \textit{cf}: \TClink{yabas}. \textit{n} (kɔ/-) onion (\citealt{Pichl1967}).

\TCheadword{sigaret} (Eng \textit{cigarette}) \textit{n} cigarette. \textit{Aa, wɔ ŋa yen ton-ton, wɔ wɔŋgul sigrɛt.} Yes, she does a few things, she sells cigarettes, cut tobacco (for pipes). \textit{A chen bɔ pin sigarɛt lɛ, ya biɛn gbo fe̹.} I am not able to buy cigarettes if I have no money (\citealt{Pichl1967}). 


\TCheadword{sii} \textit{cf}: \TClink[1]{bip}. \textit{v} fart. \textit{Yèmà kóŋ shìì yèŋɔ̀ɔ̀ì.} Yema has farted bad, stinky ones.

\TCheadword{siibii} (unspec. of \TClink{siil}) 

\TCheadword{siil} \textit{v} sting. \textit{Isilɔ hã silɛ mi.} The bees sting me (\citealt{Pichl1967}).

\TCsubword{silini} (der.) \textit{v} \textbf{1)} [sílíní] be angry, vexed (K dialect). \textbf{2)} be annoyed. \textit{A che gbo pɔng silal yɛ ya fɔs mɔ thipɛpɛ lɛ, mma silini.} I was only joking when I tapped your shoulders, don't be annoyed (\citealt{Pichl1967}). 

\TCsubword{silɔ} (der.) \textit{n} \textbf{1)} bee; \textit{siilɔ} (wɔ/hã, i) bee (\citealt{Pichl1967}); \textit{shilɔ} (pl. i) bee (\citealt{Sumner1921}). \textit{Isilɔ hã silɛ mi.} The bees sting me (\citealt{Pichl1967}). \textbf{2)} honey; \textit{siilɔ} honey (\citealt{Pichl1967}). comp. \TClink{chɛnthsilɔ} (see \TClink{chɛnth}), \TClink{silɔpɔŋkthɔ} (see \TClink{siil})

\TCsubword{silɔpɔŋkthɔ} (der.), (comp. of \TClink{silɔ}) \textit{n} giant bee species (K dialect). 

\TCsubword{siibii} (unspec.) \textit{cf}: \TClink{gbathil} (unspec. of \TClink[2]{gbath}). \textit{n} [sííbíí] punishment (K dialect). 

\TCheadword{sila} (comp. of \TClink[2]{si}, \TClink[2]{la}, see \TClink[2]{si}) 

\TCheadword{Sijismɔn} \textit{nam} Sigismund, male name given to a person. \textit{Laŋgbando akoŋ gbo pɔkɔni ilel wɔɛ, Sijismɔn, Sijismɔn wɔ ka che as bɛiyɛ, nthela, nye?} This man- I've just forgotten his name, Sigismund, Sigismund was the chief, you hear that, right?

\TCheadword{sik} \textit{cf}: \TClink[2]{panth} (der. of \TClink[1]{panth}). \textit{v} tie. \textit{Ni wɔ koi mbaŋɛ mbul-mbul, ni sik ni ayen.} And he took the ropes, one-by-one, and tied them around his middle (\citealt{Sumner1921}). \textit{Ŋ kwey sangba nyok lo ni nsïk hɔ̃ Yema gbɔl!} Take this string of corals and tie them on Yema's neck [heart?]! (\citealt{Pichl1967}). 

\TCsubword{sikni} (der.) \textit{v} tie onto oneself. \textit{N sïkni bank lo!} Tie this rope on yourself (\citealt{Pichl1967}). 

\TCheadword{sikɛ} \textit{n} [síkɛ́] doubt (K dialect). 

\TCheadword{sikni} (der. of \TClink{sik}, \TClink{-ni}, see \TClink{sik}) 

\TCheadword{sikonde} \textit{n} lovely singing voice (K dialect). 

\TCheadword{sikɔ} \textit{n} mast. \textit{Wɔn dɛ hɔ̃ pə welɛ kɔta lɛ, hɔ̃ bi gbo sukɔ, hɔ̃ sikɔ thitïŋ de pə hɔ̃ welɛ skuna.} The one (boat) that's called a cutter has only one mast, this (one) with two masts is called a schooner (\citealt{Pichl1967}). 

\TCheadword{siks} (Eng \textit{six}) \textit{Numb} six. \textit{A kaŋa ŋa nɛn thi tiŋ ai mɛkni standad siks.} I studied here for two years, and I stopped at standard six. 

\TCheadword{sikstin} (Eng \textit{sixteen}) \textit{Numb} sixteen. \textit{Pandɛ ŋɔ pɔ wɔ April, nɛndɛ ŋɔ pɔ wɔ tu thaozin ɛn sikstin.} The month they call April, the year they call two thousand and sixteen.

\TCheadword[1]{sil} \textit{cf}: \TClink[1]{bɛŋk}, \TClink[4]{bol}. \textit{n} (wɔ/hã, i) kind of maggot living in wet soil that attacks the skin of young children and animals (\citealt{Pichl1967}).

\TCheadword[2]{sil} \textit{v} continue.

\TCheadword[3]{sil} \textit{v} sting. \textit{Isilɔ hã silɛ mi.} The bees sting me (\citealt{Pichl1967}).

\TCheadword{silal} \textit{n} joke. \textit{A che gbo pɔng silal yɛ ya fɔs mɔ thipɛpɛ lɛ, mma silini.} I was only joking when I tapped your shoulders, don't be annoyed (\citealt{Pichl1967}). 

\TCheadword{silini} (der. of \TClink{siil}, \TClink{-ni}, see \TClink{siil}) 

\TCheadword{silka} \textit{v} lessen. \textit{Influɛnsa lɛ hɔ̃ tipɛ silka Kyamp ka.} Influenza has begun to lessen in Freetown (\citealt{Pichl1967}). 

\TCheadword{silɔ} (der. of \TClink{siil}, \TClink[2]{lɔ}, see \TClink{siil})

\TCheadword{silɔpɔŋkthɔ} (comp. of \TClink{silɔ} (der. of \TClink{siil}, \TClink[2]{lɔ}), see \TClink{siil}) 

\TCheadword{Simbo} \textit{nam} Simbo, name given to a person. \textit{Yaŋ a Agnɛs Jami Simbo.} I am Agnes Jamie Simbo.

\TCheadword{simɛnt} (Eng \textit{cement}) \textit{n} cement.

\TCheadword{simgbɔljɛm} (comp. of, id. of \TClink{simi}, \TClink{gbɔl}, \TClink{jɛm}, see \TClink{gbɔl}) 

\TCheadword{simi} \textit{cf}: \TClink{puthuli}, \TClink{rimi} (der. of \TClink[1]{rim}, \TClink[1]{-i}). \textit{v} \textbf{1)} spoiled. \textit{Kɛ mi lagboɛ e, a chɔŋɔmɔ sɛkɛ ŋa yɛ mɔ simiɛ mpanth ma mɔɛ.} But mother that is that, I thank you for spoiling (interrupting) your work. \textit{Apokana tirɛ ŋae hɔɛ, Taalaŋgba ki koŋ simi saba tirɛ njɛm.} The townspeople then said, This man has spoiled the town law. \textbf{2)} be poisoned. \textit{Yenjo le hɔ̃ simiɛ.} The food is spoiled by poison (\citealt{Pichl1967}). \textbf{3)} destroy. \textit{Bìkɛ̀ sìmìɛ́ kə̀llɛ̀.} The storm destroyed the house. \textbf{4)} violate. comp., id. \TClink{simgbɔljɛm} (see \TClink{gbɔl})

\TCsubword[1]{simjɛm} (comp.) \textit{v} \textbf{1)} [símjɛ́m] spoil (K dialect). \textit{Ŋ koŋ simi tamɔ lo njɛm.} You have spoiled this child (\citealt{Pichl1967}). \textit{Thalɔ, kɛ ŋa ko tha shimi njɛm.} They (the laws) are there, but they have spoiled them (due to greed). \textbf{2)} be damaged. \textit{Ŋkɔ lɛrka bot lɛ hɔ simjɛm dɛ.} Go repair the boat, it is damaged (\citealt{Pichl1967}). \textbf{3)} be discouraged. \textit{Si gbɔl hi lɛ yema simjɛm...} And then when our will is discouraged... (\citealt{Pichl1967}). \textit{Yi ma yom gbɔl hi lɛ kɔ simjɛm.} We never shall be discouraged in our hearts (\citealt{Pichl1967}). 

\TCsubword[2]{simjɛm} (comp.) \textit{n} misunderstanding. \textit{Mma ha lwɛ thiyeŋ, siminjɛm bom hɔ hani ki.} Do not go between (don't interfere), this is a big misunderstanding (\citealt{Pichl1967}). \textit{Simi-njɛm bo̹m hɔ̃ kong duk Sayprɔs Agriika lɛ thiye̹ng aña Thɔɔki lɛ.} A big misunderstanding has been created (befallen) in Cyprus between the Greeks and the Turks (\citealt{Pichl1967}). 

\TCsubword{simɔŋgama} (comp.) \textit{n} incest (brother-sister, parent-child). It is said that children begotten in incest must die. The couple who have committed incest are washed, together with a red dog, in the sea or in a river. At Tei(Krim) I was told that there was a special medicine for this purpose, but as the owner of the medicine, a woman, died, they do the washing without the medicine (\citealt{Pichl1967}). 

\TCheadword{siminji} (Soso) \textit{n} (kɔ/-) cloves (\citealt{Pichl1967}). \textit{Siminji lɛ hɔ̃ prɛs kathïl.} Cloves have a high price (\citealt{Pichl1967}). 

\TCheadword[1]{simjɛm} (comp. of \TClink{simi}, \TClink{jɛm}, see \TClink{simi}) 

\TCheadword[2]{simjɛm} (comp. of \TClink{simi}, \TClink{jɛm}, see \TClink{simi}) 

\TCheadword{simɔm} \textit{n} [símɔ́m] new graduate from a society (K dialect). 

\TCheadword{simɔŋgama} (comp. of \TClink{simi}) 

\TCheadword{sin} \textit{cf}: \TClink{gbundɛ}, \TClink[2]{mɔn}, \TClink[2]{sɔkba}, \TClink{tombo}. \textit{n} \textbf{1)} trouble; \textit{isin} (hɔ̃/-) trouble (\citealt{Pichl1967}). \textbf{2)} suffering. \textit{Ya bɛŋ isin.} I am suffering (\citealt{Pichl1967}). \textit{Bahin chala bɛ liwai igbo bɛŋ sin o.} Our father sits on his throne and we are suffering here. \textbf{3)} poverty. \textit{Isin dɛ tala mi.} Poverty depresses me (\citealt{Pichl1967}). \textbf{4)} shortage. \textit{Aa, la ko che ishin fli-o!} Ah, it has now become a real shortage-o! 

\TCheadword{sinthil} \textit{cf}: \TClink[1]{san}. \textit{n} [sìnthíl] ant species, red tree ant found in the bush that resembles \textit{san} but is red with a painful bite that draws blood (K dialect). 

\TCsubword{sithapɔm} (comp.) [sìthàpɔ́m] \textit{n} ant species, large red ant that moves in troops, for some a sign of death (K dialect). 

\TCheadword{sinthimey} (Port \textit{São Tomé} ‘São Tomé') \textit{n} (kɔ/ma) banana species, silver banana (\citealt{Pichl1967}). 

\TCheadword[1]{siŋ} (der. of \TClink[2]{siŋ}) 

\TCheadword[2]{siŋ} \textit{cf}: \TClink{ple}. \textit{v} play. \textit{A ka che siŋ bɔllɛ.} I used to play football.

\TCsubword[1]{siŋ} (der.) \textit{n} game. \textit{Siŋthi handɔ tha nkache siŋda?} What games did you used to play? \textit{Mi yɛ mɔ kache taɛ sinthɛ handɔ tha mɔ ka chɔŋ len ŋa siŋ ŋa?} Mummy, when you were small, what kind of games did you like to play?

\TCsubword{siŋɛsiŋɛ} (der.) \textit{v} play. \textit{Apimaɛ ŋa siŋɛ-siŋɛ gbo haŋ lɛ ŋa wɔ bo ŋa yema jo…} The children played around, if they say they want to eat…

\TCheadword{siŋgitha} \textit{v} be mixed up; [sìgbìthà] mix up (K dialect). \textit{Apimaɛ ha le gbo nan tee ni ayeŋ ha Ba Naɛ hɔ koŋ siŋgitha.} The children kept drawing the rope until the spider's middle became tightly small (AB: mixed up) (\citealt{Sumner1921}).

\TCsubword{siŋi} (der.) \textit{v} play. \textit{Aa asiŋ, komɔ taa wɔ gbako ni tipɛni siŋi?} Yes I played, does a little child grow up without playing?

\TCsubword{siŋil} (der.) \textit{v} play with.

\TCsubword{siŋk} (der.) \textit{v} play with.

\TCsubword{siŋma} (der.) \textit{v} play with.

\TCheadword{siŋɛsiŋɛ} (der. of \TClink[2]{siŋ}) 

\TCheadword{siŋi} (der. of \TClink[2]{siŋ}, \TClink[1]{-i}, see \TClink[2]{siŋ}) 

\TCheadword{siŋil} (der. of \TClink[2]{siŋ}, \TClink{-il}, see \TClink[2]{siŋ}) 

\TCheadword{siŋk} (der. of \TClink[2]{siŋ}, \TClink{-k}, see \TClink[2]{siŋ}) 

\TCheadword{siŋma} (der. of \TClink[2]{siŋ}, \TClink[4]{ma}, see \TClink[2]{siŋ}) 

\TCheadword{sipit} (Eng \textit{sip}) \textit{v} sip. \textit{Yema, hã hun sipit nkə lɛ!} Yema, come and sip the malombo (kɛiɛ)! (\citealt{Pichl1967}). 

\TCheadword{Sipot} \textit{nam} Seaport, name given to a place. 

\TCheadword{sipsap} (Eng \textit{sweetsop}) \textit{n} (kɔ/-) sweetsop, a wild fruit that is also planted with a sweet and sour taste and large black seeds, bubbly green exterior (\citealt{Pichl1967})

\TCheadword{Siril} \textit{nam} Cyril, male name given to a person. \textit{Ya a Siril Manli.} I am Cyril Manley.

\TCheadword{sirɔkɔ-hɔl} (unspec. of \TClink{rɔk}, \TClink[2]{ahɔl}, see \TClink{rɔk}) 

\TCheadword{sistha} (Eng \textit{sister}) \textit{nam} Sister. \textit{Sistha Kɔba ŋaha kaŋa hi mpanthoɛ.} Sister Koba is the one that taught us this work.

\TCheadword{sit} (Eng \textit{sit}) \textit{v} sit. \textit{Mɛkin dɛ ya kɔ ni sit Wasi ɛ, Kiamp ka pɛ, ni mpɛnteŋamiyɛ gbi…} Lastly after I sit the WASSCE (West African Senior School Certificate Exam), again here in Freetown, and all my brothers…

\TCheadword{sitaboŋ} \textit{n} (wɔ/hã, N) bird species, small woodpecker (Mesopicos goertae) (\citealt{Pichl1967}). 

\TCheadword{sitha} \textit{n} [sìthá] tree species (K dialect). 

\TCheadword{sithaba} \textit{n} snake species, black cobra, have hoods on the side of the head, will rarely bite (K dialect); \textit{sitaba} (wɔ/hã, N) black venomous snake (\citealt{Pichl1967}). 

\TCheadword{sithapɔm} (comp. of \TClink{sinthil}, \TClink[1]{pɔm}, see \TClink{sinthil}) 

\TCheadword{sithir} \textit{n} main sheet (nautical); \textit{sithir} (hɔ̃/tha) rope to control the sail of a boat (<Eng sheet?) (\citealt{Pichl1967}). 

\TCheadword{siza} (Eng \textit{Cesar}, i.e., Cesarean section) \textit{n} Cesarean section. \textit{Pɔɔ wɔ ŋa kɔ gbemɔ Nyamba ko kɛ lɔ pɔ ka ŋa wɔ sizaɛ, nthela nye.} They (said) she is to go to Moyamba and do the Cesarean-section there. \textit{Ye pɔ koyi kaŋdɛ pɔɛ nkegbo nɔɛ bi gballɛ kɔ ko kunwɔɛ as Sizaɛ…} When we were taught, they said if you see a mark on the belly like from a Cesarean section…

\TCheadword{sizɔs} (Eng \textit{scissors}) \textit{cf}: \TClink{cheara}. \textit{n} scissors. \textit{Abiɛ lɔni bopɛ sizɔs kunɛ abiɛ lɔni makintɔsh kunɛ.} I do not have the scissors in it, nor do I have the makintosh in it.

\TCheadword{skul} (Eng \textit{school}) \textit{cf}: \TClink[3]{kaŋ}. \textit{n} school. \textit{Awokɔ skul, akɔ ko iwɔmdɛ.} Whenever I came from school, I would go for firewood. \textit{Aka che kɔ skul, kɛ a koŋɔni.} I had been going to school, but I did not finish. \textit{Bath kanu lɔ ka che kɔ skullɛ.} It is at Bath Kanu where he went to school.

\TCheadword{skuna} (Eng \textit{schooner}) \textit{n} schooner. \textit{Wɔn dɛ hɔ̃ pə welɛ kɔta lɛ, hɔ̃ bi gbo sukɔ, hɔ̃ sikɔ thitïŋ de pə hɔ̃ welɛ skuna.} The one (boat) that's called a cutter has only one mast, this (one) with two masts is called a schooner (\citealt{Pichl1967}). 

\TCheadword[1]{so} \textit{n} bow for arrows. \textit{Mɛntɛ so lɛ kɔ sonthul.} The arrow of the bow is sharp (\citealt{Pichl1967}). 

\TCheadword{soo} \textit{cf}: \TClink{peenɛ}. \textit{n} \textbf{1)} [sóó] grain species, grows up like Guinea corn but has straight top from which seeds come (K dialect). \textit{Atipɛ yuk yekeɛ, ŋkaŋdɛ, mbinchɛ, pɛlɛ, nsowɛ, ntɔllɛ.} I start to plant cassava, corn, beans, rice, millet, Guinea corn. \textbf{2)} bulrush millet (\citealt{Pichl1967}). 

\TCheadword[2]{so} (Eng \textit{so}) \textit{coordconn} so. \textit{So labi ichɔŋ len ŋa hin chemɔ vel.} So that is why we like to call you. \textit{So wɔi munini, wɔi pɛ mina hun 1980.} So he returned, then he came back in 1980.

\TCheadword{sogboka} (unspec. of \TClink[1]{sɔ}) 

\TCheadword{soja} \textit{n} soldier.

\TCheadword{soko} \textit{n} (wɔ/hã, a) adept of Poro or other society (\citealt{Pichl1967}). \textit{Hã bonion asoko buliŋ-ni nsumoŋ dɛ.} Assemble the Sokos together with the initiates (Call of the Gbanabom to bring together the initiates to warn women and non-initiates to go away) (\citealt{Pichl1967}). 

\TCsubword{sokonɔ} (comp.) \textit{n} \textbf{1)} Poro leader. \textbf{2)} society leader. \textit{Yaa wɔ ka che sokonɔ Bondo.} Her mother was a Bondo leader. \textit{Wɛl, wɔn bɛpɛ ka cheɛ mared uman, wɔi pɛ cheɛ sokonɔ Bondo.} Well, she herself was a housewife, and she was also the head of the Bondo Society. \textbf{3)} tribal authority. \textit{soŋkənɔ} ‘principal man,' Tribal Authority, person of respect (\citealt{Hall1938}). 

\TCheadword{sokonɔ} (comp. of \TClink{soko}, \TClink{nɔ}, see \TClink{soko})

\TCheadword[1]{sokothi} \textit{cf}: \TClink{futh}, \TClink{lɛnthi}, \TClink{suth}, \TClink[2]{wɔ}. \textit{v} [sókóthí] pull out, extract a tooth (K dialect). 

\TCheadword[2]{sokothi} \textit{n} [sókóthí] passageway that is narrow or very narrow or tight room (K dialect).

\TCheadword{solom} \textit{n} gecko; \textit{solo̹m} (wɔ/hã, N, si) gecko (\citealt{Pichl1967}). 

\TCheadword{Solɔko} \textit{nam} title of the Paramount Chief of Bagroo (Lord of the South) (\citealt{Pichl1967}). 

\TCheadword{Solɔku} \textit{nam} Soloku, name given to a person. 

\TCheadword{sombɔl} \textit{n} fish species; \textit{sombɔl} (wɔ/hã, N) shortmouth (Hemirhampus braxiliensis; Hyporhampus calabaricus) (\citealt{Pichl1967}).

\TCheadword{sonthi} \textit{cf}: \TClink[1]{gbɛŋgbɛs}. \textit{v} [sónthí] pull grass, weed (K dialect). \textit{Yi koŋ gbo sonthi, pəlɛ lɛ kɔ ni pɔl len yenkəlɛŋ.} If you have weeded enough, the rice will grow well (\citealt{Pichl1967}). 

\TCsubword{sonthini} (der.) \textit{v} comb one's hair. \textit{Waŋ lɛ bi sonthok kəlɛŋ, ya bonthɔ wɔ sonthini irĩŋ wɔ lɛ.} The girl has a nice comb; I found (met) her combing her hair (\citealt{Pichl1967}). 

\TCsubword{sonthok} (der.) \textit{n} hair comb. \textit{Waŋ lɛ bi sonthok kəlɛŋ.} The girl has a nice comb (\citealt{Pichl1967}). 

\TCheadword{sonthini} (der. of \TClink{sonthi}, \TClink{-ni}, see \TClink{sonthi}) 

\TCheadword{sonthok} (der. of \TClink{sonthi}, \TClink{-k}, see \TClink{sonthi}) 

\TCheadword{sonthul} (der. of \TClink[1]{sɔnth}, \TClink{-ul}, see \TClink[1]{sɔnth})

\TCheadword{sonthuli} (der. of \TClink{sonthul} (der. of \TClink[1]{sɔnth}, \TClink{-ul}), \TClink[1]{-i}, see \TClink[1]{sɔnth}) 

\TCheadword[1]{soŋk} \textit{v} \textbf{1)} [sóŋk] heal (K dialect). \textbf{2)} get well, recover from illness (\citealt{Pichl1967}). \textit{Ya koŋ kwey saaka thigber ha nrɔmp lɔ kə ya sonkɔni.} I have made so many sacrifices for this sickness, but I have not gotten well (\citealt{Pichl1967}).

\TCsubword{soŋki} (der.) \textit{cf}: \TClink[2]{ramil}. \textit{v} heal, cure. \textit{A kɔ dɔkta lɛ ni sonki mi.} I went to the doctor and he healed me (\citealt{Pichl1967}). \textit{Ŋrɔm do ma ŋkəlɛŋ, ma bi ha sonki mɔ.} This medicine is good, it will cure you (\citealt{Pichl1967}). 

\TCheadword[2]{soŋk} \textit{n} [sòŋk] insect species (K dialect). 

\TCsubword{soŋktho} (comp.) \textit{n} [sóŋkthó] bee species, makes sweet honey, very small, needs only a small hole through which to enter into a cavity (K dialect). 

\TCheadword{soŋki} (der. of \TClink[1]{soŋk}, \TClink[1]{-i}, see \TClink[1]{soŋk}) 

\TCheadword{soŋktho} (comp. of \TClink[2]{soŋk}, \TClink[2]{tho}, see \TClink[2]{soŋk}) 

\TCheadword{sopanth} \textit{n} [sópánth] vine species, very tiny, will wrap around anything, even cassava, can prevent cassava from having good roots (K dialect); (kɔ/ma) vine species, seashore dodder (Cassytha filiformis; Merremia angustifolia) (\citealt{Pichl1967}). 

\TCheadword{sos} \textit{n} fish species, cassava fish (\citealt{Pichl1967}).

\TCheadword{Soso} \textit{nam} Soso people. \textit{Nthemdɛ ma lɔ, Asosoɛ ŋa lɔ, n shiɛ Shenge ka fishiŋ-grɔn lɔɛ.} The Themne are there, the Soso are there, you know, Shenge here is a fishing ground.

\TCheadword[1]{soso} \textit{v} \textbf{1)} flow. \textit{Tɛm hɔ gbo ken mɛn nsoso lɛ hɔ chenk anyathi gbi.} Time is like running water, it carries people away (\citealt{Pichl1967}). \textbf{2)} bleed. \textit{Ŋkɔŋ dɛ ma sɔs.} The blood is running (He is bleeding) (\citealt{Pichl1967}).

\TCheadword[2]{soso} \textit{n} society.

\TCheadword{Sotahuŋ} \textit{nam} Sotahun, name given to a place. \textit{Wante mɔɛ wɔlɔ Sotahuŋ?} Your sister is in Sotahun?

\TCheadword{soth} \textit{v} \textbf{1)} come out, as grain from stalk. \textit{Pəlɛ Kɔŋ dɛ koŋ soth.} Kong‘s rice is coming out (ready for harvesting) (\citealt{Pichl1967}). \textbf{2)} sprout. \textit{Ŋkaŋ dɛ ma soth.} The corn has sprouted (\citealt{Pichl1967}). 

\TCsubword{sothsothni} (der.) \textit{v} \textbf{1)} enter into. \textit{Ŋana tiŋ ŋa siŋ: ŋae sɔth-sothni wusɛ kunɛ.} Both of them are playing when they enter into the thatch. \textbf{2)} hide. \textit{Ŋae koŋ sɔth-sɔthni wusɛ kunɛ.} They (the rats) went and hid in the thatch.

\TCheadword{sothɔ} \textit{n} straw, stem of a plant (K dialect). 

\TCheadword{sothsothni} (der. of \TClink{soth}, \TClink{-ni}, see \TClink{soth}) 

\TCheadword[1]{sɔ} \textit{cf}: \TClink{hɛŋ}, \TClink{kakbom} (comp. of \TClink[2]{kak}, \TClink{bom}), \TClink[2]{kak}. \textit{n} south wind

\TCsubword{sogboka} (unspec.) \textit{cf}: \TClink{hɛŋ}. \textit{n} \textbf{1)} wind; breeze. \textit{Sogboka disil wɛini ŋɔe kɛnth, mmɛn dɛ mae huk yeŋwɛini.} Heavy winds (from the land) broke, the water was vexed (agitated). \textbf{2)} south wind.

\TCheadword[2]{sɔ} \textit{v} \textbf{1)} cut grass. \textit{Ŋ kɔ sɔ næ lɛ.} Go clean the grass from the road (\citealt{Pichl1967}). \textbf{2)} \textit{shɔ̀} hoe grass (\citealt{Sumner1921}).

\TCheadword[3]{sɔ} \textit{v} occur.

\TCheadword{sɔan} \textit{n} temptation. \textit{Nai wɛ ŋɔ vil ni ŋɔ chɔɔlen mɔnɛ ni sɔan ma lɔ.} The road is long and it is difficult and there are many temptations.

\TCheadword{Sɔba} \textit{nam} Sɔba, name given to third son. 

\TCheadword{sɔbul} (Eng \textit{shovel}) \textit{n} shovel; \textit{sɔbul} (hɔ/tha) shovel, spade (\citealt{Pichl1967}). 

\TCheadword{sɔi} \textit{v} mix (\citealt{Pichl1967}).

\TCsubword{pɛlɛsɔi} (comp.) \textit{n} rice husk; \textit{pəlɛ-sɔy} rice husk (\citealt{Pichl1967}).

\TCsubword{sɔisɔi} (der.) \textit{cf}: \TClink{pɛth}. \textit{adj} delicious, tasty (lit. thoroughly-mixed). \textit{Ŋa jo ŋje ma sɔisɔi gbi ŋa piŋini gbo we.} They eat delicious food, yet still they turn against us. \textit{Ŋa kul mɔi ma sɔisɔi gbi ŋa koi piŋiɛni.} They drink tasty (well-mixed) drinks, and they turn against us.

\TCsubword{sɔima} (der.) \textit{cf}: \TClink{binthima}. \textit{v} bring together, mix up (\citealt{Pichl1967}). \textit{Ŋ kɔ sɔyma pɛlɛ lɛ ni ntɔl lɛ ni nyɔk ma chɛk lɛ ko.} Go mix the rice and the Guinea corn together and take them to the farm (\citealt{Pichl1967}). 

\TCheadword{sɔik} (der. of \TClink{sɔyɛ}, \TClink{-k}, see \TClink{sɔyɛ}) 

\TCheadword{sɔisɔi} (der. of \TClink{sɔi}) 

\TCheadword{sɔk} \textit{cf}: \TClink{samak}. \textit{n} \textbf{1)} fowl; [sɔ̀k] fowl, chicken (K dialect); [chà:sɔ̀k] fowl feather (B dialect). \textbf{2)} chicken. comp. \TClink{chaasɔk} (see \TClink[1]{cha}), \TClink{husɔk} (see \TClink[1]{hu}), \TClink{tasɔk} (see \TClink{taa}) 

\TCsubword{sɔkma} (comp.) \textit{n} hen; \textit{sɔk-maa} hen (\citealt{Pichl1967})

\TCsubword{sɔkpokan} (comp.) \textit{n} cock, rooster; \textit{sɔk pokan }(wɔ/hã) cock (\citealt{Pichl1967}). \textit{Sɔkpokan dɛ wɔ woŋ.} The cock crows (\citealt{Pichl1967}). 

\TCsubword{vɛsɛksɔk} (comp.) \textit{n} chicken leg; \textit{vəsɛk sɔk} (hɔ̃/-) leg and foot of fowl (\citealt{Pichl1967}). 

\TCheadword[1]{sɔkba} (der. of \TClink[2]{sɔkba}) 

\TCheadword[2]{sɔkba} \textit{cf}: \TClink{gbundɛ}, \TClink{sin}, \TClink{tombo}. \textit{n} \textbf{1)} disturbance. \textbf{2)} trouble. \textbf{3)} problem. \textit{Kɛn bo bi ŋsɔkba la mɔ tenɛ, ha mɔn wɔ...} If you have a problem in mind and you want to talk...

\TCsubword[1]{sɔkba} (der.) \textit{v} \textbf{1)} tease. \textit{Athɛma wɔ lɛ hã sɔkba wɔ gbo, wɔ lɔ pok.} Whenever his companions teased him, he went away from them (\citealt{Pichl1967}). \textbf{2)} disturb. \textit{Lanɛ laŋ la sɔkba mɔ gbi.} That is the only one that really disturbed you. \textit{Mma mi sɔkba ya chen vee.} Don't disturb me, I am not well (\citealt{Pichl1967}). \textbf{3)} trouble (\citealt{Sumner1921}). 

\TCheadword{sɔki} \textit{v} \textbf{1)} \textit{sɔɔki} repair a leaky roof (\citealt{Pichl1967}). \textbf{2)} rethatch. \textit{Ŋ kɔ-m sɔɔkiɛ kïl mi lɛ, hɔ̃ gbɔw dul.} Go re-thatch my roof, it is leaking too much (\citealt{Pichl1967}). 

\TCheadword{sɔkma} (comp. of \TClink{sɔk}, \TClink{maa}, see \TClink{sɔk}) 

\TCheadword{sɔkɔth} \textit{cf}: \TClink{kɔysu}. \textit{n} [sɔ̀kɔ̀th] miracle that someone performs (K dialect); magic, mystery, secret (\citealt{Pichl1967}).

\TCheadword{sɔkpokan} (comp. of \TClink{sɔk}, \TClink{pokan} (unspec. of \TClink[5]{po}), see \TClink{sɔk}) 

\TCheadword{sɔku} \textit{cf}: \TClink{kɔna}, \TClink[2]{thuŋk}. \textit{n} [sɔ́kú] corner (K dialect). \textit{Kɔ bimni sɔku bullai, wɔ hɔɔl <fɔɔ fɔɔ fɔɔ> ni yeke wɔɛ che wɔn kunɔlɔ.} (She) went and bent over in one corner, she breathed <fɔɔ fɔɔ fɔɔ> (idph of panting) with the cassava (tucked) in her bosom.

\TCheadword[1]{sɔkul} \textit{v} itch. \textit{Yaŋ likɔ lɔ gbɔw sɔkul, ya bi isɔkul gber.} My skin is very itchy, I have a lot of craw-craw (\citealt{Pichl1967}). \textit{Thenthes hɔ wɛy, pə bak hɔ gbo nɔ wɔ sɔkul likɔɔ.} \TClink[1]{thenthes} is bad, they just rub it on a person, (and) it makes him scratch his skin (\citealt{Pichl1967}). 

\TCsubword[2]{sɔkul} (der.) [isɔkul] \textit{n} craw-craw (\citealt{Pichl1967}). \textit{Yaŋ likɔ lɔ gbɔw sɔkul, ya bi isɔkul gber.} My skin is very itchy, I have a lot of craw-craw (\citealt{Pichl1967}). 

\TCheadword[2]{sɔkul} (der. of \TClink[1]{sɔkul}) 

\TCheadword{sɔl} \textit{v} build a boat or canoe; \textit{sɔl} build a boat or canoe [sculpt?] (\citealt{Pichl1967}). \textit{Hã bue thɔk lɛ hã hã sɔl wɔm.} they hollowed the tree to make a canoe (\citealt{Pichl1967}). 

\TCheadword[1]{sɔlɛma} \textit{n} hassle. \textit{Ye sɔlɛmaɛ yɛ mɔ chai iroɛ, mbɔni ha paka ŋɔ.} What a hassle (it is) when you borrow something and you cannot pay it back.

\TCsubword[2]{sɔlɛma} (der.) \textit{v} bother, hassle. \textit{A-a bikɔ ma kɔ koi yen nɔ pɔ mɔi sɔlɛma.} No, do not take things from people (i.e. borrow) because then they will hassle you. 

\TCsubword[3]{sɔlɛma} (der.) \textit{adj} [sɔ̀lèmá] troublesome (K dialect). 

\TCheadword[2]{sɔlɛma} (der. of \TClink[1]{sɔlɛma}) 

\TCheadword[3]{sɔlɛma} (der. of \TClink[1]{sɔlɛma}) 

\TCheadword{sɔlɔk} \textit{n} insolence. \textit{Tamɔ ichɛklïn, wɔ ki wɔ nsɔlɔk.} This is a proud boy, he is insolent (\citealt{Pichl1967}). 

\TCheadword{sɔm} [sɔm] \textit{cf}: \TClink{chamak}, \TClink[1]{jo}. \textit{v} \textbf{1)} eat. \textit{Sakiɛ kɔn ache bɔ yuk bikɔs kulunsɛ ŋa kɔ sɔm.} The cassava leaves is what I do not plant because the goats would eat it. \textit{Yem, ŋka mi yeke mɔɛ pum ni ya sɔm, ndikɛ ma mi.} Madam, give me some of your cassava, let me eat, I am feeling hungry. \textbf{2)} chew. \textit{Lɛ nsi gbo lɔŋ, nsi gbo hɔth, mɔ sɔthɔ yen sɔmɔ.} If you know how to set traps at all, you know how to fish, you would get something to chew. \textit{Ya chen mɔ ŋɔn ka ni nsɔm.} I will give you nothing to chew.

\TCsubword{sɔm-sɔm} (der.) \textit{v} chew.

\TCheadword{sɔmbu} \textit{cf}: \TClink{bɔtakɛl} (comp. of \TClink[1]{baa}), \TClink{kɛko}. \textit{n} (wɔ/hã, si) ground squirrel (\citealt{Pichl1967}).

\TCheadword[1]{sɔn} \textit{v} dream. \textit{Ŋa lɔ ni le kunɛ wandaɛ wɔi sɔn yawɔ.} They were until one day the girl dreamt about her mother. comp. \TClink{nɔlimɛnsɔn} (see \TClink{nɔ})

\TCsubword[2]{sɔn} (der.) \textit{n} dream.

\TCheadword{Sɔna} \textit{nam} Sona, female name given by Toma Society. 

\TCheadword{sɔna} (der. of \TClink[2]{isɔ}, \TClink[2]{na}, see \TClink[2]{isɔ}) 

\TCheadword{sɔnda} \textit{cf}: \TClink{laɔn}. \textit{n} [sɔ̀ndà] lion (K dialect). 

\TCheadword{sɔnday} \textit{cf}: \TClink{ko-lɔ}. \textit{subordconn} rather than. \textit{Sɔnday gbo kə ya ma wu.} Rather than that, let me die (\citealt{Pichl1967}).

\TCheadword{Sɔnde} (Eng \textit{Sunday}) \textit{nam} Sunday. \textit{Nande sɔnde 7th ɔf fɛbwari, iya 2016.} Today is Sunday, the seventh of February, 2016.

\TCheadword{sɔnɔ} \textit{n} (kɔ/ma) small oil palm seeds that have not grown to full size (\citealt{Pichl1967}). 

\TCheadword[1]{sɔnth} \textit{v} \textbf{1)} sew. \textit{Wɔ nɔ che sɔnthɔɛ?} Is he the one that sews your clothes? \textit{Wɔ mɔ sɔnthɔ charaŋ.} He would sew it for you nicely. \textbf{2)} mend. comp. \TClink{nɔsɔnthɔ} (see \TClink{nɔ}) 

\TCsubword{sonthul} (der.) \textit{adj} sharp. \textit{Mɛntɛ so lɛ kɔ sonthul.} The arrow of the bow is sharp. der. \TClink{sonthuli} (see \TClink[1]{sɔnth})

\TCsubword{sonthuli} (der.), (der. of \TClink{sonthul}) \textit{cf}: \TClink{boŋhul}. \textit{v} sharpen. \textit{N sonthuli pɛnsil lɛ ha yaŋ.} Sharpen the pencil for me (\citealt{Pichl1967}). 

\TCsubword{sɔnthi} (der.) \textit{v} patch up old clothes (\citealt{Pichl1967}).

\TCheadword[2]{sɔnth} \textit{cf}: \TClink{kuaŋa}. \textit{adj} twentieth. \textit{Tipik huɛ seinyɛ, ŋɔ pɔ vellɛ Flaideɛ Mpothoaiɛ, nduɛ waŋnimɛnraɛ haaŋ la muɛ ko huɛ sɔnthɛ.} Beginning the first day, which they call Friday in English, from the 18\textsuperscript{th} day until the 20\textsuperscript{th} day.

\TCheadword[3]{sɔnth} \textit{n} bush.

\TCheadword{sɔnthɛ} \textit{nam} Sunday.

\TCheadword{sɔnthi} (der. of \TClink[1]{sɔnth}, \TClink[1]{-i}, see \TClink[1]{sɔnth}) 

\TCheadword{sɔnthɔ} \textit{cf}: \TClink{pɔl}, \TClink{yɔŋ}. \textit{n} [sɔ̀nthɔ̀] trap for fish (K dialect). 

\TCheadword{sɔnɡɔ} \textit{subordconn} according.

\TCheadword[1]{sɔŋ} \textit{cf}: \TClink{kolsiroŋ} (comp. of \TClink{kol}). \textit{v} bribe; \textit{sɔŋ} bribe, but more through persuasion than money (\citealt{Pichl1967}). \textit{Ŋ kɔ wɔ sɔŋ ni na-m hok ve̹le̹ŋ.} Go bribe him so that he may not expose me (\citealt{Pichl1967}). 

\TCheadword[2]{sɔŋ} \textit{cf}: \TClink{bus}. \textit{v} cut up meat, skin, undress (\citealt{Pichl1967}). \textit{Bia wɔ lɛ poinɔ di chal, hã kɔ wɔ sɔŋ.} Bia is a hunter, he has killed an antelope, (you pl.) go cut it up (\citealt{Pichl1967}). 

\TCheadword[3]{sɔŋ} \textit{n} \textbf{1)} [sɔ̀ŋ] tree species, large tree, spice used for medicine (K dialect); tree species, spice tree (\citealt{Pichl1967}). \textbf{2)} \textit{nsɔng} spice (\citealt{Pichl1967}).

\TCheadword[1]{sɔŋk} \textit{n} (kɔ/ma) cork (of a bottle), stopper (\citealt{Pichl1967}). 

\TCsubword[2]{sɔŋk} (der.) \textit{cf}: \TClink{lompu} (unspec. of \TClink{lɔŋ}) \textit{v} \textbf{1)} cork bottle (\citealt{Pichl1967}). \textbf{2)} load gun (\citealt{Pichl1967}).

\TCheadword[2]{sɔŋk} (der. of \TClink[1]{sɔŋk}) 

\TCheadword{sɔŋkɔ} (Mandinka \textit{sõgo} ‘price') \textit{cf}: \TClink{prɛs}. \textit{n} \textbf{1)} price. \textbf{2)} value. \textit{Bikɔ pomdɛ wɔ mi ni yɛthi sɔŋgɔ ma ŋɔ nɔpikan wɔ ŋa yɛthi nɔma wɔi.} Because my husband is really treating me as a husband should treat his wife.

\TCheadword{sɔŋkɔma} \textit{subordconn} \textbf{1)} as. \textit{Lɛ bi gbo ilɔɔ, pɔ wɔ di sɔŋkɔma ŋɔ saba tirɛ hɔɛ.} If he is guilty, they will kill him as the town law says. \textbf{2)} like. \textbf{3)} how.

\TCheadword{sɔŋɡɔ} \textit{subordconn} according to.

\TCheadword{Sɔɔ} \textit{nam} So, name given to third son.

\TCheadword{sɔpɔt} (Eng \textit{support}) \textit{v} support. \textit{Yɛ laio wɛ, ye mpanth mɔ ni ha ha sɔpɔt abena mɔi?} As it is, what work do you now do to support your parents? \textit{Che yi koŋ sɔpɔt bikɔs ramdɛ kɔ bom che yi koŋ sɔpɔt gbi.} She does not support us all because the family is big.

\TCheadword{sɔsɔkɔ} \textit{v} \textbf{1)} sweep away. \textit{Ŋɔ Hɔbatokɛ loliɛ taamɔtaa bul, wɔ mmɛn hukɔ ni ihɛŋ disil-disil sɔsɔkɔ.} How God saved a little boy, whom heavy waves and heavy winds swept away. \textbf{2)} carry away. \textit{La vein dɛ mae sɔsɔk wɔm dɛ.} It was not long before this water carried (washed) away the canoe.

\TCheadword{sɔthɔ} \textit{cf}: \TClink[3]{san}. \textit{v} \textbf{1)} get. \textit{Wɔn kɛndɛ vɛ wɔ asɔthɔ bo prɔblɛm.} That is the only problem I had. \textit{Nɛki gbɔl ko sɔthɔ ko, lanɛ gbi nante.} There is heartache in this world today. \textbf{2)} receive. \textbf{3)} secure. \textit{La hini ha, ŋa sɔthɔ hini-gbɔl?} What are we to do, to have peace of mind? \textbf{4)} have. \textit{Wɛl imɛmiɛni ŋa hin sɔthɔ mɔ.} Well we are happy to have you. \textbf{5)} catch. \textbf{6)} regain.

\TCheadword{sɔvaiv} (Eng \textit{survive}) \textit{v} live; survive. \textit{Boŋ cheki, ma koŋ gbako, wɔn pɛ lɔ ni sɔvaiv.} Now, they have grown (the oil palms), he is there and lives (off of it). 

\TCheadword{sɔvaiva} (Eng \textit{survivor}) \textit{n} survivor. \textit{Yaŋ ya sɔvaivaɛ ni wante mi bul wɔ lɔ bɛ Nyemɔkɔ, Bompɛ Chifdɔm.} I am the remaining one with one of my sisters, she is even in Moyeamoh, Bumpeh Chiefdom.

\TCheadword{sɔyɛ} \textit{cf}: \TClink{hothɔk}, \TClink{jɔhɔ}, \TClink{pakali} (der. of \TClink{pakil}, \TClink[1]{-i}), \TClink{woli} (der. of \TClink[1]{woi}, \TClink[1]{-i}). \textit{v} \textbf{1)} [sɔ́yɛ́] scare (someone) (K dialect). \textit{Liwu lɔ che hini sɔyɛ.} Death does not frighten us (\citealt{Pichl1967}). \textbf{2)} threaten (\citealt{Pichl1967}). 

\TCsubword{sɔik} (der.) \textit{v} scare (B dialect). \textit{Mi Mabɛl, yɛ nka che tallɛ, mbi nɔ wɔ ka che mɔ sɔikɛ?} Mammy Mabel, when you were young, did you have someone that used to scare you? \textit{Ye pɔ sɔikɛ nɔɛ.} When they scare someone. \textit{Tha sɔikɛ yenchɛkɛ, yɛ ŋa the tiŋ yɛ vɛ ŋa ɡbikini wa!} They (trawlers) scare the fish away, when they (the fish) hear the noise, they flee in a panic!

\TCsubword{sɔyɛni} (der.) \textit{v} \textbf{1)} scare (\citealt{Pichl1967}). \textbf{2)} deceive. \textit{O, hɔ̃ hwɛlɔ lɛ hɔ̃-m sɔyɛni.} Oh, how this world deceives (tempts) me (\citealt{Pichl1967}). 

\TCheadword{sɔyɛni} (der. of \TClink{sɔyɛ}, \TClink{-ni}, see \TClink{sɔyɛ}) 

\TCheadword{sɔima} (der. of \TClink{sɔi}, \TClink[4]{ma}, see \TClink{sɔi}) 

\TCheadword{spika} \textit{n} speaker. \textit{MB Baro wɔ ka che spikaɛ.} M.B. Baro, he was the speaker. \textit{Awun Spika 2013.} I became Speaker in 2013.

\TCheadword{staf} \textit{n} staff.

\TCheadword{standad} (Eng \textit{standard}) \textit{nam} standard as an educational level, compare to grade in the American system. \textit{Standad fɔ lɔ m mɛkɛni?} You stopped at standard four? \textit{A kaŋa ŋa nɛn thi tiŋ ai mɛkni standad siks.} I studied here for two years, and I stopped at standard six. 

\TCheadword{stej} (Eng \textit{stage}) \textit{n} stages. \textit{Gbemɔɛ kɔ bi stej.} Giving birth has stages.

\TCheadword{stich} (Eng \textit{stitch}) \textit{v} stitch. \textit{Yɛ ha ka stich kun wɔɛ, ko lɔ gbemɛkɛ ŋɔ ho kaɛ…} When her belly was stitched, where the baby comes out… \textit{Ŋa ŋɔi stich ahɔl.} They had stitched the exit mouth.

\TCheadword{stil} (Eng \textit{still}) \textit{cf}: \TClink{huɛŋ}, \TClink[1]{mu}. \textit{temp} still. \textit{Nɛvamaind yɛ ibiyɛn dɛ kɛ stil ai maneg bikɔs pomdɛ che ŋa mpanth, biyɛni.} Never mind that we do not have anything, but still I manage although my husband does not have work and does not have anything. \textit{Wɔn ka kaŋ Arabikɛ kɛ still ka che famalɛ kunɛ.} He learned Arabic but still he was in this farming.

\TCheadword{Stivɛn} \textit{nam} Stephen, male name given to a person.

\TCheadword{stɔ} (Eng \textit{store}) \textit{n} store. \textit{Pimdɛ kɔnɛ kɔ pɔ bia joɛ, pɔ kɔi bɛ stɔ thai kunɛ.} The remainder will be put aside for food, will be kept in stores.

\TCheadword{stret} (Eng \textit{street}) \textit{n} street. \textit{Wɔ yɛ hɔni street lɛ ibɔl hã kɔ thonkini.} When she has just dressed up, she goes out in the the street to show herself (\citealt{Pichl1967}). 

\TCheadword[1]{su} \textit{n} (kɔ/tha) finger (\citealt{Pichl1967}). \textit{Su bul kɔ chen leiŋ ila}. One finger cannot remove a louse (proverb) (\citealt{TISLL1979}). comp. \TClink{gbɛtsu} (see \TClink[2]{gbɛt}), \TClink{rɛmsupokan} (see \TClink{rɛm}) 

\TCsubword{suayeŋ} (comp.) \textit{n} (kɔ/tha) middle or ringfinger (\citealt{Pichl1967}).

\TCsubword{supokan} (comp.) \textit{n} (kɔ/tha) thumb (\citealt{Pichl1967}).

\TCsubword{suveleŋ} (comp.) \textit{n} (kɔ/tha) little finger (\citealt{Pichl1967}).

\TCheadword[2]{su} \textit{Idph} sound of moving car, motorcycle, snake, for example, <su-su-su-su-su> for the quick motion of a snake through a cassava patch (may be Krio) (B dialect). 

\TCheadword{sua} \textit{v} resist. \textit{Braima wɔe tipɛ yaath ha boŋ ihɛŋ disil wɛin dɛ sua mmɛŋ hukɔ kiai.} Braima then began to paddle to resist the dreadfully heavy winds.

\TCsubword{buŋsua} (comp.) \textit{v} resist; \textit{buŋsua} resist, oppose, go against (\citealt{Pichl1967}). \textit{Mma buŋ ba mɔ sua!} Don't oppose your father! (\citealt{Pichl1967}).

\TCheadword{suayeŋ} (comp. of \TClink[1]{su}, \TClink{ayeŋ}, see \TClink[1]{su}) 

\TCheadword{Sufian} \textit{nam} Suffian, name given to a person. \textit{Sufian Idrisa Koroma.} Suffian Idrissa Koroma.

\TCheadword{suga} (Eng \textit{sugar}) \textit{n} (hɔ̃/-) sugar (\citealt{Pichl1967}). 

\TCheadword{sui} \textit{cf}: \TClink[1]{han}, \TClink[1]{pia}. \textit{n} hand. \textit{Ŋa bi peŋa ŋan sui o.} They have guns in their hands.

\TCsubword{suibaɛ} (comp.) \textit{n} [súíbàɛ̀] palm of the hand (K dialect). 

\TCheadword{suibaɛ} (comp. of \TClink{sui}, \TClink[1]{bai}, see \TClink{sui}) 

\TCheadword{suk} \textit{cf}: \TClink{bolo}, \TClink{chocho}, \TClink{kɔŋko}, \TClink{nɔtɔ}, \TClink{thoŋku}. \textit{n} (wɔ/hã, N) kind of shell, periwinkle (\citealt{Pichl1967}). 

\TCheadword[1]{sukusɛkɛ} \textit{n} [sùkùsɛ́kɛ́] gossip, confusion (K dialect). \textit{M mam chiɛ sukusɛkɛ mɔ lɛ!} Don't bring me your confusion! (\citealt{Pichl1967}). comp. \TClink{nɔsukusɛkɛ} (see \TClink{nɔ}) 

\TCsubword[2]{sukusɛkɛ} (der.) \textit{v} [sùkùsɛ́kɛ́] spread false news, calumny, gossip (K dialect). 

\TCheadword[2]{sukusɛkɛ} (der. of \TClink[1]{sukusɛkɛ}) 

\TCheadword{Suleman} \textit{nam} Suleman, male name given to a person. \textit{Abi Suleman Bɛndu, Usman Bɛndu, Abas Bɛndu ni Muhamɛd Bɛndu.} I have Sulaiman Bendu, Usman Bendu, Abass Bendu and Mohamed Bendu.

\TCheadword{sum} \textit{n} \textbf{1)} [sùm] mouth (K dialect). \textbf{2)} lip (\citealt{Sumner1921}). \textit{Sum nɔpokan dɛ kɔ hinth.} The lip of the man is swollen (\citealt{Pichl1967}). \textbf{3)} (hɔ̃/tha) lips; mouth; beak (\citealt{Pichl1967}). comp. \TClink{sumɔhɔl} (see \TClink[1]{ahɔl}) 

\TCsubword{suma} (der.) \textit{v} [sùmà] twist mouth (K dialect). 

\TCheadword{sumoŋ} \textit{cf}: \TClink{nyama} (der. of \TClink[1]{nya}, \TClink{maa}). \textit{n} initiate.

\TCheadword{sumɔhɔl} (comp. of \TClink{sum}, \TClink[1]{ahɔl}, see \TClink[1]{ahɔl}) 

\TCheadword[1]{sun} \textit{cf}: \TClink[2]{tee}. \textit{n} pestle; \textit{suŋ} (hɔ/ma) pestle of a mortar (\citealt{Pichl1967}). \textit{Ŋ kɔ-m thɛlɛ suŋ dɛ, a yema kɔ tu.} Go trim the pestle for me, I want to go pound (the rice) (\citealt{Pichl1967}). 

\TCheadword[2]{sun} \textit{n} sand; \textit{isuuŋ} (hɔ̃/-) sand (\citealt{Pichl1967}). \textit{Hã ye tipɛ bue isuŋ doki hã hɔ thɔk hã sotho ihyɛl.} Then they began to dig the sand there, and they washed it to get salt (\citealt{Pichl1967}).

\TCheadword{Sundu} \textit{nam} Sundu, name given to a place. \textit{Wɔn wɔ lɔ Sundu ko.} He is living in Sundu.

\TCheadword{suni} \textit{v} well-cooked to the point of being soft. \textit{Yekə lɛ hɔ̃ suni.} The cassava is properly cooked (i.e. soft) (\citealt{Pichl1967}).

\TCheadword{sunth} \textit{n} a very strong rope woven of fiber found in the bush.

\TCheadword{suntha} \textit{cf}: \TClink[2]{suŋkutha} (der. of \TClink[1]{suŋkutha}). \textit{n} mixup. \textit{So nsuntha handɔ ma ko lɔ ni lɔi a?} So what is the mixup that is there now?

\TCheadword{suŋgbasa} \textit{n} boys' evening sport, kind of racing game (\citealt{Pichl1967}).

\TCheadword[1]{suŋkutha} \textit{v} \textbf{1)} destroy. \textbf{2)} mix up. \textit{Yá lá súnkúthá.} Let me mix it up, confuse things. \textit{Há yá má lá súnkúthá.} Let me not mix it up.

\TCsubword[2]{suŋkutha} (der.) \textit{cf}: \TClink{suntha}. \textit{n} unpleasantness.

\TCsubword{suŋkuthani} (der.) \textit{cf}: \TClink[1]{rim}, \TClink{tuk}. \textit{v} \textbf{1)} be destroyed (\citealt{Pichl1967}). \textbf{2)} get lost (\citealt{Pichl1967}). \textbf{3)} be spoiled. \textit{Jali Sese la koŋ sunkuthani nɛn thitiŋ do.} Sese's affairs are completely spoiled the past two years (\citealt{Pichl1967}). 

\TCheadword[2]{suŋkutha} (der. of \TClink[1]{suŋkutha}) 

\TCheadword{suŋkuthani} (der. of \TClink[1]{suŋkutha}, \TClink{-ni}, see \TClink[1]{suŋkutha}) 

\TCheadword{sup} (Eng \textit{soup}) \textit{n} \textbf{1)} soup. \textit{Pɔ ko cheth supɛ libɛn ikoŋ jo.} And the soup had been cooked long ago, we had eaten. \textbf{2)} sauce. \textit{Sùpɛ̀ hɔ́ mɛ̀n.} The soup/sauce is water. \textit{Sùp njɛ̀thíllɛ̀} the weak, tasteless sauce (NCM and Def).

\TCheadword{supokan} (comp. of \TClink[1]{su}, \TClink{pokan} (unspec. of \TClink[5]{po}), see \TClink[1]{su}) 

\TCheadword{supsap} (Eng \textit{soursop}) \textit{n} soursop, wild (K dialect).

\TCheadword{suskɔ} \textit{v} exchange. \textit{So nsuskɔɛ ma handɛ ma vɛ ka ko ki hu lɛ.} So exchanges took place for the deaths.

\TCheadword{suth} \textit{cf}: \TClink{futh}, \TClink{lɛnthi}, \TClink[1]{sokothi}, \TClink[2]{wɔ}. \textit{v} \textbf{1)} pull up. \textit{Ŋ kɔ suth puluk ɛ!} Go pull up the grass! (\citealt{Pichl1967}). \textbf{2)} pluck. \textit{Ŋ kɔ suth chæthi sɔk lɛ!} Go pluck the fowl! (\citealt{Pichl1967}).

\TCheadword{suveleŋ} (comp. of \TClink[1]{su}, \TClink[1]{veleŋ}, see \TClink[1]{su}) 

\TCheadword{swei} \textit{n} soap; \textit{nsuei} soap (\citealt{Sumner1921}). \textit{Nswɛ ki ma bi pukɔ gbe̹r.} This soap foams very much. (\citealt{Pichl1967}). comp. \TClink{lalbo-nswe} (see \TClink{lalbo}) 

\TCsubword{sweindinthɛ} (comp.) \textit{n} \textit{nswey ndinthɛ} (ma) European soap (lit. white soap) (\citealt{Pichl1967}). 

\TCsubword{sweinthi} (comp.) \textit{n} \textit{nswey nthi} (ma) black soap, local soap made from ashes, especially of the cotton tree or the husks of cotton tree nuts which are then mixed with palm oil and boiled (\citealt{Pichl1967}). 

\TCheadword{sweindinthɛ} (comp. of \TClink{swei}, \TClink{dinthɛ} (der. of \TClink{dinth}, \TClink{-ɛ}), see \TClink{swei}) 

\TCheadword{sweinthi} (comp. of \TClink{swei}, \TClink[1]{thi}, see \TClink{swei}) 

\TCheadword{swɛ} \textit{cf}: \TClink{niŋka}. \textit{n} charcoal. \textit{Iswɛ lɛ hɔ gba hink ininka.} Charcoal is different from coal (\citealt{Pichl1967}). 

\end{letter}

\begin{letter}{(Sh)}

\TCheadword{shatin} (Eng \textit{satin}) \textit{n} satin. \textit{Pis dinthɛɛ, shatin.} A piece of white cloth, satin.

\TCheadword{Sheŋge} \textit{nam} Shenge, name given to a place. \textit{Yaŋ pɔ dumɔ mi Shenge ka.} Me, I was raised in Shenge here. \textit{Ahina ŋa chan shi theli Mbolomdɛ Shenge ka.} Who (pl) knows how to speak Sherbro best in Shenge here?

\TCheadword{Sherif} \textit{nam} Sheriff, name given to a person. \textit{Anya hiɛ fɔrina ŋaɛ, Koroma Kallon, Sherif.} Our people are foreigners, Koroma, Kallon, Sheriff.

\TCheadword{Shɛrbro} \textit{nam} Sherbro.

\TCheadword{shiliŋ} (Eng \textit{shilling}) \textit{cf}: \TClink{bolthihiol} (comp. of \TClink[1]{bol}, \TClink{hiɔl}). \textit{n} shilling. \textit{Nka mi shiliŋ bul kaŋka ni a kɔ wɔtalu ko.} Give me a shilling so that I may go to Waterloo. \textit{Ŋ kɔ hoŋ ko ni m pin sɔk shiliŋ thiwaŋ.} Go to the compound and buy a fowl for ten shillings.

\TCheadword{shini} \textit{v} get used to. \textit{Labila ikonlɔ shini.} That is why we have gotten used to it.

\TCheadword{shishkɔ} \textit{cf}: \TClink[1]{piŋki}. \textit{v} change. \textit{Ja landɛ la koŋ shishkɔ?} Those things have been changed? \textit{La koŋ shishkɔ.} They have been changed.

\TCheadword{shiyɔɔɔ} \textit{Idph} of disapproval. \textit{M-m-m-m, <shiyɔɔɔ> ŋhɔ lan bɛ:<ish-sh-sh> ayo, ayo, mɔ ŋɔ sɔm!} Hm-m-m <shiyɔɔɔ> do not even say it: <ish-sh-sh> yes, yes, you will eat it!

\TCheadword{shɔp} (Eng \textit{shop}) \textit{n} shop. \textit{E-e-eh, yam bɛ a sini bikɔs a che chal telɔ shɔp pai.} Eh, myself I do not know because I do not sit at the tailor shop.

\end{letter}
\begin{letter}{T}

\TCheadword{taa} \textbf{1)} \textit{n} [tsáámì] baby boy, young child (male) (B dialect). \textbf{2)} \textit{n} \textit{taa}/\textit{apuma} child/children (\citealt{Pichl1967}). \textbf{3)} \textit{n} young person. \textit{Yɛ nka che ko tallɛ, pɔ ka che mɔ buŋ?} When you (female) were young, did they used to beat you? \textit{Kɛ yɛ laiyoɛ tamɔ ta kani nɔ santh limani.} But as it is a young boy does not give adults respect. \textbf{4)} \textit{n} junior. \textit{Yɛ wɔ wu wɛ,wɔ wɔi leyɛ thɛmko wɔ taɛ.} After she died, she left her with her junior mate. \textbf{5)} \textit{adj} small. comp. \TClink{palta} (see \TClink[3]{pal}), \TClink{rɛmtaa} (see \TClink{rɛm}) 

\TCsubword{taalaŋgbaŋ} (comp.) \textit{cf}: \TClink{tamɔ-laŋgbai} (comp. of \TClink{tamɔ}, \TClink{laŋgban}). \textit{n} young man. \textit{Taalaŋgbaŋ bul wɔ ka che lɔ, ilel wɔɛ ka cheɛ Kaiŋ Taso ka ko.} There once was a young man named Kain Tasso.

\TCsubword{taapokan} (comp.) \textit{n} [táápókán] boy (K dialect); young boy (\citealt{Pichl1967}). comp. \TClink{tamɔlaŋgbai} (see \TClink{taa}), \TClink{tamɔpokan} (see \TClink{taa})

\TCsubword{tasɔk} (comp.) \textit{n} chick; \textit{taa sɔk} chicken (\citealt{Pichl1967}). 

\TCsubword{tamɔ} (der.) \textit{cf}: \TClink{ajok}, \TClink{tak}. \textit{n} \textbf{1)} boy. \textit{Yààyɛ́ wɔ́ kɛ̀pìɛ́ tààmɔ̀ɛ̀.} The cat scratched the boy – has done it. \textit{Táàmɔ̀ɛ̀ kɔ́nth bààɛ́.} The boy caught the squirrel. \textbf{2)} child. \textit{Bami nhã ya che-lɛ tamɔ.} Lord make that I become your child (\citealt{Pichl1967}). \textit{Ta-m dɛ wɔ mɔ suy o.} My child is in your hands (\citealt{Pichl1967}). \textbf{3)} son. \textit{Wɔ tonki ta wɔ lɛ ræ.} He teaches (shows) his son (how to) write (\citealt{Pichl1967}). \textit{Tipïktipïk lɛ hɔbatokɛ ni ta wɔ lɛ Jisas Kraist hã ka che hwɛlɔ l'ay.} From everlasting, God and his son Jesus Christ were in the world (\citealt{Pichl1967}). 

\TCsubword{tamɔlaŋgbai} (der.), (comp. of \TClink{tamɔ}) \textit{cf}:\TClink{taalaŋgbaŋ} (comp. of \TClink{taa}, \TClink{laŋgban}). \textit{n} young man; \textit{tamɔ langbæ} (wɔ/hã, apuma, pl) young man (\citealt{Pichl1967}). 

\TCsubword{tamɔpokan} (der.), (comp. of \TClink{tamɔ}) \textit{n} boy; \textit{tamɔ pokan} (wɔ/hã, apuma, pl) boy; youngster (\citealt{Pichl1967}). 

\TCsubword[1]{tata} (der.) \textit{cf}: \TClink{piyɛtpiyɛt}, \TClink[2]{pumɔ}, \TClink{tonton} (der. of \TClink[1]{ton}). \textit{adj} \textbf{1)} young. \textit{Anyindɛ kache, ŋɔ pɔ kache ŋa trit a, apima atata ŋa ka bi rɛspɛkt ŋa ayin?} The people in those days, how were they treated; the children, did they have respect for people? \textit{Mɔ le bii fe, bikɔs pɔ yema di Bondo atata.} One should first have money, because one would want to initiate girls very young. \textbf{2)} small. \textit{Itataɛ pɛlɛ ton-tondɛ kɔn lɔ leɛ, amaɛ ŋa bia pɛ buŋ kɔ.} The small one that remains there, it is the women who will thresh it. \textit{A-a, wɔm thi tata bo, kɛ anya yɔl ŋa tha ŋɔth kaɛ kɛ ataims anya tiŋ.} No, they are just small boats, it is four people that fish from them, occasionally two people.

\TCheadword{taalaŋgbaŋ} (comp. of \TClink{taa}, \TClink{laŋgban}, see \TClink{taa}) 

\TCheadword{Taana} \textit{nam} Tana, name given to a person. 

\TCheadword{taapokan} (comp. of \TClink{taa}, \TClink{pokan} (unspec. of \TClink[5]{po}), see \TClink{taa}) 

\TCheadword{tafi} \textit{cf}: \TClink{loli} (der. of \TClink[2]{lol}, \TClink[1]{-i}). \textit{v} \textbf{1)} [táfí] rescue, e.g., at sea (K dialect). \textbf{2)} fish out. \textit{...ni wɔe tafi yen dinthɛ-o.} ...and fishes out the white thing.

\TCheadword[1]{tai} \textit{n} nest; \textit{tæ} (hɔ̃/tha) nest (\citealt{Pichl1967}). 

\TCsubword{taimbɛl} (comp.) \textit{n} palm nut cone; \textit{tæ mbəl} (kɔ/ma, i) cone in which the palm nuts sit (\citealt{Pichl1967}). 

\TCsubword{Taimboŋ} (comp.) \textit{nam} Pleiades constellation; \textit{tæ mboŋ} (hɔ̃/tha) Pleiades (lit. nest of songbirds) (\citealt{Pichl1967}). 

\TCsubword{taive} (comp.) \textit{n} bird nest; \textit{tæ vee} (hɔ̃/tha) bird's nest (\citealt{Pichl1967}). 

\TCheadword[2]{tai} \textit{v} \textit{tæ} fish with rod or line (\citealt{Pichl1967}). \textit{Næ bul hã hõth hɔ̃ lɛ tæ.} One way of fishing is with the rod (\citealt{Pichl1967}). \textit{Lɛn ŋkɔ chok lɛn a yema kɔ tæ.} Go twist a line, I want to go fishing (\citealt{Pichl1967}). 

\TCheadword{taimbɛl} (comp. of \TClink[1]{tai}, \TClink[2]{bɛl}, see \TClink[1]{tai})

\TCheadword{Taimboŋ} (comp. of \TClink[1]{tai}, \TClink[2]{boŋ}, see \TClink[1]{tai}) 

\TCheadword{taive} (comp. of \TClink[1]{tai}, \TClink[1]{vee}, see \TClink[1]{tai}) 

\TCheadword{tak} \textit{cf}: \TClink{ajok}, \TClink{tamɔ} (der. of \TClink{taa}). \textit{n} son. \textit{Tak Bahin yɛ wɔ isi wɔn kɛndɛ oh wɔi lɛ Jizɔs sɛ.} The son of God that we know is only Jesus.

\TCheadword{taks} (Eng \textit{tax}) \textit{nam} Hut Tax. \textit{Təm dɛ kɔ ka chɔni Pəm Taks ɛ, pə ka di Abək agbe̹r abul-abul gbo hã ka saa.} During the time of the Hut Tax War, many Krios were killed, only a few escaped (\citealt{Pichl1967}). 

\TCheadword{tal} \textit{v} be important. \textit{Ngbathïl chen tal.} Trouble has no importance (\citealt{Pichl1967}). 

\TCheadword{tala} \textit{v} depress. \textit{Isin dɛ tala mi.} Poverty depresses me (\citealt{Pichl1967}). 

\TCheadword{tama} \textit{cf}: \TClink[1]{libaŋ} (der. of \TClink[1]{li-}, \TClink[1]{baŋ}), \TClink[2]{yai}. \textit{n} \textbf{1)} [tàmà] laziness (K dialect). \textbf{2)} foolishness (\citealt{Pichl1967}). \textit{Tama ni raŋka ŋɔ mɔɛ.} It is foolishness and a curse upon you.

\TCheadword{tamɔ} (der. of \TClink{taa}) 

\TCheadword{tamɔlaŋgbai} (comp. of \TClink{tamɔ} (der. of \TClink{taa}), \TClink{laŋgban}, see \TClink{taa}) 

\TCheadword{tamɔpokan} (comp. of \TClink{tamɔ} (der. of \TClink{taa}), \TClink{pokan} (unspec. of \TClink[5]{po}), see \TClink{taa}) 

\TCheadword{Tanthbol} \textit{nam} Saturday.

\TCheadword[1]{taŋ} \textit{cf}: \TClink{gbemani}. \textit{v} cry. \textit{Wɔe pɛ po ha taŋ yɛ wɔ bosi mmɛn dɛ.} He began to cry as he was bailing water from the boat. \textit{Bahin himɔ taŋao.} Our father, we cry to you-o.

\TCsubword{taŋhil} (der.) \textit{cf}: \TClink[3]{tiŋ}. \textit{v} cry to; \textit{tə̃ŋhil} cry, complain to (\citealt{Pichl1967}). \textit{La-m dɛ chen vee, chɔli lo, wɔ-m tə̃ŋhil.} My wife is not well, she cried to me the whole of last night (\citealt{Pichl1967}). 

\TCsubword[2]{taŋ} (der.) \textit{n} mourning; \textit{taŋ} (hɔ̃/-) crying, mourning (\citealt{Pichl1967}). \textit{Pə koŋ pəl taŋ bɛɛ lɛ.} They have anounced the mourning for the chief (\citealt{Pichl1967}). 

\TCheadword[2]{taŋ} (der. of \TClink[1]{taŋ}) 

\TCheadword{taŋhil} (der. of \TClink[1]{taŋ}, \TClink{-hil}, see \TClink[1]{taŋ}) 

\TCheadword{taŋka} \textit{n} (kɔ/tha) crab pincer (\citealt{Pichl1967}).

\TCheadword{taro} \textit{cf}: \TClink{ligbem} (unspec. of \TClink{gbem}). \textit{n} descendant. \textit{Laa kuɛe, lanɛ ntaroa hiɛ ni ntaroa mɔɛ, ntaroa ŋaɛ, ŋa bia hundɛ.} That is what I mean, that is our descendant, your descendant, their descendant that is going to come.

\TCheadword{Taso} \textit{nam} Tasso. \textit{Kaiŋ Taso ni waaŋmaaɛ ŋae gbisiŋ.} Kain Tasso and the woman married.

\TCheadword{taso} \textit{n} [tàsó] bird species, does not enter its nest after 6pm because of all the medicine (K dialect); bird lives in mangrove swamps and builds large nests made of sticks (\citealt{Pichl1967}). 

\TCheadword{tasotaso} \textit{n} Poro dancing official; \textit{taso} (wɔ/hã) Poro official, the only person in Poro who dances; official dancer (\citealt{Pichl1967}). 

\TCheadword{tasɔk} (comp. of \TClink{taa}, \TClink{sɔk}, see \TClink{taa}) 

\TCheadword[1]{tata} (der. of \TClink{taa}) 

\TCheadword[2]{tata} \textit{cf}: \TClink[1]{santh}. \textit{n} small white shrimp. \textit{Santh bo̹m-bo̹m dɛ kɔ mən njɛthil l'ay kə santh ta-ta lɛ kɔn dinthɛni kɔ hɛlɛɛ ko.} The big shrimp are found in freshwater but the small and white shrimp are to be found in the sea (\citealt{Pichl1967}). 

\TCheadword[1]{tee} \textit{subordconn} \textbf{1)} up to. \textit{Ka lɔ pɔ dumɔ mi te akoŋ gbako.} I was raised here until I was grown. \textit{Lɔn lɔi le te hi koŋ gbako.} There we stayed until we were grown up. \textit{Wɛl atipɛ tɔn nɛndɛ ŋɔ Apothoɛ ŋa wɔ 2013, te mɛŋko ki amu tɔndai.} Well, I started singing in the year that white people call 2013, up to this year I'm still singing. \textbf{2)} “until.”

\TCheadword[2]{tee} \textit{cf}: \TClink[1]{sun}. \textit{n} [tèè] mortar used to pound rice (K dialect); [ìthìɛ́]/[thìthɛ́] the mortar/the mortars (B dialect); \textit{(i)te} mortar (\citealt{Sumner1921}); \textit{itee} (hɔ̃/ma) mortar (\citealt{Pichl1967}). 

\TCheadword[3]{tee} \textit{Idph} of continuing on and on. \textit{Wanthɛmdɛ ka le blid <te> ni hu.} The woman kept bleeding <te> until she died.

\TCheadword{teen} \textit{Idph} of staring intently (K dialect). \textit{Wɔ̀ lɛ̀lí <téén>.} She observed <teen> (very closely).

\TCheadword{tel} \textit{n} \textbf{1)} [tèl] vine species, rattan, esp. when made of twisted cane fiber, used to cane children (K dialect). \textbf{2)} rope, climbing belt (\citealt{Pichl1967}). \textit{Ntel lo ma ŋkəlɛŋ ha thaŋ ka wa.} This cane rope is good to climb a palm tree with (\citealt{Pichl1967}). comp. \TClink{baŋktel} (see \TClink[2]{baŋk}) 

\TCheadword{telɛ} \textit{v} wait. \textit{Wɔ hi telɛ ka muyu}. He is patiently waiting for us (\citealt{Pichl1967}. \textit{Thumɔɛ lɛ wɔ telɛ vəə lɛ apuma lɛ pɛ ha yema.} The dog is waiting for the scraps; the children, too, want them (\citealt{Pichl1967}). 

\TCheadword{telɔ} (Eng \textit{tailor}) \textit{cf}: \TClink{nɔsɔnthɔ} (comp. of \TClink{nɔ}, \TClink[1]{sɔnth}). \textit{n} tailor. \textit{Wɔ ra ichɛkɛ, wɔ telɔ, kɛ wɔ ra.} He is a farmer, and also a tailor, but he brushes. \textit{Yam bɛ a sini bikɔs a che chal telɔ shɔp pai.} I myself do not know because I do not sit at the tailor shop.

\TCheadword{tem} \textit{cf}: \TClink{gbenik} (der. of \TClink[2]{gbemi}, \TClink{-k}), \TClink{gbɛthɛhɔl}, \TClink{kun}. \textit{n} stomach. \textit{Naa lɛ wɔ bom.} The cow has a big stomach (\citealt{Pichl1967}). 

\TCheadword{temabo} \textit{n} (kɔ/ma) water lettuce (\citealt{Pichl1967}). 

\TCheadword{ten} \textit{n} bird species. 

\TCheadword{tent} \textit{cf}: \TClink[2]{hil}. \textit{n} [tént] anthill (K dialect).

\TCheadword[1]{tenti} \textit{n} [téntí] old wound (K dialect). 

\TCsubword[2]{tɛnti} (der.) \textit{v} accidentally hit or strike another person's wound or sore (\citealt{Pichl1967}).

\TCheadword{teŋ} [tèŋ] (Eng \textit{tang}) \textit{cf}: \TClink{ŋaiŋai}. \textit{adj} \textbf{1)} sour (K dialect). \textit{Sànthóŋ kɔ́ tèŋ.} A bush (used like Maggi for flavor) is sour. \textit{Ŋgbèmàŋ mpùm ma teŋ.} Some fruits are sour. \textbf{2)} sweet. \textit{Lembe lo kɔ təng chaŋ kɔnɛ chencha.} This orange is sweeter than that of yesterday (\citealt{Pichl1967}). 

\TCheadword{teŋka} \textit{cf}: \TClink{patikulali}. \textit{adj} particular.

\TCsubword[1]{teŋkateŋka} (der.) \textit{adj} \textbf{1)} important. \textit{Lɛli lɛ kɔ te̹nka-te̹nka pɔk bɛ Bo̹lo̹m dɛ.} The post-mortem is very important in the country of the Bolom chielfs (\citealt{Pichl1967}). \textbf{2)} particular. \textit{Pɛnthe-m dɛ wom ajok ko-m ka chencha hã hom mi jali te̹nkate̹nka.} My brother sent his son here to me yesterday to tell me something particular (\citealt{Pichl1967}).

\TCsubword[2]{teŋkateŋka} (der.) \textit{adv} actually. \textit{I yema ni wun ko ja tɔntho, la ivelɛmɔ teŋga-teŋgaɛ.} We want to now come to the singing aspect that we actually called you for.

\TCheadword[1]{teŋkateŋka} (der. of \TClink{teŋka}) 

\TCheadword[2]{teŋkateŋka} (der. of \TClink{teŋka}) 

\TCheadword{teŋkeli} \textit{v} be finicky; \textit{te̹nkeli} be disinclined to eat any kind of food without meat or fish (\citealt{Pichl1967}). 

\TCheadword{tep} (Eng \textit{(cassette) tape}) \textit{n} cassette player. \textit{Abibo tep, akɔ ŋɔ hok a ple.} If I have a tape, I take it out and play (it).

\TCheadword{ter} \textit{cf}: \TClink{ayeŋ}. \textit{n} [terɛ] the waist (K dialect). \textit{Yendɛ hɔ bi ni che tə ton vɛ lɛ hɔ ki.} The reason he has a small waist is this (\citealt{Sumner1921}). 

\TCheadword{Tetima} \textit{nam} Tetima, name given to a place. 

\TCheadword{tɛbul} (Eng \textit{table}) \textit{cf}: \TClink{mɛsa}. \textit{n} table; \textit{tɛbul} (hɔ̃/tha) table (ex Engl) (\citealt{Pichl1967}). \textit{Yɛ mɔ koŋ thɔk boithɛ gbi ni sɛiyɛ, mɔi bɛ tebullɛ atok.} After washing the dishes and the spoon, then you put it on the table.

\TCheadword{tɛhil} \textit{n} sweetness.

\TCheadword{tɛl} \textit{v} \textbf{1)} connect; \textit{tɛl} join (\citealt{Sumner1921}). \textit{Thinæ tha tɛlɛ trïthi hĩ lɛ.} Roads connect our towns (\citealt{Pichl1967}). \textbf{2)} join (\citealt{Sumner1921}).

\TCsubword{tɛlni} (der.) \textit{v} \textbf{1)} have in common (\citealt{Pichl1967}). \textbf{2)} be joined.

\TCheadword{tɛlni} (der. of \TClink{tɛl}, \TClink{-ni}, see \TClink{tɛl}) 

\TCheadword[1]{tɛm} (Eng \textit{time}) \textit{cf}: \TClink{bonk}, \TClink{lɔkɔ}, \TClink[1]{mɛŋk}. \textit{n} time. \textit{Ŋɔn ŋɔ biɛni standad taim.} It does not have a standard time. comp. \TClink[1]{tɛmpum} (see \TClink[2]{pum}), \TClink[2]{tɛmpum} (see \TClink[2]{pum}) 

\TCsubword{tɛmgbi} (comp.) \textit{temp} all the time, everytime (\citealt{Pichl1967}). 

\TCsubword[1]{tɛmotɛm} (der.) \textit{temp} \textbf{1)} every time, distributive (\citealt{Pichl1967}). \textbf{2)} anytime. \textit{Sɛkɛnɔ we, so Abatokɛ yemɔ gbo, tɛm-o-tem ŋɔ inɔ pɛ bia yema, iŋa ni ŋa shi la.} Thank you, so if God agrees, anytime we want you, we would let you know that. \textbf{3)} all the time. \textit{Nkeni ko mɔ lɔ che kɔ tɛm-o-tɛm?} Makeni, do you go there all the time?

\TCsubword[2]{tɛmotɛm} (der.) \textit{disco} once upon a time. \textit{Tɛn-o-tɛn, tɛn po mbawom o.} Once upon a time there was a fable, the fable rose from the ancestors (the introduction to tales) (\citealt{Pichl1967}). 

\TCheadword[2]{tɛm} \textit{interrog} when. \textit{Tɛm ndɔ ŋɔ ntipɛ gbemia?} When did you start delivering?

\TCheadword[3]{tɛm} \textit{v} bump. \textit{Lɛ ŋ kɔ gbo binthi sɔksi l'ay, n tuntni mma ki təm bo̹l mɔ.} If you go into the fowl coop, bend your head or you will bump your head (\citealt{Pichl1967}). 

\TCheadword{tɛmɛ} \textit{cf}: \TClink{tɔthian}. \textit{v} strive; struggle (\citealt{Sumner1921}). 

\TCsubword{tɛmɛn} (der.) \textit{v} strive; struggle (\citealt{Sumner1921}).

\TCsubword{tɛmɛni} (der.) \textit{v} strive. \textit{Ya kɔ tɛmɛni gbath lo hɔ̃ kath.} I go to strive for myself, the times are hard (\citealt{Pichl1967}). 

\TCsubword{tɛmɛtɛmɛ} (der.) \textit{v} struggle. \textit{Ni ŋa tɛmɛ-tɛmɛ haŋ ni wuthi wɔ ni ŋa woth wɔ ŋa yɔk wɔ kilɛ wɔ ko.} They struggled to untie him and took him to his house. 

\TCheadword{tɛmɛn} (der. of \TClink{tɛmɛ}, \TClink[2]{-n}, see \TClink{tɛmɛ}) 

\TCheadword{tɛmɛni} (der. of \TClink{tɛmɛ}, \TClink{-ni}, see \TClink{tɛmɛ}) 

\TCheadword{tɛmɛtɛmɛ} (der. of \TClink{tɛmɛ}) 

\TCheadword{tɛmgbi} (comp. of \TClink[1]{tɛm}, \TClink[3]{gbi}, see \TClink[1]{tɛm}) 

\TCheadword{tɛmi} \textit{v} bite, hook; tɛɛmi bite the hook (fish) (\citealt{Pichl1967}). \textit{Ya tɛɛmiɛ gbokbo kə kɔni, minɛn-na bɛt lɛ yenkəlɛŋ.} A gbokbo has bitten on my hood but it has gone, it didn't swallow the bait very well (\citealt{Pichl1967}). 

\TCheadword[1]{tɛmotɛm} (der. of \TClink[1]{tɛm}, \TClink{-o-}, see \TClink[1]{tɛm}) 

\TCheadword[2]{tɛmotɛm} (der. of \TClink[1]{tɛm}, \TClink{-o-}, see \TClink[1]{tɛm}) 

\TCheadword[1]{tɛmpum} (comp. of \TClink[1]{tɛm}, \TClink[2]{pum}, see \TClink[2]{pum}) 

\TCheadword[2]{tɛmpum} (comp. of \TClink[1]{tɛm}, \TClink[2]{pum}, see \TClink[2]{pum}) 

\TCheadword[1]{tɛn} \textit{n} \textbf{1)} sense; \textit{nten} sense (\citealt{Sumner1921}). \textit{Pim nɔ wɔ sɔtha nten Inglan la atheliɛ komɔko.} Maybe someone in England will understand what I said to you. \textbf{2)} mind. \textit{Kɛn bo bi ŋsɔkba la mɔ tenɛ, ha mɔn wɔ...} If you have a problem in mind and you want to talk... \textit{Nɔonɔ nten ma wɔɛ ma gbo ko feɛ mesaɛ atok.} Everyone in the court bari focused their minds on the money on the table. \textbf{3)} understanding. \textit{Nɔ shini che ko labi yendɛ yɛ mɔ la ŋa ncheyi ni nshila thiyen, ni la saŋ mɔ ntenɛ.} One does not know the future that is why when doing something you should ask so you can know it and understand it better. \textit{Nɔthiɛ nthɛkɛsiɛ wɔ ni san la ntenɛ.} Human beings clarify in order to understand things. \textbf{4)} cleverness; intelligence. \textit{Tamɔ le wɔ nthïn.} The boy is clever (\citealt{Pichl1967}). \textit{Ba ləm wɔ nthïn chaŋ nvis lɛ gbi tho ɛ ko.} The rabbit is the most clever of all the animals in the bush (\citealt{Pichl1967}). \textbf{5)} judgment. \textit{San dɛ koŋ lo nthïn.} Proverb: The otter has delivered the judgment (\citealt{Pichl1967}). \textbf{6)} wisdom; \textit{nthïn} (ma) wisdom (\citealt{Pichl1967}). 

\TCsubword[2]{tɛn} (der.) \textit{cf}: \TClink{lomani}, \TClink{lonibolɛ}, \TClink{mɛmba}. \textit{v} remember. \textit{Siŋthɛ vɛ tha nlonigbo ntenɛ lɛ nkache siŋ?} Those are the only games you remembered that you used to play? der. \TClink[2]{tɛni} (see \TClink[1]{tɛn}), \TClink{tɛnin} (see \TClink[1]{tɛn}), \TClink{tɛnini} (see \TClink[1]{tɛn})

\TCsubword[1]{tɛni} (der.) \textit{n} thought. \textit{Tɛni-m bul hɔ̃-m bo̹l lɛ, la hĩ gbo thiyeŋ yin ni Hɔbatokɛ.} One thought is only in my head. It is between us only, me [we?] and God (\citealt{Pichl1967}). der. \TClink{tɛŋkɛn} (see \TClink[1]{tɛn})

\TCsubword[2]{tɛni} (der.), (der. of \TClink[2]{tɛn}) \textit{v} remember. \textit{Nɔmaa chaɛ a: Yemi, ni ntɛniɛ mini o-o-o.} The woman sang: My lady, remember me. \textit{Amaaɛ ŋae yom: Yemi, ni ntɛniɛ mini-o, ni ntɛniɛ mini-o, ni ntɛniɛ mini-o.} The women answered: My lady, do not forget me now, and do not forget me now, and do not forget me now. der. \TClink{tɛnin} (see \TClink[1]{tɛn}), \TClink{tɛnini} (see \TClink[1]{tɛn})

\TCsubword{tɛnin} (der.), (der. of \TClink[2]{tɛni}) \textit{v} think. \textit{La mɔ tɛniɛn wɔiyowɔ ɛ.} What you are thinking everyday. \textit{Lɛ a chala si a tɛnin ya ke nɔ bɛma min.} When I used to sit down and think, I saw someone who would help me. 

\TCsubword{tɛnini} (der.), (der. of \TClink[2]{tɛni}) \textit{v} \textbf{1)} think. \textit{Tamɔ lɛ wɔ gbo hã len lifǐk chen tɛnini.} The boy does things only at random, he doesn't think (\citealt{Pichl1967}). \textbf{2)} remember. \textit{Tɛnɛni.} Remember (title of a hymn). \textit{A che bo pɛ chɛnɛni temdɛ ŋɔ huɛ.} I will not just remember the time he died.

\TCsubword{tɛŋkɛn} (der.), (der. of \TClink[1]{tɛni}) \textit{n} suspicion. \textit{Lanɛ la pə hɔmɔ mɔ lɛ hã yaŋ, la chen roŋ, ntɛnkɛn ma gbo ve.} What they told you about me is not true, it is only a suspicion (\citealt{Pichl1967}). 

\TCheadword[2]{tɛn} (der. of \TClink[1]{tɛn})

\TCheadword{tɛnɛn} \textit{v} think. \textit{Wɛl, yendɛ ŋɔ atɛnɛndɛ lɛllɛ lɔ thollɛ...} Well the reason I think that the ground is sinking... \textit{So lanɛ la yaŋ atɛnɛ, bikɔ mɛŋkɛ ŋɔ tha ka cheni wun kaɛ.} So that is what I think, because at the time they were not here.

\TCheadword[1]{tɛni} (der. of \TClink[1]{tɛn}, \TClink[1]{-i}, see \TClink[1]{tɛn})

\TCheadword[2]{tɛni} (der. of \TClink[2]{tɛn} (der. of \TClink[1]{tɛn}), \TClink[1]{-i}, see \TClink[1]{tɛn})

\TCheadword{tɛnin} (der. of \TClink[2]{tɛni} (der. of \TClink[2]{tɛn}, \TClink[1]{-i}), \TClink[2]{-n}, see \TClink[1]{tɛn}) 

\TCheadword{tɛnini} (der. of \TClink[2]{tɛni} (der. of \TClink[2]{tɛn}, \TClink[1]{-i}), \TClink{-ni}, see \TClink[1]{tɛn}) 

\TCheadword{tɛnis} (Eng \textit{tennis}) \textit{n} tennis. \textit{Wɛl i ka che ple han tɛnis bɔl.} We used to play hand tennis ball.

\TCheadword{tɛnt} \textit{cf}: \TClink[5]{ken}. \textit{Loc} nearby. \textit{La bi a bɔɔni mɔm tɛntɛ.} That is what makes me draw closer to you.

\TCsubword[1]{ntɛnt} (der.) \textit{v} be near. \textit{Kɛ kpɔnko hɔ ka che trï ko ntɛnt, hɔ nɔonɔ ka chen kɔ ai ɛ.} But there was a forest near the town, which no one entered (\citealt{Pichl1967}).

\TCsubword[2]{ntɛnt} (der.) \textit{adp} near. \textit{Bàíyɛ́ ŋɔ́ kìllɛ́ ntɛ̀nt.} The bari is near the house.

\TCheadword{tɛnthe} \textit{cf}: \TClink{thak}. \textit{n} split cane stick. \textit{Pɔ kɔ yuk ka tɛnthe.} They plant it with a split cane stick.

\TCheadword{tɛnthil} \textit{v} \textbf{1)} awaken. \textbf{2)} wake up.

\TCheadword[2]{tɛnti} (der. of \TClink[1]{tenti}) 

\TCheadword{tɛŋka} \textit{cf}: \TClink[2]{pum}. \textit{adv} maybe. \textit{Yɛlai bikɔs hin pɛ tɛŋga apima hinyɛ ha bia che hun gbɛ.} That is it, because again maybe our children will come visit. \textit{Tɛŋka tɛm lan ncheni pɛ wɔɛ, nko wu.} Maybe by that time you are not alive, you are dead.

\TCheadword{tɛŋkɛ} [tɛ̀nkɛ̀] \textit{cf}: \TClink{bɛnthɛ}, \TClink{yo}. \textit{n} \textbf{1)} bird-driving platform, [tɛ̀nkɛ̀], [thìtɛ̀nkɛ̀ɛ́] farm platform for driving birds (K dialect); \textit{tənkɛ} (kɔ/ma?) platform on the farm where children sit and drive the birds or monkeys away (\citealt{Pichl1967}). \textbf{2)} scaffold (\citealt{Sumner1921}). \textit{Thitənkə tha yi sɛmi ichɛk ay...} The scaffolds which we erect in a farm... (\citealt{Pichl1967}). 

\TCheadword{tɛŋkɛn} (der. of \TClink[1]{tɛni} (der. of \TClink[1]{tɛn}, \TClink[1]{-i}), see \TClink[1]{tɛn}) 

\TCheadword{tɛtɛk} \textit{cf}: \TClink{saŋpɛlɛ} (comp. of \TClink{saŋ}, \TClink{pɛlɛ}). \textit{n} (kɔ/-) young, not quite full rice (\citealt{Pichl1967}).

\TCheadword{tii} \textit{cf}: \TClink{kok}. \textit{n} \textbf{1)} base, e.g., of cotton tree (kapok) distinct from \textit{kok} ‘buttress of cotton tree' (K dialect). \textit{A-a, iche ma jo, ibo kɔ sɛmi thɔkɛ ti.} No, we would not eat it, we just take it to the base of the trees. \textit{Nchiɛ mi pəpə lɛ hɔ̃ ko ti lɛ.} Bring me the calabash that is at the foot (of the tree) (\citealt{Pichl1967}). \textbf{2)} home. \textit{Yɛ ya wokɔ tikomiko a kɔni yena livil we...} If I travel from home to anywhere... \textbf{3)} village. \textit{Akon gbo pɔkɔni tiɛ lɔ pɔ gbem wɔ.} I've just forgotten the village (where) he was born. \textbf{4)} town. \textit{Amaɛ ŋai hun, ŋa kɔ woth thi bolɛ, ŋa yɔk kebelthai ɔ tithai.} The women will come and carry it on their heads and take it to farm houses or towns. \textit{Pɔ koŋ gbo, ŋa koŋ kɔ gbo yɔk ti thai, pɔ kɔ pak bai thikranthikran thibombom.} After taking it to the farmhouses/towns, it would then be piled up into different sections into very big piles.

\TCsubword{tiko} (der.) \textit{Loc} in this town. \textit{Komɔ landɛ ko bɛ hani gbako, wɔ tika.} That child is now grown, she is in this town. \textit{Wa maɛ, wɔ tika, Mɔmi Prat ki wante wɔi.} A girl, she is in this town, Mummy Pratt's sister. \textit{Aa, wɔnbɛ wɔɔ nyoroko, tiko bami, ha ha le kilɛ wɔl ko.} Yes, She herself is in Nyoro, my father's village, they are the ones she left in the house. \textit{Mɔni gbo kɔ kɛkɛ tiko.} You are going to the village too early. comp. \TClink{kakitiki} (see \TClink[2]{ka}) 

\TCheadword{tiiŋni} \textit{v} \textbf{1)} [tííŋní] enter coma (K dialect). \textbf{2)} faint (\citealt{Pichl1967}). \textit{Nak lo kɔ kath, Kɔŋ wɔ gbo tini.} This illness is serious, Kong faints constantly (\citealt{Pichl1967}). 

\TCheadword[1]{tik} \textit{v} \textbf{1)} land. \textit{Bot lɛ koŋ tik bondo ko, ha mɔ telɛ han wunkiɛ.} The boat has landed; they are awaiting you to weigh anchor (\citealt{Pichl1967}). \textit{Bot lɛ koŋ tik, ha lɔ bondɔ ko.} The boat has landed, they are at the wharf (\citealt{Pichl1967}). \textbf{2)} reach shore. 

\TCheadword[2]{tik} \textit{n} antelope.

\TCheadword{tike} \textit{cf}: \TClink{gbala}. \textit{n} log; \textit{tike} (hɔ̃, i) long piece of wood, too large to be cut at the farm or to be carried on the head by women so that it has to be carried on the soulder by men (\citealt{Pichl1967}).

\TCheadword{tikɛtil} (Eng \textit{tea kettle}) \textit{n} tea kettle. \textit{Rïm dɛ kɔ hok tii-kɛtïl l'ay.} The steam comes out of the tea kettle (\citealt{Pichl1967}).

\TCheadword{tiko} (der. of \TClink{tii}, \TClink[1]{ko}, see \TClink{tii})

\TCheadword{tilaŋ} \textit{cf}: \TClink{halɛ}, \TClink[2]{pɛ}, \TClink[1]{pika} (der. of \TClink[2]{pika}), \TClink[2]{pim}. \textit{adj} \textbf{1)} other. \textit{Ŋɔ tipɛ kaŋdɛ, mpanth handɔ ma ŋaɛ, pambondɛ gbisiŋɛ, ni ja li tilan gbi.} How she started learning, what work is she doing, if she is married and other things. \textit{Pɔ yuk mansaŋhaɛ nseen si pɔ wɔm bɛ kutha pɛlɛɛ ni nyiki ntilaŋ.} They plant this egusi together with it first, before they plant rice or any other seeds. \textbf{2)} another. \textit{Lɛ nɔ yema gbo hink trï bul ay hã kɔ trï tilaŋ ay.} If somebody wants to go from one town to another town (\citealt{Pichl1967}). comp. \TClink{Ketilaŋ} (see \TClink{ke}) 

\TCheadword{tilɛni} (der. of \TClink{-ni}) 

\TCheadword{timitimi} \textbf{1)} \textit{v} weak. \textbf{2)} \textit{n} weakness.

\TCheadword{timp} \textit{n} (kɔ/ma) high cliff (\citealt{Pichl1967}). \textit{Timp lɛ Gbaŋgbaya ko ntɛnt kɔ tokɛ.} The cliff near Gbaŋgbaya is high (\citealt{Pichl1967}). 

\TCheadword{Timpla} \textit{nam} Timpla, name given to a place. 

\TCheadword{timpla} \textit{n} riverside.

\TCheadword[1]{tin} [t̪s̪ə̀n] \textit{cf}: \TClink[3]{tu}. \textit{Numb} two (B dialect). \textit{A cha fe pɔŋ tɪŋ ya hɔ munk gbəŋ.} I borrowed two and I will return them tomorrow (\citealt{Pichl1967}). \textit{Mpang nwang ni tïng man ma nɛn bul ay ɛ.} There are twelve months in one year (\citealt{Pichl1967}). comp. \TClink{mɛntiŋ} (see \TClink[1]{mɛn}), \TClink{waŋnitiŋ} (see \TClink[2]{waŋ}) 

\TCsubword{tintatu} (comp.) \textit{nam} “tin tan two,” name given to a game. \textit{Chaŋ gbo siŋthɛ tha ika che siŋdɛ, thi siŋ thalɔ pɔ tha velɛ tintatu ɛ.} Just the games that we used to play, one is called \textit{tin tan two.}
 
\TCsubword[2]{tin} (der.) [t̪s̪ə̀n] \textit{cf}: \TClink{sɛkɔn}. \textit{adj} second. \textit{Mɔikɛ tindɛ, mii gbemɛni komɔ pokan, i gbo ama.} The second thing is mother did not have male children; we are just females.

\TCheadword[2]{tin} (der. of \TClink[1]{tin}) 

\TCheadword{tintatu} (comp. of \TClink[1]{tin}, \TClink[3]{tu}, see \TClink[1]{tin}) 

\TCheadword[1]{tintin} (der. of \TClink[2]{tintin}) 

\TCheadword[2]{tintin} \textit{cf}: \TClink[1]{rɔŋ}. \textit{n} [tìntìn] truth (K dialect). 

\TCsubword[1]{tintin} (der.) \textit{cf}: \TClink[2]{ayɛn}. \textit{adj} \textbf{1)} obedient; \textit{tïntïn} straight, obedient (\citealt{Pichl1967}). \textit{Boɛ, waŋ mɔ lo chen tïntïn, koŋ bɛ yenwɛy, ŋ kɔ wɔ yi.} Boe, this your daughter is not straight. She has gone bad, go ask her (\citealt{Pichl1967}). \textbf{2)} direct straightforward. \textit{Bahin wɛ, Wɔ lɔ naiyɛ tiŋtiŋdɛ.} Our Father says He is the direct way. \textit{Yɛ mɔ theli wɔk ni nɔɛ kɔ ke sampullɛ wɔi si kɛ nɔɛ ki wɔ tintin, n thambas ɛ.} When you say something, let the person see the sample, then the person knows that this person is straightforward.

\TCheadword[1]{tiŋ} [t̪s̪ə́n] \textit{n} \textbf{1)} monkey (B dialect). \textbf{2)} \textit{ting} (wɔ/hã, si) chimpanzee (\citealt{Pichl1967}). \textbf{3)} [tə́ŋ] baboon (According to Ba Yanker there is a minimal pair but the vowel is different: tə́ŋ ‘baboon' (not ‘monkey') vs. tə̀ŋ ‘two' [vowel close to [ɪ]]) (K dialect).

\TCheadword[2]{tiŋ} \textit{cf}: \TClink{pen}. \textit{Idph} \textbf{1)} of holding fast. \textbf{2)} of tightness. \textit{\`{m}bìsì <tíŋ>!} Hold on <tiŋ> (tight)!

\TCheadword[3]{tiŋ} \textit{cf}: \TClink{taŋhil} (der. of \TClink[1]{taŋ}, \TClink{-hil}). \textit{n} \textbf{1)} nonsense. \textit{Yikiɛ ŋɔ iyema, ilap labila iyemani tiŋ.} It is our respect that we want; we are shy, that is why we do not want nonsense. \textbf{2)} noise. \textit{Woŋgomi ko ma lɔ kɔ nche lɔ bɔnth chiŋ, bikɔs yaŋ pɛ ayemani tiŋ.} In my house if you go there you will not hear any noise, because myself I do not want noise.

\TCheadword{tiŋkɔ} \textit{cf}: \TClink{yok}. \textit{n} coral species, kind of red coral that makes the most expensive beads (\citealt{Pichl1967}). 

\TCheadword{tipɛ} \textit{cf}: \TClink[6]{po}. \textit{v} \textbf{1)} begin. \textit{Hã tipɛ sɔthɔ bali hĩ ɛ.} They began to amass (lit. to get) wealth (\citealt{Pichl1967}). \textit{Langban dɛ tipɛ bɛmpa aye̹n hã kaŋ hã.} The man began to make them a place to teach them (\citealt{Pichl1967}). \textbf{2)} start. \textit{A tipɛ gbemi 1954.} I started delivering 1954. \textit{Ko lɔ pɔ tipɛ haŋ ko lɔ pɔ ko mɛkniɛ.} Where they start until the end.

\TCsubword{tipɛni} (der.) \textit{v} begin. \textit{Tipɛni fisa.} He begins to be (or: to feel) better (\citealt{Pichl1967}).

\TCsubword{tipik} (der.) \textit{n} beginning. \textit{Tipik lɛ ye ha bɔnthɛ, ha ka silan lɛ ha	bi ha kantha kil lɛ si mənk lɛ koŋhoni.} At the beginning when they met up, they did not know that they had to close up the house before the time ran out (\citealt{Pichl1967}). der. \TClink{tipiktipik} (see \TClink{tipɛ})

\TCsubword{tipiktipik} (der.), (der. of \TClink{tipik}) \textit{temp} since forever. \textit{Tipïktipïk lɛ Hɔbatokɛ ni ta wɔ lɛ Jisas Kraist hã ka che kwɛlɔ l'ay.} From everlasting, God and his son Jesus Christ were in the world (\citealt{Pichl1967}). 

\TCheadword{tipɛni} (der. of \TClink{tipɛ}, \TClink{-ni}, see \TClink{tipɛ}) 

\TCheadword{tipik} (der. of \TClink{tipɛ}, \TClink{-k}, see \TClink{tipɛ})

\TCheadword{tipiktipik} (der. of \TClink{tipik} (der. of \TClink{tipɛ}, \TClink{-k}), see \TClink{tipɛ}) 

\TCheadword[1]{tis} \textit{n} resin; \textit{ntis} (ma) resin (\citealt{Pichl1967}). comp. \TClink{tismabue} (see \TClink{boe}) 

\TCheadword[2]{tis} \textit{adj} drunk, (not Krio, good Bolom word) (B dialect). \textit{Wɔ bi tis.} He is drunk. 

\TCheadword{Tisana} \textit{nam} Tissana, name given to a place. \textit{Yɛs, bullɛ wɔ Tisana ko.} Yes, the one is at Tissana.

\TCheadword{tismabue} (comp. of \TClink[1]{tis}, \TClink{n-}, \TClink{boe}, see \TClink{boe}) 

\TCheadword{tith} \textit{adj} thick. \textit{Be̹th lo hɔ̃ tith.} This plank is thick (\citealt{Pichl1967}). 

\TCheadword[1]{to} \textit{n} \textbf{1)} [tó] tree species, large tree (K dialect); tree species, large tree found on coasts with edible fruit like coffee berries (\citealt{Pichl1967}). \textbf{2)} fig nut. comp. \TClink{togba} (see \TClink[1]{gbɔs}) 

\TCheadword[2]{to} \textit{v} \textbf{1)} climb. \textit{Ŋkɔ too waa lɛ ni ŋkɔɔ mbəl lɛ!} Go climb up the palm tree and cut the nuts! (\citealt{Pichl1967}). \textbf{2)} mount. Ŋ kɔ too waa lɛ ni ŋkɔ mbəl lɛ! Go up the palm tree and cut the nuts! (\citealt{Pichl1967}). 

\TCheadword[3]{to} \textit{n} [tó] place where animals meet (K dialect). 

\TCheadword{tobae} \textit{cf}: \TClink{mba}. \textit{n} comrade; \textit{tobæ} (wɔ/hã) comrade, equal (one male to another) (\citealt{Pichl1967}). 

\TCheadword{toɛ} \textit{v} \textbf{1)} \textit{tuɛi} put on clothes (\citealt{Sumner1921}). \textit{Ŋ kɔ toɛɛ kumba mɔ lɛ ni yi kɔ bondɔ ko.} Go put on your gown and let us go down to the wharf (\citealt{Pichl1967}). \textbf{2)} get dressed. \textit{Ŋ ka mi yen-o-yen hɔ̃ m bɔ lɛ ni a kɔ toɛ.} Give me anything you can that I go and (may) dress (\citealt{Pichl1967}). \textit{Wantɛm do koŋ gbo toɛɛ, wɔ yɛ hɔni street lɛ ibɔl hã kɔ thonkini.} This young woman, when she has just dressed up, she goes out in the street to show herself (\citealt{Pichl1967}). \textbf{3)} wear. \textit{Ki ŋɔa tɔioɛ, achen pɛ pin.} This that I am wearing, I will not buy it again.

\TCsubword{toɛya} (der.) \textit{n} clothes; \textit{toɛɛ-ya} (hɔ̃/tha) any kind of dress for males and females (\citealt{Pichl1967}). \textit{Toɛɛ-ya wɔ lɛ kɔ gbe̹r kə pəə thibeth.} He has many clothes, they fill boxes (\citealt{Pichl1967}). \textit{Ya mɔ kamɔ nje, ya mɔ tɔyɛ mɔ.} I give you food, I give you clothes (\citealt{Pichl1967}).

\TCheadword{toɛya} (der. of \TClink{toɛ}) 

\TCheadword{togba} (comp. of \TClink[1]{to}, \TClink[1]{gbɔs}, see \TClink[1]{gbɔs}) 

\TCheadword[1]{tok} \textit{cf}: \TClink[1]{lɛli}, \TClink[1]{pɛmplɛ}. \textit{v} [tók] look, watch, involves observing from afar with no real commitment, just to see what is happening (observe, check out), while \textit{lɛli} ‘look at' is really to look at, better for a pot and a football match where the attention is more focused (K dialect). \textit{Hã tokɔ tɔ lɛ thanthɛn.} In vain they watched the grave (\citealt{Pichl1967}). \textit{Taa jobɔy ntok ni nsɛli.} Weak child, watch and pray (\citealt{Pichl1967}). 

\TCsubword{Hɔbatokɛ} (der.), (comp. of \TClink[1]{tokɛ}) \textit{nam} God, lit. the voice of the Lord (or Father) in the sky (\citealt{Pichl1967}). \textit{À chɔŋɔɔ Hòbátòkɛ̀ sɛ̀kɛ́.} I give thanks to God. \textit{So sɛkɛ we, Abatokɛ chema mɔni.} So thank you, may God be with you.

\TCsubword{Tokɛ} (der.) \textit{nam} Tokeh, name given to a place; \textit{tokee} (kɔ/-) watching-place (also place-name) (\citealt{Pichl1967}).

\TCsubword[1]{tokɛ} (der.) \textit{Loc} \textbf{1)} at the top. \textit{Mbàŋsɛ̀ ŋà rɪ́k wàɛ̀ tòkɛ̀.} The weaver birds wove (their nests) at the top of the palm tree. \textbf{2)} above. \textit{Yɛ Bɛl Maaɛ koŋ thaŋni boeɛ tokɛ hiŋk wul-lɛ lɔ bin wɔɛ...} When Rat Wife had climbed above the kitchen (away) from where death had missed her... \textit{Bɛlsɛ ŋa lɔ baiɛ tokɛ.} The rats are there on top of the bari. \textit{Bɛl Maaɛ wɔe tipɛ mir-mir, wɔ mukumuku ton, ton, tokɛ ko.} Rat Wife began to watch intently, she crept little by little from above. comp. \TClink{hɔbatokɛ} (see \TClink[1]{tok}), \TClink{wɔmtokɛ} (see \TClink[2]{wɔm}), der. \TClink{tokɛtokɛ} (see \TClink[1]{tok}), id. \TClink{wɔmtokɛ} (see \TClink[2]{wɔm})

\TCsubword[2]{tokɛ} (der.) \textit{n} watching place; \textit{tokee} (kɔ/-) watching-place (also place-name) (\citealt{Pichl1967}).

\TCsubword[3]{tokɛ} (der.) \textit{adv} loudly. \textit{Ŋ hɔ tokɛɛ chaŋ vɛ ni ya the la mɔ hɔ lɛ.} Speak louder than that and let me hear what you say (\citealt{Pichl1967}). 

\TCsubword{tokɛtokɛ} (der.), (der. of \TClink[1]{tokɛ}) \textit{v} be high-up. Rɔŋ dɛ tokɛ-tokɛ. the mountain is very high (\citealt{Pichl1967}). 

\TCheadword[2]{tok} \textit{cf}: \TClink[2]{pɛn}. \textit{n} thunder (K dialect). \textit{To̹k lɛ kɔ pɛn parɛ hwɛ.} The thunder cracked the other day (\citealt{Pichl1967}). 

\TCheadword{toka} \textit{n} (kɔ/ma) iron rattles tied around the legs as, e.g., the \textit{kɔysunɔ} has while dancing (\citealt{Pichl1967}). 

\TCheadword{Tokɛ} (der. of \TClink[1]{tok}, \TClink[1]{ɛ}, see \TClink[1]{tok}) 

\TCheadword[1]{tokɛ} (der. of \TClink[1]{tok}, \TClink[1]{ɛ}, see \TClink[1]{tok})

\TCheadword[2]{tokɛ} (der. of \TClink[1]{tok}, \TClink[1]{ɛ}, see \TClink[1]{tok}) 

\TCheadword[3]{tokɛ} (der. of \TClink[1]{tok}, \TClink[1]{ɛ}, see \TClink[1]{tok}) 

\TCheadword{tokɛtokɛ} (der. of \TClink[1]{tokɛ} (der. of \TClink[1]{tok}, \TClink[1]{ɛ}), see \TClink[1]{tok}) 

\TCheadword{tokoth} \textit{cf}: \TClink{gboso}, \TClink{hakla}, \TClink{sayom}. \textit{n} [tókóth] snare for small birds and animals (K dialect); \textit{to̹kot} (kɔ/ma) trap (small tree bent down on the end of which a sling is fixed) (\citealt{Pichl1967}). 

\TCheadword{tokpɛn} \textit{n} [tòkpɛ́n] tree species just like cocoa leaf, in same family, has some red and black dots on leaves (K dialect). 

\TCheadword[1]{tol} \textit{v} \textbf{1)} assault a woman; \textit{tool} assault a woman, to make an indecent attack but not raping (\citealt{Pichl1967}). \textbf{2)} play tricks. \textit{Ntolɛ, i pɔŋ hukɛ. Ihukɛ ŋɔi pɔŋɛ, aji.} You used tricks, we threw hooks. It is the hooks that we throw, (and) we caught (fish)! comp. \TClink{nɔtolɔ} (see \TClink{nɔ}) 

\TCheadword[2]{tol} \textit{n} (wɔ/hã, N) fish species, gwangwa (Corvina nigrita) (\citealt{Pichl1967}). 

\TCheadword{tom} \textit{cf}: \TClink{gbogbotok} (unspec. of \TClink[3]{gbogbo}), \TClink{kɔm}, \TClink{maima}, \TClink[2]{wo}. \textit{n} (kɔ/tha) vagina (\citealt{Pichl1967}). 

\TCheadword[1]{Toma} \textit{nam} Toma Society, mixed society that accepts men and women (\citealt{Pichl1967}, \citealt{Hall1938}). \textit{Pə ka gbetha wɔ ifɔŋ Toma lɛ.} They swore her on the Toma medicine (\citealt{Pichl1967}). 

\TCheadword[2]{toma} \textit{n} (wɔ/hã, si) chameleon (\citealt{Pichl1967}). 

\TCheadword[3]{toma} [tómà] \textit{n} rice variety, upland variety, light brown in color (K dialect).

\TCsubword{alitoma} (der.) \textit{n} rice variety, the name \textit{Ali} with the word \textit{tómà}, upland variety, light brown in color (K dialect). 

\TCheadword{tombia} \textit{n} [tómbíá] tree species, fruit used for medicine to treat cough, suck on seeds (K dialect). 

\TCheadword{Tombo} [thɔmbɔ] \textit{nam} Tombo, name given to a place, from \textit{thombok} ‘beg' because a lot of food is grown there people come in boats to beg (B dialect). \textit{Kɛ pɔ chelɔ pɛ theli Mbolom ken Bonthiko, Thomboko, inal pimdɛ.} But they no longer speak Bolom there like in Bonthe, Tombo, and other places. 

\TCheadword{tombo} \textit{cf}: \TClink{gbundɛ}, \TClink{sin}, \TClink[2]{sɔkba}. \textit{n} [tòyòmbó] trouble (K dialect). \textit{Yɛ tombo ŋɔ mɔi gbo...} When in times of trouble.. \textit{Tombo bɔnth wɔ kɛ che bi ŋa wu.} Though he was troubled he was not destroyed.

\TCheadword[1]{ton} \textit{adj} \textbf{1)} small. \textit{Tamɔ tondɛ wɔ gbankthani kotha kathil bo̹m mɛ nɔ bɛn.} The small boy wrapped the big Kente cloth around himself as if he were a big man (\citealt{Pichl1967}). \textit{Tondɛ kɔ lɛ ituɛ kunɛ, mɔ kɔi kɔ thɔŋgul ŋa paŋdɛ.} The small bit that remains in the pot, you reserve it for the evening. \textbf{2)} fine. \textit{Yi kwey liwal, si yi chok lɛn ton, si yi panth lɛn do.} We take palm leaves, then we twist them to a fine line, then we tie this line (\citealt{Pichl1967}). \textbf{3)} little; \textit{toon} small, little (\citealt{Pichl1967}). \textit{Bɛl Maaɛ wɔe tipɛ mir-mir, wɔ mukumuku ton, ton, tokɛ ko.} Rat Wife began to watch intently, she crept little by little from above. comp. \TClink{palthon} (see \TClink[3]{pal}) 

\TCsubword[2]{ton} (der.) \textit{adv} a bit. \textit{Lagbo bɔmdai lɔɛ, pɔ kɔ ŋa gbompa ton, ɛn pɔ pɛ ka thiwonka, kaŋka kɔ ma gbompa ni bɔnɔ bul.} If it (rice field) is in a swamp, they will make it (space between plants) a little greater and make spaces so it (rice seedling) can grow without being pushed into one place. der. \TClink{tontonton} (see \TClink[1]{ton})

\TCsubword{tonton} (der.) \textit{cf}: \TClink{piyɛtpiyɛt}, \TClink[1]{tata} (der. of \TClink{taa}). \textit{adj} small. \textit{Mɔi rɛthi jɛmdɛ ton-ton.} You reduce the fire a little. \textit{Kɔ lɔ boni le ton-ton te kɔi koŋ ho.} It just remains low until it has cooked. \textit{Mi pɔ mi ka yen tontondɛ.} Mummy, they give me a little something.

\TCsubword{tontonton} (der.), (der. of \TClink[2]{ton}) \textit{temp} slowly. \textit{Yɛ hɔ ni yellɛ lanɛ mɔni kɛn keŋ-keŋdɛ, mɔkɔni kɛn ton-ton-tondɛ.} As it boils you are cutting the krain-krain slowly.

\TCheadword[2]{ton} (der. of \TClink[1]{ton})

\TCheadword{tonton} (der. of \TClink[1]{ton}) 

\TCheadword{tontonton} (der. of \TClink[2]{ton} (der. of \TClink[1]{ton}), see \TClink[1]{ton}) 

\TCheadword{toŋkandɔ} \textit{n} armpit; [toŋkandɔɛ] underarm (K dialect). \textit{Yaŋ tonkangdɔ thukul lɛ kɔ ho.} The sweat comes under my armpit (\citealt{Pichl1967}). 

\TCheadword{toŋkɔ} \textit{n} (wɔ/hã, N) fish species, cutlass fish (also: \textit{lonkɔ} freshwater fish cutlass ?) (\citealt{Pichl1967}). 

\TCheadword{too} \textit{cf}: \TClink[1]{laa}. \textit{n} \textbf{1)} [tòò] insect species, flea or louse (K dialect). \textbf{2)} (wɔ/hã, i) inspect species, fowl louse (\citealt{Pichl1967}). 

\TCheadword{toofi} \textit{cf}: \TClink{gbundagbunda} (der. of \TClink{gbunda}), \TClink{yɔk}. \textit{v} grab. \textit{Ŋchen thol kɛkɛ-kɛkɛ ni ŋkɔ toofi yekeɛ hiŋk sampaai?} Will not you climb down and quickly grab the cassava from inside the basket?

\TCheadword{tool} \textit{cf}: \TClink{gbunda}. \textit{v} [tóól] rape (K dialect). 

\TCheadword{top} \textit{n} ground pig or giant rat (Cricetomys gambianus) (\citealt{Pichl1967}). \textit{Tamɔ lɛ wɔ dẅiye ken top.} The boy is stealing like a ground-pig (\citealt{Pichl1967}). 

\TCheadword{topur} \textit{n} [tòpùr] favoritism, e.g., between children (K dialect).

\TCheadword{toto} \textit{n} [tótó] bird species (K dialect). 


\TCheadword[1]{tɔ} \textit{n} [tɔ̀] snail, identical to [tò] ‘grave' (K dialect). 

\TCheadword[2]{tɔ} \textit{cf}: \TClink[1]{mɛnɛ}. \textit{n} [tɔ̀] grave, identical to [tò] ‘snail' (K dialect). 

\TCheadword{Tɔka} \textit{nam} Tucker, name given to a person. \textit{Kɛ wanta bul ka che lɔ woŋga ka Tɔmi Tɔka ka kɛ ka che peshɛnt siza.} A girl used to be in this house of Tommy Tucker's, but she was a Cesarean-section patient.

\TCheadword{tɔkɔ} \textit{cf}: \TClink[1]{ha}, \TClink[2]{ŋal}. \textit{prep} about. \textit{Wɔ theeɛ tɔkɔ Plantiɛ.} He heard about Plantain (Island).

\TCheadword{tɔkɔli} (unspec. of \TClink[3]{hɔl}) 

\TCheadword{tɔkɔsi} \textit{v} make filthy by dragging things on the ground (\citealt{Pichl1967}). \textit{Hã ma sïnk walli nɔ lɛ ni puy, ihial ka nante; nɔ lɛ mɔ lɔ tɔkɔsi lɛ, mɔ bi hã bas lɔ.} Don't play with your palm branches and grass on the dancing place here today. The person who makes filth there will have to sweep it (\citealt{Pichl1967}). 

\TCheadword{tɔl} \textit{n} plant species; [tɔ̀l] Guinea corn (K dialect); [ntɔllɛ] Guinea corn (K dialect). \textit{Ŋ kɔ sɔyma pɛlɛ lɛ ni ntɔl lɛ ni nyɔk ma chɛk lɛ ko} Go mix the rice and the Guinea corn together and take them to the farm (\citealt{Pichl1967}). 

\TCheadword{tɔlɛ} (der. of \TClink[1]{ahɔl}) 

\TCheadword[1]{tɔm} \textit{n} idol.

\TCheadword[2]{tɔm} \textit{n} number. \textit{Nshini tɔmdɛ?} You do not know the number? \textit{A-a, ashini tɔmdɛ.} No, I don't know the number.

\TCsubword[3]{tɔm} (der.) \textit{v} count (K dialect). \textit{Yaŋ tɔm nyol ma wɔ lɛ.} I count his jewels (\citealt{Pichl1967}). 

\TCheadword{tɔmbɔ} \textit{n} (wɔ/hã, N, si) fish species, jumper mullet (\citealt{Pichl1967}). 

\TCheadword{Tɔmi} \textit{nam} Tommy, name given to second son. \textit{Kɛ wanta bul ka che lɔ woŋga ka Tɔmi Tɔka ka kɛ ka che peshɛnt siza.} A girl used to be in this house of Tommy Tucker's, but she was a Cesarean-section patient. 

\TCheadword[1]{tɔn} (der. of \TClink[2]{tɔn}) 

\TCheadword[2]{tɔn} \textit{cf}: \TClink[1]{chai}. \textit{v} sing. \textit{Ina toŋgiɛ mɔ ŋa tɔnda?} Who taught you how to sing? \textit{Wɔi kɔni pɔyko, yɛ kɔni yɛ wɔi ko sɛm ko thɔkɛ, wɔi po ŋa tɔn.} And then she goes to the stream, when she went to the stream, she stood by the tree, and then she started to sing. comp. \TClink{nɔtɔnɔ} (see \TClink{nɔ}), der. \TClink{tɔŋkwa} (see \TClink[2]{tɔn}) 

\TCsubword{tɔntho} (der.) \textit{n} singing. \textit{I yema ni wun ko ja tɔntho, la ivelɛmɔ teŋga-teŋgaɛ.} We want to now come to the singing aspect that we actually called you for.

\TCsubword[1]{tɔn} (der.) \textit{n} \textbf{1)} song. \textbf{2)} singing. \textit{Mɛŋk handɔ ŋɔ ntipɛ tɔndɛa?} What time did you start this singing? \textit{Yi chɔŋ weɛ ŋɔ mɔ tɔndɛ lendɛ.} We like the way you sing.

\TCsubword{tɔŋk} (der.) \textit{cf}: \TClink[1]{sɛli}. \textit{v} \textbf{1)} praise. \textbf{2)} serve. \textit{Man pɛŋke, ŋa tɔnk Bahin yɛ.} do not give up serving the Lord \textbf{3)} pray. \textit{Iŋa tɔnk wɔ wɛ yo wɛ.} We should pray to you every day. der. \TClink{tɔŋkwa} (see \TClink[2]{tɔn})

\TCsubword{tɔŋkwa} (der.), (der. of \TClink{tɔŋk}) \textit{cf}: \TClink[2]{hok} (der. of \TClink[1]{ho}, \TClink{-k}). \textit{v} celebrate, praise. \textit{Itɔnk wa, itɔnk wa.} Let us praise, let us praise.

\TCheadword{tɔnt} \textit{n} \textbf{1)} [tɔ̀nt] tributary, small river that leads to the main river (K dialect). \textbf{2)} creek. \textit{Ama lɛ ha hoth tɔnt l'ay ka thumɔ.} The women are fishing in the creek with the women's net (\citealt{Pichl1967}). 

\TCsubword{tɔntɛ} (der.) \textit{n} creekside town.

\TCheadword{tɔntɛ} (der. of \TClink{tɔnt}, \TClink[1]{ɛ}, see \TClink{tɔnt}) 

\TCheadword{tɔntho} (der. of \TClink[2]{tɔn}) 

\TCheadword{Tɔŋ} \textit{nam} Tong, male name given to a person (Pa Yanker knows no such name, “Tɔŋ”). \textit{Tɔŋ chiɛ pəlɛ lɛ sampa l'ay, kə koŋ kɔ sẽy kïl lɛ ko.} Tong brought the rice in the basket, but he has scattered it in the house (\citealt{Pichl1967}). \textit{Tɔŋ wɔ po̹l, wɔ gbo chaŋ-chaŋ pɔksi lɛ ay.} Tong is foolish, he goes abotu from one place to another (\citealt{Pichl1967}).

\TCheadword[1]{tɔŋ} \textit{cf}: \TClink{biŋ}, \TClink{hantha}, \TClink{waya}. \textit{n} (hɔ̃/tha) largest kind of fishing fence (\citealt{Pichl1967}); (\textit{tɔŋ} fishing fence, not known to Pa Yanker.)

\TCheadword[2]{tɔŋ} \textit{n} [tɔ̀ŋ] pillar (K dialect); \textit{tɔng kïl} (hɔ̃/tha) pillar, house post (\citealt{Pichl1967}).

\TCheadword{tɔŋha} \textit{n} [tɔ̀ŋhá] tree species with fruit, grows in old growth forest (K dialect). 

\TCheadword{tɔŋk} (der. of \TClink[2]{tɔn}, \TClink{-k}, see \TClink[2]{tɔn})

\TCheadword{tɔŋkwa} (der. of \TClink{tɔŋk} (der. of \TClink[2]{tɔn}, \TClink{-k}), see \TClink[2]{tɔn}) 

\TCheadword{tɔrɔth} \textit{Idph} emphatic ideophone. \textit{La libɛn Bɛl Maaɛ koŋ pɛ thaŋni poŋ boeɛ tokɛ wusɛ kunɛ <tɔrɔth>.} Quickly, Rat Wife had again climbed (and) disappeared above the kitchen into the thatch <tɔrɔth> (idph of emphasis). 

\TCheadword[1]{tɔth} \textit{cf}: \TClink[1]{bokoth}. \textit{v} [thɔ́th] suck, e.g., an orange (K dialect). \textit{Wɔ tɔth pak lɛ.} He sucks (the marrow out of) the bone (\citealt{Pichl1967}).\textit{ Woŋ dɛ wɔ tɔth nɔ yanɔ l'ay, lɛ sɛmɛ lɔ gbo.} The leech will suck a person if he is standing just (for a moment) in the stream (\citealt{Pichl1967}). 

\TCheadword[2]{tɔth} \textit{n} beast. \textit{Ŋha ya hɔ gboɛ ntɔthɛ gbi hɔlɔai.} Let me say, all the animals in the world.

\TCheadword{tɔthian} \textit{cf}: \TClink{tɛmɛ}. \textit{v} [tɔ́thíán] struggle to find something (K dialect).

\TCheadword{tɔthiani} \textit{v} weak.

\TCheadword{tradishɔnal} \textit{adj} traditional.

\TCheadword{traiya} (Eng \textit{try}) \textit{cf}: \TClink[1]{chɔk}, \TClink[2]{hani} (der. of \TClink{haa}, \TClink{-ni}), \TClink[1]{hɛl}. \textit{v} try. \textit{Aa, atraiya ton.} Yes, I tried a little bit.

\TCheadword{trɛn} (Eng \textit{train}) \textit{n} train.

\TCheadword{tri} \textit{n} \textbf{1)} town. \textbf{2)} village. comp., id. \TClink{paaŋtriayeŋ} (see \TClink[2]{paŋ})

\TCsubword{trihuɛ} (comp.) \textit{n} place where the dead live. \textit{Koŋ kɔni trï-huɛ} (? \textit{trï wuɛ}) He is gone where the dead live, i.e. he died (\citealt{Pichl1967}). 

\TCheadword{trihuɛ} (comp. of \TClink{tri}, \TClink[1]{ɛ}, \TClink[1]{wu}, see \TClink{tri}) 

\TCheadword{Triniti} (Eng \textit{trinity}) \textit{nam} Trinity.

\TCheadword{trit} (Eng \textit{treat}) \textit{cf}: \TClink{yɛthi}. \textit{v} treat. \textit{Anyindɛ kache, ŋɔ pɔ kache ŋa trit a?} The people in those days, how were they treated?

\TCheadword{trɔla} \textit{n} trawler.

\TCheadword[1]{tu} \textit{n} \textbf{1)} [tú] iron (K dialect). \textit{Itu lo hɔ̃ kələŋ hã cho' thigbe̹r ɛ.} This iron is good for making axes (\citealt{Pichl1967}). \textbf{2)} (hɔ̃/ma) \textit{itu} any pot (\citealt{Pichl1967}). \textit{Yɛmɔ ni hun sɛmi mɔi chi itu bɛia.} After putting it down you then bring the rice pot.

\TCsubword{kobotu} (comp.) \textit{n} \textit{kobotu} (ma) small iron pot (\citealt{Sumner1921}).

\TCsubword{tuyaka} (comp.) \textit{n}  \textit{itu-yakaa} (hɔ̃/-) iron pot (\citealt{Pichl1967}). 

\TCheadword[2]{tu} \textit{v} [tù] pound rice, peanuts, palm kernels (K dialect); [tsu] pound (B dialect); \textit{tu} pound rice (\citealt{Pichl1967}). 

\TCheadword[3]{tu} (Eng \textit{two}) \textit{cf}: \TClink[1]{tin}. \textit{Numb} two. \textit{Pandɛ ŋɔ pɔ wɔ April, nɛndɛ ŋɔ pɔ wɔ tu thaozin ɛn sikstin.} The month they call April, the year they call two thousand and sixteen. comp. \TClink{tintatu} (see \TClink[1]{tin}) 

\TCheadword[1]{tua} \textit{n} (wɔ/hã, si) fish species, bawbar Sam (\citealt{Pichl1967}). 

\TCheadword[2]{tua} \textit{n} be ashamed.

\TCheadword{tudu} (Eng \textit{to do}) \textit{v} do. \textit{I theɛn ni yeŋkɛlɛŋ kɛ no we tu du.} We do not feel good but there is nothing to do.

\TCheadword{tuk} \textit{cf}: \TClink{suŋkuthani} (der. of \TClink[1]{suŋkutha}, \TClink{-ni}). \textit{v} \textbf{1)} disappear. \textit{Nsiɛ tɛm pɛm doki yɛi chaŋ-chaŋdɛ raiyɛ ŋɔ koŋ tuk.} You know during the war how we were moving around, the document has disappeared. \textit{I koni sɔtha shiɛ lɛ Mbolomdɛ ma yema tuk ayenal gbe ko lɔ pɔ kache theli Mbolomdɛ.} We know that Bolom is disappearing in many places where they used to speak Bolom. \textbf{2)} be lost. \textit{Yen-o-yen gbi hɔ koŋ tuk.} Everything is lost. \textit{Nle kɔ bo mpɔni nwɔk mpika ntuk maɛ; labi la pɛthilɛ mini.} If you leave it and throw yourself into another language, you lose it; that is why it is not sweet to me.

\TCsubword{tuki} (der.) \textit{v} lose.

\TCheadword{tuki} (der. of \TClink{tuk}, \TClink[1]{-i}, see \TClink{tuk}) 

\TCheadword{tukum} \textit{cf}: \TClink{kuluŋ}. \textit{n} (wɔ/hã, si) antelope species, bush goat, any kind of smaller antelope (\citealt{Pichl1967}). 

\TCheadword{tukutɛkɛ} \textit{n} [túkútɛ̀kɛ̀] bird species that moves in large flocks (K dialect). 

\TCheadword{tumgbula} \textit{n} animal species.

\TCheadword[1]{tun} \textit{n} bird species, brownish bird 18 inches, sometimes shows the time (K dialect); (wɔ/hã, si) bird species, Senegal coucal (Centhropus senegalensis)(\citealt{Pichl1967}). \textit{Lɛ ntheyɛn gbo lo̹m tun mɔ sak nduɛ ay.} If you don't hear the voice of the Coucal bird you will be late in bed (proverb: If you don't act in time you will miss the opportunity) (\citealt{Pichl1967}). 

\TCheadword[2]{tun} \textit{adv} still.

\TCheadword{tunt} \textit{cf}: \TClink{bimni}, \TClink{chok}. \textit{v} \textbf{1)} bend (K dialect). \textbf{2)} twist. \textit{Apuma lɛ ha cho', santh lɛ tunt thɔm wɔ lɛ yenwɛy, ha kɔ ha koosi.} The children are fighting, the older one has badly twisted his companion, go and separate them (\citealt{Pichl1967}).

\TCsubword{tuntitunti} (der.) \textit{adj} [tùntìtùntì] crooked, can be said metaphorically of people (K dialect).

\TCsubword{tuntni} (der.) \textit{v} bend oneself.

\TCsubword{tuntɔni} (der.) \textit{v} be bent.

\TCheadword{tuntitunti} (der. of \TClink{tunt}) 

\TCheadword{tuntni} (der. of \TClink{tunt}, \TClink{-ni}, see \TClink{tunt}) 

\TCheadword{tuntɔni} (der. of \TClink{tunt}, \TClink{-ni}, see \TClink{tunt}) 

\TCheadword[1]{tuntun} \textit{cf}: \TClink{raaka}. \textit{n} plant species, a shrub similar to \textit{rakaa} (\citealt{Pichl1967}). 

\TCheadword[2]{tuntun} [túntún] \textit{cf}: \TClink{pulukɛ}. \textit{n} \textbf{1)} rubbish pile, [túntún (dɛ̀)] (the) rubbish pile (in town) (B dialect). \textbf{2)} dunghill. \textit{Ŋ kɔ pɔnki tutuŋ dɛ ato̹k.} Go throw it on the dunghill (\citealt{Pichl1967}). 

\TCheadword{Tuntuŋ} \textit{n} \textbf{1)} secret society. \textit{Ntuntuŋg} a society more highly regarded than Poro in Ndema – has images (\textit{Ketheboni}) that foretell misfortune – said to originally come from Baga (\citealt{Hall1938}). \textbf{2)} ancestral worship.

\TCheadword{tutuk} \textit{n} vine species, vine growing in swamps, fiber used for brooms, long and black, leaves used for medicine (K dialect). 

\TCheadword{tutun} \textit{v} heat.

\TCheadword{tuyaka} (comp. of \TClink[1]{tu}, \TClink[1]{ya}, see \TClink[1]{tu}) 

\TCheadword{twɛ} \textit{cf}: \TClink{lɔlma} (comp. of \TClink{lɔl}, \TClink[4]{ma}), \TClink[1]{mɔn}. \textit{v} have sex. \textit{Nɔma lɛ kong wo̹th kun kan gbo parɛ twɛ.} The woman is pregnant, she knew a man just recently (\citealt{Pichl1967}). 

\end{letter}

\newpage
\begin{letter}{Th}
\TCheadword[1]{tha} \textit{NCP} \textbf{1)} they. \textit{Pɔ koŋ gbo chakath yeŋkɛlɛŋ, pɔi chi bɛkthɛ.} They remove the stalks from the rice completely, then they bring the bags. \textit{Kilthi lɛ tha Pujoŋ kunɛ tha bom.} The houses in Pujehun are big (\citealt{Pichl1967}). \textbf{2)} them. \textit{Siŋthɛ thavɛ tha yaŋ akache siŋdɛ.} So those are the games I used to play. \textit{Yɛ ŋa ni joɛ, ŋa koni gbo jo, mɔi kɔ thɔk panthɛ gbi mminɛ tha koŋ sɛmi.} As you are now eating, after eating, you wash all the dishes and return them. \textbf{3)} it. \textbf{4)} relative pronoun: that/which. \textit{I yema ni hun ko siŋthɛ tha nkache siŋ ko tallɛ.} We want to come to the games you used to play when you were young. \textit{Siŋthɛ tha pɔ vel kukuu.} The game that is called ku-ku. \textit{Kɛ mi yaŋbɛ achɔŋɔmɔ sɛkɛe ŋa yi the tha nyiyɛ mi ɛ, Abatokɛ bɛ lɔ ruba.} But me, I thank you for the questions you have asked me, may God be with you.

\TCheadword[2]{tha} \textit{n} grandmother.

\TCsubword[1]{thetha} (der.) \textit{n} \textbf{1)} grandmother (Definitely dental, what PC Lenga is called). \textbf{2)} old woman.

\TCsubword{thethanthetha} (der.) \textit{n} great-grandchildren. \textit{N koŋ gbo le dumɔ komɔ nseiɛ haŋ komɔ thetha-thethaɛ.} You have to train the first child to the great-grandchild. \textit{Hin thivelendɛ, nrokɛ, nrekiaɛ, apima nthethanthethaɛ.} Behind us, the grandchildren, the great-grandchildren, our great-great-grandchildren. 

\TCsubword[2]{thetha} (unspec.) \textit{n} genealogical prefix ‘grand-' \textit{Yɛ mpima nthetha ha hundɛ, yɛ haŋ che veleŋkoɛ…} When the grandchildren come, since they are after (us)…

\TCheadword{thaal} \textit{n} rafter, when the building's roof is raised, the piece that goes across (K dialect). 

\TCheadword{thaale} \textit{n} \textbf{1)} (wɔ/hã, si) lobster (Panulirus regius) (\citealt{Pichl1967}). \textbf{2)} crab. \textit{Nthaalɛ lɛ ni hã che mɛnɛ ko.} The crabs and the giant snails live on the bottom of the sea (\citealt{Pichl1967}). comp. \TClink{wothaale} (see \TClink[2]{wo}) 

\TCsubword{thaaleŋgbuɔ} (comp.) \textit{n} crab species, big ocean crab (Calinectus gladiator) (\citealt{Pichl1967}). 

\TCheadword{thaaleŋgbuɔ} (comp. of \TClink{thaale}, \TClink{gbuɔ}, see \TClink{thaale}) 

\TCheadword{thaba} \textit{n} tobacco. \textit{Bikɔs hin abena hiɛ pɔ thuka ŋa bo pɔm thaba.} Because our (emph.) parents were just married with tobacco leaf. comp. \TClink{pɔmthaba} (see \TClink[1]{pɔm}) 

\TCheadword{thafɛ} (Mende ?) \textit{n} pipe.

\TCheadword{thai} \textit{n} fungal infection of animals and humans; mycosis.

\TCheadword{thak} \textit{cf}: \TClink[2]{bɛth}, \TClink[2]{kɛn}, \TClink{kɛth}, \TClink{rɔk}, \TClink{tɛnthe}. \textit{v} \textbf{1)} cut. \textit{Ŋkɔ-m thak gbasa bul.} Go cut for me one head-tie (\citealt{Pichl1967}). \textbf{2)} split.

\TCsubword[1]{thɛki} (der.) \textit{v} \textbf{1)} tear. \textbf{2)} split wood. \textit{Ŋkɔ thɛki iwɔm dɛ si ŋ kɔ yeki thɔk lɛ.} Go split the wood and then widen the split (\citealt{Pichl1967}). der. \TClink{thɛkini} (see \TClink{thak}) 

\TCsubword{thɛkini} (der.), (der. of \TClink[1]{thɛki}) \textit{v} be torn. 

\TCheadword{thakam} \textit{n} trumpet.

\TCheadword{thal} \textit{cf}: \TClink{muku}. \textit{v} creep.

\TCheadword[1]{tham} \textit{v} overcome.

\TCheadword[2]{tham} \textit{cf}: \TClink[3]{gbal}. \textit{v} write.

\TCheadword[3]{tham} \textit{v} be old enough. \textit{Yɛ kon thamdɛ, laŋgbaɛ wɛ ma pɛ lo sampa the.} When she was old enough, the man said she should stop weaving baskets. \textit{Bikɔs nɔbɛndɛ koŋ gbo tham, ko piŋgindɛ tamɔ.} Because if an old person has become old enough, she has turned into a baby.

\TCheadword{thambase} \textit{n} \textbf{1)} sign; mark. \textbf{2)} evidence; proof.

\TCheadword{thamir} \textit{cf}: \TClink{thɔi}. \textit{v} \textbf{1)} fail. \textbf{2)} drop out. \textit{Wɛl, ara ŋaa kandaɛ bul thamura mɔikɛ yɔllɛ.} Well, three are in school and one dropped out which makes it four.

\TCheadword{thamlamgbaŋ} \textit{n} [thàmlàmgbáŋ] long straight stretch of a river (K dialect). 

\TCheadword{thampel} \textit{n} [thámpél] bird species, hawk (K dialect); (wɔ/hã, N) bird species, hawk, kite (\citealt{Pichl1967}). comp. \TClink{bɛlthampel} (see \TClink[2]{bɛl}) 

\TCheadword{thamprr} \textit{n} [tham̀pr̀r̀] bird species, eagle (K dialect).

\TCheadword{thanɛ} \textit{dem} \textbf{1)} that. \textit{Hã mman tĩ, ya lɔng nui ko tɔnthi lɛ thanɛ!} Stop making noise, I am listening to that song! (\citealt{Pichl1967}). \textbf{2)} those. \textit{Thanɛ tha akache siŋdɛ?} The ones I used to play? \textit{So thanɛ gbi nka bitha?} So you used to have all those things?

\TCheadword{thanthen} \textbf{1)} \textit{adj} ordinary. \textit{Ŋa pɛ di yenchɛk a thanthendɛ.} They would also catch this ordinary fish. \textit{Ikoi baŋg li thanthɛndoki ikɔ pɛŋka.} We take this ordinary rope we jump with it. \textbf{2)} \textit{adv} in vain; for nothing. \textit{Ihɔlɔŋ hɔ̃ gbo thanthɛn.} Life is (just) in vain (\citealt{Pichl1967}).

\TCheadword{thanthɛŋkɔ} (der. of \TClink{thaŋkɔ}) 

\TCheadword{thanthɛŋkɔbɛ} (der. of \TClink{thaŋkɔ}) 

\TCheadword{thaŋ} \textit{v} \textbf{1)} go up. \textit{Hanɛ ŋa thaŋ, hanɛ ŋa thol.} Some are going up, some are going down. \textbf{2)} climb.

\TCsubword{thaŋni} (der.) \textit{v} climb up. \textit{Ni ŋa muni thaŋni, kara-kara, kara-kara, kara-kara poŋ! baiɛ tokɛ tɔrɔth.} And they returned climbed, \textit{kara-kara, kara-kara, kara-kara} (idph of scampering), disappeared! above the bari, \textit{tɔrɔth} (idph of emphasis). \textit{Yɛ Bɛl Maaɛ koŋ thaŋni boeɛ tokɛ hiŋk wul-lɛ lɔ bin wɔɛ...} When Rat Wife had climbed above the kitchen (away) from where death had missed her...

\TCheadword{thaŋgban} \textit{adj} [thàŋgbàn] much traveled, been everywhere, traveled about, e.g., water (K dialect).

\TCheadword{thaŋgbaŋ} \textit{n} rocks or cliffs near the shore on which certain kinds of saltwater shrubs grow. At high tide they are covered by the sea (\citealt{Pichl1967}). 

\TCheadword{thaŋkil} \textit{n} (wɔ/hã, N) fish species, fish smaller than a \textit{lonko}, usually found in wells (\citealt{Pichl1967}). 

\TCheadword{thaŋkir} [thàŋkǝ̀r] \textit{n} tree species (K dialect). 

\TCheadword{thaŋkɔ} \textit{cf}: \TClink{bikɔs}, \TClink{haliwɔ}, \TClink{hayɛ}. \textit{subordconn} \textbf{1)} though. \textit{Thankɔ hɔlthi nɔ-kafa lɛ bɔn hã ke gbɛng mɔ lɛ...} Though the eyes of the sinner cannot see thy glory... (\citealt{Pichl1967}). \textbf{2)} because. \textit{Thankɔ mɔ penkə hun.} Because you first came (\citealt{Pichl1967}). 

\TCsubword{thanthɛŋkɔ} (der.) \textit{subordconn} though, although, even though. 

\TCsubword{thanthɛŋkɔbɛ} (der.) \textit{subordconn} though, although, even though. \textit{Thanthɛŋkəbɛ yɛ wɔ-m diɛ, ya bi hã lanɛ wɔ.} Though he slays me, I will trust in him (\citealt{Pichl1967}). 

\TCheadword{thaŋni} (der. of \TClink{thaŋ}, \TClink{-ni}, see \TClink{thaŋ}) 

\TCheadword{thaŋthihɔn} \textit{adj} [thànthíhɔ̀n] proud (K dialect). 

\TCheadword{thaozin} (Eng \textit{thousand}) \textit{cf}: \TClink[2]{wul}. \textit{Numb} thousand. \textit{Pandɛ ŋɔ pɔ wɔ April, nɛndɛ ŋɔ pɔ wɔ tu thaozin ɛn sikstin.} The month they call April, the year they call two thousand and sixteen. \textit{Pɔ nɔi koŋ ka inshɔ, tɛmdɛ vɛ pɔ nɔi hɔm lɛ, haŋ ha thunɔ thaozin waŋ.} They would have given assurances, when they tell you the bride price is ten thousand.

\TCheadword{thapa} \textit{cf}: \TClink[2]{man}, \TClink[1]{mɛkin} (der. of \TClink[1]{mɛk}, \TClink[1]{-n}). \textit{v} \textbf{1)} stop. \textbf{2)} prevent.

\TCheadword{tharmra} \textit{cf}: \TClink{bin}. \textit{v} [thàrmrá] missed, something got away (K dialect). 

\TCheadword[1]{thath} \textit{n} [thàth] grass species, found in swamps, greenish in color (K dialect). 

\TCheadword[2]{thath} \textit{n} eye mucus. 

\TCheadword[3]{thath} \textit{n} [tháth] canoe seat (K dialect). 

\TCheadword{thatha} \textit{n} [thàthà] wall (K dialect). 

\TCheadword{the} \textit{cf}: \TClink{lɔŋnui} (unspec. of \TClink{nui}), \TClink[1]{si}. \textit{v} \textbf{1)} hear. \textbf{2)} obey. \textit{Mɔ ŋa theɛ po mɔi ken ki.} You should listen to your husband like this. \textbf{3)} understand. \textit{[N] koŋ gbo the ndumdɛ wɔnɛ gbi wɔ hundɛ ko ndum malan lɔ che.} You just have to understand character [in children] comes from the character that is already there. \textit{Bɛlsa ŋɔɛ handɔ ŋa hɔ si ŋa thee la?} What rat will speak and you understand it? \textbf{4)} smell. \textbf{5)} feel. \textit{A theɛ ni yeŋ kɛlɛŋ.} I do not feel good. \textit{Làŋgbàɛ́ thé nɛ̀kí kà bìllɛ́.} The man felt pain from yaws.

\TCsubword{chenthehwɛi} (comp.), (comp. of \TClink{theɛhwɛ}) \textit{adj} not hearing.

\TCsubword{theɛgbɔs} (comp.) \textit{n} smell.

\TCsubword{theɛhwɛ} (comp.) \textit{n} deafness. comp. \TClink{chenthehwɛi} (see \TClink{the})

\TCsubword{theɛn} (der.) \textit{v} feel. \textit{I theɛn ni yeŋkɛlɛŋ kɛ no we to du.} We do not feel good but there's nothing to do. \textit{I theɛn ni yeŋkɛlɛŋ kɛ ibiɛni weɛ ŋɔ ila bɔ kɔndɛm dɛ.} We do not feel good that we do not have a way of condemning it.

\TCsubword{thekɛ} (der.) \textit{v} feel. \textit{So ŋɔ nthekɛ lani a?} How do you feel about that? \textit{Pɔk si pim, Mbolomdɛ ma yema lɔ koŋ tuk, ŋɔ nthekɛ lani a?} In other places, the Sherbro language wants to disappear from there; how do you feel about that? der. \TClink{thekni} (see \TClink{the}) 

\TCsubword{theki} (der.) \textit{cf}: \TClink{nɛmil}. \textit{v} taste. \textit{Pɔmthi gbamdɛ lɛ ye ma kɔ gbo chɛth yeŋkɛlɛŋ ni ntheki kɔni pɛth-pɛthɛ…} Potato leaves, if you want to cook them nicely so that they taste good…

\TCsubword{thekni} (der.), (der. of \TClink{thekɛ}) \textit{v} feel. \textit{Mi, ŋɔ mɔ thekni ja Mbolom do wa?} Mi, how do you feel about this Sherbro?

\TCsubword{theni} (der.) \textit{v} \textbf{1)} feel. \textit{Wandaɛ bɛ yɛ wɔ ko theni ndikɛ wɔ yɛ kwɛ kɛmdɛ.} When the girl felt hungry, she took (the bucket?) \textbf{2)} feel ill. \textbf{3)} see oneself. comp. \TClink{theyɛn-nɛki} (see \TClink[1]{nak})

\TCheadword{thee} \textit{n} cheek, [ntheeɛ] the cheek (K dialect); \textit{nthee} (ma) cheek (\citealt{Pichl1967}). 

\TCheadword{theɛgbɔs} (comp. of \TClink{the}, \TClink[1]{gbɔs}, see \TClink{the}) 

\TCheadword{theɛhwɛ} (comp. of \TClink{the}, \TClink[1]{wɛi} (der. of \TClink[2]{wɛi}), see \TClink{the})

\TCheadword{theɛn} (der. of \TClink{the}, \TClink[1]{-n}, see \TClink{the}) 

\TCheadword{thek} \textit{n} (wɔ/hã, N) fish species, baiako (fish) (Lagocephalus laevigatus) (\citealt{Pichl1967}). 

\TCheadword{thekɛ} (der. of \TClink{the}, \TClink{-k}, see \TClink{the}) 

\TCheadword{theki} (der. of \TClink{the}, \TClink{-k}, \TClink[1]{-i}, see \TClink{the}) 

\TCheadword{thekni} (der. of \TClink{thekɛ} (der. of \TClink{the}, \TClink{-k}), \TClink{-ni}, see \TClink{the}) 

\TCheadword{theli} \textit{cf}: \TClink{gbemani}, \TClink[1]{hɔ}, \TClink[1]{lem}, \TClink{wɛ}, \TClink[2]{wɔni} (der. of \TClink[1]{hɔ}, \TClink{-ni}). \textit{v} \textbf{1)} speak. \textit{Shenge ka pɔ ŋa pɛ theli nwɔk mpim bisaid Mbolom?} Here in Shenge do they speak other languages besides Sherbro? \textit{Wɔ theli Mbolomdai, wɔ theli Mpothoai.} He spoke in Bolom, he spoke in English. \textbf{2)} say. \textit{Ashiɛ lanɛ la nko theli kiɛ.} I know what you said here. \textit{Lanɛ la yi theliowɛ labi ŋa kɔni, labi ŋa che haŋ gbɛŋ.} What we are saying here is going to stay and last forever. \textbf{3)} talk. \textit{Ŋa mam ŋa theli ŋani po mɔi.} You laugh, you talk with your husband. \textit{Yɛmɔ theli ko aŋaɛ, nwɔk mpim ma pɔ chi komɔko ma che ndumɔ, nye?} When you talk to the people, some cases they bring to you are difficult, right?

\TCsubword{theliaŋ} (der.) \textit{n} talking. \textit{Bolomnɔɛ wɔn wɔ bi ndum, yemani theliaŋ gbe.} The Sherbro man has good character, he does not want too much talking. \textit{Ikoŋ ke jao ki theliande ŋɔ ŋa koi huŋ thelimando wɛ.} We have seen this thing, this talk you have talked to us now.

\TCsubword{thelini} (der.) \textit{v} speak. \textit{Labo ma chaŋ thelini ndɔ-ndɔ, ŋɔ nkema a?} If they speak it more everywhere, how do you see it? \textit{Handɔ ma chaŋ thelinia?} Which one is widely spoken?

\TCheadword{theliaŋ} (der. of \TClink{theli}) 

\TCheadword{thelini} (der. of \TClink{theli}, \TClink{-ni}, see \TClink{theli}) 

\TCheadword{Them} \textit{n} \textbf{1)} Themne people. \textit{Nthemdɛ ma lɔ, Asosoɛ ŋa lɔ.} The Themne are there, the Soso are there. \textit{Akoroma ŋɔ cheni Them.} Koromas are not Themnes. \textbf{2)} Themne region. \textit{Nsanda ko, Them ko?} In Sanda, [is that] a Themne region? \textbf{3)} Themne language. \textit{Yɛ ŋa kɔ ŋa mi lɛŋ Nthemdai, ha ŋai lɛŋ Mbolomdai.} Whenever they would greet me in Themne, I would reply in Bolom. \textit{Ŋai hɔ i mɔm nche hɔ Nthemdɛ?} Then they would say, don't you speak Themne?

\TCsubword{Themnɔ} (comp.) \textit{n} Themne person. \textit{Koroma cheni Themnɔ.} The Koromas are not Themnes.

\TCheadword{Themanɔ} \textit{nam} Themano, name given to a place. \textit{Ko lɔ pɔ bɛ yuk bɔmthaiɛ, Themanɔ ko lɔn pɔ lɔ yuk, tiko yami.} It is only there that they plant in the muds, at Themano, my mother's village.

\TCheadword{Themdel} \textit{nam} Timdale Chiefdom. \textit{Nande ako vel laŋgba bul wɔ pɔ gbem Themdɛl ko.} Today I have called on a man who was born in Timdale (Chiefdom).

\TCheadword{Themnɔ} (comp. of \TClink{Them}, \TClink{nɔ}, see \TClink{Them}) 

\TCheadword{theni} (der. of \TClink{the}, \TClink{-ni}, see \TClink{the})

\TCheadword[1]{thenthes} (der. of \TClink[2]{thenthes}) 

\TCheadword[2]{thenthes} \textit{n} [thénthés] vine species with leaves that scratch, burn like nettles (K dialect).

\TCsubword[1]{thenthes} (der.) \textit{n} poisonous itchy leaf used by charmers to do harm to others (\citealt{Pichl1967}). \textit{Thenthes hɔ wɛy, pə bak hɔ gbo nɔ wɔ sɔkul likɔɔ.} \textit{Thenthes} is bad, they just rub it on a person, (and) it makes him scratch his skin (\citealt{Pichl1967}). 

\TCheadword{theŋgbleŋ} \textit{n} bird species in sparrow family with long tail (K dialect).

\TCheadword{theŋkil} \textit{adj} clear. \textit{Ŋ ke mən ntheŋkil lɛ.} Look how clear the water is (\citealt{Pichl1967}).

\TCheadword{theŋkleŋ} \textit{n} (wɔ/hã, N) crab species, small beach crab (Ocypada africana, Plagusia depressa) (\citealt{Pichl1967}). 

\TCheadword{thes} \textit{cf}: \TClink{binch}. \textit{n} bean. \textit{Wɔ̀ yúk (*ɛ́) thésthɛ̀. Wɔ̀ yùkɛ́.} He planted the beans. He planted.

\TCheadword[1]{thetha} (der. of \TClink[2]{tha})

\TCheadword[2]{thetha} (unspec. of \TClink[2]{tha})

\TCheadword{thethanthetha} (der. of \TClink[2]{tha})

\TCheadword{theyɛn-nɛki} (comp. of \TClink{theni} (der. of \TClink{the}, \TClink{-ni}), \TClink{nɛki} (der. of \TClink[2]{nak}, \TClink[1]{-i}), see \TClink[1]{nak}) 

\TCheadword[1]{thɛ} \textit{v} \textbf{1)} burn. \textit{Pɔ koŋ thɛɛ ŋchɛkɛ.} They finished burning the area they had brushed. \textbf{2)} roast. comp. \TClink{yekɛthɛɛ} (see \TClink{yekɛ}) 

\TCsubword{thɛsal} (comp.) \textit{n} farm burning.

\TCheadword[2]{thɛ} \textit{n} (kɔ/ma) tree species, sandpaper tree (Ficus exasperata) (\citealt{Pichl1967}). 

\TCheadword{thɛbu} \textit{n} \textbf{1)} (wɔ/na, N) kind spirits who direct and assist people when at work (\citealt{Pichl1967}) \textbf{2)} species of elves that help wood carvers (\citealt{Hall1938}).

\TCheadword{thɛɛ} [thɛ̀ɛ̀] \textit{n} insect species, edible flying ant that comes in the rainy season around June, can even be eaten without cooking, dark brown in color, mates in the air and when they fall the female loses her wings and they cluster, children will look for such congregations on the farm (K dialect); \textit{thɛ} (wɔ/hã, i) insect species, edible reddish flying white ant, four times as large as \textit{gbe̹gbe̹n}, some people eat it (\citealt{Pichl1967}). 

\TCheadword[1]{thɛk} \textit{cf}: \TClink{ham}. \textit{n} (wɔ/hã, si) lizard (Agama agama) (\citealt{Pichl1967}). 

\TCheadword[2]{thɛk} \textit{n} side.

\TCheadword{thɛkɛ} \textit{cf}: \TClink[5]{baŋ}, \TClink{blem}. \textit{v} \textbf{1)} blame. \textbf{2)} explain.

\TCheadword{thɛkɛn} \textit{n} \textit{ithɛkɛn} (hɔ̃/-) plant species, thorny shrub with big white seeds used for the warri game (\citealt{Pichl1967}). 

\TCheadword{thɛkɛsi} [thɛ́kɛ́sí] \textit{cf}: \TClink{yeyɛ}. \textit{v} \textbf{1)} interpret. \textit{Wɔ ma theli, wɔ mɔ ma thɛkɛsiɛ kunɛ yeŋkɛlɛn ba.} He can speak Sherbro, and translates it for you very well. \textbf{2)} clarify. \textit{Ma wɔ bo toŋgi, ni nchewɔ thɛkɛsiɛ ja yegbe, la chenche yeŋkɛlɛn.} Do not just show him then you do not make clear to him, it would not be good. \textit{Nɔthiɛ nthɛkɛsiɛ wɔ ni san la ntenɛ.} Human beings clarify in order to understand things. \textbf{3)} explain. \textit{Ko lɔ pɔ joɛ, pɔi hun koŋ, kendɛ ŋɔ nko kɔlo thɛkɛshi ko keŋ-keŋdɛ.} Where they eat, doing everything, just as how you had explained for the krain-krain. \textit{So hin ko thɛkɛshiɛ anyaɛ, la cheŋ pɛ hani, ni ka ko hin ko ramdɛ kunɛ.} As we have explained to the people, it does not happen anymore, even in our family.

\TCsubword{thɛkɛsini} (der.) \textit{v} watch over oneself. \textit{Mɔ gbɛ yeŋkɛlɛn, mɔ ŋa thɛkɛsini.} You should walk carefully, you should watch over yourself.

\TCheadword{thɛkɛsini} (der. of \TClink{thɛkɛsi}, \TClink{-ni}, see \TClink{thɛkɛsi}) 

\TCheadword[1]{thɛki} (der. of \TClink{thak}, \TClink[1]{-i}, see \TClink{thak}) 

\TCheadword[2]{thɛki} \textit{cf}: \TClink{mɛni}. \textit{v} \textbf{1)} kindle. \textit{Ŋ kɔ thɛɛki jɛmdi lɛ.} Kindle the fire (\citealt{Pichl1967}). \textbf{2)} incite. \textit{Nɔma lo wɔn nche nwɛy, wɔ thɛɛki lijɛm anyin thiyeŋ.} This woman is a bad one, she incites people (\citealt{Pichl1967}). 

\TCheadword{thɛkika} \textit{cf}: \TClink{berɛ}. \textit{n} axe (\textit{berɛ thɛkika} instead of \textit{thamhak}) (B dialect); \textit{thamhak} (hɔ̃/tha) axe (non-African type) (ex. Algonquin via Eng) (\citealt{Pichl1967}).

\TCheadword{thɛkini} (der. of \TClink[1]{thɛki} (der. of \TClink{thak}, \TClink[1]{-i}), \TClink{-ni}, see \TClink{thak}) 

\TCheadword{thɛl} \textit{v} \textbf{1)} trim. \textbf{2)} circumcise.

\TCheadword{thɛlɛn} \textit{cf}: \TClink{thom}, \TClink[1]{yi}. \textit{v} \textbf{1)} [thɛ̀lɛ̀n] beg, appeal to someone (K dialect). \textbf{2)} ask (\citealt{Pichl1967}). \textit{Yi thɛlɛn baal lɛ, kong balani.} We asked the chief because of the dispute about adultery, and he has consented (\citealt{Pichl1967}). 

\TCheadword{thɛm} \textit{v} hatch.

\TCheadword{thɛmba} (unspec. of \TClink{thɔm}) 

\TCheadword{thɛmkɔ} (unspec. of \TClink{thɔm}) 

\TCheadword{thɛmni} (unspec. of \TClink{-ni}) 

\TCheadword{thɛmp} \textit{n} tree species, like \textit{tel} but leaves slightly broader, used for weaving (K dialect). 

\TCheadword[1]{thɛn} \textit{n} \textbf{1)} [thɛ̀n] story (K dialect). \textit{Thɛn: Yendɛ hɔ bi ni Ba Na che tə tondɛ.} A story: Why the spider has such a small waist (\citealt{Sumner1921}). \textbf{2)} fable. \textbf{3)} affair. \textit{Chɛliɛ mi tɛn wɛy ya che kɔn pɔkɔni.} He created a bad situation for me, I shall not forget it (\citealt{Pichl1967}). \textbf{4)} proverb. \textit{…ni mgballɛ gbi maiko koiyɛ, Ithaiɛ, yen-o-yen.} …and all the writings we have taken, the proverbs, everything. \textit{Nthaɛ maMbolomdɛ}, ‘Bolom proverbs,' title of a 1979 Institute for Sierra Leonean Languages (TISLL) (Lutheran Bible Translators) ms containing 175 proverbs (\citealt{TISLL1979}).

\TCheadword[2]{thɛn} \textit{n} breeze.

\TCheadword{thɛnsunth} \textit{n} (wɔ/hã, N) fish species, ladyfish or longneck (Cynoscion) (\citealt{Pichl1967}). 

\TCheadword[1]{thɛnthɛ} (Eng \textit{tent}) \textit{n} mosquito net.

\TCheadword[2]{thɛnthɛ} \textit{n} row or line of corn, soldiers, etc. (\citealt{Pichl1967}). 

\TCheadword{thɛnthɛs} \textit{Idph} [thɛ́nthɛ́s] of noise hen makes when it is about to lay an egg (K dialect).

\TCheadword{thɛŋ} \textit{cf}: \TClink[3]{bɛk}. \textit{n} side. \textit{Mɔ lɔ che hin thɛŋɛo.} You are always by our side.

\TCsubword{thɛŋkɛi} (comp.) \textit{Loc} near.

\TCheadword{thɛŋgbɛŋ} \textit{cf}: \TClink{thɔŋkaŋ}. \textit{n} bat species. \textit{Ba Thəngbəŋ lee mathui bach lɛ veleŋ che-lɛ mɔ hunki gbo...} Mr. Bat remained hidden behind a young palm tree so that if somebody came there... (\citealt{Pichl1967}). \textit{Thɛngbɛŋ velni thɔnkaŋ kə wɔ ton chaŋ thɔnkaŋ.} The \textit{thɛŋgbɛŋ} resembles the \TClink{thɔŋkaŋ}, but it is smaller than the \TClink{thɔŋkaŋ} (\citealt{Pichl1967}). 

\TCheadword{thɛŋk} \textit{cf}: \TClink{thɔndɔ}, \TClink{yɛthɔk}. \textit{v} \textbf{1)} put up for storage. \textit{thɛŋk} put up. \textbf{2)} bring up. \textit{Beraa, hi thola ka thigbikan ni hi kɔa gbunda feɛ hiŋk mɛsaɛ atok, ni hi thɛŋk ŋɔ tokɛ ka.} Gentlemen, let us run down and grab the money on top of the table, and let us store it up here. \textbf{3)} take. \textit{“Bɛlsɛ, bɛlsɛ, bɛlsɛ, bɛlsɛ,” thanthɛn; bɛlsɛ koŋ thɛŋk feɛ gberba.} “Rats, rats, rats, rats,” nothing they can do; the rats have taken plenty of money away.

\TCsubword{thiŋgi} (der.) \textit{cf}: \TClink{tholi} (der. of \TClink{thol}, \TClink[1]{-i}). \textit{v} \textbf{1)} put down. \textit{Mɔi nɛmil hɔŋ shi gbo che hɔŋ nyemɔɛ, mɔi thiŋgi hɔ koŋ gbo lɔ, mɔi thiŋgi.} You taste it, if it is exactly as you want it, then you set it down if it has finished cooking. \textbf{2)} take off. \textit{aaa yɛ mɔ ni koŋ ha vɛ ni mɔi thiŋgi bokɛ mɔi sɛmi.} After doing all that, you take the sauce off the fire and set it down.

\TCheadword{thɛŋkɛ} \textit{n} pen.

\TCheadword{thɛŋkɛi} (comp. of \TClink{thɛŋ}, \TClink[5]{ken}, see \TClink{thɛŋ}) 

\TCheadword[1]{thɛrɛŋ} \textit{cf}: \TClink{yaŋka}. \textit{n} cave.

\TCheadword[2]{thɛrɛŋ} \textit{cf}: \TClink{pe}. \textit{n} [thɛ́rɛ́ŋ] rock or stone that is too large to pick up (K dialect).

\TCheadword{thɛsal} (comp. of \TClink[1]{thɛ}, \TClink[1]{sal}, see \TClink[1]{thɛ}) 

\TCheadword{thɛthɛ} \textit{v} coax.

\TCheadword{thɛthɛl} \textit{n} insect species, grasshopper, [thɛ̀thɛ̀l]/ [thɛ̀thɛ̀lsɛ̀] grasshopper/ grasshoppers (B dialect); [thɛthɛl] dragon fly, grasshopper (K dialect); \textit{thɛthɛɛl} (wɔ/hã, N) grasshopper (\citealt{Pichl1967}). 

\TCheadword{thɛtɛ-thɛtɛ} \textit{n} (kɔ/ma) plant species, plant with broad almost round leaves that are used for porridge (\citealt{Pichl1967}). 

\TCheadword[1]{thi} \textit{adj} black. {Nɛn doki wɔe hun chɔŋ waaŋmaa len yeŋkɛ-lɛŋba; ilel wááŋmààɛ ŋɔ ka cheɛ Yeŋken haliwɔ wááŋmàà ki jal wɔɛ ŋɔ ka che thii.} This man came to (began to) love this woman very much; the woman's name was Yanken because her skin was black. comp. \TClink{pithi} (see \TClink[1]{pi}), \TClink{sweinthi} (see \TClink{swei}), \TClink{velthi} (see \TClink[2]{vel})

\TCheadword[2]{thi} \textit{disco} please.

\TCheadword{thi-} \textit{NCM} \textit{pfx} \textit{ubd stem} noun class marker. \textit{Mɔni bɔ shi nɛnthɛ tha nko koi ko gbemiɛ?} Do you know how many years you have been delivering (babies)? \textit{Pe rɛnthɛ, Laɔn ɔf Juda.} Rock of ages, Lion of Judah. \textit{I koi pisthɛ iraparapa tha iŋakɔ mɔi bɔl.} We would take small pieces of cloth, we make it like ball. comp. \TClink{thibolɔtok} (see \TClink[1]{bol}), \TClink[2]{thiveleŋ} (see \TClink[1]{veleŋ}), der. \TClink{nɔthi} (see \TClink{nɔ}), id. \TClink{lomthibul} (see \TClink[2]{lom}) 

\TCheadword{thibeŋ} \textit{adj} improper.

\TCheadword{thibolɔtok} (comp. of \TClink{thi-}, \TClink[1]{bol}, \TClink{atok}, see \TClink[1]{bol}) 

\TCheadword{thiboŋ} \textit{Idph} [thíbóŋ] of a stone falling into water (K dialect). 

\TCheadword{thifaŋ} \textit{cf}: \TClink{bolmin} (comp. of \TClink[1]{bol}, \TClink[3]{min}). \textit{adj} idiotic.

\TCheadword{thiiŋ} \textit{Idph} of being full! \textit{Mmɛn dɛ yema bɛ pɛr wɔm dɛ <thiiŋ> mɛŋk-o-ki, ni ŋɔ yema nyuŋ.} The water is about to fill the canoe <thiiŋ> at this time, and it will capsize.

\TCheadword{thikla} \textit{cf}: \TClink{wɔŋhul} (der. of \TClink{wɔŋ}, \TClink{-ul}). \textit{v} \textbf{1)} [thíklá] betray (someone) (K dialect). \textbf{2)} sell; trade. \textit{Yɛ mɛŋk pin ni thikla awokɛ ka koŋ dɛ...} When the time of buying and selling enslaved people had finished... comp. \TClink{nɔthikla} (see \TClink{nɔ}) 

\TCheadword{thil} \textit{cf}: \TClink{kɔnaibol} (id. of, comp. of \TClink[2]{kɔ}, \TClink[1]{nai}, \TClink[1]{bol}), \TClink{sɛmɛkni} (der. of \TClink[1]{sɛm}, \TClink{-k}, \TClink{-ni}). \textit{v} urinate (\citealt{Pichl1967}). 

\TCheadword{thiliŋ} (der.) \textit{n} \textit{nthïlïng} (ma) urine (\citealt{Pichl1967}).

\TCheadword{thiliŋ} (der. of \TClink{thil}) 

\TCheadword{thim} \textit{cf}: \TClink{chok}, \TClink{pikith}, \TClink{tunt}. \textit{v} \textbf{1)} [thím] roll up a mat (K dialect). \textbf{2)} turn. \textit{Nthim bot lɛ njok ɛ, thipe tha che ko!} Turn the boats to the right side, there are rocks ahead! (\citealt{Pichl1967}). \textbf{3)} wag. \textit{Məndɛ ma thim gbəlaŋ.} The water is whirling around (\citealt{Pichl1967}). 

\TCsubword{thimkɔk} (comp.) \textit{cf}: \TClink{vunthu} (der. of \TClink{runth}). \textit{v} \textbf{1)} turn one's back. \textbf{2)} retreat.

\TCsubword{thimini} (der.) \textit{v} loiter.

\TCheadword{thimbɔs} \textit{cf}: \TClink{chiɛ}. \textit{n} shore. comp. \TClink{pethimbɔs} (see \TClink{pe}) 

\TCheadword{thimik} \textit{n} neck, [tə̀mə̀k]/[tə̀mə̀kthɛ̀] neck/necks (B dialect); [timikɛ] the neck (K dialect); \textit{thïmïk} (hɔ̃/tha) throat, neck (\citealt{Pichl1967}). 

\TCheadword{thimini} (der. of \TClink{thim}, \TClink[1]{-i}, \TClink{-ni}, see \TClink{thim}) 

\TCheadword{thimkɔk} (comp. of \TClink{thim}, \TClink{kɔk}, see \TClink{thim}) 

\TCheadword{thimni} (der. of \TClink{thim}, \TClink{-ni}, see \TClink{-ni}) 

\TCheadword{thiŋ} \textit{cf}: \TClink{yibaw}. \textit{v} foretell (\citealt{Pichl1967}); . \textit{theŋg} divine (\citealt{Hall1938}). comp. \TClink{nɔyiɛnthiŋ} (see \TClink{nɔ}) 

\TCsubword{thiŋnɔ} (comp.) \textit{cf}: \TClink{nɔyiɛnthiŋ} (comp. of \TClink{nɔ}, \TClink[1]{yi}, \TClink{thiŋ}), \TClink{nɔyiɛyibaw} (comp. of \TClink{nɔ}, \TClink[1]{yi}, \TClink{yibaw}). \textit{n} (wɔ/hã) diviner (also: \textit{nɔ lom thïng}) (\citealt{Pichl1967}).

\TCheadword{thiŋgi} (der. of \TClink{thɛŋk}, \TClink[1]{-i}, see \TClink{thɛŋk}) 

\TCheadword{thiŋk} \textit{n} root.

\TCheadword{thiŋki} \textit{v} take off fire. \textit{N thinki itu lɛ.} Take the pot off the fire (\citealt{Pichl1967}). 

\TCheadword{thiŋnɔ} (comp. of \TClink{thiŋ}, \TClink{nɔ}, see \TClink{thiŋ}) 

\TCheadword[1]{thiriŋ} \textit{v} snore.

\TCheadword[2]{thiriŋ} \textit{v} hex.

\TCheadword{thisɛm} \textit{n} [thìsɛ̀m] position (K dialect). 

\TCheadword[1]{thiveleŋ} \textit{n} without; in absence of. \textit{A chen bɔ kɔ hɔm thiveleŋ.} I cannot go without you (\citealt{Pichl1967}). \textit{Yaŋ thiveleŋ lɔ ŋhɔ lane, ncho lan bɔ hɔ'yaŋ thoɛ lɛ.} It is in my absence that you said so, you cannot say so in my presence (\citealt{Pichl1967}). 

\TCheadword[2]{thiveleŋ} (comp. of \TClink{thi-}, \TClink[1]{veleŋ}, see \TClink[1]{veleŋ}) 

\TCheadword{thiyeŋ} \textit{cf}: \TClink{yeŋthi}. \textit{post} \textbf{1)} among. \textbf{2)} between.

\TCheadword[1]{tho} \textit{v} \textbf{1)} [thó] drive off, drive away (K dialect). \textit{Ha che hɔ ha ni ha ye tho apootooa lɛ.} They fought for a long time and then they drove away the Europeans (\citealt{Pichl1967}). \textit{N tho thumɔɛ lɛ, wɔ tun gbɔs wɛy.} Drive out the dog; he smells bad (\citealt{Pichl1967}). \textbf{2)} [thó] banish (K dialect). 

\TCheadword[2]{tho} \textit{n} [thò] bush (K dialect). \textit{Nan baŋk, baŋk nan tho.} Pull a vine, and the vine pulls the bush (proverb). comp. \TClink{kɛntrithoɛ} (see \TClink{kɛntri}), \TClink{soŋktho} (see \TClink[2]{soŋk}), \TClink{yentho} (see \TClink[1]{yen}) 

\TCheadword[3]{tho} \textit{cf}: \TClink[1]{ki}, \TClink[1]{lan}, \TClink{wɔnɛ}. \textit{dem} these. \textit{Ŋa lee gbo pos, ni nɔmaa bul ŋan thiyeŋ wɔe chaɛ tɔn tho ki.} They just continue peeling, then a woman among them raised this song. 

\TCheadword{thoi} \textit{v} chase.

\TCsubword{thoiŋ} (der.) \textit{cf}: \TClink{thok}. \textit{v} [thóíŋ] chase, run after (K dialect). \textit{Tùmɔ̀ɛ̀sɛ̀ ŋà thóín vísɛ̀}. The dogs chase the animal.

\TCheadword{thoiŋ} (der. of \TClink{thoi}, \TClink{-ŋ}, see \TClink{thoi}) 

\TCheadword{thok} \textit{cf}: \TClink{thoiŋ} (der. of \TClink{thoi}, \TClink{-ŋ}). \textit{v} \textbf{1)} [thók] hunt with dogs (K dialect). \textit{Poinɔ wɔ̀ thókɔ́.} The hunter (who has only wounded the animal) will come back with dogs (and chase). \textbf{2)} [thók] hunt (K dialect). \textit{A thók ma pɛllɛ.} I hunt with a net. comp. \TClink{pɛlthook} (see \TClink[2]{pɛl}), \TClink{thumɔi-thɔkɔ} (see \TClink{thumɔɛ}) 

\TCsubword{thookɛ} (der.) \textit{n} (kɔ/-) hunting with dogs (\citealt{Pichl1967}). 

\TCheadword{thol} \textit{cf}: \TClink{duk}. \textit{v} \textbf{1)} come down. \textit{Yaŋ ya thol kɛkɛ-kɛkɛ hiŋk ka.} Let me come down quickly from here. \textbf{2)} go down. \textit{Hanɛ ŋa thaŋ, hanɛ ŋa thol.} Some are going up, some are going down. \textit{Kɛ ya chen thol gbi hiŋk ka.} But I am not going down there from here. \textbf{3)} fall. \textit{Ichɔ ŋa thaŋ, ibo thole.} The higher we climb, the more we fall. \textbf{4)} sink.

\TCsubword{tholi} (der.) \textit{cf}: \TClink{thiŋgi} (der. of \TClink{thɛŋk}, \TClink[1]{-i}). \textit{v} \textbf{1)} take down. \textbf{2)} put down. \textit{Pɔi tholi ni pɔ yɛthiɛ ŋɔ, pɔi bɛ pothɛ.} They put it down and would lower it, and then they add the dirt. unspec. \TClink{tholiɛpɔ} (see \TClink{thol})

\TCsubword{tholiɛpɔ} (der.), (unspec. of \TClink{tholi}) \textit{v} put down.

\TCheadword{tholhaa} \textit{n} [thólhàà] tree species, found in the bush, mostly used for firewood, strong, can be used for boards (K dialect). 

\TCheadword{tholi} (der. of \TClink{thol}, \TClink[1]{-i}, see \TClink{thol})

\TCheadword{tholiɛpɔ} (unspec. of \TClink{tholi} (der. of \TClink{thol}, \TClink[1]{-i}), see \TClink{thol}) 

\TCheadword{thom} \textit{cf}: \TClink{thɛlɛn}. \textit{v} \textbf{1)} [thóm] beg for something from someone (K dialect). \textit{Ya bɔnthɔ wɔ poo yekə, ya thom wɔ ni kənklɛni.} I met him sharing cassava; I begged him (for some), but he refused (\citealt{Pichl1967}). \textbf{2)} charter. (\citealt{Pichl1967}). comp. \TClink{nfinɔthomɔ} (see \TClink{nɔ}) 

\TCsubword{thomnɔ} (der.) \textit{cf}: \TClink{nfinɔthomɔ} (comp. of \TClink{nɔ}, \TClink{thom}). \textit{n} beggar.

\TCheadword{thomnɔ} (der. of \TClink{thom}, \TClink{nɔ}, see \TClink{thom}) 

\TCheadword[1]{thon} \textit{v} [thón] fry (K dialect). \textit{Aapum ŋa nuputha mbana ndriɛ ni gbɛrɛ ha thoŋ bo.} Others mix ripe bananas with flour to fry.

\TCheadword[2]{thon} \textit{cf}: \TClink[1]{mɛntɛ}. \textit{n} [thòn] inside of long stem of \textit{mɛntɛ} (bamboo) used to make mats (K dialect). 

\TCheadword{thontha} \textit{cf}: \TClink{sap}. \textit{v} [thónthá] catch, e.g., oranges when thrown down (K dialect).

\TCheadword{thonthni} (comp.) \textit{v} squat.

\TCheadword{thoŋ} \textit{n} \textbf{1)} bamboo (\citealt{Sumner1921}). \textbf{2)} (kɔ/ma) bamboo pole (\citealt{Pichl1967}). \textbf{3)} \textit{nthong} (ma) bamboo chair (\citealt{Pichl1967}).

\TCheadword{thoŋi-thoŋi} \textit{v} run after. \textit{Yɛ imath-mathnindɛ apikandɛ ŋani thoŋi-thoŋi siŋthɛ vɛ…} When we would hide and the boys would run after us, (in) those games…

\TCheadword[1]{thoŋka} \textit{cf}: \TClink[2]{hɔ} (der. of \TClink[1]{hɔ}). \textit{v} [thóŋká] argue, judge, enter a decision (K dialect). \textit{Bɛl Maaɛ ŋani poo wɔɛ ŋa lɔ thoŋka boe bom dɛ tokɛ wusɛ kunɛ.} Rat Wife and her husband are arguing in the thatch above the big kitchen. \textit{Mɔ thonka tɛm gbi, kə nchen kɔ bay ko no pə si lɛ mɔ lɛ nɔ-thonka}. You are arguing all of the time, but you don't go to court to show them that you are a lawyer (\citealt{Pichl1967}). \textit{Pɔ bɛ wɔ ŋgbekteɛ ni pɔ sɛmi wɔ bai ko anyaɛ gbi chee: lɔ pɔ bi ha thoŋka wɔ}. They put him in handcuffs and brought him to the bari in front of all the people where they will judge him. comp. \TClink{nɔthoŋka} (see \TClink{nɔ}) 

\TCsubword[2]{thoŋka} (der.) \textit{n} argument, discussion. \textit{Yɛ thoŋka ki gbi kɔ haani bɛl siatiŋ doki thiyeŋ dɛ.} When all this arguing is going on between these two rats…

\TCsubword{thoŋki} (der.) \textit{v} \textbf{1)} exclaim. \textbf{2)} proclaim. \textbf{3)} show. \textit{Ina toŋgiɛ mɔ ŋa gbemi ahindɛa?} Who showed you how to deliver people? \textbf{4)} point out. \textbf{5)} teach. \textit{Ina toŋgiɛ mɔ ŋa tɔnda?} Who taught you how to sing? \textit{Aa ama ha toŋi.} Yes, I am teaching them. \textbf{6)} summon. \textit{Bia toŋkiɛ jali Kaiŋ ha kɔnth.} Bia summoned Kayn for seizure (\citealt{Pichl1967}). der. \TClink{thoŋkini} (see \TClink{-ni})

\TCheadword{thoŋki} (der. of \TClink[1]{thoŋka}, \TClink[1]{-i}, see \TClink[1]{thoŋka}) 

\TCheadword{thoŋkini} (der. of \TClink{thoŋki} (der. of \TClink[1]{thoŋka}, \TClink[1]{-i}), \TClink{-ni}, see \TClink{-ni}) 

\TCheadword{thoŋku} \textit{cf}: \TClink{bolo}, \TClink{chocho}, \TClink{kɔŋko}, \TClink{nɔtɔ}, \TClink{suk}. \textit{n} (wɔ/hã, N) seashell type, whitish round shell, smaller than \textit{bolo} and bigger than \textit{suk} (\citealt{Pichl1967}). 

\TCheadword{thookɛ} (der. of \TClink{thok}, \TClink[1]{ɛ}, see \TClink{thok}) 

\TCheadword{thooth}\textit{n} [thóóth] bird species, bird that appears near dusk, keeping a step ahead (nightjar) (K dialect). 

\TCheadword{thosi} \textit{v} cut off branches.

\TCheadword{thoth} \textit{n} [thòth] vine species, leaves used for medicine (K dialect). 

\TCheadword{thotho} \textit{cf}: \TClink[2]{pa}. \textit{n} \textbf{1)} small sore, contrasts with \textit{pa} which is larger (K dialect). \textit{Nrɔmdɛ ma yemandɛ pɔ bɛ ko thotho mɔɛ, ma ma bɛ ko thotho thɔm mɔ.} The medicine that you don't want to be put on your sore, do not put it on the sore of your friend (proverb). \textbf{2)} wound, bruise, sore spot (\citealt{Pichl1967}).

\TCheadword{thow} \textit{n} (kɔ/ma) mushroom species, large mushroom (\citealt{Pichl1967}). 

\TCheadword{thoyaŋ} \textit{n} [thòyán] plant species, like lily plant, used for food and medicine (K dialect); \textit{thɔyan} (kɔ/ma) common plant in villages and towns, its berries used as powerful medicine against itching (\citealt{Pichl1967}).

\TCheadword{thɔ} \textit{n} (hɔ̃/tha) charpenter's adze (\citealt{Pichl1967}). \textit{Sese theyɛn-nɛki, thɔ lɛ kəth wɔ yenwɛy.} Sese hurt himself, the adze cut him badly (\citealt{Pichl1967}).

\TCheadword{thɔi} \textit{cf}: \TClink{thamir}. \textit{v} \textbf{1)} drop. \textbf{2)} drip.

\TCsubword{thuniɛni} (comp.) \textit{cf}: \TClink[1]{gbit}. \textit{v} [thúníɛ́nì] eat food dropped on the ground as mad people do or children (K dialect).

\TCheadword[1]{thɔk} \textit{v} \textbf{1)} [thɔ̀k] wash, e.g., clothes, but [thɔ́k] wash! (imperative is H, other contexts L) (K dialect). \textit{Mɔ́ má thɔ̀k gbèŋ.} You will wash them tomorrow. \textit{Hã ye tipɛ bue isuŋ doki hã hɔ thɔk hã sotho ihyɛl.} Then they began to dig the sand there, and they washed it to get salt (\citealt{Pichl1967}). \textit{Ŋ kɔ thɔk kothanthi lo, ŋkoŋ gbo ŋkɔma tha.} Go wash these clothes and when you have finished, go iron them (\citealt{Pichl1967}). \textbf{2)} wash away. \textit{Mbi hã thɔk kafa-m dɛ gbi.} You will wash away all my sins (\citealt{Pichl1967}). 

\TCsubword{thɔn} (der.) \textit{v} wash, bathe. \textit{Yɛ̀ kóŋ thɔ̀n dɛ̀, wɔ̀è bání kùáɛ́ njáláí.} After bathing she rubbed oil on her skin. der. \TClink{thɔndɛ} (see \TClink[1]{thɔk}), \TClink{thɔni} (see \TClink[1]{thɔk})

\TCsubword{thɔndɛ} (der.), (der. of \TClink{thɔn}) \textit{n} bathing.

\TCsubword{thɔni} (der.), (der. of \TClink{thɔn}) \textit{v} wash oneself. \textit{Bìmsɛ̀ ŋà thɔ́ní.} The porpoises wash themselves (because they go up and down in the water as they swim). 

\TCheadword[2]{thɔk} \textit{n} \textbf{1)} [thɔ̀k], [tòŋk] tree (K dialect). \textit{Thɔk bomdɛ kɔ lɔ vɛ ni che lɔ kɔ tɔn.} The big tree that is there, she should go there and sing. \textit{Hã bue thɔk lɛ hã hã sol wɔm.} They hollowed the tree to make a canoe (\citealt{Pichl1967}). \textbf{2)} [thɔ̀k] stick (K dialect). \textit{Hã buŋ wɔ ka thɔk.} They flogged him with a stick (\citealt{Pichl1967}). \textit{Thɔk kith lɛ, thɔk lɛ kɔ kith.} The stick is short (\citealt{Pichl1967}). \textbf{3)} wooden cross. \textit{Pɔ baŋ wɔ ko thɔkɛ, pɔ chu wɔ wɔn kumbɛ.} They nailed him on the cross, they stabbed him on his side. \textbf{4)} branch. \textbf{5)} stalk. comp. \TClink{sɛthɔk} (see \TClink{sɛɛ}) 

\TCsubword{gbɛthɛk} (comp.) \textit{cf}: \TClink{gbɛɛtigbɛɛti} (der. of \TClink[1]{gbet}, \TClink[1]{-i}). \textit{n} (hɔ̃/tha) bat for beating the washing (\citealt{Pichl1967}). 

\TCsubword{thɔkbol} (comp.) \textit{n} stick to loosen braids.

\TCsubword{thɔkihɔɔlɔŋ} (comp.) \textit{n} [thɔ̀kìhɔ́ɔ́lɔ́ŋ] tree species, bark used for malaria (lit. life tree) (K dialect). 

\TCsubword{thɔŋkanai} (comp.) \textit{n} [thɔ̀ŋkànáí] tree species, hardwood, used for firewood (K dialect). 

\TCheadword{thɔkbol} (comp. of \TClink[2]{thɔk}, \TClink[1]{bol}, see \TClink[2]{thɔk}) 

\TCheadword{thɔkihɔɔlɔŋ} (comp. of \TClink[2]{thɔk}, \TClink[2]{hɔlɔŋ} (comp. of \TClink[2]{hɔl}), see \TClink[2]{thɔk}) 

\TCheadword{thɔkɔtokgbemɔ} (comp. of \TClink{gbem}) 

\TCheadword{thɔlɛ} (der. of \TClink[1]{ahɔl}) 

\TCheadword{thɔli} \textit{cf}: \TClink[2]{pem}, \TClink{thɔnthɔ}. \textit{v} \textbf{1)} keep silent. \textbf{2)} be quiet.

\TCheadword{thɔm} \textit{n} \textbf{1)} friend. \textit{Nɛɛ kufə thɔm wɔ lɛ kɔ na lɛy lɛ.} He furtively stole the trousers of his friend while paying him a visit (\citealt{Pichl1967}.) \textit{Nchen nhã fothok thɛm mɔ nɔthi mbol.} You shall not calumniate your friends (\citealt{Pichl1967}). \textbf{2)} companion. \textit{Lɛ mɛlɛn gbo ŋke̹n, thoma mɔ lɛ ve̹lɛŋ ræ lɛ.} If you let yourself go, your companion will surpass you in the studies (\citealt{Pichl1967}). \textit{Wááŋmàɛ̀ bàmá thɔ̀mwɛ̀.} The girl lied about her companion. \textbf{3)} mate. \textit{Tha ika che siŋ, iŋa boniɛ, isiŋ ni athɔma hiɛ.} That is what we used to play, hide-and-seek, we played with our mates. \textbf{4)} [lithɛm] friendship (\citealt{Pichl1967}). \textbf{5)} love. \textit{I huni ko ja gbisiŋdɛ, yɛ pɔ panth li thɛmdɛ, ŋɔ nkela ja kache ɛ ni kenɛkiɛ?} Let us now come to the tying of love (i.e., marriage), how they used to engage couples, what was it like in the past, and nowadays?

\TCsubword{thɛmba} (unspec.) \textit{n} \textbf{1)} friend. \textbf{2)} friendship.

\TCsubword{thɛmkɔ} (unspec.) \textit{n} mate. \textit{Kɛ mi mbiɛni themkɔ nye?} But do not you have a mate? \textit{Thɛmko atiŋ ha ka che yɛ we.} Once there were two mates.

\TCheadword{Thɔmɔs} \textit{nam} Thomas, male name given to a person.

\TCheadword{thɔn} (der. of \TClink[1]{thɔk}, \TClink[2]{-n}, see \TClink[1]{thɔk}) 

\TCheadword{thɔndɛ} (der. of \TClink{thɔn} (der. of \TClink[1]{thɔk}, \TClink[2]{-n}), see \TClink[1]{thɔk})

\TCheadword{thɔndɔ} [thɔ̀ndɔ̀] \textit{cf}: \TClink{thɛŋk}, \TClink{thɔŋhul}. \textit{v} keep. \textit{Nɔ̀mààɛ̀ thɔ̀ndɔ̀ mmɛ̀ndɛ̀ bàbɔ́ŋdàì.} The woman keeps water in a large jar.

\TCheadword{thɔni} (der. of \TClink{thɔn} (der. of \TClink[1]{thɔk}, \TClink[2]{-n}), \TClink{-ni}, see \TClink[1]{thɔk})

\TCheadword{thɔnthɔ} \textit{cf}: \TClink[2]{pem}, \TClink{thɔli}. \textit{v} [thɔ́nthɔ́] calm a child (K dialect). 

\TCheadword{thɔŋhul} \textit{cf}: \TClink{thɔndɔ}. \textit{v} \textbf{1)} keep. \textit{Ŋ thɔ́ŋklɔ̀ mí yènchɛ́k àsə̀kə́l.} Keep the dried (smoked) fish for me. \textit{Yema tɔ̀ŋklɔ́ mì yenchɛk àsə̀kə́l.} Yeama kept the dried fish for me. \textbf{2)} reserve. \textit{Tondɛ kɔ lɛ ituɛ kunɛ, mɔ kɔi kɔ thɔŋgul ŋa paŋdɛ.} The small bit that remains in the pot, you reserve it for the evening.

\TCheadword{thɔŋkanai} (comp. of \TClink[2]{thɔk}) 

\TCheadword{thɔŋkaŋ} \textit{cf}: \TClink{thɛŋgbɛŋ}. \textit{n} [thɔ̀nkáŋ] bat species (K dialect). \textit{Thɛngbɛŋ velni thɔnkaŋ kə wɔ ton chaŋ thɔnkaŋ.} The \TClink{thɛŋgbɛŋ} resembles the \textit{thɔnkaŋ}, but it is smaller than the \textit{thɔnkaŋ} (\citealt{Pichl1967}). 

\TCheadword[1]{thɔŋpaŋ} \textit{cf}: \TClink{biŋkinchin}, \TClink[1]{koŋkbo} (comp. of \TClink[4]{bol}). \textit{n} (wɔ/hã, si) beetle species, synonym for \textit{kɔŋkbo} (a beetle) (\citealt{Pichl1967}). 

\TCsubword[2]{thɔŋpaŋ} (id.) \textit{cf}: \TClink[2]{bɛŋk} (id. of \TClink[1]{bɛŋk}), \TClink[2]{koŋkbo} (id. of \TClink[1]{koŋkbo}), \TClink{nɔyilɔ} (comp. of \TClink{nɔ}, \TClink[1]{yil}). \textit{n} (wɔ/hã, si) drunkard (\citealt{Pichl1967}). 

\TCheadword{Thɔɔki} (Eng \textit{Turkey}) \textit{nam} Turkey, name given to a place. \textit{Simi-njɛm bo̹m hɔ̃ kong duk Sayprɔs Agriika lɛ thiye̹ng aña Thɔɔki lɛ.} A big misunderstanding has been created (befallen) in Cyprus between the Greeks and the Turks (\citealt{Pichl1967}). 

\TCheadword[1]{thɔsuŋ} \textit{v} cough. \textit{Thɔsuŋ dɛ hɔ mi, chɔli lo ya thɔsuŋ.} I have a cough, the whole night I was coughing (\citealt{Pichl1967}).

\TCsubword[2]{thɔsuŋ} (der.) \textit{n} cough. \textit{Thɔsuŋ dɛ hɔ mi, chɔli lo ya thɔsuŋ.} I have a cough, the whole night I was coughing (\citealt{Pichl1967}).

\TCheadword[1]{thɔth} \textit{n} \textbf{1)} [thɔ̀th] buttocks (K dialect); [tɔ̀t]/ [tɔ̀t(ə)thɛ́] butt/ the butts (B dialect). \textbf{2)} stump. \textit{Ye hã bɛthi bol wɔ lɛ hɔ lee thɔt lɛ.} When they cut off the top of the tree, there is the stump which remains (\citealt{Pichl1967}). 

\TCsubword{thɔthboot} (comp.) \textit{cf}: \TClink{bolkathil} (comp. of \TClink[1]{bol}, \TClink[1]{kathil}). \textit{n} stern (of a boat). 

\TCsubword[2]{thɔth} (der.) \textit{adv} proportionally, ‘like butts when seated.' \textit{Ni bai ko, pɔ lɔ chɛli fe kasaŋ-keɛ ŋɔ leeɛ thɔth.} In the court bari, they are arranging the funeral money (contributions) proportionally.

\TCheadword{thɔthboot} (comp. of \TClink[1]{thɔth}, \TClink{bot}, see \TClink[1]{thɔth}) 

\TCheadword{thɔthɔ} \textit{n} oil palm. \textit{Bi pɛ gadin bom, gadin nthɔthɔɛ.} He also has a big garden, an oil-palm garden.

\TCheadword{Thɔzde} \textit{nam} Thursday.

\TCheadword{thri} (Eng \textit{three}) \textit{cf}: \TClink[1]{ra}. \textit{Numb} three. \textit{A mɛkɛni klas thri.} I stopped at class three.

\TCheadword{thu} \textit{v} spit.\textit{ N thu mango lɛ!} Spit out the mango! (\citealt{Pichl1967}).

\TCsubword{thuilath} (comp.) \textit{v} spit.

\TCsubword{futhul} (der.) \textit{v} spit. \textit{Yɛ bi ni mfuthul mi-a?} Why do you spit on me? (\citealt{Pichl1967}).

\TCheadword{Thua} \textit{nam} [thúá] Thua, male name given by Poro Society (K dialect). 

\TCheadword{thubi} \textit{cf}: \TClink[3]{nya}. \textit{v} poor. \textit{Ŋa hanɛ ŋa thubi yɛ, ni hanɛ ŋa biɛn dɛ o.} For the poor and for the needy.

\TCheadword{thugba} (Port \textit{tubo} ‘pipe') \textit{n} cannon.

\TCheadword{thuilath} (comp. of \TClink{thu}, \TClink[2]{lath}, see \TClink{thu}) 

\TCheadword{thuk} \textit{v} [thùk] warm (K dialect). \textit{Mən dɛ ma thuk.} The water is warm (\citealt{Pichl1967}). 

\TCsubword{thuk-thuk} (der.) \textit{cf}: \TClink[1]{dri}. \textit{adj} hot.

\TCsubword[1]{thukul} (der.) \textbf{1)} \textit{n} heat. \textbf{2)} \textit{n} urgency. \textbf{3)} \textit{v} sweat.

\TCsubword[2]{thukul} (der.) \textit{cf}: \TClink[1]{yenwɛi} (comp. of \TClink[1]{yen}, \TClink[1]{wɛi}). \textit{adj} feverish; ill. comp. \TClink{gbolnthuk} (see \TClink{gbɔl}), der. \TClink{thukuli} (see \TClink{thuk})

\TCsubword{thukuli} (der.), (der. of \TClink[2]{thukul}) \textit{v} warm. \textit{Ya ka ni hani santhɛ, isɔ bul akoŋ thukuli jomi kusɛ ayema kɔ jo…} When I had grown up, one morning after I had just warmed my rice and wanted to eat it… comp. \TClink{gbɔlthukul} (see \TClink{gbɔl})

\TCheadword{thuka} (der. of \TClink[3]{thunɔ} (der. of \TClink[1]{thunɔ}), \TClink{-k}, see \TClink[1]{thunɔ})

\TCheadword[1]{thukul} (der. of \TClink{thuk}, \TClink{-ul}, id. of \TClink[1]{ho}, see \TClink{thuk}) 

\TCheadword[2]{thukul} (der. of \TClink{thuk}, \TClink{-ul}, see \TClink{thuk})

\TCheadword{thukuli} (der. of \TClink[2]{thukul} (der. of \TClink{thuk}, \TClink{-ul}), \TClink[1]{-i}, see \TClink{thuk})

\TCheadword{thul} \textit{n} (lɔ/ma) raffia, raffia-straw (Raphia vinifera) (\citealt{Pichl1967}). \textit{Bɔ wɔ lɛ hɔ bɛmpaka lithul}. His hat is made of raffia-straw (\citealt{Pichl1967}). \textit{Thulli-kən dɛ kəlɛng chang thulli poth lɛ.} The Ken raffia is finer than the Pot raffia (\citealt{Pichl1967}).

\TCheadword{thum} \textit{n} (wɔ/hã, si) shark (\citealt{Pichl1967}). \textit{Thumsi lɛ hã gbe̹rgbe̹r.} Sharks are of many kinds (\citealt{Pichl1967}). 

\TCsubword{thumbiɔlɔ} (comp.) \textit{n} (wɔ/hã, si) shark species (\citealt{Pichl1967}). 

\TCsubword{thumgbel} (comp.) \textit{n} (wɔ/hã, si) shark species, leopard shark (\citealt{Pichl1967}). 

\TCheadword{thumbiɔlɔ} (comp. of \TClink{thum}) 

\TCheadword{thumgbel} (comp. of \TClink{thum}, \TClink[1]{gbel}, see \TClink{thum}) 

\TCheadword{thumɔ} \textit{n} [thúmɔ́] fishing net used by women, round with a stick on the upper edge (K dialect). 

\TCheadword{thumɔɛ} \textit{n} dog, [thùmɔ̀y]/ [thùmɔ̀ysɛ̀] dog/ dogs (B dialect); (wɔ/hã, si) dog (\citealt{Pichl1967}). \textit{Thumɔɛ lɛ gbos.} The dog barks (\citealt{Pichl1967}). \textit{Thumɔɛ lɛ wɔ pikïth lo̹m wɔ lɛ.} The dog wags his tail (\citealt{Pichl1967}). comp. \TClink{miliŋdithumɔɛ} (see \TClink{miliŋ}) 

\TCsubword{thumɔi-thɔkɔ} (comp.) \textit{n} hunting dog.

\TCheadword{thumɔi-thɔkɔ} (comp. of \TClink{thumɔɛ}, \TClink{thok}, see \TClink{thumɔɛ}) 

\TCheadword{thumsɔn} \textit{v} be measured.

\TCheadword{thun} \textit{n} medicine for searching.

\TCheadword{thuni} \textit{v} smell.

\TCheadword{thuniɛni} (comp. of \TClink{thɔi}) 

\TCheadword[1]{thunɔ} \textit{cf}: \TClink[2]{gbeŋgben}, \TClink{gbɛlɛŋ}, \TClink[1]{lɛli} (comp. of \TClink[3]{lɛ}). \textit{v} [thúnɔ́] search and find (something), seek (K dialect). \textit{Ŋ kɔ thunɔ nyik mam dɛ.} Go look for my keys (\citealt{Pichl1967}).

\TCsubword[2]{thunɔ} (der.) \textit{v} marry. \textit{Pɔ che thunɔ kaa fe gbe kɛ ŋa kache thunɔ ka apokas ŋaɛ hɔɛ.} They do not marry with plenty of money, but they listened to their husbands. der. \TClink{thuka} (see \TClink[1]{thunɔ}), unspec. \TClink[2]{thunɔ} (see \TClink[1]{thunɔ})

\TCsubword{thuka} (der.), (der. of \TClink[2]{thunɔ}) \textit{cf}: \TClink[2]{gbisiŋ}, \TClink[2]{path}. \textit{v} marry.

\TCsubword[3]{thunɔ} (der.), (unspec. of \TClink[2]{thunɔ}) \textit{n} price paid by a family to finalize a marriage (K dialect). \textit{...boya ni ŋa pa thunɔ waaŋmaaɛ huɛ bullɛ vɛ gbi.} ...the engagement gift, then they paid the dowry at once. \textit{Pɔ nɔi koŋ ka inshɔ, tɛmdɛ vɛ pɔ nɔi hɔm lɛ, haŋ ha thunɔ thaozin waŋ.} They would have given assurances, when they tell you the bride price is ten thousand.

\TCheadword{Thunthun} \textit{nam} Thunthun Society, the cannibals, cf. Leopard Society. \textit{Ŋa, kache, ŋa ja, rɛdilɛ, rɛdilɛ, Nthunthundɛ.} They used to do cannibalism, cannibalism, the Thunthun society.

\TCheadword{thuŋ} \textit{v} \textbf{1)} stink. \textbf{2)} smell bad.

\TCsubword{thuŋkan} (der.) \textit{v} [thùŋkàn] offend (K dialect).

\TCheadword[1]{thuŋk} \textit{cf}: \TClink[2]{bian} (der. of \TClink[1]{bian}). \textit{adj} deep, [thùùŋk] deep (K dialect). \textit{Hial lɛ kɔ thunk.} The river is deep (\citealt{Pichl1967}).

\TCheadword[2]{thuŋk} \textit{cf}: \TClink{kɔna}, \TClink{sɔku}. \textit{n} \textbf{1)} small area for storage, especially for secret stuff (K dialect); \textit{thunk-Yase, thunk-Bondo, thunk-Toma} the secret dark part of a house where the medicines or idols are kept (\citealt{Pichl1967}). \textit{Mma kɔ thunk Yase l'ay, lɛ nchen bo̹l lɛ Yase-nɔ, chen bɔ vɛ bi hã kɔnth mɔ.} Don't go into the Yase nook if you are not a member of the Yase; otherwise it will catch you (\citealt{Pichl1967}). \textbf{2)} corner; nook. \textit{Mbas thunk l'ay charaŋ!} Sweep the corner clean! (\citealt{Pichl1967}). 

\TCheadword{thuŋkan}(der. of \TClink{thuŋ}, \TClink[2]{-n}, see \TClink{thuŋ}) 

\TCheadword{thuthu} [thúthú] \textit{n} rat species (K dialect). 

\TCheadword{thuu} [thúú] \textit{v} measure (K dialect). \textit{Ŋ kɔ tuu ibəl lɛ shop lɛ ahɔl ni nhã ya si bushɛl liwɔ.} Go measure the palm kernels at the shop and let me know how many bushels (there are) (\citealt{Pichl1967}). 

\end{letter}

\begin{letter}{U}

\TCheadword{-ul} \textit{v > ???} \textit{sfx} verb extension. der. \TClink{bɔsɔli} (see \TClink[2]{bɔs}), \TClink[1]{bɔsul} (see \TClink[2]{bɔs}), \TClink{gbolnthuk} (see \TClink{gbɔl}), \TClink{kunputul} (see \TClink{kun}), \TClink{nyuhul} (see \TClink[1]{nyuŋ}), \TClink{pɔmul} (see \TClink[2]{pɔm}), \TClink{puthul} (see \TClink[3]{puth}), \TClink{puthuli} (see \TClink[3]{puth}), \TClink{sonthul} (see \TClink[1]{sɔnth}), \TClink{sonthuli} (see \TClink[1]{sɔnth}), \TClink[1]{thukul} (see \TClink{thuk}), \TClink[2]{thukul} (see \TClink{thuk}), \TClink{wɔŋhul} (see \TClink{wɔŋ})

\TCheadword{uman} (Eng \textit{woman}) \textit{cf}: \TClink[2]{laa}, \TClink{maa}, \TClink{wante} (der. of \TClink[1]{waŋ}). \textit{n} woman. \textit{Wɛl, wɔn bɛpɛ ka cheɛ mared uman, wɔi pɛ cheɛ sokonɔ Bondo.} Well, she herself was a housewife, and she was also the head of the Bondo Society.

\TCheadword{Umaru} \textit{nam} Umaru, male name given to a person. \textit{Bami kacheɛ Umaru Koroma.} My father was Umaru Koroma.

\TCheadword{Usman} \textit{nam} Usman, male name given to a person. \textit{Abi Suleman Bɛndu, Usman Bɛndu, Abas Bɛndu ni Muhamɛd Bɛndu.} I have Sulaiman Bendu, Usman Bendu, Abass Bendu and Mohamed Bendu.

\end{letter}
\begin{letter}{(V)}

\TCheadword{Vajinia} \textit{nam} Virginia, female name given to a person. \textit{Vajinia Baro.} Virginia Baro.

\TCheadword[1]{ve} \textit{n} health. \textit{Yi che live ay.} Let us be healthy (\citealt{Pichl1967})

\TCsubword[2]{ve} (der.) \textit{v} be well. \textit{Mma mi sɔkba ya chen vee.} Don't disturb me, I am not well (\citealt{Pichl1967}). 

\TCsubword{veve} (der.) \textit{v} be very well. \textit{Mɔ ve?} Are you well? \textit{Veve.} Very well.

\TCheadword[1]{vee} \textit{n} [véé] bird (generic) (K dialect). \textit{Niŋgbi lɛ wɔ lɛ ve fɔnwɛy, vɛ anyin hã hɔ.} The owl is the bird of witches, so people say (\citealt{Pichl1967}). comp. \TClink{taive} (see \TClink[1]{tai}) 

\TCsubword{veebolmin} (comp.) \textit{n} (wɔ/hã, si) bird species, swallow (lit. crazy bird) (\citealt{Pichl1967}). 

\TCheadword[2]{vee} \textit{cf}: \TClink{kɔgba}. \textit{n} [véé] oyster (K dialect). \textit{Yà kɔ̀ bón véésɛ̀.} I go harvest oysters.

\TCheadword{veebolmin} (comp. of \TClink[1]{vee}, \TClink[1]{bol}, \TClink[3]{min}, see \TClink[1]{vee}) 

\TCheadword{vei} \textit{cf}: \TClink{westaim}. \textit{v} \textbf{1)} spend a long time. \textit{Wɛl, a ko lɔ vei kunɛ ton.} Well, I have taken a long time there. \textit{Veɛni ka che Bachalɔ ko.} He did not stay long and he was staying at Bachalor. \textbf{2)} delay. \textbf{3)} be a long time. \textit{La veiɛni, Bɛl Pokan dɛ, pook Bɛl Maaɛ wɔe hun ko laa wɔɛ...} Not long after that, the Rat Husband, husband of the Rat Wife came to her and said...

\TCsubword{veio} (der.) \textit{adv} too long. \textit{Nko lɔ veio?} You have been on it for too long? \textit{Ako lɔ veio.} I have been on it for too long.

\TCsubword{vɛthiɛlɛ} (unspec.) \textit{cf}: \TClink{palɛ}. \textit{temp} some time ago.

\TCheadword{veio} (der. of \TClink{vei}) 

\TCheadword[1]{vel} \textit{cf}: \TClink{gbei}, \TClink{ku}. \textit{v} \textbf{1)} be called. \textit{Gbendi abəka lɛ ni nchə ma hã veelɛ Akrio.} The descendants of the freed slaves are called Krios (\citealt{Pichl1967}). \textbf{2)} call. \textit{Pe velɛ bul-nɔ-bul.} They called one after the other (\citealt{Pichl1967}). \textbf{3)} invite. \textit{Ŋ kɔ mi velɛ Sese ni Gbana.} Go invite Sese and Gbana for me (\citealt{Pichl1967}). \textbf{4)} summon. \textit{Bɛɛ tirɛ ni ŋgbako ma tirɛ ŋae wom ha vel Kaiŋ Taso.} The town chief and the elders then summoned Kain Tasso.

\TCheadword[2]{vel} \textit{n} (wɔ/hã, si, N) fish species, grouper (Lutjanus asa) (\citealt{Pichl1967}). \textit{Ya kɔnth bo vel bomdɛ bul yai munini.} I just caught one big fish and returned.

\TCsubword{velsa} (comp.) \textit{n} (wɔ/hã, \textit{velsi asa} pl.) red grouper (\citealt{Pichl1967}).

\TCsubword{velthi} (comp.) \textit{n} (wɔ/hã, \textit{velsi athi} pl.) black grouper (\citealt{Pichl1967}). 

\TCheadword[1]{veleŋ} \textbf{1)} \textit{post} behind. \textbf{2)} \textit{adp} after. \textit{Kɛ yendɛ ŋɔ bi lanɛvɛ, velen thilandɛ dɔktaɛ wɔ ka ŋa wɔ ɔpreshɔndɛ ka hun.} But the reason for that one, after all that, the doctor who did his operation came. \textit{Ŋa ka mu hunɔn mu, ŋa bia hundɛ, hin velendɛ...} Those that have not come yet, that are going to come after us... \textit{Ŋan bɛ, lɛ lagbandɛ wɔ gbo hun nɛn veleŋ ni ŋan bɛ ŋa shiɛ lɛ ahin ŋa lɔ ka ŋa ŋan.} Them! When the man comes next year, let them know we are here for them. \textbf{3)} \textit{post} back. \textit{Lanɛ gbi la haŋ hin Abolomaɛ ile veleŋ.} That will make us, the Sherbro, remain behind. \textbf{4)} \textit{Loc} outside.

\TCsubword[2]{thiveleŋ} (comp.) \textit{n} behind. \textit{Hanɛ ha bia kɔ hundɛ hin thivelen yɛi bia koŋ chaŋdɛ, ŋan gbi haŋa hɔŋɔ lɛŋ.} Those that will be coming behind us when we shall have past, I am greeting all of them.

\TCsubword[2]{veleŋ} (der.) \textit{cf}: \TClink[1]{baom}, \TClink[1]{bɛn}. \textit{n} ancestry. \textit{Ve̹le̹ŋ mɔ lɛ...} Your ancestry... (\citealt{Pichl1967}). 

\TCsubword{gbaŋkveleŋ} (unspec.) \textit{n} back door.

\TCheadword[1]{velia} \textit{v} \textbf{1)} [vélíà] redeem, liberate (K dialect); redeem, pardon, \textit{awoka velia}/\textit{velia wonɔ} liberated slaves/redeem a slave (\citealt{Pichl1967}). \textit{Nchoŋmalen ma chiɛ wɔ hã hun hi velia.} It is love that brought him to redeem us (\citealt{Pichl1967}). \textbf{2)} pardon. \textit{Mɔ ka velia me / mɔ velia-m.} You have given me pardon (\citealt{Pichl1967}). \textbf{3)} save, rescue. \textit{Lan gbi, velia mi yo we.} Despite all that, he rescued me e-e. \textit{Velia mi yo Jizɔs velia mi we.} Save me-o, Jesus, save me-e!

\TCsubword[2]{velia} (der.) \textit{n} (kɔ/-) redemption, pardon (\citealt{Pichl1967}). 

\TCheadword{velni} \textit{cf}: \TClink{fɛtɛ}. \textit{v} resemble. \textit{Yu lo ve̹ni ve̹lsɔk.} This fish ressembles a \textit{velsok} (\citealt{Pichl1967}). 

\TCheadword{velsa} (comp. of \TClink[2]{vel}, \TClink[1]{sa}, see \TClink[2]{vel}) 

\TCheadword{velthi} (comp. of \TClink[2]{vel}, \TClink[1]{thi}, see \TClink[2]{vel}) 

\TCheadword{ver} \textit{n} \textbf{1)} contribution. \textbf{2)} share.

\TCheadword{vethi} \textit{cf}: \TClink{bɛmpa}. \textit{v} help. \textit{Kache pabondɛ mbowɔni nwoth mɔi wɔ hu mi vethi.} In the past, if you met someone with (multiple) loads, you (would) say come help me (e.g., get this on my head). \textit{Wɔi wɔ, mi, nchi a hun mɔ hothɔ.} He would say, no, mother, let me help you.

\TCheadword{veve} (der. of \TClink[1]{ve}) 

\TCheadword{vɛ} \textbf{1)} \textit{dem} that. \textit{Wɔn kɛndɛ vɛ wɔ asɔthɔ bo prɔblɛm.} That is the only problem I had. \textit{Siŋthɛ thavɛ tha yaŋ akache siŋdɛ.} So those are the games I used to play. \textbf{2)} \textit{dem} those. \textbf{3)} \textit{adv} so. \textit{Nïŋgbi lɔ wɔ lɛ ve fɔnwɛy, vɛ anyin hã hɔ}. The owl is the bird of witches, so people say (\citealt{Pichl1967}). \textit{Vɛ la yɛ} It is so. \textit{Vɛ la yɛ?} Is it so? (\citealt{Pichl1967}) \textbf{4)} \textit{adv} thus. \textbf{5)} \textit{cop} be. \textit{La kong gbo vɛ lɔntha.} It is all finished (\citealt{Pichl1967}). 

\TCheadword[1]{vɛɛ} [vɛɛ] \textit{n} [vɛ̀ɛ̀] thorn (K dialect); \textit{vɛ} (hɔ̃/tha) thorn (\citealt{Pichl1967}). 

\TCheadword[2]{vɛɛ} \textit{v} [vɛ̀ɛ̀] stone (K dialect); \textit{vɛɛ} stone (\citealt{Pichl1967}); \textit{vɛ} stone (\citealt{Sumner1921}). \textit{Mma mi vɛ!} Don't throw (stones) at me! (\citealt{Pichl1967}). \textit{Yɛ pə ka vɛɛ wɔ theeni wu lɛ.} When they stoned him, he saw himself die (\citealt{Pichl1967}). 

\TCheadword[3]{vɛɛ} \textit{n} (hɔ̃/-) scraps of food (\citealt{Pichl1967}). \textit{Thumɔɛ lɛ wɔ telɛ vəə lɛ, apuma lɛ pɛ hã yema.} the dog is waiting for the scraps, the children, too, want them (\citealt{Pichl1967}).

\TCheadword{vɛɛthɛɛ} \textit{temp} one to six months (K dialect). 

\TCheadword{vɛkɛth} (der. of \TClink[1]{wɔk}) 

\TCheadword{vɛlvɛl} \textit{cf}: \TClink{Januari}. \textit{nam} [vɛ̀lvɛ̀l] month of January (K dialect). 

\TCheadword{vɛmp} \textit{n} (wɔ/hã, N) large sea snail, shell used to skim palm oil (\citealt{Pichl1967}). 

\TCheadword{vɛlɛni} \textit{v} face.

\TCheadword{vɛŋkɛni} \textit{v} \textbf{1)} happen. \textbf{2)} meet.

\TCheadword{vɛsɛksɔk} (comp. of \TClink{sɔk}) 

\TCheadword{vɛsiɛ} \textit{v} [vɛ́síɛ́] scratch, search for food like a hen (K dialect). 

\TCheadword{vɛthiɛlɛ} (unspec. of \TClink{vei}) 

\TCheadword{vidio} \textit{n} video.

\TCheadword{viki} \textit{v} \textbf{1)} [víkí] straighten (K dialect). \textbf{2)} stretch (\citealt{Sumner1921}). \textit{A kɔ viiki bɛŋthi-m dɛ.} I go to stretch my legs, i.e. I go for a walk (\citealt{Pichl1967}).

\TCsubword{vikini} (der.) \textit{v} stretch oneself. \textit{A yema vikini kɛ hinth lo kɔ kith hã yang.} I want to stretch but this bed is too short for me (\citealt{Pichl1967}). 

\TCheadword{vikin} \textit{n} mooring line.

\TCheadword{vikini} (der. of \TClink{viki}, \TClink{-ni}, see \TClink{viki}) 

\TCheadword[1]{vil} \textit{adj} \textbf{1)} long. \textit{Nai wɛ ŋɔ vil ni ŋɔ chɔɔlen mɔnɛ ni sɔan ma lɔ.} The road is long and it is difficult and there are many temptations. \textit{Braima koŋ haa lanɔ ki ha gbaath vil.} Braima had done this for a long time. \textbf{2)} distant. \textit{Anya hi-ɔ hã ka cheɛ pɔk livil.} Our people were living in a distant country (\citealt{Pichl1967}). \textit{Kɛnth kɔ lɔ livil.} Kent is far away from here (\citealt{Pichl1967}). \textbf{3)} tall; high. \textit{Kɔŋ wɔ vil.} Kong is tall (\citealt{Pichl1967}). \textit{Thɔk lɛ kɔ vil.} The tree is high (\citealt{Pichl1967}). comp. \TClink{kɔluŋ-vil} (see \TClink[4]{kɔ}) 

\TCsubword[2]{vil} (der.) \textit{adv} far. \textit{Aa kɛ akɔni livil.} Yes, but I did not go far. \textit{Braima chen kɔni livil ha kɔ lɔɔli pɛl yɛllɛɛ hiŋk bondɔ ko.} Braima will not go far from the wharf to inspect leggo chains.

\TCsubword{vilvil} (der.) [vílvíl] \textit{adj} very tall, [thɔ̀k mvílvíllɛ̀]/ [thɔ̀k ŋkìthkìthɛ̀] very tall tree/ very short tree (K dialect). 

\TCheadword{vila} \textit{cf}: \TClink{nyoŋkni}. \textit{v} [vílá] wither, be about to die (K dialect). \textit{Sæ lɛ kɔ kath, pɔmthi ŋkəfe lɛ koŋ vila.} The dry season is hard, the leaves of the peppers have withered (\citealt{Pichl1967}). 

\TCheadword{vilvil} (der. of \TClink[1]{vil}) 

\TCheadword{vinda} \textit{cf}: \TClink{viŋhil}. \textit{n} [víndà] medicine for trees or farms (traditional pesticides/herbicides) also known as \textit{viŋhil} (K dialect). 

\TCheadword{viŋhil} \textit{cf}: \TClink{vinda}. \textit{n} [víŋhíl] medicine for trees or farms (traditional pesticides/herbicides) also known as \textit{vinda} (K dialect). 

\TCheadword{vis} \textit{n} \textbf{1)} meat, [vəs]/[və́sɛ̀]/ [\`{m}və́sɛ̀] meat/the meat/the meat (pl) (B dialect). \textit{À bə́th və́sɛ̀.} I cut up the meat. \textit{À bɛ̀thí vìsɛ̀.} I'm cutting up the meat. \textit{À kóŋ bɛ̀thí vìsɛ̀.} I cut it up. \textbf{2)} [vəs] animal (K dialect). \textit{Anyin ŋa lɔ ŋan thiyeŋ ŋa thee ŋhɔk ma nvisɛ ni veesɛ.} There are people among them who understand the words of the animals and the birds. \textit{Ŋkɔ bus vis lɛ.} Go skin the animal (\citealt{Pichl1967}). comp. \TClink{huvis} (see \TClink[1]{hu}) 

\TCheadword{vuli} (Eng \textit{very}) \textit{cf}: \TClink{gboŋ}. \textit{adv} very.

\TCheadword{vunth} \textit{v} push. \textit{N vunth tamɔ lɛ, wɔnɛ bɔko gbɔw thi ka.} Push that boy outside, he makes too much noise here (\citealt{Pichl1967}).

\TCheadword{vunthu} (der. of \TClink{runth}) 

\end{letter}
\begin{letter}{W}

\TCheadword{wa} \textit{Idph} underscores flight.

\TCheadword[1]{waa} \textit{n} [wáá] palm tree (K dialect). \textit{Ŋkɔ too waa lɛ ni ŋkɔ mbəl lɛ!} Go climb the palm tree and cut the nuts! (\citealt{Pichl1967}). \textit{Tho lo bi iwa gber, kə hɔ biɛni ibach pɔsɔ.} This bush has many palm trees, but it has not many young palm trees (\citealt{Pichl1967}). \textit{Ntel lo ma ŋkəlɛŋ hã thaŋ ka wa.} This cane rope is good to climb a palm tree with (\citealt{Pichl1967})

\TCheadword[2]{waa} \textit{adj} much.

\TCheadword{waami} \textit{adj} between dry and fresh, [waami]/ [yenchɛk àwààmì] something in between dry and fresh/ not completely dry or raw (K dialect).

\TCheadword[1]{wai} \textit{n} \textbf{1)} supplies. \textbf{2)} bullet. \textbf{3)} lead; metal. comp. \TClink{sɛwai} (see \TClink{sɛɛ}) 

\TCheadword[2]{wai} \textit{cf}: \TClink[2]{sɛk}. \textit{adj} without sauce.

\TCheadword[3]{wai} \textit{interj} [wáí] expression of pain (K dialect). 

\TCheadword[4]{wai} \textit{adv} quietly; without any ado or celebration. \textit{Pɔ wɔ bo kɔ kɔŋ wai, pɔ sɛŋyɛ lɔni.} They would just bury him quietly, then everybody would go away.

\TCsubword{huaihuai} (der.) \textit{adv} quietly. \textit{Ni ŋa che tɔn, ya mɔe hɔmɛ, nthol huai-huai ni ŋkɔ kue yekeɛ ni ŋchii.} And they are singing, then I said to you, go down quietly and take the cassava and bring it (back).

\TCheadword[1]{wal} \textit{cf}: \TClink[2]{chak}, \TClink{wus}. \textit{n} \textbf{1)} \textit{wál} palm leaf (K dialect); \textit{liwal} palm leaf (\citealt{Sumner1921}). \textit{Yi kwey liwal, si yi chok lɛn ton, si yi panth lɛn do...} We take palm leaves, then we twist them to a fine line, then we tie this line... (\citealt{Pichl1967}). \textbf{2)} \textit{liwal} (lɔ/-) palm fiber used for making nets or lines (\citealt{Pichl1967}). \textit{Iwaa chen kəlɛng hã liwal, walli hoolɛ lɔ gbo kɔ ibach lɛ.} Palm trees are not good for palm fiber, palm fiber is found only among young palm-trees (\citealt{Pichl1967}). 

\TCheadword[2]{wal} \textit{n} \textbf{1)} [wàl] temporary place to live, farmhouse (K dialect). \textbf{2)} resting place (K dialect). \textbf{3)} area (\citealt{Pichl1967}). 

\TCsubword{walpɔ} (comp.) \textit{n} Poro warning area (\citealt{Pichl1967}). 

\TCsubword{yenal} (comp.) \textit{n} place. \textit{Y chɔŋ la len yɛ pɔ chaŋ theli Mbolomdɛ, bikɔs inal pim, Bolomko lɔɛ.} We like that because they speak Sherbro here more, because other places are Sherbro lands. \textit{Alanɛ la vili ayenal pim pɔ chelɔ pɛ theli Mbolom.} I believe that really, in other places they do not speak Bolom anymore.

\TCheadword{walpɔ} (comp. of \TClink[2]{wal}, \TClink{Pɔ}, see \TClink[2]{wal}) 

\TCheadword{wan} (Eng \textit{one}) \textit{Numb} one. \textit{Wan de asɔthɔni mu prɔblɛm ya lɔ gbemiɛ.} Not once have I had a problem delivering (a baby). \textit{Mi Shenge ka fli ya tipɛ klas wandɛ.} It is in Shenge here that I start class one.

\TCheadword{wantama} (comp. of \TClink{wante} (der. of \TClink[1]{waŋ}), \TClink{maa}, see \TClink[1]{waŋ}) 

\TCheadword{wante} (der. of \TClink[1]{waŋ}) 

\TCheadword[1]{wantiŋ} \textit{n} \textbf{1)} [wàntíŋ] flowers that appear on trees (K dialect). \textbf{2)} blossom (\citealt{Pichl1967}). 

\TCsubword[2]{wantiŋ} (der.) \textit{cf}: \TClink{puki}. \textit{v} blossom. \textit{Potɔhɔl lɛ koŋ moɛy, ngbemaŋ dɛ tipɛ wantiŋ.} When springtime has come, the fruit trees begin to blossom (\citealt{Pichl1967}). 

\TCheadword[1]{waŋ} \textit{cf}: \TClink{pumaama} (comp. of \TClink[3]{pum}, \TClink{nɔmaa}). \textit{n} \textbf{1)} girl, [wáŋ]/[wáŋmàà]/ [wàŋ]/[wáŋmà à-wàŋ] girl/daughter/ ten/ten girls (K dialect). \textbf{2)} daughter.

\TCsubword{waŋmaa} (comp.) \textit{n} \textbf{1)} young woman. \textit{Kaiŋ Taso ka mɔɛ tir bul, lɔ ka ke waaŋmaa kɛlɛŋ-kɛlɛŋ.} Kain Tasso reached a village where he saw a fine young woman. \textbf{2)} woman. \textit{Wanma wɔ kɔ bi nɔpikandɛ bawɔ, yawɔ che laŋ shi.} A woman would have a man without the knowledge of her father and mother. \textit{Baa waaŋmaaɛ wɔe wom ko komnɛ wɔɛ Kaiŋ Taso lɛ jajɛl wɔɛ koŋ wu.} The young woman's father sent a message to his son-in-law, Kain Tasso, that his mother-in-law had died. comp. \TClink{waŋmarɛs} (see \TClink[1]{waŋ})

\TCsubword{waŋmarɛs} (comp.), (comp. of \TClink{waŋmaa}) \textit{n} virgin.

\TCsubword{wantama} (der.), (comp. of \TClink{wante}) \textit{n} full-grown girl.

\TCsubword{wante} (der.) \textit{cf}: \TClink[2]{laa}, \TClink{maa}, \TClink{uman}. \textit{n} \textbf{1)} sister, [wàntsə́]/[wàntsə́mí] sister/my sister (B dialect). \textit{Yema si kump sampa chang awante Bue.} Yema knows better than her sister bue how to finish a basket (\citealt{Pichl1967}). \textbf{2)} woman, wife, includes female cousins as well (\citealt{Hall1938}). \textbf{3)} \textit{wanta}, \textit{wang-ta} (wɔ/hã, N) young girl (\citealt{Pichl1967}). \textit{Wanta sɛmɛ tho lɛ ve̹le̹ŋ.} The girl stands behind the bush (\citealt{Pichl1967}). \textbf{4)} young woman. \textit{Nwantɛm agbe̹r hã trï ka ni hã akəlɛŋ-kəlɛŋ.} There are many young women in this town and they are very beautiful (\citealt{Pichl1967}). comp. \TClink{wantama} (see \TClink[1]{waŋ}) 

\TCheadword[2]{waŋ} \textit{Numb} ten, [wàŋ]/[wáŋmà à-wàŋ] ten/ten girls (K dialect). \textit{Mpang nwang ni tïng man ma nɛn bul ay ɛ.} There are twelve months in one year (\citealt{Pichl1967}). comp. \TClink{mɛŋhiɔlniwaŋ} (see \TClink[1]{mɛn}), \TClink{waŋnihiɔl} (see \TClink{hiɔl}) 

\TCsubword{waŋnibul} (comp.) \textit{Numb} eleven. \textit{nduɛ waŋnimɛŋtiŋ dɛ, palpal lɛ, mɛŋkɛ ŋɔn waŋnibul lɛ.} The seventeenth day, noon, the eleventh hour.

\TCsubword{waŋnimɛnra} (comp.) \textit{Numb} eighteen. \textit{Huɛ Seiŋyɛ ŋɔ pɔ vel lɛ Flaideɛ Mpothoaiɛ, nduɛ waŋnimɛŋraɛ.} On \textit{Seiŋyɛ}, which they call Friday in English, the eighteenth.

\TCsubword{waŋnimɛŋtiŋ} (comp.) \textit{Numb} seventeen. \textit{nduɛ waŋnimɛŋtiŋ dɛ, palpal lɛ, mɛŋkɛ ŋɔn waŋnibul lɛ.} The seventeenth day, noon, the eleventh hour.

\TCsubword{waŋnitiŋ} (comp.) \textit{Numb} twelve. \textit{Mpang nwang ni tïng man ma nɛn bul ay ɛ.} There are twelve months in one year (\citealt{Pichl1967}). (\citealt{Pichl1967}).

\TCheadword{waŋmaa} (comp. of \TClink[1]{waŋ}, \TClink{maa}, see \TClink[1]{waŋ}) 

\TCheadword{waŋmarɛs} (comp. of \TClink{waŋmaa} (comp. of \TClink[1]{waŋ}, \TClink{maa}), \TClink{rɛs}, see \TClink[1]{waŋ}) 

\TCheadword{waŋnibul} (comp. of \TClink[2]{waŋ}, \TClink[3]{ni}, \TClink[3]{bul}, see \TClink[2]{waŋ}) 

\TCheadword{waŋnihiɔl} (comp. of \TClink[2]{waŋ}, \TClink[3]{ni}, \TClink{hiɔl}, see \TClink{hiɔl}) 

\TCheadword{waŋnimɛnra} (comp. of \TClink[2]{waŋ}, \TClink[3]{ni}, \TClink{mɛnra} (comp. of \TClink[1]{mɛn}, \TClink[1]{ra}), see \TClink[2]{waŋ}) 

\TCheadword{waŋnimɛŋtiŋ} (comp. of \TClink[2]{waŋ}, \TClink[3]{ni}, \TClink{mɛntiŋ} (comp. of \TClink[1]{mɛn}, \TClink[1]{tin}), see \TClink[2]{waŋ}) 

\TCheadword{waŋnitiŋ} (comp. of \TClink[2]{waŋ}, \TClink[3]{ni}, \TClink[1]{tin}, see \TClink[2]{waŋ}) 

\TCheadword{wawa} \textit{n} [ẁawà] tree species, leaves used for medicine (K dialect). 

\TCheadword{waya} (Eng \textit{wire}) \textit{cf}: \TClink{biŋ}, \TClink{hantha}, \TClink[1]{tɔŋ}. \textit{n} \textbf{1)} wire.

\TCheadword[1]{we} (der. of \TClink[1]{wɛi} (der. of \TClink[2]{wɛi}), see \TClink[2]{wɛi}) 

\TCheadword[2]{we} (Eng \textit{way}) \textit{n} way. \textit{Yi chɔŋ weɛ ŋɔ mɔ tɔndɛ lendɛ.} We like the way you sing. \textit{Ntoŋgi mi mu we ɛ ŋɔ pɔ gbisiŋdɛ ni boŋgo.} Show me the way they used to marry and nowadays.

\TCheadword{westaim} (Eng \textit{waste time}) \textit{cf}: \TClink{vei}. \textit{v} waste time. \textit{Pɔ wɔ kɔŋ paŋdɛ vɛ, pɔ che westaim.} They bury him, they do not waste time.

\TCheadword{wɛ} \textit{cf}: \TClink{gbemani}, \TClink[1]{hɔ}, \TClink[1]{lem}, \TClink{theli}, \TClink[2]{wɔni} (der. of \TClink[1]{hɔ}, \TClink{-ni}). \textit{v} say. \textit{Ya bi ŋa wɛ a chɔŋɔ mɔ sɛkɛ, Bahin.} I have to say thank you, Lord. \textit{Yɛ kon thamdɛ, laŋgbaɛ wɛ ma pɛ lo sampa the bikɔs koŋ tham.} When she was old enough, the man said she should stop weaving baskets because she had become old. \textit{Iwɛ awa ŋa jo, thɔkɛ ma ikaɛ.} We say ok eat, we give to the tree.

\TCheadword[1]{wɛi} \textit{cf}: \TClink[1]{kafa}, \TClink{kɛnda}. \textit{n} \textbf{1)} evil. \textit{Yaŋ ya pəkɛ gbo iwɛi.} I am (truly) filled with evil (\citealt{Pichl1967}). \textbf{2)} wickedness. comp. \TClink{jawɛi} (see \TClink[1]{ja}), \TClink{lomɔfɔnwɛy} (see \TClink{lomɔ}), \TClink{nɔncheŋwɛi} (see \TClink{nɔ}), \TClink{nɔfɔnwɛi} (see \TClink{nɔ}), \TClink{theɛhwɛ} (see \TClink{the}), \TClink[1]{yenwɛi} (see \TClink[1]{yen}), \TClink[2]{yenwɛi} (see \TClink[1]{yen})

\TCsubword[2]{wɛi} (der.) \textit{adj} \textbf{1)} bad. \textit{ndum ŋwɛiɛ} bad training (parenting), because a child asked what was in a wrapped parcel. \textit{Mbo̹lo̹m ŋwɛi ma che paalɛ bai ko, anya atïŋ dɛ hã lo̹l.} In the bad case that was recently before the court, the two men were set free (\citealt{Pichl1967}). \textbf{2)} ugly. \textit{Chɛliɛ mi tɛn wɛy ya che kɔn pɔkɔni.} He created a bad situation for me, I shall not forget it (\citealt{Pichl1967}). comp. \TClink[1]{fɔnwɛi} (see \TClink[2]{wɛi}), \TClink{hɛŋwɛi} (see \TClink[2]{wɛi}), \TClink{hwɛwɛi} (see \TClink[2]{wɛi}), \TClink{jawɛi} (see \TClink[1]{ja}), \TClink{lomɔfɔnwɛy} (see \TClink{lomɔ}), \TClink{nɔncheŋwɛi} (see \TClink{nɔ}), \TClink{nɔfɔnwɛi} (see \TClink{nɔ}), \TClink{nɔwɔi} (see \TClink[2]{wɛi}), \TClink{theɛhwɛ} (see \TClink{the}), \TClink[1]{yenwɛi} (see \TClink[1]{yen}), \TClink[2]{yenwɛi} (see \TClink[1]{yen}), der. \TClink[1]{we} (see \TClink[2]{wɛi}), \TClink[1]{wɛini} (see \TClink[2]{wɛi}), \TClink[2]{wɛini} (see \TClink[2]{wɛi})

\TCsubword[1]{fɔnwɛi} (der.), (comp. of \TClink[1]{wɛi}) \textit{cf}: \TClink{humoe}, \TClink{mane}, \TClink[3]{wɔm}, \TClink{yasi}. \textit{n} \textbf{1)} witchcraft. \textit{Nɔɛ wɔ hu ni la hɔndɛ wɔ fɔnwɔiɛ, pɔ che bia ha thisiŋ pɔ che mɛmini}. The person died and it was proven that he was a witch, they could not celebrate, they were not happy. \textbf{2)} witch. \textit{Lɛ la toŋgiɛ lɛ nɔ ki wɔ fɔnwɔi, kunɛ igbeth ka cheni tiŋ-tiŋ ki athɔma wɔ}. If it showed that the person was a witch, dirty-belly, he was not straightforward among his fellow men. \textbf{3)} potion, medicine. comp. \TClink[2]{fɔnwɛi} (see \TClink[2]{wɛi}), \TClink{lomɔfɔnwɛy} (see \TClink{lomɔ}), \TClink{nɔfɔnwɛi} (see \TClink{nɔ})

\TCsubword[2]{fɔnwɛi} (der.), (comp. of \TClink[1]{fɔnwɛi}) \textit{v} die as a witch.

\TCsubword{hɛŋwɛi} (der.), (comp. of \TClink[1]{wɛi}) \textit{cf}: \TClink[6]{hɔ}. \textit{n} bad weather.

\TCsubword{hwɛwɛi} (der.), (comp. of \TClink[1]{wɛi}) \textit{n} bad idea. 

\TCsubword{nɔwɔi} (der.), (comp. of \TClink[1]{wɛi}) \textit{n} bad person. \textit{Kacheɛ nɔwɔi, nɔwɔiɛ wɔ nɔfɔnwɔiyɛ.} He was a bad person, a bad person is a witch person.

\TCsubword[1]{we} (der.), (der. of \TClink[1]{wɛi}) \textit{prt} emphatic particle. \textit{So sɛkɛ we, Abatokɛ chema mɔni.} So thank you, may God be with you. \textit{Mi akɔlɔ e.} I go there. \textit{Kɛ, apa, lagbowɛwe.} Well, pa, goodbye. \textit{Mɔ́ŋá ŋálá wàì!} Be patient! \textit{Wɔ̀sòwéí.} Goodbye-o. \textit{Braima wɔe tipɛ yaath ha boŋ ihɛŋ disil wɛin dɛ sua mmɛŋ hukɔ kiai.} Braima then began to paddle to resist the dreadfully heavy winds.

\TCsubword[1]{wɛini} (der.), (der. of \TClink[1]{wɛi}) \textit{adj} dreadful.

\TCsubword[2]{wɛini} (der.), (der. of \TClink[1]{wɛi}) \textit{adv} awfully. 

\TCheadword[1]{wɛini} (der. of \TClink[1]{wɛi} (der. of \TClink[2]{wɛi}), \TClink{-ni}, see \TClink[2]{wɛi}) 

\TCheadword[2]{wɛini} (der. of \TClink[1]{wɛi} (der. of \TClink[2]{wɛi}), \TClink{-ni}, see \TClink[2]{wɛi}) 

\TCheadword{wɛl} (Eng \textit{well}) \textit{disco} well. \textit{Wɛl, iya waŋ ni tiŋ.} Well we are twelve. \textit{Wɛl gbem apima amɛnbul.} Well, he gave birth to six children. \textit{Wɛl, wɔn bɛpɛ ka cheɛ mared uman, wɔi pɛ cheɛ sokonɔ Bondo.} Well, she herself was a housewife, and she was also the head of the Bondo Society.

\TCheadword{Wɛzde} (Eng \textit{Wednesday}) \textit{nam} Wednesday. \textit{Nante ndɔi koŋɔnɔ ni mɛndɛ, nante Wɛzde.} Today is the twenty-fifth, today is Wednesday.

\TCheadword{wik} (Eng \textit{week}) \textit{n} week. 

\TCheadword{Wilɛm} \textit{nam} William, male name given to a person. \textit{Ba mi bi ilel, ka bi ilel Piɛ Wilɛm.} My father has a name, he used to have the name Pieh William.

\TCheadword{wini} \textit{cf}: \TClink[2]{saba}. \textit{n} Poro dance.

\TCheadword[1]{wo} \textit{v} dry up. \textit{Mən dɛ koŋ wo yanɔ l'ay.} The water in the river has dried up (\citealt{Pichl1967}). 

\TCheadword[2]{wo} \textit{cf}: \TClink{gbogbotok} (unspec. of \TClink[3]{gbogbo}), \TClink{kɔm}, \TClink{maima}, \TClink{tom}. \textit{n} \textit{iwo} (hɔ̃/-) hair on the privy parts (\citealt{Pichl1967}).

\TCsubword{wokilin} (comp.) \textit{n} \textit{iwo kïlim} (wɔ/hã) kind of hairy crab (\citealt{Pichl1967}).

\TCsubword{wothaalɛ} (comp.) \textit{n} \textit{iwo thaalɛ} (wɔ/hã) kind of hairy crab (\citealt{Pichl1967}).

\TCheadword[3]{wo} \textit{v} [wó] crow as a cock crows (B dialect); [wóŋ] crow as a cock crows (K dialect). \textit{Sɔkpokan dɛ wɔ wo.} The cock crows (\citealt{Pichl1967}). 

\TCheadword[4]{wo} \textit{v} throw net. \textit{A minɛ pɛl kɔ mɔ kɔ woɛ.} I thought it was a net that you would throw.

\TCheadword[5]{wo} \textit{n} rice stalks. \textit{Iwoɛ, iwo itataɛ pɔ ŋɔ pak ayen, pɔ ŋɔ pɛ bia buŋ.} The rice grass stalks, the immature stalks are parked somewhere, people thresh them again.

\TCheadword[6]{wo} \textit{v} live.

\TCheadword[1]{woi} [woi] \textit{v} fear. \textit{Ya wɔ wei.} I fear him.

\TCsubword{woli} (der.) \textit{cf}: \TClink{hothɔk}, \TClink{jɔhɔ}, \TClink{pakali} (der. of \TClink{pakil}, \TClink[1]{-i}), \TClink{sɔyɛ}. \textit{v} \textbf{1)} [wólí] threaten (K dialect). \textbf{2)} scare. \textit{Ya chencha kɔ faka ɛ ko, nɔma lɛ wɔ kan wu lɛ lemmi woliyɛ ni yiki tha bɔ lɛ ya-m veleŋ.} (When) I went to the village yesterday, the woman who died recently followed and scared me by shaking the bushes behind me (\citealt{Pichl1967}). 

\TCheadword[2]{woi} \textit{v} call.

\TCheadword[1]{wok} \textit{cf}: \TClink{woro}. \textit{n} \textbf{1)} shooting. \textbf{2)} target.

\TCheadword[2]{wok} \textit{n} \textbf{1)} enslaved person. \textit{Haaŋ mɛŋkɛ ŋɔ Apotho aɛ ka hun dɔ chal ha pin awok aɛ...} Until the time the Europeans came to live there to buy enslaved people... \textit{Mɛŋkɛ ŋɔ Apotho aɛ ka che pin anyaɛ hiŋk Afrikaɛ, ŋà ŋá kɔ piŋkiɛ awokɛ.} At the time when the Europeans were buying people from Africa, they turned them into enslaved people. \textbf{2)} slavery. \textit{liwok} slavery. 

\TCsubword{wonɔ} (comp.) \textit{n} enslaved person. \textit{Nyema bi mi piŋki lɛ wonɔ.} You want to make me into a slave (\citealt{Pichl1967}). 

\TCheadword{woki} \textit{v} \textbf{1)} [wókí] wonder (K dialect). \textit{Nɔmaa chaɛ a, “Ya gbo woki-o-o.”} The woman sings, “I am just wondering.” \textbf{2)} sigh. \textit{Ya bɔnthɔ wɔ taŋ, wɔ wokiɛ ya wɔ.} I met him crying, he was sighing for his mother (\citealt{Pichl1967}). 

\TCheadword{wokilin} (comp. of \TClink[2]{wo}, \TClink{kilim}, see \TClink[2]{wo}) 

\TCheadword{woko} \textit{n} \textbf{1)} grass species. \textbf{2)} bow string.

\TCheadword{wokum} (Eng \textit{oakum}) \textit{n} oakum, \textit{iwokum} (\citealt{Pichl1967}). 

\TCheadword{wolɛ} \textit{n} first rice planting.

\TCheadword{woli} (der. of \TClink[1]{woi}, \TClink[1]{-i}, see \TClink[1]{woi}) 

\TCheadword{wom} \textit{n} message; greeting. \textit{So womdɛki ŋanɛ ŋa hunɔ ni muɛ} So this greeting to those that have not come yet.

\TCsubword{womnɔ} (comp.) \textit{n} messenger.

\TCheadword{womnɔ} (comp. of \TClink{wom}, \TClink{nɔ}, see \TClink{wom}) 

\TCheadword[1]{won} \textit{cf}: \TClink{baŋkbuk} (comp. of \TClink[2]{buk}). \textit{n} bush yam. \textit{won} the real bush yam.

\TCheadword[2]{won} \textit{v} add.

\TCheadword{wonɔ} (comp. of \TClink[2]{wok}, \TClink{nɔ}, see \TClink[2]{wok}) 

\TCheadword[1]{woŋ} \textit{n} \textbf{1)} [wóŋ] leech (K dialect). \textit{Woŋ dɛ wɔ tɔth nɔ yanɔ l'ay, lɛ sɛmɛ lɔ gbo.} The leech will suck a person if he is standing just (for a moment) in the stream (\citealt{Pichl1967}) \textbf{2)} fish species, black freshwater fish like a crocus, can reach up to 18 inches in length but is usually not longer than a foot, good to eat, has no whiskers (K dialect). \textit{Wóŋ wɔ̀ kì.} This is the \textit{woŋ} (when picking out the fish in a bin with others). 

\TCheadword[2]{woŋ} \textit{v} [wóŋ] curse (K dialect). \textit{Bia wɔ nche wɛy, wɔ woŋ lol thiwɛy ko ama wɔ lɛ.} Bia has bad habits. He curses his wives with bad words (\citealt{Pichl1967}).

\TCsubword{woŋhɔ} (comp.) \textit{v} abuse. \textit{Bɛl Pokan dɛ wɔe hɔ ko laa wɔɛ, “Mba, ha yeke mɔɛ la mɔ mi woŋhɔɛ?”} Rat Husband said to his wife, “Madam, is it for your cassava that you are abusing me?”

\TCheadword{woŋhɔ} (comp. of \TClink[2]{woŋ}, \TClink[1]{hɔ}, see \TClink[2]{woŋ}) 

\TCheadword{woŋkani} \textit{v} crow.

\TCheadword{woŋki} \textit{cf}: \TClink{rɛdi}. \textit{v} be ready. \textit{Kon hɔn mu woŋki.} Not ready yet. \textit{Wɔ̀ wònkí, wɔ́ wònkìɛ́ mɛ̀nk [ə] gbí.} He will be ready, he is always ready.

\TCheadword[1]{woŋko} \textit{n} \textbf{1)} house; home; his place. \textit{Wonko ŋaɛ Nthemdɛ ma pɔ lɔ theli ɔ Mbolomdɛ?} Their houses, is it Themne they speak there or Sherbro? \textit{La nbia ŋa woŋgo wɔ ko.} The things you have to do in your home. \textit{Woŋgomi ko ma lɔ kɔ nche lɔ bɔnth chiŋ, bikɔs yaŋ pɛ ayemani tiŋ.} In my house, if you go there, you will not hear any noise, because myself I do not want noise. \textit{Yɛ koŋ woth-kun dɛ pɔɛ hɔ ma gbemɔ woŋga.} After she got pregnant, she was told no delivery at home. \textit{Kɛ lɛliɛ kɔ mɛkni ko wɔko.} But examiners (spies?) would stop at his place. \textbf{2)} space. \textit{Pɔ kɔ yuk tonton, ɛn pɔ kɔ pɛ ka thiwoŋka.} People will plant a little (here and there), and people will make space. \textit{Lagbo bɔmdai lɔɛ, pɔ kɔ ŋa gbompa ton, ɛn pɔ pɛ ka thiwonka, kaŋka kɔ ma gbompa ni bɔnɔ bul.} If it (rice field) is in a swamp, they will make it (space between plants) a little greater and make spaces so it (rice seedling) can grow without being pushed into one place. 

\TCheadword[2]{Woŋko} \textit{nam} Wong Island, name given to a place. Woŋko yɛ ache paa kɔ Dema koɛ, a yema lɔ kɔ fli abo abo ŋa nkuath ŋa yaŋ kɔlɔ. When I used to go to Dema, I really wanted to go to Wong (Island), (but) I was afraid to go there. 

\TCheadword{woŋkru} \textit{cf}: \TClink[1]{fama}, \TClink[2]{ra}. \textit{v} clear farm. \textit{Koŋ woŋkru ichɛk wɔ lɛ, hɔ ka heyɛni.} He has finished clearing his farm that was never well burnt (\citealt{Pichl1967}). 

\TCheadword{woro} \textit{cf}: \TClink[1]{wok}. \textit{v} shoot at.

\TCheadword{woso} \textit{cf}: \TClink{meni}, \TClink[2]{pot}. \textit{n} herbal clay.

\TCheadword[1]{woth} \textit{cf}: \TClink{yɔk}. \textit{v} carry.

\TCsubword{wothkun} (comp.) \textit{cf}: \TClink[3]{sɛm}. \textit{v} be pregnant (lit. carry belly). \textit{Yɛ koŋ wothkun dɛ pɔɛ hɔ ma gbemɔ woŋga.} After she got pregnant, she was told no delivery at home.

\TCsubword[2]{woth} (der.) \textit{n} load. \textit{Woth disil.} Heavy load. \textit{Kache pabondɛ mbowɔni nwoth mɔi wɔ hu mi vethi.} In the past, if you met someone with (multiple) loads, you (would) say come help me (e.g., get this on my head).

\TCheadword[2]{woth} (der. of \TClink[1]{woth}) 

\TCheadword{wothaalɛ} (comp. of \TClink[2]{wo}, \TClink{thaale}, see \TClink[2]{wo}) 

\TCheadword{wothkun} (comp. of \TClink[1]{woth}, \TClink{kun}, see \TClink[1]{woth}) 

\TCheadword[1]{wɔ} \textit{pers} \textit{NCP} \textbf{1)} 3rd person singular pronoun, noun class pronoun (wɔ): she; he; her; him; hers; his; it; its. \textit{Thetha mi ka che ŋa mpanth ma landɛ pɛŋ bifo wɔ mmu hu.} My grandmother used to do the work before she died. \textit{Nɔ ndɔndɔ wɔ yema ŋa thelaɛ wɔla the, wɔlɔka gbi.} Whoever wants to hear it, hears it, throughout the whole world. \textit{Bɛyɛ wɔn ayɛnaɛ hun, hun wɔŋ injɛkshɔn, bikɔs yaŋ ache injɛk a siŋɔ ni.} The chief himself came and gave the injection because I do not know how to do it. \textit{Ya ke wɔ ma hɔl thimdɛ, ni ya bɛŋ ma wɔ pia mi njokɛ, ni ya theli ko wɔ ko.} I saw him with my eyes, and I touched him with my right hand, and I talked to him. \textbf{2)} 3rd person relative pronoun: that; who; whom; whose. \textit{Wɔ mmɛn hukɔ ni ihɛŋ disil-disil sɔsɔkɔ.} Whom heavy waves and heavy winds swept away. \textit{Kaiŋ Taso bɛ wɔ jajɛl wɔɛ wuɛ wɔ lae theeɛ.} Kain Tasso, whose mother-in-law died, heard about it.

\TCsubword{wɔki} (comp.) \textit{dem} this one. \textit{Hina wɔki a?} Who is this one? (\citealt{Pichl1967}).

\TCheadword[2]{wɔ} \textit{cf}: \TClink{futh}, \TClink{lɛnthi}, \TClink[1]{sokothi}, \TClink{suth}. \textit{v} \textbf{1)} pluck. \textbf{2)} pick. \textit{Thɔkɛ kɔi thol wandaɛ wɔi wɔ mgbemaŋdɛ wɔ yɛ jo.} And the tree came down, the girl then picked the fruit and ate.

\TCheadword[3]{wɔ} \textit{cf}: \TClink{yɛn}. \textit{interrog} \textbf{1)} how much. \textit{Mɔ nɛnthi wɔ?} How old are you? \textbf{2)} how many. \textit{Nɛn thi wɔ?} How many years? \textit{Apima wɔ ŋa gbema?} How many children did he have? \textbf{3)} what. \textit{Mi bamɔ ilel wɔa?} Mummy, what is your father's name?

\TCsubword{liwɔ} (der.) \textit{quant} how many. 

\TCheadword{wɔch} \textit{n} watch.

\TCheadword{wɔɛ} \textit{v} \textbf{1)} be alive. \textit{Mpɛntɛ ha mɔɛ ha ba mɔ gbemdɛ, ha wɔi?} Your brothers born of the same father, are they alive? \textit{Atheɛ nwɔ kache, kɛ cheni pɛ wɔɛ?} I heard you say past, is he not alive? \textbf{2)} inhabit. \textit{Yamɔ wɔn ndɔ wɔɛa?} What about your mother, where does she live? \textbf{3)} live.

\TCheadword{wɔiowɔi} (der. of \TClink[2]{hu}, \TClink{-o-}, see \TClink{-o-}) 

\TCheadword[1]{wɔk} \textit{cf}: \TClink{sas}. \textit{v} squeeze.

\TCsubword{wɔkmmɔ} (comp.) \textit{v} milk cow.

\TCsubword{vɛkɛth} (der.) \textit{cf}: \TClink{sas}. \textit{v} squeeze.

\TCheadword[2]{wɔk} (der. of \TClink[1]{hɔ}) 

\TCheadword{wɔki} (comp. of \TClink[1]{wɔ}, \TClink[1]{ki}, see \TClink[1]{wɔ}) 

\TCheadword{wɔkmmɔ} (comp. of \TClink[1]{wɔk}, \TClink{nama} (comp. of \TClink[1]{na}, \TClink{maa}), see \TClink[1]{wɔk}) 

\TCheadword{wɔlɔ} \textit{n} \textbf{1)} world. \textit{Hɔlɔɛ gbi kunɛ.} All over the world. \textit{Siin bɛ pɛ lagboɛ wɔ hɔlɔ ka.} He does not even know anymore whether he is in this world. \textbf{2)} Earth. 

\TCheadword{Wɔlsh} \textit{nam} Walsh, name given to a person. \textit{Triniti chəəch hɔ kilkil Ani Wɔlsh skuul.} Trinity Church is opposite to Annie Walsh School (\citealt{Pichl1967}). 

\TCheadword{Wɔlta} \textit{nam} Walter, male name given to a person. \textit{Wɔlta Hanson a ka shi wɔ.} Walter Hanson, I used to know him.

\TCheadword[1]{wɔm} \textit{n} firewood. \textit{Yɛ ya wɔ hɔmɔ kənthi iwɔm dɛ wɔ ye kɔ.} When I tell him to break the firewood, he goes (\citealt{Pichl1967}). \textit{Wɔ ye tholi idïk iwɔm dɛ.} He took down the bundle of wood (\citealt{Pichl1967}).\textit{Ŋkɔ gbïl iwɔm dɛ lal l'ay kɔ, jɛmdi lɛ lɔ yema nyum.} Go put wood on the fire, the fire is about to go out (\citealt{Pichl1967}).

\TCheadword[2]{wɔm} \textit{cf}: \TClink{bot}, \TClink[1]{pampa}. \textit{n} \textbf{1)} boat, [wɔ̀m]/[wɔ̀mdɛ̀]/[wɔ̀mthɛ̀] boat/the boat/ boats (K dialect). \textit{Yɔk mi ko wɔm dɛ.} Take me there to the boat (\citealt{Pichl1967}). \textbf{2)} canoe. \textit{ Hã bue thɔk lɛ hã hã sol wɔm.} They hollowed the tree to make a canoe (\citealt{Pichl1967}). \textit{Ŋgbẽy wɔm dɛ.} Call the canoe (\citealt{Pichl1967}). \textbf{3)} ship. \textit{Manawa tha ka che hɛlɛ ko hã thapa wɔmthi anya pinɛ awoka lɛ.} the warships were kept at sea to stop the slave ships (\citealt{Pichl1967}). 

\TCsubword{wɔmchiɛ} (comp.), (id.) [wɔ́mchìɛ̀] \textit{n} car (lit. land boat) (K dialect). 

\TCsubword{wɔmgbimi} (comp.) \textit{n} (hɔ̃/tha) steamship, steamer (\citealt{Pichl1967}). 

\TCsubword{wɔmmaaŋko} (comp.) \textit{n} (hɔ̃/tha) large canoe propelled by oars (\citealt{Pichl1967}). 

\TCsubword{wɔmmbɔkul} (comp.) (hɔ̃/tha) \textit{n} large canoe, up to 3 tons, propelled by oars (\citealt{Pichl1967}). 

\TCsubword{wɔmpɛm} (comp.) (hɔ̃/tha) \textit{n} war canoe, warship (\citealt{Pichl1967}).

\TCsubword{wɔmtokɛ} (comp.), (id.) [wɔ́mtòkɛ́] \textit{cf}: \TClink[2]{balon}, \TClink{plɛn}. \textit{n} airplane (lit. sky boat) (K dialect). 

\TCheadword[3]{wɔm} \textit{cf}: \TClink[1]{fɔnwɛi} (comp. of \TClink[1]{wɛi}), \TClink{humoe}, \TClink{mane}, \TClink{yasi}. \textit{n} [\`{ŋ}wɔ̀mdɛ́] the medicine (B dialect). \textit{Ya wɔi kɔ pinɛ mwɔmdɛ, ya wɔi ka.} I went to buy her medicine and gave (it to) her. \textit{Nrɔm do ma ŋkəlɛŋ, ma bi hã sonki mɔ.} This medicine is good, it will cure you (\citealt{Pichl1967}). \textit{A bi nrɔm ka hɔ̃ mɔ bɔ ramïl apuma mo lɛ.} I have a medicine, which can cure your children (\citealt{Pichl1967}). 

\TCheadword{wɔmchiɛ} (comp. of, id. of \TClink[2]{wɔm}, \TClink{chiɛ}, see \TClink[2]{wɔm}) 

\TCheadword{wɔmgbimi} (comp. of \TClink[2]{wɔm}, \TClink{gbim}, see \TClink[2]{wɔm}) 

\TCheadword{wɔmmaŋko} (comp. of \TClink[2]{wɔm}, \TClink{lala-maŋke} (comp. of \TClink{lala}), see \TClink[2]{wɔm}) 

\TCheadword{wɔmmbɔkul} (comp. of \TClink[2]{wɔm}) 

\TCheadword{wɔmpɛm} (comp. of \TClink[2]{wɔm}, \TClink{pɛm}, see \TClink[2]{wɔm}) 

\TCheadword{wɔmtokɛ} (comp. of, id. of \TClink[2]{wɔm}, \TClink[1]{tokɛ} (der. of \TClink[1]{tok}, \TClink[1]{ɛ}), see \TClink[2]{wɔm}) 

\TCheadword{wɔnɛ} \textit{cf}: \TClink[1]{ki}, \TClink[1]{lan}, \TClink[3]{tho}. \textbf{1)} \textit{dem} this; this one. \textit{Wɔnɛ wɔ gbem wɔ.} The one who gave birth to him. \textit{Wɔnɛ pɔ bɛŋ wɔ bo, wɔi ko sɛm.} The one the ball touched would stand out. \textit{Hanɛ ha bia kɔ hundɛ hin thivelen yɛi bia koŋ chaŋdɛ, ŋan gbi haŋa hɔŋɔ lɛŋ.} Those that will be coming behind us when we shall have past, I am greeting all of them. \textbf{2)} \textit{dem} that; other one. \textit{Ma wɔ dumka igbɛth wɔnɛ bɛ hun gbo che igbɛth.} do not raise him to be spoiled (immoral), the ones coming (after him) will be spoiled. \textbf{3)} \textit{indfpro} anyone. \textit{So wɔnɛ wɔ vɛ thɔmwɔ.} So anybody that threw the ball at the other one.

\TCsubword{wɔnɛki} (comp.) \textit{dem} doubly marked demonstrative, ‘this one' (\textit{wɔ̀nà kí} this one [general rise throughout]). \textit{Hina wɔ-ki a?} Who is this one? (\citealt{Pichl1967}). 

\TCheadword{wɔnɛki} (comp. of \TClink{wɔnɛ}, \TClink[1]{ki}, see \TClink{wɔnɛ}) 

\TCheadword[1]{wɔni} \textit{subordconn} before. \textit{...paliioki tɛmpim te ki et-o-klɔk ichɔl wɔni huŋ gbemɔ.} ...the whole day, sometimes (not) until eight o'clock in the evening before giving birth.

\TCheadword[2]{wɔni} (der. of \TClink[1]{hɔ}, \TClink{-ni}, see \TClink[1]{hɔ}) 

\TCheadword{wɔŋ} \textit{cf}: \TClink[2]{bɛ}. \textit{v} \textbf{1)} [wɔ́ŋ] give (K dialect). \textit{Ya gbo che koŋ wɔŋ ihɔɔlɔŋ miɛ chelɛ ya sɔthɔ yeke ki; mɔ mie hɔm dɛ ya ka mɔ ŋɔ ni nsɔm.} I have just risked (given) my life so that I may get cassava to eat; you told me to give you some to eat. \textit{Bɛɛ lɛ Kɔng kol sirɔng hã sɔng wɔ ni kɔ wɔŋ beli li-mbul.} The chief gave Kong a corruption fee to bribe him to go and give false evidence (\citealt{Pichl1967}). \textbf{2)} give oneself. \textit{Wɔ ka wɔŋ ni kɛn ŋa koi kafaŋi yai.} He gave himself up to take away our sins. \textbf{3)} send. \textit{Pɔ koŋ kɔ gbo bɛ bɛkthai – pimdɛ kɔ pɔ bia fun-fun kai – pɔ kɔi wo tokɛko.} After putting it in bags - the other ones they will plant (in the rice nursery) - they will send it up top. \textbf{4)} blow.

\TCsubword{wɔŋgbenawi} (comp.) \textit{n} Poro announcement.

\TCsubword{wɔŋhul} (der.) \textit{cf}: \TClink{thikla}. \textit{v} \textbf{1)} [wɔ́ŋhúl] betray (K dialect). \textit{Gbana wɔ wɛi, koŋ hi wɔ̃hul ko anya hi nchenk lɛ.} Gbana is bad, he has betrayed us to our enemies (\citealt{Pichl1967}). \textbf{2)} sell. \textit{Mputh ma na lɛ pə ma wɔŋ hul.} It is the guts of the cow that they sell (\citealt{Pichl1967}). \textit{Ŋ kɔ wɔhul sɔk bul.} Go sell one fowl (\citealt{Pichl1967}). 

\TCsubword{wɔŋni} (der.) \textit{v} give oneself. \textit{A wɔŋni ŋkɛn ŋa wɔn.} I give myself to him.

\TCheadword{wɔŋgbenawi} (comp. of \TClink{wɔŋ}, \TClink{gbenɔ}, \TClink[4]{-i}, see \TClink{wɔŋ}) 

\TCheadword{wɔŋhul} (der. of \TClink{wɔŋ}, \TClink{-ul}, see \TClink{wɔŋ}) 

\TCheadword{wɔŋni} (der. of \TClink{wɔŋ}, \TClink{-ni}, see \TClink{wɔŋ}) 

\TCheadword{wɔsɔ} \textit{cf}: \TClink{lagbowɛ}, \TClink{yipio}. \textit{disco} [wɔ̀sɔ̀] goodbye (K dialect); [wɔ̀sòwéí] goodbye-o (B dialect). 

\TCheadword{Wɔtalu} \textit{nam} Waterloo, name given to a place.

\TCheadword{wɔwɔ} \textit{coordconn} [wɔ̀wɔ̀] however (K dialect). 

\TCheadword[1]{wu} \textit{cf}: \TClink[1]{nyum}. \textit{v} \textbf{1)} [wù] die (K dialect). \textit{Tamɔ lɛ ker ɛ kel wɔ ni wɔ ye wu.} The boy was bitten by a snake and died then (\citealt{Pichl1967}). \textit{Igbimi lɛ hɔ hã ya koŋ kuthni lɛ ŋɡɛyɛn gbo ya bi hã wu.} The smoke had suffocated me, if you had not come quickly, I would have died (\citealt{Pichl1967}). \textbf{2)} be destroyed. \textit{Tombo bɔnth wɔ kɛ che bi ŋa wu.} Though he was troubled he was not destroyed. comp. \TClink{nɔwu} (see \TClink{nɔ}), \TClink{trihuɛ} (see \TClink{tri}) 

\TCsubword[2]{wu} (der.) \textit{n} \textbf{1)} death. \textit{Liwu lɔ che hini sɔyɛ.} Death does not frighten us (\citealt{Pichl1967}). \textbf{2)} calamity. \textit{Liwu lɔ bɔnthɔ hĩ, gbundɛ bom koŋ duk pɔk l'ay.} Calamity has met us; a big trouble has befallen the country (\citealt{Pichl1967}). 

\TCsubword[3]{wu} (der.) \textit{n} \textit{iwuu} (hɔ̃/-) lameness, paralysis (\citealt{Pichl1967}). 

\TCsubword[4]{wu} (der.) \textit{n} \textbf{1)} dead one. \textit{Ahuɛ ko lɔ che thiyɛŋ.} The dead ones had been among them. \textbf{2)} death. \textit{Nsuskɔɛ ma handɛ ma vɛ ka ko ki hu lɛ.} Exchanges took place for the deaths.

\TCsubword{wuɛwuɛ} (der.) \textit{v} die. \textit{Kɛ ŋanɛ ŋa wuɛwuɛ ni ache pɛ mɛmba hin awɔ ile lɔ, hin awɔ ile lɔɛ... yi abaot amɛnbul.} But some have died so I do not remember how many of us remain, how many of us remain there... we are about six.

\TCsubword[1]{wul} (der.) \textit{cf}: \TClink[2]{sak}. \textit{n} \textbf{1)} wake. \textit{Kaiŋ Taso wɔe bɛmpani ni anya wɔe ŋae kɔni ko wul-lɛ.} Kain Tasso and his people prepared themselves to go to the wake. \textit{Haaŋ ni nante bɛ, pɔ mu tɔn tontho ki chɔl sakɛ ha hok saka wul-lɛ.} Even up to to the present day, people still sing these songs the night of the wake. \textbf{2)} funeral. \textit{Kaiŋ Taso koŋ pɔkɔni bɛ ko wul lijajɛl wɔɛ lɔ hunɛ.} Kain Tasso has forgotten that he came to his mother-in-law's funeral. \textbf{3)} death. \textit{Yɛ Bɛl Maaɛ koŋ thaŋni boeɛ tokɛ hiŋk wul-lɛ lɔ bin wɔɛ...} When Rat Wife had climbed above the kitchen (away) from where death had missed her... \textit{Iwɔ, ha wul lijajɛl wɔɛ la wɔ mamɛ?} Why, with the death of his mother-in-law, why is he laughing?

\TCheadword[2]{wu} (der. of \TClink[1]{wu}) 

\TCheadword[3]{wu} (der. of \TClink[1]{wu}) 

\TCheadword[4]{wu} (der. of \TClink[1]{wu}) 

\TCheadword[5]{wu} \textit{v} initiate. \textit{Buɛ Hini, ya koni hu ifɔndɛ.} (I was known as) Bue Hini, after being initiated into the society. \textit{Yɛ hu ifɔndɛ pɔ mɔi ka ilel Buɛ Hini?} When you were initiated is the time you were given the name Bue Hini?

\TCheadword{wuɛwuɛ} (der. of \TClink[1]{wu}) 

\TCheadword{wuk} \textit{cf}: \TClink{kɔŋkɔ}. \textit{n} \textit{iwuk} (hɔ̃/-) skin of cooked rice which has not turned out smooth, usually given to children (\citealt{Pichl1967}). 

\TCheadword[1]{wul} (der. of \TClink[1]{wu}) 

\TCheadword[2]{wul} \textit{cf}: \TClink{thaozin}. \textit{Numb} thousand. \textit{Pàŋ Nanɔɛ, nɛn dɛ wul bul kɛmɛ koŋhɔanya mɛŋhiɔlniwaŋ, koŋhɔanya hiɔl ni mɛŋbul.} July 1986.

\TCheadword{wumbe} \textit{n} female spirit.

\TCheadword{wumɛn} \textit{adj} impotent.

\TCheadword{wun} [wún] \textit{n} [mwún] brains; \textit{ŋwun} (ma) brains (\citealt{Pichl1967}). 

\TCheadword{wunjal} \textbf{1)} \textit{n} careless person. \textbf{2)} \textit{adj} negligent.

\TCheadword{wunthi} \textit{v} \textbf{1)} untie. \textit{N wúnthí pànthɛ́.} Untie the tied. \textbf{2)} loosen.

\TCheadword{wuŋk} \textit{v} rush through; hurry.

\TCsubword{wuŋki} (der.) \textit{v} weigh anchor; depart.

\TCheadword{wuŋki} (der. of \TClink{wuŋk}, \TClink[1]{-i}, see \TClink{wuŋk}) 

\TCheadword{wus} \textit{cf}: \TClink[1]{wal}. \textit{n} \textbf{1)} (kɔ/hɔ) palm leaf still on the tree, also other kinds of leaves or grasses used for thatching (\citealt{Pichl1967}). \textbf{2)} thatch. \textit{Yɛ̀ pɔ̀ kóŋ gbó bálón bellɛ, pɔ̀ bɛ́ wùsɛ̀, pɔ̀ ŋɔ̀ bím.} When they have finished tying the rafters of the farmhouse, they put on the thatch, they cover it. \textit{Ŋae koŋ sɔth-sɔthni wusɛ kunɛ.} They (the rats) went and hid in the thatch.

\TCheadword{wusi} \textit{v} ransack. \textit{Wusi kil lɛ chen kəlɛŋ.} To ransack a home is not good (\citealt{Pichl1967}).

\TCheadword{wuup} \textit{Idph} of slipping and falling. \textit{La vein bɛ, wɔe hɛthini hiŋk boeɛ tokɛ <wu-u-u-u-wup>.} Before long, she slips down from above the kitchen (and) <wu-u-u-u-wup> (slips and falls).

\end{letter}
\begin{letter}{Y}

\TCheadword[1]{ya} \textit{cf}: \TClink[1]{chɛth}. \textit{v} [yáá] cook (K dialect). \textit{Ŋ kɔ ya pəlɛ lɛ.} Go cook the rice (\citealt{Pichl1967}). comp. \TClink{tuyaka} (see \TClink[1]{tu}) 

\TCsubword{yakani} (der.) \textit{v} cook. \textit{La si gbo pɛ hani ni pɔ koŋ yakani tri thai than gbi mɛnk bullɛ.} It just so happened that they all finished cooking together in all the villages at the same time (\citealt{Sumner1921}). 

\TCheadword[2]{ya} \textit{cf}: \TClink[1]{mi}. \textit{pers} \textbf{1)} I. \textit{Ya bi bɛthɛkin, ya mɔ la hɔm gbəŋ.} I have a secret, I tell it to you tomorrow (\citealt{Pichl1967}). \textit{Ya bi nrɔm ka, ma mɔ bɔ ramir.} I have a medicine here, it should cure you (\citealt{Pichl1967}). \textbf{2)} me. \textit{N sonthuli pɛnsil lɛ hã yaŋ.} Sharpen the pencil for me (\citealt{Pichl1967}). \textit{Ŋ kɔ tuu ibəl lɛ shop lɛ ahɔl ni nhã ya si bushɛl liwɔ.} Go measure the palm kernels at the shop and let me know how many bushels (there are) (\citealt{Pichl1967}). \textit{Ya chen na sɛm ka ŋán chee yàŋ kɛn.} I wouldn't have been standing here before you, (me) alone. 

\TCheadword[1]{yaa} \textit{n} [yàà] mother (K dialect); [yáá]/[yáámì]/[yáámɔ̀] mother/my mother/ your mother (B dialect); \textit{yaa} mother (\citealt{Pichl1967}). \textit{Ya bɔnthɔ wɔ taŋ, wɔ wokiɛ ya wɔ.} I met him crying, he was sighing for his mother (\citealt{Pichl1967}). 

\TCheadword{yabas} \textit{cf}: \TClink{sibɔla}. \textit{n} [yàbás] onion (K dialect). \textit{Yɛ mɔ ni bɛ yabasɛ atok, mɔi gbiŋgith.} After putting the onions in, then you cover it.

\TCheadword{Yagba} \textit{nam} a name meaning something like going up and down, always in a hurry. \textit{Yagba} nickname given to a very old lady in Shenge, always calling out to people in the road.

\TCheadword{yagba} \textit{n} [yàag̀bà], [yagba] worry (K dialect). \textit{Mabi yagba gbe kɛ ma nkɛlɛŋ lɛ nɔ mɔ bo tiŋ-tiŋ.} There are many worries, but it is fine if you are straightforward with the people.

\TCheadword{yagbo} \textit{n} nephew. \textit{Yagboɛ wɔɛ wɔ bɛmi skul, bami yagbe wɔɛ.} It's my father's nephew that sent me to school.

\TCheadword[1]{yai} \textit{n} [yàí] cat (K dialect). \textit{Yàìyɛ́ wɔ́ kóthàɛ̀ àlɔ̀.} The cat is under the cloth. \textit{Yay ɛ wɔ kɛpiɛ tamɔ lɛ.} The cat scratched the boy (\citealt{Pichl1967}). 

\TCheadword[2]{yai} \textit{cf}: \TClink{tama}. \textit{n} foolishness.

\TCsubword{yaiyai} (der.) \textit{n} worthlessness. \textit{Yamɔ lɛ wɔ libaŋ, mpanth ma wɔ lɛ gbo iyay-yay.} The boy is lazy, his work is just completely worthless (\citealt{Pichl1967}). 

\TCheadword{yaiyai} (der. of \TClink[2]{yai}) 

\TCheadword{yakani} (der. of \TClink[1]{ya}, \TClink{-kani} (der. of \TClink{-k}, \TClink{-ni}), see \TClink[1]{ya}) 

\TCheadword{yala} \textit{cf}: \TClink{miliŋdigber} (id. of \TClink{miliŋ}, \TClink{gbe}). \textit{n} unreliable person.

\TCheadword{yam} \textit{v} yawn.

\TCheadword{yambɛ} \textit{Loc} around the shoulders.

\TCheadword{yambo} \textit{n} snake medicine; \textit{Yambo/Djambo} a snake medicine (\citealt{Hall1938}). 

\TCheadword{yamfa} \textit{n} nagging.

\TCheadword{yams} (Eng \textit{yams}) \textit{cf}: \TClink[2]{buk}, \TClink[2]{di}. \textit{n} yam. \textit{yàmə̀s} yam.

\TCheadword{yana} (Port \textit{ventana} ‘window') \textit{n} window. comp. \TClink{yaŋminɛ} (see \TClink[2]{min}) 

\TCsubword{yaŋmbusɛ} (comp.) \textit{n} inside of nose.

\TCheadword{yancheŋkɛ} (comp. of \TClink{yanɔ}, \TClink[2]{choŋ}, see \TClink{yanɔ}) 

\TCheadword{yanɔ} \textit{cf}: \TClink[1]{hial}. \textit{n} \textbf{1)} river. \textbf{2)} stream.

\TCsubword{yancheŋkɛ} (comp.) \textit{n} (wɔ/hã, N) fish species, large black snapper (\citealt{Pichl1967}).

\TCheadword{Yaŋka} \textit{nam} Yanker, name given to a person. \textit{Ba Yaŋka wɔ chaŋ shi theli Mbolomdɛ; wɔ kiban dɛ, wɔ chaŋ shi theli Mbolomdɛ.} Ba Yanker knows how to speak Sherbro the best; he is the expert that knows how to speak Sherbro better (than anyone).

\TCheadword{yaŋka} \textit{cf}: \TClink[1]{thɛrɛŋ}. \textit{n} cave.

\TCheadword{yaŋmbusɛ} (comp. of \TClink{yana}, \TClink[3]{bos}, \TClink[1]{ɛ}, see \TClink{yana}) 

\TCheadword{yaŋminɛ} (comp. of \TClink{yana}, \TClink[2]{min}, \TClink[1]{ɛ}, see \TClink[2]{min}) 

\TCheadword{yao} \textit{n} sea spirit.

\TCheadword{yas} \textit{cf}: \TClink[1]{kɛkɛ}, \TClink{libɛn} (der. of \TClink[2]{li-}), \TClink{yasani}. \textit{temp} \textbf{1)} [yás] quickly (K dialect). \textbf{2)} frequently.

\TCsubword{yas-yas} (der.) \textit{temp} frequently.

\TCheadword{yasani} \textit{cf}: \TClink{yas}. \textit{v} hurry. \textit{Hɔk kɔ biɛn bɛŋ, kɛ kɔ yasani.} News does not have feet, but it hurries.

\TCheadword{Yase} \textit{nam} Yase Society. \textit{Mma kɔ thunk Yase l'ay, lɛ nchen bo̹l lɛ Yase-nɔ, chen bɔ vɛ bi hã kɔnth mɔ.} Don't go into the Yase nook if you are not a member of the Yase; otherwise it will catch you (\citealt{Pichl1967}). 

\TCsubword{Yasenɔ} (comp.) \textit{n} Yase member. \textit{Mma kɔ thunk Yase l'ay, lɛ nchen bo̹l lɛ Yase-nɔ, chen bɔ vɛ bi hã kɔnth mɔ.} Don't go into the Yase nook if you are not a member of the Yase; otherwise it will catch you (\citealt{Pichl1967}). 

\TCheadword{Yasenɔ} (comp. of \TClink{Yase}, \TClink{nɔ}, see \TClink{Yase}) 

\TCheadword{yasi} \textit{cf}: \TClink[1]{fɔnwɛi} (comp. of \TClink[1]{wɛi}), \TClink{humoe}, \TClink{mane}, \TClink[3]{wɔm}. \textit{n} medicines that heal but can also harm (\citealt{Hall1938}). 

\TCheadword{yas-yas} (der. of \textbf{yas})

\TCheadword{yataŋ} \textit{n} (wɔ/hã, N) insect species, scolopendra with strong jaws, thought to be dangerous (\citealt{Pichl1967}).

\TCheadword{yath} \textit{v} \textbf{1)} row. \textbf{2)} paddle. \textit{Wɔe tipɛ yaath ha kɔ ko pɛl dukiɛ.} He then began to paddle to go to the leggo chain.

\TCheadword{yayoŋ} \textit{n} [yáyóŋ] feel uninhibited, free within oneself (K dialect). 

\TCheadword[1]{ye} \textit{v} dance. comp. \TClink{nɔyes} (see \TClink{nɔ}) 

\TCsubword{yeek} (der.) \textit{v} dance with. \textit{Lɛ ŋke yɛ amaaɛ ŋa koŋ nuik tɔn thiŋaɛ; haliwɔ yɛ ŋa tɔn dɛ, vɛ ŋa yeek bol thiŋaɛ.} If you see how the women amuse themselves with their songs; because when they sing, so do they dance with their heads. \textit{Ndɛli yɛ ŋa tɔn ni ŋa che yeek bol thiŋaɛ!} Look at them while they sing, dancing with their heads!

\TCsubword[2]{ye} (der.) \textit{n} dance. \textit{Yeethi lo tha hiniɛm gbɔl chaŋ thanɛ chencha.} This dance delights me more than that of yesterday (\citealt{Pichl1967}).

\TCheadword[2]{ye} (der. of \TClink[1]{ye})

\TCheadword[3]{ye} \textit{nam} madam. \textit{Yem, ŋka mi yeke mɔɛ pum ni ya sɔm, ndikɛ ma mi.} Madam, give me some of your cassava, let me eat, I am feeling hungry. comp. \TClink{yemi} (see \TClink[2]{mi}) 

\TCheadword[4]{ye} \textit{n} cold.

\TCheadword{Yebu} \textit{nam} Yebu, female name given to a person. \textit{Ilel wɔ ŋɔ Zainab Yebu Kumba.} Her name is Zainab Yebu Kumba.

\TCheadword{yeek} (der. of \TClink[1]{ye}, \TClink{-k}, see \TClink[1]{ye}) 

\TCheadword{yeer} \textit{v} yell. \textit{Anya gber wɛin dɛ ŋa diklɛni bai koɛ, ŋa lee gbo kue yeer tokɛ kathba ŋa hɔɛ…} Among the many people who were gathered in the bari, they remain yelling aloud, they said…

\TCheadword{yegbe} \textit{adj} better. \textit{Ko chɛkɛ vɛ lɔndɔ kache sɔthɔ, mɛŋkɛ vɛ sɔthɔ yegbe cheni.} It is in that farm that he used to get (money), that time there was not a better harvest.

\TCsubword{nɔyegbe} (comp.) \textit{n} good person. \textit{Mɔ nɔ-yegbe.} You are good.

\TCheadword{yek} (der. of \TClink[1]{yen}) 

\TCheadword{yekɛ} \textit{cf}: \TClink{hopa}. \textit{n} (hɔ̃/-) cassava (Manihot esculenta) (\citealt{Pichl1967}). \textit{Atipɛ yuk yekeɛ, ŋkaŋdɛ, mbinchɛ, pɛlɛ, nsowɛ, ntɔllɛ.} I start to plant cassava, corn, beans, rice, millet, Guinea corn. 

\TCsubword{yekɛayɛɛn} (comp.) \textit{n} \textit{yekə ayɛɛn} (hɔ̃/-) real cassava (\citealt{Pichl1967}). 

\TCsubword{yekɛkamtha} (comp.) \textit{n} \textit{yekə kamtha} (hɔ̃/-) cassava species, long-cooking cassava, used for fufu (\citealt{Pichl1967}). 

\TCsubword{yekɛkel} (comp.) \textit{n} \textit{yekə ke̹l} (hɔ̃/-) cassava species, monkey cassava, inedible (\citealt{Pichl1967}). 

\TCsubword{yekɛkus} (comp.) \textit{n} \textit{yekə kus} (hɔ̃/-) remains of cooked cassava kept for unexpected guests (\citealt{Pichl1967}). 

\TCsubword{yekɛthɛɛ} (comp.) \textit{n} \textit{yekə thɛɛ} (hɔ̃/-) roasted cassava (\citealt{Pichl1967}). 

\TCsubword{yekɛpoloŋ} (unspec.) \textit{n} cassava species, cassava whose leaves resemble that of cottonwood (\citealt{Pichl1967}). 

\TCheadword{yekɛayɛn} (comp. of \TClink{yekɛ}, \TClink{ayɛn}, see \TClink{yekɛ}) 

\TCheadword{yekɛkamtha} (comp. of \TClink{yekɛ}) 

\TCheadword{yekɛkel} (comp. of \TClink{yekɛ}, \TClink[1]{kel}, see \TClink{yekɛ}) 

\TCheadword{yekɛkus} (comp. of \TClink{yekɛ}, \TClink[2]{kus}, see \TClink{yekɛ}) 

\TCheadword{yekɛpoloŋ} (unspec. of \TClink{yekɛ}, \TClink[2]{poloŋ}, see \TClink{yekɛ}) 

\TCheadword{yekɛthɛɛ} (comp. of \TClink{yekɛ}, \TClink[1]{thɛ}, see \TClink{yekɛ}) 

\TCheadword[1]{yel} \textit{v} decrease; reduce.

\TCheadword[2]{yel} \textit{v} boil. \textit{Mɛ̀ndɛ̀ mà yéːl.} The water is boiling. \textit{Mɛndɛ ma koŋ yɪl/hɪl.} The water is boiling, the water has reached a boiling state. \textit{Mai yelmani nkuaiɛ.} It will boil together with the palm oil.

\TCheadword[3]{yel} \textit{n} island. \textit{Lanɔ ki la haani Yɛl Nsaŋha ko.} This happened on Egusi Island.

\TCsubword{Yelsaha} (comp.) \textit{cf}: \TClink{Kakir}, \TClink{Kɔka}. \textit{nam} \textbf{1)} Caulker dynasty. \textbf{2)} Plantain Island residents.

\TCheadword{yeli} \textit{v} reduce.

\TCheadword{Yelsaha} (comp. of \TClink[3]{yel}, \TClink{saha}, see \TClink[3]{yel}) 

\TCheadword{yem} \textit{n} [yém] sister (K dialect). 

\TCheadword{Yema} \textit{nam} [yémá] Yema, name given to second daughter (K dialect). \textit{Siŋ kwey sangba nyok lo ni nsik hɔ Yema gbɔl!} Take this string of corals and tie them on Yema's neck (\citealt{Pichl1967}). 

\TCheadword[1]{yema} \textit{cf}: \TClink[2]{yom}. \textit{v} \textbf{1)} [yémà] want (K dialect). \textit{Lɛ nyema-m gbo bɔnthi gbəŋ boa.} If you want to meet me, come early tomorrow (\citealt{Pichl1967}). \textit{Yema yema kɔni gbəŋ ko apook.} Yema wants to go to her husband tomorrow (\citealt{Pichl1967}). \textit{A yema maani hã kul thafɛ.} I want to stop smoking a pipe (\citealt{Pichl1967}). \textbf{2)} wish. \textbf{3)} need. \textit{Iyema mɔ wɛyowɛ.} We need you every day. \textit{Iyema mɔ gbɔlsi yai.} We need you in our heart. \textbf{4)} like. \textit{So labi ale yimani laŋgbaŋdo labo wɔla bia yema.} So that is why I am asking this man if he likes that. \textit{Wɛl, ala bɔ yema.} Well, I can like it. \textbf{5)} agree. \textit{So labi ha ŋa lemɔ yi labo nyema la ŋa yaŋ yimɔ yi thilan.} So that is why I should first ask you if you would agree, for me to ask you these questions. \textit{Abatokɛ yemɔ bo, i la lɔ le lantha we.} If God agrees, we would hang it there. \textbf{6)} approve. id. \TClink{Nyemɔ} (see \TClink{nɔ}) 

\TCsubword[2]{yema} (der.) \textit{n} permission.

\TCheadword[2]{yema} (der. of \TClink[1]{yema})

\TCheadword[3]{yema} \textit{cf}: \TClink[2]{bi}. \textit{Aux} incipient, modal ‘will' (same usage in pidgin). \textit{Ŋkɔ gbïl iwɔm dɛ lal l'ay kɔ, jɛmdi lɛ lɔ yema nyum.} Go put wood on the fire, the fire is about to go out (\citealt{Pichl1967}). \textit{Palli lɛ yema duk.} The sun is about to set (\citealt{Pichl1967}). 

\TCheadword{yemi} (comp. of \TClink[3]{ye}, \TClink[2]{mi}, see \TClink[2]{mi}) 

\TCheadword[1]{yen} \textit{indfpro} \textbf{1)} something. \textit{Yen hɔ̃ tun ka.} Something stinks here (\citealt{Pichl1967}). \textit{A ke̹ yen.} I saw something (\citealt{Pichl1967}). \textbf{2)} what. \textit{A yema ke̹ yen dɛ hɔ̃ lɔ kunɛ lɛ.} I want to see what is inside (\citealt{Pichl1967}). \textbf{3)} thing, [bèyèn]/ [yá bìyɛní] there is nothing/ I have nothing (ɛ = something (yɛ?) (K dialect). \textit{Ya mɔ ka mɔ yendɛ gbi ŋɔ yemai.} I give you everything that you want. comp. \TClink[1]{beyen} (see \TClink[1]{be}), \TClink[2]{beyen} (see \TClink[1]{be}), \TClink{yenal} (see \TClink[2]{wal}), \TClink{yenjo} (see \TClink[2]{jo}), \TClink{yeŋbul} (see \TClink[3]{bul}), \TClink[2]{yeŋkɛlɛŋ} (see \TClink[1]{kɛlɛŋ}) 

\TCsubword{yenbiɛihɔlɔŋ} (comp.) \textit{n} living being (lit. thing-gotten-life).

\TCsubword{yenchɛk} (comp.) \textit{cf}: \TClink{yu}. \textit{n} [yèènchɛ́k] the fish (pl) (B dialect); [bìnthì yènchɛ́k kɔ̀ kí] This is a fish coop (K dialect). 

\TCsubword{yendapani} (comp.) \textit{n} \textbf{1)} [yéndàpání] mercy, sorrow (K dialect). \textit{Kumɔ lo wɔ yendapani lɛ, gbo toon ni yaa wɔ wɔ wu.} This child is full of sorrow, she was just small when her mother died (\citealt{Pichl1967}). \textbf{2)} pity (\citealt{Pichl1967}). \textit{Ya ko la yendapani hã leynɔ, kə peeki cheni.} I consider it a pity to depart from you, but it cannot be helped (\citealt{Pichl1967}). 

\TCsubword{yentho} (comp.) \textit{cf}: \TClink[1]{gbel}, \TClink{hathog}. \textit{n} leopard, substitute for taboo name (lit. bush thing). \textit{Ba yentho bi lɔ hantha ka pənth lɛ ay.} There was a Mr. Leopard who had a fishing fence here in the swamp (\citealt{Pichl1967}).

\TCsubword[1]{yenwɛi} (comp.) \textit{cf}: \TClink[2]{thukul} (der. of \TClink{thuk}, \TClink{-ul}). \textit{adv} badly. \textit{yèŋwɛ̀í} badly. \textit{Boɛ, waŋ mɔ lo chen tintin, koŋ bɛ yenwɛy, ŋ kɔ wɔ yi.} Boe, this daughter of yours is not straight, she has gone bad, go ask her (\citealt{Pichl1967}).

\TCsubword[2]{yenwɛi} (comp.) \textit{adj} [yèŋwɛ̀í] ill (B dialect); \textit{yeŋwɔi} ill (\citealt{Sumner1921}). 

\TCsubword{yek} (der.) \textit{n} things. \textit{Kòní bɛ́ ǹyéék [ɪ] má kómɔ̀wɛ̀ bààlàɛ́-áí.} Koni put the child's things in the basket. \textit{Ŋɔ mɔ thɔkɛ, ŋɔ mɔ kɔ saka po mɔɛ, ŋɔ mɔ wɔ thɔkɔ nyekma wɔɛ.} How to wash things, how to make your husband's bed, how to wash his things.

\TCsubword{yenoyen} (der.) \textit{indfpro} \textbf{1)} anything. \textit{Yenoyen nche ho bɛ hɔŋ chaŋ thɔm wɔɛ, mɔ bo be shi che.} You do not put anything to supercede the other, it should be exact. \textbf{2)} everything. \textit{Ni mgballɛ gbi maiko koiyɛ, ɪthaiɛ, yen-o-yen.} And all the writings we have taken, the proverbs, everything. \textit{Futh pɛlɛ, yukɛ, ŋa yen-o-yen, haŋ i koŋ gbako.} Root rice, the planting, do everything until we have grown.

\TCsubword[1]{Yenwɛini} (der.) \textit{nam} name used by women as substitute for taboo name of Poro Society spirit who appears as a dancing masquerade (lit. horrible thing) (\citealt{Pichl1967}).

\TCsubword[2]{yenwɛini} (der.) \textit{adv} vexed; agitated. \textit{Sogboka disil wɛini ŋɔe kɛnth, mmɛn dɛ mae huk yeŋwɛini.} Heavy winds (from the land) broke, the water was vexed (agitated). 

\TCsubword[3]{yenwɛini} (der.) \textit{n} bad thing. 

\TCsubword{yɛni} (der.) \textit{indfpro} nothing. \textit{Yà bì yɛ̀ní.} I have nothing.

\TCheadword[2]{yen} \textit{cf}: \TClink{bila}, \TClink[1]{ja}, \TClink{risen}. \textit{n} reason.

\TCheadword[3]{yen} \textit{cf}: \TClink{kuu}. \textit{n} property.

\TCheadword{yenal} (comp. of \TClink[1]{yen}, \TClink[2]{wal}, see \TClink[2]{wal}) 

\TCheadword{yenbiɛihɔlɔŋ} (comp. of \TClink[1]{yen}, \TClink[1]{bi}, \TClink[2]{hɔlɔŋ} (comp. of \TClink[2]{hɔl}), see \TClink[1]{yen}) 

\TCheadword{yenchɛk} (comp. of \TClink[1]{yen}) 

\TCheadword{yendapani} (comp. of \TClink[1]{yen}, \TClink{lapan}, see \TClink[1]{yen}) 

\TCheadword{yenjo} (comp. of \TClink[1]{yen}, \TClink[2]{jo}, see \TClink[2]{jo}) 

\TCheadword{yenoyen} (der. of \TClink[1]{yen}, \TClink{-o-}, see \TClink[1]{yen}) 

\TCheadword{yentho} (comp. of \TClink[1]{yen}, \TClink[2]{tho}, see \TClink[1]{yen}) 

\TCheadword[1]{yenwɛi} (comp. of \TClink[1]{yen}, \TClink[1]{wɛi} (der. of \TClink[2]{wɛi}), see \TClink[1]{yen}) 

\TCheadword[2]{yenwɛi} (comp. of \TClink[1]{yen}, \TClink[1]{wɛi} (der. of \TClink[2]{wɛi}), see \TClink[1]{yen}) 

\TCheadword[1]{Yenwɛini} (der. of \TClink[1]{yen}, \TClink[1]{wɛini} (der. of \TClink[1]{wɛi}, \TClink{-ni}), see \TClink[1]{yen}) 

\TCheadword[2]{yenwɛini} (der. of \TClink[1]{yen}, \TClink[1]{wɛini} (der. of \TClink[1]{wɛi}, \TClink{-ni}), see \TClink[1]{yen}) 

\TCheadword[3]{yenwɛini} (der. of \TClink[1]{yen}, \TClink[1]{wɛini} (der. of \TClink[1]{wɛi}, \TClink{-ni}), see \TClink[1]{yen}) 

\TCheadword{yenyen} \textit{Idph} of quietude. \textit{Ye lai bikɔs ivin Pothonɔ ki yɔ hun ke nɔ ndɔndɔ ko wɔko, lɔ <yen-yen> pɔ che diskres nɔ.} That is it, because even when this white man came here, he saw everybody in his place, the place is <yen-yen> (quiet), they do not disgrace people.

\TCheadword{yeŋ} \textit{v} be in a dilemma.

\TCheadword{yeŋbul} (comp. of \TClink[1]{yen}, \TClink[3]{bul}, see \TClink[3]{bul}) 

\TCheadword{yeŋk} \textit{n} insect wax.

\TCheadword{Yeŋken} \textit{nam} Yanken, female name given to a person. \textit{Nɛn doki wɔe hun chɔŋ waaŋmaa len yeŋkɛ-lɛŋba; ilel wááŋmààɛ ŋɔ ka cheɛ Yeŋken haliwɔ wááŋmàà ki jal wɔɛ ŋɔ ka che thii.} This man came to (began to) love this woman very much; the woman's name was Yanken because her skin was black. 

\TCheadword[1]{yeŋkes} (Eng \textit{Yankees}) \textit{n} English person. comp. \TClink{nɔyeŋkes} (see \TClink{nɔ}) 

\TCheadword[2]{yeŋkes} (Eng \textit{Yankees}) \textit{adj} English.

\TCheadword[1]{yeŋkɛlɛŋ} (comp. of \TClink[1]{kɛlɛŋ})

\TCheadword[2]{yeŋkɛlɛŋ} (comp. of \TClink[1]{yen}, \TClink[1]{kɛlɛŋ}, see \TClink[1]{kɛlɛŋ}) 

\TCheadword{yeŋkɛlɛŋba} (der. of \TClink[1]{yeŋkɛlɛŋ} (comp. of \TClink[1]{kɛlɛŋ}), \TClink[2]{ba}, see \TClink[1]{kɛlɛŋ}) 

\TCheadword{yeŋkɛlɛŋyeŋkɛlɛŋ} (der. of \TClink[1]{yeŋkɛlɛŋ} (comp. of \TClink[1]{kɛlɛŋ}), see \TClink[1]{kɛlɛŋ}) 

\TCheadword{yeŋthi} \textit{cf}: \TClink{thiyeŋ}. \textit{prep} between.

\TCheadword{yereŋ} \textit{cf}: \TClink{pinthaŋ}. \textit{v} be confused. \textit{Koŋ yereŋ gbi.} He is completely confused. \textit{Kaiŋ Taso koŋ yereŋ.} Kain Tasso was confused.

\TCheadword{Yesefu} \textit{nam} Yesefu, male name given to a person. \textit{Yaŋ yalɔ Yesefu.} I am Yesefu.

\TCheadword{yey} \textit{cf}: \TClink[2]{gbo}, \TClink{seŋgbɛŋ} (comp. of \TClink{sɛŋ}, \TClink[1]{gbɛlaŋ}). \textit{n} children's top.

\TCheadword{yeyɛ} \textit{cf}: \TClink{thɛkɛsi}. \textit{v} \textbf{1)} translate. \textbf{2)} interpret.

\TCheadword{yeɡbe} \textit{adj} \textbf{1)} upstanding. \textbf{2)} well.

\TCheadword[1]{yɛ} \textit{cf}: \TClink[5]{che}, \TClink[3]{hɔ}, \TClink{lagbo} (comp. of \TClink[2]{la}, \TClink[1]{gbo}), \TClink[2]{la}, \TClink[2]{lɛ}, \TClink[4]{ni}, \TClink[3]{ŋa}, \TClink{pabondɛ}, \TClink[2]{si}. \textit{subordconn} \textbf{1)} since. \textit{Yɛ mpima nthetha ha hundɛ, yɛ haŋ che veleŋkoɛ...} When the grandchildren come, since they are after (us)... \textit{Mɔm, frɔm yɛpɔka gbem mɔ haŋ ma nandɛ, yɛ nko ke wɔlɔɛ frɔm kache haŋ ma nande, ŋɔ nkeni wɔlɔa?} From since you were born until today, since you have seen the world in the past up until now, how do you see the world? \textbf{2)} though. \textbf{3)} when. \textbf{4)} how. \textit{Pɔi wɔ yɛ nɔɔ ki wɔ bɔ cha chaŋchaŋ doa.} Then they would begin to say, how is this person roaming about this way? \textit{Yɛ pɔ ŋaɛ.} That is how they do. \textit{Nlɛli ye wɔ ke̹ni!} See how he looks!, i.e., what a stupid-looking face he has (\citealt{Pichl1967}). \textbf{5)} after. \textit{Yɛ koŋ woth-kun dɛ pɔɛ hɔ ma gbemɔ woŋga.} After she got pregnant, she was told no delivery at home. \textit{Yɛ̀ kóŋ thɔ̀n dɛ̀, wɔ̀è bání kùáɛ́ njáláí.} After bathing she rubbed oil on her skin. \textbf{6)} if. \textit{Bikɔs nɔthiɛ yɛ mɔ ha lendɛ, mɔ ŋa shi ha ja la mɔ la ha kai.} Because human beings, if you are making something, you should know the reason why you are making it. \textit{Ŋgɛtiɛ malɔ gbo mɔ bɛ nton.} If you have groundnut, add a little. \textit{Pɔ koŋ kɔ gbo bɛ bɛkthai, pɔ ye ma gbo jo, pɔ kɔ sɛkɛli.} After putting it in bags, if they (want to) eat it, they first dry it (in the sun). \textbf{7)} as. \textit{Yɛ mɔ kɔ ni puliɛ,mɔ koi yabasɛ nbɛlɔ atok.} As you are mixing it, you take the onion and add it in. \textit{Yɛ mɔni gbiŋgith vɛ, inkeni bo iyiɛ ŋa hun chɔŋ.} As you cover it, the next time you open it is for dishing out. \textbf{8)} that. \textit{Yɛ nkoyɛ mɛŋk mɔɛ nwun, Abatokɛ che mamɔ.} That you have taken your time and come, may God be with you. \textit{Ya gbo che koŋ wɔŋ ihɔɔlɔŋ miɛ chelɛ ya sɔthɔ yeke ki.} I have just given my life so that I may get cassava to eat. \textbf{9)} while. \textit{Yɛ thoŋka ki gbi kɔ haani bɛl siatiŋ doki thiyeŋ dɛ…} When all this arguing is going on between these two rats… \textbf{10)} when. \textit{A lomani yɛ Ba Ŋgobɛ ka che hun dɛ hwɛ lɛ hɔ lelɛ.} I remember when Mr. Ngobe was coming that it rained (\citealt{Pichl1967}). \textit{Pɔ mɔ koil ye vɛ, la kɔ kanni.} When people shout at you, that's not good. \textbf{11)} what.

\TCheadword[2]{yɛ} \textit{temp} \textbf{1)} now. \textit{So, mɔm ni po mɔ ŋaŋa ka tipɛn dɛ ɔ mɔm ni nɔ peka ŋa ni yɛ?} So, you and your husband started, or you are now with another person? \textit{Ok, a wɔni yɛ nɛnthi mɛn dɛ kunɛ lɔni yɛ.} Ok, I (would) say that it is five years I am in it now. \textbf{2)} then. \textit{Yɛ nɔ wɔ che ko kɔnaɛ, ya hundɛ wɔi hɔ “He!”} When someone would be in a corner, then I would come and she would say “Hey!” \textit{A kache dikil koŋo thi bɛlpotho wɛ, ayi bɛ isundɛ.} I used to gather coconut shells, then I would put sand (inside). \textbf{3)} once. \textit{Thɛmko atiŋ ha ka che yɛ we.} Once there were two mates.

\TCheadword[3]{yɛ} \textit{interrog} \textbf{1)} what. \textit{Yɛ laio wɛ, ye mpanth mɔ ni ha ha sɔpɔt abena mɔi?} As it is, what work do you now do to support your parents? \textit{Yɛ wɔ kache ŋaa?} What did he used to do? \textbf{2)} why.

\TCsubword{yɛbi} (comp.) \textit{interrog} \textbf{1)} why. \textbf{2)} how.

\TCsubword{yɛbilaɛ} (comp.) \textit{subordconn} because. \textit{Wɛl imɛmiɛni ŋa hin sɔthɔ mɔ, yɛbilaɛ, yɛ ŋkoŋ ndɔio ki tɔn chɔchai...} Well we are happy to have you, because after you have sung for us in the church... \textit{...yɛbilaɛ ashiɛ lanɛ la ŋa nsheɛ.} ...because I knew that that was prior.

\TCsubword{yɛbini} (comp.) \textit{interrog} why. \textit{Yɛbi chɔŋ ma len na?} Why do you like it?

\TCsubword{yɛkia} (comp.) \textit{interrog} What is this?

\TCheadword[4]{yɛ} \textit{n} \textit{yə} (kɔ/ma) tree species, tree with light wood used for making boats or planks (\citealt{Pichl1967}). 

\TCheadword{yɛbi} (comp. of \TClink[3]{yɛ}, \TClink[1]{bi}, see \TClink[3]{yɛ}) 

\TCheadword{yɛbilaɛ} (comp. of \TClink{bila}, \TClink[3]{yɛ}, see \TClink[3]{yɛ}) 

\TCheadword{yɛbini} (comp. of \TClink[3]{yɛ}, \TClink[2]{bi}, \TClink[4]{ni}, see \TClink[3]{yɛ}) 

\TCheadword{yɛɛk} \textit{cf}: \TClink{sɛɛbom} (comp. of \TClink{sɛɛ}, \TClink{bom}). \textit{n} \textbf{1)} [yə̀ə̀k] kind of utensil, small cooking spoon (K dialect); \textit{yɛk} (kɔ/ma) kitchen utensil used to stir rice or fufu (\citealt{Pichl1967}). \textbf{2)} [yə̀kɛ́] metal ladle (K dialect); [yɛ̀k], [ìyɛ̀k] metal ladle (B dialect). 

\TCheadword[1]{yɛgbɛ} \textit{n} \textbf{1)} (wɔ/hã, N) bird species, breeds on the ground. If by accident someone breaks its eggs, death or misfortune will trail him unless he is valued by the society which has the same name (\citealt{Pichl1967}). \textbf{2)} (wɔ/hã, N) disease caused by the \textit{yɛgbɛ} bird – about persons who have the disease it is said, “The bird has caught him or her.” Girls cured of this disease have “Yagbo” as their second name, boys add after their first name “ba vee” (\citealt{Pichl1967}). 

\TCheadword[2]{Yɛgbɛ} \textit{nam} Yegbe Society.

\TCheadword[1]{yɛk} \textit{n} insect species, bedbug, [yɛk]/[ŋyɛk] bedbug/bedbugs (K dialect).

\TCheadword[2]{yɛk} \textit{adj} fragile.

\TCheadword{yɛkia} (comp. of \TClink[3]{yɛ}, \TClink[1]{ki}, \TClink[1]{a}, see \TClink[3]{yɛ}) 

\TCheadword[1]{yɛl} \textit{n} liver, [yɛ́l]/[yɛ́lthɛ̀] liver/livers (K dialect); (hɔ̃/tha) liver (\citealt{Pichl1967}).

\TCheadword[2]{yɛl} \textit{v} [yɛ́l] prepare fishing rod (K dialect); \textit{yɛɛl} prepare for fishing with the rod, i.e., fix lead and hook to the line (\citealt{Pichl1967}).

\TCheadword{yɛlai} \textit{dem} that is it. \textit{Yɛlai bikɔs hin pɛ tɛŋga apima hinyɛ ha bia che hun gbɛ.} That is it, because maybe our children will come visit.

\TCheadword{yɛlkɛnth} \textit{n} puny fellow. \textit{Yɛlkənth lo wɔm hɔl.} This puny fellow insults me (\citealt{Pichl1967}).

\TCheadword{yɛllɛ} \textit{cf}: \TClink{yɔlko}. \textit{n} chain. \textit{Paŋopaŋ gbi, Braima wɔ kɔ lɔɔli pɛl dukiɛ ni yɛllɛɛ.} Every evening, Braima goes to inspect the leggo chain and the yɛllɛ chain. \textit{Braima wɔe kɔ lɔɔli pɛl yɛllɛɛ ni pɛl dukiɛ.} Braima went to inspect the net chain but the net had sunk.

\TCheadword{yɛlɔ} (Eng \textit{yellow}) \textit{cf}: \TClink{bɔŋkia}. \textit{adj} yellow.

\TCheadword{yɛmbɛ} \textit{n} waist beads.

\TCheadword{yɛn} \textit{cf}: \TClink[3]{wɔ}. \textit{interrog} how much.

\TCheadword{yɛni} (der. of \TClink[1]{yen}, \TClink[2]{ni}, see \TClink[1]{yen}) 

\TCheadword{yɛnthɛ} \textit{n} beni-seed (sesame seed) (K dialect); \textit{yɛntɛ} beni-seed (Sesamum indicum) (\citealt{Pichl1967}).

\TCheadword{Yɛŋki} \textit{nam} Yenki, female name given by Toma Society. 

\TCheadword{yɛs} \textit{cf}: \TClink{aa}, \TClink{ayo}, \TClink{ee}. \textit{disco} yes. \textit{Yɛs, bullɛ wɔ Tisana ko.} Yes, the one is at Tisana. \textit{Yɛs, ako bi nɔma.} Yes, I have got a woman.

\TCheadword{yɛthi} \textit{cf}: \TClink{trit}. \textit{v} \textbf{1)} [yɛ́thí] hold (K dialect). \textit{N yɛthi tiŋ!} Hold fast! (\citealt{Pichl1967}). \textbf{2)} fulfill. \textit{Hã bɔni yɛthi saba wɔ lɛ.} They cannot fulfill his law (\citealt{Pichl1967}). \textbf{3)} receive. \textit{Hwɛlɔ lɛ yɛthiɛ bɛ wɔ lɛ.} The world receives her king (\citealt{Pichl1967}). \textbf{4)} owe. \textit{A yɛthiɛ wɔ shiliŋ thira.} I owe her three shillings (\citealt{Pichl1967}). \textbf{5)} treat. \textit{Bikɔ pomdɛ wɔ mi ni yɛthi sɔŋgɔ ma ŋɔ nɔpikan wɔ ŋa yɛthi nɔma wɔi.} Because my husband is really treating me as a husband should treat his wife.

\TCsubword{yɛthini} (der.) \textit{v} \textbf{1)} hold tightly. \textbf{2)} cling fast.

\TCheadword{yɛthini} (der. of \TClink{yɛthi}, \TClink{-ni}, see \TClink{yɛthi}) 

\TCheadword{yɛthɔk} \textit{cf}: \TClink{thɛŋk}. \textit{v} bring by canoe.

\TCheadword[1]{yi} \textit{cf}: \TClink{thɛlɛn}. \textit{v} \textbf{1)} ask. \textbf{2)} ask for. \textit{À lɛ́líyá Bɔ̀ì, à yíyɛ́/yíɛ́ Bɔ̀ì.} I am looking for Boi, I'm asking for Boi. \textbf{3)} beseech. \textit{Bahin hi mɔ yi.} Our Father, we beseech you. comp. \TClink{nɔyiɛnthiŋ} (see \TClink{nɔ}), \TClink{nɔyiɛyibaw} (see \TClink{nɔ}) 

\TCsubword[1]{yimani} (comp.) \textit{v} ask consent. der. \TClink[2]{yimani} (see \TClink[1]{yi}) 

\TCsubword[2]{yimani} (comp.), (der. of \TClink[1]{yimani}) \textit{n} consent. \textit{So, aa ŋa le yimani ko lagbando labo yema la.} So, I should first ask the consent of this man, if he would want that. \textit{So labi ale yimani laŋgbaŋdo labo wɔla bia yema, Apa nyema la?} So that is why I am asking this man if he likes that, Pa, do you want it?

\TCsubword[2]{yi} (der.) \textit{n} question. \textit{Kɛ ayema mɔ yi yi bul.} But I just want to ask you a question.

\TCsubword[2]{yiki} (der.) \textit{v} ask. \textit{Kɛ lanɛ ki la bia humɔ yikiɛ, wɔnɛ gbi wɔ bia yema ŋa thelaɛ chɔŋ wɔla bia the.} But what I am about to ask you, anybody that wants to hear it could hear it.

\TCsubword{yiyini} (der.) \textit{v} ask themselves. \textit{Ŋae tipɛ yi-yini-ŋkɛn ŋa hɔɛ, “La taalaŋgba ki wɔ mama?”} They begin to ask themselves the same, saying, “What is this young man laughing about?”

\TCheadword[3]{yi} \textit{v} open. \textit{Haa yɛ mɔ kɔ yiɛ mɛndɛ ma shi gbo che, mɔi rɛthi jɛmdɛ ton-ton.} Then you open (the pot), if the water is just as it should be, you reduce the fire a little. \textit{Yɛ mɔni gbiŋgith vɛ, inkeni bo iyiɛ ŋa hun chɔŋ.} As you cover it, the next time you open it is for dishing out.

\TCheadword{yiars} (Eng \textit{years}) \textit{cf}: \TClink[1]{mɛŋk}, \TClink[2]{nɛn}. \textit{n} years. \textit{Tɛm landɛ ejimdɛ ŋɔ ej ɔf fɔti sɛvin yiars.} At that time, I was 47 years old.

\TCheadword{yiba} \textit{n} (wɔ/hã, si) vulture (\citealt{Pichl1967}).

\TCheadword{yibaw} \textit{cf}: \TClink{thiŋ}. \textit{n} \textbf{1)} future. \textit{Ya hun ni yiɛ mi yibaw.} I came to let you look at the ground for me, i.e., to tell me the future (\citealt{Pichl1967}). \textbf{2)} fortune telling. comp. \TClink{nɔyiɛyibaw} (see \TClink{nɔ}) 

\TCheadword{yiik} \textit{n} \textit{(i-)yiik} (hɔ̃/ma) key (\citealt{Pichl1967}). \textit{Yiik miɛ ŋɔ ki.} This is my key. \textit{Ŋ kɔ kwey nyik lɛ yaa mɔ suy...} Go take the keys from your mother's hand... (\citealt{Pichl1967}).

\TCheadword[1]{yiki} \textit{cf}: \TClink[2]{mani}, \TClink{rɛspɛkt}. \textit{n} [yíkì] respect (K dialect). \textit{Nke hin Abolomai, yikiɛ ŋɔ iyema.} You see us Sherbro, it is respect we want.

\TCheadword[2]{yiki} (der. of \TClink[1]{yi}, \TClink{-k}, see \TClink[1]{yi}) 

\TCheadword[3]{yiki} \textit{cf}: \TClink{yuki} (der. of \TClink{yuk}, \TClink[1]{-i}). \textit{n} \textbf{1)} [yìkì] seed (K dialect). \textit{Pɔ yuk mansaŋhaɛ nseen si pɔ wɔm bɛ kutha pɛlɛɛ ni nyiki ntilaŋ.} They plant this egusi together with it first, before they plant rice or any other seeds. \textbf{2)} \textit{nyiiki} vegetables (\citealt{Pichl1967}). \textbf{3)} plant. \textit{Nsaŋhaɛ ma ka che chaŋ bali ha chaŋ nyiki halɛ gbi.} The egusi grew more than all the other plants. \textbf{4)} plantation. \textit{Pɔ ŋa yuki, pɔŋa ŋa nyiki?} Do they plant here, do they make plantations? comp. \TClink{hunyiki} (see \TClink[1]{hu}) 

\TCheadword{yikisi} \textit{v} [yíkísì], [yíkíshì] walk with a wiggle, verb, woman wiggling when she walks (K dialect). 

\TCheadword[1]{yil} \textit{cf}: \TClink[1]{kul}. \textit{v} \textbf{1)} be drunk. \textbf{2)} drink alcohol to excess. comp. \TClink{nɔyilɔ} (see \TClink{nɔ}) 

\TCheadword[2]{yil} \textit{n} (wɔ/hã, N) bird species, nightjar (\citealt{Pichl1967}). 

\TCheadword[1]{yimani} (comp. of \TClink[1]{yi}, \TClink[2]{mani}, see \TClink[1]{yi}) 

\TCheadword[2]{yimani} (der. of \TClink[1]{yimani} (comp. of \TClink[1]{yi}, \TClink[2]{mani}), see \TClink[1]{yi}) 

\TCheadword{yiŋjin} \textit{n} engine.

\TCheadword{yiŋktha} \textit{cf}: \TClink{pakil}, \TClink{pakni}, \TClink{pikith}. \textit{v} [yíŋkthá] shake (K dialect). \textit{M ma yikita thɔk lɛ, thɔm mɔ lɛ wɔ lɔ tokɛ ko, ma ki duk.} Don't shake the tree, your companion is up there, he could fall (\citealt{Pichl1967}).

\TCheadword{yipio} \textit{cf}: \TClink{lagbowɛ}, \TClink{wɔso}. \textit{disco} goodbye.

\TCheadword{yiwɔ} \textit{n} problems. \textit{Yɛ ya pɛ ka bɛ iwɔ miaɛ ko ŋɔ woth gbi ŋa yan dɛ.} When I am full with our problems, he will take on the load for me.

\TCheadword{yiyini} (der. of \TClink[1]{yi}, \TClink{-ni}, see \TClink[1]{yi}) 

\TCheadword{yo} \textit{cf}: \TClink{tɛŋkɛ}. \textit{v} drive birds away.

\TCheadword{yok} \textit{cf}: \TClink{tiŋkɔ}. \textit{n} \textbf{1)} coral. \textbf{2)} beads.

\TCheadword[1]{yol} \textit{v} decorate.

\TCsubword{yolni} (der.) \textit{v} be dressed with trinkets.

\TCheadword[2]{yol} \textit{n} \textbf{1)} jewels. \textit{Yaŋ tɔm nyol ma wɔ lɛ.} I count his jewels (\citealt{Pichl1967}). \textbf{2)} trinkets. \textit{Ŋ kɔ wɔ yol ka nyol lo.} Go decorate him with these trinkets (\citealt{Pichl1967}). \textit{Pə pɔŋ hok pɔɔ lɛ gbəŋ, yi bi nyol ŋgber hã hi lɛ.} They will pull Poro tomorrow (i.e. Poro will be out); we shall have many trinkets for our candidates (\citealt{Pichl1967}). 

\TCheadword{yolni} (der. of \TClink[1]{yol}, \TClink{-ni}, see \TClink[1]{yol}) 

\TCheadword[1]{yom} \textit{v} catch, contract, \textit{yom bək} catch colic (\citealt{Pichl1967}). 

\TCheadword[2]{yom} \textit{cf}: \TClink[1]{yema}. \textit{v} \textbf{1)} allow. \textit{Kani yom ŋa yin, chaŋ yenchɛkoki ŋa sɛkɛliɛ.} She never allows us things, it was only this dried fish. \textbf{2)} answer. \textit{Lɛ velɛ-m gbo ya wɔ yomɔ.} When he calls for me, I answer him (\citealt{Pichl1967}). \textbf{3)} consult. \textbf{4)} agree. \textit{Yà/à kóŋ yòm.} I agreed. \textit{Yi koŋ yom hã kɔ kə yi yema boya.} We agree to go but we want some cold water (presents to encourage them) (\citealt{Pichl1967}). \textbf{5)} be responsible. \textit{Ya che mɔn che yomɔ.} I will hold you (\citealt{Pichl1967}).

\TCsubword{yombul} (comp.) \textit{cf}: \TClink{yeŋbul} (comp. of \TClink[1]{yen}, \TClink[3]{bul}). \textit{n} equal. \textit{Hina mɔm hi gbo yombul.} You and I are equals.

\TCheadword[3]{yom} \textit{n} [yòm] taboo (K dialect). 

\TCheadword{yombul} (comp. of \TClink[2]{yom}, \TClink[2]{bul} (der. of \TClink[3]{bul}), see \TClink[2]{yom}) 

\TCheadword[1]{yoŋ} \textit{n} (hɔ̃/tha) fish basket, simple weir basket for fish or birds (\citealt{Pichl1967}). 

\TCheadword[2]{yoŋ} \textit{v} [yóŋ] take care of someone (K dialect). 

\TCheadword{yoŋibɛk} (comp. of \TClink[1]{bɛk}) 

\TCheadword{yosokin} \textit{v} [yósókìn] whine, grumble (K dialect). 

\TCheadword{yothi} \textit{v} pinch. \textit{Wɔ yothi Kɔŋ wɔn kɔk.} He pinches Kong on his backside (\citealt{Pichl1967}).

\TCheadword{yɔk} \textit{cf}: \TClink{gbundagbunda} (der. of \TClink{gbunda}), \TClink{toofi}, \TClink[1]{woth}. \textit{v} \textbf{1)} take. \textit{Yɔk mi ko wɔm dɛ.} Take me there to the boat. \textbf{2)} carry. \textit{Amaɛ ŋai hun, ŋa kɔ woth thi bolɛ, ŋa yɔk kebelthai ɔ tithai.} The women will come and carry it on their heads, and take it to farm houses or towns. \textbf{3)} carry or take away (\citealt{Pichl1967}). \textbf{4)} grab. \textit{Təm ra lɛ moɛ gbo si ɛ yi yɔk ŋgbatho ma hĩ ɛ.} When the time for clearing the bush arrives, we grab our cutlasses (\citealt{Pichl1967}). 

\TCheadword{Yɔki} \textit{nam} Yoki, name given to 7\textsuperscript{th} daughter, per Madam Lenga and Pa Yanker.

\TCheadword{yɔktha} (unspec. of \TClink{kɛth}) 

\TCheadword{yɔl} \textit{n} (wɔ/hã, N) crab species, pounded and used as bait for catching the \textit{kontha} (\citealt{Pichl1967}). 

\TCheadword{yɔlkɔ} \textit{cf}: \TClink{yɛllɛ}. \textit{n} \textbf{1)} [yɔ̀lkɔ̀] chain net (K dialect). \textbf{2)} fishing line (\citealt{Pichl1967}). 

\TCheadword{yɔŋ} \textit{cf}: \TClink{pɔl}, \TClink{sɔnthɔ}. \textit{n} [yɔ̀ŋ] fish trap (K dialect). 

\TCsubword{yɔŋkɔ} (unspec.) \textit{n} [yɔ́ŋkɔ́] basket for catching mudskippers (K dialect).

\TCheadword{yu} \textit{cf}: \TClink{yenchɛk} (comp. of \TClink[1]{yen}). \textit{n} fish, sg. \textit{Yu lɛ kong puthul, lɛ ŋgbəŋ wɔ gbo hinɛ gbo nɔth.} The fish is rotten already, if you touch it, you will find it quite soft (\citealt{Pichl1967}). \textit{Wɔ nɛkiɛ lɛ wɔ kuyɛ yu ihuk lɛ.} He hurt himself when he took a fish from the hook (\citealt{Pichl1967}).

\TCsubword{yubom} (comp.) \textit{cf}: \TClink{niŋkta}. \textit{n} fish species, electric ray, torpedo fish (also \textit{nïnkta}) (\citealt{Pichl1967}).

\TCsubword{yuhɔtka} (comp.) [yúhɔ̀tkà] \textit{n} bait, the fish you put on a hook (K dialect). 

\TCheadword{yubom} (comp. of \TClink{yu}, \TClink{bom}, see \TClink{yu}) 

\TCheadword{yuhɔtka} (comp. of \TClink{yu}, \TClink[2]{hɔth}, \TClink{-k}, see \TClink{yu}) 

\TCheadword{yuk} \textit{v} plant. \textit{Ŋa yaŋ toŋgi ŋɔ pɔ yuk pɛlɛ.} For me to show how to plant rice. \textit{Hin lɛ pɛ sallɛ mɔi gbo asaŋ keŋkendɛ a yuk gbamdɛ.} For us, when rainy season comes, I plant krain-krain, (and) I plant potato leaves. 

\TCsubword{yuki} (der.) \textit{cf}: \TClink[3]{yiki}. \textit{v} plant. \textit{Pɔ ŋa yuki, pɔ ŋa ŋa nyiki?} Do they plant here, do they make plantations?

\TCsubword{yukyuk} (der.) \textit{v} plant. \textit{Yɛ mɔ yukyuk a?} What do you plant?

\TCheadword{yuki} (der. of \TClink{yuk}, \TClink[1]{-i}, see \TClink{yuk}) 

\end{letter}
\begin{letter}{Z}

\TCheadword{Zainab} \textit{nam} Zainab, female name given to a person. \textit{Ilel wɔ ŋɔ Zainab Yebu Kumba.} Her name is Zainab Yebu Kumba.

\TCheadword{zit} \textit{Idph} \textbf{1)} of falling down with a noise (\citealt{Pichl1967}). \textit{Ta hã thɔk lɛ tok ɛ ni wɔ ye hɛ̃thni ni duk lɛ ko zit.} The boy climbed up the tree, slipped, and fell down “thump"! (\citealt{Pichl1967}). \textbf{2)} of standing solidly (B dialect).

\end{letter}

