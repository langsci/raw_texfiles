\begin{letter}{A}

\TCheadword[1]{a} \textit{cf}: \TClink[1]{e}. \textit{prt} clause-final interrogative particle. \textit{Tɛm landɛ vɛ ŋɔ mɔiya?} How old were you then? \textit{Ilel mɔa?} What is your name? \textit{Wɔn ndɔ pɔ gbem wɔa?} Where was she born? \textit{Yɛ nwɔk aa?} What languages? comp. \TClink{yɛkia} (see \TClink[3]{yɛ})

\TCheadword[2]{a} \textit{cf}: \TClink[2]{ya}. \textit{pro} I, me. \textit{A pin yenchɛk akɔ hɔŋgul.} I buy and sell fish. \textit{Yaŋ, a chɔŋ nwɔk mamdɛ len, Mbolomdɛ.} Me, I love it (the church service) in my language, Bolom. \textit{A kaŋa ŋa nɛn thi tiŋ ai mɛkni standad siks.} I studied here for two years, and I stopped at standard six. 

\TCheadword[3]{a} \textit{disco} ah, oh. \textit{Abatokɛ.} Ah God. \textit{Amaaɛ ŋae yom, “A ye-e-e mi-i-i.”} The women answer, “Oh, my lady.”

\TCheadword[4]{a} \textit{prt} particle used to introduce quoted speech. \textit{Nɔmaa chaɛ a: “Yemi, ni ntɛniɛ mini o-o-o.”} The woman sang: “My lady, remember me.” \textit{Kɛ gbemɔ nseiɛ primi, ye pɔ hɔ primiɛ vɛ aagbemɔ landɛ kɔ kath.} But they say that giving birth first to a preemie is difficult.

\TCheadword{a-} \textit{NCM} \textit{pfx} \textit{sfx} noun class marker. \textit{Lanɛ la li kɛlɛŋ ahinŋa ŋan the.} That is very good for people to hear. \textit{Apima wɔ ŋa gbema?} How many children did he have? \textit{Amaaɛ lee gbo thoŋka lanɔ ki ŋan thiyeŋ.} The women continued arguing about this among themselves.

\TCsubword{abulabul} (der.) \textit{cf}: \TClink{buleŋ-buleŋ} (der. of \TClink{buleŋ}), \TClink{bulnɔbul} (comp. of \TClink[3]{bul}, \TClink{nɔ}). \textit{adv} few. \textit{Təm dɛ kɔ ka chɔni Pəm Taks ɛ, pə ka di Abək agbe̹r abul-abul gbo hã ka saa.} During the time of the Hut Tax War, many Krios were killed, only a few escaped (\citealt{Pichl1967}). 

\TCheadword{a-a} [aɂa] \textit{cf}: \TClink[1]{be}, \TClink{bɛaan}, \TClink{kakeiŋ}, \TClink[2]{no}, \TClink{sakoo}. \textit{disco} no. \textit{A-a, be nɔbonthɔ nɔ cheni pɛ.} No, there is no helper anymore. \textit{A-a, wɔn pɔ du mɔ wɔ ni ka.} No, he was not raised here.

\TCheadword{aa} \textit{cf}: \TClink{ayo}, \TClink{ee}, \TClink{yɛs}. \textit{disco} \textbf{1)} yes. \textit{Aa, ya Kristian.} Yes, I am a Christian. \textit{Aa, han, a bɛ ŋalɔ.} Yes, them, I put them there. \textbf{2)} ah.

\TCheadword{abaot} (Eng \textit{about}) \textit{adv} about, approximately. \textit{Kɛ ŋanɛ ŋa wuɛwuɛ ni ache pɛ mɛmba hin awɔ ile lɔ, hin awɔ ile lɔɛ... yi abaot amɛnbul.} But some have died so I do not remember how many of us remain, how many of us remain there... we are about six. \textit{Ako ni chaŋ abaut twɛntisiks ŋɔ ani ko mpantho kiɛ.} It's been about 26 years since I've been in this line of work.

\TCheadword{Abas} \textit{nam} Abas, male name given to a person. \textit{Abas Bɛndu}. Abas Bendu.

\TCheadword{Abdul} \textit{nam} Abdul, male name given to a person.

\TCheadword{Abdulai} \textit{nam} Abdulai, male name given to a person. \textit{Yaŋ yaa Abdulai Bɛndu.} I am Abdulai Bendu.

\TCheadword{Abduramani} \textit{nam} Abduramani, male name given to a person.

\TCheadword{abɛna} (der. of \TClink[1]{bɛn}) 

\TCheadword{Abu} \textit{nam} Abu, male name given to a person. \textit{Ba mi wɔ lɔ Abu Bakar Bɛndu.} My father is Abu Bakarr Bendu.

\TCheadword{abulabul} (der. of \TClink{a-}, \TClink[3]{bul}, see \TClink{a-}) 

\TCheadword{Adama} \textit{nam} Adama, female name given to a person. \textit{Ya lɔ Adama Bolomnɔ.} I am Adama Sherbro person (nickname used to differentiate her from other Adamas and emphasize her love of speaking Sherbro). 

\TCheadword{administreshɔn} (Eng \textit{administration}) \textit{n} administration. \textit{Frɔm tu thaozin tɛn, ŋɔ a lɔi ni administreshɔndɛ kunɛ.} In 2010 I entered the administration.

\TCheadword{Afrika} \textit{nam} Africa. \textit{Mɔɛyktu che ki lo hɔ̃ ko-ŋgbïnk pɔk Afrika lɛ.} This is a dilemma common to all Africa (\citealt{Pichl1967}). 

\TCheadword{aftabak} (Eng \textit{afterbirth}) \textit{n} afterbirth. \textit{Aftabakɛ ŋɔ hun gba ki <gbiŋ>.} The afterbirth came and really got stuck <gbiŋ>. \textit{Hɔ ŋa “ma blem wanthɛmdɛ, aftabakɛ nai landɛ ŋɔ kanthka gbaŋ, ŋɔ che bɔ honi."} He said, “Do not blame the woman, the way for the afterbirth was blocked, it was not able to come out.”

\TCheadword{Agnɛs} \textit{nam} Agnes, female name given to a person. \textit{Yaŋ a Agnɛs Jami Simbo.} I am Agnes Jamie Simbo.

\TCheadword{ahel} \textit{cf}: \TClink{kɛbel} (unspec. of \TClink{bel}). \textit{n} (hɔ̃/-) farmhouse (\citealt{Pichl1967}). comp. \TClink{naiahol} (see \TClink[1]{nai}) 

\TCheadword[1]{ahɔl} \textit{n} \textbf{1)} opening, door, mouth; \textit{hial, hial ahɔl} (kɔ/ma) river, mouth of a river, estuary (\citealt{Pichl1967}). \textit{Ya sɛmɛ kil lɛ ahɔl.} I am standing at the door. \textit{Ŋa ŋɔi stich ahɔl, siɛ yɛ komɔɛ wɔ hundɛ honi bo.} They had stitched the opening of the vagina, you know when the baby is about to come, after it is out. \textbf{2)} front. \textit{Yà léé pɛ̀lɛ̀ɛ́ kìllɛ̀ ɔ́hɔ́l kó.} I left the rice in front of the house. \textit{Kache yɛ n yema bo nɔma ni anyamɔɛ kɔlɔ, ŋaa ha lɛŋ kilɛ hɔl ko…} In those days, if you wanted a woman then your people would go there, they would first greet the house front. \textit{Cho koŋ kəthani, wɔ lɛ gboka-nɔ, chen bo chaŋ fay-hɔl ko yɛ theɛ min dɛ wɔ hɔ lɛ.} Cho is perplexed, he is a non-initiate, he cannot pass in front of the Poro bush when he hears the devil talking (\citealt{Pichl1967}). \textbf{3)} hole. comp. \TClink{mukɔhɔl} (see \TClink[1]{muk}) 

\TCsubword{kekɛthihɔl} (comp.) \textit{v} experience; \textit{kekɛ-thihɔl} see for oneself, behold, experience (\citealt{Pichl1967}). \textit{Lanɛ la ya ke̹kɛ-thihɔl nante lɛ la linɛkiɛ.} What I experienced today was painful (\citealt{Pichl1967}).

\TCsubword{nyɔhɔl} (comp.) \textit{n} mouth; \textit{nyɔhɔl} (/thi) mouth (\citealt{Sumner1921}). \textit{Ni <kara-kara kara-kara kara-kara> ni thaŋni boeɛ tokɛ poŋ ni yekeɛ che wɔn nyɔŋhɔl.} And <kara-kara kara-kara kara-kara> scampered up quickly with the cassava in her mouth.

\TCsubword{sumɔhɔl} (comp.) \textit{n} mouth. \textit{Lɛ gbo sïŋma thumɔɛ-ta, wɔ mõe yema nyathi sumɔhɔl.} If you play with a young dog, it will lick your mouth (proverb) (\citealt{Pichl1967}). 

\TCsubword{thɔlɛ} (der.) \textit{n} [thɔ̀lɛ̀] (human) face (K dialect); [itɔ̀lɛ́]/[tɔ̀lɛ̀ thɛ́] face/the faces (B dialect); \textit{thiɔl} (-/tha) face (pl of \textit{hɔl} mostly pronounced “th-hɔl") (\citealt{Pichl1967}).

\TCheadword[2]{ahɔl} \textit{post} \textbf{1)} in the location of, at. \textit{Ŋ kɔ tuu ibəl lɛ shop lɛ ahɔl ni nhã ya si bushɛl liwɔ.} Go measure the palm kernels at the shop and let me know how many bushels there are (\citealt{Pichl1967}) \textbf{2)} in front of. \textit{Ndɛ, ndɛ-m ya sɛmɛ kïl lɛ ahɔl!} Behold me standing at the door! (\citealt{Pichl1967}). unspec. \TClink{sirɔkɔ-hɔl} (see \TClink{rɔk}) 

\TCheadword{-ai} \textit{cf}: \TClink[4]{hɔl}, \TClink{kunɛ} (der. of \TClink{kun}, \TClink[1]{ɛ}). \textit{post} \textbf{1)} in. \textit{Nɔs gbi ŋa ka cheni eriaio ai, hɔspitalai fli nɔs ka che ŋa ni.} There was no nurse in that whole area, even in the hospital there was no nurse. \textit{Yi pɔng ihuk lɛ mən d'ay.} We threw the hook into the water (\citealt{Pichl1967}). \textit{Wɔ theli Mbolomdai, wɔ theli Mpothoai.} He spoke in Bolom, (and) he spoke in English. \textbf{2)} on. \textit{Bahin chala bɛli mɔai.} Our father sits on his throne. \textit{Ŋkɔ gbïl iwɔm dɛ lal l'ay kɔ, jɛmdi lɛ lɔ yema nyum.} Go put wood on the fire, the fire is about to go out (\citealt{Pichl1967}). \textbf{3)} inside of, within the bounds of. \textit{Mpaŋ nwaŋ ni tĩŋ man ma nɛn bul ay ɛ.} There are twelve months in one year (\citealt{Pichl1967}). \textit{Than tha yi hɛ̃y ay si yi yatha si yi kɔ trï lɛ.} In these (canoes) we embark, then we pull the oars and then we go to town (\citealt{Pichl1967}). \textbf{4)} into, to. \textit{Nchi mbɔs pɔkiyai.} Bring peace to our country. \textit{Ponk pia lal lɛ ay ko.} He put his hand into the fire (\citealt{Pichl1967}). \textbf{5)} out from, from out of. \textit{Rïm dɛ kɔ hok tii-kɛtïl l'ay.} The steam comes out of the tea kettle (\citealt{Pichl1967}).

\TCheadword{aida} (Eng \textit{either}) \textit{coordconn} either. \textit{Koromanɔ aida orijin wɔɛ wɔ Maninkanɔ, che Themnɔ wɔɛ.} Koroma, either his origin is Maninka, it is not Themne. 

\TCheadword{ajok} (Krio or Yoruba origin?) \textit{cf}: \TClink{tak}, \TClink{tamɔ} (der. of \TClink{taa}). \textit{n} son. \textit{Ŋà ké àjókwɛ̀ wɔ̀ kímɔ̀.} They saw his son running away. \textit{Pɛnthe-m dɛ wom ajok ko-m ka chencha hã hom mi jali te̹nkate̹nka.} My brother sent his son here to me yesterday to tell me something particular (\citealt{Pichl1967}).

\TCheadword{Aku} \textit{nam} (wɔ/hã, aakua, pl) Aku, name from Yoruba for Muslim Krios (\citealt{Pichl1967}). 

\TCheadword{Albat} (Eng \textit{Albert}) \textit{nam} Albert, male name given to a person. \textit{Ilel miɛ hɔɛ Albat Yanka Pothoai.} My name is Albert Yanker in English. 

\TCheadword{Alfa} \textit{nam} Alpha. \textit{Aa, Bahin, mɔ, mɔlɔ Alfa ni Omɛga.} Yes, Lord, you, you are the Alpha and Omega.

\TCheadword{Alfɔnso} \textit{nam} Alphonso, male name given to a person. \textit{Alfɔnso Kɔka.} Alphonso Caulker.

\TCheadword{alitoma} (der. of \TClink[3]{toma}) 

\TCheadword{alɔ} \textit{post} \textbf{1)} under, underneath, beneath (\citealt{Sumner1921}). \textit{Thethanyi wɔn ka gbo che powɔɛ alɔ ŋa ha mpanth ma yenchɛkɛ.} Our grandmother was just preparing fish (for a crew) under her husband, i.e. he was in charge of. \textit{Yàìyɛ́ wɔ́ kóthàɛ̀ àlɔ̀.} The cat is under the cloth.  \textbf{2)} \textit{bɛthalɛ} lower part of the loins (\citealt{Pichl1967}). \textbf{3)} to. \textit{Mi, nkɔ kil kandɛ alɔ?} Ma, did you go to school? \textit{Be o, akɔ ni kil kaŋ alɔ.} No, I did not go to school. comp. \TClink{Bachalɔ} (see \TClink{bach}) 

\TCheadword{Alusain} \textit{nam} Alusine, male name given to a person. \textit{Ya Mista Alusain.} I am Mr. Alusine.

\TCheadword{Amadu} \textit{nam} Amadu, male name given to a person. \textit{Braima wɔe hun ko kenyaa wɔɛ Ba Amadu Kamara Planti ko, wɔɛ nɔhɔthɔ.} Braima then went to his uncle, Ba Amadu Kamara at Plantain (Island), who is a fisherman.

\TCheadword{Amɛrika} \textit{cf}: \TClink[2]{mɛk}. \textit{nam} America, name given to a place. \textit{Roshia ni Amɛrika hã koŋ kɔnth lo̹mthibul le pəm kɔ koŋ.} Russia and America have made an agreement that war should cease (\citealt{Pichl1967}).

\TCheadword{Aminata} \textit{nam} Aminata, female name given to a person. \textit{Ama ŋa Kadiatu Bɛndu, Isata Bɛndu, Ramatu Bɛndu ni Aminata Bɛndu.} The women are Kadiatu Bendu, Isata Bendu, Ramatu Bendu, and Aminata Bendu.

\TCheadword{Ani} \textit{nam} Annie, female name given to a person. \textit{Triniti Chəəch hɔ̃ kïlkïl Ani Wɔlsh skuul.} Trinity Church is opposite to Annie Walsh School (\citealt{Pichl1967}). 

\TCheadword{Anshɔn} \textit{nam} Anshon, name given to a person, surname.

\TCheadword{Ansu} \textit{nam} Ansu, male name given to a person. \textit{Yaa Ansu kɛ ilel mi gbem kaɛ ŋɔ Baki.} I am Ansu, but my birth name is Baki.

\TCheadword{anti} (Eng \textit{auntie}) \textit{cf}: \TClink[3]{lok}. \textit{n} aunt, auntie. \textit{A bo ka anti miɛ, wɔ dumɔ miyɛ.} I'm just here with my auntie, she raised me.

\TCheadword{aot} (Eng \textit{out}) \textit{Loc} out. \textit{Hɔɛ ika ko hɔm wanthem dɛ woth bo kun ma gbemɔ aot, kɔ gbemɔ hɔspitulai.} He said we had told the woman that whenever she got pregnant, she should not give birth outside (the hospital), she should come to the hospital.

\TCheadword{apa} \textit{cf}: \TClink[1]{ba}. \textit{nam} \textbf{1)} [àpà], [pàpà] father (different from [pà] ‘mister') (K dialect). \textbf{2)} title of respect for a man, may be translated as father, pa, mister, sir. \textit{Apa, ilel mɔa?} Father, what is your name? \textit{Apa ŋɔ ko che kath; kache ŋɔ ka che pɛth.} Pa, it has become difficult; things used to be good. \textbf{3)} \textit{Wɛl, apa, sɛkɛ we, i chɔŋɔ mɔ sɛkɛ we, Abatokɛ che mamɔ.} Well, pa, thank you, we thank you, may God be with you. \textit{Apa, nyema la?} Sir, do you want that?

\TCheadword{apɔint} (Eng \textit{appoint}) \textit{v} appoint. \textit{Pɔ mi apɔint, ya kɔni.} I was appointed, and then I went.

\TCheadword{Arabik} (Eng \textit{Arabic}) \textit{nam} Arabic. \textit{Arabikɛ ŋɔ n kaŋaɛ?} So it is Arabic that you learnt? \textit{Wɔn ka kaŋ Arabikɛ kɛ still ka che famalɛ kunɛ.} He learned Arabic but still he was in this farming.

\TCheadword{arenj} (Eng \textit{arrange}) \textit{v} arrange. \textit{I yema pɛ ni hun yi toŋgi ŋɔ pɔ arenj, ŋɔ pɔ bɛmpa ja Bondoɛ.} We want you to come and show us how to arrange, how to prepare Bondo (rituals).

\TCheadword{Arɛalɛ} \textit{nam} Areale, female name given to a person.

\TCheadword{arijana} (Arabic {\textarab{جنّة} } \textit{jannah} ‘heaven') \textit{n} \textbf{1)} paradise. \textit{Hwɛ moɛkɛ rai ni po hink wuli lɛ ay ni kɔni arijana.} Risen from the dead the third day and gone to paradise (\citealt{Pichl1967}). \textbf{2)} heaven. \textit{Ya lanɛ hɔbatokɛ ba sɛm thibo̦m dɛ wɔ bɛmpa arijana ni hwɛlɔ lɛ.} I believe in God the Almighty Father who made heaven and earth (\citealt{Pichl1967}).

\TCheadword{as} (Eng \textit{as}) \textit{prep} as; like. \textit{Sijismɔn wɔ ka che as bɛiyɛ, nthela nye?} Sigismund was (acting as) the chief, you heard that, right? \textit{Ye pɔ koyi kaŋdɛ pɔɛ nkegbo nɔɛ bi gballɛ kɔ ko kunwɔɛ as Sizaɛ…} When we were taught, they said if you see a mark on the belly like from a Cesarean section…

\TCheadword{asthafula} (Arabic) \textit{disco} God forbid.

\TCheadword{ataims} (Eng \textit{at times}) \textit{temp} occasionally. \textit{A-a, wɔm thi tata bo, kɛ anya yɔl ŋa tha ŋɔth kaɛ kɛ ataims anya tiŋ.} No, they are just small boats, it is four people that fish from them, occasionally two people.

\TCheadword{atok} \textit{cf}: \TClink[6]{ni}. \textit{post} \textbf{1)} on top. \textit{I kɔ sɛm pethɛ atok.} We go and stand on the stones. \textit{Yɛ mɔ ni bɛ yabasɛ atok, mɔi gbiŋgith.} After putting the onions in, then you cover it. \textbf{2)} up in, up. \textit{Kə̀llɛ̀ wɔ́ thɔ̀kɛ̀ àtòk.} The monkey is up in the tree. \textit{Nhã yenkəlɛŋ ŋthɔk lɛ tok ɛ, mma pakali lɛɛ thɔk lɛ thɔm mɔ lɛ ma ki duk.} Be careful you there up in the tree, don't shake the tree branch lest your companion fall (\citealt{Pichl1967}). \textit{Nlɛli ato̹k ka!} Look up this way! (\citealt{Pichl1967}). \textbf{3)} on. \textit{Ŋ kɔ pɔnki tutuŋ dɛ ato̹k.} Go throw it on the dunghill (\citealt{Pichl1967}). \textit{Thɔk lɛ ato̹k.} On the tree/up in the tree (\citealt{Pichl1967}). \textbf{4)} over, above. \textit{Rɛthiɛ kapathi wɔ lɛ yaŋ atok.} He spread his wings over me (\citealt{Pichl1967}). \textbf{5)} about. \textit{Nandɛ ako vel laŋgbaŋ bul ŋa hun wɔ yi ŋalwɔ atokɛ.} Today I have called on a man to come in order to ask him about himself. comp. \TClink{kanaatok} (see \TClink{kana}), \TClink{thibolɔtok} (see \TClink[1]{bol}) 

\TCheadword{atɔl} (Eng \textit{at all}) \textit{quant} at all. \textit{Ŋɔ cheni pɛ bul atɔl.} It is not the same today at all.

\TCheadword{awa} (Soso \textit{awa} ‘okay') \textit{cf}: \TClink{ayo}, \TClink{oke}. \textit{disco} okay. \textit{Awa, nlelɔ lantha.} Okay, hold it there, hang on. \textit{Awa sɛkɛ-sɛkɛ we.} Okay, thank you.

\TCheadword{aya} \textit{disco} alas. \textit{Aya, amaa ŋa pos yeke ko vɛ, ni ndikɛ che mi ka}. Alas, there are women peeling cassava there, and I am hungry here.

\TCheadword{ayen} [àyɪ́n] \textit{indfpro} \textbf{1)} anywhere (K dialect). \textit{Ŋkɔni ayen gbi ha kɔ lɛliɛ yen joo, ni nsiiɛ ya kun dumɔ.} You do not go anywhere to find things to eat, and you know my belly is hard (i.e. I am about to give birth). \textit{Ba Na bɔni pɛ ha kɔ ayen gbi, sɛmɛ gbo ayenal bul.} Mr. Spider was not able to go anywhere at all, he just stood in one place (\citealt{Sumner1921}). \textit{Ya chen kɔ ayen gbi.} I'm not going anywhere. \textbf{2)} somewhere; someplace. \textit{Iwoɛ, iwo itataɛ pɔ ŋɔ pak ayen, pɔ ŋɔ pɛ bia buŋ.} The rice grass stalks, the immature stalks are parked somewhere, people thresh them again. \textit{Ayen lɔlɔ lɔi nan yenchɛkɛ tɛŋka dɔzin tin, dɔzin ra.} There is a place where we draw the fish, like two dozen, three dozen. \textbf{3)} everywhere. \textit{Mɔ wɔ bala-bala ni, wɔn bɛ wɔ mɔ balani, mɔ wɔ kis-kis yɛŋ bɛ, wɔi po ha yɛthi mmɔ ma mɔɛ.} You hug him, he hugs you, you kiss him all over, then he begins to hold your breast.

\TCheadword{ayeŋ} [àyɪ́ŋ] \textit{cf}: \TClink{ter}. \textit{n} middle of a person or creature, waist (B dialect). \textit{Ayeŋ ha wɔɛ hɔ che bisiɛ pen.} His middle is tightly tied (\citealt{Sumner1921}). \textit{Ba Na ka che ayeŋ hã bom kendɛ nvis hã hallɛ, kɛ gbɔlkajo ŋɔ siŋ ka wɔ ayeŋ vɛ.} Mr. Spider used to have a waist as big as the other animals, but gluttony affected his middle very much (\citealt{Sumner1921}). comp. \TClink{chɔlayeŋ} (see \TClink[1]{chɔl}), comp., id. \TClink{paaŋtriayeŋ} (see \TClink[2]{paŋ})

\TCheadword{ayɛn} \textit{adv} \textbf{1)} [àyɛ̀n] truly (K dialect). \textit{Ayɛn biɛ-m bisin.} Truly, he cares for me (\citealt{Pichl1967}). \textbf{2)} indeed. \textit{Ayɛɛn, mbɔn ma lɔ pɔk lo.} Indeed, there is cannibalism in this country (\citealt{Pichl1967}). comp. \TClink{peayɛn} (see \TClink{pe}), \TClink{yekɛayɛn} (see \TClink{yekɛ}) 

\TCheadword{ayɛɛn} \textit{cf}: \TClink[1]{tintin} (der. of \TClink[2]{tintin}). \textit{adj} true (\citealt{Pichl1967}). 

\TCheadword{ayo} \textit{cf}: \TClink{aa}, \TClink{awa}, \TClink{ee}, \TClink{oke}, \TClink{yɛs}. \textit{disco} \textbf{1)} okay. \textbf{2)} yes. \textit{Ay, ya ŋa kee, laa mi!} Yes, I see them, my wife! \textit{Ayo, yɛ pɔ pɛ mi ketheŋ kendɛ yekeɛ ha yeke kiɛ labi ŋhɔɛ ya ka mɔ ŋɔ ni nsɔm?} Yes, when they wanted to cut me like this cassava, that's why you said, “Let me give it to you and you chew?”

\end{letter}
\begin{letter}{B}

\TCheadword[1]{ba} \textit{cf}: \TClink[2]{bɛɛ}. \textit{n} \textbf{1)} father. \textit{Mma buŋ ba mɔ sua!} Don't oppose your father! (\citealt{Pichl1967}). \textbf{2)} master. \textit{Ba mi koŋ kɔn bias ay nante.} My master went on a journey today (\citealt{Pichl1967}). \textbf{3)} chief. \textit{Bɛɛ lɛ Kɔng kol sirɔng hã sɔng wɔ ni kɔ wɔŋ beli li-mbul.} The chief gave Kong a corruption fee to bribe him to go and give false evidence (\citealt{Pichl1967}). \textbf{4)} title of respect for a man, may be translated as father, pa, sir, but also used in Sierra Leone English. \textit{Ba Ke̹l ka hinɛn gbɔl.} Mr Monkey was not satisfied (\citealt{Pichl1967}). \textit{Ba Na sɛmi ka gbɔl bom, gbɔl ka jo.} Mr Spider stood proudly, gluttonously (\citealt{Pichl1967}). \textit{Braima wɔe hun ko kenyaa wɔɛ Ba Amadu Kamara Planti ko, wɔɛ nɔhɔthɔ.} Braima then went to his uncle, Ba Amadu Kamara at Plantain (Island), who is a fisherman. comp. \TClink{kelba} (see \TClink[1]{kel})

\TCsubword{Bahi} (comp.) \textit{nam} Lord, Christian God (lit. our father). \textit{A yiyɛ Bahin ŋa toŋi mi nai wɛ we.} I ask the Lord to show me the way. \textit{Kɔnɛ o Bahin.} Restore (unto us) Our Father. \textit{Oo, Bahin, lahi cha ba ha ba?} Oh, Our Father, what are we doing? 

\TCsubword{bami} (comp.) \textit{nam} Mister (lit. my father). \textit{Bami, yaŋ bɛ ya theeɛ la bɛlsɛ hɔɛɛ, kɛ pɔ chen laanɛ nɔ ka kakeiŋ.} Mister, I too heard what the rats said, but they will not believe anyone else. comp. \TClink{hɔbatokɛ} (see \TClink[1]{tok}) 
 
\TCheadword[2]{ba} \textit{adv} emphatic particle often used with the adverb \textit{yeŋkɛlɛŋ} ‘well.' \textit{Nchekɔ gbo kɛn ni nthɔkɔ kɔnio; mɔ kɔ thɔk yeŋkɛlɛŋ ba.} You do not just cut it without washing it; first you must wash it very well. \textit{Wɔ ma theli, wɔ mɔ ma thɛkɛsiɛ kunɛ yeŋkɛlɛn ba.} He can speak Sherbro, and translates it for you very well. der. \TClink{gbeba} (see \TClink{gbe}), \TClink{yeŋkɛlɛŋba} (see \TClink[1]{kɛlɛŋ}) 

\TCheadword[1]{baa} [bàà] \textit{n} squirrel species (K dialect); (wɔ/hã, si) squirrel species that lives in trees (\citealt{Pichl1967}). \textit{Táàmɔ̀ɛ̀ kɔ́nth bààɛ́.} The boy caught the squirrel.

\TCsubword{bɔtakɛl} (comp.) \textit{cf}: \TClink{kɛko}, \TClink{sɔmbu}. \textit{n} (wɔ/hã, si) squirrel species (\citealt{Pichl1967}).

\TCheadword[2]{baa} [báá] \textit{cf}: \TClink{balmaa}, \TClink{boka}, \TClink[1]{hɔlɔŋ}. \textit{n} [íbáá] curved knife (K dialect); \textit{iba} (hɔ̃/tha) curved knife used to cut palm trees for palm wine (\citealt{Pichl1967}). \textit{Nɔ̀sààɛ́ wɔ̀ bɛ́t bàchɛ̀ kà íbáá.} The tapster tapped the tree with a knife.

\TCheadword{baabalipal} (comp. of \TClink[2]{baba}, \TClink[1]{pal}, see \TClink[2]{baba}) 

\TCheadword{baala} [bààlá] (Port \textit{balai}?) \textit{n} cane basket (K dialect); \textit{balæ} (kɔ/tha) fancy type of cane basket with or without lid (\citealt{Pichl1967}). \textit{Kòní bɛ́ ǹyéék má kómɔ̀wɛ̀ bààlàɛ́-áí.} Koni put the child's things in the basket.

\TCheadword{baama} \textit{n} (hɔ̃/tha) lair of the society spirit who appears as a dancing masquerade and where he retires after a session (\citealt{Pichl1967}). \textit{Mtoindɛ kɔn baama.} The Mtoin spirit has gone home to rest (\citealt{Pichl1967}). 

\TCheadword{baana} \textit{n} (kɔ/ma) banana (\citealt{Pichl1967}). \textit{Abdulai bí jó bàná nrà.} Abdulai ate three bananas. comp. \TClink{boombaana} (see \TClink[1]{boo}) 

\TCheadword{baana ayɛn} \textit{n} (kɔ/ma) plantain (\citealt{Pichl1967}). 

\TCheadword{baana kathil} \textit{n} (kɔ/ma) apple banana (Musa sapientum var.) (\citealt{Pichl1967}). 

\TCheadword{baani} \textit{n} bird species, gregarious small white seabird, [bááníɛ̀], [báámíɛ̀] seabird (K dialect); \textit{baanin} (wɔ/hã, N) small whitish seabird (\citealt{Pichl1967}).

\TCheadword{baar} (Port ?) \textit{cf}: \TClink{fe}, \TClink{kɔpa}. \textit{n} (hɔ̃/tha) denomination of money equal to four shillings (\citealt{Pichl1967}).

\TCsubword{baaryeŋ} (unspec.) \textit{cf}: \TClink{fe}, \TClink{kɔpa}. \textit{n} (hɔ̃/tha) denomination of money equal to two shillings (\citealt{Pichl1967}).

\TCheadword[1]{baata} \textit{v} [bààtà] act foolishly (K dialect). \textit{Táámɔ̀ɛ̀ wɔ́ bààtà.} The boy acts the fool.

\TCsubword[2]{baata} (der.) \textit{adj} [bààtà] foolish (K dialect). 

\TCsubword{batabata} (der.) \textit{adj} fun, funny. \textit{Siŋthɛ tha kache batabata.} The games were a lot of fun.

\TCheadword{baawombaawom} (der. of \TClink[1]{bawom}) 

\TCheadword{Baba} \textit{nam} Baba, male name given to a person.

\TCheadword[1]{baba} \textit{n} (hɔ̃/tha) simple shed, often without enclosure (\citealt{Pichl1967}). 

\TCheadword[2]{baba} [bàbà] \textit{n} umbrella, [bàbà mìɛ́] my umbrella (K dialect); \textit{baaba} (hɔ̃/tha) umbrella (\citealt{Pichl1967}). 

\TCsubword{baabalipal} (comp.) \textit{n} (hɔ̃/tha) umbrella, parasol (lit. sun umbrella) (\citealt{Pichl1967}).

\TCheadword[3]{baba} [bàbà] \textit{n} good-for-nothing (K dialect). \textit{Wɔ́ bàbà!} He is worthless (a good-for-nothing)!

\TCheadword{babɔŋ} [bàbɔ́ŋ] (Port ?) \textit{n} stone jug (an old word) (K dialect); (hɔ̃/tha) stone jug of European origin, mainly used to contain medicines (\citealt{Pichl1967}). \textit{Nɔ̀mààɛ̀ thɔ̀ndɔ̀ mmɛ̀ndɛ̀ bàbɔ́ŋdàì.} The woman keeps water in the jug. 

\TCheadword{bach} \textit{n} \textbf{1)} [bàch] palm species, short palm (K dialect). \textit{Nɔ̀sààɛ́ wɔ̀ bɛ́t bàchɛ̀ kà íbáá.} The tapster tapped the tree with a knife. \textbf{2)} (hɔ̃/hɔ, i) palm species, young oil palm (\citealt{Pichl1967}). \textit{Walli hoolɛ lɔ gbo kɔ ibach lɛ}. Palm fiber is found only among young palm trees (\citealt{Pichl1967}).

\TCsubword{Bachalɔ} (comp.) \textit{nam} Bachalo, name given to a place on the western edge of Sherbro Island, Bonthe District (lit. place of short palm). \textit{Veɛni ka che Bachalɔ ko, ko kil Madamdɛ Bachalɔ ko.} He did not stay long and he was staying at Bachalo, at Madam's (Paramount Chief's) house at Bachalo.

\TCheadword{Bachalɔ} (comp. of \TClink{bach}, \TClink{alɔ}, see \TClink{bach}) 

\TCheadword{Bahi} (comp. of \TClink[1]{ba}, \TClink{hi}, see \TClink[1]{ba})

\TCheadword[1]{bai} \textit{cf}: \TClink[2]{baŋ}, \TClink{kuku}. \textit{n} \textbf{1)} bari. \textit{Pɔ bɛ wɔ ŋgbekteɛ ni pɔ sɛmi wɔ bai ko anyaɛ gbi chee lɔ pɔ bi ha thoŋka wɔ.} They put him in handcuffs and brought him to the bari in front of all the people where they will judge him. \textbf{2)} court of law. \textit{Mbo̹lo̹m ŋwɛi ma che paalɛ bai ko, anya atïŋ dɛ hã lo̹l.} In the bad case that was recently before the court, the two men were set free (\citealt{Pichl1967}). \textit{Mɔ thonka tɛm gbi, kə nchen kɔ bay ko no pə si lɛ mɔ lɛ nɔ-thonka.} You are arguing all of the time, but you don't go to court to show them that you are a lawyer (\citealt{Pichl1967}). \textbf{3)} temporary abode for Poro novices upon leaving the \textit{bankaŋ dɔ}. They stay for four days where the last formalities such as naming and solemn vows are made to the \textit{sokos} to respect their parents, older persons and to be obedient to them. After this their parents or guardians are allowed to take them home (\citealt{Pichl1967}). comp. \TClink{kaŋbay} (see \TClink[1]{kaŋ}), \TClink{kolbai} (see \TClink{kol}), \TClink{suibaɛ} (see \TClink{sui}) 

\TCheadword[2]{bai} (Eng \textit{by}) \textit{prep} by, into. \textit{Themnɔ bai koinsidɛnt ŋɔ ŋa sɔthɔ Koromaɛ vɛ.} The Themne got (the surname) Koroma by accident. \textit{Pɔ koŋ gbo, ŋa koŋ kɔ gbo yɔk ti thai, pɔ kɔ pak bai thikranthikran thibombom.} After taking it to the farmhouses/towns, it would then be piled up into different sections into very big piles.

\TCheadword{Baiyikɛ} \textit{cf}: \TClink{Barikɛ}. \textit{nam} Baiyike, name given to 5\textsuperscript{th} son.

\TCheadword[1]{bak} \textit{v} rub on, smear (\citealt{Sumner1921}). \textit{Wɔ kuyɛ pomthilɛ wɔ hɔ̃ wo ye bak bolwɔlɛ.} He takes the leaves and then rubs them into his (the other's) head (\citealt{Pichl1967}). \textit{Bak mi rɔm ndɛ.} Rub the medicine on me.

\TCsubword{bakni} (der.) \textit{v} embrocate oneself, annoint oneself, rub on oneself (\citealt{Pichl1967}). \textit{Yɛ̀ kóŋ thɔ̀n dɛ̀, wɔ̀è bání kùáɛ́ njáláí.} After bathing she rubbed oil on her skin.

\TCheadword[2]{bak} (Eng \textit{back}) \textit{adv} back. \textit{Ayema lɔ chɔlɔ, ayema lɔ chɔlɔ bak.} I want to fight for it, I want to fight back for it. 

\TCheadword{Bakan} \textit{nam} [bàkàn] Bakan, name given by Poro Society (K dialect).

\TCheadword{Bakar} \textit{nam} Bakarr, male name given to a person. \textit{Ba mi wɔ lɔ Abu Bakar Bɛndu.} My father is Abu Bakarr Bendu.

\TCheadword{Baki} [bàkí] \textit{nam} Baki, name given to 4\textsuperscript{th} son. \textit{Yaa Ansu kɛ ilel mi gbem kaɛ ŋɔ baki.} I am Ansu, but my birth name is Baki. \textit{Baki wɔ ŋkïl.} Baki is a rascal (\citealt{Pichl1967}).

\TCheadword{bakni} (der. of \TClink[1]{bak}, \TClink{-ni}, see \TClink[1]{bak}) 

\TCheadword[1]{bal} \textit{n} \textbf{1)} compensation for adultery. \textit{Hálíwɔ̀ hìn má Yèmà, wɔ̀ pín bállɛ̀ kò Chó.} Because he slept with Yema, he paid \textit{bal} to Cho. \textbf{2)} (kɔ/-) dispute before chief or elders about adultery (\citealt{Pichl1967}). \textit{Yi thɛlɛn baal lɛ, kong balani.} We asked the chief because of the adultery dispute, and he has consented (\citealt{Pichl1967}).

\TCsubword[1]{homabal} (unspec.) \textit{n} (kɔ/-) compensation for adultery, the accused has to pay an indemnity to the betrayed husband and the sum is fixed by the judge (\citealt{Pichl1967}). 

\TCheadword[2]{bal} \textit{v} become involved in an adultery dispute (\citealt{Pichl1967}).

\TCsubword[2]{homabal} (unspec.) \textit{v} become involved in an adultery dispute (\citealt{Pichl1967}). 

\TCheadword{bala} \textit{v} hug.

\TCsubword{balabala} (der.) \textit{v} hug. \textit{Mɔ wɔ bala-bala ni, wɔn bɛ wɔ mɔ balani, mɔ wɔ kis-kis yɛŋ bɛ, wɔi po ha yɛthi mmɔ ma mɔɛ.} You hug him, he hugs you, you kiss him all over, then he begins to hold your breast.

\TCsubword{balani} (der.) \textit{v} \textbf{1)} consent, accept. \textit{Yi thɛlɛn baal lɛ, kong balani.} We asked the chief because of the adultery dispute, and he has consented (\citealt{Pichl1967}). \textbf{2)} hug. \textit{Yááɛ̀ bàlàní kòmɔ̀wɛ́.} The mother hugged her child. \textit{Mɔ wɔ bala-bala ni, wɔn bɛ wɔ mɔ balani, mɔ wɔ kis-kis yɛŋ bɛ, wɔi po ha yɛthi mmɔ ma mɔɛ.} You hug him, he hugs you, you kiss him all over, then he begins to hold your breast.

\TCheadword{balabala} (der. of \TClink{bala}) 

\TCheadword{Balaka} \textit{nam} [bàlàkà] Balaka, male name given by a society (K dialect).

\TCheadword{balani} (der. of \TClink{bala}, \TClink{-ni}, see \TClink{bala}) 

\TCheadword{balansbɔl} (Eng \textit{balance ball}) \textit{n} balance ball. \textit{Dɛn yɛ ibɛ nkɔkaɛ lɛko nyɔn doki ŋɔ pɔ vellɛ balansbɔllɛ.} Then we would put our shoes on the ground (for) this thing (game) they called balance ball.

\TCheadword{Bale} \textit{nam} (wɔ/-) Bale, female name given by Toma Society (\citealt{Pichl1967}). 

\TCheadword[1]{bali} \textit{n} \textbf{1)} wealth, riches. \textit{Jizɔs, ya mɔnɛ ni mbali mi.} Jesus, I am poor so make me rich. \textbf{2)} productivity. \textit{Nsaŋhaɛ ma ka che chaŋ bali ha chaŋ nyiki halɛ gbi.} The egusi grew more than all the other plants. der. \TClink{nɔbalia} (see \TClink{nɔ}), \TClink{nɔbaliabalia} (see \TClink{nɔ})

\TCsubword[2]{bali} (der.) \textit{v} be wealthy. \textit{Sese Mpondo, ŋkong bali hiɛ̃?} Sese Mpondo, you are rich now, aren't you? (\citealt{Pichl1967}). comp. \TClink{nɔbalia} (see \TClink{nɔ}), der. \TClink{nɔbaliabalia} (see \TClink{nɔ})

\TCsubword[3]{bali} (der.) \textit{adj} prosperous, luxuriant in growth (\citealt{Sumner1921}). 

\TCheadword{balmaa} \textit{cf}: \TClink[2]{baa}, \TClink{boka}, \TClink[1]{hɔlɔŋ}, \TClink[2]{saki}. \textit{n} (kɔ/ma) small two-edged knife formerly used by men for self-defense (\citealt{Pichl1967}). \textit{Bálmá lúɛ́ lítìŋ.} The \textit{balmaa} knife is sharp on both (edges) (lit. The sharp \textit{balmaa} is double).

\TCheadword[1]{balon} \textit{v} [bálón] tie roof rafters before putting on any covering, be it straw or zinc (K dialect). \textit{Yɛ̀ pɔ̀ kóŋ gbó bálón bellɛ, pɔ̀ bɛ́ wùsɛ̀, pɔ̀ ŋɔ̀ bím.} When they have finished tying the rafters of the farmhouse, they put on the thatch, they cover it.

\TCheadword[2]{balon} (Eng \textit{balloon}) \textit{cf}: \TClink{plɛn}, \TClink{wɔmtokɛ} (comp. of, id. of \TClink[2]{wɔm}, \TClink[1]{tokɛ}). \textit{n} [bàlón] airplane (K dialect). 

\TCheadword{balun} \textit{n} (wɔ/hã, N) snake species said to be very poisonous and able to jump very high (Bothrophthalmus lineatus?) (\citealt{Pichl1967}). 

\TCheadword{bama} \textit{v} [bàmà] tell an egregious lie about someone (K dialect). \textit{Wááŋmàɛ̀ bàmá thɔ̀mwɛ̀.} The girl lied about her companion.

\TCheadword{bami} (comp. of \TClink[1]{ba}, \TClink[1]{mi}, see \TClink[1]{ba})

\TCheadword[1]{ban} \textit{n} bundle. \textit{Pɔ koŋ kɔ gbo futh, pɔ kɔi panth thiban pɔ woth kɔ bolɛ.} After they have uprooted it, they have to tie it into a sheaf and carry it on the head. 

\TCsubword[2]{ban} (der.) \textit{cf}: \TClink[1]{bas}. \textit{v} \textbf{1)} gather together. \textit{Ikoŋ gbo, iban mthɔkɛ manɛ malɔ.} When we have finished, we have to gather all of those sticks. \textbf{2)} sweep together; gather in a place to throw away, as leaves being gathered from sweeping (K dialect); sweep (\citealt{Pichl1967}).   comp. \TClink[1]{nyamban} (see \TClink[1]{ban})

\TCsubword[1]{nyamban} (comp.) \textit{v} carry away. \textit{Lalaɛ kɔ wɔe hɛthni mmɛn nyamban doai ni kɔ kɔni hiŋk wɔn.} His paddle slipped from him, the water carried it away from him.

\TCheadword[3]{ban} [bán] \textit{n} drum type (K dialect); (hɔ̃/tha) big standing drum beaten with one drumstick (\citealt{Pichl1967}). \textit{Táámòɛ̀ wɔ́ lɔ́k bándɛ̀ wɔ̀ kí.} This is the boy who plays the drum. 

\TCheadword{Banabum} [bánábùm] \textit{nam} Banabum, male name given to a person (K dialect).

\TCheadword{bandɛ} \textit{n} dressing. \textit{Lanɛ la hɔni thɔndɛaiɛ, la chen hɔni bandɛai.} What is said while bathing is different from what is said while getting dressed (proverb) (\citealt{TISLL1979}). 

\TCheadword{bandri} (Eng \textit{boundary}) \textit{n} boundary. \textit{Lɔ pɔ wɔ bandriɛ veleŋkoɛ ŋɔ lɔ sekoɛ Shechiɛ.} People there say the boundary is over there, they say it is in Sittia. 

\TCheadword{banga} \textit{n} musk cat. \textit{Gbel kɔn gbes, banga le bɔn lom.} When leopard goes east, musk cat remains lifting its tail (proverb) (\citealt{TISLL1979}). 

\TCheadword[1]{baŋ} \textit{v} be lazy (\citealt{Pichl1967}).

\TCsubword[1]{libaŋ} (der.) \textit{cf}: \TClink{tama}. \textit{n} (lɔ/-) laziness (\citealt{Pichl1967}).
.
\TCsubword[2]{libaŋ} (der.) \textit{adj} lazy (\citealt{Pichl1967}, \citealt{Sumner1921}). \textit{Tamolɛ wɔ libaŋ.} The boy is lazy (\citealt{Pichl1967}). 

\TCheadword[2]{baŋ} \textit{cf}: \TClink[1]{bai}, \TClink{kuku}. \textit{n} \textit{mbang} (ma) temporary shed erected for Poro novices when they are taken out to the bush (\citealt{Pichl1967}).

\TCheadword[3]{baŋ} [bàŋ] \textit{n} bird species, yellow and black weaver bird, also called palm bird (K dialect); (wɔ/hã) compact weaver bird, small reddish or yellowish (Pachyphantes pachyrrhynchus), if sits on the branches near the house, he brings good luck, if he flies away, it means bad luck (\citealt{Pichl1967}). \textit{Mbàŋsɛ̀ ŋà rɪ́k wàɛ̀ tòkɛ̀.} The weaver birds wove (their nests) at the top of the palm tree.

\TCsubword{baŋsakɔ} (comp.) \textit{n} [bàŋsàkɔ] bird species (K dialect). 

\TCheadword[4]{baŋ} (der. of \TClink[2]{bas}) 

\TCheadword[5]{baŋ} [báŋ] \textit{cf}: \TClink{blem}, \TClink{thɛkɛ}. \textit{v} blame (K dialect). \textit{Báŋ wɔ́ là.} Blame him/her for it.

\TCheadword[6]{baŋ} \textit{n} hole.

\TCheadword[1]{baŋchoŋ} \textit{cf}: \TClink{honchoŋ}. \textit{v} float about (\citealt{Pichl1967}).

\TCheadword[2]{baŋchoŋ} \textit{v} auction (\citealt{Pichl1967}).

\TCheadword{baŋgawa} \textit{n} (kɔ/ma) plant species, shrub (Microglossa volubilis) (\citealt{Pichl1967}). 

\TCheadword{Baŋgura} \textit{nam} Bangura, name given to a person, surname. 

\TCheadword[1]{baŋk} \textit{n} (wɔ/hã, N) fish species (\citealt{Pichl1967}).

\TCheadword[2]{baŋk} \textit{n} \textbf{1)} (kɔ/ma) rope, line (\citealt{Pichl1967}). \textit{Ni wɔ koi mbaŋɛ mbul-mbul, ni sik ni ayen.} And he took the ropes, one-by-one, and tied them around his middle (\citealt{Sumner1921}). \textit{Ni apimaɛ hã koi mbaŋgɛ bul-bul ni hã kɔ tri thɛai bul-bul.} Then his children had to take each of the ropes and go to each village (\citealt{Sumner1921}). \textbf{2)} plant species, vine used to tie things (\citealt{Sumner1921}). \textit{Nan baŋk, baŋk nan tho.} Pull a vine, and the vine pulls the bush (proverb) (\citealt{TISLL1979}). \textbf{3)} line or column of driver ants on the march, often after a heavy rain (K dialect). 

\TCsubword{baŋkgbɔl} (comp.) \textit{cf}: \TClink[2]{saŋgba}. \textit{n} (kɔ/ma) necklace, string of beads worn around the neck (\citealt{Pichl1967}). 

\TCsubword{baŋkihɔlɔŋ} (comp.) \textit{cf}: \TClink{baŋksit} (unspec. of \TClink[2]{baŋk}). \textit{n} (kɔ/ma) line used to hoist a sail, halyard (lit. life rope) (\citealt{Pichl1967}).

\TCsubword{baŋkmin} (comp.) \textit{n} [báŋkmìn] rope used to control a spirit who appears as a dancing masquerade (lit. spirit rope) (K dialect).

\TCsubword{baŋktel} (comp.) \textit{n} (kɔ/ma) rope or cord worn around the waist, belt (\citealt{Pichl1967}). 

\TCsubword{baŋksit} (unspec.) \textit{cf}: \TClink{baŋkihɔlɔŋ} (comp. of \TClink[2]{baŋk}). \textit{n} (kɔ/ma) line used to hoist a sail, halyard (\citealt{Pichl1967}).

\TCheadword[1]{baŋka} \textit{cf}: \TClink[2]{baŋka}. \textit{n} [bàŋká] bush cat that eats chickens and palm fruit (K dialect); (wɔ/hã, si) African civet (Viverra civetta) (\citealt{Pichl1967}). \textit{Bàŋkáɛ́ wɔ̀ kɔ́ náìbòl áyéná bùl.} The \textit{baŋka} civet always defecates in the same place.

\TCheadword[2]{baŋka} [bàŋká] \textit{cf}: \TClink[1]{baŋka}. \textit{n} fruit with an offensive smell, perhaps named after civet because of its offensive smell (K dialect).

\TCheadword{Baŋkaŋ} \textit{nam} Bankang, male name given to a person. \textit{Thipïk isɔ lo Bankaŋ wɔ gbo thimini ka.} On from this morning, Bankang was loitering around (\citealt{Pichl1967}). 

\TCheadword{baŋkathɔm} [bàŋkàthóm] \textit{n} wasp species, possibly Bomboli wasp (K dialect). 

\TCheadword{baŋkbuk} (comp. of \TClink[2]{buk}) 

\TCheadword{Baŋkelen} [báŋkèlèn] \textit{nam} March, time of year it is too hot to even walk on a pathway (K dialect). 

\TCheadword{baŋkgbɔl} (comp. of \TClink[2]{baŋk}) 

\TCheadword{baŋkihɔlɔŋ} (comp. of \TClink[2]{baŋk}) 

\TCheadword{baŋkiŋ} \textit{n} novice, one who is not yet a full member of Poro Society (\citealt{Pichl1967}). 

\TCheadword{baŋkmin} (comp. of \TClink[2]{baŋk}, \TClink[3]{min}, see \TClink[2]{baŋk}) 

\TCheadword{baŋksit} (unspec. of \TClink[2]{baŋk}) 

\TCheadword{baŋktel} (comp. of \TClink[2]{baŋk}, \TClink{tel}, see \TClink[2]{baŋk}) 

\TCheadword{baŋsakɔ} (comp. of \TClink[3]{baŋ}, \TClink{saŋka}, see \TClink[3]{baŋ}) 

\TCheadword{baom} \textit{cf}: \TClink[1]{bɛn}, \TClink[2]{veleŋ} (der. of \TClink[1]{veleŋ}). \textit{n} \textbf{1)} [bàóm] grandfather (B dialect); [bàhóm], [bàóm] grandfather (K dialect); \textit{baawom} (wɔ/hã, N) grandfather (\citealt{Pichl1967}). \textbf{2)} ancestor. \textit{Tɛn po mbawom o.} The fable rose from the ancestors (\citealt{Pichl1967}). \textbf{3)} \textit{mbaawom} (-/ha) Poro Society spirits who appear as dancing masquerades (\citealt{Pichl1967}).

\TCsubword{baombaom} (der.) [bàómbàóm] \textit{n} \textbf{1)} great grandfather (K dialect). \textbf{2)} great grandparents. \textbf{3)} \textit{bawombawom} (wɔ/hã) ancestor (\citealt{Pichl1967}). 

\TCheadword{barai} [bàráí] (Arabic {\textarab{بارك} } \textit{barak} ‘bless') \textit{v} bless (K dialect). \textit{So Abatokɛ barai mɔ la.} So may God bless you for that.

\TCheadword{Barikɛ} \textit{cf}: \TClink{Baiyikɛ}. \textit{nam} Barike, name given to 5\textsuperscript{th} son. \textit{Tɔŋ wɔ che sïnk thɔk kɔ tɔkɔli ka hɔl lɛ Barïkɛ.} Tong was playing with a stick which he unintentionally stuck in Barike's eye (\citealt{Pichl1967}). 

\TCheadword{Baro} \textit{nam} Baro, name given to a person. \textit{Ya mi ka lɔ pɔ gbem wɔɛ, bawɔ ilel wɔ ŋɔ Pa Baro.} My mother was born here, her father's name was Pa Baro.

\TCheadword[1]{bas} \textit{cf}: \TClink[2]{ban} (der. of \TClink[1]{ban}), \TClink{bɛs}. \textit{v} sweep. \textit{Mbas kil lɛ charaŋ!} Sweep the house clean! (\citealt{Pichl1967}). \textit{ Kenki na isɔki pɔi hɔ ha bas.} It was like this in the morning, (so) they said to sweep. 

\TCsubword{Basmachin} (comp.) \textit{nam} [bàsmàchín] August because that is when the rains come and sweep the fields clean (lit. sweep fields) (K dialect); \textit{basmanchin} (hɔ̃/-) August (\citealt{Pichl1967}). 

\TCheadword[2]{bas} [bàs] \textit{n} spear (K dialect); \textit{ibas} (hɔ̃/ma) spear, harpoon (\citealt{Pichl1967}). \textit{Báŋ ìbàsɛ̀.} Stab with a spear.

\TCsubword[4]{baŋ} (der.) [báŋ] \textit{cf}: \TClink[2]{chu} (der. of \TClink[2]{bas}). \textit{v} \textbf{1)} stab, poke, puncture (K dialect). \textit{Báŋ ìbàsɛ̀.} Stab with a spear. \textbf{2)} hammer (K dialect). \textbf{3)} nail. \textit{Pɔ baŋ wɔ ko thɔkɛ, pɔ chu wɔ wɔn kumbɛ.} They nailed him on the cross, they stabbed him on his side. 

\TCheadword{basa} \textit{n} species of eel, edible, whose head is used secretly by hunters to train their dogs, [bàsàɛ́], [mbàsàɛ́] eel (K dialect); \textit{basæ} (wɔ/hã, N) species of eel, conger eel (Conger conger) (\citealt{Pichl1967}). 

\TCheadword{Basmachin} (comp. of \TClink[1]{bas}) 

\TCheadword{batabata} (der. of \TClink[1]{baata}) 

\TCheadword{bath} \textit{n} area where rice is grown, the grass is turned with mud to make a fertilizer that can be spread on a field, turned and aerated over the course of a month (K dialect); wide grassy and swampy field mainly used for rice cultivation (\citealt{Pichl1967}). \textit{Pɛ̀lɛ̀ɛ̀ kɔ́ gbér bàthàì.} There is a lot of rice in the swamp. 

\TCheadword{bathaŋ} \textit{n} [bàthàŋ] plant species, vinelike plant with thorns, if pricked resultant sores can get infected (K dialect); (kɔ/ma) plant species (Hymenocardia lyrata) (\citealt{Pichl1967}). 

\TCheadword{Bathkanu} \textit{nam} Batkanu, name given to a place, chiefdom headquarters for Libeisaygahun Chiefdom, Bombali District. \textit{Bathkanu lɔ ka che kɔ skullɛ.} Bathkanu is where he went to school.

\TCheadword{baya} \textit{n} (wɔ/hã, N) traitor (\citealt{Pichl1967}).

\TCheadword{Baybul} (Eng \textit{Bible}) \textit{nam} Bible. \textit{Baybul lɛ hɔ lɛ Sent Pawl ka che-lɛɛ ni ke ka thihɔl yɛ pə ka vɛɛy Sent Stivin.} The Bible says that St. Paul was present and saw with his (own) eyes when they stoned St Stephen (\citealt{Pichl1967}). 

\TCheadword[1]{be} \textit{cf}: \TClink{a-a}, \TClink{kakeiŋ}. \textit{quant} \textbf{1)} no, not any, none. \textit{Be pɛ nɔ kedɛ wɔn.} There is not any like him anymore. \textit{Be pɛ nɔ cheni wɔ mɔilɛ Jizɔs Kraist.} There is no other person who compares to Jesus Christ. \textbf{2)} to no extent; not at all. \textit{Lɔn lɔ pɔ chema bo wɔ kɛt-kɛt.} It is only there (the coastal areas) where people do not at all speak it regularly. \textit{Be nɔbonthɔ nɔ cheni pɛ.} There is no longer any helper. \textbf{3)} at all. \textit{Chaŋgbo lɛ abi bo fe, akɔ pin kɔtin.} If I have (any) money at all, I will buy cotton (cloth).

\TCsubword[1]{beyen} (comp.) \textit{indfpro} nothing (K dialect). \textit{Yá bì(y)ɛní.} I have nothing. \textit{Chala bo che ŋa beyen.} She is just sitting down doing nothing.

\TCsubword[2]{beyen} (comp.) \textit{adj} empty. \textit{Bithir lɛ hɔ̃ beye̹n.} The bottle is empty (\citealt{Pichl1967}). 

\TCsubword{benɔ} (der.) \textit{indfpro} no one. \textit{I taŋ ŋa loli benɔ ŋa bɔnth.} We cry for rescue but there is no one to help.

\TCheadword[2]{be} \textit{adv} \textbf{1)} ever. \textit{Wɔn kɛndɛ vɛ wɔ asɔthɔ bo prɔblɛm.} That is the only problem I ever had. \textbf{2)} only. \textit{I ka che amɛŋra kɛ ile nibo amɛn.} We were eight but only five of us are left. \textit{Kɛ sinthɛ vɛ bo tha ŋa kache siŋɛ?} Are those games the only ones you used to play? \textbf{3)} just. \textit{I bo ŋa ka ha limani.} We just need to give them respect. \textit{Beo, a bo pin agbaŋ ŋa.} No, I just buy and smoke them.

\TCheadword{bel} \textit{n} farmhouse. \textit{Yɛ̀ pɔ̀ kóŋ gbó bálón bellɛ, pɔ̀ bɛ́ wùsɛ̀, pɔ̀ ŋɔ̀ bím.} When they have finished tying the rafters of the farmhouse, they put on the thatch, they cover it. \textit{Yi koŋ gbo rɔki si yi ɛ thɔnk pəlɛ be̹l l'ay.} After having harvested it, we put up the rice in the farmhouse (\citealt{Pichl1967}). 

\TCsubword{kɛbel} (unspec.) \textit{cf}: \TClink{ahel}. \textit{n} farm. \textit{Pi kəbe̹l ko pali lo.} He was on the farm the whole of today (\citealt{Pichl1967}).

\TCheadword{bela} [bélà] (Port \textit{vela} ‘sail') \textit{n} sail (K dialect); (hɔ̃/tha) sail (\citealt{Pichl1967}). 

\TCheadword{bele} \textit{cf}: \TClink{beleŋ}. \textit{n} [bèlè] game that involves hiding a disk, thought by some to induce laziness and sloth (K dialect). comp. \TClink{mathbele} (see \TClink{math}) 

\TCheadword{beleŋ} \textit{cf}: \TClink{bele}. \textit{n} double. \textit{Beleŋ digber kɔ chendɛ bisi.} Many double (much talking, redoubling?) is not tightening up (the issue) (proverb) (\citealt{TISLL1979}).

\TCheadword{beleŋhɛni} \textit{v} wrap up. \textit{Wɔe keɛ yen ŋɔ beleŋhɛniɛ.} And he sees this something is wrapped up.

\TCheadword{benɔ} (der. of \TClink[1]{be}, \TClink{nɔ}, see \TClink[1]{be}) 

\TCheadword{bente} [bènté] \textit{cf}: \TClink{bik}, \TClink[4]{bobo}. \textit{n} mat used to carry a corpse to the grave, though not buried with the body (K dialect). 

\TCheadword{Bentu} \textit{nam} Bentu, female name given to a person. \textit{Bentu, wɔn wɔ Nsanda ko.} Bentu, she is in Sanda.

\TCheadword{beŋbeŋ} \textit{n} \textbf{1)} [bə̀ŋbə̀ŋ] pegs used to stretch and hold taut the warp strings when weaving a mat or ropes used when building a house to keep the walls straight (K dialect). \textbf{2)} (hɔ̃/tha) heap or mound, e.g., for planting potatoes, yams, etc. (\citealt{Pichl1967}).

\TCheadword{beo} \textit{disco} no. \textit{Beo, akɔ ni kil kaŋ alɔ.} No, I did not go to school.

\TCheadword{berɛ} [bérɛ́] \textit{cf}: \TClink{thɛkika}. \textit{n} axe, [bérɛ́]/[bérɛ́thɛ́] axe/axes (K dialect). \textit{Itu lo hɔ̃ kələŋ hã cho' thigbe̹r ɛ.} This iron is good for making axes (\citealt{Pichl1967}). 

\TCheadword{bes} \textit{n} ladder, [bès]/[bèsthɛ̀] ladder/ladders (K dialect).

\TCheadword{betaa} \textit{n} (wɔ/hã, N) fish species, sea porcupine (Diodon, Chilomyctorus antennatus) (\citealt{Pichl1967}).

\TCheadword{beth} \textit{n} \textbf{1)} (hɔ̃/tha) box; chest (\citealt{Pichl1967}). \textit{Yema kɔ be̹th awante l'ay chena lɛ lɛliɛ yen koŋ wusi be̹th l'ay lɔn gbi nyək lɛ ma gbo sẽyɛni hinth l'atok.} Yema went into her sister's box to find that the box had been ransacked and all the things were scattered about on the bed (\citealt{Pichl1967}). \textbf{2)} (hɔ̃/tha) plank, wooden board (\citealt{Pichl1967}). \textit{Be̹th lo hɔ̃ tith.} This plank is thick (\citealt{Pichl1967}). 

\TCheadword[1]{beyen} (comp. of \TClink[1]{be}, \TClink[1]{yen}, see \TClink[1]{be}) 

\TCheadword[2]{beyen} (comp. of \TClink[1]{be}, \TClink[1]{yen}, see \TClink[1]{be}) 

\TCheadword[1]{bɛ} \textit{cf}: \TClink[1]{-n}, \TClink{delma}, \TClink{fili}, \TClink{ivin}, \TClink[1]{mu}. \textit{adv} \textbf{1)} imminently. \textit{Wɔ bɛ hun.} He is just now coming (\citealt{Pichl1967}). \textit{La yeyɛn bɛ pel lɛ pəloɛ.} It was just after the egg was broken (\citealt{Pichl1967}). \textbf{2)} also. \textit{Apumahiyɛ bɛ ŋa po bo ŋa labi ŋa the la.} Our children will also hear it when they grow up. \textit{Ipuluk ɛ bɛ hɔ̃ tipɛ ho.} The grass also begins to sprout (\citealt{Pichl1967}). \textbf{3)} even. \textit{Chencha bɛ ya kɔɛ akɔni poi.} Even yesterday when I went, I didn't go early. \textit{Bul bɛ nche ŋɔ loni ntenɛ?} Can you not even remember one?

\TCheadword[2]{bɛ} \textit{cf}: \TClink{wɔŋ}. \textit{v} \textbf{1)} send, usually to school. \textit{Yami ka bɛmi skul, kɛ akɔni livil.} My mother sent me to school, but I didn't go far. \textit{Aa, pɔ ka bɛ mi Arabik.} Yes, they sent me to an Arabic (school). \textbf{2)} add. \textit{Ŋgɛtiɛ malɔ gbo mɔ bɛ nton.} If you have groundnut, add a little. \textit{Pɔi tholi ni pɔ yɛthiɛ ŋɔ, pɔi bɛ pothɛ.} They put it down and lower it, and then they add the dirt. \textbf{3)} put. \textit{Yɛ mɔ bɛ lalakoɛ jɛmdɛ lɔlɔ bo shi che kɔ ma ki he.} After you put it on the hearth, the fire would be just so, so that it (rice) would not burn. \textit{Wɔ ye bə hɔ̃ pan do ki kunɛ.} He put it into that pan (\citealt{Pichl1967}). 

\TCsubword{bɛŋa} (der.) \textit{v} send. \textit{Ŋan gbi nbɛŋa kaŋdai?} Did you send all of them to school?

\TCsubword{bɛrɛ} (unspec.) \textit{v} add. \textit{Mbɛ̀rɛ̀ mì.} Add something for me. \textit{Yɛ ikoŋ mpanthɛ ma aŋaɛ, yai tipɛ pɛni pɛ ha bɛrɛ kaŋ miyɛ Champ ko ni.} When I finished the work we were doing, I started learning to add to my education in Freetown.

\TCsubword{bɛrɛlɔni} (unspec.) \textit{v} add oneself/itself. \textit{Ni jali tilaŋ gbi labiŋa bɛrɛlɔni.} And all these other things should be added on.

\TCsubword{bɛrmani-bɛrmani} (unspec.) \textit{v} help. \textit{Mmɛn dɛ ni ihɛn dɛ che wɔ bɛrmani-bɛrmani.} The water and the breeze is helping him. 

\TCheadword[3]{bɛ} \textit{post} near. \textit{A kɔni tho bɛ.} I go near the bush (\citealt{Pichl1967}). comp. \TClink{lɛbɛ} (see \TClink{lɛɛ}) 

\TCheadword[4]{bɛ} \textit{pro} emphatic reflexive pronoun. \textit{Wɔn bɛ salima ko lɔ ka cheɛ.} She herself used to be in Salima. \textit{Bikɔ ŋan bɛ ŋai lɛli kani.} Because they themselves are the ones we look up to.

\TCheadword{bɛaan} \textit{cf}: \TClink{a-a}, \TClink[2]{no}, \TClink{sakoo}. \textit{disco} no! (\citealt{Pichl1967}). 

\TCheadword[1]{bɛɛ} \textit{n} \textbf{1)} kingdom. \textit{Yɛ mɔ kɔni bɛɛ limɔai chɔŋgba.} When you go to your kingdom forever. \textbf{2)} throne. \textit{Bahin chala bɛli mɔai.} Our father sits on his throne. \textbf{3)} chieftaincy.

\TCsubword{bɛɛli} (der.) \textit{n} (lɔ/ma) chiefdom, kingdom (\textit{bɛɛli/libɛɛ}) (\citealt{Pichl1967}). 

\TCsubword{libɛɛ} (der.) \textit{n} (lɔ/ma) chiefdom, kingdom (\textit{bɛɛli/libɛɛ}) (\citealt{Pichl1967}).

\TCsubword{lomɔlibɛ} (der.), (comp. of \TClink{libɛɛ}) \textit{n} (kɔ/tha) regal robe (\citealt{Pichl1967}).

\TCheadword[2]{bɛɛ} (Themne \textit{ɔbai} ‘chief') \textit{cf}: \TClink[1]{ba}. \textit{n} \textbf{1)} chief, [bàí bɛ̀ɛ̀lɛ̀] the chief's court (K dialect); \textit{bɛɛ} (wɔ/hã, a) chief (\citealt{Pichl1967}). \textbf{2)} \textit{bɛɛ} (wɔ/hã, a) king (\citealt{Pichl1967}). \textit{Hwɛlɔ lɛ yɛthiɛ bɛ wɔ lɛ.} The world receives her king (\citealt{Pichl1967}). \textbf{3)} god. \textit{Bɛ tilaŋ gbi cheni.} There is no other Lord (\citealt{Pichl1967}). 

\TCsubword{bɛɛbom} (comp.) \textit{n} (wɔ/hã, a) paramount chief (\citealt{Pichl1967}).

\TCsubword{bɛmaa} (comp.) \textit{cf}: \TClink{kwin}. \textit{n} queen (\citealt{Pichl1967}). 

\TCheadword[3]{bɛɛ} \textit{cf}: \TClink{bɛia}. \textit{n} \textbf{1)} [bɛ́ɛ́] jug, large earthenware pot for keeping water cool (K dialect); \textit{bɛ} (hɔ̃/tha) country pot, earthenware (\citealt{Pichl1967}). \textbf{2)} [bɛ̀ɛ̀] water pot, used to reduce liquid for medicine (K dialect). 

\TCheadword{bɛɛbom} (comp. of \TClink[2]{bɛɛ}, \TClink{bom}, see \TClink[2]{bɛɛ})

\TCheadword{bɛɛli} (der. of \TClink[1]{bɛɛ}) 

\TCheadword{bɛia} \textit{cf}: \TClink[3]{bɛɛ}. \textit{n} rice pot. \textit{Yɛmɔ ni hun sɛmi mɔi chi itu bɛia.} After putting it down, you bring the rice pot. \textit{Yɛ mɔni koŋ thɔk itu bɛiaɛ vɛ, mɔi kɔ thu pɛlɛ, mɔi huŋ bɛ lalako.} After you have washed the rice pot, you measure the rice and then put it on the fire.

\TCheadword[1]{Bɛk} \textit{n} Krio person. \textit{Gbe̹mni Abəka lɛ ni nche ma hã lɛ ma fəsɛ hã ma Apotoa.} The inheritance and the way of the life of the Krios resemble those of the Europeans (\citealt{Pichl1967}). \textit{Təm dɛ kɔ ka chɔni Pəm Taks ɛ, pə ka di Abək agbe̹r abul-abul gbo hã ka saa.} During the time of the Hut Tax War, many Krios were killed, only a few escaped (\citealt{Pichl1967}). 

\TCsubword{Bɛknɔ} (der.) \textit{n} (wɔ/hã, a) Krio person (\citealt{Pichl1967}).

\TCheadword[2]{Bɛk} \textit{adj} relating to Krio culture, identity, language, etc. (\citealt{Pichl1967}). 

\TCheadword[1]{bɛk} \textit{n} (hɔ̃/-) colic, \textit{ibək/yom bək} colic/catch colic (\citealt{Pichl1967}, \citealt{Sumner1921}).

\TCsubword{yoŋibɛk} (comp.) \textit{adj} [yóŋìbɛ̀k] overfed (K dialect). 

\TCheadword[2]{bɛk} (Eng \textit{bag}) \textit{cf}: \TClink{gbamfa}, \TClink[1]{kɔ}. \textit{n} \textbf{1)} bag. \textit{Pɔ koŋ gbo chakath yeŋkɛlɛŋ, pɔi chi bɛkthɛ.} They remove the stalks from the rice completely, then they bring the bags. \textbf{2)} (hɔ̃/tha) quiver (\citealt{Pichl1967}). 



\TCheadword[3]{bɛk} [bɛ́k] \textit{cf}: \TClink{thɛŋ}. \textit{n} side (K dialect). 

\TCheadword[4]{bɛk} \textit{n} [bɛ́k] tree species with small thorns found in swamps (K dialect).

\TCheadword{Bɛknɔ} (der. of \TClink[1]{Bɛk}, \TClink{nɔ}, see \TClink[1]{Bɛk}) 

\TCheadword{Bɛkun} \textit{nam} Bekun, female name given by Yase Society (\citealt{Pichl1967}). 

\TCheadword[1]{bɛl} \textit{n} [bə́l]/[bə́lsɛ̀] rat/rats (B dialect). 

\TCheadword[2]{bɛl} \textit{n} \textbf{1)} \textit{bəl} (kɔ/ma) nut, any kind of nut (\citealt{Pichl1967}). \textbf{2)} [bɛ̀l]/[\`{m}bɛ̀l] palm nuts (B dialect); [bɛ̀l lɛ̀]/[\`{m}bɛ̀l lɛ́] the palm nut/the ripe palm nuts (K dialect). \textbf{3)} palm kernels. \textit{Ŋ kɔ tïkïl ibəl lɛ kahãy ko, hɔɛɛ lɛ yema le̹l.} Go gather the palm kernels outside, it will rain (\citealt{Pichl1967}). \textit{Ŋ kɔ tuu ibəl lɛ shop lɛ ahɔl ni nhã ya si bushɛl liwɔ.} Go measure the palm kernels at the shop and let me know how many bushels there are (\citealt{Pichl1967}). comp. \TClink{chɛnthmbɛl} (see \TClink{chɛnth}), \TClink{hoymbɛl} (see \TClink[2]{hoy}), \TClink{nɔkɔmbɛl} (see \TClink{nɔ}), \TClink{taimbɛl} (see \TClink[1]{tai}), unspec.. \TClink{choombɛl} (see \TClink{chocho}) 

\TCsubword{bɛlmagbo} (comp.) \textit{n} [mbɛlmagbo] seeds used in the warri game (K dialect); (kɔ/ma) seeds of a creeping plant found on the beach and used for the warri game (\citealt{Pichl1967}).

\TCsubword{bɛlpotho} (comp.) \textit{cf}: \TClink{konat} \textit{n} [bɛ̀lpòthò] coconut (K dialect); (kɔ/ma) coconut (\citealt{Pichl1967}). \textit{Pɛlɛ kɔ che yeɡbe ka hi fi, ken bɛl pothoɛ ki...} Rice does not grow well in our hands, like coconut... 

\TCsubword{bɛlsekiɛni} (comp.) \textit{n} \textit{ibəlsekiɛni} (kɔ/ma) broken kernels of the oil palm (\citealt{Pichl1967}).

\TCsubword{bɛlthampel} (comp.) \textit{n} \textbf{1)} [bɛ̀lthàmpél] grass species used for medicine (K dialect). \textbf{2)} (kɔ/ma) tree species, small tree (Cnestis ferruginea) (\citealt{Pichl1967}).

\TCsubword[3]{bɛl} (der.) \textit{n} \textbf{1)} \textit{ibəl} (hɔ̃/-) glands, probably from ‘palm nut' because of its shape (\citealt{Pichl1967}). \textbf{2)} \textit{ibəl} (hɔ̃/-) disease of the glands (\citealt{Pichl1967}). comp. \TClink{nakibɛl} (see \TClink[1]{nak}) 

\TCheadword[1]{bɛlɛ} [bɛ́lɛ́] \textit{cf}: \TClink{biim}, \TClink[2]{chol}. \textit{n} [bɛ́lɛ́ɛ́] fish species, name for many different species of large flatfish with short forked tails, 2--2.5 feet long (K dialect). 

\TCheadword[2]{bɛlɛ} \textbf{1)} \textit{subordconn} unless (\citealt{Sumner1921}). \textbf{2)} \textit{subordconn} until. \textit{Ya chen kɔ bɛlə ŋkɔ.} I shall not go until you go (\citealt{Pichl1967}). \textbf{3)} \textit{subordconn} except (\citealt{Pichl1967}). \textbf{4)} \textit{subordconn} before. \textbf{5)} \textit{adv} only.

\TCheadword[1]{bɛlɛn} [bɛ̀lɛ̀n] \textit{Loc} \textbf{1)} on the side, [nàɛ́ bɛ̀lɛ̀n]/[bɛ̀lɛ̀n mìɛ̀] on the side of the road/on my side (K dialect). \textit{Yà chén kɔ̀ bɛ̀lɛ̀n kò.} I will not go on (that) side. \textit{Mpang mən-bul bɛlɛng buli, mən-bul bɛlɛng hãlɛ.} Six months on one side, six months on the other side (\citealt{Pichl1967}). \textbf{2)} \textit{bɛlɛŋ} near (\citealt{Pichl1967}).

\TCheadword[2]{bɛlɛn} [bɛ̀lɛ̀n] \textit{adv} privately (K dialect); \textit{bɛlɛŋ} privately (\citealt{Pichl1967}).

\TCheadword{bɛlɛŋthi} \textit{prep} around.

\TCheadword{bɛlkɛk} \textit{n} \textit{bɛlkək} (kɔ/ma) plant species, prickly shrub (Ximenia americana) (\citealt{Pichl1967}). 

\TCheadword{bɛlma} \textit{n} [bɛ̀lmà] sling to drive birds (K dialect); (kɔ/ma) lasso, sling to catch animals (\citealt{Pichl1967}). 

\TCheadword{bɛlmagbo} (comp. of \TClink[2]{bɛl}, \TClink[3]{gbo}, see \TClink[2]{bɛl}) 

\TCheadword{bɛlpotho} (comp. of \TClink[2]{bɛl}, \TClink{Potho}, see \TClink[2]{bɛl}) 

\TCheadword{bɛlsekiɛni} (comp. of \TClink[2]{bɛl}, \TClink{sekitini} (der. of \TClink{sek}, \TClink{-ni}, \TClink[1]{-i}), see \TClink[2]{bɛl}) 

\TCheadword{bɛlthampel} (comp. of \TClink[2]{bɛl}, \TClink{thampel}, see \TClink[2]{bɛl}) 

\TCheadword{bɛmaa} (comp. of \TClink[2]{bɛɛ}, \TClink{maa}, see \TClink[3]{bɛ}) 

\TCheadword{bɛmba} [bɛ̀mbɛ̀] \textit{n} tree species, berry tree that grows in swampy areas (K dialect); (kɔ/ma) small tree on the shore (Chrysobelanus ellipticus) (\citealt{Pichl1967}).

\TCheadword{bɛmɛk} \textit{cf}: \TClink{biŋk}, \TClink[2]{nyum}. \textit{v} [bə́mə́k] be blind (K dialect). \textit{Làŋgbàɛ́ ché pàɛ̀ kə́ gbér, kɛ́, yɛ̀làìò kóŋ bə́mə́k.} The man once was seeing well, but now he is blind.

\TCheadword{Bɛmpa} \textit{nam} [bɛ̀mpá] Bempa, name given to a place (K dialect). 

\TCheadword{bɛmpa} \textit{cf}: \TClink[2]{bɛth}, \TClink[1]{chɔ}, \TClink{haa}, \TClink[2]{hɛl}, \TClink{vethi}. \textit{v} \textbf{1)} [bɛ́mpá] settle (K dialect). \textit{Hà lá bɔ̀ɔ̀ bɛ́mpá.} They will be able to settle it. \textit{Lɛ nɔ koyɛni gbo ha pɔn bɛmpa la, makɔni kɔtai.} If the person does not settle it, it goes to court. \textbf{2)} prepare. \textit{Nælɛ gbi yi bɛmpa yenjo hĩ lɛ, yi bɛmpa hɔ̃ yenkeleŋ.} In whatever way we prepare our food, let us prepare it nicely and cleanly (\citealt{Pichl1967}). \textbf{3)} arrange. \textit{Yaŋ fli ya woth laɛ ko fe ton-tondo ki ŋa aŋa mpanth lɔnlɔ abɛmpa gbi ja apimamdɛ o ja aŋamdɛ gbi fe tondo ki kunɛ.} It is me that works to arrange all of my children's affairs and my own affairs with very little money coming in. \textbf{4)} make. \textit{Langbandɛ tipɛ bɛmpa aye̹n hã kaŋ hã.} The man began to make them a place to teach them (\citealt{Pichl1967}). \textbf{5)} mend. \textit{Lamdɛ bɛmpa kumbamdɛ.} My wife mends my shirt (\citealt{Pichl1967}). \textbf{6)} help. \textit{Ya bi woth disil yaŋ atok, kɛ ya biɛn nɔ bɛma min.} I have a heavy load on my head, but I do not have anyone to help me.

\TCsubword{nɔbɛma} (comp.) \textit{cf}: \TClink[1]{kump}, \TClink{nɔbonthɔ} (comp. of \TClink{nɔ}). \textit{n} helper. \textit{Nɔ-bɛmam.} My helper (title of a hymn).

\TCsubword{bɛmpabɛmpa} (der.) \textit{v} arrange. \textit{Ŋa bɛmpa-bɛmpa ja Bondoɛ, kendɛ kiɛ ni kacheɛ ni ntoŋgi nyi ŋɔnɛ ŋɔ kathɛ.} To set up a Bondo school, these days and those days and show us the one that is hard.

\TCsubword{bɛmpaka} (der.) \textit{v} prepare. \textit{Fe wullɛ lɔ pɔ bɛmpaka wullɛ.} It is the funeral money people use to prepare the funeral (proverb) (\citealt{TISLL1979}).

\TCsubword{bɛmpani} (der.) \textit{v} \textbf{1)} prepare oneself. \textit{Kaiŋ Taso wɔe bɛmpani ni anya wɔe ŋae kɔni ko wul-lɛ.} Kain Tasso and his people prepared themselves to go to the wake. \textbf{2)} begin. \textit{Pie wɔ bəmpani hã wolɛ gbəŋ.} Pieh will begin planting his farm tomorrow (\citealt{Pichl1967}). 

\TCheadword{bɛmpaka} (der. of \TClink{bɛmpa}, \TClink[1]{ka}, see \TClink{bɛmpa}) 

\TCheadword{bɛmpani} (der. of \TClink{bɛmpa}, \TClink{-ni}, see \TClink{bɛmpa}) 

\TCheadword[1]{bɛn} \textit{cf}: \TClink{baom}, \TClink[2]{veleŋ} (der. of \TClink[1]{veleŋ}). \textit{n} \textbf{1)} ancestor. \textit{Yi kɔ hərni abɛna hĩ lɛ.} We go to worship our ancestors (\citealt{Pichl1967}). \textbf{2)} parent. \textit{Lanɛ la li kɛlɛŋ, lɛ bɛn mɔi wɔ mɔ gbo ntɛnt, mɔ ha suthra wɔ, mɔ ha toŋgiɛ lɛ wɔ gbem mɔ.} That is what is good, if your parent is near you, you should try to show that she gave birth to you. \textit{La ka che kath ŋa abɛnai.} It was difficult for our parents. \textbf{3)} age. \textit{Saɛ kɔ chen ha libɛn.} A heavy beard is not a result of age (proverb) (\citealt{TISLL1979}). comp. \TClink{kolabɛna} (see \TClink{kol}) 

\TCsubword{abɛna} (der.) \textit{cf}: \TClink{ram}. \textit{n} (-/ha) generation, forefathers, ancestors (\citealt{Pichl1967}). 

\TCheadword[2]{bɛn} \textit{n} \textbf{1)} old times. \textit{Laŋ la nante lɛ ka cheni mbɛn ɛ.} What is now happening did not happen in old times (\citealt{Pichl1967}). \textbf{2)} old ways. \textit{Ja mbɛnɛlɛ la ka koŋ bɛviɛ.} The old ways have been forgotten. \textbf{3)} distant past. \textit{Aa, li bɛn. Nshi ni nɛnthi lan?} Yes, (it was) long ago. Do you know what years?

\TCheadword[3]{bɛn} \textit{adj} old. \textit{A-a, ha wɔ ja bɛn la koŋ.} No, they say old ways have ended. \textit{Nduɛ muɛkɛ mɛŋtiŋdɛ, ni nɔmaa bɛn dɛ, wɔe wu jajɛl Kaiŋ Tasoɛ.}
On the seventh day, the old woman died, Kain Tasso’s mother-in-law. \textit{Ni wɔ ye bɔm nɔma bɛn.} And then he met an old woman (\citealt{Pichl1967}). \textit{Nɔ-pokan bən do bi nak-nchɛs.} This old man has leprosy (\citealt{Pichl1967}). \textit{Tamɔ lɛ fɛɛkiɛ mi sin dɛ ya chaŋ bawɔ bɛn.} The boy disregards me, he doesn't realize that I am older than his father (\citealt{Pichl1967}). comp. \TClink{nɔbɛn} (see \TClink{nɔ}) 

\TCsubword{bɛnbɛn} (der.) \textit{adj} very old. \textit{Ya dikil panthe, panthe bɛnbɛndɛ.} I gather the pans, the very old pans.

\TCheadword{bɛnaihyɛl} (unspec. of \TClink[2]{hɛlɛ}) 

\TCheadword{bɛnbɛn} (der. of \TClink[3]{bɛn}) 

\TCheadword{bɛn-bole} (comp. of \TClink[1]{bol})

\TCheadword{Bɛndu} \textit{nam} Bendu, name given to a person, surname. \textit{Yaŋ yaa Abdulai Bɛndu.} I am Abdulai Bendu.

\TCheadword{bɛnthɛ} \textit{cf}: \TClink{tɛŋkɛ}. \textit{n} [bɛ̀nthɛ́]/[bɛ̀nthɛ̀thɛ́] platform/platforms (B dialect).

\TCheadword{Bɛntisaya} \textit{nam} Bentisaya, name given to a place in Dema Chiefdom, Bonthe District. \textit{Kɛ kɔ tipɛ Ndema ka lɔn, Ndema Chifdom, lɔ pɔ wɔ pɔk Ndemaɛ lɔ tipɛ Bɛntsaiya.} But it starts in Ndema here, Ndema Chiefdom, where they say Ndema country starts at Bentsaiya. 

\TCheadword[1]{bɛŋ} cf: \TClink{gbɛŋ} \textit{v} \textbf{1)} \textit{bɛng, gbɛng} touch (\citealt{Pichl1967}). \textit{Ya ke wɔ ma hɔl thimdɛ, ni ya bɛŋ ma wɔ pia mi njokɛ, ni ya theli ko wɔ ko.} I saw him with my eyes, and I touched him with my right hand, and I talked to him. \textit{Yu lɛ kong puthul, lɛ ŋgbəŋ wɔ gbo hinɛ gbo nɔth.} The fish is rotten already, if you touch it, you will find it quite soft (\citealt{Pichl1967}). \textbf{2)} experience, feel. \textit{Ya bɛŋ isin.} I am suffering (lit. I feel suffering) (\citealt{Pichl1967}). \textit{Bahin i ko gbɔ bɛŋ sin o.} Our Father, we have struggled so. \textbf{3)} hit. \textit{So wɔnɛ wɔ vɛ thɔmwɔɛ, wɔnɛ pɔ bɛŋ wɔ bo, wɔi ko sɛm.} So anybody that threw the ball at the other one, the one the ball would hit would stand out.

\TCheadword[2]{bɛŋ} \textit{cf}: \TClink[1]{nɛ}. \textit{n} \textbf{1)} [bɛ̀ŋ] leg or foot (K dialect); [bàŋ]/[bàŋ thɛ́] foot, the feet (B dialect). \textit{A kɔ viiki bɛŋthi-m dɛ.} I go to stretch my legs, i.e. I go for a walk (\citealt{Pichl1967}). \textit{Nak-naka bí bɛ̀ŋ nàká.} Her leg hurts. \textit{Koŋ koŋ chu bəŋ wɔ lɛ ka vɛ.} Kong's foot was pricked by a thorn (\citealt{Pichl1967}). \textbf{2)} [bɛ́ŋ] sole of the foot (K dialect). comp. \TClink{bɛŋkajɛm} (see \TClink[3]{ka})

\TCsubword{bɛŋhil} (comp.) \textit{n} [bə̀ŋhíl] foot swelling, elephantiasis (K dialect); (hɔ̃/tha) elephantiasis in the leg (\citealt{Pichl1967}). 

\TCsubword{bɛŋkɔk} (comp.) \textit{cf}: \TClink{binthaŋ}, \TClink{gbɔŋkɔt}. \textit{n} (hɔ̃/tha) ankle (\citealt{Pichl1967}). 

\TCsubword{bɛŋpiamin} (comp.) \textit{n} (hɔ̃/tha) left leg or foot (\citealt{Pichl1967}). 

\TCsubword{bɛŋpianjok} (comp.) \textit{n} (hɔ̃/tha) right leg or foot (\citealt{Pichl1967}). 

\TCsubword{bɛŋdɔ} (der.) \textit{n} bedside (lit. at foot) (K dialect); \textit{thibəŋdɔ} (-/tha) bedside (\citealt{Pichl1967}). 

\TCheadword[3]{bɛŋ} [bɛ̀ŋ] (Port \textit{banco} ‘bank, bench, seat, stool') \textit{cf}: \TClink[3]{chal}, \TClink{chɛm}, \TClink{gbakra}. \textit{n} chair, seat (K dialect). \textit{Nlɛli bɛŋ dɛ, nchal!} Look at the chair, sit down! (\citealt{Pichl1967}). 

\TCheadword{bɛŋa} (der. of \TClink[2]{bɛ}) 

\TCheadword{bɛŋdɔ} (der. of \TClink[2]{bɛŋ}, \TClink[7]{lɔ}, see \TClink[2]{bɛŋ}) 

\TCheadword{bɛŋgɛt} \textit{v} [bə́ŋgə́t] cover (B dialect). \textit{Mbə́ŋgə́t ìʦùɛ́!} Cover the pot! 

\TCheadword{bɛŋhil} (unspec. of \TClink[2]{bɛŋ})

\TCheadword[1]{bɛŋk} [bɛ́ŋk] \textit{cf}: \TClink[4]{bol}, \TClink[1]{sil}. \textit{n} kind of maggot found in palm wine, smaller than \textit{bol} (K dialect). 

\TCsubword[2]{bɛŋk} (id.) [bɛ́ŋk] \textit{cf}: \TClink[2]{koŋkbo} (id. of \TClink[1]{koŋkbo}), \TClink{nɔyilɔ} (comp. of \TClink{nɔ}, \TClink[1]{yil}), \TClink[2]{thɔŋpaŋ} (id. of \TClink[1]{thɔŋpaŋ}). \textit{n} drunkard, term of abuse based on the fact that these maggots are never far from alcohol, i.e., palm wine (K dialect). 

\TCheadword[3]{bɛŋk} [bɛ́ŋk] \textit{n} [bɛ́ŋk], [(ì)bɛ̀ŋkɛ́] rice after cleaning, not the chaff or husk but heavier rice left in the fanner (K dialect). 

\TCheadword{bɛŋkajɛm} (comp. of \TClink[2]{bɛŋ}, \TClink[3]{ka}, \TClink{jɛm}, see \TClink[3]{ka}) 

\TCheadword{bɛŋkɔk} (comp. of \TClink[2]{bɛŋ}, \TClink{kɔk}, see \TClink[3]{bɛŋ}) 

\TCheadword{bɛŋpiamin} (comp. of \TClink[2]{bɛŋ}, \TClink{piamin} (comp. of \TClink[1]{pia}, \TClink[3]{min}), see \TClink[2]{bɛŋ}) 

\TCheadword{bɛŋpianjok} (comp. of \TClink[2]{bɛŋ}, \TClink{pianjok} (comp. of \TClink[1]{pia}, \TClink[1]{jo}), see \TClink[2]{bɛŋ}) 

\TCheadword[1]{bɛŋthisɔk} [bɛ̀ŋthìsɔ̀k] \textit{n} grass species, used to make sauce (K dialect). 

\TCheadword[2]{bɛŋthisɔk} \textit{n} (kɔ/ma) herb species (Amaranthus spinosus) (\citealt{Pichl1967}). 

\TCheadword{bɛraa} \textit{n} \textbf{1)} [bɛ̀ráá] gentlemen (K dialect). \textit{Bɛraa, hi thola ka thigbikan ni hi kɔa gbunda feɛ hiŋk mɛsaɛ atok.} Gentlemen, let us run down and grab the money on top of the table. \textit{Anyaɛ bai ko bul wɔe gbaki ni hɔɛ, “Bɛra, ŋa pokɔ mi lɔ ka.”} One of the people in the bari replied, “Gentlemen, get outta here!" \textbf{2)} people (polite). \textit{Bɛ̀ráá, nyá chàl mù.} People, people be seated. \textit{Labi bɛ bɛra ŋa che kɛkɛ-o hɔɛ, ‘pɔ gbiŋkith feɛ-o-o-o!'} That’s why people were saying just now, “let's cover the money-o!"

\TCheadword{bɛrɛ} (unspec. of \TClink[2]{bɛ})

\TCheadword{bɛrɛlɔni} (der. of \TClink{bɛrɛ} (unspec. of \TClink[2]{bɛ}), \TClink[2]{lɔ}, \TClink{-ni}, see \TClink[2]{bɛ}) 

\TCheadword{bɛrmani-bɛrmani} (unspec. of \TClink[2]{bɛ}) 

\TCheadword{bɛs} \textit{cf}: \TClink[1]{bas}. \textit{n} [bɛ̀s] broom (K dialect); (kɔ/ma) broom (\citealt{Pichl1967}). 

\TCsubword{bɛslisoko} (comp.) \textit{n} (kɔ/ma) ceremonial broom of the Sokos and other adepts (\citealt{Pichl1967}). 

\TCheadword{bɛsɛŋ} [bɛ̀sɛ̀ŋ] \textit{n} balance (K dialect); (hɔ̃/tha) balance of a boat or canoe riding the waves (\citealt{Pichl1967}). \textit{Wɔ̀mdɛ́ hɔ́ bì bɛ̀sɛ̀ŋ.} The boat is balanced.

\TCheadword{bɛslisoko} (comp. of \TClink{bɛs}, \TClink{soko}, \TClink[1]{li-}, see \TClink{bɛs}) 

\TCheadword[1]{bɛt} (Eng \textit{bait}) \textit{n} \textbf{1)} bait, fish you put on a hook (K dialect). \textit{A bi huk bul ŋɔ adukiɛ yuɛ betɛ gbo koi gbo hukɛ, a wɔi nan.} I have a hook that I use, if the fish comes for the bait on the hook, I then pull it up. \textbf{2)} \textit{beth} (kɔ/ma) fishing net used for catching small fish (\citealt{Pichl1967}). 

\TCheadword[2]{bɛt} \textit{v} [bɛ́t] tap a tree to collect sap (K dialect). \textit{Nɔ̀sààɛ́ wɔ̀ bɛ́t bàchɛ̀ kà íbáá.} The tapster tapped the tree with a knife.

\TCheadword[1]{bɛth} \textit{cf}: \TClink[3]{ke}. \textit{n} \textbf{1)} (hɔ̃/tha) loins, lower part of the loins (\citealt{Pichl1967}). \textbf{2)} hip (K dialect). \textit{Ìbɛ̀th mí ŋɔ̀ nɛ̀kí.} My hip is hurting me.

\TCsubword{bɛthɛhɔl} (comp.) [bɛ́thɛ́hɔ́l] \textit{n} lower stomach, also used for loins (K dialect).

\TCheadword[2]{bɛth} \textit{cf}: \TClink{bɛmpa}, \TClink[2]{kɛn}, \TClink{kɛth}, \TClink{rɔk}, \TClink{thak}. \textit{v} \textbf{1)} cut. \textit{Bia bɛth rəm wɔ lɛ thɛmni yenwɛy næ lɛ bol.} Bia has cut his toe, he stubbed it badly on the way (\citealt{Pichl1967}). \textit{Ni wɔ kɔ thoai ni bɛthi mbaŋ ndumɔndumɔ...} And so he went to the bush and cut very strong ropes... (\citealt{Sumner1921}). \textbf{2)} cut off. \textit{Ye hã bɛthi bo̹l wɔ lɛ hɔ̃ lee thɔt lɛ.} When they cut off the top of the tree, there is the trunk which remains (\citealt{Pichl1967}. \textit{Ye hã bɛthi bo̹l wɔ lɛ.} Then when they cut off his head (\citealt{Pichl1967}). \textbf{3)} cut down, fell. \textit{A bɛth thɔk lɛ ka bərɛ.} I cut down the tree with an axe (\citealt{Pichl1967}). \textbf{4)} butcher, cut up meat. \textit{À bɛ̀thí vìsɛ̀} I'm cutting up the meat. \textit{À kóŋ bɛ̀thí vìsɛ̀.} I cut up the meat. \textbf{5)} reduce, cut back. \textit{Kɔ koŋ gbo lɔ, mɔ lɔi bɛthi jɛmlɛ.} After it finishes, you have to reduce the fire. \textbf{6)} “cut a deal,” e.g., settle a court dispute. \textit{bɛth mbolom/bɛth ŋhɔ'/ bɛth thonka} to cut off or settle a court dispute (\citealt{Pichl1967}).

\TCsubword[2]{bɛthni} (der.) \textit{v} remove, cut off. \textit{Ni gbɔs yabasɛ hɔ bɛthni pɔmthi gbamdɛ gbɔs lan.} To remove the smell of the onion from the potato leaves.

\TCheadword{bɛthɛhɔl} (comp. of \TClink[1]{bɛth}, \TClink[1]{ahɔl}, see \TClink[1]{bɛth}) 

\TCheadword{bɛthɛkɛni} \textit{v} [bɛ̀thɛ̀kɛ̀ní] feel urgency (K dialect). \textit{Ya bɛthkɛni, ya le kɔ nai ɛ bɔl.} I am pressed, I go first to the privy (\citealt{Pichl1967}). 

\TCheadword{bɛthɛkin} [bɛ̀thɛ̀kín] \textit{n} secret (K dialect); secret affair (\citealt{Pichl1967}). \textit{À bí bɛ̀thɛ̀kín.} I have a secret. \textit{Ya bi bɛthɛkin, ya mɔ la hɔm gbəŋ.} I have a secret, I will tell it to you tomorrow (\citealt{Pichl1967}). 

\TCheadword[1]{bɛthni} \textit{v} [bɛ̀thní] be hoarse (K dialect, \citealt{Pichl1967}). \textit{Lòm mìɛ́ ŋɔ̀ bɛ̀thní.} My voice is hoarse.

\TCheadword[2]{bɛthni} (der. of \TClink[2]{bɛth}, \TClink{-ni}, see \TClink[2]{bɛth}) 

\TCheadword{bɛthpɔɔ} \textit{n} summons, preliminary official summons to any dance by a Laka, Taso, or other Poro official (\citealt{Pichl1967}). 

\TCheadword{bɛvi} \textit{v} forget. \textit{Ja mbɛnɛlɛ la ka koŋ bɛviɛ.} The old ways have been forgotten. 

\TCheadword[1]{bi} \textit{cf}: \TClink[2]{kɛna}. \textit{v} \textbf{1)} have. \textit{Ya bi bɛthɛkin, ya mɔ la hɔm gbəŋ.} I have a secret, I will tell it to you tomorrow (\citealt{Pichl1967}). \textit{Abi apuma atiŋ.} I have two children. \textit{Bi kil kɛləŋ.} He has a nice house (\citealt{Pichl1967}). \textbf{2)} own. \textit{Ŋalɛ wɔ ŋaa ina bi ka a?} And they said who owns (the land) here? \textbf{3)} come to have, get. \textit{Ye lɛ kulɔ gbo ni mən bɔsul, mɔ bi ipula mɔm kunɛ.} Then if you drink unboiled water, you will get worms. \textit{Pɛlɛ kɔi pith kɔi piŋgi, kɔi bi kun, kɔi gbemɔ.} The rice will get dark, and then it will change and swell up (lit. have a belly, i.e. be pregnant) and then tiller. \textbf{4)} cause to do something. \textit{La bi a bɔɔni mɔm tɛntɛ.} That is what makes me draw closer to you. comp. \TClink{yenbiɛihɔlɔŋ} (see \TClink[1]{yen}), \TClink{yɛbi} (see \TClink[3]{yɛ}) 

\TCsubword[1]{biyɛni} (unspec.) \textit{v} have not. \textit{Kache ŋɔn hi, mbi fe, mbiyɛni fe ha nyamɔ ŋa mɔ bɔnth.} In the past, whether you had money or not, your people would help you. der. \TClink[2]{biyɛni} (see \TClink[1]{bi})

\TCsubword[2]{biyɛni} (unspec.), (der. of \TClink[1]{biyɛni}) \textit{adj} destitute. \textit{Boŋgo che ki, nɔ mbiyɛni gbo fe nche lɔik Bondo.} These days, if one has no money, one will not enter Bondo.

\TCheadword[2]{bi} \textit{cf}: \TClink[3]{che}, \TClink[2]{ha}, \TClink[2]{ki}, \TClink[7]{kɔ}, \TClink[3]{lɔi}, \TClink[1]{ma}, \TClink{mɔs}, \TClink[2]{ŋa}, \TClink[3]{yema}. \textit{Aux} \textbf{1)} must, modal auxiliary. \textit{Ya bi ŋa wɛ a chɔŋɔ mɔ sɛkɛ, Bahin.} I have to say thank you, Lord. \textit{Labo thibɔm lɔ pɔ bia yukɛ, pɔ kɔ ni bɔm thai pɔi kɔ piŋgi bɔmdɛ ɔ pɔi gbusa.} If people have to plant where it is muddy, they will then turn the mud over or then they dig. \textbf{2)} would. \textit{Ŋ hɔmɔ-m vɛɛthiɛlɛ lɛ mbi hã hun kə ya ke mɔni.} You told me the other time that you would come, but I did not see you (\citealt{Pichl1967}). \textit{So labi ale yimani laŋgbadɛ ki labo wɔla bia chɔŋ la len.} So that is why I am first asking the consent of this man, if he would like it. \textbf{3)} should. \textit{Pum ya biɛ hã kɔ.} Perhaps I should go (\citealt{Pichl1967}). \textbf{4)} will, shall. \textit{ Ŋgbəŋ hun mi che, tɛmpum ya bi hã kɔma mɔ.} Come to me tomorrow, maybe I shall go along with you (\citealt{Pichl1967}). \textit{A che bi ŋa lɔɛ arijana.} I will never, never enter the kingdom of God. \textit{So anyaiɛ, apima iyɛ, nrokɛ, nrekiaɛ ŋa bia hundɛ.} So our people, our children, the grandchildren, the great-grandchildren that are going to come. comp. \TClink{yɛbini} (see \TClink[3]{yɛ}), der. \TClink{labi} (see \TClink[2]{la}) 

\TCheadword[3]{bi} \textit{n} Poro Society drum, [bíɛ̀] the Poro drum (K dialect); \textit{ibi} (hɔ̃/-) kind of drum, the same as \textit{ibimbi} (\citealt{Pichl1967}). 

\TCsubword[2]{bimbi} (comp.) \textit{n} Poro Society drum (K dialect); \textit{ibimbi} (hɔ̃/-) kind of drum, the same as \textit{ibi} (\citealt{Pichl1967}). 

\TCheadword{Bia} \textit{nam} Bia, name given by Poro Society. \textit{So ilel Bia Hɛlɛ ŋɔ mbɔnth kɔ wɔ?} So Bia Helleh is the one you met with him? 

\TCheadword{bia} (Eng \textit{bear}) \textit{v} bear a burden, withstand a hardship. \textit{A sɔthɔ gbo aya wɔiowɔi, a sɔthɔni gbo, ai bya ŋa wɔi ŋallɛ.} If I have (something) every day I cook, but if I do not, I'm patient for the next day ( I will bear it until the next day). 

\TCheadword{biaa} (Port \textit{via} ‘channel') \textit{n} (hɔ̃/tha) channel, deep water (\citealt{Pichl1967}). 

\TCheadword{Biahɛni} \textit{nam} male name given by a society and first name of the first Dema paramount chief. \textit{Wɔnɛ fɔs wɔ piŋgɛ yɛthi chukalɛ? Ba Biahɛni Ŋɡamaŋɡa.} Who was the first person that held the staff? Ba Biaheni Ngamanga.

\TCheadword[1]{bian} \textit{n} very deep space, hollowed out or cleaned bare, can be used for the result of erosion (K dialect). 

\TCsubword[2]{bian} (der.) \textit{cf}: \TClink[1]{thuŋk}. \textit{adj} [bìàn] deep (K dialect). \textit{Bìàn wɔ̀ lɔ̀.} There is a deep (spot) there.

\TCheadword{bias} (Port \textit{viagem} ‘voyage') \textit{cf}: \TClink{gbɛyɛ}. \textit{n} (hɔ̃/tha) journey, trip (\citealt{Pichl1967}). \textit{Ba mi koŋ kɔn bias ay nante.} My master went on a journey today (\citealt{Pichl1967}). \textit{Yi kɔ gbahã ba hĩ ka kɔn bias gbath vil ni koŋ moey.} We go to welcome our father who went on a journey long ago and has returned now (\citealt{Pichl1967}). \textit{Yà bí bìás thìrà.} I took three trips.

\TCheadword{bifo} (Eng \textit{before}) \textit{subordconn} before. \textit{Thetha mi ka che ŋa mpanth ma landɛ pɛŋ bifo wɔ mmu hu.} My grandmother used to do the work before she died. \textit{Aa, kɛ bifo dat akoni che ko administreshɔn dɛ fɔ lɔŋg.} Yes, but before that I had been in administration for a while.

\TCheadword{biim} [bììm] \textit{cf}: \TClink[1]{bɛlɛ}, \TClink[2]{chol}. \textit{n} fish species, type of flatfish, very tasty (K dialect).

\TCheadword{biisi} \textit{v} \textbf{1)} make tight. \textit{Yá bíísí bàŋkɛ̀.} I will tighten the rope. \textit{Ayeŋ wɔ lɛ che bisiɛ pe̹ŋ.} His waist is tightly strung (\citealt{Pichl1967}). \textbf{2)} hold on. \textit{Mbìsì tíŋ!} Hold on tight!

\TCheadword{bik} \textit{cf}: \TClink{bente}, \TClink[4]{bobo}. \textit{n} \textbf{1)} (hɔ̃/tha) type of mat (\citealt{Pichl1967}, \citealt{Sumner1921}). \textbf{2)} burial mat (K dialect). \textit{Pɔ bia kɔ kɔŋ nɔ ni bikɛ.} They would bury the corpse with a mat.

\TCheadword{bikɛ} \textit{n} \textbf{1)} [bìkɛ́] heavy wind with rain, often comes at night (B dialect); [bìkɛ̀] storm (K dialect); \textit{bikɛɛ} (hɔ̃/tha) storm, tornado (\citealt{Pichl1967}). \textit{Bìkɛ̀ sìmìɛ́ kə̀llɛ̀.} The storm destroyed the house. \textit{Hã kɔ chæ thɔk lɛ kɔ bikɛɛ lɛ duki chɔl na næ lɛ 'hɔl lɛ.} Go and lift the tree that the storm felled on the road last night (\citealt{Pichl1967}). \textbf{2)} wind. 

\TCheadword{bikɔs} (Eng \textit{because}) \textit{cf}: \TClink{haliwɔ}, \TClink{hayɛ}, \TClink{thaŋkɔ}. \textit{subordconn} because. \textit{Acheŋɔni pɛ lonibolɛ, bikɔs pɔ chiɛmi ka yaŋ taa.} I would not remember it because I was brought here when I was very young. \textit{Bikɔ pomdɛ wɔ mi ni yɛthi sɔŋgɔ ma ŋɔ nɔpikan wɔ ŋa yɛthi nɔma wɔi.} Because my husband is really treating me as a husband should treat his wife. \textit{Bikɔs ya mɔ lapa gbo, mɔ mɔ lapɛ.} Because if your mother gets ashamed, you have shamed yourself.

\TCheadword[1]{bil} \textit{n} (ma) rice variety, sweet with small grains (K dialect, \citealt{Pichl1967}). \textit{Yà kùthá bìllɛ́.} I planted bil (a rice variety). comp. \TClink{miliŋdibil} (see \TClink{miliŋ}) 

\TCheadword[2]{bil} \textit{n} yaws, disease involving large boils (K dialect). \textit{Bìl làŋgbàɛ́ thé nɛ̀kí kà bìllɛ́.} Yaws (is) a disease like a large boil. \textit{Komɔ lo bi mbil?} Does this child have yaws? (\citealt{Pichl1967}). \textit{Làŋgbàɛ́ thé nɛ̀kí kà bìllɛ́.} The man was in pain due to yaws.

\TCheadword[3]{bil} \textit{n} \textbf{1)} marriage. \textit{Mi gbisiŋ doki, bil loki lɔ mɔɔ kunɛ yini gbɔl ŋɔlɔ ŋa mɔm?} This engagement, this marriage that you are in, do you have peace of mind? \textit{Bìllɛ̀ kùèɛ́ gbó nɛ̀n thìrà.} The marriage lasted only three years. \textbf{2)} marital home. \textit{So, ŋɔ ke bila, pɛth-pɛth ŋɔ lɔ?} So, how do you see this marital home, is it sweet?

\TCheadword{bila} \textit{cf}: \TClink[1]{ja}, \TClink{risen}, \TClink[2]{yen}. \textit{n} reason. \textit{Yɛ bilaɛ Prɔf wɔn pɛ yema kɔ toŋgi lawɔɛ yɛ wɔ bia muniniɛ.} The reason is after he returns, Prof also wants to show his wife himself. \textit{Yɛ bila ŋan bɛ ŋa theli Nthemdɛ konɔko, nye?} The reason is they speak Themne everywhere, right? comp. \TClink{yɛbilaɛ} (see \TClink[3]{yɛ}) 

\TCheadword[1]{bim} \textit{v} cover (\citealt{Pichl1967}, \citealt{Sumner1921}). \textit{Yɛ̀ pɔ̀ kóŋ gbó bálón bellɛ, pɔ̀ bɛ́ wùsɛ̀, pɔ̀ ŋɔ̀ bím.} When they have finished tying the rafters of the farmhouse, they put on the thatch, they cover it.

\TCsubword{bimik} (der.) \textit{v} \textbf{1)} cover. \textit{Pùlùkɛ́ bə̀mə̀kɛ́ lɛ́llɛ̀.} Grass covered the ground. \textbf{2)} close (\citealt{Pichl1967}, \citealt{Sumner1921}).

\TCsubword{gbiŋkith} [gbìŋkìth] (unspec.) \textit{v} cover (K dialect). \textit{Yɛ mɔ gbiŋgithɛ, la cheŋ vei mɔi yi, mɔ kɔi puli.} After covering it, it does not take long, then you open it, then you mix it. \textit{Nke gbo tamɔ soth chaŋ, bi sum ha gbinkith ka ko.} If you see a child sprouting teeth, (be sure he) has the mouth to cover it (proverb) (\citealt{TISLL1979}).

\TCsubword{gbintik} (unspec.), (der. of \TClink{gbiŋkith}) \textit{n} (hɔ̃/tha) cover, lid (\citealt{Pichl1967}); \textit{kpinkith} cover (both \textit{n} and \textit{v}) (\citealt{Sumner1921}). 

\TCsubword{gbiŋkithni} (unspec.), (der. of \TClink{gbiŋkith}) \textit{v} cover oneself, e.g., with blanket (\citealt{Pichl1967}). \textit{Ihee hɔ̃ peyɛni mi, ya bɔnthɔ ni hin koŋ gbïnkithni wɔn thibəŋ ni wɔn bol.} Mother has a cold, I found her lying in bed, and she had covered (herself) her feet and head (\citealt{Pichl1967}). 

\TCheadword[2]{bim} [bìm] \textit{n} porpoise species, long black, dangerous when frightened, will try to capsize a boat (K dialect); (wɔ/hã) porpoise (\citealt{Pichl1967}). \textit{Bìmndɛ́ wɔ̀ chɔ́ má wɔ̀mdɛ̀.} The porpoise fought the boat.

\TCheadword[1]{bimbi} \textit{cf}: \TClink[1]{hani}. \textit{n} [bímbí] crowd (K dialect). \textit{Bímbí bòm kɔ̀ ché ná bóndɔ̀ kò.} There was a big crowd at the wharf. \textit{Bimbi lɛ paak lay hã wuli Ba Kennedy lɛ kɔ che parɛ cho gboŋ.} There was a crowd in the park because of Mr. Kennedy's death, they were plenty the other day (\citealt{Pichl1967}). 

\TCheadword[2]{bimbi} (comp. of \TClink[3]{bi})

\TCheadword{bimik} (der. of \TClink[1]{bim}, \TClink{-k}, see \TClink[1]{bim}) 

\TCheadword{bimni} \textit{cf}: \TClink{chok}, \TClink{pikith}, \TClink{thim}, \TClink{tunt}. \textit{v} [bìmnì] bow, stoop, bend over, can be permanent or temporary, used to approach the paramount chief, to show respect, or to pick something up (K dialect); stoop down (\citealt{Pichl1967}). \textit{Nɔ̀mà bɛ̀ndɛ̀ kóŋ bìmnì.} The old lady is stooped. \textit{Kɔ bimni sɔku bullai, wɔ hɔɔl <fɔɔ fɔɔ fɔɔ> ni yeke wɔɛ che wɔn kunɔlɔ.} (She) went and bent over in one corner, she breathed <fɔɔ fɔɔ fɔɔ> (idph of panting) with the cassava (tucked) in her bosom.

\TCheadword{bin} \textit{cf}: \TClink{tharmra}. \textit{v} \textbf{1)} [bín] miss (K dialect). \textit{Ŋà bín wɔ̀mdɛ̀.} They missed the boat. \textbf{2)} make a mistake (\citealt{Pichl1967}).

\TCheadword{binbis} \textit{n} \textbf{1)} [bìmbìs] welt or sore caused by whipping (K dialect). \textit{Kòmɔ̀ɛ̀ bí mbìmbìs wɔ̀n kɔ́k, wɔ̀n njàlàì gbí.} The child has sores on its back, all over its body. \textbf{2)} bump. \textit{Thambase buŋ kɔ lɛ mbinbis.} The evidence of being flogged is a bump (\citealt{Pichl1967}). 

\TCheadword{binch} (Krio \textit{binch} ‘beans') \textit{cf}: \TClink{thes}. \textit{n} bean. \textit{Atipɛ yuk yekeɛ, ŋkaŋdɛ, mbinchɛ, pɛlɛ, nsowɛ, ntɔllɛ.} I start to plant cassava, corn, beans, rice, millet, Guinea corn.

\TCheadword{bind} \textit{n} (hɔ̃/tha) canoe or boat bench with a hole, through which the mast is fixed (\citealt{Pichl1967}).

\TCheadword{binthaŋ} \textit{cf}: \TClink{bɛŋkɔk} (unspec. of \TClink[2]{bɛŋ}). \textit{n} \textbf{1)} [bìnthàŋ] ankle (K dialect). \textit{Bìnthàŋ mìɛ̀ kɔ́ nɛ̀kì.} My ankle hurts. \textbf{2)} (hɔ, kɔ/tha) heel of the foot (\citealt{Pichl1967}). 

\TCheadword{binthi} [bìnthì] \textit{n} coop for domesticated animals, livestock, or fish (K dialect); (hɔ̃/tha) coop for smaller animals or for fishing (\citealt{Pichl1967}). \textit{Bìnthì sɔ́k/kúlúŋ/yènchɛ́k kɔ̀ kí.} This is a chicken/goat/fish coop. \textit{Lɛ ŋ kɔ gbo binthi sɔksi l'ay, n tuntni mma ki təm bo̹l mɔ.} If you go into the fowl coop, bend your head or you will bump your head (\citealt{Pichl1967}). 

\TCheadword{binthima} (unspec. of \TClink[4]{ma})

\TCheadword{binthimani} (unspec. of \TClink[4]{ma})

\TCheadword{binthmabinthma} (der. of \TClink{binthima} (unspec. of \TClink[4]{ma}), see \TClink[4]{ma}) 

\TCheadword{biŋ} \textit{cf}: \TClink{hantha}, \TClink[1]{tɔŋ}, \TClink{waya}. \textit{n} (hɔ̃/tha) enclosure for catching fish (\citealt{Pichl1967}). \textit{Hìná wɔ̀ bɛ̀mpà bìŋ dó á?} Who built this fishing fence?

\TCheadword{biŋk} [bíŋk] \textit{cf}: \TClink{bɛmɛk}, \TClink[2]{nyum}. \textit{n} blindness (K dialect). 

\TCheadword{biŋkinchin} \textit{cf}: \TClink[1]{koŋkbo} (comp. of \TClink[4]{bol}), \TClink[1]{thɔŋpaŋ}. \textit{n} [bìŋkìnchín] beetle species, very large and black (K dialect).

\TCheadword[1]{bip} \textit{n} (hɔ̃/tha) fart (\citealt{Pichl1967}). \textit{Wɔ ye hun hɔɛ, ntheɛ bip?} Then he asked, “Did you hear the fart?” (\citealt{Pichl1967})

\TCsubword[2]{bip} (der.) \textit{cf}: \TClink{sii}. \textit{v} fart (\citealt{Pichl1967}).

\TCheadword[3]{bip} [bíp] \textit{Idph} of falling as mangoes when a branch is cut (K dialect).

\TCheadword{bipr} \textit{v} be present. \textit{Yà bìpɝrɛ́ lɔ̀, chè lɔ́ní.} I was present there, not there. \textit{Mrs. Kennedy ka bipr ko lɔ pə wɔ apook lɛ.} Mrs. Kennedy was on the spot when her husband was shot (\citealt{Pichl1967}). 

\TCheadword{bisaid} (Eng \textit{besides}) \textit{prep} besides. \textit{Shenge ka pɔ ŋa pɛ theli nwɔk mpim bisaid Mbolom?} Here in Shenge do they speak other languages besides Sherbro?

\TCheadword{bise} \textit{n} \textbf{1)} (kɔ/-) seeds of a fruit from which a sauce is made (\citealt{Pichl1967}). \textbf{2)} sauce made from the fruit of the same name (\citealt{Pichl1967}).

\TCheadword{bisin} (Krio \textit{bisin} ‘care for' ?) \textit{v} care for. \textit{Ayɛn biɛ-m bisin.} Truly, he cares (has care) for me. \textit{Yèmà wɔ̀ bísín hà kòmɔ́ɛ́.} Yema took care of the child.

\TCheadword{bisɔŋ} \textit{n} (kɔ/ma) small shots (\citealt{Pichl1967}).

\TCheadword{bith} \textit{cf}: \TClink[4]{ke}. \textit{n} [bìth] stick, pointed remnant of vegetation that remains after a field has been brushed and burnt (B dialect). \textit{Bìthɛ̀ ŋɔ̀ chú wɔ̀.} The stick stabbed him. \textit{I kon gbo iban mbithiɛ manɛ malɔ, man gbi.} Once we have finished, we gather those sharp sticks that are there and burn them.

\TCheadword{bithagbɔ} \textit{n} \textbf{1)} (kɔ/ma) herb species with a few pink flowers in axils (Justicia insularis) (\citealt{Pichl1967}). \textbf{2)} (kɔ/ma) sauce made from the herb of the same name (\citealt{Pichl1967}). 

\TCheadword{bithii} (Port \textit{vidro} ‘glass') \textit{n} bottle, [bìthìì]/[bìthìì wɔ́m dɛ̀] bottle/medicine bottle (K dialect); \textit{bithir} bottle (\citealt{Pichl1967}). \textit{Ikoi bithi thiseko ki, thanɛ thakoŋ pɛli vɛ.} We take the bottle of hooks, those broken ones. \textit{Bithir lɛ hɔ̃ beye̹n.} The bottle is empty (\citealt{Pichl1967}). 

\TCheadword{bitni} \textit{v} fall on one's knees, kneel down (\citealt{Pichl1967}). \textit{Pɔ̀ ànyàɛ̀ ŋá bìtìn chɔ́chàì.} People kneel in church.

\TCheadword[1]{biyɛni} (unspec. of \TClink[1]{bi}, \TClink[2]{ni}, see \TClink[1]{bi})

\TCheadword[2]{biyɛni} (der. of \TClink[1]{biyɛni} (unspec. of \TClink[1]{bi}, \TClink[2]{ni}), see \TClink[1]{bi}) 

\TCheadword{blem} (Eng \textit{blame}) \textit{cf}: \TClink[5]{baŋ}, \TClink{thɛkɛ}. \textit{v} blame. \textit{Mista, laŋgba landɛ koŋ pa hu, wɔi hun wɔ ŋai hun hɔm lɛ ŋa ma blem wanthɛm dɛ vɛo.} Mister, the man is dead, he came and he told them that you should not blame that woman. \textit{Hɔ ŋa ma blem wanthɛmdɛ, aftabakɛ nai landɛ ŋɔ kanthka gbaŋ, ŋɔ che bɔ honi.} He said, “Do not blame the woman, the way for the afterbirth was blocked, it was not able to come out.”

\TCheadword{blidin} (Eng \textit{bleeding}) \textit{n} bleeding. \textit{Aftabakɛ ŋɔ hun gba ki <gbiŋ>, blidin iŋɔi huŋyi ki fip.} The afterbirth came and really got stuck <gbiŋ>, then bleeding burst out badly. \textit{Wanthɛmdɛ ka le blid te ni hu.} The woman kept bleeding until she died.

\TCheadword[1]{bo} \textit{cf:} \TClink[3]{bɔm}. \textit{v} meet (B dialect). \textit{Kache pabondɛ mbowɔni nwoth mɔi wɔ hu mi vethi.} In the past, if you met someone with (multiple) loads, you (would) say come help me (e.g., get this on my head). \textit{Iyema la gbo shi tɛŋka iboma lɔ ni gbo.} We just want to know if maybe we just meet them now.

\TCsubword[2]{bon} (der.) \textit{n} \textbf{1)} meeting. \textit{Boon dɛ kɔ che parɛ Furabee Kɔlej kɔ koŋ sẽyni.} The meeting which was recently at Fourah Bay College has dispersed (\citealt{Pichl1967}). \textbf{2)} \textit{bòn} feast, dance (\citealt{Sumner1921}). \textit{Bɛɛ pɔkɛ wɔ ka huɛ ni bon bom kɔ huŋ duk; pɔkai gbi hɔ taŋ ŋa wɔ.} The chief of the country died and then a great feast took place; the whole country cried for him (\citealt{Sumner1921}). \textbf{3)} ceremony. \textit{Bɔn bom kɔɛ, pɔ bia lɛ siŋ haaŋ.} If it is a big ceremony, they celebrate for a long time.

\TCsubword[1]{boni} (der.) \textit{cf}: \TClink{keni} (der. of \TClink{ke}, \TClink{-ni}), \TClink[1]{lɛli} (comp. of \TClink[3]{lɛ}), \TClink{nɔɔmi}. \textit{v} \textbf{1)} meet one another; \textit{bón} meet one another (\citealt{Sumner1921}). \textit{Kisik lɛ hã hini lɛ hã pɛ bɔni nɛn sana lɛ.} At the end they decided they would meet again in the new year (\citealt{Pichl1967}). \textit{Hin gbi hĩ bon' ka.} We are all meeting each other here (\citealt{Pichl1967}). \textbf{2)} find. \textit{Chaŋ bo yɛ ikache math boni ɛ.} Only when we used to play hide-and-seek. comp. \TClink{mathboni} (see \TClink{math}) 

\TCheadword[2]{bo} \textit{prt} emphatic particle. \textit{Yaŋ bɛ agbem bo apumma mɛn.} Myself, I gave birth to five children. \textit{Bɛŋ miɛ bó kɔ̀ nɛ̀kí!} My leg hurts! 

\TCheadword{boa} \textit{v} [bòá] be early (K dialect). \textit{À bíá bòá.} I have to be early. \textit{Braima wɔe boa ha kɔ lɛɛli mpɛl lɛ ma kɔ chɛncha lɔɔli huɛ lanthgbɔl lɛ.} Braima goes out early to inspect the nets which he went to check yesterday. \textit{A chen che ka gbəŋ ipal; lɛ nyemaɛ-m gbo bɔnthi gbəŋ boa.} I shall not be here tomorrow at midday; if you want to meet me, be early tomorrow (\citealt{Pichl1967}). \textit{Burɛ, yɛ bi hã boa ki-a, kɔ ma hã bɔnthɔ mi mputhun.} Bureh, why are you so early? You have taken me unawares (\citealt{Pichl1967}). 

\TCheadword[1]{bobo} \textit{n} fish species, tenni-fish, [bóbó]/[bóbósɛ̀] fish sg/fish pl (K dialect); (wɔ/hã) tenni-fish (Albula vulpes) (\citealt{Pichl1967}). 

\TCheadword[2]{bobo} \textit{cf}: \TClink{nɔwu} (comp. of \TClink{nɔ}, \TClink[1]{wu}), \TClink[2]{pɔm}. \textit{n} corpse, [bòbò], [bòbòɛ̀] corpse (K dialect). 

\TCheadword[3]{bobo} \textit{cf}: \TClink{mumu}. \textit{n} deaf mute, [wɛ̀ɛ̀ bóbó], [bóbó wɛ̀ɛ̀] he cannot talk (either order okay) (K dialect).

\TCheadword[4]{bobo} \textit{cf}: \TClink{bente}, \TClink{bik}. \textit{n} [bóbó], [bòbò thɛ́] shroud mat (K dialect); (hɔ̃/tha) mat used for wrapping a corpse (\citealt{Pichl1967}). 

\TCheadword{bobon} [bóbòn] \textit{n} bird species, woodpecker, likes dead trees, lays eggs inside hollow it makes in trees, brown and white, some reddish, multiple species, [bóbón] (also [bóbòn])/[bóbón sɛ̀] woodpecker/woodpeckers (K dialect).

\TCheadword{boe} textit{n} [bòè] tree species, rubber tree (K dialect); \textit{bue} (kɔ/ma) rubber tree (Manihot glaziovii) (\citealt{Pichl1967}).

\TCsubword{tismabue} (comp.) \textit{cf}: \TClink{jɛiŋɛiŋ}. \textit{n} rubber (\citealt{Pichl1967}). 

\TCheadword{Boɛ} \textit{nam} Boe, female name given to a person. \textit{Boɛ, waŋ mɔ lo chen tïntïn, koŋ bɛ yenwɛy, ŋ kɔ wɔ yi.} Boe, your daughter is not straight, she has gone bad, go ask her (\citealt{Pichl1967}). 

\TCheadword{bogba} \textit{n} socks (K dialect); \textit{mbogba} (?/ma) short stockings (\citealt{Pichl1967}). 

\TCheadword{boi} \textit{cf}: \TClink{chɛnchi}, \TClink{plet}. \textit{n} dish, basin; [bóé] dish, plate, basin (K dialect); \textit{bo̹y} (hɔ̃/tha) dish, plate (\citealt{Pichl1967}). \textit{Joɛ kɔ ni ho, mɔi thok boithɛ.} After the rice is properly dry, you wash the dishes. \textit{Yɛ mɔ koŋ chɔŋ boi po mɔɛ, mɔi bɛ boi apima mɔɛ, mɔ hɔ sɛmi, mɔmbɛ mɔi lɛ.} After dishing out your husband's basin, then you put your children's basin and put it down, then it's left to you. \textit{Ŋ kɔ sankath bo̹y lɔ, hɔ̃ chen charaŋ.} Go rinse the plate there, it is not clean (\citealt{Pichl1967}). 

\TCheadword{bok} \textit{n} \textbf{1)} leaves used for making a sauce. \textit{Kɛ yɛ mɔ kɔ chi bokɛ vɛ, mɔ kɔ le thɔkɔ.} But after you have gone for leaves, you wash it first. \textbf{2)} sauce made from leaves of the same name; (kɔ/-) kind of sauce, kreŋ-kreŋ (\citealt{Pichl1967}). \textit{Pɔi chɛth bokɛ pɔiya joɛ ha yindɛ ŋai hun gbompani ŋai hun jo.} They will cook the sauce and the rice, and everybody will gather around and eat. \textit{Aaa yɛ mɔ ni koŋ ha vɛ ni mɔi thiŋgi bokɛ mɔi sɛmi.} After doing all that, you take the sauce off the fire and put it down.

\TCsubword{bok-kiin} (unspec.) \textit{n} (kɔ/\nobreakdash-) sauce type, slippery sauce (\citealt{Pichl1967}). 

\TCheadword{Boka} \textit{cf}: \TClink{Gboka}, \TClink{Gbɔkathoŋthoŋ}. \textit{nam} \textbf{1)} Boka, name given to 6\textsuperscript{th} son. \textbf{2)} Poro Society spirit who appears as a dancing masquerade (K dialect). 

\TCheadword{boka} \textit{cf}: \TClink[2]{baa}, \TClink{balmaa}, \TClink[2]{gbato}, \TClink{gbatɔ}, \TClink[1]{hɔlɔŋ}. \textit{n} [bóká] kind of scyth, cutlass, machete, [bóká]/[bóká thɛ̀] or [thìbóká] cutlass/cutlasses (K dialect). 

\TCheadword{bokanre} \textit{n} (wɔ/hã, N) fish species, parrot grouper (Cryptotomus spp.) (\citealt{Pichl1967}). 

\TCheadword{bokichal} \textit{n} (kɔ/ma) plant species, hibiscus (Hibiscus scotellii) (\citealt{Pichl1967}). 

\TCheadword{bok-kiin} (unspec. of \TClink{bok}) 

\TCheadword[1]{bokoth} \textit{cf}: \TClink[1]{tɔth}. \textit{v} suck out, e.g., the marrow of a bone (\citealt{Pichl1967}). 

\TCheadword[2]{bokoth} \textit{n} (kɔ/ma) plant species, creeping plant with small round leaves the size of a small coin, has sour taste, frequently used as a medicine (\citealt{Pichl1967}). comp. \TClink{keŋkeŋbokoth} (see \TClink{keŋken}) 

\TCheadword{bokthampel} \textit{n} (kɔ/ma) plant species, cactus, esp. opuntia (\citealt{Pichl1967}). 

\TCheadword{boku} \textit{n} (kɔ/hɔ, i) palm species, coco-palm (Cocos nucifera) (\citealt{Pichl1967}).

\TCheadword[1]{bol} \textit{n} \textbf{1)} [ból] head (K dialect). \textit{Pɔ thu wɔ ilathɛ, pɔ bɛ wɔ vɛthɛ bol.} People spat on him, and they put thorns on his head. \textit{Amaɛ ŋai hun, ŋa kɔ woth thi bolɛ, ŋa yɔk kebelthai ɔ tithai.} The women will come and carry it on their heads, and take it to farm houses or towns. \textbf{2)} top. \textit{Ye hã bɛthi bo̹l wɔ lɛ hɔ̃ lee thɔt lɛ.} When they cut off the top of the tree, there is the trunk which remains (\citealt{Pichl1967}). \textbf{3)} mind. \textit{Mà mì bénbòlɛ̀.} Do not keep me in your mind (i.e., Do not think or worry about me). \textit{Bo̹lɛɛ hɔno wɔ lɛ bi hã bali.} One's mind tells one he will be rich (\citealt{Pichl1967}). \textbf{4)} attention. \textit{Lɛm muyu gbo ni mbɛ komɔ kaŋdai, ni wɔnbɛ bɛlɔ bolwɔi, mɛkindɛ ŋɔ vɛ.} If you are patient and put your children in school, and they pay attention there, that is the end. comp. \TClink{kɔnaibol} (see \TClink[2]{kɔ}), \TClink{naibol} (see \TClink[1]{nai}), \TClink{pelbol} (see \TClink[2]{pel}), \TClink{pikith-bol} (see \TClink{pikith}), \TClink{thɔkbol} (see \TClink[2]{thɔk}), \TClink{vebolmin} (see \TClink[1]{vee})

\TCsubword{bɛn-bole} (comp.) \textit{v} plan (lit. to put into one's head) (\citealt{Pichl1967}). \textit{Ya bən bo̹le hã kɔ gbəng.} I plan to go tomorrow (\citealt{Pichl1967}).

\TCsubword{bolbooth} (comp.) \textit{n} (hɔ̃/tha) bow of canoe or boat (\citealt{Pichl1967}).

\TCsubword{boldinthɛ} (comp.) \textit{adj} white-haired (\citealt{Pichl1967}).

\TCsubword{bolgbeni} (comp.) \textit{n} (hɔ̃/tha) mask type, profane mask (\citealt{Pichl1967}). 

\TCsubword{bolgɔbɔ} (comp.) \textit{n} (hɔ̃/tha) mask type, mask of Mende origin (\citealt{Pichl1967}).

\TCsubword{bolkathil} (comp.) \textit{cf}: \TClink{thɔthboot} (comp. of \TClink[1]{thɔth}, \TClink{bot}). \textit{adj} stubborn (\citealt{Pichl1967}). \textit{Tamɔ lɔ wɔ bo̹l kathïl.} This boy is stubborn (\citealt{Pichl1967}). \textit{Tamɔ lɛ wɔ bo̹lkathïl chen thekni buŋ.} They boy is stubborn, he doesn't feel flogging (\citealt{Pichl1967}). \textit{Aa, pɔ ka che mi buŋ, aka che bolkathil.} Yes, they used to beat me, I was stubborn.

\TCsubword{bolkoŋgoli} (comp.) \textit{n} (hɔ̃/tha) mask type, Kongoli mask of Mende origin (\citealt{Pichl1967}).

\TCsubword{bollɛveleŋ} (comp.) \textit{n} \textit{boe̹le̹ŋ} (hɔ̃/-) back of the head or neck (contraction of \textit{bo̹l lɛ ve̹le̹ŋ}) (\citealt{Pichl1967}).

\TCsubword{bolmachenche} (comp.) \textit{cf}: \TClink{bolnafali} (comp. of \TClink[1]{bol}). \textit{n} (hɔ̃/tha) mask type, same as \textit{bo̹l nafali} (\citealt{Pichl1967}).

\TCsubword{bolmin} (comp.) \textit{cf}: \TClink{thifaŋ}. \textit{adj} stupid, crazy (\citealt{Pichl1967}). \textit{Bo̹lmin mɔ ɛ!} You are an idiot! (\citealt{Pichl1967}). \textit{Bo̹l-min ken tukum trï bɛ.} To be stupid like a bushgoat near the town (\citealt{Pichl1967}). 

\TCsubword{bolmɔ} (comp.) \textit{cf}: \TClink{puk}. \textit{n} (hɔ̃/tha) nipple (\citealt{Pichl1967}). 

\TCsubword{bolnafali} (comp.) \textit{cf}: \TClink{bolmachenche} (comp. of \TClink[1]{bol}). \textit{n} (hɔ̃/tha) Mende play mask, abstract made of straw or cloth in various colors (\citealt{Pichl1967}). 

\TCsubword{bolnow} (comp.) \textit{n} (hɔ̃/tha) Bondo helmet mask (\citealt{Pichl1967}). 

\TCsubword{bolpel} (comp.), (id.) \textit{cf}: \TClink{pelbol} (comp. of, id. of \TClink[2]{pel}, \TClink[1]{bol}). \textit{adj} bald (lit. head egg) (\citealt{Pichl1967}). \textit{Ya bo̹l-pel, ya biɛni iriŋ.} I am bald, I have no hair (\citealt{Pichl1967}). 

\TCsubword{bolthihiol} (comp.) \textit{cf}: \TClink{shiliŋ}. \textit{n} (hɔ̃/tha) shilling, 1 shilling equals 4 heads of tobacco (\citealt{Pichl1967}). 

\TCsubword{thibolɔtok} (comp.) \textit{n} (-/tha) head of bed (\citealt{Pichl1967}). 

\TCsubword{bolɛɛnɔ} (unspec.) \textit{n} (hɔ̃/-) mind, determination (\citealt{Pichl1967}). 

\TCheadword[2]{bol} \textit{n} (ma) lies(s), \textit{fothi mbol} tell a lie (\citealt{Pichl1967}). \textit{Nchen nhã fothok thɛm mɔ nɔthi mbol.} You shall not calumniate your friends (\citealt{Pichl1967}). \textit{Nche gbo lem thelian mbol.} You should not just lie. \textit{Ŋa ma hi gbo fothok mbol!} Do not just lie to us! comp. \TClink{fothimbol} (see \TClink{fothi}) 

\TCsubword{limbul} (unspec.) \textit{n} false evidence. \textit{Bɛɛ lɛ Kɔng kol sirɔng hã sɔng wɔ ni kɔ wɔŋ beli li-mbul.} The chief gave Kong a corruption fee to bribe him to go and give false evidence (\citealt{Pichl1967}). 

\TCheadword[3]{bol} \textit{post} on. \textit{Bia bɛth rəm wɔ lɛ thɛmni yenwɛy nai lɛ bol.} Bia has cut his toe, he stubbed it badly on the way (\citealt{Pichl1967}). 

\TCheadword[4]{bol} \textit{cf}: \TClink[1]{bɛŋk}, \TClink[1]{sil}. \textit{n} [bòl] palm maggot (K dialect); (wɔ/hã, i, N) maggot of the palm beetle found in rotten palm trees which can be roasted and eaten (\citealt{Pichl1967}). 

\TCsubword[1]{koŋkbo} (comp.) \textit{cf}: \TClink{biŋkinchin}, \TClink[1]{thɔŋpaŋ}. \textit{n} [kóŋkbó] beetle species, found in palm trees, adult form of the palm maggot \textit{bol} (K dialect); (wɔ/hã, N) beetle species, lives in rotten oil-palm trees. Children stick a straw up its anus to make it “sing” (\citealt{Pichl1967}).

\TCsubword[2]{koŋkbo} (comp.), (id. of \TClink[1]{koŋkbo}) [kóŋkbó] \textit{cf}: \TClink[2]{bɛŋk} (id. of \TClink[1]{bɛŋk}), \TClink{nɔyilɔ} (comp. of \TClink{nɔ}, \TClink[1]{yil}), \TClink[2]{thɔŋpaŋ} (id. of \TClink[1]{thɔŋpaŋ}). \textit{n} drunkard, term of abuse based on fact that beetles are never far from alcohol, i.e., palm wine (K dialect). 

\TCheadword[5]{bol} [ból] \textit{v} slip (K dialect). \textit{Wɔ̀ ból lɛ̀ɛ̀ kò.} He slipped on the ground. 

\TCheadword{bolboth} (comp. of \TClink[1]{bol}, \TClink{bot}, see \TClink[1]{bol}) 

\TCheadword{boldinthɛ} (comp. of \TClink[1]{bol}, \TClink{dinthɛ} (der. of \TClink{dinth}, \TClink{-ɛ}), see \TClink[1]{bol}) 

\TCheadword{bolɛɛnɔ} (unspec. of \TClink[1]{bol}) 

\TCheadword{bolgbeni} (comp. of \TClink[1]{bol}) 

\TCheadword{bolgɔbɔ} (comp. of \TClink[1]{bol}) 

\TCheadword{boli} \textit{n} diarrhea.

\TCheadword{bolkathil} (comp. of \TClink[1]{bol}, \TClink[1]{kathil} (der. of \TClink{kath}, \TClink{-il}), see \TClink[1]{bol}) 

\TCheadword{bolkek} [bólkèk] \textit{n} fish species, bearded, 18 inches long, edible, tasty (K dialect); \textit{bɔlke̹k} (wɔ/hã, N) beard-beard fish (Pentanemus quinquarius) (\citealt{Pichl1967}). comp. \TClink{pɛlbɔlkek}

\TCheadword{bolkoŋgoli} (comp. of \TClink[1]{bol}) 

\TCheadword{bollɛveleŋ} (comp. of \TClink[1]{bol}, \TClink[1]{veleŋ}, see \TClink[1]{bol}) 

\TCheadword{bolmachenche} (comp. of \TClink[1]{bol}) 

\TCheadword{bolmin} (comp. of \TClink[1]{bol}, \TClink[3]{min}, see \TClink[1]{bol}) 

\TCheadword{bolmɔ} (comp. of \TClink[1]{bol}, \TClink[1]{mɔ}, see \TClink[1]{bol}) 

\TCheadword{bolnafali} (comp. of \TClink[1]{bol}) 

\TCheadword{bolnow} (comp. of \TClink[1]{bol}) 

\TCheadword{bolo} [bòlò] \textit{cf}: \TClink{chocho}, \TClink{kɔŋko}, \TClink{nɔtɔ}, \TClink{suk}, \TClink{thoŋku}. \textit{n} shell of shellfish species collected for jewelry, small ones used for necklaces (K dialect). 

\TCheadword[1]{Bolom} \textit{n} \textbf{1)} the Sherbro people. \textit{Abo̹lo̹m hã lɛ aɲin hã si hõth lɛ.} The Sherbros, they know how to fish (\citealt{Pichl1967}). \textbf{2)} the Sherbro language. \textit{Shenge ka pɔ ŋa pɛ theli nwɔk mpim bisaid Mbolom?} Here in Shenge do they speak other languages besides Sherbro?

\TCsubword{Bolomnɔ} (comp.) \textit{n} Sherbro person. \textit{Lɛ mbɔn gbo hɔ mpootoo lɛ koot l'ay, mɔ lɛ Bo̹lo̹m-nɔ, Them-nɔ, Mɛnde-nɔ.} If you don't speak English at the court, there is someone who will interpret for you in your language, be you a Bolom, Themne, or Mende (\citealt{Pichl1967}). \textit{Lɛ nɔ shi la bo lɛ mɔ Bolomnɔ, nɔ ndɔndɔ wɔ mɔ ka limani.} If a person knows that you are Sherbro, everyone gives you respect.

\TCheadword[2]{Bolom} \textit{adj} pertaining to Bolom people, culture, etc. \textit{Apuma Bo̹lo̹m hãn gbi hã kaaŋ hã sakïl.} All Bolom children learn how to swim (\citealt{Pichl1967}.) \textit{Ilel mi Bolomdɛ ŋɔ lɔ Sɔ.} My Bolom name is Sor.

\TCheadword[3]{bolom} \textit{n} (ma) case, affair, matter (\citealt{Pichl1967}). \textit{A biɛn chæ pɔsɔ hã hɔ 'mbolon dɛ.} I have not much to say on this matter (\citealt{Pichl1967}). \textit{Nsey lɛ hã koŋ sey mbo̹lo̹m dɛ.} The witnesses have given evidence in the case (\citealt{Pichl1967}.) \textit{Mbo̹lo̹m ŋwɛy ma che paalɛ bai ko, anya atïŋ dɛ hã lo̹l.} In the bad case that was recently before the court, the two men were set free (\citealt{Pichl1967}).

\TCheadword{Bolomnɔ} (comp. of \TClink[1]{Bolom}, \TClink{nɔ}, see \TClink[1]{Bolom}) 

\TCheadword{boloŋk} [bòlòŋk] \textit{n} fish species, two types, one at sea, one in the river; the one at sea much bigger, six feet long, the river one is 1-2 inches in diameter, both types edible (K dialect). 

\TCheadword{bolpel} (comp. of, id. of \TClink[1]{bol}, \TClink[2]{pel}, see \TClink[1]{bol}) 

\TCheadword{bolthihiol} (comp. of \TClink[1]{bol}, \TClink{hiɔl}, see \TClink[1]{bol}) 

\TCheadword{bom} \textit{adj} \textbf{1)} big, large. \textit{Kilthi lɛ tha Pujoŋ kunɛ tha bom.} The houses in Pujehun are big (\citealt{Pichl1967}). \textit{Bɔn bom kɔɛ, pɔ bia lɛ siŋ haaŋ.} If it is a big ceremony, they celebrate for a long time. \textit{Bɛl Maaɛ ŋani poo wɔɛ ŋa lɔ thoŋka boe bom dɛ tokɛ wusɛ kunɛ.} Rat Wife and her husband are arguing in the thatch above the big kitchen. \textbf{2)} high. \textit{Mbunkluŋ dɛ ma bo̹m.} The waves are high (\citealt{Pichl1967}). \textbf{3)} important. \textit{Koŋ kueni ŋke̹n bo̹m chaŋ Thua.} Kong thinks himself more important than Thua (\citealt{Pichl1967}). \textbf{4)} mighty. \textit{Ya lanɛ Hɔbatokɛ sɛmthi bo̹m dɛ.} I believe in God the Almighty (\citealt{Pichl1967}). \textbf{5)} old. \textbf{6)} great. comp. \TClink{bɛɛbom} (see \TClink[2]{bɛɛ}), \TClink{gbanabom} (see \TClink{Gbana}), \TClink{gbɔlbom} (see \TClink{gbɔl}), \TClink{kakbom} (see \TClink[2]{kak}), \TClink{kelbom} (see \TClink[1]{kel}), \TClink{kolbom} (see \TClink{kol}), \TClink{lelbom} (see \TClink[5]{lel}), \TClink{naibom} (see \TClink[1]{nai}), \TClink{pɛlbom} (see \TClink[2]{pɛl}), \TClink{rɛmbom} (see \TClink{rɛm}), \TClink{sɛɛbom} (see \TClink{sɛɛ}), \TClink{sɛkbom} (see \TClink[1]{sɛk}), \TClink{yubom} (see \TClink{yu}), der. \TClink{palbom} (see \TClink[3]{pal}) 

\TCsubword{kueni-bom} (comp.) \textit{v} be proud, feel important (\citealt{Pichl1967}).

\TCsubword{bomba} (der.) \textit{adj} very big, very large. \textit{Yel lo kinɔ ka che bomba nɛn thigber tha koŋ chaŋ dɛ.} This island was very big many years ago.

\TCsubword{bombom} (der.) \textit{adj} big, large. \textit{Pɔ bi ha di naa tri thi bom-bom dɛai gbi.} They would have to kill cows in all the big towns (\citealt{Sumner1921}). \textit{Ihɔ̃kɔ hã lɛ bo̹mbo̹m.} Their goiters are big (\citealt{Pichl1967}). 

\TCheadword{bomba} (der. of \TClink{bom}, \TClink[2]{ba}, see \TClink{bom}) 

\TCheadword{bombo} \textit{n} (kɔ/-) smallpox (\citealt{Pichl1967}).

\TCheadword{bomboli} \textit{n} insect species, like an ant but bigger, eats people's food, moves in groups, can foretell an event when they enter a house en masse (K dialect).

\TCheadword{bombom} (der. of \TClink{bom}) 

\TCheadword{bomp} \textit{n} \textbf{1)} part of an area, e.g., where a single crop is grown (K dialect) \textbf{2)} (kɔ/ma) cape (\citealt{Pichl1967}).

\TCheadword{bompa} \textit{v} attack, jump at (\citealt{Pichl1967}). 

\TCheadword{Bompɛ} \textit{nam} \textbf{1)} Bumpeh Chiefdom. \textit{Ya gbemni Nyemɔko, Mamu Sɛkshɔn, Bompɛ Chifdɔm, Mɔyamba Distrikt.} I was born in Moyeamoh, Mamu Section, Bumpeh Chiefdom, Moyamba District. \textit{Yami pɔ gbem wɔ pɔk Rotifuŋgɛ, lɔ pɔ vel Bompɛɛ, Nyogbako.} My mother was born in Rotifunk country, which they used to call Bumpeh, Moyogba. \textbf{2)} Bompetok or Bompetoke, coastal town in Kagboro Chiefdom located southeast of Shenge, the town headquarters of Kagboro Chiefdom. \textit{Yaŋ Bompɛ ko lɔ ayɛ.} I live in Bompetoke (Kagboro Chiefdom).

\TCheadword[1]{bon} \textit{n} (kɔ/ma) tree species, swizzle-stick tree (Rauvolfia vomitoria) (\citealt{Pichl1967}). 

\TCheadword[2]{bon} (der. of \TClink[1]{bo}) 

\TCheadword[3]{bon} \textit{v} [bón] harvest oysters (K dialect); \textit{bo̹ŋ} cut or knock oysters from rocks or the roots of mangrove trees (\citealt{Pichl1967}). \textit{Yà kɔ̀ bón véésɛ̀.} I go harvest oysters. \textit{ Pənkiyɔ, pənkiyɔ, mɔ thi ka-m lala ya kɔ bong vee.} Jumping, jumping, you should rather give me a paddle to cut off oysters (song) (\citealt{Pichl1967}). 
\TCheadword[1]{Bondo} \textit{nam} Bondo Society, girls' initiation society, bush school. \textit{Boŋgo che ki, nɔ mbiyɛni gbo fe nche lɔik Bondo.} These days, if one has no money, one will not enter Bondo. \textit{La minɛ yɛpɔ lɔik wanda Bondoɛ…} It means when a girl is initiated into the Bondo Society…

\TCsubword{Bondogbaka} (comp.) \textit{nam} (hɔ̃/-) Bondo Society without a spirit who appears as a dancing masquerade, where girls are trained only to dance (\citealt{Pichl1967}).

\TCheadword[2]{Bondo} \textit{adj} pertaining to Bondo Society. \textit{Yaa wɔ ka che sokonɔ Bondo.} Her mother was a Bondo leader.

\TCheadword{Bondogbaka} (comp. of \TClink[1]{Bondo}, \TClink{gbaka}, see \TClink[1]{Bondo}) 

\TCheadword{bondɔ} (comp. of \TClink[1]{boŋ}, \TClink[7]{lɔ}, see \TClink[1]{boŋ}) 

\TCheadword[1]{boni} (der. of \TClink[1]{bo}, \TClink{-ni}, see \TClink[1]{bo}) 

\TCheadword[2]{boni} \textit{cf}: \TClink{mathboni} (comp. of \TClink{math}, \TClink[1]{boni}). \textit{n} (hɔ̃/-) children's game hide-and-seek (played by boys and girls in the evening) (\citealt{Pichl1967}). \textit{Tha ika che siŋ, iŋa boniɛ, isiŋ ni athɔma hiɛ.} That is what we used to play, hide-and-seek, we played with our mates. \textit{Wɛl chaŋbo yɛ ika che kɔni siŋ boniɛ ni athɔmamdɛ paŋdɛ.} Well except when we used to go play the hide-and-seek game with my mates in the evening.

\TCheadword{Bonth} \textit{nam} \textbf{1)} Bonthe, the largest city in Bonthe District. \textit{Kɛ pɔ chelɔ pɛ theli Mbolom ken Bonthiko, Thomboko, inal pimdɛ.} But they no longer speak Bolom there like in Bonthe, Tombo, and other places. \textit{Pə he̹rkɛ wɔ bonth ko.} He was taken across to Bonthe (\citealt{Pichl1967}). \textbf{2)} Sherbro Island. \textbf{3)} Bonthe District. 

\TCheadword{bontum} \textit{n} (wɔ/hã, N, si) wasp species, black and yellow wasp (\citealt{Pichl1967}). 

\TCheadword[1]{boŋ} \textit{n} (hɔ̃/tha) low cliff, low hill or slope (\citealt{Pichl1967}).

\TCsubword{bondɔ} (comp.) \textbf{1)} \textit{Loc} on the waterside, landing place, wharf (\citealt{Pichl1967}). \textit{Bímbí bòm kɔ̀ ché ná bóndɔ̀.} There was a big crowd at the wharf. \textit{Hã ke bondɔ ko ni hã nan wɔmdɛ chiɛ ko.} Look for the wharf and pull the canoe on shore (\citealt{Pichl1967}). \textit{Bot lɛ koŋ tïk bondo ko, hãmɔ telɛ han wunkiɛ.} The boat has landed, they are awaiting you to weigh anchor (\citealt{Pichl1967}). \textit{Bondo ka lɔ thuŋ puth, isay igbe̹r lɔ ka.} It stinks very much at the wharf; there is a lot of filth there (\citealt{Pichl1967}). \textbf{2)} \textit{n} shore. \textit{Ni bondɔ ka lɔ ki ka.} And this is the shore. 

\TCheadword[2]{boŋ} \textit{n} bird species, very small, stays in round thatch houses in the bush or town in groups, black and whitish color (K dialect); (wɔ/hã, N, si) small black and white bird (\citealt{Pichl1967}). comp. \TClink{peenɛmboŋ} (see \TClink{peenɛ}), \TClink{Taimboŋ} (see \TClink[1]{tai}) 

\TCheadword{boŋgo} \textit{temp} \textbf{1)} now. \textit{Ntoŋgi mi mu we ɛ ŋɔ pɔ gbisiŋdɛ ni boŋgo.} Show me the way they used to marry to that of now. \textit{Kɛ boŋgo nia?} What about now? \textit{Kɛ boŋgo pɔ che pɛ ka ha ŋabɛn limani gbi?} But now they do not give elders respect at all? \textbf{2)} these days. \textit{Boŋgo che ki, nɔ mbiyɛni gbo fe nche lɔik Bondo.} These days, if one has no money, one will not enter Bondo.

\TCheadword{boŋhul} \textit{cf}: \TClink{sonthuli} (der. of \TClink{sonthul}, \TClink[1]{-i}). \textit{v} whet or sharpen a knife (\citealt{Pichl1967}). \textit{Kong ma gbo chentheŋgi si yi ma-e boŋhũl.} When he has finished forging them, we sharpen them (\citealt{Pichl1967}.) 

\TCheadword{boŋk} \textit{cf}: \TClink{lɔkɔ}, \TClink[1]{mɛŋk}, \TClink[1]{tɛm}. \textit{n} (ho/-) time, certain time or day (\citealt{Pichl1967}). \textit{Bonk lo ya ke Sese.} At the time, I saw Sese. \textit{Boŋ cheki, ma koŋ gbako.} Now, they have grown (the oil palms).

\TCheadword[1]{boo} [bòò] \textit{cf}: \TClink[2]{gber}, \TClink[1]{pul}. \textit{n} \textbf{1)} bread, no plural (K dialect); \textit{boo} (hɔ̃/ma) bread (\citealt{Pichl1967}). \textit{Ŋ kɔ mi pinɛ bo lɛ.} Go buy some bread for me. \textbf{2)} rice flour. \textit{Wɔ́ bìnthìmà bòòɛ̀.} She mixed the rice flour (with water).

\TCsubword{boombaana} (comp.) \textit{n} (hɔ̃/ma) bread made of rice and bananas (\citealt{Pichl1967}). 

\TCsubword{booŋkantri} (comp.) (Eng \textit{country}) \textit{n} (hɔ̃/ma) bread made of groundnuts (lit. country bread) (\citealt{Pichl1967}). 

\TCheadword[2]{boo} [bóó] \textit{n} kitchen, [bóó]/[thìbóó] or [bóóthɛ̀] kitchen/kitchens (K dialect); \textit{boo} (hɔ̃/tha) kitchen (\citealt{Pichl1967}). \textit{Wɔ́ bàs bòé kò.} He is sweeping the kitchen. \textit{Wɔ̀ bás bòé kò.} He swept the kitchen. \textit{Ŋkɔ kwey jɛmdi lɛ bwe ko.} Go take the fire from the kitchen (\citealt{Pichl1967}).

\TCheadword{boombaana} (comp. of \TClink[1]{boo}, \TClink{baana}, see \TClink[1]{boo}) 

\TCheadword{booŋkantri} (comp. of \TClink[1]{boo}) 

\TCheadword[1]{bos} \textit{v} [bós] shave (K dialect, \citealt{Pichl1967}). 

\TCsubword{bosni} (comp.) \textit{v} shave oneself. \textit{Ya kɔ bosni.} I go to shave (myself) (\citealt{Pichl1967}). 

\TCheadword[2]{bos} \textit{cf}: \TClink{chasa}. \textit{n} [bòs] calabash (K dialect); (kɔ/ma) calabash, bottle (\citealt{Pichl1967}). \textit{Bos sɛ koŋ pɛl.} The calabash is broken.

\TCsubword{bosi} (der.) \textit{v} bail, e.g., water (K dialect). \textit{Wɔe tipɛ ha taŋ yɛ wɔ bosi mmɛn dɛ hiŋk wɔm dɛai.} He began to cry as he was bailing water from the boat.

\TCheadword[3]{bos} \textit{n} (-/ma) inner part of the nostrils (\citealt{Pichl1967}). comp. \TClink{yaŋmbusɛ} (see \TClink{yana}) 

\TCsubword{pɛlmbos} (comp.) \textit{n} bleeding of the nose (\citealt{Pichl1967}). 

\TCheadword{bosi} (der. of \TClink[2]{bos}, \TClink[1]{-i}, see \TClink[2]{bos}) 

\TCheadword{bosni} (comp. of \TClink[1]{bos}, \TClink{-ni}, see \TClink[1]{bos}) 

\TCheadword{bot} (Eng \textit{boat}) \textit{cf}: \TClink[1]{pampa}, \TClink[2]{wɔm}. \textit{n} boat. \textit{Bot lɛ kon tïk, hã lɔ bondɔ ko.} The boat has landed, they are at the wharf (\citealt{Pichl1967}). \textit{Nthim bot lɛ njok ɛ, thipe tha che ko!} Turn the boats to the right side, there are rocks ahead! (\citealt{Pichl1967}). comp. \TClink{bolboth} (see \TClink[1]{bol}), \TClink{ndɛthmaboot} (see \TClink{dɛth}), \TClink{thɔthboot} (see \TClink[1]{thɔth}) 

\TCheadword{botho} \textit{n} shoot. \textit{Lɛ banaɛ yema gbo wu, kɔ ye ho botho ha gbemɔ.} When a banana tree is about to die, it sends out shoots for further fruit (proverb) (\citealt{TISLL1979}). 

\TCheadword{boya} \textit{n} \textbf{1)} (hɔ̃/-) gift of kola nuts (\citealt{Pichl1967}). \textbf{2)} (kɔ/-) second stage of courtship (in presence of the woman's parents the young man gives a present to the woman and is then recognized as her suitor) (\citealt{Pichl1967}). \textbf{3)} token gift (may be [bo:ya]). \textit{Apum haŋ che mi paka apum hamika nsoiɛ, ha mi ka boyaɛ.} Some will not pay me, some give me soap, others give me a gift (for being their midwife). \textbf{4)} bribe. \textit{Yi koŋ yo̹m hã kɔ kə yi yema boya.} We have agreed to go, but we want a gift/bribe. (\citealt{Pichl1967}).

\TCheadword[1]{bɔ} \textit{n} (hɔ̃/tha) bar of an estuary or harbor (\citealt{Pichl1967}).

\TCsubword{bɔhɔl} (comp.) \textit{n} (hɔ̃/tha) sea bar, opening to the ocean (\citealt{Pichl1967}). 

\TCheadword[2]{bɔ} \textit{cop} probably be. \textit{Wɛl, ani bɔ che nɛnthi kwaŋa ra ni mɛn.} Well, I am probably 65 years old.

\TCheadword[3]{bɔ} \textit{Aux} \textbf{1)} be able. \textit{A chen bɔ pin sigarɛt lɛ, ya biɛn gbo fe̹.} I am not able to buy cigarettes if I have no money (\citealt{Pichl1967}). \textit{A che ŋɔ bɔ si.} I would not be able to know it. \textbf{2)} can.

\TCheadword{bɔbɔ} \textit{n} (kɔ/ma) plant species, life plant, air plant (Bryophyllum pinnatum) (\citealt{Pichl1967}).

\TCheadword{bɔfima} \textit{cf}: \TClink{humoe}, \TClink{mane}. \textit{n} medicine which contains, among other things, white of an egg, blood, fat, and other parts of a human being, the blood of a cock, and a few grains of rice (one of the most powerful medicines making its owner rich, honored by the people, and invincible in court) (\citealt{Pichl1967}). 

\TCheadword{bɔhɔl} (comp. of \TClink[1]{bɔ}) 

\TCheadword{Bɔi} \textit{nam} \textbf{1)} Boi, name given to first daughter. \textit{À lɛ́líyá Bɔ̀ì.} I'm looking for Boi. \textit{À yíyɛ́/yíɛ́ Bɔ̀ì.} I am asking for Boi. \textit{Yami kacheɛ Bɔi Kigba.} My mother was Boi Kigba.

\TCheadword[1]{bɔi} (Eng \textit{boy}) \textit{n} boy. \textit{Shenge bɔi fli wɔɔ Shenge kaɛ ya gbemiɛ wɔ yawɔ.} Even the Shenge boy, here in Shenge, I delivered him.

\TCheadword[2]{bɔi} \textit{v} get enough, be satisfied (\citealt{Pichl1967}). \textit{Ya kong bɔy, ya chen pɛ kul ya ma ki yil.} I have enough, I will drink no more lest I get drunk (\citealt{Pichl1967}). \textit{Sathia chanth lɛ koŋ bɔy mɔ lɛ, mma wɔ pɛ kuli.} Sathia's child has sucked enough, don't give him more to drink (\citealt{Pichl1967}). \textit{Yɛ mɔ ni hun minɛ puli vɛ, lɛ nke bo yabasɛ kɔ bɔ ni mɔi bɛrɛ.} When you mix it again, if you see that the onion is not enough, you add more.

\TCsubword{bɔyi} (unspec.) \textit{v} satisfy. \textit{Abɔyi ni gbo ache hun.} If I am not satisfied, I will not return.

\TCsubword{bɔini} (unspec.) \textit{v} be disgusted with. \textit{Ya koŋ bɔyni jali mɔ.} I am disgusted with you (lit. your affairs, actions) (\citealt{Pichl1967}). 

\TCheadword{Bɔima} \textit{nam} Boima, female name given to a person. \textit{Wɔlɔ Bɔima Hana.} She is Boima Hannah.

\TCheadword{bɔini} (unspec. of \TClink[2]{bɔi}, \TClink{-ni}, see \TClink[2]{bɔi}) 

\TCheadword[1]{bɔk} \textit{cf}: \TClink[1]{kɛk}, \TClink[2]{koŋ}, \TClink[2]{nya}. \textit{n} tortoise (B dialect); (wɔ/hã, N) kind of turtle (\citealt{Pichl1967}). \textit{Bɔk yema fɔs, kɛ pia wɔ kɔ kith.} The tortoise wants to punch, but its arm is short (proverb) (\citealt{TISLL1979}).

\TCheadword[2]{bɔk} \textit{n} [bɔ́k] greens (B dialect). \textit{A kə́n bɔ́kɛ̀.} I cut up the greens.

\TCheadword{bɔkla} \textit{n} (kɔ/-) gift of one piece of country cloth, one Guinea, and a brass bucket given to a woman's parents if she is determined to be a virgin by the bridegroom (\citealt{Pichl1967}). 

\TCheadword{bɔko} \textit{cf}: \TClink[1]{hoŋka}, \TClink{kahai}. \textit{post} outside (\citealt{Pichl1967}). \textit{Pɔnth lɛ hɔ̃ trï bɔko.} The swamp is outside town (\citealt{Pichl1967}). \textit{Nrus iwɔm dɛ bɔko ni pe sak kɔbɔ lɛ hã lath pəlɛ lɛ.} Push aside the wood outside and let them spread out that mat to dry the rice (\citealt{Pichl1967}). 

\TCheadword{bɔkon} [bɔ̀kón] \textit{cf}: \TClink{niŋgbi}. \textit{n} owl species, nocturnal, dark brown, about a foot high, used to be found around Shenge, but now that the forest has been broken up, it has disappeared (K dialect). 

\TCheadword{bɔkɔ} [bɔ́kɔ́] \textit{n} palm species, short palm (K dialect). 

\TCheadword{bɔkɔtok} \textit{n} (hɔ̃/tha) edge (\citealt{Pichl1967}).

\TCheadword[1]{bɔl} (Eng \textit{ball}) \textit{n} \textbf{1)} ball. \textit{Chanth lɛ wɔ sïŋk bɔɔl lɛ.} The child plays with the ball (\citealt{Pichl1967}). \textit{Wɛl i ka che ple han tɛnis bɔl.} We used to play hand tennis ball. \textbf{2)} football. \textit{Bɔllɛ ŋɔn lagbolɛ mɛŋk, mɛŋkɛ hɔ mɔigbo, ŋakɔni fillai ŋa kɔ siŋ.} The football (match) is scheduled, when the time comes, they go to the field and play.

\TCheadword[2]{bɔl} \textit{n} \textbf{1)} \textit{ibɔl} length (\citealt{Sumner1921}). \textbf{2)} height.

\TCheadword[1]{bɔm} \textit{n} \textbf{1)} swamp, [bɔ̀m]/[bɔ̀m]/ [bɔ́m] swamp/frog/meet, help (K dialect); (hɔ̃/tha) mangrove swamp (\citealt{Pichl1967}). \textit{Lagbo bɔmdai lɔɛ, pɔ kɔ ŋa gbompa ton, ɛn pɔ pɛ ka thiwonka, kaŋka kɔ ma gbompa ni bɔnɔ bul.} If it (rice field) is in a swamp, they will make it (space between plants) a little greater and make spaces so it (rice seedling) can grow without being pushed into one place. \textbf{2)} tidal mud flats. \textit{Pɔi yɔk bom dai ɔ pɔ yɔk kɔ pɔnth thai pɔi yuk.} Then people take it to the mud flats, or people will take it to a swamp to plant it. \textit{Bɔmthɛ thalɔ kɛ apim ha chelɔ yuk.} The mud flats are there, but some do not plant there.

\TCheadword[2]{bɔm} \textit{cf}: \TClink{gbɛgbɛ}. \textit{n} [bɔ̀m] frog, toad (K dialect); (wɔ/hã, N) frog (\citealt{Pichl1967}). \textit{Bɔ̀mndɛ́ ɔ̀ gbɛ́gbɛ́yɛ̀? Gbɛ́gbɛ́yɛ̀ wɔ̀ pɛ́ŋhɛ̀.} Toad or frog? It's the frog who jumps.

\TCheadword[3]{bɔm} \textit{cf}: \TClink[1]{bo}, \TClink{bɔnth}. \textit{v} \textbf{1)} [bɔ́m] meet (K dialect); meet (\citealt{Pichl1967}). \textit{Ni wɔ ye bɔm nɔma bən.} And then he met an old woman (\citealt{Pichl1967}). \textit{Wɔ munini, wɔ ye bɔnthɔ boŋ dɛ wɔn che.} She returned and found herself (lit. met herself) facing a hill (\citealt{Pichl1967}). \textbf{2)} help (K dialect).

\TCheadword{Bɔmɔtok} \textit{nam} Bomotoke, name given to headquarters of Timdale Chiefdom, Moyamba District. \textit{Wɔn pɛ gbemni Bɔmɔtok ko?} He was also born in Bomotoke (Timdale Chiefdom)?

\TCheadword[1]{bɔn} \textit{v} \textbf{1)} drag, draw along, e.g., the ground (\citealt{Pichl1967}). \textit{Kothathi wɔ lɛ tha chen bɔnni lɛɛ ko.} His clothes do not drag on the ground (\citealt{Pichl1967}). \textit{La bi a bɔɔni mɔm tɛntɛ.} That is what makes me draw closer to you. \textbf{2)} raise.

\TCheadword[2]{bɔn} \textit{n} (ma) cannibalism, to be found a cannibal at the post-mortem (\citealt{Pichl1967}). \textit{Ayɛɛn mbɔn ma lɔ pɔk lo.} Indeed, there is cannibalism in this country (\citealt{Pichl1967}). comp. \TClink{nyabɔn} (see \TClink{nɔ}) 

\TCheadword{bɔnɔ} \textit{n} place. \textit{Lagbo bɔmdai lɔɛ, pɔ kɔ ŋa gbompa ton, ɛn pɔ pɛ ka thiwonka, kaŋka kɔ ma gbompa ni bɔnɔ bul.} If it (rice field) is in a swamp, they will make it (space between plants) a little greater and make spaces so it (rice seedling) can grow without being pushed into one place. 

\TCheadword{bɔnth} \textit{cf}: \TClink[1]{bo}, \TClink[3]{bɔm}. \textit{v} \textbf{1)} meet. \textit{Mɔ lɔ bɔnth apuma mɔ ɛ han gbi}. You will meet all your children there (\citealt{Pichl1967}). \textbf{2)} help. \textit{Kache ŋɔn hi, mbi fe, mbiyɛni fe ha nyamɔ ŋa mɔ bɔnth.} In the past, whether you had money or not, your people would help you. \textit{Haŋ yɛ mɔ munini ha mɔm ko bɔnth bamɔ ŋa mpanth.} And how you came back to the town to help your father with work. comp. \TClink{nɔbonthɔ} (see \TClink{nɔ})

\TCheadword[1]{bɔŋ} \textit{n} crown-like headdress of the Taso (\citealt{Pichl1967}).

\TCheadword[2]{bɔŋ} (Eng \textit{bung}) \textit{n} (kɔ/ma) bung of barrels (\citealt{Pichl1967}). \textit{Mbɔŋ ma pipɛ ma bɛmpani iwɔm.} Barrel bungs are made of wood (\citealt{Pichl1967}).

\TCheadword{bɔŋk} \textit{cf}: \TClink{kok}. \textit{n} (kɔ/ma) testicles, scrotum (\citealt{Pichl1967}).

\TCheadword{bɔŋkia} \textit{cf}: \TClink{yɛlɔ}. \textit{adj} yellow (K dialect, \citealt{Pichl1967}).

\TCheadword{bɔɔ} \textit{cf}: \TClink[2]{gba}. \textit{n} hat, cap, [bɔ́ɔ̀]/[bɔ́ɔ̀ thə̀ tsə̀n]/[bɔ́ɔ́ thə́ ɻà]/[bɔ́ɔ́ thí yɔ̀l]/ [bɔ́ɔ́ thə́ mə̀n] hat/two hats/ three hats/ four hats/ five hats (K dialect). \textit{Bɔ wɔ lɛ hɔ bɛmpaka lithul}. His hat is made of raphia-straw (\citealt{Pichl1967}). 

\TCheadword{bɔp} \textit{n} (kɔ/ma) tree species, agidi tree (Mitragyna stipulosa) (\citealt{Pichl1967}).

\TCheadword[1]{bɔs} \textit{cf}: \TClink{peem}, \TClink[1]{pem}. \textit{n} \textbf{1)} \textit{mbɔs} (ma) peace (\citealt{Pichl1967}). \textit{Nchi mbɔs pɔkiyai.} Bring peace to our country. \textbf{2)} \textit{mbɔs} (ma) quiet (\citealt{Pichl1967}).

\TCheadword[2]{bɔs} \textit{v} \textbf{1)} be cold (\citealt{Pichl1967}). \textit{Hùɛ́ɛ̀ ŋɔ́ bɔ̀s.} The day is cold. \textbf{2)} be wet (\citealt{Pichl1967}).

\TCsubword{bɔsɔlin} (der.) \textit{v} quench, cool, satisfy, e.g., thirst (\citealt{Pichl1967}) \textit{Hã bɔsɔlin gbɔl lɛ hĩ kul mən dɛ.} To quench our thirst we drink water (\citealt{Pichl1967}).

\TCsubword[1]{bɔsul} (der.) \textit{adj} \textbf{1)} [bɔ̀súl] cold, [hùɛ̀ bɔ̀súl]/[hùɛ̀ thíbɔ̀súl] cold day/cold days (K dialect). \textbf{2)} wet, as soaked cassava (K dialect). \textbf{3)} raw, esp. unsmoked fish (K dialect). der. \TClink{bɔsɔli} (see \TClink[2]{bɔs})

\TCsubword[2]{bɔsul} (der.) \textit{n} latter, cooler part of the day. \textit{Kase lɛ wɔ hun yee palli bɔsul lɛ.} The Kase will come to dance this late afternoon (\citealt{Pichl1967}). 

\TCsubword{bɔsɔli} (der.), (der of \TClink[1]{bɔsul}) \textit{v} make wet, soak (\citealt{Pichl1967}, \citealt{Sumner1921}). 

\TCheadword{bɔsɔli} (der. of \TClink[1]{bɔsul} (der. of \TClink[2]{bɔs}, \TClink{-ul}), \TClink[1]{-i}, see \TClink[2]{bɔs}) 

\TCheadword{bɔsɔlin} (der. of \TClink[2]{bɔs}) 

\TCheadword[1]{bɔsul} (der. of \TClink[2]{bɔs}, \TClink{-ul}, see \TClink[2]{bɔs})

\TCheadword[2]{bɔsul} (der. of \TClink[2]{bɔs}) 

\TCheadword{bɔtakɛl} (comp. of \TClink[1]{baa}) 

\TCheadword[1]{bɔth} \textit{v} clean, e.g., teeth, shoes, etc (\citealt{Pichl1967}, \citealt{Sumner1921}).

\TCheadword[2]{bɔth} \textit{n} (hɔ̃/tha) handle of hoe, knife, etc. (\citealt{Pichl1967}). 

\TCheadword{bɔthaw} \textit{n} (hɔ̃/tha) fist (\citealt{Pichl1967}). 

\TCsubword{bɔthawyay} (unspec.) \textit{n} (kɔ/ma) plant species, shrub with plum-shaped red-orange fruit with velvet-like skin (\citealt{Pichl1967}).

\TCheadword{bɔthawyay} (unspec. of \TClink{bɔthaw}) 

\TCheadword{bɔthbɛrɛ} \textit{cf}: \TClink{kɔŋklɔŋ}. \textit{n} (wɔ/hã, N) millipede species, small kind of millipede (\citealt{Pichl1967}). 

\TCheadword{bɔthɔŋ} (Eng \textit{button}) \textit{n} (hɔ̃/tha) button (\citealt{Pichl1967}). 

\TCheadword{bɔyi} (unspec. of \TClink[2]{bɔi}, \TClink[1]{-i}, see \TClink[2]{bɔi}) 

\TCheadword{Braima} \textit{nam} Brima or Braima, male name given to a person. \textit{Hue bul, Braima wɔe hun tɛnini ha lee abɛna wɔ'ɛ Fuŋk ko.} One day, Braima came to think about leaving his people and Rotifunk.

\TCheadword{brawn} (Eng \textit{brown}) \textit{adj} brown (\citealt{Pichl1967}). 

\TCheadword{brɛdfrut} (Eng \textit{breadfruit}) \textit{n} (kɔ/ma) breadfruit (Artocarpus communis) (\citealt{Pichl1967}). 

\TCheadword{brɛdi} (Eng \textit{bread}) \textit{n} bread. \textit{Ŋ ka mi se̹k brɛdi.} Give me a slice of bread (\citealt{Pichl1967}). 

\TCheadword{brim} \textit{n} (wɔ/hã) fish species, red snapper (Pagrus ehrenbergi and P. pagrus) (\citealt{Pichl1967}).

\TCheadword{bu} \textit{n} (hɔ̃/tha) horn (of animals or musical instruments) (\citealt{Pichl1967}). 

\TCheadword{Bua} \textit{nam} (wɔ/-) name for second initiated Bondo Society girls (also: Bura) (\citealt{Pichl1967}).

\TCheadword{bua} (Mende \textit{bua}) \textit{disco} Mende greeting. \textit{Lɛ nwɔ gbo, ŋa “mɔi," ŋan ŋa wɔ “bua.”} If you say to them, “good afternoon," they will say, “bua" (‘greetings' in Mende).

\TCheadword{buba} (Wolof \textit{mbubu} ‘gown') \textit{n} (hɔ̃/tha) long shirt, caftan (\citealt{Pichl1967}). 

\TCheadword{Bue} \textit{nam} Bue, female name given to a person. \textit{Yema si kump sampa chang awante Bue.} Yemas knows better than her sister Bue how to finish a basket (\citealt{Pichl1967}). 

\TCheadword{bue} \textit{cf}: \TClink{gbusa}, \TClink[2]{kutha}. \textit{v} \textbf{1)} [búé] dig, e.g., well, [búé]/[búé lùɛ̀] dig/dig hole or dig well (K dialect). \textit{Yɛ pɔ bɛ mi ko kaŋdɛ, ika che kɔni ikɔ boi, iŋa mpanthɛ, iyuk yekeɛ.} When I was sent to school, we used to go, we would go dig, we did work and we planted cassava. \textbf{2)} hollow out. \textit{Hã bue thɔk lɛ hã hã sol wɔm.} They hollowed out the tree to make a canoe (\citealt{Pichl1967}). 

\TCheadword{bui} \textit{n} (wɔ/hã, N (?)) bird species, eagle (\citealt{Pichl1967}). 

\TCheadword[1]{buk} \textit{cf}: \TClink{kana}. \textit{n} (kɔ/ma) short rope, both ends of which are fixed to the mast and form a sling onto which the main yard is set (\citealt{Pichl1967}). 

\TCheadword[2]{buk} \textit{cf}: \TClink[2]{di}, \TClink{yams}. \textit{n} (kɔ/ma) yam (Dioscorea spp.) (\citealt{Pichl1967}). 

\TCsubword{baŋkbuk} (comp.) \textit{cf}: \TClink[1]{won}. \textit{n} [bàŋkbúk] plant species, bitter-tasting with edible tubers that grows in the bush (K dialect); (kɔ/ma) plant species, climbing plant, bush yam (Smilax kraussiana) (\citealt{Pichl1967}). \textit{Kúlúnsɛ̀ chɔŋ bàŋkbúkɛ́ lèn.} The goats love \textit{baŋkbuk}.

\TCheadword[3]{buk} (Eng \textit{book}) \textit{n} book (B dialect). \textit{Lomthinɔo, pikchɔthinɔo, lanɛ gbi wɔ tha chi, lipikaɛ pɔ lai ni be ki buk.} Your voice (recordings), your pictures, he will bring all of that, the rest will be put in a book.

\TCheadword[1]{bul} \textit{n} (wɔ/hã) hunchback (\citealt{Pichl1967}). 

\TCheadword[2]{bul} (der. of \TClink[3]{bul}) \textit{temp} \textbf{1)} once. \textit{lɛ nɔsɛ ha ni gbo kɛkɛ nrunth gbo mɔ gbo runth li bul komɔɛ koŋ honi.} If the nurse does not make it fast, you just push, you just push once, and the baby is out. \textbf{2)} one day, suddenly. \textit{Wɔi bul ŋɔ ka gbo sɔtha atak, wɔi hu nak.} It was only one day that he had the attack and became sick. comp. \TClink{nchembul} (see \TClink[1]{che}), \TClink{yombul} (see \TClink[2]{yom}), id. \TClink{lomthibul} (see \TClink[2]{lom})

\TCheadword[3]{bul} \textbf{1)} \textit{Numb} one. \textit{Kɛ ayema mɔ yi yi bul.} But I just want to ask you a question. \textit{Agbem apuma awaŋnimɛntiŋ, bul ko lɔ hok thiyeŋ.} I had seventeen children, the one has gone away. \textbf{2)} \textit{adj} same. \textit{Wɔn pɛ mpanth bul lɛ ma bo wɔɛ wɔ ra.} She also does the same thing farming. \textit{Ŋan gbi ko ya bullɛ?} Are they all from the same mother? \textbf{3)} \textit{adj} entire, complete. \textit{So la minɛ skul buli ŋɔɛ?} So it means Bondo is a whole school? \textit{Sɔlɛma bulli ŋɔ vɛ.} That is a complete hassle. \textbf{4)} \textit{adj} unified. comp. \TClink{mɛnbul} (see \TClink[1]{mɛn}), \TClink{waŋnibul} (see \TClink[2]{waŋ}) 

\TCsubword{bulbul} (comp.) \textit{cf}: \TClink{buleŋ-buleŋ} (der. of \TClink{buleŋ}). \textit{quant.} each, one-by-one, one after the other (\citealt{Pichl1967}). \textit{Hã ka hã ndel bul bul bul bul.} They gave (each of them/one after the other) a name (\citealt{Pichl1967}). \textit{Ni apimaɛ ha koi mbaŋgɛ bul-bul ni ha kɔ tri thɛai bul-bul.} Then his children had to take each of the ropes and go to each village (\citealt{Sumner1921}). 

\TCsubword{bulnɔbul} (comp.) \textbf{1)} \textit{adv} every person separately, one after the other (\citealt{Pichl1967}). \textit{Pə ve̹lɛ bul nɔ bul.} They called one after the other (\citealt{Pichl1967}). \textbf{2)} \textit{quant} each. 

\TCsubword{yeŋbul} (comp.) \textit{cf}: \TClink{yombul} (comp. of \TClink[2]{yom}, \TClink[2]{bul}). \textit{n} same thing. \textit{Ihun ni ko ja mbɛɛnɛ, yɛ pɔ kache gbisiŋdɛ ni boŋgo labo ŋɔ yeŋbul.} Let us now come to those days' affairs, the way they used to marry, if it is the same thing as now.

\TCheadword{buleŋ} \textit{v} be different (\citealt{Sumner1921}). 

\TCsubword{buleŋ-buleŋ} (der.) \textit{cf}: \TClink{abulabul} (der. of \TClink{a-}, \TClink[3]{bul}), \TClink{bulnɔbul} (comp. of \TClink[3]{bul}, \TClink{nɔ}). \textit{adj} different, various, [yenchɛk abuleŋ-buleŋ] different kinds of fish (\citealt{Pichl1967}). 

\TCheadword{buleŋni} (der. of \TClink{buleŋ}, \TClink[1]{ni}, see \TClink[1]{ni}) 

\TCheadword{bulkɔ} \textit{n} plant species, benni leaf (Sesamum radiatun) (\citealt{Pichl1967}). 

\TCheadword{bulnɔbul} (comp. of \TClink[3]{bul}, \TClink{nɔ}, see \TClink[3]{bul})

\TCheadword[1]{bulɔ} \textit{cf}: \TClink[1]{ja}, \TClink[1]{panth}. \textit{n} work. \textit{Bulɔ kendɛ handɔ?} What kind of work? comp. \TClink{nɔbulɔ} (see \TClink{nɔ})

\TCheadword[2]{bulɔ} \textit{cf}: \TClink{haa}, \TClink[2]{kɔ}. \textit{v} attend. \textit{Ka lɔ nkache bulɔɛ?} You used to go to school here?

\TCheadword{bum} \textit{Idph} of something falling down (K dialect).

\TCheadword{buma} (Port \textit{verruma} ‘gimlet') \textit{n} (hɔ̃/tha) gimlet (\citealt{Pichl1967}). 

\TCheadword{bundɛ} (der. of \TClink{buŋ}) 

\TCheadword{Bundu} \textit{nam} Bundu, name given to a person, surname. \textit{Ya mi wɔ lɔ Salematu Bundu.} My mother is Salaymatu Bundu.

\TCheadword{buŋ} \textit{v} \textbf{1)} flog, beat. \textit{Hã buŋ wɔ ka thɔk.} They flogged him with a stick (\citealt{Pichl1967}). \textit{Ya ki hundɛ, pɔ mi buŋ.} When I come back, they (will) flog me. \textit{Yɛ nka che ko tallɛ, pɔ ka che mɔ buŋ?} When you were young, did they used to beat you? \textbf{2)} thresh. \textit{Pɔ koŋ gbo, pɔi chi nteɛ ŋa hun buŋdɛ, pɔ buŋ.} After they have finished, they bring the mortars, they thresh the rice. \textit{Pɔ koŋ gbo buŋ, pɔi huŋ chakath.} After they have finished threshing, they come to remove the stalks. \textbf{3)} win a game. comp. \TClink{buŋsua} (see \TClink{sua}) 

\TCsubword{bundɛ} (der.) \textit{v} be beaten. \textit{La nka che ŋa labi pɔ ka che mɔ bundɛa?} What did you do that you were beaten?

\TCheadword{buŋklipal} (unspec. of \TClink[1]{li-}, \TClink[1]{pal}, see \TClink[1]{li-}) 

\TCheadword{buŋkluŋ} \textit{n} (hɔ̃/tha) wave, surf (\citealt{Pichl1967}). Mbunkluŋ dɛ ma bo̹m. The waves are high (\citealt{Pichl1967}). \textit{Sakïl bunkluŋ dɛ atok.} He swam on the waves (\citealt{Pichl1967}). 

\TCheadword{buŋsua} (comp. of \TClink{buŋ}, \TClink{sua}, see \TClink{sua}) 

\TCheadword{Burɛ} \textit{nam} Bureh, male name given to a person. \textit{Burɛ, yɛ bi hã boa ki-a, kɔ ma hã bɔnthɔ mi mputhun.} Bureh, why are you so early? You have taken me unawares (\citealt{Pichl1967}). 

\TCheadword[1]{burɔ} \textit{n} (kɔ/ma) plant species, shrub or small tree (Smeathmannia laevigata and Ouratea vogelii) (\citealt{Pichl1967}). 

\TCsubword{burɔdinthɛ} (comp.) \textit{n} (kɔ/ma) tree species, white mangrove (Avicennia nitida) (\citealt{Pichl1967}). 

\TCheadword[2]{burɔ} [bùrɔ́] \textit{n} bird species, medium-sized, brown and white with speckled chest (K dialect). 

\TCheadword{burɔdinthɛ} (comp. of \TClink[1]{burɔ}, \TClink{dinthɛ} (der. of \TClink{dinth}, \TClink{-ɛ}), see \TClink[1]{burɔ}) 

\TCheadword{bus} \textit{cf}: \TClink{jal}, \TClink[4]{kɔ}, \TClink[2]{sɔŋ}. \textit{v} skin. \textit{Ŋkɔ bus vis lɛ.} Go skin the animal! (\citealt{Pichl1967}). 

\TCsubword{busni} (der.) \textit{v} \textbf{1)} skin oneself (shed like a snake), undress (\citealt{Pichl1967}). \textbf{2)} break out (war) (\citealt{Pichl1967}). \textit{Pəmdɛ kɔ busni Mpelɛ ko.} War has broken out at Mpele (\citealt{Pichl1967}). 

\TCsubword{hosni} (der.) \textit{v} \textbf{1)} skin oneself. \textbf{2)} shed skin. \textit{Kər lɛ koŋ hõsni.} The snake has shed its skin (\citealt{Pichl1967}). 

\TCheadword{bushɛl} (Eng \textit{bushel}) \textit{n} bushel. \textit{Pɔ koŋ gbo kutha, pɔi chi pɛlɛ ken bushɛl libul ɔ litiŋ ɔ limɛn bɛ ɔ waŋ bɛ.} After the plowing, they would have to bring the rice like one or two bushels, or five, or even ten. \textit{Ŋka mi pəlɛ bushɛl ibul.} Give me one bushel of rice (\citealt{Pichl1967}). 

\TCheadword{busni} (der. of \TClink{bus}, \TClink{-ni}, see \TClink{bus}) 

\TCheadword{buth} \textit{n} (hɔ̃/tha) anus (\citealt{Pichl1967}).

\TCheadword{buthba} \textit{n} (hɔ̃/-) rice variety, dark kind of rice (\citealt{Pichl1967}). \textit{Pəl lɛ ko̹y chaŋ buthba lɛ.} The reddish rice increases more than the dark one (\citealt{Pichl1967}). 

\end{letter}

\begin{letter}{Ch}

\TCheadword[1]{cha} \textit{n} [chà] feather, [chà]/[chàthɛ́] feather/the feathers (B dialect); \textit{chææ} (hɔ̃/tha) feather (\citealt{Pichl1967}). \textit{Ŋ kɔ suth chæthi sɔk lɛ!} Go pluck the fowl! (\citealt{Pichl1967}). 

\TCsubword{chaasɔk} (comp.) \textit{n} [chààsɔ̀k] fowl feather (B dialect). 

\TCheadword[2]{cha} \textit{cf}: \TClink[3]{che}. \textit{Aux} auxiliary ‘have.' \textit{Oo, Bahin, la hi cha ko haee?} Oh, our Father, what have we done? \textit{La i cha ba ha ba?} What have we done? \textit{Labi hi mɔ yiɛ la hi cha ko ha.} That is why we are asking you what have we done. \textit{Pɔi wɔ yɛ nɔɔ ki wɔ bɔ cha chaŋchaŋ doa.} Then they would begin to say how is this person roaming about this way. \textit{Laa mi, si ŋcha thol hiŋk ka ni ŋkɔ chii yeke hiŋk ŋken dɛ ma luɛ vɛ…} My wife, if you descend from here and bring back cassava from those sharp knives… \textit{Ha kafaiyɛ, ŋɔ icha ba bɛŋsin kia.} It is for our wickedness that we are perishing.

\TCheadword{chaasɔk} (comp. of \TClink[1]{cha})

\TCheadword[1]{chai} \textit{cf}: \TClink[2]{tɔn}. \textit{v} \textbf{1)} raise a song, sing. \textit{Ni nɔmaa bul ŋan thiyeŋ wɔe chaɛ tɔn tho ki.} Then a woman among them raised this song. \textit{Nɔmaa chaɛ a: “Ya gbo woki-o-o.”} The woman sings: “I am just wondering.” \textit{Nɔmaa chaɛ a: “Yemi, ni ntɛniɛ mini o-o-o.”} The woman sang: “My lady, remember me.” \textbf{2)} say. \textit{A biɛn chæ pɔsɔ hã hɔ 'mbolon dɛ.} I have not much to say on this matter (\citealt{Pichl1967}). \textbf{3)} lift. \textit{Hã kɔ chæ thɔk lɛ kɔ bikɛɛ lɛ duki chɔl na næ lɛ 'hɔl lɛ.} Go and lift the tree that the storm felled on the road las night (\citealt{Pichl1967}). \textit{Nɛɛ gbo pulaɛ, wɔ chaɛ bol wɔɛ.} If you step on the worm, it will lift up its head (proverb) (\citealt{TISLL1979}). \textbf{4)} lend, borrow. \textit{A chæ fe.} I lent money (\citealt{Pichl1967}). \textit{Ye sɔlɛmaɛ yɛ mɔ chai iroɛ, mbɔni ha paka ŋɔ.} What a hassle (it is) when you borrow something and you cannot pay it back.

\TCsubword{chaini} (der.) \textit{v} raise oneself up. \textit{Hɔ chaini fli ŋɛ chanthɛ.} It (afterbirth) rises up again like a baby.

\TCheadword[2]{chai} \textit{adj} brackish (\citealt{Pichl1967}). 

\TCheadword{chaini} (der. of \TClink[1]{chai}, \TClink{-ni}, see \TClink[1]{chai}) 

\TCheadword[1]{chak} \textit{v} drop, drip, leak, e.g., rice from a leaky barn (\citealt{Pichl1967}). 

\TCheadword[2]{chak} \textit{cf}: \TClink[1]{wal}. \textit{n} \textit{ichak} (hɔ̃/-) piassava, stout fiber obtained from the leaf stalks of palm trees (\citealt{Pichl1967}). 

\TCheadword{chakabulla} \textit{cf}: \TClink{peŋka}. \textit{n} single barrel gun type (B dialect). 

\TCheadword{chakath} \textit{v} remove stalks, e.g., from rice plants (B dialect). \textit{Pɔ koŋ gbo buŋ, pɔi huŋ chakath.} After they have threshed, they come to remove the stalks. \textit{Pɔ koŋ gbo chakath yeŋkɛlɛŋ, pɔi chi bɛkthɛ.} They remove the stalks from the rice completely, then they bring the bags.

\TCheadword{chaktha} \textit{n} insect species, butterfly (K dialect); (wɔ/hã, N) butterfly, dragonfly or similar insect (\citealt{Pichl1967}). 

\TCheadword[1]{chal} \textit{cf}: \TClink{gbɛma}, \TClink{re}. \textit{n} [chàl] deer (K dialect); (wɔ/hã, si) any kind of larger antelope (\citealt{Pichl1967}). \textit{Bia wɔ lɛ poinɔ di chal, hã kɔ wɔ sɔŋ.} Bia is a hunter, he has killed an antelope, (you pl.) go cut it up (\citealt{Pichl1967}). \textit{Pui-nɔ lɛ chala tho l'ay wɔ mïrə chal lɛ.} The hunter sits in the bush and watches the deer (\citealt{Pichl1967}). comp. \TClink{pɛlchal} (see \TClink[2]{pɛl}) 

\TCheadword[2]{chal} \textit{cf}: \TClink[1]{chɛli}. \textit{v} \textbf{1)} sit. \textit{Cheni chali hɔ̃ ki? Nlɛli bɔŋ dɛ, nchal!} Isn't there a seat here? Look at the chair, sit down! (\citealt{Pichl1967}). \textit{Bahin chala bɛ liwai igbo bɛŋ sin o.} Our father sits on his throne and we are suffering here. \textbf{2)} reside, live. \textit{Ŋa ni lamgbantho ki ŋa chalao wɛ, ŋaŋa gbem apumma mɛn do wɛ?} You (pl) and this man you're living with, are you the ones that gave birth to (are you the parents of) these five (children)? \textit{Lɔn lɔ chala pɛ?} Is he also staying there? \textit{Haaŋ mɛŋkɛ ŋɔ Apotho aɛ ka hun dɔ chal ha pin awok aɛ...} Until the white man came there and settled to buy enslaved people... \textit{Na chala ŋa?} Do they live here?

\TCheadword[3]{chal} \textit{cf}: \TClink[3]{bɛŋ}, \TClink[1]{chɛlɛk}, \TClink{chɛm}, \TClink{gbakra}. \textit{n} seat. \textit{Cheni chali hɔ̃ ki? Nlɛli bɔŋ dɛ, nchal!} Isn't there a seat here? Look at the chair, sit down! (\citealt{Pichl1967}). 

\TCsubword{chala} (unspec.) \textit{n} (hɔ̃/tha) very fine mat used to cover chairs (\citealt{Pichl1967}). 

\TCsubword[2]{chɛlɛk} (unspec.) \textit{n} (hɔ̃/ma) seat (\citealt{Pichl1967}). 

\TCheadword{chala} (unspec. of \TClink[3]{chal}) 

\TCheadword{chalalɛ} \textit{n} (kɔ/ma) plant species, mistletoe and similar tree parasites (\citealt{Pichl1967}). 

\TCheadword{cham} \textit{v} make known publicly, announce (\citealt{Pichl1967}).

\TCheadword{chamak} \textit{cf}: \TClink[1]{jo}, \TClink{sɔm}. \textit{v} chew many things rapidly (vs. \textit{sɔm}) (B dialect). 

\TCsubword{chamakin} (der.) \textit{v} chew (\citealt{Sumner1921}). 

\TCheadword{chamakin} (der. of \TClink{chamak}, \TClink[2]{-n}, see \TClink{chamak}) 

\TCheadword{chamne} \textit{cf:} \TClink{cholnɔ}. \textit{n} (wɔ/hã, N) carpenter (\citealt{Pichl1967}). 

\TCheadword{chanth} \textit{n} baby, child. \textit{Hɔ chaini fli ŋɛ chanthɛ.} It (afterbirth) rises up again like a baby. id. \TClink{pɛŋchanth} (see \TClink[2]{pɛŋ}) 

\TCheadword[1]{chaŋ} \textit{v} \textbf{1)} surpass. \textit{Ko gbemiɛ gbi ŋɔ nko gbemiɛ handɔ ŋɔ chaŋ mɔ che fɔi?} In all the deliveries you have delivered which one was the easiest? \textit{Ahina ŋa chan shi theli Mbolomdɛ Shenge ka.} Who (pl) knows how to speak Sherbro best in Shenge here? \textit{Planti ka, mpanth handɔ, ma ayindɛ ŋaa ma chaŋ, ŋa la chaŋ mpanth-o-mpanth a?} In Plantain here what work do people do more, what is the most common job? \textit{Wɔn wɛ kɔysunɔ lɛ chaŋ atɛma wɔ lɛ.} He himself was the greatest sorcerer among his peers (\citealt{Pichl1967}). \textbf{2)} come to pass, happen, transpire. \textit{Yɛ lanɔ ki la koŋ chaŋ dɛ, abɛɛ-aɛ ni ŋgbakoɛ ŋae vel Kaiŋ Taso ha thoŋka wɔ.} After this happened, the chiefs and the elders then called Kain Tasso to judge him. \textbf{3)} pass. \textit{Ha piɛɛ koŋ nyaɛ, labi wɔ chaŋ lan puthulɛ atokɛ?} The elephant has become thin, therefore, he should pass over a rotten bridge? (proverb) (\citealt{TISLL1979}). \textbf{4)} be better. \textit{Imɔl hɔ chaŋ taŋ gber.} Sorrow is better than a lot of crying (proverb) (\citealt{TISLL1979}). comp. \TClink[1]{pɔŋchaŋchaŋ} (see \TClink{Pɔ}), \TClink[2]{pɔŋchaŋchaŋ} (see \TClink{Pɔ}), der. \TClink{nɔchancha} (see \TClink{nɔ})

\TCsubword[1]{chaŋchaŋ} (der.) \textit{v} \textbf{1)} travel around, roam. \textit{Nsiɛ tɛm pɛm doki yɛi chaŋ-chaŋdɛ raiyɛ ŋɔ koŋ tuk.} You know during the war how we were moving around, the document has disappeared. \textit{Tɔŋ wɔ po̹l, wɔ gbo chaŋ-chaŋ pɔksi lɛ ay.} Tong is foolish, he goes abotu from one place to another (\citealt{Pichl1967}). \textit{Wɔ gbo chaŋchaŋ polo̹ŋ sin la wɔ hã lɛ.} He only goes about from place to place and does not know what to do (\citealt{Pichl1967}). \textbf{2)} surpass. \textit{Wɛl anya landɛ ŋa chaŋchaŋ cheɛ.} Well, those are the people that are greater in number. der. \TClink{nɔchancha} (see \TClink{nɔ})

\TCheadword[2]{chaŋ} \textit{cf}: \TClink[2]{kɔth}. \textit{n} (kɔ/ma) tooth, especially incisors (\citealt{Pichl1967}, \citealt{Sumner1921}). \textit{Ja la gbɔw mi, nchaŋ ma mɔ lɛ ma gbɔw igɛth.} This is too hard for me, your teeth are too dirty (\citealt{Pichl1967}). 

\TCheadword[3]{chaŋ} \textit{quant} more; more than. \textit{Thɛngbɛŋ ve̹lni thɔnkaŋ kə wɔ ton chaŋ thɔnkaŋ.} The thɛngbɛŋ resembles the thɔnkaŋ but it is smaller than the thɔnkaŋ (\citealt{Pichl1967}). \textit{I ko vei ina pomdɛ o, iko bɛ chaŋ nɛnthi waŋdɛ.} We have stayed together me and my husband, now more than ten years. \textit{Way thugba kɔ bo̹m chaŋ way pe̹nka.} The cannonball is bigger than the bullet of a gun (\citealt{Pichl1967}). \textit{Yeethi lo tha hiniɛ-m gbɔl chaŋ thanɛ chencha.} This dance delights me more than that of yesterday (\citealt{Pichl1967}). 

\TCheadword[4]{chaŋ} \textit{subordconn} \textbf{1)} except. \textit{Chaŋ gbo mbithi mbullɛ malɔ leɛ, ma lɔ le sɛmdɛ.} Except the short standing sticks that stand there would remain standing there. \textbf{2)} unless.

\TCsubword{chaŋbo} (unspec.) \textit{subordconn} \textbf{1)} unless. \textit{Chaŋbo athɔni ka Min Charaŋ dɛ we...} Unless I cleanse myself with the Holy Spirit... \textbf{2)} except. \textit{Wɛl chaŋbo yɛ ika che kɔni siŋ boniɛ ni athɔmamdɛ paŋdɛ.} Well except when we used to go play the hide-and-seek game with my mates in the evening. \textit{Chaŋbo paŋdɛ ŋɔ mɔi bo pɔ hiŋ ka ja tuthɛ, than bo tha ika che kunɛ.} Except when evening came, we would be given rice pounding work, that was the work we were engaged in.

\TCheadword[5]{chaŋ} \textit{coordconn} until. \textit{Tɛmpim la koi ndɔi ntiŋ pɔ che wɔ kɔŋ, chaŋ pɔ koŋla.} Sometimes it would take two days without being buried, until the process is done.

\TCheadword{chaŋbo} (unspec. of \TClink[4]{chaŋ}) 

\TCheadword[1]{chaŋchaŋ} (der. of \TClink[1]{chaŋ}) 

\TCheadword[2]{chaŋchaŋ} \textit{adv} very well. \textit{Chaŋchaŋ, wɔ lɔ, wɔ lɔ Sotahuŋ.} Very well, she is there, she is Sotahun.

\TCheadword{chaŋchao} [chàŋchàò] \textit{n} bird species, very scarce now, not quite so big as a hen, a bit bigger than bush fowl, but not a bush hen, very tasty (K dialect). 

\TCheadword{chaŋgbaŋ} \textit{n} (hɔ̃/-) small rocky headland with small trees (\citealt{Pichl1967}). 

\TCheadword{chaŋha} \textit{quant} too much.

\TCsubword{lichaŋha} (der.) \textit{adv} too much (\citealt{Sumner1921}).

\TCheadword[1]{charaŋ} \textit{cf}: \TClink{kilia}. \textit{adv} \textbf{1)} cleanly. \textit{Mbas kil lɛ charaŋ!} Sweep the house clean! (\citealt{Pichl1967}). Ŋ kɔ sankath bo̹y lɔ, hɔ̃ chen charaŋ. Go rinse the plate there, it is not clean (\citealt{Pichl1967}). textit{Mbolom dɛ ma wɔni kilia ni charaŋ.} The Sherbro language is being spoken clearly and cleanly. \textbf{2)} nicely. \textit{Wɔ mɔ sɔnthɔ charaŋ.} He would sew it for you nicely. \textbf{3)} very well. \textit{Charaŋ.} Very well. comp. \TClink{checharaŋ} (see \TClink[6]{che}), der. \TClink[2]{licharaŋ} (see \TClink[1]{charaŋ}) 

\TCsubword[1]{charaŋcharaŋ} (der.) \textit{adv} well. \textit{Apa, wɔkɛ handɔ kɔ mɔm mɔ theli charaŋcharaŋ ŋa?} Pa, which language do you speak well?

\TCsubword[1]{licharaŋ} (der.) \textit{n} (lɔ/-) cleanliness (\citealt{Pichl1967}).

\TCsubword[2]{licharaŋ} (der.), (der. of \TClink[1]{licharaŋ}) \textit{adj} clean (\citealt{Sumner1921}). 

\TCheadword[2]{charaŋ} \textit{adj} \textbf{1)} moral. \textit{Mɔ ha dum wɔ ni wɔ hani charaŋ.} You should train him to be moral. \textit{Kɛ nkowɔ gbo dum wɔ charaŋcharaŋ, wɔnɛ bɛ wɔ kɔ hundɛ wɔ che charaŋ.} But if you have trained him to be moral, the other ones who follow will also be moral. \textbf{2)} pure, holy. \textit{La li kɛlɛŋ hi lemil Wɔ ŋa che charaŋ kendɛ Wɔn.} It is nice for us to follow Him and be pure like Him. \textit{Min Charaŋ} Holy Ghost (\citealt{Pichl1967}). 

\TCsubword[2]{charaŋcharaŋ} (der.) \textit{adj} moral. \textit{Kɛ nkowɔ gbo dum wɔ charaŋcharaŋ, wɔnɛ bɛ wɔ kɔ hundɛ wɔ che charaŋ.} But if you have trained him to be moral, the other ones who follow will also be moral.

\TCheadword[1]{charaŋcharaŋ} (der. of \TClink[1]{charaŋ}) 

\TCheadword[2]{charaŋcharaŋ} (der. of \TClink[2]{charaŋ}) 

\TCheadword{chasa} \textit{cf}: \TClink[2]{bos} \textit{n} (hɔ̃/tha) bottle gourd (Lagenaria siceraria), calabash rattle (shake-shake) (\citealt{Pichl1967}).

\TCheadword{chayoŋ} \textit{n} insect species, cricket (K dialect). 

\TCheadword[1]{che} \textit{cop} be. \textit{Kɛ tɛmdɛ vɛ aka che ton.} But at that time I was small. \textit{Gbi hɔ ka che kitɛ kunɛ.} Everything used to be in the kit. \textit{Mbo̹lo̹m ŋwɛi ma che paalɛ bai ko, anya atïŋ dɛ hã lo̹l.} In the bad case that was recently before the court, the two men were set free (\citealt{Pichl1967}).

\TCsubword[2]{che} (der.) \textit{n} \textbf{1)} being, state, condition, habit, way of life. \textit{Bia wɔ nche wɛy, wɔ woŋ lol thiwɛy ko ama wɔ lɛ.} Bia has bad habits, he curses his wives with bad words (\citealt{Pichl1967}). \textit{Gbe̹mni abəka lɛ ni nche ma hã lɛ ma fəsɛ hã ma apotoa.} The inheritance and the way of the life of the Krios resemble those of Europeans (\citealt{Pichl1967}). \textbf{2)} future. \textit{Nɔ shini che ko labi yendɛ yɛ mɔ la ŋa ncheyi ni nshila thiyen, ni la saŋ mɔ ntenɛ.} One does not know the future that is why when doing something you should ask so you can know it and understand it better. \textit{Ndɔ-lɔ che-ko bi hã bɔnth ni-a?} Where will the future find me? (\citealt{Pichl1967})comp. \TClink{nɔncheŋwɛi} (see \TClink{nɔ})

\TCsubword[1]{chelɛɛ} (der.) \textit{v} be present. \textit{Baybul lɛ hɔ lɛ Sent Pawl ka che-lɛɛ ni ke ka thihɔl yɛ pə ka vɛɛy Sent Stivin.} The Bible says that St. Paul was present and saw with his (own) eyes when they stoned St Stephen (\citealt{Pichl1967}).

\TCsubword{nchembul} (comp.) \textit{n} (ma) harmony (one existence) (\citealt{Pichl1967}). 

\TCheadword[3]{che} \textit{cf}: \TClink[2]{bi}, \TClink[2]{cha}, \TClink[2]{ki}, \TClink[7]{kɔ}. \textit{Aux} \textbf{1)} present- and past-progressive auxiliary verb. \textit{Chala bo che ŋa beyen.} She is just sitting down doing nothing. \textit{Mɔm, la nka cheni ŋaa?} You, what have you been doing? \textit{Ya chen kɔ ayen gbi.} I'm not going anywhere. \textbf{2)} future auxiliary verb; ‘will.' \textit{Ache lɔŋ kɔ gbi, ya lɔ kɔɛ a ke nɔɛ yɛ sɛmɛ kilɛ koɛ.} I will not go there at all, when I go I see the person standing in the room. \textit{Yaŋ pɛ ani bia che yaa ni.} Me, I will be cooking for myself. \textit{A che thaŋ pɛ bɔ si, bikɔs tha thi gbe.} I will not be able to know them again, because they are so many. \textbf{3)} Aux.Neg. \textit{Tem lan ma che na pɛ shimi Plantiɛ.} At that time it would not have destroyed Plantain (Island) again. \textit{Pɛlɛ kɔ che yeɡbe ka hi fi, ken bɛl pothoɛ ki...} Rice does not grow well in our hands, like coconut ... comp. \TClink{chenthehwɛi} (see \TClink{the}) 

\TCheadword[4]{che} \textit{cf}: \TClink{chɛ}. \textit{post} in front of, before. \textit{Kɛ mɛŋk ki, mamani gbi, haliwɔ wɔ bɛɛ lɛ chee.} But this time, he did not laugh at all, because he was in front of the chiefs. \textit{Atiŋ ŋa koŋ kɔni cheko, iara iwɔlɔ ka.} Two of them have gone before, we are now three in this world. comp. \TClink{kache} (see \TClink[5]{ka}) 

\TCheadword[5]{che} \textit{adv} further. \textit{Nche gbo lem thelian mbol, kɔ chen kɔ che, nɔ wɔ ŋa kek thi wɔl.} You should not just lie, it would not go further, one should see with his eyes. \textit{Ya koŋ standad siks ɛ, bami ni yami ŋa ka biɛni fɔsaɛ ŋa kɔ che, yai kɔni Champ ko.} After I finished standard six, my father and my mother did not have the resources for me to go further, so I went to Freetown.

\TCheadword[6]{che} \textit{cf}: \TClink[3]{hɔ}, \TClink[2]{la}, \TClink[2]{lɛ}, \TClink[4]{ni}, \TClink[3]{ŋa}, \TClink[1]{yɛ}. \textit{subordconn} that. \textit{Gɔmɛnt lɛ hã thonkiɛ lɛ hã yema hã saba, che lɛ tamɔ pokan gbi wɔ koŋ huth lɛ, wɔ hã paka pɔn bul hã bo̹l wɔ lɛ.} The government has proclaimed that they want to make a law that every young man who has come of age has to pay one pound as a head-tax (\citealt{Pichl1967}). 

\TCsubword[2]{chelɛ} (der.) \textit{subordconn} so that, in order that. \textit{Bami nhã ya che-lɛ tamɔ.} Lord make that I become your child (\citealt{Pichl1967}). \textit{Ba Thəngbəŋ lee mathui bach lɛ ve̹le̹ŋ che-lɛ mɔ hunki gbo...} Mr Bat remained hidden behind a young palm tree so that if somebody came there... \textit{Chii chelɛ ya hun sɔthɔ yen ha sɔm, ndikɛ koŋ mi gbɔɔ!} Bring it so that I can come and eat something, hunger is consuming me!

\TCheadword[7]{che} \textit{v} \textbf{1)} behave. \textit{So wɔkɛ kɔ mɔ wom ko ŋanɛ ja Bondoɛ, ŋɔ ŋa bia cheɛ, ndumdɛ.} So the word you send to them about Bondo, how they should behave, (about) character. \textbf{2)} become. \textit{Kenɛki ikoni shiɛ lɛ mɔ lɔ Spikaɛ, nɛn ndɔ ŋɔ nche Spika?} Now we know that you are Speaker, what year did you become Speaker? \textbf{3)} engage. \textit{Chaŋbo paŋdɛ ŋɔ mɔi bo pɔ hiŋ ka ja tuthɛ, than bo tha ika che kunɛ.} Except when evening came, we would be given rice pounding work, that was the work we were engaged in.

\TCsubword{checharaŋ} (comp.) \textit{n} cleanliness. \textit{Checharaŋ lɛ fɛsɛ ncho ma hɔbatokɛ.} Cleanliness is next to godliness (\citealt{Pichl1967}). 

\TCheadword{cheara} (Port \textit{tesoura} ‘scissors') \textit{cf}: \TClink{sizɔs}. \textit{n} (hɔ̃/tha) scissors (\citealt{Pichl1967}). 

\TCheadword{checharaŋ} (comp. of \TClink[7]{che}, \TClink[1]{charaŋ}, see \TClink[7]{che}) 

\TCheadword{cheche} \textit{n} \textbf{1)} (kɔ/tha) light, lamp (\citealt{Pichl1967}). \textit{Tipïk lɛ hɔbatokɛ bɛmpa cheche lɛ, nyənkin dɛ hɔ̃ ka bɛmpa nɔthi.} In the beginning God made the light, finally he made man (\citealt{Pichl1967}) \textbf{2)} brightness, the name of one of Tom Caulker's daughters. \textbf{3)} righteousness. \textit{Bahin, a bi ŋa che gbɛŋ mɔɛ ni cheche mɔ kunɛ we.} Lord, I should be your faithful and your righteous. \textbf{4)} shelter. \textit{Cheche yɛ rithi yai.} Shelter in the storm.

\TCheadword{chechem} \textit{n} (kɔ/ma) tree species, small tree that grows in clusters on the beach (\citealt{Pichl1967}). 

\TCheadword[1]{chelɛ} (der. of \TClink[1]{che}, \TClink[2]{lɛ}, see \TClink[1]{che}) 

\TCheadword[2]{chelɛ} (der. of \TClink[6]{che}, \TClink[1]{lɔ}, see \TClink[6]{che}) 

\TCheadword{chencha} \textit{cf}: \TClink{gbɛŋ}, \TClink{jɛk}, \TClink{nante}, \TClink{sɛmplɛŋ}. \textit{temp} yesterday. \textit{Pə kɔnthi chencha Sese wɔ lɔ yɔlko l'ay gbunda la ke Kaay lɛ.} They caught Sese yesterday, he is in chains (because) he raped Kayn's wife (\citealt{Pichl1967}). \textit{Chencha bɛ ya kɔɛ akɔni poi.} Even yesterday when I went, I didn't go early. \textit{Braima wɔe boa ha kɔ lɛɛli mpɛl lɛ ma kɔ chɛncha lɔɔli huɛ lanthgbɔl lɛ.} Braima goes out early to inspect the nets which he went to check yesterday.

\TCheadword[1]{cheŋk} \textit{cf}: \TClink[2]{cheŋk}. \textit{v} hate. \textit{Chènkɛ́ mì.} He hates me. \textit{Nɔ̀ mí chéŋk wɔ che paaɛ, kɛ yɛ laio, chɔ́ŋ mì lèn.} He hated me some time ago, but now he likes me. comp., der. \TClink{nɔncheŋk} (see \TClink{nɔ})

\TCheadword[2]{cheŋk} \textit{cf}: \TClink[1]{cheŋk}. \textit{n} \textbf{1)} (ma) hatred, enmity (\citealt{Pichl1967}). \textbf{2)} enemy. \textit{Nɔ mi chenk.} My enemy (\citealt{Pichl1967}). \textit{Gbana wɔ wɛi, koŋ hĩ wɔ̃hul ko anya hĩ nche̹nk lɛ.} Gbana is bad, he has betrayed us to our enemies (\citealt{Pichl1967}). 

\TCheadword[3]{cheŋk} \textit{v} carry away. \textit{Tɛm hɔ̃ gbo ke̹n mɛn nsoso lɛ hɔ̃ chenk anyathi gbi.} Time is like running water, it carries people away (\citealt{Pichl1967}). \textit{N chenkə hɔ̃ paraat!} Run away quickly! (\citealt{Pichl1967}). 

\TCheadword{chenth} \textit{n} bird species, grayish pigeon (K dialect); (hɔ̃/ha, N) bird species, green fruit pigeon (\citealt{Pichl1967}). 

\TCheadword{chenthehwɛi} (comp. of \TClink[2]{che}, \TClink{theɛhwɛ} (comp. of \TClink{the}, \TClink[1]{wɛi}), see \TClink{the}) 

\TCheadword{chentheŋgi} \textit{v} forge. \textit{Koŋ ma gbo chentheŋgi si yi ma-e boŋhũl.} When he has finished forging them, we sharpen them (\citealt{Pichl1967})

\TCheadword{chenthni} \textit{v} tiptoe (\citealt{Pichl1967}). 

\TCheadword{cheŋwɛ} \textit{n} [ŋ́chèŋwɛ́] rudeness (K-dialect).

\TCheadword{chereŋ} \textit{v} free. \textit{Pɔ wɔe kue ŋgbekteɛ ŋkɛnt, ni pɔ chereŋ Kaiŋ Taso.} They took the handcuffs off his hands and they freed Kain Tasso.

\TCsubword{chereŋni} (der.) \textit{n} (?/?) freedom, place of refuge (when children play hide-and-seek, the seeker, when touching one of the others says \textit{chereŋni} ‘safe' or ‘free,' i.e., now I am free while you have to catch me) (\citealt{Pichl1967}). 

\TCheadword{chereŋni} (der. of \TClink{chereŋ}, \TClink{-ni}, see \TClink{chereŋ}) 

\TCheadword{chɛ} \textit{cf}: \TClink[4]{che}. \textit{Loc} front.

\TCheadword{chɛɛki} \textit{v} throw down successively, e.g., palm kernels from a tree (\citealt{Pichl1967}). \textit{Ŋkɔ-m chɛɛkiɛ.} Go drop the palm kernels down for me (\citealt{Pichl1967}). 

\TCheadword{chɛk} \textit{n} (hɔ̃/ma) farm, \textit{chɛk/ichɛk } farm (\citealt{Pichl1967}). \textit{Mpanth ma ichɛk ma ɛ mpanth ŋkəlɛŋ.} Farm work is fine work (\citealt{Pichl1967}). \textit{Yɛŋ ni yɛŋ ŋɔ mɔ bɛ ichɛkɛ vɛ kunɛa?} What and what do you plant on your farm? \textit{A chɔŋ mpanth ma chɛk len kə ma katho.} I like farm work, but it is hard (\citealt{Pichl1967}). comp. \TClink{nɔhinyɛchɛk} (see \TClink{nɔ}), \TClink{nɔrachɛk} (see \TClink{nɔ}) 

\TCheadword{chɛkɛm} \textit{n} chin, [chɛkɛmdɛ] the chin (K dialect); \textit{chɛkəm} (-/tha) chin (\citealt{Pichl1967}).

\TCheadword{chɛklin} \textit{n} pride (\citealt{Pichl1967}). \textit{Tamɔ ichɛklïn, wɔ ki wɔ nsɔlɔk.} This is a proud boy, he is insolent (\citealt{Pichl1967}). 

\TCheadword[1]{chɛlɛk} \textit{cf}: \TClink[3]{chal}, \TClink{chɛm}. \textit{n} times (multiplication), \textit{chɛlɛk li/chɛlɛk li hyɔl } multiplied/four times (\citealt{Pichl1967}). \textit{Ŋ kɔ kɔni pïriŋni kïl lɛ vɛ chɛlka thihyɔl ni muni.} Go four times around the house and come back (\citealt{Pichl1967}). 

\TCheadword[2]{chɛlɛk} (unspec. of \TClink[3]{chal}) 

\TCheadword[1]{chɛli} \textit{cf}: \TClink[2]{chal}. \textit{n} dwelling place, living place, [(i)chɛli] home (\citealt{Pichl1967}). \textit{Nchi mbɔs i chɛliyai.} Bring peace to our homes. \textit{Yamfa hɔ̃ wɛy, hɔ̃ pəl ichɛli.} To backbite is bad, it may wreck a home (\citealt{Pichl1967}). \textit{Rait naw mpanth ma lifamalifama, la a ni kunɛ ko ŋami ichɛliɛ kunɛ.} Right now I am involved in farming work, that is what I am involved in in my household. 

\TCheadword[2]{chɛli} \textit{cf}: \TClink{fothi}, \TClink[1]{hɔ}, \TClink[1]{lem}. \textbf{1)} \textit{v} tell. \textit{Kãy, thæ lɛ kɔ ya mɔ chɛliɛ mi chencha lɛ, kɔ hiniɛ min gbɔl labiya bə la hɔm che.} Kay, the story your mother told me yesterday does not please me, hence I put the matter before you (\citealt{Pichl1967}). \textbf{2)} \textit{v} arrange. \textit{Ni bai ko, pɔ lɔ chɛli fe kasaŋ-keɛ ŋɔ leeɛ thɔth.} In the court bari, they are arranging the funeral money (contributions) proportionally. \textbf{3)} \textit{n} sitting.

\TCheadword{chɛm} \textit{cf}: \TClink[3]{bɛŋ}, \TClink[3]{chal}, \TClink[1]{chɛlɛk}, \TClink{gbakra}. \textit{n} (hɔ̃/tha) chair (\citealt{Pichl1967}). 

\TCheadword{chɛnchi} [chɛ̀nchí] \textit{cf}: \TClink{boi}, \TClink{plet}. \textit{n} plate (K dialect). 

\TCheadword{chɛnth} \textit{n} (kɔ/ma) two or three oil fruit on one stalk (\citealt{Pichl1967}). 

\TCsubword{chɛnthmbɛl} (comp.) \textit{n} (kɔ/ma) bran of the oil-fruit (\citealt{Pichl1967}). 

\TCsubword{chɛnthsilɔ} (comp.) \textit{n} (kɔ/ma) honeycomb (\citealt{Pichl1967}).

\TCheadword{chɛnthmbɛl} (comp. of \TClink{chɛnth}, \TClink[2]{bɛl}, see \TClink{chɛnth}) 

\TCheadword{chɛnthsilɔ} (comp. of \TClink{chɛnth}, \TClink{silɔ} (der. of \TClink{siil}, \TClink[2]{lɔ}), see \TClink{chɛnth}) 

\TCheadword{Chɛpo} \textit{nam} Chepo, name given to a place.

\TCheadword{chɛrchɛr} \textit{cf}: \TClink{fiyoŋfiyoŋ}, \TClink[1]{saɛ}. \textit{Idph} [chɛ̀rchɛ̀r] of the cry of the \textit{saɛ} bird, a very small bird that can foretell the future (K dialect). 

\TCheadword{Chɛrnɔ} \textit{nam} Chernor, name given to a person. 

\TCheadword[1]{chɛth} \textit{cf}: \TClink[2]{yaa}. \textit{v} \textbf{1)} cook. \textit{Pɔi chɛth bokɛ pɔi ya joɛ, ha yindɛ ŋai hun gbompani ŋai hun jo.} They will cook the sauce and the rice, and everybody will gather around and eat. \textit{Wɔŋyi huŋ toŋgi ŋɔ pɔ chɛth keŋkeŋdɛ.} She is about to show us how to cook krain-krain. \textit{Mɔ yi hun toŋgi ŋɔ pɔ chɛth pɔmthi gbamdɛ.} You should now come and show us how to cook potato leaves. \textbf{2)} boil. \textit{Ŋkɔ mi chɛthɛ yekə lɛ.} Go boil a cassava for me (\citealt{Pichl1967}). 

\TCsubword{chɛthni} (der.) \textit{v} be boiled. \textit{Sɔk lɛ gbɔw chɛthni.} The fowl is overboiled.

\TCheadword[2]{chɛth} \textit{n} bird species, swallow (K dialect). unspec. \TClink{chɛtlipalkɔ} (see \TClink[1]{pal}) 

\TCheadword{chɛthni} (der. of \TClink[1]{chɛth}, \TClink{-ni}, see \TClink[1]{chɛth}) 

\TCheadword{chɛtlipalkɔ} (unspec. of \TClink[2]{chɛth}, \TClink[1]{pal}, \TClink[2]{kɔ}, see \TClink[1]{pal}) 

\TCheadword{chi} \textit{v} bring, fetch (\citealt{Pichl1967}). \textit{Acheŋɔni pɛ lonibolɛ, bikɔs pɔ chiɛmi ka yaŋ taa.} I would not remember it because I was brought here when I was very young. \textit{Pɔ koŋ gbo pɔ chi fatalaisaɛ pɔi saŋ.} When they have finished, they will bring the fertilizer and scatter it.

\TCheadword{chich} \textit{v} be jealous (B dialect); \textit{chith} be jealous (\citealt{Pichl1967}). \textit{Mɔŋ a chich.} You are jealous. \textit{Poo-m dɛ wɔ chith.} My husband is jealous (\citealt{Pichl1967}). 

\TCheadword{chichin} (Eng \textit{teaching}) \textit{n} teaching. \textit{Aa miyo amu ŋa mpanth ma chichindɛ.} I am presently doing teaching work. \textit{Mpanth ma chichindɛ vɛ pɔ mɔ paka?} This teaching work, do they pay you?

\TCheadword{chiɛ} \textit{cf}: \TClink{thimbɔs}. \textit{n} \textbf{1)} (hɔ̃/tha) shore (\citealt{Pichl1967}). \textit{Hã ke bondɔ ko ni hã nan wɔmdɛ chiɛ ko.} Look for the wharf and pull the canoe on shore (\citealt{Pichl1967}). \textbf{2)} land. comp. \TClink{Sechiɛ} (see \TClink{Se}), comp., id. \TClink{wɔmchiɛ} (see \TClink[2]{wɔm}) 

\TCheadword{chifdɔm} (Eng \textit{chiefdom}) \textit{n} chiefdom. \textit{Chifdɔm ndɔ?} In what chiefdom? \textit{Ya gbemni Nyemɔko, Mamu Sɛkshɔn, Bompɛ Chifdɔm, Mɔyamba Distrikt.} I was born in Moyeamoh, Mamu Section, Bumpeh Chiefdom, Moyamba District.

\TCheadword{chin} \textit{n} (ma) excrement, dung (\citealt{Pichl1967}). \textit{Nchíndɛ̀ mà hónì <fop fup>.} The shit came out <fop fup> (idph of defecating).

\TCsubword{chinmana} \textit{n} (ma) cow dung (\citealt{Pichl1967}). 

\TCheadword{chinchi} \textit{n} (kɔ/ma) tree species, small tree with red fruits (Alchornea hirtella, usually \textit{thɔk chinchi}) (\citealt{Pichl1967}). 

\TCheadword{chinmana} (comp. of \TClink{chin}, \TClink[7]{ma}, \TClink[1]{na}, see \TClink{chin})

\TCheadword{chiseŋ} \textit{v} sneeze (\citealt{Pichl1967}).

\TCheadword{Cho} \textit{nam} Cho, name given to first son. \textit{Chò yòthìɛ́ mì.} Cho has pinched me. \textit{Choo Manu ŋɔ pɔ gbemka miɛ.} Cho Manu is the name I was born with.

\TCheadword{cho} \textit{v} put (\citealt{Pichl1967}). \textit{Gbam dɛ kɔ cho gbïlɛ na lɛ kong nɔthul, kɔ kong lɔɔ.} The potato which you put (on) to roast is soft already, it is roasted (\citealt{Pichl1967}).

\TCheadword{chok} \textit{cf}: \TClink{bimni}, \TClink{pikith}, \TClink{thim}, \TClink{tunt}. \textit{v} twist, spin (\citealt{Pichl1967}). \textit{Yi kwey liwal, si yi chok lɛn ton, si yi panth lɛn do...} We take palm leaves, then we twist them to a fine line, then we tie this line... (\citealt{Pichl1967}). 

\TCheadword{chokoth} \textit{n} (hɔ̃/tha) trap (\citealt{Pichl1967}). 

\TCheadword[1]{chol} \textit{n} art.

\TCsubword{cholnɔ} (der.) \textit{cf:} \TClink{chamne}, \TClink{kɛbi}. \textit{n} (wɔ/hã, pl. achol) artist, craftsman (like: sculptor, carpenter, blacksmith) (\citealt{Pichl1967}).

\TCsubword{lichol} (der.) \textit{n} (lɔ/-) art, skill (\citealt{Pichl1967}).

\TCheadword[2]{chol} [chòl] \textit{cf}: \TClink[1]{bɛlɛ}, \TClink{biim}. \textit{n} fish species, flatfish, slippery, white on bottom, dark on top, edible (K dialect). 

\TCheadword{cholnɔ} (der. of \TClink[1]{chol}, \TClink{nɔ}, see \TClink[1]{chol}) 

\TCheadword{chondal} \textit{adj} lewd (\citealt{Pichl1967}). comp. \TClink{nɔmachondal} (see \TClink{nɔ})

\TCheadword[1]{choŋ} \textit{n} (wɔ/hã, N) fish species, pepe (Pennaeus velutimus) (M tone --contrast with \TClink[2]{choŋ}) (\citealt{Pichl1967}). comp. \TClink{yancheŋkɛ} (see \TClink{yanɔ}) 

\TCheadword[2]{choŋ} \textit{n} (wɔ/hã, N) fish species, smallest kind of freshwater fish (L tone -- contrast with \TClink[1]{choŋ}) (\citealt{Pichl1967}). comp. \TClink{yancheŋkɛ} (see \TClink{yanɔ}) 

\TCsubword{choŋchoŋ} (der.) [chóŋchòŋ] \textit{n} fish species, very small, gets only as large as a baby finger, silky, edible, found only in rivers, not a fish women look for (K dialect). 

\TCheadword{choŋk} \textit{n} (kɔ/ma) herb species, stiff branches with solitary flowers, found near the coast in short grass (\citealt{Pichl1967}).

\TCheadword[1]{chɔ} \textit{cf}: \TClink{bɛmpa}, \TClink{haa}, \TClink[2]{hɛl}. \textit{v} make, fabricate (\citealt{Pichl1967}). \textit{Itu lo hɔ̃ kəlɛŋ hã chɔ thibɛrɛ ni thikaa.} This iron is good for making axes and hoes (\citealt{Pichl1967}).

\TCheadword[2]{chɔ} \textbf{1)} \textit{v} [chɔ́] fight (B dialect). \textit{Bìmndɛ́ wɔ̀ chɔ́ má wɔ̀mdɛ̀.} The porpoise fought the boat. \textit{À mɔ̀ chɔ́.} I will fight you. \textit{Nhɔ gboɛ han ni tikɛ ha chɔ, ma pɛ wei lek thiwɔi.} If you say you will fight the antelope, do not fear the horns (proverb) (\citealt{TISLL1979}). \textbf{2)} \textit{n} war (B dialect).

\TCheadword{chɔch} (Eng \textit{church}) \textit{n} church. \textit{Mɔ kɔ chɔchai?} Do you go to church? \textit{Triniti Chəəch hɔ̃ kïlkïl Ani Wɔlsh skuul.} Tinity Church is opposite to Annie Walsh School (\citealt{Pichl1967}). 

\TCheadword{chɔchɔ} \textit{cf}: \TClink{bolo}, \TClink{kɔŋko}, \TClink{nɔtɔ}, \TClink{suk}, \TClink{thoŋku}. \textit{n} (hɔ̃/hɔ, i) any kind of shell (\citealt{Pichl1967}).

\TCsubword{chɔgbɔyɔ} (unspec.) \textit{v} gamble with cowries (\textit{chɔ' gbɔyɔ} same as \textit{chɔk gbɔlɔ}) (\citealt{Pichl1967}).

\TCsubword{chɔkgbɔlɔ} (unspec.) \textit{v} gamble with cowries (\textit{chɔk gbɔlɔ} same as \textit{chɔ' gbɔyɔ}) (\citealt{Pichl1967}).

\TCsubword{chɔɔmbɛl} (unspec.) \textit{n} \textit{ichɔɔ-mbəl} (hɔ̃/-) shells of broken palm kernels (\citealt{Pichl1967}). 

\TCheadword[1]{chɔk} \textit{cf}: \TClink[2]{hani} (der. of \TClink{haa}, \TClink{-ni}), \TClink[1]{hɛl}, \TClink{traiya}. \textit{v} try. \textit{Ŋa hi tɛmdɛ kache thaozin waŋdɛ fe gbe hɔɛ, kɛ mɔ hɔŋ chɔk o haŋ ni mbɔ ŋɔ.} In those days, ten thousand was a lot of money, but you would try and get it.

\TCheadword[2]{chɔk} \textit{n} (wɔ/hã, N) fish species, skate (Trygon pastinaca) (\citealt{Pichl1967}); fish species (pl. si) (\citealt{Sumner1921}). 

\TCheadword{Chɔkɔ} \textit{cf}: \TClink{Chɔkɔ}. \textit{nam} (wɔ/-) Choko, name given to 5\textsuperscript{th} daughter (\citealt{Pichl1967}). 

\TCheadword[1]{chɔkɔ} \textit{n} half-cloth (Nd dialect). \textit{Mɔ bo tɔi chɔkɔlɛ, pɔ kɔ velɛ chɔkɔlɛ, kɔ mɔ tɔiɛ.} You just wear the half cloth, people call it half-cloth, that you wear. 

\TCheadword[2]{chɔkɔ} \textit{n} (kɔ/ma) tree species, similar to plum tree (Xmas stick, Alchornea cordifolia) (\citealt{Pichl1967})

\TCheadword{chɔkrɔ} \textit{n} (kɔ/tha) piece of cloth tied around the waist and between the legs like a bikini (\citealt{Pichl1967}). 

\TCheadword[1]{chɔl} \textit{n} (hɔ̃/tha) night (\citealt{Pichl1967}, \citealt{Sumner1921}). 

\TCsubword{chɔlayeŋ} (comp.) \textit{n} (hɔ̃/tha) midnight (\citealt{Pichl1967}).

\TCsubword{chɔli} (comp.) \textit{cf}: \TClink{chɔli} (der. of \TClink{chɔl}). \textit{n} (hɔ̃/tha) all night (\citealt{Pichl1967}, \citealt{Sumner1921}). \textit{Thɔsuŋ dɛ hɔ̃ mi, chɔli lo ya thɔsuŋ.} I have a cough, the whole night I was coughing (\citealt{Pichl1967}). \textit{La-m dɛ chen vee, chɔli lo, wɔ-m tə̃ŋhil.} My wife is not well, she cried to me the whole of last night (\citealt{Pichl1967}). 

\TCsubword{chɔlrithi} (comp.) \textit{n} (hɔ̃/tha) moonless night (\citealt{Pichl1967}).

\TCsubword[2]{chɔl} (der.) \textit{temp} at night, in the evening, [ìchɔ́l] at night (B dialect). \textit{...paliioki tɛmpim te ki et-o-klɔk ichɔl wɔni huŋ gbemɔ.} ...the whole day, some times (not) until eight o'clock in the evening before giving birth. \textit{Haaŋ ni nante bɛ, pɔ mu tɔn tontho ki chɔl sakɛ ha hok saka wul-lɛ.} Even up to the present day, people still sing these songs the night of the wake.

\TCheadword{chɔlayeŋ} (comp. of \TClink[1]{chɔl}, \TClink{ayeŋ}, see \TClink[1]{chɔl}) 

\TCheadword{chɔli} (comp. of \TClink[1]{chɔl}, \TClink[1]{-i}, see \TClink[1]{chɔl})

\TCheadword{chɔlɔŋk} \textit{n} fish species, large, wide, not too fat, found in rivers and sea (K dialect); \textit{cholonk} (wɔ/hã, N) fish species, cutterhead (Sphyrna zygaena) (\citealt{Pichl1967}). 

\TCheadword{chɔlrithi} (comp. of \TClink[1]{chɔl}, \TClink[2]{rithi} (der. of \TClink[1]{rithi}), see \TClink[1]{chɔl}) 

\TCheadword[1]{chɔŋ} \textit{v} \textbf{1)} distribute, share food; \textit{chɔŋ} dish up food (\citealt{Pichl1967}). \textit{Kɔ koŋ gbo ho, mɔi chɔŋ.} When the rice is dry, then you dish it out. \textit{Pɔ tipɛ gbo chɔŋ ni tamɔ bul wɔ nan baŋgɛ.} They just started dishing out the food when one child of his pulls a rope (\citealt{Sumner1921}). \textbf{2)} bring. \textit{Mma puthuli komo lɛ wɔ ma chɔŋ lɛɛpi.} Don't spoil the child, it will bring shame on you in the future (\citealt{Pichl1967}). \textit{Ya tɛninin hã lanɛ la-m chɔŋ bɔnth lɛ.} I think of what it will bring me in the future (\citealt{Pichl1967}). \textbf{3)} give (thanks);  \textit{chɔŋ … sɛkɛ} give thanks  (\citealt{Pichl1967}); \textit{chɔŋ … səkɛ} to give thanks (\citealt{Pichl1967}). \textit{Lɔn pɛ yi chɔŋo lɔ Abatokɛ sɛkɛ.} There again we thank God for that part. \textit{Ǹ lɔ̀llɔ́ ɲɛ̀ŋkɛ̀lɛ́ŋ? À chɔ̀ŋá Àbátùkɛ́ sàkà.} Did you sleep well? I give thanks to God.

\TCsubword{chɔŋchɔŋ} (der.) \textit{v} serve food. \textit{Ayi kɔ ŋyai mɛndɛ ko yami, ayi ya ayi chɔŋ-chɔŋ.} And I then go and fetch water for my mother, then I dish it out.

\TCsubword{chɔŋɔni} (id.) \textit{v} give thanks. \textit{Yɛlaioɛ a chɔŋɔni Abatokɛ sɛkɛ fli e.} As it is, I give thanks to God really.

\TCsubword{chɔŋ … len} (id.) \textit{v} \textbf{1)} love. \textit{Jizɔs, a chɔŋ mɔ len.} Jesus, I love you. \textit{Bahin chɔŋ mi len.} My Lord loves me. \textbf{2)} like. \textit{Yi chɔŋ weɛ ŋɔ mɔ tɔndɛ lendɛ.} We like the way you sing. \textit{Mɔm nchɔn la len?} You, do you like it? \textbf{3)} approve. \textit{Apa, nchɔŋ la len?} Father, do you approve of it?

\TCheadword[2]{chɔŋ} \textit{cf}: \TClink[2]{pel}. \textit{v} lay eggs; \textit{chɔ́ŋ} lay eggs (\citealt{Sumner1921}). \textit{Sɔk lɛ wɔ chɔŋ.} The hen lays an egg (\citealt{Pichl1967}). 

\TCheadword{chɔŋchɔŋ} (der. of \TClink[1]{chɔŋ}) 

\TCheadword{chɔŋgba} \textit{temp} forever, eternally (K dialect, \citealt{Pichl1967}). \textit{Yɛ mɔ kɔni bɛɛ limɔai chɔŋgba.} When you go to your kingdom forever (Christian). \textit{Ya bi hã chɔŋ mɔ le̹n chɔŋgba.} I shall love you forever (\citealt{Pichl1967}). \textit{Ŋka hĩ ihɔlɔŋ chɔngba hwɛlɔ lɔ ay.} Give us eternal life in this world (\citealt{Pichl1967}). 

\TCheadword{chɔŋ … len} (id. of \TClink[1]{chɔŋ}, \TClink{len}, see \TClink[1]{chɔŋ}) 

\TCheadword{chɔŋɔni} (der. of \TClink[1]{chɔŋ}, \TClink{-ni}, see \TClink[1]{chɔŋ}) 

\TCheadword{chɔɔlen} \textit{v} be difficult. \textit{Nai wɛ ŋɔ vil ni ŋɔ chɔɔlen mɔnɛ ni sɔan ma lɔ.} The road is long and it is difficult and there are many temptations.

\TCheadword{chɔɔmbɛl} (unspec. of \TClink{chocho}, \TClink[2]{bɛl}, see \TClink{chocho}) 

\TCheadword[1]{chu} \textit{n} (kɔ/ma) tree species, mangrove (\citealt{Pichl1967}). 

\TCheadword[2]{chu} \textit{cf}: \TClink[4]{baŋ} (der. of \TClink[2]{bas}). \textit{v} \textbf{1)} sting, be stung or pricked (\citealt{Pichl1967}). \textit{Kong kong chu bəng wɔ lɛ ka vɛ.} Kong's foot was pricked by a thorn (\citealt{Pichl1967}). \textbf{2)} stab. \textit{Pɔ baŋ wɔ ko thɔkɛ,pɔ chu wɔ wɔn kumbɛ.} They nailed him on the cross, they stabbed him on his side. \textbf{3)} pierce. \textit{Vɛ̀ɛ̀ chú mì.} The thorn pierced me. comp. \TClink{chuŋpaŋ} (see \TClink[1]{paŋ}) 

\TCheadword{chukra} \textit{n} (kɔ/-) walking stick (\citealt{Pichl1967}). 

\TCheadword{Chukra} \textit{nam} (wɔ/-) Chukra, name grandparents call the first-born grandson (\citealt{Pichl1967}). 

\TCheadword{chumbu} (Port \textit{chumbo} ‘lead (Pb)') \textit{n} (kɔ/\nobreakdash-) lead (metal) (\citealt{Pichl1967}).

\TCheadword[1]{chuŋ} \textit{v} give shade (\citealt{Pichl1967}). \textit{Nchuŋ kapathi mɔ lɛ.} Provide shade for us with your wings (\citealt{Pichl1967}). 

\TCheadword[2]{chuŋ} \textit{cf:} \TClink{churuŋ}. \textit{n} (kɔ/-) shade, shadow (\citealt{Pichl1967}).

\TCheadword{chuŋpaŋ} (comp. of \TClink[2]{chu}, \TClink[1]{paŋ}, see \TClink[1]{paŋ}) 

\TCheadword{chuo} \textit{adv} scarcely, hardly, seldom (\citealt{Pichl1967}). \textit{Wante mɔ lɛ wɔ chuo kə.} Your sister is hardly to be seen (\citealt{Pichl1967}). 

\TCheadword{churuŋ} \textit{cf:} \TClink[2]{chuŋ}. \textit{n} (kɔ/-) shade, shadow (\citealt{Pichl1967}).

\end{letter}
\begin{letter}{D}

\TCheadword{dal} \textit{n} (ma) soot (\citealt{Pichl1967}). 

\TCheadword{daŋkɔ} \textit{cf}: \TClink{kuai}. \textit{n} palm kernel oil from the palm nut; [ǹdàŋgò] palm nut oil (B dialect); \textit{ndankɔ} oil from palm kernel (\citealt{Sumner1921}); \textit{ndankɔ} (ma) palm oil (\citealt{Pichl1967}).

\TCheadword{Daru} \textit{nam} Daru, name given to a place. 

\TCheadword{dat} (Eng \textit{that}) \textit{cf}: \TClink[2]{dis}. \textit{dem} that. \textit{Kɛ ɔrijinali ŋan ŋa Kamara, Sise, dis dat.} But originally they were Kamara, Sesay, this, that. \textit{Aa, kɛ bifo dat akoni che ko administreshɔn dɛ fɔ lɔŋg.} Yes, but before that I had been in administration for a while.

\TCheadword{dath} \textit{n} (hɔ̃/tha) helmsman's rear thwart in canoe or boat (\citealt{Pichl1967}).

\TCheadword{datha} \textit{n} [ndatha] pan-boiled rice (B dialect); \textit{ndaatha} (ma) cleaned uncooked husk rice (\citealt{Pichl1967}). \textit{Apimdɛ ŋa kɔ lechɛthɛ, ŋa ha kɔ ndatha.} Some will first boil it and make pan-boiled rice.

\TCheadword{de} (Eng \textit{day}) \textit{n} day. \textit{Yan deɛ ŋɔ huɛ lɔkɔɛ ŋɔ hu wɛ, aka shilani.} As for me, the day he died the day he died, I did not know. \textit{Ayema la gbo shi bikɔs deɛ ŋɔi kɔlɔɛ Mbolomdɛ ma i bɔnthɔ pɔ theli.} I just want to know that, because even the day we went there, it is Sherbro that we heard them talking.

\TCheadword{debɔɛ} \textit{n} female member of Poro Society, for each chapter of Poro there is a \textit{deboi}, a hereditary position held by a woman whose job it is to cook and dress wounds (\citealt{Hall1938}); \textit{ndebɔɛ} woman initiated into Poro and theoretically regarded as a man, e.g., if a woman unintentionally violates a Poro secret, she must become initiated as a man as Poro is forbidden to women (\citealt{Pichl1967}).

\TCheadword{del} \textit{n} part. \textit{M poo shiliŋ bul ndel nra}. Divide one shilling into three parts (\citealt{Pichl1967}). 

\TCheadword{delma} \textit{cf}: \TClink[1]{-n}, \TClink[1]{bɛ}. \textit{pro} self. \textit{Bɛl Maaɛ wɔe hɔko ndelma wɔɛ...} Rat Wife then said to herself...

\TCheadword{Dema} \textit{nam} Dema Chiefdom (pronounced \textit{Ndema} with a pre-nasalized stop but spelled locally without) (Nd dialect). \textit{Ndema ko lɔ pɔ gbem mi.} I was born in Dema (Chiefdom). 

\TCheadword{dembe} \textit{cf}: \TClink{gbogbɔth}, \TClink{lembe}, \TClink{rokos}. \textit{n} lime. \textit{Ŋ swey ndembe lo ni ŋ kɔ ma wɔk ni nsas ma.} Take these limes and go squeeze them (\citealt{Pichl1967}). \textit{Ŋ kwey ndembe lo ni rokos lɛ ni ŋkɔ ma wɔk ni nsas mɔ.} Take the limes and the orange and go and squeeze them and then strain them (\citealt{Pichl1967}). 

\TCheadword{deŋkma} \textit{n} \textit{ndəŋkma} (wɔ/hã) locust (\citealt{Pichl1967}). 

\TCheadword[1]{dɛn} (Eng \textit{then}) \textit{temp} then. \textit{Dɛn yɛ ibɛ nkɔkaɛ lɛko nyɔn doki ŋɔ pɔ vellɛ balansbɔllɛ.} Then we would put our shoes on the ground (for) this thing (game) they called balance ball. \textit{Dɛn yami wokɔ pɛ ko ba yi yɛ, wɔi bi nɔ pokan pika.} Then when mother left our father, she had another husband.

\TCheadword[2]{dɛn} (Eng \textit{them}) \textit{prt} plural marker. \textit{Abolom ŋan dɛn.} All are Sherbros.

\TCheadword[3]{den} \textit{n} [ìdə̀n] hair (B dialect); \textit{idiŋ/irïŋ} (hɔ̃/-) hair (\citealt{Pichl1967}). 

\TCheadword{dɛth} \textit{n} (ma) sides; shipboard (\citealt{Pichl1967}). 

\TCsubword{ndɛthmaboot} (comp.) \textit{n} (ma) shipboard (\citealt{Pichl1967}). 

\TCheadword[1]{di} \textit{cf}: \TClink[2]{hɔth}. \textit{v} \textbf{1)} kill (in some dialects \textit{ji}) (\citealt{Pichl1967}). \textit{Ha ji aŋaiɛ.} They are killing our people. Nchen hã di nɔ. though shalt not kill (\citealt{Pichl1967}). \textbf{2)} catch fish. \textit{Ntolɛ, i pɔŋ hukɛ. Ihukɛ ŋɔi pɔŋɛ, aji.} You used tricks, we threw hooks. It is the hooks that we throw, (and) we caught (fish)! \textbf{3)} initiate into an organization. \textit{Mɔ le bii fe, bikɔs pɔ yema di Bondo atata.} You should first have money, because they want to initiate girls very young. \textit{Labi gɔvmɛntɛ ŋɔ wɔɛ nɔ mɔ le telɛ pɛŋ mɔ hɔ mu di Bondo.} That is why the government says we should wait before we initiate Bondo. \textbf{4)} mix, e.g., cement (or quicklime) with water, \textit{di simɛnt} mix cement (\citealt{Pichl1967}) comp. \TClink{nɔdiɛnɔ} (see \TClink{nɔ}) 

\TCsubword{dini} (der.) \textit{v} kill oneself. \textit{Kɛn kɔ dini ŋkɛn.} The raffia palm kills itself (proverb). The raffia palm is renowned for the delicious palm wine that it produces when tapped. Since it is frequently tapped for this purpose, it is killed by this process. It is considered to be the tree's own fault since this would not happen if the tree did not produce such fine palm wine (\citealt{TISLL1979}).

\TCheadword[2]{di} \textit{cf}: \TClink[2]{buk}, \TClink{yams}. \textit{n} (kɔ/-) yam(s) (\citealt{Pichl1967}).

\TCheadword[1]{dik} \textit{n} \textit{(i)dik} (hɔ̃/tha) bundle, e.g., of wood (\citealt{Pichl1967}, \citealt{Sumner1921}). \textit{Woth dïk iwɔm bo̹m.} He carried a big bundle of wood (\citealt{Pichl1967}). \textit{Wɔ ye tholi idïk iwɔm dɛ.} He took down the bundle of wood (\citealt{Pichl1967}). 

\TCheadword[2]{dik} \textbf{1)} \textit{n} \textit{ndik} (ma) hunger (\citealt{Pichl1967}). \textit{Yɛ nkache ko tallɛ, nkache jo kendɛ ŋɔ nyima ɔ nkache ndik?} When you were young, did you eat as you wanted or did you go hungry? \textit{Ni chii chelɛ ya hun sɔthɔ yen ha sɔm, ndikɛ koŋ mi gbɔɔ!} And bring it so that I can come and eat something, hunger is consuming me! \textbf{2)} \textit{adj} hungry. \textit{Faama wɔ ndïk.} Fama is hungry (\citealt{Pichl1967}). 

\TCheadword{dikil} \textit{cf}: \TClink{dikilni} (der. of \TClink{dikil}, \TClink{-ni}). \textit{v} \textbf{1)} gather. \textit{A koŋ gbo bas, a dikilɛ gbo ipulukɛ ai le yini a chaŋchaŋ tiko.} After sweeping, I would gather the dirty clothes and then leave them there and travel about town. \textit{Ye koŋ vɛ m'minɛ dikil panthɛ gbɛlɛ nkɔŋtha thɔk.} When he is finished eating, then you gather all the pans and wash them. \textit{Ŋ kɔ tïkïl ibəl lɛ kahãy ko, hɔɛɛ lɛ yema le̹l.} Go gather the palm kernels outside, it will rain (\citealt{Pichl1967}). \textbf{2)} recruit. \textit{Labo ja Bondoɛ la ko che kath ŋa dikil apimaɛ, la chaŋ kacheɛ?} Has it become harder to recruit girls for Bondo than in the old days?

\TCsubword{dikildikil} (der.) \textit{v} gather. \textit{So nɛn mɔikɛ raɛ, wɔi chi lan gbi la ko dikildikillɛ.} So about the third year, he will bring everything he has gathered.

\TCsubword{dikilni} (der.) \textit{cf}: \TClink{dikil}. \textit{v} gather. \textit{Yi bi hã dikilni wɔn bəŋthi.} We shall gather at his feet. \textit{Ko lɔ anyaɛ dikleni bai koɛ, anyin ŋa lɔ ŋan thiyeŋ ŋa thee ŋhɔk ma ŋvisɛ ni veesɛ.} Where the people gathered in the bari, there were people among them who understand the words of the animals and the birds.

\TCheadword{dikilni} (der. of \TClink{dikil}, \TClink{-ni}, see \TClink{dikil}) 

\TCheadword{diklia} \textit{v} declare. \textit{Raitɛ ŋa nselɛ bikɔ nko diklia...} The first right (entitlement) because you have declared...

\TCheadword{dim} \textit{n} \textit{(i)dim} (hɔ̃/-) third stage of farming after \textit{yɔktha}, when the trees are felled and burnt (\citealt{Pichl1967}, \citealt{Sumner1921}). \textit{Ichɛk theyɛ lɛ hɔ̃ yi we lɛ idïm d'ay ɛ.} The burnt farm we call by the name “idïm" (\citealt{Pichl1967}). 

\TCheadword{dini} (der. of \TClink[1]{di}, \TClink{-ni}, see \TClink[1]{di})

\TCheadword{dinth} \textit{v} \textbf{1)} gleam, shed a faint light (\citealt{Pichl1967}). \textit{Pan dɛ hɔ̃ dinth.} The moon shines (\citealt{Pichl1967}). \textbf{2)} clean. \textit{Pɔ koŋ gbo tu kɔ dinth yeŋkɛlɛŋ, pɔi chi ituɛ pɔi bɛ lalako.} After pounding the rice and cleaning it properly, they bring the pot and put it on the fire. comp. \TClink{pɛlɛdinthɛ} (see \TClink{pɛlɛ}) 

\TCsubword{dinthɛ} (der.) \textit{adj} \textbf{1)} white, [pɛ̀lɛ̀ɛ̀ kɔ́ dìnthɛ́ɛ̀]/ [kìl dìnthɛ́ɛ̀]/ [kìl thìdìnthɛ́ɛ̀] the rice is white/ white house/ white houses (K dialect). \textit{Ya bi lo̹mɔ dinthɛ.} I have a white gown (\citealt{Pichl1967}). \textbf{2)} clean. \textit{Wɔ bia toŋgiɛ lɛ nɔɛ ki wɔ kunɛ dinthɛ.} He would come and show the person was a clean-belly person. \textbf{3)} bright. comp. \TClink{boldinthɛ} (see \TClink[1]{bol}), \TClink{burɔdinthɛ} (see \TClink[1]{burɔ}), \TClink{kunɛdinthɛ} (see \TClink{kun}), \TClink{pɔkdinthɛ} (see \TClink[3]{pɔk}), \TClink{sweindinthɛ} (see \TClink{swei})

\TCsubword{dinthi} (der.) \textit{v} whiten; make quite clean (\citealt{Pichl1967}). \textit{Ŋkɔ dinthi pəlɛ!} Go and pound the rice quite clean! (\citealt{Pichl1967}). 

\TCsubword{dinthil} (der.) \textit{v} appear pale or sickly (\citealt{Pichl1967}). 

\TCsubword{fedinthɛ} (der.) \textit{n} (hɔ̃/-) silver (money) (\citealt{Pichl1967}). 

\TCheadword{dinthɛ} (der. of \TClink{dinth}, \TClink{-ɛ}, see \TClink{dinth}) 

\TCheadword{dinthi} (der. of \TClink{dinth}, \TClink[1]{-i}, see \TClink{dinth}) 

\TCheadword{dinthil} (der. of \TClink{dinth}, \TClink{-il}, see \TClink{dinth}) 

\TCheadword{dip} \textit{n} (wɔ/hã, si) porcupine (commonly Atherurus africana, but some informants insisted upon the occurrence of Hystrix cristata) (\citealt{Pichl1967}). 

\TCheadword[1]{dis} [dìs] \textit{adj} \textbf{1)} heavy (K dialect). \textit{Pánthɛ̀ mà dìs.} The work is heavy. \textit{Bəth lɛ hɔ̃ dis.} The box is heavy (\citealt{Pichl1967}). \textit{Ŋgber ɛ ma dukə dis nante} The fog fell heavy today (\citealt{Pichl1967}). \textbf{2)} strong. \textit{Hɛŋdɛ hɔ dis.} The wind is strong (\citealt{Pichl1967}). 

\TCsubword{disil} (der.) \textit{adj} heavy (K dialect). \textit{Pánthɛ̀ mà dìs/dìsìl, pánth ndìsíl} The work is heavy, heavy work. der. \TClink{disildisil} (see \TClink[1]{dis})

\TCsubword{disildisil} (der.), (der. of \TClink{disil}) \textit{adj} heavy. \textit{Ŋɔ Hɔbatokɛ loliɛ taamɔtaa bul, wɔ mmɛn hukɔ ni ihɛŋ disil-disil sɔsɔkɔ.} How God saved a little boy, whom heavy waves and heavy winds swept away.

\TCheadword[2]{dis} (Eng \textit{this}) \textit{cf}: \TClink{dat}. \textit{dem} this. \textit{Kɛ ɔrijinali ŋan ŋa Kamara, Sise, dis dat.} But originally they were Kamara, Sesay, this, that.

\TCheadword{Disɛmbaɛ} (Eng \textit{December}) \textit{nam} December. \textit{Tɛmdɛ ni ŋɔ kɔi ni hun sɛkillɛ, tɛm Novɛmba ŋa bɔnth ni Disɛmbaɛ.} The time for drying comes between November and December.

\TCheadword{disil} (der. of \TClink[1]{dis}, \TClink{-il}, see \TClink[1]{dis}) 

\TCheadword{disildisil} (der. of \TClink{disil} (der. of \TClink[1]{dis}, \TClink{-il}), see \TClink[1]{dis}) 

\TCheadword{diskres} (Eng \textit{disgrace}) \textit{n} disgrace. \textit{A bɔ sɔpɔt yami ma diskres.} Then I will be able to support my mother (and) not be a disgrace. \textit{Ye lai bikɔs ivin Pothonɔ ki yɔ hun ke nɔ ndɔndɔ ko wɔko, lɔ yen-yen, pɔ che diskres nɔ.} That is it, because even when this white man came here, he saw everybody in his place, the place is quiet, they do not disgrace people.

\TCheadword{dispɛnsa} (Eng \textit{dispenser}) \textit{n} dispenser. \textit{Dispɛnsa che ŋa ni, be nɔs che ŋa ni lanɛ bɛiyɛ wɔka che ŋa ka cheɛ dɔkta.} There was no dispenser and no nurse, but at the time the paramount chief was here, there was a doctor.

\TCheadword{distrikt} (Eng \textit{district}) \textit{n} district. \textit{Ya gbemni Nyemɔko, Mamu Sɛkshɔn, Bompɛ Chifdɔm, Mɔyamba Distrikt.} I was born in Moyeamoh, Mamu Section, Bumpeh Chiefdom, Moyamba District.

\TCheadword{Dodo} \textit{nam} Dodo, name given to a place. \textit{Pɔ gbem Manɔ ko Manɔ Dodo.} My mother was born in Mano, Mano Dodo. \textit{Yami wɔɔ Mayeni Laŋgo, Manɔ Dodo.} My mother is Mayeni Lango, Mano Dodo.

\TCheadword{dompɔm} \textit{n} \textit{ndom pɔm} (ma) plant species, medicinal leaves from which a gargle is prepared (\citealt{Pichl1967}). 

\TCheadword{dɔkta} (Eng \textit{doctor}) \textit{n} doctor. \textit{Dispɛnsa che ŋa ni, be nɔs che ŋa ni lanɛ bɛiyɛ wɔka che ŋa ka cheɛ dɔkta.} There was no dispenser and no nurse, but at the time the paramount chief was here, there was a doctor. \textit{A kɔ dɔkta lɛ ni sonki mi.} I went to the doctor and he healed me.

\TCheadword{dɔŋ} \textit{n} \textit{ndɔng} (-/ma) gold (sg.“lɔng" is scarcely used) (\citealt{Pichl1967}). 

\TCheadword{dɔɔ} \textit{Idph} of heavy rain falling, same in Mende (K dialect). 

\TCheadword{dɔzin} (Eng \textit{dozen}) \textit{n} dozen. \textit{Ayen lɔlɔ lɔi nan yenchɛkɛ tɛŋka dɔzin ra, dɔzin tin, dɔzin ra.} There is a place where we draw the fish like three dozen, two dozen, three dozen.

\TCheadword[1]{dri} \textit{cf}: \TClink[2]{fai}, \TClink{thuk-thuk} (der. of \TClink{thuk}). \textit{adj} \textbf{1)} [dɹə] ripe (pronounced with something like pharyngeal fricative accompaniment sometimes written \textit{dir} by consultants) (K dialect); \textit{dri} or \textit{drɛ} ripe (\citealt{Sumner1921}); \textit{drï} ripe (\citealt{Pichl1967}). \textit{Apum ŋa pos mbanaɛ, ni apum ŋa nuputha mbana ndriɛ ni gbɛrɛ ha thóŋ bo.} Other bananas, and others mix ripe bananas with flour to fry. \textbf{2)} red, same kind of red as \textit{sa,} except that it is acquired as part of the ripening process, \textit{sa} is something inherent, what an object started with (K dialect); \textit{drï} bright red (\citealt{Pichl1967}). \textbf{3)} ‘red hot.' \textit{Itu lɛ hɔ̃ drï.} The iron is red-hot (\citealt{Pichl1967}). 

\TCsubword[2]{dri} (der.) \textit{n} redness. \textit{A ma mɔ saka, ni nyiɛ mi yɛ driɛ mɔ thihɔlla?} I should stay awake (sacrifice) for you and then have you ask me why my eyes are red? (proverb) (\citealt{TISLL1979}). 

\TCheadword[3]{dri} \textit{cf}: \TClink[2]{dum}. \textit{v} ripen. \textit{Yɛ kɔ koŋ gbemɔɛ, kɔ koŋ gbo kɔi hun dri.} After the rice has tillered, it will ripen.

\TCheadword{du} \textit{n} \textbf{1)} fish fin. \textit{Du gbokbo lɛ bi nyam.} The fins of the catfish are poisonous (\citealt{Pichl1967}). \textbf{2)} (hɔ̃/tha) back fin of fish (\citealt{Pichl1967})

\TCheadword{dua} \textit{n} pneumonia (Nd dialect). 

\TCheadword{duba} (Arabic ‘inkpot') \textit{n} ink (\citealt{Pichl1967}). 

\TCheadword{dugbu} \textit{cf}: \TClink{pekɛ}. \textit{nam} (kɔ/-) place on Sherbro Island where the dead go after their post-mortem to be treated and healed of their surgical wounds. There is also a similar but less famous place in Sherbro country (\citealt{Pichl1967}). 

\TCheadword{dugu} \textit{n} (hɔ̃/tha) Kufu mask, a grotesque mask used by the Kufu Society (\citealt{Pichl1967}). 

\TCheadword[1]{dui} \textit{v} \textit{dẅi} steal (\citealt{Pichl1967}). \textit{Nchen nhã dẅi.} Thou shalt not steal (\citealt{Pichl1967}). \textit{Lɛ pə kɔ hã dẅi...} If one goes to steal... (\citealt{Pichl1967}). \textit{Tamɔ lɛ wɔ dẅiye ken top.} The boy is stealing like a ground-pig (\citealt{Pichl1967}). comp. \TClink{nɔdwiyɛ} (see \TClink{nɔ}) 

\TCheadword[2]{dui} \textit{n} \textit{idẅi} (hɔ̃/-) theft, stealing (\citealt{Pichl1967}). \textit{Idẅi hɔ̃ iwɛy.} Theft is bad (\citealt{Pichl1967}). 

\TCheadword{duiyɛ} \textit{n} thief. \textit{Ko-lɔ ma kɔ ko bawɔ lɛ, kɔni ko anya dẅiyɛ lɛ lɔ ndɔ pə kɔnth wɔ.} Instead of going to his father he went to a company of thieves (and) there he was caught (\citealt{Pichl1967}). 

\TCsubword{duiye-duiye} (der.) \textit{n} \textit{dẅiye-dẅiye} (wɔ/hã) habitual thief; pick-pocket (\citealt{Pichl1967}). 

\TCheadword{duk} \textit{cf}: \TClink{thol}. \textit{v} \textbf{1)} drop, fall, descend (K dialect); fall; set (sun) (\citealt{Pichl1967}). \textit{Wɔe duk sampa yekeɛ kunɛ, gbunda yekeɛ maŋchaŋma wɔɛ.} She drops into the cassava basket, grabs the cassava with her teeth. \textit{Sese duk thɔk lɛ.} Sese fell down the tree (\citealt{Pichl1967}). \textit{Thàfɛ́ (hɔ) dùkɔ́ (*-ɛ) nì hɔ̀ pɛ́l thìsék.} The pipe fell and broke into small pieces. \textit{Yɛlaio wɛ, yɛ mgbɛ ma dukɛ…} As it is, when the fog falls… \textit{Palli lɛ yema duk.} The sun is about to set (\citealt{Pichl1967}). \textbf{2)} befall. \textit{Liwu lɔ bɔnthɔ hĩ, gbundɛ bo̹m koŋ duk pɔk l'ay.} Calamity has met us, big trouble has befallen the country (\citealt{Pichl1967}). \textit{Thɔli, hã thɔli, gbundɛ bo̹m koŋ duk trï ka.} Keep silent, big trouble has befallen this town (\citealt{Pichl1967}) \textbf{4)} take place. \textit{Bɛɛ pɔkɛ wɔ ka huɛ ni bon bom kɔ huŋ duk, pɔkai gbi hɔ taŋ ŋa wɔ.} the chief of the country died and then a great feat took place, the whole country cried for him (\citealt{Sumner1921}). \textbf{5)} beat. \textit{Kɔŋ-gbɔl, kɔŋ-gbɔl wɔ lɛ kɔ duk yɛ pə wɔ ku ilel lɛ.} His heart beats when they call his name (\citealt{Pichl1967}). 

\TCsubword{dukduk} (der.) \textit{v} descend. \textit{Ŋa hɛthhɛthni ŋa dukduk hiŋk ndɔndɔ, ŋa gbundagbunda feɛ hiŋk mɛsaɛ atok.} They slipped in (descended) from all directions, they grabbed the money from on top of the table.

\TCsubword{duki} (der.) \textit{v} \textbf{1)} make fall. \textit{Hã kɔ chæ tə lɛ ko bikaa lɛ duki chɔl na næ lɛ 'hol lɛ.} Go and lift the tree that the storm felled on the road last night (\citealt{Pichl1967}). \textbf{2)} throw down (\citealt{Pichl1967}). \textbf{3)} leave. \textit{Kenyaa Braimaɛ, Ba Amadu Kamara, bi mpɛl hɔth kaɛ kek thira: mpɛl ma ŋgbampɔɛ, mpɛl ndukiɛ ni yɛlɛɛ.} Braima's uncle, Ba Amadu Kamara, has fishing nets, three different types: bonga nets, nets they leave at sea, and the chain. \textbf{4)} drop. \textit{Ka nlɛrni, wɔe duki kilikɛ.} He hurried up and dropped the anchor. \textit{Kɔiɛ, wɔe tɛniniɛ lɛ dukiɛ gbo kilikɛ…} He thought if he dropped the anchor… \textbf{5)} use. \textit{A chen duki pɛl, nhukɛ ma a dukiɛ.} I do not use a net, I use hooks.

\TCheadword{duki} (der. of \TClink{duk}, \TClink[1]{-i}, see \TClink{duk}) 

\TCheadword{dul} \textit{v} leak. \textit{Ŋkɔ-m sɔkiɛ kïl mi lɛ, hɔ̃ gbɔw dul.} Go re-thatch my roof, it is leaking too much (\citealt{Pichl1967}). 

\TCheadword{dulɔ} \textit{adj} leaky. \textit{Wɔm dulɔ.} A leaking canoe (\citealt{Pichl1967}). 

\TCheadword[1]{dum} \textit{v} \textbf{1)} raise. \textit{Wɔn dɔ pɔ du mɔ wɔa?} Where was he raised? \textit{A-a, wɔn pɔ du mɔ wɔ ni ka.} No, he was not raised here. \textit{Thetha mi wɔ ka dum miɛ.} It is my grandmother who raised me. \textit{Pɔ gbem wɔ Shenge ka pɔ wɔ ŋai dum ŋa.} She was born here at Shenge and she was raised here. \textit{Yaŋ pɔ dumɔ mi Shenge ka.} Me, I was raised here in Shenge. \textbf{2)} train or educate a child, dog, etc. (\citealt{Pichl1967}). \textit{Tamɔ lo koŋ-kosul, nche wɔn pɛ wɔ dum.} The child is inveterate beyond reform, you will not be able to train him anymore (\citealt{Pichl1967}). \textbf{3)} be raised.

\TCsubword{dumka} (der.) \textit{v} train. \textit{Ma wɔ dumka igbɛth wɔnɛ bɛ hun gbo che igbɛth.} Do not raise him to be spoiled (immoral), the ones coming (after him) will be spoiled. 

\TCsubword{dumɔni} (der.) \textit{v} \textbf{1)} raise. \textit{Che nɔ pika wɔ dumɔni yɛ, yanyi wɔn wɔ dumɔniye.} There is no one else who raised us, it is our mother who raised us. \textbf{2)} be trained; tame (\citealt{Pichl1967}). \textit{Thumɔɛ lɛ dumɔni.} The dog is tame (\citealt{Pichl1967}). 

\TCheadword[2]{dum} \textit{cf}: \TClink[3]{dri}. \textit{v} \textbf{1)} be ripe, [m̀màŋgùɛ̀ (má) kóŋ dùm], [má dùmɔ̀], [mà dúm] The mango is ripe (all three have the same interpretation) (K dialect). \textit{Mà dúm gbèŋ} It will be ripe tomorrow. \textbf{2)} be full; be in fruit (\citealt{Pichl1967}); full (as fruit, rice, etc) (\citealt{Sumner1921}). \textit{Pəlɛ lɛ tipɛ dum.} The rice begins to fill the ears (husks) (\citealt{Pichl1967}). 

\TCheadword[3]{dum} \textit{n} \textbf{1)} character. \textit{Bolomnɔɛ wɔn wɔ bi ndum, yemani theliaŋ gbe.} The Sherbro man has a good character, he does not like much talking. \textit{Mɔ ŋa koi ndumma mɔe ma pɔ dumɔ mɔi.} You should take the character you were raised up with. \textbf{2)} parenting (‘training' in West African English). \textit{Ndum ŋwɛiɛ.} Bad training (parenting)! (said of a child who asked what was in a wrapped parcel) (K dialect). \textbf{3)} behavior. 

\TCheadword{dumka} (der. of \TClink[1]{dum}, \TClink[4]{ka}, see \TClink[1]{dum}) 

\TCheadword[1]{dumɔ} \textit{v} be strong, be hard (\citealt{Sumner1921}) \textit{Ŋkɔni ayen gbi ha kɔ lɛliɛ yen joo, ni nsiiɛ ya kun dumɔ.} You do not go anywhere to find things to eat, and you know my belly is hard (i.e. I am about to give birth).

\TCsubword{dumɔ-dumɔ} (der.) \textit{adj} very strong. Ni wɔ ye kɔ thoɛi ko ni bɛthi mbank ndumɔ ndumɔ ni chi ma kilɛi wɔ ko ni thɔnghul ma. And so he went to the bush and cut very strong ropes and brought them to his house and kept them (\citealt{Sumner1921}). 

\TCheadword[2]{dumɔ} \textit{adj} difficult, hard. \textit{Yɛmɔ theli ko aŋaɛ, nwɔk mpim ma pɔ chi komɔko ma che ndumɔ, nye?} When you talk to the people, some cases they bring to you are difficult, right?

\TCheadword{dumɔndumɔ} (der. of \TClink[1]{dumɔ}) 

\TCheadword{dumɔni} (der. of \TClink[1]{dum}, \TClink{-ni}, see \TClink[1]{dum}) 

\TCheadword{Duramani} \textit{nam} Duramani, name given to a person. 

\TCheadword{duth} \textit{v} burst. \textit{Ko lɔ yenjo kɛlɛŋdɛ hɔ ma simɛnjɛmdɛ, ni kundɛ hɔ duth.} Before good food spoils, let stomach burst (proverb) (\citealt{TISLL1979}). 

\end{letter}
\begin{letter}{E}

\TCheadword[1]{e} \textit{cf}: \TClink[1]{a}. \textit{prt} clause-final interrogative particle (\citealt{Pichl1967}). 

\TCheadword[2]{e} \textit{prt} Negative (possible negative from vowel lengthening with a high tone, but may be just addition of high tone) (B dialect). \textit{Dɛ nɔɛ ŋɔth bo ka ntɛnt ɡbi chee pɛ di.} If someone fishes near here, he gets no catch. \textit{A chee ki wɔɛ mɔ tɔi wɔch.} I did not say you wear a watch.

\TCheadword{e-e} \textit{disco} expression of surprise or consternation. \textit{Ee tombo bɔnth hin.} Eh, we are in trouble. \textit{E-e-eh, yam bɛ a sini bikɔs a che chal telɔ shɔp pai.} Eh, myself I do not know because I do not sit at the tailor shop. \textit{Achɔn ma len eh, bikɔs amɔs wɔni ɛ nwɔkɛ ma pɔ yemaɛ mavɛ Mbɛkɛ vɛ.} I like it, because I must say the language they want, it is that Krio.

\TCheadword{ee} \textit{cf}: \TClink{aa}, \TClink{ayo}, \TClink{yɛs}. \textit{disco} \textbf{1)} \textit{ee, ẽẽ, eye} all right (\citealt{Pichl1967}); \textit{eye} all right (\citealt{Sumner1921}). \textbf{2)} yes. \textit{Ee, pɛth-pɛth ŋɔ lɔ.} Yes, it is sweet.

\TCheadword{ej} (Eng \textit{age}) \textit{n} age. \textit{Tɛm landɛ ejimdɛ ŋɔ ej ɔf fɔti sɛvin yias.} At that time, my age was 47 years.

\TCheadword{Emi} \textit{nam} Amy or Amie, female name given to a person. \textit{Yami ilel wɔɛ Emi Manli.} My mother's name is Amie Manley.

\TCheadword{Epril} \textit{nam} April. \textit{Pandɛ ŋɔ pɔ wɔ Epril, nɛndɛ ŋɔ pɔ wɔ tu thaozin ɛn sikstin.} The month they call April, the year they call two thousand and sixteen.

\TCheadword{eria} (Eng \textit{area}) \textit{Loc} area. \textit{Ɛlaboɛ kostal eria, halthe ntɛnt lɔ Athemaɛ ŋahun challɛ.} Just that coastal area, the seaside where the Themnes have come and settled. \textit{Nɔs gbi ŋa ka cheni eriaio ai, ɔspitalai fli nɔs ka che ŋa ni.} There was no nurse in that whole area, even in the hospital there was no nurse.

\TCheadword{etoklɔk} (Eng \textit{eight o'clock}) \textit{temp} eight o'clock. \textit{Pamdɛ kunɛ wɔɛ ŋɔ tipɛ nɛki etoklɔk oki.} If the pregnancy begins to hurt at eight o'clock, okay.

\TCheadword{ewɔ} \textit{disco} why? (\citealt{Sumner1921}). \textit{Iwɔ, ha wul lijajɛl wɔɛ la wɔ mamɛ?} Why, with the death of his mother-in-law, why is he laughing?


\end{letter}
\begin{letter}{Ɛ}

\TCheadword[1]{ɛ} \textit{post} \textbf{1)} in. \textit{Kɛn bo bi ŋsɔkba la mɔ tenɛ, ha mɔn wɔ…} But if you have a problem in mind and you want to talk… \textit{Là mí bòlɛ̀; Mà mì bénbòlɛ̀.} It is in mind; Do not keep me in your mind (Do not think or worry about me). \textbf{2)} at; on. \textit{pia njokɛ/pia minɛ} on the right hand side/on the left hand side, \textit{njok ɛ/min ɛ} at or on the right/at or on the left (\citealt{Pichl1967}) comp. \TClink{bɛnbolɛ} (see \TClink[1]{bol}), \TClink{kɛntrithoɛ} (see \TClink{kɛntri}), der. \TClink{kunɛ} (see \TClink{kun}), \TClink{mɛnɛi} (see \TClink[2]{mɛn}), \TClink{piaminɛ} (see \TClink[1]{pia}), \TClink{pianjokɛ} (see \TClink[1]{pia}), \TClink{wɔmtokɛ} (see \TClink[2]{wɔm})

\TCheadword[2]{ɛ} \textit{cf}: \TClink[2]{ndɛ}. \textit{def} definite article. \textit{Tɛm landɛ vɛ ŋɔ mɔi ya?} That time how (old) were you? \textit{Kɛ ahindɛ ŋa nko gbemiɛ ŋan gbi nshiŋa?} But the people you have delivered, do you know them all? \textit{Sistha Kɔba ŋaha kaŋa hi mpanthoɛ.} Sister Koba is the one that taught us this work. der. \TClink[3]{lanɛ} (see \TClink[1]{lan}) 

\TCheadword[3]{ɛ} \textit{prt} subordinate clause-final particle. \textit{Sɔŋkɔma ŋɔ wɔ gbo che haaɛ.} ...just like he had been doing. \textit{Igbimi chen po ko lɔ lijɛm chendɛ.} Smoke will not appear where there is no fire (proverb) (\citealt{TISLL1979}). \textit{Mpang nwang ni tïng man ma nɛn bul ay ɛ.} There are twelve months in one year (\citealt{Pichl1967}). \textit{Ba Na ni gbɔlkajo wɔ ɛ.} There was a spider who was very gluttonous (\citealt{Pichl1967}).

\TCheadword{-ɛ} \textit{v} \textit{sfx} verbal suffix denoting act or state, \textit{sɛm} to stand; \textit{sɛmɛ} to be standing; \textit{hin} to lie down; \textit{hinɛ} to be lying down (\citealt{Sumner1921}). \textit{A yiyɛ Bahin ŋa toŋi mi nai wɛ we.} I ask the Lord to show me the way. \textit{Ya sɛmɛ kil lɛ ahɔl.} I am standing at the door. \textit{Ko lɔ Kaiŋ Taso hinɛ pɛllɛaiɛ...} Where Kain Tasso was lying in the hammock... der. \TClink{hɔlɛ} (see \TClink[2]{hɔl}) 

\TCheadword{ɛm} \textit{disco} \textbf{1)} em. \textbf{2)} um.

\TCheadword{ɛn} (Eng \textit{and}) \textit{cf}: \TClink[2]{-i}, \TClink[1]{kɛ}, \TClink[4]{la}, \TClink[1]{o}. \textit{coordconn} ‘and,' usually clause-initial. \textit{Pɔ kɔ yuk tonton, ɛn pɔ kɔ pɛ ka thiwonka.} People will plant a little (here and there), and people will make space. \textit{En lanɛ la bia hu theli kaɛ, ŋanɛ gbi ŋa bia yema ŋa thela wɔlɔkaɛ ŋala bia the.} And all that he has to say here, all that would want to hear it in this world would hear it.

\end{letter}
\begin{letter}{F}

\TCheadword{faani} \textit{v} rely. \textit{Nɔ ma faani bith puthul.} One should not rely on rotten kindling (proverb) (\citealt{TISLL1979}). 

\TCheadword[1]{fai} (Themne ‘slaughter'?) \textit{n} Poro bush, area away from town where Poro Society initiations and education take place, [fai]/[fai ko] Poro bush/ in the Poro bush (Nd dialect); \textit{fay} (kɔ/tha) Poro bush (\citealt{Pichl1967}). 

\TCheadword[2]{fai} \textit{cf}: \TClink[1]{dri}. \textit{adj} [fáí] hot, as pepper (B dialect); hot; burning (\citealt{Pichl1967}). \textit{Kəfe kɔ fay.} Pepper is hot (\citealt{Pichl1967}). 

\TCheadword{fainal} (Eng \textit{final}) \textit{adj} finalized. \textit{Kɛ ako ŋɔ mu sɔthɔ fainal.} But I have been able to get it finalized. 

\TCheadword{fainali} (Eng \textit{finally}) \textit{adv} finally. \textit{Bikɔ pabondɛ nko diklia, nko siɛ lɛ fainali nko sɔthɔ.} Because if you have declared, finally you know that you have got (it). 

\TCheadword{faka} \textit{n} (kɔ/tha) village (\citealt{Pichl1967}). \textit{Trïthi hĩ lɛ ni fakathi hĩ lɛ thipum tha sɛmɛ hial ato̹k.} Our towns and villages, some are situated on rivers (\citealt{Pichl1967}). 

\TCheadword{Faama} \textit{nam} Fama, name given by Poro Society, Faama Thampel is the founder of the Kabu fishing society (\citealt{Pichl1967}). \textit{Fama wɔ ndïk.} Fama is hungry (\citealt{Pichl1967}). 

\TCheadword[1]{fama} (Eng \textit{farmer}) \textit{cf}: \TClink[2]{ra}, \TClink{woŋkru}. \textit{v} farm. \textit{Wɛl, ya fama, a ra.} Well, I farm, I brush (clear fields).

\TCheadword[2]{fama} (Eng \textit{farmer}) \textit{cf}: \TClink{nɔhinyɛchɛk} (comp. of, der. of \TClink{nɔ}, \TClink[1]{hini}, \TClink{chɛk}), \TClink{nɔra} (comp. of \TClink{nɔ}, \TClink[2]{ra}). \textit{n} \textbf{1)} farming. \textit{So ŋan fama lɛki bo laŋa kache kunɛ?} So it was just this farming that you were engaged in? \textbf{2)} farmer. \textit{Ka cheɛ fama, mpanth ma wɔ ma ka gbo cheɛ.} He was a farmer, that was his only job. textit{Wɔn bɛ ka cheɛ fama.} He himself was a farmer. 

\TCsubword{famalifama} (der.) \textit{n} [lifamalifama] farming. \textit{Rait naw mpanth ma lifama-lifama.} Right now I am involved in farming work. 

\TCheadword{Famancha} \textit{nam} (wɔ/-) Famancha, companion of the Laka speaking on his behalf (\citealt{Pichl1967}). 

\TCheadword{famili} \textit{n} family. \textit{Labondɛ ŋa cheni famili bul, ŋan ni ayindɛ vɛ, nchelɔ jo.} If you are not in one family, you and others would not eat it.

\TCheadword[1]{fan} \textit{n} animal species, cutting grass, [fààndɛ́] cutting grass (def) (B dialect); (wɔ/hã, si) cane-rat or cutting grass (Thryonomys swinderianus) (\citealt{Pichl1967}). comp. \TClink{pɛlmfan} (see \TClink[2]{pɛl}) 

\TCheadword[2]{fan} \textit{cf}: \TClink{kiptha}. \textit{n} (hɔ̃/-) or (ma) new and very sweet palm-wine (\citealt{Pichl1967}). 

\TCheadword{fani} \textit{v} depend on. \textit{Yi fani gbo nhɔk ma wɔ rɔŋ dɛ.} We depend only on the truth of his words (\citealt{Pichl1967}). 

\TCheadword{faniŋ} \textit{adj} \textit{ifanïng} (\textit{n} [ifaniŋ] used as \textit{adj}) grey-haired; grizzled (\citealt{Pichl1967}). 

\TCheadword{fatalaisa} (Eng \textit{fertilizer}) \textit{n} fertilizer. \textit{Pɔ koŋ gbo pɔ chi fatalaisaɛ pɔi saŋ.} When they have finished, they will bring the fertilizer and scatter it.

\TCheadword{fe} \textit{cf}: \TClink{baar}, \TClink{baaryeŋ} (unspec. of \TClink{baar}), \TClink{kɔpa}. \textit{n} money. \textit{Chaŋgbo lɛ abi bo fe, akɔ pin kɔtin, ayi huŋgul.} If I have (any) money at all, I will buy cotton (cloth) to sell. \textit{Kache ŋɔn hi, mbi fe, mbiyɛni fe ha nyamɔ ŋa mɔ bɔnth.} In the past, whether you had money or not, your people would help you. \textit{fe̹ bo̹l bul} one head (of tobacco) equals three pounds (lit. one head money) (\citealt{Pichl1967}). \textit{A chen bɔ pin sigarɛt lɛ, ya biɛn gbo fe̹.} I am not able to buy cigarettes if I have no money (\citealt{Pichl1967}). comp. \TClink{gbɔlmafe} (see \TClink{gbɔl}) 

\TCsubword{muŋkofe} (comp.), (id.) cf: \TClink{muŋkokol}. \textit{v} return the dowry (\citealt{Pichl1967}).

\TCheadword{fedinthɛ} (comp. of \TClink{fe}, \TClink{dinthɛ} (der. of \TClink{dinth}, \TClink{-ɛ}), see \TClink{dinth}) 

\TCheadword{ferna} \textit{n} \textit{fe̹rna} (kɔ/-) clouds, clouded sky (\citealt{Pichl1967}). 

\TCheadword{Fɛbuari} (Eng \textit{February}) \textit{nam} February. \textit{Nante ndɔi mɔikɛ waŋnibullɛ, Fɛbuari, 2016.} Today is the eleventh day of February, 2016.

\TCheadword{fɛgbɛ} \textit{Idph} of lying motionless, tired, and lying flat due to fatigue, [fɛgbɛ]/[lɛlɛ fɛgbɛ] lying motionless/ flat on the ground (also used in Mende) (K dialect).

\TCheadword{fɛki} \textit{v} \textbf{1)} disregard. \textit{Tamɔ lɛ fɛɛkiɛ mi sin dɛ ya chaŋ bawɔ bɛn.} the boy disregards me, he doesn't realize that I am older than his father (\citealt{Pichl1967}). \textbf{2)} disrespect. \textit{Tamɔɛ fɛkiɛ́ mì.} The boy disrespected me. \textit{Tàmɔ̀ɛ̀ wɔ̀ fɛ̀kiɛ́ mì.} The boy has disrespected me. \textit{Tàmɔ̀ɛ̀ wɔ́ mí fɛ̀kí.} The boy disrespects me. \textit{Tamɔɛ wɔ̀ mí fɛ̀kí.} The boy is disrespecting me. 

\TCheadword{fɛlɛɛ} \textit{n} fish species, silver in color, 4 inches, edible, people fish for them, sometimes with very small hooks (\#20), found at very top of the river where it enters the swamp (K dialect).

\TCheadword{fɛŋgbɛ} \textit{n} shroud. \textit{Yɛ laioɛ achelɔ pɛ ke bik, anibo ke fɛŋgbɛɛ.} As is it now I do not see a mat there again, we just now see a shroud.

\TCheadword{fɛt} \textit{Idph} of not doing again. \textit{Nche ma pɛ <fɛt>?} You would not go there again <fɛt>?

\TCheadword{fɛsɛ} \textit{cf}: \TClink{velni}. \textit{v} \textbf{1)} \textit{fɛs} to be neighboring, near (\citealt{Pichl1967}). \textit{Kisi lɔ fɛsɛ Kyamp ko.} Kisi is near Freetown (\citealt{Pichl1967}). \textit{Checharaŋ lɛ fɛsɛ ncho ma hɔbatokɛ.} Cleanliness is next to godliness (\citealt{Pichl1967}). \textbf{2)} \textit{fɛsɛ} be opposite to (\citealt{Sumner1921}). \textbf{3)} resemble. \textit{Gbe̹mni abəka lɛ ni nche ma hã lɛ ma fəsɛ hã ma apotoa.} The inheritance and the way of life of the Krios resembles those of the Europeans (\citealt{Pichl1967}). 

\TCsubword{fɛtɛfɛtɛ} (der.) \textit{adv} near, close. \textit{Lagbo pɔnthai pɔ che kɔ yuk fɛtɛfɛtɛ ni.} If it is in the swamp, it is not planted very close.

\TCsubword{fɛtɛn} (der.) \textit{adj} near, close. \textit{Laŋgba dɛ fli wɔ ŋa fɛtɛndɛ Lɔmli, Malama Bolomnɔ…} Even the man they are close with at Lumley, Malama, is Sherbro… \textit{Tɛm lan ikɔlɔ bɛ pa, bikɔs kil hinyɛ ŋɔ fɛtɛni bo…} Even that time we went there, because our house is just close…

\TCheadword{fiii} \textit{Idph} of a mouse or rat squeak. \textit{Bɛlsɛ ŋae tipɛ gbik-gbikni baiɛ tokɛ, <kara-kara kara-kara kara-kara> ŋa hɔɛ, <fiii fiii fiii>.} The rats began scampering up above the bari, <kara-kara kara-kara kara-kara> they were saying <fiii fiii fiii>.

\TCheadword{fiifi} \textit{cf:} \TClink[1]{san}. \textit{n} ant species (K dialect); \textit{fiifii} (wɔ/hã, N) ant species, very large (larger than \textit{san}), black, used for medical purposes (\citealt{Pichl1967}).

\TCheadword{fiithii} \textit{n} evening. \textit{Yɛ fiithiiɛ koŋ pɛriɛ sɔŋkɔma mɛŋk mɛŋraɛ…} When the night had filled the eighth hour…

\TCheadword{fik} \textit{adv} \textit{lifĩk} at random (\citealt{Pichl1967}). \textit{Tamɔ lɛ wɔ gbo ha len lifĩk, chen tɛnini.} The boy just does things at random, he does not think (\citealt{Pichl1967}). 

\TCheadword{fikthiŋ} \textit{n} (kɔ/ma) fishing rod (\citealt{Pichl1967}).

\TCheadword{fil} (Eng \textit{field}) \textit{n} field. \textit{Bɔllɛ ŋɔn lagbolɛ mɛŋk, mɛŋkɛ hɔ mɔigbo, ŋakɔni fillai ŋa kɔ siŋ.} The football (match) is scheduled, when the time comes, they go to the field and play.

\TCheadword{filɛ} (Eng \textit{feel}) \textit{v} feel. \textit{Ŋɔ nfilɛ ŋa lan?} How do you feel about that?

\TCheadword{fili} \textit{cf}: \TClink[1]{bɛ}, \TClink{ivin}, \TClink[1]{mu}. \textit{adv} \textbf{1)} really. \textit{Ya la mɛmiɛni fli ha haŋ mpanth haŋ pɔkimdɛ.} I am happy about that, to really work for my country. \textit{Rɔŋ fili wɔ mi lɛli atok.} Yes indeed he really, really cares for me. \textbf{2)} even. \textit{Aa ha ka che theli Mbolomdɛ, wɔnɛ fli ka che O Si pɔlis, Hestins.} Yes, they used to speak Sherbro, even the one who was an OC Police (officer), Hastings.

\TCsubword{flifli} (der.) \textit{adv} really. \textit{Gbemiɛ ki la mɔɔ ki kunɛ, mɔ mɛmiɛni ŋa lan ŋa mɔm che gbemi ahindɛ fli-fli?} This midwife work that you are in, are you really happy to just be delivering? \textit{Bolomnɔ flifli.} The real Sherbro man.

\TCheadword{Filip} \textit{nam} Phillip, male name given to a person.

\TCheadword{finthi} \textit{n} (hɔ̃/tha) fishing net, set as trap on the side of the river (\citealt{Pichl1967}).

\TCheadword{fintiani} \textit{v} be knotted. \textit{Baŋk lɛ koŋ fintiani.} The rope is knotted (\citealt{Pichl1967}). 

\TCheadword{fip} \textit{Idph} of bursting out. \textit{Aftabakɛ ŋɔ hun gba ki <gbiŋ>, blidin iŋɔi huŋyi ki fip.} The afterbirth came and really got stuck, then bleeding burst out badly.

\TCheadword{fisa} \textit{v} be better (\citealt{Pichl1967}). \textit{Tipɛni fisa.} He begins to be (or: to feel) better (\citealt{Pichl1967}).

\TCheadword{fishaman} (Eng \textit{fisherman}) \textit{n} fisherman. \textit{Aa, ka che fishaman.} Yes, he was a fisherman.

\TCheadword{fishiŋgraund} (Eng \textit{fishing ground}) \textit{n} fishing area. \textit{N shiɛ Shenge ka fishiŋ-graund lɔɛ.} You know, Shenge here is a fishing ground.

\TCheadword{fithnan} \textit{n} (hɔ̃/-) epidemics; sickness (\citealt{Pichl1967}). 

\TCheadword{fiyoŋfiyoŋ} \textit{cf}: \TClink{chɛrchɛr}, \TClink[1]{saɛ}. \textit{Idph} [fíyóŋfíyóŋ] of the cry of the \textit{saɛ} bird, a very small bird that can foretell the future (K dialect).

\TCheadword{Flaide} (Eng \textit{Friday}) \textit{nam} Friday. \textit{Tipik huɛ seinyɛ, ŋɔ pɔ vellɛ Flaideɛ Mpothoaiɛ...} Beginning from the first day, which they call Friday in English...

\TCheadword{flawa} (Eng \textit{flower}) \textit{n} (hɔ̃/tha) flower (\citealt{Pichl1967}).

\TCheadword{flɛg} (Eng \textit{flag}) \textit{n} (hɔ̃/tha) flag (\citealt{Pichl1967}).

\TCheadword{flifli} (der. of \TClink{fili}) 

\TCheadword{floŋ-floŋ} \textit{v} blow. \textit{Ŋɔ ihɛŋ dɛ ŋɔ floŋ-floŋ wɔm dɛ vɛ mmɛn dɛ ma pɔŋni wɔm dɛai.} The way the winds were blowing the canoe, the water poured into the canoe. 

\TCheadword{fofo} \textit{n} (kɔ/-) grass species (Rottboellia exaltata) (\citealt{Pichl1967}). 

\TCsubword{fofobaka} (unspec.) \textit{n} (kɔ/-) grass species (Ischaemum rugosum) (\citealt{Pichl1967}). 

\TCheadword{fol} \textit{cf}: \TClink{kɔnaibol} (id. of, comp. of \TClink[2]{kɔ}, \TClink[1]{nai}, \TClink[1]{bol}), \TClink{naibol} (id. of \TClink[1]{nai}, \TClink[1]{bol}). \textit{v} defecate; shit. \textit{Mɔm komɔ rɛmda ki, ya chen lan haa gbi. Mɔm komɔ kel ki, ŋchen ŋɔn fol.} You child of a viper, I will not do it--at all. You child of a monkey, you will not shit it (out). 

\TCheadword{fon} (Eng \textit{phone}) \textit{v} call. \textit{Rait naw isɔloki pɔ ko mi bɛ fon ŋa hanya tiŋ.} Right now, this morning, they have called me for two people.

\TCheadword{fonde} \textit{n} (hɔ̃/-) asthma (\citealt{Pichl1967}).

\TCheadword{Foŋke} \textit{nam} (wɔ/-) Fonke, male name given by Toma Society (\citealt{Pichl1967}). 

\TCheadword{fothi} \textit{cf}: \TClink[2]{chɛli}, \TClink[1]{hɔ}, \TClink[1]{lem}. \textit{v} tell (\citealt{Pichl1967}).

\TCsubword{fothimbol} (comp.) \textit{v} lie (lit. tell a lie) (\citealt{Pichl1967}). 

\TCheadword{fothimbol} (comp. of \TClink{fothi}, \TClink[2]{bol}, see \TClink{fothi}) 

\TCheadword{fothok} \textit{v} \textit{fothok mbol} tell lies about someone, slander (\citealt{Pichl1967}); \textit{fothok} only ever used with \textit{mbol} (K dialect). \textit{Ŋa ma hi gbo fothok mbol!} Dont just lie to us! \textit{Nchen nhã fothok thɛm mɔ nɔthi mbol.} You shall not calumniate your friends (\citealt{Pichl1967}). 

\TCheadword[1]{fɔ} (Eng \textit{four}) \textit{cf}: \TClink{hiɔl}. \textit{nam} four. \textit{Standad fɔ lɔ m mɛkɛni?} You stopped standard four? \textit{Ai mɛkni mɛŋkɛ vɛ ŋɔ pɔ ŋɔ vellɛ standad fɔ.} I stopped that time as they used to call it standard four.

\TCheadword[2]{fɔ} (Eng \textit{for}) \textit{prep} for. \textit{Aa, kɛ bifo dat akoni che ko administreshɔn dɛ fɔ lɔŋg.} Yes, but before that I had been in administration for a while.

\TCheadword{fɔi} \textit{adj} easy. \textit{Ko gbemiɛ gbi ŋɔ nko gbemiɛ handɔ ŋɔ chaŋ mɔ che fɔi?} In all the deliveries you have delivered, which one was the easiest?

\TCheadword{fɔm} (Eng \textit{form}) \textit{n} form; class or grouping of pupils in a school. \textit{A kɔ lɔni pɛ haŋ ya ko kɔni fɔm wan, ya pɛ tipɛ kɔ hɔlide.} I did not go there again until I went to form one, then I started going for holidays again.

\TCheadword{fɔn} \textit{n} \textbf{1)} society. \textit{Chen bo wu ni pɔ kɔŋ wɔ, pɔ wɔ lemɛk gbal ifɔndɛ.} He would not just die and be buried, they would complete society rites for him (lit. pass the society boundary with him). \textit{Yɛ hu ifɔndɛ pɔ mɔi ka ilel Buɛ Hini?} When you were initiated is the time you were given the name Bue Hini? \textbf{2)} secret.

\TCheadword{fɔnifɔni} (Eng \textit{funny}) \textit{n} hijinks. \textit{Ŋa haŋa thi, yɛ ŋa ŋa fɔni-fɔniɛ vɛ, ŋa mam.} For black people, when you (pl) do amusing things then you (pl) laugh.

\TCheadword[1]{fɔnwɛi} (comp. of \TClink[1]{wɛi} (der. of \TClink[2]{wɛi}), see \TClink[1]{wɛi})

\TCheadword[2]{fɔnwɛi} (comp. of \TClink[1]{fɔnwɛi} (comp. of \TClink[1]{wɛi} (der. of \TClink[2]{wɛi}), see \TClink[2]{wɛi})

\TCheadword{fɔɔ-fɔɔ-fɔɔ} \textit{Idph} of panting. \textit{Kɔ bimni sɔku bullai, wɔ hɔɔl <fɔɔ fɔɔ fɔɔ> ni yeke wɔɛ che wɔn kunɔlɔ.} (She) went and bent over in one corner, she breathed <fɔɔ fɔɔ fɔɔ> (idph of panting) with the cassava (tucked) in her bosom.

\TCheadword{fɔrina} (Eng \textit{foreigner}) \textit{n} foreigners. \textit{Anya hiɛ fɔrina ŋaɛ, Koroma, Kalon, Sherif.} Our people are foreigners, Koroma, Kallon, Sheriff.

\TCheadword[1]{fɔs} \textit{cf}: \TClink{kuŋkuŋ}. \textit{v} \textbf{1)} knock. \textit{Hã fɔs kïl lɛ hɔl ko}. They knock at the door (\citealt{Pichl1967}). \textbf{2)} strike, hit (\citealt{Sumner1921}). \textit{Fɔs mi yaŋ ŋkumbɛ ni kɛnthi gbangba-m dɛ.} He struck me on my side and broke my rib (\citealt{Pichl1967}). \textit{Fɔs gbo mindɛ, hɔlɛ hɔ hok imam.} If you hit the nose, the tears will run from the eyes (proverb) (\citealt{TISLL1979}). \textbf{3)} tap. \textit{A che gbo pɔng silal yɛ ya fɔs mɔ thipɛpɛ lɛ, mma silini.} I was only joking when I tapped your shoulders, don't be annoyed (\citealt{Pichl1967}). 

\TCheadword[2]{fɔs} \textit{n} rubble. \textit{Ya fɔs gbo tutundɛ ha si ko lo igbimiɛ hɔ hok kaɛ.} If I heat the rubble, I will know where the heat comes from (proverb) (\citealt{TISLL1979}).

\TCheadword{fɔsa} (Port \textit{força} ‘force') \textit{cf}: \TClink{kugba}. \textit{n} \textbf{1)} (hɔ̃/-) strength (\citealt{Pichl1967}). \textit{A chaŋ mɔ fɔsa.} I am stronger than you (\citealt{Pichl1967}). \textit{Ya biɛn fɔsa.} I have no strength (\citealt{Pichl1967}). \textbf{2)} power. \textit{Fɔsa Hɔbatokɛ.} By the power of God. \textit{Mɔ bi fɔsa lɛ gbi.} You have all the power (\citealt{Pichl1967}). \textbf{3)} resources. \textit{Ya koŋ standad siks ɛ, bami ni yami ŋa ka biɛni fɔsaɛ ŋa kɔ che, yai kɔni Champ ko.} After I finished standard six, my father and my mother did not have the resources for me to go further, so I went to Freetown.

\TCheadword{fɔst} (Eng \textit{first}) \textit{cf}: \TClink{sen}. \textit{n} first. \textit{Tɛm ndɔ ŋɔ pɔ gbem mɔa? Naintin fɔti tu fɔst ɔf Januari.} When were you born? First of January, 1942.

\TCheadword{Fransis} \textit{nam} Francis, male name given to a person. \textit{Ba mi ka koŋ hu, pɔ wɔ velɛ Fransis Manli.} My father has died, they called him Francis Manley.

\TCheadword{frɔm} (Eng \textit{from}) \textit{prep} from. \textit{Frɔm ko lɔpɔ tipɛ haŋ ko lɔ pɔ mɛkniɛ.} From where it starts on to where it stops. \textit{Frɔm yɛ pɔ gbem mɔ?} Since you were born? \textit{Frɔm 2010 ŋɔ a lɔi ni administrashɔndɛ kunɛ.} In 2010 I entered the administration.

\TCheadword{fufu} \textit{n} doughlike dish made of boiled and pounded yam or cassava made into balls to be eaten with soups or stews. \textit{Ŋ kɔ sas fufu lɛ.} Go strain the fufu (\citealt{Pichl1967}). 

\TCheadword{funfun} \textit{v} \textit{fũfũ} or \textit{fuŋfuŋ} plant rice in a nursery for later transplanting (\citealt{Pichl1967}). \textit{Pəlɛ lɛ kɔ ya fũŋ-fũŋ hɔ lɛ kɔ si che hã yuk paŋ Gbiminte lɛ.} The rice that I planted temporarily (in a nursery) will do for transplanting in the month of July (\citealt{Pichl1967}). \textit{Pɔ koŋ kɔ gbo bɛ bɛkthai – pimdɛ kɔ pɔ bia fun-fun kai – pɔ kɔi wo tokɛko.} After putting it in bags - the other ones they will plant (in the rice nursery) - they will send it up top. 

\TCheadword{fufuŋ} \textit{n} (hɔ̃/tha) lungs (\citealt{Pichl1967}). 

\TCheadword{Fuŋk} \textit{nam} Funkia, name of a town south of Freetown near the Atlantic coast of the Sierra Leone peninsula. \textit{Fuŋg ko.} In Funkia.

\TCheadword{fup} \textit{Idph} of defecating (K dialect). \textit{Nchíndɛ̀ mà hónì <fup>.} The shit came out like <fup>.

\TCheadword{Furabe} \textit{nam} Fourah Bay College, University of Sierra Leone. \textit{Boon dɛ kɔ che parɛ Furabee Kɔlɛj kɔ koŋ sẽyni.} The meeting which was recently at Fourah Bay College has dispersed (\citealt{Pichl1967}). 

\TCheadword{futh} \textit{cf}: \TClink{lɛnthi}, \TClink[1]{sokothi}, \TClink{suth}, \TClink[2]{wɔ}. \textit{v} \textbf{1)} uproot (Shenge pronunciation of \textit{suth}) (\citealt{Pichl1967}). \textit{Pɔ koŋ kɔ gbo futh, pɔ kɔi panth thiban pɔ woth kɔ bolɛ.} After they have uprooted it, they tie it in sheaves and carry it on their heads. \textbf{2)} take root, \textit{futh pɛlɛ} root rice (B dialect). \textit{Wɛl tɛm dɛ vɛ yɛ pɔ kɔ hun lɛli labo kɔ ko mɔi futhɛ.} At that time they will come to see if it has formed roots.

\TCheadword{futhul} (der. of \TClink{thu}) \textit{v} spit, spit or throw poison (\citealt{Pichl1967}). \textit{Kər lɛ wɔ futhul.} The snake spit its poison (\citealt{Pichl1967}). \textit{Yɛ bi ni mfuthul mi-a?} Why do you spit on me? (\citealt{Pichl1967}) 

\end{letter}
\begin{letter}{(G)}

\TCheadword{gadin} (Eng \textit{garden}) \textit{n} (hɔ̃/tha) garden (\citealt{Pichl1967}). \textit{Bi pɛ gadin bom, gadin nthɔthɔɛ.} He also has a large garden, an oil-palm garden. \textit{Killɛ bɔɔko gaadin hɔ̃ lɔ}. There is a garden outside the house (\citealt{Pichl1967}). 

\TCheadword{Gananɔ} \textit{nam} (wɔ/hã, a) Ghanaian, person from Ghana (\citealt{Pichl1967}). 

\TCheadword{gaŋgaŋ} \textit{Idph} of “bluffing,” someone walking with big strides, arms extended to the sides (“akimbo”) (K dialect). 

\TCheadword{gari} \textit{n} farina.

\TCheadword{gɛti} (Eng \textit{get}) \textit{v} get. \textit{Ŋgɛtiɛ malɔ gbo mɔ bɛ nton.} If you have groundnut, add a little. \textit{Nsɔthɔni gbo ŋgɛtiɛ mɔi bɛ ogiɛ.} If you don't have groundnut, you add ogiri.

\TCheadword{goŋgɔɔ} \textit{n} (kɔ/ma) grass species (Pennisetum subangustum) (\citealt{Pichl1967}). 

\TCheadword{Gɔdɛ-Gɔdɛ} \textit{nam} Gode-Gode, a Bondo mask. 

\TCheadword{gɔvamɛnt} (Eng \textit{government}) \textit{n} (hɔ̃/ma) government (\citealt{Pichl1967}). \textit{Labi gɔvamɛntɛ ŋɔ wɔɛ nɔ mɔ le telɛ pɛŋ mɔ hɔ mu di Bondo.} That is why the government says we should wait before we initiate Bondo. textit{Gɔmɛnt lɛ hã thoŋkiɛ lɛ hã yema hã saba, che lɛ tamɔ pokan gbi wɔ koŋ huth lɛ, wɔ hã paka pɔn bul hã bol wɔ lɛ.} The government has proclaimed that they want to make a law that every young man who has come of age has to pay one pound as a head-tax (\citealt{Pichl1967}).

\TCheadword{gɔvana} (Eng \textit{governor}) \textbf{1)} \textit{n} (wɔ/hã, a) governor (\citealt{Pichl1967}). 

\TCheadword{Grasfil} \textit{nam} Grassfield, name given to a place. \textit{Hin fli woŋgo hin ko ibi kil, Grasfil, kil hin ramdɛ.} We ourselves, we have a house at Grassfield, our family house. \textit{Hin pɛ grasfil Abolomaɛ, ŋalɔ agbei.} In Grassfield, the Sherbro are many.

\TCheadword{Gres} \textit{nam} Grace, name given to a person. \textit{Wɔ Bɔima Gres.} She is Boima Grace.

\TCheadword{griin} (Eng \textit{green}) \textit{adj} green (\citealt{Pichl1967}). 

\TCheadword{Grika} \textit{nam} Greeks, people from Greece. \textit{Simi-njɛm bo̹m hɔ̃ kong duk Sayprɔs Agriika lɛ thiye̹ng aña Thɔɔki lɛ.} A big misunderstanding has been created (befallen) in Cyprus between the Greeks and the Turks (\citealt{Pichl1967}). 

\TCheadword{guda} \textit{n} big basket for holding fish, also used in Themne and Mende (K dialect). 

\TCheadword{Guɛ} \textit{nam} (wɔ/-) Gwe, name given by Poro Society (\citealt{Pichl1967}). 

\TCheadword{guma} \textit{n} (hɔ̃/tha) fence, enclosure where Bondo girls are initiated; enclosed place (\citealt{Pichl1967}).

\TCheadword{gwava} (Port \textit{goiaba} ‘guava') \textit{n} (kɔ/ma) guava (\citealt{Pichl1967}).


\end{letter}
\begin{letter}{Gb}

\TCheadword[1]{gba} \textbf{1)} \textit{adj} [gbá] different (K dialect). \textit{Ki hɔ̃ gba.} This is something different (\citealt{Pichl1967}). \textit{Ŋɔ kache gba.} It was different. \textbf{2)} \textit{adv} separately. \textit{A wɔnɛ ko lem gba.} I will discuss this one separately. 

\TCsubword{gbagba} (der.) \textit{adj} different. \textit{Kache lanbɛ la gba-gba.} In the past it was different.

\TCheadword[2]{gba} \textit{cf}: \TClink{bɔɔ}. \textit{n} [gbà] hat (K dialect); \textit{gbaa} (hɔ̃/tha) hat, kind of helmet (\textit{gba ndoŋ} ‘golden hat' used in the Sherbro Hymn Book to translate crown) (\citealt{Pichl1967}). 

\TCheadword[3]{gba} \textit{cf}: \TClink{gbalɔ}, \TClink{kɛbi}. \textit{n} [gbà] smithy, where blacksmiths make cutlasses (K dialect); \textit{gbaa} (lɔ/-) forge; smithy (\citealt{Pichl1967}).

\TCsubword{gbaalɔ} \textit{cf}: \TClink{kɛbi}. \textit{n} (hɔ̃/tha) forge; smithy (\citealt{Pichl1967}). 

\TCheadword[4]{gba} \textit{v} [gbá] fasten, adhere, stick (K dialect); \textit{gbaa} fasten; stick fast (\citealt{Pichl1967}). \textit{Aftabakɛ ŋɔ hun gba ki <gbiŋ>, blidin iŋɔi huŋyi ki fip.} The afterbirth came and really got stuck <gbiŋ>, then bleeding burst out badly. \textit{Yɛ hã kən dɛ ba hã gbaa gbo ayena libul lɛ.} When they observed that their father stuck fast in the one place (\citealt{Pichl1967}). comp. \TClink{kilgbakɛ} (see \TClink[1]{kil}) 

\TCsubword{gbaalak} (der.) \textit{n} \textit{ŋgbaalak} (ma) writing, script (\citealt{Pichl1967}). 

\TCheadword{gbaalak} (der. of \TClink[4]{gbal}, \TClink{-k}, see \TClink[4]{gbal}) 

\TCheadword{gbaalɔ} (der. of \TClink[3]{gba}, \TClink[8]{lɔ}, see \TClink[3]{gba})

\TCheadword{gbaap} \textit{n} (wɔ/hã, N) fish species, black snapper (Lethrinus atlanticus) (\citealt{Pichl1967}). 

\TCheadword{gbabaŋ} \textit{n} (?/?) deafness (\citealt{Pichl1967}). 

\TCheadword{gbabi} [gbàbì] \textit{n} fish species, two types, one has a long neck (K dialect). 

\TCheadword{gbagba} (der. of \TClink[1]{gba}) 

\TCheadword{gbagbak} \textit{n} \textit{ŋgbagbak} (ma) paralysis, trembling in the hands (\citealt{Pichl1967}). \textit{Koŋ wɔ naka ŋgbagbak lɛ ma wɔ, nlɛliɛ pia wɔ lɛ hɔ̃ pakïl lɛ.} Kong is suffering from paralysis, see how his hand trembles (\citealt{Pichl1967}). 

\TCheadword{gbaha} \textit{v} greet someone returned from a long journey abroad. \textit{Yi kɔ gbahã ba hĩ ka kɔn bias gbath vil ni koŋ moey.} We go to welcome our father who went on a journey long ago and has returned now (\citealt{Pichl1967}). 

\TCheadword{gbak} \textit{n} (kɔ/ma) vegetable species, kind of garden egg (\citealt{Pichl1967}). 

\TCheadword{gbaka} \textit{n} \textit{igbaka} (hɔ̃/-) laughter, \textit{kɛnthi igbaka} laugh loudly and heartily (\citealt{Pichl1967}). comp. \TClink{Bondogbaka} (see \TClink[1]{Bondo}) 

\TCheadword{gbakayao} [gbàkàyáó] \textit{n} bird species, several species who build their nests from the feathers of other birds (like \textit{saagbaama}), will chase hawks or any bird that gets near the nest, it will bother you whether you take an egg or don't take an egg (K dialect). \textit{Koŋ nɔmi pel thigbakaɛ.} You have found the eggs of the bad-heart bird (proverb) (\citealt{TISLL1979}). 

\TCheadword{gbaki} \textit{v} \textbf{1)} answer, reply (\citealt{Pichl1967}). \textit{Anyalɛ ŋae gbaki ŋa hɔɛ, “Awa la likɛlɛŋ; hi sɛiŋsɛiŋnia.”} The others answered and said, “Okay, it is good; let us scatter.” \textit{Kaiŋ Taso wɔe gbaki ni hɔɛ, “Yeŋkɛlɛŋba abɛna mi.”} Kain Tasso answered, “Very well, my elders.” \textbf{2)} explain. \textit{Lɛ pɔ iyɛ wɔ gbo, ŋɔ nɔ ki che mɔ pa gbaki yia, mɔi wɔ a chelani pa gbaki yeŋkɛlɛŋ.} If they ask this person how he was responding to you, then you say he was not explaining it well. \textbf{3)} reply. \textit{Bɛl Maaɛ wɔe gbaki ni hɔ ko poo wɔɛ, “Ndɛli thumɔɛɛ.”} Rat Wife replied to her husband, “Look at the dog.”

\TCsubword{gbakia} (der.) \textit{n} answer. \textit{Bahin wɔ bi gbakia.} Jesus has the answer.

\TCheadword[1]{gbako} \textit{v} \textbf{1)} grow. \textit{Boŋ cheki, ma koŋ gbako, wɔn pɛ lɔ ni sɔvaiv.} Now, they have grown (the oil palms), he is there and lives (off of it). \textbf{2)} be grown up, be old. \textit{Lɔn lɔi le te hi koŋ gbako.} We stayed there until we were grown. \textit{Ka lɔ pɔ dumɔ mi te a koŋ gbako.} I was raised here until I was grown. \textit{Mpanthɛ ma ŋaɛ, frɔm yɛ wɔ ka che ko tallɛ haŋ koŋ gbako.} It is the work he has done since his childhood until now that he is old.

\TCheadword[2]{gbako} \textit{n} elders. \textit{Bɛɛ tirɛ ni ŋgbako ma tirɛ ŋae wom ha vel Kaiŋ Taso.} Then the town chief and the elders summoned Kain Tasso. \textit{Yɛ lanɔ ki la koŋ chaŋ dɛ, abɛɛ-aɛ ni ŋgbakoɛ ŋae vel Kaiŋ Taso ha thoŋka wɔ.} After this affair had passed, the chief and elders called Kain Tasso to judge him. 

\TCheadword{gbakra} \textit{cf}: \TClink[3]{bɛŋ}, \TClink[3]{chal}, \TClink{chɛm}, \TClink{gbala}. \textit{n} (hɔ̃/tha) chair made of sticks (\citealt{Pichl1967}).

\TCheadword[1]{gbal} \textit{n} (kɔ/ma) tree species, wild plum tree (Parinari excelsa) (\citealt{Pichl1967}). 

\TCheadword[2]{gbal} \textit{n} (kɔ/ma) cloth strip about 4-5 inches wide (\citealt{Pichl1967}). 

\TCheadword[3]{gbal} \textit{cf}: \TClink[2]{tham}. \textit{v} write. \textit{Rai hɔ pə gbal ka thənkɔ.} It is on paper one writes with a pen (\citealt{Pichl1967}). 

\TCheadword[4]{gbal} \textit{n} \textbf{1)} writings. \textit{Ni mgballɛ gbi maiko koiyɛ, ɪthaiɛ, yen-o-yen.} And all the writings we have taken, the proverbs, everything. \textbf{2)} line. \textit{Inan gballɛ, ilɔ pɛŋgipɛŋgi, i kikkik.} We draw the line, we jump there (and) kick. \textbf{3)} mark. \textit{Ye pɔ koyi kaŋdɛ pɔɛ nkegbo nɔɛ bi gballɛ kɔ ko kunwɔɛ as Sizaɛ…} When we were taught, they said if you see a mark on the belly like from a Cesarean section…

\TCheadword{gbala} \textit{cf}: \TClink{gbakra}, \TClink{tike}. \textit{n} log. \textit{Thɔŋ chiɛ mi gbala woth ko wɔn thipɛpɛ.} Thong brought a long [piece of] wood for me, he carried it on his shoulders (\citealt{Pichl1967}). 

\TCheadword{gbalɔni} (comp. of \TClink{gbani}, \TClink[1]{lɔ}, see \TClink{gbani}) 

\TCheadword{gbam} \textit{n} \textbf{1)} [gbàm] potato (B dialect); (kɔ/ma) potato (Ipomea batatas) (\citealt{Pichl1967}). \textbf{2)} potato greens. \textit{Win lɛ pɛ sallɛ mɔi gbo asaŋ keŋkendɛ a yuk gbamdɛ.} For us, when rainy season comes, I plant krain-krain, (and) I plant potato leaves. \textit{Mɔ yi hun toŋgi ŋɔ pɔ chɛth pɔmthi gbamdɛ.} You should now come and show us how to cook potato leaves.

\TCsubword{gbamsa} (comp.) \textit{n} (kɔ/ma) red kind of sweet potato (\citealt{Pichl1967}).

\TCheadword{gbamfa} \textit{cf}: \TClink[2]{bɛk}, \TClink[1]{kɔ}. \textit{n} (hɔ̃/tha) quiver, bag (\citealt{Pichl1967}). 

\TCheadword{gbampɔ} \textit{n} mullet (K dialect); (wɔ/hã, N) bonga fish (\citealt{Pichl1967}). \textit{Yɛlɛ kɔa ŋɔthɛ, gbampɔɛ kɔa ŋɔthɛ.} It is the big sea fishing that I do, I fish for mullet. comp. \TClink{pɛlgbampɔ} (see \TClink[2]{pɛl}) 

\TCheadword{gbamsa} (comp. of \TClink{gbam}, \TClink[1]{sa}, see \TClink{gbam}) 

\TCheadword{Gbana} \textit{nam} (wɔ/-) Gbana, name given by Poro Society (\citealt{Pichl1967}). 

\TCsubword[1]{gbanabom} (comp.) \textit{n} \textbf{1)} (wɔ/hã, a) interpreter for society spirit who appears as a dancing masquerade (\citealt{Pichl1967}). \textbf{2)} (wɔ/hã, a) first novice in the Poro Society; court crier (\citealt{Pichl1967}). 

\TCheadword[1]{gbanabom} (comp. of \TClink{Gbana}, \TClink{bom}, see \TClink{Gbana}) 

\TCheadword[2]{gbanabom} \textit{n} secret. \textit{Nɔ sini gbanabom mɛnɛ.} No one knows the secret of the grave (proverb) (\citealt{TISLL1979}). 

\TCheadword{gbandeŋ} \textit{n} [ŋgbándéŋ] leprosy (K dialect). 

\TCheadword{gbani} \textit{v} \textbf{1)} [gbánì] lean (something) against (something) (K dialect). \textbf{2)} walk next to, along river looking for a place to ford (K dialect). \textit{Gbani kɔ chendɛ he hal.} Walking beside the sea is not crossing it (proverb) (\citealt{TISLL1979}). 

\TCsubword{gbalɔni} (comp.) \textit{v} lean (yourself) against (separable ‘there' pronoun) (K dialect).

\TCheadword{gbanɔ} \textit{n} (hɔ̃/tha) channels between the banks (Turtle Islands) (\citealt{Pichl1967}). 

\TCheadword{gbanthima} \textit{n} \textit{ŋgbanthima} (ma) neighboring country (?Gbandi land) (\citealt{Pichl1967}).

\TCheadword[1]{gbaŋ} \textit{n} \textit{igbang} (hɔ̃/ma) grass species, kind of grass similar to sugar cane (\citealt{Pichl1967}). 

\TCheadword[2]{gbaŋ} \textit{v} smoke (fish), spread fish out to be smoked. \textit{Beo, a bo pin agbaŋ ŋa.} No, I just buy and smoke them.

\TCsubword{gbaŋk} (der.) \textit{v} smoke. \textit{Mpanthɛ gbi ma mɔ ŋaɛ wok ka ko pindɛ haŋ gbaŋkɛ.} All the work you do starting from the buying up to the smoking. 

\TCheadword[1]{gbaŋba} \textit{n} rib, [baŋba]/[\`{m}báŋbá] rib/the ribs (B dialect); \textit{gbangba} (kɔ/ma) rib (\citealt{Pichl1967}). 

\TCheadword[2]{gbaŋba} \textit{n} \textit{gbangbæɛ} (hɔ̃/tha) large flat rock in streams (\citealt{Pichl1967}). 

\TCheadword{gbaŋga} \textit{v} be put into. \textit{Pɔ gbaŋga wɔ bo pothɛ atok, pɔi nu bikɛ pɔ bim wɔ lɔ atok.} After he would be put in the ground, they would fold the mat then they would put the corpse on it.

\TCheadword{gbaŋgbaŋ} \textit{n} bird species, flies but very light, rocks when it flies, has a large beak (K dialect); \textit{gbangban} (wɔ/hã, N) bird the size of a pigeon but longer beak (\citealt{Pichl1967}). 

\TCsubword{gbaŋgbaŋsasa} (comp.) \textit{cf}: \TClink{saŋka}. \textit{n} bird species, kingfisher (K dialect); (wɔ/hã, N) common hornbill (Lophoceros fasciatus) (\citealt{Pichl1967}). 

\TCsubword{gboŋgboŋploplo} (comp.) [gbóŋgbóŋplòplò] \textit{n} bird species, colorful green bird with a long tail, black streak through eyes, male has yellow chin fading into burnt amber, long slightly curved black bill, has a chirp, flies around as if performing a mating display (K dialect). 

\TCsubword{gboŋgbotho} (comp.) [gbóŋgbóthó] \textit{n} bird species, pelican (K dialect); \textit{gbongbootho} (wɔ/hã, N) pelican (Pelecanus rufus) (\citealt{Pichl1967}). 

\TCheadword{Gbaŋgbaya} \textit{nam} Gbangbaya, name given to a place. \textit{Timp lɛ Gbaŋgbaya ko ntɛnt kɔ tokɛ.} The cliff near Gbaŋgbaya is high (\citealt{Pichl1967}). 

\TCheadword{gbaŋk} (der. of \TClink[2]{gbaŋ})

\TCheadword{gbaŋkthani} \textit{v} wrap a cloth around the shoulders or body (\citealt{Pichl1967}). \textit{Tamɔ toon dɛ wɔ gbankthani kotha kathïl bo̹o̹m mɛ nɔ bɛn.} The small boy wrapped the big kenthe cloth around himslef as if he were a big man (\citealt{Pichl1967}). 

\TCheadword{gbaŋkveleŋ} (unspec. of \TClink[1]{veleŋ}) 

\TCheadword{Gbaŋɡbato} \textit{nam} Gbangbatok, name given to a place. 

\TCheadword{gbaŋtha} \textit{n} \textit{igbangtha} (hɔ̃/ma) unripe nut of the oil palm (\citealt{Pichl1967}).

\TCheadword{gbap} \textit{n} (hɔ̃/tha) small mat kept near the door, used to take out the sweepings (\citealt{Pichl1967}). 

\TCheadword{gbasa} \textit{n} \textbf{1)} (hɔ̃/tha) hankerchief [headtie?] (\citealt{Pichl1967}). \textbf{2)} (hɔ̃/tha) headdress of Bondo girls (\citealt{Pichl1967}).

\TCheadword{gbata-gbata} (unspec. of \TClink[1]{gbath})

\TCheadword[1]{gbath} \textit{temp} time. \textit{Braima koŋ haa lanɔ ki ha gbaath vil.} Braima had done this for a long time. \textit{Yi kɔ gbahã ba hĩ ka kɔn bias gbath vil ni koŋ moey.} We go to welcome our father who went on a journey long ago and has returned now (\citealt{Pichl1967}). \textit{Ya kɔ tɛmɛni gbath lo hɔ̃ kath.} I go to strive for myself, the times are hard (\citealt{Pichl1967}). 

\TCsubword{gbata-gbata} (unspec.) \textit{adv} finally. \textit{So, mɔm, nɛnthi wɔ tha nshi niɛ lɛ Plantiɛ ki kɔ bia koi, ŋa kɔn, wok ka ni, gbata-gbata?} So do you know how many years it will take (before) Plantain (Island) disappears from here finally? 

\TCheadword[2]{gbath} \textit{v} slap with the palm of the hand (\citealt{Pichl1967}). 

\TCsubword{gbathil} (unspec.) \textit{cf}: \TClink{siibii} (unspec. of \TClink{siil}). \textit{n} \textit{ŋgbathïl} (ma) punishment; suffering, trouble (\citealt{Pichl1967}). \textit{Ŋgbathïl ma hwɛlɔ lɛ ma mɛkini.} The suffering of the world is finished (\citealt{Pichl1967}). \textit{Mathin hĩ lɛ hink ŋgbathïl gbi.} Our shelter from all the troubles (\citealt{Pichl1967}). \textit{Ŋgbathïl chen tal.} Trouble has no importance (\citealt{Pichl1967}). 

\TCheadword[1]{gbato} \textit{n} (kɔ/ma) tree species, tree with flowers resembling chestnut candles (Pterocarpus santalinoides) (\citealt{Pichl1967}). 

\TCheadword[2]{gbato} \textit{cf}: \TClink{boka}, \TClink{gbatɔ}. \textit{n} (kɔ/ma) cat o' nine tails (\citealt{Pichl1967}). 

\TCheadword{gbatɔ} \textit{cf}: \TClink{boka}, \TClink[2]{gbato}. \textit{n} cutlass, [gbatɔ]/[mgbàtɔ́] cutlass/the cutlasses (B dialect); \textit{gbathɔɔ} (kɔ/ma) cutlass; machete (\citealt{Pichl1967}).

\TCheadword{gbe} \textit{cf}: \TClink[1]{no}, \TClink{pɔs}. \textit{quant} \textbf{1)} many. \textit{Mɔm mbi ja gbe ŋa ŋanɛ ŋa hunɔni muɛ ŋa ŋan si.} You have many things for those that have not yet come to know. \textit{Ligbe ba la hun ni ŋɔ pɔ vellɛ, ŋɔi hɔni Mpothoai ɛ rilijɔndɛ la ko hundɛ, Kristiandɛ.} Many things have happened in what we call in English religion, Christianity. \textbf{2)} a lot. \textit{Pɔki Salon dɛ, pɔ ko ha jagbe.} In our country Sierra Leone, they have done a lot. \textit{Bolomnɔɛ wɔn wɔ bi ndum, yemani theliaŋ gbe.} The Sherbro man has good character, he does not like much talking. \textbf{3)} plentiful. \textit{Mith lɛ ko che gbe we.} Hatred is plentiful now. \textit{Yɛlaio wɛ, yɛ jaɛ ma ko ŋani mgbeɛ ŋɔ maredɛ kɔ bi ni prɔblɛm thɛ.} Nowadays, when things are abundant, all the marriages are full of problems.

\TCsubword{gbeba} (der.) \textit{quant} enough. \textit{Pɔ bi ha di naa tri thi bom-bom dɛai gbi, ni pɔ ya yenjo gbeba ha nɔ-o-nɔ jo.} People would have to kill cows in all the big towns and cook enough food for everyone to eat (\citealt{Sumner1921}). 

\TCsubword{gbegbe} (der.) \textit{quant} many. \textit{Kɔnɛ lɛŋa hun gbo, ŋa koi ndumdɛ ma hiŋka biɛ, ja gbe-gbe la ma lɔ che.} Please when they would come, they should take the character we had, they should be involved in many things.

\TCsubword{gberba} (der.) \textit{quant} \textbf{1)} many. \textit{Ŋae hun che billai ha nɛn thigberba.} They have been married for many years. \textit{Isɔ lan dɛ vɛ, amaa agberba ŋa diklɛni boe ko lɔ pɔ bɛmpa yenjoo si pɔ wɔm bɛ hok sakaɛ.} That morning, many women gathered in the kitchen where they prepared food before they made the sacrifice. \textbf{2)} much. \textit{Fe ki, gberba ŋɔ raɛ thifeai.} Much of this money is in paper. \textit{Bɛlsɛ koŋ thɛŋk feɛ gberba.} The rats have taken plenty of money away.

\TCsubword{gbergber} (der.) \textit{n} many different kinds. \textit{Thumsi lɛ hã gbe̹rgbe̹r.} There are many kinds of sharks (\citealt{Pichl1967}). \textit{Yenjo hɔ̃ gbe̹rgbe̹r.} Food is of different kinds (\citealt{Pichl1967}). 

\TCsubword{ligber} (der.) \textit{temp} \textbf{1)} many times. \textit{Ya koŋ wɔ məəy ligbe̹r kɛ theyɛni lo̹m mi lɛ.} I warned him many times but he did not listen to my words (\citealt{Pichl1967}). \textit{Koŋ kɔ ligbe̹r bondɔ ko mɔ̃ekə waŋ dɛ hɔ̃ ki.} He has gone many times to the wharf, till now it is the tenth time (\citealt{Pichl1967}). 
\textbf{2)} often. \textit{Ligbe̹r yi pey imam hĩ lɛ.} Often we shed our tears (\citealt{Pichl1967}). 

\TCsubword[2]{gbet} (unspec.) [gbét] \textit{Idph} of water all gone (K dialect). \textit{Mà kóŋ wɔ̀ <gbə́t>.} It (water) is all gone <gbét>. 

\TCheadword{gbeba} (der. of \TClink{gbe}, \TClink[2]{ba}, see \TClink{gbe}) 

\TCheadword{gbegbe} (der. of \TClink{gbe}) 

\TCheadword{gbei} \textit{cf}: \TClink{ku}, \TClink[1]{vel}. \textit{v} \textit{gbẽy} or \textit{gbẽyŋ} holler, call (\citealt{Pichl1967}). \textit{Ŋgbẽy wɔm dɛ.} Call the canoe (\citealt{Pichl1967}). 

\TCheadword{gbekte} \textit{n} \textit{ŋgbe̹kte} (ma) handcuffs (\citealt{Pichl1967}). \textit{Pɔ bɛ wɔ ŋgbekteɛ ni pɔ sɛmi wɔ bai ko anyaɛ gbi chee lɔ pɔ bi ha thoŋka wɔ.} They put him in handcuffs and brought him to the bari in front of all the people where they will judge him.

\TCheadword{gbekthe} \textit{n} Bondo drum (can be heard in project video but not seen) (B dialect); ibethe (hɔ̃/-) Bondo drum (\citealt{Pichl1967}).

\TCheadword[1]{gbel} \textit{cf}: \TClink{hathog}, \TClink{yentho} (comp. of \TClink[1]{yen}, \TClink[2]{tho}). \textit{n} leopard (K dialect); (wɔ/hã, si) leopard (taboo name) (usually called \textit{yen tho} or \textit{hã-thoɛ}) (\citealt{Pichl1967}). comp. \TClink{thumgbel} (see \TClink{thum}) 

\TCheadword[2]{gbel} \textit{n} (kɔ/ma) small canoe (\citealt{Pichl1967}). 

\TCheadword{gbeleŋ-gbeleŋ} \textit{Idph} of ringing out, typically of a bell but anything that reverberates in such a manner, same as in Mende (K dialect). 

\TCheadword{gbem} \textit{v} \textbf{1)} bear; bring forth (\citealt{Pichl1967}). \textit{Wɔn kɛn wɔ gbem ŋan awaŋni tindɛ?} Is she the only one that gave birth to the twelve of you? \textit{Tɛnɛni, tɛnɛni, tɛnɛni, kɛ ya wɔ gbem mɔ we.} Remember, remember, remember that your mother gave birth to you. \textit{Lanɛ la li kɛlɛŋ, lɛ bɛn mɔi wɔ mɔ gbo ntɛnt, mɔ ha suthra wɔ, mɔ ha toŋgiɛ lɛ wɔ gbem mɔ.} That is what is good, if your parent is near you, you should try to show that she gave birth to you. \textit{Ŋa ni lamgbantho ki ŋa chalao wɛ, ŋaŋa gbem apumma mɛn do wɛ?} You (pl) and this man you're living with, are you the ones that gave birth to (are you the parents of) these five (children)? \textbf{2)} be born. \textit{Wɔn ndɔ pɔ gbem wɔa?} Where was she born? \textit{Yaa Ansu kɛ ilel mi gbem kaɛ ŋɔ Baki.} I am Ansu, but my birth name is Baki. \textit{Ya mi ka lɔ pɔ gbem wɔɛ, bawɔ ilel wɔ ŋɔ Pa Baro.} My mother was born here, her father's name was Pa Baro. comp. \TClink{lagbem} (see \TClink[2]{laa}) 

\TCsubword{thɔkɔtokgbemɔ} (comp.) \textit{n} moment of childbirth (\citealt{Pichl1967}). 

\TCsubword[1]{gbemi} (der.) \textit{n} \textbf{1)} child-bearing. \textit{Ŋa hunyi toŋgi ja gbeme.} To come and tell us about child bearing. \textit{Nɛn lan agbemenimu.} That year I had not started having children yet. \textbf{2)} delivery. \textit{Ko gbemiɛ gbi ŋɔ nko gbemiɛ handɔ ŋɔ chaŋ mɔ che fɔi?} In all the deliveries you have delivered, which one was the easiest?

\TCsubword[2]{gbemi} (der.) \textit{v} \textbf{1)} to assist giving birth; deliver. \textit{Kɛ ahindɛ ŋa nko gbemiɛ ŋan gbi nshiŋa?} But the people you have delivered, do you know them all? \textit{Ko gbemiɛ gbi ŋɔ nko gbemiɛ handɔ ŋɔ chaŋ mɔ che fɔi?} In all the deliveries you have delivered, which one was the easiest? \textbf{2)} give birth. \textit{Ŋa gbemi?} Did you both have children? der. \TClink{gbenik} (see \TClink{gbem})

\TCsubword[1]{gbemni} (der.) \textit{n} inheritance. \textit{Gbe̹mni abəka lɛ ni nche ma hã lɛ ma fəsɛ hã ma apotoa.} The inheritance and the way of life of the Krios resemble those of the Europeans (\citealt{Pichl1967}). 

\TCsubword[2]{gbemni} (der.) \textit{v} be born. \textit{Wɔn pɛ gbemni Bɔmɔtok ko?} He was also born in Bomotoke (Timdale Chiefdom)? \textit{Ya gbemni Nyemɔko, Mamu Sɛkshɔn, Bompɛ Chifdɔm, Mɔyamba Distrikt.} I was born in Moyeamoh, Mamu Section, Bumpeh Chiefdom, Moyamba District.

\TCsubword{gbenik} (der.), (der. of \TClink[2]{gbemi}) \textit{cf}: \TClink{gbɛthɛhɔl}, \TClink{kun}, \TClink{tem}. \textit{n} \textit{thigbenïk} (-/tha) womb (\citealt{Pichl1967}). 

\TCsubword{gbema} (unspec.) \textit{v} be barren (also adj) (\citealt{Pichl1967}). \textit{Nɔma lɛ wɔ gbema.} The woman is barren (\citealt{Pichl1967}). 

\TCsubword{gbemɔ} (unspec.) \textit{v} bring into the world; give birth (\citealt{Pichl1967}). \textit{Ka gbe̹mɔ ya wɔ Mari lɛ.} Who was born of his mother Mary (\citealt{Pichl1967}). \textbf{2)} propagate, tiller, grow shoots (e.g. of rice). \textit{Yɛ kɔ koŋ gbemɔɛ, kɔ koŋ gbo kɔi hun dri.} After the rice has tillered, it will ripen. \textit{Pɛlɛ kɔi pith kɔi piŋgi, kɔi bi kun, kɔi gbemɔ.} The rice will get dark, and then it will change and swell up (lit. have a belly, i.e. be pregnant) and then tiller.

\TCsubword{ligbem} (unspec.) \textit{cf}: \TClink{taro}. \textit{n} (wɔ/hã) descendant (\citealt{Pichl1967}).

\TCheadword{gbema} (unspec. of \TClink{gbem}) 

\TCheadword{gbemani} \textit{cf}: \TClink{gbesani}, \TClink[1]{hɔ}, \TClink[1]{lem}, \TClink[1]{taŋ}, \TClink{theli}, \TClink{wɛ}, \TClink[2]{wɔni} (der. of \TClink[1]{hɔ}, \TClink{-ni}). \textit{v} talk, cry (\citealt{Pichl1967}). 

\TCheadword{gbemaŋ} \textit{n} \textbf{1)} (kɔ/ma) fruit (\citealt{Pichl1967}, \citealt{Sumner1921}). \textbf{2)} fruit trees. \textit{Potohɔl lɛ koŋ mõɛy, ngbe̹maŋ dɛ tipɛ wantïŋ.} When springtime has come, the fruit trees begin to blossom (\citealt{Pichl1967}). 

\TCheadword[1]{gbemi} (der. of \TClink{gbem}, \TClink[1]{-i}, see \TClink{gbem}) 

\TCheadword[2]{gbemi} (der. of \TClink{gbem}, \TClink[1]{-i}, see \TClink{gbem})

\TCheadword[1]{gbemni} (der. of \TClink{gbem}, \TClink{-ni}, see \TClink{gbem}) 

\TCheadword[2]{gbemni} (der. of \TClink{gbem}, \TClink{-ni}, see \TClink{gbem}) 

\TCheadword{gbenik} (der. of \TClink[2]{gbemi} (der. of \TClink{gbem}, \TClink[1]{-i}), \TClink{-k}, see \TClink{gbem}) 

\TCheadword{gbenɔ} \textit{n} \textbf{1)} [gbə̀nɔ̀] sister-in-law, [gbə̀nɔ̀]/ [\`{m}gbə́nɔ̀] sister-in-law/ sisters-in-law (B dialect). \textbf{2)} (wɔ/-) daughter-in-law, as addressed by her father-in-law (\citealt{Pichl1967}). comp. \TClink{wɔŋgbenawi} (see \TClink{wɔŋ}) 

\TCheadword{gbeŋ} \textit{n} (kɔ/ma) bracelet (\citealt{Pichl1967}).

\TCheadword[1]{gbeŋgben} \textit{v} clear (B dialect). \textit{Pɔ koŋ gbo, pɔi gben-gben.} When we have finished, we clear the land. \textit{Pɔ gben-gben ko lɔ pɔ koŋ thɛ.} You clear the burnt area.

\TCheadword[2]{gbeŋgben} \textit{cf}: \TClink{gbɛlɛŋ}, \TClink[1]{lɛli} (comp. of \TClink[3]{lɛ}), \TClink[1]{thunɔ}. \textit{v} research, investigate. \textit{Pɔ la gben-gben, ahin lan ŋa ka che lɔ, ika bi bul Bompɛ ko, ni ibi bul Kagbɔ ka.} They would research that, there were people for that, we had one in Bumpeh and one in Kagboro. \textit{Ala amu gbɛŋgbɛŋ.} I will investigate it. \textit{Mbia gbɛŋgbɛŋ.} You would have to investigate that. 

\TCheadword[3] {gbeŋgben} \textit{n} \textbf{1)} ant species, stinging ant, black, lives in the ground, uses a small hole as entrance (K dialect); \textit{gbeŋgbeŋ} (wɔ/hã, N) ant species, very small black ant (\citealt{Pichl1967}). \textbf{2)} \textit{gbeŋgben} (wɔ/hã, N) ant species, male flying, white, edible, roasted and eaten, after one night's flight, loses its wings (Amithermus atlanticus) (\citealt{Pichl1967}). 

\TCheadword{gbeŋgbesese} (comp. of \TClink{gbɛŋgbɛ}) 

\TCheadword{gbeŋtheŋ} \textit{cf}: \TClink[1]{kɛnt}. \textit{n} wrist, [gbeŋtheŋdɛ] the wrist (K dialect). 

\TCheadword[1]{gber} [gbɝ́r] \textit{n} tree species whose fruit is used to make a sauce (K dialect). 

\TCheadword[2]{gber} \textit{cf}: \TClink[1]{boo}, \TClink[1]{pul}. \textit{n} (kɔ/-) unsoaked rice flour (syn. lɛkɛ) (\citealt{Pichl1967}). \textit{Ni apum ŋa nuputha mbana ndriɛ ni gbɛrɛ ha thóŋ bo.} And others mix ripe bananas with flour to fry.

\TCheadword[3]{gber} \textit{n} (kɔ/-) doughy, slippery kind of ocra sauce (\citealt{Pichl1967}).

\TCheadword{gberba} (der. of \TClink{gbe}, \TClink[2]{ba}, see \TClink{gbe}) 

\TCheadword{gbereeth} \textit{n} (hɔ̃/-) marrow (\citealt{Pichl1967}). 

\TCheadword{gbergber} (der. of \TClink{gbe}) 

\TCheadword{gbes} \textit{cf}: \TClink{buŋklipal} (unspec. of \TClink[1]{li-}, \TClink[1]{pal}). \textit{n} (kɔ/-) east (\citealt{Pichl1967}). \textit{Kɔn gbe̹s ko polo̹ŋ.} He is gone far away to the east (\citealt{Pichl1967}). 

\TCheadword{gbesani} \textit{cf}: \TClink{gbemani}. \textit{v} scold, bawl at (\citealt{Pichl1967}). 

\TCheadword[1]{gbet} \textit{v} \textbf{1)} \textit{gbe̹t} give someone a slight cuff on the head (\citealt{Pichl1967}). \textbf{2)} \textit{gbɛnth} hit someone accidentally with the pestle when pounding rice (\citealt{Pichl1967}). comp. \TClink{gbɛthɛk} (see \TClink[2]{thɔk}) 

\TCsubword{gbɛɛtigbɛɛti} (der.) \textit{cf}: \TClink{gbɛthɛk} (comp. of \TClink[1]{gbet}, \TClink[2]{thɔk}). \textit{v} strike. \textit{Bɛl Pokan dɛ: Ntelɛ mi ya hun, ni ya hun mɔ gbɛɛti-gbɛɛti bol.} Rat Husband: Wait, let me come knock you on the head.

\TCheadword[2]{gbet} (unspec. of \TClink{gbe}) 

\TCheadword{gbeta} \textit{v} fall.

\TCheadword{gbetha} \textit{cf}: \TClink[4]{kɛ}. \textit{v} \textbf{1)} swear. \textit{Pə ka gbetha wɔ ifɔŋ Toma lɛ.} They swore her on the Toma medicine (\citealt{Pichl1967}). \textbf{2)} take an oath.

\TCheadword{gbɛ} \textit{v} \textbf{1)} walk; \textit{gbɛɛ} walk (\citealt{Pichl1967}). \textit{Ke ko gbɛ nai arijana lɔ wɔ che iŋɔi bomai.} He has walked the road to heaven where we will be with gladness. \textbf{2)} travel (K dialect). \textit{Yà gbɛ̀ɛ́ thìrà.} I took three trips. \textbf{3)} run. \textit{Kɛyɛ laiowɛ yɛmɔbo hɔ vethimi, wɔ gbɛ kɛkɛkɛ, ha hun mɔ vethi.} But as it is now, as you say help me (lift this to my head), he would quickly run to help you. \textbf{4)} visit. \textit{Yɛlai bikɔs hin pɛ tɛŋga apima hinyɛ ha bia che hun gbɛ.} That is it, because maybe our children will come visit.

\TCsubword{gbɛmani} (der.) \textit{v} walk alone (lit. walk with oneself) (\citealt{Pichl1967}). 

\TCsubword[2]{gbɛk} (unspec.) \textit{v} deal with (lit. walk with). \textit{Bikɔs Bolomnɔɛ mɔ gbo ŋa len mɔ ŋa shi la mɔ gbɛkɛ.} Because [for] the Bolom, [if] you are doing something you should know how to walk with it [deal with it].

\TCheadword{gbɛbeŋ} \textit{n} \textit{gbəbe̹ŋ} (hɔ̃/tha) wooden tub (\citealt{Pichl1967}). 

\TCheadword{gbɛɛkɛp} \textit{n} bird species, kite or big hawk (\citealt{Pichl1967}). 

\TCheadword{gbɛɛŋ} \textit{n} [gbɛɛŋ] glory, holiness, spirit (K dialect); \textit{gbəng} glory (\citealt{Pichl1967}). \textit{Hã ke gbəŋ mɔ lɛ.} They see thy glory (\citealt{Pichl1967}). 

\TCheadword{gbɛɛr} adj severe, strict (\citealt{Pichl1967}). \textit{Ba Sese wɔ ba gbəər.} Sese's father is a severe father (\citealt{Pichl1967}). 

\TCheadword{gbɛɛtigbɛɛti} (der. of \TClink[1]{gbet}, \TClink[1]{-i}, see \TClink[1]{gbet}) 

\TCheadword{gbɛgbɛ} \textit{cf}: \TClink[2]{bɔm}. \textit{n} frog, [gbɛ̀gbɛ̀]/ [gbɛ̀gbɛ̀sɛ̀] frog/frogs (K dialect). \textit{Bɔ̀mndɛ́ ɔ̀ gbɛ́gbɛ́yɛ̀? Gbɛ́gbɛ́yɛ̀ wɔ̀ pɛ́ŋhɛ̀.} Toad or frog? It's the frog who jumps.

\TCheadword{gbɛhɔl} \textit{n} \textit{ŋgbə hɔl} (ma) winter (\citealt{Pichl1967}).

\TCheadword[1]{gbɛk} \textit{v} \textbf{1)} \textit{gbɛɛk} lead (\citealt{Pichl1967}). \textbf{2)} operate, run. \textit{A-a, a-a, a gbɛk wɔm.} No, no, I run the boat (as transport).

\TCheadword[2]{gbɛk} (unspec. of \TClink{gbɛ}, \TClink{-k}, see \TClink{gbɛ}) 

\TCheadword{gbɛkɛbu} \textit{n} \textit{ŋgbəkɛbu} (-/ha) class of Toma Society spirits who appear as dancing masquerades (\citealt{Pichl1967}). 

\TCheadword{gbɛkɛm} \textit{n} \textit{thigbəkəm} ((hɔ)/tha) temples (\citealt{Pichl1967}).

\TCheadword{gbɛki} \textit{v} hire (\citealt{Pichl1967}). \textit{Yi gbɛki kump hã bɔnth hĩ hã rɔk.} We hire helpers to help us to harvest (\citealt{Pichl1967}). 

\TCheadword[1]{gbɛlaŋ} \textit{v} whirl (\citealt{Pichl1967}). \textit{Məndɛ ma thim gbəlaŋ.} The water is whirling around (\citealt{Pichl1967}). 

\TCsubword{seŋgbɛŋ} (comp.) \textit{cf}: \TClink[3]{gbɔ}, \TClink{yey}. \textit{n} \textit{sengbəng} (kɔ/ma) nut with a long and thin stick through its middle, used as a top for children (\citealt{Pichl1967}). 

\TCheadword[2]{gbɛlaŋ} \textit{n} \textit{ŋgbɛlaŋ} (ma) whirlpool (\citealt{Pichl1967}).

\TCheadword{gbɛlɛni} (der. of \TClink{gbɛ}, \TClink{-ni}, see \TClink{-ni}) 

\TCheadword{gbɛlɛŋ} \textit{cf}: \TClink[2]{gbeŋgben}, \TClink[1]{lɛli} (comp. of \TClink[3]{lɛ}), \TClink[1]{thunɔ}. \textit{v} \textit{gbələng} search, examine, note, observe (\citealt{Pichl1967}).

\TCheadword{gbɛm} \textit{v} \textit{gbəm} press down with one's hand (\citealt{Pichl1967}). 

\TCheadword{gbɛma} \textit{cf}: \TClink[1]{chal}, \TClink{re}. \textit{n} (wɔ/hã, si) antelope species, possibly gray duiker (\citealt{Pichl1967}). 

\TCheadword{gbɛmani} (der. of \TClink{gbɛ}, \TClink[4]{ma}, \TClink{-ni}, see \TClink{gbɛ}) 

\TCheadword{Gbɛnawi} \textit{nam} Poro song after initiation (\citealt{Pichl1967}). 

\TCheadword{Gbɛni} (Mende \textit{Gbɛni}) \textit{n} name given to society spirit who appears as a dancing masquerade (\citealt{Pichl1967}). 

\TCheadword{gbɛnth} \textit{cf}: \TClink{kikith}. \textit{v} \textit{gbənth} persist (\citealt{Pichl1967}).

\TCheadword{gbɛntrɛ} \textit{v} linger (\citealt{Pichl1967}). \textit{Hã yɛng la yi gbɛntrɛ a?} Why would we linger? (\citealt{Pichl1967}). 

\TCheadword{gbɛnu} \textit{n} \textit{gbənu} (wɔ/hã, a) wife of the Poro Society spirit who appears as a dancing masquerade (\citealt{Pichl1967}). 

\TCheadword{gbɛŋ} \textit{cf}: \TClink{chencha}, \TClink{jɛk}, \TClink{nante}. \textbf{1)} \textit{temp} tomorrow (K dialect); \textit{gbəng} tomorrow, \textit{gbəng chɔl} tomorrow night, \textit{gbəng ipal} tomorrow noon, \textit{gbəng pang} tomorrow evening, \textit{gbəng isɔ} tomorrow morning (\citealt{Pichl1967}); [gbɔ̀n] tomorrow (B dialect). \textit{Pɔ̀ há thònká gbɛŋ.} They will judge them tomorrow. \textit{N hun gbəŋ ipal.} Come tomorrow at daytime (\citealt{Pichl1967}). \textbf{2)} \textit{temp} forever. \textit{Lanɛ la yi theliowɛ labi ŋa kɔni, labi ŋa che haŋ gbɛŋ.} What we are saying here is going to stay and last forever. \textbf{3)} \textit{n} future.

\TCheadword{gbɛŋgbɛ} \textit{n} (kɔ/ma) plant species, climbing plant (Adenia lobata), fish poison made of it (\citealt{Pichl1967}). 

\TCsubword{gbeŋgbesese} (comp.) \textit{n} [gbéŋgbésésé] plant species, bitterball, like the pepper but the fruit is round, thus like a bitter ball (K dialect).

\TCheadword[1]{gbɛŋgbɛs} \textit{cf}: \TClink{gbɛthil}, \TClink{sonthi}. \textit{v} weed stiff herbs with a machete etc. (\citealt{Pichl1967}).

\TCheadword[2]{gbɛŋgbɛs} \textit{n} (kɔ/ma) weed. (\citealt{Pichl1967}). 

\TCheadword{gbɛŋbɛtɛtɛ} \textit{n} fish species, fish known to Ndema people, about 15 inches (K dialect). 

\TCheadword{gbɛŋkasɛsɛ} \textit{cf}: \TClink[3]{gbo}. \textit{v} hop on one leg while holding up the other as children do when playing (\citealt{Pichl1967}).

\TCheadword{gbɛr} \textit{n} \textbf{1)} (ma) \textit{ŋgbe̹r} dew (\citealt{Pichl1967}). \textbf{2)} \textit{ŋgbe̹r} (kɔ/ma) cloud (\citealt{Pichl1967}). \textbf{3)} \textit{ŋgbe̹r} (kɔ/ma) steam, vapor (\citealt{Pichl1967}). \textbf{4)} fog. \textit{Ŋgbe̹r ɛ ma dukə dis nante.} The fog fell heavy today (\citealt{Pichl1967}). \textit{Yɛlaio wɛ, yɛ mgbɛ ma dukɛ, yɛ iche sɔthɔ ja yenchɛkɛ.} As it is, when the fog falls, we do not have fish.

\TCheadword{gbɛrɛ} \textit{cf}: \TClink[1]{gbi}. \textit{quant} all (\citealt{Sumner1921}).

\TCheadword{gbɛrpr} \textit{v} blink, twinkle (eyes) (\citealt{Pichl1967}). 

\TCheadword{gbɛs} \textit{n} tree species, sweet smell when it flowers, women gather it to make fragrant white clay for themselves and their babies, [gbɛ̀s]/[gbɛ̀s-sɛ̀] tree/trees (K dialect). 

\TCheadword[1]{gbɛt} \textit{adv} \textbf{1)} exactly (\citealt{Pichl1967}, \citealt{Sumner1921}). \textbf{2)} only. \textit{Beo, ŋa che mi bonth, ki kaŋdɛ gbɛt lɔ ŋa mu bo.} No, they do not help me, they are just in school now. \textbf{3)} at all. \textit{A-a, ŋɔ che lɔ tik, gbɛt.} No, it does not land there at all. 

\TCheadword[2]{gbɛt} \textit{n} (kɔ/ma) ring (\citealt{Pichl1967}). \textit{Ya tukiɛ gbət-im dɛ.} I have lost my ring (\citealt{Pichl1967}). 

\TCsubword{gbɛtnui} (comp.) \textit{n} (kɔ/ma) earring (\citealt{Pichl1967}). 

\TCsubword{gbɛtsu} (comp.) \textit{n} (kɔ/ma) finger ring (\citealt{Pichl1967}). 

\TCheadword{gbɛta} \textit{v} ebb completely (\citealt{Pichl1967}). \textit{Mən dɛ kong gbəta, ama lɛ hã kɔni pə ko hã lẽy sirəmp.} The sea has ebbed completely, the women have gone on the mud to pick up konk-snails (\citealt{Pichl1967}). 

\TCheadword[1]{gbɛth} \textit{n} \textit{igbɛth} (hɔ̃/-) dirt, filth (\citealt{Pichl1967}, \citealt{Sumner1921}). \textit{Yi ma yo̹m kil l'ay thi-hĩ ko lɔn che igbɛth.} We shouldn't allow dirt in our houses (\citealt{Pichl1967}). \textit{Ja la gbɔw mi, nchaŋ ma mɔ lɛ ma gbɔw igɛth.} This is too hard for me. Your teeth are too dirty (\citealt{Pichl1967}).

\TCheadword[2]{gbɛth} \textit{adj} \textit{ibɛth} dirty, filthy (\citealt{Pichl1967}, \citealt{Sumner1921}). \textit{Lɛ la toŋgiɛ lɛ nɔ ki wɔ fɔnwɔi, kunɛ igbeth ka cheni tiŋ-tiŋ ki athɔma wɔ.} If it showed that the person was a witch, dirty-belly, he was not straightforward among his fellow men. \textit{Wɔ thɔk pɔthiɛ hɔn veleŋdɛ, ni le hɔn kunɛɛ igbɛth.} He is washing the cup outside, leaving the inside dirty (proverb) (\citealt{TISLL1979}). \textit{Nɔɛ wɔ pɔŋ pis igbɛthɛ mɛndaiɛ, chen keni bɛlɛ wɔnɛ wɔ hɔ lɔ kueɛ.} The person that puts a dirty cloth in the water is not seen, but [rather] the one that takes it out (proverb) (\citealt{TISLL1979}). \textit{Hã kul mən ŋgbɛth la ɛ ja libul la chi nakɛ.} To drink dirty water is one of the causes of (lit. which brings) sickness (\citealt{Pichl1967}). 

\TCheadword{gbɛthɛhɔl} \textit{cf}: \TClink{gbenik} (der. of \TClink[2]{gbemi}, \TClink{-k}), \TClink{kun}, \TClink{tem}. \textit{n} (hɔ̃/tha) womb; woman's belly (\citealt{Pichl1967}). 

\TCheadword{gbɛthɛk} (comp. of \TClink[1]{gbet}, \TClink[2]{thɔk}, see \TClink[2]{thɔk}) 

\TCheadword[1]{gbɛthil} \textit{v} \textbf{1)} hint (\citealt{Pichl1967}). \textbf{2)} warn secretly (\citealt{Pichl1967}).

\TCheadword[2]{gbɛthil} \textit{cf}: \TClink[1]{gbɛŋgbɛs}, \TClink{sonthi}. \textit{v} lift out cassava, potatoes, etc. without harming the plant so as to leave the younger ones to grow (\citealt{Pichl1967}). 

\TCheadword{gbɛtnui} (comp. of \TClink[2]{gbɛt}, \TClink{nui}, see \TClink[2]{gbɛt}) 

\TCheadword{gbɛtsu} (comp. of \TClink[2]{gbɛt}, \TClink[1]{su}, see \TClink[2]{gbɛt}) 

\TCheadword{gbɛyɛ} \textit{cf}: \TClink{bias}. \textit{n} \textit{thigbɛyɛ} (-/tha) journey (syn. \textit{bias}) (\citealt{Pichl1967}). 

\TCheadword[1]{gbi} \textit{cf}: \TClink{gbɛrɛ}, \TClink{gboŋ}, \TClink{pe}. \textit{adv} \textbf{1)} very much. \textit{Lanɛ laŋ la sɔkba mɔ gbi.} That is the only one that really disturbed you. \textit{Ŋa kul mɔi ma sɔisɔi gbi ŋa koi piŋiɛni.} They drink tasty, well-mixed drinks, and they turn against us. \textbf{2)} at all. \textit{Ache lɔŋ kɔ gbi, ya lɔ kɔɛ a ke nɔɛ yɛ sɛmɛ kilɛ koɛ.} I will not go there at all, when I see the person standing in the room. \textit{Mɔm komɔ rɛmda ki, ya chen lan haa gbi.} You child of a viper, I will not do it--at all. \textbf{3)} hard. \textit{Kikith ko gbi lɔ ŋcheka.} Press down hard whenever (something) is here. \textbf{4)} together. \textit{Gbi ni ŋgefeyɛ, mɔi binthma-binthma mpuliɛ-puliɛ mɔi nɛmil labo iyɛllɛ ŋɔ shilɔ che.} Together with the pepper, you mix it up, and then you taste it to know if the salt is okay.

\TCheadword[2]{gbi} \textit{v} steer a canoe or boat (\citealt{Pichl1967}). 

\TCheadword[3]{gbi} \textit{quant} \textbf{1)} all. \textit{Mɔ lɔ bɔnth apuma mɔ ɛ han gbi}. You will meet all your children there (\citealt{Pichl1967}). \textbf{2)} every. \textit{Gɔmɛnt lɛ hã thoŋkiɛ lɛ hã yema hã saba, che lɛ tamɔ pokan gbi wɔ koŋ huth lɛ, wɔ hã paka pɔn bul hã bol wɔ lɛ.} The government has proclaimed that they want to make a law that every young man who has come of age has to pay one pound as a head-tax (\citealt{Pichl1967}). \textit{Ho̹ŋ gbi kɔ hã wɔŋ ve̹r.} Every compound is to send its share (\citealt{Pichl1967}). \textbf{3)} any. \textit{Wɛl tɛmdɛ gbi ŋɔa rɛdiɛ akɔ hɛlɛ ko} Well at any time I am ready, I will go out to sea. comp. \TClink{ko-gbi} (see \TClink[2]{ko}), \TClink{lanɛ-gbi} (see \TClink[1]{lan}), \TClink{tɛmgbi} (see \TClink[1]{tɛm}) 

\TCheadword[4]{gbi} \textit{cf}: \TClink[2]{hɛlɛ}, \TClink{hɛliŋ}, \TClink{mɛnpɛyɛ} (comp. of \TClink[3]{mɛn}, \TClink[1]{pɛ}). \textit{n} \textit{gbiy} (hɔ̃/-) low tide occurring twice a month (\citealt{Pichl1967}). \textit{Kɛ ɡbiɛ ma che thaŋ.} But at low tide it does not climb up.

\TCheadword{gbiipir} [gbiipɝ] \textit{n} \textbf{1)} fish species, sting ray, wing span of four feet, seldom seen (K dialect); gbïpr (wɔ/hã, N) monster fish, leviathan (\citealt{Pichl1967}). \textit{Gbïpr kantha hal lɛ.} The gbïpr blocks the river (\citealt{Pichl1967}). 

\TCheadword{gbikan} (unspec. of \TClink{bikin})

\TCheadword{gbikɛ} \textit{n} (hɔ̃/tha) bag; bag for game (\citealt{Pichl1967}). 

\TCheadword{gbikgbikni} (der. of \TClink{gbikni}) 

\TCheadword{gbikin} \textit{cf}: \TClink{gbikan}, \TClink{gbikni}. \textit{v} run (\citealt{Sumner1921}). 

\TCsubword{gbikan} \textit{cf}: \TClink{gbikin}, \TClink{gbikni}. \textit{n} \textit{thigbïkan} (-/tha) race (\citealt{Pichl1967}). \textit{Gbïkanthi wɔ lɛ tha koŋ.} His race is finished (\citealt{Pichl1967}). \textit{Beraa, hi thola ka thigbikan ni hi kɔa gbunda feɛ hiŋk mɛsaɛ atok.} Gentlemen, let us run down and grab the money on top of the table. 

\TCheadword{gbikni} \textit{cf}: \TClink{gbikan}, \TClink{gbikin}, \TClink{kimɔ}. \textit{v} \textbf{1)} run (\citealt{Pichl1967}). \textbf{2)} run away, flee. \textit{Ba Na pɛ wɔ ye gbïkni.} Mr. Spider ran away again (\citealt{Pichl1967}).

\TCsubword{gbikgbikni} (der.) \textit{v} scamper. \textit{Bɛlsɛ ŋae tipɛ gbik-gbikni baiɛ tokɛ <kara-kara kara-kara kara-kara> ŋa hɔɛ, <fiii fiii fiii>.} The rats began scampering up above the bari <kara-kara kara-kara kara-kara> they were saying <fiii fiii fiii>. \textit{Bɛl siatiŋ do ki, ŋa gbik-gbikni tokɛ ko <kara-kara kara-kara kara-kara>.} These two rats, they run-run around above <kara-kara kara-kara kara-kara>.

\TCheadword[1]{gbil} \textit{n} (wɔ/hã, N) fish species, sea fowl (Balistes forcipatus) (\citealt{Pichl1967}). 

\TCheadword[2]{gbil} \textit{n} (kɔ/-) kind of sauce (\citealt{Pichl1967}). 

\TCheadword[3]{gbil} \textit{v} \textbf{1)} stoke a fire; put something on to roast (\citealt{Pichl1967}). \textit{Ŋkɔ gbïl iwɔm dɛ lal l'ay kɔ jɛmdi lɛ lɔ yema nyum.} Go put wood on the fire, the fire is about to go out (\citealt{Pichl1967}). \textbf{2)} roast. \textit{Gbam dɛ kɔ cho gbïlɛ na lɛ kong nɔthul, kɔ kong lɔɔ.} The potato which you put (on) to roast is soft already, it is roasted (\citealt{Pichl1967}).

\TCheadword{gbilgbil} \textit{n} [gbìlgbìl] tree species with a bitter root, cut up and left in water overnight to steep and then drunk, good for belly, headache, malaria, leaves also bitter, can be fermented, put in fire, then applied to head (K dialect); \textit{gbïlgbïl} (kɔ/ma) plant species, shrub with round flower heads 2-3 inches in diameter (Nauclea latifolia) (\citealt{Pichl1967}).

\TCheadword{gbim} \textit{n} \textit{gbïm} (hɔ̃/-) smoke (\citealt{Pichl1967}). comp. \TClink{wɔmgbimi} (see \TClink[2]{wɔm}) 

\TCheadword{gbimi} \textit{n} \textit{igbïmi} (kɔ/-) dust (\citealt{Pichl1967}).

\TCheadword{Gbiminte} \textit{nam} July. \textit{Pəlɛ lɛ kɔ ya fũŋ-fũŋ hɔ lɛ kɔ si che hã yuk paŋ Gbiminte lɛ.} The rice that I planted temporarily (in a nursery) will do for transplanting in the month of July (\citealt{Pichl1967}).

\TCheadword{gbinthim} \textit{n} \textit{gbïnthïm} (kɔ/tha) veranda (\citealt{Pichl1967}). 

\TCheadword{gbintik} (der. of \TClink{gbiŋkith} (unspec. of \TClink[1]{bim}), see \TClink[1]{bim}) 

\TCheadword{gbiŋ} \textit{Idph} of sticking. \textit{Aftabakɛ ŋɔ hun gba ki <gbiŋ>, blidin iŋɔi huŋyi ki fip.} The afterbirth came and really got stuck <gbiŋ>, then bleeding burst out badly.

\TCheadword[1]{gbiŋk} \textit{n} \textit{gbïnk} (kɔ/tha) rudder (\citealt{Pichl1967}).

\TCheadword[2]{gbiŋk} \textit{v} \textit{ŋgbïnk} be widespread, be common (\citealt{Pichl1967}). \textit{Nyɛk lo ma kong ŋgbïnk trï ka bonk bon Amoya lɛ.} Those things are all over the town during the festival of the Muslims (\citealt{Pichl1967}). comp. \TClink{koŋgbiŋk} (see \TClink[1]{ko}) 

\TCheadword{gbiŋkis} \textit{v} groan, sigh (\citealt{Pichl1967}). 

\TCheadword{gbiŋkith} (unspec. of \TClink[1]{bim})

\TCheadword{gbiŋkithni} (der. of \TClink{gbiŋkith} (unspec. of \TClink[1]{bim}), \TClink{-ni}, see \TClink[1]{bim}) 

\TCheadword{gbiŋknyaŋkuŋ} \textit{cf}: \TClink{gbiŋkra}. \textit{n} (wɔ/hã, N) crab species, small crab, usually has one pincer larger than the other, winker crab (\citealt{Pichl1967}).

\TCheadword{gbiŋkra} \textit{cf}: \TClink{gbiŋknyaŋkuŋ}. \textit{n} (wɔ/hã, N) crab species (\citealt{Pichl1967}). 

\TCheadword[1]{gbisiŋ} \textit{n} \textbf{1)} (kɔ/-) marriage (\citealt{Pichl1967}). \textit{Tɛm ndɔ ŋɔ mɔ gbisiŋɛa?} When did you get married? \textbf{2)} engagement. \textit{Mi gbisiŋ doki, bil loki lɔ mɔɔ kunɛ yini gbɔl ŋɔlɔ ŋa mɔm?} This engagement, this marriage that you are in, do you have peace of mind?

\TCheadword[2]{gbisiŋ} \textit{cf}: \TClink[2]{path}, \TClink{thuka}. \textit{v} marry. \textit{M gbisiŋɛ?} Are you married? \textit{Aa, nɔ gbisiŋɛ, abi nɔpokan.} Yes, I am married, I have a husband.

\TCheadword[1]{gbit} \textit{cf}: \TClink{thuniɛni} (comp. of \TClink{thɔi}). \textit{v} sip, lap, eat like animal (\citealt{Pichl1967}). 

\TCheadword[2]{gbit} \textit{cf}: \TClink[3]{pal}. \textit{n} (kɔ/ma) short pole for dragging net; tree trunk (\citealt{Pichl1967}).

\TCheadword[1]{gbo} \textit{adv} \textbf{1)} emphasis, gives stress to word or phrase it follows and is not to be translated or expressed in other words (\citealt{Pichl1967}). \textit{Ŋkuyɛ gbo.} Do take it! (\citealt{Pichl1967}). \textit{Ihɔlɔng hɔ̃ gbo thanthɛn.} Life is (just) vain (\citealt{Pichl1967}). \textit{Yang ya pɛkɛ gbo iwɛy.} I am (truly) filled with evil (\citealt{Pichl1967}). \textbf{2)} only. \textit{Wɔnɛ wɔ ka biyɛ gbemɛni, mi gbo wɔ gbemdɛ.} The one he had did not give birth; it is only our mother that gave birth. \textit{Ŋɔi ni ŋa fili si i mɔla chaŋ gbo ka Jizɔs sɛ.} How are we to go there, only if we pass through Jesus. \textbf{3)} just. \textit{Lɛ nɔsɛ ha ni gbo kɛkɛ nrunth gbo mɔ gbo runth li bul komɔɛ koŋ honi.} If the nurse does not make it fast, you just push, you just push once, and the baby is out. \textit{Velen thilandɛ hun gbo le chal ka ni kunɛ ŋɔ wɔi nɛki.} After that (she) just sat and felt her delivery pain. \textbf{4)} very, quite. \textit{Lɛ yɔktha sɛkilɛ gbo yenkəlɛŋ yi lo he̹r charaŋ.} When the farm with felled trees is quite dry, we burn it clean (\citealt{Pichl1967}). \textit{Rælɛ hɔ̃ gbo ləm.} The paper is very thin (\citealt{Pichl1967}). \textbf{5)} at all. \textit{Lɛ nsi gbo lɔŋ, nsi gbo hɔth, mɔ sɔthɔ yen sɔmɔ.} If you know how to set traps at all, you know how to fish at all, you would get something to chew. \textbf{6)} actually, indeed. \textit{Abɔyi ni gbo ache hun.} If I am not satisfied, I will not return. \textit{Pɛlɛ bɛ, haŋaɛ kuthai gbo, hanɛ ha han nchɛkɛ han ha kuthaɛ.} Even rice, let them indeed plow, those that make a farm must plow it. \textbf{7)} right. \textit{Poloŋ dɛ kɔ gbo kïl mi lɛ ntɛɛnt.} The cotton tree is right near my house (\citealt{Pichl1967}). \textbf{8)} simply. \textit{Wɔ gbo wunk tho l'ay ni pəlɛ l'ay.} He simply rushed through the bush and the rice (\citealt{Pichl1967}). comp. \TClink{lagbo} (see \TClink[2]{la}) 

\TCheadword[2]{gbo} \textit{temp} still. \textit{Ŋa jo ɲje ma sɔisɔi gbi ŋa piŋini gbo we.} They eat nice food, yet still they turn against us.

\TCheadword{gbogboo} \textit{n} (kɔ/tha) female sexual organ; vulva (\citealt{Pichl1967}).

\TCsubword{gbogbotok} (unspec.) \textit{cf}: \TClink{kɔm}, \TClink{maima}, \TClink{tom}, \TClink[2]{wo}. \textit{n} (kɔ/tha) mons veneris; private parts of males or females (\citealt{Pichl1967})

\TCheadword{gbogbotale} [gbógbótàlè] \textit{n} palm species, short palm, [gbógbótàlèɛ́] a short palm (K dialect).

\TCheadword{gbogbotok} (unspec. of \TClink{gbogbo}) 

\TCheadword{Gboka} \textit{cf}: \TClink{Boka}. \textit{nam} [gbòkà] Poro Society spirit who appears as a dancing masquerade, with origins among the Mende (K dialect).

\TCsubword{Gbɔkathoŋthoŋ} \textit{cf}: \TClink{Boka}. \textit{nam} (wɔ/-) Toma Society spirit who appears as a dancing masquerade (\citealt{Pichl1967}).

\TCheadword{gbokanɔ} (comp. of \TClink{Gboka}, \TClink{nɔ}, see \TClink{nɔ}) 

\TCheadword{gbokbo} \textit{cf}: \TClink{gboloŋk}, \TClink{gbɔlɔŋ}. \textit{n} (wɔ/hã) fish species, catfish (Arius latiscutatus) (\citealt{Pichl1967}). \textit{Du gbokbo lɛ bi nyam.} The fins of the catfish are poisonous (\citealt{Pichl1967}). \textit{Ŋ kɔ salenka gbokbo lo.} Go salt this catfish! (\citealt{Pichl1967}). comp. \TClink{pɛlgbokbo} (see \TClink[2]{pɛl}) 

\TCheadword{gbokoth} \textit{n} (hɔ̃/-) cowpox (\citealt{Pichl1967}).

\TCheadword{gbolbel} \textit{n} [gbòlbél] tree species (K dialect). 

\TCheadword{gboli} \textit{n} elder, [gbólì]/ [\`{m}gbòlí] elder/ elders (B dialect).

\TCheadword{gbolnthuk} (comp. of \TClink{gbɔl}, \TClink[2]{thukul} (der. of \TClink{thuk}, \TClink{-ul}), see \TClink{gbɔl}) 

\TCheadword{gbolo} \textit{n} throat; \textit{gboolo} (hɔ̃/tha) throat, gorge, gullet (\citealt{Pichl1967}).

\TCheadword{gboloŋk} \textit{cf}: \TClink{gbokbo}, \TClink{gbɔlɔŋ}, \TClink{gbuluŋk}. \textit{n} (wɔ/hã, N) fish species, Bonita fish (Thunnus Pelamys) (\citealt{Pichl1967}).

\TCheadword{gbompa} \textit{v} \textbf{1)} enlarge, grow. \textit{Lagbo bɔmdai lɔɛ, pɔ kɔ ŋa gbompa ton, ɛn pɔ pɛ ka thiwonka, kaŋka kɔ ma gbompa ni bɔnɔ bul.} If it (rice field) is in a swamp, they will make it (space between plants) a little greater and make spaces so it (rice seedling) can grow without being pushed into one place. \textbf{2)} assemble, gather together. \textit{Pɔi chɛth bokɛ pɔiya joɛ ha yindɛ ŋai hun gbompani ŋai hun jo.} They will cook the sauce and rice, and everyone will gather and eat.

\TCheadword{gbonda} \textit{n} [gbòndà] tree species used for axe and hoe handles, a special termite can infest and ruin (K dialect).

\TCheadword[1]{gboŋ} \textit{cf}: \TClink[1]{gbi}, \TClink{vuli}. \textit{adv} very (\citealt{Sumner1921}). 

\TCheadword[2]{gboŋ} [gbóŋ] \textit{cf}: \TClink{gbɛŋkasɛsɛ}. \textit{n} game played with seeds and small cups, like warri (K dialect); \textit{gbo} (hɔ̃/tha) game played with seeds (\citealt{Pichl1967}). comp. \TClink{bɛlmagbo} (see \TClink[2]{bɛl}) 

\TCheadword{gboŋgboŋploplo} (comp. of \TClink{gbaŋgbaŋ}) 

\TCheadword{gboŋgbos} \textit{n} \textit{ŋgbongbos} (ma) strong current in rivers or creeks caused by rain (\citealt{Pichl1967}). 

\TCheadword{gboŋgbotho} (comp. of \TClink{gbaŋgbaŋ}) 

\TCheadword[1]{gboo} \textit{cf}: \TClink[2]{gbogbo}. \textit{n} (hɔ̃/tha) short trousers of the Bolom (\citealt{Pichl1967}).

\TCheadword[2]{gboo} \textit{n} (hɔ̃/tha) padlock (also \textit{gbooku}) (\citealt{Pichl1967}).

\TCheadword{gbooku} \textit{n} (hɔ̃/tha) padlock (also \textit{gboo}) (\citealt{Pichl1967}).

\TCheadword{gbos} \textit{v} \textbf{1)} bark. Thumɔɛ lɛ gbos. The dog barks (\citealt{Pichl1967}). \textbf{2)} speak rudely. \textit{Ŋai yɛŋ la mɔ gbosa?} Why are you barking your words? (overheard from a motorcycle driver to his passenger customer by A Bendu). 

\TCheadword{gbosa} \textit{n} \textit{igbosa} (hɔ̃/tha) knife used to trim the face of palm cabbage (\citealt{Pichl1967}).

\TCheadword{gboso} \textit{cf}: \TClink{hakla}, \TClink{sayom}, \TClink{tokoth}. \textit{n} (hɔ̃/tha) mud-covered trap set over holes of subterranean animals (\citealt{Pichl1967}).

\TCheadword[1]{gbɔ} \textit{adj} difficult. \textit{La cheŋ gbɔ, kɛ lanɛki boŋgoo lagbɔ.} It is not difficult, but the one these days is difficult. \textit{Ja la gbɔw mi.} This is too hard for me (\citealt{Pichl1967}). 

\TCheadword[2]{gbɔ} \textit{adv} excessively. \textit{Haliwɔ wɔm dɛ ŋɔ gbɔɔ che bom ni dis.} Because the canoe was too big and heavy. \textit{Ni chii chelɛ ya hun sɔthɔ yen ha sɔm, ndikɛ koŋ mi gbɔɔ!} And bring it so that I can come and eat something, hunger is consuming me! \textit{Kïl mi lɛ hɔ gbɔw dul.} My roof is leaking too much (\citealt{Pichl1967}). 

\TCheadword[3]{gbɔ} \textit{cf}: \TClink{seŋgbɛŋ} (comp. of \TClink{sɛŋ}, \TClink[1]{gbɛlaŋ}), \TClink{yey}. \textit{n} \textit{thigbo} (-/tha) children's top (\citealt{Pichl1967}). 

\TCheadword[1]{gbɔgbɔ} \textit{n} hammer, [gbɔ̀gbɔ̀]/[gbɔ̀gbɔ̀tɛ́] hammer/hammers (B dialect); \textit{gbɔgbɔ} (hɔ̃/tha) hammer (\citealt{Pichl1967}). 

\TCheadword[2]{gbɔgbɔ} \textit{cf}: \TClink[1]{gboo}. \textit{n} (hɔ̃/tha) large kind of country cloth (\citealt{Pichl1967}).

\TCheadword{gbɔgbɔth} \textit{cf}: \TClink{dembe}, \TClink{lembe}, \TClink{rokos}. \textit{n} (kɔ/ma) sour orange (Citrus aurantium) (\citealt{Pichl1967}). 

\TCheadword{gbɔgbulɔ} \textit{n} (wɔ/hã, N) pangolin (Manis tricuspis and Manis tetradactylus)

\TCheadword{Gbɔkathoŋthoŋ} (comp. of \TClink{Gboka}) 

\TCheadword{gbɔklɔ} \textit{n} (kɔ/ma) herb species, Indian shot (Canna bidentata; Croix lacrimae jobi) (\citealt{Pichl1967}).

\TCheadword{gbɔksa} \textit{v} scrub (\citealt{Pichl1967}, \citealt{Sumner1921}). 

\TCsubword{gbɔksani} \textit{v} scrub oneself (\citealt{Pichl1967}). 

\TCheadword{gbɔl} \textit{n} \textbf{1)} heart, [gbɔ̀l]/[gbɔ̀l thɛ́] heart/the hearts (B dialect). \textit{Jizɔs, a chɔŋ mɔ len gbɔl mi yai.} Jesus, I love you with all my heart. \textit{Ba Na sɛmi ka gbɔl bo̹m, gbɔl ka jo.} Mr Spider stood proudly (lit. with big heart), gluttonously (lit. heart with food) (\citealt{Pichl1967}). \textbf{2)} resolve, will. \textit{Si gbɔl hĩ lɛ yema simjɛm.} And then when our will is discouraged (\citealt{Pichl1967}). comp. \TClink{hiŋ-gbɔl} (see \TClink{hin}), id. \TClink{Lanthgbɔl} (see \TClink{lanth}), \TClink[1]{lanthgbɔl} (see \TClink{lanth}), \TClink[2]{lanthgbɔl} (see \TClink{lanth}), \TClink{min-gbɔl} (see \TClink[1]{min}) 

\TCsubword{gbɔlbom} (comp.) \textit{n} \textbf{1)} pride (lit. big heart) (\citealt{Pichl1967}, \citealt{Sumner1921}). \textit{Tamɔ bul wo lɔ trï ka ni wɔ gbɔlbo̹m wɛyni.} There was a boy in a town and he was terribly proud (\citealt{Pichl1967}). \textbf{2)} [gbɔ̀lbóm] proud person (B dialect). 

\TCsubword[1]{gbɔlkajo} (comp.) \textit{adj} gluttonous (\citealt{Pichl1967}). \textit{Ba Na ni gbɔlkajo wɔ ɛ.} There was the spider and he was very gluttonous (\citealt{Pichl1967}, \citealt{Sumner1921}).

\TCsubword[2]{gbɔlkajo} (comp.) \textit{n} gluttony (\citealt{Sumner1921}). \textit{Ba Na ka che ayeŋ ha bom kendɛ nvis ha hallɛ, kɛ gbɔlkajo ŋɔ siŋ ka wɔ ayeŋ vɛ.} Ba Spider formerly had a big waist equivalent to the other animals, but gluttony played with his middle very much. (\citealt{Sumner1921}). 

\TCsubword{gbɔlmafe} (comp.) \textit{adj} \textit{ŋgbɔlmafe} avaricious (\citealt{Pichl1967}). 

\TCsubword{gbolnthuk} (comp.) \textit{n} madness; eccentricity (ŋgbɔl?) (\citealt{Pichl1967}). 

\TCsubword{gbɔlthukul} (comp.) [gbɔ́lthúkùl] \textit{v} be easily angered (lit. warm heart) (K dialect). 

\TCsubword{hini-gbɔl} (comp.) \textit{n} \textbf{1)} satisfaction (also be satisfied) (\citealt{Pichl1967}). \textbf{2)} peace. \textit{Huno ni ka hin hini-gbɔl.} Come and give us peace in our heart.

\TCsubword{kɔŋ-gbɔl} (comp.) \textit{n} heartbeat. \textit{Kɔŋgbɔl wɔ lɛ kɔ duk yɛ pə wɔ ku ilellɛ.} His heart beats when they call his name (\citealt{Pichl1967}). 

\TCsubword{simgbɔljɛm} (comp.), (id.) \textit{v} [símgbɔ́ljɛ́m] discourage (K dialect).

\TCsubword{ŋgbɔl} (der.) \textit{adj} avaricious (\citealt{Pichl1967}) 

\TCsubword{mintha-gbɔl} (id.) \textit{v} endure. \textit{La tamɔ lɛ }(or: \textit{ta mɔ lɛ}) \textit{ka mintha gbɔl-a?} Why has the child (or: your child) to endure this? (\citealt{Pichl1967}). 

\TCheadword{gbɔlbom} (comp. of \TClink{gbɔl}, \TClink{bom}, see \TClink{gbɔl}) 

\TCheadword[1]{gbɔlkajo} (comp. of \TClink{gbɔl}, \TClink[3]{ka}, \TClink[1]{jo}, see \TClink{gbɔl}) 

\TCheadword[2]{gbɔlkajo} (comp. of \TClink{gbɔl}, \TClink[3]{ka}, \TClink[1]{jo}, see \TClink{gbɔl}) 

\TCheadword{gbɔlmafe} (comp. of \TClink{gbɔl}, \TClink[3]{ma}, \TClink{fe}, see \TClink{gbɔl}) 

\TCheadword{gbɔlɔŋ} \textit{cf}: \TClink{gbokbo}, \TClink{gboloŋk}. \textit{n} fish species, catfish (K dialect). 

\TCheadword{gbɔlthukul} (comp. of \TClink{gbɔl}, \TClink{thukuli} (der. of \TClink[2]{thukul}, \TClink[1]{-i}), see \TClink{gbɔl}) 

\TCheadword{gbɔm} \textit{n} \textbf{1)} (hɔ̃/tha) mourning place; place of the funeral where mourners come to sympathize (\citealt{Pichl1967}).

\TCsubword{gbɔmɔlɔ} (der.) \textit{n} (hɔ̃/tha) mourning place; place of the funeral where mourners come to sympathize (\citealt{Pichl1967}). \textit{Ŋkuath ma chen gbɔmɔlɔ ko kɛ moa moa hɔ lɔ.} Fear is not at the burial house, if something to be feared is not there (proverb) (\citealt{TISLL1979}).

\TCheadword{gbɔnɔ} [gbɔ̀nɔ̀] \textit{n} tree species, fig tree (K dialect).

\TCheadword{gbɔnthi} \textit{n} (hɔ̃/-) or (ma/-) old and very strong palm wine (\citealt{Pichl1967}). 

\TCheadword{gbɔnthɔ} \textit{cf}: \TClink{pil}. \textit{n} (kɔ/ma) dregs, skin and other residue after boiling palm-nuts (\citealt{Pichl1967}).

\TCheadword{gbɔntma} \textit{cf}: \TClink{kamsa}, \TClink{kumba}. \textit{n} (hɔ̃/tha) long shirt; gown (see also: buba) (\citealt{Pichl1967}). 

\TCheadword{gbɔŋgbɔŋ} \textit{n} (hɔ̃/-) epilepsy (\citealt{Pichl1967}).

\TCheadword[1]{gbɔŋkɔ} \textit{n} (hɔ̃/tha) forest. \textit{Kɛ kpɔnkɔ hɔ̃ ka che trï ko ntɛnt, hɔ̃ nɔonɔ ka chen kɔ ai ɛ.} But there was a forest near the town, which no one entered (\citealt{Pichl1967}). \textit{Kpɔnkɔ lɛ hɔ̃ kong pinkin dɛ trï bo̹m wəyni kəlɛng.} The forest was changed into a big and beautiful town (\citealt{Pichl1967}). 

\TCheadword[2]{gbɔŋkɔ} \textit{n} (kɔ/-) rice variety (\citealt{Pichl1967}, \citealt{Sumner1921}). 

\TCheadword{gbɔŋkɔt} \textit{cf}: \TClink{bɛŋkɔk}. \textit{n} [gbɔ̀ŋkɔ̀t] ankle (K dialect). 

\TCheadword{gbɔɔ} \textit{cf}: \TClink{roth}. \textit{n} [gbɔ̀ɔ̀] eggplant (K dialect); (kɔ/ma) eggplant, garden egg (\citealt{Pichl1967}).

\TCheadword{gbɔs} \textit{n} (kɔ/-) scent, to scent, smell (\citealt{Pichl1967}). \textit{Ki hɔ̃ gbɔs wɛi.} This smells bad (\citealt{Pichl1967}). comp. \TClink{theɛgbɔs} (see \TClink{the}) 

\TCsubword{togba} (comp.) \textit{n} [tógbá] sweet-smelling tree used for herbs, ground with clay and put on babies (K dialect). 

\TCheadword{gbɔsɔ} \textit{n} (wɔ/hã, N) collective name for large fish such as sharks, tunnies, etc. (\citealt{Pichl1967}). 

\TCheadword{gbɔthɔ} \textit{n} (hɔ̃/tha) valley (\citealt{Pichl1967}).

\TCheadword{gbɔyɔ} \textit{n} cowries (\citealt{Pichl1967}).

\TCheadword{gbu} \textit{cf}: \TClink{kos}. \textit{n} \textit{thigbu} (-/tha) jaws (\citealt{Pichl1967}).

\TCheadword{gbuk} [gbùk] \textit{n} plant species, vine with thorns (K dialect). 

\TCheadword{gbuki} \textit{v} [gbúkí] uproot by force (K dialect). 

\TCheadword{gbulu} [gbùlù] \textit{n} calabash (K dialect). \textit{Hɔɛóhɔɛ wɔ̀ pɛ́l gbùlù.} Every day he breaks a calabash. \textit{Sɛ́sɛ́ wɔ̀ pɛ̀l gbùlùɛ̀.} Sese broke the calabash.

\TCheadword{gbuluk} [gbúlúk] \textit{n} snake species, egg-eating, bites people, poisonous, quiet and not so dangerous, less than a meter and not very fat, though more than a foot long, brown with some blackish specks (K dialect); (wɔ/hã, N) snake species, (some identify as egg-eating, Dasybeltis scabra, others say night adder, Causus rhombeatus) (\citealt{Pichl1967}). 

\TCheadword{gbuluŋk} \textit{cf}: \TClink{pemple}. \textit{n} fish species, kind of jumping fish living in fresh water (Periophthalmus papilio) (\citealt{Pichl1967}). 

\TCheadword{gbunda} \textit{cf}: \TClink{gbundɛ}, \TClink{tool}. \textit{v} \textbf{1)} rape. \textit{Pə kɔnthi chencha Sese wɔ lɔ yɔlko l'ay gbunda la ke Kaay lɛ.} They caught Sese yesterday, he is in chains (because) he raped Kayn's wife (\citealt{Pichl1967}). \textbf{2)} attack (\citealt{Pichl1967}). \textbf{3)} grab. \textit{Beraa, hi thola ka thigbikan ni hi kɔa gbunda feɛ hiŋk mɛsaɛ atok.} Gentlemen, let us run down and grab the money on top of the table. \textit{Wɔe duk sampa yekeɛ kunɛ, gbunda yekeɛ maŋchaŋma wɔɛ.} She drops into the cassava basket, grabs the cassava with her teeth. (\citealt{Pichl1967})

\TCsubword{gbundagbunda} (der.) \textit{cf}: \TClink{toofi}, \TClink{yɔk}. \textit{v} grab. \textit{Ŋa hɛthhɛthni ŋa dukduk hiŋk ndɔndɔ, ŋa gbundagbunda feɛ hiŋk mɛsaɛ atok.} They slipped in (descended) from all directions, they grabbed the money from on top of the table. (\citealt{Pichl1967})

\TCheadword{gbundɛ} \textit{cf}: \TClink{gbunda}, \TClink{sin}, \TClink[2]{sɔkba}, \TClink{tombo}. \textit{n} (hɔ̃/-) trouble (\citealt{Pichl1967}). \textit{Liwu lɔ bɔnthɔ hĩ, gbundɛ bo̹m koŋ duk pɔk l'ay.} Calamity has met us, big trouble has befallen the country (\citealt{Pichl1967}). \textit{Thɔli, hã thɔli, gbundɛ bo̹m koŋ duk trï ka.} Keep silent, big trouble has befallen this town (\citealt{Pichl1967}).

\TCheadword{gbuŋgbuŋ} \textit{n} (kɔ/ma) steam launch (\citealt{Pichl1967}).

\TCheadword{gbuŋkni} \textit{v} become wedged (\citealt{Pichl1967}).

\TCheadword{gbuɔ} \textit{n} \textit{ŋgbuɔ} (ma) ocean, high sea (\citealt{Pichl1967}). comp. \TClink{thaaleŋgbuɔ} (see \TClink{thaale}) 

\TCheadword{gbusa} \textit{cf}: \TClink{bue}, \TClink[2]{kutha}. \textit{v} dig. \textit{Labo thibɔm lɔ pɔ bia yukɛ, pɔ kɔ ni bɔm thai pɔi kɔ piŋgi bɔmdɛ ɔ pɔi gbusa.} If people have to plant where it is muddy, they will then turn the mud over or then they dig. \textit{Lagbo pɔnthai lɔi pɔ gbusa.} If it is in the swamp, they will dig it.

\TCheadword{gbut} \textit{n} (hɔ̃/-) end (\citealt{Pichl1967}); [mgbut] end (K dialect). \textit{La boɛ lɛkɛ-lɛkɛ mgbut.} That's it, the story ends. \textit{Rəkə rəkə gbut.} That's the end (final formula in tales) (\citealt{Pichl1967}).

\TCheadword{gbuta} \textit{v} “swear” someone (Nd dialect). 

\TCsubword{gbutaram} (comp.) \textit{n} family curse (Nd dialect). \textit{Gbutaramdɛ} curse of the family (name given to the launch bought by a local politican). 

\TCheadword{gbutaram} (comp. of \TClink{gbuta}, \TClink{ram}, see \TClink{gbuta}) 

\TCheadword{gbuth} \textit{adj} \textbf{1)} rough. \textbf{2)} savage, ill-bred (\citealt{Pichl1967}).

\TCheadword{gbuthul} \textit{adj} unripe, green, ill-bred (\citealt{Pichl1967}).

\TCheadword{gbuu} \textit{cf}: \TClink{kii}. \textit{n} (wɔ/hã, si) crocodile species, short-nosed or dwarf crocodile (Osteolaemus tetraspis) (\citealt{Pichl1967}). 

\end{letter}
\begin{letter}{H}

\TCheadword[1]{ha} \textit{cf}: \TClink[2]{ŋal}, \TClink{tɔkɔ}. [variously written \textit{ha, hã, ŋa}] \textit{prep} \textbf{1)} for. \textit{Ya la mɛmiɛni fli ha haŋ mpanth haŋ pɔkimdɛ.} I am happy about that, to really work for my country. \textit{N sonthuli pɛnsil lɛ hã yaŋ.} Sharpen the pencil for me (\citealt{Pichl1967}). \textbf{2)} about. \textit{Lanɛ la pə hɔmɔ mɔ lɛ hã yaŋ, la chen roŋ, ntɛnkɛn ma gbo vɛ.} What they told you about me is not true; it is only a suspicion (\citealt{Pichl1967}). \textit{A yema mɔ ni yi ŋa yɛ nka che ko tallɛ?} I want to now ask you about when you were young. \textbf{3)} per. \textit{Ŋa wɔɛ, Mbɛkɛ ma pɔ chan theli ɔ Mbolomdɛ?} Per day, is it Krio they speak more or Sherbro? \textit{Gɔmɛnt lɛ hã thoŋkiɛ lɛ hã yema hã saba, che lɛ tamɔ pokan gbi wɔ koŋ huth lɛ, wɔ hã paka pɔn bul hã bol wɔ lɛ.} The government has proclaimed that they want to make a law that every young man who has come of age has to pay one pound as a head-tax (\citealt{Pichl1967}). \textbf{4)} with. \textit{Bikɔs hin abena hiɛ pɔ thuka ŋa bo pɔm thaba.} Because our (emph.) parents were just married with tobacco leaf. \textbf{5)} here. \textit{Ha chala ŋa?} Do they reside here? \textit{So yɛ nwuni Shenge ka, nkaŋa ŋa pɛ?} So when you came here to Shenge, did you study here as well? \textbf{6)} of. \textit{Yèmà wɔ̀ bísín hà kòmɔ́ɛ́.} Yema took care of the child.

\TCheadword[2]{ha} \textit{cf}: \TClink[2]{bi}, \TClink[3]{lɔi}, \TClink[1]{ma}, \TClink{mɔs}, \TClink[2]{ŋa}. \textit{Aux} \textbf{1)} should. \textit{Lanɛ la li kɛlɛŋ, lɛ bɛn mɔi wɔ mɔ gbo ntɛnt, mɔ ha suthra wɔ, mɔ ha toŋgiɛ lɛ wɔ gbem mɔ.} That is what is good, if your parent is near you, you should try to show that she gave birth to you. \textit{Ncheni ha bɛ iyɛl gbe.} You should not add a lot of salt. \textit{…si ŋa wɔm bɛ hun kɔni hɔthɔ gbampɔɛ.} …if he should come go fishing for mullet. \textbf{2)} let. \textit{Kɔ̀, há kɔ̀.} He went, let him go. comp. \TClink{maha} (see \TClink[1]{ma}) 

\TCheadword[3]{ha} \textit{subordconn} \textbf{1)} functions similar to infinitival ‘to,' often begins a series of verbs. \textit{Haŋ yɛ mɔ munini ha mɔm ko bɔnth bamɔ ŋa mpanth.} And how you came back to the town to help your father with work. \textit{Bɛɛ tirɛ ni ŋgbako ma tirɛ ŋae wom ha vel Kaiŋ Taso.} The town chief and the elders then summoned Kain Tasso. \textbf{2)} because.

\TCheadword{ha-ha-ha} \textit{interj} laughing sound. \textit{Wɔe mam tokɛ tokɛ kaathba, “Ha-ha-ha-hae-e-e-e ha-ha-ha, yɛ len la ki-a-e-e-e!”} He laughed loudly, “(laughs) What a thing is this!”

\TCheadword{haa} \textit{cf}: \TClink{bɛmpa}, \TClink[2]{bulɔ}, \TClink[1]{chɔ}, \TClink[2]{hɛl}, \TClink[2]{kɔ}. \textit{v} \textbf{1)} do. \textit{Pɔki Salon dɛ, pɔ ko ha jagbe.} In our country Sierra Leone, they have done a lot. \textit{Yai po haŋ ha ja yenchɛk vɛ, fish prɔsɛsin.} I started doing fish work, fish processing. \textit{Mɔm la ŋka cheni ŋa?} What have you been doing? \textit{Ba mɔ la ŋaa?} Your father, what does he do? \textbf{2)} make. \textit{Yus-o-ki ŋa haa Braima theni yeŋkɛlɛŋ.} These fish made Braima feel fine. \textit{Wɔ ko wɔ bɔ ŋa wothim dɛ poepoe ŋɔ kɔ che mi disilɛ.} He has made this load light which was heavy on my head. \textit{Nɛn do ŋɔ ŋa nɛnthi waŋnimɛndɛ.} This year makes fifteen years. \textit{Ŋɔ pɔ ŋamɔ spika?} When you were made Speaker? \textbf{3)} attend. \textit{Apim hami ntɛnt, apim hanɛ haa kandaiɛ ŋaa Kiamp ko.} Some are near me, some of them attend school in Freetown. \textit{Awɔ ŋaa kil kaŋdɛ a?} How many of them are in school? \textbf{4)} form. \textit{I chala boɛ ni iŋa grup.} We just sat and decided to form a group. \textit{I koi pisthɛ iraparapa tha iŋakɔ mɔi bɔl.} We would take small pieces of cloth and form it like ball. \textbf{5)} perform. comp. \TClink{nɔhampanth} (see \TClink{nɔ})

\TCsubword[2]{hani} (der.) \textit{cf}: \TClink[1]{chɔk}, \TClink[1]{hɛl}, \TClink{traiya}. \textit{v} \textbf{1)} happen. \textit{Lanɛ la wɔn yemaɛ la ha hani.} What they would want is what happens. \textit{Mɔ̀ bí là hànì.} You have it happen. \textbf{2)} try. \textit{Ihani gbi, hani gbi, hani gbi, wɔi keni ken ki.} We tried and tried and we tried, then early morning came. \textbf{3)} make. \textit{Hɔ hani ki, hɔ chaini fli ŋɛ chanthɛ.} Make like this, it rises up again like a baby.

\TCsubword[2]{han} (der.) \textit{v} do. \textit{Nhɔ gboɛ han ni tikɛ ha chɔ, ma pɛ wei lek thiwɔi.} If you say you will fight the antelope, do not fear the horns (proverb) (\citealt{TISLL1979}).

\TCheadword[1]{haaa} \textbf{1)} \textit{temp} long time. \textit{Pɔ saŋ lɔ gbo haŋ pɔi mɛl.} They will scatter for some time and leave. \textbf{2)} \textit{temp} ever. \textit{Nchen hã di nɔ.} Thou shalt not kill (\citealt{Pichl1967}). \textit{Nchen nhã fothok thɛm mɔ nɔthi mbol.} You shall not calumniate your friends (\citealt{Pichl1967}). \textbf{3)} \textit{temp} forever. \textbf{4)} \textit{temp} on. \textbf{5)} \textit{Loc} a ways.

\TCheadword[2]{haaa} \textit{subordconn} \textbf{1)} until. \textit{Sistha Kɔba lan wɔ lɔ mu haŋ ma nantɛ?} Sister Koba, is she still there up to this day? \textit{Ka lɔ pɔ dumɔ mɔ haŋ nko gbako?} Did they raise you here until you grew up? \textbf{2)} unto.

\TCheadword{hakla} \textit{cf}: \TClink{gboso}, \TClink{sayom}, \TClink{tokoth}. \textit{n} trap used to direct animal toward a single opening in a fence (B dialect). 

\TCheadword{halɛ} \textit{cf}: \TClink[2]{pɛ}, \TClink[1]{pika} (der. of \TClink[2]{pika}), \TClink[2]{pim}, \TClink{tilaŋ}. \textit{adj} other. \textit{Mpang mən-bul bɛlɛng buli, mən-bul bɛlɛng hãlɛ.} Six months on the one side, six on the other side (\citealt{Pichl1967}). \textit{Nsaŋhaɛ ma ka che chaŋ bali ha chaŋ nyiki halɛ gbi.} The egusi grew more than all the other plants.

\TCheadword{hali} \textit{prep} about. \textit{Yaŋ ayɛn ya ke taamɔ ki wɔ ya lem hali wɔɛ.} I myself saw this little boy whom I am talking about.

\TCheadword{haliwɔ} \textit{cf}: \TClink{bikɔs}, \TClink{hayɛ}, \TClink{thaŋkɔ}. \textit{subordconn} because, for (K dialect); \textit{hãliwɔɔ} because (\citealt{Pichl1967}). \textit{A hã la hãliwɔɔ vɛ ŋhɔmɔ-m na.} I did it because you told me so (\citealt{Pichl1967}). \textit{Yi chɔng wɔ len hãliwɔɔ wɔ penkə hĩ chɔng len.} We love him because he first loved us (\citealt{Pichl1967}). \textit{Haliwɔ, wɔ ibi wɔn kɛn dɛ o.} For he is by our side.

\TCheadword{halthe} \textit{cf}: \TClink[2]{hɛlɛ}, \TClink[5]{lel}, \TClink[3]{mɛn}. \textit{n} \textbf{1)} sea. \textit{Ɛlaboɛ kostal eria, halthe ntɛnt lɔ Athemaɛ ŋahun challɛ.} Just the coastal areas, the seaside where the Themnes have come and settled. \textbf{2)} river.

\TCheadword{ham} \textit{cf}: \TClink[1]{thɛk}. \textit{n} (wɔ/hã, si) lizard species, Nile monitor (\citealt{Pichl1967}). 

\TCheadword[1]{han} (Eng \textit{hand}) \textit{cf}: \TClink[1]{pia}, \TClink{sui}. \textit{n} hand. \textit{Wɛl i ka che ple han tɛnis bɔl, ni iple chɔch, ni thipika.} We used to play hand tennis ball, and we play church, and other things.

\TCheadword[2]{han} (der. of \TClink{haa}) 

\TCheadword{Hana} \textit{nam} Hannah, female name given to a person. \textit{Wɔlɔ Bɔima Hana.} She is Boima Hannah.

\TCheadword{handɔ} \textit{cf}: \TClink{hina}, \TClink[5]{hɔ}, \TClink[1]{la}, \TClink{ndɔ}, \TClink[5]{ŋa}. \textit{interrog} \textbf{1)} which. \textit{Mbolomdɛ, Plantain ka lɔ mɔi kiɛ, man ni nthemdɛ handɔ mapɔ chaŋ thelia?} The Sherbro, on Plantain (Island) here where you are, Bolom or Themne, which do they speak more? \textbf{2)} what. \textit{Det handɔ lɔ pɔ gbem mɔa?} You were born on what date? \textit{Siŋthi handɔ tha nkache siŋda?} What games did you used to play? \textbf{3)} who.

\TCheadword[1]{hanɛ} \textit{quant} some. \textit{Kɛ ŋanɛ ŋa wuɛwuɛ ni ache pɛ mɛmba hin awɔ ile lɔ, hin awɔ ile lɔɛ... yi abaot amɛnbul.} But some have died so I do not remember how many of us remain, how many of us remain there... we are about six.

\TCheadword[2]{hanɛ} \textit{dem} those (ha). \textit{Ishiɛ ŋanɛ ŋa bia kɔ hundɛ…} We know that those that are going to come… \textit{Ŋanɛ gbi ŋa yema ŋa thelaɛ ŋala bia the.} Everyone that would want to hear it would hear it.

\TCheadword[1]{hani} \textit{cf}: \TClink[1]{bimbi}. \textit{n} crowd; \textit{hani} (kɔ) crowd (\citealt{Pichl1967}). comp. \TClink{kumpohani} (see \TClink{kompuŋ}) 

\TCheadword[2]{hani} (der. of \TClink{haa}, \TClink{-ni}, see \TClink{haa}) 

\TCheadword{Hanson} \textit{nam} Hanson, name given to a person. \textit{Wɔlta Hanson a ka shi wɔ.} Walter Hanson, I used to know him.

\TCheadword{hanth} \textit{v} shine brightly.\textit{ Palli lɛ hɔ̃ hanth.} The sun shines brightly (\citealt{Pichl1967}). 

\TCheadword{hantha} \textit{cf}: \TClink{biŋ}, \TClink[1]{tɔŋ}, \TClink{waya}. \textit{n} fishing fence. \textit{Ba Yentho bi lɔ hantha ka pənth lɛ ay.} Mr. Leopard had a fishing fence here in the swamp (\citealt{Pichl1967}). 

\TCsubword{hanthpɛl} (comp.) \textit{n} \textit{hãnth-pəl} (kɔ/-) size of a fishing net (\citealt{Pichl1967}).

\TCheadword{hanthpɛl} (comp. of \TClink{hantha}, \TClink[2]{pɛl}, see \TClink{hantha}) 

\TCheadword{haŋka} (Eng \textit{anchor}) \textit{cf}: \TClink{kilik}. \textit{n} anchor (\citealt{Pichl1967}).

\TCheadword{Haruna} \textit{nam} Haruna, name given to a person. \textit{Ba Haruna.} Mr. Haruna.

\TCheadword{hathog} \textit{cf}: \TClink[1]{gbel}, \TClink{yentho} (comp. of \TClink[1]{yen}, \TClink[2]{tho}). \textit{n} (wɔ/hã) leopard (substitute for taboo name) (\citealt{Pichl1967}).

\TCheadword{hato} \textit{subordconn} for; \textit{hato} for, in order to, for the sake of (\citealt{Sumner1921}).

\TCheadword{Hawɔd} \textit{nam} Howard, male name given to a person. \textit{Shenge ka fli skullɛ ŋɔ pɔ wɔ Hawɔd Mɛmorial vɛ.} It is in Shenge here in that school called Howard Memorial.

\TCheadword{hayɛ} \textit{cf}: \TClink{bikɔs}, \TClink{haliwɔ}, \TClink{thaŋkɔ}. \textit{subordconn} because. \textit{Ba mɔ koŋ silini, hãyɛ ŋkɔ wɔn lɛ̃ynɛ hã wiik bul.} Your father is annoyed at you because you did not go to compliment him for a week (\citealt{Pichl1967}). 

\TCheadword[1]{he} \textit{n} cold (sickness). \textit{Ihee hɔ̃ peyɛni mi, ya bɔnthɔ ni hin koŋ gbïnkithni wɔn thibəŋ ni wɔn bol.} Mother has a cold, I found her lying in bed, and she had covered (herself) her feet and head (\citealt{Pichl1967}). 

\TCheadword[2]{he} \textit{disco} hey. \textit{Yɛ nɔ wɔ che ko kɔnaɛ, ya hundɛ wɔi hɔ, “He!”} When someone would be in a corner, then I would come and she would say, “Hey!”

\TCheadword[1]{hei} \textit{v} burn. \textit{Iŋɔ thɛ ŋɔ he yeŋkɛlɛŋ.} We burn it (the field) for it to be burned properly.

\TCheadword[2]{hei} \textit{v} \textit{hẽy} show or expose the teeth, grin (\citealt{Pichl1967})

\TCheadword{hel} \textit{v} \textbf{1)} boil. \textit{Mɛ̀ndɛ̀ mà híl.} The water is boiling (right now). \textit{Sɔk lɛ wɔ mu hel.} The fowl is still boiling (\citealt{Pichl1967}). \textbf{2)} bubble. \textit{Mən dɛ ma hel.} The water bubbles (as it boils) (\citealt{Pichl1967}).

\TCheadword{her} \textit{cf}: \TClink[1]{hɛi}. \textit{v} \textbf{1)} \textit{he̹r} cross (\citealt{Pichl1967}). \textit{Mən dɛ koŋ ye̹l, ma chen pɛ pɔsɔ mɔ bɔ he̹r lel ko.} The water has decreased, it is not much now, you can go across to the other side (\citealt{Pichl1967}). \textbf{2)} go across.

\TCsubword{herk} (der.) \textit{v} [hèrk] take someone across (the same for ‘burn a farm') (K dialect); \textit{hèrk} take across (\citealt{Pichl1967}). \textit{Pə ke̹rkɛ wɔ Bonth ko.} He was taken across to Bonthe (\citealt{Pichl1967}). \textit{Ba Amadu Kamara wɔe herk yagbe wɔɛ Braima Nsheŋke ka.} Mr. Amadu Kamara then ferries his nephew Braima across to Shenge. der. \TClink{herkɛni} (see \TClink{her})

\TCsubword{herkɛni} (der.), (der. of \TClink{herk}) \textit{v} ferry oneself. \textit{Yà hèrkɛ́ní.} I ferried myself across (e.g., river, stream, or stretch of water). 

\TCsubword{herni} (der.) \textit{v} go over, cross. \textit{Wɔ he̹rni lel ko.} We crossed the ocean (\citealt{Pichl1967}).

\TCsubword{hereth} (unspec.) \textit{v} \textbf{1)} be watery. \textit{Sup lɛ hɔ̃ he̹re̹th.} The soup is thin (\citealt{Pichl1967}). \textbf{2)} \textit{he̹re̹th} thin, watery (\citealt{Pichl1967}). 

\TCheadword{herk} (der. of \TClink{her}, \TClink{-k}, \TClink{-ni}, see \TClink{her}) 

\TCheadword{herka} \textit{n} \textbf{1)} \textit{he̹rka} (hɔ̃/tha) ferry boat (\citealt{Pichl1967}). \textbf{2)} tree species, a tree for making rafts in the old days, wood is very light and floats easily, leaves also used for herbs for easing the delivery of babies, woman drinks solution of leaves in water (K dialect); \textit{herka} corkwood tree (\citealt{Sumner1921}).

\TCheadword{herkɛni} (der. of \TClink{herk} (der. of \TClink{her}, \TClink{-k}, \TClink{-ni}), \TClink{-ni}, see \TClink{her}) 

\TCheadword{herni} (der. of \TClink{her}, \TClink{-ni}, see \TClink{her}) 

\TCheadword{Hestins} \textit{nam} Hastings, name given to a place. \textit{Aa ha ka che theli Mbolomdɛ, wɔnɛ fli ka che O-Si pɔlis, Hestins.} Yes, they used to speak Sherbro, even the one who was an OC police (officer), Hastings.

\TCheadword[1]{hɛi} \textit{cf}: \TClink{her}. \textit{v} set off, embark (\citealt{Sumner1921}). \textit{Koŋ hɛ̃y wɔm dɛ ay.} He has set off in the canoe (\citealt{Pichl1967}). \textit{Than tha yi hɛ̃y ay si yi yatha si yi kɔ trï lɛ.} In these (canoes) we embark, then we pull the oars and then we go to town (\citealt{Pichl1967}).

\TCheadword[2]{hɛi} \textit{v} fan; \textit{hẽy} fan, winnow (\citealt{Pichl1967}). \textit{Ya kong hɛ̃y pəlɛ lɛ hã kɔ hɛ̃thi ibənkɛ lɛ.} I have fanned the rice, you (pl) go and pick out the husk (\citealt{Pichl1967}).

\TCsubword{hɛthɛ} (unspec.) \textit{n} \textit{hɛ̃thɛ} (kɔ/tha) fanner, winnowing basket (\citealt{Pichl1967}).

\TCheadword{hɛk} \textit{v} use.

\TCheadword[1]{hɛl} \textit{cf}: \TClink[1]{chɔk}, \TClink[2]{hani} (der. of \TClink{haa}, \TClink{-ni}), \TClink{traiya}. \textit{v} try (\citealt{Sumner1921}).

\TCheadword[2]{hɛl} \textit{cf}: \TClink{bɛmpa}, \TClink[1]{chɔ}, \TClink{haa}. \textit{v} \textit{həl} try, do (\citealt{Pichl1967}).

\TCheadword[3]{hɛl} \textit{n} salt. \textit{Ncheni ha bɛ iyɛl gbe.} You should not put a lot of salt.

\TCheadword{Hɛlɛ} \textit{nam} Helleh, name given to a person. \textit{Ba Bia Hɛlɛ.} Mr. Bia Helleh.

\TCheadword[1]{hɛlɛ} \textit{n} basket type, fancy type of basket made of raffia (\citealt{Pichl1967}). 

\TCheadword[2]{hɛlɛ} \textit{cf}: \TClink[4]{gbi}, \TClink{halthe}, \TClink{hɛliŋ}, \TClink[5]{lel}, \TClink{mɛnpɛyɛ} (comp. of \TClink[3]{mɛn}, \TClink[1]{pɛ}), \TClink[3]{mɛn}. \textit{n} sea. \textit{A shi ŋɔth kɛ ache kɔ hɛlɛ.} I know how to fish but I do not go out on the seas. \textit{Yɛmɔ kɔni hɛlɛ koɛ, mɔ lɔ kɔ lɔl?} When you go out to the sea, do you sleep there?

\TCsubword{hɛlɛiko} (comp.) \textbf{1)} \textit{adj} sea-related. \textit{Wɛl, yaŋ ken dɛ ki mpanth ma hɛlɛkoɛ lɔaɛ.} Well, as of now I am doing sea work. \textbf{2)} \textit{Loc} at sea.

\TCsubword{bɛnaihyɛl} (unspec.) \textit{n} fish species, cassava- or lady-fish (\citealt{Pichl1967}).

\TCheadword{hɛlɛiko} (comp. of \TClink[2]{hɛlɛ}, \TClink[1]{ko}, see \TClink[2]{hɛlɛ}) 

\TCheadword{hɛliŋ} \textit{cf}: \TClink[4]{gbi}, \TClink[2]{hɛlɛ}, \TClink{mɛnpɛyɛ} (comp. of \TClink[3]{mɛn}, \TClink[1]{pɛ}). \textit{n} high tide. \textit{Che wɔiowɔi-o, kɛ yɛ helendɛ ŋɔ che vɛ, ŋɔ mɛndɛ ma thaŋ tokɛtokɛ, mai, nyathi lɛllɛ.} It is not every day-o, but when it is high tide, the water climbs high and licks the land.

\TCheadword{hɛm} \textit{cf}: \TClink{kɛŋklɛni}. \textit{v} refuse, deny.

\TCheadword{hɛn} \textit{v} deny.

\TCheadword{hɛŋ} \textit{cf}: \TClink{kakbom} (comp. of \TClink[2]{kak}, \TClink{bom}), \TClink[2]{kak}, \TClink{sogboka} (unspec. of \TClink[1]{sɔ}), \TClink[1]{sɔ}. \textit{n} wind. \textit{Hɛ̀ŋndɛ́ ŋɔ́ [hɔ̃] bɔ̀s.} The wind is cold. \textit{Taamɔtaa bul, wɔ mmɛn hukɔ ni ihɛŋ disil-disil sɔsɔkɔ.} A little boy, whom heavy waves and heavy winds swept away. comp. \TClink{hɛŋwɛi} (see \TClink[2]{wɛi}) 

\TCsubword{hɛŋveleŋ} (comp.) \textit{n} gonorrhea; \textit{həŋ-ve̹le̹ŋ} gonorrhea (polite expression) (\citealt{Pichl1967}). 

\TCheadword{hɛŋki} \textit{v} \textbf{1)} pass something briefly over fire (\citealt{Pichl1967}). \textbf{2)} initiate infants into Poro Society (\citealt{Pichl1967}). 

\TCheadword{hɛŋveleŋ} (comp. of \TClink{hɛŋ}, \TClink[1]{veleŋ}, see \TClink{hɛŋ}) 

\TCheadword{hɛŋwɛi} (comp. of \TClink{hɛŋ}, \TClink[1]{wɛi} (der. of \TClink[2]{wɛi}), see \TClink[2]{wɛi}) 

\TCheadword{hɛrin} (Eng \textit{herring}) \textit{n} fish species, herring like bonga but smaller (K dialect).

\TCheadword{hɛrni} \textit{v} worship. \textit{ Yi kɔ hərni abɛna hĩ lɛ.} We go to worship our ancestors (\citealt{Pichl1967}).

\TCheadword{hɛrp} \textit{n} tree species with thorns, whitish in color, used for herbs (K dialect).

\TCheadword{hɛth} \textit{v} slip.

\TCheadword{hɛthɛ} (unspec. of \TClink[2]{hɛi}) 

\TCheadword{hɛthhɛthni} (der. of \TClink{hɛthni}) 

\TCheadword{hɛthi} \textit{v} pick out. \textit{Ya koŋ hɛy pəlɛ lɛ hã kɔ hɛt hi ibənkɛ lɛ.} I have fanned the rice, you go and pick out the husk (\citealt{Pichl1967}). \textit{Kɔ hɛt hi ibənkɛ lɛ.} He went to pick out the husk (of rice) (\citealt{Pichl1967}).

\TCheadword{hɛthil} [hɛ́thtǝ̀l] \textit{n} snake found in swamps, stays in the water, black like cobra, 1.5 inches diameter, about 2.5 feet long, some say it is dangerous to people (K dialect). 

\TCheadword{hɛthni} \textit{v} slip. \textit{Lalaɛ kɔ wɔe hɛthni mmɛn nyamban doai ni kɔ kɔni hiŋk wɔn.} His paddle slipped from him, the water carried it away from him.

\TCsubword{hɛthhɛthni} (der.) \textit{v} slip in. \textit{ŋa hɛthhɛthni ŋa dukduk hiŋk ndɔndɔ, ŋa gbundagbunda feɛ hiŋk mɛsaɛ atok.} They slipped in (descended) from all directions, they grabbed the money from on top of the table.

\TCheadword{hi} \textit{pers} we; our; us. \textit{Kɛ pɔk pim kɔlɔ nyanɔɛ pɔ cheŋ wɔ ka fɔsa, hin ka gbo.} But in other countries, if a stranger goes there they would not give him power, only we here do. \textit{Ŋa ŋa awɔŋɔ lɛŋ yeŋkɛlɛŋ ba, ŋa loni bolɛ in bɛ iŋaka ŋa ŋan.} They are the ones I am sending this fine greeting for; they should bear in mind that we are here for them. \textit{La Bahin ko ŋa ha yan dɛ.} What our Father has done for us. \textit{Kɔnɛ o Bahin.} Restore (unto us), our Father. \textit{Anya hiɛ fɔrina ŋaɛ, Koroma, Kallon, Sheriff.} Our people are foreigners, Koroma, Kallon, Sheriff. \textit{Oo aŋa mi isi yɛ lɛ kɛ Kraist ka wu ŋa hin.} Oh, my people, let us realize that Christ died for us. \textit{Ŋa huŋ yi toŋgi ŋalwɔ.} To come and show us about himself. \textit{Abatokɛ che ma ha ni, ha bɔnth hiŋ pɛ ihɔlɔŋ kunɛ.} May God be with them for us, that they meet us alive again.

\TCheadword[1]{hial} \textit{cf}: \TClink{yanɔ}. \textit{n} river; \textit{hial, hial ahɔl} (kɔ/ma) river, mouth of a river, estuary (\citealt{Pichl1967}). \textit{Hial lɛ kɔ thunk.} The river is deep (\citealt{Pichl1967}). 

\TCheadword[2]{hial} \textit{n} dancing area. \textit{Ha ma sink walli nɔ lɛ ni puy, ihial ka nante.} Do not play with your palm branches and grass in the dancing area here today.

\TCheadword{hiɛ} \textit{disco} clause-final particle expressing desire to affirm the validity of preceding clause, cf., ‘isn't that so?' \textit{Sese Mpondo, ŋkong bali hiɛ̃?} Sese Mpondo, you are rich now, aren‘t you? (\citealt{Pichl1967}). \textit{Yamɔ pɛ wɔn Shenge ka lɔ pɔ gbem wɔ hinye?} Your mother was also born here, right? \textit{Ba-m, ŋkoŋ bali, hĩɛ?} Sir, you are rich now, isn't it so? (\citealt{Pichl1967}). 

\TCheadword[1]{hil} \textit{v} fly; \textit{hiil} fly (\textit{yil} Shenge pronunciation for \textit{hil}, to fly) (\citealt{Pichl1967}).

\TCsubword{hilk} (der.) \textit{v} fly with. \textit{Bikɛɛ hilkɛ iteɛ kanaɛ hɛthɛ.} If the wind flies with the mortar, what about the fanner?

\TCheadword[2]{hil} (Eng \textit{hill}) \textit{cf}: \TClink{tent}. \textit{n} anthill.

\TCheadword{-hil} \textit{v > v} \textit{sfx} verb extension; meaning unknown.

\TCheadword{hilk} (der. of \TClink[1]{hil}, \TClink{-k}, see \TClink[1]{hil}) 

\TCheadword{hin} \textit{cf}: \TClink[2]{hɔl}, \TClink{rɛsth}. \textit{v} \textbf{1)} lie down. \textit{Hálíwɔ̀ hìn má Yèmà, wɔ̀ pín bállɛ̀ kò Chó.} Because he slept with Yema, he paid \textit{bal} to Cho. \textit{Kaiŋ Taso wɔ jajɛl wɔɛ wuɛ, hinɛ lɔ pɛllɛai amaaɛ ntɛnt.} Kain Tasso whose mother-in-law died, lay down in the hammock near the women. \textbf{2)} lay. \textit{Kaiŋ Taso hinɛ pɛllɛai wɔ la ke ni wɔ la theeɛ.} Kain Tasso lying in the hammock saw it and heard it. \textbf{3)} rest. der. \TClink{nɔhinyɛchɛk} (see \TClink{nɔ})

\TCsubword{hiŋ-gbɔl} (comp.) \textit{v} be satisfied.\textit{ Kə ba kel ka hinɛn gbɔl.} But Mr Monkey was not satisfied (\citealt{Pichl1967}).

\TCsubword[1]{hini} (der.) \textit{v} \textbf{1)} lay down, set down.\textit{ Ŋkɔ hini chanth lɛ!} Go lay down the baby (\citealt{Pichl1967}). \textit{Poɛ hiniɛ lɛ hɔɛ hɔ pɔ pɛɭ taŋ dɛ...} They laid down a law that the day they would break off the mourning... (\citealt{Sumner1921}). \textbf{2)} decide. \textit{Kisik lɛ hã pɛ bɔni nɛn sana lɛ.} At the end they decided they would meet again in the new year (\citealt{Pichl1967}). \textbf{3)} arrange. \textit{Ŋkɔ la hini!} Go and arrange it! (\citealt{Pichl1967}). comp. \TClink{nɔhinyɛchɛk} (see \TClink{nɔ})

\TCsubword{hɔlini} (der.) \textit{n} (kɔ/-) rest, relaxation (\citealt{Pichl1967}).

\TCheadword{hina} \textit{cf}: \TClink{handɔ}. \textit{interrog} who. \textit{Hìná wɔ̀ bɛ̀mpà bìŋ dó á?} Who built this fence? \textit{Ina lɔ ba mɔa?} Who is your father?

\TCheadword{Hini} \textit{nam} Hini, name given to a person. \textit{Yɛ hu ifɔndɛ pɔ mɔi ka ilel Buɛ Hini?} When you were initiated, you were given the name Bue Hini?

\TCheadword[1]{hini} (der. of \TClink{hin}, \TClink[1]{-i}, see \TClink{hin}) 

\TCheadword[2]{hini} \textit{v} please. comp. \TClink{hini-gbɔl} (see \TClink{gbɔl}) 

\TCheadword{hini-gbɔl} (comp. of \TClink[2]{hini}, \TClink{gbɔl}, see \TClink{gbɔl}) 

\TCheadword[1]{hinth} \textit{cf}: \TClink{koi}. \textit{v} swell; \textit{hínth} be swollen (\citealt{Sumner1921}). \textit{Mma vəkɛth su-m dɛ, kɔ hinth ni lwɛ nse, mma ki-m nɛki.} Don't squeeze my finger, it will swell and suppurate, don't hurt me! (\citealt{Pichl1967}). 

\TCheadword[2]{hinth} \textit{n} bed; \textit{hinth} (kɔ/tha) bed (\citealt{Pichl1967}). \textit{A yema vikini kɛ hinth lo kɔ kith hã yaŋ.} I want to stretch but this bed is too short for me (\citealt{Pichl1967}). 

\TCheadword[3]{hinth} \textit{n} [hínth] swelling (K dialect). 

\TCheadword{hiŋ-gbɔl} (comp. of \TClink{hin}, \TClink{gbɔl}, see \TClink{hin}) 

\TCheadword{hiŋk} \textit{cf}: \TClink[1]{hok} (der. of \TClink[1]{ho}, \TClink{-k}). \textit{prep} from.

\TCheadword{hiɔl} \textit{cf}: \TClink[1]{fɔ}. \textit{Numb} four (\citealt{Sumner1921}); [yɔ̀l] cardinal four, [bɔ́ɔ́ thí yɔ̀l] four hats (B dialect). comp. \TClink{bolthihiol} (see \TClink[1]{bol}), \TClink{mɛnhiɔl} (see \TClink[1]{mɛn}), \TClink{mɛŋhiɔlniwaŋ} (see \TClink[1]{mɛn})

\TCsubword{waŋnihiɔl} (comp.) \textit{Numb} fourteen. \textit{Nɛn thiwaŋnihiɔl, gbemni Fuŋk ko.} Fourteen years old, born in Rotifunk.

\TCheadword{hip} (Eng \textit{heap}) \textit{cf}: \TClink[1]{sal}. \textit{n} heap (B dialect).

\TCheadword{histri} \textit{n} history.

\TCheadword{hmm} \textit{disco} yes.

\TCheadword[1]{ho} \textit{v} \textbf{1)} emerge, come out. \textit{Yɛ ha ka stich kun wɔɛ, ko lɔ gbemɛkɛ ŋɔ ho kaɛ...} When her belly was stitched, where the baby comes out... \textbf{2)} sprout. \textbf{3)} emerge. id. \TClink[1]{thukul}

\TCsubword{honi} (comp.) \textit{v} \textbf{1)} go out, get out. \textbf{2)} come out. \textit{Nchíndɛ̀ mà hónì fop fup.} The shit came out like foop foop.

\TCsubword[1]{hok} (der.) \textit{cf}: \TClink{hiŋk}. \textit{v} \textbf{1)} come from. \textit{Ndɔ-lɔ ŋhokɔ-a?} Where do you come from? \textit{Yà híŋk kò Bà Yànkà.} I came from Ba Yanker. \textbf{2)} come out. \textbf{3)} take out. \textit{Po mɔɛ bɛ wɔ gbo yema jo, mɔi minɛ ko wok ŋa wɔn joɛ.} If your husband also said he wants to eat, you go and take the rice out again. \textit{Ipulukɛ gbi ma lɔɛ pɔ ma lɔ koŋ hok.} All the piles (of branches and leaves) that are there are taken out. \textbf{4)} go from. \textit{Awokɔ gbo ko mɔ ko yai hun ko Mi Adama.} After leaving you, I will go to Mami Adama. \textbf{5)} originate.

\TCsubword[2]{hok} (der.) \textit{cf}: \TClink{tɔŋkwa} (der. of \TClink{tɔŋk}). \textit{v} celebrate. \textit{Haaŋ ni nante bɛ, pɔ mu tɔn tontho ki chɔl sakɛ ha hok saka wullɛ.} Even up to the present day, people still sing these songs the night of the wake.

\TCheadword[2]{ho} \textit{n} alum bark; \textit{hoo} (hɔ̃/ma) alum bark (\citealt{Pichl1967}). 

\TCheadword[3]{ho} \textit{v} \textbf{1)} of rice when all the water has evaporated or been absorbed. \textit{Joɛ kɔ ni ho, mɔi thɔk boithɛ.} After the rice is properly dry, you wash the dishes. \textit{Kɔ koŋ gbo ho, mɔi chɔŋ, nyɔk ŋa po mɔi.} When the rice is dry, then you dish it out, you take your husband's rice in. \textbf{2)} be cooked, as of rice. \textit{Kɔ lɔ boni le ton-ton te kɔi koŋ ho.} It just remains low until it has cooked.

\TCheadword{hoɛ} \textit{cf}: \TClink{pɔɔ}. \textit{n} rain; \textit{hɔɛɛ} (hɔ̃/-) rain (\citealt{Pichl1967}). \textit{Hɔɛɛ hɔ̃ dis.} The rain is heavy (\citealt{Pichl1967}). 

\TCheadword[1]{hok} (der. of \TClink[1]{ho}, \TClink{-k}, see \TClink[1]{ho}) 

\TCheadword[2]{hok} (der. of \TClink[1]{ho}, \TClink{-k}, see \TClink[1]{ho}) 

\TCheadword{holni} \textit{v} meet.

\TCheadword[1]{homabal} (unspec. of \TClink[1]{bal}) 

\TCheadword[2]{homabal} (unspec. of \TClink[2]{bal}) 

\TCheadword{homploŋ} \textit{adj} [hòmplòŋ] empty (K dialect). 

\TCheadword{honchoŋ} \textit{cf}: \TClink[1]{baŋchoŋ}. \textit{v} float.

\TCheadword{honi} (comp. of \TClink[1]{ho}, \TClink{-ni}, see \TClink[1]{ho}) 

\TCheadword[1]{hoŋ} \textit{cf}: \TClink[1]{hul}. \textit{v} blow; \textit{hoŋ} blow a whistle or horn (\citealt{Pichl1967}).

\TCheadword[2]{hoŋ} \textit{n} whistle; \textit{hoŋ hoyuŋ} whistle (\citealt{Pichl1967}). 

\TCheadword[3]{hoŋ} \textit{n} \textbf{1)} home. \textit{Woŋ mɔ koa?} What about your house? \textbf{2)} compound. \textit{Ŋkɔ hoŋ ko ni mpin sɔk shiliŋ thiwaŋ.} Go to the compound and buy a fowl for ten shillings.

\TCheadword[1]{hoŋka} \textit{cf}: \TClink{bɔko}, \TClink{kahai}, \TClink[1]{kaŋ}. \textit{n} \textit{honka} (hɔ̃/tha) open area (\citealt{Pichl1967}).

\TCheadword[2]{hoŋka} \textit{Loc} outside. \textit{Ŋkɔ lath kotha-thi lɛ honka lɛ ay.} Go spread the clothes outside (\citealt{Pichl1967}). 

\TCheadword{hoo} [hòò] \textit{n} tree species with roots used for medicine, stomach ailments, extremely bitter (K dialect). 

\TCheadword{hopa} \textit{cf}: \TClink{yekɛ}. \textit{n} cassava field; \textit{hopa} (hɔ̃/tha) cassava farm where, after harvesting all other fruits, only cassava remains (\citealt{Pichl1967}). 

\TCheadword{hopabai} \textit{n} market; \textit{hopa bai} (hɔ̃/tha) main market, big market (\citealt{Pichl1967}). 

\TCheadword{hosni} (der. of \TClink{bus}, \TClink{-ni}, see \TClink{bus}) 

\TCheadword[1]{hoth} [hóth] \textit{cf}: \TClink{nyɛ}. \textit{n} chaff, part of rice that is blown away (K dialect). 

\TCheadword[2]{hoth} [hóth] \textit{n} gap in teeth (K dialect). 

\TCheadword[3]{hoth} [hóth] \textit{v} fill mud in lattice of sticks, mud-and-wattle construction (K dialect). 

\TCheadword{hothɔk} \textit{cf}: \TClink{jɔhɔ}, \TClink{pakali} (der. of \TClink{pakil}, \TClink[1]{-i}), \TClink{sɔyɛ}, \TClink{woli} (der. of \TClink[1]{woi}, \TClink[1]{-i}). \textit{v} \textbf{1)} frighten; \textit{} frighten by shouting suddenly (\citealt{Pichl1967}). \textbf{2)} take unawares. \textit{Min dɛ hothkɔ wɔ.} The devil took her unawares (\citealt{Pichl1967}).

\TCheadword[1]{hoy} \textit{v} make oil; \textit{ho̹y} make oil (\citealt{Pichl1967}).

\TCheadword[2]{hoy} \textit{n} oil fabrication; \textit{ho̹y} (?/-) oil fabrication (\citealt{Pichl1967}). 

\TCsubword{hoykɔkɔ} (comp.) \textit{n} coconut oil fabrication; \textit{ho̹y kɔkɔ} coconut oil fabrication (\citealt{Pichl1967}). 

\TCsubword{hoymbɛl} (comp.) \textit{n} palm oil fabrication; \textit{ho̹y mbəl} palm oil fabrication (\citealt{Pichl1967}). 

\TCheadword{hoymbɛl} (comp. of \TClink[2]{hoy}, \TClink[2]{bɛl}, see \TClink[2]{hoy}) 

\TCheadword[1]{hɔ} \textit{cf}: \TClink[2]{chɛli}, \TClink{fothi}, \TClink{gbemani}, \TClink{ku}, \TClink[1]{lem}, \TClink{theli}, \TClink{wɛ}, \TClink[2]{wɔni} (der. of \TClink[1]{hɔ}, \TClink{-ni}). \textit{v} \textbf{1)} say. \textit{Lɛ nwɔ gbo, ŋa mɔi, ŋan ŋa wɔ “bua.”} If you say to them, \textit{mɔi} (‘Good afternoon' in Sherbro), they will say, \textit{bua} (‘Greetings' in Mende). \textit{Kɛ la hɔn dɛ, nyiki'ɛ gbi ma pɔ ka yuk ŋchɛkai mɛŋkɛ vɛɛ yelloaiɛ.} But it is said, all these seeds they planted at the farm at that time on this island. \textit{Ni ŋa che tɔn, ya mɔe hɔmɛ, nthol huai-huai ni ŋkɔ kue yekeɛ ni ŋchii.} And they are singing, then I said to you, go down quietly and take the cassava and bring it (back). \textit{Bɛl Pokan dɛ wɔe gbaki ni hɔ ko laa wɔɛ, “Ndɛli la mɔm hɔm dɛ waata-o.”} Rat Husband answered saying to his wife, “Watch what you are saying, Girly-o!” \textbf{2)} tell. \textit{Li pika la ayema ni nwɔmi ŋa iwɔlɔŋ mɔɛ.} The next thing I would like you to do is to tell us the story of your life. \textbf{3)} talk. \textit{Kɛn bo bi ŋsɔkba la mɔ tenɛ, ha mɔn wɔ...} If you have a problem in mind and you want to talk... \textbf{4)} speak. \textit{Yaŋ bɛ pɛ ŋami hɔ ko mi ko haŋ.} They spoke to me for a long time. \textit{Yɛ mɔ hɔ Mbolomdɛ motoɛ kunɛ, nɔnɔ wɔ thimni wɔi yi mɔ Bolomnɔ?} When you speak Sherbro in a vehicle, everybody will turn and ask, are you Sherbro? \textbf{5)} call. \textit{Pɔ gbem mi paŋdɛ ŋɔ pɔ wɔ Sɛptɛmbaɛ, paŋ mɔikɛ mɛnyɔllɛ.} I was born in the month that they call September, the ninth month. \textit{Wɛl atipɛ tɔn nɛndɛ ŋɔ Apothoɛ ŋa wɔ 2013, te mɛŋko ki amu tɔndai.} Well, I started singing in the year that white people called 2013, up to this year I'm still singing. \textit{Nande a ko wun ko laŋgbandɛ ki ni a huŋ wɔ yi ŋalwɔ atokɛ.} Today I have called this man to come and to ask him about himself. \textbf{6)} prove. \textit{Nɔɛ wɔ hu ni la hɔndɛ wɔ fɔnwɔiɛ, pɔ che bia ha thisiŋ pɔ che mɛmini.} The person died and it was proven that he was a witch, they could not make merry, they were not happy. comp. \TClink{hɔbatokɛ} (see \TClink[1]{tok}), \TClink{woŋhɔ} (see \TClink[2]{woŋ}), comp., der. \TClink{ŋɔhɔlpok} (see \TClink{nɔ}), comp., der., id. \TClink{nɔŋhɔ} (see \TClink{nɔ})

\TCsubword[2]{hɔ} (der.) \textit{cf}: \TClink[1]{thoŋka}. \textit{n} \textbf{1)} voice. \textbf{2)} quarrel. \textit{Ŋani po wɛ ŋa bi mu nwɔ ton-ton, kɛ ŋa sɛiɛ ni mu o, ŋalɔ mu.} She and her husband had a small quarrel, but they have not separated, they are still there. \textbf{3)} palaver. \textit{Wɔ̀ lɔ́ hàà bɛ́mpá.} She is there to settle palavers.

\TCsubword{hɔma} (der.) \textit{v} \textbf{1)} compromise. \textbf{2)} argue with. \textit{Bɛl Pokan dɛ: “Mba yaŋ ya mɔ hɔm vɛ?”} Rat Husband: “Woman, is it me you are abusing like that?” \textbf{3)} talk with.

\TCsubword{hɔn} (der.) \textit{v} say. \textit{...haliwɔ la koŋ hɔn dɛ koŋ lee mmɛn dɛai.} ...because it had been said that he had drowned in the water. 

\TCsubword{hɔni} (der.) \textit{v} say to oneself. \textit{Lanɛ la hɔni thɔndɛaiɛ, la chen hɔni bandɛai.} What you say to yourself while bathing is not what you say to yourself while getting dressed (proverb) (\citealt{TISLL1979}). 

\TCsubword[2]{wɔk} (der.) \textit{n} \textbf{1)} word. \textbf{2)} language. \textit{Shenge ka nwɔk ndɔ ma pɔ chan thelia?} In Shenge here what language do they speak more? \textit{Nwɔk nra ma pɔ chaŋ theliɛ, Mbolomdɛ, Mmɛndeɛ ni Nthemdɛ.} They speak three languages here: Sherbro, Mende and Themne. \textbf{3)} case. \textit{Yɛmɔ theli ko aŋaɛ, nwɔk mpim ma pɔ chi komɔko ma che ndumɔ, nye?} When you talk to the people, some cases they bring to you are difficult, right?

\TCsubword[2]{wɔni} (der.) \textit{cf}: \TClink{gbemani}, \TClink[1]{hɔ}, \TClink[1]{lem}, \TClink{theli}, \TClink{wɛ}. \textit{v} speak. \textit{Achɔn ma len eh, bikɔs amɔs wɔni ɛ nwɔkɛ ma pɔ yemaɛ mavɛ Mbɛkɛ vɛ.} I like it, because I must say the language they want, it is that Krio.

\TCheadword[3]{hɔ} \textit{cf}: \TClink[5]{che}, \TClink[2]{la}, \TClink[2]{lɛ}, \TClink[4]{ni}, \TClink[3]{ŋa}, \TClink[1]{yɛ}. \textit{subordconn} \textbf{1)} it is. \textit{Ŋɔ koŋ gbo we ha che mi paka.} That is all, but they do not pay me. \textbf{2)} why, how. \textit{Ŋɔ̃́ŋ lɔ̀là?} How did you sleep? \textbf{3)} that. \textit{Ko ni bo sɔn yawɔ, yawɔ wɔ wɔi wɔm tɛm dɛ ŋbi ŋɔ wɔ theni bo ndik ni kɔ ni pɔyko.} As she dreamt of her mother, her mother told her that anytime she is hungry, she should go to the stream. \textbf{4)} when. \textit{Ŋɔ pɔ ŋamɔ spika?} When you were made Speaker? \textbf{5)} then. \textit{Ka hun hɔth ŋɔ hɔ ka ke yami.} He came to fish, and then he met my mother. \textit{Wɛl yami ka bɛmi skul kɛ akɔni livil, ŋɔ aka mɛl ayi pɔni ŋɔthɛ kunɛ.} Well, my mother sent me to school but I didn't go far, then I left and involved myself in fishing.

\TCheadword[4]{hɔ} \textbf{1)} \textit{indfpro} it. \textit{Bòlmìɛ́ ŋɔ̃́kì.} This is my head. \textit{Yɛ mɔ ŋɔ hun thɔŋul vɛ, lɛ apimaɛ ŋa siŋɛ-siŋɛ gbo haŋ lɛ ŋa wɔ bo ŋa yema jo…} As you keep it reserved, after the children played around, if they say they want to eat… \textbf{2)} \textit{NCP} it. \textit{Bithir lɛ hɔ̃ beye̹n.} The bottle is empty (\citealt{Pichl1967}). \textit{Pan dɛ hɔ dinth.} The moon shines (\citealt{Pichl1967}). \textit{Ŋ kwey sangba nyok lo ni nsïk hɔ Yema gbɔl!} Take this string of corals and tie it on Yema's neck! (\citealt{Pichl1967}). \textbf{3)} \textit{NCP} relative pronoun: that/which. \textit{Yi koni shi temdɛ ŋɔ pɔ gbem mɔ, ko lɔ pɔ gbemmɔ?} We already know when you were born, where were you born? \textit{Ya mɔ ka mɔ yendɛ gbi ŋɔ yemai.} I give you everything that you want.

\TCheadword[5]{hɔ} \textit{cf}: \TClink{handɔ}, \TClink[1]{la}, \TClink{ndɔ}, \TClink[5]{ŋa}. \textit{interrog} what. \textit{Ŋɔwɔ ka che huŋ ŋaa?} What did he used to come and do? \textit{Bɛlsa ŋɔɛ handɔ ŋa hɔ si ŋa thee la?} What rat will speak and you understand it?

\TCheadword[6]{hɔ} \textit{cf}: \TClink{hɛŋwɛi} (comp. of \TClink{hɛŋ}, \TClink[1]{wɛi} (der. of \TClink[2]{wɛi})) \textit{n} \textbf{1)} weather. \textit{A lomani yɛ Ba Ŋgobɛ ka che hun dɛ hwɛ lɛ hɔ lelɛ.} I remember when Mr. Ngobe was coming that it rained (\citealt{Pichl1967}). \textbf{2)} rain.

\TCheadword{hɔa} [hɔ̀à] \textit{n} tree species, leaves used for medicine (K dialect). 

\TCheadword{hɔbatokɛ} (comp. of \TClink[1]{hɔ}, \TClink[1]{ba}, \TClink[1]{tokɛ} (der. of \TClink[1]{tok}, \TClink[1]{ɛ}), see \TClink[1]{tok}) 

\TCheadword{hɔhɔ} \textit{n} (wɔ/hã, N) fish species, 6 inches long with strong jaws, lives beneath cliffs (\citealt{Pichl1967}). 

\TCheadword{hɔima} \textit{cf}: \TClink[1]{kump}, \TClink{nɔhampanth} (comp. of \TClink{nɔ}, \TClink{haa}, \TClink[1]{panth}). \textit{n} \textit{ŋhɔyma} (-/ha) workers for hire for harvesting rice; a union of men who hire themselves out for harvest work (\citealt{Pichl1967}). 

\TCheadword{hɔkɔ} \textit{n} goiter. \textit{Ihɔkɔ hã lɛ bombom.} Their goiters are big (\citealt{Pichl1967}). 

\TCheadword[1]{hɔl} \textit{v} \textbf{1)} insult. \textit{Yɛlkənth lo wɔ-m hɔl.} This puny fellow insults me (\citealt{Pichl1967}). \textbf{2)} \textit{hɔ́l} rebuke (\citealt{Sumner1921}); \textit{hɔ̃l} rebuke, scold (\citealt{Pichl1967}). 

\TCheadword[2]{hɔl} \textit{cf}: \TClink{hin}, \TClink{rɛsth}. \textit{v} breathe. \textit{Ha bɔnthɔ baha yinɛ ni che hɔl lok, lok.} They met their father lying breathing with great difficulty (\citealt{Sumner1921}). \textit{Kɔ bimni sɔku bullai, wɔ hɔɔl <fɔɔ fɔɔ fɔɔ> ni yeke wɔɛ che wɔn kunɔlɔ.} (She) went and bent over in one corner, she breathed <fɔɔ fɔɔ fɔɔ> (idph of panting) with the cassava (tucked) in her bosom.

\TCsubword[2]{hɔlɔŋ} (comp.) \textit{n} life. \textit{Ihɔlɔŋ hɔ gbo thanthɛn.} Life is (just) in vain (\citealt{Pichl1967}). \textit{Li pika la ayema ni nwɔmi ŋa iwɔlɔŋ mɔɛ.} The next thing I would like you to do is to tell us the story of your life. \textit{Ihɔlɔŋ hɔ imɔl.} Life is sad (\citealt{Pichl1967}). comp. \TClink{baŋkihɔlɔŋ} (see \TClink[2]{baŋk}), \TClink{nanihɔlɔŋ} (see \TClink{nan}), \TClink{thɔkihɔɔlɔŋ} (see \TClink[2]{thɔk}), \TClink{yenbiɛihɔlɔŋ} (see \TClink[1]{yen}) 

\TCsubword{hɔlɛ} (der.) \textit{n} whisper. \textit{Ya hun mɔ hɔm thihɔlɛ-hɔla.} I come to tell you something secretly (\citealt{Pichl1967}). 

\TCheadword[3]{hɔl} \textit{n} eye, [wɔ́l]/[wɔ́llɛ̀]/[wɔ̀lthɛ́] eye/ the eye/the eyes (B dialect). \textit{Kɔŋdɛ kɔ akekɛ thiwɔllɛ yɛ laiyoɛ hɔ cheni pɛ bul.} The burial that I have seen (with my) eyes, it is not still the same. \textit{Nɔonɔ yellɛ ko ŋa hun ha kek Braima thihɔl...} Everyone on the island came to see Braima with (thier own) eyes... comp. \TClink{pɛlmahɔl} (see \TClink[1]{pɛl}) 

\TCsubword{tɔkɔli} (unspec.) \textit{v} hurt someone's eye.

\TCheadword[4]{hɔl} \textit{cf}: \TClink{-ai}, \TClink{kunɛ} (der. of \TClink{kun}, \TClink[1]{ɛ}). \textit{post} into, inside. \textit{Pəŋ huu lɛ ni kɔni kïl lɛ hɔl ko.} He jumped over the fence and went into the house (\citealt{Pichl1967}). 

\TCheadword[5]{hɔl} \textit{n} rest; \textit{ŋhɔ̃l} (ma) rest (\citealt{Pichl1967}). \textit{Lami wɔ abi habi wɔn pɛ wɔ che wɔl libul-libul.} My wife I would rest some time (\citealt{Pichl1967}). 

\TCheadword{hɔlɛ} (der. of \TClink[2]{hɔl}, \TClink{-ɛ}, see \TClink[2]{hɔl}) 

\TCheadword{hɔlide} (Eng \textit{holiday}) \textit{temp} holiday time. \textit{A kɔ lɔni pɛ haŋ ya ko kɔni fɔm wan, ya pɛ tipɛ kɔ hɔlide.} I did not go there again until I went to form one, then I started going for holidays again.

\TCheadword{hɔlini} (der. of \TClink[5]{hɔl}, \TClink{hin}, see \TClink{hin}) 

\TCheadword[1]{hɔlɔŋ} \textit{cf}: \TClink[2]{baa}, \TClink{balmaa}, \TClink{boka}. \textit{n} (hɔ̃/tha) curved knife used for cutting palm trees for palm wine (syn. \textit{iba}) (\citealt{Pichl1967}). 

\TCheadword[2]{hɔlɔŋ} (comp. of \TClink[2]{hɔl})

\TCheadword{hɔm} \textit{prep} from.

\TCheadword{hɔma} (der. of \TClink[1]{hɔ}, \TClink[4]{ma}, see \TClink[1]{hɔ}) 

\TCheadword{hɔn} (der. of \TClink[1]{hɔ})

\TCheadword{hɔni} (der. of \TClink[1]{hɔ}, \TClink{-ni}, see \TClink[1]{hɔ})

\TCheadword{hɔntlɔn} \textit{n} plant species; \textit{hɔntlɔn} (hɔ̃/tha or hɔ̃/hɔ̃, i) beach convolvus, long rope-like trailer with 2-3 large purple flowers on each stalk (\citealt{Pichl1967}). 

\TCheadword{hɔspith} (Eng \textit{hospice}) \textit{cf}: \TClink{kilpekɛ} (comp. of \TClink[1]{kil}, \TClink{pekɛ}). \textit{n} hospital. \textit{Nɔs gbi ŋa ka cheni eriaio ai, hɔspitalai fli nɔs ka che ŋa ni.} There was no nurse in that whole area, even in the hospital there was no nurse. \textit{Ya che ko taallɛ, acheni ve, ya naka naka, pɔ mi yɔk hɔspithai ni asoŋ.} When I was a young, I was not well, they took me to the hospital to get well.

\TCheadword[1]{hɔth} (der. of \TClink[2]{hɔth}) 

\TCheadword[2]{hɔth} \textit{cf}: \TClink[1]{di}. \textit{v} fish. \textit{Ka hun hɔth ŋɔ hɔ ka ke yami.} He came to fish, and then he met my mother. comp. \TClink{nɔhɔnthɛ} (see \TClink{nɔ}), \TClink{yuhɔtka} (see \TClink{yu}) 

\TCsubword[1]{hɔth} (der.) \textit{n} fishing. \textit{A shi ŋɔth kɛ ache kɔ hɛlɛ.} I know how to fish but I do not go out on the seas.

\TCsubword{hɔthka} (der.) \textit{n} fishing. \textit{ŋa wɔe pel mawɔm wɔɛ, wɔm hɔthka bom dɛai ŋɔ bi yiŋjin dɛ.} They loaded it into his boat, the big fishing boat that has an engine. 

\TCheadword{hɔthka} (der. of \TClink[2]{hɔth}, \TClink{-k}, see \TClink[2]{hɔth}) 

\TCheadword[1]{hu} \textit{n} \textbf{1)} yard; \textit{huu} (kɔ/tha?) yard, corral (\citealt{Pichl1967}). \textbf{2)} fence. \textit{Pəŋ huu lɛ ni kɔni kïl lɛ hɔl ko.} He jumped over the fence and went into the house (\citealt{Pichl1967}). 

\TCsubword{hunyiki} (comp.) \textit{n} plantation; \textit{huu nyiiki} (kɔ/tha?) plantation (\citealt{Pichl1967}). 

\TCsubword{husɔk} (comp.) \textit{n} chicken yard; \textit{huu sɔk} (kɔ/tha?) fowl yard (\citealt{Pichl1967}). 

\TCsubword{huvis} (comp.) \textit{n} cattle yard; \textit{huu vis} (kɔ/tha?) cattle yard (\citealt{Pichl1967}). 

\TCheadword[2]{hu} \textit{cf}: \TClink[2]{isɔ}, \TClink[1]{saaka}. \textit{n} \textbf{1)} day. \textit{Hùɛ́ɛ̀ ŋɔ́ [hɔ̃] bɔ̀s/bɔ̀súl.} The day is cold. \textit{Iyema mɔ wɛyowɛ.} We need you every day. \textbf{2)} morning. \textit{Wɔyɛ ŋɔ keni gbo, apa wɔ tɛnini.} Early in the morning, father will do some thinking. der. \TClink{wɔiowɔi} (see \TClink{-o-}\nobreakdash)

\TCsubword{hwɛpi} (comp.) \textit{cf}: \TClink{rithilɛhɔl} (comp. of \TClink[2]{rithi}, \TClink[1]{ahɔl}). \textit{temp} dusk. comp. \TClink{kɔsmahwɛ} (see \TClink{kɔs})

\TCheadword{huaihuai} (der. of \TClink[4]{wai}) 

\TCheadword{huɛ} \textit{v} be glad (\citealt{Sumner1921}). 

\TCheadword{huɛŋ} \textit{cf}: \TClink[1]{mu}, \TClink[1]{ni}, \TClink[3]{pɛ}, \TClink{stil}. \textit{temp} yet, still. \textit{Huɛŋ anyaɛ ŋa gbiŋkithɛni feɛ.} Still they did not cover the money. 

\TCheadword{huhuu} \textit{Idph} animal cry. \textit{Tumgbula lɛ wɔ hɔ chɔl lɛ, nthe lom wɔ lɛ hɔ ki ɛ hu-huu.} The \textit{tumgbula} cries in the night, you will hear his voice, it is so, hú-huù (\citealt{Pichl1967}).

\TCheadword[1]{huk} (Eng \textit{hook}) \textit{cf}: \TClink{seko}. \textit{n} hook. \textit{A chen duki pɛl, nhukɛ ma a dukiɛ.} I do not use a net, I use hooks.

\TCheadword[2]{huk} \textit{n} bush spider. \textit{Huksi atïŋ hã che kïl lɛ kunɛ.} There are two bush spiders in the house (\citealt{Pichl1967}). 

\TCheadword{hukum} \textit{n} wasp species (K dialect); (wɔ/hã, N) kind of bush wasp (\citealt{Pichl1967}).

\TCheadword[1]{hul} \textit{cf}: \TClink[1]{hoŋ}. \textit{v} blow on for cooling, e.g., a child's cut (\citealt{Pichl1967}). 

\TCheadword[2]{hul} \textit{cf}: \TClink{kuum}, \TClink{mutmut}. \textit{n} mosquito (K dialect); \textit{hũl} (wɔ/hã, i) mosquito (\citealt{Pichl1967}). 

\TCheadword{humoe} \textit{cf}: \TClink{bɔfima}, \TClink[1]{fɔnwɛi} (comp. of \TClink[1]{wɛi}), \TClink{mane}, \TClink[3]{wɔm}, \TClink{yasi}. \textit{n} cleansing, purifying medicine used to “wash” guilty parties (lovers who copulate in the bush) and the bush for farming after sacrilege (\citealt{Hall1938}). 

\TCheadword{humɔ} \textit{v} send for. \textit{So lagboɛ nɔ wu, ramdɛ kɔ kɔ lomthibul pɔi humɔ nɔɛ vɛ.} So if a person dies, the family will make a unanimous agreement and send for that person.

\TCheadword{Humwɛ} \textit{n} Humwe Society, a mixed society; \textit{humwɛ} (kɔ) a mixed society that benefits the washing of the bush to secure good crops (\citealt{Pichl1967}). 

\TCheadword[1]{hun} \textit{cf}: \TClink[5]{min}, \TClink{muni}, \TClink{muŋk}. \textit{v} \textbf{1)} come. \textit{Wɔ bɛ hun.} He is just now coming (\citealt{Pichl1967}). \textit{Tɛmdo mɔ choŋ hun a?} At about what time will you come? (\citealt{Pichl1967}). \textit{A yema ni i wun ko ja Mbolomdɛ.} I want us now to come to Bolom matters. \textbf{2)} become. \textit{Wɛl awun Spika 2013.} Well, I became Speaker in 2013. \textbf{3)} return. \textbf{4)} start on.

\TCsubword[3]{hun} (der.) \textit{n} visit, trip. \textit{Hun sendɛ ŋɔ hundɛ, hun 1978.} The first time he came was in 1978.

\TCheadword[2]{hun} \textit{Aux} Incipient particle on its way to becoming grammaticalized. \textit{Wɔ ye hun hɔɛ, ntheɛ bip?} Then he asked, “Did you hear the fart?” (\citealt{Pichl1967}). \textit{Velen thilandɛ hun gbo le chal ka ni kunɛ ŋɔ wɔi nɛki.} After that (she) just sat and felt her delivery pain. \textit{Mɔ yi hun toŋgi ŋɔ pɔ chɛth pɔmthi gbamdɛ.} You should now come and show us how to cook potato leaves.

\TCheadword[3]{hun} (der. of \TClink[1]{hun}) 

\TCheadword{hunyiki} (comp. of \TClink[1]{hu}, \TClink[3]{yiki}, see \TClink[1]{hu}) 

\TCheadword{husɔk} (comp. of \TClink[1]{hu}, \TClink{sɔk}, see \TClink[1]{hu}) 

\TCheadword{huth} \textit{v} come of age. \textit{Gɔmɛnt lɛ hã thoŋkiɛ lɛ hã yema hã saba, che lɛ tamɔ pokan gbi wɔ koŋ huth lɛ, wɔ hã paka pɔn bul hã bol wɔ lɛ.} The government has proclaimed that they want to make a law that every young man who has come of age has to pay one pound as a head-tax (\citealt{Pichl1967}). comp. \TClink{nɔmɔkhuth} (see \TClink{nɔmɔk}) 

\TCheadword{huvis} (comp. of \TClink[1]{hu}, \TClink{vis}, see \TClink[1]{hu}) 

\TCheadword{hwai} \textit{cf}: \TClink{lɛlɛ}. \textit{temp} slowly.

\TCsubword{hwaihwai} (der.) \textit{temp} slowly.

\TCheadword{hwaihwai} (der. of \TClink{hwai}) 

\TCheadword{hwaini} \textit{adj} have a disfigured nose like a leper (caused by Toma medicine and also healed by it) (\citealt{Pichl1967}).

\TCheadword{hwe} \textit{cf}: \TClink[2]{pukɔ}. \textit{n} \textbf{1)} froth, (kɔ) froth (\citealt{Pichl1967}). \textbf{2)} foam. \textit{Məndɛ ma hɔ̃ hwe.} There is foam on the water (\citealt{Pichl1967}). 

\TCheadword{hwɛpi} (comp. of \TClink[2]{hu}, \TClink[1]{pi}, see \TClink[2]{hu})

\TCheadword{hwɛwɛi} (comp. of \TClink[1]{wɛi} (der. of \TClink[2]{wɛi}), see \TClink[2]{wɛi}) 


\end{letter}


\begin{letter}{I}

\TCheadword{i-} \textit{NCM} \textit{pfx} noun class marker (hɔ). \textit{Ilel mɔa?} What is your name? \textit{Mɔi bɛ ituɛ kunɛ.} You put it in the pot. \textit{Akoŋ gbo bas, adikilɛ gbo ipulukɛ ai le yini achaŋ-chaŋ tiko.} After sweeping, I will gather the dirty clothes and then leave them there and travel about town. der. \TClink{ipal} (see \TClink[1]{pal})

\TCheadword[1]{-i} \textit{v} \textit{sfx} \textbf{1)} causative. \textbf{2)} repeated action. der. \TClink{bɛsekiɛni} (see \TClink[2]{bɛl}), \TClink{bosi} (see \TClink[2]{bos}), \TClink{bɔsɔli} (see \TClink[2]{bɔs}), \TClink{duki} (see \TClink{duk}), \TClink[1]{gbemi} (see \TClink{gbem}), \TClink[2]{gbemi} (see \TClink{gbem}), \TClink{gbenik} (see \TClink{gbem}), \TClink{gbɛɛtigbɛɛti} (see \TClink[1]{gbet}), \TClink[1]{hini} (see \TClink{hin}), \TClink{jɛthɛli} (see \TClink[2]{jɛth}), \TClink{jɛthɛlini} (see \TClink[2]{jɛth}), \TClink{kabani} (see \TClink{kaban}), \TClink{kɛnthi} (see \TClink{kɛnth}), \TClink{lepi} (see \TClink[1]{lap}), \TClink{lɛpi} (see \TClink[1]{lap}), \TClink{loli} (see \TClink[2]{lol}), \TClink[1]{mani} (see \TClink[2]{man}), \TClink{mɛnklɛni} (see \TClink{mɛnkilɛn}), \TClink{nani} (see \TClink{nan}), \TClink{nɛkɛli} (see \TClink[1]{nak}), \TClink{nɛki} (see \TClink[1]{nak}), \TClink{nɔhinyɛchɛk} (see \TClink{nɔ}), \TClink{nɔloliɛ} (see \TClink{nɔ}), \TClink{nyumi} (see \TClink[1]{nyum}), \TClink{pakali} (see \TClink{pakil}), \TClink{pɛni} (see \TClink[2]{pɛn}), \TClink{pɔkɔni} (see \TClink{pɔkɔn}), \TClink{pɔŋki} (see \TClink[2]{pɔŋ}), \TClink{pɔŋkiɛn} (see \TClink[2]{pɔŋ}), \TClink{puthi} (see \TClink[3]{puth}), \TClink{puthuli} (see \TClink[3]{puth}), \TClink{puthuni} (see \TClink{puthun}), \TClink{rɛthi} (see \TClink{rɛth}), \TClink{rimi} (see \TClink[1]{rim}), \TClink{rɔki} (see \TClink{rɔk}), \TClink{sekitini} (see \TClink{sek}), \TClink{sɛmi} (see \TClink[1]{sɛm}), \TClink{siŋi} (see \TClink[2]{siŋ}), \TClink{sonthuli} (see \TClink[1]{sɔnth}), \TClink{soŋki} (see \TClink[1]{soŋk}), \TClink{sɔnthi} (see \TClink[1]{sɔnth}), {}\TClink[1]{tɛni} (see \TClink[1]{tɛn}), \TClink[2]{tɛni} (see \TClink[1]{tɛn}), \TClink{tɛnin} (see \TClink[1]{tɛn}), \TClink{tɛnini} (see \TClink[1]{tɛn}), \TClink{tɛŋkɛn} (see \TClink[1]{tɛn}), \TClink{tuki} (see \TClink{tuk}), \TClink{theki} (see \TClink{the}), \TClink{theyɛn-nɛki} (see \TClink[1]{nak}), \TClink[1]{thɛki} (see \TClink{thak}), \TClink{thɛkini} (see \TClink{thak}), \TClink{thimini} (see \TClink{thim}), \TClink{thiŋgi} (see \TClink{thɛŋk}), \TClink{tholi} (see \TClink{thol}), \TClink{thoŋki} (see \TClink[1]{thɔŋka}), \TClink{thoŋkini} (see \TClink[1]{thɔŋka}), \TClink{thukuli} (see \TClink{thuk}), \TClink{woli} (see \TClink[1]{woi}), \TClink{wuŋki} (see \TClink{wuŋk}), \TClink{yuki} (see \TClink{yuk}), unspec. \TClink{bɔyi} (see \TClink[2]{bɔi}), \TClink{dinthi} (see \TClink{dinth}), \TClink{tholiɛpɔ} (see \TClink{thol})

\TCheadword[2]{-i} \textit{cf}: \TClink{ɛn}, \TClink[1]{kɛ}, \TClink[4]{la}, \TClink[1]{o}. \textit{coordconn} \textbf{1)} a conjunction particle, usually attached to a pronoun early in a clause, usually subject but also object. \textit{Wɔi pɛ muni wɔi hun gbemɔ wantemdɛ ka ŋɔ ba mi ka wuwɛ.} She came back here to deliver my sister when my father died. \textit{So wɔi munini, wɔi pɛ mina hun 1980.} So he returned, then he came back in 1980. \textit{Anyaɛ ŋani gbo vel yel lo ɛ Planti ko haaŋ ni manante.} People have been calling it Plantain ever since. \textbf{2)} then. \textit{Pɔi hun saŋ pɛlɛ.} Then they come and scatter the rice. \textit{Lagbo pɔnthai lɔi pɔ gbusa.} If it is in the swamp, they will dig (plow) it. \textit{Pɔ koŋ gbo raa pɔi piŋgi kaŋka inallɛ lɔ ŋa ni kɛlɛn.} After brushing, they have to turn over the soil so that it becomes clean.

\TCheadword[3]{-i} \textit{pers} \textit{sfx} subject pronoun. \textit{Mɔ ŋa koi ndumma mɔe ma pɔ dumɔ mɔi.} You should take the character you were raised up with. \textit{Laŋgba lo, Jɔn Planten, wɔe munini pɔk wɔɛ Potho ko.} This man, John Plantain, then returned to his country, to the whites.

\TCheadword[4]{-i} \textit{pro-form} \textit{sfx} \textbf{1)} obj pro. \textit{Pɔ koŋ kɔ gbo futh, pɔ kɔi panth thiban pɔ woth kɔ bolɛ.} After they have uprooted it, they have to tie it into a sheaf and carry it on the head. \textit{Mɔi chal ni nkoŋkɔ kɛn yeŋkɛlɛŋ, mɔ kɔi bɛ pandɛ kunɛ.} You now sit and cut them nicely, then you put them in a pan. \textbf{2)} possessive particle. \textit{Apuma mɔi ŋa bɛŋsin no we.} Your children are suffering a lot. comp. \TClink{wɔŋgbenawi} (see \TClink{wɔŋ}) 

\TCheadword[1]{ibɔl} [ìbɔ́l] \textit{adj} high identical in meaning to \textit{víl} (K dialect). 

\TCheadword[2]{ibɔl} \textit{post} along. comp. \TClink{naibɔl} (see \TClink[1]{nai}) 

\TCheadword{Ibrahim} \textit{nam} Ibrahim, male name given to a person. \textit{Ba mi ilel wɔ ŋɔ Ibrahim Kumba.} My father's name is Ibrahim Kumba.

\TCheadword{Idrisa} \textit{nam} Idrissa, male name given to a person. \textit{Sufian Idrisa Koroma.} Suffian Idrissa Koroma.

\TCheadword{-il} \textit{v > adj} \textit{sfx} changes verbs into adjectives, verb extension? der. \TClink{bolkathil} (see \TClink[1]{bol}), \TClink{dinthil} (see \TClink{dinth}), \TClink{disil} (see \TClink[1]{dis}), \TClink{disildisil} (see \TClink[1]{dis}), \TClink[2]{jɛthil} (see \TClink[2]{jɛth}), \TClink{jɛthɛli} (see \TClink[2]{jɛth}), \TClink{jɛthɛlini} (see \TClink[2]{jɛth}), \TClink[1]{kathil} (see \TClink{kath}), \TClink{lemil} (see \TClink[1]{lem}), \TClink{pɛthil} (see \TClink{pɛth}), \TClink[3]{sɛkil} (see \TClink[2]{sak}), \TClink[1]{sɛkil} (see \TClink[2]{sɛk}), \TClink[2]{sɛkil} (see \TClink[2]{sɛk}), \TClink{sɛmil} (see \TClink[1]{sɛm}), \TClink{siŋil} (see \TClink[2]{siŋ})

\TCheadword{in} \textit{disco} no.

\TCheadword{infat} \textit{disco} in fact.

\TCheadword{influɛnsa} (Eng \textit{influenza}) \textit{n} influenza.

\TCheadword{injɛk} (Eng \textit{inject}) \textit{v} inject. \textit{Bɛyɛ wɔn ayɛnaɛ hun, hun wɔŋ injɛkshɔn, bikɔs yaŋ ache injɛk a siŋɔ ni.} The chief himself came, he came and gave an injection because I do not know how to do it.

\TCheadword{injɛkshɔn} (Eng \textit{injection}) \textit{n} injection. \textit{Bɛyɛ wɔn ayɛnaɛ fli wɔi hun wɔŋ injɛkshɔnaa, wan thɛmdɛ wɔi huŋ hu.} The chief came and no sooner he came and injected the girl, the girl died.

\TCheadword{inshɔ} \textit{v} insure. \textit{Pɔ nɔi koŋ ka inshɔ, tɛmdɛ vɛ pɔ nɔi hɔm lɛ, haŋ ha thunɔ thaozin waŋ.} They would have given assurances, when they tell you the bride price is ten thousand.

\TCheadword{intrɛst} (Eng \textit{interest}) \textit{n} interest. \textit{Ŋa bi intrɛst ko lanɛ laŋ nsiɛ.} They have an interest in what you know.

\TCheadword{Iŋglan} (Eng \textit{England}) \textit{nam} England, name given to a place. \textit{Pim nɔ wɔ sɔtha nten Inglan la atheliɛ komɔko.} Maybe someone in England will understand what I said to you.

\TCheadword{ipal} (der. of \TClink{i-}, \TClink[1]{pal}, see \TClink[1]{pal}) 

\TCheadword{Isata} \textit{nam} Isata, female name given to a person. \textit{Ama ŋa Kadiatu Bɛndu, Isata Bɛndu, Ramatu Bɛndu ni Aminata Bɛndu.} The women are Kadiatu Bendu, Isata Bendu, Ramatu Bendu and Aminata Bendu.

\TCheadword{ish-sh-sh} \textit{Idph} of disapproval. \textit{M-m-m-m shiyɔɔɔ, ŋhɔ lan bɛ: ish-sh-sh, ayo, ayo, mɔ ŋɔ sɔm!} Hm-m-m shiyɔɔɔ (expression of disapproval), do not even say it: ish-sh-sh, yes, yes, you will eat it!

\TCheadword[1]{isɔ} \textit{temp} in the morning, [isɔo] in the morning (K dialect). \textit{Gbeŋ isɔ.} Tomorrow morning (\citealt{Pichl1967}).

\TCheadword[2]{isɔ} \textit{cf}: \TClink[2]{hu}, \TClink[1]{saaka}. \textit{n} (hɔ̃/-) morning (\citealt{Pichl1967}); [ìshɔ́ɔ́] morning (B dialect); \textit{shɔ́, ishɔ} morning (\citealt{Sumner1921}). \textit{Thipïk isɔ lo Bankaŋ wɔ gbo thimini ka.} On from this morning, Bankang was loitering around (\citealt{Pichl1967}). 

\TCsubword{sɔna} (der.) \textit{temp} this morning; \textit{isɔ na} this morning (\citealt{Pichl1967}). 

\TCheadword{ivin} (Eng \textit{even}) \textit{cf}: \TClink[1]{bɛ}, \TClink{fili}, \TClink[1]{mu}. \textit{adv} even. \textit{Ivin paŋ-o-paŋ.} Even every month. \textit{Ye lai bikɔs ivin Pothonɔ ki yɔ hun ke nɔ ndɔndɔ ko wɔko.} That is it, because even when this white man came here, he saw everybody in his place.

\end{letter}

\begin{letter}{J}

\TCheadword[1]{ja} \textit{cf}: \TClink{bila}, \TClink[1]{bulɔ}, \TClink[1]{panth}, \TClink{risen}, \TClink[2]{yen}. \textit{n} \textbf{1)} (lɔ/ma, pl. nyek) cause (\citealt{Pichl1967}). \textit{Hã kul mən ŋgbɛth la ɛ ja libul la chi nak lɛ.} To drink dirty water is one of the causes of (lit. which brings) sickness (\citealt{Pichl1967}). \textbf{2)} matter. \textit{Iyema ni hun ko ja yenchɛkɛ?} We want to now come to the matter of fish? \textbf{3)} affair. \textit{Ya koŋ bɔyni jali mɔ.} I am disgusted with you (lit. your affairs, actions) (\citealt{Pichl1967}). \textit{Yaŋ fli ya woth laɛ ko fe ton-tondo ki ŋa aŋa mpanth lɔnlɔ abɛmpa gbi ja apimamdɛ o ja aŋamdɛ gbi fe tondo ki kunɛ.} It is me that works to arrange all of my children's affairs and my own affairs with very little money coming in. \textbf{4)} thing. \textit{Mɔm mbi ja gbe ŋa ŋanɛ ŋa hunɔni muɛ ŋa ŋan si.} You have many things for those who have not come yet to know. \textit{Jizɔs ŋa ja bom ba ŋa yaŋ, yɛ peyɛ nkɔŋ ma wɔlɛ.} Jesus did a big thing for me when He shed His blood. \textbf{5)} reason. \textit{Jaɛ labi imɛmiɛni ŋa yaŋ chemɔ vel ŋa che mɔ huŋ yi ɛ…} The reason why we are happy to be calling to come and ask you… \textbf{6)} work. \textit{Yai po haŋ ha ja yenchɛk vɛ, fish prɔsɛsin.} I started doing fish work, fish processing. \textit{Chaŋbo paŋdɛ ŋɔ mɔi bo pɔ hiŋ ka ja tuthɛ, than bo tha ika che kunɛ.} Except when evening came, we would be given rice pounding work, that was the work we were engaged in. \textbf{7)} business. \textit{Mi Adama, ko ja nchethɛ ikoŋɔlɔnmu.} Mami Adama, we have not finished the cooking business. 

\TCsubword{jawɛi} (comp.) \textit{n} \textit{jawɛy} bad event (\citealt{Pichl1967}).

\TCsubword[2]{ja} (unspec.) \textit{indfpro} something. \textit{Nɔ chen ŋyẽy thanthɛn; pum ke̹ ja kɛlɛŋ ɔ the ikɛlɛŋ wɔ hunɛ hã wɔn, labi ni che mɛmilni.} One does not smile for nothing; perhaps he sees something good or hears of some good news in store for him, hence he smiles (\citealt{Pichl1967}). 

\TCheadword{jaajo} [jààjò] \textit{n} vine species, fruit has long seeds that used to be used on ankle bracelets that rattled when girls danced (K dialect). 

\TCheadword{jajɛl} \textit{n} mother-in-law, [jàjɛ̀l]/ [ǹjàjɛ̀l] mother-in-law/ mothers-in-law (B dialect); (wɔ/hã, N) mother-in-law (\citealt{Pichl1967}). \textit{Ni ha lɛŋ yɛ komnɛ ɛ, ha lɛŋ yɛ jajɛllɛ, ni pɔi bɛ fe.} Then they greet the father-in-law, then they greet the mother-in-law, then they ‘put' the money (i.e. offer gift as confirmation of marriage engagement). 

\TCheadword{jal} \textit{cf}: \TClink{bus}, \TClink[4]{kɔ}. \textit{n} \textbf{1)} \textit{njal} (ma) body (\citealt{Pichl1967}, \citealt{Sumner1921}). \textit{Ya che palɛ njal thukul} I was feverish the other day (\citealt{Pichl1967}). \textit{Kòmɔ̀ɛ̀ bí mbìmbìs wɔ̀n kɔ́k, wɔ̀n njàlàì gbí.} The child has sores on its back, all over its body. \textbf{2)} \textit{njal} (ma) flesh (\citealt{Pichl1967}). \textit{Njal lɛ ma-m dɛ ma gbo jɛth.} All my flesh is weak (\citealt{Pichl1967}). \textbf{3)} skin. \textit{Ilel wááŋmààɛ ŋɔ ka cheɛ Yeŋken, haliwɔ wááŋmàà ki jal wɔɛ ŋɔ ka che thii.} The woman's name was Yanken because her skin was black. \textit{Yɛ̀ kóŋ thɔ̀n dɛ̀, wɔ̀è bání kùáɛ́ njáláí.} After bathing she rubbed oil on her skin. 

\TCheadword{Jalikatu} \textit{nam} Jalikatu, female name given to a person. \textit{Ŋa mi ilellɛ ŋɔ Jalikatu B Kumba.} My own name is Jalikatu B Kumba.

\TCheadword{Jambo} \textit{n} (hɔ̃/-) Jambo Society, mixed male and female society that owns a medicine against snake bite. The same medicine, misused, can cause a person to be bitten by a snake. The adepts of this society are snake charmers (\citealt{Pichl1967}).

\TCheadword{Jami} \textit{nam} Jamie, name given to a person. \textit{Yaŋ a Agnɛs Jami Simbo.} I am Agnes Jamie Simbo.

\TCheadword{Januari} (Eng \textit{January}) \textit{cf}: \TClink{vɛlvɛl}. \textit{nam} January. \textit{Naintin fɔti tu fɔst ɔf Januari.} 1942, first of January. \textit{Tipɛ ko mɛŋkɛ vɛ haŋ ŋa mɔi yɛlaio ɛ nɛnthɛ tha koni che kuanya yɔl ni nɛn thitin, paŋdo ki ŋɔ chaŋ paoɛ, Januari.} Start from that time up to where I am now I am eighty two years (old), as of that month that just past, January.

\TCheadword{Jasa} \textit{nam} (wɔ/-) Jasa, female name given to a person (\citealt{Pichl1967}). 

\TCheadword{jawɛi} (comp. of \TClink[1]{ja}, \TClink[2]{wɛi} (der. of \TClink[1]{wɛi}), see \TClink[1]{ja}) 

\TCheadword{jɛiŋɛiŋ} [jɛ́íŋɛ̀ìŋ] \textit{cf}: \TClink{tismabue} (comp. of \TClink[1]{tis}, \TClink{n-}, \TClink{boe}). \textit{n} tree species, rubber tree, more like a vine, its sap used to make a ball that children can play with, sap also used for medicine (K dialect).

\TCheadword{jɛk} \textit{cf}: \TClink{chencha}, \TClink{gbɛŋ}, \TClink{gbɛŋ}, \TClink{nante}. \textit{temp} day after tomorrow (B dialect, \citealt{Pichl1967}, \citealt{Sumner1921}). 

\TCheadword{jɛm} \textit{cf}: \TClink{bɛŋkajɛm} (comp. of \TClink[2]{bɛŋ}, \TClink[3]{ka}, \TClink{jɛm}), \TClink{lal}. \textit{n} \textbf{1)} \textit{jɛm/lijɛm} (lɔ/-) fire (\citealt{Pichl1967}). \textit{Ŋkɔ gbïl iwɔm dɛ lal l'ay kɔ, jɛmdi lɛ lɔ yema nyum.} Go put wood on the fire, the fire is about to go out (\citealt{Pichl1967}). \textit{A kɔ hã kwey lijɛm kɛ jɛmdi lɔ lɔ ithïheng.} I went to take some fire but the fire there was not proper (\citealt{Pichl1967}). \textbf{2)} \textit{jɛm/lijɛm} (lɔ/-) firebrand (\citealt{Pichl1967}). comp. \TClink{bɛŋkajɛm} (see \TClink[3]{ka}), \TClink{rɔŋjɛmdi} (see \TClink[2]{rɔŋ}), \TClink{simgbɔljɛm} (see \TClink{gbɔl}), \TClink[1]{simjɛm} (see \TClink{simi}), \TClink[2]{simjɛm} (see \TClink{simi}), id. \TClink{simgbɔljɛm} (see \TClink{gbɔl}) 

\TCheadword{jɛŋjɛi} \textit{n} [jɛ́ŋjɛ́í] tree species, rubber tree more like a vine, sap used to make a ball that children can play with, sap also used for medicine (K dialect). 

\TCheadword[1]{jɛth} \textit{n} weakness. \textit{Yi pɛkɛ kafa ni jɛth.} We are filled with evil and weakness (\citealt{Pichl1967}).

\TCsubword[2]{jɛth} (der.) \textit{cf}: \TClink{jobɔi}, \TClink{pool}. \textit{adj} \textbf{1)} weak, lazy, disinclined to do much work. \textit{Thɔ̀kɛ̀ kɔ̀ jɛ̀th.} The stick is weak. \textbf{2)} tasteless, insipid, insufficiently salty. \textit{Sup lɛ hɔ jɛth.} The soup is not salted enough (\citealt{Pichl1967}). 

\TCsubword{jɛthɛli} (der.) \textit{v} slacken. \textit{Wɔn wɔ gbo nani, aha lɛ hã jɛthɛli hã ma hã mbank lɛ.} While he is pulling hard, the others should slacken their ropes (\citealt{Pichl1967}). 

\TCsubword{jɛthɛlini} (der.) \textit{n} relaxation; (kɔ/-) relaxation (\citealt{Pichl1967}). 

\TCsubword[2]{jɛthil} (der.) \textit{adj} weak, \textit{sùp njɛ̀thíllɛ̀} the weak, tasteless sauce (K dialect). \textit{Hi ma nɛmɛni chek maɛ bɛ la kɛkɛ I ko jɛthil.} Let us not follow our habit, for we are too quick to become weak.

\TCheadword[2]{jɛth} (der. of \TClink[1]{jɛth}) 

\TCheadword{jɛthɛli} (der. of \TClink[1]{jɛth}, \TClink{-il}, \TClink[1]{-i}, see \TClink[1]{jɛth}) 

\TCheadword{jɛthɛlini} (der. of \TClink[1]{jɛth}, \TClink{-il}, \TClink[1]{-i}, \TClink{-ni}, see \TClink[1]{jɛth}) 

\TCheadword[1]{jɛthil} \textit{adj} fresh (water). \textit{Santh bombom dɛ kɔ mən njɛthil l'ay kə santh tata lɛ kɔn dinthɛni kɔ hɛlɛɛ ko.} The big shrimp are found in freshwater but the small and white shrimp are to be found in the sea (\citealt{Pichl1967}). 

\TCheadword[2]{jɛthil} (der. of \TClink[1]{jɛth}, \TClink{-il}, see \TClink[1]{jɛth}) 

\TCheadword{jika} (Eng \textit{chigger}) \textit{n} chigger; (wɔ/hã) chigger, sand flea (\citealt{Pichl1967}). 

\TCheadword{jith} \textit{temp} four days hence (\citealt{Pichl1967}, \citealt{Sumner1921}); \textit{joth} next next next tomorrow, day after the day after the day after tomorrow (K dialect). 

\TCheadword{Jizɔs} \textit{nam} Jesus, male name given to a person. \textit{Jizɔs, a chɔŋ mɔ len.} Jesus, I love you. \textit{Jizɔs, ya mɔnɛ ni mbali mi.} Jesus, I am poor so make me rich.

\TCheadword{Jo} \textit{nam} Joe, name given to a person.

\TCheadword[1]{jo} \textit{cf}: \TClink{chamak}, \TClink{sɔm}. \textit{v} \textbf{1)} eat something soft like cassava, rice, bread, etc., also pronounced \textit{je} in Shenge [K dialect] (\citealt{Pichl1967}). \textit{Ba kel hɔ lɛ, “Nɛn, yaŋ ya tipɛ jo-o.”} Mr. Monkey said, “Man, I begin to eat” (\citealt{Pichl1967}). \textbf{2)} consume. \textit{Koŋ hã jo muk.} He has eaten (his money), i.e., he has wasted all his money (\citealt{Pichl1967}). comp. \TClink[1]{gbɔlkajo} (see \TClink{gbɔl}), \TClink{pianjok} (see \TClink[1]{pia}) 

\TCheadword[2]{jo} \textit{n} \textbf{1)} rice that has just been cooked, or is being cooked. \textit{Joɛ kɔ ni ho, mɔi thɔk boithɛ.} After the rice is properly dry, you wash the dishes. \textbf{2)} cooked rice with a sort of gravy or soup on top (\citealt{Pichl1967}). \textbf{ 3)} food. \textit{Ya mɔ kamɔ nje, ya mɔ tɔyɛ mɔ.} I give you food, I give you clothes. \textit{Chiyɛ pɛ njo lo ki.} She brings the food once more (\citealt{Pichl1967}). comp. \TClink{pulijo} (see \TClink{puli}) 

\TCsubword{jokus} (comp.) \textit{n} leftover rice; \textit{jo kus} cold rice, remains of rice (\citealt{Pichl1967}). 

\TCsubword{yenjo} (comp.) \textit{n} food. \textit{Næ thi gbe̹r that lɔ hã bɛmpa yenjo hĩ lɛ.} There are many ways of preparing our food, we eat it soaked (\citealt{Pichl1967}). \textit{Nælɛ gbi yi bɛmpa yenjo hĩ lɛ, yi bɛmpa hɔ̃ yenkeleŋ.} In whatever way we prepare our food, let us prepare it nicely and cleanly (\citealt{Pichl1967}). 

\TCsubword{jɔbɔ} (unspec.) \textit{n} rice variety (\citealt{Pichl1967}, \citealt{Sumner1921}).

\TCheadword{jobɔi} \textit{cf}: \TClink[2]{jɛth} (der. of \TClink[1]{jɛth}), \TClink{pool}. \textit{adj} weak, feeble.

\TCheadword{joho} [jõŋhɔ̃] \textit{n} [jóŋhɔ̀] fish like \textit{bolkek} but without a beard, pointed nose (K dialect); \textit{jõhõ} fish species, shynose (Gerros melanopterus) (\citealt{Pichl1967}).

\TCheadword{joi} \textit{n} better.

\TCheadword{jokus} (comp. of \TClink[2]{jo}, \TClink[2]{kus}, see \TClink[2]{jo}) 

\TCheadword{joɔ} \textit{adj} fattest.

\TCheadword{Josɛf} \textit{nam} Joseph, male name given to a person. \textit{Josɛf Bɛndu.} Joseph Bendu.

\TCheadword{joth} \textit{n} uterus; (hɔ̃/-) uterus (\citealt{Pichl1967}). 

\TCheadword{jɔbɔ} (unspec. of \TClink[2]{jo}) 

\TCheadword{jɔhɔ} \textit{cf}: \TClink{hothɔk}, \TClink{pakali} (der. of \TClink{pakil}, \TClink[1]{-i}), \TClink{sɔyɛ}, \TClink{woli} (der. of \TClink[1]{woi}, \TClink[1]{-i}). \textit{v} threaten.

\TCheadword{Jɔn} \textit{nam} John, male name given to a person. \textit{Bami wɔlɔ Jɔn Nɛtɛ.} My father is John Netteh.

\TCheadword{Juda} \textit{nam} Judah, name given to a place. \textit{Pe rɛnthɛ, Laɔn ɔf Juda, ko mɔ ko lɔ ibɛ lanɛ iyɛ oo.} Rock of ages, Lion of Judah, in you we put our trust.

\TCheadword{Julai} \textit{nam} July.

\end{letter}
\begin{letter}{K}

\TCheadword{-k} \textbf{1)} \textit{v} \textit{sfx} instrumental, applicative suffix. \textbf{2)} \textit{v > ?} \textit{sfx} suffix denoting action. comp. \TClink{kilgbakɛ} (see \TClink[1]{kil}), \TClink{yuhɔtka} (see \TClink{yu}), der. \TClink{bimik} (see \TClink[1]{bim}), \TClink{gbalak} (see \TClink[4]{gbal}), \TClink{gbenik} (see \TClink{gbem}), \TClink{herk} (see \TClink{her}), \TClink{kerkɛni} (see \TClink{her}), \TClink{hilk} (see \TClink[1]{hil}), \TClink[1]{hok} (see \TClink[1]{ho}), \TClink[2]{hok} (see \TClink[1]{ho}), \TClink{hɔthka} (see \TClink[2]{hɔth}), \TClink{-kani} (see \TClink{-ni}), \TClink[1]{kek} (see \TClink{ke}), \TClink{kekɛthihɔl} (see \TClink[1]{ahɔl}), \TClink{lemɛk} (see \TClink[1]{lem}), \TClink{lɔik} (see \TClink[1]{lɔi}), \TClink[1]{muɛkɛ} (see \TClink{muɛ}), \TClink[2]{muɛkɛ} (see \TClink{muɛ}), \TClink{pikɛ} (see \TClink[1]{pi}), \TClink{paaŋpikɛ} (see \TClink[2]{paŋ}), \TClink{pɔŋki} (see \TClink[2]{pɔŋ}), \TClink{pɔŋkiɛn} (see \TClink[2]{pɔŋ}), \TClink{sɛmɛkni} (see \TClink{sɛm}), \TClink{siŋk} (see \TClink[2]{siŋ}), \TClink{sonthok} (see \TClink{sonthi}), \TClink{sɔik} (see \TClink{sɔyɛ}), \TClink{tipik} (see \TClink{tipɛ}), \TClink{tipiktipik} (see \TClink{tipɛ}), \TClink{tɔŋk} (see \TClink[2]{tɔn}), \TClink{tɔŋkwa} (see \TClink[2]{tɔn}), \TClink{thekɛ} (see \TClink{the}), \TClink{theki} (see \TClink{the}), \TClink{thekni} (see \TClink{the}), \TClink{thuka} (see \TClink[1]{thunɔ}), \TClink{yakani} (see \TClink[1]{ya}), \TClink{yeek} (see \TClink[1]{ye}), \TClink[2]{yiki} (see \TClink[1]{yi}), unspec. \TClink[2]{gbɛk} (see \TClink{gbɛ}), \TClink{pɛŋkiyɔ} (see \TClink[2]{pɛŋ}) 

\TCheadword[1]{ka} \textit{n} hoe. \textit{Ká búl, ká thə̀tsə̀ŋ, ká thə̀rà.} One hoe, two hoes, three hoes. \textit{Ká kó wù.} The hoe is dull. \textit{Ká kó chènì wù.} The hoe is not dull – for ‘dull' use the word ‘dead.'

\TCheadword[2]{ka} \textit{cf}: \TClink[2]{ko}. \textit{Loc} here, proximal demonstrative locative; \textit{kà} here (\citealt{Sumner1921}). \textit{Ka koŋ che Ngendema ko.} He (here) had been at Gendema. \textit{Shenge bɔi fli wɔɔ Shenge kaɛ ya gbemiɛ wɔ yawɔ.} Even the Shenge boy, here in Shenge, I delivered him. comp. \TClink{lelka} (see \TClink[2]{lel}) 

\TCsubword{kaki} (comp.) \textit{Loc} right here. comp. \TClink{kakitiki} (see \TClink[2]{ka})

\TCsubword{kakitiki} (comp.), (comp. of \TClink{kaki}) \textit{Loc} right here. \textit{Nchi bo lɛ kakitiki.} Bring the bread here to this place and not to another (\citealt{Pichl1967}). 

\TCheadword[3]{ka} \textit{prep} \textbf{1)} of, with. \textit{hã buŋ wɔ ka thɔk.} They flogged him with a stick (\citealt{Pichl1967}). \textit{A bɛth thɔk lɛ ka bɛrɛ.} I cut the tree with an axe (\citealt{Pichl1967}). \textit{Bɔ wɔ lɛ hɔ bɛmpaka lithul.} His hat is made of raphia-straw (\citealt{Pichl1967}). \textbf{2)} through. \textit{Ŋɔi ni ŋa fili si i mɔla chaŋ gbo ka Jizɔs sɛ.} How are we to go there, only if we pass through Jesus. \textbf{3)} by. \textit{Chaŋbo athɔni ka Min Charaŋ dɛ we...} Unless I cleanse myself with the Holy Spirit... \textbf{4)} because of, from. \textit{Làŋgbàɛ́ thé nɛ̀kí kà bìllɛ́.} The man was in pain from yaws. comp. \TClink[1]{gbɔlkajo} (see \TClink{gbɔl}) 

\TCsubword{bɛŋkajɛm} (comp.) [bɛ́ŋkájɛ̀m] \textit{cf}: \TClink{jɛm}. \textit{n} firestick (K dialect); \textit{bəngkajɛm} firebrand, torch, glowing log to bring light (\citealt{Pichl1967}).

\TCheadword[4]{ka} \textit{v} \textbf{1)} provide, give. \textit{Apum haŋ che mi paka, apum hamika nsoiɛ, ha mi ka boyaɛ.} Some will not pay me, some give me soap, others give me a gift. \textit{Ka hin Jizɔs Kraist.} He gave us Jesus Christ. \textbf{2)} show. \textit{Lɛ nɔ shi la bo lɛ mɔ Bolomnɔ, nɔ ndɔndɔ wɔ mɔ ka limani.} If a person knows that you are Sherbro, everybody gives you respect.

\TCheadword[5]{ka} \textit{cf}: \TClink[2]{na}, \TClink[1]{pa}. \textbf{1)} \textit{prt} once long ago, in the remote past. \textit{Kaiŋ Taso ka mɔɛ tir bul, lɔ ka ke waaŋmaa kɛlɛŋ-kɛlɛŋ}. Kain Tasso reached a village where he saw a fine young woman. \textit{Kɛ kpɔnko hɔ ka che trï ko ntɛnt, hɔ nɔonɔ ka chen kɔ ai ɛ.} But there was a forest near the town, which no one entered (\citealt{Pichl1967}). \textbf{2)} \textit{Aux} had. \textbf{3)} \textit{Aux} used to. \textbf{4)} \textit{Aux} once.

\TCsubword{kache} (comp.) \textit{temp} \textbf{1)} in those days. \textit{So yen che vɛ yɛlaio ɛ ni kache kendɛ ŋɔ yaŋ ashila.} So that is it between as it is now and those days. \textit{Apa, ŋɔ ko che kath; kache ŋɔ ka che pɛth.} Pa, it has become difficult; things used to be good. \textit{Mɔni hun cɔmpia boŋgo ni kacheɛ.} You should come and compare these days and those days. \textbf{2)} long ago; [kache] twenty years ago and beyond (K dialect). 

\TCheadword[6]{ka} \textit{post} \textbf{1)} on. \textit{Ya koŋ che boe-o tokɛ ka ha nduɛ ŋra gbi ya sɔthɔni yen ha joo.} I have been here above this kitchen for three days. I did not get anything to eat. \textbf{2)} in. \textit{Yaŋ pɔ dumɔ mi Shenge ka.} Me I was raised in Shenge here. \textit{Atiŋ ŋa koŋ kɔni cheko, iara iwɔlɔ ka.} Two of them have gone before, we are now three in this world. \textbf{3)} to. \textbf{4)} with.

\TCheadword[7]{ka} \textit{Aux} used to. \textit{A ka bi pɛl kɔ a ka che yɔk hɛlɛ koɛ, kɛ iŋɛŋdɛ ka bɔnth mi lɔ yay ŋyun.} I had a net I used to go out with to sea, but the wind met me there once. \textit{Wɔn bɛ salima ko lɔ ka cheɛ.} She herself used to be in Salima.

\TCheadword{kaa} \textit{n} [kaar] fish species, croakers, like \textit{brim}, 18 inches when big and sweet to eat, even when dry, but when over dry, inedible (K dialect); \textit{kaa} (wɔ/hã) fish species, krokus or gunugu fish (Pomadasys jubelini) (\citealt{Pichl1967}). 

\TCheadword{kaakaa} \textit{n} crab species, hermit crab (\citealt{Pichl1967}). 

\TCheadword{kaam} \textit{n} insect species, tsetse fly (K dialect); insect species, fly, possibly tsetse, very painful bite, bigger than \textit{lɛl} (\citealt{Pichl1967}). 

\TCheadword{kaana} \textit{cf}: \TClink[3]{kɛn}. \textit{n} \textbf{1)} [kàànà] bamboo species used to make ladders, leaves used for medicine (K dialect). \textbf{2)} Flute-like bamboo musical instrument played at Bondo celebrations (as heard (but not seen) in video made for SLC project) (B dialect).

\TCheadword{Kaare} \textit{nam} Kaare, male name given by a society. 

\TCheadword{kaark} \textit{cf}: \TClink{keke}. \textit{n} [kààrk] tree species used for making locking windows, extremely hard, long-lasting wood (K dialect); (kɔ/ma) tree species, hard wood resists saltwater and used for the keel of boats, planks, and furniture (\citealt{Pichl1967}). 

\TCheadword{kabalo} (Port \textit{cavalo} ‘horse') \textit{n} horse.

\TCheadword{kaban} \textit{adj} \textbf{1)} wonderful. \textit{ Ŋhɔk ma ihɔlɔŋ kaban.} Wonderful word of life (\citealt{Pichl1967}). \textit{Nchɔŋmalen ŋkaban dɛ ma ka hwɛ nɔ wɔn dɛ.} The wonderful love which he had promised to his people (\citealt{Pichl1967}). \textbf{2)} surprising (\citealt{Pichl1967}). 

\TCsubword{kabani} (der.) \textit{v} \textbf{1)} be surprised. \textit{Anya lɛ hã kɔ ke hã kabani.} People who saw it were surprised (or marvelled) (\citealt{Pichl1967}). \textbf{2)} wonder.

\TCheadword{kabande} \textit{n} miracle.

\TCheadword{kabani} (der. of \TClink{kaban}, \TClink[1]{-i}, see \TClink{kaban}) 

\TCheadword{kabɔya} \textit{n} bird species, nicely colored blue, several species of same bird, a little bit bigger than palm bird, lives in groups as well, used to set traps using berries from Christmas tree to catch them (K dialect). 

\TCheadword{Kabu} \textit{nam} Kabu, fishing society that goes out in their canoes and fixes net poles in the mud nearest the ebb tide limit, and when the bank is dry, they return to take the fish out of the net. The founder of this society was Fama Thampel from Kabu, Yawri Bay (\citealt{Pichl1967}). 

\TCheadword{kache} (comp. of \TClink[5]{ka}, \TClink[3]{che}, see \TClink[5]{ka}) 

\TCheadword{Kadiatu} \textit{nam} Kadiatu, female name given to a person. \textit{Yami wɔlɔ Kadiatu Bɛndu.} My mother is Kadiatu Bendu.

\TCheadword[1]{kafa} \textit{cf}: \TClink{kɛnda}, \TClink[1]{wɛi}. \textit{n} \textbf{1)} evil. \textit{Yi pɛkɛ kafa ni jɛth.} We are filled with evil and weakness (\citealt{Pichl1967}). \textbf{2)} sin. \textit{Wɔ ka Wɔŋ ni kɛn ŋa koi kafaŋi yai.} He gave Himself up to take away our sins. \textbf{3)} wickedness. \textit{Ha kafaiyɛ, ŋɔ icha ba bɛŋsin kia.} It is for our wickedness that we are perishing. comp. \TClink{nɔkafa} (see \TClink{nɔ})

\TCsubword[2]{kafa} (der.) \textit{v} sin.

\TCheadword[2]{kafa} (der. of \TClink[1]{kafa}) 

\TCheadword{kafri} (Arabic {\textarab{كافر} } \textit{kafir} ‘unbeliever, atheist') \textit{n} non-believer; \textit{kaafri} (wɔ/hã, a) pagan (\citealt{Pichl1967}).

\TCheadword{Kagbɔrɔ} \textit{nam} Kagboro Chiefdom. \textit{Ka che Spika ha Kagboɛ.} He was the Speaker for Kagboro Chiefdom.

\TCheadword{kahai} \textit{cf}: \TClink{bɔko}, \TClink[1]{hoŋka}. \textit{Loc} outside.

\TCheadword{Kaiŋ} \textit{nam} Kain, name given to a person. \textit{Bia toŋkiɛ jali Ka̰ḭn hã kɔnth.} Bia summoned Kayn for seizure (\citealt{Pichl1967}). \textit{Kaiŋ Taso koŋ yereŋ.}
Kain Tasso was confused.

\TCheadword{kais} \textit{n} saltpond.

\TCheadword[1]{kak} \textit{n} monkey.

\TCheadword[2]{kak} \textit{cf}: \TClink{hɛŋ}, \TClink{kakbom} (comp. of \TClink[2]{kak}, \TClink{bom}), \TClink[1]{sɔ}. \textit{n} east wind.

\TCsubword{kakbom} (comp.) \textit{cf}: \TClink{hɛŋ}, \TClink[2]{kak}, \TClink[1]{sɔ}. \textit{n} southeast wind, only during the rainy season.

\TCheadword{kakali} \textit{v} [kàkàlì] roast (B dialect). 

\TCheadword{kakao} \textit{n} [kàkàó] cocoa, its fruit used for sauce, can be dried for use as a flour (K dialect).

\TCheadword{kakbom} (comp. of \TClink[2]{kak}, \TClink{bom}, see \TClink[2]{kak}) 

\TCheadword{kakeiŋ} \textit{cf}: \TClink{a-a}, \TClink[1]{be}. \textit{disco} not at all. \textit{Kakeiŋ ya chen mɔ ŋɔn ka kith bɛ.} Not at all, I'm not going to give you even half. \textit{Nɔ halɛ wɔe hɔɛ, “Bami, yaŋ bɛ ya theeɛ la bɛlsɛ hɔɛɛ, kɛ pɔ chen laanɛ nɔ ka kakeiŋ.”} One person then said, “Mister, I, too, heard what the rats said, but they will not believe anybody else” (for emphasis kakeiŋ).

\TCheadword{kaki} (comp. of \TClink[2]{ka}, \TClink[1]{ki}, see \TClink[2]{ka})

\TCheadword{kakian} \textit{n} bird species, black, bigger than a hawk, regarded as a state or heraldic bird that nobody has the right to kill. Witches are in charge of it. This bird is allegedly found from Bom to Bonthe (Yon) Island. If a village on the island wants to nest the birds, they have to pay a certain sum of money. Then a small ceremony will be performed that enables the custodians of the birds to climb the tree and take out the nestlings (\citealt{Pichl1967}). 

\TCheadword{kakim} \textit{v} stutter.

\TCheadword{Kakir} \textit{cf}: \TClink{Kɔka}, \TClink{Yelsaha} (comp. of \TClink[3]{yel}, \TClink{saha}). \textit{nam} Caulker, name given to a clan.

\TCheadword{kakitiki} (comp. of \TClink{kaki} (comp. of \TClink[2]{ka}, \TClink[1]{ki}), \TClink{tiko} (der. of \TClink{tii}, \TClink[1]{ko}), see \TClink[2]{ka}) 

\TCheadword{kako} \textit{prep} beside.

\TCheadword{kakyaŋ} [kàkyàŋ] \textit{n} bird species, shorebird found on beaches, feeds on fish, some species grey, some white (K dialect).

\TCheadword{kal} \textit{n} bundle with the cutting instruments and other ceremonial materials of the Bondo Society (B dialect). 

\TCheadword{Kallon} \textit{nam} Kallon, name given to a person. \textit{Anya hiɛ fɔrina ŋaɛ, Koroma, Kallon, Sheriff.} Our people are foreigners, Koroma, Kallon, Sheriff.

\TCheadword{kalom} \textit{n} palm wine. comp. \TClink{mɔɛŋkalom} (see \TClink[1]{mɔɛ}) 

\TCheadword{kamando} \textit{n} game, “hiding the cork” similar to water ball. One man has a cork, as used for the nets, in his hand that he dips deep in the water. The others have to guess where the cork will come up to knock it with their feet (\citealt{Pichl1967}). 

\TCheadword{kamanthi} \textit{cf}: \TClink{pɛlgbampɔ} (comp. of \TClink[2]{pɛl}, \TClink{gbampɔ}). \textit{n} casting net.

\TCheadword{Kamara} \textit{nam} Kamara, name given to a person. \textit{Taamɔtaa ki, ilel wɔɛ ŋ hɔɛ Braima Kamara.} This small boy's name was Braima Kamara.

\TCheadword{kamato} \textit{n} fishing medicine that provides good results with fishing, consists of a special kind of stone (\citealt{Pichl1967}). 

\TCheadword{kamba} \textit{n} Yase messenger.

\TCheadword{kambwe} (Port \textit{câmbio} ‘exchange') \textit{n} pan for cooking salt, large flat 2–3 inches deep, formerly of brass, now of iron or zinc (\citealt{Pichl1967}). 

\TCheadword{kamɔ} \textit{n} Arabic teacher. \textit{Mpanthɛ vɛ ma ikache ŋa ko kamɔ miyɛ.} These are the things we used to do for my Arabic teacher.

\TCheadword{kamsa} (Port \textit{camisa} ‘shirt') \textit{cf}: \TClink{gbɔntma}, \TClink{kumba}. \textit{n} shirt. 

\TCheadword[1]{kan} \textit{cf}: \TClink[2]{lo}, \TClink{rik}. \textit{v} begin to plait a basket or net (\citealt{Pichl1967}). 

\TCheadword[2]{kan} \textit{cf}: \TClink{parɛ} (der. of \TClink[1]{pal}) \textbf{1)} \textit{temp} recently. \textit{Ya chencha kɔ faka ɛ ko, nɔma lɛ wɔ kan wu lɛ lem mi woliyɛ ni yiki tha bɔ lɛ yam veleŋ.} When I went to the village yesterday, the woman who died recently followed and scared me by shaking the bushes behind me (\citealt{Pichl1967}). \textit{Nɔma lɛ koŋ wothkun kan gbo parɛ twɛ.} The woman is pregnant; she knew a man just recently (\citealt{Pichl1967}). \textbf{2)} \textit{Loc} somewhere.

\TCheadword{kana} (Port \textit{cana} ‘cane stick, rod, staff') \textit{cf}: \TClink[1]{buk}. \textit{n} yard (sailing).

\TCsubword{kanaatok} (comp.) \textit{n} upper yard (sailing).

\TCheadword{kanaatok} (comp. of \TClink{kana}, \TClink{atok}, see \TClink{kana}) 

\TCheadword{kand} \textit{cf}: \TClink{karaŋ}, \TClink[2]{lan}. \textit{n} school. \textit{Mi nkɔ kil kandɛ alɔ?} Did you attend school? \textit{Yɛ n ka che ko tallɛ pɔ ka bɛ mɔ kil kandai?} When you were young, were you sent to school? \textit{Yɛ pɔ bɛ mi kaŋdaɛ hɔ pɔ kami ilellɛ vɛ.} When I was sent to school when I was given that name.

\TCheadword{kande} \textit{n} [kándè] paramount chief (B dialect). 

\TCheadword{-kani} (der. of \TClink{-k}, \TClink{-ni}, see \TClink{-ni}) 

\TCheadword{kanth} \textit{n} reed species.

\TCsubword{kanthiŋkɔ} (comp.) \textit{n} plant species, (euphorbiaceae sp) (\citealt{Pichl1967}). 

\TCheadword[1]{kantha} \textit{v} close; shut. \textit{Ŋkantha rɛnth lɛ!} Close the door!

\TCheadword[2]{kantha} \textit{n} training for paramount chiefs before they are inaugurated (B dialect).

\TCheadword{kanthiŋkɔ} (comp. of \TClink{kanth}) 

\TCheadword{kanthka} \textit{v} block. \textit{Hɔ ŋa ma blem wanthɛmdɛ, aftabakɛ nai landɛ ŋɔ kanthka gbaŋ, ŋɔ che bɔ honi.} He said, “Do not blame the woman, the way for the afterbirth was blocked, it was not able to come out.”

\TCheadword{Kanu} \textit{nam} Kanu, name given to a place. \textit{Bath Kanu lɔ ka che kɔ skullɛ.} It is at Bath Kanu where he went to school.

\TCheadword[1]{kanya} \textit{n} [kànyà] rice flour beaten with groundnuts (K dialect); (hɔ̃/-) flour made of groundnuts and rice (\citealt{Pichl1967}).

\TCheadword[2]{kanya} (Mende \textit{kanya} ‘gonorrhea') \textit{n} [kányá] gonorrhea (K dialect); (hɔ̃/-) gonorrhea, syphilis (\citealt{Pichl1967}).

\TCheadword[1]{kaŋ} \textit{cf}: \TClink[1]{hoŋka}. \textit{n} open place.

\TCsubword{kaŋbay} (comp.) \textit{n} dancing area.

\TCheadword[2]{kaŋ} \textit{n} corn. \textit{Atipɛ yuk yekeɛ, ŋkaŋdɛ, mbinchɛ, pɛlɛ, nsowɛ, ntɔllɛ.} I start to plant cassava, corn, beans, rice, millet, Guinea corn.

\TCheadword[3]{kaŋ} \textit{cf}: \TClink{skul}. \textit{n} \textbf{1)} school; schooling. \textit{Ŋan gbi nbɛŋa kaŋdai?} Did you send all of them to school? \textit{Apa, kɔ kil kaŋdɛ alɔ?} Pa, did you go to school? \textit{Ndɔ mɛkɛni kaŋ mɔa?} Where did you finish your schooling? \textbf{2)} learning. \textit{So pɛth-pɛth kaŋdɛ kɔvɛ.} So that is the sweetness of learning. \textit{Ŋkɔ kil kaŋdɛ?} Did you go to the learning house? comp. \TClink{kilkaŋ} (see \TClink[1]{kil}) 

\TCheadword{kaŋaloma} \textit{cf}: \TClink{lelena}. \textit{n} praying mantis.

\TCheadword{kaŋbay} (comp. of \TClink[1]{kaŋ}, \TClink[1]{bai}, see \TClink[1]{kaŋ}) 

\TCheadword[1]{kaŋk} \textit{n} ant hill (K dialect). 

\TCheadword[2]{kaŋk} \textit{n} ant species, white ant (\citealt{Pichl1967}). 

\TCheadword[1]{kaŋka} \textit{subordconn} so that. \textit{Pɔ koŋ gbo raa pɔi piŋgi kaŋka inallɛ lɔ ŋa ni kɛlɛn.} After brushing, they have to turn over the soil so that it becomes clean.

\TCheadword[2]{kaŋka} \textit{Aux} may.

\TCheadword{kaŋkafiuŋ} \textit{n} \textbf{1)} puzzle. \textbf{2)} riddle.

\TCheadword{kaŋkagbet} \textit{n} [kàŋkàgbét] scorpion (K dialect). 

\TCheadword{kaŋko} \textit{n} [kàŋkó] fish species, edible small fish found in swamps, 4 inches (K dialect). 

\TCheadword{kapa} \textit{n} \textbf{1)} wing, sail (\citealt{Pichl1967}). \textit{Rɛthiɛ kapathi wɔ lɛ yaŋ atok.} He spread his wings over me (\citealt{Pichl1967}). \textit{Nchuŋ kapathi mɔ lɛ.} Provide shade for us with your wings (\citealt{Pichl1967}). \textbf{2)} hand.

\TCheadword{karakara} (der.) \textit{Idph} of scampering, scurrying, sound rats make as they move through a thatch roof (K dialect). \textit{La veiɛni, bɛlsɛ ŋae kinda baiɛ tokɛ: <kara-kara kara-kara kara-kara>.} It was not long after when the rats ran up above the bari: <kara-kara kara-kara kara-kara>. 

\TCheadword{karaŋ} \textit{cf}: \TClink{kand}, \TClink[2]{lan}. \textit{v} \textbf{1)} learn. \textbf{2)} read. \textbf{3)} teach. \textit{Sistha Kɔba ŋaha kaŋa hi mpanthoɛ.} Sister Koba is the one that taught us this work. \textit{So yɛ pɔ lɔik wanda maɛ, pɔ wɔi ko kaŋ len-o-len.} So when they initiate a girl, they teach her everything. \textbf{4)} study. \textit{So yɛ nwuni Shenge ka, nkaŋa ŋa pɛ?} So when you came to Shenge here, did you study here as well?

\TCheadword{Kari} \textit{nam} [kàrí] Kari, male name given by a society (K dialect). 

\TCheadword[1]{kasa} \textit{cf}: \TClink[1]{sampa}. \textit{n} (kɔ/tha) basket type, large, bigger than a \textit{sampa} (\citealt{Pichl1967}). 

\TCheadword[2]{kasa} \textit{n} kind of eczema found mostly on the head where it destroys the roots of the hair, sometimes it is found as light patches on the skin of the body (\citealt{Pichl1967}). 

\TCheadword{kasaŋke} \textit{n} payment or contributions made at a burial ceremony. \textit{Ni bai ko, pɔ lɔ chɛli fe kasaŋ-keɛ ŋɔ leeɛ thɔth.} In the court bari, they are arranging the funeral money (contributions) proportionally. \textit{Yɛ ŋa muɛ tirɛ lɔ ŋa ha bɛ kasaŋkeɛ ŋae lɔl…} When they reached the village where they had to greet the burial ceremony…

\TCheadword{Kase} \textit{nam} Kase Society. 

\TCheadword{kase} \textit{n} \textbf{1)} (hɔ̃/tha) fault (\citealt{Pichl1967}, \citealt{Sumner1921}). \textbf{2)} blame. \textit{Kase che wɔn.} He is blameless (\citealt{Pichl1967}). 

\TCheadword{kasɛt} (Eng \textit{cassette}) \textit{n} cassette. \textit{La pɔ koŋ rɛkɔdɛ pɔ bia ŋa kasɛt lan.} We have recorded that, we have to bring the cassette.

\TCheadword{Kasilan} \textit{nam} Kasilan, Sherbro spirit whose main residence is at Bonthe-Borhol (\citealt{Pichl1967}). comp. \TClink{rɔŋkasilan} (see \TClink[2]{rɔŋ}) 

\TCheadword{kasuu} (Men \textit{kasilu} ‘spider') \textit{n} spider species, contraction of the Mende word \textit{kasilu} ‘spider,' sometimes used instead of the Bolom word, \textit{na} (\citealt{Pichl1967}).

\TCheadword{katamɛn} (comp. of \TClink[3]{mɛn}) 

\TCheadword{katata} \textit{n} [kàtàtà] vine species, long vine that grows along the ground and wraps around trees, medicine from the leaves, vines can be used to tie bundles of wood (K dialect). 

\TCheadword{kath} \textit{adj} \textbf{1)} hard. \textit{Ya kɔ tɛmɛni gbath lo hɔ̃ kath.} I go to strive for myself, the times are hard (\citealt{Pichl1967}). \textit{Apa, ŋɔ ko che kath.} Pa, it has become difficult. \textbf{2)} difficult. \textbf{3)} strong. \textit{Han hã jɛth kɛ wɔn wɔ kath.} They are weak but he is strong (\citealt{Pichl1967}). \textbf{4)} tough. \textit{Sɔk lɛ wɔ kath.} The (meat of the) fowl is tough (\citealt{Pichl1967}). \textbf{5)} loud. \textit{Ŋhɔ̃ kath.} A loud (strong) voice (\citealt{Pichl1967}). \textbf{6)} serious. \textit{Nak lo kɔ kath, Kɔŋ wɔ gbo tïni.} This illness is serious, Kong faints constantly (\citealt{Pichl1967}). \textit{Ya hã kɔ nante, nak ya-m dɛ kɔ kath.} I have to go today, my mother's illness is serious (\citealt{Pichl1967}). \textbf{7)} tight.

\TCsubword{kathba} (der.) \textit{adv} \textbf{1)} loudly. \textit{Amaaɛ ŋa bɛmpa ŋjeeɛ ha sakaɛ ŋae thee yɛ Kaiŋ Taso mam kaathbaɛ.} The women who were preparing the food for the sacrifice heard Kain Tasso laughing loudly. \textbf{2)} seriously. \textit{Huɛ bul, pɔe wom ko Kaiŋ Tasoɛ jajɛl, wɔɛ wɔ naka kathba.} One day, they sent a message to Kain Tasso that his mother-in-law was very sick.

\TCsubword[1]{kathil} (der.) \textit{adj} \textbf{1)} difficult, [mà kàthíl] very hard (K dialect). \textit{Ya hɔmɔ wɔ ja-m kathil le lan gbi.} I tell him all my difficulties (lit. difficult thing) (\citealt{Pichl1967}). \textbf{2)} hard. \textbf{3)} serious. \textit{Nrɔmp lɛ ma ŋkathil.} The sickness is serious (\citealt{Pichl1967}). \textit{Kaiŋ Taso wɔe bɛmpa laa wɔɛ ni wɔm wɔ ha kɔ muɛ ko nak kathillɛ.} Kain Tasso prepared his wife and sent her to go attend to her mother's sickness. \textbf{4)} high. \textit{Siminji lɛ hɔ̃ prɛs kathïl.} Cloves have a high price (\citealt{Pichl1967}). comp. \TClink{bolkathil} (see \TClink[1]{bol}) 

\TCheadword{kathba} (der. of \TClink{kath}, \TClink[2]{ba}, see \TClink{kath}) 

\TCheadword[1]{kathil} (der. of \TClink{kath}, \TClink{-il}, see \TClink{kath}) 

\TCheadword[2]{kathil} \textit{nam} Kente cloth. \textit{Tamɔ tondɛ wɔ gbaŋkthani kotha kathil bom mɛ nɔ bɛn.} The small boy wrapped the big Kente cloth around himself as if he were a big man (\citealt{Pichl1967}).

\TCheadword{kathkath} \textit{adj} strong.

\TCheadword{Kay} \textit{nam} Kãy, name given by Poro Society (\citealt{Pichl1967}). \textit{Piye chaŋ Kãy ntɛn}. Piye is more clever than Kayn (\citealt{Pichl1967}). 

\TCheadword{ke} \textit{v} \textbf{1)} see. \textit{Anya lɛ hã kɔ ke hã kabani.} People who saw it were surprised (or marvelled) (\citealt{Pichl1967}). \textit{Hunna kə ya ke wɔn na.} He came but I did not see him (\citealt{Pichl1967}). \textbf{2)} look.

\TCsubword{Ketilaŋ} (comp.) \textit{nam} Ketilang, female name given to a person (lit. she saw another). (\citealt{Pichl1967}). 

\TCsubword[1]{kek} (der.) \textit{v} see. \textit{Kɔŋdɛ kɔ akekɛ thiwɔllɛ yɛ laiyoɛ hɔ cheni pɛ bul.} The burial that I have seen (with my) eyes, it is not still the same. comp. \TClink{kekɛthihɔl} (see \TClink[1]{ahɔl})

\TCsubword[3]{ken} (der.) \textit{v} \textbf{1)} be seen (\citealt{Sumner1921}). \textbf{2)} reveal oneself. \textit{Huɛɛ ŋɔ ken gbo, Braima wɔ le kɔ lɛliɛ mpɛl lo ki pɛiŋ.} Just as day breaks, Braima first goes to inspect these fishing lines.

\TCsubword{keni} (der.) \textit{cf}: \TClink[1]{boni} (der. of \TClink[1]{bo}, \TClink{-ni}), \TClink[1]{lɛli} (comp. of \TClink[3]{lɛ}), \TClink{nɔɔmi}. \textit{v} \textbf{1)} find. \textit{Ŋɔ nkeni Mbolomdɛ kenɛki a ŋɔ pɔ ma theli?} How do you find Bolom now compared to how they used speak it? \textit{Tamɔ lɛ koŋ keni mɛnklɛni.} The boy has gone to find protection (\citealt{Pichl1967}). \textbf{2)} appear, look. \textit{Wɔ kɛni imɔl.} He is looking sorrowful (\citealt{Pichl1967}). \textbf{3)} be visible. \textbf{4)} break. \textbf{5)} compare.

\TCsubword{kenin} (der.) \textit{v} dawn. \textit{Yɛ wɔiyɛ ŋɔ kenindɛ mɔi chɔŋɔ Abatokɛ sɛkɛ mɔi wɔ achɔŋɔ Abatokɛ sɛkɛe.} When the day breaks you give thanks to God, you say I give thanks to God. \textit{Yɛ huɛɛ ŋɔ keni, huɛɛ ŋɔ pɔ hok saka nduɛ ŋraɛ.} When the day broke, the day when the people came from the third day sacrifice.

\TCsubword{keche} (unspec.) \textit{v} see. \textit{Nɔ inyun dɛ koŋ keche.} The blind man was finally able to see (\citealt{Pichl1967}). comp. \TClink{nɔikeche} (see \TClink{nɔ})

\TCheadword{kɛbel} (unspec. of \TClink{bel}) 

\TCheadword{keche} (unspec. of \TClink{ke}) 

\TCheadword[1]{kee} \textit{cf}: \TClink[1]{bɛth}. \textit{n} (hɔ̃/tha) hip (\citealt{Pichl1967}). 

\TCheadword[2]{kee} \textit{cf}: \TClink{bithi}. \textit{n} \textit{ikee} (hɔ̃/-) stalks or rice or else remaining in the field after harvesting (\citealt{Pichl1967}). 

\TCheadword[1]{keeya} \textit{n} [kèèyá] fish species, found in rivers near swamps, edible, 4–5 inches long (K dialect).

\TCheadword[2]{keeya} \textit{n} [kééyà] person who acts as a go-between or matchmaker, convincing, e.g., a woman to consent to a marriage, esp if the woman is from the area (K dialect). 

\TCheadword[1]{kek} (der. of \TClink{ke}, \TClink{-k}, see \TClink{ke})

\TCheadword[2]{kek} \textit{n} type. \textit{Kenyaa Braimaɛ, Ba Amadu Kamara, bi mpɛl hɔth kaɛ kek thira: mpɛl ma ŋgbampɔɛ, mpɛl ndukiɛ ni yɛlɛɛ.} Braima's uncle, Ba Amadu Kamara, has fishing nets, three different types: bonga nets, nets they leave at sea, and the chain.

\TCheadword{keke} \textit{cf}: \TClink{kaark}. \textit{n} [kèkè] tree species used for making plank boats (K dialect). 

\TCheadword{kekɛthihɔl} (comp. of \TClink[1]{kek} (der. of \TClink{ke}, \TClink{-k}), \TClink[1]{ahɔl}, see \TClink[1]{ahɔl}) 

\TCheadword[1]{kel} [kə́l] \textit{n} monkey, [kə̀l]/[kə̀l sɛ̀] monkey/the monkeys (B dialect); \textit{ke̹l} (wɔ/hã, si) monkey (\citealt{Pichl1967}). \textit{Kə̀llɛ̀ wɔ́ thɔ̀kɛ̀ àtòk.} The monkey is up in the tree. \textit{Hĩn-gbɔl, kə Ba Ke̹l ka hinɛn gbɔl.} But Mr Monkey was not satisfied (\citealt{Pichl1967}). comp. \TClink{nuikel} (see \TClink{nui}), \TClink{yekɛkel} (see \TClink{yekɛ}) 

\TCsubword{kelbaa} (comp.) \textit{n} \textit{ke̹l baa} (wɔ/hã, si) monkey species (\citealt{Pichl1967}). 

\TCsubword{kelbom} (comp.) \textit{n} \textit{ke̹l bo̹m} (wɔ/hã, si) monkey species, putty-nosed colobus (Cercopithecus nictitans) (\citealt{Pichl1967}). 

\TCsubword{kelmesinya} (unspec.) \textit{n} \textit{ke̹l mesiña} (wɔ/hã, si) monkey species, red colobus (Procolobus badius) (\citealt{Pichl1967}). 

\TCsubword{kelsadin} (unspec.) \textit{n} \textit{ke̹l sadin} (wɔ/hã, si) monkey species, black colobus (Colobus polykomos) (\citealt{Pichl1967}). 

\TCheadword[2]{kel} \textit{v} bite. \textit{À mɔ̀ kə́l.} I bite you. \textit{Kə́l lɛ̀ kə́l kə́llɛ̀.} The monkey bit the monkey. \textit{Tamɔ lɛ ke̹r ɛ ke̹l wɔ ni wɔ ye wu.} The boy was bitten by a snake and died (\citealt{Pichl1967}). 

\TCheadword{kelba} (comp. of \TClink[1]{kel}, \TClink[1]{ba}, see \TClink[1]{kel}) 

\TCheadword{kelbom} (comp. of \TClink[1]{kel}, \TClink{bom}, see \TClink[1]{kel}) 

\TCheadword{kelmesinya} (unspec. of \TClink[1]{kel}) 

\TCheadword{kelnimɔf} \textit{v} press or tighten the lips under great strain (\citealt{Pichl1967}). 

\TCheadword{kelsadin} (unspec. of \TClink[1]{kel}) 

\TCheadword{kem} \textit{n} tree species, red camwood, dyewood (\citealt{Pichl1967}). 

\TCheadword{kemba} \textit{n} plant species, shrub (Solanum nodiflorum) (called \textit{ɛfɔ-odu} in Krio) (\citealt{Pichl1967}). 

\TCheadword[1]{ken} \textit{adj} \textbf{1)} equivalent. \textit{So yen che vɛ yɛlaio ɛ ni kache kendɛ ŋɔ yaŋ ashila.} So that is it between as it is now and those days. \textbf{2)} similar to. \textit{Yan aka bo minɛ mɔ gbemi kilɛ ko ni pɔmthɛ ken aŋa bɛndɛ ŋa ŋaɛ.} I always think you just deliver (babies) in the home, with the leaves, like our first people did it. \textit{Bulɔ kendɛ handɔ?} The work is similar to what? (What kind of work?) \textbf{3)} same. \textit{Ŋe tipɛ yi-yini-ŋkɛn ŋa hɔɛ, “La taalaŋgba ki wɔ mama?”} They begin to ask themselves the same thing, saying, “What is this young man laughing about?”

\TCheadword[2]{ken} \textit{cf}: \TClink{kɛth}. \textit{n} knife, [kə̀nd]/[kə̀ndɛ̀] knife/the knife (B dialect); \textit{like̹n} (lɔ/ma) country knife (\citealt{Pichl1967}). \textit{Ke̹n di lɛ lɔ luɛ.} The knife is sharp (\citealt{Pichl1967}). 

\TCheadword[3]{ken} (der. of \TClink{ke}, \TClink[1]{-n}, see \TClink{ke}) 

\TCheadword[4]{ken} \textit{cf}: \TClink{kendɛ}, \TClink[5]{ni}, \TClink{ŋɛ}. \textbf{1)} \textit{subordconn} as. \textit{Mbolomdɛ man kendɛ ichalao ɛ.} The Bolom as we are seated now. \textbf{2)} \textit{coordconn} like. \textit{Wɔ kəlɛng ke̹n yaa wɔ.} She is beautiful like her mother (\citealt{Pichl1967}).

\TCheadword[5]{ken} \textit{cf}: \TClink{tɛnt}. \textit{Loc} next to. \textit{Haliwɔ, wɔ ibi wɔn kɛn dɛ o.} For he is by our side. comp. \TClink{thɛŋkɛi} (see \TClink{thɛŋ}) 

\TCheadword[6]{ken} \textit{prep} like. \textit{Tɛm hɔ gbo ken mɛn nsoso lɛ hɔ chenk anyathi gbi.} Time is like running water, it carries people away (\citealt{Pichl1967}). 

\TCheadword{kendɛ} \textit{cf}: \TClink[4]{ken}, \TClink[5]{ni}, \TClink{ŋɛ}. \textit{prep} like. \textit{Kɛ la leɛ ni bo kendɛ vɛ, nɔ bul chen bo lem.} But it does not just remain like that, a person cannot just talk. \textit{Tak Bahin yɛ Wɔ i si, bepɛ nɔ kendɛ Wɔn.} The son of God we know, there is no other God like Him.

\TCheadword{Kenedi} \textit{nam} Kennedy, name given to a person. 

\TCheadword{kenɛki} \textit{temp} \textbf{1)} nowadays. \textit{Kɛ ŋɔ nke wɔlɔ ka che ni kenɛkia?} How do you see the world these days and in the past? \textit{So kɔŋ dɛ kache ni kenɛki yɛ ŋɔi yɛ?} So that is how burial in the past and now is? \textbf{2)} right now. \textit{Kenɛki mɔ kɔ skul?} Right now do you go to school? \textit{Ishiɛ lɛ mɔ Koroma nɔ, kenɛki pɛ mɔlɔ Spikaɛ Kagbɔ ka.} We know you are a Koroma, and now you are the Speaker of Kagboro.

\TCsubword{kenɛki-kenɛki} (der.) \textit{temp} \textbf{1)} nowadays. \textit{Kenɛki-kenɛki wantɛ yi bɛndɛ wɔ pɔk Potho wɔ yi sɔpɔt.} This time now, we have our sister in the whiteman's country who supports us. \textbf{2)} right now. \textit{Ŋanɛki kenɛki-kenɛki ɛn ŋanɛ ŋa bia kɔ hundɛ.} The ones right now and those that are going to come.

\TCheadword{keni} (der. of \TClink{ke}, \TClink{-ni}, see \TClink{ke}) 

\TCheadword{kenin} (der. of \TClink{ke}, \TClink{-ni}, \TClink[2]{-n}, see \TClink{ke}) 

\TCheadword[1]{keŋkeŋ} (Krio \textit{krenkren} ‘leaf sauce') \textit{n} \textbf{1)} krain-krain, [kǝ́ŋkǝ́ŋ] bush and its leaves that are used to make a slippery sauce eaten with rice or fufu (K dialect). \textit{Hin lɛ pɛ sallɛ mɔi gbo asaŋ keŋkendɛ a yuk gbamdɛ.} For us, when rainy season comes, I plant krain-krain, (and) I plant potato leaves. \textit{Wɔŋyi huŋ toŋgi ŋɔ pɔ chɛth keŋkeŋdɛ.} She is about to show us how to cook krain-krain.

\TCsubword{keŋkeŋbokoth} (comp.) \textit{n} [kéŋkéŋbókóth] plant species, bush with leaves used for sauce (K dialect).

\TCheadword[2]{keŋkeŋ} \textit{n} [kéŋkéŋ] bird species, shorebird about 15 inches high (K dialect); \textit{ke̹nke̹ŋ} (wɔ/hã, N) bird species, small whitish seabird, plover? (\citealt{Pichl1967}). 

\TCheadword{keŋkeŋbokoth} (comp. of \TClink{keŋken}, \TClink[2]{bokoth}, see \TClink{keŋken}) 

\TCheadword[1]{ker} \textit{n} snake (generic). \textit{Tamɔ lɛ ker ɛ kel wɔ ni wɔ ye wu.} The boy was bitten by a snake and died then (\citealt{Pichl1967}). \textit{Mɔm komɔ kɛr ki.} You are the son of a snake!

\TCheadword[2]{ker} \textit{v} be tired.

\TCheadword[3]{ker} \textit{n} [kér] tree species, small tree used for setting traps, bends easily and bounces back well, used in rivers (K dialect).

\TCheadword{kete} \textit{n} dance for men and women accompanied by \textit{igbethe} (\citealt{Pichl1967}). 

\TCheadword{keth} \textit{cf}: \TClink{kumbɛ}. \textit{n} chest (K dialect). 

\TCheadword{ketheboni} \textit{n} images foretelling misfortune. \textit{Ntuntuŋg}, a secret society, has images (\textit{ketheboni}) that foretell misfortune – said to originally come from Baga (\citealt{Hall1938}). 

\TCheadword{ketheŋ} (der. of \TClink{kɛth}) 

\TCheadword{Ketilaŋ} (comp. of \TClink{ke}, \TClink{tilaŋ}, see \TClink{ke}) 

\TCheadword{Keway} \textit{nam} Keway, name given by Yase Society (\citealt{Pichl1967}). 

\TCheadword{Kɛ} \textit{nam} [kɛ́] Que, male name given by Poro Society (K dialect). 

\TCheadword[1]{kɛ} \textit{cf}: \TClink[2]{-i}, \TClink{ɛn}, \TClink[4]{la}, \TClink[1]{o}. \textit{coordconn} \textbf{1)} but. \textbf{2)} and. \textit{Kɛ pɔ yuk pɛlɛ pɔnthai ɔ bɔmthai?} And do they plant rice in the swamps or muds? \textbf{3)} then.

\TCheadword[2]{kɛ} \textit{disco} well. \textit{Kɛ, apa, lagbowɛwe.} Well, pa, goodbye.

\TCheadword[3]{kɛ} \textit{subordconn} \textbf{1)} that. \textit{Oo aŋa mi isi yɛ lɛ kɛ Kraist ka wu ŋa hin.} Oh, my people, let us realize that Christ died for us. \textit{Tɛnɛni, tɛnɛni, tɛnɛni, kɛ ya wɔ gbem mɔ we.} Remember, remember, remember that your mother gave birth to you. \textbf{2)} for. \textit{A kɛ lokimdɛ wɔi pɔ bi bɛ ha hu ŋ saka wɔi, ŋgasumana ko, fakai ko.} Because he is my in-law, we even have to make his sacrifice (tithe) in Mokainsumana, in the village. \textbf{3)} about.

\TCheadword[4]{kɛ} \textit{cf}: \TClink{gbetha}. \textit{n} \textit{ikə} oath (\citealt{Pichl1967}).

\TCheadword{kɛbɛlini} (der. of \TClink{kɛyɛni}) 

\TCheadword{kɛbɛŋ} \textit{cf}: \TClink{kiminmi} (comp. of \TClink{kii}, \TClink[1]{min}, \TClink[1]{mi}). \textit{n} [kǝ̀bǝ̀ŋ] fish species, found in rivers, six inches, edible, people fish for it (K dialect); \textit{kəbəng} (wɔ/hã, si) fish species, sheephead or benda-benda (Chaetodipterus lippei, Drepane punctata) (\citealt{Pichl1967}).

\TCheadword{Kɛbi} \textit{nam} Kebi, name given to a person (\citealt{Pichl1967}). 

\TCheadword{kɛbi} \textit{cf}: \TClink{cholnɔ}, \TClink[3]{gba}, \TClink{gbalɔ}. \textit{n} blacksmith.

\TCheadword{kɛɛ} \textit{n} [kɛ̀ɛ̀] vine species with a reddish fruit, starts off green, turns yellow, then red when ripe, people eat fruit, roots used for medicine (K dialect).

\TCheadword{Kɛfɛ} \textit{nam} Kefe, female name given to a person. 

\TCheadword{kɛfɛ} \textit{n} pepper. \textit{Kəfe kɔ fay.} Pepper is hot (\citealt{Pichl1967}).

\TCsubword{kɛfɛgbokru} (comp.) \textit{n} sweet pepper.

\TCsubword{kɛfɛtonton} (comp.) \textit{n} chili peppers.

\TCheadword{kɛfɛgbokru} (comp. of \TClink{kɛfɛ}) 

\TCheadword{kɛfɛtonton} (comp. of \TClink{kɛfɛ}, \TClink{tonton} (der. of \TClink[1]{ton}), see \TClink{kɛfɛ}) 

\TCheadword{kɛi} \textit{v} deny.

\TCheadword{kɛiɛ} \textit{n} kind of wild fruit, malombo, citrus-like, the seeds are what one eats or actually sucks (B dialect); \textit{kɛɛ} (kɔ/ma) stone fruit, maloubo (\citealt{Pichl1967}). 

\TCheadword[1]{kɛk} \textit{cf}: \TClink[1]{bɔk}, \TClink[2]{koŋ}, \TClink[2]{nya}. \textit{n} turtle species.

\TCheadword[2]{kɛk} \textit{n} stocks (instrument of punishment) (\citealt{Pichl1967}).

\TCheadword[1]{kɛkɛ} \textit{cf}: \TClink{libɛn} (der. of \TClink[2]{li-}), \TClink{yas}. \textit{temp} \textbf{1)} quickly (\citealt{Pichl1967}). \textbf{2)} immediately (\citealt{Pichl1967}, \citealt{Sumner1921}). \textbf{3)} \textit{kɛtkɛt} fast (\citealt{Sumner1921}). \textit{Nsiɛ Abolomaɛ kɛkɛ ŋako wɔ nwɔk ma nɔ.} You know the Bolom spoke another person's language fast. \textbf{4)} regularly. \textit{Lɔn lɔ pɔ chema bo wɔ kɛt-kɛt.} It is only there where people do not speak it regularly.

\TCsubword{kɛkɛkɛ} (comp.) \textit{temp} quickly. \textit{Wɔi wɔ mi nchi a hun mɔ hothɔ, Kɛyɛ laiowɛ yɛmɔbo hɔ vethimi, wɔ gbɛ kɛkɛkɛ, ha hun mɔ vethi.} He would say, no mother, let me carry it, but as it is now, as you say help me (lift this to my head), he would quickly run to help you.

\TCheadword[2]{kɛkɛ} \textit{temp} \textbf{1)} short time. \textit{Nkeni gbo nkɔni ayen-o-yen, mɔni gbo kɔ kɛkɛ.} If you see somebody go somewhere, you just go there for a short time. \textbf{2)} just now. \textit{Labi bɛ bɛra ŋa che kɛkɛ-o hɔɛ, ‘pɔ gbiŋkith feɛ-o-o-o!'} That’s why people were saying just now, ‘let's cover the money-o!' \textit{Kɛkɔo bɛ nkɔ gbo mɔ bɔnth gadinthai mbokɛ ma lɔ.} Even now if you just go you will find leafy plants (used for making sauces) in gardens there.

\TCsubword{kɛkɛlɔ} (unspec.) \textit{temp} \textbf{1)} immediately. \textit{Ya mɔ loli kɛkɛlɔ.} I (shall) save you immediately (\citealt{Pichl1967}). \textbf{2)} quickly. \textit{Ihɔlɔŋ dɛ hɔ̃ mɛkin kɛkɛlɔ.} Life ends quickly (\citealt{Pichl1967}). \textbf{3)} fast. \textit{Rithi lɛ kɔ mɔ̃e kɛkɛlɔ}. The darkness is fast coming to an end (\citealt{Pichl1967}). 

\TCheadword{kɛkɛkɛ} (comp. of \TClink[1]{kɛkɛ}) 

\TCheadword{kɛkɛlɔ} (unspec. of \TClink[2]{kɛkɛ}) 

\TCheadword{kɛkɛŋ} \textit{n} skull. \textit{“Muuliaeɛ” lɔ mmɛn dɛ ma kɔ kuŋkuŋ dɛ, yen kendɛ kɛkɛŋ thianyin.} “Muuli” where the water will carry (you) over things that resemble human skulls.

\TCheadword{kɛko} \textit{cf}: \TClink{bɔtakɛl} (comp. of \TClink[1]{baa}), \TClink{sɔmbu}. \textit{n} squirrel species.

\TCheadword{kɛkoŋ} \textit{n} bamboo pole, strong palm rib (\citealt{Pichl1967}). 

\TCheadword{kɛl} \textit{n} snakeskin.

\TCheadword[1]{kɛlɛŋ} \textit{n} goodness. \textit{Sɔlɛma hɔ cheni kelen.} It is not good to have the hassle. 

\TCsubword[1]{yeŋkɛlɛŋ} (comp.) \textit{adv} \textbf{1)} nicely. \textit{Pɔmthi gbamdɛ lɛ ye ma kɔ gbo chɛth yeŋkɛlɛŋ ni ntheki kɔni pɛth-pɛthɛ…} Potato leaves, if you want to cook them nicely so that they taste good… \textbf{2)} well. \textit{N lɔ̀llɔ́ ɲɛ̀ŋkɛ̀lɛ́ŋ? À chɔ̀ŋá Àbátùkɛ́ sàkà.} Did you sleep well? I give thanks to God. \textit{Ŋɔ́hɔ́lpòkɛ̀ wɔ̀ thɛ́kɛ́sí sàbàɛ́ yèŋwɛ̀í/yèŋkɛ̀lɛ́ŋ.} The judge interpreted the law badly/well. \textbf{3)} carefully. \textit{Mɔ gbɛ yeŋkɛlɛn, mɔ ŋa thɛkɛsini.} You should walk carefully, you should watch over yourself. \textbf{4)} properly. \textit{Ŋa mina kɔ pɛ sɛkɛli yeŋkɛlɛŋ.} They will then dry it properly. \textit{Iŋɔ thɛ ŋɔ he yeŋkɛlɛŋ.} We burn it (the field) for it to be burned properly. \textbf{5)} completely. \textit{Pɔ koŋ gbo chakath yeŋkɛlɛŋ, pɔi chi bɛkthɛ.} They remove the stalks from the rice completely, then they bring the bags. der. \TClink{yeŋkɛlɛŋba} (see \TClink[1]{kɛlɛŋ}), \TClink{yeŋkɛlɛŋyeŋkɛlɛŋ} (see \TClink[1]{kɛlɛŋ})

\TCsubword[2]{yeŋkɛlɛŋ} (comp.) \textit{adj} good. \textit{Yeŋkɛlɛŋ, ba mi.} Very good, my father. \textit{Bikɔs gbɔsɛ kɔlɔ bo ncheni theni yeŋkɛlɛŋ.} Because if the smell is there, you would not feel good.

\TCsubword{yeŋkɛlɛŋba} (comp.), (der. of \TClink[1]{yeŋkɛlɛŋ}) \textit{adv} \textbf{1)} very much. \textit{Nɛn doki wɔe hun chɔŋ waaŋmaa len yeŋkɛ-lɛŋba.} This man began to love this woman very much. \textbf{2)} very well. \textit{Kaiŋ Taso wɔe gbaki ni hɔɛ, “Yeŋkɛlɛŋba, abɛna mi.”} Kain Tasso answered, “Very well, my elders.”

\TCsubword{yeŋkɛlɛŋyeŋkɛlɛŋ} (comp.), (der. \TClink[1]{yeŋkɛlɛŋ}) \textit{adv} \textbf{1)} thoroughly. \textit{Ira thoɛ yeŋkɛlɛŋ-yeŋkɛlɛŋ mɛnɛ ko.} We brush the bush thoroughly, right under. \textbf{2)} very well. \textit{So ni ikancheya pɛni tɔnthe kaŋga chɔche ŋɔ kɔ che ni ithe Mbolomdɛ yeŋkɛlɛŋ-yeŋkɛlɛŋ.} So we should be practicing singing for the church and for us to know Sherbro really well.

\TCheadword[2]{kɛlɛŋ} \textit{adj} \textbf{1)} nice. \textit{Næthi lɛ thipum tha thikəlɛŋ.} Some roads are fine (\citealt{Pichl1967}). \textbf{2)} fine. \textit{Kong ka che tamɔ kələŋ.} Kong was a fine boy (\citealt{Pichl1967}). \textbf{3)} good. \textit{Itu lo hɔ̃ kələŋ hã cho' thigbe̹r ɛ.} This iron is good for making axes (\citealt{Pichl1967}). \textbf{4)} beautiful. \textit{Kpɔnkɔ lɛ hɔ̃ kong pinkin dɛ trï bo̹m wəyni kəlɛng.} The forest was changed into a big and beautiful town (\citealt{Pichl1967}). \textbf{5)} wonderful. \textit{Itɔnk wa, ŋa mpanth ma wɔ kɛlɛn dɛ.} Celebrate for the wonderful work he has done. \textbf{6)} well.

\TCsubword{kɛlɛŋkɛlɛŋ} (der.) \textit{adj} fine, beautiful. \textit{Kaiŋ Taso ka mɔɛ tir bul, lɔ ka ke waaŋmaa kɛlɛŋ-kɛlɛŋ.} Kain Tasso reached a village where he saw a beautiful young woman. \textit{Nwantɛm agber hã trï ka ni hã akəlɛŋkəlɛŋ.} There are many young women in this town and they are very beautiful (\citealt{Pichl1967}). 

\TCheadword{kɛlɛŋkɛlɛŋ} (der. of \TClink[2]{kɛlɛŋ}) 

\TCheadword[1]{kɛm} \textit{n} bucket. \textit{La gbem dɛ woth chanth wɔ lɛ wɔn veleŋ ni muni woth kəm mmən wɔn bol.} The nursing mother carries her child on her back and she also carries a bucket of water on her head (\citealt{Pichl1967}). 

\TCheadword[2]{kɛm} \textit{n} metal. 

\TCsubword{kɛmsa} (comp.) \textit{n} \textit{kəmsa} (hɔ̃/tha) brass, copper (\citealt{Pichl1967}).

\TCheadword{Kɛma} \textit{nam} Kema, female name given to a person.

\TCheadword{kɛmɛ} \textit{Numb} hundred. \textit{Pàŋ Nanɔɛ, nɛn dɛ wul bul kɛmɛ koŋhɔanya mɛŋhiɔlniwaŋ, koŋhɔanya hiɔl ni mɛŋbul.} July 1986. 

\TCheadword{kɛmɛkɛ} \textit{cf}: \TClink[4]{ko}, \TClink[4]{min}, \TClink{tɛnin} (der. of \TClink[2]{tɛni}, \TClink[2]{-n}). \textit{v} think.

\TCheadword{kɛmsa} (comp. of \TClink[2]{kɛm}, \TClink[1]{sa}, see \TClink[2]{kɛm}) 

\TCheadword[1]{kɛn} \textit{v} be alone. \textit{Bahin yo, Bahin yo we, mɔm kɛn gbo mɔ i lanɛ.} Our Father, our Father o-o e-e, in you alone we trust. \textit{Yàŋ kə́n.} I am alone. \textit{À kə̀ní \`{ŋ}kə̀n.} I'm lonesome.

\TCsubword[4]{kɛn} (der.) \textit{adj} only. \textit{Wɔn kɛn wɔ gbem ŋan awaŋni tindɛ?} Is she the only one that gave birth to the twelve of you?

\TCheadword[2]{kɛn} \textit{cf}: \TClink[2]{bɛth}, \TClink{kɛth}, \TClink{rɔk}, \TClink{thak}. \textit{v} [kə́n] dice, cut into small pieces, esp. greens not used for meat. \textit{À kə́n bɔ́kɛ̀.} I chop into pieces. \textit{Yɛ mɔ koŋ kɛn bokɛ vɛ, kɛ yɛ mɔ kɔ chi bokɛ vɛ, mɔ kɔ le thɔkɔ.} After cutting the leaves, but after you have brought the leaves, you wash them first. \textit{Mɔi chal ni nkoŋkɔ kɛn yeŋkɛlɛŋ, mɔ kɔi bɛ pandɛ kunɛ.} You now sit and cut them nicely, then you put them in a pan.

\TCheadword[3]{kɛn} \textit{cf}: \TClink{kaana}. \textit{n} \textbf{1)} [kɛ̀n] tree species, rafia bamboo palm whose sap is drunk as palm wine and young leaves are used for raffia (K dialect). \textbf{2)} palm wine from tree of the same name, not as tasty as other palm wines (K dialect). 

\TCsubword{mɔɛŋkɛn} (comp.) \textit{n} bamboo wine.

\TCheadword[4]{kɛn} (der. of \TClink[1]{kɛn}) 

\TCheadword[1]{kɛna} \textit{n} rainbow.

\TCheadword[2]{kɛna} \textit{cf}: \TClink[1]{bi}. \textit{v} own. \textit{Ya kəna rai lo.} I own this book (\citealt{Pichl1967}). 

\TCheadword{kɛnda} \textit{cf}: \TClink[1]{kafa}, \TClink[1]{wɛi}. \textit{n} sin. \textit{I mbo sɛli we ŋa kɛnda ma iyɛ.} We have come to pray, Lord, for our sins.

\TCheadword{kɛnde} (Port \textit{candeia} ‘lamp') \textit{n} candle.

\TCheadword[1]{kɛnt} \textit{cf}: \TClink{gbeŋtheŋ}. \textit{n} wrist. \textit{Pɔ wɔe kue ŋgbekteɛ ŋkɛnt, ni pɔ chereŋ Kaiŋ Taso.} They took the handcuffs off his wrists and they freed Kain Tasso.

\TCheadword[2]{kɛnt} \textit{n} [kǝ́nt] tree species, most desirable tree for roofing and for mud and wattle walls, very difficult to find nowadays (K dialect). 

\TCheadword[3]{kɛnt} \textit{n} fruit type, monkey apple (\citealt{Pichl1967}). 

\TCheadword{kɛntak} (Eng \textit{canticle}) \textit{n} \textit{kɛntək} (hɔ̃/tha) Bible verse (\citealt{Pichl1967}). 

\TCheadword{Kɛnth} \textit{nam} Kent, name given to a village located on the southern tip of the Sierra Leone peninsula. \textit{Kɛnth kɔ lɔ livil.} Kent is far from here (\citealt{Pichl1967}). 

\TCheadword{kɛnth} \textit{v} \textbf{1)} break along length (intrans) (\citealt{Sumner1921}). \textit{Thafɛ lɛ kɔ dukɔɛ ni kɔ kənth.} The pipe fell down and broke (\citealt{Pichl1967}). \textbf{2)} break.

\TCsubword{kɛnthi} (der.) \textit{v} \textbf{1)} break in parts (transitive) (\citealt{Sumner1921}). \textit{Fɔs mi yaŋ ŋkumbɛ ni kɛnthi gbangba-m dɛ.} He struck me on my side and broke my rib (\citealt{Pichl1967}). \textit{ Yɛ ya wɔ hɔmɔ kənthi iwɔm dɛ wɔ ye kɔ.} When I tell him to break the firewood, he goes. \textbf{2)} break into, metaphorically, as into laughter, \textit{kɛnthi igbaka} ‘laugh loud and heartily' (\citealt{Pichl1967}). \textbf{3)} cut up.

\TCheadword{kɛnthi} (der. of \TClink{kɛnth}, \TClink[1]{-i}, see \TClink{kɛnth}) 

\TCheadword{kɛntri} (Mandinka \textit{kantiga} ‘groundnut') \textit{cf}: \TClink{malɔ}. \textit{n} \textit{kɛntrï} (kɔ/ma) groundnut (Arachis hypogaea) (\citealt{Pichl1967}); \textit{kɛntir} groundnut (\citealt{Sumner1921}). 

\TCsubword{kɛntripootoo} (comp.) \textit{n} \textit{kɛntrï pootoo} (kɔ/ma) breadnut (Artocarpus communis) (\citealt{Pichl1967}). 

\TCsubword{kɛntrithoɛ} (comp.) \textit{n} \textit{kɛntrï thoɛ} (kɔ/ma) bush groundnut (Desmodium adscendens) (\citealt{Pichl1967}). 

\TCheadword{kɛntripootoo} (comp. of \TClink{kɛntri}, \TClink{Potho}, see \TClink{kɛntri}) 

\TCheadword{kɛntrithoɛ} (comp. of \TClink{kɛntri}, \TClink[2]{tho}, \TClink[1]{ɛ}, see \TClink{kɛntri}) 

\TCheadword{kɛnya} \textit{n} uncle, [kɛ̀ɲà]/[\`{ŋ}kɛ̀ɲà] uncle/uncles (B dialect); \textit{ke̹ña} (wɔ/hã, N) uncle, father's brother (\citealt{Pichl1967}). \textit{Braima wɔe hun ko kenyaa wɔɛ Ba Amadu Kamara Planti ko, wɔɛ nɔhɔthɔ.} Braima then went to his uncle, Ba Amadu Kamara at Plantain (Island), who is a fisherman.

\TCheadword{kɛŋkɛn} \textit{n} grass species.

\TCheadword{kɛŋklɛni} \textit{cf}: \TClink{hɛm}. \textit{v} refuse; deny. \textit{Ya bɔnthɔ wɔ poo yekə, ya thom wɔ ni kənklɛni.} I met him sharing cassava; I begged him (for some), but he refused (\citealt{Pichl1967}). 

\TCheadword{kɛpi} \textit{v} scratch. \textit{Yayɛ wɔ kɛpiɛ tamɔ lɛ.} The cat scratched the boy (\citealt{Pichl1967}). 

\TCheadword{kɛrkɛr} [kɝkɝ] \textit{n} bird species, witch bird, seldom seen, but when it is, it is a harbinger of a baby's death, only known by its voice, something like a crow, when you hear it, the sound is disturbing, induces fear (K dialect).

\TCheadword{kɛth} \textit{cf}: \TClink[2]{bɛth}, \TClink[2]{kɛn}, \TClink[2]{mu}, \TClink{rɔk}, \TClink{thak}. \textit{v} \textbf{1)} cut. \textit{kɛth} cut. \textit{Sese theyɛn-nɛki, thɔ lɛ kəth wɔ yenwɛy.} Sese hurt himself, the adze badly cut him (\citealt{Pichl1967}). \textbf{2)} fell. comp. \TClink{palli-chɛthɛ} (see \TClink[1]{pal})

\TCsubword{ketheŋ} (der.) \textit{v} cut up. \textit{Yɛ pɔ pɛ mi ketheŋ kendɛ yekeɛ ha yeke kiɛ labi ŋhɔɛ, “Ya ka mɔ ŋɔ ni nsɔm”?} When they wanted to cut me like this cassava, that's why you said, “Let me give it to you and you chew”? \textit{Bɛl Maaɛ wɔe hɔ ko poo wɔɛ, amaaɛ ŋa pos, ŋa ketheŋ yekeɛ.} Rat Wife said to her husband, “Women are peeling, and they are dicing cassava.”

\TCsubword{yɔktha} (unspec.) \textit{n} \textbf{1)} the tree-cutting stage before the burning. \textbf{2)} farm with felled trees. 

\TCheadword{kɛthani} \textit{adj} perplexed. \textit{Cho koŋ kəthani, wɔ lɛ gboka-nɔ, chen bo chaŋ fay-hɔl ko yɛ theɛ min dɛ wɔ hɔ lɛ.} Cho is perplexed, he is a non-initiate, he cannot pass in front of the Poro bush when he hears the (Poro) spirit is talking (\citealt{Pichl1967}).

\TCheadword[1]{kɛthkɛth} (der. of \TClink[2]{kɛthkɛth}) 

\TCheadword[2]{kɛthkɛth} \textit{n} clock.

\TCsubword[1]{kɛthkɛth} (der.) \textit{temp} frequently. \textit{Plɛn dɛ kɔn poto kɛthkɛth hink Kyamp ka.} The plane goes frequently from Freetown to Europe (\citealt{Pichl1967}). 

\TCheadword{kɛtil} (Eng \textit{kettle}) \textit{n} kettle.

\TCheadword[1]{kɛi} \textit{v} burp, belch.

\TCheadword[2]{kɛi} \textit{v} come. \textit{Igbimi lɛ hɔ hã ya koŋ kuthni lɛ ŋgɛyɛn gbo ya bi hã wu.} The smoke had suffocated
me, if you had not come quickly, I would have died (\citealt{Pichl1967}). 

\TCheadword{kɛyɛni} \textit{v} avoid.

\TCheadword{kɛbɛlini} (der.) \textit{v} avoid.

\TCheadword[1]{ki} \textit{cf}: \TClink[1]{lan}, \TClink[3]{tho}, \TClink{wɔnɛ}. \textit{dem} \textbf{1)} this. \textbf{2)} these. \textit{Amaa ki, apum ŋa pos gbam dɛ, apum ŋa pos yekeɛ.} These women, some were peeling potatoes, others peeling cassava. \textit{Huɛɛ ŋɔ ken gbo, Braima wɔ le kɔ lɛliɛ mpɛl lo ki pɛiŋ.} Just as day breaks, Braima first goes to inspect these fishing lines. \textbf{3)} demonstrative suffix. comp. \TClink{kaki} (see \TClink[2]{ka}), \TClink{kakitiki} (see \TClink[2]{ka}), \TClink{wɔki} (see \TClink[1]{wɔ}), \TClink{wɔnɛki} (see \TClink{wɔnɛ}), \TClink{yɛkia} (see \TClink[3]{yɛ}), unspec. \TClink{mɛŋkoki} (see \TClink[1]{mɛŋk}) 

\TCsubword[2]{loki} (comp.) \textit{dem} \textbf{1)} this. \textit{Dɛn yɛ ibɛ nkɔkaɛ lɛko nyɔn doki ŋɔ pɔ vellɛ balansbɔllɛ.} Then we would put our shoes on the ground (for) this thing (game) they called balance ball. \textit{Mi gbisiŋ doki, bil loki lɔ mɔɔ kunɛ yini gbɔl ŋɔlɔ ŋa mɔm?} This engagement, this marriage that you are in, do you have peace of mind? \textbf{2)} these. \textit{Bɛl siatiŋ doki, ŋa gbik-gbikni tokɛ ko: kara, kara, kara, kara, kara, kara.} These two rats scamper above (the kitchen): kara, kara, kara, kara, kara, kara. \textit{Yɛ thoŋka ki gbi kɔ haani bɛl siatiŋ doki thiyeŋ dɛ…} When all this arguing is going on between these two rats… \textbf{3)} this place. \textit{Ichɔŋ la len bikɔs ka Bolom ka lɔki.} We like that because this place is Sherbro country. 

\TCsubword{kinɔ} (der.) \textit{cf}: \TClink[1]{kɔnɛ}. \textit{dem} this. \textit{Yel Nsaŋha ko, yel lo kinɔ ka che bomba nɛn thigber tha koŋ chaŋ dɛ.} The Island of Egusi, this island was very big many years ago.

\TCsubword{lanɛki} (der.) \textit{indfpro} this thing; this matter. \textit{Kɛ lanɛki lamɔ bia hun theli kiɛ…} But this thing you are coming to say… \textit{La cheŋ gbɔ, kɛ lanɛki boŋgoo lagbɔ.} It is not dificult, but the one these days is difficult.

\TCheadword[2]{ki} \textit{cf}: \TClink[2]{bi}, \TClink[3]{che}, \TClink[7]{kɔ}. \textit{prt} particle for definite near future; \textit{ki} used with verbs to express future time (\citealt{Sumner1921}). \textit{Yi ki che fay ko.} We will be in the Poro bush (\citealt{Pichl1967}). \textit{Lɛ ŋ kɔ gbo binthi sɔksi l'ay, n tuntni mma ki təm bo̹l mɔ.} If you go into the fowl yar, bend your head or you will bump your head (\citealt{Pichl1967}). 

\TCheadword[3]{ki} \textit{cf}: \TClink{gbuu}. \textit{n} \textit{kï} (wɔ/hã, si) crocodile (Crocodilus niloticus and cataphractus) (\citealt{Pichl1967}). comp. \TClink{lolki} (see \TClink[3]{lol}) 

\TCsubword{kiminmi} (comp.) \textit{cf}: \TClink{kɛbɛŋ}. \textit{n} fish species, (lit. ‘crocodile, swallow me'), nickname for the \textit{kəbəŋ} fish because it is said that he invites the crocodile to swallow him with his call, and when swallowed, inflates himself in the crocodile's stomach so that the crocodile dies (\citealt{Pichl1967}). 

\TCheadword{Kiamp} \textit{nam} Freetown, name given to a place. \textit{Ashiɛlɛ nkɔ pɛ Kiamp ko nsheɛ, so nwɔm yi len ŋa lan.} And I know you went to Freetown early on, so tell us something about that. \textit{Plɛn dɛ kɔn poto kɛthkɛth hink Kyamp ka.} The plane goes frequently from Freetown to Europe (\citealt{Pichl1967}). \textit{Yɛ ikoŋ mpanthɛ ma aŋaɛ, yai tipɛ pɛni pɛ ha bɛrɛ kaŋ miyɛ Champ ko ni.} When I finished the work we were doing, I started learning to add to my education in Freetown.

\TCheadword{kiban} \textit{n} expert. \textit{Ba Yaŋka wɔ chaŋ shi theli Mbolomdɛ; wɔ kiban dɛ, wɔ chaŋ shi theli Mbolomdɛ.} Ba Yanker knows how to speak Sherbro the best; he is the expert that knows how to speak Sherbro better (than anyone).

\TCheadword{kibaŋ} \textit{n} promontory.

\TCheadword{kibiŋ} \textit{n} mound.

\TCheadword{Kichom} \textit{nam} Kichom, name given to a place. \textit{Kichom lɔ mpəŋ atok yeŋthi Kɔnakri ni pok Kyamp.} Kichom is on the border between Guinea and Sierra Leone (\citealt{Pichl1967}).

\TCheadword{kichɛn} (Eng \textit{kitchen}) \textit{n} [kɪchən] kitchen (B dialect). 

\TCheadword{Kigba} \textit{nam} Kigba, name given to a person. \textit{Yami kacheɛ bɔi kigba.} My mother used to be Boi Kigba.

\TCheadword{kikith} \textit{cf}: \TClink{gbɛnth}. \textit{v} persist, continue, press on. \textit{Kikith} press down (hymn). \textit{Kikith ko gbi lɔ ŋcheka.} Press down whenever (something) is here (hymn). 

\TCheadword{kikkik} (Eng \textit{kick}) \textit{v} kick. \textit{Inan gballɛ, ilɔ pɛŋgipɛŋgi, i kikkik.} We draw the line, we jump there (and) kick.

\TCheadword[1]{kil} \textit{n} \textbf{1)} house. \textit{Kilthi lɛ tha Pujoŋ kunɛ tha bom.} The houses in Pujehun are big (\citealt{Pichl1967}). \textit{Ni wɔ ye kɔ killɛai wɔ ko.} And then he went into his house (his place) (\citealt{Pichl1967}). \textbf{2)} home. \textit{Nɔ ncheni ŋa gbemi wɔ kilɛko, gbemi hɔspitul koɛ.} No one should give birth at home, [they should] give birth at the hospital. comp. \TClink{nɔkil} (see \TClink{nɔ}) 

\TCsubword{kilbaŋkaŋ} (comp.) \textit{cf}: \TClink[2]{krikri} (der. of \TClink[1]{krikri}). \textit{n} house type, round house.

\TCsubword{kilchantha} (comp.) \textit{n} house type, rectangular house.

\TCsubword{kilɛihɔl} (comp.) \textit{cf}: \TClink{rɛnth}. \textit{n} \textbf{1)} door. \textbf{2)} doorway.

\TCsubword{kilgbakɛ} (comp.) \textit{n} wattle-and-stick house.

\TCsubword{kilkaŋ} (comp.) \textit{n} school. \textit{Mɔm nka kɔ kilkaŋdɛ?} Did you go to school? \textit{Aa, a tipɛ kilkaŋdɛ Nfɔs ko.} Yes, I started school in Mofos.

\TCsubword{kilpekɛ} (comp.) \textit{cf}: \TClink{hɔspith}. \textit{n} hospital.

\TCsubword{kilrithi} (comp.) \textit{n} prison.

\TCsubword{kilthipe} (comp.) \textit{n} house type, stone or cement house. 

\TCheadword[2]{kil} \textit{Idph} of traveling, sound of many tapping feet or traffic. 

\TCheadword[3]{kil} \textit{cf}: \TClink{nɔmpithika} (comp. of \TClink{nɔ}, \TClink{pithika}), \TClink{nɔsukusɛkɛ} (comp. of \TClink{nɔ}, \TClink[1]{sukusɛkɛ}). \textit{n} \textbf{1)} rascal. \textit{Baki wɔ ŋkil.} Baki is a rascal (\citealt{Pichl1967}). \textbf{2)} rascality.

\TCheadword{kilbaŋkaŋ} (comp. of \TClink[1]{kil}) 

\TCheadword{kilchantha} (comp. of \TClink[1]{kil}) 

\TCheadword{kilɛihɔl} (comp. of \TClink[1]{kil}, \TClink[1]{ɛ}, \TClink[1]{ahɔl}, see \TClink[1]{kil}) 

\TCheadword{kilgbakɛ} (comp. of \TClink[1]{kil}, \TClink[4]{gba}, \TClink{-k}, see \TClink[1]{kil}) 

\TCheadword{kilia} (Eng \textit{clear}) \textit{cf}: \TClink[1]{charaŋ}. \textit{adv} clearly. \textit{Mbolom dɛ ma wɔni kilia ni charaŋ.} The Sherbro language is being spoken clearly and cleanly.

\TCheadword{kilik} \textit{cf}: \TClink{haŋka}. \textit{n} anchor. \textit{Ka nlɛrni, wɔe duki kilikɛ.} He hurried up and dropped the anchor. \textit{Kɛ be, kilikɛ ŋɔ ton ha bɔɔ yɛthi wɔm dɛ mmɛn nyamban dɛai huɛ vɛ.} But no, the anchor was (too) small to hold the canoe in the rough sea that day.

\TCheadword{kilim} \textit{n} crab species, kind of inland crab (\citealt{Pichl1967}). comp. \TClink{wokilin} (see \TClink[2]{wo}) 

\TCheadword{kiliŋ} \textit{n} drum type, large, made of one tree with 2-3 slots and beaten with two sticks (\citealt{Pichl1967}). 

\TCheadword{kilkaŋ} (comp. of \TClink[1]{kil}, \TClink[3]{kaŋ}, see \TClink[1]{kil}) 

\TCheadword{kilkil} \textit{prep} opposite to. Triniti chəəch hɔ kilkil Ani Wɔlsh skuul. Trinity Church is opposite to Annie Walsh School (\citealt{Pichl1967}). 

\TCheadword{kilpekɛ} (comp. of \TClink[1]{kil}, \TClink{pekɛ}, see \TClink[1]{kil}) 

\TCheadword{kilrithi} (comp. of \TClink[1]{kil}, \TClink[1]{rithi}, see \TClink[1]{kil}) 

\TCheadword{kilthipe} (comp. of \TClink[1]{kil}, \TClink{pe}, see \TClink[1]{kil}) 

\TCheadword{kiminmi} (comp. of \TClink[3]{ki}, \TClink[1]{min}, \TClink[1]{mi}, see \TClink[3]{ki}) 

\TCheadword{kimɔ} \textit{cf}: \TClink{gbikni}, \TClink{parat}. \textit{v} run away, flee, retreat. \textit{Ŋà ké àjókwɛ̀ wɔ̀ kímɔ̀.} They saw his son running away.

\TCheadword[1]{kin} \textit{n} fish species, small kuta fish (\citealt{Pichl1967}). 

\TCheadword[2]{kin} \textit{n} [kín] vine species, leaves used for sauce (K dialect). 

\TCheadword{kinda} \textit{v} run up. \textit{La veiɛni, bɛlsɛ ŋae kinda baiɛ tokɛ: <kara-kara kara-kara kara-kara>.} It was not long after when the rats ran up above the bari: <kara-kara kara-kara kara-kara> (idph of scampering). 

\TCheadword{kinɔ} (der. of \TClink[1]{ki}, \TClink{nɔ}, see \TClink[1]{ki}) 

\TCheadword{kipkip} \textit{n} [kǝ́pkǝ́p] tree species (K dialect). 

\TCheadword{kiptha} \textit{cf}: \TClink[2]{fan}. \textit{n} fermenting palm wine.

\TCheadword{Kisi} \textit{nam} Kissy, name given to a place. \textit{Kisi lɔ fɛsɛ Kyamp ko.} Kissy is near Freetown.

\TCheadword{kisi} \textit{n} [kìsì] plant species, lily type, fragrant roots ground and used for an ointment for babies and women (K dialect).

\TCheadword{kisik} \textit{cf}: \TClink{nyɛŋkin}. \textit{temp} finally.

\TCheadword[1]{kisiŋ} \textit{n} animal species.

\TCheadword[2]{kisiŋ} \textit{n} [kìsǝ̀ŋ] tree species, plum tree (K dialect). 

\TCheadword{kiskis} \textit{v} kiss. \textit{Mɔ wɔ bala-bala ni, wɔn bɛ wɔ mɔ balani, mɔ wɔ kis-kis yɛŋ bɛ, wɔi po ha yɛthi mmɔ ma mɔɛ.} You hug him, he hugs you, you kiss him all over, then he begins to hold your breast.

\TCheadword{kit} (\textit{Eng kit}) \textit{n} kit. \textit{Kɛ kitɛ ŋɔmi fi.} But I still have the kits. \textit{Gbi hɔ ka che kitɛ kunɛ.} It all used to be in the kit.

\TCheadword[1]{kith} \textit{adj} short. \textit{Thɔk lɛ kɔ kith.} The stick is short (\citealt{Pichl1967}). \textit{A yema vikini kɛ hinth lo kɔ kith hã yang.} I want to stretch but this bed is too short for me (\citealt{Pichl1967}). comp. \TClink{paaŋkith} (see \TClink[2]{paŋ}), \TClink{nɔkith} (see \TClink{nɔ})

\TCsubword{kithkith} (der.) [kìthkìth] \textit{adj} very short, [thɔ̀k ŋkìthkìthɛ̀] very short tree (K dialect). 

\TCsubword{likith} (der.) \textit{n} shortness. \textit{La gbo likith ken hwɛ lɛ.} It is only as short as a day (\citealt{Pichl1967}). 

\TCheadword[2]{kith} \textit{v} \textbf{1)} be brackish (\citealt{Pichl1967}). \textbf{2)} [kìth] hard to swallow, as babies will not swallow, not bitter (K dialect).

\TCheadword{kithkith} (der. of \TClink[1]{kith}) 

\TCheadword{kithni} \textit{adj} \textbf{1)} tight. \textbf{2)} crowded.

\TCheadword{kiyan} (Eng \textit{can}) \textit{n} can.

\TCheadword{klampis} [klámpís] \textit{n} whale (K dialect); \textit{krampïs} (wɔ/hã, N) kind of whale (Eng. grampus) (\citealt{Pichl1967}). 

\TCheadword{klas} (Eng \textit{class}) \textit{n} class. \textit{Atipɛ ko klas wan.} I started in class one. \textit{Wɛl, a mɛkɛni klas thri.} Well, I stopped at class three.

\TCheadword[1]{ko} \textbf{1)} \textit{adp} to. \textit{Awokɔ gbo ko mɔ ko yai hun ko Mi Adama.} After leaving you, I will go to Mami Adama. \textit{Kɛ kpɔnko hɔ ka che trï ko ntɛnt, hɔ nɔonɔ ka chen kɔ ai ɛ.} But there was a forest near the town, which no one entered (\citealt{Pichl1967}). \textit{Pəmdɛ kɔ busni Mpelɛ ko.} War has broken out at Mpele (\citealt{Pichl1967}). \textbf{2)} \textit{adp} with. \textit{Kɛ wɔ ko bamɔ?} But is she with your father? \textit{Atipɛ komɔko.} I will begin with you. \textbf{3)} \textit{adp} from. \textit{Yà hínk kò Bà Yànkà.} I came from Ba Yanker. \textbf{4)} \textit{prep} by. \textit{Wɔi kɔni pɔyko, yɛ kɔni yɛ wɔi ko sɛm ko thɔkɛ, wɔi po ŋa tɔn.} When she went to the stream, she stood by the tree, and then she started to sing. \textit{Yɛ le wɔ lɔɛ, thɔmko taɛ mpanthɛ man gbi wandaɛ wɔ ma ko nŋa woŋgo.} When she left her there, the junior mate, all the house work had been done by the girl alone. \textbf{5)} \textit{post} on. \textit{Yɛ mɔni koŋ thɔk itu bɛiaɛ vɛ, mɔi kɔ thu pɛlɛ, mɔi huŋ bɛ lalako.} After you have washed the rice pot, you measure the rice and then put it on the fire. \textbf{6)} \textit{post} at. \textit{A-a, Themdel ko, tiko wɔ ko lɔɔ Nsanda ko.} No, at Timdale (Chiefdom), his town is called Nsanda. \textit{Ka lɔ pɔ bɛ bia huŋa sakaɛ, lel ko, ŋgasumana ko.} It is here that they would have to come and do his sacrifice, at Mokainsumana. \textit{Mbuɛ ko.} At Mbueh. \textbf{7)} \textit{post} in. \textit{Nande ako vel laŋgba bul wɔ pɔ gbem Themdɛl ko.} Today I have called on a man who was born in Timdale (Chiefdom). \textit{Wɔn pɔ gbem wɔ Nra ko.} She was born in Ra (village). \textbf{8)} \textit{post} in front of; before. \textit{Mbo̹lo̹m ŋwɛi ma che paalɛ bai ko, anya atïŋ dɛ hã lo̹l.} In the bad case that was recently before the court, the two men were set free (\citealt{Pichl1967}). \textbf{9)} \textit{post} into. comp. \TClink{hɛlɛiko} (see \TClink[2]{hɛlɛ}), \TClink{lelko} (see \TClink[2]{lel}), der. \TClink{tiko} (see \TClink{tii}), \TClink{kakitiki} (see \TClink[2]{ka}), unspec. \TClink{mɛŋkoki} (see \TClink[1]{mɛŋk}) 

\TCsubword{koŋgbiŋk} (comp.) \textit{adj} common. \textit{Mɔɛyktu che ki lo hɔ̃ ko-ŋgbïnk pɔk Afrika lɛ.} This is a dilemma common to all Africa (\citealt{Pichl1967}). 

\TCheadword[2]{ko} \textit{cf}: \TClink[2]{ka}. \textit{Loc} \textbf{1)} ahead. \textbf{2)} yonder. \textbf{3)} there. \textit{Nɛki gbɔl ko sɔthɔ ko, lanɛ gbi nante.} There is heartache in this world today. \textit{Pɔ koŋ gbo pɔi gbɛki amaɛ, ŋai kɔni ko futh pɛlɛ.} When they have finished, they hire the women to go and uproot the rice. \textbf{4)} general locative particle. \textit{I chɔŋ la len yɛ pɔ chaŋ theli Mbolomdɛ, bikɔs inal pim, Bolomko lɔɛ.} We like that because they speak Sherbro here more, because other places are Sherbro lands. \textit{Anyaɛ kani gbo che vel yelloɛ “Yel nsaŋha ko.”} The people were only now calling this island “Island of Egusi.” comp. \TClink{lɔko} (see \TClink[1]{lɔ}) 

\TCsubword{ko-gbi} (comp.) \textit{pro-form} wherever. \textit{Ko-gbi lɔ gbo lɛ, Hɔbatokɛ wɔ lɔ.} Wherever I may go, there is God (\citealt{Pichl1967}).

\TCsubword{koki} (der.) \textbf{1)} \textit{pro-form} to that place. \textit{Loc} yonder. unspec. \TClink{kokitiki} (see \TClink[2]{ko})

\TCsubword{kokitiki} (der.), (unspec. of \TClink{koki}) \textit{pro-form} to that very place.

\TCheadword[3]{ko} \textit{n} compound.

\TCheadword[4]{ko} \textit{cf}: \TClink{kɛmɛkɛ}, \TClink[4]{min}, \TClink{tɛnin} (der. of \TClink[2]{tɛni}, \TClink[2]{-n}). \textit{v} consider.

\TCheadword{ko-gbi} (comp. of \TClink[2]{ko}, \TClink[3]{gbi}, see \TClink[2]{ko}) 

\TCheadword{ko-lɔ} \textit{cf}: \TClink{sɔnday}. \textit{subordconn} rather than, instead of.

\TCheadword{koa} \textit{cf}: \TClink{kola}, \TClink{kolo}. \textit{n} fish species, tarpon (\citealt{Pichl1967}). 

\TCheadword{koba} \textit{n} [kóbà] tree species, spindly tree, very light, leaves used for medicine (K dialect). 

\TCheadword{kobo-mɛn} (comp. of \TClink[3]{mɛn}) 

\TCheadword{kobotu} (comp. of \TClink[1]{tu}) 

\TCheadword{Kofuŋ} \textit{nam} Kofung. \textit{Anya kofuŋ dɛ hã bi sɔkɔth gber, bul hɔ lɛ pə pɔŋ wɔ gbo kil lɛ ko, pə kantha hɔ ka gbooku, wɔ honi si pə be yi kil lɛ.} The Kofung people have many magical powers, one is they throw him into a house which they lock with a padlock, he will get out and the house is not opened (\citealt{Pichl1967}).

\TCheadword{koi} \textit{cf}: \TClink[1]{hinth}. \textit{v} swell; increase in volume.

\TCheadword{koinsidɛnt} (Eng \textit{coincidence}) \textit{n} coincidence. \textit{Themnɔ bai koinsidɛnt ŋɔ ŋa sɔthɔ Koromaɛ vɛ.} The Themne got (the surname) Koroma by accident.

\TCheadword{koiye} \textit{cf}: \TClink{pɛnɛk}. \textit{v} \textbf{1)} scream. \textit{Ŋa ka che mi sɔiɛ, akoiye.} They used to scare me (so that) I would scream. \textbf{2)} shout. \textit{Pɔ mɔ koil ye vɛ, la kɔ kanni.} When people shout at you, it does not look good.

\TCheadword{kok} \textit{cf}: \TClink{bɔŋk}, \TClink{tii}. \textit{n} \textbf{1)} [kòk] buttress, e.g., of a large cotton tree (kapok) distinct from \textit{tii} ‘base of a cotton tree' (K dialect). \textit{Ŋ̀kók mà pòlòndɛ́.} The buttresses of the cotton tree. \textbf{2)} scrotum (\citealt{Pichl1967}). 

\TCsubword{kokkunɛ} (comp.) \textit{cf}: \TClink{lua}. \textit{n} hernia.

\TCheadword{koki} (der. of \TClink[2]{ko}, \TClink[1]{ki}, see \TClink[2]{ko}) 

\TCheadword{kokitiki} (unspec. of \TClink{koki} (der. of \TClink[2]{ko}, \TClink[1]{ki}), see \TClink[2]{ko}) 

\TCheadword{kokkunɛ} (comp. of \TClink{kok}, \TClink{kunɛ} (der. of \TClink{kun}, \TClink[1]{ɛ}), see \TClink{kok}) 

\TCheadword{kokovaia} \textit{n} rice variety (\citealt{Pichl1967}, \citealt{Sumner1921}). 

\TCheadword{kol} \textit{n} \textbf{1)} [kól] kola nut, used in propitiating ancestors, has importance as a gift of significance in many contexts, e.g., marriage, initial greeting to town chief (K dialect). \textbf{2)} kola tree, can be used for lumber, has an attractive brown color (K dialect). \textbf{3)} gift (\citealt{Pichl1967}). 

\TCsubword{kolabɛna} (comp.) \textit{n} gift (often a drink) for parents or elders (\citealt{Pichl1967}). 

\TCsubword{kolbai} (comp.) \textit{n} court fee (\citealt{Pichl1967}). 

\TCsubword{kolbom} (comp.) \textit{n} last stage of engagement, present given to the parents on this occasion (\citealt{Pichl1967}). 

\TCsubword{kollɛnyɛ} (comp.) \textit{n} [kóllɛ̀nyɛ̀] greeting gift, usually money, that an outsider makes to the notables of a village (B dialect). 

\TCsubword{kolsiroŋ} (comp.) \textit{cf}: \TClink[1]{sɔŋ}. \textit{n} corruption fee. \textit{Bɛɛ lɛ Kɔng kol sirɔng hã sɔng wɔ ni kɔ wɔŋ beli li-mbul.} The chief gave Kong a corruption fee to bribe him to go and give false evidence (\citealt{Pichl1967}). 

\TCsubword{muŋkokol} (id.), (comp.) \textit{v} return the dowry (lit. return the kola) (\citealt{Pichl1967}). 

\TCheadword{kola} \textit{cf}: \TClink{koa}, \TClink{kolo}. \textit{n} fish species, reddish found in swamps, caught to eat, two inches at its biggest (K dialect).

\TCheadword{kolabɛna} (comp. of \TClink{kol}, \TClink[1]{bɛn}, see \TClink{kol}) 

\TCheadword{kolbai} (comp. of \TClink{kol}, \TClink[1]{bai}, see \TClink{kol}) 

\TCheadword{kolbom} (comp. of \TClink{kol}, \TClink{bom}, see \TClink{kol}) 

\TCheadword{kollɛnyɛ} (comp. of \TClink{kol}, \TClink{lɛnyɛ} (unspec. of \TClink[1]{lɛŋ}), see \TClink{kol}) 

\TCheadword{kolo} \textit{cf}: \TClink{koa}, \TClink{kola}. \textit{n} fish species, kaima fish (\citealt{Pichl1967}). 

\TCheadword{Kolone} \textit{nam} Kolone, female name given by a society. 

\TCheadword[1]{koloŋ} \textit{n} (wɔ/hã, N) ant species, sugar ant (\citealt{Pichl1967}, \citealt{Sumner1921}). 

\TCheadword[2]{koloŋ} \textit{n} \textit{kolo̹ng} (kɔ/ma) testicles (vulgar) (\citealt{Pichl1967}). 

\TCheadword[3]{koloŋ} \textit{n} cockroach (K dialect); \textit{kolung} cockroach (Blatta spp) (\citealt{Pichl1967}). 

\TCheadword{kolsiroŋ} (comp. of \TClink{kol}) 

\TCheadword{kombutha} \textit{n} peel, shell, e.g., of groundnuts. 

\TCheadword{komnɛ} (Themne) \textit{n} \textbf{1)} father-in-law, [kòmnɛ́]/[\`{ŋ}kòmnɛ́] father- or brother-in-law/pl. (B dialect). \textit{Hi kɔ la hɔm komnɛ wɔɛ.} Let us go and tell his father-in-law. \textbf{2)} son-in-law. \textit{Baa waaŋmaaɛ wɔe wom ko komnɛ wɔɛ Kaiŋ Taso lɛ jajɛl wɔɛ koŋ wu.} The young woman's father sent a message to his son-in-law, Kain Tasso, that his mother-in-law had died.

\TCheadword[1]{komɔ} \textit{n} \textbf{1)} child. \textit{Yááɛ̀ bàlàní kòmɔ̀wɛ́.} The mother hugged her child. \textbf{2)} baby. \textit{Ŋa ŋɔi stich ahɔl, siɛ yɛ komɔɛ wɔ hundɛ honi bo.} They had stitched the exit mouth, you know when the baby is about to come, after it is out.

\TCheadword[2]{komɔ} \textit{n} your place. \textit{Yɛmɔ theli ko aŋaɛ, nwɔk mpim ma pɔ chi komɔko ma che ndumɔ, nye?} When you talk to the people, some cases they bring to you are difficult, right?

\TCheadword{komplo} \textit{n} bird species with sweet song, wings something like a bat's, found in the bush, seldom seen (K dialect). 

\TCheadword{komptha} \textit{n} [kòmpthà] tree species (K dialect). 

\TCheadword{kompuŋ} \textit{Idph} of falling in water, given as the Sherbro equivalent of a Mende ideophone (K dialect). 

\TCsubword{kumpohani} (comp.) \textit{v} plunge into the water and enjoy oneself by frolicking (several persons) (\citealt{Pichl1967}). 

\TCheadword{Kona} \textit{nam} Kona, name given to third daughter. 

\TCheadword{konat} \textit{cf}: \TClink{bɛlpotho} (comp. of \TClink[2]{bɛl}, \TClink{Potho}). \textit{n} coconut. \textit{A yuk ikonatɛ.} I plant coconut.

\TCheadword{Koni} \textit{nam} Koni, name given to third daughter. \textit{Koni} name for third daughter of a man, both \textit{Koni} and \textit{Kona} short for \textit{Konima} (per Abdulai Bendu). \textit{Kòní bɛ́ ǹyéék [ɪ] má kómɔ̀wɛ̀ bààlàɛ́-áí.} Koni put the child's things in the basket.
 \TCheadword{koni} (der. of \TClink[1]{koŋ}, \TClink{-ni}, see \TClink[1]{koŋ}) 

\TCheadword{kont} \textit{n} small wasp species (K dialect).

\TCheadword[1]{kontho} \textit{n} \textbf{1)} flying fish. \textbf{2)} mudskipper. \textit{Mɔ gbo chɔ pu konthoɛ, ha ni pothɛ kɔ kek mɔni.} If you fight with the mudskipper, then let the mud be seen on you (proverb) (\citealt{TISLL1979}).

\TCheadword[2]{kontho} \textit{cf}: \TClink{kɔŋko}. \textit{n} tortoise shell, emblem of the Poro society for various purposes, e.g., for the \textit{gbanabom}, the \textit{famancha}, the disciples of the Taso or Kase (\citealt{Pichl1967}). 

\TCheadword{Koŋ} \textit{nam} Kong, male name given to a person. 

\TCheadword[1]{koŋ} \textit{v} \textbf{1)} finish. \textbf{2)} end. \textit{Ko lɔ mpanth ma pɛlɛ ma ni hun koŋdɛ.} Where the rice farm work comes to an end.

\TCsubword{koni} (der.) \textit{Aux} perfect. \textit{Yi koni shi temdɛ ŋɔ pɔ gbem mɔ, ko lɔ pɔ gbemmɔ?} We already know when you were born, where were you born? \textit{So nɛnthi wɔ tha nkoni, ok, nkoyi ni toŋgi Nenthɛ tha nkolɔ ni koi yɛ.} So how many have you got, OK, you have shown us the number of years you have taken.

\TCheadword[2]{koŋ} \textit{cf}: \TClink[1]{bɔk}, \TClink[1]{kɛk}, \TClink[2]{nya}. \textit{n} turtle species, big kind of sea turtle (Dermochelys coriacea or Chelone mydas) (\citealt{Pichl1967}). 

\TCheadword[3]{koŋ} \textit{Aux} perfect aspect marker. \textit{Ya ka ni hani santhɛ, isɔ bul a koŋ thukuli jomi kusɛ ayema kɔ jo…} When I had grown up, one morning after I had just warmed my rice and wanted to eat it… \textit{I amɛn bullɛ ka koŋ wu.} We are five, one died a while ago. \textit{Jizɔs, a chɔŋ mɔ len ŋa lanɛ la ko ŋa ha yaŋ.} Jesus, I love you for what you have done for me. \textit{Nande ako vel laŋgba bul wɔ pɔ gbem Themdɛl ko.} Today I have called on a man who was born in Timdale (Chiefdom). \textit{So lan la ako ha ŋkuath ha ŋɔth.} So that is how I became afraid of fishing.

\TCheadword[4]{koŋ} \textit{cf}: \TClink[1]{lɛ}. \textit{n} palm branch.

\TCsubword{nyamkoŋ} (comp.) \textit{n} palm frond rib.

\TCheadword{koŋgbiŋk} (comp. of \TClink[1]{ko}, \TClink[2]{gbiŋk}, see \TClink[1]{ko}) 

\TCheadword[1]{koŋkbo} (comp. of \TClink[4]{bol}) 

\TCheadword[2]{koŋkbo} (id. of \TClink[1]{koŋkbo} (comp. of \TClink[4]{bol}), see \TClink[4]{bol}) 

\TCheadword{koŋkbos} \textit{n} cucumber, [koŋkbos]/ [ŋ̀kɔ̀ŋkbósɛ̀] cucumber/the cucumbers (B dialect); [kòmbòs] cucumber (K dialect).

\TCheadword{koŋko} \textit{n} (hɔ̃/tha) room type, small separate room, an addition to a house, kiosk (\citealt{Pichl1967}). 

\TCheadword{koŋkonani} \textit{n} [kóŋkónání] vine species that bears yellow fruit, leaves used medicine to treat malaria (K dialect).

\TCheadword{koŋkonya} (comp. of \TClink{kɔŋko}, \TClink[2]{nya}, see \TClink{kɔŋko}) 

\TCheadword{koŋ-kosul} \textit{adj} inveterate, obstinate, beyond reform.

\TCheadword{koŋkothuba} \textit{n} [kóŋkóthùbà] plant species, lily-like plant that can grow as high as five feet, has roots used for medicine (K dialect). 

\TCheadword{koŋkowalia} \textit{n} cross (\citealt{Pichl1967}). 

\TCheadword{Koroma} \textit{nam} Koroma, name given to a person. \textit{Lɔkɔoai pɔ wɔ velɛ Mista Koroma a shini.} He is called Mr. Koroma, I do not know. \textit{Akoroma ŋɔ cheni them.} Koromas are not Themnes.

\TCsubword{Koromanɔ} (der.) \textit{n} a Koroma person; the Koroma people. \textit{Koromanɔ aida ɔrijin wɔɛ wɔ Maninkanɔ... che Themnɔ wɔɛ.} Koroma, either the origin is Maninka... it is not Themne. \textit{Mɛndenɔ gbi wɔ Koromanɔ wɔ Maniŋka.} All the Mende people who are Koroma people are Maninka people.

\TCheadword{koromanɔ} (der. of \TClink{Koroma}, \TClink{nɔ}, see \TClink{Koroma}) 

\TCheadword{kos} \textit{cf}: \TClink[1]{gbu}. \textit{n} [kòs], [\`{ŋ}kòsɛ́] jaw (B dialect).

\TCheadword{kosi} \textit{cf}: \TClink[4]{po}, \TClink{sɛin}. \textit{v} separate; part. \textit{Apuma lɛ hã chɔ' yenwɛy, hã kɔ koosi.} The children are fighting badly, do go part them (\citealt{Pichl1967}). \textit{Apuma lɛ hã cho', santh lɛ tunt thɔm wɔ lɛ yenwɛy, hã kɔ hã koosi.} The children are fighting, the older one has badly twisted his companion, go and separate them (\citealt{Pichl1967}). 

\TCheadword{kostal} (Eng \textit{coastal}) \textit{adj} coastal. \textit{Ɛlaboɛ kostal eria, halthe ntɛnt lɔ Athemaɛ ŋahun challɛ.} Just the coastal areas, the seaside where the Themnes have come and settled.

\TCheadword{kotha} \textit{n} \textbf{1)} [kóthà] clothes. \textit{Kothathi wɔ lɛ tha chen bɔnni lɛɛ kɔ.} His clothes don't drag on the ground (\citealt{Pichl1967}). \textit{Ŋkɔ lath kotha-thi lɛ honka lɛ ay.} Go spread the clothes outside (\citealt{Pichl1967}). \textbf{2)} cloth (K dialect). \textit{Yàìyɛ́ wɔ́ kóthàɛ̀ àlɔ̀.} The cat is under the cloth. \textit{Lɛ nɔɔmiɛ gbo kotha lɛ hɔ thuk lɛ, ya bi hã paka mɔ.} If you should find the cloth that was lost, I shall pay you a reward (\citealt{Pichl1967}). 

\TCsubword{kothasampa} (comp.) \textit{n} [kòthàsámpà] bush species (lit. bush cloth), leaves woven together and worn tied around the waist by the woman who announces the women's society (K dialect). 

\TCheadword{kothasampa} (comp. of \TClink{kotha}, \TClink[1]{sampa}, see \TClink{kotha}) 

\TCheadword{kothikothi} \textit{n} [kòthìkòthì] tree species, clumping tree, used for setting traps, leaves used as medicine (K dialect). 

\TCheadword{Kothuŋ} \textit{nam} Kothung, name given to 6\textsuperscript{th} son.

\TCheadword{koy} \textit{n}\textit{ŋkoy} (ma) false face, mask (\citealt{Pichl1967}). 

\TCheadword{koyɛ} \textit{v} accept. \textit{Lɛ nɔ koyɛni gbo ha pɔn bɛmpa la, makɔni kɔtai, lokal kɔt.} If the person does not accept the settlement, they go to the court, the local court.

\TCheadword[1]{kɔ} \textit{cf}: \TClink[2]{bɛk}, \TClink{gbamfa}. \textit{n} \textbf{1)} cover (B dialect). \textbf{2)} quiver (\citealt{Pichl1967}).

\TCheadword[2]{kɔ} \textit{cf}: \TClink[2]{bulɔ}, \TClink{haa}. \textit{v} \textbf{1)} go. \textit{Ya ka ni hani santhɛ, isɔ bul akoŋ thukuli jomi kusɛ ayema kɔ jo…} When I had grown up, one morning after I had just warmed my rice and wanted to eat it… \textit{Pɔɔ wɔ ŋa kɔ gbemɔ Nyamba ko kɛ lɔ pɔ ka ŋa wɔ sizaɛ, nthela nye.} They (said) she is to go to Moyamba and do the Cesarean-section there. \textit{Ishiɛ ŋanɛ ŋa bia kɔ hundɛ…} We know that those that are going to come… \textbf{2)} attend (school). \textit{Skul handɔ ŋɔ nkache kɔa?} Which school did you attend? \textbf{3)} leave. \textit{Wɔn bɛ yɛ wɔ kɔɛ, ndum malan maa yɔk.} When she leaves, it is with that training that she will go. \textbf{4)} will. comp. \TClink{nɔkɔmbɛl} (see \TClink{nɔ}), unspec. \TClink{chɛtlipalkɔ} (see \TClink[1]{pal}) 

\TCsubword[1]{kɔma} (der.) \textit{v} go with. \textit{Ŋ kɔma mi.} Go with me (\citealt{Pichl1967}).\textit{ Ngbəŋ hun mi che, tɛmpum ya bi hã kɔma mɔ.} Come to me tomorrow, maybe I shall go along with you (\citealt{Pichl1967}).

\TCsubword{kɔni} (der.) \textit{v} go. \textit{Hanya pikaɛ, pɔ gbɛki hanyaɛ ha kɔni bɔm thai.} Some other people, they will hire people to go to their mud plots. \textit{Ko kɔni kɛ wɔ pɛ hunɛ.} He has gone but is coming again.

\TCsubword{kɔnaibol} (id.), (comp.) \textit{cf}: \TClink{fol}, \TClink{naibol} (id. of \TClink[1]{nai}, \TClink[1]{bol}), \TClink{sɛmɛkni} (der. of \TClink[1]{sɛm}, \TClink{-k}, \TClink{-ni}), \TClink{thil}. \textit{v} \textit{kɔ-næ-bol} urinate/defecate, ‘relieve oneself' (polite) (lit. go to the head of the road) (\citealt{Pichl1967}).

\TCheadword[3]{kɔ} \textit{NCP} \textbf{1)} it. \textit{Pəmdɛ kɔ busni Mpelɛ ko.} War has broken out at Mpele (\citealt{Pichl1967}). \textit{Kəfe kɔ fay.} Pepper is hot (\citealt{Pichl1967}). \textbf{2)} they. \textit{Yɛlaio wɛ, yɛ jaɛ ma ko ŋani mgbeɛ ŋɔ maredɛ kɔ bi ni prɔblɛm thɛ.} Nowadays, when things are abundant, all the marriages are full of problems. Kəfɛ lɛ kɔ nyonkni. The peppers are shrinking (as they dry) (\citealt{Pichl1967}). \textbf{3)} relative pronoun; that. \textit{A minɛ pɛl kɔ mɔ kɔ woɛ.} I thought it was a net that you would throw. \textit{Tondɛ kɔ lɛ ituɛ kunɛ, mɔ kɔi kɔ thɔŋgul ŋa paŋdɛ.} The small bit that remains in the pot, you reserve it for the evening.

\TCheadword[4]{kɔ} \textit{cf}: \TClink{bus}, \TClink{jal}. \textit{n} \textbf{1)} body.\textit{ Wanta lɛ wɔ kəlɛŋ wɔ lɛ kɔ vil.} The girl is nice, her body is long (\citealt{Pichl1967}). \textbf{2)} skin. \textit{Thenthes hɔ wɛy, pə bak hɔ gbo nɔ wɔ sɔkul likɔɔ.} \TClink[1]{thenthes} is bad, they just rub it on a person, (and) it makes him scratch his skin (\citealt{Pichl1967}). \textit{Yaŋ likɔ lɔ gbɔw sɔkul, ya bi isɔkul gber.} My skin is very itchy, I have a lot of craw-craw (\citealt{Pichl1967}). comp. \TClink{pokɔ} (see \TClink[1]{po}) 

\TCsubword{kɔluŋ-vil} (comp.) \textit{n} body along spine.

\TCsubword{kufu} (unspec.) \textit{n} \textbf{1)} skin irritation. \textbf{2)} rash.

\TCheadword[5]{kɔ} \textit{v} cut palm nuts. \textit{Ŋ kɔ too wa lɛ ni ŋ kɔ mbəl lɛ!} Mount the palm tree and cut the nuts! (\citealt{Pichl1967}). 

\TCheadword[6]{kɔ} \textit{cf}: \TClink[2]{paka}, \TClink[1]{pin}. \textit{v} pay. \textit{Abɛna wɔɛ ŋae bɛmpani yeŋkɛlɛŋ ba ni ŋae kɔ path.} His parents prepared themselves well and engaged the lady (paid the bride price).

\TCheadword[7]{kɔ} \textit{cf}: \TClink[2]{bi}, \TClink[3]{che}, \TClink[2]{ki}. \textit{Aux} future auxiliary verb; modal ‘will.' \textit{Yɛ kɔ koŋ gbemɔɛ, kɔ koŋ gbo kɔi hun dri.} After the rice has tillered, it will ripen. \textit{So lagboɛ nɔ wu, ramdɛ kɔ kɔ lomthibul pɔi humɔ nɔɛ vɛ.} So if a person dies, the family will make a unanimous agreement and send for that person.

\TCheadword{Kɔba} \textit{nam} Koba, name given to a person. \textit{Sistha Kɔba ŋaha kaŋa hi mpanthoɛ.} Sister Koba is the one that taught us this work.

\TCheadword{Kɔbɔ} \textit{nam} society spirit who appears as a dancing masquerade wearing a a \textit{kɔbɔ} mat (\citealt{Pichl1967}). 

\TCheadword{kɔbɔ} (Mende \textit{kɔbɔ ‘floor mat'}) \textit{n} largest kind of rough mat for covering large area, e.g., the floor (\citealt{Pichl1967}). \textit{Nrus iwɔm dɛ bɔko ni pe sak kɔbɔ lɛ hã lath pəlɛ lɛ.} Push aside the wood outside and let them spread out the mat to dry the rice (\citealt{Pichl1967}). 

\TCheadword{kɔfe} (coffee) \textit{n} coffee. \textit{Kɔfe lɛ hɔ̃ lo̹l.} The coffee is bitter (\citealt{Pichl1967}).

\TCheadword{kɔfin} (Eng \textit{coffin}) \textit{n} coffin.

\TCheadword{kɔfo} \textit{cf}: \TClink{kɔysu}. \textit{n} \textit{kɔfõ} (wɔ/hã) powerful society, its members can go through walls and cannot be kept in prison, they never lean against a wall for fear of falling through it (\citealt{Pichl1967}). 

\TCheadword{kɔgba} \textit{cf}: \TClink[2]{vee}. \textit{n} pearl oyster.

\TCheadword{kɔjia} \textit{cf}: \TClink{kɔysu}. \textit{n} magician.

\TCheadword{kɔk} \textit{n} \textbf{1)} back. \textit{Kòmɔ̀ɛ̀ bí mbìmbìs wɔ̀n kɔ́k, wɔ̀n njàlàì gbí.} The child has sores on his back, all over his body. \textbf{2)} buttocks. comp. \TClink{thimkɔk} (see \TClink{thim}) 

\TCheadword{Kɔka} \textit{cf}: \TClink{Kakir}, \TClink{Yelsaha} (comp. of \TClink[3]{yel}, \TClink{saha}). \textit{nam} Caulker, name given to a person, name given to a clan. \textit{Alfɔnso Kɔka.} Alphonso Caulker.

\TCheadword{kɔka} \textit{n} shoe. \textit{Dɛn yɛ ibɛ nkɔkaɛ lɛko nyɔn doki ŋɔ pɔ vellɛ balansbɔllɛ.} Then we would put our shoes on the ground (for) this thing (game) they called balance ball.

\TCheadword{kɔkia} \textit{n} (kɔ/ma) plant species (Olyra latifolia) (\citealt{Pichl1967}).

\TCheadword{kɔkɔyɛ} \textit{n} (wɔ/hã, si) bird species, smaller kind of bush fowl (Francolinus spp.) (\citealt{Pichl1967}). 

\TCheadword{Kɔlɛj} (Eng \textit{college}) \textit{nam} College.

\TCheadword{kɔlma} \textit{n} \textbf{1)} [kɔ̀lmà] tree species, lily-type of tree, grows near swamps, can be woven into a fine mat (K dialect); (kɔ/-) kind of grass (\citealt{Pichl1967}). \textbf{2)} mat made from plant of the same name (K dialect, \citealt{Pichl1967}).

\TCheadword{kɔlɔ} \textit{n} cockle.

\TCheadword{kɔluŋ} \textit{n} fruit pit.

\TCheadword{kɔluŋ-vil} (comp. of \TClink[4]{kɔ}, \TClink[1]{vil}, see \TClink[4]{kɔ}) 

\TCheadword{kɔm} \textit{cf}: \TClink{gbogbotok} (unspec. of \TClink[3]{gbogbo}), \TClink{maima}, \TClink{tom}, \TClink[2]{wo}. \textit{n} penis (\citealt{Pichl1967}).

\TCsubword{kɔmpɔth} (unspec.) \textit{n} (wɔ/hã) uncircumcised (a serious offense) (\citealt{Pichl1967}).

\TCheadword[1]{kɔma} (der. of \TClink[2]{kɔ}, \TClink[4]{ma}, see \TClink[2]{kɔ}) 

\TCheadword[2]{kɔma} \textit{v} iron.

\TCheadword{kɔmpɛn} (Eng \textit{company}) \textit{n} company; society of which the members hire themselves out to work on farms or at the erection of houses (\citealt{Pichl1967}); \textit{kɔpe} a work company, used to build houses (\citealt{Hall1938}).

\TCheadword{kɔmpia} (Eng \textit{compare}) \textit{v} compare. \textit{Mɔni hun kɔmpia boŋgo ni kacheɛ.} You should come compare now and the past.

\TCheadword{kɔmpɔth} (unspec. of \TClink{kɔm}) 

\TCheadword{kɔmtha} \textit{n} [kɔ̀mthá] tree species used for caulking, as tar, to seal a split in a canoe, peeled and beaten until it is pliable (K dialect). 

\TCheadword[1]{kɔn} \textit{n} \textbf{1)} [kɔ̀n] innocence test used to find out if person is righteous (K dialect); “sasswood ordeal”: A suspected person is made to drink an infusion of sasswood bark in which the heart of a fowl has been boiled. If he vomits the liquid, it proves his innocence (\citealt{Pichl1967}). \textbf{2)} [kɔ̀n] test to find out if a girl is a virgin (K dialect). 

\TCheadword[2]{kɔn} \textit{n} [kɔ̀n]tree species used for mortars, if straight used for canoes, wood quite heavy (K dialect). 

\TCheadword{kɔna} (Eng \textit{corner}) \textit{cf}: \TClink{sɔku}, \TClink[2]{thuŋk}. \textit{n} corner. \textit{Yɛ nɔ wɔ che ko kɔnaɛ, ya hundɛ wɔi hɔ “He!”} When someone would be in a corner, then I would come and she would say “Hey!”

\TCheadword{kɔnaibol} (id. of, comp. of \TClink[2]{kɔ}, \TClink[1]{nai}, \TClink[1]{bol}, see \TClink[2]{kɔ}) 

\TCheadword{Kɔnakri} \textit{nam} Conakry, name given to a place. 

\TCheadword{kɔndɛm} (Eng \textit{condemn}) \textit{v} condemn. \textit{I theɛn ni yeŋkɛlɛŋ kɛ ibiɛni weɛ ŋɔ ila bɔ kɔndɛm dɛ.} We do not feel good that we do not have a way of condemning it.

\TCheadword{kɔndishɔn} (Eng \textit{condition}) \textit{n} condition.

\TCheadword{Kɔndɔlɔ} \textit{nam} Kondoloh, name given to a place. \textit{Kɔndɔlɔ Sɛkshɔn.} Kondoloh Section.

\TCheadword[1]{kɔnɛ} \textit{cf}: \TClink{kinɔ} (der. of \TClink[1]{ki}, \TClink{nɔ}) \textit{dem} that. \textit{Pimdɛ kɔnɛ kɔ pɔ bia joɛ, pɔ kɔi bɛ stɔ thai kunɛ.} The remainder will be put aside for food, will be kept in stores.

\TCheadword[2]{kɔnɛ} \textit{cf}: \TClink{may}. \textit{v} \textbf{1)} restore. \textit{Kɔnɛ o Bahin.} Restore (unto us) our Father. \textbf{2)} forgive. \textit{Kɔnɛ mai we, ŋa kafa hin yɛ we.} Please forgive us for all that has been damaged. \textbf{3)} please. \textit{Kɔnɛ lɛŋa hun gbo, ŋa koi ndumdɛ ma hiŋka biɛ.} Please when they would come, they should take the character we had.

\TCheadword{Kɔni} \textit{nam} Koni, female name given by a society. 

\TCheadword{kɔni} (der. of \TClink[2]{kɔ}, \TClink{-ni}, see \TClink[2]{kɔ}) 

\TCheadword{kɔnkɔ} \textit{n} crust.

\TCheadword[1]{kɔnth} \textit{cf}: \TClink{kuu}. \textit{n} seizure (theft? robbery?) (\citealt{Pichl1967}). (\citealt{Pichl1967}). \textit{Bia toŋkiɛ jali Kaỹn hã kɔnth.} Bia summoned Kayn for seizure (\citealt{Pichl1967}).

\TCheadword[2]{kɔnth} \textit{v} \textbf{1)} catch. \textit{Tùmɔ̀ɛ̀sɛ̀ thòìnyɛ́ vísɛ̀ nì ŋà kɔ́nth [ə] wɔ́.} The dogs chased the animal and they grabbed it. \textit{Pə kɔnthi chencha Sese wɔ lɔ yɔlko l'ay gbunda la ke Kaay lɛ}. They caught Sese yesterday, he is in chains (because) he raped Kayn's wife (\citealt{Pichl1967}). \textbf{2)} reach. \textit{Roshia ni Amɛrika hã koŋ kɔnth lomthibul le pəm kɔ koŋ.} Russia and America have reached an agreement that war should cease (\citealt{Pichl1967}). 

\TCheadword[1]{kɔŋ} \textit{n} blood. \textit{Jizɔs ŋa ja bom ba ŋa yaŋ, yɛ peyɛ nkɔŋ ma Wɔlɛ.} Jesus has done a big thing for me when He shed his blood. comp. \TClink{kɔŋ-gbɔl} (see \TClink{gbɔl}) 

\TCheadword[2]{kɔŋ} \textit{v} [kɔ́ŋ] bury (B dialect). \textit{I kɔ́ŋ nɔ́éwɛ̀.} We buried the corpse. \textit{Chen bo wu ni pɔ kɔŋ wɔ, pɔ wɔ lemɛk gbal ifɔndɛ.} He would not just die and be buried, they would complete society rites for him (lit. pass the society boundary with him). 

\TCsubword[3]{kɔŋ} (der.) \textit{n} burial. \textit{So, wɛl, ihun ni ko ja kɔŋdɛ, yɛ ayindɛ ŋa wuɛ yɛ pɔ kɔŋdɛ, kache ni kenɛkiɛ.} So, well, let us now come to burial, when people die how they bury them, in the past and nowadays. \textit{Kɔŋdɛ kɔ akekɛ thiwɔllɛ yɛ laiyoɛ hɔ cheni pɛ bul.} The burial that I have seen (with my) eyes, it is not still the same.

\TCheadword{kɔŋ-gbɔl} (comp. of \TClink[1]{kɔŋ}, \TClink{gbɔl}, see \TClink{gbɔl}) 

\TCheadword{kɔŋklɔŋ} \textit{cf}: \TClink{bɔthbɛrɛ}. \textit{n} millipede species (K dialect).

\TCheadword{kɔŋko} \textit{cf}: \TClink{bolo}, \TClink{chocho}, \TClink[2]{kontho}, \TClink{nɔtɔ}, \TClink{suk}, \TClink{thoŋku}. \textit{n} [kɔ́ŋkò], [kɔ̀ŋkóthɛ̀] shell, e.g., of a tortoise or snail (B dialect). \textit{A kache dikil koŋo thi bɛl potho wɛ, ayi bɛ isundɛ.} I used to gather coconut shells, then I would put sand (inside).

\TCsubword{koŋkonya} (comp.) \textit{n} shell of the \textit{nya} turtle (\citealt{Pichl1967}). 

\TCheadword{kɔŋkɔ} \textit{cf}: \TClink{wuk}. \textit{n} rice crust.

\TCheadword{kɔŋkɔkula} \textit{n} monkey species, red colobus (\citealt{Pichl1967}). 

\TCheadword{kɔŋlɔŋ} \textit{n} ant species.

\TCheadword{kɔɔ} [kɔ̀ɔ̀] \textit{n} snake species, big snake, brownish, lives in trees, more than five feet long, not poisonous but people will still kill it (K dialect); \textit{kɔɔ} (wɔ/hã, si) snake species, black snake living in palm trees (\citealt{Pichl1967}). 

\TCheadword{kɔɔt} \textit{n} bird species, pastor bird, has white around neck, crow (K dialect); \textit{kɔt} (wɔ/hã, si) pied crow (Corvus albus), when it sits on the side of the house which looks towards the Poro-bush, a male member of the family will die. If it sits on the side which looks towards the Bondo-bush, a female will die (\citealt{Pichl1967}). 

\TCheadword{kɔp} (Eng \textit{cup}) \textit{n} cup. \textit{Mɔi ya jowɛ, yɛ mɔ ya joɛ ken pɛlɛ kɔp litin, nkoŋ kɔ thɔk yeŋkɛlɛn.} Then you cook the rice, when cooking the rice like two cups, you have to wash it properly.

\TCheadword{kɔpa} (Eng \textit{copper}) \textit{cf}: \TClink{baar}, \TClink{baaryeŋ} (unspec. of \TClink{baar}), \TClink{fe}. \textit{n} copper, money (\citealt{Pichl1967}).

\TCheadword{kɔpra} \textit{v} collect, call in debts. 

\TCheadword{kɔs} \textit{n} (wɔ/hã) fish species, freshwater fish similar to catfish (\citealt{Pichl1967}). 

\TCsubword{kɔsmahwɛ} (comp.) \textit{n} \textit{ŋkɔs ma hwɛ} (ma) cooking late in the night (When crews come home from fishing, they prepare some fish and cassava, and the remnants are left for boys and girls) (\citealt{Pichl1967}). 

\TCheadword{kɔsmahwɛ} (comp. of \TClink{kɔs}, \TClink[4]{ma}, \TClink{hwɛpi} (comp. of \TClink[2]{hu}, \TClink[1]{pi}), see \TClink{kɔs}) 

\TCheadword{kɔsul} \textit{v} be fixed in one's habits.

\TCheadword{kɔt} (Eng \textit{court}) \textit{n} court. \textit{Lɛ nɔ koyɛni gbo ha pɔn bɛmpa la, makɔni kɔtai, lokal kɔt.} If the person does not accept the settlement, they go to the court, the local court.

\TCheadword{kɔt-kɔt} (Krio \textit{kɔt-kɔt} 'cut into small bits') \textit{n} cut tobacco. \textit{Aa, wɔ ŋa yen ton-ton, wɔ wɔŋgul sigrɛt, kɔt-kɔt.} Yes, she does a few things, she sells cigarettes, cut tobacco (for pipes).

\TCheadword{kɔta} (Eng \textit{cutter}) \textit{n} type of ship; cutter.

\TCheadword[1]{kɔth} \textit{n} (hɔ̃/tha) dry bark of a tree (also: kɔth thɔk ɛ) (\citealt{Pichl1967}).

\TCheadword[2]{kɔth} \textit{cf}: \TClink[2]{chaŋ}. \textit{n} molar tooth.

\TCheadword{kɔtin} (Eng \textit{cotton}) \textit{cf}: \TClink{nɔmafuuŋk} (comp. of \TClink{nɔma}). \textit{n} cotton. \textit{Chaŋgbo lɛ abi bo fe, akɔ pin kɔtin, ayi huŋgul.} If I have (any) money at all, I will buy cotton (cloth) to sell. 

\TCheadword{kɔysu} \textit{cf}: \TClink{kɔfo}, \TClink{kɔjia}, \TClink{sɔkɔth}. \textit{n} (kɔ/-) magic; hypnotism (\citealt{Pichl1967}).

\TCsubword{kɔysunɔ} (comp.) \textit{cf}: \TClink{nɔfɔnwɛi} (comp. of \TClink{nɔ}, \TClink[1]{fɔnwɛi}). \textit{n} (wɔ/hã) hypnotist; sorcerer; one who makes people see things which really don't happen (\citealt{Pichl1967}). \textit{Wɔn wɛ kɔysunɔ lɛ chaŋ atɛma wɔ lɛ.} He himself was the greatest sorcerer among his peers (\citealt{Pichl1967}).

\TCheadword{kɔysunɔ} (comp. of \TClink{kɔysu}, \TClink{nɔ}, see \TClink{kɔysu}) 

\TCheadword{kran} \textit{cf}: \TClink{pulukɛ}. \textit{n} pile. \textit{Pɔ koŋ gbo, ŋa koŋ kɔ gbo yɔk ti thai, pɔ kɔ pak bai thikranthikran thibombom.} After taking it to the farmhouses/towns, it would then be piled up into different sections into very big piles.

\TCheadword{Krayst} \textit{nam} Christ. \textit{Oo aŋa mi isi yɛ lɛ kɛ Kraist ka wu ŋa hin.} Oh, my people, let us realize that Christ died for us.

\TCheadword[1]{krikri} \textit{adj} round.

\TCheadword[2]{krikri} (der.) (Men kili-kili) \textit{cf}: \TClink{kilbaŋkaŋ} (comp. of \TClink[1]{kil}). \textit{n} house type, round house.

\TCheadword[2]{krikri} (Men kili-kili) (der. of \TClink[1]{krikri}) 

\TCheadword{Krim} \textit{nam} Krim, name given to a place. \textit{Pə hɔmɔm parɛ lɛ ŋkɔ vɛthiɛ Themdel ko ni Krim ko.} I was told the other day you went to Timdale and to Krim some time ago.

\TCheadword{Krio} \textit{nam} Krio people. \textit{Gbendi abəka lɛ ni nchə ma hã veelɛ Akrio.} The descendants of the freed slaves are called Krios (\citealt{Pichl1967}).

\TCheadword{Krismɛs} \textit{nam} Christmas. \textit{Yɛ pɔ ŋa haŋ tɛm Krismɛsɛ ŋɔi hun.} This is what will happen up until Christmas comes.

\TCheadword{Kristian} \textit{n} \textbf{1)} Christian person. \textit{Mɔm mɔ Kristian?} You, are you Christian? \textit{Aa, ya Kristian.} Yes, I am a Christian. \textbf{2)} Christianity. \textit{Ligbe ba la hun ni ŋɔ pɔ vellɛ, ŋɔi hɔni Mpothoai ɛ rilijɔndɛ la ko hundɛ, Kristiandɛ.} Many things have happened in what we called in English religion, Christianity.

\TCheadword{kritikal} (Eng \textit{critical}) \textit{adj} critical. \textit{Aa, ashila manɛ maŋa chiɛ maa kritikallɛ.} Yes, I know that, the ones they bring to us are critical.

\TCheadword{kronik} \textit{adj} chronic.

\TCheadword{ku} \textit{cf}: \TClink{gbei}, \TClink[1]{hɔ}, \TClink[1]{vel}. \textit{v} \textbf{1)} call. \textit{Kɔŋgbɔl wɔ lɛ kɔ duk yɛ pə wɔ ku ilellɛ.} His heart beats when they call his name (\citealt{Pichl1967}). \textbf{2)} to name, as in a paternity suit (K dialect). \textit{Wɛl, nku mu bul?} Well, name one?

\TCheadword{kuai} \textit{cf}: \TClink{daŋkɔ}. \textit{n} [ǹkùàɛ̀] palm oil, the oil from the fruit itself (K dialect). \textit{Yɛ̀ kóŋ thɔ̀n dɛ̀, wɔ̀è bání kùáɛ́ njáláí.} After bathing she rubbed oil on her skin.

\TCheadword{kuamu} \textit{n} fish species, kamus fish (\citealt{Pichl1967}). 

\TCheadword{kuaŋa} \textit{cf}: \TClink[2]{sɔnth}. \textit{Numb} twenty. \textit{Nɛnthi kuaŋa tiŋ ŋɔ niɛ?} It is now forty years. \textit{Wɛl, ani bɔ che nɛnthi kwaŋa ra ni mɛn.} Well, I am probably 65 years old. \textit{Ŋ kɔm thɔŋhulɔ fe lo hɔ pɔn kuhɔnɔ.} Go keep this money for me, it is twenty pounds (\citealt{Pichl1967}). 

\TCheadword{kuath} \textit{n} fear.

\TCsubword{nɔŋkwath} (comp.) \textit{n} coward.

\TCheadword{kuba} \textit{n} cover, [kùbá]/ [kùbá tɛ̀] cover/ the covers (B dialect). \textit{Yɛ mɔ bɛ pɔmthi gbamdɛ, mɔ kɔi kuba kɔi koŋ vela yeŋkɛlɛŋ lɔn atok.} As you are putting in the potato leaves, then you take the cover when it is going nicely on top.

\TCheadword{kueindau} \textit{n} accompanying present.

\TCheadword{kueni} \textit{v} feel; think oneself. \textit{Kong kueni ŋke̹n bo̹m chaŋ Thua.} Kong thinks himself more important than Thua (\citealt{Pichl1967}). comp. \TClink{kueni-bom} (see \TClink{bom}) 

\TCheadword{kueni-bom} (comp. of \TClink{bom}, \TClink{kueni}, see \TClink{bom}) 

\TCheadword{kuɛ} \textit{v} \textbf{1)} take. \textbf{2)} make (sacrifice). \textbf{3)} take away. \textit{Wɔ ka wɔŋ ni kɛn ŋa koi kafaŋi yai.} He gave himself up to take away our sins. \textit{Pɔ wɔe kue ŋgbekteɛ ŋkɛnt, ni pɔ chereŋ Kaiŋ Taso.} They took the handcuffs off his hands and they freed Kain Tasso. \textbf{4)} spend (time). \textit{Nɛn thi-wɔ tha mɔ ko lɔ kwe ya?} How many years have you spent there? \textbf{5)} elect. \textit{Pɔ̀ kùɛ́ wááŋwɛ̀ɛ̀ bɛ̀ɛ̀.} They elected the daughter chief.

\TCheadword{kuɛe} \textit{v} mean; signify. \textit{Laa kuɛe, lanɛ ntaroa hiɛ ni ntaroa mɔɛ, ntaroa ŋaɛ, ŋa bia hundɛ.} That is what I mean, that is our descendant, your descendant, their descendant that is going to come.

\TCheadword{kufɛ} \textit{n} \textbf{1)} clothes. \textbf{2)} pants.

\TCheadword{kufu} (unspec. of \TClink[4]{kɔ}) 

\TCheadword{kugba} \textit{cf}: \TClink{fɔsa}. \textit{n} \textbf{1)} warrior. \textbf{2)} strength. \textit{Kugba limɔɛ lɔ gbo nɔmaa atok}. Your strength can only best a woman (meant as a taunt (lit. your strength is just above a woman's).

\TCsubword{kugbanɔ} (comp.) \textit{n} warrior.

\TCheadword{kugbanɔ} (comp. of \TClink{kugba}, \TClink{nɔ}, see \TClink{kugba}) 

\TCheadword{kuku} \textit{cf}: \TClink[1]{bai}, \TClink[2]{baŋ}. \textit{n} (hɔ̃/tha) Hut erected for Poro Society assemblies made of palm straw, for sleeping, sometimes with several rooms (\citealt{Pichl1967}).

\TCheadword{kukuu} \textit{n} game. \textit{Siŋthɛ tha pɔ vel kukuu} The game that is called ku-ku.

\TCheadword[1]{kul} \textit{cf}: \TClink[1]{yil}. \textit{v} \textbf{1)} smoke, \textit{kul thafɛ} smoke a pipe (\citealt{Pichl1967}). \textbf{2)} drink. \textit{Hã bɔsɔlin gbɔl lɛ hĩ kul mən dɛ.} To quench our thirst we drink water (\citealt{Pichl1967}). \textit{Ŋa kul mɔi ma sɔisɔi gbi ŋa koi piŋiɛni.} They drink tasty drinks and they turn against us.

\TCsubword{kulmmɛn} (comp.) \textit{n} thirst. \textit{Kul mmən hɔ̃ mi.} I am thirsty (\citealt{Pichl1967}).

\TCheadword[2]{kul} \textit{n} (wɔ/hã, si) fish species, sole (Cynoglossus, Citharichthys stampflii, Siacium micrurum) (\citealt{Pichl1967}). 

\TCsubword{kultapoo} (unspec.) \textit{n} (wɔ/hã, si) fish species, kind of sole with fins on its back (\citealt{Pichl1967}). 

\TCheadword{kulbeŋ} \textit{n} [kúlbéŋ] locust tree, used for food, long pods with yellow seeds inside, some people will boil to soften and even add sugar (K dialect).

\TCheadword{kulmmɛn} (comp. of \TClink[1]{kul}, \TClink[3]{mɛn}, see \TClink[1]{kul}) 

\TCheadword{kultapoo} (unspec. of \TClink[2]{kul}) 

\TCheadword{kuluŋ} \textit{cf}: \TClink{tukum}. \textit{n} goat, [kúlúŋ]/ [kúlúŋsɛ̀] goat/goats (B dialect); (wɔ/hã, si) goat (\citealt{Pichl1967}). \textit{Kúlúnsɛ̀ chɔŋ bànkbúkɛ́ lèn.} The goats love \textit{bànkbúkɛ́}. 

\TCheadword{kum} \textit{n} (wɔ/hã, N) fly species, night fly, sand fly (\citealt{Pichl1967}). 

\TCheadword{Kumba} \textit{nam} Kumba, name given to a person. \textit{Ba mi ilel wɔ ŋɔ Ibrahim Kumba.} My father's name is Ibrahim Kumba.

\TCheadword{kumba} \textit{cf}: \TClink{gbɔntma}, \TClink{kamsa}. \textit{n} \textbf{1)} shirt. \textbf{2)} gown.

\TCheadword{kumbɛ} \textit{cf}: \TClink{keth}. \textit{n} side of chest. \textit{Fɔs mi yaŋ ŋkumbɛ ni kɛnthi gbaŋgba-m dɛ.} He struck me on my side and broke my rib (\citealt{Pichl1967}). \textit{Pɔ baŋ wɔ ko thɔkɛ, pɔ chu wɔ wɔn kumbɛ.} They nailed him on the cross, they stabbed him in his side.

\TCheadword[1]{kump} \textit{cf}: \TClink{hɔima}, \TClink{nɔbɛma} (comp. of \TClink{nɔ}, \TClink{bɛmpa}), \TClink{nɔbonthɔ} (comp. of \TClink{nɔ}), \TClink{nɔhampanth} (comp. of \TClink{nɔ}, \TClink{haa}, \TClink[1]{panth}). \textit{n} helper, worker, labor. \textit{Yi gbɛki kump hã bɔnth hi hã rɔk.} We hire helpers to help us to harvest (\citealt{Pichl1967}). 

\TCheadword[2]{kump} \textit{v} do the final work in plaiting a basket or a net, esp. plaiting the edges (\citealt{Pichl1967}). \textit{Yema si kump sampa chang awante Bue.} Yema knows better than her sister Bue how to finish a basket.

\TCheadword{kumpohani} (comp. of \TClink{kompuŋ}, \TClink[1]{hani}, see \TClink{kompuŋ}) 

\TCheadword{kun} \textit{cf}: \TClink{gbenik} (der. of \TClink[2]{gbemi}, \TClink{-k}), \TClink{gbɛthɛhɔl}, \TClink{tem}. \textit{n} \textbf{1)} belly. \textit{Ŋkɔni ayen gbi ha kɔ lɛliɛ yen joo, ni nsiiɛ ya kun dumɔ.} You do not go anywhere to find things to eat, and you know my belly is hard (i.e. I am about to give birth). \textit{Ye lɛ kulɔ gbo ni mən bɔsul, mɔ bi ipula mɔm kunɛ.} Then if you drink unboiled water, you will have worms in the belly. \textbf{2)} womb. \textit{Tɛnɛni yo pɛnte tɛnɛni, kɛ ya wɔ woth kun mɔi.} Remember, oh brother, remember, that it is mother who carried you in her womb. \textit{Kan sɔnthɔ ta kun bɔnthma.} To hurry to sew a dress for the child in the womb (proverb). Used, e.g., if someone buys a lottery ticket and without knowing whether or not it will win enters into negotiations to buy a house (\citealt{Pichl1967}). \textbf{3)} pregnancy. \textit{Kundɛ ko gbo che mpaŋ mɛntiŋ, mɔ kɔ kul.} If the pregnancy has reached its seventh month, you drink it. \textit{Te yɛ woth kundɛ wɔ mi hun gbemɔ ka.} She got pregnant and was brought to Shenge and gave birth with me (as midwife). comp. \TClink{wothkun} (see \TClink[1]{woth}) 

\TCsubword{kunɛdinthɛ} (comp.) \textit{n} (hɔ̃/tha) white or clean belly (to be found with a clean belly at the post mortem says that the deceased was neither wizard, witch nor cannibal) (\citealt{Pichl1967}).

\TCsubword{kunɔlɔ} (comp.) \textit{cf}: \TClink[1]{mɔ}. \textit{n} \textbf{1)} (hɔ̃/tha) lower part of the belly (\citealt{Pichl1967}). 2) bossom. \textit{Kɔ bimni sɔku bullai, wɔ hɔɔl <fɔɔ fɔɔ fɔɔ> ni yeke wɔɛ che wɔn kunɔlɔ.} (She) went and bent over in one corner, she breathed <fɔɔ fɔɔ fɔɔ> (idph of panting) with the cassava (tucked) in her bosom.

\TCsubword{kunputul} (comp.) \textit{n} [kùnpùtúl] gluttonous (lit. rotten belly) (B dialect).

\TCsubword{kunɛ} (der.) \textit{cf}: \TClink{-ai}, \TClink[4]{hɔl}. \textit{post} \textbf{1)} inside. \textit{Kïlthi lɛ tha Pujɔng kunɛ tha bo̹m.} The houses in Pujehun are big (\citealt{Pichl1967}). \textit{Mɔi bɛ ituɛ kunɛ.} You put it in the pot. \textbf{2)} in. \textit{Pang bul kunɛ yi bi ihɛɛling itïng.} In one month, we have two spring tides (\citealt{Pichl1967}). \textit{Huksi atïŋ hã che kïl lɛ kunɛ.} There are two bush spiders in the house (\citealt{Pichl1967}). \textbf{3)} within. \textit{Ŋa theɛnigbo ndum ŋan bɛ ŋa hun la haŋ cheɛ kunɛ.} They do not quite understand good character, the ones who are coming up (younger people), it should be within. comp. \TClink{kokkunɛ} (see \TClink{kok}), der. \TClink{kunɛ} (see \TClink{kun})

\TCheadword{kunani} \textit{v} report.

\TCheadword{kunɛ} (der. of \TClink{kun}, \TClink[1]{ɛ}, see \TClink{kun})

\TCheadword{kunɛdinthɛ} (comp. of \TClink{kun}, \TClink{dinthɛ} (der. of \TClink{dinth}, \TClink{-ɛ}), see \TClink{kun}) 

\TCheadword{kunɔlɔ} (comp. of \TClink{kun}, \TClink[7]{lɔ}, see \TClink{kun}) 

\TCheadword{kunputul} (comp. of \TClink{kun}, \TClink{puthul} (der. of \TClink[2]{puth}, \TClink{-ul}), see \TClink{kun}) 

\TCheadword[1]{kunth}\textit{n} [kùnth] tree species on whose stem leaves grow, used for fishing, used for drying tenni fish because it does not break (K dialect); (kɔ/ma) seeds used for oil that can be used to treat sores on the legs (\citealt{Pichl1967}). 

\TCheadword[2]{kunth} \textit{n} [kùnth] palm species with fruit that people eat, just suck them (K dialect); \textit{kuth} small palm trees along shore, fruits like small dates, leaves used to make straw hats (\citealt{Pichl1967}).

\TCheadword{kuŋk} \textit{n} \textbf{1)} bush where chiefs are carried and where they stay for some time to undergo grooming. Presumably, one of the ceremonies is cleansing by a vapor bath (\citealt{Pichl1967}). \textbf{2)} vapor bath.

\TCsubword{kuŋkuni} (der.) \textit{v} sweat in vapor bath.

\TCheadword{kuŋkbɛ} \textit{n} submarine (see Sumner 1928).

\TCheadword{kuŋkuni} (der. of \TClink{kuŋk}, \TClink{-ni}, see \TClink{kuŋk}) 

\TCheadword{kuŋkuŋ} \textit{cf}: \TClink[1]{fɔs}. \textit{v} \textbf{1)} knock; rap. \textbf{2)} carry.

\TCheadword{kupɔ} \textit{n} \textbf{1)} eyelid. \textit{kupͻɛ} the eyebrow.

\TCheadword[1]{kus} \textit{n} vomit. \textit{kus} vomit.

\TCheadword[2]{kus} \textit{n} leftovers. \textit{Ya ka ni hani santhɛ, isɔ bul akoŋ thukuli jomi kusɛ ayema kɔ jo…} When I had grown up, one morning after I had just warmed my rice and wanted to eat it… \textit{Yeke kusɛ ŋɔ ka cheni konk.} The cassava leftovers would never end. comp. \TClink{jokus} (see \TClink[2]{jo}), \TClink{mɛnŋkus} (see \TClink[3]{mɛn}), \TClink{yekɛkus} (see \TClink{yekɛ}) 

\TCheadword{kusɔ} \textit{cf}: \TClink{lal}. \textit{n} the three rocks on which pots sit; hearth.

\TCheadword{kut} \textit{n} fish species, hognose (\citealt{Pichl1967}). 

\TCheadword{kuta} \textit{n} a piece of wrap-around cloth worn by women and tied at the waist; lappa. \textit{kùtà, kùtàthɛ̀} wrapper (lappa worn by women), wrappers.


\TCheadword[1]{kutha} \textit{n} fish species, large kuta, barracuda? (\citealt{Pichl1967}). 

\TCheadword[2]{kutha} \textit{cf}: \TClink{bue}, \TClink{gbusa}. \textit{v} dig the soil but only on its surface to plant rice or other seeds (\citealt{Pichl1967}). \textit{Pɔ telɛ wik bul mɔike tindɛ pɔi kutha.} They will wait one or two weeks to plow the land. \textit{Yà kùthá bìllɛ́.} I planted bil (a rice variety). \textit{Pɔ yuk mansaŋha'ɛ nseen si pɔ wɔm bɛ kutha pɛlɛ'ɛ ni nyiki ntilaŋ.} They plant this egusi together with it first, before they plant rice or any other seeds.

\TCheadword{kuthni} \textit{v} be suffocated. \textit{Igbimi lɛ hɔ hã ya koŋ kuthni lɛ ŋgɛyɛn gbo ya bi hã wu.} The smoke had suffocated me, if you had not come quickly, I would have died (\citealt{Pichl1967}). 

\TCheadword{kuu} \textit{cf}: \TClink[1]{kɔnth}, \TClink[3]{yen}. \textit{n} \textbf{1)} property (K dialect). \textbf{2)} \textit{kuu} (hɔ̃/tha) estate of deceased (\citealt{Pichl1967}).

\TCheadword{kuum} \textit{cf}: \TClink[2]{hul}, \TClink{mutmut}. \textit{n} insect species, like a mosquito but smaller, bite similar (K dialect).

\TCheadword{kwin} (Eng \textit{queen}) \textit{cf}: \TClink{bɛmaa} (comp. of \TClink[2]{bɛɛ}, \TClink{maa}). \textit{n} queen. \textit{Lɔkɔ bul kwiin Yenke̹s lɛ kong hun Kyamp ka.} Once the English queen came to Freetown (\citealt{Pichl1967}). 


\end{letter}
\begin{letter}{L}

\TCheadword[1]{la} \textit{cf}: \TClink{handɔ}, \TClink[5]{hɔ}, \TClink{ndɔ}, \TClink[5]{ŋa}. \textit{interrog} \textbf{1)} why. \textit{La tamɔ lɛ ka mintha-gbɔl-a? }Why has the child to endure this? (\citealt{Pichl1967}). \textbf{2)} what. \textit{A chɔŋhɔlen la ha ke chencha.} I like what I saw yesterday. \textit{Mɔm la nka cheni ŋa?} What have you been doing? \textit{Ba mɔ, la ŋaa?} Your father, what does he do?



\TCheadword[2]{la} \textit{cf}: \TClink[5]{che}, \TClink[3]{hɔ}, \TClink[2]{lɛ}, \TClink[4]{ni}, \TClink[3]{ŋa}, \TClink[1]{yɛ}. \textit{subordconn} \textbf{1)} that. \textit{Gbemiɛ ki la mɔɔ ki kunɛ, mɔ mɛmiɛni ŋa lan ŋa mɔm che gbemi ahindɛ fli-fli?} This midwife work that you are in, are you really happy to just be delivering? \textit{Wɔi chi lan gbi la ko dikildikillɛ.} He would bring everything that he has gathered. \textit{Jali bul la ya yema nsi yɛ o.} One thing I want you to know. \textbf{2)} how. \textit{Bikɔs Bolomnɔɛ mɔ gbo ŋa len mɔ ŋa shi la mɔ gbɛkɛ.} Because [for] the Bolom, [if] you are doing something, you should know how to walk with it [deal with it]. \textbf{3)} whether; if. \textit{Aŋa yi mɔ la labo nyema la.} I should ask you if you like that. \textit{La gbo ŋa kɔni ŋɔthɛ, la gbo nbontho ba mɔ ŋa mpanthɛ.} If you went fishing, or if you helped your father in the field. \textbf{4)} what. \textit{I ŋa tɛnɛni we, la ya ko ŋa ha mɔn dɛ.} We have to remember what I have done for you. \textbf{5)} why. \textit{Ha wul lijajɛl wɔɛ la wɔ mamɛ?} Why, with the death of his mother-in-law, why is he laughing? \textit{Bikɔs nɔthiɛ yɛ mɔ ha lendɛ, mɔ ŋa shi ha ja la mɔ la ha kai.} Because, as human beings, if you are making something, you should know why you are making it. comp. \TClink{sila} (see \TClink[2]{si}) 

\TCsubword{lagbo} (comp.) \textit{cf}: \TClink[2]{lɛ}, \TClink{pabondɛ}, \TClink[2]{si}, \TClink[1]{yɛ}. \textit{subordconn} \textbf{1)} if. \textit{So, aa ŋa le yimani ko lagbando labo yema la.} So, I should first ask the consent of this man, if he would want that. \textit{Pɔ wɔ ko lɛli han gbeŋ, lagboɛ nɔɛ vɛ ka cheɛ nɔ charaŋ wɔn kunɛ…} They would examine him the next day, if the person is clean inside… \textbf{2)} whether. \textit{Labo ja Bondoɛ la ko che kath ŋa dikil apimaɛ, la chaŋ kacheɛ…} Whether the Bondo business has become harder to gather children more than in the past…

\TCsubword{lanlabi} (comp.) \textit{coordconn} therefore. \textit{Bikɔs yɛ ha ka yɔkim yɛ, ha wɔ ha Kagbɔrɔ, lanlabi akɔ akɔ che gbemiɛ.} Because when they took me, they said it was for Kagboro, that is why I am, I am delivering (children).

\TCsubword{labi} (der.) \textit{coordconn} \textbf{1)} that is why. \textit{Labi gɔvmɛntɛ ŋɔ wɔɛ nɔ mɔ le telɛ pɛŋ mɔ hɔ mu di Bondo.} That is why the government says we should wait before we initiate Bondo. \textit{So labi ichɔŋ len ŋa hin chemɔ vel.} So that is why we like to call you. \textit{Labi isɔloki alewɔ ka yendɛ hɔ bɛni kunɛ.} That is why this morning I have given her something to put in her stomach. \textbf{2)} it is what. comp. \TClink{lanlabi} (see \TClink[2]{la}), der. \TClink{labiya} (see \TClink[2]{la})

\TCsubword{labiya} (der.), (der. of \TClink{labi}) \textit{cf}: \TClink[4]{lanɛ} (der. of \TClink[2]{la}). \textit{coordconn} \textbf{1)} therefore. \textit{Ni jali tilaŋ gbi labiŋa bɛrɛlɔni.} And all these other things should be added on. \textit{Labila hɔlaiɛ gbi Shenge ka lɔ pɔ dumɔni yɛ lɔi si.} That is why no matter how it is, it is in Shenge we were raised and there we know. \textit{Lanɛ gbi lipika la pɛ biya yema ŋa theli ɛ.} All other things that he would want to speak about. \textbf{2)} that is why. \textit{Labila ikonlɔ shini.} That is why we have gotten used to it. \textit{Nke hin Abolomai, yikiɛ ŋɔ iyema, ilap labila iyemani tiŋ.} You see the Sherbro man, it is our respect that we want; we are shy, that is why we do not want nonsense.

\TCsubword[4]{lanɛ} (der.) \textit{cf}: \TClink{labiya} (der. of \TClink{labi}) \textit{dem} \textbf{1)} so. \textbf{2)} that. \textit{Lanɛ la li kɛlɛŋ ahinŋa ŋan the.} That is very good for people to hear. \textit{Lanɛ laŋ la sɔkba mɔ gbi.} That is the only one that really disturbed you.

\TCheadword[3]{la} \textit{indfpro} \textbf{1)} it. \textit{La yeyɛn bɛ pel lɛ pəloɛ.} It was not long after the egg was broken. \textit{Ŋkɔ la hini!} Go and arrange it! (\citealt{Pichl1967}). \textbf{2)} something. \textit{Nkela bo la li kɛlɛŋ, ala bɔ yema.} If you see it as something good, I will want it. \textbf{3)} \textit{Rel}.

\TCheadword[4]{la} \textit{cf}: \TClink[2]{-i}, \TClink{ɛn}, \TClink[1]{kɛ}, \TClink[1]{o}, \TClink{ɔ}. \textit{coordconn} \textbf{1)} correlative conjunction used with another la. \textit{La gbo ŋa kɔni ŋɔthɛ, la gbo nbontho ba mɔ ŋa mpanthɛ.} if you went fishing, or if you helped your father in the field. \textbf{2)} and.

\TCheadword[1]{laa} \textit{cf}: \TClink{too}. \textit{n} louse (K dialect); (wɔ/hã, i) louse (\citealt{Pichl1967}). \textit{Su bul kɔ chen leiŋ ila}. One finger cannot remove a louse (proverb) (\citealt{TISLL1979}).

\TCheadword[2]{laa} \textit{cf}: \TClink{maa}, \TClink{wante} (der. of \TClink[1]{waŋ}), \TClink{uman}. \textit{n} \textbf{1)} wife. \textbf{2)} woman \textbf{3)} female.

\TCsubword{lagbem} (comp.) \textit{n} nursing mother.

\TCheadword{laathama} \textit{n} (wɔ/hã, N) fish species (Ilisha melanota) (\citealt{Pichl1967}). 

\TCheadword{Labaŋ} \textit{nam} Toma Society spirit who appears as a dancing masquerade (\citealt{Pichl1967}).

\TCheadword{labi} (der. of \TClink[2]{la}, \TClink[2]{bi}, see \TClink[2]{la})

\TCheadword{labiya} (der. of \TClink{labi} (der. of \TClink[2]{la}, \TClink[2]{bi}), see \TClink[2]{la}) 

\TCheadword{lagbem} (comp. of \TClink[2]{laa}, \TClink{gbem}, see \TClink[2]{laa}) 

\TCheadword{lagbo} (comp. of \TClink[2]{la}, \TClink[1]{gbo}, see \TClink[2]{la}) 

\TCheadword{lagbowɛ} \textit{cf}: \TClink{wɔso}, \TClink{yipio}. \textit{disco} goodbye. \textit{Kɛ, apa, lagbowɛwe.} Well, pa, goodbye.

\TCheadword{lai} \textit{coordconn} \textbf{1)} it is. \textit{Yɛ lai.} That is it. \textit{Ŋɔ lai ni ko lɔ pɔ gbem mɔa?} How is it about where you are born? \textbf{2)} what it is. \textit{Ŋa hun, ŋa yɛ nsimiyɛ mpanth ma mɔɛ ŋa hun chal ka ŋa theɛ lanɛ lai thelio ɛ.} To come, to spoil your work, to come and sit hear what I was saying.

\TCheadword{laio} \textit{temp} now; as it is. \textit{Yɛlaio wɛ, yɛ mgbɛ ma dukɛ, yɛ iche sɔthɔ ja yenchɛkɛ.} As it is, when the fog falls, we do not have fish. \textit{Yɛ laio ɛ, apim ŋa bi ndum, apim ŋa biɛni ndum.} As it is, some have good character, some do not have good character. \textit{Kɛ yɛ laiyoɛ tamɔ ta kani nɔ santh limani.} But as it is a young boy does not give adults respect. \textit{Yɛ laioɛ ache kɔ skul.} As it is now, I do not go to school.

\TCheadword{laka} \textit{n} Poro messenger.

\TCheadword{lakapɛm} (unspec. of \TClink{pɛm}) 

\TCheadword{lal} \textit{cf}: \TClink{jɛm}, \TClink{kusɔ}. \textit{n} \textbf{1)} fire. \textit{Pɔ koŋ gbo tu kɔ dinth yeŋkɛlɛŋ, pɔi chi ituɛ pɔi bɛ lalako.} After pounding the rice and clean it properly, they then bring the pot and put it on the fire. \textbf{2)} hearth.

\TCheadword{lala} \textit{n} (kɔ/ma) oar, paddle, fin of fish (\citealt{Pichl1967}). \textit{Lalaɛ kɔ wɔe hɛthni mmɛn nyamban doai ni kɔ kɔni hiŋk wɔn.} His paddle slipped from him, the water carried it away from him.

\TCsubword{lala-maŋke} (comp.) \textit{n} (kɔ/ma) boat oar (\citealt{Pichl1967}). comp. \TClink{wɔmmaŋko} (see \TClink[2]{wɔm}) 

\TCsubword{lal-patil} (comp.) (Eng \textit{paddle}) \textit{n} \textit{lal patïl} (kɔ/ma) paddle (\citealt{Pichl1967}). 

\TCheadword{lalbo} \textit{n} (wɔ/hã, N) fish species, black Billy (\citealt{Pichl1967}). 

\TCsubword{lalbo-nswe} (comp.) \textit{n} fish species, soapy black Billy (\citealt{Pichl1967}). 

\TCheadword{lalbo-nswe} (comp. of \TClink{lalbo}, \TClink{swei}, see \TClink{lalbo}) 

\TCheadword{lalpatil} (comp. of \TClink{lala}) 

\TCheadword[1]{lam} \textit{cf}: \TClink{lɛmɛ}. \textit{n} fish species, flat mullet (\citealt{Pichl1967}). 

\TCheadword[2]{lam} \textit{v} stick somewhere, be fastened. 

\TCheadword[3]{lam} \textit{n} pumpkin.

\TCheadword{lamp} (Eng \textit{lamp}) \textit{n} lamp. \textit{Lamp ɛ kɔ man.} The lamp is burning. \textit{Lamp ɛ kɔ mithil.} The lamp
shines.

\TCheadword[1]{lan} \textit{cf}: \TClink[1]{ki}, \TClink[3]{tho}, \TClink{wɔnɛ}. \textit{dem} \textbf{1)} this. \textit{Lanɔ ya gbemiɛ fli.} That is why I am really delivering. \textit{So lan la ako ha nkuath ha ŋɔth.} So that is how I became afraid of fishing. \textit{So ŋan fama lɛki bo laŋa kache kunɛ?} So it was just this farming that you were engaged in? \textbf{2)} that. \textbf{3)} these; those. \textit{Yɛ wɔ koŋ thɔkɛ pagbondɛ chiɛ nyekma lan ni sɛmiyɛ ma kilɛ ko, a.} When she has washed (the corpse), if (she) brought those things and set them inside the house, I (was afraid). \textit{So labi ha ŋa lemɔ yi labo nyema la ŋa yaŋ yimɔ yi thilan.} So that is why I should first ask you if you would agree, for me to ask you these questions. comp. \TClink{lanlabi} (see \TClink[2]{la}) 

\TCsubword{landɛ} (der.) \textit{dem} \textbf{1)} that. \textit{Komɔ landɛ ko bɛ hani gbako, wɔ tika.} That child is now grown, she is in this town. \textit{Tɛm landɛ vɛ che ndum mai.} That time it is not good character. \textbf{2)} those. \textit{Nsi sin thilandɛ?} Do you know those games? \textit{Wɛl anya landɛ ŋa chaŋchaŋ cheɛ.} Well, those are the people that are more in number.

\TCsubword[3]{lanɛ} (der.) \textit{coordconn} it is that (\citealt{Sumner1921}). \textit{Dispɛnsa che ŋa ni, be nɔs che ŋa ni lanɛ bɛiyɛ wɔka che ŋa ka cheɛ dɔkta.} There was no dispenser and no nurse, but at the time the paramount chief was here, there was a doctor. comp. \TClink{lanɛ-gbi} (see \TClink[1]{lan})

\TCsubword{lanɛ-gbi} (der.), (comp. of \TClink[3]{lanɛ}) \textit{coordconn} in spite of it all. Kɛ lanɛ-bi hã yang la nwu. But in spite of it all, it is (a fact that) you died for me (\citealt{Pichl1967}). 

\TCheadword[2]{lan} (Eng \textit{learn}) \textit{cf}: \TClink{kand}, \TClink{karaŋ}. \textit{v} learn; teach. \textit{Wɛl ŋa ma wɔ, bikɔz ama ha lan.} Well they speak it, because I am teaching them.

\TCheadword{landɛ} (der. of \TClink[1]{lan}) 

\TCheadword[1]{lanɛ} (der. of \TClink[2]{lanɛ}) 

\TCheadword[2]{lanɛ} \textit{n} trust. \textit{Pe rɛnthɛ, Laɔn ɔf Juda ko mɔ ko lɔ ibɛ lanɛ iyɛ oo.} Rock of ages, Lion of Judah, in you we put our trust.

\TCsubword[1]{lanɛ} (der.) \textit{v} \textbf{1)} believe. \textit{Iŋɔɛ ni pɛth ŋɔ lɔ ayen kɛ lanɛ gbo hi lɔ ke.} Sweet, sweet things are in heaven and if we believe we will see them. \textbf{2)} trust.

\TCheadword[3]{lanɛ} (der. of \TClink[1]{lan}, \TClink[2]{ɛ}, see \TClink[1]{lan})

\TCheadword[4]{lanɛ} (der. of \TClink[2]{la}) 

\TCheadword{lanɛ-gbi} (comp. of \TClink[3]{lanɛ} (der. of \TClink[1]{lan}, \TClink[2]{ɛ}), \TClink[3]{gbi}, see \TClink[1]{lan}) 

\TCheadword{lanɛki} (der. of \TClink{len}, \TClink[1]{ki}, see \TClink[1]{ki}) 

\TCheadword{lanlabi} (comp. of \TClink[1]{lan}, \TClink{labi} (der. of \TClink[2]{la}, \TClink[2]{bi}), see \TClink[2]{la}) 

\TCheadword{lanɔki} \textit{dem} this.

\TCheadword{lanth} \textit{v} hang. \textit{Wɛl ayema ilanth lɔ we?} Well I want us to hang it up there, okay?

\TCsubword{Lanthgbɔl} (id.) \textit{nam} Thursday. \textit{La hun haani huɛ bul, huɛ Lanthgbɔllɛ, ŋɔ pɔ vel lɛ Thɔsdeɛ Mpothoaiɛ…} It happened one day, the day of Lanthgbol, which white people call Thursday…

\TCsubword[1]{lanthgbɔl} (id.) \textit{v} \textbf{1)} be worried. \textbf{2)} consult.

\TCsubword[2]{lanthgbɔl} (id.) \textit{cf}: \TClink{ŋgbelŋgbel}. \textit{n} anxiety.

\TCsubword{Lanthpol} (id.) \textit{nam} Saturday. 

\TCsubword{lɛnthɛkni} (unspec.) \textit{v} cling; hang on. \textit{Wɔ lɛnthɛkni lɛɛ thɔk lɛ.} He clung to the branch of a tree (\citealt{Pichl1967}). 

\TCheadword{Lanthgbɔl} (id. of \TClink{lanth}, \TClink{gbɔl}, see \TClink{lanth}) 

\TCheadword[1]{lanthgbɔl} (id. of \TClink{lanth}, \TClink{gbɔl}, see \TClink{lanth}) 

\TCheadword[2]{lanthgbɔl} (id. of \TClink{lanth}, \TClink{gbɔl}, see \TClink{lanth}) 

\TCheadword{Lanthpol} (id. of \TClink{lanth}) 

\TCheadword{laŋ} \textit{n} \textbf{1)} bridge. \textbf{2)} matchmaker.

\TCheadword{laŋgban} (Themne \textit{ɔlaŋgba} ‘man') \textit{n} man. \textit{Nande ako vel laŋgba bul wɔ pɔ gbem Themdɛl ko.} Today I have called on a man who was born in Timdale (Chiefdom). comp. \TClink{taalaŋgbaŋ} (see \TClink{taa}), \TClink{tamɔlaŋgbai} (see \TClink{taa}) 

\TCheadword{Laŋgo} \textit{nam} Lango, name given to a person. \textit{Ka cheɛ Yema Laŋgo.} She was Yema Lango.

\TCheadword{laŋgwaj} (Eng \textit{language}) \textit{n} language. \textit{A koŋ thekindɛ kɛ language ɛ lɔ pɔ gbem hi pɔko ɛ, kɔ koŋ tuk, tuk kɔ hɔŋ tukɛ.} I have felt that our language that we are born into, is getting lost, it is getting lost.

\TCheadword{laŋthibul} (unspec.) \textit{n} kind of game.

\TCheadword{laɔn} (Eng \textit{lion}) \textit{cf}: \TClink{sɔnda}. \textit{n} lion. \textit{Pe rɛnthɛ, Laɔn ɔf Juda.} Rock of ages, Lion of Judah. \textit{Pe rɛnthɛ, Laɔn ɔf Juda ko mɔ ko lɔ ibɛ lanɛ iyɛ oo.} Rock of ages, Lion of Judah, in you we put our trust.

\TCheadword[1]{lap} \textit{v} \textbf{1)} be ashamed. \textit{A koŋ lap.} I am ashamed. \textit{Yɛ chen laapɛ…} Since you are not ashamed… \textbf{2)} be shy. \textit{Nke hin Abolomai, yikiɛ ŋɔ iyema, ilap labila iyemani tiŋ.} You see the Sherbro man, it is our respect that we want; we are shy, that is why we do not want nonsense.

\TCsubword{lepi} (der.) \textit{cf}: \TClink{piylɛ}. \textit{v} disgrace.\textit{ Nɔsanth wɔ ki, m ma wɔ lepi.} He is an elder, do not disgrace him.

\TCsubword{lɛpi} (der.) \textit{v} make ashamed. \textit{Mma puthuli komo lɛ wɔ ma choŋ lɛɛpi.} Don‘t spoil the child; it will make you ashamed in the future (\citealt{Pichl1967}). 

\TCheadword[2]{lap} \textit{n} \textbf{1)} shame. \textit{Mbiɛn ndap.} You have no shame. \textbf{2)} shyness.

\TCheadword{lapan} \textit{v} pity. \textit{Kei ndapani we, ha kafa iyɛ we.} See us (and) pity us for our wickedness. comp. \TClink{yendapani} (see \TClink[1]{yen}) 

\TCheadword[1]{lath} \textit{cf}: \TClink{rɛthi} (der. of \TClink{rɛth}, \TClink[1]{-i}), \TClink[1]{sak}. \textit{v} spread out to dry. \textit{Ŋkɔ lath kotha-thi lɛ honka lɛ ay.} Go spread the clothes outside (\citealt{Pichl1967}). 

\TCsubword{lathni-nser} (comp.), (der.) \textit{v} lie flat on the back.

\TCheadword[2]{lath} \textit{n} spittle, expectorant, saliva. \textit{Pɔ thu wɔ ilathɛ,pɔ bɛ wɔ vɛthɛ bol.} People spat on him, and they put thorns on his head. comp. \TClink{thuilath} (see \TClink{thu}) 



\TCheadword{lathaŋ} \textit{n} (hɔ̃/tha) thigh (\citealt{Pichl1967}). 

\TCheadword{lathni-nser} (comp. of, der. of \TClink[1]{lath}, \TClink{-ni}, see \TClink[1]{lath}) 

\TCheadword{Lawana} \textit{nam} Lawana, name given to a place. \textit{Lawana ko lɔ pɔ gbem wɔ.} She was born in Lawana.

\TCheadword[1]{le} \textit{n} (kɔ/ma) star (\citealt{Pichl1967}).

\TCsubword{lemmɛnnhyɛl} (comp.) \textit{n} \textit{le mmən nhyɛl} (wɔ/hã) starfish (\citealt{Pichl1967}).

\TCsubword{lensakahɔl} (comp.) \textit{n} \textit{le nsakahɔl} morning star (\citealt{Pichl1967}).

\TCheadword[2]{le} \textit{v} \textbf{1)} remain. \textit{Tondɛ kɔ lɛ ituɛ kunɛ, mɔ kɔi kɔ thɔŋgul ŋa paŋdɛ.} The small bit that remains in the pot, you reserve it for the evening. \textbf{2)} stay. \textit{Ba Na lee mathini.} Mr. Spider stayed behind to hide himself (\citealt{Pichl1967}). \textbf{3)} continue. \textit{Amaaɛ ŋa lee theli lanɔ ki ŋan thiyeŋ.} The women continue talking about it among themselves. \textbf{4)} leave. \textbf{5)} leave up to. \textit{Lai le ko wɔ kolɛ.} The matter was for him to deal with.

\TCsubword{leni} (der.) \textit{v} remain. \textit{Ŋa kaŋ Mbolomdɛ, lɔ leni gboɛ wɔi sɔthɔ atak wɔi hu.} They were learning Bolom, they were there when he was attacked and died.

\TCheadword[3]{le} \textit{Aux} auxiliary verb forming anterior tense; ‘first.' \textit{Kɛ a le yiɛ nɔmaɛ ki pamdɛ chɔŋ la len.} But I would first ask this woman if she approves (likes it). \textit{Pɔ le raa pɔnthɛ.} They first brush the swamp. \textit{So labi ha ŋa lemɔ yi labo nyema la ŋa yaŋ yimɔ yi thilan.} So that is why I should first ask you if you would agree, for me to ask you these questions.

\TCheadword[1]{lee} \textit{n} [lèè] tree species with fruit that is very sweet but has thorns on its inner seed, can be boiled if it gets too mature, add sugar (K dialect); plum, hog plum (Spondias mombin) (\citealt{Pichl1967}). 

\TCsubword{leemin} (comp.) \textit{n} [lèèmín] fruit with a very nice scent and sweet taste (K dialect).

\TCheadword[2]{lee} \textit{v} drown.

\TCheadword{leemin} (comp. of \TClink[1]{lee}, \TClink[3]{min}, see \TClink[1]{lee}) 

\TCheadword{leiŋ} \textit{v} remove. \textit{Su bul kɔ chen leiŋ ila}. One finger cannot remove a louse (proverb) (\citealt{TISLL1979}).

\TCheadword{lek} \textit{n} \textit{le̹k} (hɔ̃/tha) horn of an animal or musical instrument (\citealt{Pichl1967}). 

\TCheadword[1]{lel} \textit{n} name. \textit{Ntoŋgi mi ilel maŋaɛ.} Show me their names. \textit{Ilel wɔ lɛ hɔ ka che Sese.} His name was Sese (\citealt{Pichl1967}). 

\TCheadword[2]{lel} \textit{Loc} across. \textit{Ka lɔ pɔ bɛ bia huŋa sakaɛ, lel ko, ŋgasumana ko.} It is here that they would have to come and do his sacrifice at Mokainsumana.

\TCsubword{lelka} (comp.) \textit{Loc} this side. \textit{Wɔ hunɛ lelka.} He came to this side (\citealt{Pichl1967}). 

\TCsubword{lelko} (comp.) \textit{n} other side. \textit{A kɔ lelko.} I went to that (other) side (\citealt{Pichl1967}).

\TCheadword[3]{lel} \textit{v} become caught in a net.

\TCheadword[4]{lel} \textit{v} rain. \textit{A lomani yɛ Ba Ŋgobɛ ka che hun dɛ hwɛ lɛ hɔ lelɛ.} I remember when Mr. Ngobe was coming that it rained (\citealt{Pichl1967}).

\TCheadword[5]{lel} \textit{cf}: \TClink{halthe}, \TClink[2]{hɛlɛ}, \TClink[3]{mɛn}. \textit{n} \textbf{1)} sea. \textit{Braima wɔe pɛrni ha che kɔ duki mpɛl lo ki ndelma wɔ.} Braima then practiced to go to leave the nets at sea. \textbf{2)} ocean.

\TCsubword{lelbom} (comp.) \textit{n} ocean.

\TCheadword{lelbom} (comp. of \TClink[5]{lel}, \TClink{bom}, see \TClink[5]{lel}) 

\TCheadword{lele} \textit{n} (wɔ/hã, si) jellyfish (\citealt{Pichl1967}). 

\TCsubword{lelegboŋ} (comp.) \textit{n} jellyfish species, greenish kind of jelly fish, the touch of its stinging capsules can be very painful (\citealt{Pichl1967}). 

\TCheadword{lelegboŋ} (comp. of \TClink{lele}) 

\TCheadword{lelena} \textit{cf}: \TClink{kaŋaloma}. \textit{n} praying mantis.

\TCheadword{lelka} (comp. of \TClink[2]{lel}, \TClink[2]{ka}, see \TClink[2]{lel}) 

\TCheadword{lelko} (comp. of \TClink[2]{lel}, \TClink[1]{ko}, see \TClink[2]{lel}) 

\TCheadword[1]{lem} \textit{cf}: \TClink[2]{chɛli}, \TClink{fothi}, \TClink{gbemani}, \TClink[1]{hɔ}, \TClink{theli}, \TClink{wɛ}, \TClink[2]{wɔni} (der. of \TClink[1]{hɔ}, \TClink{-ni}). \textit{v} \textbf{1)} follow. \textbf{2)} go. \textit{Aa, a lɔ lem kɔ.} Yes, I do go there. \textbf{3)} talk (about). \textit{Yaŋ ayɛn ya ke taamɔ ki wɔ ya lem hali wɔɛ.} I myself saw this little boy whom I am talking about. \textbf{4)} tell. \textit{Aaa che la bɔ lem lan gbi.} I cannot tell it all. \textit{Ŋha ya lemɛ nɔ len la haani, rɔŋ ayén Planti ko.} Let me tell you something that happened, a true story at Plantain (Island). 

\TCsubword[2]{lem} (der.) \textit{n} \textbf{1)} follower, disciple, apostle. \textit{Nha hun hã cheɛ alema wɔ lɛ.} They came to be his followers (\citealt{Pichl1967}). \textit{Oo i mbo sɛli we, ŋa alema iyɛ.} Oh, we are praying for our disciples. \textbf{2)} explanation. comp. \TClink{lemnɔ} (see \TClink[1]{lem})

\TCsubword{lemɛk} (der.) \textit{v} pass with. \textit{Chen bo wu ni pɔ kɔŋ wɔ, pɔ wɔ lemɛk gbal ifɔndɛ.} He would not just die and be buried; they would complete society rites for him (lit. pass the society boundary with him).

\TCsubword{lemil} (der.) \textit{v} follow. \textit{Thetha miyɛ ka che kɔ chɛkaiɛ ha wɔ lemil, ayi kɔ chi iwɔmdɛ.} When my grandmother used to go to the farm I used to follow her, then I would get fire wood.

\TCsubword{lemnɔ} (der.), (comp. of \TClink[2]{lem}) \textit{n} follower, disciple, apostle (\citealt{Pichl1967}).

\TCheadword[2]{lem} (der. of \TClink[1]{lem})

\TCheadword{lembe} \textit{cf}: \TClink{dembe}, \TClink{gbogbɔth}, \TClink{rokos}. \textit{n} orange. \textit{Lembe lo kɔ təng chaŋ kɔnɛ chencha.} This orange is sweeter than that of yesterday (\citealt{Pichl1967}). 

\TCheadword{lemɛ} \textit{v} explain. \textit{Yaŋ Jalikatu B Kumba ahun yi nɔmaɛ ki ni lemɛ mi jaliwɔ atokɛ.} I, Jalikatu B Kumba, am about to ask this woman about herself. \textit{A hun yi lamŋan dɛ ki ni lemɛmi jaliwɔ atokɛ, lenolen la wɔ si ŋa wɔndɛ.} I have come to ask this man to discuss himself, everything that he knows about himself.

\TCheadword{lemɛk} (der. of \TClink[1]{lem}, \TClink{-k}, see \TClink[1]{lem}) 

\TCheadword{lemil} (der. of \TClink[1]{lem}, \TClink{-il}, see \TClink[1]{lem}) 

\TCheadword{lemmɛnnhyɛl} (comp. of \TClink[1]{le}, \TClink[2]{mɛn}, \TClink[2]{hɛlɛ}, see \TClink[1]{le}) 

\TCheadword{lemnɔ} (comp. of \TClink[2]{lem} (der. of \TClink[1]{lem}), \TClink{nɔ}, see \TClink[1]{lem}) 

\TCheadword{len} \textit{indfpro} \textbf{1)} thing. \textit{Anyaɛ gbi bai ko ŋae hɔɛ, “Anya mi, yɛ len la ki-a?”} All the people in the bari say, “My people, what thing is this?” \textbf{2)} something. \textit{Bikɔs nɔthiɛ yɛ mɔ ha lendɛ, mɔ ŋa shi ha ja la mɔ la ha kai.} Because, as human beings, if you are making something, you should know why you are making it. id. \TClink{chɔŋ … len} (see \TClink[1]{chɔŋ}) 

\TCsubword{lenolen} (der.) \textit{indfpro} \textbf{1)} everything. \textit{So yɛ pɔ lɔik wanda maɛ, pɔ wɔi ko kaŋ len-o-len.} So when they initiate a girl, they teach her everything. \textit{A hun yi lamŋan dɛ ki ni lemɛmi jaliwɔ atokɛ, lenolen la wɔ si ŋa wɔndɛ.} I have come to ask this man to discuss himself, everything that he knows about himself. \textbf{2)} anything.

\TCheadword{leni} (der. of \TClink[2]{le}, \TClink{-ni}, see \TClink[2]{le}) 

\TCheadword{lenolen} (der. of \TClink{len}, \TClink{-o-}, see \TClink{len}) 

\TCheadword{lensakahɔl} (comp. of \TClink[1]{le}, \TClink{sakahɔl} (comp. of, id. of \TClink[1]{saaka}, \TClink[1]{ahɔl}), see \TClink[1]{le}) 

\TCheadword{leŋ} \textit{adv} \textbf{1)} openly. \textit{Pɔ tɔm feɛ pɔ ŋɔ dikil mɛsa bom dɛ atok leiŋ.} They are counting the money, gathering it openly on the big table. \textbf{2)} publicly.

\TCheadword{lepi} (der. of \TClink[1]{lap}, \TClink[1]{-i}, see \TClink[1]{lap}) 

\TCheadword{leynɔ} \textit{v} depart. \textit{Yipioo, ya ko la yendapani hã leynɔ, kə peeki cheni.} Goodbye! I consider it a pity to depart from you but it cannot be helped (\citealt{Pichl1967}). 

\TCheadword[1]{lɛ} \textit{cf}: \TClink[4]{koŋ}. branch (of a tree). \textit{Nha yenkəlɛŋ thɔk lɛ tok ɛ, mma pakali lɛɛ thɔk lɛ thɔm mɔ lɛ ma ki duk.} Be careful you there up in the tree, don't make the tree branch shake lest your companion fall (\citealt{Pichl1967}). 

\TCheadword[2]{lɛ} \textit{cf}: \TClink[5]{che}, \TClink[3]{hɔ}, \TClink{lagbo} (comp. of \TClink[2]{la}, \TClink[1]{gbo}), \TClink[2]{la}, \TClink[4]{ni}, \TClink[3]{ŋa}, \TClink{pabondɛ}, \TClink[2]{si}, \TClink[1]{yɛ}. \textit{subordconn} \textbf{1)} if. \textit{Lɛ nwɔ gbo, ŋa mɔi, ŋan ŋa wɔ “bua.”} If you say to them, \textit{mɔi} (‘Good afternoon' in Sherbro), they will say, \textit{bua} (‘Greetings' in Mende). \textit{Chaŋgbo lɛ abi bo fe, akɔ pin kɔtin, ayi huŋgul.} If I have (any) money at all, I will buy cotton (cloth). \textbf{2)} that. \textit{Oo aŋa mi isi yɛ lɛ kɛ Kraist ka wu ŋa hin.} Oh, my people, let us realize that Christ died for us. \textit{Siŋthɛ vɛ tha nlonigbo ntenɛ lɛ nkache siŋ?} Those are the only games you remembered that you used to play? \textbf{3)} when. \textit{Lɛ̀ yɔ̀kthà (hɔ̀) sɛ̀kìlɛ́ gbó, yí thɛ̀ɛ̀ ìchɛ̀kɛ́.} Once the newly cut brush is completely dry, we burn the farm. \textbf{4)} after. \textit{Lɛ apimaɛ ŋa siŋɛ-siŋɛ gbo haŋ lɛ ŋa wɔ bo ŋa yema jo…} After the children played around, if they say they want to eat… der. \TClink[1]{chelɛ} (see \TClink[5]{che}) 

\TCheadword[3]{lɛ} \textit{v} look; behold. \textit{Pɛnthe mi nlə-m lanɔ la bɔnthə-m dɛ.} Brother, look at what has happened to me (lit. what met me) (\citealt{Pichl1967}). \textit{Ndɛ, ndɛ-m ya sɛmɛ kïl lɛ ahɔl!} Behold me standing at the door! (\citealt{Pichl1967}). comp. \TClink{nɔlɛli} (see \TClink{nɔ})

\TCsubword[1]{lɛli} (comp.) \textit{cf}: \TClink[1]{boni} (der. of \TClink[1]{bo}, \TClink{-ni}), \TClink[2]{gbeŋgben}, \TClink{gbɛlɛŋ}, \TClink{keni} (der. of \TClink{ke}, \TClink{-ni}), \TClink{nɔɔmi}, \TClink[1]{thunɔ}. \textit{v} \textbf{1)} look. \textit{Nlɛli bɔŋ dɛ, nchal!} Look at the chair, sit down! (\citealt{Pichl1967}). \textit{Abiɛ lɔni nɔndo ŋɔ pɔ lɛli kunthɛ.} I do not have that thing they use to look inside pregnant women. Nlɛli mpanth lɛ ma wɔ hã lɛ! Look at the work he does! (\citealt{Pichl1967}). \textbf{2)} regard. \textbf{3)} appear. \textit{Nlɛli ye wɔ keni!} See how he looks! i.e., what a stupid-looking face he has (\citealt{Pichl1967}). \textbf{4)} search for. \textit{Nɔma lɛ wɔ hun chencha ka lɛ koŋ tuk, pə wɔ lɛliɛ.} The woman that came yesterday is lost, they are searching for her (\citealt{Pichl1967}). \textbf{5)} find. \textit{Yema kɔ gboth awante l'ay chena lɛ lɛliɛ yen koŋ wusi gboth l'ay lɔn gbi nyək lɛ ma gbo seyɛni hinth l'atok.} Yema went into her sister's box to find that the box had been ransacked and all the things were scattered about on the bed (\citealt{Pichl1967}). \textit{Wɔe kɔni pɔk livil poŋ ha kɔ lɛliɛ waaŋmaa.} He went far away to find a woman. \textbf{6)} look after; care for. \textit{Rɔŋ fili wɔ mi lɛli atok.} Yes indeed he really, really cares for me. \textbf{7)} examine. \textit{Wɛl tɛm dɛ vɛ yɛ pɔ kɔ hun lɛli labo kɔ ko mɔi futhɛ.} At that time they will come to see if it has formed roots. \textit{Pɔ wɔ ko lɛli han gbeŋ, lagboɛ nɔɛ vɛ ka cheɛ nɔ charaŋ wɔn kunɛ…} They would examine him the next day, if the person is clean inside… comp. \TClink{nɔlɛli} (see \TClink{nɔ}), der. \TClink[2]{lɛli} (see \TClink[3]{lɛ}), \TClink[1]{ndɛ} (see \TClink[3]{lɛ}), unspec. \TClink{lɛliya} (see \TClink[3]{lɛ})

\TCsubword[2]{lɛli} (comp.), (der. of \TClink[1]{lɛli}) \textit{n} \textbf{1)} post-mortem exam. \textbf{2)} spy. \textit{Kɛ lɛliɛ kɔ mɛkni ko wɔko.} But examiners (spies?) would stop at his place.

\TCsubword[1]{ndɛ} (comp.), (der. of \TClink[1]{lɛli}) \textit{interj} behold! \textit{Ndɛm ya sɛmɛ kïl lɛ ahɔl!} Behold me standing at the door! (the imperative of lɛli look) (\citealt{Pichl1967}). 

\TCsubword{lɛliya} (comp.), (unspec. of \TClink[1]{lɛli}) \textit{v} look for. \textit{A lɛ́líyá Bɔ̀ì, à yíyɛ́/yíɛ́ Bɔ̀ì.} I am looking for Boi, I'm asking for Boi.

\TCheadword[4]{lɛ} \textit{prep} about.

\TCheadword{lɛbɛ} (comp. of \TClink{lɛɛ}, \TClink[2]{bɛ}, see \TClink{lɛɛ}) 

\TCheadword{lɛɛ} \textit{n} \textbf{1)} ground. \textit{Pùlùkɛ́ bə̀mə̀kɛ́ lɛ́llɛ̀.} Grass covered the ground. \textbf{2)} floor. \textbf{3)} land. \textbf{4)} soil.

\TCsubword{lɛbɛ} (comp.) \textit{Loc} low; near the ground. \textit{Nlanth kɔ lɛbɛ.} Hang it near the ground (not high). \textit{Nlanth kɔ lɛlɛ.} Hang it low.

\TCheadword{lɛfa} \textit{n} fan for fire.

\TCheadword{lɛiu} \textit{v} \textbf{1)} accost. \textbf{2)} compliment.

\TCheadword{lɛka} \textit{n} charm.

\TCheadword{lɛkɛ} \textit{n} \textbf{1)} rice flour gruel. \textbf{2)} gari.

\TCheadword{lɛkɛlɛkɛ} \textit{cf}: \TClink[1]{mɛk}. \textit{disco} signals the completion of a story, finality. \textit{La boɛ lɛkɛ-lɛkɛ mgbut.} That's it, the story ends.

\TCheadword{lɛki} \textit{n} fly species, size of an ordinary housefly but bites painfully and often (\citealt{Pichl1967}). 

\TCheadword{lɛlɛ} \textit{cf}: \TClink{hwai}. \textit{temp} slowly.

\TCheadword{lɛlɛn} \textit{n} [lɛ̀lɛ̀n] tree species, small tree that grows very slowly, used formerly for night fishing, can light it and fish will come close to be scooped up, very scarce in bush (K dialect). 

\TCheadword[1]{lɛli} (comp. of \TClink[3]{lɛ}) 

\TCheadword[2]{lɛli} (der. of \TClink[1]{lɛli} (comp. of \TClink[3]{lɛ}), see \TClink[3]{lɛ}) 

\TCheadword{lɛliya} (unspec. of \TClink[1]{lɛli} (comp. of \TClink[3]{lɛ}), see \TClink[3]{lɛ}) 

\TCheadword[1]{lɛm} \textit{adj} thin.

\TCheadword[2]{lɛm} \textit{n} rabbit. \textit{Ba ləm wɔ nthin chaŋ nvis lɛ gbi tho ɛ ko.} The rabbit is the most clever of all the animals in the bush (\citealt{Pichl1967}). 

\TCheadword{lɛmɛ} \textit{cf}: \TClink[1]{lam}. \textit{n} fish species, (Cichlidae spp.) (\citealt{Pichl1967}). 

\TCheadword[1]{lɛn} \textit{n} \textbf{1)} bird species, small black birds, stay around dug toilets – people use them to make charms (K dialect); (wɔ/hã, si) bird species, palm swift, swallow (perhaps lən small bat sp) (\citealt{Pichl1967}). \textbf{2)} lən (wɔ/hã, si) small kind of bat (\citealt{Pichl1967}). 

\TCheadword[2]{lɛn} (Eng \textit{line}) \textit{n} line. \textit{Yi kwey liwal, si yi chok lɛn ton, si yi panth lɛn do.} We take palm leaves, then we twist them to a fine line, then we tie this line.

\TCheadword{lɛnthɛkni} (unspec. of \TClink{lanth}) 

\TCheadword{lɛnthi} \textit{cf}: \TClink{futh}, \TClink[1]{sokothi}, \TClink{suth}, \TClink[2]{wɔ}. \textit{v} pluck. \textit{Ŋkɔm lɛnthiɛ nrokos ntiŋ ni mpakai nhiɔl!} Go pluck me two oranges and four papayas (\citealt{Pichl1967}).

\TCheadword{lɛnyɛ} (unspec. of \TClink[1]{lɛŋ}) 

\TCheadword[1]{lɛŋ} \textit{v} greet. \textit{Yɛ ŋa kɔ ŋa mi lɛŋ Nthemdai, ha ŋai lɛŋ Mbolomdai.} Whenever they would greet me in Themne, I would reply in Bolom.

\TCsubword[2]{lɛŋ} (der.) \textit{n} greeting. \textit{Bikɔs nɔɛ wɔŋɔ mɔ gbo lɛŋ, chɔŋ mɔ len.} Because if someone sends a greeting to you, he loves you. \textit{Mi Adama, i yema pɛ ni nwun wom lɛŋ ko ŋanɛ ŋa hunɔnimuɛ.} Mami Adama, I want also for you to come send greetings to your descendants.

\TCsubword{lɛŋlɛŋ} (der.) \textit{v} greet. \textit{Ŋa hun kɔ theɛ lomthi hiɛ hi ŋa lɛŋlɛŋ likɛlɛŋ.} The ones that know us, that would come and hear our voices, we are sending our fine greetings.

\TCsubword{lɛnyɛ} (unspec.) \textit{v} greet. \textit{Kɛ mɔm yɛ mɔ ŋa boniyɛ, Nthemdɛ lɔ ŋa lɛŋyɛ?} But when you meet them, is it in Themne you greet them? comp. \TClink{kollɛnyɛ} (see \TClink{kol}) 

\TCheadword[2]{lɛŋ} (der. of \TClink[1]{lɛŋ}) 

\TCheadword{lɛŋlɛŋ} (der. of \TClink[1]{lɛŋ}) 

\TCheadword{lɛpi} (der. of \TClink[1]{lap}, \TClink[1]{-i}, see \TClink[1]{lap}) 

\TCheadword{lɛrka} \textit{v} \textbf{1)} repair.\textit{ Ŋkɔ lɛrka bot lɛ hɔ simjɛm dɛ.} Go repair the boat, it is damaged (\citealt{Pichl1967}). \textbf{2)} make.

\TCheadword{lɛrni} \textit{n} haste. \textit{Ka nlɛrni, wɔe duki kilikɛ.} He hurried up and dropped the anchor.

\TCheadword{lɛsa} \textit{v} hold and guide in a gentle and careful manner.

\TCheadword[1]{li-} \textit{NCM} \textit{pfx} \textit{ubd stem} \textit{sfx} noun class marker. \textit{Lanɛ gbi lipika la pɛ biya yema ŋa theli ɛ.} All other things that he would want to speak about. \textit{Lanɛ la li kɛlɛŋ ahinŋa ŋan the.} That is very good for people to hear. \textit{Nkela bo la li kɛlɛŋ, ala bɔ yema.} If you see it as something good, I will want it. comp. \TClink{bɛslisoko} (see \TClink{bɛs}), \TClink{miliŋdibil} (see \TClink{miliŋ}), der. \TClink{famalifama} (see \TClink[2]{fama}), \TClink[1]{libaŋ} (see \TClink[1]{baŋ}), \TClink[3]{libaŋ} (see \TClink[1]{baŋ}), \TClink{libɛɛ} (see \TClink[1]{bɛɛ}), \TClink[1]{licharaŋ} (see \TClink[1]{charanŋ}), \TClink[2]{licharaŋ} (see \TClink[1]{charanŋ}), \TClink{lichol} (see \TClink[1]{chol}), \TClink{likith} (see \TClink[1]{kith}), \TClink{lomɔlibɛ} (see \TClink[1]{bɛɛ})

\TCsubword{buŋklipal} (unspec.) \textit{cf}: \TClink{gbes}. east (\citealt{Pichl1967}). 

\TCheadword[2]{li-} \textit{prt pfx} adverbializer. \textit{Bálmá lúɛ́ lítìŋ.} The \textit{balmaa} knife is sharp on both sides (lit. The sharp \textit{balmaa} is double (edged)). der. \TClink{lichaŋha} (see \TClink{chaŋha}), \TClink{ligber} (see \TClink{gbe}), \TClink{liwɔ} (see \TClink[3]{wɔ})

\TCsubword{libɛn} (der.) \textit{cf}: \TClink[1]{kɛkɛ}, \TClink{yas}. \textit{temp} quickly. \textit{La libɛn Bɛl Maaɛ koŋ pɛ thaŋni poŋ boeɛ tokɛ wusɛ kunɛ <tɔrɔth>.} Quickly, Rat Wife had again climbed (and) disappeared above the kitchen into the thatch <tɔrɔth> (idph of emphasis). 

 \TCsubword{libul-libul} (der.) \textit{temp} sometimes; occasionally; once in a while. \textit{Lami, wɔ abi habi wɔn pɛ wɔ che wɔl libul-libul.} My wife that I have she has to be resting occasionally. \textit{Acheŋ lɔl hɛlɛ ko, libul-libul bo ŋɔ a lɔl hɛlɛ koɛ.} I do not sleep at sea, I only sleep at sea once in a while. 

\TCheadword[1]{libaŋ} (der. of \TClink[1]{li-}, \TClink[1]{baŋ}, see \TClink[1]{baŋ}) 

\TCheadword[2]{libaŋ} (der. of \TClink[1]{li-}, \TClink[1]{baŋ}, see \TClink[1]{baŋ}) 

\TCheadword[3]{libaŋ} \textit{n} anger (\citealt{Pichl1967}).

\TCsubword[4]{libaŋ} (der.) \textit{adj} angry (\citealt{Pichl1967}).

\TCheadword{libɛɛ} (der. of \TClink[1]{li-}, \TClink[1]{bɛɛ}, see \TClink[1]{bɛɛ})

\TCheadword{libɛn} (der. of \TClink[2]{li-}) 

\TCheadword{libul-libul} (der. of \TClink[2]{li-}, \TClink[2]{bul} (der. of \TClink[3]{bul}), see \TClink[2]{li-}) 

\TCheadword{lichaŋha} (der. of \TClink[2]{li-}, \TClink{chaŋha}, see \TClink{chaŋha}) 

\TCheadword[1]{licharaŋ} (der. of \TClink[1]{li-}, \TClink[1]{charaŋ}, see \TClink[1]{charaŋ})

\TCheadword[2]{licharaŋ} (der. of \TClink[1]{licharaŋ} (der. of \TClink[1]{li-}, \TClink[1]{charaŋ}), see \TClink[1]{charaŋ}) 

\TCheadword{lichol} (der. of \TClink[1]{li-}, \TClink[1]{chol}, see \TClink[1]{chol}) 

\TCheadword{ligbem} (unspec. of \TClink{gbem}) 

\TCheadword{ligber} (der. of \TClink[2]{li-}, \TClink{gbe}, see \TClink{gbe}) 

\TCheadword{likith} (der. of \TClink[1]{li-}, \TClink[1]{kith}, see \TClink[1]{kith}) 

\TCheadword{Limba} \textit{nam} Limba people. \textit{Anyindɛ gbi ha lɔ lɔi, kɛnyɛ gbi ŋa lɔ lɔi, Athemaɛ ha ha, ha Limbaɛ gbi ha ha.} All the people enter there, the Themnes are here, the Limbas themselves are here.

\TCheadword{limbul} (unspec. of \TClink[2]{bol}) 

\TCheadword{Lipalai} \textit{nam} Lipalai, name given to a place. \textit{Ka Lipalaiko.} Here in Lipalai (in Sittia Chiefdom).


\TCheadword{liwɔ} (der. of \TClink[2]{li-}, \TClink[3]{wɔ}) 

\TCheadword[1]{lo} \textit{dem} \textbf{1)} this. \textit{Nɛn do ŋɔ ŋa nɛnthiwaŋnimɛn dɛ.} This year makes fifteen years. \textbf{2)} these. \textit{Ŋa ni lamgbantho ki ŋa chalao wɛ ŋaŋa gbem apumma mɛn do wɛ?} You (pl.) and this man you're living with, are you the ones that gave birth to (are you the parents of) these five (children)? \textit{Huɛɛ ŋɔ ken gbo, Braima wɔ le kɔ lɛliɛ mpɛl lo ki pɛiŋ.} Just as day breaks, Braima first goes to inspect these fishing lines.

\TCheadword[2]{lo} \textit{cf}: \TClink[1]{kan}, \TClink{rik}. \textit{v} weave; plait. \textit{Mi ka che vɛ thɛ lo sampatha.} My mother used to weave baskets.

\TCheadword[3]{lo} \textit{v} deliver. \textit{San dɛ koŋ lo nthin.} The otter has delivered the judgement (proverb) (\citealt{Pichl1967}). 

\TCheadword[4]{lo} \textit{indfpro} \textbf{1)} it. \textbf{2)} them. \textit{Iŋɔɛ ni pɛth ŋɔ lɔ ayen kɛ lanɛ gbo hi lɔ ke.} Sweet, sweet things are in heaven and if we believe we will see them.

\TCheadword{loɛ} \textit{cf}: \TClink{lɔl}. \textit{n} \textbf{1)} sleep. \textit{Lee gbo nduɛai, hwɛ ton-ton ŋɔɛ tipɛ lel, mmɛn dɛ ma pɔni wɔm dɛ kunɛ.} He remained sleeping, then a light rain began to fall, the water entering the boat. \textit{Lɛ hɛn gbo lom tendɛ, mbi ha sak ndɔɛ.} If you ignore the song of the bird, you will oversleep (proverb) (\citealt{TISLL1979}). \textbf{2)} sleepiness. \textbf{3)} day.

\TCheadword[1]{lok} \textit{n} type of tool, flat, wooden, carved in the shape of a knife, used for making mats (\citealt{Pichl1967}). 

\TCheadword[2]{lok} \textit{Idph} sound of breathing with difficulty. \textit{Ha bɔnthɔ baha yinɛ ni che hɔl <lok lok>.} They met their father lying breathing <lok lok> (with great difficulty) (\citealt[txt28]{Sumner1921}). 

\TCheadword[3]{lok} \textit{cf}: \TClink{anti}. \textit{n} aunt.

\TCheadword{lokal} (Eng \textit{local}) \textit{adj} local. \textit{Lɛ nɔ koyɛni gbo ha pɔn bɛmpa la, makɔni kɔtai, lokal kɔt.} If the person does not accept the settlement, they go to the court, the local court.

\TCheadword[1]{loki} \textit{n} in-law. \textit{A kɛ lokimdɛ wɔi pɔ bi bɛ ha hu ŋ saka wɔi, ŋgasumana ko, fakai ko.} Because he is my in-law, we even have to make his sacrifice (tithe) in Mokainsumana, in the village.

\TCheadword[2]{loki} (comp. of \TClink[1]{lo}, \TClink[1]{ki}, see \TClink[1]{ki}) 

\TCheadword[1]{lol} \textit{n} bitterness. \textit{Bia wɔ nche wɛy, wɔ woŋ lol thiwɛy ko ama wɔ lɛ.} Bia has bad habits, he curses at his wives with bad words. 

\TCheadword[2]{lol} \textit{v} be set free, be saved. \textit{Mbo̹lo̹m ŋwɛi ma che paalɛ bai ko, anya atïŋ dɛ hã lo̹l.} In the bad case that was recently before the court, the two men were set free (\citealt{Pichl1967}). der. \TClink{nɔloliɛ} (see \TClink{nɔ}) 

\TCsubword{loli} (der.) \textit{cf}: \TClink{tafi}. \textit{v} \textbf{1)} save. \textit{Wɔ koŋ mi loli.} He saved me (\citealt{Pichl1967}). \textbf{2)} rescue. \textit{I taŋ ŋa loli benɔ ŋa bɔnth.} We cry for rescue, no one to help. comp. \TClink{nɔloliɛ} (see \TClink{nɔ}) 

\TCheadword[3]{lol} \textit{n} gall bladder.

\TCsubword{lolki} (comp.) \textit{n} [lòlkí] vine species which can be used for poisoning people (lit. crocodile bile) (K dialect). 

\TCheadword[4]{lol} \textit{adj} bitter. \textit{Kɔfe lɛ hɔ lol.} The coffee is bitter (\citealt{Pichl1967}).

\TCheadword{lolan} \textit{v} be the last to arrive. \textit{Wɔ lolan mɔɛ.} He is the last to arrive (\citealt{Pichl1967}). 

\TCheadword{loli} (der. of \TClink[2]{lol}, \TClink[1]{-i}, see \TClink[2]{lol})

\TCheadword{lolki} (comp. of \TClink[3]{lol}, \TClink[3]{ki}, see \TClink[3]{lol}) 

\TCheadword{lolom} \textit{n} \textbf{1)} [lòlòm] grass species, a very small grass found near swamps used for medicine, palm wine tapsters use it to sift palm wine, as a sieve, also used as medicine, used to combat small insects that bother hens when they are laying (K dialect). \textbf{2)} plant species, climbing plant, regarded as an emblem of the Poro Society (\citealt{Pichl1967}). 

\TCheadword[1]{lom} \textit{n} tail. \textit{Tumɔɛ lɛ wɔ pikïth lom wɔ lɛ.} The dog wags his tail (\citealt{Pichl1967}). 

\TCheadword[2]{lom} \textit{n} \textbf{1)} voice. \textit{Pɔ ple rɛkɔd mɔɛ, pɔi theɛ lom mɔɛ ŋɔ nche pa theliɛ.} They will play your recordings, then they will hear your voice, how you used to talk. \textbf{2)} words. \textbf{3)} insulting language.

\TCsubword{lomthibul} (id.) \textit{n} \textbf{1)} agreement. \textit{Roshia ni Amɛrika hã koŋ kɔnth lomthibul le pəm kɔ koŋ.} Russia and America have made an agreement that war should cease (\citealt{Pichl1967}). \textbf{2)} unanimity.

\TCheadword[3]{lom} \textit{adj} mean, unkind. \textit{Thumɔɛ lo̹m.} You mean dog (an insult) (\citealt{Pichl1967}).

\TCheadword{lomani} \textit{cf}: \TClink{lonibolɛ}, \TClink{mɛmba}, \TClink[2]{tɛn} (der. of \TClink[1]{tɛn}). \textit{v} \textbf{1)} remember. \textit{Be, apa ni loman ja ŋɔth.} No, I do not remember knowing how to fish. \textit{A lomani yɛ Ba Ŋgobɛ ka che hun dɛ hwɛ lɛ hɔ lelɛ.} I remember when Mr. Ngobe was coming that it rained (\citealt{Pichl1967}). \textbf{2)} recognize.

\TCheadword{lomos} \textit{v} decide to undergo difficulties. \textit{Ya koŋ lomos ŋgbathil ma hwɛlɔ lɛ.} I have decided to yield to the difficulties of the world (\citealt{Pichl1967}). 

\TCheadword{lomothiŋ} \textit{n} grass species (\citealt{Pichl1967}). 

\TCheadword{lomɔ} \textit{n} (kɔ/tha) garment, gown (\citealt{Pichl1967}). \textit{Ya bi lo̹mɔ dinthɛ.} I have a white gown (\citealt{Pichl1967}). comp. \TClink{lomɔlibɛ} (see \TClink[1]{bɛɛ}) 

\TCsubword{lomɔfɔnwɛy} (comp.) \textit{n} witch's gown.

\TCheadword{lomɔfɔnwɛy} (comp. of \TClink{lomɔ}, \TClink[1]{fɔnwɛi} (comp. of \TClink[2]{wɛi}), see \TClink{lomɔ}) 

\TCheadword{lomɔlibɛ} (comp. of \TClink{lomɔ}, \TClink{libɛɛ} (der. of \TClink[1]{li-}, \TClink[1]{bɛɛ}), see \TClink[1]{bɛɛ}) 

\TCheadword{lompu} (unspec. of \TClink{lɔŋ}) 

\TCheadword{lomthibul} (id. of \TClink[2]{lom}, \TClink{thi-}, \TClink[2]{bul} (der. of \TClink[3]{bul}), see \TClink[2]{lom}) 

\TCheadword{loni} \textit{v} plait.

\TCheadword{lonibolɛ} \textit{cf}: \TClink{lomani}, \TClink{mɛmba}, \TClink[2]{tɛn} (der. of \TClink[1]{tɛn}). \textit{v} \textbf{1)} remember. \textit{Siŋthɛ vɛ tha nlonigbo ntenɛ lɛ nkache siŋ?} Those are the only games you remembered that you used to play? \textit{Acheŋɔni pɛ lonibolɛ, bikɔs pɔ chiɛmi ka yaŋ taa.} I would not remember it because I was brought here when I was very young. \textit{Nche ni loni bolɛ bul bɛ?} Would you not be able to remember even one? \textbf{2)} consider; bear in mind. \textit{Ŋa ŋa awɔŋɔ lɛŋ yeŋkɛlɛŋ ba, ŋa loni bolɛ in bɛ iŋaka ŋa ŋan.} They are the ones I am sending this fine greeting for; they should bear in mind that we are here for them. \textbf{3)} realize. \textit{Iko loni bolɛ lɛ ayen gbe ko lɔpɔ kache theli Mbolomdɛ, pɔ che lɔ pɛ theli Mbolomdɛ.} We have realized that in many places where they used to speak Sherbro, they no longer speak Sherbro.

\TCheadword{lonth} \textit{n} plucking pole, long pole or stick with a two- pronged fork on one end, used to pluck fruit, esp. kola (\citealt{Pichl1967}). 

\TCheadword{lontho} \textit{n} [lónthó] okra (K dialect); (kɔ/ma) okra (Hibiscus esculentus)(\citealt{Pichl1967}).

\TCheadword{loŋk} \textit{n} (hɔ̃/tha) knee (\citealt{Pichl1967}).

\TCheadword{Lɔ} \textit{nam} \textbf{1)} Lord. \textit{Lɔ Jizɔs sɛmɛ ŋa loli aŋa wɔ.} Lord Jesus rises to save his people. \textbf{2)} Law. \textit{Yaŋ ilel miɛ hɔn Mabɛl Lɔ.} Me my name is Mabel Law.

\TCheadword[1]{lɔ} \textit{Loc} there. \textit{Mɔ lɔ bɔnth apuma mɔ ɛ han gbi}. You will meet all your children there (\citealt{Pichl1967}). \textit{Ika chelɔ mpaŋ bul.} We were there for one month. comp. \TClink{gbalɔni} (see \TClink{gbani}), der. \TClink{gbɔmɔlɔ} (see \TClink{gbɔm}) 

\TCsubword{lɔko} (comp.) \textit{Loc} there. \textit{Kàá kó ŋàlà, kàá kó lɔ̀kò.} The hoe is here, the hoe is there.

\TCsubword{lɔolɔ} (der.) \textit{Loc} place. \textit{Ayen lɔlɔ lɔi nan yenchɛkɛ tɛŋka dɔzin ra, dɔzin tin.} There is a place where we draw the fish, like three dozen, two dozen.

\TCheadword[2]{lɔ} \textit{NCP} \textbf{1)} it. \textit{Ŋkɔ gbïl iwɔm dɛ lal l'ay kɔ, jɛmdi lɛ lɔ yema nyum.} Go put wood on the fire, the fire is about to go out (\citealt{Pichl1967}). \textit{A kɔ hã kwey lijɛm kɛ jɛmdi lɔ lɔ ithïheng.} I went to take some fire but the fire there was not proper (\citealt{Pichl1967}) \textit{Ken di lɛ lɔ luɛ.} The knife is sharp (\citealt{Pichl1967}). \textbf{2)} where. der. \TClink{bɛrɛlɔni} (see \TClink[2]{bɛ}), \TClink{silɔ} (see \TClink{siil}) 

\TCheadword[3]{lɔ} \textit{cf}: \TClink{chɛtlipalkɔ} (unspec. of \TClink[2]{chɛth}, \TClink[1]{pal}, \TClink[2]{kɔ}) \textit{Loc} west.

\TCheadword[4]{lɔ} \textit{cop} be. \textit{Wɔn bɛ Salima ko lɔ ka cheɛ.} She herself used to be in Salima. \textit{Abatokɛ bɛ lɔ ruba.} May God be with you. \textit{Ko lɔ pɔ gbem wɔi lɔpɔ wɔ pɛ mina pɛ dum?} Where she was born is where she was also raised? \textit{Bɔllɛ ŋɔn lagbolɛ mɛŋk, mɛŋkɛ hɔ mɔigbo, ŋakɔni fillai ŋa kɔ siŋ.} The football (match) is scheduled, when the time comes, they go to the field and play. \textit{Mɔmɔ lɔ nɔsendɛ?} Are you the first one? \textit{Alɔ S. I. Koroma.} I am S.I. Koroma. \textit{Tak Bahin yɛ wɔ isi wɔn kɛndɛ oh wɔi lɛ Jizɔs sɛ.} The son of God that we know is only Jesus. \textit{Beo che lɛ ni pɛ.} No he is not there anymore.

\TCheadword[5]{lɔ} \textit{interrog} where. \textit{Wɔn dɔ pɔ dumɔ wɔa?} Where was she raised? comp. \TClink{ndɔlɔ} (see \TClink{ndɔ}) 

\TCheadword[6]{lɔ} \textbf{1)} \textit{subordconn} where. \textit{Ko lɔ Kaiŋ Taso hinɛ pɛllɛaiɛ, wɔ ke ni wɔ thee la bɛl siatiŋ dɛ ŋa thoŋka ŋan thiyeŋ dɛ.} Where Kain Tasso was lying in the hammock, he sees and hears what the two rats are arguing about. \textit{Triai lɔ pɔ piŋɛ gbo koŋ yaɛ…} In the town where they had first done the cooking… \textbf{2)} \textit{NCP} relative pronoun. 

\TCheadword[7]{lɔ} \textit{post} below. \textit{Oŋ dɔ (bondɔ)...} Below the cliff, on the waterside, landing place, wharf. comp. \TClink{bondɔ} (see \TClink[1]{boŋ}), \TClink{kunɔlɔ} (see \TClink{kun}) 

\TCheadword[8]{lɔ} \textit{ncm} fill out based on other markers. 
 
\TCheadword[1]{lɔi} \textit{v} enter (B dialect); \textit{luɛi} come or go in (\citealt{Sumner1921}). \textit{Mma hã lwɛ thiyeng, siminjɛm bo̹m hɔ̃ hani ki.} Do not enter between (don't interfere), this is a big misunderstanding (\citealt{Pichl1967}). 

\TCsubword{lɔik} (der.) \textit{v} enter. \textit{So yɛ pɔ lɔik wanda maɛ, pɔ wɔi ko kaŋ len-o-len.} So when they initiate a girl, they teach her everything. \textit{Boŋgo che ki, nɔ mbiyɛni gbo fe nche lɔik Bondo.} These days, if one has no money, one will not enter Bondo.

\TCheadword[2]{lɔi} \textit{v} produce. \textit{Mma vəkɛth su-m dɛ, kɔ hinth ni lwɛ nse, mma ki-m nɛki.} Don't squeeze my finger, it will swell and produce pus, don't hurt me! (\citealt{Pichl1967}). 

\TCheadword[3]{lɔi} \textit{cf}: \TClink[2]{bi}, \TClink[2]{ha}, \TClink[1]{ma}, \TClink{mɔs}, \TClink[2]{ŋa}. \textit{Aux} have to; must. \textit{Kɔ koŋ gbo lɔ, mɔ lɔi bɛthi jɛmlɛ.} After it has cooked, you have to reduce the fire.

\TCheadword{lɔik} (der. of \TClink[1]{lɔi}, \TClink{-k}, see \TClink[1]{lɔi}) 

\TCheadword[1]{lɔk} \textit{n} drumstick.

\TCsubword[2]{lɔk} (der.) \textit{v} beat a drum. \textit{Táámòɛ̀ wɔ́ lɔ́k bándɛ̀ wɔ̀ kí.} This is the boy who plays the drum. 

\TCheadword{lɔko} (comp. of \TClink[1]{lɔ}, \TClink[2]{ko}, see \TClink[1]{lɔ}) 

\TCheadword{lɔkɔ} \textit{cf}: \TClink{bonk}, \TClink[1]{mɛŋk}, \TClink[1]{tɛm}. \textit{n} day; time; period. \textit{Yan deɛ ŋɔ huɛ lɔkɔɛ ŋɔ hu wɛ, aka shilani.} As for me, the day he died the day he died, I did not know.

\TCsubword{lɔkɔolɔkɔ} (der.) \textit{cf}: \TClink{wɔiowɔi} (der. of \TClink[2]{hu}, \TClink{-o-}). \textit{temp} all the time; always. \textit{Lɔkɔɔlɔkɔ hɔ ya hun dɛ, ya bɔnth wɔ hã mpanth.} Always when I come, I meet him at work (\citealt{Pichl1967}). 

\TCheadword{lɔkɔlɔkɔ} \textit{n} [lɔ̀kɔ̀lɔ̀kɔ̀] vine species, light grey, bears fruit, several can be intertwined together in a clump (K dialect). 

\TCheadword{lɔkɔolɔkɔ} (der. of \TClink{lɔkɔ}, \TClink{-o-}, see \TClink{lɔkɔ}) 

\TCheadword{lɔl} \textit{cf}: \TClink{loɛ}. \textit{v} sleep. \textit{Ŋɔ̃́ŋ lɔ̀là?} How did you sleep? \textit{N lɔ̀llɔ́ ɲɛ̀ŋkɛ̀lɛ́ŋ?} Did you sleep well?

\TCsubword{lɔlma} (comp.) \textit{cf}: \TClink[1]{mɔn}, \TClink{twɛ}. \textit{v} copulate; sleep with a woman (\citealt{Pichl1967}). 

\TCsubword{lɔlni} (der.) \textit{v} be sleepy, be drowsy (\citealt{Pichl1967}). \textit{Sak lɛ kɔ bi ni ya che lɔlni.} On account of dancing the whole night, I am sleepy (\citealt{Pichl1967}). 

\TCheadword{lɔlma} (comp. of \TClink{lɔl}, \TClink[4]{ma}, see \TClink{lɔl}) 

\TCheadword{lɔlni} (der. of \TClink{lɔl}, \TClink{-ni}, see \TClink{lɔl}) 

\TCheadword{lɔlɔk} \textit{n} [lɔ̀lɔ̀k] duck species, domesticated ducks used for charms against people, to make people sick. Soothsayers will find out by referring sickness to this kind of duck – “I do not have children like a duck” (my family does not behave as ducks do), children go in front, not behind (something to do with parents dying first) (K dialect). 

\TCheadword{Lɔmli} \textit{nam} Lumley, name given to a place. \textit{Laŋgba dɛ fli wɔ ŋa fɛtɛndɛ Lɔmli, Malama Bolomnɔ.} Even the man they are close with at Lumley, Malama, is Sherbro.

\TCheadword{lɔntha} \textit{n} \textbf{1)} end. \textit{Lɔntha ya mɛkɛni.} There I end (\citealt{Pichl1967}). \textbf{2)} finish. \textit{La koŋ gbo vɛ lɔntha.} It is all finished (\citealt{Pichl1967}). 

\TCheadword{lɔŋ} \textit{v} \textbf{1)} set a trap. \textit{Lɛ nsi gbo lɔŋ, nsi gbo hɔth, mɔ sɔthɔ yen sɔmɔ.} If you know how to set traps and how to fish, you will have something to chew. \textit{Ahɔth, alɔŋ.} I went fishing, I set traps. \textbf{2)} set.

\TCsubword{lompu} (unspec.) \textit{cf}: \TClink[2]{sɔŋk} (der. of \TClink[1]{sɔŋk}). \textit{v} \textbf{1)} set a trap. \textbf{2)} cock a gun.

\TCheadword{lɔŋg} (Eng \textit{long}) \textit{temp} a while, a long time. \textit{Aa, kɛ bifo dat akoni che ko administreshɔn dɛ fɔ lɔŋg.} Yes, but before that I had been in administration for a while.

\TCheadword{lɔŋnui} (unspec. of \TClink{nui}) 

\TCheadword{lɔolɔ} (der. of \TClink[1]{lɔ}, \TClink{-o-}, see \TClink[1]{lɔ}) 

\TCheadword[1]{lɔɔ} \textit{adj} \textbf{1)} ugly. \textit{Wɔ ilɔɔ.} He is ugly (\citealt{Pichl1967}). \textbf{2)} bad. 3) wrong. \textit{Bi ilɔɔ.} He is in the wrong (as in court) (\citealt{Pichl1967}).

\TCheadword[2]{lɔɔ} \textit{n} guilt. \textit{Lɛ bi gbo ilɔɔ, pɔ wɔ di sɔŋkɔma ŋɔ saba tirɛ hɔɛ.} If he is guilty, they will kill him as the town law says. \textit{Taalaŋgbaɛ biɛni ilɔɔ gbi.} The young man has no guilt at all.

\TCheadword{lua} \textit{cf}: \TClink{kokkunɛ} (comp. of \TClink{kok}, \TClink{kunɛ}). \textit{n} hernia.

\TCheadword{luba} \textit{n} (kɔ/ma) plant species, ringworm shrub, craw-craw plant (\citealt{Pichl1967}). 

\TCheadword[1]{luɛ} \textit{v} [lúɛ́] sharp, minimal pair with [lùɛ̀] hole (K dialect). \textit{Bálmá lúɛ́ lítìŋ.} The \textit{balmaa} knife is sharp on both sides (lit. The sharp \textit{balmaa} is double (edged)).

\TCheadword[2]{luɛ} \textit{n} \textbf{1)} [lùɛ̀] hole (K dialect); \textit{luɛi} (pl. thi) well (\citealt{Sumner1921}). \textit{Búé lùɛ̀.} Dig or hollow out a hole in the ground or a tree, put in a hole. \textbf{2)} well. \textit{Búé ùɛ̀.} Dig a well.



\end{letter}
